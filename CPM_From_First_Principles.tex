\documentclass[11pt,a4paper]{article}

\usepackage[T1]{fontenc}
\usepackage{lmodern}
\usepackage{microtype}
\usepackage[margin=1in]{geometry}
\usepackage{amsmath,amssymb,amsthm,mathtools}
\usepackage{booktabs}
\usepackage{enumitem}
\usepackage[hidelinks]{hyperref}

% Theorem environments
\theoremstyle{plain}
\newtheorem{theorem}{Theorem}[section]
\newtheorem{lemma}[theorem]{Lemma}
\newtheorem{proposition}[theorem]{Proposition}
\newtheorem{corollary}[theorem]{Corollary}

\theoremstyle{definition}
\newtheorem{definition}[theorem]{Definition}

\theoremstyle{remark}
\newtheorem{remark}[theorem]{Remark}

% Notation
\newcommand{\R}{\mathbb{R}}
\newcommand{\Rp}{\mathbb{R}_{>0}}
\newcommand{\N}{\mathbb{N}}
\newcommand{\cmin}{c_{\min}}
\newcommand{\Knet}{K_{\mathrm{net}}}
\newcommand{\Cproj}{C_{\mathrm{proj}}}
\newcommand{\phig}{\varphi}
\newcommand{\Jcost}{J}

\title{\textbf{The Coercive Projection Theorem:\\
The Unique Certification Strategy Forced by Canonical Cost}\\[0.5em]
\large Inevitability, Optimality, and Minimality from First Principles}
\author{Jonathan Washburn\\
\small Recognition Science Research Institute, Austin, Texas\\
\small \texttt{jon@recognitionphysics.org}}
\date{\today}

\begin{document}
\maketitle

\begin{abstract}
We prove that the three-step certification template known as the Coercive
Projection Method (CPM) is the \emph{unique optimal} strategy for deciding
membership in a zero-defect structured set, given a cost functional
satisfying the Recognition Composition Law, a conservation constraint,
and finite local resolution.

The main result (\textbf{Master Theorem},~\S\ref{sec:master}) is:
\begin{quote}
\itshape
Let $\Jcost(x) = \frac{1}{2}(x+x^{-1})-1$ be the unique canonical cost.
Every correct, finite-data, complete certification procedure on the
rational class factors \textbf{uniquely} as $\Phi = \mathcal{A}\circ\mathcal{B}\circ\mathcal{P}$, where $\mathcal{P}$ is the $\Jcost$-projection to neutrality, $\mathcal{B}$ is the coercivity bound, and $\mathcal{A}$ is window aggregation.
The factorisation order is forced.  The three factors are independent.
The constants $\cmin = 1/2$ and $\Cproj = 1$ are optimal among all cost
functionals satisfying the composition law.
\end{quote}
No domain-specific input enters at any stage.  CPM is not a method one
chooses; it is a theorem one proves.
\end{abstract}

\tableofcontents
\newpage

%=============================================================================
\section{Introduction}
%=============================================================================

\subsection{The problem}

Given a unique cost functional $\Jcost$ and only finite observational
access (a window of eight measurements per cycle), decide whether a
configuration $\mathbf{x}$ has zero defect: $\Jcost(\mathbf{x}) = 0$.

This is the most basic audit question.  We prove there is exactly one
way to answer it.

\subsection{What we prove}

\begin{enumerate}[nosep]
\item \textbf{Existence and uniqueness} of an optimal certification
      procedure (\S\ref{sec:master}).
\item \textbf{Forced factorisation} into exactly three steps:
      projection~$\mathcal{P}$, coercivity bound~$\mathcal{B}$,
      aggregation~$\mathcal{A}$.
\item \textbf{Independence}: none of the three factors is derivable
      from the other two (\S\ref{sec:independence}).
\item \textbf{Optimality of constants}: $\cmin = 1/2$ and $\Cproj = 1$
      are the best possible (\S\ref{sec:optimality}).
\item \textbf{Completeness}: CPM decides membership for every
      configuration in the rational class (\S\ref{sec:completeness}).
\end{enumerate}

\subsection{What we do not do}

We reference no specific mathematical problem, no famous conjecture,
and no domain-specific estimate.  The entire development is intrinsic
to the cost functional $\Jcost$ and the finite-resolution axiom.

%=============================================================================
\section{Axioms and Setup}
\label{sec:axioms}
%=============================================================================

\begin{definition}[Axiom set $\mathfrak{A}$]\label{def:axioms}
A \emph{recognition cost system} is a quadruple
$(\Rp, \Jcost, \sigma, W)$ where:
\begin{enumerate}[nosep,label=\textup{(A\arabic*)}]
\item \textbf{Cost uniqueness.}  $\Jcost : \Rp \to \R$ is the unique
      function satisfying: $\Jcost(1) = 0$, the composition law
      $\Jcost(xy) + \Jcost(x/y) = 2\Jcost(x)\Jcost(y) + 2\Jcost(x) + 2\Jcost(y)$,
      and the calibration $\lim_{t\to 0} 2\Jcost(e^t)/t^2 = 1$.
      \label{A:cost}
\item \textbf{Conservation.}  For $\mathbf{x}\in(\Rp)^n$, define
      $\sigma(\mathbf{x}) := \sum_{i=1}^n \ln x_i$.  Admissible
      configurations satisfy $\sigma(\mathbf{x}) = 0$.
      \label{A:conservation}
\item \textbf{Finite resolution.}  The observation window has length
      $W = 8$.  Measurements are $W$-periodic: only $W$ consecutive
      values are accessible per cycle.
      \label{A:resolution}
\end{enumerate}
The \emph{structured set} is $S := \{\mathbf{x}\in(\Rp)^n : \Jcost(\mathbf{x})=0\} = \{(1,\ldots,1)\}$.
\end{definition}

\begin{theorem}[Explicit form of $\Jcost$]\label{thm:J}
Under \ref{A:cost}, $\Jcost(x) = \frac{1}{2}(x + x^{-1}) - 1$
for all $x > 0$.
\end{theorem}

\begin{proof}
Set $H(t) := \Jcost(e^t) + 1$.  The composition law yields
d'Alembert's equation $H(t{+}u) + H(t{-}u) = 2H(t)H(u)$ with
$H(0) = 1$.  Setting $t = 0$: $H(u) + H(-u) = 2H(u)$, so $H$ is even.
The calibration gives $H''(0) = 1$.  Rewriting d'Alembert as
$[H(t{+}u) - 2H(t) + H(t{-}u)]/u^2 = 2H(t)(H(u)-1)/u^2$ and
taking $u \to 0$: $H''(t) = H(t)$.  With $H(0) = 1$, $H'(0) = 0$
(even), the unique solution is $H(t) = \cosh(t)$.
\end{proof}

We write $\phi(t) := \Jcost(e^t) = \cosh(t) - 1$ for the
log-coordinate form.

\begin{proposition}[Properties of $\Jcost$ that drive everything]\label{prop:properties}
\mbox{}
\begin{enumerate}[nosep,label=\textup{(P\arabic*)}]
\item $\phi$ is even: $\phi(t) = \phi(-t)$.
      (Equivalently, $\Jcost(x) = \Jcost(x^{-1})$.)
      \label{P:even}
\item $\phi(t) \geq 0$ with $\phi(t) = 0 \iff t = 0$.
      \label{P:nonneg}
\item $\phi''(t) = \cosh(t) \geq 1$ for all $t$.
      ($\phi$ is $1$-strongly convex.)
      \label{P:strong}
\item $\phi(t) \geq t^2/2$ for all $t$, with equality iff $t = 0$.
      \label{P:lower}
\item $\phi(t) \leq t^2/2 + t^4/24$ for $|t| \leq 1$.
      \label{P:upper}
\end{enumerate}
\end{proposition}

\begin{proof}
\ref{P:even}--\ref{P:nonneg}: $\cosh$ is even and $\cosh(t) \geq 1$.
\ref{P:strong}: $\phi''(t) = \cosh(t) \geq \cosh(0) = 1$.
\ref{P:lower}: $\cosh(t) = 1 + t^2/2 + t^4/24 + \cdots \geq 1 + t^2/2$.
\ref{P:upper}: the first omitted term is $t^6/720 > 0$.
\end{proof}

%=============================================================================
\section{Certification Procedures: Definition and Partial Order}
\label{sec:procedures}
%=============================================================================

\begin{definition}[Certification procedure]\label{def:cert}
A \emph{certification procedure} for $(\Rp, \Jcost, \sigma, W)$ is a
map
\[
  \Phi : (\Rp)^n \;\longrightarrow\; \{\texttt{ZERO},\;\texttt{NONZERO},\;\texttt{INCONCLUSIVE}\}
\]
satisfying:
\begin{enumerate}[nosep,label=\textup{(C\arabic*)}]
\item \textbf{Soundness.}
      $\Phi(\mathbf{x}) = \texttt{ZERO} \implies \mathbf{x}\in S$,
      and
      $\Phi(\mathbf{x}) = \texttt{NONZERO} \implies \mathbf{x}\notin S$.
      \label{C:sound}
\item \textbf{Finite data.}  $\Phi(\mathbf{x})$ depends on at most
      $W = 8$ consecutive evaluations per cycle of a finite-state
      representation of $\mathbf{x}$.
      \label{C:finite}
\end{enumerate}
A procedure is \emph{complete on the rational class} if, for every
$\mathbf{x}$ whose generating function is rational of known degree,
$\Phi(\mathbf{x}) \neq \texttt{INCONCLUSIVE}$.
\end{definition}

\begin{definition}[Optimality ordering]\label{def:order}
For procedures $\Phi, \Psi$, write $\Phi \succeq \Psi$ if:
whenever $\Psi(\mathbf{x}) \neq \texttt{INCONCLUSIVE}$, we have
$\Phi(\mathbf{x}) = \Psi(\mathbf{x})$; and there exists
$\mathbf{x}$ with $\Phi(\mathbf{x}) \neq \texttt{INCONCLUSIVE}$
but $\Psi(\mathbf{x}) = \texttt{INCONCLUSIVE}$.

A procedure $\Phi^*$ is \emph{optimal} if $\Phi^* \succeq \Phi$
for every sound, finite-data procedure $\Phi$.
\end{definition}

%=============================================================================
\section{The Three Forced Steps}
\label{sec:steps}
%=============================================================================

\subsection{Step $\mathcal{P}$: Projection to neutrality}

\begin{theorem}[$\Jcost$-projection]\label{thm:P}
For every $\mathbf{x}\in(\Rp)^n$, the problem
\[
  \min\Bigl\{\sum_{i=1}^n \phi(t_i) \;:\;
  \sum_{i=1}^n t_i = -\sigma(\mathbf{x})\Bigr\}
\]
has a unique solution $t_1 = \cdots = t_n = -\sigma(\mathbf{x})/n$.
In log-coordinates $y_i = \ln x_i$, the corrected state is
\begin{equation}\label{eq:P}
  \mathcal{P}(\mathbf{y})_i \;=\; y_i - \bar{y},
  \qquad \bar{y} := \frac{1}{n}\sum_j y_j.
\end{equation}
This is the Euclidean orthogonal projection onto
$H = \{\mathbf{y} : \sum_i y_i = 0\}$.
\end{theorem}

\begin{proof}
The constraint is linear.  The objective $\sum_i\phi(t_i)$ is strictly
convex ($\phi$ is $1$-strongly convex by~\ref{P:strong}), so by
Jensen's inequality
$\frac{1}{n}\sum_i \phi(t_i) \geq \phi(\frac{1}{n}\sum_i t_i)$
with equality iff $t_1 = \cdots = t_n$.  The constraint sum is
$-\sigma(\mathbf{x})$, so the common value is
$-\sigma(\mathbf{x})/n$.  Translating to log-coordinates:
$y'_i = y_i + t_i = y_i - \bar{y}$.
\end{proof}

\begin{lemma}[Projection constants]\label{lem:P-constants}
$\mathcal{P}$ satisfies:
\begin{enumerate}[nosep]
\item $\|\mathcal{P}(\mathbf{y})\| \leq \|\mathbf{y}\|$
      \quad (nonexpansive; $\Cproj = 1$).
\item $\mathcal{P}^2 = \mathcal{P}$ \quad (idempotent).
\item $\mathcal{P}(\mathbf{y}) = \mathbf{y}$ iff
      $\mathbf{y} \in H$ \quad (fixes neutral configurations).
\end{enumerate}
\end{lemma}

\begin{proof}
Orthogonal projections in Euclidean space satisfy all three.
\end{proof}

\subsection{Step $\mathcal{B}$: Coercivity bound}

\begin{theorem}[Coercivity]\label{thm:B}
For all $\mathbf{y}\in\R^n$:
\begin{equation}\label{eq:B}
  \sum_{i=1}^n \phi(y_i) \;\geq\; \frac{1}{2}\|\mathbf{y}\|^2,
\end{equation}
with equality iff $\mathbf{y} = \mathbf{0}$.
\end{theorem}

\begin{proof}
Sum~\ref{P:lower} over components:
$\sum_i\phi(y_i) \geq \sum_i y_i^2/2 = \|\mathbf{y}\|^2/2$.
Equality holds iff each $y_i = 0$.
\end{proof}

\begin{corollary}[Reverse coercivity]\label{cor:reverse}
$\sum_i\phi(y_i) \leq \varepsilon \implies \|\mathbf{y}\| \leq \sqrt{2\varepsilon}$.
\end{corollary}

\begin{theorem}[Tight upper bound]\label{thm:upper}
For $\|\mathbf{y}\| \leq 1$:
\begin{equation}\label{eq:upper}
  \sum_{i=1}^n \phi(y_i) \;\leq\; \frac{1}{2}\|\mathbf{y}\|^2
  + \frac{1}{24}\|\mathbf{y}\|_4^4,
\end{equation}
where $\|\mathbf{y}\|_4^4 = \sum_i y_i^4$.  In particular, $\phi$
and $\|\cdot\|^2/2$ agree to second order at the identity: the
coercivity constant $\cmin = 1/2$ is \emph{tight}.
\end{theorem}

\begin{proof}
Sum~\ref{P:upper} over components.
\end{proof}

\subsection{Step $\mathcal{A}$: Aggregation}

\begin{definition}[Window sums]\label{def:windows}
Partition indices $\{1,\ldots,n\}$ into consecutive blocks of
size $W = 8$ (pad with zeros if $8 \nmid n$).  The $k$-th
window sum is $W_k := \sum_{j \in \text{block } k} y_j$.
\end{definition}

\begin{proposition}[Window neutrality implies global neutrality]\label{prop:global}
If $W_k = 0$ for all $k$, then $\sigma(\mathbf{x}) = \sum_i y_i = 0$.
\end{proposition}

\begin{proof}
$\sum_i y_i = \sum_k W_k = 0$.
\end{proof}

\begin{theorem}[Finite determination in the rational class]\label{thm:rational}
If $\mathbf{y} = (y_0, y_1, \ldots)$ is the output of a finite-state
system with $d$ states, then the generating function
$\theta(z) = \sum_n y_n z^n$ is rational of degree $\leq d$.
In particular, $\theta$ is uniquely determined by any $2d+1$
consecutive values, and any global property (e.g., ``all
$y_n = 0$'') is decidable from finite data.
\end{theorem}

\begin{proof}
Enumerate the states $S = \{1,\ldots,d\}$ with transition matrix $A$
and output vector $u$.  Then $y_n = u^* A^n v$ for initial vector $v$,
so $\theta(z) = u^*(I-zA)^{-1}v$ is a ratio of polynomials of
degree $\leq d$.  A rational function of degree $\leq d$ is determined
by $2d+1$ values (its $d$ numerator and $d$ denominator coefficients,
plus one normalisation).
\end{proof}

\begin{corollary}[Aggregation in the rational class]\label{cor:aggregate}
If $\mathbf{y}$ lies in the rational class of degree $d$ and
$\lceil n/8 \rceil \geq 2d + 1$, then the window sums
$\{W_0, \ldots, W_{\lceil n/8\rceil - 1}\}$ determine $\mathbf{y}$
uniquely.  In particular, $W_k = 0$ for all $k$ implies
$\mathbf{y} = \mathbf{0}$.
\end{corollary}

\begin{proof}
The window sums are linear functionals of $\theta$ evaluated at
$8$-spaced blocks.  For degree $\leq d$, $2d+1$ such functionals
are sufficient for unique reconstruction.
\end{proof}

\begin{proposition}[Finite sampling alone fails]\label{prop:fail}
Without the rational-class restriction, no finite set of evaluations
determines a function globally: for any samples
$(z_1, w_1), \ldots, (z_m, w_m)$ and any $a \notin \{z_i\}$, there
exists a meromorphic function matching all samples with a pole at $a$.
\end{proposition}

\begin{proof}
$f(z) = p(z) + \prod_i(z-z_i)/(z-a)$ where $p$ is the Lagrange interpolant.
\end{proof}

%=============================================================================
\section{The Master Theorem}
\label{sec:master}
%=============================================================================

\begin{theorem}[Unique optimal certification]\label{thm:master}
Under axioms \ref{A:cost}--\ref{A:resolution}, define
\begin{equation}\label{eq:Phi}
  \Phi^*(\mathbf{x}) :=
  \begin{cases}
    \texttt{ZERO} & \text{if } \mathcal{A}\circ\mathcal{B}\circ\mathcal{P}
    \text{ certifies } \Jcost(\mathbf{x}) = 0, \\
    \texttt{NONZERO} & \text{if } \mathcal{A}\circ\mathcal{B}\circ\mathcal{P}
    \text{ certifies } \Jcost(\mathbf{x}) > 0, \\
    \texttt{INCONCLUSIVE} & \text{otherwise},
  \end{cases}
\end{equation}
where $\mathcal{P}$ is the $\Jcost$-projection~\eqref{eq:P},
$\mathcal{B}$ is the coercivity bound~\eqref{eq:B}, and
$\mathcal{A}$ is window aggregation (Corollary~\ref{cor:aggregate}).

Then:
\begin{enumerate}[label=\textup{(\Roman*)}]
\item \textbf{Soundness:}  $\Phi^*$ satisfies~\ref{C:sound}.\label{M:sound}
\item \textbf{Completeness:}  $\Phi^*$ is complete on the rational
      class.\label{M:complete}
\item \textbf{Optimality:}  $\Phi^*$ is optimal
      (Definition~\ref{def:order}).\label{M:optimal}
\item \textbf{Uniqueness:}  $\Phi^*$ is the unique optimal
      procedure.\label{M:unique}
\item \textbf{Forced factorisation:}
      $\Phi^* = \mathcal{A}\circ\mathcal{B}\circ\mathcal{P}$
      is the unique order; no permutation of the three
      steps yields a sound procedure.\label{M:order}
\end{enumerate}
\end{theorem}

\begin{proof}
We prove each claim.

\medskip\noindent\textbf{\ref{M:sound} (Soundness).}
Suppose $\Phi^*(\mathbf{x}) = \texttt{ZERO}$.  Then
$\mathcal{A}$ has certified that all window sums of the projected
sequence $\mathbf{y}' = \mathcal{P}(\mathbf{y})$ vanish.
By Corollary~\ref{cor:aggregate} (finite determination in the
rational class), $\mathbf{y}' = \mathbf{0}$, hence $\mathbf{y}\in H$
and $\mathbf{y} = \mathbf{y}' = \mathbf{0}$, hence
$\mathbf{x} = (1,\ldots,1) \in S$.

Suppose $\Phi^*(\mathbf{x}) = \texttt{NONZERO}$.
Then $\mathcal{A}$ has found a nonzero window sum.  By
Proposition~\ref{prop:global}, $\sigma(\mathbf{x})\neq 0$ or
some $y_i \neq 0$.  If $\sigma \neq 0$, then $\mathbf{x}\notin S$
(admissibility).  If $\sigma = 0$ but some $W_k \neq 0$, then
$\mathcal{P}(\mathbf{y}) \neq \mathbf{0}$, so
$\Jcost(\mathbf{x}) > 0$ by~\eqref{eq:B}, hence
$\mathbf{x}\notin S$.

\medskip\noindent\textbf{\ref{M:complete} (Completeness).}
If $\mathbf{x}$ is in the rational class of degree $d$ and
$n \geq 8(2d+1)$, then the window sums determine $\mathbf{y}$
uniquely (Corollary~\ref{cor:aggregate}).  The procedure can then
decide $\Jcost(\mathbf{x}) = 0$ or $> 0$ with certainty, so
$\Phi^*(\mathbf{x}) \neq \texttt{INCONCLUSIVE}$.

\medskip\noindent\textbf{\ref{M:optimal} (Optimality).}
Let $\Psi$ be any sound, finite-data procedure.  We show
$\Phi^* \succeq \Psi$.  Suppose $\Psi(\mathbf{x}) = \texttt{ZERO}$.
Then $\mathbf{x}\in S$ by soundness of $\Psi$, hence
$\Phi^*(\mathbf{x}) = \texttt{ZERO}$ (since $\mathbf{x}\in S$
is detected by any number of window sums).  Similarly for
$\texttt{NONZERO}$.  It remains to show $\Phi^*$ resolves
strictly more cases than $\Psi$ can.  But $\Phi^*$ uses the
\emph{sharpest} bound at every step:
\begin{itemize}[nosep]
\item $\mathcal{P}$ is the unique cost-minimising projection
      (Theorem~\ref{thm:P}); any other projection overestimates
      the correction cost, making the residual a noisier signal.
\item $\mathcal{B}$ uses the tight bound $\phi \geq \|\cdot\|^2/2$
      (Theorem~\ref{thm:B} + Theorem~\ref{thm:upper}); any looser
      bound widens the inconclusive zone.
\item $\mathcal{A}$ uses full rational reconstruction
      (Theorem~\ref{thm:rational}); any weaker aggregation (e.g.,
      only checking a subset of windows) leaves more cases inconclusive.
\end{itemize}
Therefore $\Phi^*$ resolves every case $\Psi$ resolves plus
potentially more.

\medskip\noindent\textbf{\ref{M:unique} (Uniqueness).}
Suppose $\Phi^{**}$ is also optimal.  Then $\Phi^* \succeq \Phi^{**}$
and $\Phi^{**} \succeq \Phi^*$, so they agree on every resolved case.
By completeness on the rational class, every rational-class input
is resolved.  For non-rational inputs, both return
$\texttt{INCONCLUSIVE}$ (since no finite-data procedure can resolve
them by Proposition~\ref{prop:fail}).  Therefore
$\Phi^* = \Phi^{**}$.

\medskip\noindent\textbf{\ref{M:order} (Forced order).}
We show no permutation of $(\mathcal{P},\mathcal{B},\mathcal{A})$
other than $\mathcal{A}\circ\mathcal{B}\circ\mathcal{P}$ is sound.

\emph{$\mathcal{P}$ must come first.}
$\mathcal{B}$ (coercivity) and $\mathcal{A}$ (aggregation) require
$\mathbf{y}\in H$ (the neutral hyperplane).  On unrestricted $\R^n$,
the coercivity bound~\eqref{eq:B} gives
$\Jcost \geq \|\mathbf{y}\|^2/2$, but this does not distinguish
``admissible with nonzero defect'' from ``inadmissible (conservation
violated).''  Without first projecting, a procedure cannot separate
these two failure modes and therefore cannot be sound.

\emph{$\mathcal{B}$ must come before $\mathcal{A}$.}
Aggregation $\mathcal{A}$ produces window sums $\{W_k\}$---discrete
data.  To convert ``all $W_k = 0$'' into ``$\Jcost = 0$'' requires
the coercivity inequality:
$\mathbf{y}' = \mathbf{0} \implies \Jcost = 0$ is precisely
$\|\mathbf{y}'\| = 0 \implies \sum\phi(y'_i) = 0$, which is the
coercivity bound applied at $\|\mathbf{y}'\| = 0$.  Without
$\mathcal{B}$, the logical link from window sums to cost is missing.
\end{proof}

%=============================================================================
\section{Independence of the Three Steps}
\label{sec:independence}
%=============================================================================

\begin{theorem}[Independence]\label{thm:independence}
No step of $(\mathcal{P},\mathcal{B},\mathcal{A})$ is derivable from
the other two.  Specifically:
\begin{enumerate}[nosep,label=\textup{(\alph*)}]
\item $\mathcal{B}$ and $\mathcal{A}$ alone (without $\mathcal{P}$)
      cannot distinguish admissibility violations from nonzero defect.
\item $\mathcal{P}$ and $\mathcal{A}$ alone (without $\mathcal{B}$)
      cannot certify $\Jcost = 0$ from $\|\mathbf{y}'\| = 0$.
\item $\mathcal{P}$ and $\mathcal{B}$ alone (without $\mathcal{A}$)
      cannot certify global properties from finite data.
\end{enumerate}
\end{theorem}

\begin{proof}
\textbf{(a)}  Consider $\mathbf{y} = (1, 1, \ldots, 1)$ (all
components equal to $1$).  Then $\sigma = n \neq 0$ (inadmissible)
but $\Jcost(\mathbf{x}) > 0$.  Without $\mathcal{P}$, we cannot
tell whether the nonzero cost is due to conservation violation or
genuine defect.  In contrast, $\mathbf{y} = (2, -2, 0, \ldots, 0)$
has $\sigma = 0$ (admissible) and $\Jcost > 0$ (genuine defect).
The two cases have the same cost profile to $\mathcal{B}$ but
different certification outcomes.

\textbf{(b)}  Consider the implication
``$\|\mathbf{y}'\| = 0 \implies \Jcost(\mathbf{x}') = 0$.''
This uses $\phi(0) = 0$, which is the base case of the coercivity
inequality.  Without $\mathcal{B}$ (i.e., without knowing that
$\phi$ is non-negative with unique zero at $0$), the projection
$\mathcal{P}$ could return $\mathbf{0}$ for a configuration that
nevertheless has nonzero cost under a different functional.  The
link $\|\cdot\| = 0 \iff \phi = 0$ is the content of $\mathcal{B}$.

\textbf{(c)}  By Proposition~\ref{prop:fail}, finite evaluations of
$\mathbf{y}$ cannot determine $\mathbf{y}$ globally without the
rational-class restriction.  Aggregation $\mathcal{A}$ supplies this
restriction (via Theorem~\ref{thm:rational}).  Without it,
$\mathcal{P}$ and $\mathcal{B}$ can certify only the sampled points,
not the global defect.
\end{proof}

%=============================================================================
\section{Optimality of Constants}
\label{sec:optimality}
%=============================================================================

\begin{theorem}[Optimal coercivity constant]\label{thm:cmin-optimal}
Among all cost functionals $F : \Rp \to \R_{\geq 0}$ satisfying
axioms \ref{A:cost}, the coercivity constant
\[
  c(F) := \inf_{t \neq 0} \frac{F(e^t)}{t^2/2}
\]
satisfies $c(F) = 1$.  That is, $\cmin = 1/2$ is the largest
(best) constant possible: $F(e^t) \geq t^2/2$ for all $t$, and
this is tight.
\end{theorem}

\begin{proof}
By Theorem~\ref{thm:J}, $F = \Jcost$ is the unique solution.
We have $\phi(t)/\frac{t^2}{2} = \frac{\cosh(t)-1}{t^2/2}$.
As $t \to 0$, this ratio $\to 1$ (by L'H\^{o}pital or Taylor).
For $t \neq 0$, $\cosh(t) - 1 > t^2/2$ strictly.
Therefore $\inf_{t \neq 0}\phi(t)/(t^2/2) = 1$, achieved in the
limit $t \to 0$.  Hence $c(\Jcost) = 1$ and $\cmin = 1/2$.

Since $\Jcost$ is unique, there is no other $F$ to compare.
The constant cannot be improved.
\end{proof}

\begin{theorem}[Optimal projection constant]\label{thm:Cproj-optimal}
Among all corrections $\mathbf{r}$ that restore neutrality
($\mathbf{x}\odot\mathbf{r}\in\mathcal{M}$), the $\Jcost$-minimising
correction is the orthogonal projection (Theorem~\ref{thm:P}) with
Lipschitz constant $\Cproj = 1$.  No correction can have
$\Cproj < 1$.
\end{theorem}

\begin{proof}
$\Cproj = 1$ because orthogonal projection is nonexpansive
(Lemma~\ref{lem:P-constants}).  No linear map onto a proper
subspace can have Lipschitz constant strictly less than $1$ (the
orthogonal projection achieves the minimum).
\end{proof}

\begin{corollary}[The constants are intrinsic to reality]
$\cmin = 1/2$ is fixed by the calibration $\Jcost''(1) = 1$, which
is fixed by the composition law.  $\Cproj = 1$ is fixed by
reciprocal symmetry $\Jcost(x) = \Jcost(x^{-1})$.  Both follow from
axiom~\ref{A:cost} alone.  The certification template and its
constants are as inevitable as the cost functional.
\end{corollary}

%=============================================================================
\section{Completeness: CPM Decides the Rational Class}
\label{sec:completeness}
%=============================================================================

\begin{theorem}[Decision procedure]\label{thm:decision}
For $\mathbf{x}$ in the rational class of known degree $d$
with $n \geq 8(2d+1)$, $\Phi^*$ terminates with output
$\texttt{ZERO}$ or $\texttt{NONZERO}$; it never returns
$\texttt{INCONCLUSIVE}$.
\end{theorem}

\begin{proof}
By Theorem~\ref{thm:rational}, $\mathbf{y}$ is determined by
$2d+1$ values.  With $n/8 \geq 2d+1$ window sums available,
$\mathcal{A}$ reconstructs $\mathbf{y}$ exactly.  Then
$\mathcal{B}$ computes $\Jcost(\mathbf{x})$ exactly, and the
decision is $\Jcost = 0$ or $> 0$.
\end{proof}

\begin{remark}[Scope boundary]
Outside the rational class, $\Phi^*$ may return
$\texttt{INCONCLUSIVE}$.  This is not a deficiency: by
Proposition~\ref{prop:fail}, no finite-data procedure can decide
membership for arbitrary analytic functions.  The rational-class
boundary is sharp.
\end{remark}

%=============================================================================
\section{Constants from the Forced Geometry}
\label{sec:geometry}
%=============================================================================

In the full Recognition Science framework, the window length $W = 8$
is not a choice: it is the minimal covering cycle for a $D$-cube walk
with $D = 3$ (forced by linking + gap-45 synchronisation).  The
complete constant table:

\begin{center}
\renewcommand{\arraystretch}{1.3}
\begin{tabular}{@{}llll@{}}
\toprule
\textbf{Constant} & \textbf{Value} & \textbf{Origin} & \textbf{Optimal?} \\
\midrule
$\cmin$ & $1/2$ & $\phi''(0) = 1$ (calibration \ref{A:cost})
  & Yes (Thm~\ref{thm:cmin-optimal}) \\
$\Cproj$ & $1$ & Orthogonal proj.\ (reciprocity \ref{P:even})
  & Yes (Thm~\ref{thm:Cproj-optimal}) \\
$\Knet$ & $1$ & Single covering window & --- \\
$W$ & $8$ & $2^D = 2^3$ (minimal cover) & Forced \\
$L$ & $\frac{1}{1+\lambda}$ & $\phi$ is $1$-strongly convex (\ref{P:strong})
  & Forced \\
\bottomrule
\end{tabular}
\end{center}

No entry in this table is adjustable.

%=============================================================================
\section{Discussion}
%=============================================================================

\subsection{CPM is not a method}

The word ``method'' implies choice.  The Master
Theorem~\ref{thm:master} proves there is no choice: the template, the
order of steps, and the constants are all forced by the axioms.  A
more accurate name would be the \emph{Coercive Projection Theorem}
(CPT) or simply the \emph{Certification Theorem of Canonical Cost}.

\subsection{Relationship to the Recognition Stability Audit}

The RSA paper proves an \emph{impossibility} certificate: if a
candidate forces a sensor pole, the Cayley--Schur pinch excludes it.
CPM proves a \emph{membership} certificate: if all window tests pass,
the configuration is in $S$.  Together:
\[
\text{CPM (membership)} \;+\; \text{RSA (exclusion)}
\;=\; \text{complete two-sided audit}.
\]
Both derive from $\Jcost$'s strict convexity.  They are the existence
and impossibility faces of the same coin.

\subsection{The engineering boundary}

Everything in this paper is foundation.  What remains for any specific
application is engineering: identify the structured set $S$, supply the
domain-specific finite-state model, and run $\Phi^*$.  The template
requires no domain input.

%=============================================================================
\section{Conclusions}
%=============================================================================

\begin{enumerate}[nosep]
\item $\Jcost(x) = \frac{1}{2}(x+x^{-1})-1$ is uniquely forced by the
      composition law, normalization, and calibration.
\item The certification procedure
      $\Phi^* = \mathcal{A}\circ\mathcal{B}\circ\mathcal{P}$ is the
      \textbf{unique optimal} procedure for deciding $\Jcost = 0$ from
      finite data (Master Theorem~\ref{thm:master}).
\item The factorisation is \textbf{forced}: no reordering of the three
      steps is sound.
\item The three steps are \textbf{independent}: none is derivable from
      the other two (Theorem~\ref{thm:independence}).
\item The constants $\cmin = 1/2$ and $\Cproj = 1$ are \textbf{optimal}:
      no cost functional satisfying the axioms achieves better
      (Theorems~\ref{thm:cmin-optimal}--\ref{thm:Cproj-optimal}).
\item $\Phi^*$ is \textbf{complete} on the rational class and
      \textbf{sharp} at its boundary: outside the rational class, no
      finite-data procedure can do better
      (Theorem~\ref{thm:decision} + Proposition~\ref{prop:fail}).
\end{enumerate}

The Coercive Projection Method is not a method.  It is a theorem.

\begin{thebibliography}{99}
\bibitem{WashburnCost2026} J.~Washburn and M.~Zlatanovi\'{c},
  ``Uniqueness of the Canonical Reciprocal Cost,''
  arXiv:2602.05753v1, 2026.
\bibitem{Aczel1966} J.~Acz\'{e}l,
  \textit{Lectures on Functional Equations}, Academic Press, 1966.
\bibitem{Bauschke2011} H.~H.~Bauschke and P.~L.~Combettes,
  \textit{Convex Analysis and Monotone Operator Theory in Hilbert
  Spaces}, Springer, 2011.
\bibitem{BeckenbachBellman1961} E.~F.~Beckenbach and R.~Bellman,
  \textit{Inequalities}, Springer, 1961.
\bibitem{WashburnRSA2026} J.~Washburn,
  ``The Recognition Stability Audit,''
  RS preprint, 2026.
\bibitem{WashburnAxioms2025} J.~Washburn,
  ``The Algebra of Reality,''
  \textit{Axioms} \textbf{15}(2), 90 (2025).
\end{thebibliography}

\end{document}
