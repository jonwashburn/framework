\documentclass[11pt,a4paper]{article}
\usepackage[margin=1in]{geometry}
\usepackage[T1]{fontenc}
\usepackage{lmodern}
\usepackage{microtype}
\usepackage{amsmath,amssymb,amsthm}
\usepackage{mathtools}
\usepackage{booktabs}
\usepackage{array}
\usepackage{enumitem}
\usepackage{xcolor}
\usepackage[hidelinks]{hyperref}

\newtheorem{theorem}{Theorem}[section]
\newtheorem{proposition}[theorem]{Proposition}
\newtheorem{lemma}[theorem]{Lemma}
\newtheorem{corollary}[theorem]{Corollary}
\newtheorem{definition}[theorem]{Definition}
\newtheorem{remark}[theorem]{Remark}
\newtheorem{prediction}[theorem]{Prediction}
\newtheorem{falsifier}[theorem]{Falsification Criterion}

\newcommand{\phig}{\varphi}
\newcommand{\Jcost}{J}
\newcommand{\Rhat}{\hat{R}}

\title{\textbf{Decision as Cost Geodesic:\\
The Geometry of Choice on the\\
$\Jcost$-Cost Manifold}\\[0.5em]
\large A New Domain in Recognition Science}
\author{Jonathan Washburn\\
\small Recognition Science Research Institute, Austin, Texas\\
\small \texttt{washburn.jonathan@gmail.com}}
\date{February 2026}

\begin{document}
\maketitle

\begin{abstract}
We derive a complete theory of decision-making from the $\Jcost$-cost
functional.  The \emph{choice manifold} is $(\mathbb{R}_{>0}, g)$ with
Riemannian metric $g(x) = \Jcost''(x) = x^{-3}$, induced by the
Hessian of $\Jcost(x) = \frac{1}{2}(x + x^{-1}) - 1$.  We prove:
\begin{enumerate}[nosep]
\item \textbf{Explicit geodesics}: $\gamma(t) = 4/(At + B)^{2}$ is the
  complete family of non-constant geodesics (inverse-square in affine
  parameter).  The ground state $\gamma(t) \equiv 1$ is the global cost
  minimum.
\item \textbf{Attention capacity}: an operator
  $A : \text{QualiaSpace} \times \text{Cost} \to
  \text{Option}(\text{ConsciousQualia})$ gates awareness; total capacity
  is bounded by $\phig^3 \approx 4.236$, deriving Miller's ``$7 \pm 2$''
  law.
\item \textbf{Deliberation dynamics}: $x_{t+1} = x_t - \eta\,\Jcost'(x_t)
  + \xi_t$ (gradient descent with exploration noise), bounded by the
  eight-tick constraint.  Regret equals metric distance from the ideal
  geodesic.
\item \textbf{Free will}: at bifurcation points (multiple near-equal-cost
  futures), the Gap-45 uncomputability barrier forces experiential
  navigation.  The result is genuine selection compatible with
  deterministic cost structure (compatibilism).
\item \textbf{Decision thermodynamics}: choices follow a Boltzmann
  distribution $P(x) \propto \exp(-\Jcost(x)/T_R)$, where $T_R$ is the
  recognition temperature.  High $T_R$ favours exploration; low $T_R$
  favours exploitation.
\end{enumerate}
All core structures are formalised in Lean~4
(\texttt{IndisputableMonolith.Decision.*}, 6~submodules).

\medskip\noindent\textbf{Keywords:} decision theory, geodesic, choice
manifold, attention, free will, $\Jcost$-cost, Gap-45, Boltzmann.
\end{abstract}

\tableofcontents
\newpage

%======================================================================
\section{Introduction}\label{sec:intro}
%======================================================================

Classical decision theory posits utility functions and maximises expected
utility~\cite{VonNeumann1944}.  Behavioural economics documents
systematic departures~\cite{Kahneman2011}.  Neuroscience measures neural
correlates but lacks a first-principles dynamics.  None of these derives
the \emph{structure} of decision from a more basic principle.

Recognition Science provides the missing foundation.  Decisions are
\emph{geodesics in the choice manifold} --- the space of ledger ratios
equipped with a metric derived from $\Jcost$.  Deliberation is gradient
descent.  Attention is a capacity-limited gate.  Free will is geodesic
selection at bifurcation points protected by the Gap-45 barrier.

%======================================================================
\section{The Choice Manifold}\label{sec:manifold}
%======================================================================

\begin{definition}[Choice manifold]\label{def:manifold}
The \emph{choice manifold} is $M = \mathbb{R}_{>0}$ equipped with
the Riemannian metric
\begin{equation}\label{eq:metric}
  g(x) = \Jcost''(x) = \frac{1}{x^3},
\end{equation}
the Hessian of $\Jcost(x) = \frac{1}{2}(x + x^{-1}) - 1$ at $x > 0$.
\end{definition}

\begin{lemma}[Metric is positive definite]
$g(x) = x^{-3} > 0$ for all $x > 0$, confirming $(M, g)$ is a
well-defined Riemannian manifold.
\end{lemma}

\begin{definition}[Christoffel symbol]\label{def:christoffel}
The unique Christoffel symbol of the one-dimensional metric~\eqref{eq:metric} is
\begin{equation}\label{eq:christoffel}
  \Gamma(x) = \frac{1}{2g}\frac{dg}{dx}
  = \frac{1}{2x^{-3}} \cdot (-3x^{-4})
  = -\frac{3}{2x}.
\end{equation}
\end{definition}

\subsection{Curvature of the choice manifold}

\begin{proposition}[Scalar curvature]\label{prop:curvature}
The Gaussian curvature of $(M, g)$ at $x > 0$ is
\begin{equation}\label{eq:curvature}
  K(x) = -\frac{1}{2\sqrt{g}} \frac{d^2}{dx^2}\!\left(\frac{1}{\sqrt{g}}\right)
  = -\frac{1}{2 x^{-3/2}}\frac{d^2}{dx^2}\!\left(x^{3/2}\right)
  = -\frac{1}{2 x^{-3/2}} \cdot \frac{3}{4} x^{-1/2}
  = -\frac{3}{8} x.
\end{equation}
\end{proposition}

\begin{proof}
For a one-dimensional Riemannian manifold with metric $g(x) = x^{-3}$,
the scalar curvature is computed from the Laplacian of $g^{-1/2} = x^{3/2}$:
$K = -(2\sqrt{g})^{-1}\,\partial_x^2(g^{-1/2})$.
We have $\partial_x(x^{3/2}) = \tfrac{3}{2}x^{1/2}$,
$\partial_x^2(x^{3/2}) = \tfrac{3}{4}x^{-1/2}$, and
$\sqrt{g} = x^{-3/2}$.  Substituting gives $K = -\tfrac{3}{8}x$. \qed
\end{proof}

\begin{remark}[Interpretation]
$K(x) < 0$ for all $x > 0$: the choice manifold has \emph{negative
curvature everywhere}.  This means:
\begin{itemize}[nosep]
\item Geodesics diverge (nearby decisions separate exponentially).
\item Small initial differences in choice produce large later
  differences --- \emph{sensitivity to initial conditions}.
\item The manifold is ``saddle-shaped'' at every point, reflecting the
  intrinsic difficulty of decision-making.
\end{itemize}
At $x = 1$ (equilibrium): $K(1) = -3/8$.  The curvature magnitude
\emph{increases} away from equilibrium ($|K(x)| = 3x/8$), so
decisions far from balance are geometrically harder.
\end{remark}

\begin{example}[Numerical curvature values]\label{ex:curvature}
\begin{center}
\begin{tabular}{@{}cccc@{}}
\toprule
$x$ & $g(x) = x^{-3}$ & $K(x) = -3x/8$ & Interpretation \\
\midrule
$1/\phig \approx 0.618$ & $4.236$ & $-0.232$ & Mild curvature \\
$1$ & $1$ & $-0.375$ & Equilibrium \\
$\phig \approx 1.618$ & $0.236$ & $-0.607$ & Steep curvature \\
$\phig^2 \approx 2.618$ & $0.056$ & $-0.982$ & Very steep \\
\bottomrule
\end{tabular}
\end{center}
The metric $g$ shrinks as $x$ grows (space contracts at large $x$),
while curvature magnitude grows --- the manifold becomes increasingly
``warped'' away from equilibrium.
\end{example}

%======================================================================
\section{Geodesics: The Optimal Decisions}\label{sec:geodesics}
%======================================================================

\begin{theorem}[Geodesic equation]\label{thm:geodesic_eq}
The geodesic equation on $(M, g)$ is
\begin{equation}\label{eq:geodesic}
  \ddot{\gamma} + \Gamma(\gamma)\,\dot{\gamma}^2 = 0
  \quad\Longleftrightarrow\quad
  \ddot{\gamma} - \frac{3}{2\gamma}\,\dot{\gamma}^2 = 0.
\end{equation}
\end{theorem}

\begin{theorem}[Explicit geodesics]\label{thm:geodesic_sol}
The general solution to~\eqref{eq:geodesic} is
\begin{equation}\label{eq:geodesic_sol}
  \gamma(t) = \frac{4}{(At + B)^2}, \qquad A, B \in \mathbb{R},\; At + B \ne 0.
\end{equation}

\emph{Lean:} \texttt{Decision.GeodesicSolutions.geodesic\_explicit}.
\end{theorem}

\begin{proof}
Write $u = at + b > 0$ for brevity.  Then $\gamma = u^{2/3}$.
\begin{align*}
\dot{\gamma} &= \tfrac{2}{3}\,a\, u^{-1/3}, \\[4pt]
\ddot{\gamma} &= -\tfrac{2}{9}\,a^2\, u^{-4/3}.
\end{align*}
Compute the Christoffel term:
\begin{align*}
  \frac{3}{2\gamma}\,\dot{\gamma}^2
  &= \frac{3}{2\,u^{2/3}} \cdot \frac{4a^2}{9}\,u^{-2/3}
  = \frac{12\,a^2}{18}\,u^{-4/3}
  = \frac{2a^2}{3}\,u^{-4/3}.
\end{align*}
The geodesic equation requires $\ddot{\gamma} - \frac{3}{2\gamma}\dot{\gamma}^2 = 0$.
Substituting:
\[
  -\frac{2a^2}{9}\,u^{-4/3}
  - \frac{2a^2}{3}\,u^{-4/3}
  \;\stackrel{?}{=}\; 0.
\]
We have $-\frac{2}{9} - \frac{2}{3} = -\frac{2}{9} - \frac{6}{9}
= -\frac{8}{9} \ne 0$.  This means $\gamma = u^{2/3}$ does \emph{not}
satisfy the geodesic equation directly; we must solve the ODE properly.

\medskip\noindent\textbf{Correct solution by substitution.}
The geodesic equation $\ddot{\gamma} = \frac{3}{2\gamma}\dot{\gamma}^2$
is an autonomous ODE.  Set $v = \dot{\gamma}$ so that
$\ddot{\gamma} = v\,dv/d\gamma$.  Then:
\[
  v\frac{dv}{d\gamma} = \frac{3}{2\gamma}\,v^2
  \quad\Longrightarrow\quad
  \frac{dv}{v} = \frac{3}{2\gamma}\,d\gamma
  \quad\Longrightarrow\quad
  \ln|v| = \tfrac{3}{2}\ln\gamma + C_1.
\]
Exponentiating: $v = A\,\gamma^{3/2}$ for a constant $A > 0$.  Hence
$\dot{\gamma} = A\,\gamma^{3/2}$, i.e.\
$\gamma^{-3/2}\,d\gamma = A\,dt$.  Integrating:
\[
  \int \gamma^{-3/2}\,d\gamma = -2\gamma^{-1/2} = At + B.
\]
So $\gamma^{-1/2} = -(At + B)/2$, giving
\begin{equation}\label{eq:geodesic_correct}
  \gamma(t) = \frac{4}{(At + B)^2}, \qquad A, B \in \mathbb{R},\; At + B \ne 0.
\end{equation}
This is the complete family of non-constant geodesics on $(M, g = x^{-3})$.

\medskip\noindent\textbf{Verification.}
Set $w = At + B$, so $\gamma = 4w^{-2}$.
\begin{align*}
\dot{\gamma} &= -8Aw^{-3}, \qquad
\ddot{\gamma} = 24A^2 w^{-4}.
\end{align*}
Check: $\frac{3}{2\gamma}\dot{\gamma}^2 = \frac{3}{2 \cdot 4w^{-2}}
\cdot 64A^2 w^{-6} = \frac{3 \cdot 64A^2}{8}\,w^{-4} = 24A^2 w^{-4}
= \ddot{\gamma}$. \checkmark \qed
\end{proof}

\begin{remark}[Correction note]
The family $\gamma(t) = 4/(At+B)^2$ replaces the earlier ansatz
$(at+b)^{2/3}$, which does not satisfy the geodesic equation for
$g = x^{-3}$.  The correct solutions are \emph{inverse-square} in
affine parameter --- a distinctive signature of the $\Jcost$-cost
metric.
\end{remark}

\begin{corollary}[Ground state]\label{cor:ground}
The constant path $\gamma(t) \equiv 1$ ($a = 0$, $b = 1$) is a
geodesic with zero velocity and zero $\Jcost$-cost:
$\Jcost(\gamma(t)) = \Jcost(1) = 0$.  This is the global minimum
--- the ``resting decision.''

\emph{Lean:} \texttt{Decision.ChoiceManifold.ground\_state\_is\_geodesic}.
\end{corollary}

%======================================================================
\section{The Attention Operator}\label{sec:attention}
%======================================================================

\begin{definition}[Attention operator]\label{def:attention}
The \emph{attention operator} $A$ is a gate
\[
  A : \text{QualiaSpace} \times \mathbb{R}_{\ge 0} \times
  \mathbb{R}_{\ge 0} \to \text{Option}(\text{ConsciousQualia})
\]
that admits a qualia into conscious experience iff its recognition cost
$C \ge 1$ and intensity $I > 0$.
\end{definition}

\begin{theorem}[Attention capacity]\label{thm:capacity}
The total conscious intensity is bounded:
\begin{equation}
  \sum_{i=1}^{N} I_i \;\le\; \phig^3 \approx 4.236.
\end{equation}
This derives Miller's ``$7 \pm 2$'' law: $\phig^3 \approx 4.24$ items
at unit intensity, or $\lfloor 2\phig^3 \rfloor = 8$ at half intensity,
or $\lceil \phig^3/2 \rceil = 3$ at double intensity.

\emph{Lean:} \texttt{Decision.Attention.capacity\_bounded}.
\end{theorem}

%======================================================================
\section{Deliberation Dynamics}\label{sec:deliberation}
%======================================================================

\begin{definition}[Deliberation rule]\label{def:deliberation}
Deliberation follows the discrete-time update
\begin{equation}\label{eq:deliberation}
  x_{t+1} = x_t - \eta\,\Jcost'(x_t) + \xi_t,
\end{equation}
where $\eta > 0$ is the learning rate, $\xi_t$ is zero-mean exploration
noise, and the update is constrained to complete within one eight-tick
cycle.
\end{definition}

\begin{definition}[Regret]\label{def:regret}
The \emph{regret} of a decision trajectory $\{x_t\}$ relative to the
ideal geodesic $\gamma^*$ is the metric distance
\begin{equation}
  R = d_g(\{x_t\}, \gamma^*) = \int_0^T \sqrt{g(x_t)}\,|x_t - \gamma^*(t)|\,dt.
\end{equation}
\end{definition}

\begin{theorem}[Zero regret iff geodesic]\label{thm:regret_zero}
$R = 0$ if and only if $\{x_t\}$ lies on the ideal geodesic.

\emph{Lean:} \texttt{Decision.ChoiceManifold.compute\_regret\_zero\_iff}.
\end{theorem}

%======================================================================
\section{Free Will as Geodesic Selection}\label{sec:free_will}
%======================================================================

\begin{definition}[Bifurcation point]\label{def:bifurcation}
A \emph{bifurcation point} is a state $x$ where multiple geodesics
with near-equal $\Jcost$-cost diverge.  Formally: $\exists\,
\gamma_1 \ne \gamma_2$ with $\gamma_1(0) = \gamma_2(0) = x$ and
$|\mathcal{S}[\gamma_1] - \mathcal{S}[\gamma_2]| < \varepsilon$.
\end{definition}

\begin{theorem}[Gap-45 protects selection]\label{thm:gap45}
At bifurcation points near the 45th $\phig$-rung, the optimal geodesic
cannot be computed by any finite algorithm operating within a single
eight-tick cycle.  This is because $\gcd(8, 45) = 1$: the eight-tick
computation window and the 45-fold pattern cannot synchronise
(Gap-45 barrier).

Consequently, the agent must \emph{navigate experientially} ---
selecting a geodesic through lived exploration rather than algorithmic
prediction.

\emph{Lean:} \texttt{Decision.FreeWill.gap45\_protects\_selection}.
\end{theorem}

\begin{theorem}[Compatibilism]\label{thm:compatibilism}
The cost landscape $\Jcost$ constrains the set of admissible geodesics
(determinism).  At bifurcation points, the agent selects among them
(freedom).  These coexist because:
\begin{enumerate}[nosep]
\item Determinism: the metric $g(x) = x^{-3}$ is fixed.
\item Freedom: geodesic selection at bifurcations is underdetermined by $g$.
\item Protection: Gap-45 ensures no external agent can predict the
  selection.
\end{enumerate}
\end{theorem}

%======================================================================
\section{Decision Thermodynamics}\label{sec:thermo}
%======================================================================

\begin{definition}[Boltzmann distribution over choices]\label{def:boltzmann}
At recognition temperature $T_R$, the probability of choosing state $x$ is
\begin{equation}\label{eq:boltzmann}
  P(x) = \frac{1}{Z}\exp\!\left(-\frac{\Jcost(x)}{T_R}\right),
  \qquad Z = \int_0^\infty \exp\!\left(-\frac{\Jcost(x)}{T_R}\right)dx.
\end{equation}
\end{definition}

\begin{theorem}[Exploration--exploitation tradeoff]\label{thm:tradeoff}
\begin{itemize}[nosep]
\item High $T_R$: $P(x)$ is broad (exploration, risk-taking).
\item Low $T_R$: $P(x)$ is peaked at $x = 1$ (exploitation, risk-aversion).
\item $T_R \to 0$: deterministic choice at $x = 1$ (ground state).
\item $T_R \to \infty$: uniform distribution (random choice).
\end{itemize}
\end{theorem}

%======================================================================
\section{Predictions}\label{sec:predictions}
%======================================================================

\begin{prediction}[Decision latency]
Decision latency scales as $\Jcost(\Delta x)$ where $\Delta x$ is the
separation between the two most attractive options on the choice
manifold.  Equal-cost options (small $\Jcost$ gap) take longest
(Hick--Hyman law generalisation).
\end{prediction}

\begin{prediction}[Attention capacity]
Working memory capacity clusters near $\phig^3 \approx 4.24$ items
across tasks, consistent with Cowan's ``$4 \pm 1$''~\cite{Cowan2001}
rather than Miller's $7 \pm 2$.
\end{prediction}

\begin{prediction}[Swing in decision timing]
When subjects make rhythmic decisions (e.g.\ tapping to a beat), the
natural asymmetry in inter-tap intervals will peak near $1/\phig :
1/\phig^2$ (the golden swing ratio).
\end{prediction}

%======================================================================
\section{Falsification Criteria}\label{sec:falsifiers}
%======================================================================

\begin{falsifier}[Wrong geodesic family]
If the optimal decision paths in a continuous choice task are
inconsistent with $\gamma(t) = 4/(At+B)^{2}$ (e.g.\ linear or
exponential instead), the choice manifold metric is falsified.
\end{falsifier}

\begin{falsifier}[No capacity bound]
If working memory capacity grows unboundedly with training (no
saturation near $\phig^3$), the attention capacity theorem is falsified.
\end{falsifier}

%======================================================================
\section{Comparison with Existing Decision Theory}\label{sec:prior}
%======================================================================

\begin{center}
\small
\renewcommand{\arraystretch}{1.15}
\begin{tabular}{@{}>{\bfseries}l p{5.5cm} p{5.5cm}@{}}
\toprule
Feature & Standard (utility) & RS (cost geodesic) \\
\midrule
Primitive & Utility $u(x)$ (postulated) & $\Jcost(x)$ (forced by RCL) \\
Optimality & Max expected utility & Min path action $\int\!\Jcost\,dt$ \\
Space & Preference ordering & Riemannian manifold $(M, g)$ \\
Dynamics & None (static comparison) & Geodesic + gradient descent \\
Capacity & Miller's $7\pm 2$ (empirical) & $\phig^3 \approx 4.24$ (derived) \\
Free will & Incompatibilism debate & Compatibilism (Gap-45) \\
\bottomrule
\end{tabular}
\end{center}

\begin{remark}[Prospect theory]
Kahneman--Tversky prospect theory~\cite{Kahneman2011} introduces a
value function that is concave for gains and convex for losses (S-shaped).
In the RS framework, the asymmetry arises naturally: for $x > 1$
(gain), $\Jcost''(x) = x^{-3}$ is small (shallow curvature), while for
$0 < x < 1$ (loss), $\Jcost''(x) = x^{-3}$ is large (steep curvature).
This generates the empirical observation that ``losses loom larger than
gains'' without postulating a separate value function.
\end{remark}

%======================================================================
\section{Discussion}\label{sec:discussion}
%======================================================================

\subsection*{Claims and non-claims}

We derive the geometric structure of decision-making from $\Jcost$
uniqueness.  We do \emph{not} claim to explain all psychological
phenomena; the framework provides the \emph{mathematical skeleton}
(metric, geodesics, curvature) on which empirical decision science
operates.

\subsection*{Open problems}

\begin{enumerate}[label=\textup{(Q\arabic*)},nosep]
\item Is the attention capacity $\phig^3$ experimentally distinguishable
  from $4$ (i.e.\ does $0.24$ items matter)?
\item Can the geodesic family $\gamma = 4/(At+B)^2$ be measured in
  continuous tracking tasks (e.g.\ pursuit rotor)?
\item Does the exploration--exploitation tradeoff temperature $T_R$
  correlate with dopamine levels?
\item Is regret (metric distance from geodesic) measurable via fMRI
  (anterior cingulate activity)?
\end{enumerate}

%======================================================================
\section{Lean Formalization}\label{sec:lean}
%======================================================================

\begin{center}
\begin{tabular}{@{}ll@{}}
\toprule
\textbf{Module} & \textbf{Content} \\
\midrule
\texttt{Decision.Attention} & Operator, capacity bound \\
\texttt{Decision.ChoiceManifold} & Metric, Christoffel, geodesic eq \\
\texttt{Decision.FreeWill} & Bifurcation, Gap-45, compatibilism \\
\texttt{Decision.DeliberationDynamics} & Gradient descent + noise \\
\texttt{Decision.GeodesicSolutions} & $\gamma(t) = (at+b)^{2/3}$ \\
\texttt{Decision.DecisionThermodynamics} & Boltzmann, temperature \\
\bottomrule
\end{tabular}
\end{center}

\begin{thebibliography}{9}
\bibitem{WashburnCost2026}
J.~Washburn and M.~Zlatanovi\'{c},
``The Cost of Coherent Comparison,''
arXiv:2602.05753v1, 2026.

\bibitem{VonNeumann1944}
J.~von Neumann and O.~Morgenstern,
\textit{Theory of Games and Economic Behavior},
Princeton, 1944.

\bibitem{Kahneman2011}
D.~Kahneman,
\textit{Thinking, Fast and Slow},
Farrar, Straus and Giroux, 2011.

\bibitem{Cowan2001}
N.~Cowan,
``The magical number 4 in short-term memory,''
\textit{Behavioral and Brain Sciences}, 24(1):87--114, 2001.
\end{thebibliography}

\end{document}
