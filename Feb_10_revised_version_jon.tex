\documentclass[11pt]{article}

\usepackage[margin=1in]{geometry}
\usepackage[T1]{fontenc}
\usepackage[utf8]{inputenc}
\usepackage{lmodern}
\usepackage{microtype}
\usepackage{amsmath,amssymb,amsthm,mathtools}
\usepackage[colorlinks=true,linkcolor=blue,citecolor=blue,urlcolor=blue]{hyperref}
%\usepackage[nameinlink]{cleveref}
\usepackage{booktabs}
\usepackage{enumitem}
\setlist{nosep}
\usepackage{xcolor}

% ---- Colored-change helpers (teal = additions, red = deletions) ----
\newcommand{\ADD}[1]{\textcolor{teal}{#1}}
\newcommand{\DEL}[1]{\textcolor{red}{\sout{#1}}}
\usepackage[normalem]{ulem}  % for \sout

% Theorem environments
\newtheorem{theorem}{Theorem}[section]
\newtheorem{lemma}[theorem]{Lemma}
\newtheorem{proposition}[theorem]{Proposition}
\newtheorem{corollary}[theorem]{Corollary}
\newtheorem{definition}[theorem]{Definition}
\newtheorem{remark}[theorem]{Remark}
\newtheorem{example}[theorem]{Example}
\newtheorem{axiom}[theorem]{Axiom}

% Notation
\newcommand{\C}{\mathcal{C}}
\newcommand{\E}{\mathcal{E}}
\newcommand{\CR}{\mathcal{C}_R}
\newcommand{\M}{\mathcal{M}}
\newcommand{\Z}{\mathbb{Z}}
\newcommand{\R}{\mathbb{R}}
\newcommand{\N}{\mathbb{N}}
\newcommand{\Q}{\mathbb{Q}}
\newcommand{\lcmop}{\operatorname{lcm}}
\newcommand{\gcdop}{\gcd}
\newcommand{\lk}{\operatorname{lk}}
\newcommand{\dimop}{\dim}

\title{Dimensional Rigidity as a Selection Principle in Recognition Geometry}
\author{
  \ADD{Sebastian Pardo-Guerra\thanks{Recognition Physics Institute, Austin, TX, USA. \texttt{sebas@recognitionphysics.org}},} \\
  \ADD{Anil Thapa\thanks{Recognition Physics Institute, Austin, TX, USA. \texttt{athapa@recognitionphysics.org}},} \\
  \ADD{Jonathan Washburn\thanks{Recognition Physics Institute, Austin, TX, USA. \texttt{jon@recognitionphysics.org}}}
}
\date{\today}

\begin{document}

\maketitle

\begin{abstract}
In this work, we address the question: \emph{why is physical space three--dimensional?} Rather than assuming spatial dimensionality \emph{a priori}, we show that \(D=3\) is uniquely selected when observable space is constructed from measurement processes themselves. Within this
perspective, dimensionality emerges as a consequence of operational and structural constraints, rather than as a primitive input.

We work within the framework of \emph{Recognition Geometry} (RG), in which observable space \(\CR\) arises as a recognition quotient \(\C/\!\sim_R\) of configurations \(\C\) under measurement--induced equivalence relations, without reference to a pre--existing ambient
geometry. In this setting, we identify three genuinely independent constraints---topological, dynamical, and geometric---each of which permits multiple candidate dimensions when considered in isolation.
Specifically, we analyze: (A) same--dimension linking of extended objects, which requires odd spatial dimensions; (B) Green--kernel orbital stability, which excludes dimensions \(D\ge 4\); and (C) the existence of non--abelian rotation groups, which requires \(D\ge 3\).

While none of these constraints alone fixes the dimension uniquely, their simultaneous enforcement yields a single consistent solution. The intersection of the admissible sets,
\[
\mathcal{A}_A \cap \mathcal{A}_B \cap \mathcal{A}_C = \{3\},
\]
provides a convergent selection mechanism that isolates three
dimensions.

Our main theorem establishes that any recognition quotient satisfying these three independent constraints necessarily has \(\dimop(\CR)=3\). This result links measurement--first operational requirements---topological complexity, dynamical stability, and geometric richness---to the classical three--dimensional structure of observable physical space, offering a principled explanation for spatial
dimensionality rooted in recognition and measurement rather than assumption.
\end{abstract}

\noindent\textbf{Keywords:} Recognition Geometry, dimensional rigidity, linking number, Kepler dynamics, synchronization, selection principles

\noindent\textbf{MSC 2020:} 51A05, 57K10, 70F05, 05C45, 68V15

\section{Introduction}

The question of why physical space appears to have three spatial dimensions has intrigued mathematicians and physicists since antiquity. While empirical observation consistently confirms $D=3$, the deeper question remains: is three-dimensionality a contingent fact of our universe, or does it follow from fundamental structural constraints inherent in the very nature of observation? In this work, we address this question from a radically different starting point: rather than assuming an ambient space $\mathbb{R}^D$ and verifying which $D$ supports physical complexity, we construct observable space from measurement processes themselves and show that three independent operational constraints uniquely force $D=3$.

Our approach rests on \emph{Recognition Geometry} (RG) \cite{WashburnZlatanovicAllahyarov2026}, a measurement-first framework in which observable space $\mathcal{C}_R$ emerges as a \emph{recognition quotient} $\mathcal{C}/\!\sim_R$ of configurations under measurement-induced equivalence relations. Within this paradigm, we identify three genuinely independent constraints---topological, dynamical, and geometric---each admitting multiple candidate dimensions when considered in isolation. Their simultaneous enforcement, however, yields a unique solution:
\[
\mathcal{A}_A \cap \mathcal{A}_B \cap \mathcal{A}_C = \{3,5,7,\dots\} \cap \{2,3\} \cap \{3,4,5,\dots\} = \{3\},
\]
where $\mathcal{A}_A$ (same-dimension linking) forces odd dimensions, $\mathcal{A}_B$ (Green-kernel orbital stability) forces low dimensions, and $\mathcal{A}_C$ (non-abelian rotation groups) forces high dimensions. Only $D=3$ survives all three filters, providing a convergent selection mechanism that isolates three dimensions without fine-tuning or anthropic reasoning.

Before presenting our framework and results, we situate our contribution within the broader landscape of dimensional arguments in physics and mathematics, showing how the recognition-first paradigm offers a fundamentally new approach to this ancient question.

\ADD{This paper builds on the axiomatic characterization of ratio-induced mismatch costs established
in~\cite{WashburnRahnamaiBarghi2026}.  There it was shown that the assumptions of inversion symmetry,
strict convexity, coercivity, and a multiplicative d'Alembert compatibility identity uniquely
force $J(x)=\tfrac{1}{2}(x^{a}+x^{-a})-1$ for some $a>0$ (with $a$ absorbable into the scale maps).
We take this cost-kernel result as given and focus on the downstream topological, dynamical, and
geometric consequences that determine spatial dimension.}

\subsection{Prior Approaches to Dimensional Selection}

For over two millennia, the mathematical narrative has been dominated by what may be termed the \emph{space-first paradigm}. In this view, geometry begins with a set of points equipped with pre-existing structure—a smooth manifold $M$ with topology $T$, differential structure $A$, and metric tensor $g$. Objects are "located" in this space, and measurement is modeled as a function $f(x) \in \mathbb{R}$ assigning an observable value to a pre-existing state $x$. This continuum framework—from Euclidean geometry to the smooth 4-dimensional spacetime of General Relativity \cite{Lee2013, Wald1984}—treats spatial dimension as an input rather than an output. Even in Quantum Mechanics, the underlying Hilbert space remains a continuous structure built over complex numbers \cite{Riesz1990}, with dimension assumed rather than derived.

Within this space-first paradigm, multiple lines of research have identified special properties of $D=3$, each revealing constraints that prefer or single out three dimensions. These arguments, while illuminating, share a common limitation: they assume the existence of ambient space and verify compatibility, rather than deriving dimensionality from operational principles.

The earliest systematic analysis is due to \emph{Ehrenfest} \cite{Ehrenfest1917}, who argued that stable planetary orbits and atomic structures require $D=3$: in $D>3$, the inverse-power-law potential becomes too steep, causing orbital instability; in $D=2$, no inverse-square force law emerges from Gauss's law. \emph{Barrow and Tipler} \cite{BarrowTipler1986} surveyed anthropic constraints, noting that biological complexity (stable chemistry, information processing) appears to require $D=3$. \emph{Tegmark} \cite{Tegmark1997} systematically analyzed dimensions $D=1$ through $D=10$, concluding that only $D=3,4$ permit stable structures, with $D=3$ uniquely supporting both stable orbits and rich topology. These classical arguments ask: "For which $D$ do physical laws support complexity?" They demonstrate compatibility but do not explain why space \emph{has} dimension $D$ in the first place.

From pure mathematics, deep results reveal unique features of three-dimensional topology.
\ADD{Freedman~\cite{Freedman1982} classified simply connected closed topological 4-manifolds via their unimodular intersection forms.  Combined with Donaldson's smooth rigidity results (and later work), this implies that $\mathbb{R}^4$ admits uncountably many \emph{exotic} smooth structures, a phenomenon unique to dimension~4.}
\ADD{Classical knot theory of embeddings $S^1\hookrightarrow\mathbb{R}^3$ is special; in higher dimensions the behavior changes dramatically, though higher-dimensional knot theory (e.g.\ codimension-2 sphere knots) exists and is non-trivial.}
\ADD{In quantum field theory, anomaly cancellation in gauge theories imposes dimensional constraints; for instance, gravitational and gauge anomalies cancel in $D=10$ for the superstring, and analogous constraints appear in lower-dimensional models.}
These results establish special topological and algebraic properties of $D=3$ but do not explain why the \emph{observable universe} selects this dimension.

String theory takes a different approach \cite{Green1987, Polchinski1998}, postulating $D=10$ or $D=11$ spacetime dimensions, with $D-4$ dimensions "compactified" on a small manifold, leaving $3+1$ observable dimensions. While mathematically elegant, this framework assumes an ambient high-dimensional space and requires fine-tuning of compactification geometry (Calabi-Yau manifolds, moduli stabilization). It provides a mechanism by which extra dimensions could be hidden but not a derivation of why $D=3$ emerges as the fundamental observable structure.

In summary, prior work divides into three categories: \emph{(i)} compatibility arguments that verify which dimensions support physical complexity \cite{Ehrenfest1917, BarrowTipler1986, Tegmark1997}, assuming ambient space; \emph{(ii)} mathematical results showing special properties of $D=3$ \cite{Freedman1982}, without explaining observational selection; and \emph{(iii)} high-dimensional frameworks with compactification \cite{Green1987, Polchinski1998}, requiring fine-tuning. None derive dimensionality from operational measurement principles.

\subsection{The Recognition Geometry Paradigm: From Measurement to Space}

\emph{Recognition Geometry} \cite{WashburnZlatanovicAllahyarov2026} proposes a fundamental inversion of the traditional relationship between measurement and space. Rather than locating measurements within pre-existing geometry, RG posits that \textbf{recognition is primitive, and space is derived}. This measurement-first philosophy shares deep roots with operational approaches to quantum theory—from Von Neumann's measurement postulates \cite{vonNeumann1955} to Rovelli's relational interpretation \cite{Rovelli1996}, which suggests that states are not absolute but relative to observers—but formalizes these ideas into a complete geometric framework.

The starting point is a \emph{configuration space} $\mathcal{C}$ representing "what the world does" and a \emph{recognizer} $R: \mathcal{C} \to \mathcal{E}$ mapping configurations to observable events in an \emph{event space} $\mathcal{E}$, representing "what the observer sees." Crucially, $\mathcal{C}$ is not assumed to have any a priori topological or metric structure; instead, locality is introduced through a neighborhood system defined on configurations themselves, encoding which states are "nearby" in configuration dynamics rather than in spatial geometry.

Observable space emerges as the \emph{recognition quotient} $\mathcal{C}_R := \mathcal{C}/\!\sim_R$, where two configurations are equivalent ($c_1 \sim_R c_2$) if they are observationally indistinguishable ($R(c_1) = R(c_2)$). States in the quotient space are uniquely identified by their measurement outcomes, establishing that "observable reality" is precisely the information available to the recognizer—no more, no less. A central tenet of RG is the \emph{finite local resolution axiom} (RG3), which formalizes the fact that any physical observer can distinguish only finitely many outcomes in a local region. This ensures that the emergent geometry is necessarily discrete or granular at the fundamental level, smoothing into a manifold-like continuum only in the limit of high resolution.

Within this framework, dimension is not an input but an \emph{emergent property}: the recognition dimension $D$ is the number of independent coordinates required to locally parameterize the space of distinguishable events. The question "why $D=3$?" becomes: what operational requirements on $\mathcal{C}_R$ force this specific dimension? Our main result establishes that three genuinely independent constraints—topological, dynamical, and geometric—uniquely select $D=3$.

\subsection{Three Independent Selection Principles}

The transition from a space-first to a recognition-first paradigm requires a new way of understanding the "selection" of physical parameters. If dimension is not a given property of a container but an emergent property of a quotient, we must ask what constraints force our specific observable reality. We identify three complementary requirements that constrain dimension in genuinely independent ways, formalized through the concept of \emph{convergent independence}.

\begin{definition}[Allowed Dimension Sets]\label{def:allowed_sets}
For any criterion $(X)$, we define its \emph{allowed-dimension set}
\[
\mathcal{A}_X \;:=\;\{D\in \mathbb{N}: \text{criterion $(X)$ holds in dimension $D$}\}.
\]
A family of constraints $(X_1),\dots,(X_n)$ exhibits \emph{convergent independence} for selecting $D^\star$ if:
\begin{enumerate}
\item \textbf{Nontriviality:} Each $\mathcal{A}_{X_i}$ contains more than one candidate dimension,
\item \textbf{Convergence:} The intersection is a singleton: $\bigcap_{i=1}^n \mathcal{A}_{X_i} = \{D^\star\}$.
\end{enumerate}
\end{definition}

This pattern distinguishes genuine multi-constraint selection from single-constraint characterizations. A convergent-independent family carves out overlapping but non-coinciding regions of dimension space, with only one dimension surviving all filters. Our three constraints achieve this with $D^\star=3$.

\textbf{Constraint (A): Same-Dimension Topological Linking.} For extended objects (submanifolds) of intrinsic dimension $p$ to admit integer-valued linking invariants with other objects of the \emph{same} dimension $p$, standard homological bookkeeping requires $p+p=D-1$, forcing $D=2p+1$ to be odd. The familiar case of \emph{loop-loop linking} ($p=1$, circles in space) gives $D=3$, but the general constraint admits all odd dimensions $D\ge 3$:
\[
\mathcal{A}_A = \{3, 5, 7, 9, \dots\}\quad\text{(odd dimensions with }D\ge 3\text{)}.
\]
This topological requirement ensures that observable space can support entangled extended structures—field lines, flux tubes, polymer chains—with robust integer-valued topological charges stable under perturbations.

\textbf{Constraint (B): Green-Kernel Orbital Stability.} In Recognition Geometry, physical potentials emerge from information costs required to distinguish configurations. Under isotropy and scale-freeness, the natural dynamics is governed by Green-kernel potentials (logarithmic for $D=2$, power-law $V_D(r)\propto -r^{2-D}$ for $D\ge 3$). For near-circular orbits under such potentials to be \emph{stable} (small perturbations produce bounded oscillations, not runaway spiraling), the effective potential must satisfy $U''_{\text{eff}}(r_0)>0$ at the circular orbit radius $r_0$. Separate analyses for the logarithmic case ($D=2$) and power-law case ($D\ge 3$) both confirm stability, while $D=1$ is excluded (no angular momentum in one dimension) and $D\ge 4$ is unstable, yielding:
\[
\mathcal{A}_B = \{2, 3\}\quad\text{(low dimensions with stable orbits)}.
\]
This dynamical requirement ensures that recognition structures—atoms, planetary systems, bound vortex pairs—can persist as stable, repeating patterns. In $D\ge 4$, such bound states are perturbatively unstable and cannot exist.

\textbf{Constraint (C): Non-Abelian Rotation Group.} For the observable space to support genuinely independent rotations—non-commuting transformations encoding distinct orientational degrees of freedom—\ADD{the local orthonormal frame rotation group} $SO(D)$ must be non-abelian. It is a standard fact from Lie theory that $SO(1)$ is trivial and $SO(2)\cong S^1$ is abelian (all rotations commute), while $SO(D)$ is non-abelian for $D\ge 3$:
\[
\mathcal{A}_C = \{3, 4, 5, 6, \dots\}\quad\text{(higher dimensions)}.
\]
This geometric requirement excludes the degenerate low-dimensional cases $D=1,2$ where the geometry lacks sufficient "room" for complex rotational dynamics—gyroscopic precession, angular momentum coupling, spinor representations.

These three constraints address orthogonal aspects of physical reality: (A) is topological (dimension parity for entanglement), (B) is dynamical (upper bound for stability), and (C) is geometric (lower bound for complexity). Each admits multiple dimensions individually, but their conjunction is dramatically more restrictive.

We note that Section~\ref{sec:specializations} discusses sharper formulations: constraint (T) specializes (A) to loop-loop linking ($p=1$), directly giving $D=3$; constraint (K) strengthens (B) from stability to non-precession (Bertrand's theorem), directly giving $D=3$; and constraint (S) provides a computational efficiency consideration (minimizing synchronization overhead) that is a boundary optimization rather than an independent selector. The dimensional conclusion $D=3$ relies entirely on the genuinely independent constraints (A), (B), (C).

\subsection{Convergence to $D=3$: The Main Result}

The key observation is that these three allowed sets, each non-singleton, have a unique common element:
\[
\boxed{\mathcal{A}_A \cap \mathcal{A}_B \cap \mathcal{A}_C = \{3,5,7,\dots\} \cap \{2,3\} \cap \{3,4,5,\dots\} = \{3\}.}
\]
This intersection pattern provides a \emph{genuinely convergent} selection mechanism: no single constraint determines $D=3$, but their conjunction uniquely forces it. Constraint (A) eliminates all even dimensions, leaving $\{3,5,7,\dots\}$. Constraint (B) eliminates high dimensions ($D\ge 4$) and the one-dimensional case (no angular momentum), leaving only $\{2,3\}$. Constraint (C) eliminates the degenerate low-dimensional cases, requiring $D\ge 3$. Only $D=3$ survives all three filters.

Our main theorem formalizes this convergence within the Recognition Geometry framework:

\begin{theorem}[Dimensional Rigidity in Recognition Geometry]\label{thm:main}
Let $(\mathcal{C}, \mathcal{E}, R)$ be a recognition geometry with quotient $\mathcal{C}_R = \mathcal{C}/\sim_R$. Assume \ADD{$(\C,\E,R)$ admits an effective manifold model $\M$ in the sense of Definition~\ref{def:two_scale}} and admits enough structure for constraints (A), (B), and (C) to be formulated. If \ADD{$\M$} satisfies all three constraints, then \ADD{$\dimop(\M)=3$} (equivalently, the recognition dimension of $\mathcal{C}_R$ is~3).

Conversely, $D=3$ satisfies all three constraints under appropriate regularity hypotheses.
\end{theorem}

This result cleanly separates the \emph{foundational} axioms of Recognition Geometry—which describe how observable space emerges from measurement processes—from the \emph{selective} constraints—which determine which emergent spaces are physically viable. The rigidity of $D=3$ is thus not a contingent accident or an anthropic coincidence, but a mathematical necessity for any recognition-based world that supports topological entanglement, dynamical stability, and geometric richness.

\begin{remark}[Sharper Specializations]\label{rem:specializations}
Each of the three independent constraints (A), (B), (C) admits sharper formulations that directly force $D=3$:
\begin{itemize}
  \item \textbf{(T) Loop-loop linking.} Specialize (A) to $p=1$ (circles). Then $D=2(1)+1=3$ uniquely, yielding $\mathcal{A}_T=\{3\}$.
  \item \textbf{(K) Non-precessing orbits.} Strengthen (B) from stability to exact closure (apsidal angle $\Delta\theta=2\pi$). This requires the orbit equation to be linear in the Binet variable $u=1/r$, forcing $D=3$ uniquely, yielding $\mathcal{A}_K=\{3\}$.
  \item \textbf{(S) Computational consideration.} Given the lower bound $D\ge 3$ from constraint (C) and an external parameter $N$ (odd), the synchronization period $\mathrm{lcm}(2^D,N)=N\cdot 2^D$ is strictly increasing in $D$. Minimizing over $D\ge 3$ yields $D=3$ (the boundary value). This is a boundary optimization, not an independent selector; it applies for any odd $N$ and therefore does not constrain the theory.
\end{itemize}
Constraints (T) and (K) are \emph{characterizing} specializations that individually imply $D=3$ and provide additional physical insight (observable loop entanglement, stable atoms). Constraint (S) is a computational efficiency consideration that does not contribute to dimensional selection. None of (T), (K), (S) exhibit the convergent-independence pattern of Definition~\ref{def:allowed_sets}. The body of the paper focuses on the genuinely independent constraints (A), (B), (C), with (T), (K), (S) discussed as specializations in Section~\ref{sec:specializations}.
\end{remark}

\subsection{Contributions and Structure}

Our approach differs fundamentally from prior work in three key respects:

\textbf{(1) Ontological inversion.} We do not assume an ambient space $\mathbb{R}^D$ and verify which $D$ supports physical complexity. Instead, we construct observable space $\mathcal{C}_R$ as a recognition quotient from measurement processes and \emph{derive} $D=3$ as an emergent property. Dimension is not a container but a consequence of operational distinguishability.

\textbf{(2) Genuinely independent constraints.} Unlike classical arguments \cite{Ehrenfest1917, Tegmark1997} that focus primarily on orbital stability, or mathematical results that identify isolated special properties of $D=3$, we formulate three constraints addressing orthogonal aspects of physical reality. Only their intersection collapses to $\{3\}$, exhibiting the convergent-independence pattern of Definition~\ref{def:allowed_sets}.

\textbf{(3) Structural clarity.} Key results—linking dimension formula $D=2p+1$ (Theorem~\ref{thm:same_dim_linking}), stability condition $D<4$ (Theorem~\ref{thm:stability_power}), non-abelian bound $D\ge 3$ (Proposition~\ref{prop:nonabelian})—are stated in a modular way, checkable within standard mathematical frameworks independent of physical interpretation.

The remainder of this paper establishes these results rigorously. Section~2 reviews the foundational framework of Recognition Geometry. Sections~3--5 prove constraints (A), (B), (C) and compute their allowed dimension sets. Section~6 establishes the main dimensional rigidity theorem. Section~7 discusses sharper specializations (T), (K), (S). Section~8 concludes with discussion of open problems and extensions to quantum recognition geometries.

\section{Preliminaries: Recognition Geometry Foundations}

We now establish the formal framework underlying our dimensional analysis. This section summarizes the axiomatic structure of Recognition Geometry—configuration and event spaces, recognizers, quotients, locality, and finite resolution—providing the minimal apparatus needed to formulate constraints (A), (B), and (C). Proofs of stated results can be found in \cite{WashburnZlatanovicAllahyarov2026}; we include only the definitions and theorems necessary for our analysis. Readers familiar with the RG framework may skim to Section~3.

\subsection{Basic Structures}

Recognition Geometry rests on four primitives: a configuration space $\C$ (states the world can occupy), an event space $\E$ (observable outcomes), a recognizer $R:\C\to\E$ (measurement process), and an indistinguishability relation $\sim_R$ (equality of outcomes). The recognition quotient $\CR=\C/\!\sim_R$ represents observable reality.

\begin{definition}[Configuration and Event Spaces]
A \emph{configuration space} $\C$ is a nonempty set of states. An \emph{event space} $\E$ is a set of observable outcomes.
\end{definition}

\begin{definition}[Recognizer]
A \emph{recognizer} is a map $R: \C \to \E$ assigning an observable event to each configuration.
\end{definition}

\begin{definition}[Indistinguishability]
Configurations $c_1, c_2 \in \C$ are \emph{observationally indistinguishable} with respect to $R$, denoted $c_1 \sim_R c_2$, if $R(c_1) = R(c_2)$.
\end{definition}

The relation $\sim_R$ is an equivalence relation whose equivalence classes $[c]_R$ are called \emph{resolution cells}.

\begin{definition}[Recognition Quotient]
The \emph{recognition quotient} is the quotient space $\CR = \C / \sim_R$.
\end{definition}

\begin{theorem}[Injectivity of Observable Map {\cite{WashburnZlatanovicAllahyarov2026}}]
The induced map $\overline{R}: \CR \to \E$ defined by $\overline{R}([c]_R) = R(c)$ is injective.
\end{theorem}

\subsection{Locality and Finite Resolution}

Topology is not assumed on $\C$ but emerges through a neighborhood system. Crucially, the finite local resolution axiom (RG3) ensures that observers cannot distinguish infinitely many outcomes locally—a physical constraint with deep consequences for dimensional selection.

\begin{definition}[Neighborhood System]
A \emph{locality structure} on $\C$ assigns to each $c\in\C$ a nonempty collection $\mathcal{N}(c)\subseteq\mathcal{P}(\C)$ of \emph{neighborhoods} satisfying:
\begin{enumerate}
    \item[(i)] (Reflexivity) $c\in U$ for every $U\in\mathcal{N}(c)$;
    \item[(ii)] (Intersection closure) for all $U,V\in\mathcal{N}(c)$, $\exists W\in\mathcal{N}(c)$ with $W\subseteq U\cap V$;
    \item[(iii)] (Local refinement) for all $U\in\mathcal{N}(c)$ and $c'\in U$, $\exists V\in\mathcal{N}(c')$ with $V\subseteq U$.
\end{enumerate}
\end{definition}

\begin{axiom}[RG3: Finite Local Resolution {\cite{WashburnZlatanovicAllahyarov2026}}]
For every $c\in\C$ and recognizer $R:\C\to\E$, there exists $U\in\mathcal{N}(c)$ such that $|R(U)|<\infty$.
\end{axiom}

\smallskip
\noindent\textbf{Why RG3 is needed.}
RG3 formalizes an operational fact: any physical observer has bounded resources (finite time, energy, memory, and noise tolerance) and therefore can distinguish only finitely many outcomes within any sufficiently small neighborhood of configurations. Mathematically, this axiom rules out pathologies where arbitrarily fine local distinctions would be observable, and it ensures that recognition quotients have an intrinsic \emph{granularity} compatible with discrete registers and computational costs.

\ADD{\subsection{Imported Cost-Kernel Characterization}}

\ADD{\begin{proposition}[Unique Mismatch Penalty; Washburn--Rahnamai Barghi~\cite{WashburnRahnamaiBarghi2026}]\label{prop:cost_kernel}
Let $J:(0,\infty)\to[0,\infty)$ satisfy
\emph{(i)}~inversion symmetry $J(x)=J(1/x)$,
\emph{(ii)}~strict convexity,
\emph{(iii)}~normalization $J(1)=0$,
\emph{(iv)}~coercivity $J(x)\to\infty$ as $x\to 0^+$ or $x\to\infty$, and
\emph{(v)}~the multiplicative d'Alembert identity
$(1+J(xy))+(1+J(x/y))=2(1+J(x))(1+J(y))$.
Then there exists $a>0$ such that $J(x)=\cosh(a\log x)-1=\tfrac{1}{2}(x^{a}+x^{-a})-1$.
The parameter~$a$ is absorbed by rescaling $\iota_{S},\iota_{O}\mapsto\iota_{S}^{a},\iota_{O}^{a}$,
yielding the canonical form $J(x)=\tfrac{1}{2}(x+x^{-1})-1$ without loss of generality.
\end{proposition}}

\ADD{\noindent
This result provides the unique cost functional underlying the recognition
framework and the Green-kernel potentials analyzed in Section~\ref{sec:constraint_B}.
The novelty of the present work lies in the geometric and topological
consequences of this cost kernel, specifically the forcing of $D=3$
spatial dimensions via linking constraints, rather than in the derivation
of~$J$ itself.}

\subsection{Composite Recognizers and Symmetries}

\begin{definition}[Composite Recognizers]
Given recognizers $R_1: \C \to \E_1$ and $R_2: \C \to \E_2$, their \emph{composition} is $(R_1 \otimes R_2)(c) = (R_1(c), R_2(c))$.
\end{definition}

\begin{theorem}[Refinement {\cite{WashburnZlatanovicAllahyarov2026}}]\label{thm:refinement}
The quotient $\mathcal{C}_{R_1 \otimes R_2}$ refines $\mathcal{C}_{R_1}$ and $\mathcal{C}_{R_2}$, increasing distinguishing power.
\end{theorem}

\begin{definition}[Recognition Symmetries]\label{def:rec_symmetries}
A transformation $g: \C \to \C$ is a \emph{recognition symmetry} if $R(g(c)) = R(c)$ for all $c \in \C$. Configurations related by symmetries are \emph{gauge equivalent}.
\end{definition}

\subsection{Two-Scale Structure: Finite Resolution and Effective Manifolds}

The interplay between RG3 (finite local resolution) and smooth differential geometry requires careful treatment. Individual recognition quotients $\CR$ have intrinsic granularity; differential-geometric arguments apply to an \emph{effective manifold} $\M$ arising in the refinement limit.

\begin{lemma}[RG3 Implies Finite Local Equivalence Classes]\label{lem:rg3_finite_classes}
Fix $c\in\C$ and choose $U\in \mathcal{N}(c)$ with $|R(U)|<\infty$ (existence guaranteed by RG3).
Then the image $\pi_R(U)\subseteq \CR$ contains at most $|R(U)|$ equivalence classes, hence is finite.
\end{lemma}

\begin{proof}
Points of $\CR$ are equivalence classes $[c']_R$. On $U$, the function $R$ takes only finitely many values. Each equivalence class in $\pi_R(U)$ corresponds to a distinct value in $R(U)$, so $|\pi_R(U)|\le |R(U)|<\infty$.
\end{proof}

This lemma reveals that individual quotients $\mathcal{C}_R$ are necessarily \emph{granular} at the scale set by finite resolution. To perform classical differential geometry (Laplace operators, Green kernels, orbital mechanics) and algebraic topology (homology, linking), we work with an effective smooth limit.

\begin{definition}[Directed Refinement and Effective Manifold]\label{def:two_scale}
Let $R_1\preceq R_2\preceq\cdots$ be a directed system of recognizers with increasing distinguishing power (e.g., via composite recognizers $R_{i+1}=R_i\otimes S_i$ for auxiliary recognizers $S_i$).
Let $\mathcal{C}_{R_i}:=\C/\!\sim_{R_i}$ be the finite-resolution quotients.

An \emph{effective manifold model} is a smooth $D$-manifold $\M$ equipped with coarse-graining maps $\phi_i:\mathcal{C}_{R_i}\to \M$ such that:
\begin{enumerate}[label=(\alph*)]
\item Each $\phi_i$ is well-defined (respects equivalence classes) and maps finite-resolution states into smooth regions;
\item As $i\to\infty$, the images $\phi_i(\mathcal{C}_{R_i})$ become dense in $\M$ and the induced geometric structures converge (in a suitable sense: Gromov--Hausdorff, manifold atlas convergence, etc.).
\end{enumerate}
All differential-geometric constructions (Laplacians, Green kernels, $SO(D)$, orbital stability) are performed on the effective manifold $\M$, not on any fixed finite-resolution quotient $\mathcal{C}_{R_i}$.
\end{definition}

\begin{example}[Lattice Convergence to Continuum]\label{ex:lattice}
Take $\C=\R^D$ with usual topology and define $R_i$ as rounding to a grid of spacing $2^{-i}$:
\[
R_i(x)=2^{-i}\bigl(\lfloor 2^i x_1\rfloor,\dots,\lfloor 2^i x_D\rfloor\bigr)\in (2^{-i}\Z)^D.
\]
Then $\mathcal{C}_{R_i}$ can be identified with the discrete lattice $(2^{-i}\Z)^D$, which is granular at scale $2^{-i}$.
As $i\to\infty$, the lattice becomes dense in $\R^D$, and the effective manifold model is $\M=\R^D$.
This is the textbook pattern ``discrete micro / smooth macro'' resolving the tension between RG3 and smooth manifold structure.
\end{example}

\begin{remark}[On Terminology and Notation]\label{rem:notation}
In the sequel, when we write ``$\CR$ is a smooth $D$-manifold'' or apply differential geometry, we mean \emph{the effective manifold $\M$ arising in the refinement limit}, not a single finite-resolution quotient.
For notational simplicity, we often write $\CR$ when $\M$ is understood from context; where disambiguation is essential, we use $\M$ explicitly.
The \emph{recognition dimension} $D$ is the dimension of $\M$, the number of independent smooth coordinates required to parameterize the space of distinguishable events in the refinement limit.
\end{remark}

\section{Constraint (A): Same-Dimension Linking Forces Odd Dimensions}\label{sec:constraint_A}

The first selection principle concerns the topological capacity of the observable space to support entangled extended structures. We show that the requirement for integer-valued linking of extended objects of the \emph{same} intrinsic dimension forces the ambient space dimension to be odd.\footnote{Throughout this section, arguments involving homology, Alexander duality, and smooth embeddings apply to the effective manifold $\M$ (Definition~\ref{def:two_scale}), not to finite-resolution quotients $\mathcal{C}_{R_i}$.}

\subsection{Physical Motivation: Dimension Parity and Entanglement}

In Recognition Geometry, configurations in $\C$ may correspond to field lines, flux tubes, polymer chains, or other extended structures. Two such objects are observationally linked if no local measurement or continuous rearrangement (without tearing or intersection) can separate them. The linking number provides an integer-valued topological charge that is stable under perturbations and serves as a robust observable invariant.

The capacity to support integer-valued linking is sensitive not just to the ambient dimension $D$, but to the \emph{relative} dimensions of the objects being linked. The classical case—two loops (1-dimensional circles) in 3-dimensional space—is special because the dimensions balance correctly. But the underlying principle is more general and reveals a fundamental dimension-parity constraint.

\subsection{The Dimension Formula for Same-Dimension Linking}

We first establish the dimension-counting constraint, then provide sufficient conditions for the existence of a well-defined linking number.

\begin{lemma}[Dimension Formula for Intersection]\label{lem:intersection_dim}
Let $W$ be a $(p+1)$-dimensional submanifold and $B$ a $p$-dimensional submanifold in an oriented $D$-manifold, intersecting transversely. Then the intersection dimension is
\[
\dim(W\cap B) = (p+1) + p - D = 2p+1-D\ADD{.}
\]
For a signed intersection number $W\cdot B\in\mathbb{Z}$ (a signed count of points) to exist, we require $\dim(W\cap B)=0$, which forces $D=2p+1$.
\end{lemma}

\begin{proof}
Standard transversality: for transverse submanifolds of dimensions $a$ and $b$ in dimension $D$, the intersection has dimension $a+b-D$. Here $a=p+1$, $b=p$, so $\dim(W\cap B) = 2p+1-D$. A signed count is defined only when this equals zero.
\end{proof}

\begin{theorem}[Same-Dimension Linking Formula]\label{thm:same_dim_linking}
Let $\CR$ be a closed oriented $D$-manifold with vanishing intermediate homology:
\[
H_i(\CR;\mathbb{Z}) = 0 \quad\text{for all}\quad 0 < i < D
\]
(satisfied, for instance, if $\CR$ has the integral homology of $S^D$). Let $A,B\subset \CR$ be disjoint, closed, oriented $p$-submanifolds with $0 < p < D$. Then a canonical well-defined $\mathbb{Z}$-valued linking number $\lk(A,B)\in\mathbb{Z}$ exists if and only if $D = 2p + 1$.
\end{theorem}

\begin{proof}
We establish four claims: existence of bounding chain, necessity of dimension formula, independence of choice, and failure when $D\neq 2p+1$.

\smallskip
\noindent\textbf{(1) Existence of bounding chain $W$.}
Since $0<p<D$ and $H_p(\CR;\mathbb{Z})=0$ by hypothesis, we have $[A]=0$ in homology. Thus there exists a $(p+1)$-chain $W$ with $\partial W=A$.

\smallskip
\noindent\textbf{(2) Dimension formula forces $D=2p+1$ (necessity).}
By Lemma~\ref{lem:intersection_dim}, the intersection $W\cap B$ has dimension $2p+1-D$. For $W\cdot B$ to be a signed integer (count of points), we require $2p+1-D=0$, i.e., $D=2p+1$. If $D\neq 2p+1$, the intersection is generically higher-dimensional and cannot yield an integer-valued invariant via point-counting.

\smallskip
\noindent\textbf{(3) Independence of choice of $W$ (sufficiency when $D=2p+1$).}
Assume $D=2p+1$. If $W$ and $W'$ both satisfy $\partial W=\partial W'=A$, then $Z:=W-W'$ is a $(p+1)$-cycle. Since $p+1<D$ and $H_{p+1}(\CR;\mathbb{Z})=0$ by hypothesis, there exists a $(p+2)$-chain $Q$ with $\partial Q=Z$.

\ADD{In an oriented closed $D$-manifold, the intersection number $Z\cdot B$ depends only on the homology class $[Z]\in H_{p+1}(\CR;\Z)$ (see, e.g., Rolfsen~\cite{Rolfsen1976}).
Since $H_{p+1}(\CR;\Z)=0$ by hypothesis and $Z$ is a cycle, we have $[Z]=0$, hence $Z\cdot B=0$.  Therefore:}
\[
(W\cdot B) - (W'\cdot B) = (W-W')\cdot B = Z\cdot B = 0.
\]
Thus $\lk(A,B):=W\cdot B$ is independent of the choice of bounding chain.

\smallskip
\noindent\textbf{(4) Conclusion.}
The linking number $\lk(A,B)$ is well-defined if and only if $D=2p+1$, forcing $D$ to be odd.
\end{proof}

\subsection{The Allowed Dimension Set}

\begin{proposition}[Constraint (A): Odd Dimensions]\label{prop:odd_dimensions}
If the recognition quotient $\CR$ admits nontrivial integer-valued linking invariants between extended objects of the same intrinsic dimension $p$, then $\dimop(\CR)=2p+1$ for some integer $p\ge 1$. Thus the allowed-dimension set is
\[
\mathcal{A}_A = \{D\in\mathbb{N} : D \text{ is odd and } D\ge 3\} = \{3, 5, 7, 9, \dots\}.
\]
\end{proposition}

\begin{proof}
By Theorem~\ref{thm:same_dim_linking}, same-dimension linking requires $D=2p+1$ for some integer \ADD{$p\ge 1$} (since Theorem~\ref{thm:same_dim_linking} assumes $0<p<D$). For $p=1$ (loops), $D=3$; for $p=2$ (surfaces), $D=5$; \ADD{for $p=3$, $D=7$;} and so on. All such $D$ are odd. Conversely, any odd $D\ge 3$ can be written as $D=2p+1$ for $p=(D-1)/2\ge 1$, so $\mathcal{A}_A=\{3,5,7,\dots\}$.
\end{proof}

\begin{remark}[Physical Interpretation: Codimension-2 Defects]\label{rem:codim2}
The constraint requires that \emph{for some sector} of extended structures with intrinsic dimension $p=(D-1)/2$, same-dimension linking must be possible. In Recognition Geometry, extended objects—field lines, flux tubes, membranes, cosmic strings, domain walls—arise naturally as configurations in $\C$ that project to submanifolds in the quotient $\CR$. The requirement is not that \emph{all} such objects link, but that the emergent geometry supports topologically entangled states for \emph{some} class of equal-dimension defects.

\ADD{Same-dimension linking requires $p=(D-1)/2$, so the codimension of such objects is $D-p=(D+1)/2$, which equals~$2$ \emph{only} when $D=3$.  Thus codimension-2 defects (a physically distinguished class, e.g.\ vortex lines, cosmic strings) directly force $D=3$.  In general, the same-dimension linking constraint forces dimensional parity: observable space must be odd-dimensional to support robust integer-valued topological charges for same-dimension extended structures, without singling out codimension~2 for $D\neq 3$.}
This excludes even dimensions $D=2,4,6,\dots$ where no such entanglement sector exists, yielding $\mathcal{A}_A=\{3,5,7,\dots\}$.
\end{remark}

\begin{remark}[The Loop Case $p=1$]
The familiar case of \emph{loop-loop linking} corresponds to $p=1$, giving $D=2(1)+1=3$. This is the topological entanglement observed in magnetic flux tubes, polymer chains, and the Hopf link. This special case (constraint (T) in Section~\ref{sec:specializations}) directly forces $D=3$, but the general Constraint (A) allows $p$ to vary, admitting $D\in\{3,5,7,\dots\}$. The physical justification for this flexibility is provided in Remark~\ref{rem:codim2}: recognition geometries generically support extended structures at various dimensions, and the requirement is that \emph{some} same-dimension linking sector exists. The selection of $D=3$ emerges only when combined with constraints (B) and (C).
\end{remark}

\subsection{Physical Interpretation in Recognition Geometry}

In the RG paradigm, linking is not a primitive property of the configuration space $\C$, but an \emph{observable} property of the quotient $\CR$. Extended structures (field lines, polymer strands) correspond to embedded submanifolds in $\CR$. The constraint (A) formalizes the principle that \emph{observable space must support topologically distinct entangled states}. If such states exist for objects of intrinsic dimension $p$, the emergent ambient dimension must satisfy $D=2p+1$, forcing $D$ to be odd.

\section{Constraint (B): Green-Kernel Stability Forces $D<4$}\label{sec:constraint_B}

The second selection principle addresses the dynamical stability of bound states. In any recognition-based world, physical potentials are not fundamental properties of space but emerge from the \emph{information cost} required to distinguish configurations in the recognition quotient $\CR$. We show that the requirement for \emph{stable} (not necessarily non-precessing) circular orbits under Green-kernel potentials forces $D<4$.\footnote{All differential-geometric constructions in this section (Laplace--Beltrami operator, Green kernels, effective potentials, orbital mechanics) are performed on the effective manifold $\M$ (Definition~\ref{def:two_scale}), not on finite-resolution quotients.}

\subsection{Physical Motivation: Stable Bound States}

A recognition structure—an atom, planetary system, or bound vortex pair—requires \emph{radial stability}: small perturbations should produce bounded oscillations, not runaway spiraling. In Recognition Geometry, potentials emerge from information costs; under isotropy and scale-freeness, the natural dynamics is governed by Green-kernel potentials.

The question is: for which dimensions $D$ do such potentials admit stable circular orbits? This is a weaker requirement than \emph{closure} (non-precession), which would force $D=3$ uniquely. Here we ask only for \emph{stability}, admitting a broader class of dimensions.

\begin{remark}[Domain of Applicability]\label{rem:orbital_domain}
The orbital stability analysis requires angular momentum and circular orbits, which presuppose at least two-dimensional motion (a plane of rotation). Therefore, $D=1$ is \emph{excluded from the domain of applicability}: in one-dimensional space, there is no rotation group, no angular momentum $\ell$, and no centrifugal barrier. The constraint applies only to $D\ge 2$.
\end{remark}

\subsection{Emergent Potentials from Recognition Costs}

\begin{proposition}[RG Derivation of Central Potentials]\label{prop:central_potentials}
If the recognition structure is rotationally symmetric and satisfies an additivity principle for information costs, the emergent potential $V_D(r)$ in a $D$-dimensional recognition quotient is given by the Green's function of the Laplacian:
\[
V_D(r) \propto 
\begin{cases} 
k\ln(r) & D=2 \\
-\frac{k}{r^{D-2}} & D \ge 3 
\end{cases}
\]
where $k>0$ sets the strength (attractive sign chosen for bound states).
\end{proposition}

The derivation follows standard Green-kernel methods (see Appendix~\ref{app:green_derivation} for details). Note the qualitative difference: the $D=2$ case is \emph{logarithmic}, while for $D\ge 3$ we have an \emph{inverse-power law}
\[
V_D(r) = -\frac{k}{r^{n}}, \quad\text{where}\quad n := D-2 \ge 1.
\]
These cases require separate stability analyses.

\subsection{Stability Analysis for Circular Orbits}

We analyze stability separately for $D\ge 3$ (power-law potentials) and $D=2$ (logarithmic potential).

\begin{theorem}[Green-Kernel Stability for $D\ge 3$]\label{thm:stability_power}
Let $\CR$ be a smooth $D$-manifold with $D\ge 3$ and Green-kernel potential $V_D(r) = -k/r^n$ where $n=D-2\ge 1$ and $k>0$. A circular orbit at radius $r_0$ is stable (i.e., $U''_{\text{eff}}(r_0)>0$) if and only if $n<2$, equivalently $D<4$. For integer $D\ge 3$, only $D=3$ satisfies this condition.
\end{theorem}

\begin{proof}
We analyze the effective potential for a particle of mass $m$ with angular momentum $\ell$ moving under the central force $F(r)=-k n/r^{n+1}$ (where $n=D-2\ge 1$ for $D\ge 3$):
\[
U_{\text{eff}}(r) = \frac{\ell^2}{2m r^2} - \frac{k}{r^n}\ADD{.}
\]

\smallskip
\noindent\textbf{Step 1: Circular orbit condition.}
A circular orbit occurs at a critical point $r_0$ where $U'_{\text{eff}}(r_0)=0$:
\[
U'_{\text{eff}}(r) = -\frac{\ell^2}{m r^3} + \frac{k n}{r^{n+1}}.
\]
Setting $U'_{\text{eff}}(r_0)=0$ gives
\[
\frac{\ell^2}{m} = k n\, r_0^{2-n}\ADD{.}
\]

\smallskip
\noindent\textbf{Step 2: Stability condition.}
Stability requires $U''_{\text{eff}}(r_0)>0$. Compute:
\[
U''_{\text{eff}}(r) = \frac{3\ell^2}{m r^4} - \frac{k n(n+1)}{r^{n+2}}.
\]
At $r=r_0$, substitute $\ell^2/m = k n r_0^{2-n}$:
\begin{align*}
U''_{\text{eff}}(r_0) &= \frac{3 k n r_0^{2-n}}{r_0^4} - \frac{k n(n+1)}{r_0^{n+2}} \\
&= \frac{k n}{r_0^{n+2}} \bigl(3 - (n+1)\bigr) \\
&= \frac{k n (2-n)}{r_0^{n+2}}.
\end{align*}

Since $k>0$, $n\ge 1$ (for $D\ge 3$), and $r_0>0$, we have $U''_{\text{eff}}(r_0)>0$ if and only if $2-n>0$, i.e., $n<2$.

\smallskip
\noindent\textbf{Step 3: Translate to dimension.}
Since $n=D-2$, the condition $n<2$ becomes $D-2<2$, i.e., $D<4$. For integer dimensions $D\ge 3$, only $D=3$ satisfies this.
\end{proof}

\begin{proposition}[Logarithmic Potential Stability for $D=2$]\label{prop:stability_log}
For $D=2$ with logarithmic Green-kernel potential $V_2(r) = k\ln(r)$ (where $k>0$ gives attractive force $F(r)=-k/r$), circular orbits are stable.
\end{proposition}

\begin{proof}
The effective potential is
\[
V_{\text{eff}}(r) = k\ln(r) + \frac{\ell^2}{2mr^2}.
\]
Differentiate: $V'_{\text{eff}}(r) = k/r - \ell^2/(mr^3)$. A circular orbit satisfies $V'_{\text{eff}}(r_0)=0$, giving
\[
\frac{k}{r_0} = \frac{\ell^2}{mr_0^3} \quad\Longrightarrow\quad \ell^2 = mk r_0^2.
\]
For stability, compute $V''_{\text{eff}}(r) = -k/r^2 + 3\ell^2/(mr^4)$. At $r=r_0$, substitute $\ell^2 = mk r_0^2$:
\[
V''_{\text{eff}}(r_0) = -\frac{k}{r_0^2} + \frac{3mk r_0^2}{mr_0^4} = -\frac{k}{r_0^2} + \frac{3k}{r_0^2} = \frac{2k}{r_0^2} > 0.
\]
Thus $D=2$ admits stable circular orbits under logarithmic potentials.
\end{proof}

\begin{corollary}[Allowed Set for Orbital Stability]\label{cor:allowed_B}
Combining Remark~\ref{rem:orbital_domain} (excluding $D=1$), Theorem~\ref{thm:stability_power} ($D=3$ for power laws), and Proposition~\ref{prop:stability_log} ($D=2$ for logarithmic), the allowed-dimension set for central-force orbital stability under Green-kernel potentials is
\[
\mathcal{A}_B = \{2, 3\}.
\]
\end{corollary}

\begin{remark}[Higher Dimensions are Unstable]
For $D=4$, we have $n=2$, so $U''_{\text{eff}}(r_0)=0$ (marginal stability). For $D>4$, $U''_{\text{eff}}(r_0)<0$ (unstable). In these dimensions, near-circular orbits are perturbatively unstable: small radial perturbations grow, causing the orbit to spiral inward or outward. Recognition structures (atoms, planetary systems) relying on stable bound states cannot exist in $D\ge 4$ under Green-kernel power-law dynamics.
\end{remark}

\begin{remark}[Why $D=2$ and $D\ge 3$ Differ]
The qualitative difference between logarithmic ($D=2$) and power-law ($D\ge 3$) potentials reflects the Green-kernel singularity structure: in two dimensions, the Laplacian's fundamental solution is logarithmic, while in higher dimensions it follows a power law. Both cases admit stable orbits, but the $D=2$ case requires separate calculation and cannot be obtained by setting $n=0$ in the power-law formula.
\end{remark}

\subsection{Physical Interpretation in Recognition Geometry}

Constraint (B) formalizes the principle that \emph{observable space must support stable dynamical structures}. In the recognition paradigm, stable bound states represent repeating, recognizable configurations—atoms whose electrons return to the same orbital pattern, planetary systems with predictable periods. If the emergent potential destabilizes circular orbits (as happens for $D\ge 4$), such structures cannot persist, and the notion of a "stable recognition pattern" loses operational meaning.

The constraint admits $\mathcal{A}_B=\{2,3\}$ (Corollary~\ref{cor:allowed_B}), a non-singleton set. Note that $D=1$ is excluded not because orbits are unstable but because circular orbits and angular momentum are undefined in one-dimensional space (Remark~\ref{rem:orbital_domain}). Combined with constraint (A) (odd dimensions) and constraint (C) (non-abelian rotations), only $D=3$ survives.

\section{Constraint (C): Non-Abelian Rotations Force $D\ge 3$}\label{sec:constraint_C}

The third selection principle concerns the geometric richness of the observable space. We show that requiring the rotation group to be non-abelian—so that rotations in different planes do not commute—forces $D\ge 3$.\footnote{In the smooth Riemannian limit, oriented orthonormal frames form a principal $SO(D)$-bundle over $\M$; non-abelianity of $SO(D)$ captures local rotational complexity.}

\subsection{Physical Motivation: Rotational Complexity}

In a one-dimensional space ($D=1$), there is no notion of rotation at all; in fact,  the orthogonal group $SO(1)$ is trivial. In a two-dimensional space ($D=2$), rotations are parameterized by a single angle $\theta\in S^1$, and all rotations commute: $SO(2)\cong S^1$ is abelian. Only starting from $D=3$ do we have multiple independent rotation planes (e.g., rotations about the $x$, $y$, and $z$ axes), and these rotations \emph{do not commute}: $SO(3)$ is non-abelian.

In Recognition Geometry, the emergent space $\CR$ must support \emph{operationally distinguishable rotational transformations}. If all rotations commute, the space lacks the geometric complexity needed for rich dynamics (e.g., gyroscopic precession, angular momentum coupling). Hence, the requirement for non-abelian rotations provides a natural lower bound on dimension.

\subsection{The Non-Abelian Constraint}

\begin{proposition}[Non-Abelian \ADD{Local Frame} Rotation Group]\label{prop:nonabelian}
The \ADD{local orthonormal frame} rotation group $SO(D)$ is non-abelian if and only if $D\ge 3$. Thus the allowed-dimension set for constraint (C) is
\[
\mathcal{A}_C = \{D\in\mathbb{N} : D\ge 3\} = \{3, 4, 5, 6, 7, \dots\}.
\]
\end{proposition}

\begin{proof}
It is a standard fact from Lie theory:
\begin{itemize}
\item $SO(1)$ is the trivial group (no rotations in one dimension).
\item $SO(2)\cong S^1$ is isomorphic to the circle group of rotations by angle $\theta$. Since rotations commute (rotation by $\theta_1$ followed by $\theta_2$ equals rotation by $\theta_1+\theta_2=\theta_2+\theta_1$), $SO(2)$ is abelian.
\item For $D\ge 3$, $SO(D)$ is non-abelian. For instance, in $SO(3)$, rotations about the $x$-axis and $y$-axis do not commute: $R_x(\alpha)R_y(\beta) \neq R_y(\beta)R_x(\alpha)$ for generic $\alpha,\beta$.
\end{itemize}
Thus $SO(D)$ is non-abelian if and only if $D\ge 3$.
\end{proof}

\subsection{Physical Interpretation in Recognition Geometry}

Constraint (C) formalizes the principle that \emph{observable space must support geometrically independent transformations}. In RG, recognition symmetries (Definition~\ref{def:rec_symmetries}) correspond to transformations of configurations that leave measurement outcomes unchanged. If the quotient $\CR$ is to exhibit non-trivial rotational symmetry (e.g., isotropy of physical laws), the rotation group must be rich enough to encode multiple independent directions.

The condition $D\ge 3$ excludes the degenerate low-dimensional cases $D=1,2$ where the geometry is "too simple" to support the rotational complexity observed in the physical world. This provides the lower bound that complements the upper bound from constraint (B).

\section{Main Result: Convergence to $D=3$}

We now synthesize the three independent constraints, establishing that their intersection uniquely selects $D=3$.

\begin{theorem}[Dimensional Rigidity in Recognition Geometry---Full Statement]\label{thm:full}
Let $(\C,\E,R)$ be a recognition geometry with a directed system of recognizers $(R_i)_{i\in\N}$ and effective manifold $\M$ (Definition~\ref{def:two_scale}). Assume $\M$ has sufficient regularity for constraints (A), (B), and (C) to be formulated. If $\M$ satisfies all three constraints, then $\dimop(\M)=3$.

Conversely, if $\dimop(\M)=3$ and $\M$ admits:
\begin{itemize}
    \item smooth embeddings of $p$-manifolds (for same-dimension linking),
    \item a rotationally symmetric metric inducing a Green-kernel potential (for orbital stability),
    \item a non-abelian rotation group $SO(D)$ (for geometric richness),
\end{itemize}
then constraints (A), (B), and (C) are satisfied.
\end{theorem}

\begin{proof}
\noindent\textbf{Sufficiency: constraints imply $D=3$.}

By Proposition~\ref{prop:odd_dimensions}, constraint (A) gives $\dimop(\M)\in\mathcal{A}_A=\{3,5,7,\dots\}$ (odd dimensions with $D\ge 3$).

By Corollary~\ref{cor:allowed_B}, constraint (B) gives $\dimop(\M)\in\mathcal{A}_B=\{2,3\}$ (dimensions with stable orbits under Green-kernel potentials).

By Proposition~\ref{prop:nonabelian}, constraint (C) gives $\dimop(\M)\in\mathcal{A}_C=\{3,4,5,6,\dots\}$ (dimensions with non-abelian rotations).

The intersection is
\[
\mathcal{A}_A \cap \mathcal{A}_B \cap \mathcal{A}_C = \{3,5,\dots\} \cap \{2,3\} \cap \{3,4,5,\dots\} = \{3\}.
\]
Thus $\dimop(\M)=3$.

\smallskip
\noindent\textbf{Necessity: $D=3$ satisfies all constraints.}

Assume $\dimop(\M)=3$ and verify each constraint:

\textbf{Constraint (A):} Since $3=2(1)+1$, we have $D=2p+1$ for $p=1$. By Theorem~\ref{thm:same_dim_linking}, same-dimension linking of 1-manifolds (loops) is possible. For instance, two disjoint embedded circles in $\M\cong S^3$ can have nontrivial linking number $\lk(\gamma_1,\gamma_2)\in\mathbb{Z}$. Thus constraint (A) is satisfied.

\textbf{Constraint (B):} For $D=3$, we have $n=D-2=1$, so $n<2$. By Theorem~\ref{thm:stability_power}, circular orbits under the Green-kernel potential $V_3(r)=-k/r$ are stable ($U''_{\text{eff}}(r_0)>0$). Thus constraint (B) is satisfied.

\textbf{Constraint (C):} Since $3\ge 3$, the rotation group $SO(3)$ is non-abelian (Proposition~\ref{prop:nonabelian}). For instance, rotations about different axes do not commute. Thus constraint (C) is satisfied.

This completes the proof.
\end{proof}

\begin{corollary}[No Alternative Dimensions]
There is no $D\neq 3$ satisfying all three constraints simultaneously.
\end{corollary}

\begin{proof}
The intersection $\mathcal{A}_A\cap\mathcal{A}_B\cap\mathcal{A}_C=\{3\}$ is a singleton.
\end{proof}

\subsection{Summary of Selection Principles}

Table~\ref{tab:constraints} summarizes the three independent constraints and their allowed-dimension sets.

\begin{table}[h]
\centering
\begin{tabular}{p{1.5cm}p{2.5cm}p{4.5cm}p{3.5cm}}
\toprule
\textbf{Constraint} & \textbf{Type} & \textbf{Physical Requirement} & \textbf{Allowed Set $\mathcal{A}_X$} \\
\midrule
(A) Linking & Topological & Same-dimension entanglement $D=2p+1$ & $\{3,5,7,\dots\}$ \\[6pt]
(B) Stability & Dynamical & Stable Green-kernel orbits ($D\ge 2$, $D<4$) & $\{2,3\}$ \\[6pt]
(C) Rotations & Geometric & Non-abelian $SO(D)$ for $D\ge 3$ & $\{3,4,5,6,\dots\}$ \\
\midrule
\textbf{Intersection} & \textbf{---} & \textbf{All three} & $\mathbf{\{3\}}$ \\
\bottomrule
\end{tabular}
\caption{Three genuinely independent selection principles with non-singleton allowed sets $\mathcal{A}_A$, $\mathcal{A}_B$, $\mathcal{A}_C$ converging to $\dim(\CR)=3$.}
\label{tab:constraints}
\end{table}

\section{Sharper Specializations (T), (K), (S)}\label{sec:specializations}

While constraints (A), (B), (C) provide a genuinely convergent selection mechanism, each admits sharper formulations that directly characterize $D=3$. These specializations offer additional physical insight and connect to well-known results (Alexander duality, Bertrand's theorem, computational efficiency).

\subsection{(T) Loop-Loop Linking via Alexander Duality}

\textbf{Specialization of (A):} Fix $p=1$ (circles). Then constraint (A) forces $D=2(1)+1=3$ uniquely.

\begin{theorem}[Alexander Duality for Loop Complements]\label{thm:alexander}
Let $\CR$ be a locally contractible homology $D$-manifold with the integral homology of $S^D$. Let $K\subset \CR$ be an embedded circle. Then
\[
H_1(\CR\setminus K)\cong\Z \iff D=3.
\]
\end{theorem}

\begin{proof}
By Alexander duality,
\[
\widetilde H_1(\CR\setminus K)\ \cong\ \widetilde H^{D-2}(S^1).
\]
Since $\widetilde H^{q}(S^1)$ is $\Z$ for $q=1$ and $0$ otherwise, the right-hand side is $\Z$ precisely when $D-2=1$, i.e., $D=3$.
\end{proof}

The nontrivial $H_1(\CR\setminus K)\cong\mathbb{Z}$ provides the algebraic source for integer-valued linking: a second disjoint loop $\gamma$ defines a class $[\gamma]\in H_1(\CR\setminus K)$, and $\lk(K,\gamma)\in\mathbb{Z}$ measures the winding. This is the familiar Hopf link, Borromean rings, etc.

\subsection{(K) Non-Precessing Kepler Orbits}

\textbf{Strengthening of (B):} This requires not just stability, but exact \emph{closure} (no precession, apsidal angle $\Delta\theta=2\pi$).

Before stating the theorem, we establish our convention for measuring orbital precession and justify the use of planar coordinates.

\begin{definition}[Apsidal Angle]\label{def:apsidal}
For a bound orbit $r(\theta)$ in a central potential, define the \emph{apsidal angle} $\Delta\theta$ as the angular separation between successive \emph{periapsides} (perihelion passages):
\[
\Delta\theta := \theta_{k+1} - \theta_k \quad\text{where}\quad r(\theta_k) = r(\theta_{k+1}) = r_{\min}.
\]
Equivalently, $\Delta\theta$ is the angular advance over one full radial oscillation (peri $\to$ apo $\to$ peri). For a non-precessing orbit, $\Delta\theta = 2\pi$ (the periapsis returns to the same angular position after one radial cycle).
\end{definition}

Although we work in $D$-dimensional space, orbital motion under central forces is inherently planar:

\begin{lemma}[Planarity of Central-Force Motion]\label{lem:planar}
Let $\mathbf{r}(t) \in \mathbb{R}^D$ with $D\ge 2$ satisfy Newton's equation $m\ddot{\mathbf{r}} = F(\|\mathbf{r}\|)\,\mathbf{r}/\|\mathbf{r}\|$ for a central force. Then the motion lies in the two-dimensional subspace $P := \mathrm{span}\{\mathbf{r}(0), \dot{\mathbf{r}}(0)\}$ spanned by the initial position and velocity.
\end{lemma}

\begin{proof}
The angular momentum bivector $L := \mathbf{r} \wedge \mathbf{p} \in \Lambda^2(\mathbb{R}^D)$ (where $\mathbf{p} = m\dot{\mathbf{r}}$) is conserved: $\dot{L} = \dot{\mathbf{r}} \wedge \mathbf{p} + \mathbf{r} \wedge \dot{\mathbf{p}} = 0$ since $\dot{\mathbf{r}} \wedge (m\dot{\mathbf{r}}) = 0$ and $\mathbf{r} \wedge (F\mathbf{r}/\|\mathbf{r}\|) = 0$. By uniqueness of solutions to the second-order ODE with initial data in $P$, the trajectory remains in $P$ for all time.
\end{proof}

Thus, using polar coordinates $(r,\theta)$ in the invariant plane is mathematically rigorous even when the ambient space has $D \ge 3$. We can now state the characterization:

\begin{theorem}[Kepler Non-Precession Principle]\label{thm:kepler}
Let $\CR$ be a smooth $D$-manifold with Green-kernel potential $V_D(r) \propto -r^{2-D}$ for $D\ge 3$. Near-circular orbits are non-precessing (Definition~\ref{def:apsidal}: $\Delta\theta = 2\pi$) if and only if $D=3$.
\end{theorem}

\begin{proof}
By Lemma~\ref{lem:planar}, motion reduces to a plane with polar coordinates $(r,\theta)$. The orbit equation in Binet form with $u(\theta) := 1/r(\theta)$ is
\[
u'' + u = \alpha\, u^{D-3}, \quad\text{where}\quad \alpha := \frac{mk(D-2)}{\ell^2}.
\]
(See Appendix~\ref{app:apsidal} for derivation.)

\textbf{Case $D=3$:} The equation becomes linear, $u'' + u = \alpha$ (constant right-hand side). Solutions are conic sections; bounded orbits are closed ellipses with $\Delta\theta = 2\pi$ (no precession).

\textbf{Case $D\neq 3$:} The equation is nonlinear. A circular orbit $u(\theta) \equiv u_0$ satisfies $\alpha = u_0^{2-n}$ where $n := D-2$. Linearizing $u = u_0 + \delta(\theta)$ with $|\delta| \ll 1$ yields (see Appendix~\ref{app:apsidal} for details):
\[
\delta'' + (2-n)\delta = 0, \quad\text{i.e.,}\quad \delta'' + (4-D)\delta = 0.
\]
For $D<4$ (the stable regime from Constraint~\ref{thm:stability_power}), this has solutions $\delta(\theta) = A\cos(\omega\theta) + B\sin(\omega\theta)$ with frequency $\omega = \sqrt{4-D}$. The perturbation $\delta(\theta)$ is periodic in $\theta$ with period $2\pi/\omega$. Since periapsis occurs at maxima of $u$ (minima of $r$), successive periapsides are separated by one full period:
\[
\Delta\theta = \frac{2\pi}{\omega} = \frac{2\pi}{\sqrt{4-D}}.
\]

For non-precession, Definition~\ref{def:apsidal} requires $\Delta\theta = 2\pi$. Thus:
\[
\frac{2\pi}{\sqrt{4-D}} = 2\pi \quad\Longrightarrow\quad \sqrt{4-D} = 1 \quad\Longrightarrow\quad D = 3.
\]

Conversely, $D=3$ gives $\omega = 1$ and $\Delta\theta = 2\pi$ as required.
\end{proof}

This strengthens constraint (B): not only are orbits stable in $D=3$, they are \emph{exactly periodic} (Bertrand's theorem). This is essential for "stable atoms" and planetary systems—recognition structures that repeat their configuration exactly.

\subsection{(S) Computational Consideration: A Boundary Optimization}

\textbf{Not an independent selector:} Given the lower bound $D\ge 3$ already established by constraint (C), we examine a computational efficiency consideration involving synchronization periods.

In Recognition Science applications, a $D$-dimensional recognizer operates on a discrete register with $2^D$ states (internal period), and external dynamics exhibit gap periods. An external parameter $N=45$ is taken from Recognition Science (not derived within this paper) as a phenomenological input. We emphasize that the following analysis does not contribute to the dimensional selection $D=3$, which rests entirely on constraints (A), (B), (C).

\begin{definition}[Synchronization Period]
For a gap period $N$ and dimension $D$, the \emph{synchronization period} is
\[
S(D,N) := \lcmop(2^D, N)\ADD{,}
\]
representing the computational overhead for phase-locking internal dyadic cycles with external rhythms.
\end{definition}

\begin{theorem}[Boundary Optimization on $\mathcal{A}_C$]\label{thm:sync}
Given the lower bound $D\ge 3$ established by constraint (C), and a gap period $N=45$ (odd), the synchronization period $S(D,45):=\mathrm{lcm}(2^D,45)$ satisfies $S(D,45)=45\cdot 2^D$, which is strictly increasing in $D$. Therefore the minimal synchronization overhead on the admissible regime $D\ge 3$ occurs at the boundary: $D=3$, giving $S(3,45)=360$.
\end{theorem}

\begin{proof}
Since $45=9\times 5$ is odd and $2^D$ is a power of 2, we have $\gcdop(2^D,45)=1$ for all $D\in\mathbb{N}$. By the identity $\mathrm{lcm}(a,b)=ab/\gcd(a,b)$, we obtain
\[
S(D,45)=\mathrm{lcm}(2^D,45)=\frac{2^D\cdot 45}{1}=45\cdot 2^D.
\]
For any $D\in\mathbb{N}$, 
\[
S(D+1,45)-S(D,45)=45\cdot 2^{D+1}-45\cdot 2^D=45\cdot 2^D>0,
\]
so $S(\cdot,45)$ is strictly increasing. Under the constraint $D\ge 3$, the unique minimizer is the smallest feasible value: $D=3$, yielding $S(3,45)=45\cdot 8=360$.
\end{proof}

\begin{remark}[Tautological Structure and Status of N=45]\label{rem:tautology}
This result is tautological in the following sense: for \emph{any} odd $N$ and \emph{any} lower bound $D_{\min}$, the minimizer of $\mathrm{lcm}(2^D,N)=N\cdot 2^D$ over $D\ge D_{\min}$ is always $D=D_{\min}$ (the boundary value). Thus constraint (S) does not independently select $D=3$; it merely confirms that $D=3$ (already established by constraints (A), (B), (C)) is the boundary minimum for synchronization overhead.

The value $N=45$ is taken from Recognition Science applications as a phenomenological parameter; \textbf{it is not derived within this paper}. Importantly, the dimensional selection $D=3$ from Theorem~\ref{thm:full} is completely independent of the specific value of $N$: the argument applies for any odd $N$. The synchronization period $360=45\cdot 2^3$ is therefore a \emph{consequence} of the input choice $N=45$, not a theoretical prediction. With $N=43$ we would obtain $344$; with $N=47$ we would obtain $376$; the dimensional conclusion $D=3$ remains unchanged.

We include constraint (S) as a computational efficiency consideration (minimal synchronization overhead given $D\ge 3$), not as an independent dimensional selector. The value $360$ can be noted for its high divisibility ($360=2^3\times 3^2\times 5$, with 24 divisors), which may be computationally convenient, but this is an a posteriori observation rather than a derivation.
\end{remark}

\subsubsection{Parameter-Dependent Formulation}\label{sec:param_dep}

The boundary optimization above can be made into a genuine interior optimization by correcting the sign of the capacity term. Consider the functional
\[
G(D,N) = \alpha \cdot \lcmop(2^D, N) - \beta \cdot D,
\]
where $\alpha>0$ penalizes synchronization latency and $\beta>0$ \emph{rewards} representational capacity (with the negative sign ensuring that larger $D$ reduces the objective in a minimization problem, consistent with "rewarding" capacity).

\begin{proposition}[Parameter-Dependent Optimum]\label{prop:param_opt}
For $N=45$ (odd), we have $G(D,45)=\alpha\cdot 45\cdot 2^D - \beta D$. The minimizer over $D\ge D_{\min}$ is
\[
D^\star = \max\left(D_{\min}, \left\lceil \log_2\left(\frac{\beta}{45\alpha}\right)\right\rceil\right).
\]
In particular, $D^\star=3$ (given $D_{\min}=3$) if and only if the parameters satisfy $4 < \beta/(45\alpha) \le 8$, equivalently $180\alpha < \beta \le 360\alpha$.
\end{proposition}

\begin{proof}
The forward difference is $G(D+1,45)-G(D,45)=\alpha\cdot 45\cdot 2^D-\beta$. This is negative (function decreasing) when $45\alpha\cdot 2^D<\beta$, and positive (function increasing) when $45\alpha\cdot 2^D\ge\beta$. Thus the minimum occurs at the smallest $D$ with $45\alpha\cdot 2^D\ge\beta$, which is $D=\lceil\log_2(\beta/(45\alpha))\rceil$, subject to $D\ge D_{\min}$.

For $D^\star=3$ (assuming $D_{\min}\le 3$), we need $45\alpha\cdot 2^2<\beta$ (so $D=2$ is not optimal) and $45\alpha\cdot 2^3\ge\beta$ (so $D=3$ is optimal). This gives $180\alpha<\beta\le 360\alpha$.
\end{proof}

\begin{remark}[Interpretation of Parameter Dependence]
Proposition~\ref{prop:param_opt} shows that with the corrected sign, the optimization becomes honest but reveals that $D=3$ is no longer a universal constant—it depends on the ratio $\beta/\alpha$. The original formulation (Theorem~\ref{thm:sync}) effectively takes $\beta=0$ (pure latency minimization), which makes the minimizer always the lower bound, rendering (S) tautological as a selection principle. We retain Theorem~\ref{thm:sync} as a tie-breaking rationale: given $D\ge 3$ from constraint (C), computational efficiency considerations favor $D=3$ as the boundary minimum. The parameter-dependent formulation (Proposition~\ref{prop:param_opt}) demonstrates that a genuine trade-off between capacity and synchronization overhead exists, but requires parameter tuning to select $D=3$.
\end{remark}

\begin{remark}[On the Specializations]\label{rem:spec_summary}
Constraints (T), (K), (S) each give $\mathcal{A}_T=\mathcal{A}_K=\mathcal{A}_S=\{3\}$ (singletons), so they do not exhibit convergent independence in the sense of Definition~\ref{def:allowed_sets}. They provide additional physical insights but are not independent selectors:
\begin{itemize}
\item (T) specializes (A) by fixing $p=1$ (loops), connecting to Alexander duality and observable entanglement (Hopf link, Borromean rings),
\item (K) strengthens (B) from stability to non-precession, connecting to Bertrand's theorem and exactly periodic bound states (stable atoms, planetary systems),
\item (S) is a boundary optimizer: given $D\ge 3$ from (C), minimizing the strictly increasing function $N\cdot 2^D$ (for any odd $N$) trivially yields $D=3$. The value $N=45$ is taken from Recognition Science applications as a phenomenological parameter and is not derived within this paper. The resulting synchronization period $360=8\times 45$ depends entirely on the choice $N=45$; different values of $N$ yield proportionally different periods without affecting the dimensional conclusion. Constraint (S) provides a computational efficiency interpretation (minimal overhead at the boundary) but does not contribute to the dimensional selection.
\end{itemize}

The primary dimensional selection is achieved exclusively by the genuinely independent constraints (A), (B), (C); constraints (T), (K) provide sharper physical insights, while (S) offers a subsidiary computational consideration.
\end{remark}

\section{Discussion}

\textbf{The two-scale structure as a feature.}
The interplay between finite-resolution quotients $\mathcal{C}_{R_i}$ and the effective manifold $\M$ (Definition~\ref{def:two_scale}) is not a technical complication but a conceptual strength of the recognition paradigm. It captures a fundamental duality present in all physical theories: the tension between operational foundations (what observers can actually distinguish) and mathematical idealization (the smooth structures enabling calculation).

At finite resolution, individual quotients $\mathcal{C}_{R_i}$ are necessarily granular (Lemma~\ref{lem:rg3_finite_classes}), reflecting the bounded resources of any physical observer—finite time, energy, memory, and noise tolerance. This granularity is not a pathology but an honest representation of measurement physics: RG3 (finite local resolution) formalizes the impossibility of distinguishing infinitely many outcomes locally. The discrete lattice structure at this scale is analogous to the atomistic view in molecular dynamics or the lattice regularization in quantum field theory.

Differential-geometric arguments (Laplacians, Green kernels, orbital mechanics) and topological constructions (homology, linking) require smooth structure and therefore apply to the effective manifold $\M$, the continuum limit arising as resolution is refined. This is not a departure from measurement-first principles but their natural completion: $\M$ represents the idealized geometry approached as observers increase their distinguishing power without bound. The convergence $\mathcal{C}_{R_i}\to\M$ (via coarse-graining maps $\phi_i$) formalizes the well-known physical pattern of discrete micro-scale descriptions flowing to smooth macro-scale effective theories.

Crucially, the dimensional selection $D=3$ occurs at the level of the effective manifold $\M$, not at any fixed finite resolution. This is physically appropriate: constraints (A), (B), (C) involve smooth embeddings, differential equations, and Lie groups—structures that only make sense in the continuum limit. The RG framework thus provides a rigorous two-scale ontology: operationally discrete (finite quotients) and effectively continuous (limit geometry), with dimensional rigidity emerging from constraints on the latter while being grounded in the former.

The connection to Recognition Science provides concrete physical grounding for our abstract framework. In that theory, the ledger space $\mathcal{L}$—representing the complete ontological state of the recognition-based universe—serves as the configuration space $\C$. Observable physical space emerges through a position recognizer $R_{\text{pos}}:\mathcal{L}\to\mathbb{R}^3$ extracting spatial coordinates from ledger states. By the injectivity theorem, the recognition quotient $\mathcal{L}/\!\sim_{R_{\text{pos}}}$ embeds into $\mathbb{R}^3$. Our dimensional rigidity result explains why: if the emergent space is to support same-dimension linking (A), stable dynamical structures (B), and non-abelian rotational symmetry (C), then it must satisfy $\dim(\CR)=3$. The choice of $\mathbb{R}^3$ as target space is not an arbitrary modeling decision but a structural requirement imposed by these three independent constraints.

Our results apply specifically to recognition geometries satisfying the hypotheses underlying each constraint. Constraint (A) requires that $\CR$ admit smooth embeddings of $p$-manifolds with sufficient regularity for Alexander duality. Constraint (B) assumes isotropy—a recognition-symmetry group acting transitively on metric spheres—justifying the radial Green-kernel potential. Constraint (C) requires that the emergent geometry support a non-abelian rotation group. Each assumption can be justified within RG—isotropy via refinement and symmetry averaging, linking via observable entanglement, non-abelian rotations via geometric capacity—but they remain structural hypotheses rather than logical necessities. The theorem establishes sufficiency: if $\CR$ satisfies all three constraints under their regularity conditions, then $\dim(\CR)=3$. It does not claim that dimensional alternatives are impossible when one or more hypotheses fail, nor does it extend immediately to quantum recognizers where indistinguishability involves divergence measures.

Several directions warrant investigation. Can constraint (A) extend to higher odd dimensions with analogous dynamical constraints? How robust is the intersection pattern under perturbations of the allowed sets? Can we formalize the observation that sharper specializations (T) and (K) provide additional overdetermined selection of $D=3$? What happens in quantum recognition geometries with stochastic recognizers $R:\C\to\Delta(\E)$—do constraints still force $D=3$, or does quantum indeterminacy permit higher dimensions?

\section{Conclusion}

We have established that spatial dimension $D=3$ emerges uniquely from Recognition Geometry as a mathematical necessity. Theorem~\ref{thm:full} proves that if a recognition quotient $\CR$ satisfies three genuinely independent constraints---same-dimension linking (A), Green-kernel stability (B), and non-abelian rotations (C)---each admitting multiple candidate dimensions, then $\dimop(\CR)=3$, with no alternative (Corollary 5.1). The intersection pattern 
\begin{center}
    $\mathcal{A}_A\cap\mathcal{A}_B\cap\mathcal{A}_C=\{3,5,\dots\}\cap\{2,3\}\cap\{3,4,5,\dots\}=\{3\}$
\end{center}
exhibits convergent independence: topology forces odd dimensions, dynamics forces low dimensions (excluding $D=1$ where angular momentum is undefined and $D\ge 4$ where orbits are unstable), and geometry forces high dimensions; only $D=3$ survives all three filters.

The dimensional rigidity theorem provides the first rigorous derivation of spatial dimension from measurement-first foundations. Unlike classical arguments that assume ambient space $\mathbb{R}^D$ and verify consistency, we construct observable space as a recognition quotient $\C/\!\sim_R$ and show that operational constraints on topological complexity, dynamical stability, and geometric richness uniquely determine its dimension. This connects Recognition Geometry to foundational questions—why is space three-dimensional? why do inverse-square laws emerge?—with an answer rooted in structural necessity: these features are mathematical consequences of the recognition paradigm required for a world supporting entangled structures, stable orbits, and non-trivial rotational symmetry.

Sharper specializations (T) and (K)—loop-loop linking and non-precessing Kepler orbits—provide additional physical insights and each individually characterize $D=3$, offering an overdetermined reinforcement of the main result. The computational consideration (S) is a boundary optimization that applies for any odd external parameter $N$ and does not contribute to dimensional selection; it is included for completeness but should not be viewed as supporting evidence for $D=3$. Open problems include extending the framework to higher odd dimensions with $p>1$ linking, proving perturbative robustness theorems, and adapting the analysis to quantum recognition geometries.

\section*{Acknowledgments}



\begin{thebibliography}{99}

\bibitem{WashburnZlatanovicAllahyarov2026}
J.~Washburn, M.~Zlatanovi\'{c}, and E.~Allahyarov,
\emph{Recognition Geometry},
Axioms (2026), accepted.

\bibitem{WashburnRahnamaiBarghi2026}
\ADD{J.~Washburn and A.~Rahnamai Barghi,
\emph{Reciprocal Convex Costs for Ratio Matching: Axiomatic Characterization},
Axioms (2026).
\texttt{doi:10.3390/axioms1010000}.}

\bibitem{Alexander1923}
J.W.~Alexander,
\emph{On the chains of a complex and their duals},
Proc. Nat. Acad. Sci. USA \textbf{10} (1924), 168--172.

\bibitem{BarrowTipler1986}
J.D.~Barrow and F.J.~Tipler,
\emph{The Anthropic Cosmological Principle},
Oxford University Press, 1986.

\bibitem{Binet1845}
J.~Binet,
\emph{M\'emoire sur l'int\'egration des \'equations diff\'erentielles de la m\'ecanique},
J. Math. Pures Appl. \textbf{10} (1845), 457--470.

\bibitem{Ehrenfest1917}
P.~Ehrenfest,
\emph{In what way does it become manifest in the fundamental laws of physics that space has three dimensions?},
Proc. Amsterdam Acad. \textbf{20} (1917), 200--209.

\bibitem{Freedman1982}
M.H.~Freedman,
\emph{The topology of four-dimensional manifolds},
J. Differential Geom. \textbf{17} (1982), 357--453.

\bibitem{Green1987}
M.B.~Green, J.H.~Schwarz, and E.~Witten,
\emph{Superstring Theory, Volume 1: Introduction},
Cambridge University Press, 1987.

\bibitem{Polchinski1998}
J.~Polchinski,
\emph{String Theory, Volume 1: An Introduction to the Bosonic String},
Cambridge University Press, 1998.

\bibitem{Lee2013}
J.M.~Lee,
\emph{Introduction to Smooth Manifolds},
3rd ed., Springer, 2013.

\bibitem{Riesz1990}
F.~Riesz and B.~Sz.-Nagy,
\emph{Functional Analysis},
Dover Publications, 1990.

\bibitem{Rolfsen1976}
D.~Rolfsen,
\emph{Knots and Links},
Publish or Perish, 1976.

\bibitem{Rovelli1996}
C.~Rovelli,
\emph{Relational Quantum Mechanics},
Int. J. Theor. Phys. \textbf{35} (1996), 1637--1678.

\bibitem{Tegmark1997}
M.~Tegmark,
\emph{On the dimensionality of spacetime},
Classical and Quantum Gravity \textbf{14} (1997), L69--L75.

\bibitem{vonNeumann1955}
J.~von Neumann,
\emph{Mathematical Foundations of Quantum Mechanics},
Princeton University Press, 1955.

\bibitem{Wald1984}
R.M.~Wald,
\emph{General Relativity},
University of Chicago Press, 1984.

\end{thebibliography}

\appendix

\section{Detailed Derivation of Green-Kernel Potentials}\label{app:green_derivation}

We provide the complete derivation of the dimension-dependent Green-kernel potential from Recognition Geometry principles.

\subsection{From Recognition Costs to Laplacian Operators}

In RG, a comparative recognizer assigns a \emph{cost} $J:\CR\times\CR\to\mathbb{R}_{\ge 0}$ to distinguish two observable states. Under isotropy (recognition symmetries acting transitively on spheres), this cost depends only on the distance $r=d(x,x_0)$ from a source $x_0$.

Locality and additivity of information costs motivate a linear, local second-order differential operator governing equilibrium influence profiles. In the manifold-like regime with Riemannian metric $g$, this is the Laplace--Beltrami operator $\Delta_g$.

\subsection{Radial Solution in $\mathbb{R}^D$}

For a radial function $V(r)$ on $\mathbb{R}^D$, the Laplacian is
\[
\Delta V(r) = V''(r) + \frac{D-1}{r}V'(r), \quad r>0.
\]
Away from the source, $\Delta V=0$, so
\[
V''(r) + \frac{D-1}{r}V'(r) = 0.
\]
Let $W(r):=V'(r)$. Then $W'(r)+(D-1)W(r)/r=0$, giving
\[
\frac{W'(r)}{W(r)} = -\frac{D-1}{r}.
\]
Integrating: $\ln|W(r)|=-(D-1)\ln r + \ln C$, so $W(r)=C r^{1-D}$.

Integrating again:
\[
V(r) = \int C r^{1-D} dr = 
\begin{cases}
C\ln r + C_0, & D=2, \\
\frac{C}{2-D} r^{2-D} + C_0, & D\ge 3.
\end{cases}
\]

\ADD{Choosing the constant so that $F=-\nabla V$ is inward (attractive), i.e.\ $C>0$,} and dropping the additive constant gives
\[
V_D(r) \propto \begin{cases} \ln r, & D=2, \\ -r^{2-D}, & D\ge 3. \end{cases}
\]

\subsection{Normalization via Flux}

The proportionality constant is fixed by the Green-kernel normalization:
\[
\int_{\partial B_r(x_0)} \partial_n V\, dS = -c\ADD{,}
\]
which encodes the source strength in the distributional equation $\Delta V = -c\,\delta_{x_0}$.

\section{RG Conditions for Alexander Duality}\label{app:rg_duality}

We establish minimal Recognition Geometry hypotheses ensuring that the quotient $\CR$ possesses sufficient regularity for Alexander duality to apply.

\begin{proposition}[RG Conditions for Duality]
Let $(\C,\E,R)$ be a recognition geometry with locality structure $\mathcal{N}$. If:
\begin{enumerate}
    \item The topology $\tau_{\mathcal{N}}$ on $\C$ is locally contractible,
    \item The quotient map $\pi_R:(\C,\tau_{\mathcal{N}})\to(\CR,\tau_R)$ is closed with contractible fibers,
    \item The quotient topology $\tau_R$ makes $\CR$ Hausdorff and second-countable,
    \item Local contractions descend to the quotient,
\end{enumerate}
then $\CR$ is locally contractible, enabling Alexander duality for embedded submanifolds.
\end{proposition}

The proof follows standard quotient-topology arguments (see original draft for details).

\section{Explicit Computation: Apsidal Angle for $V(r)=-k/r^n$}\label{app:apsidal}

We provide a complete derivation of the apsidal angle formula for the general power-law potential $V(r)=-k/r^n$ with $n=D-2$, establishing the result used in Theorem~\ref{thm:kepler}. We present two complementary approaches: the time-domain frequency-ratio method and the angle-domain Binet equation. Both yield the same formula, providing a cross-validation of the result.

\subsection{Method 1: Frequency-Ratio Formula (Time-Domain)}

This approach uses the physical frequencies of radial and angular motion.

\begin{proposition}[Frequency-Ratio Formula]\label{prop:freq_ratio}
Assume a stable circular orbit exists at radius $r_0$ in a central potential $V(r)$. Let $\Omega$ be the angular frequency of the circular orbit (radians per unit time) and $\kappa$ the radial small-oscillation frequency (radians per unit time). Define the dimensionless frequency ratio $\omega := \kappa/\Omega$. Then the apsidal angle (Definition~\ref{def:apsidal}) is
\[
\Delta\theta = \frac{2\pi}{\omega}.
\]
\end{proposition}

\begin{proof}
By Lemma~\ref{lem:planar}, the motion is planar with polar coordinates $(r,\theta)$. The Lagrangian is $L = \frac{m}{2}(\dot{r}^2 + r^2\dot{\theta}^2) - V(r)$. Conserved angular momentum gives $\ell = mr^2\dot{\theta}$, so $\dot{\theta} = \ell/(mr^2)$.

Define the effective potential $V_{\mathrm{eff}}(r) = V(r) + \ell^2/(2mr^2)$. The radial equation is $m\ddot{r} = -V'_{\mathrm{eff}}(r)$. A circular orbit at $r=r_0$ satisfies $V'_{\mathrm{eff}}(r_0)=0$.

Perturb $r(t) = r_0 + \rho(t)$ with $|\rho| \ll 1$. Taylor expanding: $V'_{\mathrm{eff}}(r_0+\rho) = V''_{\mathrm{eff}}(r_0)\rho + O(\rho^2)$. The linearized radial equation is $m\ddot{\rho} + V''_{\mathrm{eff}}(r_0)\rho = 0$, giving radial frequency $\kappa = \sqrt{V''_{\mathrm{eff}}(r_0)/m}$ and radial period $T_r = 2\pi/\kappa$.

On the circular orbit, $\dot{\theta} = \Omega := \ell/(mr_0^2)$ is constant. The angular advance during one full radial period (peri $\to$ apo $\to$ peri) is
\[
\Delta\theta = \Omega \cdot T_r = \frac{\ell}{mr_0^2} \cdot \frac{2\pi}{\kappa} = \frac{2\pi}{\kappa/\Omega} = \frac{2\pi}{\omega}.
\]
The factor $2\pi$ arises because $T_r$ is the full period of radial oscillation, corresponding to peri-to-peri as defined in Definition~\ref{def:apsidal}.
\end{proof}

\ADD{\begin{remark}[Consistency of $\omega$ across methods]
The dimensionless frequency ratio $\omega:=\kappa/\Omega$ appearing in Method~1 coincides with
$\omega=\sqrt{2-n}=\sqrt{4-D}$ obtained from the Binet linearization in Method~2.
Both denote the same physical quantity, the number of radial oscillations per orbit, expressed
in time-domain and angle-domain variables respectively.
\end{remark}}

\subsection{Method 2: Binet Equation (Angle-Domain)}

This approach works directly with the angle $\theta$ as independent variable, providing geometric insight.

\begin{proposition}[Binet Form and Linearization for Inverse-Power Forces]\label{prop:binet}
Consider a central force $F(r)$ in the invariant plane (Lemma~\ref{lem:planar}), with $u(\theta) := 1/r(\theta)$. The orbit equation is
\[
u'' + u = -\frac{m}{\ell^2 u^2}\,F(1/u),
\]
where primes denote derivatives with respect to $\theta$. If $F(r) = -kn/r^{n+1}$ (attractive inverse-power law), then
\[
u'' + u = \alpha u^{n-1}, \quad\text{where}\quad \alpha := \frac{mkn}{\ell^2}.
\]
A circular orbit $u(\theta) \equiv u_0$ satisfies $\alpha = u_0^{2-n}$, and linearization $u = u_0 + \delta$ yields
\[
\delta'' + (2-n)\delta = 0.
\]
\end{proposition}

\begin{proof}
From $\ell = mr^2\dot{\theta}$ and $r = 1/u$, we have $\dot{\theta} = \ell u^2/m$. Differentiating $r$ with respect to time using the chain rule:
\[
\dot{r} = \frac{dr}{d\theta}\dot{\theta} = -\frac{u'}{u^2} \cdot \frac{\ell u^2}{m} = -\frac{\ell}{m}u'.
\]
Differentiating again:
\[
\ddot{r} = -\frac{\ell}{m}\frac{du'}{dt} = -\frac{\ell}{m}u'' \cdot \frac{\ell u^2}{m} = -\frac{\ell^2}{m^2}u''u^2.
\]
The radial equation $m(\ddot{r} - r\dot{\theta}^2) = F(r)$ becomes, using $r\dot{\theta}^2 = u^{-1}(\ell u^2/m)^2 = \ell^2 u^3/m^2$:
\[
m\left(-\frac{\ell^2}{m^2}u''u^2 - \frac{\ell^2}{m^2}u^3\right) = F(1/u).
\]
Multiplying by $-m/(\ell^2 u^2)$ gives the Binet equation.

For $F(r) = -kn/r^{n+1}$, we have $F(1/u) = -kn u^{n+1}$, so
\[
u'' + u = \frac{mkn}{\ell^2}u^{n-1} = \alpha u^{n-1}.
\]

A circular orbit requires $u'' = 0$ and $u = u_0$ constant, so $u_0 = \alpha u_0^{n-1}$, yielding $\alpha = u_0^{2-n}$.

Linearizing $u = u_0 + \delta$ with $|\delta| \ll 1$ and using $(u_0+\delta)^{n-1} = u_0^{n-1} + (n-1)u_0^{n-2}\delta + O(\delta^2)$:
\[
\delta'' + u_0 + \delta = \alpha\bigl(u_0^{n-1} + (n-1)u_0^{n-2}\delta\bigr) + O(\delta^2).
\]
Using $\alpha u_0^{n-1} = u_0$ and $\alpha u_0^{n-2} = u_0^{2-n}u_0^{n-2} = 1$:
\[
\delta'' + \delta = (n-1)\delta \quad\Longrightarrow\quad \delta'' + (2-n)\delta = 0.
\]
\end{proof}

\begin{corollary}[Apsidal Angle from Binet Linearization]\label{cor:apsidal_from_binet}
If $\delta'' + \omega^2\delta = 0$ with $\omega = \sqrt{2-n}$ (valid for $n<2$, i.e., $D<4$), then $\delta(\theta) = A\cos(\omega\theta) + B\sin(\omega\theta)$ has period $2\pi/\omega$ in the angle $\theta$. Periapsis occurs at maxima of $u(\theta) = u_0 + \delta(\theta)$ (minima of $r(\theta)$), so successive periapsides are separated by
\[
\Delta\theta = \frac{2\pi}{\omega} = \frac{2\pi}{\sqrt{2-n}}.
\]
\end{corollary}

\begin{proof}
The general solution $\delta(\theta) = A\cos(\omega\theta) + B\sin(\omega\theta)$ is periodic with period $2\pi/\omega$ in $\theta$. Maxima of $u$ (hence minima of $r$, i.e., periapsides) repeat once per period.
\end{proof}

\subsection{Specialization to $D$-Dimensional Green-Kernel Dynamics}

For the Green-kernel potential in $D$ dimensions ($D\ge 3$), we have $V_D(r) \propto -1/r^{D-2}$, corresponding to exponent $n = D-2$.

\begin{corollary}[Kepler Precession Formula in Terms of $D$]\label{cor:kepler_D}
Let $D\ge 3$ and $V_D(r) \propto -1/r^{D-2}$. In the stable regime $D<4$, the apsidal angle (Definition~\ref{def:apsidal}) is
\[
\Delta\theta = \frac{2\pi}{\sqrt{4-D}}.
\]
Non-precession (Definition~\ref{def:apsidal}: $\Delta\theta = 2\pi$) requires $\sqrt{4-D} = 1$, forcing $D=3$.
\end{corollary}

\begin{proof}
Substituting $n = D-2$ into Corollary~\ref{cor:apsidal_from_binet}:
\[
\Delta\theta = \frac{2\pi}{\sqrt{2-n}} = \frac{2\pi}{\sqrt{2-(D-2)}} = \frac{2\pi}{\sqrt{4-D}}.
\]
For no precession, $\Delta\theta = 2\pi$ requires $\sqrt{4-D} = 1$, so $4-D = 1$ and $D=3$.
\end{proof}

\textbf{Consistency check:} Both derivation methods yield the identical formula $\Delta\theta = 2\pi/\omega$ with $\omega = \sqrt{4-D}$, confirming that the peri-to-peri convention (Definition~\ref{def:apsidal}) is maintained throughout. The non-precession condition $\Delta\theta = 2\pi$ (one full rotation per radial cycle) uniquely selects $D=3$, as stated in Theorem~\ref{thm:kepler}.

\end{document}
