\documentclass[11pt]{article}

\usepackage[margin=1in]{geometry}
\usepackage[T1]{fontenc}
\usepackage[utf8]{inputenc}
\usepackage{lmodern}
\usepackage{microtype}
\usepackage{amsmath,amssymb,amsthm,mathtools}
\usepackage[colorlinks=true,linkcolor=blue,citecolor=blue,urlcolor=blue]{hyperref}
\usepackage{booktabs}
\usepackage{enumitem}
\usepackage{xcolor}
\usepackage{longtable}
\setlist{nosep}

\newcommand{\OLD}[1]{\textcolor{red}{#1}}
\newcommand{\NEW}[1]{\textcolor{teal}{#1}}
\newcommand{\fileloc}[1]{\texttt{\small #1}}

\title{Revision Notes for\\[4pt]
\emph{Dimensional Rigidity as a Selection Principle\\in Recognition Geometry}\\[6pt]
{\normalsize Changes applied to \texttt{Feb\_10\_revised\_version.tex}}}
\author{Prepared for internal review}
\date{\today}

\begin{document}
\maketitle

\section*{Overview}

This document catalogs every change made to the manuscript
\emph{Dimensional Rigidity as a Selection Principle in Recognition Geometry}
in the February 13, 2026 revision.  Changes fall into three categories:

\begin{enumerate}[label=\textbf{\Roman*.}]
  \item \textbf{Error corrections} (comments 1--9 from the Version-3 Comment-1 review),
  \item \textbf{Published-paper integration} (importing the now-accepted Axioms paper),
  \item \textbf{Formatting and style} (equation punctuation, author order, Unicode fix).
\end{enumerate}

\noindent
All additions appear in \NEW{teal} in the revised manuscript PDF so that
co-authors and reviewers can locate them instantly.

\bigskip
\hrule
\bigskip

%======================================================================
\section{Error Corrections (from v3\_comment\_1 review)}
%======================================================================

%----------------------------------------------------------------------
\subsection{Change 1: Author order --- alphabetical by last name}

\noindent\textbf{Location:} Title page, \verb|\author{...}|.

\medskip\noindent\textbf{Problem:}
The author list was ordered Washburn, Pardo-Guerra, Thapa.
Comment~12 of the review requests alphabetical ordering by last name.

\medskip\noindent\textbf{Before:}
\begin{quote}\OLD{Jonathan Washburn, Sebastian Pardo-Guerra, Anil Thapa}\end{quote}

\noindent\textbf{After:}
\begin{quote}\NEW{Sebastian Pardo-Guerra, Anil Thapa, Jonathan Washburn}\end{quote}

%----------------------------------------------------------------------
\subsection{Change 2: Replace ``manifold-like'' with precise definition reference}

\noindent\textbf{Location:} Theorem~1.2 (Main Theorem statement in the Introduction), and Theorem~6.1 (full statement in Section~6).

\medskip\noindent\textbf{Problem:}
The phrase ``Assume $\mathcal{C}_R$ is manifold-like'' was never defined as a standalone term.
The paper already introduces the notion of an ``effective manifold model $\mathcal{M}$'' in Definition~2.5 (formerly 2.12),
but Theorem~1.2 did not reference it.

\medskip\noindent\textbf{Before:}
\begin{quote}\OLD{Assume $\mathcal{C}_R$ is manifold-like and admits enough structure\ldots}\end{quote}

\noindent\textbf{After:}
\begin{quote}\NEW{Assume $(\mathcal{C},\mathcal{E},R)$ admits an effective manifold model $\mathcal{M}$
in the sense of Definition~2.5} and admits enough structure\ldots\end{quote}

\medskip\noindent\textbf{Why:} Makes the paper internally consistent with the two-scale story (finite-resolution quotients $\to$ effective manifold limit).

%----------------------------------------------------------------------
\subsection{Change 3: Freedman exotic $\mathbb{R}^4$ --- corrected attribution}

\noindent\textbf{Location:} Section~1.1 (Prior Approaches), paragraph on pure mathematics.

\medskip\noindent\textbf{Problem:}
The original text stated ``Freedman's exotic $\mathbb{R}^4$ theorem shows\ldots $\mathbb{R}^4$ admits uncountably many distinct smooth structures.''
This is an oversimplification.  Freedman's theorem is about \emph{topological} 4-manifolds
(homeomorphism classification).  The exotic smooth structures on $\mathbb{R}^4$ follow from combining
Freedman's work with Donaldson's gauge-theoretic invariants and later results of Taubes.

\medskip\noindent\textbf{After:}
\begin{quote}\NEW{Freedman~[6] classified simply connected closed topological 4-manifolds via
their unimodular intersection forms. Combined with Donaldson's smooth rigidity results
(and later work), this implies that $\mathbb{R}^4$ admits uncountably many
\emph{exotic} smooth structures---a phenomenon unique to dimension~4.}\end{quote}

%----------------------------------------------------------------------
\subsection{Change 4: ``Knot theory only in $D=3,4$'' --- corrected}

\noindent\textbf{Location:} Same paragraph as Change~3.

\medskip\noindent\textbf{Problem:}
The claim ``Knot theory is nontrivial only in dimensions $D=3,4$'' is false.
The very next clause (``surfaces link in $D=5$'') contradicts it.
Higher-dimensional knot theory (codimension-2 sphere knots, etc.)\ is a well-established field.

\medskip\noindent\textbf{After:}
\begin{quote}\NEW{Classical knot theory of embeddings $S^1\hookrightarrow\mathbb{R}^3$ is special;
in higher dimensions the behavior changes dramatically, though higher-dimensional knot theory
(e.g.\ codimension-2 sphere knots) exists and is non-trivial.}\end{quote}

%----------------------------------------------------------------------
\subsection{Change 5: Chiral anomalies statement --- made precise}

\noindent\textbf{Location:} Same paragraph as Changes~3--4.

\medskip\noindent\textbf{Problem:}
The original text ``chiral anomalies vanishing only in specific dimensions (e.g., $D=2,6,10$\ldots)''
was vague and likely incorrect as stated.

\medskip\noindent\textbf{After:}
\begin{quote}\NEW{In quantum field theory, anomaly cancellation in gauge theories imposes dimensional
constraints; for instance, gravitational and gauge anomalies cancel in $D=10$ for the superstring,
and analogous constraints appear in lower-dimensional models.}\end{quote}

%----------------------------------------------------------------------
\subsection{Change 6: Boundary/intersection identity in linking proof --- fixed}

\noindent\textbf{Location:} Proof of Theorem~3.1, Step~(3) (Independence of choice of $W$).

\medskip\noindent\textbf{Problem:}
The proof wrote:
\[
\OLD{(\partial Q)\cdot B \;=\; Q\cdot(\partial B).}
\]
This is \emph{not} the correct boundary/intersection compatibility identity.
The correct Leibniz-type identity has an extra $\partial(Q\pitchfork B)$ term.
The chain-level argument was therefore invalid.

\medskip\noindent\textbf{After:}
The argument is replaced with a standard \emph{homology-level} argument that avoids
chain-level sign complications entirely:
\begin{quote}\NEW{In an oriented closed $D$-manifold, the intersection number $Z\cdot B$ depends
only on the homology class $[Z]\in H_{p+1}(\mathcal{C}_R;\mathbb{Z})$.
Since $H_{p+1}(\mathcal{C}_R;\mathbb{Z})=0$ by hypothesis and $Z$ is a cycle, we have $[Z]=0$,
hence $Z\cdot B=0$.}\end{quote}

\noindent\textbf{Why this is correct:}
Intersection numbers are bilinear pairings on homology (not chains).
Since $Z=W-W'$ is a cycle in a trivial homology group, it is null-homologous,
so its intersection number with any cycle vanishes.
This is a standard technique (see Rolfsen, \emph{Knots and Links}, 1976).

%----------------------------------------------------------------------
\subsection{Change 7: $p=0$ listing and codimension-2 remark --- fixed}

\noindent\textbf{Location:} Proof of Proposition~3.2 and Remark~3.3.

\medskip\noindent\textbf{Problem (a):}
The proof listed ``For $p=0$ (points), $D=1$'' as a valid case.
But Theorem~3.1 assumes $0<p<D$, so $p=0$ falls outside its domain.

\medskip\noindent\textbf{Fix:}
The proof now restricts to \NEW{$p\ge 1$} (consistent with the theorem hypothesis),
and the allowed set is written consistently as \NEW{$\mathcal{A}_A=\{3,5,7,\dots\}$}
throughout the manuscript.

\medskip\noindent\textbf{Problem (b):}
Remark~3.3 called objects ``codimension-2 defects'' but the codimension is
$(D+1)/2$, which equals~2 only when $D=3$.  The remark was circular.

\medskip\noindent\textbf{Fix:}
Rewritten to state clearly that \NEW{codimension-2 is specific to $D=3$};
the general same-dimension linking constraint forces odd $D$ via the formula
$D-p=(D+1)/2$, without singling out codimension~2 for arbitrary $D$.

%----------------------------------------------------------------------
\subsection{Change 8: Green-kernel sign convention --- fixed}

\noindent\textbf{Location:} Appendix~A (Detailed Derivation of Green-Kernel Potentials).

\medskip\noindent\textbf{Problem:}
The appendix wrote ``Choosing $C<0$ for an attractive potential.''
But in the main text, $V_2(r)=k\ln r$ with $k>0$ is attractive
(giving $F=-k/r$ inward).  For $V(r)=C\ln r$, attraction requires
$F=-V'=-C/r$ inward, i.e.\ $C>0$.

\medskip\noindent\textbf{After:}
\begin{quote}\NEW{Choosing the constant so that $F=-\nabla V$ is inward (attractive), i.e.\ $C>0$,}
and dropping the additive constant\ldots\end{quote}

%----------------------------------------------------------------------
\subsection{Change 9: $SO(D)$ --- local frame rotations, not global isometry group}

\noindent\textbf{Location:} Section~5 (Constraint~C), Proposition~5.1, and surrounding text.

\medskip\noindent\textbf{Problem:}
The paper spoke of ``the rotation group $SO(D)$'' as if it were the global isometry group
of $\mathcal{M}$.  A generic manifold need not have $SO(D)$ as its isometry group.
What is meant is the \emph{structure group of the oriented orthonormal frame bundle}.

\medskip\noindent\textbf{After:}
Changed to \NEW{``the local orthonormal frame rotation group $SO(D)$''} in both the
proposition statement and surrounding discussion.

%----------------------------------------------------------------------
\subsection{Change 10: $\omega$ consistency remark in appendix}

\noindent\textbf{Location:} Appendix~C, between Method~1 and Method~2.

\medskip\noindent\textbf{Problem:}
The symbol $\omega$ is defined as $\kappa/\Omega$ in Method~1 (time-domain)
and as $\sqrt{2-n}=\sqrt{4-D}$ in Method~2 (Binet angle-domain).
These are the same physical quantity, but a reader might worry about a name collision.

\medskip\noindent\textbf{After:}
A new \NEW{Remark (Consistency of $\omega$ across methods)} is inserted,
confirming the two definitions agree.

%======================================================================
\section{Published-Paper Integration (Axioms paper)}
%======================================================================

Now that \emph{Reciprocal Convex Costs for Ratio Matching: Axiomatic Characterization}
(Washburn \& Rahnamai Barghi, \emph{Axioms}~2026;
\texttt{doi:10.3390/axioms1010000}) is accepted and published,
we import its main result to anchor the cost functional in a
\emph{peer-reviewed, published theorem} rather than re-deriving it.

%----------------------------------------------------------------------
\subsection{Change 11: Introduction bridge paragraph}

\noindent\textbf{Location:} End of the Introduction preamble (after ``\ldots fundamentally new approach to this ancient question'').

\medskip\noindent\textbf{Added text:}
\begin{quote}\NEW{This paper builds on the axiomatic characterization of ratio-induced mismatch costs
established in~[WashburnRahnamaiBarghi2026].
There it was shown that the assumptions of inversion symmetry,
strict convexity, coercivity, and a multiplicative d'Alembert compatibility identity uniquely
force $J(x)=\tfrac{1}{2}(x^{a}+x^{-a})-1$ for some $a>0$ (with $a$ absorbable into the scale maps).
We take this cost-kernel result as given and focus on the downstream topological, dynamical, and
geometric consequences that determine spatial dimension.}\end{quote}

\noindent\textbf{Why:}
Creates a visible dependency spine between publications.
Reduces reviewer friction on ``why this cost?'' by pointing to published peer-reviewed proof.

%----------------------------------------------------------------------
\subsection{Change 12: Imported Proposition~2.5 (cost-kernel uniqueness)}

\noindent\textbf{Location:} Section~2 Preliminaries, new subsection ``Imported Cost-Kernel Characterization''
inserted before the Composite Recognizers subsection.

\medskip\noindent\textbf{Added:}
\begin{quote}
\NEW{\textbf{Proposition 2.5} (Unique Mismatch Penalty; Washburn--Rahnamai Barghi [2026]).}
\NEW{Let $J:(0,\infty)\to[0,\infty)$ satisfy
(i)~inversion symmetry $J(x)=J(1/x)$,
(ii)~strict convexity,
(iii)~normalization $J(1)=0$,
(iv)~coercivity, and
(v)~the multiplicative d'Alembert identity
$(1+J(xy))+(1+J(x/y))=2(1+J(x))(1+J(y))$.
Then there exists $a>0$ such that $J(x)=\cosh(a\log x)-1$.
The parameter~$a$ is absorbed by rescaling the scale maps,
yielding $J(x)=\tfrac{1}{2}(x+x^{-1})-1$ without loss of generality.}
\end{quote}

\noindent Followed by a \NEW{scope sentence}: ``The novelty of the present work lies in the geometric and topological
consequences of this cost kernel---specifically the forcing of $D=3$
spatial dimensions via linking constraints---rather than in the derivation
of~$J$ itself.''

\medskip\noindent\textbf{Why:}
\begin{itemize}
  \item Anchors the cost functional in a published theorem (not an assumption).
  \item Makes notation consistent across the two papers ($\iota_S, \iota_O, J$).
  \item Lets reviewers evaluate only the \emph{new} contribution (topology/dimension/stability).
  \item Builds a visible publication chain: cost law $\to$ structural consequences $\to$ physics.
\end{itemize}

%----------------------------------------------------------------------
\subsection{Change 13: Bibliography entry}

\noindent\textbf{Location:} Bibliography.

\medskip\noindent\textbf{Added:}
\begin{quote}\NEW{\texttt{[WashburnRahnamaiBarghi2026]}
J.~Washburn and A.~Rahnamai Barghi,
\emph{Reciprocal Convex Costs for Ratio Matching: Axiomatic Characterization},
Axioms (2026).
\texttt{doi:10.3390/axioms1010000}.}\end{quote}

%======================================================================
\section{Formatting and Style}
%======================================================================

%----------------------------------------------------------------------
\subsection{Change 14: Equation punctuation}

\noindent\textbf{Location:} Multiple display equations throughout the paper.

\medskip\noindent\textbf{Rule applied:}
Every display equation must end with a comma or period, matching the surrounding sentence grammar.
Terminal punctuation was added (in teal) to the following equations:
\begin{itemize}
  \item Intersection dimension formula (Lemma~3.1),
  \item Effective potential definition (Theorem~4.1),
  \item Circular orbit condition (Theorem~4.1, Step~1),
  \item Synchronization period definition (Definition~7.2),
  \item Green-kernel flux normalization (Appendix~A).
\end{itemize}

%----------------------------------------------------------------------
\subsection{Change 15: Unicode character fix}

\noindent\textbf{Location:} Remark title in Section~4.

\medskip\noindent\textbf{Problem:}
The remark title contained a raw Unicode $\ge$ character (\texttt{U+2265})
which caused a \LaTeX\ error.

\medskip\noindent\textbf{Fix:}
Replaced with math-mode \verb|$D\ge 3$| inside the remark's optional argument.

%======================================================================
\section{Post-Review Consistency Corrections (This Pass)}
%======================================================================

After a strict second-pass audit focused only on edited material, four additional
consistency fixes were applied.

\subsection*{Change 16: Correct imported Axioms-paper bibliographic metadata}

\noindent\textbf{Issue:} The prior revision notes listed the wrong DOI/issue metadata for the
imported cost-kernel paper.

\noindent\textbf{Fix applied:} Updated the manuscript citation entry to:
\NEW{\texttt{doi:10.3390/axioms1010000}} and removed the incorrect
\NEW{Axioms 15(2), 90} metadata from this citation.

\subsection*{Change 17: Harmonize the linking allowed set with $p\ge 1$}

\noindent\textbf{Issue:} Some formulas still displayed
\OLD{$\mathcal{A}_A=\{1,3,5,\dots\}$} even after restricting the theorem domain to $p\ge 1$.

\noindent\textbf{Fix applied:} Replaced these with
\NEW{$\mathcal{A}_A=\{3,5,7,\dots\}$} in the Introduction, theorem summaries,
Section~3 statements, the synthesis theorem, Table~1, and the Conclusion.

\subsection*{Change 18: Align Theorem~1.2 with the two-scale formulation}

\noindent\textbf{Issue:} The edited theorem referenced an effective manifold model $\M$ but still
applied constraints directly to $\mathcal{C}_R$ in the same sentence.

\noindent\textbf{Fix applied:} The theorem now states constraints on \NEW{$\M$}, concludes
\NEW{$\dim(\M)=3$}, and explicitly notes this is equivalent to recognition dimension~3 for
$\mathcal{C}_R$.

\subsection*{Change 19: Strengthen Section~5 rotational footnote}

\noindent\textbf{Issue:} One footnote still loosely suggested a global $SO(D)$ action.

\noindent\textbf{Fix applied:} Replaced with a local frame-bundle formulation:
\NEW{oriented orthonormal frames form a principal $SO(D)$-bundle over $\M$}.

%======================================================================
\section{Items Noted but Not Yet Applied}
%======================================================================

The following items from v3\_comment\_1 are noted for a future pass.
They require a full-text audit rather than targeted insertions:

\begin{enumerate}
  \item \textbf{Equation numbers:} Verify that \emph{every} numbered display equation in the
    paper has a label and that no important equation is left unnumbered
    (currently many use \verb|\[...\]| instead of \verb|\begin{equation}|).
  \item \textbf{QED boxes:} The review requests removing the tombstone ($\square$) at the
    end of every proof.  The current file uses \texttt{amsthm}'s default, which
    inserts $\square$ automatically.  Suppressing it requires adding
    \verb|\renewcommand{\qedsymbol}{}| in the preamble---\emph{confirm with co-authors
    before applying}, as some journals require the symbol.
  \item \textbf{Equation-ending punctuation (exhaustive pass):} The targeted punctuation
    changes above cover the most prominent cases; a line-by-line audit of all
    $\sim\!80$ display equations is recommended before final submission.
\end{enumerate}

%======================================================================
\section{Summary Table}
%======================================================================

\begin{longtable}{@{}clp{7.5cm}@{}}
\toprule
\textbf{\#} & \textbf{Type} & \textbf{Description} \\
\midrule
\endhead
1  & Fix    & Author order $\to$ alphabetical by last name \\
2  & Fix    & ``manifold-like'' $\to$ reference Definition~2.5 \\
3  & Fix    & Freedman attribution corrected \\
4  & Fix    & ``Knot theory only in $D\!=\!3,4$'' rewritten \\
5  & Fix    & Chiral anomalies statement made precise \\
6  & Fix    & Boundary compatibility $\to$ homology-level argument \\
7  & Fix    & $p\!\ge\! 1$ restriction; codimension-2 remark reframed \\
8  & Fix    & Green-kernel sign: $C\!>\!0$ (not $C\!<\!0$) \\
9  & Fix    & $SO(D)$ $\to$ local orthonormal frame rotation group \\
10 & Fix    & $\omega$ consistency remark added in appendix \\
11 & Import & Intro bridge paragraph citing Axioms paper \\
12 & Import & Proposition~2.5: imported cost-kernel uniqueness theorem \\
13 & Import & Bibliography entry for Axioms paper \\
14 & Style  & Terminal punctuation on key display equations \\
15 & Style  & Unicode $\ge$ $\to$ \verb|$\ge$| in remark title \\
16 & Fix    & Corrected imported Axioms citation metadata (DOI and issue data) \\
17 & Fix    & Harmonized $\mathcal{A}_A$ to $\{3,5,7,\dots\}$ for $p\ge 1$ \\
18 & Fix    & Theorem~1.2 now states constraints on effective manifold $\M$ consistently \\
19 & Fix    & Section~5 footnote rewritten in frame-bundle language \\
\bottomrule
\end{longtable}

\end{document}
