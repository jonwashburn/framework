\documentclass[12pt]{article}
\usepackage[margin=1in]{geometry}
\usepackage{amsmath,amssymb,amsthm}
\usepackage{graphicx}
\usepackage{enumitem}
\usepackage{array}
\usepackage{hyperref}

% Simple page style
\pagestyle{plain}

\newtheorem{theorem}{Theorem}
\newtheorem{lemma}[theorem]{Lemma}
\newtheorem{definition}{Definition}
\newtheorem{corollary}[theorem]{Corollary}

\begin{document}

\begin{center}
\textbf{\LARGE PATENT APPLICATION}\\[0.5cm]
\textbf{\Large Method and System for Formal Verification Bridge\\in Nuclear Fusion Control Systems}\\[1cm]

\begin{tabular}{rl}
\textbf{Application Type:} & Utility Patent \\
\textbf{Filing Date:} & January 25, 2026 \\
\textbf{Inventor:} & Jonathan Washburn \\
\textbf{Technology Field:} & Fusion Energy / Regulatory Compliance / Formal Verification \\
\textbf{International Class:} & G21B 1/00; G06F 21/57; G05B 9/02 \\
\end{tabular}
\end{center}

\vspace{1cm}
\hrule
\vspace{0.5cm}

\section*{ABSTRACT}

A method and system for bridging the gap between formal mathematical proofs and empirical physical measurements in safety-critical fusion control systems. The invention introduces a ``Formal Verification Bridge'' that explicitly isolates the empirical assumptions required to connect a certified control logic (e.g., a ledger-based controller) to physical reality (e.g., diagnostic sensors). The system defines a ``Traceability Hypothesis'' data structure that encapsulates a calibration envelope (scale, offset, and related validity conditions) as a formal hypothesis predicate. This hypothesis is used in a machine-checked proof (e.g., in Lean 4) that guarantees a stated bound on a physical observable \textit{conditional} on the hypothesis holding. At runtime, the system emits an auditable artifact that records the active hypothesis identifier (e.g., a Lean predicate name), relevant calibration metadata, and a cryptographic digest (e.g., SHA-256 over canonicalized inputs), enabling an auditor to determine which empirical envelope was assumed for each control action. Optional external signing may be applied as an integration step when non-repudiation is required.

\vspace{0.5cm}
\hrule
\vspace{0.5cm}

\section{BACKGROUND OF THE INVENTION}

\subsection{Technical Field}

This invention relates generally to the verification of control systems for nuclear fusion reactors, and specifically to methods for integrating formal mathematical proofs with empirical sensor data in a regulatory compliance framework.

\subsection{Description of Related Art}

Safety-critical systems (avionics, nuclear power) increasingly use formal methods to prove that software behaves correctly. However, a fundamental gap exists: mathematical proofs operate on abstract symbols, while physical systems operate on noisy, analog signals.

In fusion control, a theorem might prove that ``minimizing the Symmetry Ledger $L$ improves implosion symmetry.'' However, the controller cannot measure $L$ directly; it measures voltages from a camera. The translation from voltage to $L$ involves empirical calibration.

Current approaches typically hard-code this translation, breaking the chain of formal verification. If the calibration is wrong, the proof is irrelevant. There is no standard mechanism to ``carry'' the empirical assumption into the proof environment so that the safety guarantee is explicitly conditional on the calibration validity.

There is a need for a mechanism that bridges this gap, allowing formal proofs to reason about empirical uncertainty in a structured, auditable way.

\section{SUMMARY OF THE INVENTION}

The present invention provides a \textbf{Formal Verification Bridge} that connects empirical diagnostics to certified control logic.

The core innovation is the \textbf{Traceability Hypothesis}, a formal structure that:
\begin{enumerate}
    \item \textbf{Encapsulates the Seam:} It defines the mathematical relationship between the physical observable (e.g., ``observable asymmetry'') and the formal state (e.g., ``certified ledger'').
    \item \textbf{Quantifies the Envelope:} It includes parameters for scale, offset, and uncertainty bounds (e.g., ``observable $\le$ ledger / scale + offset'').
    \item \textbf{Enables Conditional Proof:} It serves as a hypothesis in a formal theorem (e.g., ``Theorem: IF the Traceability Hypothesis holds, THEN the controller guarantees safety'').
\end{enumerate}

The system implements this bridge by:
\begin{itemize}
    \item Defining the hypothesis as a type in a theorem prover (e.g., Lean 4).
    \item Requiring the facility operator to provide the envelope parameters (scale, offset) based on validation shots.
    \item Recording, in the runtime audit artifact, an explicit reference to the hypothesis and the calibration envelope (and optionally the numeric parameters when provided).
    \item Proving a ``Traceability Theorem'' that links the runtime certificate's PASS status to a physical bound on asymmetry.
\end{itemize}

This architecture ensures that the safety case is complete: the logic is proven, and the assumptions are logged and auditable.

\section{BRIEF DESCRIPTION OF THE DRAWINGS}

\begin{itemize}
    \item \textbf{FIG. 1} illustrates the conceptual bridge between the Physical Domain and the Formal Domain.
    \item \textbf{FIG. 2} shows the structure of the Traceability Hypothesis data object.
    \item \textbf{FIG. 3} depicts the workflow for generating a certificate using the bridge.
\end{itemize}

\section{DETAILED DESCRIPTION OF EMBODIMENTS}

\subsection{Definitions}

\begin{itemize}
    \item \textbf{Formal Domain:} The realm of mathematical logic and machine-checked proofs (e.g., Lean 4 code).
    \item \textbf{Physical Domain:} The realm of sensors, actuators, and plasma physics.
    \item \textbf{Traceability Hypothesis:} A formal definition of the relationship between a physical quantity and its formal representation, including error bounds.
    \item \textbf{Calibration Envelope:} The set of parameters (scale, offset) that define the valid range of the Traceability Hypothesis.
\end{itemize}

\subsection{The Traceability Hypothesis Structure}

In the preferred embodiment (Lean 4), the hypothesis is defined as:

\begin{verbatim}
structure TraceabilityHypothesis (cfg : BridgeConfig) where
  lower_bound : Real
  lower_bound_pos : 0 < lower_bound
  offset : Real
  offset_nonneg : 0 <= offset
  observable_le : \forall meas,
    observableAsymmetry meas cfg.modes <= diagnosticLedger cfg meas / lower_bound + offset
\end{verbatim}

This structure does three things:
\begin{enumerate}
    \item It declares the parameters \texttt{lower\_bound} and \texttt{offset}.
    \item It defines the semantic meaning of these parameters via the predicate \texttt{observable\_le}.
    \item It forces any proof using this hypothesis to explicitly handle the case where the physical measurement deviates from the ideal ledger by the specified amount.
\end{enumerate}

\subsection{The Traceability Theorem}

The system includes a machine-checked proof of the following theorem:

\textbf{Theorem (Pass Implies Observable Bound):}
Given a configured Bridge and a valid Traceability Hypothesis, if a runtime certificate reports \texttt{passed = true} with a threshold $T$, then the physical observable asymmetry is strictly bounded by $T / \text{lower\_bound} + \text{offset}$.

This theorem is the ``Bridge'': it translates a software boolean (\texttt{passed}) into a guarantee about the physical world, contingent only on the explicitly stated hypothesis.

\subsection{Runtime Integration}

When the fusion reactor operates:
\begin{enumerate}
    \item \textbf{Configuration:} The operator loads the active Calibration Envelope (e.g., ``Scale=1.0, Offset=0.05'').
    \item \textbf{Execution:} The controller computes the Ledger value $L$ and checks $L \le T$.
    \item \textbf{Certification:} The system generates a runtime artifact containing (at least) $L$, $T$, and a pass/fail flag, and records the active calibration/hypothesis identifier used for traceability. In Lean, a \texttt{DiagnosticCertificate} includes the threshold, pass flag, ledger and observable values, and calibration version; the envelope parameters are supplied as a separate hypothesis (\texttt{TraceabilityHypothesis}) used by the traceability theorem.
    \item \textbf{Reference:} The runtime artifact includes a reference to the formal hypothesis predicate (e.g., \texttt{TraceabilityHypothesis.observable\_le}) so an auditor can locate the exact Lean statement whose satisfaction is assumed.
\end{enumerate}

\subsection{Audit and Compliance}

Regulators audit the system by:
\begin{enumerate}
    \item \textbf{Verifying the Proof:} Checking the Lean code to ensure the Traceability Theorem is logically valid.
    \item \textbf{Validating the Envelope:} Reviewing empirical data (calibration shots) to confirm that the parameters (Scale, Offset) logged in the certificates are consistent with reality.
\end{enumerate}
This separates the concern of ``logic correctness'' (math) from ``sensor accuracy'' (physics), simplifying the certification process.

\section{CLAIMS}

\begin{enumerate}
    \item \textbf{A method for verifying safety properties in a physical control system, comprising:}
    \begin{enumerate}
        \item defining a formal data structure representing a hypothesis about the relationship between a physical measurement and a control variable;
        \item populating said data structure with calibration parameters derived from empirical validation;
        \item executing a machine-checked mathematical proof that guarantees a safety property of the system conditional on said hypothesis;
        \item generating a runtime artifact for each control action that includes the calibration parameters and a reference to the machine-checked proof; and
        \item controlling the physical system based on the verified safety property.
    \end{enumerate}

    \item The method of claim 1, wherein the hypothesis defines an inequality bounding the physical measurement by a function of the control variable and an uncertainty offset.

    \item The method of claim 1, wherein the machine-checked proof is verified by an automated theorem prover.

    \item The method of claim 1, wherein the runtime artifact includes a cryptographic digest (e.g., SHA-256) over a canonical representation of the inputs including calibration metadata, and wherein the artifact is optionally externally signed by a facility key.

    \item \textbf{A system for bridging formal verification and empirical control, comprising:}
    \begin{enumerate}
        \item a hypothesis registry storing formal definitions of sensor error models;
        \item a calibration store containing empirical parameters for said error models;
        \item a proof engine configured to verify theorems dependent on said error models; and
        \item a certificate generator configured to produce audit records linking runtime control decisions to the verified theorems and the stored calibration parameters.
    \end{enumerate}

    \item The system of claim 5, wherein the hypothesis registry includes a definition for a traceability hypothesis linking a symmetry ledger value to an observable asymmetry metric.

    \item \textbf{A non-transitory computer-readable medium storing instructions that, when executed by a processor, cause a system to:}
    \begin{enumerate}
        \item load a formal specification of a diagnostic bridge;
        \item instantiate the specification with facility-specific calibration values;
        \item compute a safety certificate based on the instantiated specification; and
        \item inhibit reactor operation if the safety certificate indicates a violation of the formal specification.
    \end{enumerate}
\end{enumerate}

\section*{APPENDIX: Implementation Evidence}

The core logic of this invention is implemented in the accompanying software artifacts:
\begin{itemize}
    \item \textbf{Formal Definition:} The \texttt{TraceabilityHypothesis} structure and the \texttt{pass\_implies\_observable\_bound} theorem are defined in Lean 4 in \texttt{IndisputableMonolith/Fusion/DiagnosticsBridge.lean}.
    \item \textbf{Runtime Logging (seam-first):} The control demos emit a seam note that references the Lean predicate used for the calibration-envelope assumption (e.g., \texttt{TraceabilityHypothesis.observable\_le}), in:
      \texttt{fusion/simulator/control/jag\_demo.py}, \texttt{paper\_modes\_demo.py}, and \texttt{image\_folder\_demo.py}.
\end{itemize}

\end{document}
