\documentclass[12pt]{article}
\usepackage[margin=1in]{geometry}
\usepackage{amsmath,amssymb,amsthm}
\usepackage{graphicx}
\usepackage{enumitem}
\usepackage{array}
\usepackage{hyperref}

% Simple page style
\pagestyle{plain}

\newtheorem{theorem}{Theorem}
\newtheorem{lemma}[theorem]{Lemma}
\newtheorem{definition}{Definition}
\newtheorem{corollary}[theorem]{Corollary}

\begin{document}

\begin{center}
\textbf{\LARGE PATENT APPLICATION}\\[0.5cm]
\textbf{\Large Method and System for Generating Auditable Viability Certificates\\for Nuclear Fusion Ignition Using Solvable Thresholds}\\[1cm]

\begin{tabular}{rl}
\textbf{Application Type:} & Utility Patent \\
\textbf{Filing Date:} & January 25, 2026 \\
\textbf{Inventor:} & Jonathan Washburn \\
\textbf{Technology Field:} & Fusion Energy / Reactor Control / Formal Verification \\
\textbf{International Class:} & G21B 1/00; G05B 19/00; G06F 17/11 \\
\end{tabular}
\end{center}

\vspace{1cm}
\hrule
\vspace{0.5cm}

\section*{ABSTRACT}

A method and system for certifying \textit{model-layer viability} of nuclear fusion operating points under committed, auditable proxy models. The invention introduces a rigorous method for calculating explicit, solvable thresholds for temperature ($T^*$) and reaction rate enhancement ($E^*$) that are sufficient to guarantee a positive net-power inequality in a specified loss/heating proxy model. By committing to an explicit model of bremsstrahlung and transport losses and an explicit deposited-heating proxy, the system derives a conservative sufficient condition: if $T \ge T^*$ and the enhancement factor $E \ge E^*$ (with required domain conditions such as $n>0$ and positive deposition/heating coefficients), then the formal viability inequality holds in the committed proxy model. The system generates an auditable certificate bundle for each operating point, including a cryptographic input digest (e.g., SHA-256 over canonicalized inputs), computed thresholds, pass/fail status, and references to machine-checked theorems (e.g., in Lean 4) proving the stated sufficiency result. This provides a deterministic mechanism for gating high-power operations \textit{under an explicit seam-labeled model}, without claiming facility-independent physical ignition.

\vspace{0.5cm}
\hrule
\vspace{0.5cm}

\section{BACKGROUND OF THE INVENTION}

\subsection{Technical Field}

This invention relates generally to the control and certification of nuclear fusion reactors, and specifically to methods for determining and verifying the conditions required for self-sustaining fusion burn (ignition).

\subsection{Description of Related Art}

Fusion ignition occurs when the energy deposited by fusion products (e.g., alpha particles) exceeds the energy lost to radiation (bremsstrahlung) and transport. The classic Lawson criterion ($n\tau T$) provides a general figure of merit but is often insufficient for real-time control because it relies on simplified assumptions about loss scaling.

In advanced fusion concepts (like Coherence-Controlled Fusion, PF-01), the reaction rate is enhanced by a factor $E$ due to barrier suppression. Operators need to know: ``For a given enhancement $E$, what is the minimum temperature $T$ required for net gain?''

Existing control systems typically use look-up tables or complex numerical simulations (hydrocodes) to answer this. These methods are:
\begin{itemize}
    \item \textbf{Opaque:} The assumptions inside the code are hidden.
    \item \textbf{Unverified:} There is no formal proof that the code's output guarantees ignition.
    \item \textbf{Slow:} Hydrocodes are too computationally expensive for kHz-rate control loops.
\end{itemize}

There is a need for a method that provides \textbf{explicit, solvable thresholds} for ignition that can be computed instantly and are backed by a rigorous mathematical proof.

\section{SUMMARY OF THE INVENTION}

The present invention provides a \textbf{Viability Certification System} that computes guaranteed ignition thresholds based on a formally verified power balance model.

The method comprises:
\begin{enumerate}
    \item \textbf{Model Commitment:} Explicitly defining the loss power $P_{\text{loss}}(T, n, Z_{\text{eff}})$ (bremsstrahlung + transport) and the fusion heating power $P_{\text{heat}}(T, n)$.
    \item \textbf{Threshold Derivation:} Calculating two critical values:
    \begin{itemize}
        \item \textbf{Temperature Threshold ($T^*$):} The temperature above which the Gamow tunneling exponent behaves favorably (specifically, $\eta(T) \le 1$). This is derived from nuclear constants: $T^* = \max(1, \text{gamowCoeff}^2)$.
        \item \textbf{Enhancement Threshold ($E^*$):} The minimum reaction rate multiplier required to overcome losses at the target density. $E^* = \frac{k_{\text{brem}}Z_{\text{eff}} + k_{\text{tr}}/n}{A} + 1$, where $A$ is a fusion constant.
    \end{itemize}
    \item \textbf{Certification:} Evaluating the reactor state $(T, E)$. If $T \ge T^*$ and $E \ge E^*$, the system issues a \textbf{Viability Certificate}.
    \item \textbf{Formal Guarantee:} The certificate references a machine-checked theorem (e.g., \texttt{viable\_of\_T\_ge\_T\_star...}) proving that these conditions are \textit{sufficient} for $P_{\text{heat}} > P_{\text{loss}}$.
\end{enumerate}

This approach replaces heuristic guesses with algebraic guarantees, enabling safe, automated reactor startup sequences.

\section{BRIEF DESCRIPTION OF THE DRAWINGS}

\begin{itemize}
    \item \textbf{FIG. 1} plots the power balance curves $P_{\text{heat}}$ vs $P_{\text{loss}}$ and identifies the ignition point.
    \item \textbf{FIG. 2} shows the viability region in the $(T, E)$ plane defined by $T^*$ and $E^*$.
    \item \textbf{FIG. 3} is a block diagram of the certification module.
\end{itemize}

\section{DETAILED DESCRIPTION OF EMBODIMENTS}

\subsection{Definitions}

\begin{itemize}
    \item \textbf{Viability (model-layer):} The condition where the deposited fusion-heating proxy exceeds the total loss proxy, i.e. $P_{\text{heat}} > P_{\text{loss}}$ under the committed proxy model.
    \item \textbf{Enhancement Factor ($E$):} The ratio of the actual reaction rate to the classical baseline rate (e.g., due to RS barrier scaling).
    \item \textbf{Gamow Coefficient:} A nuclear constant determining the stiffness of the tunneling barrier.
    \item \textbf{Power Balance Parameters:} Coefficients for bremsstrahlung ($k_{\text{brem}}$), transport ($k_{\text{tr}}$), and fusion energy ($k_{\text{fus}}$).
    \item \textbf{Domain conditions:} For the closed-form thresholds used here, the system requires (at minimum) $n>0$, $Z_{\text{eff}}\ge 0$, and positive deposited-heating coefficients (e.g., $f_{\text{dep}}>0$ and $k_{\text{fus}}>0$) so that the enhancement threshold is well-defined.
\end{itemize}

\subsection{The Power Balance Model}

The system commits to the following model (in normalized units):
\[
P_{\text{loss}} = k_{\text{brem}} Z_{\text{eff}} n^2 \sqrt{T} + k_{\text{tr}} n T
\]
\[
P_{\text{heat}} = E \cdot f_{\text{dep}} \cdot k_{\text{fus}} n^2 \cdot \text{proxy}(T)
\]
where the proxy for fusion reactivity is $\text{proxy}(T) = T \exp(-\eta(T))$.

\subsection{Threshold Computation}

The system computes the thresholds algebraically.

\subsubsection{Temperature Threshold $T^*$}
Derived from the condition that the Gamow exponent $\eta(T) \le 1$ to ensure the reactivity proxy behaves monotonically.
\[
T^* = \max(1, (31.3 Z_1 Z_2 \sqrt{\mu})^2)
\]
(Units: keV in this committed proxy convention. The coefficient $31.3$ is an explicit constant of the model-layer Gamow proxy used by the certified sufficiency theorem; its physical interpretation/calibration is a seam if mapped to facility units.)

\subsubsection{Enhancement Threshold $E^*$}
Derived from a conservative lower-bound regime used in the proof (specifically $T\ge 1$ and $\eta(T)\le 1$, which are ensured by $T\ge T^*$). The resulting closed-form $E^*$ is independent of $T$ and requires $n>0$ and a positive baseline heating factor.
\[
E^* = \frac{k_{\text{brem}} Z_{\text{eff}} + k_{\text{tr}}/n}{f_{\text{dep}} k_{\text{fus}} e^{-1}} + 1
\]
This formula guarantees that if the enhancement is at least $E^*$, the fusion curve sits above the loss curve in the relevant regime.

\subsection{Certification Workflow}

\begin{enumerate}
    \item \textbf{Input:} Receive real-time plasma parameters: $T$ (temperature), $n$ (density), $Z_{\text{eff}}$ (impurity), and $E$ (current coherence enhancement).
    \item \textbf{Compute:} Calculate $T^*$ and $E^*$ using the fixed nuclear constants and current density.
    \item \textbf{Check:}
    \begin{itemize}
        \item Is $T \ge T^*$?
        \item Is $E \ge E^*$?
    \end{itemize}
    \item \textbf{Issue:}
    \begin{itemize}
        \item If YES: Generate a \texttt{ViabilityCertificate} with \texttt{passed=true}.
        \item If NO: Generate a certificate with \texttt{passed=false} and the required gap (e.g., ``Need +2 keV'').
    \end{itemize}
\end{enumerate}

\subsection{Formal Verification Link}

The validity of this method relies on the theorem:
\[
(T \ge T^*) \land (E \ge E^*) \implies P_{\text{heat}}(T) > P_{\text{loss}}(T)
\]
This theorem is formally proven in the accompanying Lean 4 codebase (\texttt{IndisputableMonolith/Fusion/ViabilityThresholds.lean}). The runtime certificate records theorem identifiers (references) and an input digest binding the certificate to the evaluated operating point and parameters; optional external signing and/or proof-hash pinning can be added as an integration step if required by a facility.

\section{CLAIMS}

\begin{enumerate}
    \item \textbf{A method for certifying nuclear fusion viability, comprising:}
    \begin{enumerate}
        \item defining a power balance model comprising a loss function and a fusion heating function parameterized by an enhancement factor $E$;
        \item deriving an explicit temperature threshold $T^*$ based on nuclear Gamow coefficients;
        \item deriving an explicit enhancement threshold $E^*$ based on the ratio of loss coefficients to fusion coefficients;
        \item measuring a physical temperature $T$ and a current enhancement factor $E$ of a fusion plasma; and
        \item generating a digital certificate affirming viability if $T \ge T^*$ and $E \ge E^*$.
    \end{enumerate}

    \item The method of claim 1, wherein the temperature threshold is calculated as $T^* = \max(1, C_G^2)$, where $C_G$ is a Gamow coefficient.

    \item The method of claim 1, wherein the enhancement threshold is calculated as $E^* = (L_0 / H_0) + 1$, where $L_0$ represents baseline losses and $H_0$ represents baseline heating.

    \item The method of claim 1, wherein the digital certificate includes a reference to a machine-checked mathematical proof that the conditions $T \ge T^*$ and $E \ge E^*$ are sufficient to guarantee net positive power in the defined model.

    \item \textbf{A control system for a fusion reactor, comprising:}
    \begin{enumerate}
        \item a parameter estimator configured to determine plasma density, temperature, and impurity content;
        \item a threshold engine configured to compute a required minimum temperature and minimum reaction rate enhancement for ignition;
        \item a comparator configured to verify if the current plasma state exceeds said minimums; and
        \item a permissive logic module configured to enable high-power operations only upon successful verification.
    \end{enumerate}

    \item The system of claim 5, wherein the threshold engine implements an algebraic solution derived from a convex bound on the power balance inequality.

    \item \textbf{A non-transitory computer-readable medium storing instructions that, when executed by a processor, cause a system to:}
    \begin{enumerate}
        \item ingest plasma state variables;
        \item calculate sufficient conditions for ignition using a formally verified inequality model;
        \item output a viability status indicating whether the plasma is in a guaranteed burn regime; and
        \item log the viability status and the calculated thresholds to an audit trail.
    \end{enumerate}
\end{enumerate}

\section*{APPENDIX: Implementation Evidence}

The core logic of this invention is implemented in the accompanying software artifacts:
\begin{itemize}
    \item \textbf{Python Implementation:} \texttt{fusion/simulator/fusion/viability\_thresholds.py} implements the \texttt{compute\_thresholds} and \texttt{guaranteed\_viable} functions.
    \item \textbf{Certificate Emission (Python):} \texttt{fusion/simulator/fusion/certificate.py} can emit a certificate bundle for the viability-threshold check, including an input hash and theorem references.
    \item \textbf{Formal Verification:} The sufficiency theorem is proven in Lean 4 in \texttt{IndisputableMonolith/Fusion/ViabilityThresholds.lean}, specifically the theorem \texttt{viable\_of\_T\_ge\_T\_star\_and\_E\_ge\_E\_star}.
\end{itemize}

\end{document}
