\documentclass[11pt,a4paper]{article}
\usepackage[utf8]{inputenc}
\usepackage{amsmath}
\usepackage{amsfonts}
\usepackage{amssymb}
\usepackage{amsthm}
\usepackage{geometry}
\usepackage{hyperref}

\geometry{margin=1in}

\title{\textbf{Logic From Physical Cost: Deriving Consistency as the Ground State of the Recognition Potential}}
\author{Jonathan Washburn \\ \textit{Recognition Science Research Institute}}
\date{\today}

\begin{document}

\maketitle

\begin{abstract}
We demonstrate that the laws of logic are not axiomatic but thermodynamic. In Recognition Science, the cost functional $J(x) = \frac{1}{2}(x + x^{-1}) - 1$ is the unique potential governing reality. This function has a global minimum at $x=1$, where $J(1)=0$. Interpreting $x=1$ as "Identity" or "Consistency," we show that logic emerges because consistency is the zero-cost ground state of the universe, while contradiction ($x \neq 1$) carries a positive energy penalty.
\end{abstract}

\section{Introduction}

In standard formulations of logic, the Law of Identity ($A=A$) is posited as an axiom. In a recognition-first setting, identity at the observable level is supplied operationally: in Recognition Geometry \cite{RecognitionGeometry}, a recognizer induces an equivalence relation of observational indistinguishability and an observable quotient, where identity is equality of equivalence classes. The cost functional $J$ then acts as a selection principle that singles out self-match ($x=1$) as the unique zero-cost fixed point.

\section{Deriving Logic (T0)}

\subsection{Consistency as the Ground State}
We have derived the unique cost function $J(x) = \frac{1}{2}(x + x^{-1}) - 1$. We observe that the global minimum of this function occurs uniquely at $x=1$:
\begin{equation}
    J(1) = \frac{1}{2}(1 + 1) - 1 = 0
\end{equation}
In this framework, the state $x=1$ represents perfect ratio match (self-match) and therefore the operational notion of identity or consistency. The fact that $J(1)=0$ means that self-match is ``free.'' It is the ground state of the ontology.

\subsection{The Cost of Contradiction}
Consider a state of contradiction or inconsistency, represented by any deviation $x \neq 1$. Due to the strict convexity of $J(x)$ on $\mathbb{R}_+$, we have:
\begin{equation}
    \forall x \neq 1, \quad J(x) > 0
\end{equation}
Any deviation from identity incurs a positive cost penalty. For small deviations $x = 1+\epsilon$, the cost rises quadratically ($J \approx \epsilon^2/2$). For large deviations (gross contradictions), the cost grows linearly or exponentially depending on the regime.

Therefore, within this model, identity-consistent recognition corresponds to the unique zero-cost equilibrium of the cost landscape. If physical evolution is cost-minimizing, descriptions that maintain self-match ($x=1$) are selected as stable equilibria, while inconsistent descriptions ($x\neq 1$) carry strictly positive cost and are selected against.

\subsection{The Gödel Dissolution}
Foundational theories are often challenged by Gödel's Incompleteness Theorems, which state that any sufficiently powerful formal system contains undecidable propositions. Does this mean a complete theory of reality is impossible?

We argue that Gödel's theorems do not obstruct the closure of Recognition Science because they apply to different domains:
\begin{itemize}
    \item \textbf{Gödel's Domain:} The \textit{provability} of arithmetic sentences within a formal axiomatic system.
    \item \textbf{RS Domain:} The \textit{selection} of physical configurations via cost minimization.
\end{itemize}

The universe does not compute the digits of $\pi$ to infinity or attempt to prove all true theorems of arithmetic. Instead, it settles into configurations that minimize the cost functional $J$. The process of reality is \textbf{Selection}, not \textbf{Proof}. By reframing reality as a physical selection process rather than a formal axiomatic system, RS is inoculated against logical paradoxes. Self-referential queries that lead to undecidability in logic simply correspond to high-cost or unstable configurations in physics, which are naturally filtered out by the minimization dynamic.

\begin{thebibliography}{99}
\bibitem{RecognitionGeometry}
Washburn, J.; Zlatanović, M.; Allahyarov, E.
Recognition Geometry.
\textit{Submitted}, \textbf{2026}.
\end{thebibliography}

\end{document}
