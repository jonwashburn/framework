\documentclass[11pt]{article}

\usepackage[margin=1in]{geometry}
\usepackage{amsmath,amssymb}
\usepackage[colorlinks=true,linkcolor=blue,citecolor=blue,urlcolor=blue]{hyperref}

\title{\Large RS Mass Mechanism\\
Note by Anil}
\date{}

% Simple checkbox macro (no extra packages)
\newcommand{\todo}{\(\square\)}
\newcommand{\done}{\(\blacksquare\)}
\newcommand{\Pzero}{\textbf{P0}}
\newcommand{\Pone}{\textbf{P1}}
\newcommand{\Ptwo}{\textbf{P2}}

\begin{document}
\maketitle

Replacing ``mass-from-Yukawa-coupling'' with ``mass-as-geometry'' on a discrete ladder is great idea. The cost functional $J$ and its uniqueness claim is concrete and it applicability here is also a great approach.   

\subsection*{A. }
\begin{itemize}
  \item If any integer offsets, sector assignments, anchor scale, or normalization constants are chosen rather than derived, we need to list them as explicit hypothesis (not implied theorems).
  \item Many objects are named but not defined. Some of it are in other papers but we need to define it here as well, perhaps in appendix. Basically, a self-contained ``Definitions'' section (even if proofs go to appendices/supplement/refer-to-other-paper):
  \begin{enumerate}
  \item the ledger (state space, adjacency, what ``tick'' means),
  \item ``recognition boundary'' (localization criterion, invariance criterion, eight-tick neutrality criterion),
  \item anchor $\mu^\star$ (what it is, how it is fixed, and why it is unique or canonical),
  \item sector yardstick $A_{\rm sector}$ (definition + derivation inputs),
  \item rung $r$ (how computed, what constraints restrict it),
  \item the $Z$-map / ``charge integerization'' map (exact formula and justification),
  \item the band function $\mathrm{gap}(Z)$ (exact formula, monotonicity/concavity facts).
  \end{enumerate}
  \item For each hard-coded constant used in the sector yardsticks and counting layer (e.g. $W=17$), we need to show a derivation from prior axioms/lemmas (not new axioms though). More importantly, instead of writing the formula, we should have a mechanism that allows us to write this specific unique formula (after all we claim everything is derived). Also, if I modify the RS theory how does the formula change?
  \item At minimum we need the following:
  \begin{itemize}
  \item alternative ladder base $b$ vs.\ $\varphi$ (model selection score vs.\ data),
  \item alternative octave reference (why $-8$ is canonical; we need to show performance drop if shifted),
  \item alternative $Z$-map choices (we need to show the chosen map is uniquely consistent with constraints).
\end{itemize}
  \item We need to state and justify the anchor electroweak identification. If $v\simeq 246\,$GeV is mapped to an electroweak yardstick $A_{\rm EW}$, we need to explain the calibration step and unit conventions as well as short derivation and an explicit numeric evaluation.
\end{itemize}


\section*{B. Bridge is missing. (Big one and some part of it can be separate paper.) }
Currently, the paper asserts ``Higgs is effective'' and ``Hamiltonian emerges'' but does not yet provide the minimum field-theory bridge (symmetries, EFT limit, renormalization/scale issues, what ``mass'' means operationally).

\subsection*{B1. Bridge to field theory: the ``action/Lagrangian'' equivalent in RS}
\begin{enumerate}
    \item Lets define an RS action functional: Even if fundamental dynamics is discrete minimization, we can define a discrete action
    \begin{equation}
        S_{\rm RS}[\gamma] := \sum_{t\in \mathbb{Z}} J\!\left(x_t(\gamma)\right)
    \end{equation}
for a clearly defined ratio/observable $x_t$ along a history $\gamma$. We need to then specify how the recognition operator $\hat R$ implements a local step that decreases $S_{\rm RS}$.
%%
    \item We need to show the stationary-action limit explicitly. We currently state that in the quadratic regime $J(x)\approx \tfrac12(x-1)^2$ cost minimization reduces to stationary action. We need derivation for discrete Euler--Lagrange / steepest descent / continuum limit etc to make it robust. 
%%
    \item If RS reproduces quantum amplitudes via $w\sim e^{-S}$ or similar, we need to state that mapping cleanly and show how it recovers the Born rule for a minimal toy system (two-path interference). One worked example is perfect. 
\end{enumerate}


\subsection*{B2. Symmetry \& invariance (Lorentz/gauge)}
\begin{enumerate}
    \item We need to state what symmetry is fundamental vs emergent. If Lorentz invariance is emergent in a continuum limit, then we need to explicitly say so and specify the limit (show all the math). If gauge invariance is enforced by ledger constraints (e.g.\ discrete cochain exactness), we need to mention and show the proof for that bridge. 
    \item What is the EFT story in RS framework? We need to provide a map from {\bf RS discrete variables $\to$ coarse-grained fields $\to$ effective Lagrangian ${\cal L}_{\rm eff}$}. Note we don't need full standard model (SM) derivation in this paper, but we do need a convincing outline with at least one explicit term derived. 
\end{enumerate}


\subsection*{B3. Higgs mechanism as derived/effective}
We need to explicitly say what ``Higgs is effective'' means operationally (need to show explicit equations and state any approximation made). To the bare minimum we need to 
\begin{itemize}
  \item identify the effective scalar degree of freedom (what coarse variable?),
  \item derive (or motivate) an effective potential with a nonzero VEV,
  \item show how the standard relation $m_f = y_f v/\sqrt{2}$ arises as an EFT identity at low energy.
  \item We need to check against the LHC Higgs coupling measurement. Does RS predicts deviations or not, and if yes at what scale? 
\end{itemize}

\subsection*{B4. How do RS particle interact with each other?}
The interaction between particle gives us all the measurement done in experiment. Thus, we need to explicitly have the Lagrangian equivalent in RS. The motivation is this will tell how particles interact with each other. 


\section*{C. There are many other comments to close the gap between Recognition Science Math and Physics. But it is a separate question and worth separate new papers. }

\end{document}
