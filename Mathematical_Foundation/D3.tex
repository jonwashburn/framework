\documentclass[11pt]{article}

\usepackage[margin=1in]{geometry}
\usepackage[T1]{fontenc}
\usepackage[utf8]{inputenc}
\usepackage{lmodern}
\usepackage{microtype}
\usepackage{amsmath,amssymb,amsthm,mathtools}
\usepackage[colorlinks=true,linkcolor=blue,citecolor=blue,urlcolor=blue]{hyperref}
\usepackage[nameinlink]{cleveref}
\usepackage{enumitem}
\setlist{nosep}

% Theorem environments
\newtheorem{theorem}{Theorem}[section]
\newtheorem{lemma}[theorem]{Lemma}
\newtheorem{proposition}[theorem]{Proposition}
\newtheorem{corollary}[theorem]{Corollary}
\newtheorem{definition}[theorem]{Definition}
\newtheorem{remark}[theorem]{Remark}

% Notation
\newcommand{\C}{\mathcal{C}}
\newcommand{\E}{\mathcal{E}}
\newcommand{\CR}{\mathcal{C}_R}
\newcommand{\M}{\mathcal{M}}
\newcommand{\Z}{\mathbb{Z}}
\newcommand{\R}{\mathbb{R}}
\newcommand{\N}{\mathbb{N}}
\newcommand{\lk}{\operatorname{lk}}
\newcommand{\SO}{\operatorname{SO}}

\title{Version-2 Comment-final on D3}
\date{}

\begin{document}
\maketitle

\section{\texorpdfstring{Fix 3: Constraint (B) allowed set must exclude $D=1$ and treat $D=2$ separately}{Fix 3 (non-negotiable): Constraint (B) allowed set must exclude D=1 and treat D=2 separately}}

This is in Section \textbf{``Constraint (B): Green-Kernel Stability Forces $D<4$''} where the allowed-dimension set is stated as including $D=1$.

\subsection{\texorpdfstring{Why $D=1$ is outside the model domain}{Why D=1 is outside the model domain}}

The  proof uses:
\begin{itemize}
\item angular momentum $\ell$,
\item a centrifugal term $\ell^2/(2mr^2)$,
\item and ``circular orbit'' language.
\end{itemize}
All of that requires at least $D\ge 2$ (a plane of rotation).
In $D=1$ there is no rotation group and no angular momentum, so $D=1$ must be excluded from the allowed set.

\subsection{\texorpdfstring{$D\ge 3$: stability forces $D=3$ under Green-kernel potentials}{D>=3: stability forces D=3 under Green-kernel potentials}}

For $D\ge 3$, the Green-kernel potential has the inverse-power form
\[
V_D(r)=-\frac{k}{r^{D-2}},\quad k>0.
\]
Put $n=D-2>0$ so $V=-k/r^n$.
From frequency ratio formula (from earlier comment-2) and the effective-potential computation, stability requires $V_{\mathrm{eff}}''(r_0)>0$.
For inverse-power laws one gets:
\[
V_{\mathrm{eff}}''(r_0)=\frac{nk(2-n)}{r_0^{n+2}}.
\]
Because $n,k,r_0^{n+2}>0$, stability is equivalent to $2-n>0$, i.e.\ $n<2$, i.e.\ $D<4$.
With $D\ge 3$, the only stable dimension is $D=3$.

\subsection{\texorpdfstring{$D=2$: logarithmic Green kernel must be handled separately}{D=2: logarithmic Green kernel must be handled separately}}

For $D=2$, the Green kernel is logarithmic.
Choose the attractive sign as
\[
V_2(r)=k\log r,\quad k>0,
\]
so the force $F(r)=-V'(r)=-k/r$ is attractive.

The effective potential is
\[
V_{\mathrm{eff}}(r)=k\log r+\frac{\ell^2}{2mr^2}.
\]
Differentiate:
\[
V_{\mathrm{eff}}'(r)=\frac{k}{r}-\frac{\ell^2}{mr^3}.
\]
A circular orbit satisfies $V_{\mathrm{eff}}'(r_0)=0$, hence
\[
\frac{k}{r_0}=\frac{\ell^2}{mr_0^3}
\quad\Rightarrow\quad
\ell^2=mk r_0^2.
\]
Differentiate again:
\[
V_{\mathrm{eff}}''(r)=-\frac{k}{r^2}+\frac{3\ell^2}{mr^4}.
\]
Substitute $\ell^2=mk r_0^2$ at $r=r_0$:
\[
V_{\mathrm{eff}}''(r_0)= -\frac{k}{r_0^2}+\frac{3mk r_0^2}{m r_0^4}
= -\frac{k}{r_0^2}+\frac{3k}{r_0^2}
=\frac{2k}{r_0^2}>0.
\]
Thus stable circular orbits exist in $D=2$.

\subsection{Correct allowed set for (B)}

So the correct non-singleton allowed set, \emph{in the central-force orbital sense}, is
\[
\mathcal{A}_B=\{2,3\}.
\]
The intersection logic remains unchanged:
\[
\{1,3,5,\dots\}\cap\{2,3\}\cap\{3,4,5,\dots\}=\{3\}.
\]

\subsection{Global consistency}

\begin{itemize}
\item Replace ``$\mathcal{A}_B=\{1,2,3\}$'' with ``$\mathcal{A}_B=\{2,3\}$ (in the central-force orbital sense).'' 
\item Add a one-line remark: ``$D=1$ is excluded because angular momentum/circular orbit is not defined.''
\item Clean the algebra in the stability proof if any nonsense intermediate exponent line remains.
\end{itemize}


\section{Fix 4: Same-dimension linking theorem overclaims an ``iff'' unless hypotheses are stated}

This is in Section \textbf{``Constraint (A): Same-Dimension Linking Forces Odd Dimensions''} at the theorem named ``Same-Dimension Linking Formula.''

The draft states (paraphrase):
\[
\text{``A canonical $\Z$-valued linking number exists iff }p+p=D-1\text{.''}
\]
As written, that is false without extra hypotheses, because the standard construction requires $A$ to bound and the result to be independent of choices.

\subsection{What is needed for the standard intersection-based construction}

To define $\lk(A,B)$ by intersection with a bounding chain:
\begin{enumerate}[label=(\arabic*)]
\item We need $[A]=0\in H_p(\CR;\Z)$ so $A$ bounds some $(p+1)$-chain $W$ with $\partial W=A$.
\item We need $(p+1)+p=D$ (equivalently $2p+1=D$) so that $W\cap B$ is $0$-dimensional and can be counted with sign.
\item We need independence of the choice of $W$; a sufficient condition is $H_{p+1}(\CR;\Z)=0$.
\end{enumerate}

\subsection{\texorpdfstring{Why $D=2p+1$ is necessary for a signed count?}{Dimension bookkeeping (why D=2p+1 is necessary for a signed count)}}

\begin{lemma}[Transverse intersection dimension]\label{lem:intersection_dim}
If $W$ is a $(p+1)$-dimensional submanifold/chain and $B$ is a $p$-dimensional submanifold/cycle in an oriented $D$-manifold and they intersect transversely, then
\[
\dim(W\cap B)=(p+1)+p-D=2p+1-D.
\]
Thus a signed intersection number $W\cdot B\in\Z$ (a signed count of points) exists only if $2p+1=D$.
\end{lemma}

Proof:
Standard transversality dimension formula: for transverse submanifolds of dimensions $a$ and $b$ in dimension $D$, the intersection has dimension $a+b-D$.
Here $a=p+1$ and $b=p$, so $\dim(W\cap B)=2p+1-D$.
A signed \emph{count} is defined only when this dimension equals $0$, i.e.\ $2p+1=D$.

\subsection{Correction!}

\begin{theorem}[Same-dimension linking in a homology $D$-sphere]\label{thm:link_hsphere}
Let $\CR$ be a closed oriented $D$-manifold with the integral homology of $S^D$:
\[
H_i(\CR;\Z)\cong
\begin{cases}
\Z,& i=0,D,\\
0,& 0<i<D.
\end{cases}
\]
Let $A,B\subset\CR$ be disjoint, closed, oriented $p$-submanifolds with $0<p<D$.
Then the standard construction defines a well-defined integer linking number
\[
\lk(A,B):=W\cdot B,\qquad \partial W=A,
\]
if and only if $2p+1=D$.
\end{theorem}

Proof:

\textbf{(1) existence of a bounding chain $W$.}
Because $0<p<D$ and $\CR$ has the homology of a sphere, we have $H_p(\CR)=0$.
Hence $[A]=0$ in homology, so there exists a $(p+1)$-chain $W$ with $\partial W=A$.

\textbf{(2) integer-valuedness forces $2p+1=D$.}
By Lemma~\ref{lem:intersection_dim}, $W\cdot B$ is a signed $0$-dimensional count only if $2p+1=D$.

\textbf{(3) independence of the choice of $W$ when $2p+1=D$.}
Assume $2p+1=D$.
If $W$ and $W'$ both satisfy $\partial W=\partial W'=A$, then $Z:=W-W'$ is a $(p+1)$-cycle.
Since $p+1<D$, sphere homology gives $H_{p+1}(\CR)=0$, hence $Z=\partial Q$ for some $(p+2)$-chain $Q$.

We use the standard boundary-compatibility of the intersection pairing:
\[
(\partial Q)\cdot B = Q\cdot (\partial B),
\]
up to an overall sign depending on conventions.
Because $B$ is a cycle, $\partial B=0$, so $(\partial Q)\cdot B=0$.
Therefore
\[
(W\cdot B)-(W'\cdot B)=(W-W')\cdot B=Z\cdot B=(\partial Q)\cdot B=0.
\]
So $\lk(A,B)$ is independent of the choice of bounding chain.

\textbf{(4) failure when $2p+1\neq D$.}
If $2p+1\neq D$, the intersection $W\cap B$ has nonzero dimension generically, so the ``signed count of points'' construction is not available as an integer-valued invariant.


\subsection{Global consistency}

\begin{itemize}
\item Rewrite the ``Same-Dimension Linking Formula'' theorem with explicit hypotheses (homology-sphere or, more generally, $A$ null-homologous and the choice-independence condition).
\item If we keep $\mathcal{A}_A=\{1,3,5,\dots\}$ by allowing $p=(D-1)/2$ to float, add one paragraph explaining why the theory expects \emph{some} same-dimension linking sector to exist (e.g.\ codimension-2 defects), otherwise it reads like a parity trick.
\end{itemize}
\end{document}