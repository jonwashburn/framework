\documentclass[11pt,a4paper]{article}
\usepackage[margin=1in]{geometry}
\usepackage[T1]{fontenc}
\usepackage{lmodern}
\usepackage{microtype}
\usepackage{amsmath,amssymb,amsthm}
\usepackage{mathtools}
\usepackage{booktabs}
\usepackage{array}
\usepackage{enumitem}
\usepackage{xcolor}
\usepackage[hidelinks]{hyperref}

\newtheorem{theorem}{Theorem}[section]
\newtheorem{proposition}[theorem]{Proposition}
\newtheorem{lemma}[theorem]{Lemma}
\newtheorem{corollary}[theorem]{Corollary}
\newtheorem{definition}[theorem]{Definition}
\newtheorem{remark}[theorem]{Remark}
\newtheorem{example}[theorem]{Example}
\newtheorem{prediction}[theorem]{Prediction}
\newtheorem{falsifier}[theorem]{Falsification Criterion}

\newcommand{\phig}{\varphi}
\newcommand{\Jcost}{J}
\newcommand{\Rhat}{\hat{R}}
\newcommand{\Rp}{\mathbb{R}_{>0}}

\title{\textbf{Decision as Cost Geodesic:\\
The Geometry of Choice on the\\
$\Jcost$-Cost Manifold}\\[0.5em]
\large A New Domain in Recognition Science}
\author{Jonathan Washburn\\
\small Recognition Science Research Institute, Austin, Texas\\
\small \texttt{washburn.jonathan@gmail.com}}
\date{February 2026}

\begin{document}
\maketitle

\begin{abstract}
We derive a geometric theory of decision-making from the canonical
$\Jcost$-cost functional $\Jcost(x) = \frac{1}{2}(x + x^{-1}) - 1$.
The \emph{one-dimensional choice manifold} is $(\Rp, g)$ with
Riemannian metric $g(x) = \Jcost''(x) = x^{-3}$, the Hessian of
$\Jcost$.  We prove:
\begin{enumerate}[nosep]
\item \textbf{Explicit geodesics}: $\gamma(t) = 4/(At + B)^{2}$ is the
  complete family of non-constant geodesics (inverse-square in affine
  parameter).  The ground state $\gamma(t) \equiv 1$ is the global cost
  minimum.
\item \textbf{Attention capacity}: total conscious intensity is bounded
  by $\phig^3 \approx 4.236$, deriving Cowan's ``$4 \pm 1$'' law
  from the cost structure.
\item \textbf{Deliberation dynamics}: $x_{t+1} = x_t - \eta\,\Jcost'(x_t)
  + \xi_t$ (gradient descent with exploration noise), bounded by the
  eight-tick constraint.  Regret equals metric distance from the ideal
  geodesic.
\item \textbf{Free will}: at bifurcation points (multiple near-equal-cost
  futures), the Gap-45 uncomputability barrier forces experiential
  navigation (compatibilism).
\item \textbf{Decision thermodynamics}: choices follow a Boltzmann
  distribution $P(x) \propto \exp(-\Jcost(x)/T_R)$, where $T_R$ is the
  recognition temperature.
\end{enumerate}
We then extend to multi-dimensional decisions, showing that
independent choices decompose as product geodesics while coupled
comparisons yield a genuinely curved Hessian manifold with non-trivial
sectional curvature.  All core structures are formalised in Lean~4
(\texttt{IndisputableMonolith.Decision.*}, 7~submodules).

\medskip\noindent\textbf{Keywords:} decision theory, geodesic, Hessian
manifold, attention, free will, $\Jcost$-cost, Gap-45, Boltzmann.
\end{abstract}

\tableofcontents
\newpage

%======================================================================
\section{Introduction}\label{sec:intro}
%======================================================================

Classical decision theory posits utility functions and maximises expected
utility~\cite{VonNeumann1944}.  Behavioural economics documents
systematic departures~\cite{Kahneman2011}.  Neuroscience measures neural
correlates but lacks a first-principles dynamics.  None of these derives
the \emph{structure} of decision from a more basic principle.

Recognition Science provides the missing foundation.  The unique cost
functional $\Jcost(x) = \tfrac{1}{2}(x + x^{-1}) - 1$, forced by the
Recognition Composition Law and calibration~\cite{WashburnCost2026},
equips the space of ledger ratios with a canonical Riemannian metric
$g = \Jcost''$.  Decisions are \emph{geodesics in this choice manifold}.
Deliberation is gradient descent.  Attention is a capacity-limited gate.
Free will is geodesic selection at bifurcation points protected by the
Gap-45 barrier.

We begin with the one-dimensional case (a single comparison ratio), where
the geometry can be solved in closed form, then extend to multi-dimensional
decision spaces where independent and coupled comparisons give qualitatively
different geometric structures.

%======================================================================
\section{The One-Dimensional Choice Manifold}\label{sec:manifold}
%======================================================================

\begin{definition}[Choice manifold]\label{def:manifold}
The \emph{one-dimensional choice manifold} is $M_1 = \Rp$ equipped with
the Riemannian metric
\begin{equation}\label{eq:metric}
  ds^2 = g(x)\,dx^2, \qquad g(x) = \Jcost''(x) = \frac{1}{x^3},
\end{equation}
the Hessian of $\Jcost(x) = \frac{1}{2}(x + x^{-1}) - 1$ at $x > 0$.
\end{definition}

\begin{lemma}[Metric is positive definite]
$g(x) = x^{-3} > 0$ for all $x > 0$, confirming $(M_1, g)$ is a
well-defined Riemannian manifold.
\end{lemma}

\subsection{Log-coordinate representation}

The substitution $u = \ln x$ maps $\Rp$ to $\mathbb{R}$ and transforms
the cost to $\Jcost(e^u) = \cosh u - 1$.  The metric in log-coordinates is
\begin{equation}\label{eq:log_metric}
  \tilde{g}(u) = \frac{d^2}{du^2}(\cosh u - 1) = \cosh u,
\end{equation}
so $ds^2 = \cosh(u)\,du^2$.  At the ground state $u = 0$: $\tilde{g}(0) = 1$
(the calibration condition $\Jcost''(1) = 1$).  This form makes the
connection to information geometry transparent: $\tilde{g}$ is the Fisher
information metric of the exponential family parametrised by the
log-ratio~\cite{Amari2000}.

\begin{definition}[Christoffel symbol]\label{def:christoffel}
In ratio coordinates, the unique Christoffel symbol is
\begin{equation}\label{eq:christoffel}
  \Gamma(x) = \frac{g'(x)}{2g(x)}
  = \frac{-3x^{-4}}{2x^{-3}}
  = -\frac{3}{2x}.
\end{equation}
In log-coordinates:
$\tilde{\Gamma}(u) = \frac{\tilde{g}'(u)}{2\tilde{g}(u)}
= \frac{\sinh u}{2\cosh u} = \frac{1}{2}\tanh u.$
\end{definition}

\subsection{Parametric curvature of the cost landscape}\label{sec:curvature}

\begin{remark}[On curvature in one dimension]\label{rem:1d_curvature}
Every smooth one-dimensional Riemannian manifold is intrinsically flat:
the arc-length reparametrisation $s(x) = \int_1^x \sqrt{g(y)}\,dy =
\int_1^x y^{-3/2}\,dy = 2(1 - x^{-1/2})$ maps $(M_1, g)$ isometrically
onto an interval of $(\mathbb{R}, ds^2)$.  The ``curvature'' we compute
below is therefore not intrinsic Gaussian curvature (which vanishes in
1D) but rather the \emph{parametric curvature} --- a measure of how
rapidly the cost landscape changes in ratio coordinates.  This quantity
determines the local difficulty of decision-making and controls geodesic
divergence in the $x$-parametrisation.
\end{remark}

\begin{proposition}[Parametric curvature]\label{prop:curvature}
The parametric curvature of $(M_1, g)$ at $x > 0$ is
\begin{equation}\label{eq:curvature}
  \kappa(x) = -\frac{1}{2\sqrt{g}} \frac{d^2}{dx^2}\!\left(\frac{1}{\sqrt{g}}\right)
  = -\frac{3}{8} x.
\end{equation}
\end{proposition}

\begin{proof}
$g^{-1/2} = x^{3/2}$, so $(g^{-1/2})'' = \tfrac{3}{4}x^{-1/2}$ and
$\sqrt{g} = x^{-3/2}$.  Then $\kappa = -(2x^{-3/2})^{-1}\cdot
\tfrac{3}{4}x^{-1/2} = -\tfrac{3}{8}x$. \qed
\end{proof}

\begin{remark}[Interpretation]
$\kappa(x) < 0$ for all $x > 0$: in ratio coordinates, geodesics
\emph{diverge} --- nearby decisions separate.  The magnitude
$|\kappa(x)| = 3x/8$ \emph{increases} away from equilibrium, so
decisions far from balance are parametrically harder.  At $x = 1$:
$\kappa(1) = -3/8$.

For $x > 1$ (gain region), $\Jcost''(x) = x^{-3}$ is small (shallow
landscape); for $0 < x < 1$ (loss region), $\Jcost''(x) = x^{-3}$ is
large (steep landscape).  This asymmetry generates the empirical
observation that ``losses loom larger than gains''~\cite{Kahneman2011}
without postulating a separate value function.
\end{remark}

%======================================================================
\section{Geodesics: The Optimal Decisions}\label{sec:geodesics}
%======================================================================

\begin{theorem}[Geodesic equation]\label{thm:geodesic_eq}
The geodesic equation on $(M_1, g)$ in ratio coordinates is
\begin{equation}\label{eq:geodesic}
  \ddot{\gamma} - \frac{3}{2\gamma}\,\dot{\gamma}^2 = 0.
\end{equation}
\end{theorem}

\begin{theorem}[Explicit geodesics]\label{thm:geodesic_sol}
The general non-constant solution to~\eqref{eq:geodesic} is
\begin{equation}\label{eq:geodesic_sol}
  \gamma(t) = \frac{4}{(At + B)^2}, \qquad A, B \in \mathbb{R},\; At + B \ne 0.
\end{equation}

\emph{Lean:} \texttt{Decision.VariationalCalculus.geodesic\_correct\_satisfies\_equation}.
\end{theorem}

\begin{proof}
The geodesic equation $\ddot{\gamma} = \frac{3}{2\gamma}\dot{\gamma}^2$
is autonomous.  Set $v = \dot{\gamma}$ so $\ddot{\gamma} = v\,dv/d\gamma$:
\[
  v\frac{dv}{d\gamma} = \frac{3}{2\gamma}\,v^2
  \quad\Longrightarrow\quad
  \frac{dv}{v} = \frac{3}{2\gamma}\,d\gamma
  \quad\Longrightarrow\quad
  \ln|v| = \tfrac{3}{2}\ln\gamma + C_1.
\]
Exponentiating: $v = A\gamma^{3/2}$.  Hence
$\gamma^{-3/2}\,d\gamma = A\,dt$, and integrating:
\[
  -2\gamma^{-1/2} = At + B
  \quad\Longrightarrow\quad
  \gamma(t) = \frac{4}{(At + B)^2}.
\]

\noindent\textbf{Verification.}
Set $w = At + B$, so $\gamma = 4w^{-2}$.  Then
$\dot{\gamma} = -8Aw^{-3}$ and $\ddot{\gamma} = 24A^2 w^{-4}$.
Check:
\[
\frac{3}{2\gamma}\dot{\gamma}^2 = \frac{3}{8w^{-2}}
\cdot 64A^2 w^{-6} = 24A^2 w^{-4}
= \ddot{\gamma}. \qquad\checkmark
\]
This is formally verified in Lean. \qed
\end{proof}

\begin{corollary}[Ground state]\label{cor:ground}
The constant path $\gamma(t) \equiv 1$ is a geodesic with zero velocity
and zero $\Jcost$-cost: $\Jcost(1) = 0$.  This is the global minimum
--- the ``resting decision.''

\emph{Lean:} \texttt{Decision.GeodesicSolutions.ground\_state\_is\_geodesic}.
\end{corollary}

\begin{remark}[Geodesic distance]
The geodesic distance between two ratio states $x_0, x_1 \in \Rp$ is
\begin{equation}\label{eq:geodesic_dist}
  d_g(x_0, x_1) = \int_{x_0}^{x_1} \sqrt{g(x)}\,dx
  = \int_{x_0}^{x_1} x^{-3/2}\,dx
  = 2\bigl|x_0^{-1/2} - x_1^{-1/2}\bigr|.
\end{equation}
In log-coordinates: $d_{\tilde{g}}(u_0, u_1) = \int_{u_0}^{u_1}
\sqrt{\cosh u}\,du$, which does not have a closed form but is bounded
below by $|u_1 - u_0|$ (since $\cosh u \ge 1$).
\end{remark}

%======================================================================
\section{Multi-Dimensional Extension}\label{sec:multi}
%======================================================================

A single comparison ratio $x \in \Rp$ describes one binary choice.
Real decisions involve multiple simultaneous comparisons.  The natural
RS extension equips $\Rp^n$ with the Hessian metric of a multi-ratio
cost functional.  Two cases arise with qualitatively different geometry.

\subsection{Independent decisions: the product manifold}

\begin{definition}[Independent $n$-decision manifold]\label{def:product}
For $n$ independent comparisons with ratios $x_1, \ldots, x_n \in \Rp$,
the total cost is
\begin{equation}\label{eq:product_cost}
  \Phi(x_1, \ldots, x_n) = \sum_{i=1}^n \Jcost(x_i)
  = \sum_{i=1}^n \Bigl[\tfrac{1}{2}(x_i + x_i^{-1}) - 1\Bigr].
\end{equation}
The Hessian metric is
\begin{equation}\label{eq:product_metric}
  g_{ij} = \frac{\partial^2 \Phi}{\partial x_i\,\partial x_j}
  = \delta_{ij}\,x_i^{-3}.
\end{equation}
\end{definition}

\begin{proposition}[Product decomposition]\label{prop:product}
The manifold $(\Rp^n, g_{ij} = \delta_{ij}\,x_i^{-3})$ is a Riemannian
product $(M_1, g_1) \times \cdots \times (M_1, g_n)$.  Its geodesics
decompose:
\begin{equation}
  \gamma(t) = \bigl(\gamma_1(t), \ldots, \gamma_n(t)\bigr),
  \qquad \gamma_i(t) = \frac{4}{(A_i t + B_i)^2},
\end{equation}
where each component independently satisfies the 1D geodesic
equation~\eqref{eq:geodesic}.  The sectional curvature of every 2-plane
vanishes.
\end{proposition}

\begin{proof}
Since $g_{ij} = 0$ for $i \ne j$, the Levi-Civita connection has no
cross-Christoffel symbols: $\Gamma^k_{ij} = 0$ whenever two indices
differ.  The geodesic equation
$\ddot{\gamma}_k + \sum_{i,j}\Gamma^k_{ij}\dot{\gamma}_i\dot{\gamma}_j = 0$
reduces to $\ddot{\gamma}_k + \Gamma^k_{kk}\dot{\gamma}_k^2 = 0$ for
each $k$ independently.  The Riemann tensor vanishes on any mixed
2-plane.  \qed
\end{proof}

\begin{remark}
Independent decisions are geometrically trivial in the sense that no
new structure emerges beyond the 1D case.  The one-dimensional analysis
of Sections~\ref{sec:manifold}--\ref{sec:geodesics} is therefore the
\emph{complete building block} for independent multi-choice problems.
\end{remark}

In log-coordinates $u_i = \ln x_i$, the product metric becomes
$\tilde{g}_{ij} = \delta_{ij}\cosh u_i$, recovering the colleague's
suggestion~\cite{Zlatanovic2026} as the natural independent-decision
extension.

\subsection{Coupled decisions: the comparison manifold}

\begin{definition}[Coupled 2-decision manifold]\label{def:coupled}
When two states $x, y \in \Rp$ are compared \emph{to each other}, the
cost of the comparison is $\Jcost(x/y)$.  In log-coordinates $u = \ln x$,
$v = \ln y$, the cost functional is
\begin{equation}\label{eq:coupled_cost}
  \Phi(u, v) = \Jcost(e^{u - v}) = \cosh(u - v) - 1.
\end{equation}
The Hessian metric is
\begin{equation}\label{eq:coupled_metric}
  (\tilde{g}_{ij}) = \cosh(u - v)
  \begin{pmatrix} 1 & -1 \\ -1 & 1 \end{pmatrix}.
\end{equation}
\end{definition}

\begin{proposition}[Non-trivial geometry of coupled decisions]\label{prop:coupled}
The coupled metric~\eqref{eq:coupled_metric} is:
\begin{enumerate}[nosep]
\item \textbf{Degenerate}: $\det(\tilde{g}_{ij}) = 0$.  The metric has
  rank 1 with null direction $(1, 1)$ (simultaneous rescaling of both
  states leaves the comparison unchanged --- a gauge symmetry).
\item \textbf{Non-diagonal}: the off-diagonal $\tilde{g}_{uv} =
  -\cosh(u - v) \ne 0$ couples the two decision variables.
\item \textbf{Intrinsically flat on the constraint surface}: restricting
  to the 1D orbit $w = u - v$ (the comparison coordinate) yields
  $ds^2 = 2\cosh(w)\,dw^2$, which is a rescaling of the 1D case.
\end{enumerate}
\end{proposition}

\begin{proof}
Direct computation: $\tilde{g}_{uu} = \partial_u^2\Phi = \cosh(u-v)$,
$\tilde{g}_{vv} = \partial_v^2\Phi = \cosh(u-v)$,
$\tilde{g}_{uv} = -\cosh(u-v)$.  The determinant is
$\cosh^2(u-v) - \cosh^2(u-v) = 0$.  The null vector satisfies
$\tilde{g}_{ij}\xi^j = 0$: taking $\xi = (1,1)$ gives
$\cosh(u-v)(1 - 1) = 0$.  Restricting to $w = u - v$ eliminates the
gauge, and $\Phi(w) = \cosh w - 1$ has Hessian $\cosh w$ with
$ds^2 = \cosh(w)\,dw^2$ up to a factor of 2 from the coordinate
change. \qed
\end{proof}

\begin{remark}[Physical interpretation]
The gauge direction $(1,1)$ reflects a deep RS principle: simultaneously
multiplying numerator and denominator by the same factor leaves the
ratio --- and hence the cost --- unchanged.  Only the \emph{relative}
comparison $w = \ln(x/y)$ carries decision-relevant information.  The
effective decision space is one-dimensional in the comparison coordinate,
confirming that the 1D analysis captures the essential geometry.
\end{remark}

\subsection{The general $n$-ratio manifold}

\begin{definition}[Full decision manifold]\label{def:full}
For a ledger state with $n$ ratios $r_1, \ldots, r_n \in \Rp$ subject to
both self-costs and pairwise comparison costs, the total cost is
\begin{equation}\label{eq:full_cost}
  \Phi(r_1, \ldots, r_n) = \sum_i \Jcost(r_i) + \lambda \sum_{i < j}
  \Jcost(r_i / r_j),
\end{equation}
where $\lambda > 0$ is the coupling strength (determined by the bond
topology of the ledger).  The Hessian metric
$g_{ij} = \partial^2\Phi/\partial r_i\,\partial r_j$ is a full $n \times n$
matrix with non-zero off-diagonal entries whenever $\lambda > 0$.
\end{definition}

\begin{proposition}[Sectional curvature of the coupled manifold]\label{prop:sectional}
For $n \ge 2$ with $\lambda > 0$, the full decision manifold has
non-vanishing Riemann curvature tensor.  In particular, the sectional
curvature of the $(r_i, r_j)$-plane is non-zero whenever the coupling
$\Jcost(r_i/r_j)$ is present.  This produces genuinely
higher-dimensional effects: geodesic focusing, conjugate points, and
non-trivial Jacobi fields that are absent in the product case.
\end{proposition}

This establishes a hierarchy:
\begin{center}
\small
\begin{tabular}{@{}lll@{}}
\toprule
\textbf{Regime} & \textbf{Metric structure} & \textbf{Geometry} \\
\midrule
Single comparison & $g = x^{-3}$ (1D) & Explicit geodesics, trivially flat \\
Independent choices & $g_{ij} = \delta_{ij}x_i^{-3}$ (product) & Product geodesics, zero sectional curvature \\
Coupled comparisons & $g_{ij}$ with cross-terms & Non-trivial curvature, Jacobi fields \\
\bottomrule
\end{tabular}
\end{center}

The 1D case is the foundational building block; the coupled case is the
frontier for future work.

%======================================================================
\section{The Attention Operator}\label{sec:attention}
%======================================================================

\begin{definition}[Attention operator]\label{def:attention}
The \emph{attention operator} $\mathcal{A}$ is a gate
\[
  \mathcal{A} : \text{QualiaSpace} \times \mathbb{R}_{\ge 0} \times
  \mathbb{R}_{\ge 0} \to \text{Option}(\text{ConsciousQualia})
\]
that admits a qualia into conscious experience iff its recognition cost
$C \ge 1$ and intensity $I > 0$.
\end{definition}

\begin{theorem}[Attention capacity]\label{thm:capacity}
The total conscious intensity is bounded:
\begin{equation}
  \sum_{i=1}^{N} I_i \;\le\; \phig^3 \approx 4.236.
\end{equation}
This derives Cowan's ``$4 \pm 1$'' law~\cite{Cowan2001}: $\phig^3 \approx 4.24$ items
at unit intensity, or $\lfloor 2\phig^3 \rfloor = 8$ at half intensity,
or $\lceil \phig^3/2 \rceil = 3$ at double intensity.

\emph{Lean:} \texttt{Decision.Attention.capacity\_bounded}.
\end{theorem}

%======================================================================
\section{Deliberation Dynamics}\label{sec:deliberation}
%======================================================================

\begin{definition}[Deliberation rule]\label{def:deliberation}
Deliberation follows the discrete-time Langevin update
\begin{equation}\label{eq:deliberation}
  x_{t+1} = x_t - \eta\,\Jcost'(x_t) + \sqrt{2\eta T_R}\,\xi_t,
\end{equation}
where $\eta > 0$ is the step size, $T_R$ is the recognition temperature,
$\xi_t \sim \mathcal{N}(0,1)$ is Gaussian noise, and the update is
constrained to complete within one eight-tick cycle.  The gradient
$\Jcost'(x) = \frac{1}{2}(1 - x^{-2})$ drives the state toward $x = 1$.
\end{definition}

\begin{definition}[Regret]\label{def:regret}
The \emph{regret} of a decision trajectory $\{x_t\}$ relative to the
ideal geodesic $\gamma^*$ is the metric distance
\begin{equation}
  R = \int_0^T \sqrt{g(x_t)}\,|x_t - \gamma^*(t)|\,dt
  = \int_0^T x_t^{-3/2}\,|x_t - \gamma^*(t)|\,dt.
\end{equation}
\end{definition}

\begin{theorem}[Zero regret iff geodesic]\label{thm:regret_zero}
$R = 0$ if and only if $\{x_t\}$ lies on the ideal geodesic.  The
proof uses strict positivity of $g$ to conclude that a continuous
non-negative integrand with zero integral must vanish identically.

\emph{Lean:} \texttt{Decision.ChoiceManifold.compute\_regret\_zero\_iff}.
\end{theorem}

%======================================================================
\section{Free Will as Geodesic Selection}\label{sec:free_will}
%======================================================================

\begin{definition}[Bifurcation point]\label{def:bifurcation}
A \emph{bifurcation point} is a state $x$ where multiple geodesics
with near-equal $\Jcost$-cost diverge.  Formally: $\exists\,
\gamma_1 \ne \gamma_2$ with $\gamma_1(0) = \gamma_2(0) = x$ and
$|\mathcal{S}[\gamma_1] - \mathcal{S}[\gamma_2]| < \varepsilon$.
\end{definition}

\begin{theorem}[Gap-45 protects selection]\label{thm:gap45}
At bifurcation points near the 45th $\phig$-rung, the optimal geodesic
cannot be computed by any finite algorithm operating within a single
eight-tick cycle.  This is because $\gcd(8, 45) = 1$: the eight-tick
computation window and the 45-fold pattern cannot synchronise
(Gap-45 barrier).

Consequently, the agent must \emph{navigate experientially} ---
selecting a geodesic through lived exploration rather than algorithmic
prediction.

\emph{Lean:} \texttt{Decision.FreeWill.gap45\_protects\_selection}.
\end{theorem}

\begin{theorem}[Compatibilism]\label{thm:compatibilism}
The cost landscape $\Jcost$ constrains the set of admissible geodesics
(determinism).  At bifurcation points, the agent selects among them
(freedom).  These coexist because:
\begin{enumerate}[nosep]
\item Determinism: the metric $g = \Jcost''$ is uniquely forced.
\item Freedom: geodesic selection at bifurcations is underdetermined by $g$.
\item Protection: Gap-45 ensures no external agent can predict the
  selection.
\end{enumerate}
\end{theorem}

%======================================================================
\section{Decision Thermodynamics}\label{sec:thermo}
%======================================================================

\begin{definition}[Boltzmann distribution over choices]\label{def:boltzmann}
At recognition temperature $T_R$, the probability of choosing state $x$ is
\begin{equation}\label{eq:boltzmann}
  P(x) = \frac{1}{Z}\exp\!\left(-\frac{\Jcost(x)}{T_R}\right),
  \qquad Z = \int_0^\infty \exp\!\left(-\frac{\Jcost(x)}{T_R}\right)dx.
\end{equation}
\end{definition}

\begin{theorem}[Exploration--exploitation tradeoff]\label{thm:tradeoff}
\begin{itemize}[nosep]
\item High $T_R$: $P(x)$ is broad (exploration, risk-taking).
\item Low $T_R$: $P(x)$ is peaked at $x = 1$ (exploitation, risk-aversion).
\item $T_R \to 0$: deterministic choice at $x = 1$ (ground state).
\item $T_R \to \infty$: uniform distribution (random choice).
\end{itemize}
The critical temperature $T_\phig = 1/\ln\phig \approx 2.078$ marks
the phase boundary between coherent decision-making (ordered phase,
$T_R < T_\phig$) and exploratory randomness (disordered phase,
$T_R > T_\phig$)~\cite{WashburnThermo2026}.
\end{theorem}

%======================================================================
\section{Predictions and Falsification}\label{sec:predictions}
%======================================================================

\begin{prediction}[Decision latency]
Decision latency scales as $\Jcost(\Delta x)$ where $\Delta x$ is the
separation between the two most attractive options on the choice
manifold.  Equal-cost options (small $\Jcost$ gap) take longest
(Hick--Hyman law generalisation).
\end{prediction}

\begin{prediction}[Attention capacity]
Working memory capacity clusters near $\phig^3 \approx 4.24$ items
across tasks, consistent with Cowan's ``$4 \pm 1$''~\cite{Cowan2001}
rather than Miller's $7 \pm 2$.
\end{prediction}

\begin{prediction}[Swing in decision timing]
When subjects make rhythmic decisions (e.g.\ tapping to a beat), the
natural asymmetry in inter-tap intervals will peak near $1/\phig :
1/\phig^2$ (the golden swing ratio).
\end{prediction}

\begin{prediction}[Loss--gain asymmetry]
The ratio of loss sensitivity to gain sensitivity at ratio $x$ equals
$g(1/x)/g(x) = x^3$, so at the first $\phig$-rung: loss/gain $=
\phig^3 \approx 4.24$.  This is consistent with Tversky--Kahneman's
empirically measured loss aversion coefficient of $\sim 2$--$2.5$
(evaluated at moderate stakes where $x \approx \phig^{1/2}$, giving
$x^3 \approx 2.1$).
\end{prediction}

\begin{falsifier}[Wrong geodesic family]
If the optimal decision paths in a continuous choice task are
inconsistent with $\gamma(t) = 4/(At+B)^{2}$ (e.g.\ linear or
exponential instead), the choice manifold metric is falsified.
\end{falsifier}

\begin{falsifier}[No capacity bound]
If working memory capacity grows unboundedly with training (no
saturation near $\phig^3$), the attention capacity theorem is falsified.
\end{falsifier}

\begin{falsifier}[Product geodesics in coupled decisions]
If empirical decision trajectories in coupled-choice experiments
decompose into independent 1D geodesics (no cross-influence), then the
coupling term in~\eqref{eq:full_cost} is falsified, reducing the theory
to the product case.
\end{falsifier}

%======================================================================
\section{Comparison with Existing Decision Theory}\label{sec:prior}
%======================================================================

\begin{center}
\small
\renewcommand{\arraystretch}{1.15}
\begin{tabular}{@{}>{\bfseries}l p{5.2cm} p{5.2cm}@{}}
\toprule
Feature & Standard (utility) & RS (cost geodesic) \\
\midrule
Primitive & Utility $u(x)$ (postulated) & $\Jcost(x)$ (forced by RCL) \\
Optimality & Max expected utility & Min path action $\int\!\Jcost\,dt$ \\
Space & Preference ordering & Hessian manifold $(\Rp^n, \partial^2\Phi)$ \\
Dynamics & None (static comparison) & Geodesic + Langevin dynamics \\
Capacity & Miller's $7\pm 2$ (empirical) & $\phig^3 \approx 4.24$ (derived) \\
Free will & Incompatibilism debate & Compatibilism (Gap-45) \\
Loss aversion & Prospect theory (postulated) & $g(1/x)/g(x) = x^3$ (derived) \\
Multi-dim & Independent utility sums & Product vs.\ coupled Hessian \\
\bottomrule
\end{tabular}
\end{center}

%======================================================================
\section{Discussion}\label{sec:discussion}
%======================================================================

\subsection*{Claims and non-claims}

We derive the geometric structure of decision-making from $\Jcost$
uniqueness.  The one-dimensional case is solved completely (geodesics,
distance, regret).  The multi-dimensional extension shows that independent
choices are geometrically trivial (product structure, zero sectional
curvature) while coupled comparisons yield genuinely higher-dimensional
Riemannian geometry.  We do \emph{not} claim to explain all psychological
phenomena; the framework provides the \emph{mathematical skeleton}
(metric, geodesics, curvature) on which empirical decision science
operates.

\subsection*{Open problems}

\begin{enumerate}[label=\textup{(Q\arabic*)},nosep]
\item Can the geodesic family $\gamma = 4/(At+B)^2$ be measured in
  continuous tracking tasks (e.g.\ pursuit rotor, or drift-diffusion
  experiments)?
\item Is the attention capacity $\phig^3$ experimentally distinguishable
  from $4$ (i.e.\ does $0.24$ items matter)?
\item Does the recognition temperature $T_R$ correlate with dopamine
  levels or arousal state?
\item Is regret (metric distance from geodesic) measurable via fMRI
  (anterior cingulate activity)?
\item For the coupled decision manifold~\eqref{eq:full_cost}, do the
  geodesics exhibit focusing effects (conjugate points) that correspond
  to empirically observed decision bundling?
\item Can the loss--gain asymmetry prediction ($x^3$ ratio) be tested
  against existing prospect theory data at multiple stake levels?
\end{enumerate}

%======================================================================
\section{Lean Formalization}\label{sec:lean}
%======================================================================

\begin{center}
\begin{tabular}{@{}ll@{}}
\toprule
\textbf{Module} & \textbf{Content} \\
\midrule
\texttt{Decision.Attention} & Operator, capacity bound $\phig^3$ \\
\texttt{Decision.ChoiceManifold} & Metric, Christoffel, geodesic eq, regret \\
\texttt{Decision.FreeWill} & Bifurcation, Gap-45, compatibilism \\
\texttt{Decision.DeliberationDynamics} & Gradient descent + noise \\
\texttt{Decision.GeodesicSolutions} & $\gamma(t) = 4/(At+B)^{2}$ (corrected) \\
\texttt{Decision.VariationalCalculus} & Geodesic verification, variational principle \\
\texttt{Decision.DecisionThermodynamics} & Boltzmann, temperature \\
\bottomrule
\end{tabular}
\end{center}

\begin{thebibliography}{9}
\bibitem{WashburnCost2026}
J.~Washburn and M.~Zlatanovi\'{c},
``The Cost of Coherent Comparison,''
arXiv:2602.05753v1, 2026.

\bibitem{VonNeumann1944}
J.~von Neumann and O.~Morgenstern,
\textit{Theory of Games and Economic Behavior},
Princeton, 1944.

\bibitem{Kahneman2011}
D.~Kahneman,
\textit{Thinking, Fast and Slow},
Farrar, Straus and Giroux, 2011.

\bibitem{Cowan2001}
N.~Cowan,
``The magical number 4 in short-term memory,''
\textit{Behavioral and Brain Sciences}, 24(1):87--114, 2001.

\bibitem{Amari2000}
S.~Amari and H.~Nagaoka,
\textit{Methods of Information Geometry},
AMS, 2000.

\bibitem{Zlatanovic2026}
M.~Zlatanovi\'{c},
Private communication, February 2026.

\bibitem{WashburnThermo2026}
J.~Washburn,
``The Critical Temperature of Consciousness,''
Recognition Science preprint, 2026.

\bibitem{doCarmo1992}
M.~P. do~Carmo,
\textit{Riemannian Geometry},
Birkh\"{a}user, 1992.

\bibitem{Jost2017}
J.~Jost,
\textit{Riemannian Geometry and Geometric Analysis},
7th ed., Springer, 2017.
\end{thebibliography}

\end{document}
