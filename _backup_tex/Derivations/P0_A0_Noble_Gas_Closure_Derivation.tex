\documentclass[11pt]{article}

\usepackage[margin=1in]{geometry}
\usepackage{amsmath, amssymb, amsthm}
\usepackage{booktabs}
\usepackage{hyperref}
\usepackage{enumitem}

\hypersetup{
  colorlinks=true,
  linkcolor=blue,
  urlcolor=blue,
  citecolor=blue
}

\newtheorem{theorem}{Theorem}
\newtheorem{lemma}{Lemma}
\newtheorem{definition}{Definition}

\title{P0-A0 Noble Gas Closure Theorem\\(Derivation Notes + Lean/Artifact Cross-References)}
\author{Recognition Science Derivation Campaign}
\date{2026-01-17}

\begin{document}
\maketitle

\begin{abstract}
This note documents the \emph{fit-free} periodic-table scaffold used in the repository,
and derives (from the defining primitives) the Noble Gas Closure facts formalized in Lean:
(i) noble gases have \texttt{distToNextClosure} equal to zero,
(ii) noble gases have complete shells (\texttt{valenceElectrons} equals \texttt{periodLength}),
(iii) the noble-gas gap sequence forces the period lengths $\{2,8,8,18,18,32\}$.
We also include the preregistered validator output that checks these properties on $Z\le 118$.
\end{abstract}

\section{Claim (P0-A0)}
The project’s chemistry layer treats \textbf{noble-gas closure} as the chemical manifestation
of the RS eight-tick \emph{neutral rest} constraint: closure occurs when the system returns to a
distinguished equilibrium boundary (a ``rest'' between periods).

In the Lean implementation, this is represented by a deterministic closure map on atomic number $Z$.
The core theorems proved are listed in \S\ref{sec:lean}.

\section{First principles used in this derivation}
This document is intentionally \emph{structural}: it does not fit parameters to chemical data.
The ``first principles'' here are the \emph{deterministic primitives} used by the engine:
\begin{itemize}[leftmargin=*]
  \item \textbf{Discrete atomic number} $Z\in\mathbb{N}$.
  \item \textbf{A fixed set of closure boundaries} (noble-gas endpoints) used to define
    the current period and valence position.
  \item \textbf{Derived quantities} such as ``distance to closure'', ``valence electrons'',
    and ``period length'' are defined in terms of the closure boundaries.
\end{itemize}
Once these primitives are fixed, the closure theorems are straightforward consequences.

\section{Definitions (mathematical form of the Lean primitives)}
\label{sec:defs}
Let the (first-six) noble-gas endpoints be the set
\[
  \mathcal{N} := \{2,10,18,36,54,86\}.
\]
Define the previous closure function $\mathrm{prev}(Z)$ and next closure function $\mathrm{next}(Z)$
as piecewise constants that select the adjacent endpoints surrounding $Z$ (with $\mathrm{next}(Z)$
extended to $118$ for $Z>86$ in the code base).

\begin{definition}[Distance to closure]
Define
\[
  \mathrm{dist}(Z) := \mathrm{next}(Z) - Z.
\]
\end{definition}

\begin{definition}[Valence electrons and period length]
Define
\[
  v(Z) := Z - \mathrm{prev}(Z),\qquad
  L(Z) := \mathrm{next}(Z) - \mathrm{prev}(Z).
\]
Intuitively, $v(Z)$ counts electrons beyond the previous closed shell, and $L(Z)$ is the length
of the period containing $Z$.
\end{definition}

\section{Derivations}

\subsection{Noble gases are exactly closure points (forward direction)}
\begin{lemma}
If $Z\in \mathcal{N}$, then $\mathrm{next}(Z)=Z$ and therefore $\mathrm{dist}(Z)=0$.
\end{lemma}
\begin{proof}
By definition, each $Z\in\{2,10,18,36,54,86\}$ is itself a designated closure endpoint.
The closure map $\mathrm{next}$ returns the endpoint of the current period; at a designated endpoint,
that endpoint is $Z$ itself. Thus $\mathrm{next}(Z)=Z$ and $\mathrm{dist}(Z)=\mathrm{next}(Z)-Z=0$.
\end{proof}

\subsection{Noble gases have complete shells}
\begin{lemma}
If $Z\in \mathcal{N}$, then $v(Z)=L(Z)$.
\end{lemma}
\begin{proof}
At a period endpoint $Z$, the next closure equals $Z$ itself, so
$L(Z)=\mathrm{next}(Z)-\mathrm{prev}(Z)=Z-\mathrm{prev}(Z)=v(Z)$.
\end{proof}

\subsection{Period lengths from noble-gas gaps}
Let the sequence of noble-gas endpoints be
\[
  (n_0,n_1,n_2,n_3,n_4,n_5)=(2,10,18,36,54,86).
\]
Define the derived gap lengths
\[
  \Delta_0 := n_0,\qquad \Delta_k := n_k - n_{k-1}\ \ (k=1,\dots,5).
\]
Then explicitly,
\[
  [\Delta_0,\Delta_1,\Delta_2,\Delta_3,\Delta_4,\Delta_5]
  = [2, 8, 8, 18, 18, 32].
\]
This is the period-length sequence for periods $1$ through $6$.

\section{Validation (prereg script + artifact)}
The preregistered validator \texttt{scripts/analysis/chem\_noble\_gas\_closure.py}
writes the artifact \texttt{artifacts/chem\_noble\_gas\_closure.json}.
The recorded run in the repository reports \textbf{PASS (4/4 tests)}.

\subsection{Test table: \texttt{distToNextClosure} at noble gases}
\begin{center}
\begin{tabular}{@{}r l r r@{}}
\toprule
$Z$ & Element & $\mathrm{dist}(Z)$ & Pass \\
\midrule
2  & He & 0 & true \\
10 & Ne & 0 & true \\
18 & Ar & 0 & true \\
36 & Kr & 0 & true \\
54 & Xe & 0 & true \\
86 & Rn & 0 & true \\
\bottomrule
\end{tabular}
\end{center}

\subsection{Test table: complete shell identity $v(Z)=L(Z)$ at noble gases}
\begin{center}
\begin{tabular}{@{}r l r r r@{}}
\toprule
$Z$ & Element & $v(Z)$ & $L(Z)$ & Pass \\
\midrule
2  & He & 2  & 2  & true \\
10 & Ne & 8  & 8  & true \\
18 & Ar & 8  & 8  & true \\
36 & Kr & 18 & 18 & true \\
54 & Xe & 18 & 18 & true \\
86 & Rn & 32 & 32 & true \\
\bottomrule
\end{tabular}
\end{center}

\section{Lean cross-references}
\label{sec:lean}
Lean module:
\begin{itemize}[leftmargin=*]
  \item \texttt{IndisputableMonolith/Chemistry/PeriodicTable.lean}
\end{itemize}

Key proved theorems (names as exported in the module):
\begin{itemize}[leftmargin=*]
  \item \texttt{noble\_gas\_at\_closure}: \texttt{distToNextClosure Z = 0} for \texttt{isNobleGas Z}
  \item \texttt{noble\_gas\_complete\_shell}: \texttt{valenceElectrons Z = periodLength Z} for \texttt{isNobleGas Z}
  \item \texttt{shell\_sum\_to\_noble}: the list of (first six) cumulative closures matches \texttt{[2,10,18,36,54,86]}
  \item \texttt{period\_lengths\_from\_noble\_gaps}: $[2,10{-}2,\dots,86{-}54]=[2,8,8,18,18,32]$
\end{itemize}

\paragraph{Artifact reference.}
Validation output: \texttt{artifacts/chem\_noble\_gas\_closure.json}.

\end{document}

