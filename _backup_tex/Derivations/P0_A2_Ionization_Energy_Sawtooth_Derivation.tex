\documentclass[11pt]{article}

\usepackage[margin=1in]{geometry}
\usepackage{amsmath, amssymb, amsthm}
\usepackage{booktabs}
\usepackage{hyperref}
\usepackage{enumitem}

\hypersetup{
  colorlinks=true,
  linkcolor=blue,
  urlcolor=blue,
  citecolor=blue
}

\newtheorem{theorem}{Theorem}
\newtheorem{lemma}{Lemma}
\newtheorem{definition}{Definition}

\title{P0-A2 Ionization Energy Sawtooth\\(Derivation Notes + Lean/NIST Validation Cross-References)}
\author{Recognition Science Derivation Campaign}
\date{2026-01-17}

\begin{document}
\maketitle

\begin{abstract}
This note documents the ionization-energy ``sawtooth'' claim (P0-A2) as implemented in the repository.
The Lean layer defines a \emph{fit-free} ionization proxy based on valence position inside a period,
and proves ordering theorems that force the sawtooth shape: alkali minima, noble maxima,
strict monotonicity within a fixed period, and the reset at a period boundary.
We append the preregistered validator tables comparing these predictions to NIST first ionization energies.
\end{abstract}

\section{Claim (P0-A2)}
First ionization energy $I_1(Z)$ exhibits a distinctive sawtooth pattern across the periodic table:
within each period it rises from the alkali metal to the noble gas, and then drops sharply at the next period.

In this repository, the fit-free, testable content is the \textbf{ordering}, not the absolute scale.
The ordering is mediated by a deterministic \emph{ionization proxy} defined from period/valence structure.

\section{First principles used in this derivation}
The ``first principles'' here are the deterministic period primitives inherited from P0-A0:
\begin{itemize}[leftmargin=*]
  \item The closure maps $\mathrm{prev}(Z)$ and $\mathrm{next}(Z)$, which determine the period boundaries.
  \item The valence position $v(Z)=Z-\mathrm{prev}(Z)$ and period length $L(Z)=\mathrm{next}(Z)-\mathrm{prev}(Z)$.
\end{itemize}
The ionization proxy is set equal to $v(Z)$.

\section{Definitions (mathematical form of the Lean primitives)}
\label{sec:defs}
\begin{definition}[Ionization proxy]
Define the ionization proxy
\[
  P(Z) := v(Z) = Z-\mathrm{prev}(Z).
\]
\end{definition}

\begin{definition}[Normalized and $\phi$-scaled display seams]
The Lean module also defines a normalized proxy and a $\phi$-scaled display quantity:
\[
  \mathrm{norm}(Z) := \frac{P(Z)}{L(Z)}\in[0,1],
\qquad
  \mathrm{scaled}(Z) := \phi^{2\cdot \mathrm{period}(Z)}\cdot \mathrm{norm}(Z),
\]
and a dimensionful display \texttt{predictedI1\_eV} obtained by multiplying by a universal anchor.
The ordering theorems in \S\ref{sec:theorems} are proved at the proxy level.
\end{definition}

\section{Derivations (proxy-level ordering)}
\label{sec:theorems}

\subsection{Alkali metals are period minima}
\begin{lemma}
If an element has $v(Z)=1$ (one electron beyond the previous closure), then $P(Z)=1$.
\end{lemma}
\begin{proof}
Immediate from the definition $P(Z)=v(Z)$.
\end{proof}
This is the formal content behind ``alkali metals have minimum proxy'':
alkalis are defined by the condition $v(Z)=1$ (excluding hydrogen).

\subsection{Noble gases are period maxima (proxy)}
\begin{lemma}
If $Z$ is a noble-gas endpoint (a closure), then $P(Z)=L(Z)$.
\end{lemma}
\begin{proof}
At a closure, $v(Z)=L(Z)$ (complete shell identity from P0-A0). Since $P(Z)=v(Z)$, we have $P(Z)=L(Z)$.
\end{proof}
Thus the proxy attains the maximum possible value in its period at the noble gas.

\subsection{Strict increase within a fixed period}
\begin{lemma}
If $Z_1<Z_2$ and $\mathrm{prev}(Z_1)=\mathrm{prev}(Z_2)$ (i.e.\ no period boundary is crossed),
then $P(Z_1)<P(Z_2)$.
\end{lemma}
\begin{proof}
With the shared previous closure constant $c:=\mathrm{prev}(Z_1)=\mathrm{prev}(Z_2)$,
\[
  P(Z_1)=Z_1-c < Z_2-c = P(Z_2).
\]
\end{proof}

\subsection{Sawtooth reset at a boundary}
\begin{lemma}
If $Z_\mathrm{noble}$ is a noble-gas endpoint and $Z_\mathrm{alkali}=Z_\mathrm{noble}+1$,
then $P(Z_\mathrm{alkali}) < P(Z_\mathrm{noble})$.
\end{lemma}
\begin{proof}
At the first element of a new period, the previous closure is the noble gas: $\mathrm{prev}(Z_\mathrm{alkali})=Z_\mathrm{noble}$.
Hence
\[
  P(Z_\mathrm{alkali}) = Z_\mathrm{alkali}-\mathrm{prev}(Z_\mathrm{alkali})
  = (Z_\mathrm{noble}+1)-Z_\mathrm{noble}=1.
\]
Meanwhile $P(Z_\mathrm{noble})=L(Z_\mathrm{noble})\ge 2$ for periods beyond the trivial case, so $1<P(Z_\mathrm{noble})$.
\end{proof}

\section{Validation against NIST first ionization energies}
The preregistered validator \url{scripts/analysis/chem_ionization_energy_compare.py}
writes \url{artifacts/chem_ionization_energy_comparison.json}.
The recorded run in the repository reports \textbf{PASS (3/3 hard tests)} against a built-in NIST table
for $Z\le 86$.

\subsection{Hard test 1: alkali minimum within period (main group)}
\begin{center}
\begin{tabular}{@{}r l r l l@{}}
\toprule
Alkali $Z$ & Element & $I_1$ (eV) & Period & Pass \\
\midrule
3  & Li & 5.392 & 3--10  & true \\
11 & Na & 5.139 & 11--18 & true \\
19 & K  & 4.341 & 19--36 & true \\
37 & Rb & 4.177 & 37--54 & true \\
55 & Cs & 3.894 & 55--86 & true \\
\bottomrule
\end{tabular}
\end{center}

\subsection{Hard test 2: noble maximum within period}
\begin{center}
\begin{tabular}{@{}r l r l l@{}}
\toprule
Noble $Z$ & Element & $I_1$ (eV) & Period & Pass \\
\midrule
2  & He & 24.587 & 1--2   & true \\
10 & Ne & 21.565 & 3--10  & true \\
18 & Ar & 15.760 & 11--18 & true \\
36 & Kr & 14.000 & 19--36 & true \\
54 & Xe & 12.130 & 37--54 & true \\
86 & Rn & 10.749 & 55--86 & true \\
\bottomrule
\end{tabular}
\end{center}

\subsection{Hard test 3: sawtooth reset across boundary}
\begin{center}
\begin{tabular}{@{}r l r@{\qquad}r l r@{\qquad}l@{}}
\toprule
Alkali $Z$ & Element & $I_1$ (eV) &
Prev. noble $Z$ & Element & $I_1$ (eV) &
Pass \\
\midrule
3  & Li & 5.392 &
2  & He & 24.587 & true \\
11 & Na & 5.139 &
10 & Ne & 21.565 & true \\
19 & K  & 4.341 &
18 & Ar & 15.760 & true \\
37 & Rb & 4.177 &
36 & Kr & 14.000 & true \\
55 & Cs & 3.894 &
54 & Xe & 12.130 & true \\
\bottomrule
\end{tabular}
\end{center}

\subsection{Note on ``proxy ordering'' violations}
The validator additionally counts local anomalies where the strictly increasing proxy does not match
strict increase in the NIST values (e.g.\ Be $\to$ B, N $\to$ O). These are recorded as informational
and are not treated as a hard falsifier in the current prereg.

\section{Lean cross-references}
Lean modules:
\begin{itemize}[leftmargin=*]
  \item \texttt{IndisputableMonolith/Chemistry/IonizationEnergy.lean}
  \item \texttt{IndisputableMonolith/Chemistry/PeriodicTable.lean} (period/valence primitives)
\end{itemize}

Key proved theorems (names as exported in the module):
\begin{itemize}[leftmargin=*]
  \item \texttt{alkali\_min\_ionization}: if \texttt{valenceElectrons Z = 1} then \texttt{ionizationProxy Z = 1}
  \item \texttt{noble\_max\_ionization}: if \texttt{isNobleGas Z} then \texttt{ionizationProxy Z = periodLength Z}
  \item \texttt{ionization\_monotone\_within\_period}: same-period implies strict increase in proxy
  \item \texttt{sawtooth\_reset}: if \texttt{Zalkali = Znoble + 1} then \texttt{proxy(Zalkali) < proxy(Znoble)}
\end{itemize}

\paragraph{Artifact reference.}
Validation output: \url{artifacts/chem_ionization_energy_comparison.json}.

\end{document}

