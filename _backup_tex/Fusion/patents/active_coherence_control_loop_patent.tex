\documentclass[12pt]{article}
\usepackage{geometry}
\usepackage{fancyhdr}
\usepackage{amsmath}
\usepackage{graphicx}
\usepackage{hyperref}

\geometry{a4paper, margin=1in}
\pagestyle{fancy}
\fancyhead[L]{Patent Application: Active Coherence Control Loop}
\fancyhead[R]{Confidential}

\title{SYSTEM AND METHOD FOR ACTIVE COHERENCE CONTROL IN FUSION REACTORS VIA FEED-FORWARD PHASE CORRECTION}
\author{Inventor: Jonathan Washburn}
\date{\today}

\begin{document}

\maketitle

\begin{abstract}
A system and method for feed-forward timing correction in pulsed-energy driver systems (including inertial confinement fusion (ICF) lasers). The controller measures (or receives as an input) a per-pulse timing error \(\delta t\) relative to a reference clock and issues a compensatory timing correction (e.g., \(-\delta t\), optionally with a declared deadband and saturation) before the pulse reaches the target. The method emits an auditable, hash-based runtime artifact binding inputs and outputs. In the committed Recognition Science (RS) fusion proxy stack, the implemented \(\varphi\)-coherence metric \(C_\varphi\in[0,1]\) is computed from timing errors and phases; reducing timing-error RMS increases the \texttt{timingScore} factor and may increase \(C_\varphi\) when other terms are held fixed, thereby decreasing the model-layer barrier-scale proxy \(S=1/(1+C_\varphi+C_\sigma)\) used by tunneling-rate enhancement proxies. Hardware measurement, actuation dynamics, and any claim about physical yield enhancement are treated as facility-specific calibration seams unless separately validated.
\end{abstract}

\section{Field of the Invention}
The present invention relates to nuclear fusion reactor control systems, specifically to timing/phase synchronization architectures for pulsed-energy drivers (including inertial confinement fusion (ICF) lasers). Any claim about an achieved timing precision regime (e.g., ``attosecond-scale'') is a facility-specific seam unless separately validated.

\section{Background}
Standard inertial confinement fusion (ICF) relies on coordinated delivery of pulsed energy across multiple beamlines. In the committed RS proxy stack used in this repository, timing/phase alignment is summarized by a dimensionless \(\varphi\)-coherence metric \(C_\varphi\) computed from expected vs.\ measured event times and channel phases, and combined with a ledger-sync metric \(C_\sigma\) to compute a barrier-scale proxy \(S=1/(1+C_\varphi+C_\sigma)\). Achieving high \(C_\varphi\) requires reducing timing errors; the measurement modality, the actuation mechanism, and any mapping from the proxy model to facility burn/yield remain explicit seams.

\section{Summary of the Invention}
The invention provides a "Measure-and-Delay" feed-forward architecture that corrects timing jitter in real-time on a pulse-by-pulse basis.

The system comprises:
\begin{enumerate}
    \item \textbf{Timing reference (optional):} any stable reference clock (e.g., an optical frequency comb) providing a timing/phase reference. (Integration seam)
    \item \textbf{Timing-error input:} a measurement device or interface providing a per-pulse timing error \(\delta t\). (Integration seam)
    \item \textbf{Compute unit:} a deterministic computation that maps \(\delta t\) and declared parameters (tolerance and saturation) to a correction command \(\Delta t_{\mathrm{corr}}\).
    \item \textbf{Optional delay path:} an optical/electrical delay element providing sufficient time for measurement and computation before target interaction. (Integration seam)
    \item \textbf{Optional actuator:} a timing/phase actuation element that attempts to apply \(\Delta t_{\mathrm{corr}}\) to the pulse. (Integration seam)
    \item \textbf{Audit artifact emitter:} a hash-based runtime artifact binding inputs and outputs for audit/replay.
\end{enumerate}

\section{Detailed Description}

\subsection{Definitions}
\begin{itemize}
  \item \textbf{Timing error (\(\delta t\))}: difference between a measured pulse time and its scheduled/expected time, \(\delta t := t_{\mathrm{meas}}-t_{\mathrm{exp}}\) (seconds).
  \item \textbf{Timing tolerance (\(\tau_{\mathrm{tol}}\))}: declared deadband for timing error (seconds), where \(|\delta t|\le \tau_{\mathrm{tol}}\Rightarrow \Delta t_{\mathrm{corr}}:=0\).
  \item \textbf{Saturation (\(\Delta t_{\max}\))}: declared maximum absolute correction command (seconds), where \(|\Delta t_{\mathrm{corr}}|\le \Delta t_{\max}\).
  \item \textbf{Feed-forward correction}: a computed correction command applied (or requested) before target interaction, e.g.,
  \(\Delta t_{\mathrm{corr}} := \mathrm{clamp}_{[-\Delta t_{\max},\Delta t_{\max}]}(-\delta t)\) with optional deadband \(\tau_{\mathrm{tol}}\).
  \item \textbf{\(\varphi\)-coherence (\(C_\varphi\))}: an implemented metric \(C_\varphi=\mathrm{clamp}_{[0,1]}(R\cdot \mathrm{timingScore}\cdot \mathrm{skewScore})\) computed from timing errors and phase alignment (Lean-aligned executable interface).
  \item \textbf{LedgerSync (\(C_\sigma\))}: an implemented synchronization metric \(C_\sigma\in[0,1]\) computed from a symmetry ledger and declared parameters (Lean-aligned executable interface).
  \item \textbf{Barrier scale (\(S\))}: model-layer proxy \(S = 1/(1 + C_\varphi + C_\sigma)\) used by the committed tunneling enhancement proxy (Lean-aligned executable interface).
  \item \textbf{Certificate bundle}: a structured, auditable runtime artifact including an input hash, outputs, and pass/fail status (hash-based by default; external signing optional seam).
\end{itemize}

\subsection{The Feed-Forward Architecture}
Unlike feedback loops which correct \textit{subsequent} pulses based on past errors, this invention uses a feed-forward topology to correct \textit{the current pulse}. The main pulse traverses an optional delay path while the measurement unit provides the error and the compute unit calculates a correction command. The required delay length and actuator response characteristics are facility-specific integration seams.

\subsection{Optional Multi-Stage Correction (Integration Seam)}
To accommodate facility-specific dynamic range and bandwidth requirements, an implementation may employ a coarse stage (large-range, low-bandwidth) and a fine stage (small-range, high-bandwidth). Specific actuator technologies, ranges, and achieved timing precision are facility/hardware seams and are not asserted by the proxy computation.
\begin{itemize}
    \item \textbf{Coarse stage (example)}: a delay element for large-range correction.
    \item \textbf{Fine stage (example)}: an electro-optic phase element for small-range correction.
\end{itemize}

\subsection{Single-Shot Measurement}
The timing-error input may be supplied by a measurement unit employing single-shot cross-correlation (e.g., spectral interferometry or non-linear cross-correlation) to estimate the timing offset between a pulse and a reference. The measurement modality and its accuracy/latency are integration seams.

\subsection{Model-Layer Barrier-Scale Proxy and Seams}
In the committed RS proxy model, the barrier-scale proxy is computed as \(S = 1/(1 + C_\varphi + C_\sigma)\) with \(C_\varphi,C_\sigma\in[0,1]\). This disclosure claims the implemented computation of timing correction commands and auditable records, and the use of the Lean-aligned \(C_\varphi\) and \(S\) computations as a control objective/proxy. Any physical mechanism interpretation (e.g., vacuum polarization or dielectric screening) and any claim of facility-independent fusion yield enhancement are explicit seams unless separately validated.

\section{Claims}

\begin{enumerate}
\item A computer-implemented control system for generating feed-forward timing correction commands for a pulsed-energy driver, comprising:
    \begin{itemize}
        \item An input interface configured to receive a per-pulse timing error \(\delta t\) relative to a reference clock;
        \item A processor configured to compute a correction command \(\Delta t_{\mathrm{corr}}\) as a deterministic function of \(\delta t\) and one or more declared parameters including a tolerance \(\tau_{\mathrm{tol}}\) and a saturation limit \(\Delta t_{\max}\); and
        \item An artifact generator configured to emit a hash-based certificate bundle binding the received \(\delta t\), the declared parameters, and the computed \(\Delta t_{\mathrm{corr}}\).
    \end{itemize}
    
    \item The system of claim 1, further comprising an interface to a timing/phase actuator configured to attempt to apply \(\Delta t_{\mathrm{corr}}\) to a pulse, wherein any achieved physical timing precision is a facility-specific seam.
    
    \item The system of claim 1, wherein the deterministic function computes \(\Delta t_{\mathrm{corr}} := -\delta t\) subject to a deadband \(|\delta t|\le \tau_{\mathrm{tol}}\Rightarrow \Delta t_{\mathrm{corr}}:=0\) and saturation \(|\Delta t_{\mathrm{corr}}|\le \Delta t_{\max}\).
    
    \item The system of claim 1, wherein the certificate bundle includes an input hash and pass/fail status, and external signing is an optional integration seam.
    
    \item A computer-implemented method for generating feed-forward timing correction commands and auditable proxy-metric records, comprising:
    \begin{itemize}
        \item Receiving a timing error \(\delta t\) and declared parameters including \(\tau_{\mathrm{tol}}\) and \(\Delta t_{\max}\);
        \item Computing a correction command \(\Delta t_{\mathrm{corr}}\) and an estimated residual timing error under the command;
        \item Optionally computing an implemented \(\varphi\)-coherence metric \(C_\varphi\) from expected vs.\ measured times and channel phases, and computing a model-layer barrier-scale proxy \(S=1/(1+C_\varphi+C_\sigma)\) from \(C_\varphi\) and an input \(C_\sigma\); and
        \item Recording a hash-based certificate bundle binding the inputs and outputs, treating any mapping from \(C_\varphi\) or \(S\) to physical yield as an explicit seam.
    \end{itemize}
\end{enumerate}

\section*{APPENDIX: Implementation Evidence}
The proxy-layer computations referenced by this disclosure are implemented in:
\begin{itemize}
  \item \texttt{fusion/simulator/coherence/phi\_coherence.py}: \texttt{compute\_phi\_coherence} (\(C_\varphi\), Lean-aligned)
  \item \texttt{fusion/simulator/coherence/barrier\_scale.py}: \texttt{compute\_rs\_barrier\_scale} (\(S\), Lean-aligned)
  \item \texttt{fusion/simulator/fusion/certificate.py}: \texttt{CertificateBundle}, \texttt{compute\_input\_hash} (hash-based audit artifacts)
  \item \texttt{fusion/simulator/control/feedforward\_phase\_correction.py}: feed-forward correction computation + hash-based certificate emission (implementation evidence for this filing)
\end{itemize}

\end{document}
