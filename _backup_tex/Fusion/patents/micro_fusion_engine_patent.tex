\documentclass[12pt]{article}
\usepackage{geometry}
\usepackage{fancyhdr}
\usepackage{amsmath}
\usepackage{hyperref}
\usepackage{enumitem}

\geometry{letterpaper, margin=1in}
\pagestyle{fancy}
\fancyhead[L]{Patent Draft: Micro-Scale Fusion Control System}
\fancyhead[R]{Confidential}
\fancyfoot[C]{\thepage}
\setlength{\headheight}{14.5pt}

\title{\textbf{SYSTEMS AND METHODS FOR CONTROL OF MICRO-SCALE FUSION VIA COHERENCE-BASED SCALING LAWS}}
\author{Inventor: Jonathan Washburn}
\date{\today}

\begin{document}
\maketitle

\begin{abstract}
A system and method for controlling micro-scale fusion targets using high-repetition-rate optical pulses and auditable proxy-model scaling. In the committed Recognition Science (RS) proxy model, a barrier-scale factor \(S = 1/(1 + C_\phi + C_\sigma)\) is computed from coherence metrics; this model predicts that high coherence (\(S \approx 1/3\)) enhances tunneling probability by orders of magnitude, theoretically reducing the required confinement parameter (\(\rho R\)) and enabling micro-scale regimes. The invention claims the method of computing these scaling requirements, generating synchronized pulse schedules (e.g., \(\phi\)-spaced), and emitting hash-based audit artifacts. Physical realization of ignition, vacuum polarization mechanisms, and specific driver hardware (e.g., milliJoule lasers) are treated as explicit facility seams unless separately validated.
\end{abstract}

\section{Technical Field}
The present disclosure relates to nuclear fusion control systems, specifically to methods for computing scaling laws and synchronizing drivers for micro-scale inertial confinement fusion (ICF) under a coherence-based proxy model.

\section{Background}
Traditional Inertial Confinement Fusion (ICF) relies on thermal compression of large fuel pellets to achieve ignition. This approach typically requires massive driver energy. Recognition Science (RS) theory proposes a model where coherent electromagnetic fields modify effective tunneling probabilities via a barrier-scale factor \(S\). This disclosure describes a control architecture based on this model.

\section{Summary}
The invention provides a control architecture for a "Micro-Fusion Engine" that operates on the principle of high-frequency target targeting, governed by RS scaling laws.

\subsection*{Core Concept (Model-Layer)}
The system computes control parameters for a stream of micro-scale targets. A driver controller generates phase-coherent pulse schedules intended to maximize the RS coherence metrics \(C_\phi\) and \(C_\sigma\), thereby minimizing the barrier scale \(S\). The proxy model predicts that this enables fusion at reduced confinement scales.

\subsection*{Seams and Assumptions}
\begin{itemize}
    \item \textbf{Physical Mechanism}: The physical mechanism (e.g., vacuum polarization) is a model assumption.
    \item \textbf{Hardware}: Specific driver technologies (lasers, injectors) are integration seams.
    \item \textbf{Yield}: Actual fusion yield is an empirical result, not guaranteed by the proxy computation.
\end{itemize}

\section{Detailed Description}

\subsection{Scaling Laws (Proxy Model)}
The invention leverages the RS scaling law for barrier reduction: \(S = 1 / (1 + C_\phi + C_\sigma)\). In the committed proxy model (implemented in \texttt{simulator.coherence}), high coherence (\(S \approx 1/3\)) increases the effective tunneling probability. The system computes the required confinement parameter \(\rho R\) based on this enhanced probability. For example, the model predicts a reduction in required \(\rho R\) by orders of magnitude under optimal coherence, theoretically allowing target radius reduction.

\subsection{System Architecture}
The system comprises:
\begin{enumerate}
    \item \textbf{Target Interface}: A control interface for a micro-target injector (hardware seam).
    \item \textbf{Coherent Driver Controller}: A computing unit that generates \(\phi\)-spaced pulse schedules and phase commands (implemented in \texttt{pulse\_scheduler.py}).
    \item \textbf{Synchronization Unit}: Logic to align laser firing with target tracking data (integration seam).
    \item \textbf{Audit Emitter}: Generates hash-based certificates binding scaling parameters and pulse schedules.
\end{enumerate}

\subsection{Operating Regime (Target)}
\begin{itemize}
    \item \textbf{Fuel}: D-T, D-D, or p-B11 (parameter).
    \item \textbf{Target Size}: Micron-scale (parameter).
    \item \textbf{Pulse Schedule}: \(\phi\)-spaced train (computed).
    \item \textbf{Repetition Rate}: High frequency (parameter).
\end{itemize}

\section{Claims}

\begin{enumerate}[leftmargin=1.6em]
    \item A control system for micro-scale fusion, comprising:
    \begin{itemize}
        \item A computing unit configured to calculate a barrier-scale proxy \(S = 1/(1 + C_\phi + C_\sigma)\) from declared coherence parameters;
        \item A scheduler configured to generate a sequence of optical pulse times derived from the Golden Ratio (\(\phi\));
        \item An interface configured to output control signals for a driver targeting micro-scale fuel targets;
        \item An artifact generator configured to emit a hash-based certificate binding the calculated scaling factors and pulse schedule.
    \end{itemize}

    \item The system of claim 1, wherein the computing unit calculates a required confinement parameter reduction factor based on the barrier-scale proxy.

    \item The system of claim 1, wherein the scheduler generates a pulse train intended for a repetition rate exceeding 100 Hertz.

    \item The system of claim 1, wherein the control signals are formatted for a driver delivering energy to targets with a radius less than 50 microns.

    \item A method for configuring a fusion driver, comprising:
    \begin{itemize}
        \item Defining a target micro-scale fuel radius;
        \item Computing a required barrier-scale proxy \(S\) to enable ignition at said radius under a proxy model;
        \item Generating a phase-coherent pulse schedule to achieve said barrier-scale proxy;
        \item Recording the parameters and schedule in an auditable certificate.
    \end{itemize}
\end{enumerate}

\section*{APPENDIX: Implementation Evidence}
The proxy-layer computations referenced by this disclosure are implemented in:
\begin{itemize}
  \item \texttt{fusion/simulator/coherence/barrier\_scale.py}: \texttt{compute\_rs\_barrier\_scale} (\(S\), Lean-aligned)
  \item \texttt{fusion/simulator/simulate\_micro\_scaling.py}: \texttt{compute\_micro\_scaling}, \texttt{generate\_micro\_scaling\_certificate} (scaling law implementation)
  \item \texttt{fusion/simulator/fusion/pulse\_scheduler.py}: \texttt{generate\_phi\_schedule} (schedule generation)
\end{itemize}

\end{document}
