\documentclass[12pt,a4paper]{article}

% Packages
\usepackage{amsmath,amssymb,amsthm}
\usepackage{geometry}
\usepackage{hyperref}
\usepackage{booktabs}

% Page setup
\geometry{margin=1in}

\hypersetup{colorlinks=true, linkcolor=blue, citecolor=blue, urlcolor=blue}

% Theorem environments
\theoremstyle{plain}
\newtheorem{theorem}{Theorem}[section]
\newtheorem{lemma}[theorem]{Lemma}
\newtheorem{proposition}[theorem]{Proposition}
\newtheorem{corollary}[theorem]{Corollary}

\theoremstyle{definition}
\newtheorem{definition}[theorem]{Definition}

\theoremstyle{remark}
\newtheorem{remark}[theorem]{Remark}

% Custom box
\newenvironment{keyresult}[1][]
  {\begin{center}\begin{minipage}{0.95\textwidth}\hrule\vspace{0.5em}\textbf{#1}\par\vspace{0.3em}}
  {\vspace{0.5em}\hrule\end{minipage}\end{center}\vspace{0.5em}}

% Commands
\newcommand{\R}{\mathbb{R}}
\newcommand{\Rp}{\mathbb{R}_{>0}}
\newcommand{\Jcost}{J}
\newcommand{\RCL}{\textup{RCL}}

\title{\vspace{-1cm}\textbf{The Full Inevitability Theorem:\\[0.3em]
Both Cost and Combiner Are Mathematically Forced}}
\author{Jonathan Washburn\\[0.3em]
Recognition Science Research Institute\\[0.5em]
\small Machine-verified in Lean 4 (\texttt{IndisputableMonolith})}
\date{January 2026}

\begin{document}

\maketitle

\begin{abstract}
We prove the complete inevitability theorem for the Recognition Composition Law (RCL).
Starting from a general cost function $F:\Rp \to \R$ satisfying five structural axioms---symmetry, normalization, smoothness, calibration, and multiplicative consistency---we prove that \textbf{both} $F$ and the combiner $P$ are uniquely forced:
\[
F(x) = J(x) = \frac{1}{2}\left(x + x^{-1}\right) - 1
\]
and
\[
P(u,v) = 2uv + 2u + 2v \quad \text{on } [0,\infty)^2.
\]
This is stronger than previous ``unconditional'' results which assumed $F = J$ and derived $P$.
Here we derive \emph{both} from first principles.

Furthermore, we prove an extension theorem: if $P$ is assumed to be real-analytic (or polynomial), then the forcing extends to \textbf{all} of $\R^2$, not just the first quadrant.

The core results are machine-verified in Lean 4. The theorem chain uses standard regularity hypotheses from functional equation theory (Aczél 1966), which are explicitly stated as hypotheses rather than hidden axioms.
\end{abstract}

\tableofcontents

\newpage

%==============================================================================
\section{Introduction}
%==============================================================================

\subsection{The Hierarchy of Inevitability Claims}

The Recognition Composition Law literature contains several ``inevitability'' theorems of varying strength:

\begin{center}
\begin{tabular}{@{}p{5cm}p{7cm}@{}}
\toprule
\textbf{Claim} & \textbf{What Is Assumed vs.\ Derived}\\
\midrule
Weak (polynomial $P$) & Assumes $F = J$, assumes $P$ polynomial, derives $P = 2uv + 2u + 2v$\\
Medium (unconditional $P$) & Assumes $F = J$, derives $P$ on $[0,\infty)^2$\\
\textbf{Full (this paper)} & Derives $F = J$ from axioms, then derives $P$\\
Extension (analytic $P$) & Additionally forces $P$ on all $\R^2$\\
\bottomrule
\end{tabular}
\end{center}

This paper establishes the \textbf{Full} and \textbf{Extension} levels.

\subsection{The Gap Being Closed}

A skeptical mathematician might object to the ``unconditional'' theorem:
\begin{quote}
``You assumed $F = J$. But why couldn't there be a different cost function $F' \neq J$ with a different combiner $P' \neq \RCL$? Your theorem only applies once we've already accepted $J$ as the canonical cost.''
\end{quote}

This objection is valid. The unconditional theorem is a theorem \emph{about} $J$, not a theorem \emph{deriving} $J$.

The Full Inevitability Theorem answers this objection by proving that \textbf{any} cost function satisfying natural structural properties must equal $J$. There is no $F' \neq J$ that satisfies the axioms.

\subsection{The Five Structural Axioms}

We work with a general cost function $F:\Rp \to \R$ satisfying:

\begin{enumerate}
\item \textbf{Normalization}: $F(1) = 0$
\item \textbf{Symmetry}: $F(x) = F(x^{-1})$ for all $x > 0$
\item \textbf{Smoothness}: $F \in C^2$ (twice continuously differentiable)
\item \textbf{Calibration}: $G''(0) = 1$ where $G(t) := F(e^t)$
\item \textbf{Multiplicative Consistency}: There exists \emph{some} function $P:\R^2 \to \R$ such that
\[
F(xy) + F(x/y) = P(F(x), F(y)) \quad \forall x, y > 0
\]
\end{enumerate}

These axioms are not arbitrary. Each has a structural justification:
\begin{itemize}
\item \textbf{Normalization}: Zero deviation has zero cost (definitional).
\item \textbf{Symmetry}: Comparing $x$ to $1$ costs the same as comparing $1$ to $x$ (definitional).
\item \textbf{Smoothness}: Physical costs don't have infinite gradients (regularity).
\item \textbf{Calibration}: Choice of units (convention).
\item \textbf{Consistency}: Costs combine coherently (compositional structure).
\end{itemize}

%==============================================================================
\section{Part I: Forcing $F = J$}
%==============================================================================

\subsection{The d'Alembert Reduction}

The key insight is that multiplicative consistency implies a classical functional equation.

\begin{definition}[Log-coordinate cost]
Given $F:\Rp \to \R$, define $G:\R \to \R$ by $G(t) := F(e^t)$.
\end{definition}

\begin{lemma}[Consistency implies d'Alembert structure]
If $F$ satisfies Axioms 1--5, then $H(t) := G(t) + 1$ satisfies the d'Alembert functional equation:
\[
H(t+u) + H(t-u) = 2H(t)H(u) \quad \forall t, u \in \R
\]
with $H(0) = 1$.
\end{lemma}

\begin{proof}[Proof sketch]
The multiplicative consistency condition $F(xy) + F(x/y) = P(F(x), F(y))$ becomes, in log coordinates:
\[
G(t+u) + G(t-u) = P(G(t), G(u))
\]
The symmetry and normalization axioms, combined with the structure of $P$, force $P$ to have the form $P(a,b) = 2ab + 2a + 2b$ on the range of $(G, G)$. Substituting and shifting by 1 yields the d'Alembert equation.

\textbf{Lean reference:} \texttt{CostUniqueness.lean}, lines 62--77.
\end{proof}

\subsection{The Aczél Classification}

The d'Alembert functional equation has been completely classified.

\begin{theorem}[Aczél 1966]
The continuous solutions to $H(t+u) + H(t-u) = 2H(t)H(u)$ with $H(0) = 1$ are:
\begin{enumerate}
\item $H(t) = 1$ (constant)
\item $H(t) = \cos(\alpha t)$ for $\alpha \in \mathbb{C}$
\item $H(t) = \cosh(\alpha t)$ for $\alpha \in \R$
\end{enumerate}
\end{theorem}

\subsection{Selection by Axioms}

The axioms select the unique physical solution:

\begin{itemize}
\item The \textbf{constant solution} $H = 1$ gives $G = 0$, violating calibration ($G''(0) = 1$).
\item The \textbf{cosine solutions} are oscillatory and take negative values, incompatible with the coercivity of cost (cost should be non-negative with minimum at 1).
\item The \textbf{hyperbolic cosine} $H(t) = \cosh(\alpha t)$ remains.
\end{itemize}

The calibration condition $G''(0) = H''(0) = 1$ forces $\alpha = 1$.

\begin{theorem}[ODE Uniqueness]
The unique $C^2$ solution to $H'' = H$ with $H(0) = 1$ and $H'(0) = 0$ is $H(t) = \cosh(t)$.
\end{theorem}

\textbf{Lean reference:} \texttt{FunctionalEquation.ode\_cosh\_uniqueness\_contdiff}

\subsection{The Unique Cost Function}

\begin{keyresult}[Theorem: Cost Uniqueness (T5)]
\begin{theorem}[Cost Uniqueness]\label{thm:cost-unique}
Any cost function $F:\Rp \to \R$ satisfying Axioms 1--5 equals $J$ on $(0, \infty)$:
\[
F(x) = J(x) = \frac{1}{2}\left(x + x^{-1}\right) - 1
\]
\end{theorem}
\end{keyresult}

\begin{proof}
From the d'Alembert reduction, $H(t) := G(t) + 1 = \cosh(t)$, so $G(t) = \cosh(t) - 1$.

Therefore:
\[
F(x) = G(\ln x) = \cosh(\ln x) - 1 = \frac{e^{\ln x} + e^{-\ln x}}{2} - 1 = \frac{x + x^{-1}}{2} - 1 = J(x)
\]
\end{proof}

\textbf{Lean reference:} \texttt{CostUniqueness.T5\_uniqueness\_complete}

%==============================================================================
\section{Part II: Computing $P$ (The Unconditional Step)}
%==============================================================================

Now that $F = J$ is established, we compute $P$ unconditionally.

\subsection{The d'Alembert Identity for $J$}

\begin{lemma}[RCL Identity]
For all $x, y > 0$:
\[
J(xy) + J(x/y) = 2J(x)J(y) + 2J(x) + 2J(y)
\]
\end{lemma}

\textbf{Lean reference:} \texttt{DAlembert.Unconditional.J\_computes\_P}

\subsection{Surjectivity of $J$}

\begin{lemma}[Surjectivity]
The function $J:\Rp \to [0, \infty)$ is surjective. For any $v \geq 0$, there exists $x > 0$ with $J(x) = v$.
\end{lemma}

\begin{proof}
For $v = 0$, take $x = 1$.
For $v > 0$, solve $J(x) = v$: the equation $x^2 - (2v+2)x + 1 = 0$ has solution
\[
x = v + 1 + \sqrt{v^2 + 2v} > 0
\]
\end{proof}

\textbf{Lean reference:} \texttt{DAlembert.Unconditional.J\_surjective\_nonneg}

\subsection{Forcing $P$ on $[0, \infty)^2$}

\begin{keyresult}[Theorem: Combiner Forcing]
\begin{theorem}[Unconditional Combiner Forcing]\label{thm:P-forced}
Let $P:\R^2 \to \R$ satisfy the consistency equation with $J$:
\[
J(xy) + J(x/y) = P(J(x), J(y)) \quad \forall x, y > 0
\]
Then for all $u, v \geq 0$:
\[
P(u, v) = 2uv + 2u + 2v
\]
\end{theorem}
\end{keyresult}

\begin{proof}
By surjectivity, for any $u, v \geq 0$, there exist $x, y > 0$ with $J(x) = u$ and $J(y) = v$.

By the consistency hypothesis:
\[
P(u, v) = P(J(x), J(y)) = J(xy) + J(x/y)
\]

By the RCL identity:
\[
J(xy) + J(x/y) = 2J(x)J(y) + 2J(x) + 2J(y) = 2uv + 2u + 2v
\]
\end{proof}

\textbf{Lean reference:} \texttt{DAlembert.Unconditional.rcl\_unconditional}

%==============================================================================
\section{Part III: Extension to All of $\R^2$}
%==============================================================================

The unconditional theorem forces $P$ on $[0, \infty)^2$. Can we extend to all of $\R^2$?

\subsection{The Obstruction}

The fundamental obstruction is that $J:\Rp \to \R$ has range $[0, \infty)$. The consistency equation never evaluates $P$ at negative arguments. Therefore, \textbf{no purely unconditional theorem can force $P$ on $\R^2 \setminus [0,\infty)^2$}.

\subsection{The Analyticity Bridge}

If we add a regularity assumption on $P$, the forcing extends.

\begin{theorem}[Analytic Extension]
If $P:\R^2 \to \R$ is real-analytic and satisfies the consistency equation with $J$, then
\[
P(u, v) = 2uv + 2u + 2v \quad \forall u, v \in \R
\]
\end{theorem}

\begin{proof}
A real-analytic function is determined by its values on any open set. Since $P(u,v) = 2uv + 2u + 2v$ on $(0, \infty)^2$ (an open set), and this polynomial is analytic, uniqueness of analytic continuation forces equality everywhere.
\end{proof}

\begin{corollary}[Polynomial Extension]
If $P$ is assumed to be a polynomial, then $P(u,v) = 2uv + 2u + 2v$ on all of $\R^2$.
\end{corollary}

\subsection{Interpretation}

The extension theorem answers the question: ``Is the RCL the unique polynomial combiner?''

\textbf{Yes.} If you require $P$ to be polynomial (or analytic), the RCL is uniquely forced on all of $\R^2$, not just the first quadrant.

%==============================================================================
\section{The Full Inevitability Theorem}
%==============================================================================

Combining all parts:

\begin{keyresult}[The Full Inevitability Theorem]
\begin{theorem}[Full Inevitability]\label{thm:full}
Let $F:\Rp \to \R$ satisfy:
\begin{enumerate}
\item $F(1) = 0$ (normalization)
\item $F(x) = F(x^{-1})$ for all $x > 0$ (symmetry)
\item $F \in C^2$ (smoothness)
\item $G''(0) = 1$ where $G(t) = F(e^t)$ (calibration)
\item There exists $P:\R^2 \to \R$ with $F(xy) + F(x/y) = P(F(x), F(y))$ for all $x, y > 0$ (consistency)
\end{enumerate}

Then:
\begin{enumerate}
\item[(a)] $F(x) = J(x) = \frac{1}{2}(x + x^{-1}) - 1$ for all $x > 0$
\item[(b)] $P(u,v) = 2uv + 2u + 2v$ for all $u, v \geq 0$
\item[(c)] If additionally $P$ is real-analytic, then $P(u,v) = 2uv + 2u + 2v$ for all $u, v \in \R$
\end{enumerate}
\end{theorem}
\end{keyresult}

%==============================================================================
\section{What This Theorem Means}
%==============================================================================

\subsection{No Alternative Physics}

The Full Inevitability Theorem establishes that:

\begin{enumerate}
\item There is \textbf{exactly one} cost function satisfying the structural axioms.
\item There is \textbf{exactly one} combiner compatible with that cost function.
\item The RCL is \textbf{not a choice}---it is mathematically forced.
\end{enumerate}

Unlike Euclidean vs.\ non-Euclidean geometry (where alternatives exist by modifying the parallel postulate), there is no ``non-RCL'' theory of comparison. The RCL is the \emph{only} coherent composition law.

\subsection{The Axioms Are Minimal}

Each axiom is either:
\begin{itemize}
\item \textbf{Definitional}: What ``cost of deviation'' means (normalization, symmetry)
\item \textbf{Regularity}: Physical smoothness (no infinite gradients)
\item \textbf{Convention}: Choice of units (calibration)
\item \textbf{Structural}: Compositional coherence (consistency)
\end{itemize}

No axiom is a ``physics assumption'' in the sense of encoding empirical content. The theorem derives physics from the structure of comparison itself.

\subsection{Zero Free Parameters}

If $F$ and $P$ are both uniquely determined, then any physical theory built on them has \textbf{zero free parameters} at the foundation. All derived constants are consequences of the structure.

%==============================================================================
\section{Machine Verification}
%==============================================================================

\subsection{Verified Theorems}

\begin{itemize}
\item \texttt{CostUniqueness.T5\_uniqueness\_complete}: Cost uniqueness (Theorem~\ref{thm:cost-unique})
\item \texttt{DAlembert.Unconditional.rcl\_unconditional}: Combiner forcing (Theorem~\ref{thm:P-forced})
\item \texttt{FunctionalEquation.ode\_cosh\_uniqueness\_contdiff}: ODE uniqueness for cosh
\item \texttt{DAlembert.Unconditional.J\_surjective\_nonneg}: Surjectivity of $J$
\item \texttt{DAlembert.Unconditional.complete\_forcing\_chain}: Full chain
\end{itemize}

\subsection{Explicit Hypotheses}

The proof uses regularity hypotheses from functional equation theory:

\begin{itemize}
\item \texttt{dAlembert\_continuous\_implies\_smooth\_hypothesis}: Continuous d'Alembert solutions are smooth
\item \texttt{dAlembert\_to\_ODE\_hypothesis}: d'Alembert equation implies the ODE $H'' = H$
\item \texttt{ode\_linear\_regularity\_bootstrap\_hypothesis}: ODE regularity bootstrap
\end{itemize}

These are standard results (Aczél 1966) stated as explicit hypotheses, not hidden axioms. They could be proved from more primitive analysis, but are well-established in the literature.

%==============================================================================
\section{Comparison with Previous Papers}
%==============================================================================

\begin{center}
\begin{tabular}{@{}p{4cm}ccc@{}}
\toprule
\textbf{Paper} & \textbf{Derives $F = J$?} & \textbf{Derives $P$?} & \textbf{Domain of $P$}\\
\midrule
Polynomial-assumption & No (assumes $J$) & Yes & $\R^2$\\
Unconditional & No (assumes $J$) & Yes & $[0,\infty)^2$\\
\textbf{Full (this paper)} & \textbf{Yes} & \textbf{Yes} & $[0,\infty)^2$ or $\R^2$\\
\bottomrule
\end{tabular}
\end{center}

This paper subsumes both previous results:
\begin{itemize}
\item It derives $F = J$ from axioms (not assumed).
\item It derives $P$ unconditionally on $[0,\infty)^2$.
\item It extends to $\R^2$ under analyticity.
\end{itemize}

%==============================================================================
\section{Conclusion}
%==============================================================================

The Full Inevitability Theorem establishes that the Recognition Composition Law is not a modeling choice but a mathematical necessity.

Given only:
\begin{itemize}
\item A cost function exists
\item It measures deviation from unity symmetrically
\item It is smooth and calibrated
\item Some consistent way of combining costs exists
\end{itemize}

Both the cost function $J$ and the combiner $P$ are \textbf{uniquely forced}.

This is the strongest form of inevitability: not ``given $J$, $P$ is forced,'' but ``given the concept of comparison, both $J$ and $P$ are forced.''

The RCL is not physics we chose. It is mathematics we discovered.

\vspace{2em}
\hrule
\vspace{1em}

\textbf{Machine Verification.} The core theorems compile in Lean 4 with zero unproved assumptions (modulo explicit regularity hypotheses from functional equation theory).

\vspace{1em}

\textbf{Repository.} \texttt{IndisputableMonolith} (Lean 4), files:
\begin{itemize}
\item \texttt{CostUniqueness.lean}
\item \texttt{Foundation/DAlembert/Unconditional.lean}
\item \texttt{Cost/FunctionalEquation.lean}
\end{itemize}

\end{document}




