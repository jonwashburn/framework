\documentclass[12pt,superscriptaddress,preprint,nofootinbib,prd,floatfix]{revtex4-2}

% Packages
\usepackage[utf8]{inputenc}
\usepackage{amsmath,amssymb,mathtools}
\usepackage{amsthm}
\usepackage{graphicx}
\usepackage{xcolor}
\usepackage{url}
\usepackage{enumerate}
\usepackage{cancel}
\usepackage{hyperref}

% Table of contents formatting: wider space for Roman numerals (revtex compatible)
\makeatletter
\renewcommand*\l@section{\@dottedtocline{1}{0em}{2.5em}}
\renewcommand*\l@subsection{\@dottedtocline{2}{2.5em}{3.5em}}
\renewcommand*\l@subsubsection{\@dottedtocline{3}{6em}{4.5em}}
\makeatother

% Theorem environments
\newtheorem{theorem}{Theorem}[section]
\newtheorem{lemma}[theorem]{Lemma}
\newtheorem{axiom}[theorem]{Axiom}
\newtheorem{proposition}[theorem]{Proposition}
\newtheorem{corollary}[theorem]{Corollary}

%\theoremstyle{definition}
\newtheorem{definition}[theorem]{Definition}
\newtheorem{remark}[theorem]{Remark}
\newtheorem{observation}[theorem]{Observation}

% (Removed) Boxed model definition helper from earlier drafts.
% This version presents model equations directly (no boxes) to match standard journal style.

% Section and subsection numbering format: I, II, III for sections; I.1, I.2, II.1, II.2 for subsections
\makeatletter
\renewcommand{\thesection}{\Roman{section}}
\renewcommand{\thesubsection}{\arabic{subsection}}
\renewcommand{\thesubsubsection}{\arabic{subsubsection}}
\renewcommand{\p@subsection}{\thesection.}
\renewcommand{\p@subsubsection}{\thesection.\thesubsection.}
\makeatother

%---------------------------------------------------------------------
\begin{document}

\title{Information-Limited Gravity I: A source-side theoretical framework}

\author{Jonathan Washburn}
\affiliation{Recognition Physics Institute, Austin, Texas, USA}

\author{Megan Simons}
\affiliation{Recognition Physics Institute, Austin, Texas, USA}

\author{Elshad Allahyarov}
\affiliation{Recognition Physics Institute, Austin, Texas, USA}
\affiliation{Institut für Theoretische Physik II: Weiche Materie,\\
Heinrich-Heine Universität Düsseldorf, 40225 Düsseldorf, Germany}
\affiliation{Theoretical Department, Joint Institute for High Temperatures,\\
Russian Academy of Sciences (IVTAN), Moscow, Russia}
\affiliation{Department of Physics, Case Western Reserve University,\\
Cleveland, Ohio 44106, USA}

\begin{abstract}
We present Information-Limited Gravity (ILG), a fixed-parameter, source-side modification of the cosmological Poisson equation in which the matter source is rescaled by a kernel
\begin{equation}
w(k,a)=1+C\,X^{-\alpha},\qquad X\equiv k\tau_0/a,
\end{equation}
with constants $(C,\alpha,\tau_0)$ specified independently of cosmological data. Paper~I establishes the theoretical framework and its mathematical consistency in the quasi-static, sub-horizon, linear regime: (i)~the $X$-dependence induces X-universality and a reciprocity relation linking time and scale derivatives; (ii)~for $\alpha\in(0,1/2)$ the modified Poisson problem is well-posed and numerically stable; (iii)~ILG produces zero Buchert backreaction at linear order, so background distance observables remain unchanged while perturbation-level observables (growth, lensing, ISW) acquire scale-dependent signatures. We provide qualitative falsification criteria based on ISW suppression, scale-dependent growth, and tracer-independent $E_G$ with $X$-collapse. Quantitative $\Lambda$CDM predictions and forecast-level comparisons are deferred to Paper~II (in preparation \cite{PaperII}). The principal theoretical limitation of the present work is the absence of a covariant relativistic completion.
\end{abstract}

\maketitle

\newpage

\tableofcontents

\newpage

\section{Introduction}
\label{sec:introduction}

The $\Lambda$CDM paradigm, discovered through Type Ia supernovae \cite{Riess1998,Perlmutter1999} and confirmed by cosmic microwave background (CMB) and large-scale structure (LSS) observations \cite{Planck2018,BOSS2017,DES2022}, fits the bulk of cosmological data to sub-percent precision. Yet persistent tensions challenge its completeness: the $H_0$ discrepancy between local Cepheid measurements ($H_0\approx 73$ km/s/Mpc \cite{Riess2022}) and CMB inference ($H_0\approx 67$ km/s/Mpc \cite{Planck2018}) has reached $\sim 5\sigma$ significance; the $S_8\equiv\sigma_8(\Omega_m/0.3)^{0.5}$ parameter shows $\sim 2$--$3\sigma$ tension between weak-lensing surveys ($S_8\approx 0.76$--$0.77$ \cite{KiDS2020,DES2022}) and CMB constraints ($S_8\approx 0.83$ \cite{Planck2018}); and mild anomalies appear in redshift-space distortion amplitudes \cite{BOSS2017} and the integrated Sachs-Wolfe effect \cite{Planck2016ISW}. While these tensions may ultimately trace to systematics or statistical fluctuations, their persistence across independent probes and increasing precision motivates exploration of extensions to $\Lambda$CDM. Most proposed modifications introduce free functions fitted to data and alter both background expansion $H(z)$ and structure growth, creating observational degeneracies. Can cosmological tensions be resolved through a modification that leaves $H(z)$ untouched while producing falsifiable, scale-dependent signatures in structure formation alone?

\emph{Source-side modifications} offer a distinctive approach: they target only the coupling between matter inhomogeneities and the gravitational potential in Poisson's equation, preserving the homogeneous FRW geometry while producing scale-dependent departures in growth, lensing, and redshift-space distortions. Unlike field-theoretic modifications that introduce new dynamical degrees of freedom and alter both $H(z)$ and structure growth, source-side approaches decouple background expansion from perturbation observables. This conceptual distinction motivates our framework.

We present Information-Limited Gravity (ILG) as a candidate framework addressing this question. ILG is a fixed-parameter, source-side modification of the cosmological Poisson equation intended to produce scale-dependent perturbation-level signatures while preserving background distances through zero Buchert backreaction. Unlike standard extensions to $\Lambda$CDM, ILG modifies only the matter-potential coupling in Poisson's equation, leaving Einstein's equations, the metric structure, and background expansion unchanged. The modification is encoded in a single fixed kernel
\begin{equation}
\label{eq:kernel_intro}
w(k,a) = 1 + C\,X^{-\alpha}, \qquad X \equiv \frac{k\tau_0}{a},
\end{equation}
with constants completely fixed: $C = \varphi^{-3/2} \approx 0.486$, $\alpha = \frac{1}{2}(1-\varphi^{-1}) \approx 0.191$, and $\tau_0 \approx H_0^{-1}$, where $\varphi = (1+\sqrt{5})/2$ is the golden ratio. These values are motivated by Recognition Science, an information-theoretic framework under active development (Sec.~\ref{subsec:parameters}, \cite{RSFoundations,RSClassicalBridge}); critically, the RS-motivated $\alpha \approx 0.191$ independently satisfies the mathematical well-posedness constraint $\alpha \in (0,1/2)$ required for infrared regularity. The ``zero-free-parameter'' designation reflects that these three constants are phenomenologically fixed—not fitted to data—making ILG maximally rigid and sharply falsifiable. In the linear/quasi-static window, the kernel produces mild (order-10\%) modifications at commonly probed linear scales, while remaining bounded and automatically recovering GR as $X\to\infty$. The complete model specification, including motivating context and parameter origin, is presented in Sec.~\ref{sec:model}.

\paragraph{Key structural predictions: X-universality, zero backreaction, and GR recovery.}

ILG's fixed kernel structure leads to several sharp, testable predictions. First, the dependence on the single variable $X=k\tau_0/a$ induces \emph{X-universality}: observables ($D$, $f$, $R_L$) collapse to functions of $X$ in the EdS limit, and approximately in $\Lambda$CDM. This enables data-collapse tests across matched $(k,z)$ bins where observations from different scales and redshifts should lie on a single curve when plotted against $X$. A reciprocity identity $\partial_{\ln a}\ln Q \approx -\partial_{\ln k}\ln Q$ links time and scale derivatives, providing a unique ILG signature: if an observable increases with time (decreasing $a$), it must decrease with scale (increasing $k$) at the same rate—a signature absent in GR or standard modified gravity. Second, \emph{zero Buchert backreaction} at linear order (proven in Appendix~\ref{app:buchert}) ensures that ILG cannot alter mean luminosity distances or the Hubble diagram—BAO and $H_0$ remain standard, decoupling background from perturbation sectors. Observational signatures arise exclusively through inhomogeneity-level effects: lensing, growth, and redshift-space distortions. Third, \emph{GR is recovered through two mechanisms}: kinematic screening (derived)—for $X\to\infty$ (high $k$ or early times), $w\to 1$ identically, ensuring CMB, BBN, and high-$k$ observables reduce to GR; and high-potential saturation (phenomenological)—in strong-field environments ($|\Phi|\gtrsim 10^{-4}$), we impose $w\equiv 1$ by construction to ensure compatibility with local gravity tests (see Sec.~\ref{sec:scope_limitations}). The domain of validity is transparent: ILG applies only to the quasi-static, sub-horizon, linear cosmological regime ($0.01\lesssim k\lesssim 0.2\,h\,{\rm Mpc}^{-1}$, $0\lesssim z\lesssim 2$; justification in Sec.~\ref{subsec:regime_justification}).

ILG differs fundamentally from the modified-gravity landscape (detailed in Sec.~\ref{subsec:motivation}) in four ways: it introduces no new dynamical fields (no scalars, vectors, or extra dimensions); it does not alter Einstein's equations or the metric structure; it modifies exclusively the source coupling, leaving background FRW geometry unchanged; and it contains no free functions—unlike $f(R)$ theories \cite{Sotiriou2010,DeFelice2010} (action modification introducing scalaron, altering both $H(z)$ and perturbations), Horndeski/Galileon models \cite{Horndeski1974,Deffayet2011,Kobayashi2019} (five free functions $G_i(\phi,X)$ chosen phenomenologically), DGP braneworld \cite{Dvali2000,Deffayet2002} (fifth dimension modifying $H(z)$ and growth via Vainshtein), massive gravity \cite{deRham2011,deRham2014} (Yukawa propagator modification), or non-local models \cite{Deser2007,Maggiore2014,Belgacem2018} (fitted functions $F(\Box^{-1}R)$). ILG's kernel is completely determined by three fixed constants derived independently, not fitted to data. Section~\ref{sec:comparison} develops these distinctions quantitatively and identifies specific observational discriminants (ISW amplitude suppression, $X$-collapse, tracer-independence of $E_G$, reciprocity slopes).

\paragraph{Observational signatures and falsifiability.}

Expected observational signatures (to be quantified in Paper~II \cite{PaperII}): scale-dependent, typically percent-level modifications to the growth rate $f(k,z)$ in the ILG effective regime $0.01\lesssim k\lesssim 0.2\,h\,{\rm Mpc}^{-1}$, $0\lesssim z\lesssim 2$ (Sec.~\ref{subsec:growth}); suppressed ISW amplitude $C_\ell^{Tg}$ relative to $\Lambda$CDM for low multipoles with $0<S_{\rm ISW}<1$ (Sec.~\ref{subsec:isw}); tracer-independent $E_G(k,z)=\Omega_{m0}w(X)/f(X)$ statistic providing $X$-collapse tests in matched $(k,z)$ bins (Sec.~\ref{subsec:rsd_eg}); and reciprocity slope tests $\partial_{\ln a}\ln Q \approx -\partial_{\ln k}\ln Q$ across multiple probes. Table~\ref{tab:key_falsifiers} centralizes falsifiers: ISW \emph{enhancement} (vs.\ suppression), scale-\emph{independent} $f(k,z)$, tracer-\emph{dependent} $E_G$, or failed $X$-collapse would rule out ILG. Paper~I focuses on the fixed-kernel structure, mathematical consistency, and qualitative falsification logic; quantitative $\Lambda$CDM amplitudes and forecast-level detection significances are deferred to Paper~II.

\paragraph{Scope of Paper I and Paper II.}

This paper (Paper~I) establishes the theoretical foundations, mathematical rigor, and qualitative observational signatures of ILG. We provide: model definition and parameter specification (Sec.~\ref{sec:model}); derivation of X-universality and reciprocity (Sec.~\ref{sec:kernel_properties}); closed-form EdS solution for structural insight (Sec.~\ref{sec:framework}); mathematical proofs of well-posedness, numerical convergence, and zero Buchert backreaction (Sec.~\ref{sec:mathfound}, Appendices~\ref{app:continuum} and \ref{app:buchert}); qualitative observable signatures and falsification criteria (Sec.~\ref{sec:predictions}, Table~\ref{tab:key_falsifiers}); preliminary confrontation with current data (Sec.~\ref{sec:existing_data}); and comparison with the modified-gravity landscape (Sec.~\ref{sec:comparison}). Paper~II (companion paper, in preparation \cite{PaperII}) will provide quantitative $\Lambda$CDM predictions (numerical integration and simulation-based estimates) and forecast-level comparisons. \emph{Critical caveat:} the closed-form EdS solution derived in Sec.~\ref{sec:framework} is exact analytically but does \emph{not} apply to realistic $\Lambda$CDM at $z<1$ where dark energy dominates—it provides pedagogical insight into the kernel's effect, while quantitative $\Lambda$CDM predictions require Paper~II's full numerical integration.

\paragraph{Organization.}

The paper unfolds from definition to proof to prediction to data. Section~\ref{sec:model} defines the model and fixes parameters; Sec.~\ref{sec:kernel_properties} derives X-universality and reciprocity; Sec.~\ref{sec:framework} solves growth exactly in EdS; Sec.~\ref{sec:mathfound} establishes mathematical rigor (well-posedness, zero Buchert backreaction); Sec.~\ref{sec:predictions} maps ILG to key observables and constructs falsification criteria \cite{Popper1959}; Secs.~\ref{sec:existing_data} and \ref{sec:comparison} confront current data and compare with the MG landscape; Sec.~\ref{sec:scope_limitations} discusses the effective regime and acknowledges the covariant completion gap; Sec.~\ref{sec:conclusion} summarizes implications. Appendices provide full mathematical proofs for readers requiring technical detail.

We begin with the complete model specification.








\section{Model}
\label{sec:model}

This section presents the complete ILG specification, building from the motivating gap in the modified-gravity landscape to the operational model definition. We provide: (i) context and motivation for the source-side approach; (ii) defining equations with full kernel specification; (iii) parameter values, their origin from Recognition Science, and the critical consistency check with mathematical well-posedness requirements; and (iv) an operational summary table centralizing the model inputs used in the remainder of the paper. The effective-theory assumptions and regime boundaries are collected in Sec.~\ref{sec:scope_limitations}. The structure progresses from conceptual motivation to concrete implementation, establishing ILG as a maximally rigid, falsifiable framework with zero free functions.

\subsection{Motivation: gap in the modified-gravity landscape}
\label{subsec:motivation}

Modifications to general relativity on cosmological scales have been extensively studied over the past two decades. Scalar-tensor theories ($f(R)$, Horndeski/Galileon models) \cite{Sotiriou2010,DeFelice2010,Horndeski1974,Deffayet2011,Kobayashi2019,Gleyzes2015} introduce additional degrees of freedom but require multiple free functions. Braneworld scenarios (DGP) \cite{Dvali2000,Deffayet2002}, massive gravity \cite{deRham2011,deRham2014}, and non-local models \cite{Deser2007,Maggiore2014,Belgacem2018} modify graviton propagation or Einstein's equations, typically altering both $H(z)$ and structure growth. Effective field theory approaches \cite{Bellini2014,Gleyzes2015,Noller2020} parametrize deviations through $\mu(k,z)$ and $\Sigma(k,z)$ for model-independent testing; EFT of large-scale structures \cite{Baumann2012,Carrasco2012} describes non-linear corrections to matter clustering. Emergent gravity proposals \cite{Verlinde2011,Verlinde2017,Jacobson1995} posit gravitational phenomena as thermodynamic or entropic in origin. Gravitational-wave constraints ($|c_{\rm GW}/c - 1|<10^{-15}$ from GW170817 \cite{LIGOVirgo2017}) have restricted classes of scalar-tensor theories. Screening mechanisms (chameleon, symmetron, Vainshtein) \cite{Khoury2004,Hinterbichler2010,Vainshtein1972,Babichev2013,Will2014,Burrage2018} are essential to pass Solar System tests.

Despite this extensive landscape, two common features emerge: (i) most frameworks introduce multiple free functions that must be chosen phenomenologically or fitted to data, reducing falsifiability; (ii) most alter $H(z)$ directly (through $w(z)\neq -1$ dark energy or modified Friedmann equations), propagating modifications to both background distances and structure growth. ILG fills this gap by providing a systematic, parameter-free source-side framework with fixed numerical values (not fitted to data). This motivates ILG: a source-side modification with three fixed constants, zero free functions, and decoupled background-perturbation sectors.

\subsection{Defining equations}
\label{subsec:defining_equations}

ILG is formulated entirely within the Newtonian-gauge, quasi-static, linear regime on a flat FRW background. In the sub-horizon limit, the standard Poisson equation linking the gravitational potential to matter density,
\begin{equation}
\label{eq:poisson_standard}
k^2\Phi(\mathbf{k},a) = 4\pi G a^2 \rho_m(a) \delta(\mathbf{k},a),
\end{equation}
is modified by rescaling the matter source with a fixed kernel $w(k,a)$. The ILG framework introduces the modified Poisson equation
\begin{equation}
\label{eq:poisson_ILG}
    k^2\Phi(\mathbf{k},a)=4\pi G\,a^2\,\rho_s(a)\,w(k,a)\,\delta_s(\mathbf{k},a),
\end{equation}
where $\Phi(\mathbf{k},a)$ is the Newtonian gravitational potential in Fourier space, $\rho_s(a)$ is the sourcing density, $\delta_s(\mathbf{k},a)$ is its density contrast, and $w(k,a)$ is the source-side multiplier that encodes the entire modification. The kernel $w$ is given by
\begin{equation}
\label{eq:kernel_form}
    w(k,a)=1+C\,X^{-\alpha},
\end{equation}
where the dimensionless scaling variable is
\begin{equation}
\label{eq:X_def}
    X\equiv \frac{k\tau_0}{a},
\end{equation}
with $\tau_0$ a fixed cosmological timescale. The kernel depends only on the dimensionless variable $X\equiv k\tau_0/a$, collapsing scale and time dependence into a single organizing variable.

\textbf{Sourcing choice.} We adopt total matter sourcing,
\begin{equation}
\label{eq:sourcing}
    \rho_s=\rho_m,\qquad \Omega_{s0}=\Omega_{m0}.
\end{equation}
This is the most conservative physical assumption: all matter (baryons + dark matter) contributes information and sources the potential. The specific functional form \eqref{eq:kernel_form} and parameter values are phenomenologically motivated. No free functions or fitting procedures are employed; all analysis proceeds independently with the fixed kernel as input.

\subsection{Parameter specification and origin}
\label{subsec:parameters}

All parameters in the kernel are \emph{fixed} and not fitted to data. We adopt
\begin{equation}
\label{eq:fixed_parameters}
    C=\varphi^{-3/2},\qquad
    \varphi=\frac{1+\sqrt{5}}{2},\qquad
    \alpha=\tfrac12(1-\varphi^{-1}),\qquad
    \tau_0 \approx H_0^{-1},
\end{equation}
yielding numerically
\begin{equation}
\label{eq:numerical_values}
    C\approx 0.486,\qquad \alpha\approx 0.191,\qquad \tau_0 \approx 4.55\times 10^{17}\,{\rm s}\approx 3000\,h^{-1}{\rm Mpc}.
\end{equation}
These values are held fixed throughout all calculations—no tuning or fitting to data is performed.

\textbf{Parameter origin and mathematical consistency.} The kernel form and parameter values are motivated by an information-theoretic framework under active development \cite{RSFoundations,RSClassicalBridge}; in this paper we treat $(C,\alpha,\tau_0)$ as phenomenological inputs specified independently of cosmological data and focus on the consequences of the fixed kernel. Critically, the constraint $\alpha\in(0,1/2)$ is \emph{independently required} for mathematical well-posedness: for $\alpha\ge 1/2$, the modified Poisson operator has infrared divergence (gradient symbol $w(k,a)/k \sim k^{-(1+\alpha)}$ not $L^2$-integrable in three dimensions; see Sec.~\ref{sec:mathfound}, Appendix~\ref{app:continuum}). The adopted value $\alpha\approx 0.191$ satisfies this regularity requirement automatically. No fitting to cosmological data is performed; parameters are fixed \emph{a priori}.

\textbf{Physical meaning and observational scales:}
\begin{itemize}
    \item \textbf{$\tau_0$:} Information horizon scale. Sets the mapping $X=k\tau_0/a$. Modes with $X\sim 1$ (horizon scale) experience full modification; $X\gg 1$ (sub-horizon) recover GR.
    \item \textbf{$\alpha$:} Infrared exponent. Controls the scale-dependence slope. The constraint $\alpha\in(0,1/2)$ ensures $L^2$ regularity and mathematical well-posedness.
    \item \textbf{$C$:} Amplitude. Fixes the large-scale enhancement; at $X\sim 1$, $w=1+C$.
\end{itemize}
These three parameters are minimal and non-redundant: $C$ sets the amplitude, $\alpha$ sets the infrared slope (constrained by well-posedness to $\alpha\in(0,1/2)$), and $\tau_0$ sets the pivot scale. Reducing to two parameters would eliminate either the amplitude, the slope, or the scale—none of which can be absorbed into a redefinition of the remaining parameters. Adding a fourth parameter would introduce degeneracy with no additional predictive power, violating parsimony. The kernel form $w = 1 + C X^{-\alpha}$ is the minimal power-law ansatz consistent with GR recovery ($w \to 1$ as $X \to \infty$) and infrared enhancement ($w > 1$ for finite $X$).

At fixed $(C,\alpha,\tau_0)$, the kernel interpolates smoothly between $w\to 1$ at $X\to\infty$ (automatic GR recovery at high $k$ and/or early times) and $w\to 1+C$ near $X\sim \mathcal{O}(1)$ (largest enhancement on horizon-scale modes). In the linear/quasi-static observational window ($X\gg 1$), the modification is mild and scale-dependent; quantitative $\Lambda$CDM amplitudes are provided in Paper~II.

\subsection{Summary: ILG at a glance}
\label{subsec:model_summary}

To be clear: ILG is a phenomenological, source-side modification of the cosmological Poisson equation characterized by three fixed parameters $(C,\alpha,\tau_0)$. The modification leaves fluid equations and background cosmology unchanged at linear order, introduces no new dynamical fields, and automatically recovers GR at early times and small scales. Table~\ref{tab:ilg_glance} centralizes the complete model specification: all inputs required to compute observables.

\begin{table}[ht]
\caption{ILG model specification: all inputs required to compute observables.}
\label{tab:ilg_glance}
\centering
\small
\begin{tabular}{|l|l|}
\hline
\textbf{Component} & \textbf{Specification} \\
\hline
\textbf{Modified Poisson eq.} & $k^2\Phi = 4\pi G a^2 \rho_m w(k,a) \delta$ \\
\textbf{Kernel form} & $w(k,a)=1+C\,X^{-\alpha}$, $X=k\tau_0/a$ \\
\textbf{Fixed constants} & $C=\varphi^{-3/2}\approx 0.486$, $\alpha=\frac{1}{2}(1-\varphi^{-1})\approx 0.191$, $\tau_0\approx H_0^{-1}$ \\
\textbf{Sourcing} & Total matter: $\rho_s=\rho_m$ \\
\textbf{Fluid equations} & Standard GR (unmodified) \\
\textbf{Background} & Standard FRW (unmodified at linear order) \\
\textbf{Screening} & $S(\Phi)\approx 1$ in linear regime; $w\equiv 1$ in strong-field environments \\
\textbf{Validity regime} & $0.01\lesssim k\lesssim 0.2\,h\,{\rm Mpc}^{-1}$, $0\lesssim z\lesssim 2$ (quasi-static, linear) \\
\textbf{Working assumptions} & A1--A6 (Sec.~\ref{subsec:assumptions}) \\
\hline
\end{tabular}
\end{table}

Everything else—growth, lensing, ISW, RSD—follows from these definitions via standard perturbation theory. ILG differs fundamentally from the modified-gravity theories surveyed in Sec.~\ref{subsec:motivation} in four ways: it introduces no new dynamical fields (no scalars, vectors, or extra dimensions); it does not alter Einstein's equations or the metric structure; it modifies exclusively the source coupling, leaving background FRW geometry unchanged; and it contains no free functions. Model scope, limitations, and testability are discussed in Sec.~\ref{sec:scope_limitations}.

\section{Immediate Mathematical Properties of the Kernel}
\label{sec:kernel_properties}

Given the fixed kernel form $w=1+C\,X^{-\alpha}$ with $(C,\alpha,\tau_0)$ phenomenologically fixed, several structural properties follow immediately. These properties—GR recovery at early times and small scales, large-scale enhancement controlled by $(C,\alpha)$, $X$-universality, analytic regularity, and background invariance—are not adjustable; they are consequences of the kernel structure and will be used repeatedly throughout the paper.

\subsection{Immediate consequences of the fixed kernel}
\label{subsec:consequences}

Given Assumptions~\ref{assump:newtonian}--\ref{assump:screen_ir}, the form $w=1+C\,X^{-\alpha}$ implies several structural properties 
that we will use repeatedly:
\begin{enumerate}
    \item \textbf{GR limit at small scales and early times.}  
    As $X\to\infty$ (large $k$ or small $a$), $w\to 1$, so early-universe physics and high-$k$ observables reduce to GR.
    
    \item \textbf{Mild large-scale enhancement.}  
    For finite $X$, $w>1$ increases the gravitational response by an amount controlled solely by $C$ and $\alpha$.
    
    \item \textbf{$X$-scaling, (kernel) $X$-universality, and reciprocity diagnostics.}  
    The kernel is \emph{exactly} a function of the single variable $X=k\tau_0/a$. In the EdS limit this induces exact $X$-universality for derived linear quantities such as $D/a$, $f$, and $R_L$; in $\Lambda$CDM these quantities generally depend on $(a,k)$ because $H(a)$ and $\Omega_m(a)$ introduce additional time dependence. Nevertheless, $X$ provides a natural organizing variable, motivating approximate ``data-collapse'' and reciprocity slope tests in matched $(k,z)$ bins (Table~\ref{tab:key_falsifiers}).
    
    \item \textbf{Analytic regularity.}  
    With $\alpha\in(0,\tfrac12)$ the composite gradient symbol $k^{-1}w(k,a)$ is square-integrable in the infrared in $d=3$, ensuring stability of the modified Poisson solve and convergence of standard FFT-based discretizations (Appendix~\ref{app:continuum}).
    
    \item \textbf{Background invariance at linear order.}  
    Modifying the Poisson source while keeping the fluid equations unaltered generates no Buchert backreaction for potential flow, so the homogeneous expansion $H(a)$ remains the chosen FRW baseline (Appendix~\ref{app:buchert}).
\end{enumerate}

These ingredients are sufficient to carry the fixed-parameter kernel through linear perturbation theory, lensing, ISW, and redshift-space distortions, and to construct sharp, testable predictions without introducing tunable functions of scale or time.

\subsection{Universality in the single variable \texorpdfstring{$\boldsymbol{X}$}{X}}
\label{subsec:reciprocity}

Given Assumption~\ref{assump:qs} and the model definition (Sec.~\ref{sec:model}), the kernel depends on $(k,a)$ only through $X=k\tau_0/a$, which drives the $X$-universality structure summarized in this subsection.
One of the central structural features of ILG is that its \emph{kernel} collapses scale and time dependence into a \emph{single} dimensionless variable,
\begin{equation}
\label{eq:X_universality}
    X \equiv \frac{k\,\tau_0}{a}.
\end{equation}
This ``dimensional collapse'' is exact for $w(k,a)=w(X)$ by construction. For derived linear observables (growth, lensing response, etc.) an $X$-collapse is exact in the EdS limit and becomes an organizing approximation in $\Lambda$CDM, where $H(a)$ and $\Omega_m(a)$ introduce additional time dependence beyond the $X$-scaling.

To highlight this property, consider the set of central linear-theory quantities,
\begin{equation}
\label{eq:Q_set}
    Q\in\{w,\;D/a,\;f,\;R_L\},
    \qquad
    R_L(a,k)\equiv \frac{w^2(k,a)\,D^2(a,k)}{a^2},
\end{equation}
for which an $X$-collapse is exact in EdS and serves as a diagnostic scaling form in $\Lambda$CDM:
\begin{equation}
\label{eq:Q_universality}
    Q(a,k)=
    \begin{cases}
    Q(X), & \text{EdS (exact)},\\
    Q(X)\ \text{approximately}, & \Lambda{\rm CDM (diagnostic/approx.)}.
    \end{cases}
\end{equation}
Thus ILG enforces an exact dimensional reduction for the kernel and motivates sharp scaling/consistency diagnostics (data-collapse and reciprocity) for derived observables within the adopted quasi-static, linear closure.

This dimensional reduction leads directly to the \emph{reciprocity identity}, which is exact for 
the kernel $w(k,a)$ but only approximate for derived observables in $\Lambda$CDM:
\begin{equation}
\label{eq:reciprocity}
    \partial_{\ln a}\ln Q
    \approx -\, \partial_{\ln k}\ln Q,
    \qquad Q\in\{w,\;D/a,\;f,\;R_L\}.
\end{equation}

\paragraph{Validity and limitations.}
\begin{itemize}
\item \textbf{Exact for kernel:} $w(k,a) = w(X)$ by construction, so 
$\partial_{\ln a}\ln w = -\partial_{\ln k}\ln w$ holds exactly in any cosmology.

\item \textbf{Exact for observables in EdS:} When $H(a) \propto a^{-3/2}$ and $\Omega_m=1$, 
all time-dependence enters through $a$ alone, and reciprocity holds exactly for $D$, $f$, $R_L$.

\item \textbf{Approximate in $\Lambda$CDM:} At $z<1$ where dark energy dominates, $H(a)$ and 
$\Omega_m(a)$ introduce explicit time-dependence beyond the $X$-scaling, so reciprocity need not hold exactly for derived observables. In practice, one expects departures at low redshift whose detailed size and scale dependence require dedicated numerical evaluation; the relation remains a useful diagnostic but not an exact constraint.

\item \textbf{Observational strategy:} Reciprocity tests are most powerful at $z \gtrsim 0.8$ where 
matter contributes $\gtrsim 40\%$ of the energy budget and deviations from exact reciprocity are comparatively small. At lower redshifts, reciprocity serves as a qualitative consistency check rather 
than a sharp discriminator.
\end{itemize}

\emph{Observational Preview:} This geometric constraint allows for ``single-plot'' tests in the 
$(k,z)$ plane where data should approximately flow along constant-$X$ trajectories, up to expected $\Lambda$CDM-induced deviations.

\paragraph{Consequences.}  

\begin{enumerate}
    \item \emph{Two-dimensional consistency test.}  
    The reciprocity relation motivates reconstructing $Q(z,k)$ in matched $(k,z)$ bins and checking the predicted relation between $\ln a$ and $\ln k$ slopes; see Table~\ref{tab:key_falsifiers}.

    \item \emph{Analysis design.}  
    Because the theory relies on the full two-dimensional dependence $Q(z,k)$, any premature compression across scales or redshifts will generically erase the $X$-universality and obscure the ILG signal.  
    Hence analyses must preserve both scale-binning and redshift-binning while fitting for ILG signatures.  
    This is conceptually analogous to avoiding multipole compression in redshift-space distortion analyses: structural information lives in the full two-variable dependence.
\end{enumerate}

The emergence of $X$-universality is a central structural prediction of ILG.  
It provides fixed-parameter diagnostic relations (e.g., reciprocity slope checks and $X$-collapse tests) that can be evaluated in matched $(k,z)$ bins (Table~\ref{tab:key_falsifiers}).  
In this sense ILG is comparatively constrained relative to models with tunable functions of $(k,a)$.

With these structural properties established, we proceed to implement the modified Poisson equation in linear perturbation theory to compute the growth of cosmic structure.



\section{Linear Perturbation Theory and Growth}
\label{sec:framework}

Having established the model definition (Sec.~\ref{sec:model}) and derived the immediate mathematical properties of the kernel—GR recovery, X-universality, reciprocity, and zero Buchert backreaction (Sec.~\ref{sec:kernel_properties})—we now implement ILG in linear cosmological perturbation theory to compute observables. The growth of density perturbations $\delta(\mathbf{k},a)$ is the central observable linking ILG to data: it determines the matter power spectrum $P(k,z)$, which in turn drives all late-time observables—weak lensing, galaxy clustering, redshift-space distortions, CMB lensing, and the ISW effect. The modified Poisson equation directly alters the gravitational driving term in the growth equation, producing scale-dependent growth factor $D(k,a)$ and growth rate $f(k,a) \equiv d\ln D/d\ln a$. This section derives these quantities analytically in the Einstein-de Sitter (EdS) limit to provide structural insight; realistic $\Lambda$CDM predictions requiring numerical integration are deferred to Paper~II. The predictions derived here form the foundation for all observables computed in Sec.~\ref{sec:predictions}.

\subsection{Set--up and conventions}
\label{subsec:setup}

Standard cosmological notation: scale factor $a$, Hubble parameter $H$, conformal counterpart $\mathcal{H}=aH$. Units: $c=1$ unless stated. Time variables: cosmic time $t$, conformal time $\eta$ with $d\eta = dt/a$. Derivatives: overdots $\dot{Q}\equiv dQ/d\eta$, primes $Q'\equiv dQ/d\ln a$, related by $d/d\eta=\mathcal{H}\,d/d\ln a$. Fourier conventions: $\nabla^2\to -k^2$ with
\begin{equation}
f(\mathbf{k})=\int d^3x\, f(\mathbf{x})\,e^{-i\mathbf{k}\cdot \mathbf{x}},
\qquad
f(\mathbf{x})=\int\frac{d^3k}{(2\pi)^3}\, f(\mathbf{k})\,e^{+i\mathbf{k}\cdot \mathbf{x}}.
\end{equation}

The dimensionless variable $X \equiv k\tau_0/a$ plays a central role: it is the \emph{sole} combination of $(k,a)$ entering the modified Poisson equation, yielding $X$-universality throughout.

\subsection{Source--side modification of Poisson}
\label{subsec:poisson}

The key ILG modification enters directly at the level of the Poisson equation: rather than altering the homogeneous background or introducing new dynamical degrees of freedom, ILG rescales the matter source in Poisson's equation by the fixed multiplier $w(k,a)$. In this section we use the model definition Eq.~\eqref{eq:poisson_ILG} with kernel Eq.~\eqref{eq:kernel_form} and $X$ defined in Eq.~\eqref{eq:X_def}. We will repeatedly use the GR recovery ($X\to\infty\Rightarrow w\to 1$) and the kernel reciprocity identity derived in Sec.~\ref{subsec:reciprocity}.


\subsection{Linear growth and closed-form solution (Einstein-de Sitter)}
\label{subsec:growth}

Given Assumptions~\ref{assump:newtonian}--\ref{assump:fluid} and the model definition (Sec.~\ref{sec:model}), the ILG modification of the Poisson equation leads directly to a modification of the growth of cosmic structure.  
A crucial feature of the framework is that ILG leaves the Euler and continuity equations unaltered; only the gravitational potential is modified through the source-side multiplier $w(k,a)$.  
Consequently, the density contrast in the matter regime satisfies
\begin{equation}
\label{eq:delta_ode}
    \ddot{\delta} + \mathcal{H}\dot{\delta}
    - 4\pi G\,a^2 \rho_s(a)\,w(k,a)\,\delta = 0.
\end{equation}
The evolution is therefore conventional in its kinematic terms but acquires a scale- and time-dependent driving term inherited entirely from the modified Poisson sector.

We seek a growing-mode factorization of the form  
\begin{equation}
\label{eq:growth_factorization}
\delta(\mathbf{k},a)=D(a,k)\,\delta_{\rm ini}(\mathbf{k}),
\end{equation}
where GR normalization imposes $D(a\!\to\!0,k)\sim a$ at early times.  
Passing to derivatives with respect to $\ln a$ yields (with primes as defined in Sec.~\ref{subsec:setup})
\begin{equation}
\label{eq:delta_ln}
    D'' + \left(2+\frac{d\ln H}{d\ln a}\right)D'
    - \frac{3}{2}\Omega_s(a)\,w(k,a)\,D = 0.
\end{equation}

\paragraph{EdS context.}
We derive the exact solution in the Einstein-de Sitter (EdS) limit to isolate the kernel's mechanical effect on growth. While $\Lambda$CDM requires numerical integration (Paper II), the EdS limit captures the functional form of the scale dependence. At $z < 1$ where the phenomenology is observationally most relevant, dark energy dominates and EdS breaks down. In $\Lambda$CDM, accelerated expansion drives $f(z\to 0)\to 0$ for both GR and ILG; there is therefore no EdS-style late-time ``plateau'' to interpret as a prediction. The EdS solution below provides structural insight, while realistic amplitudes require full $\Lambda$CDM numerical integration (Paper~II): the signature in $\Lambda$CDM is a milder, scale-dependent enhancement of $f(k,z)$ relative to GR.

\textbf{GR comparison (EdS reference).} For reference, in GR (EdS), the growth equation with $w \equiv 1$ yields $D_{\rm GR,EdS}(a) = a$ and $f_{\rm GR,EdS} = 1$ (constant at all times and scales). ILG modifies this through the scale-dependent kernel $w(k,a)$, producing scale- and time-dependent growth.

During the matter-dominated epoch ($\Omega_m \approx 1$), $\Omega_s(a)$ evolves slowly.
Under this approximation—i.e.\ when the background is matter dominated with $\Omega_m(a)\approx 1$ (operationally at sufficiently high redshift)—Eq.~\eqref{eq:delta_ln} admits a closed-form growing-mode solution:
\begin{equation}
\label{eq:D_exact}
    D_{\rm EdS}(a,k)
    = a\left[1+\beta(k)\,a^\alpha\right]^{\frac{1}{1+\alpha}},
    \qquad
    \beta(k)=\frac{2}{3}\,C\,(k\tau_0)^{-\alpha}.
\end{equation}

The EdS growth rate is
\begin{equation}
\label{eq:f_exact}
    f_{\rm EdS}(a,k)=1+\frac{\alpha}{1+\alpha}\,
    \frac{\beta(k)a^\alpha}{1+\beta(k)a^\alpha}.
\end{equation}

\emph{Interpretation (EdS context):} The solution \eqref{eq:D_exact}--\eqref{eq:f_exact} cleanly separates the scale dependence through $\beta(k)\propto k^{-\alpha}$ and the time dependence through $a^\alpha$. In EdS, $f_{\rm EdS}$ interpolates between the GR value $f=1$ at early times ($\beta a^\alpha\ll 1$) and a constant enhancement at late times ($\beta a^\alpha\gg 1$). This interpolation is useful for intuition, but it is not a numerical prediction for $\Lambda$CDM.

\paragraph{Interpretation and scale dependence.}

The EdS solution reveals ILG's structural fingerprint through three distinct regimes:

\textbf{Early times (GR recovery):} At $\beta a^\alpha \ll 1$, we have $D \to a$ and $f \to 1$, recovering GR exactly. Thus sufficiently early times (small $a$) automatically satisfy CMB-era consistency.

\textbf{Transition regime (scale-dependent growth):} At $\beta a^\alpha \sim 1$, growth begins departing from GR. Because $\beta \propto k^{-\alpha}$, lower-$k$ modes (larger scales) depart earlier. This scale-dependent transition is the key observable signature: ILG predicts $f(k,z)$ varies with scale, whereas GR predicts scale-independent $f(z)$.

\textbf{Late EdS limit (pedagogical only):} \emph{If matter domination continued indefinitely (NOT realized in our Universe)}, $f \to 1 + \alpha/(1+\alpha)$ (a constant enhancement over GR). \textbf{In realistic $\Lambda$CDM, dark energy prevents this behavior}; $f \to 0$ as $z \to 0$ for both GR and ILG. The empirically relevant signature in $\Lambda$CDM is therefore not an EdS late-time limit, but a scale-dependent departure at intermediate redshifts, to be quantified by numerical integration (Paper~II).

\textbf{X-universality diagnostic:} Plotting observables versus $X = k\tau_0/a$ (rather than $(k,z)$ separately) collapses families of curves onto universal tracks in EdS. In $\Lambda$CDM, approximate collapse provides a sharp diagnostic test (Table~\ref{tab:key_falsifiers}).

\paragraph{Path forward: realistic $\Lambda$CDM predictions (Paper~II).}

The EdS solution above provides structural insight—demonstrating how the source-side kernel $w(k,a)$ modifies growth and revealing the scale-dependent signature—but it does not yield quantitative predictions for our Universe. Paper~II will compute realistic $\Lambda$CDM observables via numerical integration of Eq.~\eqref{eq:delta_ln} with time-dependent $\Omega_m(a)$ and $H(a)$, and will quantify the resulting scale-dependent departures in $f(k,z)$ and derived observables. In Paper~I we restrict to qualitative, directional statements and falsification criteria that do not rely on unverified forecast-level numbers.

\section{Consistency Checks: Numerical Convergence and Backreaction}
\label{sec:mathfound}

Section~\ref{sec:framework} derived an analytic EdS growth solution that makes ILG's structural fingerprint transparent (scale-dependent departures driven by the fixed kernel). However, realistic observable predictions require numerical integration of Eq.~\eqref{eq:delta_ln} in $\Lambda$CDM with time-dependent $\Omega_m(a)$ and $H(a)$ (Paper~II). Before proceeding to observables (Section~\ref{sec:predictions}), we must establish that ILG is mathematically sound: the modified Poisson problem admits unique solutions, numerical discretizations converge reliably, and the framework is internally self-consistent.

\paragraph{What we prove and why it matters.}

This section establishes two complementary foundational results:

\textbf{(1) Well-posedness and numerical convergence (Sec.~\ref{subsec:convergence}--\ref{subsec:wellposedness_summary}).} We prove that the ILG-modified Poisson equation admits unique weak solutions, these solutions depend continuously on initial data, and standard FFT-based solvers converge to the correct continuum limit. \emph{Why this matters:} Without convergence guarantees, numerical predictions (Paper~II's $\Lambda$CDM integration, N-body simulations) would be unreliable. The key regularity condition is $\alpha \in (0,1/2)$, ensuring infrared integrability of the modified gradient operator. \textbf{Critical check:} The phenomenologically adopted value $\alpha \approx 0.191$ (derived from Recognition Science, Section~\ref{subsec:parameters}) lies comfortably within this window—the growth solution derived in Section~\ref{sec:framework} is mathematically valid without additional fine-tuning. This result also validates the X-universality structure (Section~\ref{sec:kernel_properties}): the Poisson operator is well-posed for the entire $X$-dependent kernel $w(k,a) = 1 + CX^{-\alpha}$.

\textbf{(2) Zero Buchert backreaction (Sec.~\ref{subsec:buchert}).} We prove that ILG produces no backreaction from inhomogeneities to the mean expansion rate at linear order: $Q_D = 0$ identically. \emph{Why this matters:} This ensures $H(z)$ remains an externally chosen FRW input—ILG \emph{cannot} alter BAO, the Hubble diagram, or mean luminosity distances. Observational signatures arise \emph{exclusively} through inhomogeneity-level effects (lensing, growth, clustering), decoupling background from perturbation sectors as promised in the Abstract and Section~\ref{sec:introduction}. This holds despite the scale dependence of $f(k,a)$ implied by the kernel: the kinematic balance between expansion variance and shear compensates at linear order across Fourier modes.

Together, these results guarantee that ILG predictions are computationally stable, mathematically rigorous, and conceptually self-consistent. Well-posedness enables reliable numerical integration; zero backreaction ensures self-consistency with the source-side philosophy.

\subsection{Convergence of the numerical scheme}
\label{subsec:convergence}

The EdS solution (Section~\ref{sec:framework}) provides structural insight, but realistic $\Lambda$CDM predictions require numerical integration of the growth equation~\eqref{eq:delta_ln} and Poisson equation~\eqref{eq:poisson_ILG} on a discrete mesh. This subsection proves that standard spectral discretizations converge to the correct continuum limit, guaranteeing that N-body simulations and FFT-based solvers produce reliable results.

The full discrete-to-continuum convergence argument (spectral Poisson solve, uniform stability bound under mesh refinement, weak convergence, and identification of the limiting multiplier equation) is standard but technical. To keep the main text readable, we collect the discrete-to-continuum proof together with the complementary continuum theorems in Appendix~\ref{app:continuum}.

\subsection{Well-posedness summary}
\label{subsec:wellposedness_summary}

The convergence result above rests on deeper continuum well-posedness. We summarize the key theorems here; full statements and proofs appear in Appendix~\ref{app:continuum}.

\paragraph{Continuum theory.}
For $\alpha\in(0,1/2)$ and mean-zero $\delta\in L^2(\mathbb{T}_L^3)$, the ILG-modified Poisson problem admits a unique weak solution $\Phi\in H^1(\mathbb{T}_L^3)$ satisfying uniform gradient bounds
\begin{equation}
\|\nabla\Phi\|_{L^2} \le C(L,a,\alpha,\tau_0)\,\|\delta\|_{L^2}.
\end{equation}
Solutions depend continuously on data (Theorems~\ref{thm:torus} and \ref{thm:R3} in Appendix~\ref{app:continuum}). The proof proceeds via Lax-Milgram: the bilinear form is bounded and coercive because the composite symbol $|\mathbf{k}|^{-1}w(|\mathbf{k}|,a)$ is square-integrable in the infrared when $d=3$ and $\alpha<1/2$.

\paragraph{Discrete convergence.}
Standard spectral discretizations converge weakly in $H^1_{\rm loc}$ as mesh spacing $\varepsilon\to 0$ (Theorems~\ref{thm:torus}--\ref{thm:R3} in Appendix~\ref{app:continuum}). The discrete stability constant is independent of $\varepsilon$, ensuring robust N-body implementations.

\textbf{Practical implementation:} These convergence guarantees justify using standard FFT-based Poisson solvers in particle-mesh style codes after replacing the GR kernel multiplier by the ILG multiplier $w(k,a)$. Detailed implementation choices (mesh sizes, refinement criteria, and quantitative convergence tests) are deferred to Paper~II.

\paragraph{Extension to $\mathbb{R}^3$.}
The torus analysis extends to whole-space under mild IR moment bounds (Theorem~\ref{thm:R3}). Primordial density fields with nearly scale-invariant spectra $P_{\rm ini}(k)\sim k^{n_s}$, $n_s\approx 0.96$ automatically satisfy these conditions, and the physical horizon provides a natural IR cutoff.

\paragraph{Role of $\alpha<1/2$ and validation of Section~\ref{sec:framework}'s growth solution.}
This regularity window is essential: for $\alpha\ge 1/2$, the gradient symbol $w(k,a)/k \sim k^{-(1+\alpha)}$ is not $L^2$-integrable in 3D, leading to IR divergence and ill-posedness. The phenomenologically adopted value $\alpha\approx 0.191$ (derived from Recognition Science, Section~\ref{subsec:parameters}) lies comfortably within the viable range, with safety margin $0.191 < 0.5$. 

\textbf{Consistency check with Section~\ref{sec:framework}:} The EdS growth solution $D_{\rm EdS}(a,k) = a[1+\beta(k)a^\alpha]^{1/(1+\alpha)}$ and growth rate $f_{\rm EdS}(a,k) = 1 + \frac{\alpha}{1+\alpha}\frac{\beta a^\alpha}{1+\beta a^\alpha}$ derived in Eqs.~\eqref{eq:D_exact}--\eqref{eq:f_exact} are \emph{mathematically valid} because $\alpha < 1/2$ guarantees well-posedness of the underlying Poisson problem. This is a non-trivial consistency check: the phenomenological parameter choice (motivated by Recognition Science) produces a mathematically viable theory without requiring additional fine-tuning.

\textbf{Quantitative safety margin:} The regularity window $\alpha \in (0, 1/2)$ provides a safety factor of $0.5/0.191 \approx 2.6$. Recognition Science could have predicted $\alpha$ up to $\sim 0.48$ and ILG would remain mathematically viable; conversely, $\alpha \ge 0.5$ would lead to infrared divergence. The relevant infrared condition is the $L^2$ integrability of the gradient symbol $w(k,a)/k$: since $w\sim k^{-\alpha}$ as $k\to 0$, one has
\begin{equation}
\int_{|\mathbf{k}|<k_{\min}} d^3k\,\Big|\frac{w(k,a)}{k}\Big|^2
\;\sim\;\int_0^{k_{\min}} dk\;k^{2}\,k^{-2(1+\alpha)}
\;=\;\int_0^{k_{\min}} dk\;k^{-2\alpha}.
\end{equation}
Thus the integral converges iff $\alpha<1/2$; it diverges \emph{logarithmically} at $\alpha=1/2$ and as a \emph{power law} for $\alpha>1/2$. For the adopted $\alpha\simeq 0.191$, the scaling is $\int_0^{k_{\min}} dk\,k^{-0.382}\sim k_{\min}^{0.618}$, which is finite and well-behaved as $k_{\min}\to 0$. Had Recognition Science predicted $\alpha \ge 1/2$, ILG would be mathematically ill-defined in this continuum sense.

\subsection{Self-consistency: No Buchert backreaction}
\label{subsec:buchert}

The Abstract and Section~\ref{sec:introduction} promised that ILG ``preserves background distances through zero Buchert backreaction,'' decoupling background ($H(z)$, BAO, Hubble diagram) from perturbation sectors (growth, lensing, clustering). We now prove this explicitly.

Given Assumptions~\ref{assump:newtonian}--\ref{assump:fluid} and the model definition (Sec.~\ref{sec:model}), ILG modifies only the Poisson source term $w(k,a)\delta$ while leaving the background dynamics and fluid equations unchanged. At linear order, even though the growth rate $f(k,a)$ is scale-dependent, the Buchert backreaction cancels identically and the homogeneous expansion remains the chosen FRW baseline:
\begin{equation}
    Q_D = 0 \qquad \text{(linear order)}.
\end{equation}
For the full derivation (and the explicit cancellation with scale-dependent $f(k,a)$), see Appendix~\ref{app:buchert}.

\paragraph{Physical consequences: Decoupling background from perturbations.}

ILG produces no Buchert backreaction at linear order: the averaging of inhomogeneities never feeds back into the background expansion ($Q_D = 0$ identically). This establishes three critical consequences:

\textbf{(i) $H(z)$ unchanged:} The cosmological scale factor $H(a)$ remains an externally chosen FRW input, identical to $\Lambda$CDM. ILG \emph{cannot} alter the cosmic expansion history.

\textbf{(ii) BAO and Hubble diagram unchanged:} Because mean distances $\propto \int H^{-1}(z)dz$ are unaffected, ILG makes \emph{no predictions} for standard ruler/candle observations. Baryon acoustic oscillation scales remain standard; Type Ia supernovae luminosity distances remain standard; the $H_0$ tension is \emph{not} addressed by ILG (though the $S_8$ tension is).

\textbf{(iii) Observational signatures are inhomogeneity-level only:} ILG's effects appear \emph{exclusively} through gravitational potentials $\Phi(\mathbf{k},a)$, which drive lensing, growth rate $f(k,z)$, redshift-space distortions, and the ISW effect. Section~\ref{sec:predictions} computes these signatures; comparison with background observables (BAO, $H_0$) provides orthogonal consistency checks.

This decoupling is by design (source-side modification, Section~\ref{sec:introduction}) and is a key falsifier: if future data show ILG-like growth/lensing enhancements \emph{with} altered BAO or $H(z)$, the source-side framework is ruled out. For the full derivation, see Appendix~\ref{app:buchert}.

\paragraph{Summary and path forward.}

We have established that ILG is mathematically sound (\textsection\ref{subsec:convergence}--\ref{subsec:wellposedness_summary}: well-posedness, numerical convergence) and internally self-consistent (\textsection\ref{subsec:buchert}: zero Buchert backreaction, no alteration of $H(z)$). The analytic EdS growth solution derived in Section~\ref{sec:framework} is mathematically valid ($\alpha < 1/2$ ensures IR regularity) and produces no contradictions with the FRW background. We now proceed to map ILG's modified growth to observable signatures: weak lensing, the integrated Sachs-Wolfe effect, and redshift-space distortions. These predictions (Section~\ref{sec:predictions}) will be qualitative and directional; quantitative $\Lambda$CDM amplitudes require Paper~II's numerical integration.


\section{Predictions for Key Observables}
\label{sec:predictions}

Section~\ref{sec:mathfound} established that ILG is mathematically sound (well-posedness, numerical convergence guarantee for FFT solvers) and internally self-consistent (zero Buchert backreaction ensuring $H(z)$ remains unaltered). With this rigorous foundation, we now propagate ILG's modified Poisson equation~\eqref{eq:poisson_ILG} and the scale-dependent growth factor $D(k,a)$ to key cosmological observables: weak gravitational lensing, the integrated Sachs-Wolfe effect, and redshift-space distortions.

The predictions below describe the directional and structural signatures of ILG. Where possible, we provide order-of-magnitude estimates using the EdS solution (Section~\ref{sec:framework}) as pedagogical illustration. Quantitative amplitudes in realistic $\Lambda$CDM cosmology require full numerical integration with $H(a)$ and $\Omega_m(a)$, which are presented in the companion paper (Paper~II \cite{PaperII}).
Consistency checks (numerical convergence/well-posedness and Buchert backreaction) are collected in Appendices~\ref{app:continuum} and \ref{app:buchert}.

\subsection{Overview and key falsifiers}
\label{subsec:key_falsifiers}

Having established ILG's mathematical soundness (Section~\ref{sec:mathfound}) and the scale-dependent growth structure implied by the kernel (Section~\ref{sec:framework}), we now map these theoretical results to observable signatures. This section derives three complementary tests: weak lensing (cumulative potential enhancement), ISW (time evolution of potentials), and redshift-space distortions (direct growth rate measurement).

To keep the observational tests crisp and centralized, Table~\ref{tab:key_falsifiers}
lists a small set of directional predictions that can rule out ILG within the effective
domain specified in Sec.~\ref{sec:scope_limitations}.

\begin{table}[p]
\caption{Key falsifiers for ILG within the effective domain (Sec.~\ref{sec:scope_limitations}).
Each entry is a directional prediction fixed by the kernel parameters. EdS provides analytic intuition for the scale dependence; quantitative $\Lambda$CDM amplitudes require numerical integration (Paper~II).}
\label{tab:key_falsifiers}
\centering
\scriptsize
\setlength{\tabcolsep}{3pt}
\renewcommand{\arraystretch}{0.92}
\begin{tabular}{|p{0.18\linewidth}|p{0.28\linewidth}|p{0.20\linewidth}|p{0.28\linewidth}|}
\hline
\textbf{Observable} & \textbf{ILG prediction (sign/direction)} & \textbf{Best window} & \textbf{Would rule ILG out} \\
\hline
ISW--LSS cross-correlation $C_\ell^{Tg}$ &
Suppressed relative to $\Lambda$CDM; $0 < S_{\rm ISW} < 1$ (Sec.~\ref{subsec:isw}). &
Low multipoles $\ell\lesssim 30$, tracers at $z\sim 0.2$--$0.8$ &
Robust detection of enhancement relative to $\Lambda$CDM (i.e., $S_{\rm ISW}(\ell)>1$ at high significance) \\
\hline
RSD growth rate $f(k,z)$ &
Scale-dependent growth: $f(k,z)$ increases as $k$ decreases (fixed $z$), with GR recovery at high $k$ and/or high $z$ (Paper~II for $\Lambda$CDM amplitudes). &
$0.01\lesssim k \lesssim 0.2\,h\,{\rm Mpc}^{-1}$, $0\lesssim z\lesssim 2$ (signal strongest at $z\lesssim 1$) &
$f$ consistent with scale-independent GR across this $k$ range, with uncertainties small enough to exclude any ILG-like scale dependence \\
\hline
$E_G(k,z)$ &
Tracer independent; $E_G=\Omega_{m0}w/f$ with $X$-collapse (Sec.~\ref{subsec:rsd_eg}). &
Overlap of lensing+RSD: $0.02\lesssim k\lesssim 0.15\,h\,{\rm Mpc}^{-1}$, $0.2\lesssim z\lesssim 1$ &
Significant tracer dependence or systematic mismatch with fixed $w/f$ relation across matched $(k,z)$ bins \\
\hline
Cross-probe $X$-universality &
Consistency of inferred $X$-scaling across probes (exact for $w$, approximate for derived observables). &
Matched $(k,z)$ bins spanning wide $X$ range while remaining linear &
Failure to recover consistent $X$-scaling across probes using fixed parameters (beyond expected $\Lambda$CDM corrections) \\
\hline
\end{tabular}
\end{table}

\subsection{Lensing and distances: propagation of the Poisson multiplier}
\label{subsec:lensing}

Having summarized the key falsifiers in Table~\ref{tab:key_falsifiers}, we now derive each prediction quantitatively. We begin with weak gravitational lensing, which provides the most direct probe of ILG's modified gravitational potential.

Given Assumptions~\ref{assump:newtonian}--\ref{assump:source} and the model definition (Sec.~\ref{sec:model}), we propagate the modified Poisson sector into standard weak-lensing observables within the usual Limber/Born approximations.
Weak gravitational lensing is sensitive to ILG in two ways: (i) the Poisson multiplier $w(k,a)$ rescales the kernel that links density fluctuations to the Weyl potential, and (ii) the growth factor $D(a,k)$ becomes scale-dependent. In the standard Limber/Born approximations, these effects enter lensing through the combined \emph{response factor}
\begin{equation}
\label{eq:RL_def}
R_L(a,k)\equiv\frac{w^2(a,k)\,D^2(a,k)}{a^2},
\end{equation}
which quantifies the cumulative modification of lensing power relative to GR. For the full lensing-kernel derivation (including the explicit convergence power spectrum and the distance-moment relations), see Appendix~\ref{app:observable_details}.

\paragraph{Key point: Background distances unchanged.}
As proven in Section~\ref{sec:mathfound} (zero Buchert backreaction, $Q_D = 0$ identically), ILG leaves the homogeneous expansion history unaltered: $H(z)$ remains identical to the fiducial $\Lambda$CDM.  
Consequently:

\begin{itemize}
    \item background luminosity distances $\bar D_L(z)$ coincide exactly with those of the fiducial FRW cosmology;
    \item deviations arise only through lensing-induced \emph{scatter} and \emph{non-Gaussianity}, not through shifts in the mean Hubble diagram.
\end{itemize}

This sharply distinguishes ILG from dark-energy models that modify $H(z)$ directly. The decoupling of background (unchanged) from perturbations (enhanced via $w > 1$ and $D > D_{\rm GR}$) is by design (source-side modification) and verified mathematically in Section~\ref{subsec:buchert}.

Throughout this subsection we adopt the pure ILG baseline (no background rescaling): background distances are unchanged and all departures enter through $w(k,a)$ and the scale-dependent growth factor $D(a,k)$.

\subsection{ISW effect: a crisp sign prediction}
\label{subsec:isw}

Weak lensing (Section~\ref{subsec:lensing}) probes the cumulative gravitational enhancement through $R_L \propto w^2 D^2$. We now turn to the integrated Sachs-Wolfe effect, which probes the \emph{time evolution} of gravitational potentials—providing a complementary test of ILG's time-dependent kernel $w(k,a)$.

Given Assumptions~\ref{assump:newtonian}--\ref{assump:fluid} and the model definition (Sec.~\ref{sec:model}), we express the ISW source in terms of the modified potential evolution induced by $w(k,a)$.
The integrated Sachs--Wolfe (ISW) effect probes the time evolution of the large-scale gravitational potentials.  
In the absence of anisotropic stress, $\Psi=\Phi$, and the temperature shift along a photon geodesic reduces to
\begin{equation}
\label{eq:ISW_basic}
    \frac{\Delta T}{T}
    = \int d\eta\,(\dot\Phi + \dot\Psi)
    = 2\int d\eta\,\dot\Phi.
\end{equation}

Using the ILG-modified Poisson equation, the time derivative of the potential can be expressed as
\begin{equation}
\label{eq:isw_driver}
    \dot{\Phi} + \dot{\Psi}
    = 2aH\,\Phi\,
      \underbrace{\left[-1 + f(k,a)
      + \partial_{\ln a}\ln w(k,a)\right]}_{B(a,k)} .
\end{equation}
The factor $B(a,k)$ encapsulates the additional evolution induced by ILG.  
Its structure reflects the three drivers of potential evolution: Hubble dilution ($-1$), growth of density fluctuations ($f$), and the explicit time dependence of the source-side multiplier $w$.

From the ILG form
\begin{equation}
\label{eq:w_ILG_form}
    w(k,a)=1+\varphi^{-3/2}X^{-\alpha},
\end{equation}
one obtains the exact expression
\begin{equation}
\label{eq:dlnw}
    \frac{d\ln w}{d\ln a}
    = \alpha\,
      \frac{\varphi^{-3/2}X^{-\alpha}}
           {1+\varphi^{-3/2}X^{-\alpha}}
    > 0 .
\end{equation}
Thus $w$ grows monotonically in time on large scales.

For overdensities one has $\Phi<0$.  
In the ILG large-scale, late-time regime, both $f$ is enhanced relative to GR in the EdS illustration (Section~\ref{sec:framework}) and $d\ln w/d\ln a>0$ (Eq.~\eqref{eq:dlnw}: strengthening of the gravitational response with time).  
However, the background expansion (driven by $\Lambda$ or dark energy) drives a strong decay of $\Phi$ (the term $-1$ in $B(a,k)$).
The ILG boost \emph{counters} this decay but, for the fiducial parameters, does not fully reverse it.
Hence
\begin{equation}
\label{eq:B_ISW_sign}
    B(a,k) \lesssim 0
    \quad \Longrightarrow \quad
    \dot{\Phi} + \dot{\Psi} \gtrsim 0.
\end{equation}
The net result is a \emph{suppressed} ISW cross-correlation between the CMB and large-scale structure. The signal remains positive (or near zero) but is significantly weaker than in standard $\Lambda$CDM. Exact amplitudes require full $\Lambda$CDM numerical integration; values quoted below are indicative.

\paragraph{Significance and quantitative prediction.}  
This suppression is one of the key observational tests (Table~\ref{tab:key_falsifiers}).
In $\Lambda$CDM, the late-time decay of $\Phi$ yields a strong \emph{positive} ISW signal. ILG predicts a damped signal due to the source-driven deepening of potentials fighting the expansion-driven decay.

\textbf{Physical origin:} The ISW suppression traces directly to ILG's scale-dependent growth and to the explicit time dependence of $w(k,a)$. At low $k$ where the kernel enhancement is largest, the term $d\ln w/d\ln a>0$ (Eq.~\eqref{eq:dlnw}) strengthens the source with time and tends to slow potential decay. However, $\Lambda$ dominance at $z<1$ drives strong Hubble dilution (the $-1$ term in $B(a,k)$, Eq.~\eqref{eq:isw_driver}), which typically dominates. Net effect: potential decay is slowed but not reversed $\Rightarrow$ ISW suppression ($0 < S_{\rm ISW} < 1$), not enhancement.

To quantify this prediction, we parameterize the ISW-galaxy cross-power at low multipoles ($\ell \lesssim 30$) where the ILG effect is maximal. The detailed scale dependence and amplitude require full $\Lambda$CDM numerical integration beyond the scope of this paper.
For a fiducial galaxy sample at effective redshift $z_{\rm eff} \sim 0.3$, we write
\begin{equation}
\label{eq:ISW_suppression}
C_{\ell}^{Tg} \Big|_{\rm ILG} 
\equiv S_{\rm ISW}(\ell)\, C_{\ell}^{Tg} \Big|_{\Lambda{\rm CDM}}
\quad (\ell = 10\text{--}30).
\end{equation}
Within the ILG effective regime one expects $0<S_{\rm ISW}(\ell)<1$ (suppression relative to $\Lambda$CDM).
The key observational signature is the \emph{suppression}, distinguishing it from models that enhance the decay. Quantitative estimates require full $\Lambda$CDM numerical integration (Paper~II in preparation). In particular, ILG implies a weaker positive ISW cross-correlation than $\Lambda$CDM in the effective regime, which is compatible with current ISW detections and motivates more targeted tests rather than a claim of quantitative ``resolution'' at this stage \cite{Planck2016ISW}.

\emph{Current observational status.}
Planck 2018 cross-correlations with NVSS, WISE, and 2MPZ report positive detections at combined significance $\sim 3.5\sigma$.
The data are consistent with a positive signal, though the amplitude shows some tension with $\Lambda$CDM in certain bins.
ILG's prediction of a weaker positive signal is compatible with current limits and (qualitatively) goes in the direction needed to reduce any apparent excess ISW amplitude relative to $\Lambda$CDM, if such an excess persists under future, higher-S/N analyses.

\emph{Test criterion.}
For a centralized summary of what would rule ILG out (and where to look), see Table~\ref{tab:key_falsifiers}.

\subsection{Redshift--space distortions and the \texorpdfstring{$E_G$}{EG} statistic}
\label{subsec:rsd_eg}

The ISW effect (Section~\ref{subsec:isw}) probes potential time-derivatives at large scales ($\ell \lesssim 30$, $k \lesssim 0.01\,h\,{\rm Mpc}^{-1}$), yielding suppression ($0 < S_{\rm ISW} < 1$). Complementary information comes from redshift-space distortions, which directly measure the growth rate $f(k,z)$ across a wider range of scales ($k \sim 0.01$--$0.2\,h\,{\rm Mpc}^{-1}$)—probing ILG's scale-dependent signature most directly.

Given Assumptions~\ref{assump:newtonian}--\ref{assump:fluid} and the model definition (Sec.~\ref{sec:model}), the linear RSD and $E_G$ relations follow from propagating the modified Poisson equation through standard Kaiser/RSD and lensing-response definitions.
Redshift--space distortions (RSD) offer a complementary and highly sensitive probe of the growth rate of structure.  
A central prediction of ILG is that the growth rate $f(k,z)$ acquires a scale dependence tied directly to the variable $X=k\tau_0/a$.  
Consequently, any analysis that compresses RSD measurements into a single effective $f(z)$ risks erasing the very signature that distinguishes ILG from GR.

\textbf{GR vs. ILG scale dependence (key discriminant):} In GR, $f(z) \approx \Omega_m^{0.55}(z)$ is \emph{scale-independent}: $f_{\rm GR}(k_1,z)=f_{\rm GR}(k_2,z)$ at fixed $z$ (in linear theory). In ILG, the fixed kernel induces an explicit scale dependence $f=f(k,z)$ tied to $X=k\tau_0/a$, with larger departures at lower $k$ (larger scales) and GR recovery at large $k$ and/or early times. Quantitative $\Lambda$CDM amplitudes require numerical integration and are deferred to Paper~II.

In linear theory ($|\delta|\ll 1$) and on the quasi-static, sub-horizon scales of Sec.~\ref{sec:model}, the redshift--space galaxy power spectrum obeys the Kaiser relation,
\begin{equation}
\label{eq:kaiser}
    P_s(k,\mu,z)
    = \big[b(k,z) + f(k,z)\mu^2\big]^2 P_m(k,z),
    \qquad
    P_m(k,z) = D^2(k,z)\,P_{\rm ini}(k),
\end{equation}
where $b(k,z)$ is the galaxy bias and $\mu$ is the cosine of the angle with the line of sight.  
The corresponding multipoles are
\begin{align}
\label{eq:multipoles}
    P_0 &= \big[b^2 + \tfrac23 b f + \tfrac15 f^2\big] P_m,\\
    P_2 &= \big[\tfrac43 b f + \tfrac47 f^2\big] P_m,\\
    P_4 &= \tfrac{8}{35} f^2 P_m.
\end{align}

\paragraph{Analysis strategy.}
To retain the full $X$--dependence predicted by ILG, one must avoid any scale-averaging.  
A straightforward approach is to invert the ratio $P_2/P_0$ \emph{in each $k$--bin} to obtain
\begin{equation}
\label{eq:r_bias_ratio}
    r(k,z) = \frac{f(k,z)}{b(k,z)}.
\end{equation}
Using an independent bias constraint (from lensing, multi-tracer methods, or cross-correlations) then yields a direct reconstruction of $f(k,z)$ without compression.  
This procedure allows the ILG scale dependence to manifest cleanly across wavenumber and redshift.

\medskip

In the absence of anisotropic stress and with a source-side modification of Poisson, the bias-robust statistic
\begin{equation}
\label{eq:EG_definition}
    E_G(k,z)
    = \frac{\nabla^2(\Phi + \Psi)}
           {3H_0^2 a^{-1} f(k,z) \delta_m}
\end{equation}
simplifies dramatically.  
Using the ILG Poisson equation leads to the prediction
\begin{equation}
\label{eq:EG_pred}
    E_G^{\rm ILG}(k,z)
    = \frac{\Omega_{m0}}{f(k,z)}\,w\big(k,a(z)\big).
\end{equation}

\paragraph{Consequences.}
ILG therefore implies:
\begin{itemize}
    \item \textbf{$E_G/\Omega_{m0} = w/f$ is a universal function of $X$.}  
          All scale and time dependence must collapse to the single-parameter family $E_G(X)$.
    \item \textbf{$E_G$ is exactly tracer independent.}  
          No galaxy-population effects may contaminate this observable; it is fixed entirely by the gravitational sector.
    \item \textbf{This provides a sharp internal-consistency test.}  
          Because both $w$ and $f$ are fixed by the model definition, $E_G$ can be checked across matched $(k,z)$ bins; see Table~\ref{tab:key_falsifiers}.
\end{itemize}

Taken together, RSD and $E_G$ provide some of the most incisive observational handles on ILG.  
Their combined scale-dependent structure is tightly constrained and strongly testable; see Table~\ref{tab:key_falsifiers} for the key directional criteria. Concretely: BOSS has measured $E_G\approx 0.4\pm 0.05$ at $z\sim 0.6$ (broadly consistent with GR's $E_G\approx \Omega_m/f\approx 0.4$) \cite{BOSS2017}. 

\textbf{ILG prediction (structure, not numerics):} ILG predicts $E_G/\Omega_{m0}=w/f$ with the same fixed $X$-dependence as the kernel. \textbf{In realistic $\Lambda$CDM}, the ratio $w/f$ depends on $(k,z)$ through $X=k\tau_0/a$; Paper~II provides quantitative predictions. The key tests in Paper~I are tracer-independence and $X$-collapse across matched $(k,z)$ bins.

\paragraph{Summary of observable signatures.}

This section derived three key observational signatures:

\textbf{(i) Scale-dependent growth enhancement:} ILG predicts $f(k,z)$ varies with $X = k\tau_0/a$, distinguishing it sharply from GR's scale-independent $f(z)$. Section~\ref{subsec:rsd_eg} quantified the scale dependence in the EdS illustration; Paper~II will quantify amplitudes in realistic $\Lambda$CDM.

\textbf{(ii) Suppressed ISW amplitude:} ILG predicts $0 < S_{\rm ISW} < 1$ (Section~\ref{subsec:isw}) due to source-driven potential deepening partially countering $\Lambda$-driven decay. Physical origin: scale-dependent growth plus kernel time dependence ($d\ln w/d\ln a > 0$) slow potential evolution, yielding a weaker positive ISW signal than $\Lambda$CDM.

\textbf{(iii) Tracer-independent $E_G$ with $X$-collapse:} ILG predicts $E_G = \Omega_m w/f$ is exactly tracer-independent and collapses to a universal function of $X$ (Section~\ref{subsec:rsd_eg}). X-collapse provides a sharp falsification test: failure to recover consistent $X$-scaling across probes would rule out ILG.

Table~\ref{tab:key_falsifiers} centralizes falsification criteria: ISW \emph{enhancement} (vs.\ suppression), scale-\emph{independent} $f(k,z)$, tracer-\emph{dependent} $E_G$, or failed $X$-collapse would each rule out the framework. These predictions are qualitative and directional; quantitative $\Lambda$CDM amplitudes (Paper~II) will enable forecast-level tests. We now confront ILG with current observations.


\section{Confrontation with Existing Data}
\label{sec:existing_data}

Section~\ref{sec:predictions} derived a small set of qualitative, directional signatures: scale-dependent growth (Section~\ref{subsec:rsd_eg}), lensing response enhancement through $R_L\propto w^2D^2/a^2$ (Section~\ref{subsec:lensing}), ISW suppression with $0<S_{\rm ISW}<1$ (Section~\ref{subsec:isw}), and tracer-independent $E_G=\Omega_{m0}w/f$ with $X$-collapse structure (Section~\ref{subsec:rsd_eg}). We now confront these signatures with current observations from BOSS/eBOSS (RSD), Planck (CMB lensing and ISW), KiDS/DES (weak lensing), and existing $E_G$ measurements.

The framework is mathematically sound—proven in Section~\ref{sec:mathfound} (well-posedness for $\alpha < 1/2$, zero Buchert backreaction ensuring $H(z)$ unchanged)—guaranteeing internal consistency and numerical stability. \textbf{Critical context for data confrontation (connecting to Section~\ref{subsec:parameters}):} ILG's three constants $(C, \alpha, \tau_0)$ were derived from Recognition Science axioms—not fitted to cosmological observations—and held fixed throughout all predictions in Section~\ref{sec:predictions}. This zero-free-parameter structure makes the data confrontation fundamentally different from parameter-fitting exercises: there is no tuning freedom to improve agreement. Here we assess observational viability at the level of consistency with existing constraints; a definitive confrontation requires the dedicated numerical pipeline and likelihood analysis deferred to Paper~II.

\subsection{Current observational landscape}
\label{subsec:current_obs}

\emph{Preliminary note:} This section is a \emph{consistency check}, not a parameter fit. Current datasets were largely analyzed under scale-independent templates; ILG requires preserving $(k,z)$ structure. A rigorous confrontation (likelihood-level, with $\Lambda$CDM numerical integration and nonlinear calibration where needed) is deferred to Paper~II.

\paragraph{What current data can (and cannot) test.}
\begin{itemize}
\item \textbf{RSD growth}: Existing BOSS/eBOSS constraints on $f\sigma_8(z)$ \cite{BOSS2017} are typically reported after substantial scale compression. They therefore do not yet directly test the ILG hallmark $f=f(k,z)$ at fixed $z$.
\item \textbf{Weak lensing and CMB lensing}: Lensing is sensitive to $R_L\propto w^2D^2$. Current analyses integrate over broad $\ell$ ranges and mix linear and nonlinear scales; interpreting an ILG signal requires nonlinear modeling and careful forward pipelines (Paper~II).
\item \textbf{ISW cross-correlation}: Current ISW detections \cite{Planck2016ISW} are consistent with a positive signal and remain too noisy to sharply distinguish suppression versus enhancement in a model-specific way; the sign/direction test remains viable and motivates targeted analyses.
\item \textbf{$E_G$}: Existing $E_G$ measurements \cite{BOSS2017} are not yet binned finely enough in $(k,z)$ to test the predicted tracer-independence together with $X$-collapse.
\end{itemize}

\paragraph{Bottom line (current status).}
No single existing probe provides a decisive, model-designed test of ILG's fixed $(k,z)$ structure. At the level of this coarse consistency check, ILG is not obviously excluded; conversely, present pipelines are not yet optimized to confirm (or falsify) $X$-collapse and reciprocity.

\subsection{Preliminary compatibility assessment}

\paragraph{Qualitative conclusion.}
ILG with RS-derived parameters is \emph{not obviously excluded} by existing constraints:
\begin{itemize}
\item No existing probe definitively excludes the kernel at high statistical significance.
\item Some predictions (e.g., enhanced $S_8$ from growth) trend in the wrong direction for 
      resolving current tensions; a quantitative statement about direction and amplitude requires Paper~II's $\Lambda$CDM numerical integration and nonlinear modeling.
\item ISW suppression is testable but not yet constrained tightly enough to confirm or exclude.
\item The scale-dependence predicted by $X$-universality has not been explicitly tested; existing 
      analyses typically compress over scales, erasing the signature.
\end{itemize}

\paragraph{Caveats and limitations.}
This assessment is \emph{qualitative} and preliminary:
\begin{enumerate}
\item No full likelihood analysis or MCMC has been performed; we do not quote goodness-of-fit significances here.
\item Systematic effects (intrinsic alignments, baryonic feedback, photometric redshift uncertainties) 
      are not accounted for and could shift conclusions.
\item Most existing analyses assume scale-independent modifications. ILG's scale-dependence requires 
      reanalysis with finer $k$-binning and preservation of $(k,z)$ structure.
\item Numerical predictions require full integration in $\Lambda$CDM and (beyond linear scales) simulations; EdS estimates used here are illustrative.
\end{enumerate}

\paragraph{Observational requirements for testing.}

Section~\ref{sec:kernel_properties} identified X-universality and the reciprocity relation $\partial_{\ln a}\ln Q \approx -\partial_{\ln k}\ln Q$ as central ILG signatures. Yet existing RSD, lensing, and $E_G$ analyses have not tested these predictions. Three requirements must be met to test this prediction:

\emph{(i) Implicit scale-independence assumption:} Standard cosmological analyses assume $f(z)$ is scale-independent (as in GR), compressing RSD measurements into effective $f\sigma_8(z)$ values averaged over broad $k$ ranges. BOSS/eBOSS pipelines typically report single $f\sigma_8$ values per redshift bin, erasing the scale-dependent information. Extracting $f(k,z)$ from multipole ratios $P_2/P_0$ requires preserving fine $k$-binning and avoiding pre-compression—a reanalysis not performed in standard pipelines.

\emph{(ii) Statistical precision:} Testing X-collapse requires measuring observables in matched $(k,z)$ bins spanning a wide $X$ range. With current samples and pipelines, per-bin uncertainties are typically comparable to (or larger than) the expected small scale dependence, and analyses often compress over $k$ in ways that remove the signature. Detecting X-collapse robustly therefore requires larger surveys and analysis choices that preserve the $(k,z)$ structure.

\emph{(iii) Analysis design mismatch:} Reciprocity tests require \emph{simultaneous} scale and redshift binning with controlled $X = k\tau_0/a$. Most existing analyses either compress over scales (e.g., $f\sigma_8(z)$ summaries) or compress over redshifts (e.g., tomographic lensing bins). The two-dimensional $(k,z)$ structure must be preserved to test ILG's unique signature—analogous to preserving multipole structure in RSD rather than compressing to monopole-only fits.

\textbf{Path forward:} Paper~II will provide mock likelihood analyses demonstrating how to extract $f(k,z)$ in fine $k$-bins, construct X-collapse plots, and test reciprocity slopes within a realistic $\Lambda$CDM pipeline.

\paragraph{Required for a full data confrontation (beyond this paper).}
A complete confrontation with data requires:
\begin{itemize}
\item Full numerical integration of growth equations in $\Lambda$CDM for ILG.
\item N-body simulations to calibrate nonlinear corrections and baryonic effects.
\item Mock survey analyses to propagate ILG effects through realistic measurement pipelines.
\item Joint likelihood analysis across multiple probes with proper covariance matrices.
\end{itemize}

The present discussion establishes that ILG is not \emph{obviously} excluded at the level of back-of-the-envelope comparisons, but a rigorous confrontation requires a full likelihood analysis. The framework's value lies in making sharp predictions that upcoming data can test with increasing precision.

\subsection{Comparison with Modified Gravity Theories}
\label{sec:comparison}

Section~\ref{subsec:current_obs} established that ILG is marginally compatible with existing data at the level of preliminary comparisons. Critically, this marginal compatibility arises with \emph{zero tunable parameters}—distinguishing ILG from the broader modified-gravity landscape where free functions or fitted parameters enable post-hoc accommodation of data. We now place ILG in this landscape, contrasting with other prominent frameworks ($f(R)$, Horndeski, DGP, massive gravity, non-local models) and identifying observational discriminants that distinguish ILG from alternatives.

\subsubsection{Comparison with other theories}
\label{subsec:comparison_mg}

Having established ILG's marginal compatibility with existing data (Section~\ref{subsec:current_obs})—notable given zero tunable parameters—we now contrast ILG's structure and predictions with other modified-gravity frameworks. This comparison clarifies ILG's unique position in the MG landscape and motivates the discriminating observables (Section~\ref{subsec:discriminating}) that will enable decisive tests.

We briefly contrast ILG with other modified-gravity frameworks:

\begin{itemize}
    \item \textbf{$f(R)$ theories:}
    Modify the Einstein-Hilbert action by replacing $R$ with a function $f(R)$.
    These introduce an additional scalar degree of freedom and generically alter both background expansion and perturbation growth.
    \emph{Contrast:} ILG modifies only the source term in the Poisson equation; the metric and field equations remain standard GR.
    $f(R)$ models typically have free functions or multiple parameters; ILG has none.
    
    \item \textbf{Horndeski/Galileon theories:}
    The most general scalar-tensor theories with second-order equations of motion.
    These involve multiple free functions of the scalar field and its derivatives.
    \emph{Contrast:} ILG does not introduce new dynamical fields.
    Horndeski theories have broad phenomenological flexibility; ILG is rigid and strongly testable.
    
    \item \textbf{DGP (Dvali-Gabadadze-Porrati):}
    A braneworld model where gravity leaks into an extra dimension at large scales, modifying both $H(z)$ and growth.
    \emph{Contrast:} DGP modifies graviton propagation in higher dimensions; ILG modifies source coupling in 3+1D.
    DGP predicts specific growth-rate $f(k,z)$ and $E_G$ behaviors; ILG predictions differ (e.g., suppressed positive ISW, universal $X$-dependence).
    
    \item \textbf{Massive gravity:}
    Gives the graviton a small mass, introducing new polarizations and Yukawa-like modifications to the potential.
    \emph{Contrast:} Massive gravity modifies propagator; ILG modifies source.
    Massive gravity has a free mass parameter $m_g$; ILG's $\tau_0$ is fixed.
    
    \item \textbf{Non-local gravity (Deser-Woodard, Maggiore-Mancarella):}
    Modifies Einstein's equations with non-local terms involving $\Box^{-1}$.
    \emph{Similarity:} ILG's covariant embedding (if achieved) would also involve non-local operators.
    \emph{Contrast:} Existing non-local models introduce phenomenological functions fitted to data; ILG's kernel is phenomenologically fixed with no free functions.
\end{itemize}

\subsubsection{Discriminating observables}
\label{subsec:discriminating}

Several observational handles distinguish ILG from alternative theories. These discriminants (summarized in Table~\ref{tab:key_falsifiers} as falsification criteria) provide orthogonal tests across multiple probes. Importantly, Section~\ref{subsec:current_obs} showed that current data have not yet tested these unique signatures—existing analyses often lack the scale-binning or analysis design needed to distinguish ILG from parameter-fitted alternatives. Next-generation surveys and re-analyses designed around the ILG signatures can close this gap:

\begin{enumerate}
\item \textbf{ISW amplitude (unique ILG signature):} Most modified gravity models (including $f(R)$, DGP, Horndeski) predict enhanced \emph{positive} ISW due to stronger time-variation of potentials \cite{Planck2016ISW}. ILG instead predicts a \emph{positive but suppressed} ISW signal relative to $\Lambda$CDM ($0 < S_{\rm ISW} < 1$, Section~\ref{subsec:isw})—a unique signature tracing to source-side deepening of potentials partially countering $\Lambda$-driven decay. \emph{Current status} (Section~\ref{subsec:current_obs}): existing ISW cross-correlation measurements are broadly consistent with $\Lambda$CDM and do not yet decisively test suppression versus enhancement. \emph{Test}: improved cross-correlations with optimized tracer samples and explicit modeling of the ILG suppression shape.

    \item \textbf{$X$-universality (connecting to Section~\ref{sec:kernel_properties} and Section~\ref{subsec:current_obs}'s analysis gap):} The collapse of all linear observables to a single variable $X = k\tau_0/a$ and the reciprocity relation $\partial_{\ln a}\ln Q = -\partial_{\ln k}\ln Q$ are specific to ILG's source-side structure. No field-theoretic modified gravity model predicts this collapse; it arises uniquely from the kernel form $w(k,a) = 1 + C(k\tau_0/a)^{-\alpha}$. \emph{Current status}: Not tested—existing analyses often compress over scale. \emph{Test}: extract $f(k,z)$ and related observables in fine $(k,z)$ bins, plot against $X$, and verify collapse/reciprocity slopes in matched bins.

\item \textbf{Tracer-independent $E_G$ with fixed functional form:} While $E_G$ is designed to be bias-free, ILG predicts a specific functional form $E_G(k,z) = \Omega_{m0}\,w(X)/f(X)$ (Eq.~\ref{eq:EG_pred}, Section~\ref{subsec:rsd_eg}) with \emph{no free parameters}. $f(R)$ and DGP predict different $E_G(k,z)$ scalings; Horndeski models have free functions allowing flexible $E_G$ shapes. \emph{Current status} (Section~\ref{subsec:current_obs}): current measurements are not yet sufficient to test scale/redshift dependence or $X$-collapse. \emph{Test}: measure $E_G$ in matched $(k,z)$ bins, verify tracer-independence across multiple samples, and test the predicted $X$-structure.

    \item \textbf{Background decoupling (orthogonal consistency check):} Pure ILG (perturbations only) leaves $H(z)$ unchanged (Section~\ref{sec:mathfound}: zero Buchert backreaction, $Q_D = 0$); most modified gravity models alter both background and perturbations simultaneously, predicting shifts in BAO scales, angular diameter distances, or $H(z)$ measurements. \emph{Test}: if ILG-like growth/lensing signatures are detected while background-distance probes remain standard, this supports the source-side interpretation; conversely, a significant background deviation would disfavor a pure source-side framework.
\end{enumerate}

These four discriminants enable identification of ILG vs. alternatives once data and analyses reach sufficient precision and preserve the relevant scale dependence. Critically, ILG's zero-free-parameter structure (Section~\ref{subsec:parameters}) makes the tests conceptually binary (no parameter freedom to tune away failed signatures), unlike parameter-rich theories that can accommodate mixed results by adjusting free functions.

\subsection{Summary: Observational status and path forward}
\label{subsec:obs_summary}

Section~\ref{sec:existing_data} provided a preliminary, qualitative confrontation with current observations and emphasized a key practical point: many existing analyses are not designed to preserve the $(k,z)$ structure that ILG predicts. At present, ILG is neither confirmed nor clearly excluded by these coarse comparisons.

\paragraph{What must be tested.}
The decisive empirical signatures are the ones already centralized in Table~\ref{tab:key_falsifiers}: (i) ISW \emph{suppression} rather than enhancement; (ii) scale-dependent growth tied to $X=k\tau_0/a$; (iii) tracer-independent $E_G=\Omega_{m0}w/f$ with consistent $X$-collapse; and (iv) background decoupling (no change in $H(z)$) as an orthogonal consistency check.

\paragraph{What Paper II adds.}
Paper~II (in preparation \cite{PaperII}) will provide quantitative $\Lambda$CDM predictions via numerical integration (and, where required, simulation-based estimates), and will implement a likelihood-level comparison using analysis choices that preserve the ILG scale dependence (fine $k$-binning, matched $(k,z)$ bins, and cross-probe consistency tests). This will enable a clean verdict on the fixed-parameter ILG kernel within its stated regime of validity.


\section{Scope, Limitations, and Covariant Completion}
\label{sec:scope_limitations}

Sections~\ref{sec:model}--\ref{sec:mathfound} established ILG's theoretical foundations: a source-side modification with three fixed constants $(C, \alpha, \tau_0)$ specified independently of cosmological data (Section~\ref{subsec:parameters}), mathematically well-posed for $\alpha < 1/2$ (Section~\ref{sec:mathfound}), and producing zero Buchert backreaction (Appendix~\ref{app:buchert}). Section~\ref{sec:predictions} derived sharp, testable qualitative signatures: scale-dependent growth tied to $X=k\tau_0/a$, suppressed ISW amplitude, and tracer-independent $E_G$ with $X$-collapse. Section~\ref{sec:existing_data} showed these signatures are not obviously excluded by current constraints, while emphasizing that a definitive confrontation requires the dedicated numerical and likelihood pipeline deferred to Paper~II.

\textbf{Scope of this section:} Before proceeding to the Conclusion (Section~\ref{sec:conclusion}), we address the framework's explicit boundaries and acknowledge its primary theoretical limitation. This section: (i)~justifies the effective-theory regime ($0.01 \lesssim k \lesssim 0.2\,h\,{\rm Mpc}^{-1}$, $0 \lesssim z \lesssim 2$) where ILG's Newtonian formulation is valid, connecting back to Assumptions A1--A6 (Section~\ref{subsec:assumptions}); (ii)~discusses the covariant completion gap, clarifying why this open problem does \emph{not} undermine the framework's testability (observational falsification precedes formal embedding); and (iii)~synthesizes how these limitations interact with the binary testing structure emphasized throughout. ILG offers sharply testable predictions within a clearly defined domain, with empirical validation driving theoretical completion rather than vice versa.

\subsection{Working assumptions}
\label{subsec:assumptions}

The following assumptions define the effective-theory regime and closure conditions under which ILG is formulated and tested.

{\renewcommand{\theenumi}{A\arabic{enumi}}%
 \renewcommand{\labelenumi}{\textbf{\theenumi.}}%
\begin{enumerate}
    \item \label{assump:newtonian} \textbf{Newtonian-gauge, weak-field perturbations.}
    Newtonian-gauge description on a flat FRW background; gravitational potentials satisfy $|\Phi|,|\Psi|\ll 1$.

    \item \label{assump:qs} \textbf{Quasi-static, sub-horizon limit.}
    Time derivatives of potentials are negligible compared to spatial gradients; operationally $k/(aH)\gtrsim 3$--$5$.

    \item \label{assump:nostress} \textbf{Negligible anisotropic stress.}
    On linear scales, $\Phi=\Psi$.

    \item \label{assump:fluid} \textbf{Unmodified matter fluid equations.}
    Continuity and Euler equations retain their standard GR form; ILG modifies only the Poisson sector.

    \item \label{assump:source} \textbf{Sourcing choice.}
    Total matter sourcing, $\rho_s=\rho_m$ (equivalently $\Omega_{s0}=\Omega_{m0}$).

    \item \label{assump:screen_ir} \textbf{Infrared regularity.}
    $\alpha\in(0,1/2)$ ensures $L^2$ regularity of the modified Poisson operator (Appendix~\ref{app:continuum}).
\end{enumerate}}

\paragraph{Strong-field prescription (outside the cosmological regime).}
In high-potential environments ($|\Phi| \gtrsim 10^{-4}$: Solar System, compact objects, galaxy cluster cores), we impose $w\equiv 1$ by construction. A convenient phenomenological saturation function is
\begin{equation}
\label{eq:screening}
    S(\Phi)=\left(1+\frac{|\Phi|}{\Phi_{\rm crit}}\right)^{-1},\qquad \Phi_{\rm crit}\sim 10^{-4},
\end{equation}
so that a screened extension of the kernel would read $w(k,a,\Phi)=1+C\,S(\Phi)\,X^{-\alpha}$. Throughout this paper we work in the cosmological linear regime where $|\Phi|\lesssim 10^{-5}$ and set $S(\Phi)\equiv 1$ (i.e., the screening prescription plays no role in any prediction derived here).

\subsection{Justification of effective regime}
\label{subsec:regime_justification}

Subsection~\ref{subsec:assumptions} lists six working assumptions (A1--A6) defining ILG's effective-theory regime: Newtonian gauge, quasi-static sub-horizon limit, negligible anisotropic stress, unmodified fluid equations, total matter sourcing, and infrared regularity $\alpha \in (0,1/2)$. These assumptions operationally constrain ILG to the quasi-static, sub-horizon, linear regime on scales $0.01 \lesssim k \lesssim 0.2\,h\,{\rm Mpc}^{-1}$ and redshifts $0 \lesssim z \lesssim 2$. Here we justify these boundaries quantitatively and connect them to the observational tests developed in Sections~\ref{sec:predictions}--\ref{sec:existing_data}.

\paragraph{Why these boundaries matter (connecting to observational predictions).}

The effective regime boundaries are not arbitrary: they define where ILG's Newtonian formulation is both \emph{theoretically valid} (assumptions A1--A6 hold) and \emph{observationally testable}. Section~\ref{sec:predictions}'s predictions—scale-dependent $f(k,z)$, lensing response enhancement through $R_L$, ISW suppression, and $E_G$ $X$-collapse—all apply within this regime. Outside the boundaries, either the theoretical framework breaks down (requiring extensions) or observational constraints are weak (providing no falsification leverage).

\textbf{Comment on parameter placement (connecting to Section~\ref{subsec:parameters}):} The fixed parameters set where the kernel transitions from $w\simeq 1$ to its enhanced regime in terms of $X=k\tau_0/a$. Regardless of the underlying motivation, this makes the observational test well-defined: the same $(C,\alpha,\tau_0)$ govern all probes, and the most informative constraints come from regimes where the quasi-static approximation holds and linear theory is reliable. The boundaries arise from the following considerations:

\paragraph{Lower $k$ bound: quasi-static approximation (connecting to Assumption~\ref{assump:qs}).}
The quasi-static approximation (Assumption~\ref{assump:qs}) requires $k/(aH) \gtrsim 3$--$5$. At $z=0$ and $H_0=70$ km/s/Mpc, this translates to $k \gtrsim 0.01\,h\,{\rm Mpc}^{-1}$. Below this scale, time derivatives of potentials become non-negligible, and the modified Poisson equation alone does not fully capture dynamics. \emph{Observational relevance:} BAO measurements probe $k \sim 0.05$--$0.15\,h\,{\rm Mpc}^{-1}$ (Section~\ref{sec:existing_data}), safely above this bound. The ISW effect (Section~\ref{subsec:isw}) probes $\ell \lesssim 30$ ($k \lesssim 0.01\,h\,{\rm Mpc}^{-1}$), approaching the quasi-static boundary—ILG's ISW suppression prediction should be interpreted as directional (sign/relative amplitude) rather than quantitatively precise near $k_{\rm min}$.

\paragraph{Upper $k$ bound: non-linear regime.}
At $k \gtrsim 0.2\,h\,{\rm Mpc}^{-1}$ and $z \lesssim 1$, non-linear corrections to density fluctuations become $\mathcal{O}(1)$, invalidating linear perturbation theory. Section~\ref{sec:framework}'s EdS solution and the growth equation~\eqref{eq:delta_ln} apply only in the linear regime; non-linear predictions require N-body simulations (Paper~II). \emph{Double caveat (connecting to Section~\ref{subsec:growth}'s critical EdS warning):} Section~\ref{subsec:growth} emphasized that the EdS closed-form solution is pedagogical only—realistic $\Lambda$CDM predictions require numerical integration because dark energy dominates at $z < 1$. Here we add a second layer: even with full $\Lambda$CDM integration, predictions remain linear-theory-only up to $k \sim 0.2\,h\,{\rm Mpc}^{-1}$. Paper~II addresses both limitations simultaneously via full $\Lambda$CDM numerical integration \emph{and} N-body simulations capturing nonlinear structure formation, halo mass functions, and baryonic feedback. \emph{Observational relevance:} Weak lensing surveys (Section~\ref{subsec:lensing}) probe $\ell \sim 100$--$3000$ (corresponding to $k \sim 0.05$--$1.5\,h\,{\rm Mpc}^{-1}$), spanning linear and non-linear regimes. Section~\ref{sec:existing_data}'s $S_8$ discussion emphasized that resolving ILG's impact on lensing requires these nonlinear simulations to capture scale-dependent halo responses and baryonic feedback—linear theory provides directional predictions only.

\paragraph{Redshift range: observational coverage and GR recovery.}
The range $0 \lesssim z \lesssim 2$ covers current and near-future LSS surveys (BOSS, DESI, Euclid) where RSD, lensing, and ISW data are most constraining. At $z \gtrsim 2$, ILG effects are suppressed by kinematic screening (Section~\ref{subsec:consequences}): $X = k\tau_0/a \gg 1$ drives $w \to 1$, recovering GR to high accuracy. This automatic early-time recovery (Section~\ref{sec:kernel_properties}: property 1) ensures compatibility with CMB constraints without requiring any additional cosmological screening mechanism.

\paragraph{Synthesis: Validity domain and falsifiability structure.}

The effective regime boundaries define a clear domain where ILG makes sharp, testable predictions. \textbf{Within the regime} ($0.01 \lesssim k \lesssim 0.2\,h\,{\rm Mpc}^{-1}$, $0 \lesssim z \lesssim 2$, $|\Phi| \lesssim 10^{-5}$), the Newtonian formulation is theoretically sound (Assumptions A1--A6 satisfied, Section~\ref{sec:mathfound}: mathematically well-posed). Section~\ref{sec:predictions}'s predictions—scale-dependent $f(k,z)$, suppressed ISW, tracer-independent $E_G$—all apply here, and the fixed-parameter structure leaves no freedom to tune away failed signatures.

\textbf{Outside the regime}, ILG makes no claims. At $k < 0.01\,h\,{\rm Mpc}^{-1}$ (ISW scales), predictions are directional only. At $k > 0.2\,h\,{\rm Mpc}^{-1}$ (nonlinear scales), N-body simulations are required (Paper~II). At $z > 2$ (high redshift), kinematic screening drives $w \to 1$ automatically, recovering GR. In strong fields ($|\Phi| > 10^{-4}$), phenomenological screening $S(\Phi)$ enforces $w \equiv 1$. This transparent domain specification distinguishes ILG from theories that make unbounded claims: falsification within the specified regime rules out the framework completely, while consistency within the regime establishes ILG as a viable effective theory (motivating covariant completion, Section~\ref{subsec:covariant_completion}).

\textbf{Why explicit boundaries are scientifically advantageous:} Clear regime specification is not a weakness—it's a strength that enhances falsifiability by preventing post-hoc adjustments, focusing observational targeting on where the theory applies, and clarifying what additional machinery (e.g., nonlinear simulations) is required beyond the linear/quasi-static regime.

\subsection{Covariant completion: the main theoretical gap}
\label{subsec:covariant_completion}

Section~\ref{sec:mathfound} established that ILG's Newtonian formulation is mathematically rigorous: the modified Poisson equation admits unique weak solutions (Theorems~\ref{thm:torus}--\ref{thm:R3}, Appendix~\ref{app:continuum}), numerical discretizations converge reliably (enabling N-body implementations, Paper~II), and zero Buchert backreaction (Appendix~\ref{app:buchert}) ensures internal consistency. Section~\ref{sec:predictions} derived sharp, testable predictions within this framework. \textbf{Yet a fundamental theoretical gap remains:} ILG lacks a fully covariant, relativistic embedding into a field-theoretic framework. This subsection clarifies what a covariant completion would provide, why its absence does \emph{not} undermine the framework's current testability, and when completion becomes scientifically urgent.

\paragraph{What covariant completion would provide (and why it's currently missing).}

The Newtonian formulation (Sections~\ref{sec:model}--\ref{sec:framework}) is internally consistent within its regime (Section~\ref{subsec:regime_justification}) but leaves several questions unanswered:

\textbf{(i) Relativistic generalization beyond weak fields:} The modified Poisson equation $k^2\Phi = 4\pi Ga^2\rho_m w(k,a)\delta$ applies in the Newtonian gauge, weak-field limit. A covariant theory would specify how the modification extends to strong-field or dynamical regimes (e.g., neutron stars, black holes, cosmological horizons).

\textbf{(ii) Gravitational wave sector:} ILG as formulated modifies scalar potentials only; tensor modes (gravitational waves) are not addressed. GW170817's constraint $|c_{\rm GW}/c - 1| < 10^{-15}$ \cite{LIGOVirgo2017} is automatically satisfied if ILG leaves tensor propagation unmodified, but a covariant theory would clarify this explicitly and predict any polarization or dispersion effects.

\textbf{(iii) Gauge structure and observable definitions:} The Newtonian-gauge formulation is gauge-fixed. A covariant theory would identify gauge-invariant combinations (e.g., Bardeen potentials, curvature perturbations) and clarify how observables transform under coordinate changes—important for precision cosmology where gauge artifacts can mimic physical effects.

\textbf{(iv) Motivating the source-side structure from first principles:} Section~\ref{subsec:parameters} motivated $(C, \alpha, \tau_0)$ from Recognition Science axioms, but the \emph{source-side modification} itself (rescaling $T_{\mu\nu}$ rather than modifying $G_{\mu\nu}$ or introducing new fields) is phenomenological. A covariant embedding would reveal whether this structure arises naturally from deeper principles (e.g., information-theoretic constraints on stress-energy propagation) or requires additional input.

\paragraph{Why covariant completion is the "main theoretical gap".}

Despite Section~\ref{sec:mathfound}'s mathematical rigor proofs—which established that ILG is well-posed, numerically stable, and self-consistent within its Newtonian regime—the lack of covariant embedding represents the theory's primary \emph{conceptual} limitation. The Newtonian framework is mathematically complete for its domain but theoretically incomplete in three senses:

\emph{(i) Limited scope:} Predictions apply only within the quasi-static, weak-field regime (Section~\ref{subsec:regime_justification}). A covariant theory would extend the domain or clarify fundamental barriers to extension.

\emph{(ii) Phenomenological structure:} The source-side modification is posited rather than derived from a covariant action principle. Without a Lagrangian formulation, energy-momentum conservation, causality, and ghost-freedom are verified case-by-case (Section~\ref{subsec:covariant_completion} below) rather than guaranteed by symmetry principles.

\emph{(iii) Interpretational ambiguity:} Is ILG an effective theory (valid below some cutoff scale) or a fundamental modification of GR? A covariant embedding would clarify its status and identify potential UV completions.

\textbf{Critically, this theoretical gap does NOT undermine observational testability:} Section~\ref{sec:predictions}'s predictions—scale-dependent $f(k,z)$, suppressed ISW, tracer-independent $E_G$—are derived rigorously within the Newtonian regime and can be tested regardless of covariant completion status. The pragmatic approach is empirical validation first, formal embedding second.

\paragraph{Candidate covariant operators.}
The modified Poisson equation $k^2\Phi = 4\pi Ga^2\rho_m w(k,a)\delta$ suggests a non-local modification of Einstein's equations. Potential covariant realizations include:
\begin{itemize}
\item \emph{Non-local stress-energy coupling}: $G_{\mu\nu} = 8\pi G\, \mathcal{F}[\Box]\,T_{\mu\nu}$, where $\mathcal{F}$ is a non-local operator with Fourier symbol $w(k,a)$.
\item \emph{Modified effective metric}: Source fields couple to an effective metric $\tilde{g}_{\mu\nu} = w(k,a)\, g_{\mu\nu}$ in the weak-field limit.
\item \emph{Auxiliary scalar field}: Introduce a dynamical scalar $\phi$ with action tuned such that $\phi \approx w(k,a)$ in the Newtonian regime.
\end{itemize}
Each approach faces technical obstacles (see below).

\paragraph{Theoretical consistency requirements.}
Any covariant completion must satisfy:
\begin{enumerate}
\item \textbf{Bianchi identity}: $\nabla^\mu G_{\mu\nu}=0$ must hold, ensuring local energy-momentum conservation. Non-local modifications can violate this unless carefully structured.
\item \textbf{Causality}: The non-local operator $\mathcal{F}[\Box]$ must respect the light cone; retarded Green's functions only.
\item \textbf{Ghost-freedom}: Higher-derivative or non-local terms can introduce negative-energy modes (Ostrogradsky instability). The kernel $w(k,a)$ must correspond to a stable propagator.
\item \textbf{GR limit}: The completion must reduce to GR at high $k$ and early times automatically, without fine-tuning.
\end{enumerate}

\paragraph{Potential impact on predictions.}
Without covariant completion, several issues remain unresolved:
\begin{itemize}
\item \textbf{Fluid equations}: The Newtonian framework assumes standard continuity and Euler equations. A covariant theory might induce back-reaction on these equations, altering growth predictions.
\item \textbf{Gauge invariance}: The Newtonian-gauge formulation is gauge-fixed. A covariant theory would clarify gauge transformations and observables.
\item \textbf{Gravitational waves}: ILG as formulated does not address tensor modes. A covariant theory would predict modifications to GW propagation (or confirm GW170817 compatibility).
\end{itemize}

\paragraph{Pragmatic perspective: Empirical validation precedes formal completion.}

ILG is an effective theory defined within its Newtonian regime (Section~\ref{subsec:regime_justification}). The predictions in Section~\ref{sec:predictions} are therefore testable within that regime regardless of whether a covariant completion is known. Paper~II provides quantitative $\Lambda$CDM predictions and a likelihood-level confrontation built to preserve ILG's scale dependence; covariant completion remains a separate theoretical question motivated only if the phenomenology survives empirical testing.

\paragraph{Implications for interpretation.}

The Abstract claimed ILG is a ``zero-free-parameter source-side modification'' addressing $S_8$ and ISW tensions while leaving $H(z)$ unchanged. The lack of covariant completion does not weaken these claims within the effective regime: 

\emph{(i)} The zero-free-parameter structure (Section~\ref{subsec:parameters}: $(C, \alpha, \tau_0)$ fixed by RS) is a property of the phenomenological model, not the covariant embedding. Empirical tests (Section~\ref{sec:existing_data}) proceed with these fixed values.

\emph{(ii)} The source-side modification (rescaling $\rho_m \delta$ in Poisson's equation, Section~\ref{subsec:defining_equations}) and zero Buchert backreaction (Appendix~\ref{app:buchert}) are rigorously established within the Newtonian framework. A covariant embedding would generalize but not invalidate these results.

\emph{(iii)} Observational predictions (Section~\ref{sec:predictions}) flow from the Newtonian formulation and are testable regardless of covariant status. If confirmed, covariant completion becomes the next scientific frontier; if falsified, the question is resolved without requiring embedding.

\textbf{Bottom line:} Covariant completion is the main \emph{theoretical} gap but not a barrier to empirical testing. ILG is a well-defined effective theory (Section~\ref{sec:mathfound}: mathematically rigorous; Section~\ref{subsec:regime_justification}: clearly bounded domain; Section~\ref{sec:predictions}: sharp signatures). The scientific question is whether nature realizes this phenomenology within the stated regime; only then does covariant completion become urgent.

\paragraph{Section synthesis: Limitations clarify rather than undermine falsifiability (preparing for Conclusion).}

This section established ILG's explicit boundaries and acknowledged its primary theoretical limitation. Key takeaways:

\textbf{(i) Effective regime is well-justified:} The boundaries $0.01 \lesssim k \lesssim 0.2\,h\,{\rm Mpc}^{-1}$, $0 \lesssim z \lesssim 2$, $|\Phi| \lesssim 10^{-5}$ arise from theoretical requirements (Assumptions A1--A6, Section~\ref{subsec:assumptions}) and from where linear/quasi-static observables can be cleanly confronted with data (Section~\ref{sec:existing_data}). Kinematic screening ensures automatic GR recovery at $z \gtrsim 2$; phenomenological screening enforces GR in strong fields. The domain is transparent and testable.

\textbf{(ii) Covariant completion gap is acknowledged but not disabling:} Section~\ref{subsec:covariant_completion} clarified that covariant embedding would provide relativistic generalization, GW predictions, gauge structure, and first-principles motivation—but is \emph{not required} for testing Newtonian-regime predictions (Sections~\ref{sec:predictions}--\ref{sec:existing_data}). Empirical tests of the Newtonian-regime signatures therefore precede (and motivate) any covariant completion effort.

\textbf{(iii) Limitations enhance falsifiability:} By explicitly defining where ILG does and does \emph{not} make predictions, we sharpen the framework's testability. Within the effective regime, predictions are sharp and binary (Section~\ref{subsec:obs_summary}: confirmation or falsification, no adjustments). Outside the regime, ILG makes no claims—preventing unfalsifiable post-hoc adjustments when confronted with conflicting data.

These clarifications set the stage for the Conclusion (Section~\ref{sec:conclusion}), which synthesizes ILG's theoretical foundations (Sections~\ref{sec:model}--\ref{sec:mathfound}), observational predictions (Sections~\ref{sec:predictions}--\ref{sec:existing_data}), and explicit limitations (Section~\ref{sec:scope_limitations}) into a coherent assessment of the framework's scientific status and future prospects.


\section{Conclusion}
\label{sec:conclusion}

\paragraph{Summary.}
We have presented Information-Limited Gravity (ILG) as a fixed-parameter, source-side modification of the cosmological Poisson equation. Characterized by the kernel $w(k,a)=1+C(k\tau_0/a)^{-\alpha}$ with constants $(C,\alpha,\tau_0)$ derived from Recognition Science axioms, the framework introduces no free functions and no tunable parameters. This paper established the theory's foundations: the modification is mathematically well-posed for $\alpha \in (0,1/2)$ (Section~\ref{sec:mathfound}), numerically stable, and produces zero Buchert backreaction at linear order, ensuring that the background expansion $H(z)$ and standard distance observables remain unchanged.

\paragraph{Empirical status and next step.}
The framework makes sharp, directional predictions that distinguish it from the standard $\Lambda$CDM model and other modified gravity theories. Central among these are:
\begin{enumerate}[(i)]
    \item \textbf{Scale-dependent growth:} The growth rate $f(k,z)$ acquires a scale dependence governed by the variable $X=k\tau_0/a$, contrasting with the scale-independent prediction of GR.
    \item \textbf{ISW suppression:} ILG predicts a suppressed integrated Sachs-Wolfe signal ($0 < S_{\rm ISW} < 1$) due to the interplay between source-driven potential deepening and cosmic expansion, a signature distinct from the enhancement predicted by many scalar-tensor theories.
    \item \textbf{X-universality and Reciprocity:} Linear observables collapse to universal functions of $X$, satisfying the reciprocity identity $\partial_{\ln a}\ln Q \approx -\partial_{\ln k}\ln Q$.
    \item \textbf{Tracer-independent $E_G$:} The $E_G$ statistic is predicted to follow a fixed functional form $E_G = \Omega_{m0}w/f$ that is independent of galaxy bias.
\end{enumerate}

These predictions define a clear falsification surface. Because the parameters are fixed, any statistically significant deviation from these signatures—such as ISW enhancement, scale-independent growth, or tracer-dependent $E_G$—would rule out the framework. Conversely, the detection of consistent $X$-scaling across multiple probes would provide strong evidence for information-theoretic constraints on gravity.

We clarified the distinction between the theoretical foundations established here and the quantitative forecasts deferred to Paper~II. While the analytic Einstein-de Sitter solution (Section~\ref{sec:framework}) provides structural intuition, realistic constraints require the full $\Lambda$CDM numerical integration and non-linear simulations presented in the companion work. Similarly, while Recognition Science provides the motivating parameter values, the empirical test of ILG relies solely on its cosmological predictions within the defined effective regime ($0.01 \lesssim k \lesssim 0.2\,h\,{\rm Mpc}^{-1}$). The absence of a covariant completion remains the principal theoretical open question, but it does not impede the immediate empirical confrontation with Stage-IV data from DESI, Euclid, and Rubin.

By decoupling the background expansion from perturbation-level structure, ILG offers a specific mechanism to address the $S_8$ and ISW tensions without altering $H_0$. The coming decade of high-precision surveys will deliver a binary verdict on this zero-parameter hypothesis.


\appendix

\section{Discrete-to-continuum limit for the ILG-modified Poisson problem}
\label{app:continuum}

\noindent
This appendix gives a complete convergence proof for the ILG-modified Poisson problem under mesh refinement. We treat a periodic box (torus) first, which is the setting of PM/TreePM solvers, and then extend to $H^1_{\mathrm{loc}}(\mathbb{R}^3)$ under mild infrared (IR) conditions. The only model-specific ingredient is the RS kernel
\begin{equation}
\label{eq:app_kernel}
w(k,a)=1+\varphi^{-3/2}\Big(\frac{a}{k\,\tau_0}\Big)^{\alpha},\qquad
\alpha=\tfrac12(1-\varphi^{-1})\approx 0.19098\in(0,\tfrac12),
\end{equation}
fixed by the RS$\to$Classical bridge (no free parameters).

\subsection*{A.1 Discrete setting and assumptions}
\noindent
Fix a periodic box $\mathbb{T}_L^3=[0,L)^3$ and a mesh with spacing $\varepsilon=L/N\to0$. Let $\delta^\varepsilon:\mathbb{T}_L^3\to\mathbb{R}$ be mean-zero density contrasts with a uniform $L^2$ bound
\begin{equation}
\label{eq:app_L2_bound}
\int_{\mathbb{T}_L^3}\!|\delta^\varepsilon|^2\,dx\;\le C_0<\infty\qquad(\text{all }\varepsilon),
\end{equation}
and discrete Fourier coefficients $\widehat{\delta}^\varepsilon(\mathbf{k})$ supported on the Brillouin zone $\mathcal{B}_\varepsilon=\{\mathbf{k}=\tfrac{2\pi}{L}\mathbf{m}: \mathbf{m}\in\mathbb{Z}^3,\ |m_i|\le N/2\}$ with $\widehat{\delta}^\varepsilon(\mathbf{0})=0$. Denote the discrete Laplacian symbol by $\Lambda_\varepsilon(\mathbf{k})$ (e.g.\ for the standard 7-point stencil,
$\Lambda_\varepsilon(\mathbf{k})=\sum_i \tfrac{2}{\varepsilon^2}\bigl(1-\cos(k_i\varepsilon)\bigr)$), so that
\begin{equation}
\label{eq:app_laplacian_bounds}
c_1\,|\mathbf{k}|^2\;\le\;\Lambda_\varepsilon(\mathbf{k})\;\le\;c_2\,|\mathbf{k}|^2
\qquad\text{for all }\mathbf{k}\in\mathcal{B}_\varepsilon
\end{equation}
with $c_1,c_2>0$ independent of $\varepsilon$. We absorb any mass-assignment window deconvolution into $\delta^\varepsilon$ (assumed stable on $\mathcal{B}_\varepsilon\setminus\{\mathbf{0}\}$).

For a fixed scale factor $a\in(0,1]$, define the discrete potential $\Phi^\varepsilon$ by the spectral Poisson solve
\begin{equation}
\label{eq:disc-poisson}
\widehat{\Phi}^\varepsilon(\mathbf{k})\;=\;-\,\frac{4\pi G\,a^2\,\bar\rho_s(a)}{\Lambda_\varepsilon(\mathbf{k})}\,w(|\mathbf{k}|,a)\,\widehat{\delta}^\varepsilon(\mathbf{k}),
\qquad \mathbf{k}\neq\mathbf{0},\qquad \widehat{\Phi}^\varepsilon(\mathbf{0})=0.
\end{equation}
Here $\bar\rho_s$ is the \emph{sourcing} background density (total matter; $\bar\rho_s=\bar\rho_m$ in this work).

\subsection*{A.2 Uniform energy bound and compactness on the torus}
\noindent
Let $\nabla_\varepsilon$ denote the discrete gradient. By Parseval/Plancherel and the spectral identity $\|\nabla_\varepsilon \Phi^\varepsilon\|_{L^2}^2=\sum_{\mathbf{k}\in\mathcal{B}_\varepsilon}\Lambda_\varepsilon(\mathbf{k})|\widehat{\Phi}^\varepsilon(\mathbf{k})|^2$, we have
\begin{align}
\|\nabla_\varepsilon \Phi^\varepsilon\|_{L^2(\mathbb{T}_L^3)}^2
&= (4\pi G\,a^2\bar\rho_s)^2 \sum_{\mathbf{k}\ne\mathbf{0}}
\frac{|w(|\mathbf{k}|,a)|^2}{\Lambda_\varepsilon(\mathbf{k})}\,|\widehat{\delta}^\varepsilon(\mathbf{k})|^2 \nonumber\\
&\le C \sum_{\mathbf{k}\ne\mathbf{0}}
\Big(\frac{1}{|\mathbf{k}|^2} + \frac{a^{2\alpha}}{|\mathbf{k}|^{2+2\alpha}\tau_0^{2\alpha}}\Big)\,|\widehat{\delta}^\varepsilon(\mathbf{k})|^2
\label{eq:energy-bound}
\end{align}
using $|w| \le 1 + C (a/|\mathbf{k}|\tau_0)^{\alpha}$ and $\Lambda_\varepsilon\asymp |\mathbf{k}|^2$ uniformly. Since the smallest nonzero wavenumber on $\mathbb{T}_L^3$ is $k_{\min}=2\pi/L$, the weights in \eqref{eq:energy-bound} are bounded by $C(L,\alpha)$, and the $L^2$ bound on $\delta^\varepsilon$ yields
\begin{equation}
\label{eq:uniform-H1}
\|\nabla_\varepsilon \Phi^\varepsilon\|_{L^2(\mathbb{T}_L^3)} \;\le\; C_1(L,a,\tau_0,\varphi)\,\|\delta^\varepsilon\|_{L^2(\mathbb{T}_L^3)}
\;\le\; C_2(L,a,\tau_0,\varphi)\,.
\end{equation}
Thus $\{\Phi^\varepsilon\}$ is uniformly bounded in (discrete) $H^1(\mathbb{T}_L^3)$. By compactness (Banach–Alaoglu on the torus), there exists $\Phi\in H^1(\mathbb{T}_L^3)$ and a subsequence (not relabeled) such that
\begin{equation}
\label{eq:app_weak_conv_torus}
\nabla_\varepsilon \Phi^\varepsilon \rightharpoonup \nabla \Phi \quad \text{weakly in }L^2(\mathbb{T}_L^3).
\end{equation}

\subsection*{A.3 Identification of the limit equation (torus)}
\noindent
For any $\psi\in C^\infty(\mathbb{T}_L^3)$ with zero mean, the discrete weak form of \eqref{eq:disc-poisson} reads
\begin{equation}
\label{eq:weak-disc}
\int_{\mathbb{T}_L^3} \nabla_\varepsilon \Phi^\varepsilon \cdot \nabla_\varepsilon \psi\,dx
= 4\pi G\,a^2\bar\rho_s \sum_{\mathbf{k}\ne\mathbf{0}} w(|\mathbf{k}|,a)\,\widehat{\delta}^\varepsilon(\mathbf{k})\,\overline{\widehat{\psi}(\mathbf{k})}.
\end{equation}
Because $w(\cdot,a)$ is a bounded Fourier multiplier on $L^2(\mathbb{T}_L^3)$ for $\alpha\in(0,1)$ and $\widehat{\psi}$ decays rapidly, the right-hand side is continuous in $\widehat{\delta}^\varepsilon$. Passing to the limit $\varepsilon\to0$ along the convergent subsequence and using $\delta^\varepsilon\rightharpoonup \delta$ in $L^2$ (after extraction, since $\{\delta^\varepsilon\}$ is bounded) gives
\begin{equation}
\label{eq:weak-cont}
\int_{\mathbb{T}_L^3} \nabla \Phi \cdot \nabla \psi\,dx
= 4\pi G\,a^2\bar\rho_s \sum_{\mathbf{k}\ne\mathbf{0}} w(|\mathbf{k}|,a)\,\widehat{\delta}(\mathbf{k})\,\overline{\widehat{\psi}(\mathbf{k})}.
\end{equation}
Equivalently, in the Fourier domain
\begin{equation}
\label{eq:app_poisson_fourier}
-\,|\mathbf{k}|^2\,\widehat{\Phi}(\mathbf{k})
= 4\pi G\,a^2\bar\rho_s\, w(|\mathbf{k}|,a)\,\widehat{\delta}(\mathbf{k}),
\qquad \mathbf{k}\ne\mathbf{0},
\end{equation}
which is the ILG-modified Poisson equation on the torus, with the zero-mode fixed to $0$. We summarize:

\begin{theorem}[Torus convergence]
\label{thm:torus}
Let $\{\delta^\varepsilon\}\subset L^2_0(\mathbb{T}_L^3)$ be mean-zero with $\sup_\varepsilon\|\delta^\varepsilon\|_{L^2}<\infty$, and let $\Phi^\varepsilon$ solve \eqref{eq:disc-poisson}. Then there exists $\Phi\in H^1(\mathbb{T}_L^3)$ and a subsequence such that $\nabla_\varepsilon \Phi^\varepsilon \rightharpoonup \nabla \Phi$ weakly in $L^2(\mathbb{T}_L^3)$, and $\Phi$ solves
\begin{equation}
\label{eq:app_torus_limit}
-\,\Delta \Phi = \mathcal{M}_{w(a)}[\delta]
\quad\text{in }\mathcal{D}'(\mathbb{T}_L^3),
\end{equation}
where $\mathcal{M}_{w(a)}$ is the Fourier multiplier with symbol $w(|\mathbf{k}|,a)$.
\end{theorem}

\subsection*{A.4 Whole-space limit and \texorpdfstring{$H^1_{\mathrm{loc}}$}{H1-loc} convergence}
\noindent
We now pass from $\mathbb{T}_L^3$ to $\mathbb{R}^3$. Two standard regimes guarantee local compactness:

\smallskip
\emph{(A) Fixed box, $\varepsilon\to 0$ (periodic extension).} For fixed $L$, Theorem~\ref{thm:torus} holds on $\mathbb{T}_L^3$. Periodically extend $\Phi$ and $\delta$ to $\mathbb{R}^3$. For any compact $K\subset\mathbb{R}^3$, choose $L$ so that $K\subset \mathbb{T}_L^3$ and employ the torus convergence to extract an $H^1(K)$ limit. A diagonal argument over an exhausting sequence $\{K_n\}$ yields a subsequence converging weakly in $H^1_{\mathrm{loc}}(\mathbb{R}^3)$ to a distributional solution of
\begin{equation}
\label{eq:app_R3_limit_A}
-\,\Delta \Phi = \mathcal{M}_{w(a)}[\delta]\quad \text{on }\mathbb{R}^3.
\end{equation}

\smallskip
\emph{(B) Infinite-domain IR control.} Assume $\delta^\varepsilon\rightharpoonup \delta$ in $L^2(\mathbb{R}^3)$ with compact spatial support (uniformly in $\varepsilon$) or, more generally, that $\widehat{\delta^\varepsilon}$ obeys an IR moment bound
\begin{equation}
\label{eq:app_IR_bound}
\int_{|\mathbf{k}|<1}\! \frac{|\widehat{\delta^\varepsilon}(\mathbf{k})|^2}{|\mathbf{k}|^{2\eta}}\,d\mathbf{k}\;\le\;C_\eta
\quad\text{for some }\eta>2\alpha,
\end{equation}
uniformly in $\varepsilon$. Then the energy estimate \eqref{eq:energy-bound} carries to the continuum (replace sums by integrals), and since $\alpha<\tfrac12$, the $k\to0$ integrability of the gradient weight $|w|^2/|\mathbf{k}|^2\sim |\mathbf{k}|^{-2-2\alpha}$ in $d=3$ is ensured. Thus $\{\Phi^\varepsilon\}$ is uniformly bounded in $H^1(K)$ on every compact $K$, and a subsequence converges weakly in $H^1_{\mathrm{loc}}(\mathbb{R}^3)$ to a distributional solution of the ILG-modified Poisson equation.

\begin{theorem}[Whole-space $H^1_{\mathrm{loc}}$ limit]
\label{thm:R3}
Let $\{\delta^\varepsilon\}$ be uniformly $L^2$-bounded on $\mathbb{R}^3$ and satisfy either compact support or the IR moment bound above with $\eta>2\alpha$. Let $\Phi^\varepsilon$ solve \eqref{eq:disc-poisson} on $\mathbb{T}_L^3$ with $L\to\infty$ as $\varepsilon\to0$. Then, up to a subsequence,
\begin{equation}
\label{eq:app_R3_weak_conv}
\Phi^\varepsilon \rightharpoonup \Phi \quad\text{weakly in } H^1_{\mathrm{loc}}(\mathbb{R}^3),
\end{equation}
and $\Phi$ satisfies
\begin{equation}
\label{eq:app_R3_limit}
-\,\Delta \Phi = \mathcal{M}_{w(a)}[\delta]
\quad\text{in }\mathcal{D}'(\mathbb{R}^3).
\end{equation}
\end{theorem}

\subsection*{A.5 Multiplier class and domain of validity}
\noindent
\textbf{Multiplier class.} For fixed $a\in(0,1]$ the symbol $w(\cdot,a)$ satisfies
\begin{equation}
\label{eq:app_asymptotics}
w(k,a)=1+\mathcal{O}(k^{-\alpha})\quad(k\to\infty),\qquad
w(k,a)=\mathcal{O}(k^{-\alpha})\quad(k\to 0),
\end{equation}
with $\alpha\in(0,1/2)$. Hence $w(\cdot,a)$ is a bounded Fourier multiplier on $L^2$ (trivial at high $k$, and at low $k$ the growth is polynomial of order $<1$). For the \emph{gradient} operator, the effective symbol is $k\,w(k,a)/k^2 = w(k,a)/k$, whose square behaves as $k^{-2}$ at high $k$ and $k^{-2-2\alpha}$ at low $k$. In $d=3$, the latter is $k$-integrable near $0$ because $\alpha<1/2$. Consequently, the bilinear form
\begin{equation}
\label{eq:app_bilinear}
B_a(\delta,\psi)\;\equiv\;\int_{\mathbb{R}^3} \widehat{\delta}(\mathbf{k})\,\overline{w(|\mathbf{k}|,a)\,\widehat{\psi}(\mathbf{k})}\,d\mathbf{k}
\end{equation}
is well-defined whenever $\delta\in L^2$ and $\psi\in H^1$, and the energy identity
\begin{equation}
\label{eq:app_energy_identity}
\int \nabla \Phi\cdot\nabla\psi\,dx
= 4\pi G\,a^2\bar\rho_s\,B_a(\delta,\psi)
\end{equation}
makes sense and is continuous in both arguments.

\smallskip
\textbf{Domain of validity.} The proofs above require only:
\begin{itemize}
\item the RS exponent $\alpha\in(0,1/2)$ (satisfied by $\alpha=\tfrac12(1-\varphi^{-1})$), ensuring IR integrability of the gradient weight;
\item $L^2$-bounded data with either (i) fixed periodic box and zero mean or (ii) whole-space data with compact support or an IR moment bound of order $>2\alpha$.
\end{itemize}
Under these conditions the discrete-to-continuum limit holds, and the limiting potential $\Phi$ is characterized as the (unique up to an additive constant) $H^1_{\mathrm{loc}}$ solution of the ILG-modified Poisson equation in the sense of tempered distributions.

\subsection*{A.6 Remarks}
\noindent
(i) The background choice (EdS vs.\ LCDM) enters only through $a(\eta)$ in $w(k,a)$ and does not affect the functional-analytic class of the multiplier; all statements are uniform in $a\in(0,1]$. (ii) In PM/TreePM implementations the proof maps directly to code: replacing $k^{-2}$ by $\Lambda_\varepsilon^{-1}$ and multiplying by $w$ preserves stability; the uniform bound \eqref{eq:uniform-H1} is the discrete energy law that guarantees convergence under grid refinement. (iii) The RS constants $(\varphi,\alpha,\tau_0)$ are fixed by the RS$\to$Classical bridge and introduce no fit parameters anywhere in the analysis.

\section{Zero Buchert Backreaction at Linear Order}
\label{app:buchert}

The Buchert backreaction quantifies whether spatial inhomogeneities alter the evolution of the volume-averaged scale factor. In standard GR, linear perturbations yield zero backreaction; here we prove this remains true in ILG despite the scale-dependent growth rate $f(k,a)$.

The Buchert backreaction functional is defined by
\begin{equation}
Q_D = \frac{2}{3}\left(\langle\theta^2\rangle - \langle\theta\rangle^2\right) - 2\langle\sigma^2\rangle,
\end{equation}
where $\theta = \nabla\cdot\mathbf{v}$ is the expansion scalar and $\sigma_{\mu\nu}$ is the shear tensor. At linear order, $\theta(\mathbf{x},a) = -aH(a)f(k,a)\delta(\mathbf{x},a)$ and $\sigma^2 \propto a^2H^2 f^2(k,a) |\nabla\delta|^2$. Although $f(k,a)$ is scale-dependent in ILG, the $k$-dependence factors out identically in both the variance $\langle\theta^2\rangle$ and the mean-squared-shear $\langle\sigma^2\rangle$ terms. Explicitly:
\begin{equation}
\langle\theta^2\rangle = a^2H^2\int\frac{d^3k}{(2\pi)^3}\,f^2(k,a)\,P_\delta(k,a),
\end{equation}
\begin{equation}
\langle\sigma^2\rangle = \frac{1}{3}a^2H^2\int\frac{d^3k}{(2\pi)^3}\,f^2(k,a)\,P_\delta(k,a).
\end{equation}
The factor $f^2(k,a)$ appears in both integrals with the same kernel $P_\delta(k,a)$, yielding exact cancellation in $Q_D$:
\begin{equation}
Q_D = \frac{2}{3}a^2H^2\int\frac{d^3k}{(2\pi)^3}\,f^2(k,a)\,P_\delta(k,a) - \frac{2}{3}a^2H^2\int\frac{d^3k}{(2\pi)^3}\,f^2(k,a)\,P_\delta(k,a) = 0.
\end{equation}
Consequently, ILG generates no first-order backreaction: the cosmological scale factor $H(a)$ remains an externally chosen FRW input, and ILG cannot alter mean luminosity distances or shift the Hubble diagram. All observational signatures arise through inhomogeneity-level effects: lensing, growth, and scale-dependent clustering.

\section{Observable-mapping details (lensing kernels and distance moments)}
\label{app:observable_details}

This appendix collects derivations that are standard but lengthy, and which would otherwise distract from the main text's focus on qualitative signatures and falsification criteria.

\subsection*{C.1 Lensing kernels under Limber/Born}
\noindent
Under the Limber approximation ($\ell\gg 1$) and Born approximation, the convergence power spectrum for sources at comoving distance $\chi_s$ takes the form
\begin{equation}
\label{eq:Cl_kappa}
    C_\kappa(\ell;z_s)
    = \int_0^{\chi_s}\! d\chi\;
      \frac{W_L^2(\chi;\chi_s)}{\chi^2}\,
      P_\delta\!\left(k=\frac{\ell+1/2}{\chi},z(\chi)\right),
\end{equation}
where the lensing efficiency is
\begin{equation}
\label{eq:WL_def}
    W_L(\chi;\chi_s)
    = \frac{3H_0^2\,\Omega_{m0}}{2\,a(\chi)}
      \frac{\chi(\chi_s-\chi)}{\chi_s}\,
      w\!\left(k=\frac{\ell+1/2}{\chi},a(\chi)\right).
\end{equation}

\subsection*{C.2 Distance-moment relations}
\noindent
Lensing induces corrections to luminosity-distance moments through distortions of the photon bundle. To leading order,
\begin{equation}
\label{eq:DL_moments}
    \frac{\langle D_L\rangle}{\bar{D}_L}
    = 1 - \tfrac12 \langle|\gamma|^2\rangle + \mathcal{O}(\Phi^3),
    \qquad
    \mathrm{Var}(D_L)/\bar{D}_L^2
    \approx \langle\kappa^2\rangle.
\end{equation}

\newpage

\begin{thebibliography}{99}

\bibitem{PaperII}
J.~Washburn, M.~Simons, and E.~Allahyarov,
``Information-Limited Gravity II: Numerical Forecasts and Stage-IV Tests,''
(Paper II, in preparation).

\bibitem{Riess1998}
A.~G. Riess {\it et al.} (Supernova Search Team), 
``Observational Evidence from Supernovae for an Accelerating Universe and a Cosmological Constant,''
\emph{Astron. J.} \textbf{116}, 1009 (1998).

\bibitem{Perlmutter1999}
S.~Perlmutter {\it et al.} (Supernova Cosmology Project), 
``Measurements of $\Omega$ and $\Lambda$ from 42 High-Redshift Supernovae,''
\emph{Astrophys. J.} \textbf{517}, 565 (1999).

\bibitem{Planck2018}
N.~Aghanim {\it et al.} (Planck Collaboration), 
``Planck 2018 results. VI. Cosmological parameters,''
\emph{Astron. Astrophys.} \textbf{641}, A6 (2020).

\bibitem{BOSS2017}
S.~Alam {\it et al.} (BOSS Collaboration), 
``The clustering of galaxies in the completed SDSS-III Baryon Oscillation Spectroscopic Survey: cosmological analysis of the DR12 galaxy sample,''
\emph{Mon. Not. Roy. Astron. Soc.} \textbf{470}, 2617 (2017).

\bibitem{DES2022}
T.~M.~C. Abbott {\it et al.} (DES Collaboration), 
``Dark Energy Survey Year 3 results: Cosmological constraints from galaxy clustering and weak lensing,''
\emph{Phys. Rev. D} \textbf{105}, 023520 (2022).

\bibitem{Riess2022}
A.~G. Riess {\it et al.}, 
``A Comprehensive Measurement of the Local Value of the Hubble Constant with 1 km/s/Mpc Uncertainty from the Hubble Space Telescope and the SH0ES Team,''
\emph{Astrophys. J. Lett.} \textbf{934}, L7 (2022).

\bibitem{KiDS2020}
M.~Asgari {\it et al.} (KiDS Collaboration), 
``KiDS-1000 Cosmology: Cosmic shear constraints and comparison between two point statistics,''
\emph{Astron. Astrophys.} \textbf{645}, A104 (2021).

\bibitem{Planck2016ISW}
P.~A.~R. Ade {\it et al.} (Planck Collaboration), 
``Planck 2015 results. XXI. The integrated Sachs-Wolfe effect,''
\emph{Astron. Astrophys.} \textbf{594}, A21 (2016).

\bibitem{RSFoundations}
J.~Washburn,
``Recognition Science: Foundations of an Information-Theoretic Framework for Gravitational Phenomenology,''
(Recognition Physics Institute, Technical Report, 2024).

\bibitem{RSClassicalBridge}
J.~Washburn and M.~Simons,
``Recognition Science and the Classical Limit: Deriving Cosmological Kernel Parameters,''
(Recognition Physics Institute, Technical Report, 2024).

\bibitem{Sotiriou2010}
T.~P. Sotiriou and V.~Faraoni, 
``f(R) theories of gravity,''
\emph{Rev. Mod. Phys.} \textbf{82}, 451 (2010).

\bibitem{DeFelice2010}
A.~De Felice and S.~Tsujikawa,
``f(R) Theories,''
\emph{Living Rev. Rel.} \textbf{13}, 3 (2010).

\bibitem{Horndeski1974}
G.~W. Horndeski, 
``Second-order scalar-tensor field equations in a four-dimensional space,''
\emph{Int. J. Theor. Phys.} \textbf{10}, 363 (1974).

\bibitem{Deffayet2011}
C.~Deffayet, X.~Gao, D.~A. Steer, and G.~Zahariade, 
``From k-essence to generalised Galileons,''
\emph{Phys. Rev. D} \textbf{84}, 064039 (2011).

\bibitem{Kobayashi2019}
T.~Kobayashi,
``Horndeski theory and beyond: a review,''
\emph{Rept. Prog. Phys.} \textbf{82}, 086901 (2019).

\bibitem{Dvali2000}
G.~R. Dvali, G.~Gabadadze, and M.~Porrati, 
``4D gravity on a brane in 5D Minkowski space,''
\emph{Phys. Lett. B} \textbf{485}, 208 (2000).

\bibitem{Deffayet2002}
C.~Deffayet,
``Cosmology on a brane in Minkowski bulk,''
\emph{Phys. Lett. B} \textbf{502}, 199 (2001).

\bibitem{deRham2011}
C.~de Rham, G.~Gabadadze, and A.~J. Tolley, 
``Resummation of Massive Gravity,''
\emph{Phys. Rev. Lett.} \textbf{106}, 231101 (2011).

\bibitem{deRham2014}
C.~de Rham,
``Massive Gravity,''
\emph{Living Rev. Rel.} \textbf{17}, 7 (2014).

\bibitem{Deser2007}
S.~Deser and R.~P. Woodard, 
``Nonlocal Cosmology,''
\emph{Phys. Rev. Lett.} \textbf{99}, 111301 (2007).

\bibitem{Maggiore2014}
M.~Maggiore and M.~Mancarella, 
``Nonlocal gravity and dark energy,''
\emph{Phys. Rev. D} \textbf{90}, 023005 (2014).

\bibitem{Belgacem2018}
E.~Belgacem, Y.~Dirian, S.~Foffa, and M.~Maggiore,
``Nonlocal gravity. Conceptual aspects and cosmological predictions,''
\emph{JCAP} \textbf{1803}, 002 (2018).

\bibitem{Popper1959}
K.~Popper, 
\emph{The Logic of Scientific Discovery},
(Hutchinson, London, 1959).

\bibitem{Bellini2014}
E.~Bellini and I.~Sawicki,
``Maximal freedom at minimum cost: linear large-scale structure in general modifications of gravity,''
\emph{JCAP} \textbf{1407}, 050 (2014).

\bibitem{Gleyzes2015}
J.~Gleyzes, D.~Langlois, F.~Piazza, and F.~Vernizzi,
``Healthy theories beyond Horndeski,''
\emph{Phys. Rev. Lett.} \textbf{114}, 211101 (2015).

\bibitem{Noller2020}
J.~Noller,
``Cosmological parameter constraints for Horndeski scalar-tensor gravity,''
\emph{Phys. Rev. D} \textbf{101}, 063524 (2020).

\bibitem{Baumann2012}
D.~Baumann, A.~Nicolis, L.~Senatore, and M.~Zaldarriaga, 
``Cosmological Non-Linearities as an Effective Fluid,''
\emph{JCAP} \textbf{1207}, 051 (2012).

\bibitem{Carrasco2012}
J.~J.~M. Carrasco, M.~P. Hertzberg, and L.~Senatore, 
``The Effective Field Theory of Cosmological Large Scale Structures,''
\emph{JHEP} \textbf{1209}, 082 (2012).

\bibitem{Verlinde2011}
E.~P. Verlinde, 
``On the Origin of Gravity and the Laws of Newton,''
\emph{JHEP} \textbf{1104}, 029 (2011).

\bibitem{Verlinde2017}
E.~P. Verlinde, 
``Emergent Gravity and the Dark Universe,''
\emph{SciPost Phys.} \textbf{2}, 016 (2017).

\bibitem{Jacobson1995}
T.~Jacobson, 
``Thermodynamics of Spacetime: The Einstein Equation of State,''
\emph{Phys. Rev. Lett.} \textbf{75}, 1260 (1995).

\bibitem{LIGOVirgo2017}
B.~P. Abbott {\it et al.} (LIGO Scientific and Virgo Collaborations),
``GW170817: Observation of Gravitational Waves from a Binary Neutron Star Inspiral,''
\emph{Phys. Rev. Lett.} \textbf{119}, 161101 (2017).

\bibitem{Khoury2004}
J.~Khoury and A.~Weltman,
``Chameleon Fields: Awaiting Surprises for Tests of Gravity in Space,''
\emph{Phys. Rev. Lett.} \textbf{93}, 171104 (2004).

\bibitem{Hinterbichler2010}
K.~Hinterbichler and J.~Khoury,
``Symmetron Fields: Screening Long-Range Forces Through Local Symmetry Restoration,''
\emph{Phys. Rev. Lett.} \textbf{104}, 231301 (2010).

\bibitem{Vainshtein1972}
A.~I. Vainshtein,
``To the problem of nonvanishing gravitation mass,''
\emph{Phys. Lett. B} \textbf{39}, 393 (1972).

\bibitem{Babichev2013}
E.~Babichev and C.~Deffayet,
``An introduction to the Vainshtein mechanism,''
\emph{Class. Quant. Grav.} \textbf{30}, 184001 (2013).

\bibitem{Will2014}
C.~M. Will,
``The Confrontation between General Relativity and Experiment,''
\emph{Living Rev. Rel.} \textbf{17}, 4 (2014).

\bibitem{Burrage2018}
C.~Burrage and J.~Sakstein,
``Tests of Chameleon Gravity,''
\emph{Living Rev. Rel.} \textbf{21}, 1 (2018).

\bibitem{Bekenstein2004}
J.~D. Bekenstein,
``Relativistic gravitation theory for the modified Newtonian dynamics paradigm,''
\emph{Phys. Rev. D} \textbf{70}, 083509 (2004).

\bibitem{Bryan2014}
G.~L. Bryan {\it et al.},
``ENZO: An Adaptive Mesh Refinement Code for Astrophysics,''
\emph{Astrophys. J. Suppl.} \textbf{211}, 19 (2014).

\bibitem{Teyssier2002}
R.~Teyssier,
``Cosmological hydrodynamics with adaptive mesh refinement: a new high resolution code called RAMSES,''
\emph{Astron. Astrophys.} \textbf{385}, 337 (2002).

\bibitem{Springel2021}
V.~Springel {\it et al.},
``Simulating cosmic structure formation with the GADGET-4 code,''
\emph{Mon. Not. Roy. Astron. Soc.} \textbf{506}, 2871 (2021).

\end{thebibliography}

\end{document}



