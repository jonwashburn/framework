\documentclass[11pt]{article}

\usepackage{amsmath,amssymb,amsthm}

\newtheorem{definition}{Definition}
\newtheorem{theorem}{Theorem}
\newtheorem{lemma}{Lemma}
\newtheorem{corollary}{Corollary}

\title{The Law of Existence: Derived Meta-Principle and Ontology Predicates}
\author{
Jonathan Washburn\\
Recognition Science\\
Recognition Physics Institute\\
Austin, Texas, USA\\
\texttt{jon@recognitionphysics.org}
}
\date{}

\begin{document}
\maketitle

\begin{abstract}
Recognition Science (RS) treats \emph{cost} as primitive and uses it to define ontology before any physics is introduced.
This paper isolates a single meta-structure: a uniquely forced cost functional $J$ on positive ratios, its interpretation as a
\emph{defect} (existence cost), the resulting \emph{Law of Existence} (a unique defect-zero element), and a derived
meta-principle: $\lim_{x\to 0^+} J(x)=\infty$, i.e.\ ``nothing cannot exist'' because ``nothing'' carries infinite cost.
On top of this cost foundation we define RS-native predicates for \emph{exists}, \emph{true}, and \emph{real}:
existence is defect-zero, truth is stability under recognition iteration, and reality is existence plus membership in a discrete
RS skeleton (here represented by a $\varphi$-generated ladder).
\end{abstract}

\section*{1.\ Scope}
This paper is deliberately pre-physical.
It does not introduce spacetime, energy, or any empirical calibration.
The goal is only to state, in RS terms, what it means for something to \emph{exist}, for a statement to be \emph{true}, and for
something to be \emph{real}.

\section*{2.\ Cost-first foundation and the forced form of $J$}
RS begins with a cost functional $J$ over positive ratios.
The key move in this paper is that the meta-principle (MP) is \emph{not} assumed as a primitive slogan; it is recovered as a
theorem about the boundary behavior of $J$.

\subsection*{2.1\ Axioms on $J$ (log-coordinates)}
Let $J:\mathbb{R}_{>0}\to\mathbb{R}$ be a cost functional.
Assume:

\begin{definition}[Cost axioms]
Define $j:\mathbb{R}\to\mathbb{R}$ by $j(t):=J(e^t)$.
We assume:
\begin{enumerate}
  \item (Normalization) $J(1)=0$.
  \item (Recognition composition law) For all $x,y>0$,
  \[
    J(xy)+J(x/y)=2J(x)J(y)+2J(x)+2J(y).
  \]
  \item (Calibration) $j$ is twice differentiable at $0$ and $j''(0)=1$.
\end{enumerate}
\end{definition}

\subsection*{2.2\ Deriving the canonical $J$}
\begin{theorem}[Canonical form of $J$]
Under the cost axioms above (and sufficient differentiability to justify the steps below),
\[
J(x)=\frac12\left(x+\frac1x\right)-1\qquad (x>0).
\]
\end{theorem}

\begin{proof}
Define $F:\mathbb{R}\to\mathbb{R}$ by $F(t):=1+J(e^t)=1+j(t)$.
Substitute $x=e^t$ and $y=e^s$ into the composition law and rewrite in terms of $F$:
\[
F(t+s)+F(t-s)=2F(t)F(s).
\]
Normalization gives $F(0)=1$.
Setting $t=0$ yields $F(s)+F(-s)=2F(0)F(s)=2F(s)$, hence $F$ is even and $F'(0)=0$.
Differentiate the functional equation twice with respect to $s$ and then set $s=0$:
\[
F''(t)+F''(t)=2F(t)F''(0)\quad\Rightarrow\quad F''(t)=F''(0)\,F(t).
\]
Since $F(t)=1+j(t)$, calibration $j''(0)=1$ gives $F''(0)=1$, hence $F''(t)=F(t)$.
With $F(0)=1$ and $F'(0)=0$, the unique solution is $F(t)=\cosh(t)$.
Therefore $J(e^t)=F(t)-1=\cosh(t)-1=\tfrac12(e^t+e^{-t})-1$, i.e.
\[
J(x)=\frac12\left(x+\frac1x\right)-1.
\]
\end{proof}

\section*{3.\ Defect and the Law of Existence}
RS uses the cost functional itself as an \emph{existence defect}.

\begin{definition}[Defect]
Define the defect function $\operatorname{defect}:\mathbb{R}_{>0}\to\mathbb{R}_{\ge 0}$ by
\[
\operatorname{defect}(x):=J(x)=\frac12\left(x+\frac1x\right)-1.
\]
\end{definition}

\begin{lemma}[Nonnegativity and unique zero]
For all $x>0$, $\operatorname{defect}(x)\ge 0$, and $\operatorname{defect}(x)=0$ iff $x=1$.
\end{lemma}

\begin{proof}
Compute
\[
\operatorname{defect}(x)=\frac12\left(x+\frac1x-2\right)
=\frac12\cdot\frac{(x-1)^2}{x}
=\frac{(x-1)^2}{2x}\ge 0.
\]
Equality holds iff $(x-1)^2=0$, i.e.\ $x=1$.
\end{proof}

\begin{theorem}[Law of Existence (scalar form)]
In the scalar cost model, existence is equivalent to defect-zero:
\[
x\ \text{exists}\ \Longleftrightarrow\ \operatorname{defect}(x)=0\ \Longleftrightarrow\ x=1.
\]
\end{theorem}

\begin{proof}
Immediate from the previous lemma.
\end{proof}

\section*{4.\ Derived Meta-Principle: ``nothing cannot exist''}
The RS meta-principle is expressed as a limit theorem about the cost of the boundary object $x\to 0^+$.

\begin{theorem}[Derived MP in cost form]
\[
\lim_{x\to 0^+} J(x)=\infty.
\]
\end{theorem}

\begin{proof}
From $J(x)=\tfrac12(x+x^{-1})-1$, the term $x^{-1}$ diverges as $x\to 0^+$, hence the limit is $\infty$.
\end{proof}

\begin{definition}[``Nothing'' as a cost boundary]
We use the symbol $0$ as shorthand for the \emph{boundary} $x\to 0^+$ (not an element of $\mathbb{R}_{>0}$).
In RS cost language, ``nothing'' means ``the vanishing-ratio boundary''.
\end{definition}

\begin{corollary}[Nothing cannot exist]
``Nothing cannot exist'' is the theorem that the boundary $x\to 0^+$ is excluded by infinite defect:
\[
\text{(nothing)}\ \not\models\ \operatorname{defect}=0,
\qquad\text{because}\qquad
\lim_{x\to 0^+}\operatorname{defect}(x)=\infty.
\]
\end{corollary}

\begin{corollary}[Existence is forced (in the scalar model)]
There exists (and is unique) a defect-zero element, namely $x=1$.
Thus existence is witnessed internally by the cost structure:
\[
\exists!\,x>0\ \text{such that}\ \operatorname{defect}(x)=0.
\]
\end{corollary}

\begin{proof}
We have $\operatorname{defect}(1)=0$ and the zero is unique by the Law of Existence.
\end{proof}

\section*{5.\ Ontology predicates in RS}
The prior sections motivate three RS-native predicates: \emph{exists}, \emph{true}, and \emph{real}.

\subsection*{5.1\ Existence}
\begin{definition}[RS existence predicate]
Define
\[
\mathrm{RSExists}(x)\ :\Longleftrightarrow\ (0<x)\ \wedge\ \operatorname{defect}(x)=0.
\]
\end{definition}

\begin{theorem}[Uniqueness of RS existence (scalar form)]
For all $x>0$, $\mathrm{RSExists}(x)$ iff $x=1$.
In particular, there exists exactly one RS-existent scalar element.
\end{theorem}

\begin{proof}
By the Law of Existence, $\operatorname{defect}(x)=0$ iff $x=1$.
\end{proof}

\subsection*{5.2\ Truth as stability under recognition iteration}
Truth in RS is not defined as ``correspondence to a hidden thing-in-itself.''
Instead, truth is the stability of a statement under repeated recognition.

\begin{definition}[Recognition iteration (abstract)]
Let $\mathcal{C}$ be a nonempty configuration space and let $\mathcal{R}:\mathcal{C}\to\mathcal{C}$ be an iteration operator
representing one step of recognition-driven updating.
Write $\mathcal{R}^n$ for the $n$-fold iterate.
\end{definition}

\begin{definition}[Stabilization of a predicate]
Let $P:\mathcal{C}\to\{\mathrm{false},\mathrm{true}\}$ be a predicate and let $c\in\mathcal{C}$.
We say $P$ \emph{stabilizes under recognition iteration at $c$} if there exists $N\in\mathbb{N}$ such that for all $n\ge N$,
\[
P(\mathcal{R}^n(c))=P(\mathcal{R}^N(c)).
\]
\end{definition}

\begin{definition}[RS truth predicate]
Let $c_\star$ denote the RS-existent configuration (in the scalar model, $c_\star=1$).
Define
\[
\mathrm{RSTrue}(P)\ :\Longleftrightarrow\ P(c_\star)\ \wedge\ \bigl(P\ \text{stabilizes under recognition iteration at}\ c_\star\bigr).
\]
\end{definition}

This is a definition, not a claim that every predicate stabilizes.
The point is that RS identifies \emph{truth} with what survives iterated recognition, rather than what can be asserted once.

\subsection*{5.3\ Reality as existence plus discreteness}
RS distinguishes ``exists'' from ``real'' by adding a discreteness condition.
This makes ``real'' a stronger predicate than mere defect-zero.

\begin{definition}[Discrete RS skeleton (one canonical choice)]
Let $\varphi:=\frac{1+\sqrt{5}}{2}$ be the unique positive solution of $\varphi^2=\varphi+1$.
Define a discrete multiplicative skeleton
\[
\mathcal{D}:=\{\varphi^n : n\in\mathbb{Z}\}\subset\mathbb{R}_{>0}.
\]
This is a simple stand-in for ``discrete (algebraic in $\varphi$).''
\end{definition}

\begin{definition}[RS reality predicate]
Define
\[
\mathrm{RSReal}(x)\ :\Longleftrightarrow\ \mathrm{RSExists}(x)\ \wedge\ (x\in\mathcal{D}).
\]
\end{definition}

In the scalar existence model, $\mathrm{RSExists}(x)$ already forces $x=1=\varphi^0$, so $\mathrm{RSReal}(x)$ holds exactly at $x=1$.
The purpose of $\mathrm{RSReal}$ is not to add content to the scalar theorem, but to establish a predicate that can later be applied
to richer configuration spaces where defect-zero does not collapse to a single scalar and where the discrete skeleton becomes a
substantive constraint.

\section*{6.\ Summary}
The ontological core can be stated without physics.
Existence is defect-zero: $\mathrm{RSExists}(x)\iff \operatorname{defect}(x)=0$.
The Law of Existence is the uniqueness of the defect-zero element (scalar form: $x=1$).
The meta-principle ``nothing cannot exist'' is derived as $\lim_{x\to 0^+}J(x)=\infty$.
Truth is stability of propositions under recognition iteration: $\mathrm{RSTrue}(P)$.
Reality is existence plus discreteness: $\mathrm{RSReal}(x)$.

\end{document}
