\documentclass[11pt]{article}
\usepackage{amsmath,amssymb,amsthm,mathtools}
\usepackage[margin=1in]{geometry}

\newtheorem{definition}{Definition}
\newtheorem{lemma}{Lemma}
\newtheorem{proposition}{Proposition}
\newtheorem{theorem}{Theorem}
\theoremstyle{remark}
\newtheorem{remark}{Remark}

\title{
\large \bf Why the $\mu\to\tau$ ``local cellwise'' uniqueness claim remains conditional and non-resolving}
\author{}
\date{}

\begin{document}
\maketitle

\subsection*{1. What the new response actually proves (and what it does not)}

The note in the response defines an admissible mechanism class $M=\{M_k:k\in\{0,1,2,3\}\}$ on the 3-cube ledger, and defines a rule 
\begin{equation}
    g(M_k)=\sum_{m\in \text{Mediators}(M_k)}\frac{1}{|\text{Anchors}(m)|}
      =\frac{\#(\text{$k$-cells})}{\#(\text{vertices in a $k$-cell})},
\end{equation}
and computes
\begin{equation}
   g(M_0)=8,\quad g(M_1)=6,\quad g(M_2)=\frac32,\quad g(M_3)=\frac18, 
\end{equation}
and concludes that $3/2$ is ``unique'' in this class and therefore face mediation is uniquely selected.

{\bf Yes, the arithmetic and injectivity on a 4-element set are correct}. But none of that establishes what my critique demanded:\\
\emph{The physical framework should uniquely forces (i) this exact mechanism class, (ii) this exact rule $g$,
and (iii) the identification of the $\mu\to\tau$ correction with $g(M_2)$ rather than any alternative local rule.}

\subsection*{2. Key point: uniqueness inside a hand-chosen class is not a resolution}

A genuine resolution of the ``many alternatives'' / ``hand-specific'' objection must supply:
\begin{enumerate}
\item a \emph{framework-forced} admissible mechanism class $\mathcal{M}$ (not merely one convenient choice),
\item a \emph{framework-forced} rule $g:\mathcal{M}\to\mathbb{R}$ (not merely one convenient choice),
\item and a proof that the observed coefficient is uniquely produced in $\mathcal{M}$ under $g$.
\end{enumerate}
If (1) or (2) is itself discretionary, the original objection is merely relocated into the choice of $\mathcal{M}$ or $g$.

\begin{remark}
The response explicitly concedes that uniqueness claims are conditional and depend on the
``right admissible class'' and whether the framework commits to it. This is exactly where
the new note fails: it does not prove the framework commits to \emph{this} class and \emph{this} $g$.
\end{remark}

\subsection*{3. Failure 1: The response admits the core ingredient is a modeling hypothesis}

The note argued that ``inverse measure'' was already present in the $e\to\mu$ step and that the
$\mu\to\tau$ rule extends it discretely. However it explicitly states:
\begin{center}
\emph{The modeling hypothesis is that the RS tick dynamics uses the integrated-vs-differential split.}
\end{center}

That admission is decisive: if the split is a modeling hypothesis rather than a proved consequence, then
the induced ``inverse measure'' structure cannot be used to claim a uniquely derived law.

\underline{Non-negotiable point.}
You cannot answer ``your coefficient looks fit'' by saying:
``once we assume the split, the inverse measure rule is tautological.''
That is equivalent to:
\[
\text{assumption tailored to enforce a normalization} \;\Rightarrow\; normalization.
\]
This does not remove underdetermination; it shifts it into the assumed split.

\subsection*{3. Failure 2: The admissible mechanism class $M$ is artificially tiny and not shown to be forced}

The class $M$ contains exactly four mechanisms $M_k$ indexed by cell-dimension.
Injectivity on a 4-element set is mathematically trivial once you compute four distinct values.

The real question is not:
\[
\text{Is $g$ injective on this four-element set?}
\]
The real question is:
\[
\text{Why is this four-element set the \emph{entire} admissible mechanism class forced by RS?}
\]

{\bf \underline{Explicit counterexample:}}\\

The response excludes $E/V_{\text{cube}}=12/8$ by declaring it ``cross-level'' and therefore inadmissible.

But consider the following alternative mechanism class, which is \emph{still local, still cellwise, still cubic-ledger based}:\\

\underline{Alternative admissible class $\mathcal{M}'$:}
Fix the cube ledger. A mechanism is specified by:
\begin{itemize}
\item a mediator cell-dimension $k\in\{0,1,2,3\}$, and
\item an anchor cell-dimension $j\in\{0,1,\dots,k\}$,
\end{itemize}
with local rule:
\[
g'(M_{k,j}) := \frac{\#(\text{$k$-cells})}{\#(\text{$j$-cells in a single $k$-cell})}.
\]


This is exactly the same style of ``local per mediator'' normalization as in the response, except it does
\emph{not} hard-code the choice $j=0$ (vertices) as anchors.

Now compute in the cube:
\[
g'(M_{2,0})=\frac{6}{4}=\frac32,\qquad g'(M_{2,1})=\frac{6}{4}=\frac32.
\]
Thus the same value $3/2$ arises from \emph{two distinct local mechanisms} inside an equally natural class.


Thus, the value $3/2$ is not uniquely tied to ``vertex anchors'' unless one \emph{stipulates} $j=0$.
Therefore the response has not removed arbitrariness; it has encoded it in a type-choice.

\subsection*{4. Failure 3: The ``anchors are vertices'' claim is not derived, and in $D=3$ it does not disambiguate the number}

The response claim that the integer $4$ is unambiguous because anchors are 0-cells (vertices), not 1-cells (edges).
But in the actual physical dimension $D=3$, a square face has 4 vertices and 4 edges.
So the choice does not actually fix the denominator numerically; it only renames it.

The response attempts to justify the choice using $D>3$ behavior (counts differ in higher dimensions).
But if the framework independently asserts that $D=3$ is the realized world, then $D>3$ behavior is:
\begin{enumerate}
\item not empirically testable, and
\item not logically available as a uniqueness discriminator unless the theory proves cross-$D$ covariance
is required for the intermediate mechanism objects.
\end{enumerate}


Therefore, in $D=3$, choosing ``vertex anchors'' over ``edge anchors'' does not remove the numerical ambiguity,
and invoking higher-dimensional distinctions is irrelevant unless the framework proves those distinctions
are mandatory constraints on $D=3$ formulas.

\subsection*{5. Failure 4: ``$W$ is 2D, therefore $k=2$'' is explicitly not a theorem}

The response lists checklist item (2) as:
\begin{quote}
``A theorem that the tau step must be face-mediated.'' \\
Then immediately: ``This is a modeling claim: the $\mu\to\tau$ correction uses $W=17$ wallpaper groups, a 2D constant, so the mediator must be 2D.''
\end{quote}

This is not a theorem. It is an interpretive association.
Moreover it is dangerously close to circularity:
\begin{enumerate}
\item The model inserts $W$ into the tau correction.
\item Then it infers the mediator dimension from the fact $W$ is 2D.
\end{enumerate}
But the point at issue is \emph{why $W$ should enter in that way at all}.
Using the presence of $W$ to justify the mediator dimension does not derive the presence of $W$.

\subsection*{6. Failure 5: The proposed falsifier is not logically clean because other hypothesis terms exist}

The response claims that replacing face mediation ($k=2$) by edge mediation ($k=1$) changes the exponent by
\[
(6-\tfrac32)\alpha \approx 4.5\alpha \approx 0.0328,
\]
which would miss the lepton table unless new tuning knobs are introduced.

But this assumes the rest of the pipeline is frozen as proved structure.
In the mass paper, the lepton chain break and step formulas are explicitly labeled as \emph{hypotheses}:
the break $\delta_e$ and step formulas $S_{e\to\mu}$ and $S_{\mu\to\tau}$ are not Proved.

Therefore, the alleged falsifier is conditional on \emph{not} changing other hypothesis components.
Using the ``no tuning knobs'' contract to block compensating changes is not a falsifier; it is
treating the contract itself as an additional axiom.
But the contract is exactly what is under dispute when the critique is ``these look hand-picked.''

\subsection*{7. What this response actually achieves!}

The response achieves only the following conditional statement:
\begin{quote}
If we \emph{define} the admissible class to be $M=\{M_k\}_{k=0}^3$,
\emph{define} anchors to be vertices, and \emph{define} the local coefficient rule to be
$g(M_k)=\#(k\text{-cells})/\#(\text{vertices in a $k$-cell})$,
then the value $3/2$ appears only at $k=2$.
\end{quote}

That is a correct lemma about a chosen class.

It is not, by itself, a resolution of the underdetermination criticism,
because it does not prove that the framework forces these definitions rather than selecting them.

\subsection*{8. Non-negotiable closure conditions (what must be proven to actually end the debate)}

To close the loophole rather than relocate it, the following must be proved (not asserted):
\begin{enumerate}
\item A theorem from core RS axioms that the $\mu\to\tau$ correction coefficient is computed by a local cellwise rule of the specific form $g$ (and not any other local cellwise functional).
\item A theorem that anchors for the relevant normalization must be 0-cells (and that using 1-cells is forbidden by axiom, not taste).
\item A theorem that the step is mediated by $k=2$ cells that does not rely on the already-chosen appearance of $W$ in the step (avoid circular justification).
\item A theorem that the $W$-coupling itself (the $+W$ part of $C_\tau=W+D/2$) is forced, including its sign and linearity, rather than chosen because it reproduces $18.5$.
\end{enumerate}

Until those exist, the mechanism note is best described as:
\begin{center}
\emph{a clarifying hypothesis refinement with a Lean-checked arithmetic lemma},
\end{center}
not a derivation that upgrades the tau-step coefficient to a uniquely forced law.

\end{document}
