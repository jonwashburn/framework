\documentclass[11pt]{article}

\usepackage{amsmath,amssymb,amsthm,mathtools}
\usepackage{geometry}
\usepackage{microtype}
\usepackage[hidelinks]{hyperref}
\geometry{margin=1in}

\theoremstyle{plain}
\newtheorem{theorem}{Theorem}[section]
\newtheorem{lemma}[theorem]{Lemma}
\newtheorem{proposition}[theorem]{Proposition}
\newtheorem{corollary}[theorem]{Corollary}

\theoremstyle{definition}
\newtheorem{definition}[theorem]{Definition}
\newtheorem{remark}[theorem]{Remark}

\title{\textbf{Meaning is Forced:}\\
\large A Certificate Bridge from Closure to Semantics in the Universal Light Language}
\author{Jonathan Washburn\\Recognition Physics Institute}
\date{\today}

\begin{document}
\maketitle

\begin{abstract}
Recognition Science (RS) and its closure certificates constrain admissible structure, invariants, and canonical representations, but (as a common caveat notes) \emph{closure alone does not yet imply meaning}.
This paper isolates a minimal bridge that \emph{does} force a meaning object: once (i) reference is defined internally as minimization of a canonical reciprocal mismatch cost \(J\) and (ii) representational non-identifiabilities are quotiented out as gauge, a closed RS-compatible system determines a unique meaning \emph{up to gauge}.

Concretely, we formalize a token-layer meaning map for the Universal Light Language (ULL) in which an eight-beat window is assigned a forced WToken signature: the \textbf{mode family} is selected by a frozen classifier, the \textbf{\(\varphi\)-level} is forced as an argmin over the RS \(\varphi\)-lattice under the canonical cost, and the remaining \(\tau\)-offset freedom is shown to carry no semantic degree of freedom beyond global phase---so meaning naturally lives in the quotient space ``\(\tau\) modulo gauge''.
We provide a certificate-style statement: the resulting meaning map is definitionally invariant under gauge and is forced by the stated axioms, without introducing interpreter primitives.
The core forcing lemmas are stated and proved in classical mathematical terms; they are designed to be self-contained and intended to be cited by the main ULL periodic-table paper.
\end{abstract}

\tableofcontents

\section{Introduction}
\label{sec:intro}

Recognition Science (RS) has a distinctive kind of ambition: to derive a closed, zero-parameter framework in which admissible structure is \emph{forced} by explicit axioms rather than chosen by design~\cite{RG2026}.
In that setting it is natural to pursue a corresponding goal for semantics: a universal language of meaning in which tokens, grammar, and denotations are not conventions but consequences of the same closure logic.
The Universal Light Language (ULL) periodic-table manuscript aims to give such a system-level construction.

However, as a common (and fair) peer-review caveat notes, even very strong closure results typically imply only \emph{structure, invariants, and canonical representations}---not meaning itself.
One can have a closed dynamical system with conserved quantities and unique normal forms while still lacking an internal reason to call any output ``about'' anything.
In short: \emph{closure is not yet semantics}.
This paper isolates the missing link and packages it as a small, audit-friendly theorem that the ULL paper can cite.

\subsection{The gap: why closure does not yet imply meaning}
The obstruction is not mysterious: ``meaning'' is an additional layer of \emph{selection} and \emph{identifiability}.
Even if a framework uniquely determines a set of legal trajectories, it may still leave open:
\begin{itemize}
  \item whether any trajectory should be interpreted as a \emph{denotation} (rather than just a canonical representative), and
  \item which differences between representatives are \emph{semantically irrelevant} (and therefore should be quotiented out).
\end{itemize}
These are not mere philosophical add-ons; they are mathematical choices that must be made explicit to be reviewed.

\subsection{Bridge principle: meaning as forced argmin, modulo gauge}
Our bridge has two ingredients, each already natural in the RS/ULL ecosystem.

\paragraph{(1) A universal selection rule (argmin).}
We treat meaning as an \emph{internal optimization statement}: a configuration means the object (or signature) that minimizes a fixed mismatch cost.
In the ratio-induced setting relevant to RS, the mismatch penalty is canonical (reciprocal, strictly convex, inversion-symmetric), yielding the familiar normalized form
\[
J(x)=\tfrac12(x+x^{-1})-1.
\]
This canonical form is forced (up to a scale exponent that can be absorbed into the scale maps) by a functional-equation characterization~\cite{RCC2026}.
With such a \(J\) fixed, ``meaning'' is not an interpretive primitive; it is the output of a forced minimization rule.

\paragraph{(2) A forced identifiability quotient (gauge).}
Optimization alone does not yield a \emph{unique} meaning object unless one also specifies which degrees of freedom are observationally/semantically identifiable.
In physics, gauge degrees of freedom are not physical data; they are quotient directions.
In the ULL token model, an analogous phenomenon appears: the \(\tau\)-offset variant in the self-conjugate family is not distinguishable beyond a global phase factor, so \(\tau\) cannot be semantic data for any construction invariant under phase.
Accordingly, meaning must live in the quotient ``\(\tau\) modulo gauge''.

\subsection{What is proved (and what is not)}
This paper is intentionally narrow: it proves the minimal forcing result needed to make ``meaning is forced'' a literal theorem statement rather than a slogan.

\paragraph{What is proved.}
At the token/signature layer for neutralized eight-beat windows, we construct a meaning map whose output is forced \emph{up to gauge}:
\begin{itemize}
  \item a deterministic classifier forces a \emph{mode family} when it returns an exact class;
  \item for any extracted radius \(r>0\), the \(\varphi\)-level is forced as an argmin over the finite \(\varphi\)-ladder under the canonical cost \(J\);
  \item \(\tau\) contributes no semantic degree of freedom beyond phase gauge, so the meaning object is a quotient class in ``\(\tau\) modulo gauge''.
\end{itemize}
These claims are proved in standard mathematical terms and exposed as a small set of lemmas suitable for citation and audit.

\paragraph{What is \emph{not} proved.}
We do \emph{not} prove, in this paper, the full end-to-end ULL story:
\begin{itemize}
  \item we do not prove completeness/minimality of the full motif grammar or LNAL normal-form semantics for arbitrary signals,
  \item we do not prove a global ``unique perfect language'' theorem for all semantics layers,
  \item and we do not validate empirically that the classifier and radius extraction are correct in every domain.
\end{itemize}
Those are separate (and larger) projects; here we isolate the precise forcing bridge that converts closure plus canonical-cost semantics into a meaning object \emph{within the modeled layer}.

\subsection{How this supports the ULL periodic-table manuscript}
The ULL periodic-table manuscript can cite this paper for the specific claim that ``meaning is forced'' at the token/signature layer:
the meaning map is not an arbitrary convention, but the consequence of (i) a fixed mismatch geometry and (ii) a forced gauge quotient.
This narrows the remaining peer-review burden for ULL: what remains is to justify the system-level pipeline (windowing, invariants, grammar, empirics), not the logical possibility of meaning within a closed RS-compatible model.

\subsection{Roadmap}
Section~\ref{sec:background} records the prerequisites from RS, canonical mismatch costs, and meaning-as-argmin semantics.
Section~\ref{sec:interface} states the abstract forcing interface in certificate form.
Section~\ref{sec:main} proves the core theorem ``meaning is forced up to gauge'' for the token/signature layer.
Section~\ref{sec:ull} instantiates the interface to ULL WToken signatures.
Section~\ref{sec:relation} explains how this bridge interacts with CPM closure and with optimization-based symbol grounding.
Section~\ref{sec:proofs} provides the detailed mathematical proofs and proof structure for each formal statement.
Section~\ref{sec:limits} records the scope limits and the next proof obligations for a full ``perfect language'' result.

\section{Background and prerequisites}
\label{sec:background}

This section fixes the minimal background used by the forcing bridge: (i) the recognition-quotient viewpoint on observables and gauge, and (ii) the canonical reciprocal mismatch cost and the associated argmin semantics.
Both ingredients are designed to be citation-friendly: they are not ULL-specific choices but reusable interfaces.

\subsection{Recognition quotients and gauge: why ``observable = quotient''}
\label{subsec:background-quotient}
The ``observable = quotient'' principle is a standard mathematical fact once one takes \emph{recognizers} (measurements) as primitive.
Let \(\mathcal{C}\) be a configuration space and let \(\mathcal{E}\) be an event space.
A recognizer is a map
\[
R:\mathcal{C}\to\mathcal{E},
\]
and it induces an indistinguishability relation
\[
c_1 \sim_R c_2 \quad\Longleftrightarrow\quad R(c_1)=R(c_2).
\]
Because \(\sim_R\) is defined by equality in \(\mathcal{E}\), it is an equivalence relation.
The associated \emph{recognition quotient}
\[
\mathcal{C}_R := \mathcal{C}/{\sim_R}
\]
is therefore the space of \emph{observable states} relative to \(R\): two configurations define the same observable state if and only if no outcome of \(R\) can distinguish them.

This quotient viewpoint is not philosophical decoration; it is the canonical way to remove unobservable degrees of freedom.
If an analyst writes down a ``state space'' larger than what \(R\) can distinguish, the quotient \(\mathcal{C}_R\) is the object that records exactly the observable content of the model.
In this sense, any semantics intended to be ``physics-level'' must respect recognition quotients: it should be constant on \(\sim_R\)-classes, i.e.\ factor through \(\mathcal{C}_R\).
This quotient-first viewpoint (recognizers as primitive, observables as quotient classes, and gauge as a source of indistinguishability) is developed in the Recognition Geometry axiomatization~\cite{RG2026}.

\paragraph{Gauge as a source of quotients.}
In physical theories, a distinguished class of transformations (the gauge group) acts on configuration space while preserving observables.
Abstractly, one specifies a group \(\mathcal{G}_R\) of admissible recognition-preserving automorphisms and defines a gauge equivalence relation
\[
c_1 \sim_{\mathrm{gauge}} c_2 \quad\Longleftrightarrow\quad \exists\,T\in\mathcal{G}_R,\; T(c_1)=c_2.
\]
By definition, gauge-equivalent configurations are observationally indistinguishable for the corresponding recognizer(s), so gauge naturally produces quotients.
For our purposes, the lesson is simple: whenever a degree of freedom can be shown to vary only along such a gauge direction, it cannot be semantic data in any construction that is required to be gauge-invariant.

\subsection{Canonical reciprocal mismatch cost and meaning as argmin}
\label{subsec:background-J-argmin}
The second prerequisite is an explicit selection rule for meaning built from a canonical mismatch geometry.
We summarize the ratio-induced framework from the reciprocal-convex-cost results.

\paragraph{Ratio-induced reference costs.}
One fixes:
\begin{itemize}
  \item a configuration/token space \(S\) and an object space \(O\),
  \item positive scale maps \(\iota_S:S\to\mathbb{R}_{>0}\) and \(\iota_O:O\to\mathbb{R}_{>0}\),
  \item and a mismatch penalty \(J:(0,\infty)\to[0,\infty)\).
\end{itemize}
The induced mismatch (reference) cost is the ratio cost
\[
c(s,o) := J\!\left(\frac{\iota_S(s)}{\iota_O(o)}\right), \qquad (s,o)\in S\times O.
\]
Under the stated axioms (inversion symmetry, strict convexity, normalization at \(1\), and a multiplicative d'Alembert identity), the penalty is essentially forced:
there exists \(a>0\) such that
\[
J(x)=\cosh(a\log x)-1=\tfrac12(x^{a}+x^{-a})-1,\qquad x>0,
\]
and the parameter \(a\) can be absorbed into the scale maps (by raising \(\iota_S,\iota_O\) to the power \(a\)).
Accordingly, one may normalize to the canonical form
\begin{equation}\label{eq:J-canonical}
J(x)=\tfrac12(x+x^{-1})-1,\qquad x>0.
\end{equation}
We refer to the reciprocal-convex-cost characterization for the precise axioms and classification theorem~\cite{RCC2026}.

\paragraph{Meaning as minimization (argmin).}
Given a cost \(c\), the meaning relation is defined by minimization:
a configuration \(s\in S\) \emph{means} an object \(o\in O\) if \(o\) is a minimizer of \(o'\mapsto c(s,o')\).
Equivalently, one can take meaning to be the set-valued map
\[
\mathrm{Mean}(s):=\arg\min_{o\in O}\; c(s,o).
\]
This definition is deliberately non-mentalistic: it replaces ``interpretation'' with an intrinsic optimization statement.
Uniqueness is not automatic (ties may occur), and existence of minimizers requires separate attainment hypotheses (e.g.\ compactness/continuity).

\paragraph{Why this matters for ULL forcing.}
In the ULL token layer, the relevant ``object'' is not an external referent in the world but an internal signature class.
The forcing mechanism used later in this paper is a finite-instance of the same idea:
we minimize a canonical cost over a finite lattice (the \(\varphi\)-ladder), which gives a provably well-defined argmin selection.

\subsection{Why closure still needs a meaning functional and an identifiability story}
\label{subsec:background-closure-gap}
With these prerequisites in view, the earlier caveat becomes precise.
Closure results---even when they yield canonical normal forms---typically establish that a system has:
(i) admissible configurations, (ii) invariants, and (iii) canonical representatives (e.g.\ unique normal forms under a reduction relation).
But none of these alone is a meaning claim unless one additionally specifies:
\begin{itemize}
  \item a \emph{meaning functional} (a cost or objective whose minimizers are declared to be semantic targets), and
  \item an \emph{identifiability criterion} determining which distinctions are semantic and which lie in a gauge/quotient direction.
\end{itemize}
The bridge proved in this paper is exactly the statement that, once these two pieces are fixed in a canonical way, meaning becomes a forced object rather than an extra postulate.

\section{The meaning-forcing interface (abstract certificate)}
\label{sec:interface}

This section isolates the \emph{minimal interface} needed to state and prove a forcing theorem of the form
\[
\text{(closure + canonical mismatch geometry + gauge)} \;\Longrightarrow\; \text{(meaning object is forced up to gauge)}.
\]
The purpose of the interface is twofold:
(i) it makes explicit the exact points where ``meaning'' enters (a selection functional and an identifiability quotient), and
(ii) it keeps the dependence surface small enough that the resulting forcing lemma can serve as a reusable certificate component.

\subsection{Window space and global phase gauge}
\label{subsec:interface-window-gauge}
At the token layer, ULL operates on neutralized eight-beat windows.
Abstractly, we model a window as a vector in \(\mathbb{C}^8\), written \(v\in\mathbb{C}^8\).
The key non-identifiability in this model is global phase: multiplying a window by a unit-modulus complex scalar does not change any construction intended to be phase-invariant.
We therefore define a phase (gauge) equivalence relation.

\begin{definition}[Global phase equivalence]\label{def:phase-eq}
For \(v_1,v_2\in\mathbb{C}^8\), we write \(v_1 \sim_{\mathrm{phase}} v_2\) if there exists \(c\in\mathbb{C}\) with \(|c|=1\) such that \(v_2 = c\cdot v_1\) (scalar multiplication).
\end{definition}

The relation \(\sim_{\mathrm{phase}}\) is an equivalence relation.
Its quotient \(\mathbb{C}^8/{\sim_{\mathrm{phase}}}\) is the canonical carrier for any phase-gauge-invariant semantics at the window level.
In the ULL periodic-table setting we will not quotient \emph{windows} directly, but rather quotient the \emph{signature representations} of tokens; nevertheless, Definition~\ref{def:phase-eq} is the basic gauge notion used throughout.

\subsection{Signature space (mode, phi-level, tau) and legality}
\label{subsec:interface-signatures}
ULL token signatures carry three components:
\begin{itemize}
  \item a \emph{mode family} \(m\) in a finite set \(\mathsf{Mode}\),
  \item a \(\varphi\)-quantization \emph{level} \(\ell\) in a finite set \(\mathsf{PhiLevel}\),
  \item and a \(\tau\)-offset \(\tau\) in a small set \(\mathsf{Tau}\).
\end{itemize}
The only \(\tau\)-variants permitted depend on the mode family.
Abstractly, we package this as a legality predicate \(\mathsf{TauLegal}(m,\tau)\) stating that \(\tau\) is allowed for mode family \(m\).
In the ULL token model, non-self-conjugate families admit only \(\tau=\tau_0\), while the self-conjugate family admits two offsets (e.g.\ \(\tau_0\) and \(\tau_2\)).

\begin{definition}[\(\tau\)-extended signatures]\label{def:tau-signature}
Let \(\mathsf{Mode}\), \(\mathsf{PhiLevel}\), and \(\mathsf{Tau}\) be finite sets and let \(\mathsf{TauLegal}:\mathsf{Mode}\times\mathsf{Tau}\to\{\text{true},\text{false}\}\) be a legality predicate.
Define the set of \(\tau\)-extended signatures
\[
\mathsf{Sig}_\tau := \{(m,\ell,\tau)\in\mathsf{Mode}\times\mathsf{PhiLevel}\times\mathsf{Tau}:\ \mathsf{TauLegal}(m,\tau)\}.
\]
\end{definition}

\paragraph{Basis representatives.}
To each \(\tau\)-extended signature we associate a canonical representative vector (a ``basis waveform'')
\[
\mathsf{basis}:\mathsf{Sig}_\tau \to \mathbb{C}^8.
\]
In the ULL implementation, \(\mathsf{basis}\) is a frozen DFT8-backed construction; for our interface, it is simply part of the data.

\subsection{Gauge equivalence on signatures and the quotient meaning object}
\label{subsec:interface-quotient}
We now formalize the idea that \(\tau\) should not be treated as semantic data beyond gauge.
We do so by defining a gauge relation on signatures and quotienting by it.

\begin{definition}[Gauge equivalence on \(\tau\)-extended signatures]\label{def:gaugeeq-sig}
For \(s,s'\in\mathsf{Sig}_\tau\), write \(s \sim_{\mathrm{gauge}} s'\) if:
\begin{enumerate}
  \item \(s\) and \(s'\) have the same mode family \(m\),
  \item \(s\) and \(s'\) have the same \(\varphi\)-level \(\ell\), and
  \item \(\mathsf{basis}(s)\sim_{\mathrm{phase}} \mathsf{basis}(s')\) in the sense of Definition~\ref{def:phase-eq}.
\end{enumerate}
\end{definition}

The quotient
\[
\mathsf{TauModuloGauge} \;:=\; \mathsf{Sig}_\tau/{\sim_{\mathrm{gauge}}}
\]
is the \emph{meaning object} used in this paper: it is a \(\tau\)-extended token signature in which the \(\tau\)-degree of freedom has been eliminated as a gauge direction.

\subsection{Forced selection rules: classifier for mode, argmin for phi-level, and a canonical tau representative}
\label{subsec:interface-forced-rules}
To force a meaning object we need (i) a deterministic extraction of the discrete mode family (when possible), and (ii) a deterministic selection of the \(\varphi\)-level by minimization of the canonical mismatch cost.

\paragraph{Mode-family classifier.}
We assume a deterministic partial classifier\footnote{We write \(f:X\rightharpoonup Y\) for a partial function from \(X\) to \(Y\), i.e., a function defined on some subset of \(X\).}
\[
\mathsf{forcedMode}:\mathbb{C}^8 \rightharpoonup \mathsf{Mode},
\]
which either returns an exact mode family (when defined) or is undefined (ambiguous/invalid input).
In ULL this is derived from a frozen basis-class classifier.

\begin{remark}[Phase invariance of the classifier]\label{rem:forcedMode-phase}
To respect the window-level phase gauge (Definition~\ref{def:phase-eq}), a mode-family classifier should be invariant under global phase:
if \(v_2=c\cdot v_1\) with \(|c|=1\), then \(\mathsf{forcedMode}(v_1)\) is defined if and only if \(\mathsf{forcedMode}(v_2)\) is defined, and when defined the returned mode families agree.
This paper's forcing theorem is conditional on \(\mathsf{forcedMode}\) being defined; the above invariance is an additional (checkable) property of the concrete classifier used in ULL.
\end{remark}

\paragraph{\(\varphi\)-level selection by argmin.}
Fix a golden ratio constant \(\varphi>1\), and fix integer exponents \(e:\mathsf{PhiLevel}\to \mathbb{Z}\) encoding the finite \(\varphi\)-ladder.
Assume \(\mathsf{PhiLevel}\) is nonempty, and fix a total order \(\preceq\) on \(\mathsf{PhiLevel}\) (used only to break ties deterministically).
For a positive scalar \(r>0\), define the ladder cost
\[
\mathsf{cost}_\varphi(r,\ell) := J\!\left(\frac{r}{\varphi^{e(\ell)}}\right),
\]
with \(J\) the canonical mismatch penalty \eqref{eq:J-canonical}.
Define \(\mathsf{forcedPhiLevel}(r)\in\mathsf{PhiLevel}\) to be the \(\preceq\)-least element of the argmin set
\[
\arg\min_{\ell\in\mathsf{PhiLevel}}\; \mathsf{cost}_\varphi(r,\ell),
\]
which is nonempty since \(\mathsf{PhiLevel}\) is finite and nonempty (Lemma~\ref{lem:finite-argmin}).
Equivalently, the forcing condition is the argmin inequality:
\begin{equation}\label{eq:phi-argmin}
\mathsf{cost}_\varphi(r,\mathsf{forcedPhiLevel}(r)) \le \mathsf{cost}_\varphi(r,\ell)\qquad \text{for all }\ell\in\mathsf{PhiLevel}.
\end{equation}

\paragraph{Canonical \(\tau\)-representative.}
Because \(\tau\) will be eliminated in the quotient \(\mathsf{TauModuloGauge}\), any legal choice of \(\tau\) could be used as a representative.
For concreteness, we fix a canonical choice \(\tau_0\) and use it whenever \(\mathsf{forcedMode}(v)\) is defined and equals some mode family \(m\).

\subsection{The forced meaning map}
\label{subsec:interface-meaning-map}
With these ingredients, the token-layer meaning map is a partial function
\[
\mathsf{Meaning}:\mathbb{C}^8\times\mathbb{R}_{>0}\rightharpoonup \mathsf{TauModuloGauge}
\]
defined by:
\begin{equation}\label{eq:meaning-def}
\mathsf{Meaning}(v,r) :=
\begin{cases}
[(m,\mathsf{forcedPhiLevel}(r),\tau_0)]_{\sim_{\mathrm{gauge}}}, & \text{if }\mathsf{forcedMode}(v)=m\text{ is defined},\\
\text{undefined}, & \text{if }\mathsf{forcedMode}(v)\text{ is undefined}.
\end{cases}
\end{equation}
In words: when the classifier returns an exact mode family, the \(\varphi\)-level is forced by argmin and the \(\tau\)-degree is quotiented out, producing a meaning object that is invariant under the signature-level gauge relation.

\begin{lemma}[Window-level phase invariance of \(\mathsf{Meaning}\)]\label{lem:Meaning-phase}
Assume the phase-invariance property of Remark~\ref{rem:forcedMode-phase}.
Fix \(r>0\) and \(v_1,v_2\in\mathbb{C}^8\) with \(v_1\sim_{\mathrm{phase}} v_2\).
Then \(\mathsf{Meaning}(v_1,r)\) is defined if and only if \(\mathsf{Meaning}(v_2,r)\) is defined, and when defined the two values in \(\mathsf{TauModuloGauge}\) are equal.
\end{lemma}

\begin{proof}
Choose \(c\in\mathbb{C}\) with \(|c|=1\) such that \(v_2=c\cdot v_1\).
By Remark~\ref{rem:forcedMode-phase}, \(\mathsf{forcedMode}(v_1)\) is defined if and only if \(\mathsf{forcedMode}(v_2)\) is defined, and when defined the returned mode family \(m\) is the same.
The conclusion follows immediately from Definition~\eqref{eq:meaning-def}.
\end{proof}

\subsection{Certificate obligations (what must be proved)}
\label{subsec:interface-obligations}
To certify that Definition~\ref{eq:meaning-def} is genuinely ``forced'' (and not a hidden convention), two facts must be established:
\begin{enumerate}
  \item \textbf{\(\varphi\)-level forcing:} the argmin inequality \eqref{eq:phi-argmin} holds for all \(r>0\);
  \item \textbf{\(\tau\) carries no semantic freedom:} for any fixed \((m,\ell)\) and any two legal \(\tau,\tau'\), the two signatures \((m,\ell,\tau)\) and \((m,\ell,\tau')\) are gauge equivalent (Definition~\ref{def:gaugeeq-sig}), hence define the same element of \(\mathsf{TauModuloGauge}\).
\end{enumerate}
These are exactly the two certificate obligations that must be discharged: \(\varphi\)-level minimizing reference cost on the finite ladder, and \(\tau\)-invariance up to global phase gauge for legal variants at fixed mode+\(\varphi\).
Both are proved in Section~\ref{sec:main}.

\section{Main theorem: Meaning is forced up to gauge}
\label{sec:main}

We now discharge the two certificate obligations from Section~\ref{subsec:interface-obligations} and package them as a single forcing statement.
The result is intentionally scoped to the token/signature layer: it shows that, once the mode family is fixed (when classification succeeds) and the mismatch geometry is fixed (canonical \(J\) and the \(\varphi\)-ladder), there is no remaining semantic degree of freedom in the \(\tau\)-offset beyond phase gauge.

\subsection{Forcing the phi-level by argmin on the finite ladder}
\label{subsec:main-phi}

\begin{proposition}[\(\varphi\)-level forcing (argmin on the ladder)]\label{prop:phi-forcing}
For every \(r>0\), the forced \(\varphi\)-level satisfies the argmin inequality \eqref{eq:phi-argmin}:
\[
\mathsf{cost}_\varphi(r,\mathsf{forcedPhiLevel}(r)) \le \mathsf{cost}_\varphi(r,\ell)\qquad \text{for all }\ell\in\mathsf{PhiLevel}.
\]
\end{proposition}

\begin{proof}
Because \(\mathsf{PhiLevel}\) is finite and nonempty, Lemma~\ref{lem:finite-argmin} implies the argmin set is nonempty.
By definition, \(\mathsf{forcedPhiLevel}(r)\) is chosen from that argmin set (with ties broken by \(\preceq\)), hence it satisfies the displayed inequality.
\end{proof}

\begin{remark}[Ties and determinism]\label{rem:phi-ties}
The argmin need not be unique; ties can occur at boundary values of \(r\).
The forcing statement used in this paper is therefore the inequality form \eqref{eq:phi-argmin}, not uniqueness of the minimizer.
When a deterministic semantics is desired, ties are resolved by a fixed convention encoded in the definition of \(\mathsf{forcedPhiLevel}\).
\end{remark}

\subsection{Tau is eliminated by gauge: uniqueness in TauModuloGauge}
\label{subsec:main-tau}

\begin{remark}[Standing \(\tau\)-phase assumption]\label{rem:tau-phase-assumption}
Throughout this section we assume the following concrete property of the chosen basis representatives:
whenever \((m,\ell,\tau)\) and \((m,\ell,\tau')\) are both legal signatures in \(\mathsf{Sig}_\tau\), their basis vectors are phase equivalent,
\[
\mathsf{basis}(m,\ell,\tau)\sim_{\mathrm{phase}}\mathsf{basis}(m,\ell,\tau').
\]
In the ULL DFT8-backed basis construction this holds by an explicit computation in the self-conjugate family (Section~\ref{subsec:proofs-dft}); outside that family, legality forces \(\tau=\tau'=\tau_0\).
\end{remark}

\begin{proposition}[\(\tau\)-invariance up to phase gauge]\label{prop:tau-gauge}
Fix a mode family \(m\in\mathsf{Mode}\) and a \(\varphi\)-level \(\ell\in\mathsf{PhiLevel}\).
Let \(\tau,\tau'\in\mathsf{Tau}\) be legal offsets, i.e.\ \(\mathsf{TauLegal}(m,\tau)\) and \(\mathsf{TauLegal}(m,\tau')\).
Then the corresponding signatures are gauge equivalent:
\[
(m,\ell,\tau)\sim_{\mathrm{gauge}}(m,\ell,\tau') \quad\text{in the sense of Definition~\ref{def:gaugeeq-sig}.}
\]
Equivalently, they define the same element of the quotient meaning object:
\[
[(m,\ell,\tau)]_{\sim_{\mathrm{gauge}}} = [(m,\ell,\tau')]_{\sim_{\mathrm{gauge}}}
\qquad \text{in }\mathsf{TauModuloGauge}.
\]
\end{proposition}

\begin{proof}
Let \(s=(m,\ell,\tau)\) and \(s'=(m,\ell,\tau')\).
By construction, \(s\) and \(s'\) have the same mode family and the same \(\varphi\)-level.
By Remark~\ref{rem:tau-phase-assumption}, their basis vectors are phase equivalent.
Therefore \(s\sim_{\mathrm{gauge}} s'\) by Definition~\ref{def:gaugeeq-sig}, which is exactly the claim.
\end{proof}

\begin{corollary}[Independence of the \(\tau_0\) representative]\label{cor:tau-rep-independent}
In Definition~\eqref{eq:meaning-def}, replacing the fixed representative \(\tau_0\) by any other legal choice \(\tau\) does not change the resulting meaning object in \(\mathsf{TauModuloGauge}\).
\end{corollary}

\begin{proof}
Immediate from Proposition~\ref{prop:tau-gauge}.
\end{proof}

\subsection{Meaning is forced up to gauge (token/signature layer)}
\label{subsec:main-forced}

\begin{theorem}[Meaning is forced up to gauge]\label{thm:meaning-forced-up-to-gauge}
Fix \(r>0\) and an eight-beat window \(v\in\mathbb{C}^8\).
\begin{enumerate}
  \item If \(\mathsf{forcedMode}(v)\) is undefined, then \(\mathsf{Meaning}(v,r)\) is undefined.
  \item If \(\mathsf{forcedMode}(v)=m\) is defined, let \(\ell^*:=\mathsf{forcedPhiLevel}(r)\).
  Then there exists a unique meaning object \(M\in\mathsf{TauModuloGauge}\) such that for every legal \(\tau\),
  \[
  M = [(m,\ell^*,\tau)]_{\sim_{\mathrm{gauge}}}.
  \]
  Moreover, the meaning map \eqref{eq:meaning-def} is defined and equals exactly this object:
  \[
  \mathsf{Meaning}(v,r) = M.
  \]
\end{enumerate}
\end{theorem}

\begin{proof}
(1) is immediate from Definition~\eqref{eq:meaning-def}.

For (2), let \(M:=[(m,\ell^*,\tau_0)]_{\sim_{\mathrm{gauge}}}\).
Then \(\mathsf{Meaning}(v,r)=M\) by Definition~\eqref{eq:meaning-def}.
If \(\tau\) is any other legal offset, Proposition~\ref{prop:tau-gauge} gives
\([(m,\ell^*,\tau)]_{\sim_{\mathrm{gauge}}}=M\), proving the defining property.
Uniqueness follows because the defining property forces any such \(M'\) to equal \(M\) (take \(\tau=\tau_0\)).
\end{proof}

\begin{remark}[What this theorem does and does not claim]\label{rem:main-scope}
Theorem~\ref{thm:meaning-forced-up-to-gauge} is a \emph{meaning-forcing bridge} at the token/signature layer.
It does not claim that the mode-family classifier is infallible, nor that end-to-end meanings for arbitrary signals have been fully reduced to a unique LNAL normal form.
What it does claim is sharper and more audit-friendly: \emph{conditional on successful mode classification}, the remaining semantic degrees of freedom are fixed by a canonical argmin rule and a forced gauge quotient, so \(\tau\) is not available as a free semantic parameter.
\end{remark}

\section{Instantiation to ULL WTokens}
\label{sec:ull}

This section instantiates the abstract interface of Section~\ref{sec:interface} to the concrete token vocabulary used by ULL.
The aim is not to re-derive the ULL periodic table empirically; rather, it is to show that once a fixed WToken vocabulary and basis model are in place, the forcing theorem of Section~\ref{sec:main} becomes a literal statement about ULL meaning objects.

\subsection{The ULL token vocabulary and its signature components}
\label{subsec:ull-vocab}
ULL posits a finite dictionary of \emph{semantic atoms} called WTokens.
In this presentation, the vocabulary is a fixed finite set of 20 elements, each equipped with a signature decomposition.
Each WToken carries three pieces of signature data:
\begin{itemize}
  \item a \emph{mode family} \(m\) (four families in the eight-beat model),
  \item a \(\varphi\)-quantization level \(\ell\) (a small finite ladder),
  \item and a \(\tau\)-offset \(\tau\) subject to a legality constraint depending on \(m\).
\end{itemize}
We emphasize a claims-hygiene point: this section treats the vocabulary and basis model as \emph{given} data.
The empirical and RS-motivated arguments for why this vocabulary is the right one belong in the ULL periodic-table manuscript; the present paper only proves that, \emph{conditional on this vocabulary and its invariances}, meaning is forced up to gauge.

\subsection{Concrete basis representatives in eight-dimensional complex space}
\label{subsec:ull-basis}
To connect signatures to windows, ULL uses a frozen basis construction
\[
\mathsf{basis}:\mathsf{Sig}_\tau\to\mathbb{C}^8
\]
as in Section~\ref{subsec:interface-signatures}.
In the ULL implementation, \(\mathsf{basis}\) is DFT8-backed: each mode family selects a discrete Fourier mode (or a conjugate pair), and the remaining signature parameters select a canonical representative within that family.
The only structural feature we use in this paper is the following: for the self-conjugate family, the two legal \(\tau\)-variants differ by multiplication by a unit-modulus complex scalar (a global phase).
This is the origin of the gauge quotient in the meaning object.

\subsection{Mode-family forcing (when classification succeeds)}
\label{subsec:ull-mode-forcing}
The abstract partial classifier \(\mathsf{forcedMode}:\mathbb{C}^8\rightharpoonup\mathsf{Mode}\) from Section~\ref{subsec:interface-forced-rules} is instantiated in ULL by a frozen basis-class classifier that analyzes the eight-beat window and either:
\begin{itemize}
  \item returns an exact mode family (classifier is defined), or
  \item is undefined (ambiguous or invalid input).
\end{itemize}
Accordingly, the ULL meaning map is partial at the token layer: it is defined only when the classifier is defined.
This is the correct interface for peer review: any robustness claims about the classifier are empirical and must be evaluated as such; the forcing theorem is conditional on the classifier being defined.

\subsection{Phi-level forcing from RS mismatch geometry}
\label{subsec:ull-phi-forcing}
The \(\varphi\)-level component is forced by an argmin rule of the form \eqref{eq:phi-argmin}.
In the ULL setting, the relevant scalar \(r>0\) is an extracted radius/amplitude parameter derived from the window (or from a window-level summary statistic) and is treated as external input to the token-layer forcing theorem.
Once \(r>0\) is fixed, the forced level \(\ell^*=\mathsf{forcedPhiLevel}(r)\) is the (deterministically selected) minimizer of the canonical mismatch penalty \(J(r/\varphi^{e(\ell)})\) over the finite ladder.
This exactly matches the general meaning-as-argmin semantics of Section~\ref{subsec:background-J-argmin}, specialized to a finite hypothesis class.

\subsection{Tau modulo gauge: the ULL meaning object}
\label{subsec:ull-tau-mod-gauge}
The last signature component, \(\tau\), is where ``forced meaning'' meets gauge.
ULL allows multiple \(\tau\)-variants only in the self-conjugate family; and in that case, the basis representatives differ only by a global phase factor.
Therefore any semantics required to be invariant under global phase cannot treat \(\tau\) as semantic content.
Instead, meanings must live in the quotient
\[
\mathsf{TauModuloGauge} := \mathsf{Sig}_\tau/{\sim_{\mathrm{gauge}}},
\]
where \(\sim_{\mathrm{gauge}}\) fixes \((m,\ell)\) and identifies \(\tau\)-variants whose basis vectors are phase equivalent.

Under this quotient, the meaning object is genuinely unique:
for a fixed \((m,\ell)\), all legal \(\tau\)-variants define the same element of \(\mathsf{TauModuloGauge}\) (Proposition~\ref{prop:tau-gauge}).
This is the precise mathematical sense in which \(\tau\) is \emph{not} a semantic degree of freedom in ULL at the token layer.

\subsection{Resulting ULL token-layer meaning map}
\label{subsec:ull-meaning-map}
With these instantiations, the abstract meaning map \eqref{eq:meaning-def} becomes a literal definition for ULL windows:
given a window \(v\in\mathbb{C}^8\) and a scalar \(r>0\),
\begin{itemize}
  \item classify \(v\) to obtain a mode family \(m\) (or fail),
  \item compute \(\ell^*=\mathsf{forcedPhiLevel}(r)\) by argmin on the \(\varphi\)-ladder,
  \item return the quotient class \([(m,\ell^*,\tau_0)]\in \mathsf{TauModuloGauge}\).
\end{itemize}
Theorem~\ref{thm:meaning-forced-up-to-gauge} then reads as a forcing statement about ULL: when mode classification succeeds, ULL's token-layer meaning is uniquely determined in \(\mathsf{TauModuloGauge}\), and is independent of any \(\tau\)-offset choice beyond global phase.

\begin{remark}[Where the ULL periodic-table manuscript should cite this paper]\label{rem:ull-cite}
The ULL periodic-table manuscript should cite this paper specifically for the claim:
\emph{meaning is forced (unique up to gauge) at the WToken signature layer}.
This allows the ULL manuscript to separate the logical forcing claim from the larger engineering and empirical questions of classifier accuracy, grammar completeness, and cross-modal evaluation.
\end{remark}

\section{Relationship to CPM closure and to symbol grounding}
\label{sec:relation}

This paper is intentionally a \emph{bridge}: it does not replace existing closure results or the reference/aboutness layer, but explains how they fit together to answer the reviewer-style question ``where does meaning come from?''.
We therefore clarify the roles of CPM method closure and optimization-based symbol grounding relative to the forcing theorem proved here.

\subsection{CPM closure: what it gives, and what it does not}
\label{subsec:relation-cpm}
The Coercive Projection Method (CPM) is a quantitative closure kernel: from a small family of inequalities (projection--defect, energy control, dispersion/tests) it derives global coercivity and aggregation bounds with explicit constants~\cite{CPM2026}.
In particular, CPM closure is well-suited to certifying properties of \emph{optimization pipelines}:
it prevents vacuity (by bundling a witness), and it gives the standard implication pattern
\[
\text{local/test control}\ \Longrightarrow\ \text{global defect control}\ \Longrightarrow\ \text{stability or non-degeneracy}.
\]

At the same time, CPM closure is not, by itself, a semantics theorem.
The CPM inequalities control \emph{defect, energy gaps, and tests}; they do not specify:
(i) what the semantic objects are, (ii) which cost functional is the meaning rule, or (iii) which non-identifiable degrees of freedom should be quotiented out.
This is exactly the caveat raised in the motivating feedback: CPM closure contributes to a closed dynamical system, but closure alone does not yet imply meaning.

\subsection{How CPM supports a meaning pipeline without defining meaning}
\label{subsec:relation-cpm-support}
Although CPM closure does not define meaning, it plays two important supporting roles for ULL-style systems:
\begin{enumerate}
  \item \textbf{Non-vacuity and interface consistency.}
  Certificate-style work benefits from explicit non-vacuity witnesses: they ensure that ``for all models satisfying assumptions'' statements are not empty implications.
  \item \textbf{Stability margins for discovery.}
  Many empirical steps in token discovery can be cast as ``minimize cost subject to legality/tests''.
  CPM-style coercivity and aggregation bounds are the right tool to argue that such minimizers cannot drift arbitrarily far from the structured set, and that passing local checks forces global structure.
\end{enumerate}
In other words: CPM is the quantitative engine that can justify that a learned dictionary of atoms is \emph{stable, non-degenerate, and auditable} under the chosen constraints.
But CPM does not tell you what it would mean for such atoms to be \emph{about} anything.

\subsection{Symbol grounding and aboutness: why argmin semantics is meaning (in this model)}
\label{subsec:relation-grounding}
The missing ingredient is supplied by the ratio-based reference/aboutness layer: meaning is defined as an \emph{argmin} of a fixed mismatch cost.
This gives a semantics that is internal to the system dynamics (no interpreter primitive): ``\(s\) means \(o\)'' becomes a verifiable inequality statement comparing \(c(s,o)\) to \(c(s,o')\) for competing objects \(o'\).

The optimization-based symbol-grounding view strengthens this in a way that directly addresses the regress objection.
Meaning is not postulated as a mapping chosen by an external agent; it is a minimization rule under a universal mismatch geometry.
In particular, once \(J\) is fixed canonically (Section~\ref{subsec:background-J-argmin}), the semantics layer becomes a forced choice of objective, not an arbitrary design knob.
See~\cite{SGP2026} for a dedicated symbol-grounding treatment in this optimization semantics.

\subsection{Where this paper sits: ``meaning is forced'' = argmin + quotient}
\label{subsec:relation-bridge}
The forcing theorem of Section~\ref{sec:main} isolates the precise additional step needed to convert ``closed system + invariants'' into ``meaning object'':
\begin{itemize}
  \item \textbf{Selection:} fix an internal objective (canonical mismatch cost) and define meaning as argmin;
  \item \textbf{Identifiability:} quotient out gauge directions so that non-identifiable degrees of freedom cannot be smuggled in as semantic parameters.
\end{itemize}
The \(\tau\)-offset is a representative example: in the self-conjugate family it is a pure phase-gauge degree of freedom, so treating it as semantic would violate the observable=quotient discipline.
Quotienting produces a meaning object that is unique in the appropriate sense (unique \emph{up to gauge}).

This is the sense in which the bridge closes the peer-review gap:
once the cost functional and quotient structure are fixed canonically, meaning is no longer an extra postulate.
It becomes a forced object derived from the same closure discipline as the rest of the RS/ULL stack.

\section{Mathematical foundations and proof structure}
\label{sec:proofs}

The forcing bridge in this paper is not presented as an informal philosophical principle.
It is packaged as a small set of definitions and lemmas stated in standard mathematical terms, with explicit proof structure corresponding to the obligations in Section~\ref{subsec:interface-obligations}.
This section records the mathematical foundations at a level intended to be useful for referees and auditors.

\subsection{Equivalence-relation foundations}
\label{subsec:proofs-equiv}
The proofs rely on standard equivalence-relation and quotient-space constructions from elementary set theory and abstract algebra.

\begin{lemma}[Phase equivalence is an equivalence relation]\label{lem:phase-equiv}
The relation \(\sim_{\mathrm{phase}}\) on \(\mathbb{C}^8\) (Definition~\ref{def:phase-eq}) is an equivalence relation.
\end{lemma}

\begin{proof}
\emph{Reflexivity:} For any \(v\in\mathbb{C}^8\), we have \(v = 1 \cdot v\) with \(|1|=1\), so \(v\sim_{\mathrm{phase}} v\).

\emph{Symmetry:} If \(v_2 = c \cdot v_1\) with \(|c|=1\), then \(v_1 = c^{-1} \cdot v_2\) with \(|c^{-1}|=1\).

\emph{Transitivity:} If \(v_2 = c_1 \cdot v_1\) and \(v_3 = c_2 \cdot v_2\) with \(|c_1|=|c_2|=1\), then \(v_3 = (c_2 c_1) \cdot v_1\) with \(|c_2 c_1|=1\).
\end{proof}

\begin{lemma}[Gauge equivalence on signatures is an equivalence relation]\label{lem:gauge-equiv}
The relation \(\sim_{\mathrm{gauge}}\) on \(\mathsf{Sig}_\tau\) (Definition~\ref{def:gaugeeq-sig}) is an equivalence relation.
\end{lemma}

\begin{proof}
Conditions (1) and (2) of Definition~\ref{def:gaugeeq-sig} are equality conditions, hence automatically reflexive, symmetric, and transitive.
Condition (3) is phase equivalence of basis vectors, which is an equivalence relation by Lemma~\ref{lem:phase-equiv}.
The conjunction of three equivalence relations on a product structure is again an equivalence relation.
\end{proof}

\subsection{Well-definedness of the quotient meaning object}
\label{subsec:proofs-quotient}
Given that \(\sim_{\mathrm{gauge}}\) is an equivalence relation, the quotient
\[
\mathsf{TauModuloGauge} := \mathsf{Sig}_\tau / {\sim_{\mathrm{gauge}}}
\]
is a well-defined set.
Each element of \(\mathsf{TauModuloGauge}\) is an equivalence class \([s]_{\sim_{\mathrm{gauge}}}\) for some signature \(s\in\mathsf{Sig}_\tau\).
Two signatures represent the same meaning object if and only if they are gauge equivalent.

\subsection{Argmin existence on finite sets}
\label{subsec:proofs-argmin}
The \(\varphi\)-level forcing (Proposition~\ref{prop:phi-forcing}) relies on a basic fact from optimization theory.

\begin{lemma}[Argmin exists on nonempty finite sets]\label{lem:finite-argmin}
Let \(X\) be a nonempty finite set and let \(f:X\to\mathbb{R}\) be any function.
Then \(\arg\min_{x\in X} f(x)\) is nonempty, and any deterministic selection from this set yields a minimizer satisfying
\[
f(x^*) \le f(x) \quad \text{for all } x\in X.
\]
\end{lemma}

\begin{proof}
Since \(X\) is finite and nonempty, the image \(f(X)\subset\mathbb{R}\) is a nonempty finite subset of \(\mathbb{R}\), hence has a minimum.
Any \(x^*\) achieving this minimum satisfies the displayed inequality.
\end{proof}

Proposition~\ref{prop:phi-forcing} follows immediately: \(\mathsf{PhiLevel}\) is a nonempty finite set, and \(\mathsf{forcedPhiLevel}(r)\) is defined as a deterministic selection from the argmin set.

\subsection{DFT8 basis structure and phase relations}
\label{subsec:proofs-dft}
The proof of Proposition~\ref{prop:tau-gauge} relies on the structure of the discrete Fourier transform on \(\mathbb{Z}/8\mathbb{Z}\).

\paragraph{DFT8 modes and conjugacy.}
The DFT8 has frequency modes \(k\in\{0,1,\ldots,7\}\).
Mode \(k\) and mode \(8-k\) are complex conjugates of each other.
Specifically:
\begin{itemize}
  \item Modes \((1,7)\), \((2,6)\), and \((3,5)\) form conjugate pairs (non-self-conjugate families);
  \item Mode \(4\) is self-conjugate: \(8-4=4\).
\end{itemize}

\paragraph{Basis vectors for non-self-conjugate families.}
For conjugate pairs, the canonical basis vector is uniquely determined (up to an overall normalization convention), so there is only one legal \(\tau\)-variant (\(\tau_0\)).
In this case, Proposition~\ref{prop:tau-gauge} is trivially satisfied: if \(\tau=\tau'=\tau_0\), then the signatures are identical.

\paragraph{Basis vectors for the self-conjugate family (mode 4).}
Let \(\omega:=e^{2\pi i/8}\).
The mode-4 DFT8 waveform is
\[
u_4(n):=\omega^{4n}=e^{\pi i n}=(-1)^n,\qquad n\in\{0,1,\ldots,7\}.
\]
The two legal \(\tau\)-representatives in the self-conjugate family may be taken as the two phase conventions
\[
b_{\tau_0}(n):=u_4(n),\qquad b_{\tau_2}(n):=i\,u_4(n).
\]
Then \(b_{\tau_2}=i\cdot b_{\tau_0}\) and \(|i|=1\), hence \(b_{\tau_0}\sim_{\mathrm{phase}} b_{\tau_2}\) by Definition~\ref{def:phase-eq}.
If the ULL basis construction also includes a positive amplitude factor depending on \(\ell\), the same conclusion holds since multiplication by a positive real commutes with multiplication by \(i\).

\subsection{Proof summary for the main theorem}
\label{subsec:proofs-main}
Theorem~\ref{thm:meaning-forced-up-to-gauge} combines the above ingredients:
\begin{enumerate}
  \item \textbf{Argmin selection:} By Lemma~\ref{lem:finite-argmin}, \(\mathsf{forcedPhiLevel}(r)\) is well-defined and satisfies the argmin inequality.
  \item \textbf{Gauge quotient well-defined:} By Lemma~\ref{lem:gauge-equiv}, \(\mathsf{TauModuloGauge}\) is a valid quotient space.
  \item \textbf{\(\tau\)-independence:} By the DFT8 analysis (Section~\ref{subsec:proofs-dft}), all legal \(\tau\)-variants at fixed \((m,\ell)\) are gauge equivalent, hence define the same element of \(\mathsf{TauModuloGauge}\).
\end{enumerate}
Therefore the meaning map \(\mathsf{Meaning}\) is well-defined, and its output (when classification succeeds) is a unique element of \(\mathsf{TauModuloGauge}\) that is independent of the \(\tau\)-representative chosen.

\subsection{Classical correspondences}
\label{subsec:proofs-correspondences}
The following table summarizes the correspondence between paper objects and standard mathematical constructions:

\begin{center}
\begin{tabular}{p{0.42\textwidth} p{0.50\textwidth}}
\hline
\textbf{Paper object} & \textbf{Classical mathematical construction} \\
\hline
Window space \(\mathbb{C}^8\) & 8-dimensional complex vector space \\
\hline
Phase gauge \(\sim_{\mathrm{phase}}\) & Equivalence relation: \(v\sim w\) iff \(w=e^{i\theta}v\) for some \(\theta\in\mathbb{R}\) \\
\hline
\(\tau\)-extended signatures \(\mathsf{Sig}_\tau\) & Finite subset of \(\mathsf{Mode}\times\mathsf{PhiLevel}\times\mathsf{Tau}\) satisfying legality predicate \\
\hline
Gauge equivalence \(\sim_{\mathrm{gauge}}\) & Equivalence relation on \(\mathsf{Sig}_\tau\) (Lemma~\ref{lem:gauge-equiv}) \\
\hline
\(\tau\) modulo gauge (meaning object) & Quotient set \(\mathsf{Sig}_\tau/{\sim_{\mathrm{gauge}}}\) \\
\hline
Meaning map \(\mathsf{Meaning}\) & Partial function \(\mathbb{C}^8\times\mathbb{R}_{>0}\rightharpoonup\mathsf{TauModuloGauge}\) \\
\hline
\(\varphi\)-level forcing & Argmin on finite set (Lemma~\ref{lem:finite-argmin}) \\
\hline
\(\tau\)-invariance up to gauge & Phase equivalence of DFT8 basis vectors (Section~\ref{subsec:proofs-dft}) \\
\hline
\end{tabular}
\end{center}

\section{Limits and roadmap}
\label{sec:limits}

This paper is a deliberately narrow bridge: it resolves the specific objection ``closure does not imply meaning'' by exhibiting a meaning object that is forced (unique up to gauge) once the canonical mismatch geometry and identifiability quotient are made explicit.
To keep the claim auditable, we record here both the scope of what is proved and the remaining work required for a full end-to-end ULL semantics theorem.

\subsection{Scope of the proved claim}
\label{subsec:limits-scope}
The main result (Theorem~\ref{thm:meaning-forced-up-to-gauge}) is a token/signature-layer statement:
given an eight-beat window \(v\in\mathbb{C}^8\) and a scalar \(r>0\), if mode-family classification succeeds then the meaning object is forced in the quotient set \(\mathsf{TauModuloGauge}\).
Equivalently, \(\tau\) is not available as a semantic parameter beyond phase gauge.

In particular, this paper does \emph{not} claim:
\begin{itemize}
  \item \textbf{Classifier correctness.} We do not prove that \(\mathsf{forcedMode}(v)\) returns the ``right'' family for every physically meaningful window, nor quantify its robustness to noise.
  \item \textbf{Extraction of \(r\).} We treat \(r>0\) as an input to the token-layer meaning map. Proving that a specific extraction procedure is correct and stable is separate work.
  \item \textbf{Global uniqueness of full ULL semantics.} The result does not assert that there exists exactly one complete language semantics for all signals, programs, and motifs.
  \item \textbf{End-to-end normal forms.} We do not prove termination/confluence of a full LNAL reduction system or uniqueness of normal forms for arbitrary signals.
\end{itemize}
These omissions are intentional: they are the larger projects that a system paper must address, whereas this bridge paper isolates the missing mathematical ingredient for ``meaning'' itself.

\subsection{What remains for a full ULL semantics theorem}
\label{subsec:limits-open}
To upgrade from token-layer forcing to a full periodic-table semantics claim, one needs additional layers beyond this paper:
\begin{enumerate}
  \item \textbf{From windows to signals.} Define a signal-level meaning object (e.g.\ sequences of token-layer meanings, motifs, or normal forms) and prove invariance under the intended equivalences (carrier changes, admissible rewrites, legality constraints).
  \item \textbf{Grammar/reduction theory.} Prove that the reduction system used to compute canonical representatives terminates and is confluent on the admissible class, yielding unique normal forms.
  \item \textbf{Completeness/minimality of the atom set.} Establish that the chosen WToken dictionary is sufficient to generate admissible behavior (completeness) and that no strict subdictionary suffices (minimality), under explicitly stated constraints.
  \item \textbf{Quantitative stability margins.} Where empirical procedures are used (classifier thresholds, ladder tolerances), provide margin theorems or robustness certificates that justify the stability regions away from decision boundaries.
  \item \textbf{Empirical validation.} Cross-modal persistence and the periodic-table claims (distance banding, legality rates) are empirical statements and should be reported with reproducibility artifacts and falsification criteria.
\end{enumerate}
The point of the present paper is that all of the above can be pursued without re-opening the conceptual question ``what is meaning?'': the meaning object is already fixed at the token layer as an argmin-plus-quotient construction.

\subsection{Roadmap: extending the mathematical framework}
\label{subsec:limits-roadmap}
From a mathematical standpoint, the next natural extensions are:
\begin{itemize}
  \item \textbf{Classifier soundness on canonical bases.} Prove that \(\mathsf{forcedMode}\) succeeds on the canonical basis representatives (and returns the intended family), giving a minimal nontrivial correctness witness.
  \item \textbf{Stability away from boundaries.} Prove margin-style statements: when \(r\) lies away from ladder decision boundaries, the forced \(\varphi\)-level is locally constant under perturbations.
  \item \textbf{Signal composition.} Define a signal-level meaning object as a quotient of sequences and prove that window-wise forcing composes coherently under concatenation and the admissible rewrite system.
\end{itemize}
These steps would provide a clean path from the present forcing bridge to a complete normal-form semantics layer.

\subsection{Recommended citation language for the ULL periodic-table paper}
\label{subsec:limits-cite}
To keep claims literally true and reviewer-proof, the ULL periodic-table manuscript should cite this paper for the following statement (or a close paraphrase):
\begin{quote}
At the WToken signature layer, meaning is forced \emph{up to gauge}: whenever mode-family classification succeeds, the \(\varphi\)-level is selected as an argmin under the canonical mismatch cost, and the remaining \(\tau\)-offset freedom is eliminated by quotienting modulo global phase. Hence \(\tau\) is not a semantic parameter, and the token-layer meaning object is unique in the quotient set ``\(\tau\) modulo gauge''.
\end{quote}
Claims of full-language uniqueness or end-to-end normal-form semantics should be stated separately and only when the corresponding proofs are in place.

\appendix

\section{Proof index and audit checklist}
\label{sec:appendix-proof-index}
The table below records the logical structure of the paper: each claim is paired with the section containing its proof or justification, and the key mathematical ingredients used.
This is designed to make audits fast: each claim should have a single canonical proof location.

\begin{center}
\begin{tabular}{p{0.36\textwidth} p{0.28\textwidth} p{0.28\textwidth}}
\hline
\textbf{Paper statement} & \textbf{Proof location} & \textbf{Key ingredients} \\
\hline
Phase gauge \(\sim_{\mathrm{phase}}\) is an equivalence relation &
Lemma~\ref{lem:phase-equiv} &
Unit-modulus closure under multiplication and inversion \\
\hline
Gauge equivalence \(\sim_{\mathrm{gauge}}\) is an equivalence relation &
Lemma~\ref{lem:gauge-equiv} &
Conjunction of equality conditions and phase equivalence \\
\hline
\(\mathsf{TauModuloGauge}\) is a well-defined quotient set &
Section~\ref{subsec:proofs-quotient} &
Lemma~\ref{lem:gauge-equiv} \\
\hline
\(\varphi\)-level forcing (Proposition~\ref{prop:phi-forcing}) &
Section~\ref{subsec:main-phi} &
Lemma~\ref{lem:finite-argmin} (argmin on finite sets) \\
\hline
\(\tau\)-invariance up to gauge (Proposition~\ref{prop:tau-gauge}) &
Section~\ref{subsec:main-tau} &
Remark~\ref{rem:tau-phase-assumption} and Definition~\ref{def:gaugeeq-sig}; DFT8 justification in Section~\ref{subsec:proofs-dft} \\
\hline
Window-level phase invariance of \(\mathsf{Meaning}\) (Lemma~\ref{lem:Meaning-phase}) &
Section~\ref{subsec:interface-meaning-map} &
Remark~\ref{rem:forcedMode-phase} and Definition~\eqref{eq:meaning-def} \\
\hline
Independence of \(\tau_0\) representative (Corollary~\ref{cor:tau-rep-independent}) &
Section~\ref{subsec:main-tau} &
Immediate from Proposition~\ref{prop:tau-gauge} \\
\hline
Main theorem: meaning is forced up to gauge (Theorem~\ref{thm:meaning-forced-up-to-gauge}) &
Section~\ref{subsec:main-forced} &
Propositions~\ref{prop:phi-forcing} and~\ref{prop:tau-gauge} \\
\hline
\end{tabular}
\end{center}

\paragraph{Audit checklist.}
A reviewer can verify the paper by checking:
\begin{enumerate}
  \item Definition~\ref{def:phase-eq} correctly defines an equivalence relation (Lemma~\ref{lem:phase-equiv});
  \item Definition~\ref{def:gaugeeq-sig} correctly defines an equivalence relation (Lemma~\ref{lem:gauge-equiv});
  \item The argmin on \(\mathsf{PhiLevel}\) exists and is well-defined (Lemma~\ref{lem:finite-argmin});
  \item The DFT8 basis vectors for the self-conjugate family differ by a unit-modulus phase (Section~\ref{subsec:proofs-dft});
  \item The main theorem follows logically from the above (Section~\ref{subsec:proofs-main}).
  \item (Optional) Window-level phase gauge invariance of \(\mathsf{Meaning}\) follows from the classifier property in Remark~\ref{rem:forcedMode-phase} (Lemma~\ref{lem:Meaning-phase}).
\end{enumerate}

\section*{References}
\begin{thebibliography}{99}

\bibitem{RG2026}
J.~Washburn, M.~Zlatanovi\'c, and E.~Allahyarov.
\newblock \emph{Recognition Geometry}.
\newblock \emph{Axioms}, accepted (2026). Preprint source in this repository: \texttt{axioms-4100358.tex}.

\bibitem{RCC2026}
J.~Washburn and A.~Rahnamai Barghi.
\newblock \emph{Reciprocal Convex Costs for Ratio Matching: Functional-Equation Characterization and Decision Geometry}.
\newblock Accepted (2026). Preprint PDF in this repository: \texttt{pre-view-arxiv-submission.pdf}.

\bibitem{SGP2026}
J.~Washburn.
\newblock \emph{Optimization-Based Reference: A Cost-Theoretic Resolution of the Symbol Grounding Problem}.
\newblock 2026. Preprint source in this repository: \texttt{Optimization\_Based\_Reference\_Symbol\_Grounding.tex}.

\bibitem{CPM2026}
J.~Washburn.
\newblock \emph{CPM Method Closure: A Domain-Agnostic Certificate for Coercivity and Aggregation}.
\newblock 2026. Preprint source in this repository: \texttt{CPM\_Method\_Closure.tex}.

\end{thebibliography}

\end{document}

