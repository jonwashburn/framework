\documentclass[11pt,twocolumn]{article}

% --- Packages ---
\usepackage[utf8]{inputenc}
\usepackage[T1]{fontenc}
\usepackage{amsmath,amssymb,amsthm}
\usepackage{mathtools}
\usepackage{graphicx}
\usepackage{booktabs}
\usepackage{hyperref}
\usepackage{geometry}
\usepackage{xcolor}
\usepackage{enumitem}

\geometry{margin=1in}

% --- Theorem environments ---
\newtheorem{definition}{Definition}[section]
\newtheorem{theorem}{Theorem}[section]
\newtheorem{lemma}[theorem]{Lemma}
\newtheorem{corollary}[theorem]{Corollary}
\newtheorem{proposition}[theorem]{Proposition}
\newtheorem{remark}{Remark}[section]

% --- Custom commands ---
\newcommand{\R}{\mathbb{R}}
\newcommand{\Z}{\mathbb{Z}}
\newcommand{\N}{\mathbb{N}}
\newcommand{\phival}{\varphi}

% --- Title ---
\title{%
  Solid-State Virtual Rotor Systems:\\
  Architecture, Control Surfaces, and Metrology\\
  for High-Q Rotating-Field Experiments%
}

\author{%
  Recognition Science Research Institute\\
  \texttt{[correspondence address]}
}

\date{February 1, 2026}

\begin{document}

\maketitle

% ============================================================================
% ABSTRACT
% ============================================================================
\begin{abstract}
Any claimed anomalous effect in rotating-field experiments will be correctly dismissed unless experiments are instrumented to reject thermal, vibrational, and electromagnetic interference (EMI) artifacts. This paper specifies a modular solid-state architecture---the ``Virtual Rotor''---comprising stationary coils pulsed in sequence to create a rotating magnetic field without moving parts. We present the mathematical formalization of the virtual rotor geometry, the control surfaces for resonance locking, and a comprehensive metrology stack including force measurement, thermal mapping, EMI characterization, and vibration monitoring.

The paper defines a complete null-test suite for artifact rejection, an experimental runbook, and a data-analysis pipeline that produces publication-grade figures regardless of whether results are positive or null. The framework is designed to survive skeptical peer review by pre-registering acceptance criteria and uncertainty quantification methods.

\textbf{Keywords:} phased array, rotating magnetic field, commutation, EMI hardening, null tests, metrology, resonance locking, uncertainty quantification
\end{abstract}

% ============================================================================
% 1. INTRODUCTION
% ============================================================================
\section{Introduction}
\label{sec:intro}

\subsection{Motivation}

The history of unconventional propulsion claims is littered with experiments that failed to adequately control for mundane effects: thermal convection, vibration coupling, electromagnetic pickup, and measurement drift. Any future experiment must be designed from the ground up to survive skeptical review.

This paper provides the engineering specification and metrology protocol for a class of experiments using \emph{solid-state virtual rotors}---phased arrays of stationary coils that create rotating magnetic fields without moving mass. The virtual rotor concept offers several advantages over mechanical rotors:

\begin{itemize}
    \item No mechanical wear or failure modes
    \item Precise electronic control of frequency and phase
    \item Instantaneous reversal capability for sign-flip tests
    \item Ability to achieve effective ``rotation speeds'' exceeding mechanical limits
\end{itemize}

\subsection{Scope Boundaries}

This paper is a \textbf{methods, apparatus, and measurement} specification. It does not report experimental results or claim any effect magnitude. Claims of effect are explicitly \emph{data-gated} and belong in separate results publications.

This paper connects to:
\begin{itemize}
    \item \textbf{NM-0 (Foundation)}: Defines the $\phival$-log-spiral geometry and 8-tick scheduling discipline used here.
    \item \textbf{NM-2 (Power and Safety)}: Defines generator-mode operation and safety governors referenced in the safety integration section.
\end{itemize}

\subsection{System Overview}

The complete experimental system comprises six subsystems (Figure~\ref{fig:block}):

\begin{enumerate}
    \item \textbf{Geometry}: $\phival$-spiral coil layout (NM-0)
    \item \textbf{Scheduling}: 8-tick neutral drive patterns (NM-0)
    \item \textbf{Driver/Power Stage}: High-speed switching electronics
    \item \textbf{Field Generation}: Virtual rotor phased array
    \item \textbf{Sensors}: Force, thermal, EMI, vibration
    \item \textbf{Safety Governor}: Detune/shutdown control (NM-2)
\end{enumerate}

\begin{figure}[h]
\centering
\fbox{\parbox{0.9\columnwidth}{
\textbf{Block Diagram (Conceptual)}\\[0.5em]
\texttt{Geometry (NM-0)} $\rightarrow$ \texttt{Schedule (NM-0)}\\
\hspace*{2em}$\downarrow$\\
\texttt{Driver Stage} $\rightarrow$ \texttt{Coil Array} $\rightarrow$ \texttt{Field}\\
\hspace*{2em}$\downarrow$\\
\texttt{Sensors} $\rightarrow$ \texttt{Data Logging}\\
\hspace*{2em}$\downarrow$\\
\texttt{Governor (NM-2)} $\leftarrow$ \texttt{Safety Interlocks}
}}
\caption{System block diagram showing subsystem interconnections.}
\label{fig:block}
\end{figure}

% ============================================================================
% 2. SYSTEM ARCHITECTURE
% ============================================================================
\section{System Architecture}
\label{sec:architecture}

\subsection{Virtual Rotor: Mathematical Formalization}

The virtual rotor is a phased array of stationary electromagnetic coils positioned according to a $\phival$-log-spiral and pulsed in sequence to create a rotating field pattern.

\begin{definition}[Coil Element]
\label{def:coil}
A coil element $C_i$ is characterized by a tuple:
\begin{equation}
    C_i = (i, \theta_i, r_i, \psi_i)
\end{equation}
where:
\begin{itemize}
    \item $i \in \{0, 1, \ldots, n-1\}$ is the coil index
    \item $\theta_i \in [0, 2\pi)$ is the angular position
    \item $r_i > 0$ is the radial position
    \item $\psi_i \in \{0, 1, \ldots, 7\}$ is the phase offset (8-tick)
\end{itemize}
\end{definition}

\begin{definition}[Spiral Array]
\label{def:spiralarray}
For $n$ coils, base radius $r_0 > 0$, and pitch parameter $\kappa \in \Z$, the \emph{spiral array} is the set of coils:
\begin{equation}
    \mathcal{A}(n, r_0, \kappa) = \{C_i : i = 0, \ldots, n-1\}
\end{equation}
where each coil is positioned at:
\begin{align}
    \theta_i &= \frac{2\pi i}{n} \label{eq:theta}\\
    r_i &= r_0 \cdot \phival^{\,\kappa \theta_i / (2\pi)} = r_0 \cdot \phival^{\,\kappa i / n} \label{eq:radius}\\
    \psi_i &= i \mod 8 \label{eq:phase}
\end{align}
\end{definition}

\noindent The phase assignment~\eqref{eq:phase} ensures that the 8-tick neutrality constraint from NM-0 is automatically satisfied when all coils are driven in their assigned phase slots.

\begin{definition}[Field Velocity]
\label{def:fieldvel}
The effective ``velocity'' of the rotating field pattern is:
\begin{equation}
    v_{\text{field}} = \frac{2\pi r}{T_{\text{period}}} = \frac{2\pi r}{n \cdot \tau_{\text{pulse}}}
    \label{eq:fieldvel}
\end{equation}
where $r$ is a characteristic radius, $n$ is the number of coils, and $\tau_{\text{pulse}}$ is the pulse width for each coil.
\end{definition}

\begin{remark}[Relativistic Field Speeds]
If $\tau_{\text{pulse}}$ is in the nanosecond range (MHz switching), the field velocity~\eqref{eq:fieldvel} can approach a significant fraction of the speed of light $c$. For example, with $r = 0.1$ m, $n = 8$ coils, and $\tau_{\text{pulse}} = 1$ ns:
\begin{equation}
    v_{\text{field}} = \frac{2\pi \times 0.1}{8 \times 10^{-9}} \approx 7.85 \times 10^7 \text{ m/s} \approx 0.26c.
\end{equation}
\end{remark}

\subsection{Physical Embodiments}

The virtual rotor can be implemented in several physical forms:

\begin{enumerate}
    \item \textbf{Multi-layer PCB traces}: Spiral traces on a printed circuit board, with each layer carrying one phase.
    \item \textbf{Wound coils on substrate}: Discrete inductors wound on a non-conductive substrate at the positions specified by~\eqref{eq:theta}--\eqref{eq:radius}.
    \item \textbf{Hybrid}: PCB traces for lower-power signals with discrete inductors for high-current phases.
\end{enumerate}

\subsection{Driver/Power Stage Interface}

The driver stage must meet the following interface requirements:

\begin{table}[h]
\centering
\caption{Driver Stage Interface Requirements}
\label{tab:driver}
\begin{tabular}{@{}lll@{}}
\toprule
Parameter & Requirement & Notes \\
\midrule
Supply voltage & [Configurable] & Depends on coil design \\
Max switching rate & $\geq 10$ MHz & For high $v_{\text{field}}$ \\
Rise/fall time & $< 0.1 \times \tau_{\text{pulse}}$ & For clean edges \\
Current sensing & Per-channel & Overcurrent protection \\
Jitter (RMS) & $< 0.01 \times \tau_{\text{pulse}}$ & Coherence requirement \\
\bottomrule
\end{tabular}
\end{table}

\subsection{Timing and Clock Integrity}

\begin{definition}[Coherence Requirement]
\label{def:coherence}
The pulse sequence satisfies the \emph{coherence requirement} if the drive signal $w : \N \to \{-1, +1\}$ satisfies 8-gate neutrality:
\begin{equation}
    \sum_{k=0}^{7} w(t_0 + k) = 0 \quad \forall\, t_0 \in \N.
    \label{eq:coherence}
\end{equation}
\end{definition}

\noindent Clock integrity requirements:
\begin{itemize}
    \item Single master clock with distribution network
    \item Matched path lengths to each driver channel
    \item Jitter measurement logged for every run
    \item Deterministic replay: ability to execute identical schedules across runs
\end{itemize}

\subsection{Shielding and Mechanical Integration}

\textbf{EMI Shielding:}
\begin{itemize}
    \item Faraday enclosure around the virtual rotor
    \item Shielded cables with proper grounding
    \item Common-mode chokes on power and signal lines
\end{itemize}

\textbf{Mechanical Integration:}
\begin{itemize}
    \item Vibration-isolated mounting platform
    \item Symmetric mass distribution
    \item Vacuum-compatible materials if chamber testing is planned
\end{itemize}

% ============================================================================
% 3. CONTROL SURFACES
% ============================================================================
\section{Control Surfaces}
\label{sec:control}

\subsection{Schedule Generation}

The control software implements the 8-tick scheduling discipline from NM-0:

\begin{enumerate}
    \item \textbf{Schedule families}: Parameterized by $(f_{\text{drive}}, \phi_{\text{offset}}, \text{direction})$.
    \item \textbf{Phase reversal}: First-class operation for sign-flip tests---swap the drive direction instantaneously.
    \item \textbf{Parameter encoding}: Each run is tagged with a unique schedule ID and full parameter snapshot.
\end{enumerate}

\subsection{Resonance Lock Controller}

The resonance lock controller operates in two modes:

\textbf{Open-Loop Mode (Sweeps):}
\begin{enumerate}
    \item Coarse sweep: Step through candidate frequencies from the resonance map (NM-0).
    \item Fine sweep: High-resolution scan around peaks identified in coarse sweep.
    \item Dwell time: Configurable hold at each frequency for sensor integration.
\end{enumerate}

\textbf{Closed-Loop Mode (Locking):}
\begin{enumerate}
    \item Monitor a ``resonance score'' (e.g., coherence between drive and sensor signals).
    \item Nudge frequency/phase to maximize the score.
    \item Respect safety limits: maximum power, maximum temperature, governor overrides.
\end{enumerate}

\subsection{Safety Integration Points}

The control system exposes the following interfaces to the safety governor (detailed in NM-2):

\begin{itemize}
    \item \texttt{DETUNE}: Command to shift drive frequency off-resonance (soft stop).
    \item \texttt{SHUTDOWN}: Immediate drive disable (hard stop).
    \item \texttt{LIMITS}: Configurable thresholds for overcurrent, overtemperature, over-vibration.
\end{itemize}

\subsection{Data Logging and Provenance}

Every experimental run must log:
\begin{itemize}
    \item \textbf{Timebase}: Monotonic timestamps synchronized to GPS or NTP.
    \item \textbf{Config snapshot}: Full parameter set, firmware version, git commit hash.
    \item \textbf{Sensor streams}: Raw data at full sampling rate.
    \item \textbf{Checksums}: SHA-256 hash of each data file for tamper evidence.
\end{itemize}

% ============================================================================
% 4. METROLOGY STACK
% ============================================================================
\section{Metrology Stack}
\label{sec:metrology}

\subsection{Force/Thrust Measurement}

\textbf{Primary Modality: Precision Load Cell Platform}

\begin{table}[h]
\centering
\caption{Load Cell Requirements}
\label{tab:loadcell}
\begin{tabular}{@{}ll@{}}
\toprule
Parameter & Requirement \\
\midrule
Resolution & $< 1$ mN \\
Bandwidth & $\geq 100$ Hz \\
Drift (24h) & $< 10$ mN \\
Cross-axis rejection & $> 100:1$ \\
\bottomrule
\end{tabular}
\end{table}

\noindent Calibration procedure:
\begin{enumerate}
    \item Multi-point calibration with NIST-traceable masses.
    \item Temperature coefficient characterization.
    \item Cross-axis sensitivity mapping.
    \item Baseline drift characterization (no drive, 24h).
\end{enumerate}

\textbf{Secondary Modality: Torsion Pendulum}

For higher sensitivity measurements (sub-mN), a torsion pendulum with optical displacement sensing may be used. Requirements:
\begin{itemize}
    \item Period $> 100$ s (for low-frequency sensitivity)
    \item Optical encoder resolution $< 1$ $\mu$rad
    \item Vacuum operation capability
\end{itemize}

\subsection{Thermal Measurement}

\begin{enumerate}
    \item \textbf{Near-field mapping}: Thermistors or RTDs at $\geq 8$ locations around the virtual rotor.
    \item \textbf{IR camera}: If available, with calibrated emissivity and reflection correction.
    \item \textbf{Calorimetric accounting}: For generator-mode tests, measure input power, output power, and heat flow.
\end{enumerate}

\subsection{EMI Measurement}

\begin{enumerate}
    \item \textbf{Near-field probes}: E-field and H-field probes at critical sensor locations.
    \item \textbf{Spectrum snapshots}: Capture EMI spectrum before, during, and after each run.
    \item \textbf{Magnetometer placement}: 3-axis magnetometers to detect stray fields.
\end{enumerate}

\subsection{Vibration and Environmental Sensors}

\begin{enumerate}
    \item \textbf{Accelerometers}: 3-axis on platform and on chamber (if applicable).
    \item \textbf{Pressure gauge}: For vacuum tests.
    \item \textbf{Ambient monitoring}: Temperature, humidity, airflow.
\end{enumerate}

\subsection{Synchronization and Sampling}

\begin{itemize}
    \item All sensors share a common clock or are synchronized to $< 1$ $\mu$s.
    \item Anti-aliasing filters: $f_{\text{cutoff}} < f_{\text{sample}}/2$.
    \item Latency measurement: Characterize and log sensor latencies for each channel.
\end{itemize}

% ============================================================================
% 5. NULL TESTS
% ============================================================================
\section{Null Tests and Controls}
\label{sec:nulltests}

A rigorous experiment must include null tests that would produce \emph{no effect} if the claimed phenomenon is real. The following suite is mandatory.

\subsection{Geometry Controls}

\begin{enumerate}
    \item \textbf{Non-$\phival$ spiral}: Replace $\phival$ with an arbitrary constant (e.g., 1.5) in~\eqref{eq:radius}.
    \item \textbf{Randomized layout}: Coils at random positions (not spiral).
    \item \textbf{Scrambled phase}: Same geometry but random phase assignments $\psi_i$.
\end{enumerate}

\textbf{Prediction:} No effect should appear with control geometries.

\subsection{Scheduling Controls}

\begin{enumerate}
    \item \textbf{Off-resonance}: Drive at frequencies not in the resonance map.
    \item \textbf{Random schedule}: Same duty cycle and power but randomized timing.
    \item \textbf{Phase reversal}: Reverse the drive direction.
\end{enumerate}

\textbf{Prediction:} Off-resonance and random schedules show no effect; phase reversal shows sign flip.

\subsection{Power/Thermal Controls}

\begin{enumerate}
    \item \textbf{Matched heating}: Dissipate the same power through resistors (no rotating field).
    \item \textbf{Thermal gradient injection}: Artificially create thermal gradients to map buoyancy coupling.
\end{enumerate}

\textbf{Prediction:} Matched heating produces no force signal.

\subsection{Vibration Controls}

\begin{enumerate}
    \item \textbf{External shaker}: Inject known vibration and measure force channel response.
    \item \textbf{Correlation analysis}: Compute coherence between accelerometer and force channels.
\end{enumerate}

\textbf{Prediction:} If vibration coupling is controlled, coherence should be low during actual runs.

\subsection{EMI Coupling Controls}

\begin{enumerate}
    \item \textbf{Decoupled drive}: Run drivers into dummy loads (no field generated).
    \item \textbf{EMI injection}: Inject known EMI patterns and measure sensor response.
\end{enumerate}

\textbf{Prediction:} Decoupled drive produces no force signal.

% ============================================================================
% 6. EXPERIMENTAL PROCEDURES
% ============================================================================
\section{Experimental Procedures}
\label{sec:procedures}

\subsection{Pre-Flight Checklist}

Before each experimental session:
\begin{enumerate}
    \item Mechanical inspection: shielding, mounting, connections.
    \item Sensor calibration verification.
    \item Baseline drift measurement (no drive, $\geq 10$ minutes).
    \item Clock synchronization verification.
\end{enumerate}

\subsection{Sweep Protocol}

\begin{enumerate}
    \item \textbf{Coarse sweep}: Step through resonance-map frequencies; dwell 10 s each; log sensor peaks.
    \item \textbf{Fine sweep}: 0.1\% frequency steps around identified peaks; dwell 30 s each.
    \item \textbf{Replication}: Repeat on different days; rotate apparatus 90°; vary ambient temperature.
\end{enumerate}

\subsection{Data Products}

Every run must produce:
\begin{enumerate}
    \item Config snapshot and schedule ID.
    \item Raw sensor streams (full rate).
    \item Processed summary: spectra, coherence plots, drift metrics.
    \item Null-test comparison plots.
\end{enumerate}

% ============================================================================
% 7. DATA ANALYSIS
% ============================================================================
\section{Data Analysis Pipeline}
\label{sec:analysis}

\subsection{Signal Processing}

\begin{enumerate}
    \item \textbf{Spectral analysis}: FFT of force signal; identify peaks at drive harmonics.
    \item \textbf{Lock-in detection}: Demodulate force signal at drive frequency.
    \item \textbf{Coherence}: Compute coherence between drive and force channels.
\end{enumerate}

\subsection{Uncertainty Quantification}

Error bars combine:
\begin{enumerate}
    \item Calibration uncertainty (from calibration curve).
    \item Drift uncertainty (from baseline measurements).
    \item Statistical uncertainty (from repeated measurements).
\end{enumerate}

\textbf{Multiple-hypothesis correction:} When scanning multiple frequencies, apply Bonferroni or Benjamini-Hochberg correction to avoid false discovery.

\subsection{Acceptance Criteria}

\textbf{Pre-registered thresholds:}
\begin{enumerate}
    \item Signal-to-noise ratio $> 5$ at predicted frequency.
    \item Null-test comparison: control runs show SNR $< 2$.
    \item Minimum replication: $\geq 3$ independent runs on different days.
    \item Artifact rejection: EMI, vibration, and thermal coupling all below detection threshold.
\end{enumerate}

% ============================================================================
% 8. SAFETY
% ============================================================================
\section{Safety Considerations}
\label{sec:safety}

\subsection{Electrical Safety}

\begin{itemize}
    \item High-voltage and high-current hazards from driver stage.
    \item Proper insulation, interlocks, and emergency stop.
    \item No exposed conductors during operation.
\end{itemize}

\subsection{Vacuum Safety}

If operating in a vacuum chamber:
\begin{itemize}
    \item Rated viewport and feedthrough seals.
    \item Controlled pump-down and vent procedures.
    \item No personnel in exclusion zone during operation.
\end{itemize}

\subsection{Thermal Safety}

\begin{itemize}
    \item Overtemperature interlocks on all coils.
    \item Thermal runaway detection with automatic shutdown.
\end{itemize}

\subsection{Governor Integration}

Before any high-power testing:
\begin{itemize}
    \item Safety governor (NM-2) must be integrated and tested.
    \item Governor must have independent power and control path.
    \item Governor detune function must be verified to stop resonance within 1 cycle.
\end{itemize}

% ============================================================================
% 9. CONCLUSION
% ============================================================================
\section{Conclusion}
\label{sec:conclusion}

This paper has presented a complete engineering and metrology specification for solid-state virtual rotor experiments. The key contributions are:

\begin{enumerate}
    \item Mathematical formalization of the virtual rotor geometry with $\phival$-spiral positioning and 8-tick phase assignment.
    \item Control surface specifications for schedule generation, resonance locking, and safety integration.
    \item A comprehensive metrology stack covering force, thermal, EMI, and vibration measurements.
    \item A mandatory null-test suite for artifact rejection.
    \item Pre-registered acceptance criteria and uncertainty quantification methods.
\end{enumerate}

This framework is designed to produce publication-grade results regardless of whether the outcome is positive or null. The emphasis on metrology, null tests, and pre-registration ensures that any claimed effect will survive skeptical peer review.

% ============================================================================
% APPENDICES
% ============================================================================
\appendix

\section{Interface Specifications}
\label{app:interface}

\subsection{Telemetry Schema (JSON)}

\begin{verbatim}
{
  "run_id": "NM1-2026-02-01-001",
  "timestamp_utc": "2026-02-01T12:00:00Z",
  "config": {
    "n_coils": 8,
    "r0_mm": 50,
    "kappa": 1,
    "f_drive_hz": 1000,
    "tau_pulse_ns": 1000
  },
  "sensors": {
    "force_mN": [...],
    "temp_C": [...],
    "accel_g": [...],
    "emi_dBm": [...]
  },
  "checksum_sha256": "..."
}
\end{verbatim}

\section{Calibration Worksheets}
\label{app:calibration}

\subsection{Load Cell Calibration}

\begin{enumerate}
    \item Zero: Record output with no load for 60 s.
    \item Span: Apply 5 calibration masses; record output for 30 s each.
    \item Fit: Linear regression to obtain sensitivity (mN/V).
    \item Residual check: All calibration points within $\pm 0.5\%$.
\end{enumerate}

\subsection{Thermistor Calibration}

\begin{enumerate}
    \item Ice bath (0°C) and boiling water (100°C) reference points.
    \item Additional points at 25°C and 50°C.
    \item Steinhart-Hart fit for each thermistor.
\end{enumerate}

\section{Example Plot Templates}
\label{app:plots}

\subsection{Sweep Curve Template}

\begin{verbatim}
X-axis: Frequency (Hz)
Y-axis: Force signal (mN, lock-in amplitude)
Error bars: 1-sigma from 3 replications
Vertical lines: Predicted resonance frequencies
\end{verbatim}

\subsection{Null Test Comparison Template}

\begin{verbatim}
Bar chart: 
  - "Resonant drive" (expected signal)
  - "Off-resonance" (null)
  - "Matched heating" (null)
  - "Random schedule" (null)
Error bars: 1-sigma
Threshold line: Detection limit
\end{verbatim}

% ============================================================================
% REFERENCES
% ============================================================================
\begin{thebibliography}{9}

\bibitem{commutation}
T.~A.~Lipo,
\textit{Introduction to AC Machine Design}.
Wiley, 2017.

\bibitem{emi}
H.~W.~Ott,
\textit{Electromagnetic Compatibility Engineering}.
Wiley, 2009.

\bibitem{metrology}
BIPM,
\textit{Guide to the Expression of Uncertainty in Measurement (GUM)}.
JCGM 100:2008, 2008.

\bibitem{nulltests}
E.~Fischbach and C.~Talmadge,
\textit{The Search for Non-Newtonian Gravity}.
Springer, 1999.

\end{thebibliography}

\end{document}
