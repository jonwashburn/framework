\documentclass[11pt]{article}

% Keep packages minimal for TeX Live "basic" installs.
\usepackage[utf8]{inputenc}
\usepackage[T1]{fontenc}
\usepackage{geometry}
\usepackage{hyperref}
\usepackage{amsmath,amssymb}
\usepackage{graphicx}
\usepackage{booktabs}
\usepackage{xcolor}
\usepackage{enumitem}
\usepackage{array}

\geometry{margin=1in}
\hypersetup{
  colorlinks=true,
  linkcolor=blue,
  urlcolor=blue
}

% ---------------------------------------------------------------------------
% Convenience macros (avoid Unicode Greek in text; use LaTeX math symbols)
% ---------------------------------------------------------------------------
\newcommand{\R}{\mathbb{R}}
\newcommand{\Z}{\mathbb{Z}}
\newcommand{\N}{\mathbb{N}}
\newcommand{\phival}{\varphi}

\newcommand{\PatentTitle}{Solid-State Virtual Rotor Phased Electromagnetic Arrays with Golden-Ratio Logarithmic-Spiral Sampling and Phase-Mapped Commutation}
\newcommand{\Docket}{NTL-PROV-007}
\newcommand{\Inventors}{[Inventor Names]}
\newcommand{\Assignee}{[Assignee / Organization]}
\newcommand{\FilingDate}{February 1, 2026}

\begin{document}

\begin{center}
{\LARGE \textbf{\PatentTitle}}\\[0.75em]
{\large \textbf{Docket:} \Docket}\\[0.25em]
{\large \textbf{Inventors:} \Inventors}\\[0.25em]
{\large \textbf{Assignee:} \Assignee}\\[0.25em]
{\large \textbf{Date:} \FilingDate}\\[0.75em]
\end{center}

\vspace{-0.5em}
\hrule
\vspace{0.75em}

% ===========================================================================
% ABSTRACT (PATENT)
% ===========================================================================
\section*{Abstract}

Disclosed are apparatus, systems, methods, and non-transitory computer-readable media for implementing a \emph{solid-state virtual rotor} using a phased electromagnetic array whose element locations are generated by a golden-ratio (\(\phival\)) logarithmic-spiral scaffold and whose elements are driven according to a phase-mapped commutation pattern. In various embodiments, an array comprises a plurality of electromagnetic elements (coils, traces, inductive segments, or equivalent) indexed by \(i\in\{0,\dots,n-1\}\), positioned at angular locations \(\theta_i\) and radii \(r_i\) derived from
\[
r(\theta)=r_0\cdot \phival^{\kappa\theta/(2\pi)},
\]
where \(r_0>0\) and \(\kappa\in\Z\). Each element is assigned a phase offset \(p_i\) (e.g., \(p_i=i\bmod 8\)) such that sequential or phase-ordered excitation of the elements produces a rotating magnetic field pattern without moving mass. The disclosure further provides multi-layer PCB embodiments, via-stitched or laminated coil embodiments, shielding and ground-plane integration, and mapping of the element index/phase metadata into driver-channel wiring and software configuration. Closed-form relationships are provided for an effective field-pattern angular velocity and field-pattern linear velocity, enabling computation of candidate operating regions and integration with safety envelopes and metrology protocols.

% ===========================================================================
% TECHNICAL FIELD
% ===========================================================================
\section*{Technical Field}

The present disclosure relates to electromagnetic devices and rotating-field synthesis, and more particularly to phased coil arrays and commutation methods configured to emulate a rotor using stationary electromagnetic elements with geometry derived from a golden-ratio logarithmic-spiral scaffold and with explicit phase mapping for commutation.

% ===========================================================================
% BACKGROUND
% ===========================================================================
\section*{Background}

Mechanical rotors and rotating magnets are widely used to generate rotating fields, but such systems are limited by bearing losses, vibration, mechanical fatigue, and maximum safe RPM. Solid-state phased electromagnetic arrays can synthesize rotating fields without moving mass, but practical implementations typically rely on simple circular or evenly spaced layouts and ad hoc phase mapping.

In applications where the geometry of the field generator is treated as a primary design parameter and where reproducibility across scales and builds is important, there is a need for a compact, deterministic method to generate element placements and to define a phase mapping that is compatible with a multi-phase commutation schedule.

Accordingly, the present disclosure provides a virtual rotor architecture in which (i) the element locations are derived from a golden-ratio logarithmic spiral and (ii) phase mapping is tied to element index and/or angular position, enabling deterministic generation, fabrication, wiring, and control.

% ===========================================================================
% SUMMARY
% ===========================================================================
\section*{Summary}

This disclosure provides a solid-state virtual rotor as a phased electromagnetic array defined by:
\begin{itemize}[leftmargin=*]
  \item a spatial scaffold (golden-ratio logarithmic spiral) for element placement;
  \item a discrete sampling rule to convert the continuous scaffold into \(n\) element locations;
  \item a phase mapping \(p_i\) assigning each element to a commutation phase group;
  \item an excitation/commutation method that produces a rotating field pattern.
\end{itemize}

In one aspect, the array is fabricated as a multi-layer printed circuit board, with planar traces, via stitching, and shielding layers. In another aspect, the array is fabricated as discrete coils mounted on a substrate. In another aspect, multiple stacked arrays are used to form a three-dimensional field generator.

% ===========================================================================
% BRIEF DESCRIPTION OF DRAWINGS
% ===========================================================================
\section*{Brief Description of the Drawings}

Drawings may be provided later. For purposes of this specification:
\begin{itemize}[leftmargin=*]
  \item \textbf{FIG. 1} depicts a virtual rotor array with element positions sampled from a golden-ratio logarithmic spiral.
  \item \textbf{FIG. 2} depicts phase mapping \(p_i=i\bmod P\) with \(P=8\) and phase-ordered excitation.
  \item \textbf{FIG. 3} depicts a multi-layer PCB embodiment including coil traces, via stitching, and shielding planes.
  \item \textbf{FIG. 4} depicts alternative embodiments including multi-arm and stacked-layer arrays.
  \item \textbf{FIG. 5} depicts an example software/hardware mapping from element indices to driver channels and configuration files.
  \item \textbf{FIG. 6} depicts an effective field-pattern velocity model and derived operating parameters.
\end{itemize}

% ===========================================================================
% DEFINITIONS
% ===========================================================================
\section*{Definitions and Notation}

Unless otherwise indicated:
\begin{itemize}[leftmargin=*]
  \item \(\phival=(1+\sqrt{5})/2\) is the golden ratio.
  \item \(r_0>0\) is a base radius.
  \item \(\kappa\in\Z\) is an integer pitch-family parameter.
  \item \(n\in\N\) is the number of array elements.
  \item \(P\in\N\) is the number of commutation phases (e.g., \(P=8\)).
  \item An \emph{electromagnetic element} is any field-generating structure including a coil, trace loop, inductor segment, or equivalent.
  \item A \emph{phase mapping} is a function \(p:\{0,\dots,n-1\}\to\{0,\dots,P-1\}\) assigning an element to a phase.
  \item A \emph{commutation cycle} refers to a repeated excitation pattern across phases and/or elements that produces an apparent rotation of the field.
\end{itemize}

% ===========================================================================
% DETAILED DESCRIPTION
% ===========================================================================
\section*{Detailed Description}

\subsection*{1. Virtual Rotor Array Overview}

In one embodiment, a virtual rotor system comprises:
\begin{itemize}[leftmargin=*]
  \item an array of \(n\) electromagnetic elements arranged in a plane or volume;
  \item a mapping from element index to a geometric location \((x_i,y_i,z_i)\);
  \item a phase mapping \(p_i\) assigning each element to one of \(P\) phases;
  \item one or more driver stages configured to energize elements according to a commutation schedule.
\end{itemize}

The elements are stationary; the rotating-field effect is synthesized electronically.

\subsection*{2. Golden-Ratio Logarithmic-Spiral Sampling for Element Placement}

\paragraph{2.1 Base spiral scaffold.}
In one embodiment, a spiral scaffold is defined by:
\begin{equation}
  r(\theta) = r_0 \cdot \phival^{\kappa\theta/(2\pi)}.
  \label{eq:spiral}
\end{equation}

\paragraph{2.2 Sampling rule (uniform in angle; non-limiting).}
In one embodiment, element indices \(i=0,\dots,n-1\) map to angles:
\begin{equation}
  \theta_i = \theta_{\text{start}} + \frac{2\pi i}{n}.
  \label{eq:theta_i}
\end{equation}
Then:
\begin{equation}
  r_i = r(\theta_i) = r_0 \cdot \phival^{\kappa\theta_i/(2\pi)}.
  \label{eq:r_i}
\end{equation}
Convert to Cartesian coordinates (planar embodiment):
\begin{equation}
  (x_i,y_i) = (r_i\cos\theta_i,\; r_i\sin\theta_i).
  \label{eq:xy_i}
\end{equation}

\paragraph{2.3 Alternative sampling rules (non-limiting).}
Sampling may be uniform in \(\theta\), uniform in arc length, uniform in chord length, or adaptive to manufacturing constraints (minimum spacing, curvature). The mapping may also include clipping to a boundary or annulus.

\paragraph{2.4 Multi-arm and multi-layer placement (optional).}
An array may comprise multiple spiral arms or multiple layers. For arm \(j\) with angular offset \(\alpha_j\):
\[
r_{j}(\theta) = r_0\cdot \phival^{\kappa(\theta+\alpha_j)/(2\pi)}.
\]
For layer \(\ell\), parameters \((r_{0,\ell},\kappa_\ell,\theta_{\text{start},\ell})\) may differ.

\subsection*{3. Phase Mapping and Commutation}

\paragraph{3.1 Phase mapping definition.}
Define a phase mapping:
\[
p_i = p(i)\in\{0,\dots,P-1\}.
\]
In one embodiment:
\begin{equation}
  p_i = i \bmod P.
  \label{eq:phase_mod}
\end{equation}
This assigns each successive element to successive phases, repeating every \(P\) elements.

\paragraph{3.2 Phase groups.}
Define the set of elements in phase \(q\) as:
\[
\mathcal{I}_q := \{\, i \in \{0,\dots,n-1\} : p_i=q \,\}.
\]
In one embodiment, the driver energizes elements phase-by-phase (or interleaves phases) according to a schedule.

\paragraph{3.3 Rotating-field synthesis (conceptual).}
When phases are energized in a phase-ordered sequence, the locus of peak field magnitude moves around the array, producing an apparent rotation. Direction reversal may be implemented by reversing phase order (e.g., \(q\to (P-1-q)\)).

\paragraph{3.4 Variants of phase mapping (non-limiting).}
Phase mapping may be a function of element angle rather than index (e.g., \(p_i=\lfloor P\theta_i/(2\pi)\rfloor\)), may include permutation, may include skipping or repeated phases, and may include multi-level phase weights for amplitude shaping.

\subsection*{4. Effective Field-Pattern Velocity Models}

\paragraph{4.1 Commutation-cycle time and angular velocity.}
Let \(T_{\text{cycle}}\) denote the time for one full cycle of the commutation pattern that corresponds to one full rotation of the field pattern. Define:
\[
\omega_{\text{field}} := \frac{2\pi}{T_{\text{cycle}}}.
\]

\paragraph{4.2 Sequential element activation (non-limiting).}
If elements are activated sequentially with per-element pulse width \(\tau\) and one activation corresponds to advancing the field pattern by one element step, then:
\[
T_{\text{cycle}} = n\tau,
\quad
\omega_{\text{field}} = \frac{2\pi}{n\tau}.
\]
Define an effective linear velocity at radius \(r\):
\begin{equation}
  v_{\text{field}}(r) = r\,\omega_{\text{field}} = \frac{2\pi r}{n\tau}.
  \label{eq:vfield}
\end{equation}

\paragraph{4.3 Phase-group activation (non-limiting).}
If phases are energized as groups (e.g., 8 phases) with per-phase dwell \(\tau_p\), then \(T_{\text{cycle}}=P\tau_p\) and \(v_{\text{field}}(r)=2\pi r/(P\tau_p)\). Hybrid schedules may be represented by an effective cycle time derived from the schedule.

\subsection*{5. Physical Embodiments}

\paragraph{5.1 Multi-layer PCB embodiment.}
In one embodiment, the array is implemented as a multi-layer printed circuit board comprising:
\begin{itemize}[leftmargin=*]
  \item one or more copper layers containing spiral-sampled conductive traces forming inductive loops;
  \item via stitching to create vertical current paths and increase effective inductance;
  \item ground planes and shielding layers to reduce EMI and control return paths;
  \item connector pads mapping element indices and phase groups to driver harnesses.
\end{itemize}

\paragraph{5.2 Discrete coil embodiment.}
In one embodiment, discrete coils are mounted on a substrate at sampled positions \((x_i,y_i)\). Coils may be air-core, ferrite-core, or composite-core. Coils may be potted, mechanically constrained, or embedded.

\paragraph{5.3 Hybrid embodiment.}
In one embodiment, planar traces provide low-inductance elements and discrete inductors provide high-inductance elements, with mixed placement derived from the same spiral scaffold.

\paragraph{5.4 Three-dimensional array embodiment.}
In one embodiment, multiple planar arrays are stacked to form a volumetric field generator, optionally with alternating layer orientations and shielding layers between stacks.

\subsection*{6. Index/Phase Metadata and Wiring}

In one embodiment, each element \(i\) has associated metadata:
\[
\mathcal{M}_i = (i,\theta_i,r_i,x_i,y_i,p_i,\text{layer},\text{net name}).
\]
The metadata is used to:
\begin{itemize}[leftmargin=*]
  \item generate fabrication files (CAD/EDA);
  \item generate wiring harness tables and connector pinouts;
  \item generate firmware configuration mapping driver channels to phase groups;
  \item generate validation/inspection checklists.
\end{itemize}

% ===========================================================================
% CLAIMS (DRAFT / PROVISIONAL-STYLE)
% ===========================================================================
\section*{Claims (Draft)}

\textbf{Note:} The following claims are an initial, non-limiting claim set intended to preserve multiple fallback positions. Final claim strategy should be reviewed by counsel.

\subsection*{Independent Claims}

\begin{enumerate}[leftmargin=*]
  \item \textbf{(Apparatus)} A solid-state rotating-field generator comprising: a plurality of electromagnetic elements indexed by \(i\in\{0,\dots,n-1\}\); a placement rule configured to position the electromagnetic elements at locations derived from a golden-ratio logarithmic spiral radius profile \(r(\theta)=r_0\phival^{\kappa\theta/(2\pi)}\); and a phase mapping that assigns each electromagnetic element to one of \(P\) commutation phases such that sequential excitation of phases synthesizes a rotating magnetic field.

  \item \textbf{(Method)} A method of synthesizing a rotating magnetic field without moving mass, the method comprising: generating a set of electromagnetic element locations by sampling a golden-ratio logarithmic spiral scaffold; assigning each element a phase offset according to a phase mapping; and driving the electromagnetic elements according to a commutation schedule that energizes phases in an ordered sequence to produce a rotating field pattern.

  \item \textbf{(Non-transitory medium)} A non-transitory computer-readable medium storing instructions that, when executed by one or more processors, cause the one or more processors to: receive parameters \(r_0\), \(\kappa\), \(n\), and \(P\); compute element angles \(\theta_i=\theta_{\text{start}}+2\pi i/n\); compute radii \(r_i=r_0\phival^{\kappa\theta_i/(2\pi)}\); compute Cartesian coordinates \((x_i,y_i)\); assign phases \(p_i=i\bmod P\); and output at least one of a fabrication file or a configuration artifact encoding the element locations and phase mapping.
\end{enumerate}

\subsection*{Dependent Claims (Examples; Non-Limiting)}

\begin{enumerate}[leftmargin=*]
  \setcounter{enumi}{3}
  \item The apparatus of claim 1, wherein \(P=8\).
  \item The apparatus of claim 1, wherein the phase mapping is \(p_i=i\bmod P\).
  \item The apparatus of claim 1, wherein the plurality of electromagnetic elements comprise planar conductive traces on a multi-layer printed circuit board.
  \item The apparatus of claim 6, further comprising via stitching coupling conductive traces across layers.
  \item The apparatus of claim 1, further comprising shielding layers and ground planes configured to control return-current paths.
  \item The method of claim 2, further comprising reversing a rotation direction by reversing a phase order.
  \item The method of claim 2, further comprising computing an effective field-pattern velocity \(v_{\text{field}}(r)=2\pi r/(n\tau)\) for a pulse width \(\tau\).
  \item The apparatus of claim 1, wherein the electromagnetic elements are arranged in multiple stacked planes to form a three-dimensional field generator.
  \item The non-transitory medium of claim 3, wherein outputting comprises outputting a connector pinout mapping element indices and phases to driver channels.
  \item The non-transitory medium of claim 3, wherein assigning phases comprises assigning phases based on \(\theta_i\) rather than \(i\).
  \item The apparatus of claim 1, wherein the placement rule comprises a multi-arm spiral having multiple angular offsets.
\end{enumerate}

% ===========================================================================
% FALLBACK POSITIONS / ADDITIONAL EMBODIMENTS
% ===========================================================================
\section*{Additional Embodiments and Fallback Positions (Non-Limiting)}

\begin{itemize}[leftmargin=*]
  \item Electromagnetic elements may be coils, inductors, segmented loops, conductive traces, or any combination thereof.
  \item The placement rule may be clipped to annular regions or arbitrary boundaries and may be adaptively resampled to satisfy spacing and manufacturability constraints.
  \item Phase mapping may include permutations, phase skipping, or weighted phases to shape the field pattern.
  \item Multiple arrays may be stacked and driven with inter-layer phase offsets to shape a three-dimensional rotating field.
  \item The system may store a canonical mapping of element IDs, positions, phases, and layers to support deterministic rebuilds and configuration provenance.
\end{itemize}

\vspace{1em}
\hrule
\vspace{0.75em}
\noindent \textbf{End of Specification (Draft)}

\end{document}

