\documentclass[11pt]{article}

% Packages
\usepackage[utf8]{inputenc}
\usepackage[T1]{fontenc}
\usepackage{geometry}
\usepackage{hyperref}
\usepackage{amsmath,amssymb}
\usepackage{graphicx}
\usepackage{booktabs}
\usepackage{xcolor}
\usepackage{enumitem}
\usepackage{fancyhdr}
\usepackage{lineno}

% Geometry
\geometry{margin=1in}

% Hyperref setup
\hypersetup{
  colorlinks=true,
  linkcolor=darkblue,
  urlcolor=darkblue,
  citecolor=darkblue
}
\definecolor{darkblue}{rgb}{0,0,0.5}

% Header/Footer
\pagestyle{fancy}
\fancyhf{}
\rhead{\textbf{Project Nautilus} | NTL-PROV-030}
\lhead{Entropic Heat Harvesting Propulsion}
\cfoot{\thepage}
\setlength{\headheight}{14pt}
\addtolength{\topmargin}{-2pt}

% Line numbering for legal review
\linenumbers

% Title
\title{\textbf{PROVISIONAL PATENT APPLICATION}\\
\large \textbf{Propulsion System Utilizing Entropic Heat Harvesting for Self-Sustaining High-Velocity Flight}}
\author{Project Nautilus Engineering Team}
\date{February 2, 2026}

\begin{document}

\maketitle

\begin{abstract}
A propulsion system and thermodynamic cycle that utilizes environmental friction heat as a primary energy source for high-velocity travel. Unlike conventional aerospace vehicles where aerodynamic heating is a parasitic load requiring dissipation, this system actively absorbs thermal energy (phonons) from the vehicle's boundary layer and converts it into coherent metric work (thrust) via an "Entropic Pump" mechanism. The system utilizes a metamaterial hull coupled to a resonant electromagnetic drive core to rectify thermal noise into ordered field energy. This results in a vehicle that exhibits a negative thermal signature ("cold wake") and achieves higher efficiency at higher velocities, effectively using the chaotic motion of the medium as fuel.
\end{abstract}

\tableofcontents
\newpage

\section{Background of the Invention}

\subsection{Field of the Invention}
The present invention relates to thermodynamic cycles for propulsion, specifically to systems that harvest waste heat from aerodynamic or hydrodynamic friction to power electromagnetic drive systems.

\subsection{Description of Related Art}
High-speed travel through an atmosphere creates intense heat due to friction and compression.
\begin{itemize}
    \item \textbf{The Heat Barrier:} For aircraft exceeding Mach 3, the skin temperature can melt standard materials. This necessitates heavy thermal protection systems (TPS) like ablative tiles or active cooling loops that dump heat into fuel.
    \item \textbf{Energy Loss:} This heat represents a massive loss of kinetic energy. The vehicle pushes the air, the air gets hot, and that energy is dissipated into the wake.
\end{itemize}

Thermodynamic laws typically forbid converting ambient heat completely into work (Second Law). However, if the system can export entropy to a different reservoir (e.g., the vacuum metric itself) or operates as an open system ordering a flow, local entropy reduction is possible.

There is a need for a propulsion system that can turn the "heat barrier" into a "power source," enabling sustained hypersonic flight without thermal limits.

\section{Summary of the Invention}

The present invention provides an "Entropic Heat Harvesting" propulsion system.

It is based on the Recognition Science principle that the "Vacuum Pump" drive (NTL-PROV-013) is fundamentally an ordering device. To create a coherent metric field (low entropy), the device must shed entropy.
\begin{itemize}
    \item \textbf{Conventional Operation:} The device sheds entropy by heating up (ohmic losses).
    \item \textbf{Harvesting Operation:} The device is coupled thermally to the hot skin of the vehicle. It absorbs the high-entropy thermal phonons from the skin and uses the drive field to "sort" them, converting their energy into the coherent rotation of the virtual rotor.
\end{itemize}

The result is a regenerative cycle:
\begin{enumerate}
    \item The vehicle moves fast, creating friction heat.
    \item The drive absorbs the heat to power the metric field.
    \item The intensified field increases thrust and reduces drag (via the Superfluid Sheath, NTL-PROV-029).
    \item The vehicle flies faster, generating more heat (fuel).
\end{enumerate}

The external signature of this vehicle is a "Cold Wake"—the air or water left behind is colder than the ambient temperature because its thermal energy has been extracted.

\section{Detailed Description of the Invention}

\subsection{Theoretical Basis (The Entropic Diode)}
The core innovation is the treatment of heat not as "waste" but as "disordered information."
\begin{itemize}
    \item \textbf{Mechanism:} The drive's $\phi$-spiral field imposes a strict geometric order on the local region.
    \item \textbf{Rectification:} Thermal vibrations (phonons) in the hull material are coupled to the magnetic flux lines of the drive. The strong, coherent magnetic field dampens the random thermal vibrations, effectively "cooling" the lattice, and absorbs that energy into the field's stored potential.
    \item \textbf{Thermodynamics:} The entropy is not destroyed; it is exported to the metric field gradient (the "vacuum wake"), satisfying the Second Law.
\end{itemize}

\subsection{System Architecture}

\subsubsection{1. Thermal-Electric Metamaterial Hull}
The vehicle skin is constructed from a composite material designed to conduct phonons inward while blocking external radiation.
\begin{itemize}
    \item \textbf{Structure:} A layered dielectric/conductor stack that acts as a "phonon waveguide," funneling thermal energy from the surface to the drive core.
    \item \textbf{Seebeck Effect:} Integrated thermoelectric layers provide immediate voltage conversion for auxiliary systems, but the primary harvesting is direct phonon-magnon coupling.
\end{itemize}

\subsubsection{2. The Regenerative Drive Controller}
The drive controller (NTL-PROV-009) is modified to include a "Thermal Regeneration Mode."
\begin{itemize}
    \item \textbf{Logic:} As skin temperature rises, the controller increases the drive frequency and coupling efficiency.
    \item \textbf{Feedback:} The system monitors the temperature gradient across the hull. It adjusts the drive phase to maximize the heat flux *into* the core, effectively actively refrigerating the skin.
\end{itemize}

\subsubsection{3. The Heat Sink Buffer}
To start the cycle (before friction heat is available), the system requires an initial energy source.
\begin{itemize}
    \item \textbf{Start-Up:} An onboard battery or capacitor bank powers the initial acceleration.
    \item \textbf{Transition:} Once sufficient velocity (and friction) is achieved, the system blends in the harvested thermal energy, reducing draw from the battery. At cruise velocity, the system can become self-sustaining (ram-scoop mode).
\end{itemize}

\subsection{Operational Modes}

\subsubsection{Mode A: Cold Start}
Powered by internal storage. The hull is at ambient temperature.

\subsubsection{Mode B: Hypersonic Cruise}
The vehicle travels at Mach 5+. Skin friction generates plasma. The drive absorbs this energy, maintaining the hull at safe temperatures while boosting thrust. The wake signature shows a temperature drop relative to ambient air.

\subsubsection{Mode C: Re-Entry Harvesting}
During atmospheric re-entry from orbit, the massive heat load (normally a danger) is used to recharge the energy buffers. The vehicle brakes against the atmosphere, converting orbital kinetic energy $\to$ heat $\to$ stored electrical/metric energy.

\section{Claims}

What is claimed is:

\begin{enumerate}
    \item A propulsion method comprising:
    \begin{enumerate}
        \item Generating a coherent electromagnetic metric field to produce thrust;
        \item Thermally coupling said field generation system to the exterior surface of the vehicle;
        \item Absorbing thermal energy generated by friction with the surrounding medium into the field generation system;
        \item Converting said absorbed thermal energy into increased field intensity and thrust.
    \end{enumerate}

    \item The method of Claim 1, wherein the vehicle exhibits a negative thermal signature (cooling wake) relative to the ambient medium during high-velocity travel.

    \item A propulsion apparatus comprising:
    \begin{enumerate}
        \item A solid-state drive core configured to generate a metric modification field;
        \item A hull structure comprising a phonon-conducting metamaterial configured to funnel thermal energy from the exterior surface to the drive core;
        \item A controller configured to modulate the drive field to maximize the absorption of thermal phonons from the hull structure.
    \end{enumerate}

    \item The apparatus of Claim 3, functioning as an "Entropic Pump" that orders the random thermal motion of the boundary layer into coherent kinetic energy.

    \item A method of atmospheric re-entry wherein the heat of compression is harvested to recharge the vehicle's energy storage, rather than dissipated via ablative shields.

    \item The system of Claim 3, wherein the drive core utilizes direct phonon-magnon coupling to convert lattice vibrations into magnetic field energy.

    \item A hypersonic vehicle capable of sustained flight without thermal damage to the airframe, utilizing active entropic cooling powered by the propulsion drive itself.
\end{enumerate}

\end{document}
