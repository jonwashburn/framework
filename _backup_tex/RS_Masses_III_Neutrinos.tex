\documentclass[11pt,a4paper]{article}
\usepackage[margin=1in]{geometry}
\usepackage[T1]{fontenc}
\usepackage{lmodern}
\usepackage{microtype}
\usepackage{amsmath,amssymb,amsthm}
\usepackage{mathtools}
\usepackage{booktabs}
\usepackage{enumitem}
\usepackage{xcolor}
\usepackage[hidelinks]{hyperref}
\newtheorem{theorem}{Theorem}[section]
\newtheorem{proposition}[theorem]{Proposition}
\newtheorem{definition}[theorem]{Definition}
\newtheorem{remark}[theorem]{Remark}
\newcommand{\phig}{\varphi}
\newcommand{\Ecoh}{E_{\mathrm{coh}}}
\newcommand{\muStar}{\mu_{\star}}
\newcommand{\mRS}{m^{\mathrm{RS}}}
\newcommand{\kappaEV}{\kappa_{\mathrm{eV}}}
\newcommand{\Sigmanu}{\Sigma m_\nu}
\newcommand{\Epass}{E_{\mathrm{passive}}}
\newcommand{\RS}{Recognition Science}
\newcommand{\SM}{Standard Model}
\newcommand{\PROVED}{\textcolor{blue!70!black}{\footnotesize\textsf{[PROVED]}}}
\newcommand{\HYP}{\textcolor{orange!80!black}{\footnotesize\textsf{[HYP]}}}
\newcommand{\CERT}{\textcolor{teal}{\footnotesize\textsf{[CERT]}}}
\newcommand{\VAL}{\textcolor{purple!70!black}{\footnotesize\textsf{[VAL]}}}

\title{\textbf{Neutrino Masses from the Deep $\phig$-Ladder:\\
Edge-Level Confinement, the $4{+}7{=}11$ Decomposition,\\and the $\phig^7$ Ratio}\\[0.5em]
\large Paper III of VI: The Neutrino Sector}
\author{Jonathan Washburn\\
\small Recognition Science Research Institute, Austin, Texas\\
\small \texttt{washburn.jonathan@gmail.com}}
\date{\today}

\begin{document}
\maketitle

\begin{abstract}
The charged fermion mass framework of Papers~I--II organizes nine particles
at a single anchor using integer rungs on the $\phig$-ladder with generation
torsion $\{0,11,17\}$.  Applied naively to neutrinos ($Z_\nu=0$), the integer
rungs $(0,11,19)$ produce a splitting ratio $R_\Delta\approx 2{,}207$---more
than 65 times the observed $R_\Delta\approx 33.8$.  This paper resolves the
no-go by showing that the generation coupling framework of Paper~VI
\emph{predicts} the resolution: neutrinos, lacking a charge band ($Z=0$),
are \textbf{confined to the edge level} of the 3-cube hierarchy and couple
at \textbf{half resolution}, yielding fractional rungs.

The key structural result is that the deep-ladder rung differences, when
doubled, exhaust the passive edge count: $4+7=11=\Epass$.  The sub-decomposition
$4=2^{D-1}$ (one direction's edges) and $7=\Epass-2^{D-1}$ (remaining passive
edges) provides the same kind of structural derivation for neutrinos that the
$\Epass+F=17=W$ identity provides for charged fermions.

This yields a specific rung triple $(r_1,r_2,r_3)=(-239/4,-231/4,-217/4)$ with:
absolute masses $m_1\!\approx\!0.00354$, $m_2\!\approx\!0.00926$,
$m_3\!\approx\!0.0499\,\mathrm{eV}$; mass sum $\Sigmanu\!\approx\!0.063\,\mathrm{eV}$;
normal ordering forced by rung ordering; the seam-free prediction
$m_3^2/m_2^2=\phig^7\approx 29.03$ (where $7=\Epass-2^{D-1}$);
and $R_\Delta=(\phig^{11}-1)/(\phig^4-1)\approx 33.82$
(where $11=\Epass$ and $4=2^{D-1}$).  All exponents now have
cube-geometric provenance.
\end{abstract}

\tableofcontents
\newpage

%=============================================================================
\section{Introduction}
%=============================================================================

Neutrinos present the deepest challenge in the RS mass program.  Oscillation
experiments measure $\Delta m^2_{21}\approx 7.4\times 10^{-5}\,\mathrm{eV}^2$
and $\Delta m^2_{31}\approx 2.5\times 10^{-3}\,\mathrm{eV}^2$ with
remarkable precision, but the absolute mass scale, the ordering, and the
mechanism producing these specific numbers remain open.

The charged-sector mass framework (Papers~I--II) uses integer rungs on the
$\phig$-ladder with generation torsion $\{0,\Epass,W\}=\{0,11,17\}$.  For
neutrinos ($Q=0,\;Z_\nu=0$), the charge band vanishes and integer torsion
produces $R_\Delta\approx 2{,}207$---an order-of-magnitude failure.
Paper~III (original version) resolved this by introducing fractional
(quarter-step) rungs, but the specific rung triple was constrained by
oscillation data rather than derived from structure.

This updated paper incorporates the generation coupling framework of
Paper~VI, which explains \emph{why} fractional rungs are needed and
\emph{what specific fractions} are forced.  The neutrino no-go is not a
failure of the framework; it is a structural prediction about the
difference between charged and neutral sectors.


%=============================================================================
\section{The No-Go for Integer Rungs}
%=============================================================================

\subsection{The integer attempt}

Applying the charged-sector torsion $\{0,11,17\}$ (or the neutrino variant
$\{0,11,19\}$ from the constructor) yields rung differences
$\Delta_{1\to 2}=11$, $\Delta_{2\to 3}=8$, total $\Delta_{1\to 3}=19$.
The splitting ratio: \PROVED{}
\begin{equation}
  R_\Delta^{\mathrm{integer}} = \frac{\phig^{38}-1}{\phig^{22}-1}
  \approx 2{,}207.
\end{equation}
This is 65$\times$ larger than the observed $\sim 33.8$.  Both orderings fail.

\subsection{Diagnosis: the steps are too large}

In the charged sectors, the gap function $\mathrm{gap}(Z)$ provides a large
exponent shift ($\sim 6$--$14$) that separates families.  For $Z_\nu=0$,
$\mathrm{gap}(0)=0$---no band correction exists.  Integer torsion produces
mass ratios of $\phig^{11}\approx 199$ between generations, vastly too large
for the observed neutrino hierarchy where $\sqrt{\Delta m^2_{31}/\Delta m^2_{21}}\approx 5.8$.


%=============================================================================
\section{The Structural Resolution: Edge-Level Confinement}
\label{sec:resolution}
%=============================================================================

\subsection{Coupling levels and the charge band}

Paper~VI derives the generation torsion from the \emph{coupling level} of
a recognition boundary to the 3-cube's combinatorial hierarchy:
\begin{itemize}[nosep]
  \item Gen~1: active edge only ($\tau_1=0$),
  \item Gen~2: passive edge network ($\tau_2=\Epass=11$),
  \item Gen~3: face structure ($\tau_3=\Epass+F=11+6=17=W$).
\end{itemize}
This works for charged fermions because $Z\neq 0$ provides a \textbf{locking
potential}: the charge band ``grips'' the face structure, enabling rigid
integer-step coupling at each level.

\subsection{$Z=0$ blocks face coupling}

For neutrinos, $Z_\nu=0$ and no charge-band locking potential exists.
Without it, the boundary cannot grip the 2-dimensional face structure:
\begin{quote}
\textbf{Face coupling is blocked for neutral boundaries.}
\end{quote}
The neutrino hierarchy is therefore \emph{confined to the edge level},
with total available span $\Epass=11$ rather than $\Epass+F=17$.

\subsection{Half-resolution coupling}

Charged boundaries lock to edge channels at integer strength (one full
rung per channel).  Neutral boundaries, lacking the charge-band grip,
couple at half strength: \HYP{}
\begin{equation}
\boxed{
  \text{Neutrino total span} = \frac{\Epass}{2} = \frac{11}{2} = 5.5\text{ rungs}.
}
\end{equation}
The factor of $1/2$ is the impedance mismatch between a neutral boundary
and the edge network that a charged boundary traverses at full strength.

\subsection{The $4+7=11$ edge decomposition}

Within the edge level, the passive edge network has internal structure.
The 3-cube's 12 edges decompose by direction: 4 edges along each of the
3 spatial axes.  With 1 active edge removed, the 11 passive edges split as:
\begin{itemize}[nosep]
  \item $2^{D-1}=4$ edges (one full directional slot),
  \item $\Epass-2^{D-1}=11-4=7$ edges (remaining passive edges).
\end{itemize}
This sub-partition produces the neutrino generation steps (in doubled
coordinates): \HYP{}
\begin{equation}
  2\times\Delta_{1\to 2} = 4 = 2^{D-1},
  \qquad
  2\times\Delta_{2\to 3} = 7 = \Epass - 2^{D-1}.
\end{equation}
Dividing by 2 (the half-resolution factor): \HYP{}
\begin{equation}
  \Delta_{1\to 2} = 2,\qquad
  \Delta_{2\to 3} = \frac{7}{2},\qquad
  \Delta_{1\to 3} = \frac{11}{2}.
\end{equation}

\subsection{The rung triple}

With a deep-ladder baseline (the absolute position is set by the eV
calibration seam), the rung triple becomes: \HYP{}
\begin{equation}
  (r_1,r_2,r_3) = \left(-\frac{239}{4},\;-\frac{231}{4},\;-\frac{217}{4}\right).
\end{equation}

\subsection{Comparison: charged vs.\ neutrino generation structure}

\begin{center}
\begin{tabular}{lccc}
\toprule
& Charged sector & Neutrino (actual) & Neutrino ($\times 2$) \\
\midrule
Gen 1$\to$2 step & $\Epass=11$ & $2$ & $4=2^{D-1}$ \\
Gen 2$\to$3 step & $F=6$ & $7/2$ & $7=\Epass-2^{D-1}$ \\
Total span & $W=17$ & $11/2$ & $11=\Epass$ \\
Coupling level & Edge+Face & Edge only & Edge only \\
Charge band & $Z\neq 0$ (locked) & $Z=0$ (unlocked) & -- \\
\bottomrule
\end{tabular}
\end{center}


%=============================================================================
\section{Mass Predictions}
%=============================================================================

\subsection{The eV reporting seam}

Absolute masses require a global calibration seam: \CERT{}
$\kappaEV = 2^{-22}\phig^{51}\times 10^6\,\mathrm{eV}
\approx 1.086\times 10^{10}\,\mathrm{eV}$.

\subsection{Mass law}

$m_i^{\mathrm{pred}} = \kappaEV\cdot\phig^{r_i}$ for $i\in\{1,2,3\}$. \HYP{}

\subsection{Predicted values}

\begin{align}
  m_1 &\approx 0.00354\,\mathrm{eV}, &
  m_2 &\approx 0.00926\,\mathrm{eV}, &
  m_3 &\approx 0.0499\,\mathrm{eV}. \notag
\end{align}
Mass sum: $\Sigmanu\approx 0.063\,\mathrm{eV}$ (below cosmological bound $\lesssim 0.12\,\mathrm{eV}$). \VAL{}


%=============================================================================
\section{The $\phig^7$ Ratio and Splitting Predictions}
%=============================================================================

\subsection{The exact squared-mass ratio}

The seam cancels: \PROVED{}
\begin{equation}
\boxed{
  \frac{m_3^2}{m_2^2} = \phig^{2\times 7/2} = \phig^7 \approx 29.03.
}
\end{equation}
The exponent $7 = \Epass - 2^{D-1} = 11 - 4$ is now structurally derived:
it is the second step of the edge-level sub-partition.

\subsection{Splitting ratio}

\begin{equation}
  R_\Delta = \frac{\Delta m^2_{31}}{\Delta m^2_{21}}
  = \frac{\phig^{11}-1}{\phig^4-1} \approx 33.82.
  \quad\text{(NuFIT: $\approx 33.8$)} \;\VAL{}
\end{equation}
The exponents $11=\Epass$ and $4=2^{D-1}$ are passive-edge sub-counts.

\subsection{Numerical splittings}

$\Delta m^2_{21}\approx 7.33\times 10^{-5}\,\mathrm{eV}^2$,\quad
$\Delta m^2_{31}\approx 2.48\times 10^{-3}\,\mathrm{eV}^2$.\quad
Both within NuFIT windows. \VAL{}


%=============================================================================
\section{Normal Ordering}
%=============================================================================

Since $\phig>1$ and $r_1<r_2<r_3$, monotonicity gives
$m_1<m_2<m_3$ (normal ordering). \PROVED{}
This is not a fit choice; it is forced by the rung assignment.


%=============================================================================
\section{Why the Resolution Works}
%=============================================================================

\subsection{The three structural observations}

\begin{enumerate}
  \item \textbf{$Z=0$ blocks face coupling.}  Without a charge band, the
        neutral boundary cannot lock to the face structure.  The hierarchy
        is confined to $\Epass=11$ rather than $W=17$.

  \item \textbf{Half resolution from impedance mismatch.}  Without charge-band
        locking, edge coupling operates at half the integer strength,
        producing $\frac{1}{2}$-integer rungs.

  \item \textbf{Edge-level internal structure.}  The passive edges have
        a sub-partition $4+7=11$ reflecting the directional structure of
        the 3-cube.  This accommodates three neutrino generations within
        the reduced span.
\end{enumerate}

\subsection{What was data-constrained is now structurally derived}

In the original Paper~III, the fractional rung convention was motivated
by resolution needs and octave compatibility, and the specific rung triple
was constrained by NuFIT data.  With the generation coupling framework:
\begin{itemize}[nosep]
  \item The half-integer convention is \emph{derived} from $Z=0$ face-blocking,
  \item The total span $11/2$ is \emph{derived} from $\Epass/2$,
  \item The step decomposition $2+7/2=11/2$ is \emph{derived} from the
        $2^{D-1}+(11-2^{D-1})=11$ sub-partition,
  \item The $\phig^7$ ratio is \emph{derived} from $\Epass-2^{D-1}=7$.
\end{itemize}
Only the absolute rung position (the overall baseline on the deep ladder)
remains set by the calibration seam.


%=============================================================================
\section{Falsifiers}
%=============================================================================

\textbf{Seam-free:}
(F1)~$R_\Delta\neq(\phig^{11}-1)/(\phig^4-1)$;
(F2)~Inverted ordering established;
(F3)~$m_3^2/m_2^2\neq\phig^7$.

\textbf{Scale-dependent:}
(F4)~$\Delta m^2$ outside NuFIT windows;
(F5)~$\Sigmanu<0.062\,\mathrm{eV}$ from cosmology;
(F6)~Direct mass detection above predicted window.


%=============================================================================
\section{Conclusions}
%=============================================================================

The neutrino no-go is resolved by the same cube geometry that explains
the charged-sector generations.  The key insight is that $Z=0$ confines
neutrinos to the edge level of the 3-cube hierarchy, producing half-integer
rungs with the sub-decomposition $4+7=11=\Epass$.  Every exponent in the
neutrino predictions ($7$, $11$, $4$) now traces to passive-edge sub-counts
of the 3-cube, providing the same level of structural derivation achieved
for the charged sectors.

The neutrino sector is no longer the ``weakest link'' of the RS mass program.
It is a structurally necessary consequence of the cube partition theorem:
charged boundaries couple to edges and faces ($\Epass+F=W=17$); neutral
boundaries couple to edges only ($\Epass=11$), at half resolution ($11/2$),
with internal sub-structure ($4+7$).

\begin{thebibliography}{99}
\bibitem{PDG2024} R.~L.~Workman \textit{et al.} [Particle Data Group],
  Prog.\ Theor.\ Exp.\ Phys.\ \textbf{2022}, 083C01 (2022) and 2024 update.
\bibitem{NuFIT} I.~Esteban \textit{et al.}, NuFIT~5.x (2024);
  \url{http://www.nu-fit.org}.
\bibitem{Washburn2025} J.~Washburn,
  ``The Algebra of Reality,'' \textit{Axioms} \textbf{15}(2), 90 (2025).
\bibitem{PaperVI} J.~Washburn, Paper~VI of this series (Generation Structure).
\end{thebibliography}

\end{document}
