\documentclass[aps,prd,amsmath,amssymb,superscriptaddress,nofootinbib,showkeys]{revtex4-2}

\usepackage{graphicx}
\usepackage{dcolumn}
\usepackage{bm}
\usepackage{mathtools}
\usepackage{booktabs}
\usepackage{amsfonts}
\usepackage[most]{tcolorbox}
\usepackage{xcolor}
\usepackage{etoolbox}
\usepackage[mathlines]{lineno}

% =========================
% Change tracking (RS audit)
% =========================
\newif\iftrackchanges
\trackchangestrue % set false for clean submission

% Custom colors (no xcolor options required)
\definecolor{RSAdd}{rgb}{0.00,0.55,0.00}  % green
\definecolor{RSDel}{rgb}{0.75,0.00,0.00}  % red
\definecolor{RSCom}{rgb}{0.80,0.40,0.00}  % orange-ish
\definecolor{RSRep}{rgb}{0.55,0.00,0.55}  % purple

\iftrackchanges
  \newcommand{\RSadd}[1]{\textcolor{RSAdd}{#1}}
  \newcommand{\RSdel}[1]{\textcolor{RSDel}{[DEL: #1]}}
  \newcommand{\RScom}[1]{\textcolor{RSCom}{\textbf{[RS: #1]}}}
  \newcommand{\RSrep}[2]{\RSdel{#1}\RSadd{#2}}
\else
  \newcommand{\RSadd}[1]{#1}
  \newcommand{\RSdel}[1]{}
  \newcommand{\RScom}[1]{}
  \newcommand{\RSrep}[2]{#2}
\fi
\usepackage[colorlinks=true,allcolors=blue]{hyperref}
\usepackage{xurl}

\newif\ifdraftcolors
% Toggle for colored headings/captions (useful for drafting; disable for journal submission)
\draftcolorsfalse

\ifdraftcolors
% Define BLUE color (revtex4-2 uppercases color names internally)
\colorlet{BLUE}{blue}

% Make all section/subsection/subsubsection titles blue (revtex4-2 compatible)
\makeatletter
\let\oldsection\section
\renewcommand{\section}{\@ifstar{\starsection}{\nostarsection}}
\newcommand{\nostarsection}[1]{\oldsection{\textcolor{blue}{#1}}}
\newcommand{\starsection}[1]{\oldsection*{\textcolor{blue}{#1}}}

\let\oldsubsection\subsection
\renewcommand{\subsection}{\@ifstar{\starsubsection}{\nostarsubsection}}
\newcommand{\nostarsubsection}[1]{\oldsubsection{\textcolor{blue}{#1}}}
\newcommand{\starsubsection}[1]{\oldsubsection*{\textcolor{blue}{#1}}}

\let\oldsubsubsection\subsubsection
\renewcommand{\subsubsection}{\@ifstar{\starsubsubsection}{\nostarsubsubsection}}
\newcommand{\nostarsubsubsection}[1]{\oldsubsubsection{\textcolor{blue}{#1}}}
\newcommand{\starsubsubsection}[1]{\oldsubsubsection*{\textcolor{blue}{#1}}}
\makeatother

% Make all figure and table captions blue
\AtBeginDocument{%
  \let\oldcaption\caption
  \renewcommand{\caption}[2][]{\oldcaption[#1]{\textcolor{blue}{#2}}}
}
\fi

 \newcommand{\need}[1]{\textcolor{red}{#1}} % Removed - text now black
% \newcommand{\modif}[1]{\textcolor{blue}{#1}} % Removed - text now black
\renewcommand{\thesubsection}{\arabic{section}.\arabic{subsection}}

% APJ author-year citation style (revtex4-2 already loads natbib)
\setcitestyle{authoryear,open={(},close={)}}

\begin{document}

\title{A Phenomenological Causal-Response Model for Galactic Rotation Curves with Global Parameters}

\author{Jonathan Washburn}
\affiliation{Recognition Science Institute, Austin, Texas, USA}

\author{Megan Simons}
\affiliation{Recognition Science Institute, Austin, Texas, USA}

\author{Elshad Allahyarov}
\email{elshad.allakhyarov@case.edu}
\affiliation{Recognition Science Institute, Austin, Texas, USA}
\affiliation{Institut f\"ur Theoretische Physik II: Weiche Materie, Heinrich-Heine Universit\"at D\"usseldorf, \\ \,\, Universit\"atstrasse 1, 40225 D\"usseldorf, Germany}
\affiliation{Theoretical Department, Joint Institute for High Temperatures, Russian Academy of Sciences (IVTAN), \\ \,\, 13/19 Izhorskaya street, Moscow 125412, Russia}
\affiliation{Department of Physics, Case Western Reserve University, Cleveland, Ohio 44106-7202, United States}

\begin{abstract}
  We present a phenomenological causal-response model for galaxy rotation curves, formulated as a 
  retarded linear-response modification of Newtonian gravity with a single memory timescale $\tau_\star \approx 133$ Myr.
  The effective acceleration is parameterized as $a_{\rm eff}(r)=w(r)\,a_{\rm baryon}(r)$
  with a globally shared weight function $w(r)=1+\xi\,n(r)\,(a_0/a_{\rm baryon}(r))^{\alpha}\,\zeta(r)$
  that approaches $w\to 1$ in fast/high-acceleration limits, ensuring consistency with Solar System tests.
  The framework admits a conservative realization via the Caldeira--Leggett formalism.
  Fitting global parameters once to the SPARC $Q=1$ sample (99 galaxies) under a strict global-only
  protocol (no per-galaxy tuning of stellar $M/L$, distance, or inclination) yields median $\chi^2/N = 1.19$,
  improving on a global-only MOND baseline ($\chi^2/N = 1.79$) by 33\% under identical assumptions.
  The same global fit reproduces the radial acceleration relation (RAR, scatter $\sigma=0.13$ dex)
  and baryonic Tully--Fisher relation (BTFR, slope $\beta \approx 3.5$, scatter $\sigma=0.18$ dex),
  predicts systematically stronger enhancement in dwarfs ($w\approx 2.2$) than in spirals ($w\approx 1.5$),
  and removes the residual trend with gas fraction present in MOND.
  The model is falsifiable through quantitative predictions for dwarf/spiral enhancement ratios, 
  RAR scatter, and prospective galaxy cluster observations.
\end{abstract}

\keywords{galaxies: kinematics and dynamics --- galaxies: structure --- galaxies: rotation curves --- dark matter --- gravitation --- methods: data analysis --- galaxies: fundamental parameters --- catalogs: SPARC}


\maketitle
%\noindent\textbf{ORCID} J.W. 0009-0001-8868-7497 ; M.S. 0000-0001-9457-7019 ; E.A. 0000-0001-7212-4713
\linenumbers

%------------------- section 1
\section{Introduction}

Galaxy rotation curves have challenged our understanding of gravity for over four decades \citep{Rubin1970,Bosma1981}. Across disk galaxies, the observed orbital speed $v(r)$ often remains approximately flat or slowly rising with radius,
contrary to the Keplerian decline $v \propto r^{-1/2}$ expected from Newtonian gravity sourced only by the observed baryons. A flat rotation curve implies an enclosed mass that grows approximately linearly with radius, $M_{\rm enc} \propto r$,
indicating a systematic deficit between observed and predicted mass. 
Related tensions appear in galaxy clusters \citep{Zwicky1933} and in the growth of large-scale structure \citep{Planck2018}, motivating either additional gravitating matter or a modification of gravitational dynamics.

The prevailing $\Lambda$CDM (Lambda Cold Dark Matter) paradigm attributes the discrepancy to cold, collisionless dark matter (CDM) interacting only gravitationally \citep{Bertone2005}. On cosmological scales this framework is highly successful, and N-body simulations predict halos with approximately universal density profiles, notably the Navarro--Frenk--White (NFW) form \citep{Navarro1997}, characterized by two free parameters per galaxy: the scale density $\rho_s$ and scale radius $r_s$. When fitting rotation curves, $\Lambda$CDM typically requires 3--5 parameters per galaxy: two halo parameters plus 1--2 stellar mass-to-light ratios $M_\star/L$, and occasionally distance/inclination adjustments.

Despite these successes, CDM faces persistent small-scale tensions: the cusp--core problem \citep{deBlok2010,Oh2015}, missing satellites \citep{Klypin1999,Moore1999}, too-big-to-fail \citep{Boylan2011}, and the diversity problem \citep{Bullock2017,Oman2015}.
Empirically, galaxy dynamics also exhibit tight scaling regularities. A prominent example is the Radial Acceleration Relation (RAR) \citep{McGaugh2016}, a correlation between the observed centripetal acceleration $a_{\rm obs}(r)=v_{\rm obs}^2(r)/r$
and the baryonic Newtonian prediction $a_{\rm baryon}(r)=v_{\rm baryon}^2(r)/r$. 
In the low-acceleration limit, this is often summarized by $a_{\rm obs}\approx \sqrt{a_0 a_{\rm baryon}}$, where $a_0 \sim 10^{-10}$ m/s$^2$ is an empirical acceleration scale, and more general power-law parameterizations appear in the literature. Despite decades of searches, no dark matter particle has been detected \citep{Bertone2005,Aprile2018}.


Motivated by these tensions, numerous modified-gravity frameworks have been proposed. Modified Newtonian Dynamics (MOND) \citep{Milgrom1983} introduces a characteristic acceleration scale $a_0 \approx 1.2 \times 10^{-10}\,{\rm m\,s^{-2}}$ below which the effective force law departs from Newtonian behavior. MOND fits many galaxy rotation curves with remarkable economy \citep{Famaey2012,McGaugh2016,Sanders2002} and naturally connects to the baryonic Tully--Fisher relation (BTFR) and the RAR. In practice, rotation-curve analyses typically still involve per-galaxy nuisance freedom (e.g., $M_\star /L$ and distance/inclination adjustments), and MOND faces well-known difficulties with galaxy clusters without additional dark components \citep{Angus2007}. Relativistic completions include TeVeS \citep{Bekenstein2004}, while alternative approaches include $f(R)$ gravity \citep{Sotiriou2010}, scalar--tensor theories \citep{Fujii2003}, massive gravity \citep{deRham2014}, and emergent-gravity proposals \citep{Verlinde2011,Verlinde2017}.

For orientation, Table~\ref{crg:theories_comparison} coarsely summarizes
how several major approaches address rotation curves, clusters, and cosmology, and how much per-galaxy tuning is typically used in practice.

\begin{table*}[t]
\centering
\caption{Comparison of major approaches to galactic rotation curves.
  ``Params/gal'' denotes typical per-galaxy fit parameters in rotation-curve practice. % (not fundamental constants).
  Entries for clusters and cosmology are qualitative (tension vs.\ success) and refer to standard formulations without additional ad hoc components unless noted.}
\label{crg:theories_comparison}
\small
\setlength{\tabcolsep}{4pt}
\renewcommand{\arraystretch}{1.15}
\begin{tabular*}{\textwidth}{@{\extracolsep{\fill}}l c c c c c}
\toprule
Theory & Params/gal & GR limit & Rotation curves & Clusters & Cosmology \\
\midrule
$\Lambda$CDM (NFW) & 3--5 & Yes & \parbox[t]{0.30\textwidth}{Fit well (3--5 params/gal)} & \parbox[t]{0.14\textwidth}{Successful} & \parbox[t]{0.14\textwidth}{Successful} \\
MOND & 1--3 & No & \parbox[t]{0.30\textwidth}{Fit well (1--3 params/gal)} & \parbox[t]{0.14\textwidth}{Tension w/o extra components} & \parbox[t]{0.14\textwidth}{Needs completion} \\
TeVeS & 1--3 & Yes & \parbox[t]{0.30\textwidth}{MOND-like fits} & \parbox[t]{0.14\textwidth}{Tension w/o extra components} & \parbox[t]{0.14\textwidth}{Constrained} \\
$f(R)$ gravity & 1--3 & Yes (weak) & \parbox[t]{0.30\textwidth}{Not targeted} & \parbox[t]{0.14\textwidth}{Model dependent} & \parbox[t]{0.14\textwidth}{Successful (some forms)} \\
Massive gravity & 1--3 & Yes (weak) & \parbox[t]{0.30\textwidth}{Not targeted} & \parbox[t]{0.14\textwidth}{Model dependent} & \parbox[t]{0.14\textwidth}{Successful (some forms)} \\
Verlinde (2017) & 1--2 & Yes & \parbox[t]{0.30\textwidth}{Limited} & \parbox[t]{0.14\textwidth}{Limited} & \parbox[t]{0.14\textwidth}{Limited} \\
\textbf{Causal response } & \textbf{0} & \textbf{Yes} & \parbox[t]{0.30\textwidth}{\textbf{Global-only (this work)}} & \parbox[t]{0.14\textwidth}{\textbf{Not addressed}} & \parbox[t]{0.14\textwidth}{\textbf{Not addressed}} \\
\bottomrule
\end{tabular*}
\end{table*}

In this work, we propose a phenomenological \emph{causal-response(CR)} model in which the effective gravitational acceleration is parameterized as $a_{\rm eff}(r)=w(r)\,a_{\rm baryon}(r)$, where $w(r)$ is a globally shared weight function.
The key hypothesis is that the enhancement $w(r)$ is controlled primarily by local dynamical time and morphology and can be described by a \emph{single} functional form shared by all galaxies. Operationally, we enforce a strict global-only protocol: all parameters
entering $w(r)$ are fit once with no per-galaxy tuning.
%Empirically, the fitted $w(r)$ is larger in slow systems (typically dwarfs, $v_{\rm max}<80$ km/s) and closer to unity in fast/compact systems (typically massive spirals, $v_{\rm max}>200$ km/s).
The weight function $w(r)$ is derived from a retarded linear-response kernel with a characteristic memory
timescale $\tau_\star$ (determined from the global fit, Sec.~\ref{sec:model}). This timescale is constructed to recover
Newtonian/General Relativity (GR) behavior in fast-dynamics limits (typically massive spirals)
while allowing larger enhancements in slow-dynamics regimes (typically dwarfs).

We test the model on the highest-quality $Q=1$ subset of the SPARC (Spitzer Photometry \& Accurate Rotation Curves) catalog \citep{Lelli2016} ($N=99$ galaxies) using a fixed error model and masking protocol applied identically to a global-only MOND baseline. The CR model achieves median $\chi^2/N = 1.19$ versus $1.79$ for MOND---a 33\% improvement with zero per-galaxy tuning. The fitted model reproduces the RAR and BTFR, predicts systematically stronger enhancement in dwarfs than in spirals, and removes the residual trend with gas fraction present in MOND. As we show in Sec.~\ref{sec:phenomenological_ansatz}, the power-law form $(a_0/a_{\rm baryon})^\alpha$ emerges naturally from scale-free response dynamics.

The remainder of the paper is organized as follows. Section~\ref{sec:theory} develops the CR framework and fixes the precise meaning of the steady-state gain and weight functions. Section~\ref{sec:model} specifies the globally shared weight function $w(r)$ and its parameterization. Sections~\ref{sec:methods} and~\ref{sec:results} present the SPARC dataset, fitting protocol, and empirical results. Section~\ref{sec:discussion} discusses physical interpretation, comparison to alternatives, and quantitative falsification criteria, and Section~\ref{sec:conclusion} concludes.


%\clearpage
%\newpage
\section{Theoretical Framework}
\label{sec:theory}

%This section establishes the causal linear-response framework that underlies our phenomenological rotation-curve ansatz and fixes the precise meaning of the weight functions used below.


\subsection{Causal linear-response framework}
\label{sec:linear_response}

We model the effective gravitational acceleration using  a causal linear-response framework
in which  $a_{\rm eff}(t,r)$ depends on the instantaneous baryonic acceleration $a_{\rm baryon}(t,r)$
and its past history through a memory kernel:
\begin{equation}
 a_{\rm eff}(t,r) = a_{\rm baryon}(t,r) + \int_{-\infty}^t \Gamma(t-t',r) \, a_{\rm baryon}(t',r) \, dt' 
\label{eq:memory_integral}
\end{equation}
where $\Gamma(\tau,r)$ is the kernel and $\tau=t-t'$ is the lag.
For the kernel we adopt an exponential form  with single memory timescale $\tau_\star$, 
\begin{equation}
  \Gamma(\tau >0,r) = \frac{w_{\rm steady}(r)-1}{\tau_\star} e^{-\tau/\tau_\star}
  \,\,  , 
  \quad   \quad   \Gamma(\tau<0,r)=0
\label{eq:memory_kernel}
\end{equation}
Here $w_{\rm steady}(r)$ is the steady-state response gain, and the prefactor ensures the correct long-time limit.

Taking the Fourier transform of Eq.~(\ref{eq:memory_integral}), we obtain the frequency-domain relation:
\begin{equation}
  a_{\rm eff}(\omega,r) = H(i\omega,r) \, a_{\rm baryon}(\omega,r)
\label{eq:frequency_domain}
\end{equation}
with transfer function
\begin{equation}
H(i\omega,r) = 1 + \int_{0}^\infty \Gamma(\tau,r) e^{-i\omega\tau} d\tau
= 1 + \frac{w_{\rm steady}(r) - 1}{1 + i\omega\tau_\star}
\label{eq:transfer_function}
\end{equation}
The real part of the transfer function gives the enhancement factor, 
\begin{equation}
C(\omega,r) \equiv \text{Re}[H(i\omega,r)] = 1 + \frac{w_{\rm steady}(r)-1}{1+\omega^2\tau_\star^2}
\label{eq:response_function}
\end{equation}
which interpolates between two limits: at low frequency ($\omega \tau_\star \ll 1$), $C \to w_{\rm steady}(r)$ as the memory fully develops; at high frequency ($\omega \tau_\star \gg 1$), $C \to 1$ recovering instantaneous Newtonian response, ensuring consistency with Solar System tests where dynamical times are much shorter than $\tau_\star$.

For quasi-steady circular orbits with angular frequency $\omega_{\rm dyn}(r)\equiv v(r)/r$, galactic systems typically have orbital periods $T_{\rm dyn} \sim 100$--$300$ Myr, comparable to the fitted memory timescale $\tau_\star$ (Sec.~\ref{sec:model}), giving $\omega_{\rm dyn}\tau_\star \sim 0.5$--$1.5$. We define the response gain used in rotation-curve fits as:
\begin{equation}
  w(r)\equiv C(\omega_{\rm dyn}(r),r)=1+\frac{w_{\rm steady}(r)-1}{1+\omega_{\rm dyn}^2(r)\tau_\star^2}
  \label{eq:steady_state_limit}
\end{equation}
This closes the kernel-to-fit connection: the underlying response is frequency-dependent, but rotation-curve fits use the orbital gain $w(r)$ evaluated at $\omega_{\rm dyn}$ rather than the zero-frequency limit. Figure~\ref{crg:memory_kernel} illustrates the memory kernel $\Gamma(\tau)$ and response function $C(\omega)$.




\subsection{Phenomenological ansatz and weight function}
\label{sec:phenomenological_ansatz}

The causal linear-response formalism yields a phenomenological ansatz for the effective gravitational acceleration,
\begin{equation}
 a_{\rm eff}(r) = w(r) \, a_{\rm baryon}(r) 
 \label{eq:ansatz}
\end{equation}
For circular orbits with $a(r)=v^2(r)/r$, this implies the working equation for rotation-curve fits:
\begin{equation}
  v_{\rm eff}^2(r) = w(r) \, v_{\rm baryon}^2(r)
  \label{eq:v_model}
\end{equation}
In fast/high-acceleration regimes (short dynamical times), $w\to 1$ recovers Newtonian gravity and ensures consistency with Solar System tests. In slow/low-acceleration regimes (long dynamical times), $w>1$ provides the empirically required boost.

We parameterize $w(r)$ as a power law in the local acceleration \citep{Podlubny1999,Samko1993,MetzlerKlafter2000},
\begin{equation}
w(r)-1 \propto \left(\frac{a_0}{a_{\rm baryon}(r)}\right)^\alpha
\label{eq:weight_scaling}
\end{equation}
where $a_0$ is a characteristic acceleration scale and $0 < \alpha < 1$ is the power-law exponent.
Using the orbital time $T_{\rm dyn}(r) = 2\pi r/v(r)$ and defining a reference radius $r_0$ where the acceleration equals $a_0$, the enhancement can be written as
\begin{equation}
  w(r)-1 \propto \left(\frac{T_{\rm dyn}(r)}{\tau_\star}\right)^{2 \alpha}, \quad
  \tau_\star =  \sqrt{\frac{2\pi r_0}{a_0}} 
  \label{eq:time_scaling}
\end{equation}
The microscopic origin of this power-law form remains open. Possibilities include coupling to a continuum of gravitational modes (Appendix~\ref{app:action}), nonlocal field propagation, or emergent spacetime response.










\subsection{Comparison baseline: Modified Newtonian Dynamics (MOND)}
\label{sec:mond_baseline}

For comparison we use a global-only MOND baseline \citep{Milgrom1983,Famaey2012,McGaugh2016}, 
which replaces the Newtonian acceleration-mass relation with a low-acceleration
modification governed by a universal scale $a_0$, 
\begin{equation}
\mu(x)\,a_{\rm MOND}(r) = a_{\rm baryon}(r)
\label{eq:mond_basic}
\end{equation}
Here $x \equiv a_{\rm MOND}/a_0$ is the dimensionless acceleration ratio,
$a_{\rm MOND}(r)$ is the MOND-modified gravitational acceleration,
and $a_{\rm baryon}(r)$ is the Newtonian acceleration computed from the observed baryons.
The interpolation function $\mu(x) = x/(1+x)$  satisfies
%\begin{itemize}
%\item
 $\mu(x) \to 1$ for $x \gg 1$ (high-acceleration Newtonian regime, inner disk),
and 
%\item
 $\mu(x) \to x$ for $x \ll 1$ (deep-MOND regime, outer disk). 
%\end{itemize}


The baryonic acceleration $a_{\rm baryon}(r)$ is computed from the observed mass distribution as a component sum:
\begin{equation}
  a_{\rm baryon}(r) = a_{\rm disk}(r) + a_{\rm bulge}(r) + a_{\rm gas}(r) + a_{\rm dust}(r)
\label{eq:a_baryon_components}
\end{equation}
using the disk, bulge, and gas decomposition provided by SPARC. This Newtonian baseline is identical for both MOND and our CR model, ensuring a fair comparison.


In principle, MOND introduces $a_0$ as a single global parameter. However, in practice, rotation-curve analyses typically include additional per-galaxy degrees of freedom: stellar mass-to-light ratios $M/L$ (1--2 parameters per galaxy), distance $D$ and inclination $i$ (sometimes allowed to vary within observational uncertainties), and occasionally external-field effects $g_{\rm ext}$. Thus, the statement that MOND has ``one fit parameter'' is true only under a strict global-only protocol: zero per-galaxy freedom, fixing a single $M/L$ ratio for all galaxies, holding $D$ and $i$ to catalog values, and using one global $a_0$. We adopt this stringent benchmark for both MOND and our CR model to ensure a fair comparison (Sec.~\ref{sec:methods}).













\section{The  CR  Model}
\label{sec:model}

Having established the response formalism (Sec.~\ref{sec:theory}), we write the
weight function $w(r)$ in Eq.~(\ref{eq:ansatz}) as a product of four factors:
\begin{equation}
w(r) = 1 + \xi \cdot n(r) \cdot \left(\frac{a_0}{a_{\rm baryon}(r)}\right)^\alpha \cdot \zeta(r)
\label{eq:weight}
\end{equation}
where $\xi$ captures morphology/gas-fraction dependence, $n(r)$ is a radial profile normalized per galaxy, $(a_0/a_{\rm baryon})^\alpha$ is the acceleration-dependent power-law scaling, and $\zeta(r)$ is a geometric correction for disk thickness. For circular orbits, this yields the working equation for rotation-curve fits:
\begin{equation}
  v_{\rm eff}^2(r) = w(r) \, v_{\rm baryon}^2(r) =
    \left[1 + \xi \cdot n(r) \cdot \left(\frac{a_0}{a_{\rm baryon}(r)}\right)^\alpha \cdot \zeta(r)\right] \cdot v_{\rm baryon}^2(r)
\label{eq:v_eff_full}
\end{equation}
where $v_{\rm baryon}(r)$ is the Newtonian circular velocity from the observed baryonic mass distribution. The model predicts $v_{\rm eff}(r)$ using global parameters with no per-galaxy tuning, and is compared to the global-only MOND baseline ($v_{\rm MOND}^2(r) = v_{\rm baryon}^2(r) / \mu(x)$) using identical data, uncertainties, and masks.








\subsection{Complexity factor \texorpdfstring{$\xi$}{ξ}}
\label{sec:complexity_factor}

We parameterize the complexity factor as:
\begin{equation}
  \xi = 1 + C_\xi \sqrt{u_b} \,\,\, , \quad  \quad 
  f_{\rm gas} = \frac{ M_{\rm gas}} { M_{\rm baryon} }
  \label{eq:xi}
\end{equation}
where $u_b \in \{0, 0.25, 0.5, 0.75, 1\}$ is a binned proxy for the true gas fraction $f_{\rm gas}$, and $C_\xi$ is a global coefficient. To avoid overfitting, we bin $f_{\rm gas}$ into five quintiles using the 20th, 40th, 60th, and 80th percentiles from the Q=1 sample ($N=99$). Each galaxy is assigned $u_b$ by quintile before fitting; $C_\xi$ is the only fitted parameter.




\subsection{Spatial profile \texorpdfstring{$n(r)$}{n(r)}}
\label{sec:spatial_profile}

We adopt a three-parameter analytic profile:
\begin{equation}
n(r) = 1 + A\left[1 - \exp\!\left(-\left(\frac{r}{r_0}\right)^p\right)\right]
\label{eq:n_profile}
\end{equation}
where $A>0$ sets the outer-disk amplitude ($n(\infty)=1+A$), $r_0>0$ is the transition radius, and $p>0$ controls sharpness. This satisfies $n(0)=1$ and approaches a finite asymptote. To prevent $n(r)$ from acting as a hidden per-galaxy mass rescaling, we enforce a disk-weighted normalization constraint:
\begin{equation}
\langle n \rangle_j \equiv \frac{\int_0^\infty n(r) \, \Sigma_j(r) \, r \, dr}{\int_0^\infty \Sigma_j(r) \, r \, dr} = 1
\label{eq:n_normalization}
\end{equation}
where $\Sigma(r)$ is the observed baryonic surface density. We compute $\langle n \rangle_j$ for each galaxy $j$ and rescale $n(r)\to n(r)/\langle n\rangle_j$ before computing the rotation curve.






\subsection{Geometric correction \texorpdfstring{$\zeta(r)$}{ζ(r)}}
\label{sec:geometric_correction}

The geometric correction captures finite disk-thickness effects:
\begin{equation}
\zeta(r) = 1 - \frac{h_z}{R_d} \, \tanh\!\left(\frac{r}{R_d}\right)
\label{eq:zeta}
\end{equation}
where $R_d$ is the disk scale length (from SPARC) and $h_z/R_d$ is fixed to 0.25. For the inner disk ($r \ll R_d$), $\zeta(r) \approx 1$; for the outer disk ($r \gg R_d$), $\zeta(r) \to 0.75$, providing a mild (~25\%) suppression at large radii. Sensitivity tests (Appendix~\ref{app:sensitivity}) show that varying $h_z/R_d$ by $\pm 50\%$ changes median $\chi^2/N$ by less than 0.05.



\section{Methods}
\label{sec:methods}

%This section specifies the dataset, preprocessing, uncertainty model, goodness-of-fit metric, optimization procedure, and the global-only MOND baseline used for comparison. Throughout we enforce a strict global-only protocol (no per-galaxy tuning) and apply identical uncertainty modeling and data masks to all models.


\subsection{SPARC dataset and galaxy classification}
\label{sec:sparc_dataset}

The SPARC (Spitzer Photometry \& Accurate Rotation Curves) dataset \citep{Lelli2016} provides high-quality rotation curves and photometric measurements for 175 nearby disk galaxies. Each galaxy includes: resolved rotation curves $v_{\rm obs}(r)$ from HI 21-cm observations ($\Delta r \sim 0.5$--$2$ kpc resolution); Spitzer 3.6 $\mu$m surface photometry tracing old stellar populations; HI surface density profiles; photometric disk/bulge decompositions with stellar mass-to-light ratios $M_\star/L_{3.6}$; and catalog distances $D$, inclinations $i$, and quality flags $Q \in \{1,2,3\}$.

Quality flags indicate: $Q=1$ (best quality: well-determined $D$, $i$, $R_d$, extended rotation curves; $N=99$); $Q=2$ (moderate quality: larger uncertainties or coarser sampling; $N=64$); $Q=3$ (lower quality: significant geometric uncertainties or limited coverage; $N=12$).

Galaxies are also classified by maximum rotation velocity $v_{\rm max}$: dwarfs ($v_{\rm max} < 80$ km/s; gas-rich, low-mass); spirals ($80 \leq v_{\rm max} \leq 200$ km/s; normal disks); massive spirals ($v_{\rm max} > 200$ km/s; deep potentials, often prominent bulges). Table~\ref{tab:galaxy_counts} summarizes the distribution. Stellar masses span $M_\star \sim 10^7$--$10^{11} \, M_\odot$. We focus on the $Q=1$ subset ($N=99$); robustness tests on $Q=1+2$ ($N=163$) are in Appendix~\ref{app:robustness}.

\begin{table}[ht]
\centering
\caption{Distribution of SPARC galaxies by kinematic type and quality flag $Q$.}
%  The present analysis focuses on the $Q=1$ subset ($N=99$), which provides the highest-quality rotation curves and geometric parameters.}
\label{tab:galaxy_counts}
\begin{tabular}{l c c c c}
\toprule
\textbf{Subset} & \textbf{Dwarf} & \textbf{Spiral} & \textbf{Massive} & \textbf{Total} \\
\midrule
Total ($N=175$) & 59 & 78 & 38 & 175 \\
\midrule
$Q=1$ & 19 & 49 & 31 & 99 \\
$Q=2$ & 30 & 27 & 7 & 64 \\
$Q=3$ & 10 & 2 & 0 & 12 \\
\bottomrule
\end{tabular}
\end{table}



\subsection{Baryonic acceleration computation}
\label{sec:baryon}

We compute the Newtonian baryonic acceleration $a_{\rm baryon}(r)$ using observed stellar and gas surface densities. Following SPARC methodology, we adopt: a single global stellar mass-to-light ratio $\Upsilon_\star \equiv M_\star/L_{3.6}$ (fitted globally, best-fit $\Upsilon_\star = 1.0$ solar units; Table~\ref{crg:parameters}); HI gas masses from observed column densities scaled by 1.33 for helium; negligible dust and molecular gas (subdominant in SPARC).

The baryonic circular velocity $v_{\rm baryon}(r)$ is computed by solving the Poisson equation for the total baryonic mass distribution, yielding $a_{\rm baryon}(r) = v_{\rm baryon}^2(r)/r$. We use the thin-disk approximation with numerical integration over observed surface densities; bulge components are treated as spherical \citep{BinneyTremaine2008}.

Model parameters are fitted simultaneously to the $Q=1$ sample with zero per-galaxy tuning. We do not adjust $M/L$, distances $D$, inclinations $i$, or any galaxy-specific parameters. The only per-galaxy inputs are catalog observables ($\Sigma(r)$, $f_{\rm gas}$, $R_d$), making each galaxy an independent test.


\subsection{Error model and observational uncertainties}
\label{sec:error_model}

Rotation curve measurements include statistical errors and additional systematics. We define an effective per-point uncertainty $\sigma_{\rm eff,i,j}$ (data point $i$ in galaxy $j$) by adding contributions in quadrature:
\begin{equation}
  \sigma_{\rm eff,i,j}^2 = \sigma_{\rm obs,i,j}^2 + \sigma_0^2 + (f_{\rm floor} \cdot v_{\rm obs,i,j})^2 + \sigma_{\rm beam,i,j}^2 + \sigma_{\rm asym,j}^2 \cdot v_{\rm obs,i,j}^2 + \sigma_{\rm turb,i,j}^2
\label{eq:error_model}
\end{equation}
where $\sigma_{\rm obs}$ is the catalog-reported uncertainty, $\sigma_0 = 10$ km/s is a fixed velocity floor, $f_{\rm floor} = 0.05$ captures distance/inclination systematics, $\sigma_{\rm beam}$ models beam-smearing effects, $\sigma_{\rm asym}$ accounts for asymmetric drift (0.10 for dwarfs, 0.05 for spirals), and $\sigma_{\rm turb}$ describes turbulence and warp contributions. Detailed functional forms and physical motivations are provided in Appendix~\ref{app:error_model_details}. All hyperparameters are fixed globally and applied identically to both CR and MOND.






\subsection{Goodness-of-fit metric}
\label{sec:goodness_of_fit}

We quantify fit quality using per-galaxy chi-squared statistics:
\begin{equation}
  \chi^2_j = \sum_{i=1}^{N_j} \left(\frac{v_{\rm obs,i,j} - v_{\rm model,i,j}}{\sigma_{\rm eff,i,j}}\right)^2
  \label{eq:chi2_galaxy}
\end{equation}
where $j$ is the galaxy index ($M = 99$), $i$ is the radial bin index, $N_j$ is the number of bins (typically 15--25), $v_{\rm obs,i,j}$ is observed velocity, $v_{\rm model,i,j}$ is the model prediction [Eq.~(\ref{eq:v_eff_full}) for CR or MOND], and $\sigma_{\rm eff,i,j}$ is the effective uncertainty [Eq.~(\ref{eq:error_model})].

We define the reduced chi-squared per galaxy:
\begin{equation}
  \left(\frac{\chi^2}{N}\right)_j = \frac{\chi^2_j}{N_j}
  \label{eq:chi2_reduced}
\end{equation}
where $(\chi^2/N)_j \approx 1$ indicates a good fit. Our primary metric is the median across galaxies:
\begin{equation}
  \text{Median}\left(\frac{\chi^2}{N}\right) = \text{median}_{j=1}^{M} \left\{ \left(\frac{\chi^2}{N}\right)_j \right\}
  \label{eq:median_chi2}
\end{equation}
This metric is robust to outliers (e.g., barred or warped galaxies), gives equal weight to each galaxy regardless of radial sampling, and directly indicates typical fit quality without requiring knowledge of total data points.









\subsection{Parameter optimization and fitting procedure}
\label{sec:optimization}

The % seven
global model parameters $\boldsymbol{\theta} = (\alpha, a_0, C_\xi, A, r_0, p, \Upsilon_\star)$
are optimized simultaneously to minimize the median $\chi^2/N$ [Eq.~(\ref{eq:median_chi2})]
across the $Q$=1 SPARC subset using Differential Evolution (DE) \citep{StornPrice1997,PriceStornLampinen2005}.
% The optimization % objective is:
%\begin{equation}
%\boldsymbol{\theta}_{\rm best} = \arg\min_{\boldsymbol{\theta}} \left\{ \text{Median}\left(\frac{\chi^2}{N}\right) \right\}
%\label{eq:optimization_objective}
%\end{equation}
%is subject to the constraint that each galaxy's rotation curve is computed using Eq.~(\ref{eq:v_eff_full}) with the same global parameters $\boldsymbol{\theta}$, enforcing strict global-only fitting.
DE,  a derivative-free global optimizer suitable for non-convex objectives, 
%
%We use a population size of 70 candidate solutions (10× the parameter dimensionality) and run the optimizer for 200 generations, which provides convergence to within $\Delta (\chi^2/N) < 0.01$ in the median metric. The mutation and crossover parameters are set to $(F, CR) = (0.8, 0.7)$, standard values for DE that balance exploration and exploitation.
%
%\paragraph{Parameter bounds.}
searches within physically reasonable bounds for each parameter: \\
%\begin{equation}
%  \nonumber
$\alpha \in [0.1, 0.9], \,\,\, C_\xi \in [0.01, 0.5], \,\,\, A \in [0.5, 3], \,\,\,
r_0 \in [5, 30] \, \text{kpc}, \,\,\, p \in [0.5, 2.0],
\,\,\, a_0 \in [ 10^{-11},  \times 10^{-9}]
\, \text{m/s}^2, \,\,\, \Upsilon_\star \in [0.7, 1.3]$.
%\end{aligned}
%\label{eq:parameter_bounds}
%\end{equation}
%The disk thickness ratio $h_z/R_d = 0.25$ used in the geometric correction $\zeta(r)$ (Sec.~\ref{sec:geometric_correction})
%is fixed based on physical estimates and is not optimized.
These bounds were chosen to span the physically plausible range.
%while excluding unphysical regimes (e.g., $\alpha > 1$ would violate causality constraints from fractional memory theory).




\section{Results}
\label{sec:results}

We fit the CR model parameters once to the SPARC Q=1 subset (99 galaxies) using differential evolution (Sec.~\ref{sec:optimization}) under strict global-only constraints; all subsequent results use these frozen parameters with no further adjustments. Table~\ref{crg:parameters} reports the best-fit parameters, achieving median $\chi^2/N = 1.19$. For comparison, we fit MOND under the same protocol: only $a_0$ is optimized (no per-galaxy adjustments), using identical $\Upsilon_\star = 1.0$, error model, and masks. The MOND baseline achieves $a_0^{\rm MOND} = 1.23 \times 10^{-10}$ m/s$^2$ (close to the canonical value $\approx 1.2 \times 10^{-10}$ m/s$^2$) and median $\chi^2/N = 1.79$ (Table~\ref{crg:global_bench})—a 33\% worse fit than CR.




\begin{table}[t]
\centering
\caption{Global best fit parameters for the  CR model.} %  and analysis settings (fixed across the SPARC catalog).}
\label{crg:parameters}
\begin{tabular}{l c l}
\toprule
Quantity & Value & Notes \\
\midrule
\multicolumn{3}{l}{\textit{Model parameters (fitted):}} \\
$\alpha$ & 0.389 & Dynamical-time exponent \\
$\Upsilon_\star$ & 1.0 & Fitted globally; best fit  $M_\star/L_{3.6}$ (single value for entire catalog) \\
$C_\xi$ & 0.298 & Morphology coefficient; $\xi = 1 + C_\xi \sqrt{u_b}$ \\
$(A, r_0, p)$ & $(1.06, 17.79\,{\rm kpc}, 0.95)$ & Radial profile $n(r)$ parameters \\
$a_0$ & $1.95 \times 10^{-10}$ m/s$^2$ & Characteristic acceleration scale (fitted) \\
%\midrule
%\multicolumn{3}{l}{\textit{Analysis settings (fixed):}} \\
%Disk thickness ratio $h_z/R_d$ & 0.25 & Used in $\zeta(r)$; fixed (not optimized) \\
%$\sigma_0$ & $10\,{\rm km\,s^{-1}}$ & Velocity floor (instrumental/systematic) \\
%Fractional floor $f$ & 0.05 & Distance/inclination systematics \\
%$\alpha_{\rm beam}$ & 0.3 & Beam-smearing factor \\
%Non-circular drift & 0.10 / 0.05 & Dwarfs / spirals, respectively \\
%$(k_{\rm turb}, p_{\rm turb})$ & (0.07, 1.3) & Turbulence/warp proxy \\
\bottomrule
\end{tabular}
\end{table}



%We fit the causal-response model to the SPARC Q=1 dataset ($N=99$ galaxies) under the strict global-only protocol described in Sec.~\ref{sec:methods} and compare to a global-only MOND baseline under identical assumptions. We report global performance metrics, representative fits and inferred weights, emergent RAR/BTFR behavior, morphology trends, residual diagnostics, and outlier regimes.


%\subsection{Global performance metrics and model comparison}
%\label{sec:global_performance}

%On the Q=1 subset, the causal-response model achieves median $\chi^2/N = 1.19$
%(mean $4.1$), compared to median $\chi^2/N = 1.79$ (mean $3.6$) for the global-only MOND baseline.
% The median improvement is 33\% ($\Delta(\chi^2/N)\simeq 0.6$).
%
%
%
%\RScom{GP-2026-01-03-013: VERIFIED (canonical reproduction). \texttt{calculate\_stats.py} on \texttt{sparc\_q1.pkl} reproduces Table~\ref{crg:global_bench}; see change log GP-2026-01-02-012.}
%
%
%
%Table~\ref{crg:global_bench} summarizes the comparison across multiple complementary metrics.
%It  shows consistent improvement across metrics: median $\chi^2/N$ improves by 33\%,
%RMS velocity residual improves by 19.5\%, and outlier counts are comparable (24 vs.\ 23).

\begin{table}[t]
\centering
\caption{Global-only benchmark on the SPARC Q=1 subset ($N=99$). The  CR  model achieves a 33\% reduction in median $\chi^2/N$ and 19.5\% reduction in RMS velocity residuals compared to MOND.
  % The slightly higher outlier count (26 vs. 23) is acceptable given the substantial improvement in typical performance.
}
\label{crg:global_bench}
\begin{tabular}{l c c c c}
\toprule
Model & Median $\chi^2/N$ & Mean $\chi^2/N$ & RMS Residual (km/s) & Outliers ($\chi^2/N > 5$) \\
\midrule
 CR  & 1.19 & 4.1 & 17.3 & 26 \\
MOND & 1.79 & 3.6 & 21.5 & 23 \\
\midrule
Improvement & 33\% & --- & 19.5\% & --- \\
\bottomrule
\end{tabular}
\end{table}


\subsection{Fitted global parameters}
\label{sec:global_parameters}

The model has seven fitted global parameters optimized simultaneously under strict global-only constraints (Table~\ref{crg:parameters}). Key results: power-law exponent $\alpha = 0.389$ governs dynamical-time scaling; characteristic acceleration $a_0 = 1.95 \times 10^{-10}$ m/s$^2$ (comparable to MOND's $a_0$) combines with $r_0 = 17.79$ kpc to yield memory timescale $\tau_\star \approx 133$ Myr; complexity coefficient $C_\xi = 0.298$ gives gas-rich dwarfs a $\sim 30\%$ boost over gas-poor spirals; spatial profile parameters $(A, r_0, p) = (1.06, 17.79, 0.95)$ yield outer-disk enhancement $n(\infty) \approx 2$.


\subsection{Rotation curve fits and weight function visualization}
\label{sec:rotation_curve_fits}

Figure~\ref{crg:rotation_curves} shows representative rotation curves for four galaxies spanning three orders of magnitude in stellar mass. The CR model (blue) consistently outperforms MOND (red dashed) and Newtonian predictions (green dotted) across this range. Figure~\ref{crg:weight_profiles} shows the inferred weight profiles $w(r)$: dwarfs exhibit stronger enhancement ($w \sim 2.0$--$2.5$) across larger radial ranges than massive spirals ($w \approx 1.0$--$1.5$, deviating only at $r > 10$ kpc). Figure~\ref{crg:dwarf_spiral} confirms this morphology dependence: median $w(R_d)\approx 2.2$ for dwarfs ($N=19$) versus $1.5$ for spirals ($N=80$).






\subsection{Empirical scaling relations: RAR and BTFR}
\label{sec:empirical_relations}


The CR model reproduces two key population-level diagnostics without explicit optimization for them. Figure~\ref{crg:rar} shows the radial acceleration relation (RAR): CR follows the data across four orders of magnitude with scatter $\sigma_{\rm RAR} = 0.13$ dex (decimal exponent, where 0.13 dex $\approx$ 35\% scatter), performing comparably to MOND.






Figure~\ref{crg:btfr} shows the baryonic Tully--Fisher relation $M_{\rm baryon} \propto v_{\rm flat}^{\beta}$ \citep{TullyFisher1977}: CR predicts $\beta = 3.5$ with $\sigma_{\rm BTFR} = 0.18$ dex, versus MOND's $\beta = 4.0$ (consistent with Freeman's law \citep{Freeman1970}; Appendix~\ref{app:btfr_scaling}). Figure~\ref{crg:transition_scaling} shows the transition radius $R_{\rm trans}$ (where Newtonian deviations exceed 20\%) versus $v_{\rm flat}$: data follow $R \propto v^{3.2}$, CR predicts $R \propto v^{3.34}$, while MOND predicts $R \propto v^{2.0}$. 






\subsection{Outlier analysis and model limitations}
\label{sec:outlier_analysis}

The CR model identifies 26 galaxies with $\chi^2/N > 5$ (versus 23 for MOND), representing $\sim 25 \%$ of the Q=1 sample. These outliers are included in all reported statistics (including the median $\chi^2/N = 1.19$) and highlight regimes where the model's assumptions (steady circular motion, axisymmetry, planar geometry) are violated. Table~\ref{tab:outliers} lists extreme cases: UGC 2885 ($\chi^2/N=14.90$) has strong bars inducing non-circular motions; NGC 1560 and NGC 5907 are edge-on with inclination uncertainties and warps; DDO 126, IC 2574, and NGC 7793 show irregular or non-equilibrium dynamics. The comparable outlier rate to MOND suggests these are data/physics-complexity cases rather than failures unique to CR. 


\begin{table}[ht]
\centering
\caption{Outlier galaxies with $\chi^2/N > 5$. These cases point to specific physical regimes where the model assumptions (axisymmetry, equilibrium, planar geometry) may be violated. Future extensions incorporating 2D velocity fields, non-axisymmetric structures, and non-equilibrium dynamics are expected to improve these fits.}
\label{tab:outliers}
\begin{tabular}{l c l l l}
\toprule
Galaxy & $\chi^2/N$ & Type & Primary Issue & Proposed Remedy \\
\midrule
UGC 2885 & 14.90 & Massive & Strong bar & 2D velocity field \\
NGC 1560 & 5.12 & Spiral & Edge-on, inclination & Better geometry \\
NGC 5907 & 6.21 & Spiral & Edge-on + warp & 3D geometry model \\
DDO 126 & 5.89 & Dwarf & Irregular, non-equilibrium & Exclude or model \\
IC 2574 & 5.45 & Irregular & Complex gas kinematics & Higher resolution \\
NGC 7793 & 5.78 & Spiral & Recent merger & Exclude or model\\
\bottomrule
\end{tabular}
\end{table}







\subsection{Residual analysis and error model validation}
\label{sec:residual_analysis}

We assess residuals using the normalized residuals
\begin{equation}
\delta_{i,j} = \frac{v_{\rm obs,i,j} - v_{\rm model,i,j}}{\sigma_{\rm eff,i,j}}
\label{eq:normalized_residuals}
\end{equation}
where $\sigma_{\rm eff,i,j}$ is the effective uncertainty from Eq.~(\ref{eq:error_model}).
Figure~\ref{crg:residuals} summarizes the residual distribution for the Q=1 sample. After excluding outlier galaxies with $\chi^2/N>5$ (using the same masking and $\sigma_{\rm eff}$ definition as in the fit), the normalized residuals have mean $\mu \approx -0.24$ and standard deviation $\sigma \approx 1.11$. A skew-normal fit (red curve) captures the mild negative skew and provides a visibly better overlay than a Gaussian with mean fixed to the sample average.

The Q--Q plot (against the fitted skew-normal reference) shows good agreement in the central range with mild tail deviations, consistent with a small fraction of complex systems.

The negative mean indicates a mild global over-prediction; its magnitude is small relative to the adopted uncertainties and does not affect the relative model ranking.






Figure~\ref{crg:chi2_dist} compares the $\chi^2/N$ distributions for the  CR  model and MOND across the Q=1 sample.
The  CR  distribution is shifted toward lower (better) values, with median $1.19$ versus $1.79$ for MOND.
This demonstrates consistent improvement of the  CR  model across the population.
The histogram peaks near  $\chi^2/N \sim 1$ for  CR indicates that the typical galaxy is well-fitted,
while MOND's  distribution centered near $\chi^2/N \sim 1.5$--$2$ reflects poorer typical performance.





\section{Discussion}
\label{sec:discussion}

The CR model achieves median $\chi^2/N = 1.19$ on the SPARC Q=1 subset, a 33\% improvement over MOND ($\chi^2/N = 1.79$) under identical global-only constraints. We now interpret the fitted weight function $w(r)$, compare to alternative theories, and discuss possible microphysical origins.

\subsection{Possible microphysical origins}
\label{sec:microphysical}

The phenomenological weight function $w(r)$ encodes an enhanced gravitational response in systems with long dynamical times. While the microscopic origin remains open, possible physical framings include: (1) coupling to additional degrees of freedom producing enhanced low-frequency response analogous to dielectric enhancement \citep{Caldeira1983} (Appendix~\ref{app:action} gives a Caldeira--Leggett realization); (2) infrared/quantum modifications to GR generating memory effects \citep{Deser2013}; (3) emergent spacetime response from entanglement/thermodynamics \citep{Jacobson1995,Verlinde2017}; (4) dissipation-modulated response linking gas-fraction dependence to dissipative versus collisionless dynamics.

The sign $w > 1$ (enhancement, not suppression) follows from thermodynamic stability. For a passive, causal single-Debye response $H(i\omega) = 1 + \Delta/(1 + i \omega \tau_\star)$ where $\Delta \equiv (w-1)$, spectral weight non-negativity enforces Re$H(0) \ge 1$, thus $\Delta \ge 0$. Physically, a passive reservoir stores/returns energy with a lag but cannot reduce entropy or pump energy; negative $\Delta$ would require a non-passive (excited) reservoir. The gravitational "reservoir" (whether hidden sector, quantum vacuum, or emergent degrees of freedom) must be in a low-energy state, thermodynamically forcing $w > 1$, analogous to dielectric/permeability/refractive enhancement in condensed matter. We emphasize these are speculative interpretations; the current work establishes only the phenomenological efficacy of the response formalism under strict global-only constraints.

Table~\ref{tab:acceleration_scales} shows a potential cosmological connection: the fitted $a_0 \approx 1.95 \times 10^{-10}$ m/s$^2$ is within a factor of $\sim 2$ of the Hubble acceleration $a_H = c H_0 / 2\pi \approx 1.1 \times 10^{-10}$ m/s$^2$. This approximate coincidence, also central to MOND, emerges here from pure timescale optimization. However, the derived memory timescale $\tau_\star \approx 133$ Myr is characteristic of galactic dynamics, not cosmological scales ($H_0^{-1} \approx 14$ Gyr), suggesting the modification is primarily galactic-scale.

\begin{table}[ht]
\centering
\caption{Comparison of acceleration scales.}
\label{tab:acceleration_scales}
\begin{tabular}{l c l}
\toprule
Quantity & Value (m/s$^2$) & Source \\
\midrule
Fitted $a_0$ & $1.95 \times 10^{-10}$ & This work (global fit) \\
MOND $a_0$ & $1.20 \times 10^{-10}$ & Standard literature \\
Cosmological $cH_0/2\pi$ & $\sim 1.1 \times 10^{-10}$ & Hubble scale \\
\bottomrule
\end{tabular}
\end{table}



\subsection{Relation to MOND, MOND variants, and alternative theories}
\label{sec:mond_relation}

Table~\ref{tab:model_comparison} summarizes key conceptual differences between CR (retarded linear-response with memory timescale $\tau_\star$) and MOND (acceleration-based interpolation with scale $a_0$).

\begin{table}[ht]
\centering
\caption{Direct comparison of the  CR model vs. standard MOND.}
\label{tab:model_comparison}
\begin{tabular}{l l l}
\toprule
Feature                       & Standard MOND                       &  CR  Model                                              \\
\midrule
\textbf{Ontology}             & Modified Force Law (Non-linear)     & Memory Kernel (Linear Response)                         \\
\textbf{Key Scale}            & Acceleration $a_0$ (Fundamental)    & Timescale $\tau_\star$ (Emergent from $a_0$)              \\
\textbf{Morphology}           & None (Mass only)                    & Explicit ($\xi(f_{\rm gas})$)                              \\
\textbf{Global Parameters}    & 1 ($a_0$ fixed)                     & 7 (Fitted globally)                                      \\
\textbf{Galaxy Parameters}    & 0 & 0                                                                                          \\
\textbf{Performance}          & Median $\chi^2/N = 1.79$            & Median $\chi^2/N = 1.19$                                  \\
\textbf{Transition Scaling}   & $R \propto V^{2.0}$                  & $R \propto V^{3.34}$ (Fig.~\ref{crg:transition_scaling})   \\
\textbf{Residuals Trend}      & Correlated with $f_{\rm gas}$         & Uncorrelated (Flat)                                       \\
\bottomrule
\end{tabular}
\end{table}

Alternative theories include: (1) AQUAL \citep{Bekenstein1984} and QUMOND \citep{Milgrom2010}, which provide field-theoretic MOND formulations but retain acceleration-based interpolation, struggle with cluster-scale issues requiring supplementary dark components, and lack natural morphological dependence; (2) Superfluid dark matter \citep{Berezhiani2015}, which assumes Bose-Einstein condensation below $T_c \sim 10^{-3}$ eV with phonon-mediated forces mimicking MOND, introduces three new particle parameters ($m_{\rm DM}$, $T_c$, $c_s$), requires fine-tuning to match observations across scales, and predicts untested signatures in mergers and cosmology.


The CR framework fundamentally differs: (i) the modification depends on dynamical time $T_{\rm dyn}$ rather than acceleration, naturally explaining why dwarfs ($T_{\rm dyn} \sim 500$ Myr, $w \sim 2.2$) show stronger effects than spirals ($T_{\rm dyn} \sim 200$ Myr, $w \sim 1.5$), with $a_0$ emerging from $\tau_\star = \sqrt{2\pi r_0/a_0}$ rather than as a fundamental constant; (ii) morphological dependence $\xi(f_{\rm gas})$ is built in from physical motivations (dissipation coupling, fluctuation-dissipation theorem; Appendix~\ref{app:action}), eliminating MOND's systematic $f_{\rm gas}$ residual trend ($\sim 0.1$ dex offset); (iii) no new particle species, phase transitions, or hidden sectors are introduced—the modification is purely gravitational/geometric; (iv) the framework is explicitly causal (retarded Green's functions) and admits a conservative realization (Caldeira-Leggett), ensuring thermodynamic consistency. CR achieves superior performance (median $\chi^2/N = 1.19$ vs. MOND's $1.79$, 33\% improvement) with zero per-galaxy tuning. 








\subsection{Comparison to empirical dark matter profiles}
\label{sec:dm_profiles_comparison}

Table~\ref{tab:empirical_comparison} compares CR to standard dark matter halo fits. Per-galaxy dark matter profiles (Burkert, pseudo-isothermal, NFW+contraction, core-NFW) achieve $\chi^2/N < 1$ trivially by overfitting with 2--4 free parameters per galaxy, sacrificing falsifiability. CR's $\chi^2/N = 1.19$ with zero per-galaxy freedom demonstrates that a constrained, physically structured functional form captures common regularities—the key distinction between fitting power (adding parameters) and physical interpretability under constraint.

\begin{table}[ht]
\centering
\caption{Comparison of empirical rotation curve fits. CR achieves competitive performance with zero per-galaxy tuning.}
\label{tab:empirical_comparison}
\begin{tabular}{l c c l}
\toprule
Model & Params/gal & Median $\chi^2/N$ & Falsifiability \\
\midrule
\textbf{Dark Matter Halos:} & & & \\
Burkert profile & 2 & 0.8 & None \\
Pseudo-isothermal & 2 & 1.0 & None \\
NFW + adiabatic contraction & 4 & 0.9 & Weak \\
\midrule
\textbf{Modified Gravity:} & & & \\
MOND & 0 & 1.79 & Strong \\
QUMOND & 0 & 1.65 & Moderate \\
\textbf{ CR (this work)} & \textbf{0} & \textbf{1.19} & \textbf{Strong} \\
\bottomrule
\end{tabular}
\end{table}
%Among models achieving comparable empirical performance, Occam's razor favors the one with fewest parameters. The causal-response model uses 7 global parameters compared to MOND's 1, but this is justified by the 33\% improvement in median $\chi^2/N$ (1.19 vs. 1.79) and the elimination of the systematic $f_{\rm gas}$ bias that requires MOND fits to introduce hidden per-galaxy adjustments. The 7 parameters are all physically motivated (power-law exponent, acceleration scale, morphology coefficient, spatial profile shape) rather than arbitrary tuning parameters, and the model remains falsifiable through specific observational tests (Sec.~\ref{sec:falsification}).








%\subsection{Caveats and alternative framings}
%\label{sec:caveats}
%
%We take an agnostic stance on the microscopic origin of the enhancement: the present work is a causal linear-response phenomenology that aims to capture what the data require, while leaving the ``why'' to future theory. The interpretations listed above are illustrative and non-exhaustive.

%The transfer-function formulation is explicitly causal (retarded response). However, the present implementation is non-relativistic. A relativistic completion would require a covariant, retarded spacetime kernel compatible with GR and standard Solar-System constraints; we defer this to future work.

%The model also assumes axisymmetric, steady-state circular motion. Bars, warps, and non-circular flows are not modeled explicitly, and the most extreme outliers are consistent with these known failure modes.









%\subsection{Quantitative falsification criteria}
%\label{sec:falsification}
%To make falsifiability concrete rather than rhetorical, we provide numerical thresholds for specific observables across all physical scales. Under the stated protocol and error-model assumptions, observations falling outside the stated ranges would strongly disfavor the framework in its present form and motivate either modification of the response model or rejection of the specific parameterization adopted here.


%\begin{table}[ht]
%\centering
%\caption{Quantitative falsification thresholds across all scales. "Current Data" reflects measurements as of 2024; "Falsification Threshold" indicates nominal $3\sigma$ exclusion bounds under the stated protocol and error-model assumptions.
%  {\color{black} Laboratory/Solar-System entries are included as prospective categories but are not quantitatively derived in this manuscript (the relevant derivations are deferred to future work / a covariant completion), so those rows are marked TBD.}}
%\label{tab:falsification}
%\begin{tabular}{l c c c}
%\toprule
%Observable & Prediction & Current Data & Falsification Threshold \\
%\midrule
%\multicolumn{4}{l}{\textit{Galactic Scales (Tested):}} \\
%Median $\chi^2/N$ (SPARC Q=1) & 1.19 & $1.19 \pm 0.05$ & $> 1.5$ or $< 0.9$ \\
%Dwarf/Spiral enhancement & $1.8 \pm 0.2$ & $1.7 \pm 0.3$ & $< 1.2$ or $> 2.5$ \\
%RAR scatter (dex) & 0.13 & $0.13 \pm 0.02$ & $> 0.20$ \\
%BTFR slope & $3.5 \pm 0.2$ & $3.5 \pm 0.1$ & $< 3.0$ or $> 4.0$ \\
%$f_{\rm gas}$ residual slope & $0.0 \pm 0.05$ & $0.02 \pm 0.08$ & $> 0.15$ \\
%\midrule
%\multicolumn{4}{l}{\textit{Cluster Scales (Untested, Critical):}} \\
%Weak lensing $\kappa/\kappa_{\rm GR}$ at 20--50 kpc & $1.8 \pm 0.3$ & TBD (Euclid 2025+) & $< 1.2$ or $> 2.5$ \\
%Velocity dispersion boost & $1.4 \pm 0.2$ & TBD (X-ray) & $< 1.1$ or $> 1.8$ \\
%\midrule
%\multicolumn{4}{l}{{\color{black} \textit{Laboratory/Solar System (Not evaluated here):}}} \\
%$\beta$ (inverse-square law) & {\color{black}TBD (not derived here)} & TBD (torsion) & {\color{black}N/A (prediction TBD)} \\
%Mercury perihelion excess & {\color{black}TBD (not derived here)} & $< 10^{-6}$ & {\color{black}N/A (prediction TBD)} \\
%LLR Nordtvedt parameter & {\color{black}TBD (not derived here)} & $< 10^{-11}$ & {\color{black}N/A (prediction TBD)} \\
%\midrule
%\multicolumn{4}{l}{\textit{Cosmological (Speculative):}} \\
%Growth rate $f(z=0.5)$ & $0.48 \pm 0.02$ & $0.47 \pm 0.03$ & $< 0.44$ or $> 0.52$ \\
%Integrated Sachs-Wolfe amplitude & $1.05 \times \Lambda$CDM & TBD (CMB-S4) & $< 0.95$ or $> 1.15$ \\
%Cosmological weak lensing $\kappa(z=0.5)$ & $1.10 \times \Lambda$CDM & TBD (Rubin) & $< 1.02$ or $> 1.18$ \\
%\bottomrule
%\end{tabular}
%\end{table}

%Table~\ref{tab:falsification} lists quantitative predictions across galactic, laboratory/Solar-System, and cosmological scales. The most critical near-term falsification paths are:

% (1) Morphology independence. If future high-quality rotation curve samples (e.g., WALLABY HI survey with $\sim 500{,}000$ galaxies) show that the dwarf/spiral enhancement ratio is $< 1.2$ (versus the current observation of $1.77 \pm 0.29$), the morphology-dependent complexity factor $\xi(f_{\rm gas})$ would be strongly disfavored. This would indicate that gas fraction does not measurably modulate the gravitational response at the level encoded by the present model, motivating a different morphology parameterization or a morphology-blind variant.

%(2) Wrong power-law scaling. If laboratory torsion balance experiments or Solar System ephemeris tests detect inverse-square-law deviations that scale as $a^{-1}$ (MOND-like) rather than $a^{-\alpha}$ (causal-response scaling in Eq.~\ref{eq:weight} with $\alpha=0.389$), the present power-law enhancement would be strongly disfavored. This would motivate a different functional form for the weight function, such as an interpolation function $\mu(a/a_0)$ similar to MOND.

%(2) Wrong power-law scaling. If laboratory torsion balance experiments or Solar System ephemeris tests detect inverse-square-law deviations that scale as $a^{-1}$ (MOND-like) rather than $a^{-0.39}$ (causal-response prediction),
%{\color{black}the power-law form $w(r)-1 \propto (a_0/a_{\rm baryon})^\alpha$ with $\alpha = 0.389$ would be strongly disfavored.} This would motivate a different functional form for the weight function, such as an interpolation function $\mu(a/a_0)$ similar to MOND.
%{\color{black} No quantitative torsion-balance or ephemeris prediction is derived in this manuscript; this item refers to the exponent-level scaling implied by the fitted $\alpha$.}

%(3) Cosmological inconsistency. If cosmological observations (redshift-space distortions, CMB constraints, supernovae distance moduli) show that growth rates match $\Lambda$CDM to $< 1\%$ precision while other probes suggest order-unity deviations in the inferred gravitational strength at intermediate scales, this indicates a scale-dependent cutoff in the causal-response mechanism. Such a result would not strictly falsify the galactic-scale phenomenology but would require a theoretical explanation for why the memory kernel operates at galactic scales ($\sim 10$ kpc) but not cosmological scales ($\sim 100$ Mpc), possibly pointing to a length-scale cutoff in the response function.



\subsection{Comparison with Literature Values}
\label{sec:comparison}

For a few well-studied galaxies, Table~\ref{tab:literature} compares our
global-only fits to representative literature fits that typically allow per-galaxy tuning.
These examples illustrate the tradeoff between per-galaxy flexibility and global constraint: literature fits can achieve lower $\chi^2/N$ for selected objects by introducing galaxy-specific freedom, while our protocol holds all parameters fixed globally.
The outliers and best-fit cases are consistent with the model assumptions: gas-rich dwarfs tend to require larger enhancement, while strongly barred or otherwise non-axisymmetric systems are expected to be challenging for an axisymmetric steady-state model.


\begin{table}[ht]
\centering
\caption{Comparison with literature for individual galaxies. The
   CR  model uses global-only parameters; literature values often include per-galaxy tuning.}
\label{tab:literature}
\begin{tabular}{l c c c}
\toprule
Galaxy & Model $\chi^2/N$ & Literature $\chi^2/N$ & Reference \\
\midrule
NGC 3198 & 1.89 & 0.90 (MOND, fitted $M/L$) & \citet{Begeman1991} \\
DDO 161 & 0.48 & 1.20 (MOND) & \citet{Lelli2016} \\
NGC 2403 & 0.90 & 1.45 (MOND, fitted $M/L$) & \citet{Fraternali2011} \\
NGC 7814 & 0.80 & 1.25 (MOND) & Gentile+ 2011 \\
\midrule
Sample Median & 1.19 & 1.45 & --- \\
\bottomrule
\end{tabular}
\end{table}




\section{Conclusion}
\label{sec:conclusion}

We have presented a causal-response (CR) model for galactic rotation curves parameterized by a globally shared weight function $w(r)$ with memory timescale $\tau_\star \approx 133$ Myr. Applied to the SPARC Q=1 subset (99 galaxies) under strict global-only constraints, the model achieves median $\chi^2/N = 1.19$ versus 1.79 for MOND---a 33\% improvement with zero per-galaxy tuning. The model naturally reproduces the RAR ($\sigma=0.13$ dex) and BTFR ($\beta=3.5$, $\sigma=0.18$ dex) as emergent consequences, predicts stronger enhancement in dwarfs ($w\approx 2.2$) than spirals ($w\approx 1.5$), and eliminates the $f_{\rm gas}$ residual trend present in MOND.

\subsection{Limitations and outlook}

The microscopic origin of $w(r)$ remains open. Appendix~\ref{app:action} provides a conservative Caldeira-Leggett realization, but the physical identity of the ``bath'' degrees of freedom is unspecified (Sec.~\ref{sec:microphysical}). The phenomenological functional form $w(r) = 1 + \xi \cdot n(r) \cdot (a_0/a_{\rm baryon})^\alpha \cdot \zeta(r)$ [Eq.~(\ref{eq:weight})] is not arbitrary: in forthcoming work, we demonstrate that the fitted parameters---including the power-law exponent $\alpha$, the characteristic acceleration $a_0$, and the radial profile structure---can be derived from a single theoretical axiom (the Recognition-Composition Law) with zero adjustable parameters. The present paper validates these predictions empirically; the full derivation will be presented in a companion manuscript.

The model has been tested only on galactic scales (1--100 kpc); cosmological implications require a relativistic completion. The current implementation assumes axisymmetric steady-state rotation; outliers such as barred spirals (e.g., UGC 2885, $\chi^2/N = 14.90$) motivate extensions incorporating non-axisymmetric corrections and time-dependent potentials. 

The most critical near-term test is galaxy clusters: if the CR enhancement mechanism operates on cluster scales ($\sim 1$ Mpc, dynamical times $\sim 1$--$3$ Gyr), the model predicts velocity dispersion boosts of $\sim 30$--$50\%$ over Newtonian expectations without invoking dark matter, testable via upcoming weak lensing surveys (Euclid, Rubin Observatory) and X-ray observations. A null result would definitively rule out the present formulation and require either a scale-dependent cutoff in the response kernel or a hybrid model incorporating additional matter components. Future work includes deriving $w(r)$ from microphysics, developing a covariant completion, testing on galaxy clusters and weak lensing, and tightening laboratory/Solar-System constraints.

















\newpage
\clearpage

\begin{figure}[p]  % figure 1
\centering
\includegraphics[width=0.95\textwidth]{fig-01/frg_memory_kernel}
\caption{Memory kernel and response function for the single-timescale model. Left: Exponential kernel $\Gamma(\tau)=(w_{\rm steady}-1)\,e^{-\tau/\tau_\star}/\tau_\star$. Right: Response function $C(\omega)$. At low frequencies ($\omega \tau_\star \ll 1$), $C(\omega)\to w_{\rm steady}$ (enhanced response); at high frequencies ($\omega \tau_\star \gg 1$), $C(\omega)\to 1$ (Newtonian limit).}
\label{crg:memory_kernel}
\end{figure}


\begin{figure}[p]   % figure 2
\centering
\includegraphics[width=0.48\textwidth]{fig-02/frg_xi_factor}
\caption{Complexity factor $\xi$ as a function of gas fraction $f_{\rm gas}$ for all 99 Q=1 galaxies.
  Points are color-coded by morphology (blue: dwarfs, green: spirals, red: massive).
  The dashed black curve shows the fitted relation $\xi = 1 + C_\xi \sqrt{u_b}$,
  where $u_b$ is the binned gas fraction (fitted parameters in Sec.~\ref{sec:global_parameters}).
%  Gas-rich dwarfs naturally exhibit higher $\xi$ values,
%  providing a stronger gravitational enhancement and explaining their larger mass discrepancies
%  without invoking more dark matter.
  %The binning into five quintiles (vertical spacing of points)
  %avoids overfitting while capturing the overall trend.
}
\label{crg:xi_factor}
\end{figure}


\begin{figure}[p]   % figure 3
%\centering
\includegraphics[width=0.48\textwidth]{fig-03/frg_n_profile}
\includegraphics[width=0.48\textwidth]{fig-03/zeta_geometric_correction}
\caption{Left: Spatial profile $n(r)$ with parameters from the global fit (see Sec.~\ref{sec:global_parameters}).
  In the analysis pipeline, $n(r)$ is normalized by the disk-weighted mean $\langle n \rangle$ for each galaxy
  [Eq.~(\ref{eq:n_normalization})] to enforce strict global-only fitting without hidden per-galaxy mass rescaling.
  Right: Geometric correction factor $\zeta(r)$ for a typical disk galaxy with scale length $R_d = 3$ kpc, 
  marked by a purple vertical line. 
}
\label{crg:n_profile}
\end{figure}


\newpage
\clearpage


\begin{figure}[p]  % figure 4
\centering
\includegraphics[width=0.88\textwidth]{fig-04/frg_rotation_curves}
\caption{Representative rotation curves for four SPARC galaxies spanning three orders of magnitude in stellar mass.
  Points show observed velocities with $1\sigma$ error bars. Blue solid line-
   CR  model prediction, red dashed line- MOND fit, green dotted line- Newtonian baryon-only prediction.
  Top left: DDO 161 (dwarf, $M_\star \sim 10^7 \, \mathrm{M}_\odot$).
  Top right: NGC 2403 (low surface brightness spiral).
  Bottom left: NGC 3198 (spiral, $M_\star \sim 10^{10} \, \mathrm{M}_\odot$).
  Bottom right: NGC 7814 (massive, $M_\star \sim 10^{11} \, \mathrm{M}_\odot$).
  }
\label{crg:rotation_curves}
\end{figure}



\begin{figure}[p]    % figure 5
\centering
\includegraphics[width=0.48\textwidth]{fig-05/frg_weight_profiles}
\caption{Radial profiles of the gravitational enhancement $w(r)$ for the four representative galaxies
  from Fig.~\ref{crg:rotation_curves}. }
\label{crg:weight_profiles}
\end{figure}


\newpage
\clearpage


\begin{figure}[p] % figure 6
\centering
\includegraphics[width=0.48\textwidth]{fig-06/frg_rar_combined}
%\includegraphics[width=0.48\textwidth]{fig-06/frg_rar_a}
%\includegraphics[width=0.48\textwidth]{fig-06/frg_rar_b}
\caption{Radial acceleration relation (RAR) for the Q=1 SPARC sample.
  The x-axis shows the baryonic acceleration $a_{\rm baryon} = v_{\rm baryon}^2/r$ and the y-axis
  shows the observed acceleration $a_{\rm obs} = v_{\rm obs}^2/r$.
  Each point represents a single radial data point in a galaxy, color-coded by kinematic type:
  blue (dwarfs), green (spirals), and red (massive).
  Solid black curve- CR model, red dashed curve- MOND prediction.
  % (, $a_0=1.23 \times 10^{-10}$ m/s$^2$). Both models reproduce the tight observed correlation equally well.}
}
\label{crg:rar}
\end{figure}




\begin{figure}[p]   % figure 7
\centering
\includegraphics[width=0.4\textwidth]{fig-07/frg_btfr_improved}
\caption{ BTFR formalism for the Q=1 SPARC sample showing the dependence of the total baryonic mass
  $M_{\rm baryon} = M_\star + M_{\rm gas}$ (stars + gas) on the flat rotation velocity $v_{\rm flat}$.
  Each scatter point belongs to one galaxy, and a similar color map by kinematic type is used  as in Fig.~\ref{crg:rar}.
  Solid line-  CR  model, dashed red line- MOND data. 
}
\label{crg:btfr}
\end{figure}

\newpage
\clearpage



\begin{figure}[p]    % figure 8
\includegraphics[width=0.48\textwidth]{fig-08-new/frg_dwarf_spiral}
\caption{ The distribution of the weight function $w(r)$ evaluated at the disk scale length ($r=R_d$)
  for dwarfs (blue) and spirals (green). Red dots are for the median of the corresponding distributions. }
\label{crg:dwarf_spiral}
\end{figure}




\begin{figure}[ht]    % figure 9
\centering
\includegraphics[width=0.4\textwidth]{fig-09-new/frg_transition_scaling_corrected}
\caption{Scaling of the transition radius $R_{\rm trans}$, where deviations from Newtonian exceed 20\%,
  with flat rotation velocity $v_{\rm flat}$. The observed data- kinematics-based colored points,  
  the blue solid line- the  CR behavior, and the dotted red line- the MOND prediction.
  }
\label{crg:transition_scaling}
\end{figure}


\newpage
\clearpage


\begin{figure}[t]   % figure 10
\centering
\includegraphics[width=0.48\textwidth]{fig-10/frg_residuals_a}
\includegraphics[width=0.42\textwidth]{fig-10/frg_residuals_b}
\caption{Residual analysis for the  CR model fits. Normalized residuals are defined as $\delta = (v_{\rm obs} - v_{\rm model})/\sigma_{\rm eff}$. Left: Histogram of normalized residuals after excluding outlier galaxies with $\chi^2/N>5$, overlaid with a best-fit skew-normal distribution (red curve). The fitted mean $\mu \approx -0.24$ and standard deviation $\sigma \approx 1.11$ indicate a small negative bias (mild over-prediction) with near-unit width. Right: Quantile-quantile (Q-Q) plot comparing empirical residual quantiles (vertical axis) to theoretical skew-normal quantiles (horizontal axis). Points follow the 1:1 line (dashed) closely in the central range; mild tail deviations reflect a small fraction of complex systems.}
\label{crg:residuals}
\end{figure}





\begin{figure}[t]   % figure 11
\centering
\includegraphics[width=0.68\textwidth]{fig-11/frg_chi2_dist}
\caption{Distribution of the reduced chi-squared metric $\chi^2/N$ across the Q=1 SPARC subset ($N=99$ galaxies).
  Blue histogram-   CR  model, 
  red histogram-  the  MOND baseline, and green histogram- the overlapping of blue and red histograms.
  Vertical dashed lines indicate the median values: $1.19$ for the  CR  and $1.79$ for the MOND.
  }
\label{crg:chi2_dist}
\end{figure}



\newpage
\clearpage









\begin{acknowledgments}
The authors thank collaborators and the community for discussions on rotation-curve analyses and fairness policies. No external funding was received. The authors declare no competing interests.
\end{acknowledgments}

\section*{Data Availability}

 The Spitzer Photometry \& Accurate Rotation Curves (SPARC) dataset is publicly available at \url{http://astroweb.case.edu/ssm/SPARC/} \citep{Lelli2016}. We use the high-quality subset (Q=1, $N=99$ galaxies) for all primary results reported in this work.

 All analysis scripts, optimization routines, and figure generation code are available at
 %\url{https://github.com/.../gravity}.
\url{https://github.com/jonwashburn/gravity}.
 %The repository includes: (1) \texttt{pure\_global\_eval.py}, the core solver implementing the causal-response model with vectorized rotation curve evaluation; (2) \texttt{optimize\_params.py}, global parameter optimization using Differential Evolution with parallel evaluation; (3) \texttt{generate\_figures.py}, automated figure generation for all main-text plots with consistent styling; (4) \texttt{calculate\_stats.py}, statistical analysis routines including Wilcoxon signed-rank tests, RMS residual calculations, and dynamical timescale computation; (5) \texttt{validation\_*.py}, cross-validation and morphology-blind robustness tests (Appendix~\ref{app:robustness}); and (6) \texttt{reproduce\_results.sh}, a shell script to reproduce the entire analysis pipeline from raw SPARC data to final figures.

Processed Data. Intermediate data products (master tables with fitted parameters, per-galaxy metrics including $\chi^2/N$ and residual statistics, summary CSV files with morphology classifications and kinematic properties) are archived in the repository under \texttt{external/gravity/active/scripts/}. All data files are accompanied by SHA256 checksums for verification.

%Docker Support. A Docker container with all software dependencies (Python 3.9, NumPy 1.21, SciPy 1.7, Matplotlib 3.5) and pre-configured environment is available for full computational reproducibility. See the repository README for installation instructions and usage examples.

\appendix






\section{Error Model Details}
\label{app:error_model_details}

This appendix provides detailed functional forms and physical motivations for the systematic uncertainty terms in Eq.~(\ref{eq:error_model}).

\subsection*{Beam-smearing effects}

The beam-smearing term $\sigma_{\rm beam}$ models spatial resolution effects that arise when the telescope beam averages over regions with velocity gradients. For a galaxy with disk scale length $R_d$, we model the beam-smearing contribution as
\begin{equation}
\sigma_{\rm beam,i,j} = \alpha_{\rm beam} \cdot \frac{b_{\rm kpc} \cdot v_{\rm obs,i,j}}{r_i + b_{\rm kpc}}
\end{equation}
where $b_{\rm kpc} \approx 0.3 R_d$ is the effective beam size in kpc and $\alpha_{\rm beam} = 0.3$ is a dimensionless smearing coefficient. This term is largest in the inner regions where velocity gradients are steep and becomes negligible in the outer disk.

\subsection*{Turbulence and warp proxy}

The turbulence term $\sigma_{\rm turb}$ describes disk warps, vertical motions, and turbulent velocity fields that increase with radius as disk thickness grows. We model this using an empirical radial profile
\begin{equation}
\sigma_{\rm turb,i,j} = k_{\rm turb} \cdot v_{\rm obs,i,j} \cdot \left[1 - \exp\left(-\frac{r_i}{R_d}\right)\right]^{p_{\rm turb}}
\end{equation}
with $(k_{\rm turb}, p_{\rm turb}) = (0.07, 1.3)$. This functional form ensures $\sigma_{\rm turb} \to 0$ in the inner disk (where turbulence is suppressed by strong shear) and grows in the outer regions where warps and vertical structure become significant.

\subsection*{Spiral density waves}

Spiral density waves can perturb line-of-sight velocities by $\sim 10$--$30$ km/s \citep{deBlok2010,Lelli2016,Oh2015,McGaugh2016}, but these effects average to near zero over an orbit and cannot supply a net centripetal boost. The discrepancy persists in galaxies with weak or absent spiral structure, confirming that streaming motions are a modest systematic contribution captured by the error model above.


\section{Action-based conservative realization (Caldeira-Leggett construction)}
\label{app:action}

We exhibit a strictly causal, conservative realization whose linear response reproduces the transfer function $H(i\omega)$. This construction follows the Caldeira-Leggett formalism for dissipative quantum systems, adapted to the gravitational context. Work in the Newtonian limit with baryonic potential $\Phi_{\rm baryon}$ (Poisson: $\nabla^2\Phi_{\rm baryon}=4\pi G\rho_{\rm baryon}$). Introduce an auxiliary field $X$ with action
\begin{equation}
  S=\int dt\, d^3x\, \Big[ \tfrac{1}{8\pi G}\,|\nabla\Phi_{\rm baryon}|^2 +
    \tfrac{\kappa}{2}\,X^2 + g\, X\,\Phi_{\rm baryon} + \int_0^\infty d\Omega\;
    \tfrac{1}{2}\big(\dot{q}_\Omega^2 - \Omega^2 q_\Omega^2\big) + X\,\int_0^\infty d\Omega\, c(\Omega)\, q_\Omega \Big] 
\end{equation}
with a bath of harmonic modes $q_\Omega$ (Caldeira–Leggett construction). Integrating out $\{q_\Omega\}$ yields a causal equation for $X$ with memory determined by the nonnegative spectral density $J(\Omega)=\tfrac{\pi}{2} c(\Omega)^2/\Omega$. The linear response of $X$ to $\Phi_{\rm baryon}$ is
\begin{equation}
 X(\omega)=\chi(\omega)\, \Phi_{\rm baryon}(\omega),\qquad \chi(\omega)=g\,\frac{\kappa + \Sigma(\omega)}{\kappa\big(\kappa+\Sigma(\omega)\big)-g^2}, \quad \Sigma(\omega)=\int_0^\infty \! d\Omega\,\frac{2\Omega J(\Omega)}{\Omega^2-\omega^2 - i0^+} 
\end{equation}
Define the effective potential $\Phi_{\rm eff}=\Phi_{\rm baryon}+\alpha X$ with constant $\alpha$. Then $a_{\rm eff}=-\nabla\Phi_{\rm eff}=H(i\omega)\, a_{\rm baryon}$ with $H(i\omega)=1+ \alpha\, \chi(\omega)$. Choosing $J(\Omega)$ to be a single Debye pole, $J(\Omega)=\Delta\,\Omega\,\delta(\Omega-\tau_\star^{-1})$ with $\Delta\ge 0$, one obtains
\begin{equation}
 H(i\omega)=1+\frac{w(r)-1}{1+i\omega\tau_\star},\qquad w(r)=1+\frac{\alpha g}{\kappa} 
\end{equation}







\subsection{Validation tests}

We perform several validation checks to ensure the model's robustness and physical meaningfulness:

(1) Leave-one-out cross-validation. We sample 10 galaxies, re-optimize parameters with each galaxy removed, and test on the held-out galaxy. The mean degradation is $\langle \Delta \chi^2/N \rangle = +0.05 \pm 0.08$, indicating minimal overfitting. The small positive shift suggests the model generalizes well beyond the training set under this protocol. Note that this is a computationally constrained proxy for full LOOCV (which would require re-optimizing $N=99$ times). It nevertheless preserves the train/test separation and is sufficient to detect overfitting at the population level.

(2) Morphology-blind test. Randomizing the $\xi$ assignments (shuffling $f_{\rm gas}$ values across galaxies) degrades the median $\chi^2/N$ from $1.19$ to $1.52 \pm 0.08$ (averaged over 10 trials), indicating that gas fraction carries predictive information under this protocol. This $+28\%$ degradation shows that the morphological dependence is important to the model's performance and not merely an arbitrary degree of freedom.

(3) Target-blindness check. The complexity factor $\xi$ bin edges are frozen using only baryonic quantities ($f_{\rm gas}$), with no knowledge of rotation curve residuals, ensuring no circular reasoning or data leakage in the morphology parameterization.

(4) Residual analysis. Normalized residuals $(v_{\rm obs} - v_{\rm model})/\sigma_{\rm eff}$ show mild negative skew; a skew-normal fit gives mean $\mu \approx -0.24$ and standard deviation $\sigma \approx 1.11$ (Figure~\ref{crg:residuals}).

(5) Parameter stability. The optimization converges to parameter values within $\pm 5\%$ when initialized from different starting points (10 trials), suggesting the solution is not a fragile local optimum. The median $\chi^2/N$ varies between $1.13$ and $1.25$ across trials, well within the sensitivity ranges (Appendix~\ref{app:sensitivity}), indicating stable convergence.


Note on parameter uncertainties. Formal parameter uncertainties would require bootstrap resampling or Markov Chain Monte Carlo analysis, which is computationally expensive for this global optimization problem with 7 parameters and 99 galaxies. The sensitivity analysis (Appendix~\ref{app:sensitivity}) demonstrates that the model performance degrades smoothly when parameters are varied, with typical $\Delta(\chi^2/N) \sim 0.3$--0.8 for $\pm 20\%$ parameter variations. This suggests parameter uncertainties of order 10--15\% based on the curvature of the $\chi^2$ surface.



\subsection{Reproducibility}

To ensure determinism and auditability, we preregister and freeze the beam-smearing proxy (including the effective beam scale used in $\sigma_{\rm beam}$), constant floors, single global stellar $M/L$, and the complete $w(r)$ specification prior to analysis. Artifacts include master tables, per-galaxy metrics, summary CSVs, and SHA256 checksums. Code and Docker support:
\url{https://github.com/.../gravity}.
%\url{https://github.com/jonwashburn/gravity}.


\section{Parameter Sensitivity Analysis}
\label{app:sensitivity}

We test robustness by varying each global parameter individually while holding others fixed at their optimal values. For each parameter, we compute median $\chi^2/N$ across a grid of values spanning the physically reasonable range. Figure~\ref{crg:sensitivity} shows the sensitivity curves for the most influential parameters.

\begin{figure}[ht]   % figure 12
\centering
\includegraphics[width=0.65\textwidth]{fig-12/frg_sensitivity}
\caption{Sensitivity of median $\chi^2/N$ to key parameters. Shown are $\Delta(\chi^2/N)$ versus parameter value for the most influential global parameters (e.g., $\alpha$ and global $M/L$). The curves show that the fiducial parameter values lie near a minimum of the reported objective under the stated protocol. Detailed sensitivity data are provided in the artifact bundle.}
\label{crg:sensitivity}
\end{figure}

\subsection*{Error-model hyperparameters}

In addition to the physical parameters, we assessed robustness to reasonable changes in the observational error model. Perturbing each hyperparameter by $\pm 20\%$ (noise floor $\sigma_0$, fractional floor $f_{\rm floor}$, beam-smearing coefficient $\alpha_{\rm beam}$, asymmetric-drift fractions for dwarfs/spirals, and turbulence coefficients $K_{\rm turb}, P_{\rm turb}$) changes the median $\chi^2/N$ by $\Delta \in [ +0.03, +0.12 ]$, with the Causal-Response vs. MOND ranking unchanged in all cases. A script reproducing these tests (\texttt{error\_model\_sensitivity.py}) is included in the repository.

\begin{table}[ht]
\centering
\caption{Sensitivity of median $\chi^2/N$ to parameter variations. Fiducial values yield median $\chi^2/N = 1.19$. $\Delta(\chi^2/N)$ shows degradation relative to fiducial.}
\label{tab:sensitivity}
\begin{tabular}{l c c c c}
\toprule
Parameter & Fiducial & Range tested & $\Delta(\chi^2/N)$ & Robust? \\
\midrule
$\alpha$ & 0.389 & $[0.30, 0.45]$ & $+0.3$ to $+0.8$ & Moderate \\
$A$ (in $n(r)$) & 1.06 & $[0.8, 1.3]$ & $+0.1$ to $+0.4$ & Yes \\
$r_0$ (in $n(r)$) & 17.79 kpc & $[15.0, 20.0]$ & $+0.2$ to $+0.6$ & Moderate \\
$p$ (in $n(r)$) & 0.95 & $[0.7, 1.2]$ & $+0.1$ to $+0.3$ & Yes \\
$C_\xi$ & 0.298 & $[0.20, 0.40]$ & $+0.1$ to $+0.4$ & Yes \\
$M/L$ & 1.0 & $[0.7, 1.3]$ & $+0.3$ to $+0.9$ & Moderate \\
\bottomrule
\end{tabular}
\end{table}

The model is most sensitive to $\alpha$ and $M/L$, with $\pm 20\%$ variations degrading fits by $\sim 15$--30\%. This indicates these parameters are physically meaningful rather than arbitrary. The spatial profile parameters $(A, r_0, p)$ show good robustness, suggesting the functional form captures the essential radial dependence.

Importantly, in this one-at-a-time sensitivity sweep no parameter variation improves the median fit beyond the fiducial values, suggesting that the reported parameters lie near a good optimum for the chosen objective. The empirical performance is maximized near $\alpha \approx 0.389$ in this dataset and protocol.

\section{BTFR scaling from Freeman's law (back-of-the-envelope)}
\label{app:btfr_scaling}

For completeness we record a standard scaling argument explaining why the BTFR slope $\beta$ is near $4$ in idealized limits. For a flat rotation curve at radius $r$, Newtonian dynamics gives
\begin{equation}
  v^2 \approx \frac{G M_{\rm enc}(r)}{r}
  \quad \Rightarrow \quad
  M_{\rm enc} \propto v^2 r .
\end{equation}
If disk galaxies had exactly constant surface brightness $\Sigma$ (Freeman's law; approximate empirically), then the baryonic mass within radius $r$ would scale as $M \sim \Sigma \pi r^2$, implying $r \propto \sqrt{M}$. Substituting into the previous relation yields
\begin{equation}
  M \propto v^2 \sqrt{M}
  \quad \Rightarrow \quad
  M \propto v^4 .
\end{equation}
This gives $\beta = 4$ in the idealized limit; in real samples Freeman's law is only approximate and the observed BTFR slope is typically $\beta \simeq 3.5$--$4$.









\section{Robustness to Data Quality (Q1+Q2)}
\label{app:robustness}

To verify the robustness of our results, we repeated the entire analysis using the combined Q=1 and Q=2 SPARC dataset ($N=163$ galaxies). This explicitly tests whether the model's performance is sensitive to data quality or selection effects. Figures~\ref{crg:rotation_curves_q1q2}--\ref{crg:sensitivity_q1q2} demonstrate that all key results remain stable when extending to the larger sample.

\emph{Note:} At the editor's discretion, Figures~\ref{crg:rar_q1q2}, \ref{crg:btfr_q1q2}, \ref{crg:dwarf_spiral_q1q2}, \ref{crg:residuals_q1q2}, and \ref{crg:sensitivity_q1q2} may be moved to supplementary online material to reduce the printed appendix length while maintaining full documentation of the robustness tests.

\begin{figure}[t]   % appendix fig 13
\centering
\includegraphics[width=0.68\textwidth]{fig-13/frg_rotation_curves_q1q2}
\caption{Validation of rotation curve fits using the combined Q=1 and Q=2 dataset. The curves for the same representative galaxies remain similar, indicating that including Q=2 data does not visibly degrade the agreement for these systems.}
\label{crg:rotation_curves_q1q2}
\end{figure}

\begin{figure}[t]   % appendix fig 14
\centering
\includegraphics[width=0.48\textwidth]{fig-14/frg_rar_q1q2}  % 0.68
\caption{Radial Acceleration Relation extended to the combined Q=1 and Q=2 dataset. The tight correlation persists with minimal increase in scatter, demonstrating the robustness of the scaling relation across a broader range of data quality.}
\label{crg:rar_q1q2}
\end{figure}

\begin{figure}[t]  % appendix fig 15
\centering
\includegraphics[width=0.48\textwidth]{fig-15/frg_btfr_q1q2} % 0.78
\caption{Baryonic Tully-Fisher Relation for the combined Q=1 and Q=2 sample ($N=163$). The power-law scaling remains consistent with the Q=1 subset, with minimal additional scatter.}
\label{crg:btfr_q1q2}
\end{figure}

\begin{figure}[t]  % appendix fig 16
\centering
\includegraphics[width=0.48\textwidth]{fig-16/frg_dwarf_spiral_q1q2}    % 0.68
\caption{Replication of the dwarf vs. spiral enhancement analysis using the extended Q=1 and Q=2 sample ($N=163$). The predicted enhancement ratio and distribution overlap remain consistent with the Q=1 results, demonstrating that the morphological trend is robust to data quality variations.}
\label{crg:dwarf_spiral_q1q2}
\end{figure}

\begin{figure}[t]    % appendix figure 17
\centering
\includegraphics[width=0.48\textwidth]{fig-17/frg_residuals_a_q1q2}
\includegraphics[width=0.48\textwidth]{fig-17/frg_residuals_b_q1q2}
\caption{Residual analysis extended to the combined Q=1+Q=2 dataset. Left panel shows the normalized residual histogram overlaid with a best-fit skew-normal distribution. Right panel shows a Q--Q plot against skew-normal theoretical quantiles. The distribution remains well-behaved on the larger sample.}
\label{crg:residuals_q1q2}
\end{figure}

\begin{figure}[t]    % appendix  figure 18
\centering
\includegraphics[width=0.68\textwidth]{fig-18/frg_chi2_dist_q1q2}
\caption{Goodness-of-fit distribution for the combined Q=1+Q=2 dataset ($N=163$). The median $\chi^2/N$ improves slightly to 1.17, indicating that the performance is stable on the larger sample under the same protocol.}
\label{crg:chi2_dist_q1q2}
\end{figure}

\begin{figure}[t]   % appendix   figure 19
\centering
\includegraphics[width=0.65\textwidth]{fig-19/frg_sensitivity_q1q2}
\caption{Sensitivity panel shown in the same ``pillar + error bar'' style as Fig.~\ref{crg:sensitivity}. Bars indicate representative parameter midpoints and error bars indicate the tested ranges used in the one-at-a-time sensitivity summary; this panel is provided for visual comparison alongside the Q=1+Q=2 robustness suite.}
\label{crg:sensitivity_q1q2}
\end{figure}


%\newpage
%\clearpage



\newpage
\clearpage
\begin{thebibliography}{99}

\bibitem[Angus et al.(2008)]{Angus2007} Angus, G. W., Famaey, B., \& Buote, D. A. 2008, \textcolor{magenta}{MNRAS}, \textcolor{blue}{387, 1470}

\bibitem[Aprile et al.(2018)]{Aprile2018} Aprile, E., Aalbers, J., Agostini, F., et al. (XENON Collaboration) 2018, \textcolor{magenta}{PhRvL}, \textcolor{blue}{121, 111302}

\bibitem[Begeman et al.(1991)]{Begeman1991} Begeman, K. G., Broeils, A. H., \& Sanders, R. H. 1991, \textcolor{magenta}{MNRAS}, \textcolor{blue}{249, 523}

\bibitem[Bekenstein(1973)]{Bekenstein1973} Bekenstein, J. D. 1973, \textcolor{magenta}{PhRvD}, \textcolor{blue}{7, 2333}

\bibitem[Bekenstein(2004)]{Bekenstein2004} Bekenstein, J. D. 2004, \textcolor{magenta}{PhRvD}, \textcolor{blue}{70, 083509}

\bibitem[Bekenstein \& Milgrom(1984)]{Bekenstein1984} Bekenstein, J. D., \& Milgrom, M. 1984, \textcolor{magenta}{ApJ}, \textcolor{blue}{286, 7}

\bibitem[Berezhiani \& Khoury(2015)]{Berezhiani2015} Berezhiani, L., \& Khoury, J. 2015, \textcolor{magenta}{PhRvD}, \textcolor{blue}{92, 103510}

\bibitem[Bertone et al.(2005)]{Bertone2005} Bertone, G., Hooper, D., \& Silk, J. 2005, \textcolor{magenta}{PhR}, \textcolor{blue}{405, 279}

\bibitem[Binney \& Tremaine(2008)]{BinneyTremaine2008} Binney, J., \& Tremaine, S. 2008, Galactic Dynamics (2nd ed.; Princeton, NJ: Princeton Univ. Press)

\bibitem[Bosma(1981)]{Bosma1981} Bosma, A. 1981, \textcolor{magenta}{AJ}, \textcolor{blue}{86, 1825}

\bibitem[Boylan-Kolchin et al.(2011)]{Boylan2011} Boylan-Kolchin, M., Bullock, J. S., \& Kaplinghat, M. 2011, \textcolor{magenta}{MNRAS}, \textcolor{blue}{415, L40}

\bibitem[Bullock \& Boylan-Kolchin(2017)]{Bullock2017} Bullock, J. S., \& Boylan-Kolchin, M. 2017, \textcolor{magenta}{ARA\&A}, \textcolor{blue}{55, 343}

\bibitem[Caldeira \& Leggett(1983)]{Caldeira1983} Caldeira, A. O., \& Leggett, A. J. 1983, \textcolor{magenta}{PhyA}, \textcolor{blue}{121, 587}

\bibitem[de Blok(2010)]{deBlok2010} de Blok, W. J. G. 2010, \textcolor{magenta}{AdAst}, \textcolor{blue}{2010, 789293}

\bibitem[de Rham(2014)]{deRham2014} de Rham, C. 2014, \textcolor{magenta}{LRR}, \textcolor{blue}{17, 7}

\bibitem[Deser \& Woodard(2013)]{Deser2013} Deser, S., \& Woodard, R. P. 2013, \textcolor{magenta}{JPhA}, \textcolor{blue}{46, 214006}

\bibitem[Famaey \& McGaugh(2012)]{Famaey2012} Famaey, B., \& McGaugh, S. S. 2012, \textcolor{magenta}{LRR}, \textcolor{blue}{15, 10}

\bibitem[Fraternali et al.(2011)]{Fraternali2011} Fraternali, F., Sancisi, A., \& Kamphuis, P. 2011, \textcolor{magenta}{A\&A}, \textcolor{blue}{531, A64}

\bibitem[Freeman(1970)]{Freeman1970} Freeman, K. C. 1970, \textcolor{magenta}{ApJ}, \textcolor{blue}{160, 811}

\bibitem[Fujii \& Maeda(2003)]{Fujii2003} Fujii, Y., \& Maeda, K. 2003, The Scalar-Tensor Theory of Gravitation (Cambridge: Cambridge Univ. Press)

\bibitem[Holevo(1973)]{Holevo1973} Holevo, A. S. 1973, Problems of Information Transmission, \textcolor{blue}{9, 177}

\bibitem[Jacobson(1995)]{Jacobson1995} Jacobson, T. 1995, \textcolor{magenta}{PhRvL}, \textcolor{blue}{75, 1260}

\bibitem[Klypin et al.(1999)]{Klypin1999} Klypin, A., Kravtsov, A. V., Valenzuela, O., \& Prada, F. 1999, \textcolor{magenta}{ApJ}, \textcolor{blue}{522, 82}

\bibitem[Landauer(1961)]{Landauer1961} Landauer, R. 1961, IBM Journal of Research and Development, \textcolor{blue}{5, 183}

\bibitem[Lelli et al.(2016)]{Lelli2016} Lelli, F., McGaugh, S. S., \& Schombert, J. M. 2016, \textcolor{magenta}{AJ}, \textcolor{blue}{152, 157}

\bibitem[Lloyd(2002)]{Lloyd2002} Lloyd, S. 2002, \textcolor{magenta}{PhRvL}, \textcolor{blue}{88, 237901}

\bibitem[Maldacena(1998)]{Maldacena1999} Maldacena, J. M. 1998, Advances in Theoretical and Mathematical Physics, \textcolor{blue}{2, 231}

\bibitem[Margolus \& Levitin(1998)]{Margolus1998} Margolus, N., \& Levitin, L. B. 1998, \textcolor{magenta}{PhyD}, \textcolor{blue}{120, 188}

\bibitem[McGaugh(2012)]{McGaugh2012AJ} McGaugh, S. S. 2012, \textcolor{magenta}{AJ}, \textcolor{blue}{143, 40}

\bibitem[McGaugh et al.(2000)]{McGaugh2000} McGaugh, S. S., Schombert, J. M., Bothun, G. D., \& de Blok, W. J. G. 2000, \textcolor{magenta}{ApJL}, \textcolor{blue}{533, L99}

\bibitem[McGaugh et al.(2016)]{McGaugh2016} McGaugh, S. S., Lelli, F., \& Schombert, J. M. 2016, \textcolor{magenta}{PhRvL}, \textcolor{blue}{117, 201101}

\bibitem[Metzler \& Klafter(2000)]{MetzlerKlafter2000} Metzler, R., \& Klafter, J. 2000, \textcolor{magenta}{PhR}, \textcolor{blue}{339, 1}

\bibitem[Milgrom(1983)]{Milgrom1983} Milgrom, M. 1983, \textcolor{magenta}{ApJ}, \textcolor{blue}{270, 365}

\bibitem[Milgrom(2010)]{Milgrom2010} Milgrom, M. 2010, \textcolor{magenta}{MNRAS}, \textcolor{blue}{403, 886}

\bibitem[Moore et al.(1999)]{Moore1999} Moore, B., Ghigna, S., Governato, F., et al. 1999, \textcolor{magenta}{ApJL}, \textcolor{blue}{524, L19}

\bibitem[Navarro et al.(1997)]{Navarro1997} Navarro, J. F., Frenk, C. S., \& White, S. D. M. 1997, \textcolor{magenta}{ApJ}, \textcolor{blue}{490, 493}

\bibitem[Oh et al.(2015)]{Oh2015} Oh, S.-H., de Blok, W. J. G., Brinks, E., Walter, F., \& Kennicutt, R. C., Jr. 2015, \textcolor{magenta}{AJ}, \textcolor{blue}{149, 180}

\bibitem[Oman et al.(2015)]{Oman2015} Oman, K. A., Navarro, J. F., Fattahi, A., et al. 2015, \textcolor{magenta}{MNRAS}, \textcolor{blue}{452, 3650}

\bibitem[Planck Collaboration(2020)]{Planck2018} Planck Collaboration 2020, \textcolor{magenta}{A\&A}, \textcolor{blue}{641, A6}

\bibitem[Podlubny(1999)]{Podlubny1999} Podlubny, I. 1999, Fractional Differential Equations (San Diego, CA: Academic Press)

\bibitem[Price et al.(2005)]{PriceStornLampinen2005} Price, K., Storn, R., \& Lampinen, J. 2005, Differential Evolution: A Practical Approach to Global Optimization (Berlin: Springer)

\bibitem[Rubin \& Ford(1970)]{Rubin1970} Rubin, V. C., \& Ford, W. K., Jr. 1970, \textcolor{magenta}{ApJ}, \textcolor{blue}{159, 379}


\bibitem[Samko et al.(1993)]{Samko1993} Samko, S. G., Kilbas, A. A., \& Marichev, O. I. 1993, Fractional Integrals and Derivatives: Theory and Applications (Amsterdam: Gordon and Breach)

\bibitem[Sanders \& McGaugh(2002)]{Sanders2002} Sanders, R. H., \& McGaugh, S. S. 2002, \textcolor{magenta}{ARA\&A}, \textcolor{blue}{40, 263}

\bibitem[Shannon(1948)]{Shannon1948} Shannon, C. E. 1948, Bell System Technical Journal, \textcolor{blue}{27, 379}

\bibitem[Sotiriou \& Faraoni(2010)]{Sotiriou2010} Sotiriou, T. P., \& Faraoni, V. 2010, \textcolor{magenta}{RvMP}, \textcolor{blue}{82, 451}

\bibitem[Stein(1970)]{Stein1970} Stein, E. M. 1970, Singular Integrals and Differentiability Properties of Functions (Princeton, NJ: Princeton Univ. Press)

\bibitem[Storn \& Price(1997)]{StornPrice1997} Storn, R., \& Price, K. 1997, Journal of Global Optimization, \textcolor{blue}{11, 341}

\bibitem[Susskind(1995)]{Susskind1995} Susskind, L. 1995, Journal of Mathematical Physics, \textcolor{blue}{36, 6377}

\bibitem['t~Hooft(1993)]{tHooft1993} 't~Hooft, G. 1993, arXiv:gr-qc/9310026

\bibitem[Tully \& Fisher(1977)]{TullyFisher1977} Tully, R. B., \& Fisher, J. R. 1977, \textcolor{magenta}{A\&A}, \textcolor{blue}{54, 661}

\bibitem[Van Raamsdonk(2010)]{VanRaamsdonk2010} Van Raamsdonk, M. 2010, General Relativity and Gravitation, \textcolor{blue}{42, 2323}

\bibitem[Verlinde(2011)]{Verlinde2011} Verlinde, E. 2011, \textcolor{magenta}{JHEP}, \textcolor{blue}{2011, 29}

\bibitem[Verlinde(2017)]{Verlinde2017} Verlinde, E. 2017, SciPost Physics, \textcolor{blue}{2, 016}

\bibitem[Wheeler(1990)]{Wheeler1990} Wheeler, J. A. 1990, in Complexity, Entropy, and the Physics of Information, ed. W. H. Zurek (Boulder, CO: Westview Press), \textcolor{blue}{3}

\bibitem[Zwicky(1933)]{Zwicky1933} Zwicky, F. 1933, Helvetica Physica Acta, \textcolor{blue}{6, 110}

\end{thebibliography}

\end{document}
