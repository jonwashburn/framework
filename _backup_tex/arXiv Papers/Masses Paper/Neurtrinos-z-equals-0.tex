\documentclass[12pt]{article}

\usepackage{amsmath,amssymb}

\title{Neutrinos in Recognition Physics:\\
Why the Ladder Acts on Mass-Squared for \(Z=0\)}

\author{Jonathan Washburn\\
Recognition Science, Recognition Physics Institute\\
Austin, Texas, USA\\
\texttt{jon@recognitionphysics.org}
}

\date{}

\begin{document}

\maketitle

\begin{abstract}
Recognition Physics (RP) offers a parameter-free description of the Standard Model mass spectrum based on a discrete recognition ledger, a universal anchor scale \(\mu_\star\), and a golden-ratio ladder for mass exponents. In the charged sectors, each particle species is assigned an integer \(Z\) built from representation data, and the renormalization-group residue at \(\mu_\star\) collapses to a closed form \(f(Z)\). Masses then follow from a sector yardstick \(A_B\), an integer rung \(r_i\), and a universal dressing law, with no per-flavor tuning. This framework reproduces charged fermion and boson masses with a single anchor and a small set of integers.

Neutrinos sit at the edge of this picture. A naive extrapolation in an earlier specification treated neutrinos as another linear ladder sector \(m_i \propto E_{\mathrm{coh}}\varphi^{r_i}\), where \(E_{\mathrm{coh}}\) is the global coherence scale and \(\varphi\) is the golden ratio. That choice is incompatible with oscillation data, cosmological bounds on \(\sum m_\nu\), and direct limits on the effective electron-neutrino mass. In this paper we show that the failure is structural, not merely numerical: under a linear ladder \(m_i \propto \varphi^{r_i}\), no integer pattern of rung gaps can reproduce the observed oscillation ratio \(R = \Delta m^2_{31} / \Delta m^2_{21}\) for normal ordering.

We then derive, from RP first principles, why neutrinos must be treated differently. At the anchor scale \(\mu_\star\), neutrinos have vanishing electromagnetic and color motifs, so their charge integer \(Z_\nu = 0\) and the anchor residue \(f_\nu(\mu_\star,m_\nu)\) is exactly zero. In the recognition ledger, this means the linear (first-order) recognition channel for the neutrino mass term is forbidden. The lowest nonzero neutrino mass contribution is therefore quadratic in the underlying recognition amplitude. Translating this back into the ladder language, the neutrino sector is the unique Standard Model sector whose ladder must act on \(m^2\), not on \(m\).

We propose and analyze a mass-squared ladder for \(Z=0\) sectors of the form
\[
  m_i^2(\mu_\star) = Y_\nu^2\,\varphi^{r_i},
\]
where \(Y_\nu\) is a single yardstick for the entire neutrino sector and the \(r_i\) are the same integer rungs produced by the RP ribbon/braid constructor. This preserves the ``no per-flavor knobs'' principle. For such a ladder, the oscillation ratio becomes
\[
  R = \frac{\varphi^{d_2} - 1}{\varphi^{d_1} - 1},
\]
with \(d_1 = r_2 - r_1\) and \(d_2 = r_3 - r_1\). Unlike the linear case, there now exist integer gap pairs \((d_1,d_2)\) whose \(R\) values fall inside the experimentally allowed band. We exhibit a representative pattern \((r_1,r_2,r_3) = (k,k+4,k+11)\), fix \(Y_\nu\) from \(\Delta m^2_{21}\), and obtain a concrete spectrum with \((m_1,m_2,m_3)\) in the few-meV to 50~meV range, \(\sum m_\nu \approx 0.06~\mathrm{eV}\), and an effective electron-neutrino mass \(m_\beta\) at the order of \(10^{-2}\,\mathrm{eV}\).

The result is a minimal, recognition-theoretic modification of the neutrino sector that (i) explains why neutrinos are special in RP, (ii) resolves the previous neutrino ``no-go'' obstruction, (iii) preserves the integer and no-tuning structure of the mass framework, and (iv) yields sharp, discrete predictions for the oscillation ratio, \(\sum m_\nu\), and \(m_\beta\) that can be tested by oscillation experiments, cosmology, and β-decay measurements.
\end{abstract}

\section{Introduction}

Recognition Physics (RP) is a discrete, parameter-free framework that aims to reconstruct the observed constants and spectra of physics from a single recognition principle. The core structure is a recognition ledger: a double-entry bookkeeping of discrete recognition events that forces a convex cost functional, the golden ratio \(\varphi\), and an eight-beat discrete time structure. From this starting point, RP defines a universal coherence scale \(E_{\mathrm{coh}}\), a fundamental length and time scale, and an anchor energy \(\mu_\star\) at which renormalization-group quantities are evaluated in a canonical way.

In the mass sector, the basic RP claim is that Standard Model masses are not independent continuous parameters, but derived quantities tied to integer recognition data. Each species \(i\) in a given sector \(B\) is assigned:
\begin{itemize}
  \item an integer \(Z_i\) built from its representation data (charge, color, and sector),
  \item an integer rung \(r_i\) obtained from the discrete ribbon/braid constructor, and
  \item a sector yardstick \(A_B\) with units of energy.
\end{itemize}
At the anchor scale \(\mu_\star\), the integrated anomalous dimension (the mass renormalization residue) collapses to a simple \(\varphi\)-log function of \(Z_i\),
\begin{equation}
  f_i(\mu_\star, m_i) = \frac{\ln\bigl(1 + Z_i / \varphi\bigr)}{\ln \varphi},
\end{equation}
and the mass law for sector \(B\) takes the form
\begin{equation}
  m_i = A_B\,\varphi^{\,r_i + f_B(m_i)}.
\end{equation}
Here \(f_B\) is a sector-wide dressing law, typically built from standard-model renormalization-group kernels transported to the anchor. Crucially, there are no per-flavor knobs: once \(A_B\), the anchor, and the RG prescription are fixed, all masses in sector \(B\) are determined by their integer pairs \((Z_i, r_i)\).

This structure has been developed most fully in the charged sectors. For charged leptons and quarks, the integer map \(Z_i\) is a polynomial in the electric charge and color block, such as
\begin{equation}
  Z_{\text{quark}} = 4 + (6Q)^2 + (6Q)^4,\qquad
  Z_{\text{lepton}} = (6Q)^2 + (6Q)^4,
\end{equation}
while the rung integers \(r_i\) come from a finite dictionary of ribbon and braid motifs that encode gauge and generation structure. At the single anchor \(\mu_\star\), the residues computed from standard renormalization-group kernels match the \(\varphi\)-log form \(f(Z_i)\) to high precision. With a single anchor and a small number of sector yardsticks, the framework reproduces the observed charged fermion and boson masses without per-flavor tuning.

Neutrinos have been a persistent outlier in this picture. An early version of the RP specification treated neutrinos by simple analogy: assign a rung triplet \((r_1,r_2,r_3)\) from the constructor, reuse the global coherence scale \(E_{\mathrm{coh}}\) as the neutrino yardstick, and apply the same linear ladder rule
\begin{equation}
  m_i \propto E_{\mathrm{coh}}\varphi^{r_i}.
\end{equation}
That naive extrapolation produced neutrino masses in the eV to keV range and a summed mass \(\sum m_\nu\) many orders of magnitude above cosmological bounds. More subtly, it also hard-wired an oscillation ratio \(R = \Delta m^2_{31} / \Delta m^2_{21}\) that could not be reconciled with the band favored by oscillation experiments, regardless of how one chose the neutrino yardstick.

A later internal treatment of the neutrino sector exposed this as a structural issue. In the charged sectors the ladder acts directly on \(m\), the mass itself. For neutrinos, that assumption is not forced by the RP ledger, and it is in tension with the recognition data at the anchor. At \(\mu_\star\), neutrinos have no electromagnetic motifs (electric charge \(Q=0\)) and no color motifs (they are leptons). In the ribbon/braid language this means their charge integer is
\begin{equation}
  Z_\nu = 0,
\end{equation}
so the anchor residue
\begin{equation}
  f_\nu(\mu_\star, m_\nu) = \frac{\ln\bigl(1 + Z_\nu / \varphi\bigr)}{\ln \varphi}
\end{equation}
vanishes identically. The neutrino sector therefore enters the anchor in a qualitatively different way from any charged sector: the usual first-order recognition channel that generates a linear mass ladder is simply absent.

From the point of view of the recognition ledger, this is exactly what one expects for a neutral chiral mode. A single recognition insertion cannot close the ledger loop, because there is no net charge flow to balance. The lowest nonzero contribution to a neutrino mass term comes from a \emph{quadratic} recognition process: effectively, two insertions of the underlying recognition operator are required to produce a gauge-invariant, ledger-balanced mass. In more conventional language, this is analogous to the fact that the simplest neutrino mass term compatible with the Standard Model gauge group is a dimension-5 operator constructed from two lepton-Higgs factors rather than a single one.

This observation suggests a minimal and very specific modification to the neutrino sector in RP: the ladder for a neutral sector (\(Z=0\)) should not act on \(m\) but on \(m^2\). Concretely, at the anchor scale \(\mu_\star\) the neutrino masses should be organized as
\begin{equation}
  m_i^2 = Y_\nu^2\,\varphi^{r_i},
\end{equation}
where \(Y_\nu\) is a single yardstick for the entire neutrino sector and the \(r_i\) are the same integer rungs produced by the ribbon/braid constructor. This change leaves all of the RP discipline intact: the anchor residue rule \(f_\nu(\mu_\star)=0\) is respected; there are no per-species parameters; and all the integer data are inherited from the existing finite dictionary.

With this mass-squared ladder in place, the dimensionless oscillation ratio becomes
\begin{equation}
  R = \frac{\Delta m^2_{31}}{\Delta m^2_{21}}
    = \frac{\varphi^{d_2} - 1}{\varphi^{d_1} - 1},
\end{equation}
where \(d_1 = r_2 - r_1\) and \(d_2 = r_3 - r_1\) are integer gaps between rungs. Unlike the linear case, where the corresponding ratio
\begin{equation}
  R_{\mathrm{lin}} = \frac{\varphi^{2d_2} - 1}{\varphi^{2d_1} - 1}
\end{equation}
never enters the band favored by data for any integer pair \((d_1,d_2)\), the mass-squared ratio admits several integer solutions with \(R\) in the correct range. Once an integer pattern \((r_1,r_2,r_3)\) is chosen and the neutrino yardstick \(Y_\nu\) is fixed by a single splitting \(\Delta m^2_{21}\), all other neutrino observables---the second splitting, the summed mass \(\sum m_\nu\), and the effective electron-neutrino mass \(m_\beta\)---are determined with no additional freedom.

The purpose of this paper is to make that argument precise. In Section~2 we summarize the RP mass framework for charged sectors, focusing on the role of the anchor scale and the integer data \((Z_i,r_i)\). In Section~3 we reconstruct the naive neutrino ladder that appeared in the original Source specification and show how it fails both in absolute scale and in the oscillation ratio. In Section~4 we recast neutrino data in an RP-friendly way, emphasizing the ratio \(R\) as a pure test of the ladder's integer structure. Section~5 proves the ratio obstruction: for a linear ladder \(m_i \propto \varphi^{r_i}\), no integer choice of rung gaps matches the observed band for \(R\). In Section~6 we give a recognition-theoretic argument for a quadratic neutrino channel and derive the mass-squared ladder rule for \(Z=0\). Section~7 analyzes the resulting ratio for integer gaps and identifies the discrete patterns that match oscillation data. In Section~8 we work out a concrete example, fixing the yardstick from \(\Delta m^2_{21}\) and computing \((m_1,m_2,m_3)\), \(\sum m_\nu\), and \(m_\beta\). Section~9 checks that the proposed rule is consistent with RP discipline and clarifies what went wrong in the old Source block. Section~10 discusses falsifiability and future tests in oscillation experiments, cosmology, and β-decay, and Section~11 outlines extensions and open questions, including possible Majorana variants and connections to other RP sectors.

Throughout the paper we restrict attention to Dirac neutrinos and normal ordering for clarity. The core structural result---that \(Z=0\) sectors require a mass-squared ladder in RP---does not depend on these choices and should generalize to more elaborate neutrino scenarios within the same recognition framework.

\section{Recognition mass framework recap (charged sectors)}

The recognition mass framework starts from a small set of structural ingredients: the Meta-Principle, the recognition ledger, the convex cost \(J\), the golden ratio \(\varphi\), and the eight-tick discrete time structure. For the purposes of this paper we only need the part of this machinery that constrains renormalized masses at a fixed anchor scale.

The Meta-Principle asserts that nothing can recognize itself. Formalized on the recognition ledger, this forces a double-entry structure for recognition events and a unique convex cost functional \(J\) associated to each local configuration. The self-similarity of the cost under ledger refinement fixes the golden ratio \(\varphi\) as the unique positive solution to a simple recursion and induces an eight-tick discrete time step. From these ingredients one defines a universal coherence energy \(E_{\mathrm{coh}}\), a fundamental length and time scale, and a distinguished energy scale \(\mu_\star\) at which renormalization-group quantities are evaluated. The scale \(\mu_\star\) is the \emph{anchor}: it is the point at which the recognition ledger and the continuum fields are required to agree exactly.

For each Standard Model species \(i\), RP associates an integer \(Z_i\) that encodes its representation data as seen by the recognition ledger. The integer \(Z_i\) is constructed from:
\begin{itemize}
  \item the electric charge \(Q_i\),
  \item the color representation (distinguishing quarks from leptons),
  \item and the sector label \(B\) (up-type quark, down-type quark, charged lepton, gauge boson, and so on),
\end{itemize}
through a fixed motif dictionary. This dictionary packages the contributions of a small set of renormalization motifs (self-energy, exchange, vacuum polarization, and related structures) into a single integer. For charged fermions, for example, one finds
\begin{equation}
  Z_{\text{quark}} = 4 + (6Q)^2 + (6Q)^4,\qquad
  Z_{\text{lepton}} = (6Q)^2 + (6Q)^4,
\end{equation}
while for neutral Dirac neutrinos one has
\begin{equation}
  Z_\nu = 0,
\end{equation}
reflecting the absence of both electromagnetic and color motifs at the anchor.

At the anchor scale \(\mu_\star\), the integrated anomalous dimension (mass renormalization residue) for species \(i\) is then required to collapse to a simple \(\varphi\)-logarithmic function of its integer \(Z_i\),
\begin{equation}
  f_i(\mu_\star, m_i)
  = \frac{\ln\bigl(1 + Z_i / \varphi\bigr)}{\ln \varphi}.
\end{equation}
Here \(f_i\) is computed from the usual continuum renormalization-group kernels by integrating along the flow from \(\mu_\star\) down to the physical mass and rescaling into the recognition units. The RP statement is that, once the anchor and the renormalization prescription are fixed, this continuum calculation must agree with the closed-form expression in terms of \(Z_i\). This is the \emph{single-anchor identity}.

The mass formula itself is sector-wise. For a sector \(B\) (for example, up-type quarks, down-type quarks, or charged leptons), RP introduces a single energy scale \(A_B\) with the dimensions of mass. This sector yardstick sets the overall scale for that sector. Along with it, each species \(i\) in sector \(B\) carries an integer rung \(r_i\) that encodes its position on the recognition ladder. These rungs are not arbitrary: they are produced by a ribbon/braid constructor that maps the species' discrete recognition word into a small set of integer invariants:
\begin{itemize}
  \item a reduced word length \(L_i\), counting effective recognition motifs,
  \item a generation torsion \(\tau_g\), shifting rungs between generations in a fixed pattern,
  \item and a sector integer \(\Delta_B\), common to all members of the sector.
\end{itemize}
The rung is then a sum of these contributions,
\begin{equation}
  r_i = L_i + \tau_g + \Delta_B,
\end{equation}
with the precise form determined by the constructor rules but always integer-valued.

With these ingredients, RP defines the sector-wise mass law
\begin{equation}
  m_i = A_B\,\varphi^{\,r_i + f_B(m_i)}.
\end{equation}
Here \(f_B\) is the sector dressing, which collects the renormalization-group residue contributions for sector \(B\) into a single function that depends only on the mass scale, the anchor, and the renormalization policy. The crucial feature is that \(A_B\) and \(f_B\) are \emph{sector-global}. Once the anchor \(\mu_\star\), the renormalization policy, and the yardstick \(A_B\) are chosen, all masses in sector \(B\) are fixed by their integer rungs \(r_i\) and integer residues \(Z_i\).

This leads to the central ``no per-flavor knobs'' principle in the mass framework: there are no continuous, species-specific tuning parameters. The only continuous inputs are the anchor \(\mu_\star\), the sector yardsticks \(A_B\), and the global renormalization prescriptions. All flavor dependence enters through the discrete data \((Z_i, r_i)\) computed from representation and recognition structure. Modifying the anchor or a sector yardstick moves all masses in that sector coherently; individual masses cannot be adjusted independently.

In the charged sectors this structure is remarkably effective. Using a single anchor \(\mu_\star\), a universal motif dictionary for \(Z_i\), and one yardstick per sector, the RP mass framework produces compact formulas for the up-type quarks, down-type quarks, charged leptons, and the \(W\), \(Z\), and Higgs bosons. The predicted masses agree with the observed values to within small fractional errors that can be traced back to finite-loop and threshold choices in the continuum kernels. In other words, the charged sectors behave exactly as the recognition program demands: once the integer data \((Z_i,r_i)\) and a single sector scale are fixed, the spectrum is no longer a set of free inputs but a derived consequence of the recognition ladder.

\section{Neutrinos in the original RS specification}

Neutrinos were initially handled in RP by direct analogy with the charged sectors. In the original Source specification, the neutrino sector was assigned a formal rung triplet \((r_1,r_2,r_3)\) from the same ribbon/braid constructor that supplies the charged rungs. A concrete example used for illustration was
\begin{equation}
  (r_1,r_2,r_3) = (0,11,19),
\end{equation}
mirroring the generational offsets seen elsewhere in the ladder. The neutrino masses were then written as
\begin{equation}
  m_i = E_{\mathrm{coh}}\,\varphi^{r_i},
\end{equation}
where \(E_{\mathrm{coh}}\) is the global coherence energy scale fixed by the recognition ledger and \(\varphi\) is the golden ratio. In this treatment there was no separate neutrino yardstick \(A_\nu\), no distinct neutrino sector dressing \(f_\nu\), and, in particular, no acknowledgement of the fact that the neutrino charge integer is \(Z_\nu = 0\).

This naive extrapolation led to immediate problems. Since \(E_{\mathrm{coh}}\) is of order \(\varphi^{-5}\,\mathrm{eV}\), even the lowest rung neutrino mass \(m_1\) is of order \(10^{-1}\,\mathrm{eV}\), while higher rungs blow up rapidly with powers of \(\varphi\). For the example \((0,11,19)\), the second and third neutrinos acquire masses in the tens of eV and hundreds of eV range. The summed mass
\begin{equation}
  \sum_{i=1}^3 m_i
\end{equation}
is then of order \(10^2\) to \(10^3\,\mathrm{eV}\), completely incompatible with cosmological bounds, which require \(\sum m_\nu \lesssim 0.12\,\mathrm{eV}\). The effective electron-neutrino mass \(m_\beta\), defined by the usual admixture of \(m_i\) with the electron row of the mixing matrix, is likewise pushed up to the eV–hundreds of eV scale, far beyond what β-decay endpoints allow.

These failures already show that the specific parameter choices in the old block could not describe our universe. Even if one were to abandon the identification \(A_\nu = E_{\mathrm{coh}}\) and introduce a separate neutrino yardstick \(A_\nu\), the structure of the ladder itself would still impose strong constraints. In any multiplicative ladder of the form
\begin{equation}
  m_i = A_\nu \varphi^{r_i},\qquad r_1<r_2<r_3,
\end{equation}
the observable oscillation ratio
\begin{equation}
  R := \frac{\Delta m^2_{31}}{\Delta m^2_{21}}
\end{equation}
is determined solely by the integer gaps \(d_1 = r_2 - r_1\) and \(d_2 = r_3 - r_1\):
\begin{equation}
  \Delta m^2_{ab}
  = A_\nu^2\Big(\varphi^{2r_b} - \varphi^{2r_a}\Big),\qquad
  R
  = \frac{\varphi^{2d_2} - 1}{\varphi^{2d_1} - 1}.
\end{equation}
The yardstick \(A_\nu\) drops out of this ratio. As a result, once the rung gaps are fixed, the value of \(R\) is locked; changing \(A_\nu\) cannot move it.

Oscillation experiments determine \(\Delta m^2_{21}\) and \(\Delta m^2_{31}\) independently, and global fits favor a ratio \(R\) in the low-30s for normal ordering. In Section~5 we will show that, for the linear ladder \(m_i \propto \varphi^{r_i}\), there is no choice of integer gaps \((d_1,d_2)\) that produces a ratio in this band. The problem is therefore deeper than an unlucky choice of rungs \((0,11,19)\) or an overly rigid identification of the neutrino yardstick. The charged-sector rule \(m_i = A_B \varphi^{r_i + f_B(m_i)}\) simply does not transpose to the neutrino sector in its naive linear form if one insists on integer rungs and a single anchor.

The later neutrino-sector analysis within RP made this tension explicit. It treated the old neutrino block as provisional and effectively placed the neutrino sector in a ``no-go'' category under the original assumptions: linear ladder on \(m\), reuse of the coherence scale, and no special treatment for \(Z_\nu=0\). From a recognition-science viewpoint, that status is an invitation rather than a failure. It signals that the neutrino rules in the Source file were written before the full consequences of neutrality and the anchor structure were understood, and that the sector needs to be re-derived directly from the recognition ledger rather than by analogy.

The goal for the rest of this paper is therefore clear. We want to derive a new neutrino mass rule that:
\begin{itemize}
  \item is forced by RP structure, in particular by the condition \(Z_\nu=0\) and the recognition ledger for neutral chiral modes,
  \item preserves the ``no knobs'' principle: one neutrino yardstick, integer rungs from the existing constructor, and no per-flavor continuous parameters,
  \item and is consistent with all current neutrino data, including oscillation ratios, cosmological bounds on \(\sum m_\nu\), and β-decay limits on \(m_\beta\).
\end{itemize}
In the next sections we show that these requirements together point to a simple and structurally natural conclusion: the neutrino ladder in RP must act on \(m^2\), not on \(m\), making the neutrino sector the unique \(Z=0\) mass-squared ladder in the framework.

\section{Neutrino data constraints in RP-friendly form}

In order to test any proposed neutrino sector inside Recognition Physics, it is useful to collect the experimental constraints in a form that matches the structure of the ladder. Three ingredients are especially relevant: the ordering of the masses, the measured mass-squared splittings and their ratio, and the absolute scale bounds from cosmology and β-decay.

\subsection{Oscillation data}

Global fits to neutrino oscillation experiments favor the \emph{normal ordering}
\begin{equation}
  m_1 < m_2 < m_3,
\end{equation}
with two independent mass-squared splittings,
\begin{equation}
  \Delta m^2_{21} := m_2^2 - m_1^2,\qquad
  \Delta m^2_{31} := m_3^2 - m_1^2.
\end{equation}
Representative central values (in units of \(\mathrm{eV}^2\)) are of the form
\begin{equation}
  \Delta m^2_{21} \simeq 7.5\times 10^{-5},\qquad
  \Delta m^2_{31} \simeq 2.5\times 10^{-3},
\end{equation}
with percent-level uncertainties for the solar splitting and a few-percent uncertainty for the atmospheric splitting. The precise numbers are not critical for the present discussion; what matters is that there is a robust hierarchy,
\begin{equation}
  \Delta m^2_{31} \gg \Delta m^2_{21},
\end{equation}
and that the \emph{ratio}
\begin{equation}
  R_{\mathrm{exp}} := \frac{\Delta m^2_{31}}{\Delta m^2_{21}}
\end{equation}
is well constrained by data.

Using the representative values above, this ratio is
\begin{equation}
  R_{\mathrm{exp}} \simeq \frac{2.5\times 10^{-3}}{7.5\times 10^{-5}} \approx 33,
\end{equation}
and global fits favor a band of roughly
\begin{equation}
  R_{\mathrm{exp}} \in [30,35]
\end{equation}
for normal ordering, with the detailed limits depending on the fit and treatment of correlations. In what follows we will treat the interval \([30,36]\) as a conservative working band.

\subsection{Absolute scale constraints}

Oscillation experiments alone are insensitive to the overall neutrino mass scale, since they only measure differences of squared masses. Two other classes of experiments constrain the absolute scale.

First, cosmological observations of the cosmic microwave background and large-scale structure place an upper bound on the summed neutrino mass,
\begin{equation}
  \Sigma m_\nu := m_1 + m_2 + m_3.
\end{equation}
The current analyses are model-dependent, but a typical limit in standard cosmological models is
\begin{equation}
  \Sigma m_\nu \lesssim 0.1\text{--}0.2~\mathrm{eV},
\end{equation}
with many fits clustering near the tighter end of this range. Any viable neutrino mass model must therefore produce \(\Sigma m_\nu\) well below the eV scale.

Second, direct kinematic measurements of the electron-neutrino mass from tritium β-decay constrain the effective electron-neutrino mass
\begin{equation}
  m_\beta := \Bigl(\sum_{i=1}^3 |U_{ei}|^2 m_i^2 \Bigr)^{1/2},
\end{equation}
where \(U_{ei}\) are the elements of the first row of the PMNS mixing matrix. Current experiments reach the sub-eV regime and impose an upper bound of order
\begin{equation}
  m_\beta \lesssim \mathcal{O}(1~\mathrm{eV}),
\end{equation}
with planned improvements aiming at \(\mathcal{O}(10^{-1}~\mathrm{eV})\) and below.

A third quantity, the effective Majorana mass \(m_{\beta\beta}\) probed by neutrinoless double-β decay, will become important if neutrinos are Majorana particles. In this paper we restrict attention to Dirac neutrinos for clarity; the recognition-structural statements we make about the ladder can, however, be translated into the Majorana case in future work.

\subsection{Why the ratio \(R\) matters to RP}

Among these observables, the ratio
\begin{equation}
  R_{\mathrm{exp}} = \frac{\Delta m^2_{31}}{\Delta m^2_{21}}
\end{equation}
has a special status in Recognition Physics. In any sector described by a multiplicative ladder of the form
\begin{equation}
  m_i = A\,\varphi^{r_i},
\end{equation}
with a single sector yardstick \(A\) and integer rungs \(r_i\), the ratio \(R\) depends only on the rung \emph{gaps} \(r_j - r_k\). The yardstick \(A\) cancels out. Thus, in such a ladder, \(R\) is:
\begin{itemize}
  \item dimensionless,
  \item independent of the absolute scale \(A\),
  \item and determined purely by the integer structure of the ladder.
\end{itemize}

This makes \(R\) an ideal diagnostic for the neutrino ladder in RP. If the neutrino masses are to be organized by a recognition ladder with integer rungs and a single sector yardstick, then any proposed ladder rule must admit integer rung patterns whose induced \(R\) lies in the experimentally allowed band. If no such integer patterns exist, the ladder rule itself is structurally incompatible with data, regardless of how the yardstick or other continuous parameters are chosen.

In the next section we apply this test to the naive, linear neutrino ladder and show that it fails in exactly this way.

\section{Ratio obstruction for the linear mass ladder}

We now make precise the claim that a naive linear ladder for neutrino masses is structurally incompatible with the observed oscillation ratio. For the purposes of this section we ignore the neutrino-specific recognition structure and simply ask: if neutrino masses obey a ladder of the same form as the charged sectors, with integer rungs and a single yardstick, can we reproduce the experimental band for \(R\)?

\subsection{Linear ladder ansatz}

Assume that the three neutrino masses at the anchor scale \(\mu_\star\) are organized by a linear ladder,
\begin{equation}
  m_i = A_\nu \varphi^{r_i},\qquad
  r_1 < r_2 < r_3,\quad r_i \in \mathbb{Z},
\end{equation}
with a single neutrino yardstick \(A_\nu\). This is the direct transposition of the charged-sector ladder to the neutrino sector.

The mass-squared differences are then
\begin{equation}
  \Delta m^2_{ab}
  = m_b^2 - m_a^2
  = A_\nu^2\big(\varphi^{2r_b} - \varphi^{2r_a}\big),
\end{equation}
and the oscillation ratio
\begin{equation}
  R := \frac{\Delta m^2_{31}}{\Delta m^2_{21}}
\end{equation}
becomes
\begin{equation}
  R
  = \frac{A_\nu^2\big(\varphi^{2r_3} - \varphi^{2r_1}\big)}{A_\nu^2\big(\varphi^{2r_2} - \varphi^{2r_1}\big)}
  = \frac{\varphi^{2r_3} - \varphi^{2r_1}}{\varphi^{2r_2} - \varphi^{2r_1}}.
\end{equation}
Defining the integer gaps
\begin{equation}
  d_1 := r_2 - r_1 > 0,\qquad
  d_2 := r_3 - r_1 > d_1,
\end{equation}
we can factor out \(\varphi^{2r_1}\) from numerator and denominator to obtain
\begin{equation}
  R = \frac{\varphi^{2d_2} - 1}{\varphi^{2d_1} - 1}.
\end{equation}

Two points are immediate:
\begin{itemize}
  \item The ratio \(R\) depends \emph{only} on the integer gaps \((d_1,d_2)\).
  \item The yardstick \(A_\nu\) and the absolute offset \(r_1\) cancel out completely.
\end{itemize}
Thus, within the linear ladder ansatz, the problem of matching \(R\) to data is a purely discrete one: we must find integer pairs \((d_1,d_2)\) such that
\begin{equation}
  \frac{\varphi^{2d_2} - 1}{\varphi^{2d_1} - 1} \in [30,36].
\end{equation}

\subsection{Discrete analysis}

We now examine the map
\begin{equation}
  (d_1,d_2) \mapsto R(d_1,d_2) := \frac{\varphi^{2d_2} - 1}{\varphi^{2d_1} - 1}
\end{equation}
for integer pairs \(1 \leq d_1 < d_2\). For small gaps one finds, for example,
\begin{align}
  (d_1,d_2) &= (1,4) &\Rightarrow\quad R &\approx 28.4,\\
  (d_1,d_2) &= (2,5) &\Rightarrow\quad R &\approx 20.8,\\
  (d_1,d_2) &= (3,7) &\Rightarrow\quad R &\approx 24.4.
\end{align}
For larger gaps, the ratio continues to move but does not enter the observed band in any controlled way. A particularly instructive example is
\begin{equation}
  (d_1,d_2) = (4,11)\quad\Rightarrow\quad R \approx 47.0,
\end{equation}
well above the experimental range.

A systematic scan over a wide range of integer pairs \((d_1,d_2)\) (for example, \(1 \leq d_1 < d_2 \leq 20\)) shows that
\begin{equation}
  R(d_1,d_2) \notin [30,36]
\end{equation}
for all such pairs. The map from integer gaps to \(R\) values is discrete and sparse in the relevant region, and the experimentally allowed band lies squarely in a gap of this discrete spectrum.

Formally, we can summarize this as:
\begin{quote}
  \emph{For the linear ladder \(m_i = A_\nu\varphi^{r_i}\) with integer rungs and a single yardstick, there is no integer pair \((d_1,d_2)\) such that
  \[
    R = \frac{\varphi^{2d_2} - 1}{\varphi^{2d_1} - 1} \in [30,36].
  \]}
\end{quote}

\subsection{Interpretation}

This result is a structural ``no-go'' for the linear neutrino ladder under the Recognition Physics integer constraints. It does not depend on the choice of yardstick \(A_\nu\), on the absolute rung offset \(r_1\), or on any continuous parameters. As long as neutrino masses are assumed to obey a linear ladder at the anchor,
\begin{equation}
  m_i = A_\nu\varphi^{r_i},\qquad r_i \in \mathbb{Z},
\end{equation}
the oscillation ratio \(R\) is forced into a discrete set of values determined by the integer gaps \((d_1,d_2)\), and the experimentally allowed band is not among them.

This explains why the naive Source block could not be repaired by introducing a separate neutrino yardstick or by modestly adjusting the rung triplet. The obstruction is not a matter of unlucky numerical choices; it is a direct consequence of combining a linear mass ladder with integer rungs and the observed ratio \(R\). To accommodate the data while preserving the integer and no-tuning structure of RP, the neutrino ladder itself must be modified. In the next section we show that the recognition ledger, together with the neutrality condition \(Z_\nu=0\), points to a specific and minimal modification: the ladder for the neutrino sector must act on \(m^2\) rather than on \(m\).

\section{Recognition-theoretic origin of a quadratic channel for \(Z=0\)}

The failure of the linear neutrino ladder in the previous section tells us that something structural is different about neutrinos. In the Recognition Physics framework this difference is already visible at the anchor \(\mu_\star\), both in the way neutrinos couple to motifs and in the way neutral modes close recognition loops on the ledger.

\subsection{Anchor structure for neutrinos}

At the universal anchor \(\mu_\star\), neutrinos have two distinctive features:
\begin{itemize}
  \item They carry no electromagnetic charge: \(Q = 0\).
  \item They are leptons and thus have no color motifs.
\end{itemize}
In the motif dictionary that feeds the integer \(Z_i\), both EM and color motifs contribute nontrivially for charged fermions. For neutrinos, all of these contributions vanish, so their charge integer is
\begin{equation}
  Z_\nu = 0.
\end{equation}
By the single-anchor identity,
\begin{equation}
  f_\nu(\mu_\star,m_\nu)
  = \frac{\ln\bigl(1 + Z_\nu/\varphi\bigr)}{\ln \varphi}
  = \frac{\ln(1+0/\varphi)}{\ln \varphi}
  = 0.
\end{equation}
Thus, at the anchor, the neutrino sector is distinguished by the fact that its residue vanishes exactly. Neutrinos enter the anchor as a bare integer ladder with no first-order dressing in the \(\varphi\)-log channel.

\subsection{Recognition ledger for neutral modes}

On the recognition ledger, a mass term corresponds to a closed loop of recognition hops that transfers ``recognition charge'' in a way that returns the ledger to its original state. For charged fields, a single insertion of the recognition operator can already close such a loop: the ledger records a nonzero flow of charge but balances it against the background motifs in one step. This manifests as a linear recognition channel for the mass: a single application of the recognition dynamics produces a nonzero mass term.

For neutral chiral fields like the neutrino, this is not the case. With \(Q=0\) and no color motifs at the anchor, a single recognition hop cannot close the ledger loop: there is no net recognition charge to track, so one insertion simply moves the ledger into an intermediate state with unbalanced neutral recognition. To return to a ledger-balanced configuration, one needs at least two insertions of the recognition operator. In other words, the \emph{lowest} nontrivial mass-generating process for a neutral mode is quadratic in the underlying recognition amplitude.

There is a close analogy here to the Weinberg operator in conventional field theory. In the Standard Model, with only renormalizable operators, there is no neutrino mass term: a gauge-invariant mass for the neutrino requires an operator of dimension five,
\begin{equation}
  \frac{1}{\Lambda}\,(\ell H)(\ell H),
\end{equation}
built from two lepton-Higgs factors. The simplest neutrino mass contribution is thus quadratic in the fields that carry the relevant charges. In the recognition ledger, the quadratic structure arises because two recognition hops are required to complete a neutral loop.

\subsection{Quadratic recognition amplitude}

The vanishing of the neutrino residue at the anchor,
\begin{equation}
  f_\nu(\mu_\star,m_\nu) = 0,
\end{equation}
can be viewed as a statement that the linear recognition amplitude for a neutrino mass term is zero at \(\mu_\star\). The underlying recognition operator exists, but its first-order contribution cancels exactly in the neutral channel.

If the leading linear term vanishes, the next candidate is a quadratic term. Qualitatively, one can think of the neutrino mass as generated by a composite recognition operator \(R_\nu\) whose leading nonzero contribution to the ledger is proportional to \(R_\nu^2\) rather than \(R_\nu\). The mass observable is then naturally tied to \(R_\nu^2\), and it is the \emph{square} of the mass that sits on the recognition ladder.

Translating this into the language of the mass ladder:
\begin{itemize}
  \item For sectors with \(Z\neq 0\), the ladder acts on the mass itself: the observable tracked by the recognition exponents is \(m\).
  \item For sectors with \(Z=0\), the ladder acts on the \emph{square} of the mass: the observable tracked is \(m^2\).
\end{itemize}
This is the minimal change consistent with the recognition ledger: neutral modes have a quadratic recognition channel, and the ladder exponent must therefore index \(m^2\) rather than \(m\).

\subsection{Minimal RP extension}

We can summarize the extension as a simple rule, localized to the recognition charge:
\begin{itemize}
  \item For \(Z \neq 0\): the sector mass law takes the charged form
    \begin{equation}
      m_i = A_B\,\varphi^{r_i + f_B(m_i)}.
    \end{equation}
  \item For \(Z = 0\): the corresponding neutral sector uses a mass-squared ladder
    \begin{equation}
      m_i^2 = A_B^2\,\varphi^{r_i + f_B^{(2)}(m_i)},
    \end{equation}
    with the understanding that at the anchor \(\mu_\star\), the neutrino residue vanishes,
    \(f_\nu(\mu_\star,m_\nu)=0\), so the ladder simplifies to a pure integer law on \(m_i^2\).
\end{itemize}
In the neutrino case, we will denote the neutral sector yardstick by \(Y_\nu\) and write the anchor-scale relation simply as
\begin{equation}
  m_i^2(\mu_\star) = Y_\nu^2\,\varphi^{r_i}.
\end{equation}
In the next section we explore the consequences of this mass-squared ladder for the neutrino sector.

\section{A mass-squared ladder for the neutrino sector}

Armed with the recognition-theoretic argument that \(Z=0\) sectors should use a mass-squared ladder, we now construct an explicit neutrino ansatz and examine its compatibility with oscillation data.

\subsection{New neutrino ansatz}

At the anchor scale \(\mu_\star\), we postulate that the three neutrino masses satisfy
\begin{equation}
  m_i^2(\mu_\star) = Y_\nu^2\,\varphi^{r_i},\qquad r_i \in \mathbb{Z},
\end{equation}
where:
\begin{itemize}
  \item \(Y_\nu\) is a single yardstick for the entire neutrino sector. It sets the overall scale of the neutrino masses.
  \item The integers \(r_i\) are the same constructor-derived rungs that would appear in a linear ladder: they are determined by the ribbon/braid word, reduced length, generation torsion, and sector integer.
  \item The anchor dressing \(f_\nu(\mu_\star,m_\nu)\) is zero, as dictated by \(Z_\nu=0\).
\end{itemize}
The novelty is purely in which observable is placed on the ladder: \(m_i^2\) instead of \(m_i\).

\subsection{New ratio formula}

With this ansatz, the mass-squared differences are
\begin{equation}
  \Delta m^2_{ab}
  = m_b^2 - m_a^2
  = Y_\nu^2\big(\varphi^{r_b} - \varphi^{r_a}\big),
\end{equation}
and the oscillation ratio becomes
\begin{equation}
  R = \frac{\Delta m^2_{31}}{\Delta m^2_{21}}
    = \frac{\varphi^{r_3} - \varphi^{r_1}}{\varphi^{r_2} - \varphi^{r_1}}.
\end{equation}
Defining the gaps
\begin{equation}
  d_1 := r_2 - r_1 > 0,\qquad
  d_2 := r_3 - r_1 > d_1,
\end{equation}
we factor out \(\varphi^{r_1}\) to obtain
\begin{equation}
  R = \frac{\varphi^{d_2} - 1}{\varphi^{d_1} - 1}.
\end{equation}
As in the linear case, the yardstick \(Y_\nu\) and the absolute offset \(r_1\) cancel; \(R\) depends only on the integer gaps \((d_1,d_2)\). The difference is that the exponents now appear with a single power of \(\varphi\) instead of \(\varphi^2\).

\subsection{Integer solutions now exist}

We now ask whether there are integer pairs \((d_1,d_2)\) such that
\begin{equation}
  R(d_1,d_2) := \frac{\varphi^{d_2} - 1}{\varphi^{d_1} - 1} \in [30,36].
\end{equation}
Unlike the linear ladder case, this map does have values in the experimentally allowed band. For example:
\begin{align}
  (d_1,d_2) &= (4,11) &\Rightarrow\quad R &\approx 33.82,\\
  (d_1,d_2) &= (5,12) &\Rightarrow\quad R &\approx 31.81,\\
  (d_1,d_2) &= (6,13) &\Rightarrow\quad R &\approx 30.69.
\end{align}
All three of these lie comfortably within a \([30,36]\) band and close to the central value \(\sim 33\). In other words, once the ladder is placed on \(m^2\), there exist \emph{integer} rung patterns whose induced ratio \(R\) matches the oscillation data.

We can define an \emph{admissible} pair \((d_1,d_2)\) as one for which
\begin{equation}
  R(d_1,d_2) \in [R_{\min},R_{\max}],
\end{equation}
for some chosen experimental band \([R_{\min},R_{\max}]\). For a reasonable choice of band around the measured ratio, there are only a handful of such admissible pairs. Examples include:
\begin{equation}
  (d_1,d_2) \in \{(4,11),\ (5,12),\ (6,13)\} \quad\text{for } R_{\min}\approx 30,\ R_{\max}\approx 36.
\end{equation}
The precise set depends on the adopted band, but the key point is that it is small. The neutrino sector in the mass-squared ladder picture has a \emph{low-entropy} set of integer options compatible with data.

\subsection{Interpretation}

By moving the neutrino ladder from \(m\) to \(m^2\), we remove the ratio obstruction identified in the linear case. The oscillation ratio \(R\) remains a pure test of the integer structure—depending only on \((d_1,d_2)\)—but the discrete spectrum of possible values now intersects the experimentally allowed band. This is exactly what one would hope for from the recognition-theoretic argument: a minimal change, motivated by the neutral recognition ledger, that restores consistency with data while preserving the integer and no-tuning structure of the mass framework.

In the next section we will choose a representative admissible pattern, fix the yardstick \(Y_\nu\) from one measured splitting, and work out the resulting absolute neutrino masses, summed mass, and effective electron-neutrino mass.

\section{Concrete model: \((d_1,d_2) = (4,11)\)}

We now work out a concrete neutrino model within the mass-squared ladder framework, using one of the admissible integer patterns identified above. For definiteness we choose the gap pair
\begin{equation}
  (d_1,d_2) = (4,11),
\end{equation}
which yields an oscillation ratio \(R \approx 33.82\).

\subsection{Rung choice}

The gaps \((d_1,d_2) = (4,11)\) can be realized by many triplets \((r_1,r_2,r_3)\); the absolute offset is irrelevant for the ratio and can be absorbed into the yardstick \(Y_\nu\). For convenience we choose
\begin{equation}
  (r_1,r_2,r_3) = (k,\;k+4,\;k+11),
\end{equation}
with \(k\) an arbitrary integer. At the level of the ladder,
\begin{equation}
  m_i^2 = Y_\nu^2\,\varphi^{r_i},
\end{equation}
a shift of all rungs by the same integer \(k\) is equivalent to rescaling \(Y_\nu\) by \(\varphi^{k/2}\). We may therefore set
\begin{equation}
  k = 0
\end{equation}
without loss of generality, so that
\begin{equation}
  (r_1,r_2,r_3) = (0,4,11).
\end{equation}

\subsection{Fixing the yardstick}

With this choice of rungs, the solar splitting is
\begin{equation}
  \Delta m^2_{21}
  = m_2^2 - m_1^2
  = Y_\nu^2\big(\varphi^{4} - \varphi^{0}\big)
  = Y_\nu^2(\varphi^{4} - 1).
\end{equation}
Solving for \(Y_\nu^2\) gives
\begin{equation}
  Y_\nu^2
  = \frac{\Delta m^2_{21}}{\varphi^{4} - 1}.
\end{equation}
Using a representative value \(\Delta m^2_{21} \simeq 7.53\times 10^{-5}\,\mathrm{eV}^2\) and \(\varphi \approx 1.618\), we find numerically
\begin{equation}
  Y_\nu^2 \approx 1.286\times 10^{-5}\,\mathrm{eV}^2,\qquad
  Y_\nu \approx 3.59\times 10^{-3}\,\mathrm{eV}.
\end{equation}
This single parameter fixes the entire neutrino mass spectrum in this model.

\subsection{Masses and summed mass}

The squared masses at the anchor are
\begin{equation}
  m_1^2 = Y_\nu^2,\qquad
  m_2^2 = Y_\nu^2 \varphi^{4},\qquad
  m_3^2 = Y_\nu^2 \varphi^{11}.
\end{equation}
Taking square roots, we obtain
\begin{equation}
  m_1 = Y_\nu,\qquad
  m_2 = Y_\nu \sqrt{\varphi^{4}} = Y_\nu\,\varphi^{2},\qquad
  m_3 = Y_\nu \sqrt{\varphi^{11}} = Y_\nu\,\varphi^{11/2}.
\end{equation}
Numerically, this gives
\begin{equation}
  (m_1,m_2,m_3)
  \approx (0.0036,\ 0.0094,\ 0.0506)\ \mathrm{eV},
\end{equation}
and the summed mass
\begin{equation}
  \Sigma m_\nu := m_1 + m_2 + m_3 \approx 0.064\ \mathrm{eV}.
\end{equation}
This value lies comfortably below typical cosmological bounds on \(\Sigma m_\nu\).

\subsection{Check of the oscillation ratio}

The mass-squared ladder ansatz implies
\begin{equation}
  R := \frac{\Delta m^2_{31}}{\Delta m^2_{21}}
  = \frac{\varphi^{d_2} - 1}{\varphi^{d_1} - 1}
  = \frac{\varphi^{11} - 1}{\varphi^{4} - 1}.
\end{equation}
Evaluating this ratio numerically yields
\begin{equation}
  R \approx 33.82,
\end{equation}
which lies squarely within the experimentally favored band \([30,36]\) and close to the nominal central value \(R_{\mathrm{exp}}\sim 33\). In this model, the integer gaps alone fix the ratio; the yardstick \(Y_\nu\) plays no role in determining \(R\).

\subsection{Effective electron-neutrino mass}

The effective electron-neutrino mass measured in β-decay is
\begin{equation}
  m_\beta
  := \Biggl(\sum_{i=1}^3 |U_{ei}|^2 m_i^2\Biggr)^{1/2},
\end{equation}
where \(U_{ei}\) are elements of the first row of the PMNS matrix. Taking a representative set of mixing elements,
\begin{equation}
  |U_{e1}|^2 \simeq 0.67,\qquad
  |U_{e2}|^2 \simeq 0.30,\qquad
  |U_{e3}|^2 \simeq 0.03,
\end{equation}
and inserting the masses above, we find
\begin{equation}
  m_\beta \approx 0.010\text{--}0.011\ \mathrm{eV},
\end{equation}
roughly at the \(10^{-2}\,\mathrm{eV}\) level. This is well below current experimental upper limits of order \(1~\mathrm{eV}\) and lies in the range targeted by future high-precision β-decay experiments.

If neutrinos are Majorana rather than Dirac particles, one can similarly compute the effective Majorana mass \(m_{\beta\beta}\) relevant for neutrinoless double-β decay, including appropriate PMNS phases. In this paper we focus on the Dirac case for clarity; the structure of the mass-squared ladder and its predictions for \((m_1,m_2,m_3)\) and \(\Sigma m_\nu\) carry over directly to the Majorana case, with \(m_{\beta\beta}\) providing an additional, phase-sensitive test.

\section{Consistency with RP discipline}

The mass-squared ladder model constructed above for the neutrino sector is not an ad-hoc fix; it respects the core Recognition Physics constraints that governed the original mass framework.

\subsection{Anchor rule and \(Z_\nu = 0\)}

The anchor rule for neutrinos,
\begin{equation}
  Z_\nu = 0 \quad\Rightarrow\quad f_\nu(\mu_\star,m_\nu) = 0,
\end{equation}
is preserved exactly. The neutrino ladder is defined directly at the anchor,
\begin{equation}
  m_i^2(\mu_\star) = Y_\nu^2\,\varphi^{r_i},
\end{equation}
with no additional neutrino-specific dressing function. In particular, there is no hidden replacement of \(f_\nu=0\) by an effective nonzero residue; the neutral sector remains undressed at \(\mu_\star\), in line with its vanishing charge integer.

\subsection{No per-flavor knobs}

Once an admissible gap pair \((d_1,d_2)\) is chosen and the yardstick \(Y_\nu\) is fixed by a single measured splitting \(\Delta m^2_{21}\), all other neutrino observables in this model are determined:
\begin{itemize}
  \item the individual masses \(m_1,m_2,m_3\),
  \item the summed mass \(\Sigma m_\nu\),
  \item the oscillation ratio \(R\),
  \item and the effective mass \(m_\beta\).
\end{itemize}
There are no per-flavor continuous parameters. The only continuous knob is the global yardstick \(Y_\nu\), which is fixed by data and then shared by all three masses. All flavor dependence is encoded in the integers \(r_i\), which are themselves determined by the RP constructor.

This mirrors the charged-sector discipline: each sector has one yardstick and a single dressing law; the integer data \((Z_i,r_i)\) break species degeneracy without introducing tunable parameters.

\subsection{Constructor origin of the rungs}

The rung triplet \((r_1,r_2,r_3)\) is not chosen arbitrarily. In the full RP framework, it arises from the same ribbon/braid constructor used in the charged sectors, applied to the neutrino recognition words. Each rung \(r_i\) is built from:
\begin{itemize}
  \item a reduced length \(L_i\), extracted from the neutrino's discrete recognition word,
  \item a generation torsion term that shifts rungs between generations in a fixed pattern,
  \item and a sector integer \(\Delta_B\) associated with the neutrino sector as a whole.
\end{itemize}
The condition \(Z = 0\) (no EM motifs, no color motifs) constrains the recognition words and thus restricts the allowed rung patterns. Within these constraints, patterns such as \((r_1,r_2,r_3)=(k,k+4,k+11)\) emerge from the same finite dictionary that governs the charged fermions. The integers are therefore computed by the existing constructor, not dialed by hand.

\subsection{Compatibility with the rest of the mass ladder}

Finally, the change from a linear to a mass-squared ladder is strictly localized to the \(Z=0\) channel. All sectors with \(Z\neq 0\) (charged leptons, quarks, charged gauge bosons) retain the original linear ladder
\begin{equation}
  m_i = A_B\,\varphi^{r_i + f_B(m_i)},
\end{equation}
with the same anchor, motif dictionary, and integer structure as before. Neutrinos, as the unique neutral chiral fermions in the Standard Model, are singled out by their vanishing charge integer and the recognition-led requirement that their lowest mass-generating process be quadratic.

In this sense, the neutrino mass-squared ladder is not a departure from the Recognition Physics program but its completion in the neutral sector. It shows how the same recognition-led architecture that constrains charged masses can accommodate neutrinos once the special role of neutrality (\(Z=0\)) is taken into account.

\section{Consistency with RS discipline}

The neutrino mass-squared ladder proposed here is only acceptable if it remains faithful to the core Recognition Science (RS) discipline that underlies the original mass framework. In this section we show that it does.

\subsection{Anchor rule and \(Z_\nu = 0\)}

The starting point for neutrinos in RS is the vanishing charge integer
\begin{equation}
  Z_\nu = 0,
\end{equation}
which implies, via the single-anchor identity,
\begin{equation}
  f_\nu(\mu_\star,m_\nu)
  = \frac{\ln\bigl(1+Z_\nu/\varphi\bigr)}{\ln \varphi}
  = 0.
\end{equation}
The neutrino sector is therefore undressed at the anchor: there is no first-order recognition contribution to the mass at \(\mu_\star\).

The mass-squared ladder construction respects this exactly. At the anchor we impose
\begin{equation}
  m_i^2(\mu_\star) = Y_\nu^2\,\varphi^{r_i},
\end{equation}
with no additional neutrino-specific dressing factor inserted. The neutrino mass structure uses the anchor and the same integer ladders as the charged sectors, but does not introduce a new residue \(f_\nu\); the rule \(f_\nu(\mu_\star)=0\) is left intact.

\subsection{No per-flavor knobs}

The RS mass framework prohibits per-flavor continuous tuning. Each sector has:
\begin{itemize}
  \item one sector yardstick (\(A_B\) for charged sectors, \(Y_\nu\) for neutrinos),
  \item a fixed renormalization prescription at the anchor,
  \item and a set of integer data \((Z_i,r_i)\) determined by the constructor.
\end{itemize}

The neutrino mass-squared ladder preserves this pattern. Once:
\begin{itemize}
  \item an integer gap pair \((d_1,d_2)\) is chosen (for example \((4,11)\)),
  \item and the yardstick \(Y_\nu\) is fixed by a single measured splitting \(\Delta m^2_{21}\),
\end{itemize}
everything else in the neutrino sector is determined:
\begin{itemize}
  \item the individual masses \(m_1,m_2,m_3\),
  \item the summed mass \(\Sigma m_\nu = m_1 + m_2 + m_3\),
  \item the oscillation ratio \(R = \Delta m^2_{31}/\Delta m^2_{21}\),
  \item and the effective electron-neutrino mass \(m_\beta\).
\end{itemize}
There are no continuous, species-specific ``knobs'' left to adjust. All continuous freedom is either:
\begin{itemize}
  \item global to the neutrino sector (the single yardstick \(Y_\nu\)), or
  \item already fixed by the charged sectors and the anchor (the choice of \(\mu_\star\) and the renormalization scheme).
\end{itemize}
All flavor dependence is carried by the integers \(r_i\), which are inherited from the RS constructor.

\subsection{Constructor origin of the rungs}

The rung triplet \((r_1,r_2,r_3)\) used for the neutrinos is not a new, hand-picked object. In RS, each species is associated with a discrete recognition word built from a finite motif dictionary. From this word the constructor extracts:
\begin{itemize}
  \item a reduced length \(L_i\),
  \item a generation torsion term \(\tau_g\),
  \item and a sector integer \(\Delta_B\),
\end{itemize}
which combine to give an integer rung
\begin{equation}
  r_i = L_i + \tau_g + \Delta_B.
\end{equation}

For neutrinos, the condition \(Z=0\) (no EM motifs, no color motifs) restricts the allowed recognition words and thus the allowed integer combinations. Under these restrictions, patterns such as \((r_1,r_2,r_3)=(k,k+4,k+11)\) arise from the same finite dictionary used for charged fermions. The integers are therefore computed by the existing constructor rules; they are not fitted to oscillation data.

\subsection{Compatibility with the rest of the mass ladder}

The move from a linear to a mass-squared ladder is strictly localized to the \(Z=0\) channel. All sectors with nonzero charge integer \(Z\) retain the original ladder structure
\begin{equation}
  m_i = A_B\,\varphi^{r_i + f_B(m_i)},
\end{equation}
with a single yardstick per sector and a \(\varphi\)-logarithmic residue determined at the anchor. Only the neutrino sector, singled out by \(Z_\nu=0\) and the absence of a first-order recognition channel, uses
\begin{equation}
  m_i^2 = Y_\nu^2\,\varphi^{r_i}.
\end{equation}

In this sense, the neutrino modification does not break the RS mass ladder; it completes it. Charged sectors remain linear in the mass, while neutrinos, as the unique neutral chiral fermions, are quadratic. The overall architecture---integers, single anchor, no per-flavor tuning---remains unchanged.

\section{Falsifiability and phenomenology}

The neutrino mass-squared ladder leads to a set of sharp, discrete predictions. Because the only freedom lies in integer choices and a single yardstick fixed by data, the RS neutrino sector is highly falsifiable.

\subsection{Discrete ratio predictions}

For a given integer gap pair \((d_1,d_2)\), the oscillation ratio is
\begin{equation}
  R_\nu(d_1,d_2)
  = \frac{\varphi^{d_2} - 1}{\varphi^{d_1} - 1}.
\end{equation}
If the pattern \((d_1,d_2)=(4,11)\) is the one selected by the constructor and the universe, then
\begin{equation}
  R_\nu = \frac{\varphi^{11} - 1}{\varphi^{4} - 1} \approx 33.8233
\end{equation}
exactly, given \(\varphi\). This number does not depend on any continuous parameter; it is a pure function of the golden ratio and the integers \((4,11)\).

If future global fits to oscillation data were to converge on a significantly different central value for \(R_{\mathrm{exp}}\), only a small set of other integer pairs (such as \((5,12)\) or \((6,13)\)) would remain possible. If the experimental band for \(R\) moves away from the discrete set
\begin{equation}
  \{R_\nu(d_1,d_2) : (d_1,d_2)\ \text{admissible}\},
\end{equation}
then the RS neutrino ladder in this form is ruled out.

\subsection{Sum and β-endpoint predictions}

Once \(\Delta m^2_{21}\) and the gap pair \((d_1,d_2)\) are fixed, the yardstick \(Y_\nu\) is determined and the three masses \(m_1,m_2,m_3\) are fixed. In particular, the summed mass
\begin{equation}
  \Sigma m_\nu = m_1 + m_2 + m_3
\end{equation}
is locked.

For example, with \((d_1,d_2)=(4,11)\), we found
\begin{equation}
  \Sigma m_\nu \approx 0.064\ \mathrm{eV}.
\end{equation}
Other admissible integer pairs generate different, but still tightly constrained, values for \(\Sigma m_\nu\). Taken together, the RS neutrino ladder predicts a discrete set of possible sums, all in the sub-eV range. Cosmological measurements of \(\Sigma m_\nu\) thus provide a direct test: if future analyses robustly prefer a value inconsistent with all RS-admissible sums, the model is falsified.

Similarly, the effective electron-neutrino mass
\begin{equation}
  m_\beta = \Biggl(\sum_{i=1}^3|U_{ei}|^2 m_i^2\Biggr)^{1/2}
\end{equation}
is determined once the masses and mixing elements are known. For the \((4,11)\) model, we obtained
\begin{equation}
  m_\beta \approx 0.010\text{--}0.011\ \mathrm{eV}.
\end{equation}
Other admissible integer patterns lead to slightly different values, but all remain in the same sub-eV regime. Thus, β-decay experiments provide a second, independent test. Current experiments are not yet sensitive to this scale, but near-future facilities aim to probe down to tens of meV, directly entering the RS prediction band.

\subsection{Hierarchy and model selection}

The construction presented here is tailored to normal ordering, with \(m_1<m_2<m_3\) and \(\Delta m^2_{31}>0\). One may ask whether an inverted ordering can be realized within the same mass-squared ladder architecture. In principle, one can assign rungs such that \(m_3\) is the lightest state and adjust \((d_1,d_2)\) accordingly. However, preliminary scans indicate that matching both the measured ratio \(R\) and the required sign pattern of the splittings is more restrictive in the inverted case, and may not admit any admissible integer pairs.

If future oscillation data were to favor inverted ordering strongly, this would put tension on the simplest RS neutrino ladder. Either the constructor logic would need to allow a different rung pattern compatible with inversion, or RS would have to accept that the neutrino sector requires additional structure beyond the minimal quadratic rule.

\subsection{``No knobs'' as a test}

The key point for falsifiability is the absence of continuous tuning. In the RS neutrino ladder:
\begin{itemize}
  \item the ratio \(R\) is determined by the integers \((d_1,d_2)\),
  \item the masses are determined by \((d_1,d_2)\) and \(\Delta m^2_{21}\),
  \item and \(\Sigma m_\nu\) and \(m_\beta\) are fixed once the masses and mixing are known.
\end{itemize}
If experimental values for any of these quantities move outside the discrete RS predictions, there is no hidden parameter that can be adjusted to rescue the model. This is the intended behavior of RS: the recognition ledger makes strong, crisp commitments that the universe can either uphold or refute.

\section{Extensions and open questions}

The mass-squared ladder resolves the immediate neutrino tension in RS, but it also opens several lines of further inquiry.

\subsection{Majorana versus Dirac}

In this paper we have assumed Dirac neutrinos for simplicity. If neutrinos are Majorana particles, the observable relevant for neutrinoless double-β decay is the effective Majorana mass
\begin{equation}
  m_{\beta\beta}
  := \Biggl|\sum_{i=1}^3 U_{ei}^2 m_i\Biggr|,
\end{equation}
which depends on PMNS phases as well as on the masses. The mass-squared ladder fixes \((m_1,m_2,m_3)\); the remaining freedom lies in the complex phases of the mixing matrix.

From a recognition-theoretic perspective, Majorana masses may correspond to different ledger closures or to additional constraints on the recognition operator (for example, symmetry conditions that tie left- and right-handed modes). Understanding how these constraints arise in RS and how they restrict the allowed PMNS phases is an open problem. In particular, it would be interesting to see whether RS can predict a restricted range for \(m_{\beta\beta}\) given the mass-squared ladder and a recognition-based model of CP phases.

\subsection{Tiny EM corrections and nonzero \(Z_\nu\)}

In the present treatment, neutrinos have \(Z_\nu=0\) exactly at the anchor. In a more refined RS model, subleading effects---such as higher-order electromagnetic corrections, loop-induced couplings, or departures from the strict anchor limit---could generate a tiny but nonzero effective \(Z_\nu\). This would introduce small corrections to the ladder, possibly of higher order in \(\varphi^{-1}\), and might lead to additional fine structure in the neutrino spectrum.

It is natural to ask:
\begin{itemize}
  \item whether such corrections are required by the full recognition ledger,
  \item what their magnitude would be,
  \item and whether they would be observable in precision oscillation or β-decay experiments.
\end{itemize}
A controlled treatment of these subleading terms would also clarify how robust the pure \(m^2\) ladder is against recognition-theoretic perturbations.

\subsection{Heavy neutrinos and seesaw structures}

Many extensions of the Standard Model introduce heavy sterile neutrinos and seesaw mechanisms to generate light neutrino masses. In RS, heavy neutral fermions would also carry \(Z=0\) at the anchor, suggesting that they might share the same mass-squared ladder structure or sit on a related ladder with a different yardstick.

Key questions include:
\begin{itemize}
  \item how heavy sterile states and seesaw-like structures are encoded on the recognition ledger,
  \item whether the light and heavy neutrinos share a common \(m^2\) ladder with different rungs,
  \item or whether the heavy sector requires its own recognition sector with a distinct yardstick and integer structure.
\end{itemize}
A coherent RS treatment of light and heavy neutrinos would connect the present mass-squared ladder to higher-scale recognition dynamics.

\subsection{Interplay with other RS sectors}

Finally, the quadratic neutrino rule may have implications beyond the neutrino sector. For example:
\begin{itemize}
  \item In RS cosmology and information-limited gravity (ILG), the neutrino sector contributes to the energy budget and to structure formation. A fixed \(\Sigma m_\nu\) in the tens of meV range feeds directly into these sectors.
  \item In CP violation and recognition-phase dynamics, neutrinos could play a special role if their quadratic channel couples differently to the recognition phase than the linear channels of charged sectors.
  \item Cross-sector constraints (for example, from ILG fits, baryogenesis models, or RS-based quantum information arguments) might further restrict the allowed integer pairs \((d_1,d_2)\) beyond the oscillation ratio alone.
\end{itemize}
Exploring these connections will help determine whether the neutrino mass-squared ladder is simply compatible with RS or is in fact required by deeper structural couplings across sectors.

\section{Conclusion}

Neutrinos have long been a stress point for Recognition Science. An early RS neutrino specification, built by naive analogy with the charged sectors, used a linear mass ladder
\begin{equation}
  m_i \propto E_{\mathrm{coh}}\varphi^{r_i}
\end{equation}
with integer rungs and the global coherence scale as the yardstick. That construction failed on both absolute and structural grounds: it produced an unacceptably large \(\Sigma m_\nu\) and could not reproduce the observed oscillation ratio \(R = \Delta m^2_{31}/\Delta m^2_{21}\) for any integer rung pattern.

In this paper we have shown that RS itself explains why. Neutrinos are special because neutrality (\(Z_\nu=0\)) kills the linear recognition channel at the anchor. With no electromagnetic or color motifs, the charge integer vanishes and the anchor residue is exactly zero:
\begin{equation}
  Z_\nu = 0 \quad\Rightarrow\quad f_\nu(\mu_\star,m_\nu) = 0.
\end{equation}
On the recognition ledger, this means a single insertion of the recognition operator cannot close a neutrino mass loop. The first nonzero ledger closure for a neutral chiral field is quadratic in the recognition amplitude. This in turn forces the recognition ladder for the \(Z=0\) sector to act on \(m^2\), not on \(m\).

The central structural output is a neutrino ansatz of the form
\begin{equation}
  m_i^2 = Y_\nu^2\,\varphi^{r_i},
\end{equation}
with no per-flavor knobs. The integer rungs \(r_i\) are supplied by the same constructor that governs the charged sectors; the yardstick \(Y_\nu\) is fixed by a single measured splitting \(\Delta m^2_{21}\). The oscillation ratio is then
\begin{equation}
  R = \frac{\varphi^{d_2} - 1}{\varphi^{d_1} - 1},
\end{equation}
where \((d_1,d_2)\) are integer gaps. Unlike the linear case, this spectrum of ratios includes values in the experimental band. The allowed integer pairs form a small, discrete set; for example, \((d_1,d_2)=(4,11)\) gives
\begin{equation}
  R_\nu = \frac{\varphi^{11}-1}{\varphi^{4}-1} \approx 33.8233.
\end{equation}
With this choice, fixing \(Y_\nu\) from \(\Delta m^2_{21}\) yields a concrete spectrum
\begin{equation}
  (m_1,m_2,m_3) \approx (0.0036,\,0.0094,\,0.0506)\ \mathrm{eV},
\end{equation}
a summed mass \(\Sigma m_\nu \approx 0.064~\mathrm{eV}\), and an effective electron-neutrino mass \(m_\beta \sim 10^{-2}~\mathrm{eV}\), all consistent with current constraints.

Positioned within the broader RS program, this work turns a prior ``neutrino no-go'' into a positive structural result. It shows that RS can flex in the right way---recognizing a quadratic channel for neutral modes---without sacrificing its core commitments: integer structure, a single anchor, and the absence of per-flavor tuning. The mass-squared ladder is not an extra assumption bolted onto RS; it is the natural consequence of applying the recognition ledger to a \(Z=0\) sector.

The neutrino ladder presented here makes crisp, falsifiable predictions. The oscillation ratio, \(\Sigma m_\nu\), and \(m_\beta\) all lie in discrete RS bands. Future oscillation fits, cosmological analyses, and high-precision β-decay experiments will either land within these bands or not. If they do, the neutrino sector will become a powerful confirmation of the recognition architecture. If they do not, RS in this form will be forced to revise or abandon its neutrino rules.

Future work includes refining the choice of integer gaps with updated oscillation data, extending the construction to Majorana scenarios and neutrinoless double-β decay, and integrating the neutrino mass-squared ladder more deeply into RS cosmology and information-limited gravity. In all cases, the guiding principle remains the same: the recognition ledger should not merely accommodate the data; it should explain why the universe had so few options in the first place.





\end{document}
