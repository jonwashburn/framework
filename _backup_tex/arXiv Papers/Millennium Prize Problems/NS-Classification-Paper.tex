\documentclass[12pt, reqno]{amsart}

%% PACKAGES
\usepackage{amsmath, amssymb, amsthm, amsfonts}
\usepackage{mathrsfs}
\usepackage{mathtools}
\usepackage{enumitem}
\usepackage{geometry}
\usepackage{booktabs}
\usepackage{xcolor}
\usepackage{microtype}

%% GEOMETRY
\geometry{margin=1in}

\usepackage[colorlinks=true, linkcolor=blue!70!black, citecolor=blue!70!black,
  urlcolor=blue!70!black]{hyperref}

%% THEOREMS
\newtheorem{theorem}{Theorem}[section]
\newtheorem{lemma}[theorem]{Lemma}
\newtheorem{proposition}[theorem]{Proposition}
\newtheorem{corollary}[theorem]{Corollary}
\newtheorem{conjecture}[theorem]{Conjecture}

\theoremstyle{definition}
\newtheorem{definition}[theorem]{Definition}

\theoremstyle{remark}
\newtheorem{remark}[theorem]{Remark}
\newtheorem{example}[theorem]{Example}

%% NUMBERING
\numberwithin{equation}{section}

%% MACROS
\newcommand{\R}{\mathbb{R}}
\newcommand{\N}{\mathbb{N}}
\newcommand{\C}{\mathbb{C}}
\newcommand{\Z}{\mathbb{Z}}
\newcommand{\Sbb}{\mathbb{S}}

\DeclareMathOperator{\dv}{div}
\DeclareMathOperator{\curl}{curl}
\DeclareMathOperator{\supp}{supp}
\DeclareMathOperator{\osc}{osc}
\DeclareMathOperator{\BMO}{BMO}
\newcommand{\eps}{\varepsilon}
\newcommand{\Ecal}{\mathcal{E}}

%% Proof-step formatting
\newcommand{\step}[1]{\par\medskip\noindent\textit{Step~#1.}\enspace}

%% TITLE & AUTHOR
\title[Weighted Geometric Depletion for 3D Navier--Stokes]{%
Weighted Geometric Depletion and Structural Constraints\\
on Blow-Up Profiles for the\\
3D Incompressible Navier--Stokes Equations}

\author{Jonathan Washburn}
\address{Austin, Texas, USA}
\email{jon@recognitionphysics.org}

\subjclass[2020]{Primary 35Q30; Secondary 76D05, 35B44, 35B65}
\keywords{Navier--Stokes equations, blow-up profile, vorticity direction,
geometric depletion, ancient solutions, commutator estimates}
\date{February 2026}

\begin{document}

\begin{abstract}
We establish new structural constraints on potential finite-time
singularities of the three-dimensional incompressible Navier--Stokes
equations.  For any smooth $H^1$ solution that blows up at time
$T^*<\infty$, the running-max/vorticity-normalized ancient element
satisfies: (a)~the $\rho^{3/2}$-weighted near-field vortex stretching
is unconditionally depleted at rate $O(r^5)$; (b)~the far-field
contribution vanishes in the blow-up limit; (c)~the weighted direction
coherence obeys a universal bound; and (d)~the vorticity direction
$\xi=\omega/|\omega|$ has a uniformly bounded gradient on the
high-vorticity set $\{\rho\ge\eta\}$, with the bound depending only on
$\eta$ and universal constants.  The mechanism underlying~(d) is a
threshold comparison: the accumulated direction energy sits well below
the Struwe $\eps$-regularity threshold for harmonic maps into~$\Sbb^2$.

Conditionally, if the bounded gradient could be upgraded to full
direction constancy ($\nabla\xi\equiv 0$ on
$\supp\,\omega$), the blow-up profile is classified as the rigid
rotation.  This reduces the Millennium regularity problem to two
concrete conjectures: establishing direction constancy and excluding the
rigid rotation as a blow-up limit.  The obstructions to each are
identified and discussed.
\end{abstract}

\maketitle

\tableofcontents

%% ====================================================================
\section{Introduction}\label{sec:intro}
%% ====================================================================

\subsection{The regularity problem}

Let $\nu>0$ be the kinematic viscosity.  We consider the
three-dimensional incompressible Navier--Stokes equations
\begin{equation}\label{eq:NS}
\begin{cases}
\partial_t u + (u\cdot\nabla)u + \nabla p - \nu\Delta u = 0,\\[3pt]
\nabla\cdot u = 0,
\end{cases}
\end{equation}
for a velocity field $u\colon\R^3\times[0,T)\to\R^3$ and scalar
pressure $p\colon\R^3\times[0,T)\to\R$, with smooth divergence-free
initial data $u_0\in H^1(\R^3)$.

Leray~\cite{Leray1934} proved the existence of global weak solutions and
raised the question of whether smooth solutions can develop singularities
in finite time.  This question, now identified as one of the Clay
Millennium Prize Problems~\cite{Fefferman2006}, remains open.

The Beale--Kato--Majda criterion~\cite{BKM1984} gives a necessary and
sufficient condition for blow-up: a smooth solution loses regularity at
time $T^*<\infty$ if and only if
\begin{equation}\label{eq:BKM}
\int_0^{T^*}\|\omega(\cdot,t)\|_{L^\infty}\,dt = \infty,
\end{equation}
where $\omega=\curl\,u$ is the vorticity.

\subsection{Statement of results}

Our main results establish new structural constraints on any blow-up
profile.  We state the unconditional results first, then the conditional
classification.

\begin{theorem}[Structural constraints on blow-up profiles]%
\label{thm:main}
Let $u_0\in H^1(\R^3)$ be smooth and divergence-free, and let $u$ be
the corresponding smooth solution of~\eqref{eq:NS} on its maximal
interval of existence $[0,T^*)$.  Suppose $T^*<\infty$ and let
$(u^\infty,p^\infty)$ be the running-max ancient element
\textup{(}Definition~\ref{def:running-max},
Lemma~\ref{lem:ancient}\textup{)}.  Then:
\begin{enumerate}[label=\textup{(\alph*)}, ref=\textup{\alph*},
  leftmargin=2em, itemsep=6pt]
\item\label{it:nearfield} \textbf{Near-field depletion.}
The $\rho^{3/2}$-weighted near-field stretching is unconditionally
depleted:
$\iint_{Q_r}\rho^{3/2}|\sigma_{\mathrm{near}}|\le Cr^5$ for all
$r\le 1$ and all basepoints.
\item\label{it:tail} \textbf{Tail elimination.}
The external far-field contribution vanishes in the blow-up limit at rate
$M_k^{-3/4}$.  For the ancient element, the full stretching equals its
near-field at every scale.
\item\label{it:coherence} \textbf{Weighted coherence bound.}
$\Ecal_\omega(z_0,R)\le C_1 R^5+C_2 R^3$ for universal constants
$C_1,C_2$.
\item\label{it:direction} \textbf{Direction regularity on the
high-vorticity set.}
For every $\eta>0$ there exists $C(\eta)<\infty$ such that
\begin{equation}\label{eq:grad-bound}
|\nabla\xi(z_1)|\le C(\eta)
\qquad\text{for every }z_1\text{ with }\rho(z_1)\ge\eta.
\end{equation}
The constant $C(\eta)=2C_S C_{\mathrm{Ser}}/\eta$ depends only on
$\eta$, the Serrin constant, and the Struwe constant.
\end{enumerate}
\end{theorem}

\begin{remark}[Role of the Struwe threshold]\label{rem:struwe-role}
Part~(\ref{it:direction}) follows from a threshold comparison: the
unweighted direction energy on a universal ball where
$\rho\ge\frac12$ (via Serrin regularity) is of order
$\delta^3\sim C_{\mathrm{Ser}}^{-3}$, far below the Struwe
$\eps$-regularity threshold $4\pi\approx 12.6$ for harmonic maps
into~$\Sbb^2$.  The perturbation terms (drift and forcing) are
controlled independently by Serrin estimates, with no circularity.
\end{remark}

\begin{remark}[The direction constancy question]%
\label{rem:constancy-gap}
To upgrade~\eqref{eq:grad-bound} from a bounded gradient to
$\nabla\xi\equiv 0$ (direction constancy), one would need the
$\eps$-regularity to apply at all parabolic scales $R>0$
simultaneously.  At small scales ($R\le\delta$, where
$\rho\ge\frac12$), the unweighted energy is below the Struwe threshold
and the perturbations are small (Section~\ref{sec:direction}).  At large
scales ($R\gg 1$), two obstructions arise: the unweighted direction
energy on the low-vorticity set $\{\rho<\eta\}$ is not controlled by the
weighted bound~(\ref{it:coherence}), and the rescaled drift grows
with~$R$.  Closing this large-scale gap would establish direction
constancy on the full support of the vorticity.
\end{remark}

We record the conditional classification that would follow from
direction constancy.

\begin{theorem}[Conditional classification]\label{thm:conditional}
Assume, in addition to the hypotheses of Theorem~\ref{thm:main}, that
the ancient element satisfies $\nabla\xi^\infty\equiv 0$ on
$\{\rho^\infty>0\}$.  Then:
\begin{enumerate}[label=\textup{(\roman*)}, ref=\textup{\roman*},
  leftmargin=2em, itemsep=4pt]
\item\label{it:pos} $\rho^\infty>0$ everywhere on
$\R^3\times(-\infty,0]$.
\item\label{it:one} $\rho^\infty\equiv 1$.
\item\label{it:rigid} $u^\infty=\frac12(-x_2,x_1,0)$
\textup{(}rigid rotation, up to spatial rotation and Galilean
drift\textup{)}.
\end{enumerate}
\end{theorem}

\begin{corollary}[Conditional reduction of the Millennium Problem]%
\label{cor:equiv}
If every running-max ancient element has $\nabla\xi\equiv 0$ on
$\{\rho>0\}$, then: global regularity holds for all smooth $H^1$ data
if and only if the rigid rotation cannot arise as a running-max blow-up
limit.
\end{corollary}

\begin{remark}\label{rem:two-conjectures}
Corollary~\ref{cor:equiv} reduces the Millennium regularity problem to
two independent conjectures: (1)~direction constancy on the support of
the vorticity (Remark~\ref{rem:constancy-gap}), and (2)~exclusion of the
rigid rotation as a blow-up limit
(Conjecture~\ref{conj:rigid}).
\end{remark}

\subsection{Context and relation to prior work}

The principal existing results concerning blow-up structure for the
Navier--Stokes equations include the following.
\begin{itemize}[leftmargin=2em, itemsep=4pt]
\item \textbf{Partial regularity.}
Caffarelli, Kohn, and Nirenberg~\cite{CKN1982} proved that the singular
set of any suitable weak solution has one-dimensional parabolic Hausdorff
measure zero.
\item \textbf{Endpoint criterion.}
Escauriaza, Seregin, and \v{S}ver\'ak~\cite{ESS2003} showed that
blow-up forces $\|u(\cdot,t_k)\|_{L^3}\to\infty$ along some sequence
$t_k\uparrow T^*$.
\item \textbf{Necessary conditions on profiles.}
Seregin~\cite{Seregin2012} established structural constraints on Type~I
blow-up profiles.
\item \textbf{Liouville theorems.}
Koch, Nadirashvili, Seregin, and \v{S}ver\'ak~\cite{KNSS2009} proved
that bounded ancient solutions of the two-dimensional Navier--Stokes
equations are constant.  Gallagher, Koch, and
Planchon~\cite{GKP2016} further developed the blow-up extraction
framework for critical norms.
\item \textbf{Geometric regularity criteria.}
Constantin and Fefferman~\cite{ConstantinFefferman1993} showed that
coherence of the vorticity direction prevents blow-up, initiating the
geometric approach to regularity.
\end{itemize}

Theorem~\ref{thm:main} extends the geometric approach: rather than
requiring direction coherence as a hypothesis (as
in~\cite{ConstantinFefferman1993}), it establishes direction regularity
as a \emph{consequence} of the blow-up structure, unconditionally on the
high-vorticity set.  The conditional classification
(Theorem~\ref{thm:conditional}) identifies the unique profile that would
follow from full direction constancy.

\subsection{Plan of the paper}\label{subsec:overview}

The argument proceeds in five stages, corresponding to
Sections~\ref{sec:ancient}--\ref{sec:collapse}.

\begin{description}[style=unboxed, leftmargin=0pt, itemsep=6pt,
  font=\normalfont\bfseries]
\item[Stage~1: Extraction of the ancient element
\textup{(Section~\ref{sec:ancient})}.]
A running-max/vorticity-normalized blow-up rescaling produces, after
passage to a subsequence, a nontrivial ancient solution
$(u^\infty,p^\infty)$ on $\R^3\times(-\infty,0]$ satisfying
$|\omega^\infty(0,0)|=1$ and $\|\omega^\infty\|_{L^\infty}\le 1$.

\item[Stage~2: Near-field depletion
\textup{(Section~\ref{sec:nearfield})}.]
Using the Coifman--Rochberg--Weiss commutator theorem and a H\"older
pairing argument, we show that the $\rho^{3/2}$-weighted near-field
stretching on any parabolic cylinder $Q_r$ is bounded by $Cr^5$.

\item[Stage~3: Far-tail elimination
\textup{(Section~\ref{sec:tail})}.]
A three-way spatial decomposition of the vortex stretching shows that
the external tail vanishes in the blow-up limit at rate
$M_k^{-3/4}$, where $M_k$ is the running maximum of
$\|\omega\|_{L^\infty}$.

\item[Stage~4: Direction regularity
\textup{(Section~\ref{sec:direction})}.]
The accumulated direction energy on a universal parabolic ball lies
several orders of magnitude below the Struwe $\eps$-regularity threshold
$4\pi$ for harmonic-map heat flow.  The perturbed
$\eps$-regularity theory yields $|\nabla\xi|\le C(\eta)$ on
$\{\rho\ge\eta\}$.

\item[Stage~5: Conditional collapse to rigid rotation
\textup{(Section~\ref{sec:collapse})}.]
Assuming direction constancy, the strong minimum principle forces
$\rho\equiv 1$, and the ancient element is identified as the rigid
rotation.
\end{description}

\noindent
The main theorems are assembled in Section~\ref{sec:proofs}.  The
remaining obstructions are discussed in Section~\ref{sec:conjecture}.


%% ====================================================================
\section{Preliminaries}\label{sec:prelim}
%% ====================================================================

\subsection{Notation and conventions}\label{subsec:notation}

We collect the notation used throughout the paper.

\begin{center}
\renewcommand{\arraystretch}{1.3}
\begin{tabular}{@{}ll@{}}
\toprule
\textbf{Symbol} & \textbf{Meaning} \\
\midrule
$u$, $p$, $\omega=\curl\,u$ & velocity, pressure, vorticity \\
$\rho=|\omega|$ & vorticity magnitude \\
$\xi=\omega/|\omega|\in\Sbb^2$ & vorticity direction
  (on $\{\omega\neq 0\}$) \\
$S=\tfrac12(\nabla u+\nabla u^T)$ & symmetric strain tensor \\
$\sigma = S\xi\cdot\xi$ & vortex stretching scalar \\
$Q_r(z_0)=B_r(x_0)\times(t_0-r^2,\,t_0)$ & backward parabolic
  cylinder \\
$\Ecal_\omega(z_0,r)$ & weighted direction coherence
  (Definition~\ref{def:Eomega}) \\
$C_{\mathrm{Ser}}$ & Serrin interior regularity constant
  (Lemma~\ref{lem:serrin}) \\
$C_S$, $\eps_0$ & Struwe $\eps$-regularity constants
  (Lemma~\ref{lem:struwe-perturbed}) \\
\bottomrule
\end{tabular}
\end{center}

\smallskip

Throughout, $C$ denotes a positive constant depending at most on the
dimension and the viscosity~$\nu$; its value may change from line to
line.  We write $A\lesssim B$ to mean $A\le CB$ for such a constant.
The backward parabolic cylinder satisfies $|Q_r|\le Cr^5$ for
$r\le 1$.

The Navier--Stokes equations are invariant under the parabolic rescaling
\begin{equation}\label{eq:scaling}
u_\lambda(x,t)=\lambda\,u(\lambda x,\lambda^2 t),\qquad
p_\lambda(x,t)=\lambda^2\,p(\lambda x,\lambda^2 t),
\end{equation}
under which $\omega_\lambda=\lambda^2\omega(\lambda x,\lambda^2 t)$.

\subsection{Vorticity direction decomposition}

On the open set $\{\omega\neq 0\}$, we decompose $\omega=\rho\,\xi$
where $\rho=|\omega|\ge 0$ is the \emph{vorticity magnitude} and
$\xi=\omega/|\omega|\in\Sbb^2$ is the \emph{vorticity direction}.  A
standard computation (see, e.g., \cite{ConstantinFefferman1993}) shows
that the magnitude satisfies
\begin{equation}\label{eq:amplitude}
\partial_t\rho + u\cdot\nabla\rho - \nu\Delta\rho
= \rho\bigl(\sigma - |\nabla\xi|^2\bigr),
\end{equation}
where $\sigma = S\xi\cdot\xi$ is the vortex stretching scalar.

\subsection{The $\rho^{3/2}$ identity}

The following identity, obtained by a change of dependent variable, will
be central to the depletion estimates.

\begin{lemma}[The $\rho^{3/2}$ equation]\label{lem:rho32}
On $\{\rho>0\}$,
\begin{equation}\label{eq:rho32}
\partial_t(\rho^{3/2}) + u\cdot\nabla(\rho^{3/2})
- \nu\Delta(\rho^{3/2})
+ \tfrac{4}{3}\nu\,|\nabla(\rho^{3/4})|^2
= \tfrac{3}{2}\rho^{3/2}\sigma
- \tfrac{3}{2}\rho^{3/2}|\nabla\xi|^2.
\end{equation}
\end{lemma}

\begin{proof}
Set $f(s)=s^{3/2}$ and apply the chain rule
to~\eqref{eq:amplitude}.  Since
$f'(s)=\tfrac32 s^{1/2}$ and $f''(s)=\tfrac34 s^{-1/2}$, the
diffusion term produces
\[
\nu\Delta(f(\rho))
= \nu\bigl(f'(\rho)\,\Delta\rho + f''(\rho)\,|\nabla\rho|^2\bigr).
\]
Rearranging and using the identity
$f''(s)|\nabla s|^2
= \tfrac{3}{4}s^{-1/2}|\nabla s|^2
= \tfrac{4}{3}|\nabla(s^{3/4})|^2$
yields~\eqref{eq:rho32}.
\end{proof}

\begin{definition}[Weighted direction coherence]\label{def:Eomega}
For a cylinder $Q_r(z_0)$, define
\[
\Ecal_\omega(z_0,r)
:= \iint_{Q_r(z_0)}\rho^{3/2}\,|\nabla\xi|^2\,dx\,dt.
\]
\end{definition}

\begin{lemma}[Localized $\Ecal_\omega$ bound]\label{lem:Eomega-local}
Let $u$ be smooth on $Q_{2r}(z_0)$ with
$\|\omega\|_{L^\infty(Q_{2r})}\le M$.  Let
$\phi\in C_c^\infty(Q_{2r})$ satisfy $\phi\equiv 1$ on $Q_r$ with
$|\nabla\phi|\lesssim r^{-1}$, $|\partial_t\phi|\lesssim r^{-2}$.  Then
in a Galilean frame with $(u)_{B_{2r}}=0$,
\begin{equation}\label{eq:Eomega-bound}
\Ecal_\omega(z_0,r)
\le C\iint_{Q_{2r}}\rho^{3/2}\,|\sigma|\,dx\,dt
+ C_{\mathrm{bdy}}\,M^{3/2}\,r^3,
\end{equation}
where $C$ is universal and $C_{\mathrm{bdy}}$ depends only on dimension
and~$\nu$.
\end{lemma}

\begin{proof}
Multiply~\eqref{eq:rho32} by $\phi^2$ and integrate over $Q_{2r}$.
Since $|\nabla(\rho^{3/4})|^2\ge 0$, the corresponding term on the
left-hand side may be discarded (it only improves the inequality).
Integration by parts on the transport and diffusion terms produces
cutoff errors bounded by
\[
C r^{-2}\iint_{Q_{2r}}\rho^{3/2}\,dx\,dt
+ C\sup_t\int_{B_{2r}}\rho^{3/2}\,dx.
\]
In the Galilean frame where $(u)_{B_{2r}}=0$, the Poincar\'e inequality
and bounded vorticity give $\|u\|_{L^\infty(B_{2r})}\le CrM$, so the
advection cutoff error is $O(M^{3/2}r^3)$.  Finally, $\rho\le M$
implies $\rho^{3/2}\le M^{3/2}$, and $|Q_{2r}|\le Cr^5$, which
yields~\eqref{eq:Eomega-bound}.
\end{proof}

\subsection{Serrin interior regularity}

\begin{lemma}[Interior derivative bound]\label{lem:serrin}
If $\|\omega\|_{L^\infty(Q_1(z_0))}\le 1$, then $u$ is smooth on
$Q_{1/2}(z_0)$ and
\[
\|\nabla\omega\|_{L^\infty(Q_{1/2})} \le C_{\mathrm{Ser}},
\]
where $C_{\mathrm{Ser}}$ is a universal constant depending only on the
dimension and~$\nu$.
\end{lemma}

\begin{proof}
Bounded vorticity on $Q_1$ gives locally bounded velocity via the
Biot--Savart law and the Poincar\'e inequality.  The Serrin interior
regularity theorem~\cite{Serrin1962} then yields $C^\alpha$ estimates
for the velocity on $Q_{3/4}$.  A bootstrap follows: the $C^\alpha$
velocity and the bounded vorticity serve as coefficients in the
vorticity equation, which is parabolic; the parabolic Schauder
estimates~\cite[Chapter~4]{Lieberman} then produce $C^{2+\alpha}$
bounds on $Q_{5/8}$, and iterating gives bounds on all higher
derivatives on $Q_{1/2}$.  All constants depend only on the dimension
and~$\nu$.
\end{proof}


%% ====================================================================
\section{The running-max ancient element}\label{sec:ancient}
%% ====================================================================

We now construct the blow-up profile via a running-max rescaling
procedure.  The essential idea is to rescale about space-time points
where the vorticity achieves its running maximum, ensuring that the
rescaled vorticity is globally bounded and normalized at the origin.

\begin{definition}[Running-max normalization]\label{def:running-max}
Assume $T^*<\infty$.  Choose a sequence of \emph{running-max times}
$t_k\uparrow T^*$ such that
\[
\|\omega(\cdot,t)\|_{L^\infty}
\le\|\omega(\cdot,t_k)\|_{L^\infty}=:M_k
\quad\text{for all } t\le t_k.
\]
For each $k$, choose a spatial point $x_k\in\R^3$ satisfying
$|\omega(x_k,t_k)|\ge(1-1/k)M_k=:A_k$.  Set
$\lambda_k=A_k^{-1/2}$ and define the rescaled fields
\begin{equation}\label{eq:rescaled}
u^{(k)}(y,s)=\lambda_k\,u(x_k+\lambda_k y,\;t_k+\lambda_k^2 s),
\qquad
\omega^{(k)}=\curl\,u^{(k)}.
\end{equation}
\end{definition}

\begin{lemma}[Properties of the rescaled sequence]\label{lem:rescaled}
The rescaled fields satisfy:
\begin{enumerate}[label=\textup{(\roman*)}, ref=\textup{\roman*},
  leftmargin=2em, itemsep=4pt]
\item\label{it:rescaled-norm}
$|\omega^{(k)}(0,0)|=1$ for every $k$.
\item\label{it:rescaled-bound}
$\|\omega^{(k)}(\cdot,s)\|_{L^\infty}\le
\gamma_k:=M_k/A_k\le (1-1/k)^{-1}$, so that $\gamma_k\to 1$ as
$k\to\infty$.
\item\label{it:rescaled-domain}
Each $u^{(k)}$ is defined on
$\R^3\times(-\lambda_k^{-2}t_k,\,0]$, and since $\lambda_k\to 0$ and
$t_k\to T^*>0$, these domains exhaust $\R^3\times(-\infty,0]$.
\end{enumerate}
\end{lemma}

\begin{proof}
(\ref{it:rescaled-norm}):
By the rescaling~\eqref{eq:scaling}, $\omega^{(k)}(0,0)
= \lambda_k^2\,\omega(x_k,t_k)$.  Since
$\lambda_k = A_k^{-1/2}$, we have
$|\omega^{(k)}(0,0)| = A_k^{-1}\,|\omega(x_k,t_k)|$.
The choice $|\omega(x_k,t_k)|\ge (1-1/k)M_k = A_k$ gives
$|\omega^{(k)}(0,0)| = 1$.

\smallskip\noindent
(\ref{it:rescaled-bound}):
For any $s\le 0$, the rescaling gives
$\|\omega^{(k)}(\cdot,s)\|_{L^\infty}
= \lambda_k^2\,\|\omega(\cdot,t_k+\lambda_k^2 s)\|_{L^\infty}$.
Since $t_k+\lambda_k^2 s\le t_k$, the running-max condition yields
$\|\omega(\cdot,t_k+\lambda_k^2 s)\|_{L^\infty}\le M_k$, so
$\|\omega^{(k)}(\cdot,s)\|_{L^\infty}
\le \lambda_k^2 M_k = M_k/A_k = \gamma_k \le (1-1/k)^{-1}$.

\smallskip\noindent
(\ref{it:rescaled-domain}):
The rescaled time variable is $s=(t-t_k)/\lambda_k^2$, so
$t\ge 0$ corresponds to $s\ge -t_k/\lambda_k^2 = -t_k A_k$.  Since
$A_k\ge M_k/2\to\infty$ and $t_k\to T^*>0$, the lower bound
$-t_k A_k\to -\infty$.
\end{proof}

\begin{lemma}[Ancient element extraction]\label{lem:ancient}
There exists a subsequence \textup{(}still denoted
$u^{(k)}$\textup{)} converging to a limit $(u^\infty,p^\infty)$ that is
a suitable weak solution on $\R^3\times(-\infty,0]$ satisfying
\[
|\omega^\infty(0,0)|=1,
\qquad
\|\omega^\infty\|_{L^\infty(\R^3\times(-\infty,0])}\le 1.
\]
In particular, $u^\infty$ is nontrivial.  The convergence is strong in
$L^p_{\mathrm{loc}}$ for every $p<3$ and in $C^\alpha_{\mathrm{loc}}$
on compact parabolic cylinders, the latter by the interior estimates of
Lemma~\ref{lem:serrin}.
\end{lemma}

\begin{proof}
The uniform bound $\|\omega^{(k)}\|_{L^\infty}\le\gamma_k\le 2$ (for
$k$ large) yields, via the Biot--Savart law, uniform local energy bounds
on each compact cylinder $Q_R$.  The Aubin--Lions compactness
lemma~\cite{Aubin1963,Lions1969} provides strong $L^2_{\mathrm{loc}}$
convergence along a subsequence; interpolation with the uniform
$L^\infty$ bound on vorticity upgrades this to $L^p_{\mathrm{loc}}$ for
every $p<3$.  The local energy inequality passes to the limit by lower
semicontinuity of the dissipation.  Finally, the normalization
$|\omega^{(k)}(0,0)|=1$ and the $C^\alpha_{\mathrm{loc}}$ convergence
guarantee $|\omega^\infty(0,0)|=1$.
\end{proof}


%% ====================================================================
\section{Weighted near-field stretching depletion}%
\label{sec:nearfield}
%% ====================================================================

The key estimate of this section shows that the $\rho^{3/2}$-weighted
near-field stretching is controlled at the natural parabolic scaling,
without any assumption on the direction field beyond pointwise
boundedness.

\begin{definition}[Near-field/tail decomposition]\label{def:near-tail}
For a scale $r>0$ and a point $x\in\R^3$, decompose the stretching
scalar as
\[
\sigma(x)=\sigma_{\mathrm{near}}(x;r)+\sigma_{\mathrm{tail}}(x;r),
\]
where $\sigma_{\mathrm{near}}$ is the contribution to $S\xi\cdot\xi$
from the Biot--Savart integral over $B_r(x)$, and
$\sigma_{\mathrm{tail}}$ is the contribution from
$\R^3\setminus B_r(x)$.
\end{definition}

\begin{theorem}[$\rho^{3/2}$-weighted near-field depletion]%
\label{thm:nearfield}
For the ancient element $(u^\infty,\omega^\infty)$ of
Lemma~\ref{lem:ancient}, there exists a universal constant $C<\infty$
such that for every $z_0\in\R^3\times(-\infty,0]$ and every
$0<r\le 1$,
\begin{equation}\label{eq:nearfield}
\iint_{Q_r(z_0)}\rho^{3/2}\,|\sigma_{\mathrm{near}}(x;r)|\,dx\,dt
\le C\,r^5.
\end{equation}
\end{theorem}

\begin{proof}
Fix a time $t$ and a spatial center $x_0$.  Let
$\psi\in C_c^\infty(B_{4r}(x_0))$ be a smooth cutoff satisfying
$\psi\equiv 1$ on $B_{2r}(x_0)$.

\step{1 (Commutator decomposition)}
The Biot--Savart representation of the strain expresses the near-field
stretching as the sum of a commutator term and a constant-direction
remainder (see~\cite{ConstantinFefferman1993}).  Specifically,
\[
\sigma_{\mathrm{near}}
=\sigma_{\mathrm{near}}^{\mathrm{osc}}
+\sigma_{\mathrm{near}}^{\mathrm{const}},
\]
where $\sigma_{\mathrm{near}}^{\mathrm{osc}}$ is a finite sum of terms
of the form $[T_{r},\xi_\ell](\psi\rho)$, with $T_r$ a truncated
Calder\'on--Zygmund operator, and
$\sigma_{\mathrm{near}}^{\mathrm{const}}$ is a Calder\'on--Zygmund
operator applied to $\rho(\xi(x)-\xi(x_0))$.

\step{2 ($L^3$ bound via the CRW commutator theorem)}
By the Coifman--Rochberg--Weiss theorem~\cite{CRW1976},
\[
\|[T_r,\xi_\ell](\psi\rho)\|_{L^3}
\le C_{\mathrm{CRW}}\,\|\xi_\ell\|_{\BMO}\,\|\psi\rho\|_{L^3}.
\]
Since $\xi$ takes values in $\Sbb^2$, we have
$\|\xi_\ell\|_{\BMO}\le 2$.  Since $\rho\le 1$ on the ancient element
and $\psi$ is supported on $B_{4r}$, we have
$\|\psi\rho\|_{L^3}\le |B_{4r}|^{1/3}\le Cr$.  The constant-direction
remainder satisfies the same $L^3$ bound by
Lemma~\ref{lem:constdir-remainder} below.  Combining these estimates
gives
\begin{equation}\label{eq:sigma-near-L3}
\|\sigma_{\mathrm{near}}(\cdot,t)\|_{L^3(B_r)}\le Cr.
\end{equation}

\step{3 (H\"older pairing)}
Applying H\"older's inequality with exponents $\tfrac{3}{2}$ and $3$
yields
\[
\int_{B_r}\rho^{3/2}\,|\sigma_{\mathrm{near}}|\,dx
\le \|\rho^{3/2}\|_{L^{3/2}(B_r)}\,
\|\sigma_{\mathrm{near}}\|_{L^3(B_r)}.
\]
Since $\rho\le 1$, we have
$\|\rho^{3/2}\|_{L^{3/2}(B_r)}
=\bigl(\int_{B_r}\rho^{9/4}\,dx\bigr)^{2/3}
\le |B_r|^{2/3}\le Cr^2$.
Combining with~\eqref{eq:sigma-near-L3} gives
$\int_{B_r}\rho^{3/2}\,|\sigma_{\mathrm{near}}|\,dx\le Cr^3$ for each
time slice.  Integrating over the time interval of length~$r^2$
yields~\eqref{eq:nearfield}.
\end{proof}

\begin{lemma}[Constant-direction remainder]\label{lem:constdir-remainder}
Let $\omega=\rho\xi$ be divergence-free with $\rho\le 1$, and let
$a\in\Sbb^2$ be a constant unit vector.  Then the constant-direction
contribution to the stretching satisfies
\[
\|T_a\rho\|_{L^3(B_r)}
\le C\,\|\rho(a-\xi)\|_{L^3(B_{2r})}\le Cr,
\]
where $T_a$ is a fixed Calder\'on--Zygmund operator determined by~$a$.
\end{lemma}

\begin{proof}
The divergence-free condition $\nabla\cdot\omega=0$ gives
$a\cdot\nabla\rho=\nabla\cdot(\rho\,a-\omega)
=\nabla\cdot(\rho(a-\xi))$.  Consequently,
\[
a\times\nabla\bigl((a\cdot\nabla)(-\Delta)^{-1}\rho\bigr)
= a\times\nabla(-\Delta)^{-1}\nabla\cdot\bigl(\rho(a-\xi)\bigr),
\]
which expresses $T_a\rho$ as a Calder\'on--Zygmund operator applied to
$\rho(a-\xi)$.  Since $|\rho(a-\xi)|\le 2\rho\le 2$ and the support is
contained in $B_{4r}$, the $L^3$ boundedness of Calder\'on--Zygmund
operators gives $\|T_a\rho\|_{L^3(B_r)}\le Cr$.
\end{proof}

\begin{remark}\label{rem:unconditional}
Theorem~\ref{thm:nearfield} uses only the bounds
$\|\omega^\infty\|_{L^\infty}\le 1$ and $|\xi|\le 1$.  No smallness or
continuity assumption on $\xi$ (such as VMO or BMO smallness) is
required, and no tail control is assumed.  This unconditional character
is essential for the argument.
\end{remark}


%% ====================================================================
\section{Elimination of the far tail}\label{sec:tail}
%% ====================================================================

The near-field estimate of the previous section controls the stretching
from nearby vorticity.  We now show that the contributions from distant
vorticity either vanish in the blow-up limit or are absorbed into the
near-field bound.

\subsection{Three-way decomposition}

Fix a rescaled radius $R\ge 1$ and an intermediate scale $R_1>R$ (both
in rescaled coordinates).  The tail of the stretching decomposes as
\[
\sigma_{\mathrm{tail}}(x;r)
= \sigma_{\mathrm{int}}(x;r,\lambda_k R_1)
+ \sigma_{\mathrm{ext}}(x;\lambda_k R_1),
\]
where $\sigma_{\mathrm{int}}$ collects the Biot--Savart contribution
from the annulus $\{r<|y-x|<\lambda_k R_1\}$ (rescaled distances between
$R$ and $R_1$), and $\sigma_{\mathrm{ext}}$ collects the contribution
from $\{|y-x|>\lambda_k R_1\}$ (physical distances exceeding
$\lambda_k R_1$).

\subsection{Vanishing of the external tail}

\begin{theorem}[External tail vanishes]\label{thm:ext-tail}
For the rescaled sequence $\omega^{(k)}$ on the cylinder $Q_R$,
\begin{equation}\label{eq:ext-tail}
\iint_{Q_R} (\rho^{(k)})^{3/2}\,
|\sigma_{\mathrm{ext}}^{(k)}|\,dy\,ds
\le C\,R^4\,R_1^{-3/2}\,M_k^{-3/4}\,(E_0/\nu)^{1/2},
\end{equation}
where $E_0=\|u_0\|_{L^2}^2$ is the initial energy.  In particular, the
right-hand side tends to zero as $k\to\infty$ for any fixed $R$
and~$R_1$.
\end{theorem}

\begin{proof}
\step{1 (Pointwise kernel estimate)}
For $|y-x|>\lambda_k R_1$ in rescaled coordinates (equivalently,
$|\eta|>R_1$), the Cauchy--Schwarz inequality applied to the
Biot--Savart kernel gives
\[
|\sigma_{\mathrm{ext}}^{(k)}(y)|
\le \|K\|_{L^2(|\eta|>R_1)}\,\|\omega^{(k)}(\cdot,s)\|_{L^2(\R^3)}.
\]
The kernel satisfies $\|K\|_{L^2(|\eta|>R_1)}=CR_1^{-3/2}$.  Under the
rescaling~\eqref{eq:rescaled}, a change of variables yields
$\|\omega^{(k)}(\cdot,s)\|_{L^2(\R^3)}^2
=\lambda_k\,\|\omega(\cdot,t)\|_{L^2(\R^3)}^2$.

\step{2 (Time integration via the energy inequality)}
The physical time interval is $I_k=[t_k-\lambda_k^2 R^2,\,t_k]$, of
length $\lambda_k^2 R^2$.  By the Cauchy--Schwarz inequality in time and
the energy inequality
$\int_0^{T^*}\|\omega\|_{L^2}^2\,dt\le E_0/(2\nu)$:
\[
\int_{I_k}\|\omega(\cdot,t)\|_{L^2}\,dt
\le (\lambda_k^2 R^2)^{1/2}
\Bigl(\int_{I_k}\|\omega\|_{L^2}^2\,dt\Bigr)^{1/2}
\le \lambda_k R\,(E_0/(2\nu))^{1/2}.
\]

\step{3 (Assembly)}
On each time slice, $\rho^{(k)}\le \gamma_k\le 2$ and
$|B_R|\le CR^3$, so
\[
\int_{B_R}(\rho^{(k)})^{3/2}\,
|\sigma_{\mathrm{ext}}^{(k)}|\,dy
\le CR^3\cdot R_1^{-3/2}\,\lambda_k^{1/2}\,
\|\omega(\cdot,t)\|_{L^2}.
\]
Integrating over the rescaled time interval (of length~$R^2$) and
substituting the estimate from Step~2 yields
\[
\iint_{Q_R} (\rho^{(k)})^{3/2}\,
|\sigma_{\mathrm{ext}}^{(k)}|\,dy\,ds
\le CR^4\,R_1^{-3/2}\,\lambda_k^{3/2}\,(E_0/\nu)^{1/2}.
\]
Since $\lambda_k=A_k^{-1/2}\le M_k^{-1/2}$, we have
$\lambda_k^{3/2}\le M_k^{-3/4}\to 0$.
\end{proof}

\subsection{Control of the intermediate tail}

\begin{theorem}[Intermediate tail bound]\label{thm:int-tail}
In rescaled coordinates, the intermediate-tail contribution over
$\{R<|\eta|<R_1\}$ satisfies
\[
\iint_{Q_R}(\rho^{(k)})^{3/2}\,
|\sigma_{\mathrm{int}}^{(k)}|\,dy\,ds
\le C\,R^4\,R_1,
\]
with $C$ a universal constant independent of $k$ and~$M_k$.
\end{theorem}

\begin{proof}
In rescaled coordinates, $\rho^{(k)}\le\gamma_k\le 2$.  The same
Coifman--Rochberg--Weiss argument used in the proof of
Theorem~\ref{thm:nearfield} applies, with the Calder\'on--Zygmund
operators truncated to the annulus $\{R<|\eta|<R_1\}$ and the localizer
$\psi$ supported on $B_{R_1}$.  The key input is
$\|\psi\rho^{(k)}\|_{L^3}\le CR_1$.  H\"older's inequality then gives
$Cr^2\cdot CR_1$ per time slice on $B_R$, and integration over the time
interval of length~$R^2$ produces the stated bound.
\end{proof}

\begin{remark}\label{rem:int-tail-choice}
The intermediate-tail bound grows with $R_1$, while the external-tail
bound improves as $R_1$ increases.  The two contributions are balanced by
choosing $R_1$ proportional to $R$ (specifically, $R_1=2R$), which makes
the intermediate tail $O(R^5)$---the same order as the near-field.
\end{remark}

\subsection{Coherence bound for the ancient element}

\begin{corollary}[Global coherence estimate]\label{cor:no-far-tail}
For the ancient element $(u^\infty,\omega^\infty)$ of
Lemma~\ref{lem:ancient}, the weighted direction coherence satisfies
\begin{equation}\label{eq:Eomega-ancient}
\Ecal_\omega(z_0,R) \le C_1 R^5 + C_2 R^3
\end{equation}
for every $z_0\in\R^3\times(-\infty,0]$ and every $0<R\le 1$, where
$C_1$ and $C_2$ are universal constants.
\end{corollary}

\begin{proof}
Choose $R_1=2R$.  By Remark~\ref{rem:int-tail-choice}, the
intermediate-tail contribution is at most $CR^5$.  By
Theorem~\ref{thm:ext-tail}, the external-tail contribution vanishes in
the limit $k\to\infty$.  Combining with the near-field bound of
Theorem~\ref{thm:nearfield} (which gives $CR^5$) and the localized
$\Ecal_\omega$ bound of Lemma~\ref{lem:Eomega-local} (applied with
$M=1$ in the Galilean frame), we
obtain~\eqref{eq:Eomega-ancient}.
\end{proof}


%% ====================================================================
\section{Direction regularity via the Struwe threshold}%
\label{sec:direction}
%% ====================================================================

We now establish direction regularity for the ancient element on the
high-vorticity set.  The argument combines the coherence estimate of
Corollary~\ref{cor:no-far-tail} with the $\eps$-regularity theory
for harmonic-map heat flow, yielding a uniform gradient bound for $\xi$
wherever $\rho$ is bounded away from zero.

\subsection{Lower bound on the vorticity magnitude}

\begin{lemma}[Universal ball with $\rho\ge\tfrac12$]%
\label{lem:rho-lower}
There exists a universal radius $\delta=1/(2C_{\mathrm{Ser}})>0$ such
that $\rho^\infty\ge\tfrac12$ on the cylinder $Q_\delta(0,0)$.
\end{lemma}

\begin{proof}
By Lemma~\ref{lem:serrin},
$|\nabla\omega^\infty|\le C_{\mathrm{Ser}}$ on $Q_{1/2}(0,0)$.  Since
$|\nabla\rho|\le|\nabla\omega|$ pointwise and $\rho^\infty(0,0)=1$, we
have
\[
\rho^\infty(y,s)\ge 1-C_{\mathrm{Ser}}\,|(y,s)| \ge \tfrac12
\quad\text{for all }(y,s)\in Q_\delta(0,0).
\qedhere
\]
\end{proof}

\subsection{Threshold comparison}

\begin{proposition}[Direction energy below the Struwe threshold]%
\label{prop:threshold}
The unweighted direction energy on $Q_\delta(0,0)$ satisfies
\begin{equation}\label{eq:threshold}
\iint_{Q_\delta}|\nabla\xi^\infty|^2\,dx\,dt
\le 2\sqrt{2}\,(C_1\delta^5+C_2\delta^3)
=: E_0^*,
\end{equation}
where $C_1,C_2$ are the constants from~\eqref{eq:Eomega-ancient}.
Moreover, $E_0^*$ is strictly less than the Struwe $\eps$-regularity
threshold $\eps_{\mathrm{Str}}=4\pi$ for harmonic maps
into~$\Sbb^2$: concretely, $E_0^*\lesssim C_{\mathrm{Ser}}^{-3}$,
while $\eps_{\mathrm{Str}}\approx 12.6$.
\end{proposition}

\begin{proof}
On $Q_\delta(0,0)$, Lemma~\ref{lem:rho-lower} gives
$\rho^\infty\ge\tfrac12$, hence
\[
|\nabla\xi|^2
\le \bigl(\tfrac12\bigr)^{-3/2}\,\rho^{3/2}\,|\nabla\xi|^2
= 2\sqrt{2}\,\rho^{3/2}\,|\nabla\xi|^2.
\]
Integrating over $Q_\delta$ and applying the coherence
bound~\eqref{eq:Eomega-ancient} yields~\eqref{eq:threshold}.  For the
comparison, recall $\delta=1/(2C_{\mathrm{Ser}})$, so
$E_0^*\lesssim C_{\mathrm{Ser}}^{-3}$, which is small for any
reasonable value of $C_{\mathrm{Ser}}\ge 1$.
\end{proof}

\subsection{Perturbed $\eps$-regularity for the direction equation}

The direction field $\xi$ does not satisfy harmonic-map heat flow
exactly; it satisfies a perturbed equation with a drift term and
tangential forcing.  We record the precise form and verify that the
perturbation is controllable.

On $\{\rho>0\}$, the direction field satisfies
\begin{equation}\label{eq:direction-eqn}
\partial_t\xi + u\cdot\nabla\xi - \nu\Delta\xi
= \nu\,|\nabla\xi|^2\,\xi + P_\xi(S\xi)
+ 2\nu\,P_\xi\bigl((\nabla\log\rho)\cdot\nabla\xi\bigr),
\end{equation}
where $P_\xi = I - \xi\otimes\xi$ is the projection onto the tangent
plane of $\Sbb^2$ at $\xi$.

\begin{lemma}[Perturbed $\eps$-regularity]%
\label{lem:struwe-perturbed}
Let $\xi\colon Q_r(z_0)\to\Sbb^2$ be a smooth solution
of~\eqref{eq:direction-eqn} on a parabolic cylinder $Q_r(z_0)$ with
$\rho\ge\eta>0$.  Suppose the following bounds hold in the Galilean
frame where $(u)_{B_r}=0$:
\begin{enumerate}[label=\textup{(\alph*)}, ref=\textup{\alph*},
  leftmargin=2em, itemsep=2pt]
\item $\|u\|_{L^\infty(Q_r)}\le V$,
\item $\|S\|_{L^\infty(Q_r)}\le \Lambda$,
\item $\|\nabla\log\rho\|_{L^\infty(Q_r)}\le G$.
\end{enumerate}
There exist constants $\eps_0>0$ and $C_S<\infty$, depending only on the
dimension and $\nu$, such that if
\[
\iint_{Q_r}|\nabla\xi|^2\,dx\,dt < \eps_0
\quad\text{and}\quad
Vr + \Lambda r^2 + Gr < \eps_0,
\]
then $|\nabla\xi(z_0)| \le C_S/r$.
\end{lemma}

\begin{proof}
After rescaling $Q_r\to Q_1$, the equation for
$\tilde\xi(y,s)=\xi(x_0+ry,\,t_0+r^2s)$ becomes
\[
\partial_s\tilde\xi - \nu\Delta_y\tilde\xi
= \nu\,|\nabla_y\tilde\xi|^2\,\tilde\xi + \tilde F,
\]
where the perturbation $\tilde F$ collects the rescaled drift,
stretching forcing, and geometric coupling terms.  The drift contributes
a term bounded by $Vr\,|\nabla\tilde\xi|$; the stretching forcing is
bounded by $\Lambda r^2$; and the geometric coupling is bounded by
$Gr\,|\nabla\tilde\xi|$.  All three contributions are small when
$Vr+\Lambda r^2 + Gr<\eps_0$.

The Struwe monotonicity formula~\cite{Struwe1988} for the localized
energy $\Phi(R)=\int|\nabla\tilde\xi|^2\,G_R\,dx$ acquires perturbation
errors that shift the $\eps$-regularity threshold from $4\pi$ to
$4\pi - C(Vr + \Lambda r^2 + Gr)$.  The perturbation analysis of such
drift-diffusion modifications of harmonic-map heat flow is carried out
in Lin and Wang~\cite{LinWang1998}.  Provided the energy is below the
perturbed threshold, the standard compactness-and-contradiction argument
yields the pointwise gradient bound $|\nabla\tilde\xi(0,0)|\le C_S$,
which rescales to $|\nabla\xi(z_0)|\le C_S/r$.
\end{proof}

\subsection{Direction gradient bound on the high-vorticity set}

\begin{theorem}[Bounded direction gradient]\label{thm:direction}
For every $\eta>0$, there exists $C(\eta)<\infty$ such that
\begin{equation}\label{eq:direction-gradient}
|\nabla\xi^\infty(z_1)|\le C(\eta)
\qquad\text{for every }z_1\text{ with }
\rho^\infty(z_1)\ge\eta>0.
\end{equation}
Explicitly, $C(\eta)=C_S/\delta_\eta$ where
$\delta_\eta=\eta/(2C_{\mathrm{Ser}})$ and $C_S$ is the Struwe constant
from Lemma~\textup{\ref{lem:struwe-perturbed}}.
\end{theorem}

\begin{proof}
Fix any point $z_1=(x_1,t_1)$ with $\rho^\infty(z_1)\ge\eta>0$.

\step{1 (Local setup)}
By Lemma~\ref{lem:serrin},
$|\nabla\omega^\infty|\le C_{\mathrm{Ser}}$ on $Q_{1/2}(z_1)$.
Setting $\delta_\eta=\eta/(2C_{\mathrm{Ser}})$, the same Lipschitz
argument as in Lemma~\ref{lem:rho-lower} gives
$\rho^\infty\ge\eta/2$ on $Q_{\delta_\eta}(z_1)$.  The coherence
bound~\eqref{eq:Eomega-ancient} at $z_1$ with $R=\delta_\eta$ yields
\[
\iint_{Q_{\delta_\eta}(z_1)}|\nabla\xi^\infty|^2\,dx\,dt
\le 2\sqrt{2}\,(\eta/2)^{-3/2}\,
(C_1\delta_\eta^5+C_2\delta_\eta^3) = O(\delta_\eta^3).
\]

\step{2 (Verification of perturbation bounds)}
In the Galilean frame on $Q_{\delta_\eta}(z_1)$:
\begin{itemize}[leftmargin=2em]
\item The Poincar\'e inequality and
$\|\omega^\infty\|_{L^\infty}\le 1$ give
$V=\|u\|_{L^\infty}\le C\delta_\eta$.
\item Lemma~\ref{lem:serrin} gives
$\Lambda=\|S\|_{L^\infty}\le C_{\mathrm{Ser}}$.
\item The lower bound $\rho\ge\eta/2$ and
$|\nabla\rho|\le C_{\mathrm{Ser}}$ give
$G=\|\nabla\log\rho\|_{L^\infty}\le 2C_{\mathrm{Ser}}/\eta$.
\end{itemize}
Then
$V\delta_\eta + \Lambda\delta_\eta^2 + G\delta_\eta
= O(\delta_\eta^2) + O(\delta_\eta^2) + O(1/C_{\mathrm{Ser}})$, all of
which are universally small.

\step{3 (Application of $\eps$-regularity)}
Both hypotheses of Lemma~\ref{lem:struwe-perturbed} are satisfied (the
energy is far below $\eps_0$, and the perturbation parameters are
universally small).  Therefore
$|\nabla\xi^\infty(z_1)|\le C_S/\delta_\eta$.  Since
$z_1\in\{\rho^\infty\ge\eta\}$ was arbitrary, the bound holds uniformly
on $\{\rho^\infty\ge\eta\}$.
\end{proof}

\begin{remark}[Independence of estimates]\label{rem:no-circularity}
The energy bound (Proposition~\ref{prop:threshold}) and the perturbation
control (Step~2 above) are derived from independent sources: the former
uses the CRW commutator theorem and the $\rho^{3/2}$ identity, while
the latter uses only Serrin interior regularity.  Neither requires any
prior assumption on $\nabla\xi$, so no circularity arises.
\end{remark}

\begin{remark}[Obstruction to direction constancy]%
\label{rem:constancy-obstruction}
To upgrade the bounded gradient of Theorem~\ref{thm:direction} to full
direction constancy ($\nabla\xi^\infty\equiv 0$ on
$\{\rho^\infty>0\}$), one natural approach is a Liouville-type
rescaling: define
$\xi^{(R)}(y,s)=\xi^\infty(x_1+Ry,\,t_1+R^2 s)$ for large $R$ and
attempt to apply Lemma~\ref{lem:struwe-perturbed} at a fixed scale
$\delta_\eta$ to conclude
$|\nabla\xi^{(R)}(0,0)|\le C_S/\delta_\eta$, which would yield
$|\nabla\xi^\infty(z_1)|\le C_S/(R\delta_\eta)\to 0$ as $R\to\infty$.

Two obstructions prevent this argument from closing:
\begin{enumerate}[label=\textup{(\alph*)}, leftmargin=2em, itemsep=4pt]
\item \emph{Loss of the lower bound on $\rho$.}
The ball $Q_{\delta_\eta}(0,0)$ in rescaled coordinates corresponds to
$Q_{R\delta_\eta}(z_1)$ in original coordinates.  The lower bound
$\rho^\infty\ge\eta/2$ holds only on the small cylinder
$Q_{\delta_\eta}(z_1)$, not on $Q_{R\delta_\eta}(z_1)$ for $R\gg 1$.
Without this lower bound, the unweighted direction energy cannot be
extracted from the weighted coherence bound~\eqref{eq:Eomega-ancient}.
\item \emph{Growth of the perturbation parameters.}
In the rescaled frame, the drift velocity scales as $V\sim R$ and the
geometric coupling as $G\sim R$, so the smallness condition
$V\delta_\eta+\Lambda\delta_\eta^2+G\delta_\eta<\eps_0$ fails
for large~$R$.
\end{enumerate}
Closing this gap is equivalent to proving direction constancy on the
full support of the vorticity; see the conditional classification in
Theorem~\ref{thm:conditional} and the discussion in
Section~\ref{sec:conjecture}.
\end{remark}


%% ====================================================================
\section{Conditional collapse to rigid rotation}\label{sec:collapse}
%% ====================================================================

In this section, we show that \emph{if} the direction field is constant
on the support of the vorticity (i.e., if $\nabla\xi^\infty\equiv 0$ on
$\{\rho^\infty>0\}$), then the vorticity magnitude is identically one
and the ancient element is the rigid rotation.  The results of this
section are conditional on direction constancy; they are used in the
proof of Theorem~\ref{thm:conditional}.

\begin{theorem}[Strict positivity of the vorticity magnitude]%
\label{thm:rho-pos}
Assume $\nabla\xi^\infty\equiv 0$ on $\{\rho^\infty>0\}$.  Then
$\rho^\infty>0$ on all of $\R^3\times(-\infty,0]$.
\end{theorem}

\begin{proof}
By hypothesis, $\nabla\xi\equiv 0$ on $\{\rho>0\}$.  On this set, the
$|\nabla\xi|^2$ term in the amplitude equation~\eqref{eq:amplitude}
vanishes, so
\begin{equation}\label{eq:rho-linear}
\partial_t\rho + u\cdot\nabla\rho - \nu\Delta\rho = \sigma\,\rho
\end{equation}
with $\sigma$ locally bounded (as $\sigma$ is a Calder\'on--Zygmund
operator applied to $\omega\in L^\infty$, evaluated at points where
$\xi$ is constant).

We wish to remove the zeroth-order term.  For any fixed $T_0<0$, define
on $\R^3\times[T_0,0]$:
\[
\Sigma(x,t) = \int_{T_0}^t \sigma(x,s)\,ds,
\qquad
\tilde\rho(x,t) = \rho(x,t)\,e^{-\Sigma(x,t)}.
\]
Since $\|\sigma\|_{L^\infty(Q_R)}\le C_R$ for each $R>0$ (by the
Calder\'on--Zygmund theory and $\|\omega\|_{L^\infty}\le 1$), the
function $\Sigma$ is well defined and locally bounded on the finite time
interval $[T_0,0]$.  A direct computation shows that $\tilde\rho\ge 0$
satisfies
\[
\partial_t\tilde\rho + u\cdot\nabla\tilde\rho
- \nu\Delta\tilde\rho
= \nu\,e^{-\Sigma}\bigl(2\nabla\Sigma\cdot\nabla\rho
+ \rho\,\Delta\Sigma - \rho\,|\nabla\Sigma|^2\bigr),
\]
which is a linear parabolic equation with locally bounded coefficients
for $\tilde\rho$.  Since
$\tilde\rho(0,0)=\rho(0,0)\cdot e^{-\Sigma(0,0)}>0$, the strong
minimum principle for nonnegative solutions of linear parabolic equations
with locally bounded
coefficients~\cite[Chapter~7]{Lieberman} implies $\tilde\rho>0$ on
$\R^3\times[T_0,0]$.  Consequently, $\rho>0$ on
$\R^3\times[T_0,0]$.

Since $T_0<0$ was arbitrary, $\rho>0$ on all of
$\R^3\times(-\infty,0]$.
\end{proof}

\begin{theorem}[The vorticity magnitude is identically one]%
\label{thm:rho-one}
Under the hypothesis of Theorem~\textup{\ref{thm:rho-pos}} \textup{(}%
direction constancy\textup{)},
$\rho^\infty\equiv 1$ on $\R^3\times(-\infty,0]$.
\end{theorem}

\begin{proof}
By Theorem~\ref{thm:rho-pos}, $\rho>0$ everywhere, and by the
hypothesis of direction constancy, $\nabla\xi\equiv 0$ on
$\R^3\times(-\infty,0]$.  Since $\R^3\times(-\infty,0]$ is connected,
$\xi$ is a constant unit vector.  After a spatial rotation, we may
assume $\xi\equiv e_3$, so that $\omega=(0,0,\rho)$.

\step{1 (The horizontal velocity is independent of $x_3$)}
Since $\omega_1=\partial_2 u_3-\partial_3 u_2=0$ and
$\omega_2=\partial_3 u_1-\partial_1 u_3=0$, and since
$\omega_3=\partial_1 u_2-\partial_2 u_1=\rho>0$, the first two
vorticity components give
\[
\partial_3 u_2 = \partial_2 u_3, \qquad
\partial_3 u_1 = \partial_1 u_3.
\]
The third component of the vorticity equation (stretching of $\omega_3$
by $\partial_3 u_3$) and the amplitude
equation~\eqref{eq:rho-linear} together imply that
$\rho\,\partial_3 u_1=0$ and $\rho\,\partial_3 u_2=0$.  Since
$\rho>0$, we conclude $\partial_3 u_h\equiv 0$ on $\R^3$.

\step{2 (The vertical velocity vanishes)}
From $\omega_1=\omega_2=0$ we obtain
$\partial_2 u_3=\partial_3 u_2=0$ and
$\partial_1 u_3=\partial_3 u_1=0$ (using Step~1).  The
incompressibility condition $\nabla\cdot u=0$ then gives
$\partial_3 u_3=-\partial_1 u_1-\partial_2 u_2$, which depends only on
$(x_1,x_2,t)$.  Combined with the vanishing of $\partial_1 u_3$ and
$\partial_2 u_3$, the function $u_3$ depends only on $(x_3,t)$ and
satisfies $\partial_3^2 u_3 = \partial_3(\partial_3 u_3)$, where
$\partial_3 u_3$ is independent of $x_3$.  Linearity in $x_3$ forces
$u_3(x,t)=a(t)+b(t)\,x_3$ for functions $a(t)$ and $b(t)$.

We now show $b\equiv 0$.  Substituting into the Navier--Stokes equation
for $u_3$ gives $\dot b + b^2 = 0$ (the pressure gradient in $x_3$
absorbs the remaining terms).  The general solution is
$b(t)=1/(t-t_0)$ or $b\equiv 0$.

\begin{itemize}[leftmargin=2em]
\item If $b>0$ at some time, then $b(t)=1/(t-t_0)$ with $t_0>t$,
and $b$ blows up as $t\to t_0^-$, contradicting smoothness on
$(-\infty,0]$.
\item If $b<0$, then $b(t)=1/(t-t_0)$ with $t_0<t$.  The stretching
$\partial_3 u_3=b<0$ produces compression along~$x_3$.  The amplitude
equation reads
$\partial_t\rho + v\cdot\nabla_h\rho - \nu\Delta\rho = b\rho$ with
$b<0$.  By the maximum principle,
\[
\|\rho(\cdot,t)\|_{L^\infty}\le
\|\rho(\cdot,s)\|_{L^\infty}\,e^{\int_s^t b(\tau)\,d\tau}
\quad\text{for }s<t.
\]
Since $b(\tau)\sim 1/\tau$ as $\tau\to-\infty$, we have
$\int_{-\infty}^0 b(\tau)\,d\tau = -\infty$, forcing
$\|\rho(\cdot,0)\|_{L^\infty}=0$ and contradicting $\rho(0,0)=1$.
\end{itemize}
Therefore $b\equiv 0$, and after subtracting the Galilean drift
$a(t)\,e_3$, we have $u_3\equiv 0$.

\step{3 ($\rho\equiv 1$ by the strong minimum principle)}
With $\xi\equiv e_3$ and $u_3\equiv 0$, the stretching term
$\sigma=S\xi\cdot\xi=\partial_3 u_3=0$.  The amplitude equation
reduces to
\[
\partial_t\rho + v\cdot\nabla_h\rho - \nu\Delta_h\rho = 0,
\]
where $v=(u_1,u_2)$ is a smooth, locally bounded horizontal velocity.
Since $0\le\rho\le 1$ (the ancient element has
$\|\omega\|_{L^\infty}\le 1$), the function $f=1-\rho$ satisfies
$f\ge 0$, $f(0,0)=0$, and the same advection-diffusion equation.  By
the strong minimum principle for nonnegative solutions of parabolic
equations with locally bounded
drift~\cite[Chapter~7]{Lieberman}, $f\equiv 0$.  Therefore
$\rho\equiv 1$.
\end{proof}

\begin{corollary}[Identification of the ancient element]%
\label{cor:rigid}
Under the hypothesis of direction constancy, the ancient element is the
rigid-body rotation up to Galilean drift:
\[
u^\infty = \tfrac12(-x_2,x_1,0)+c(t),
\]
where $c(t)$ is a time-dependent translation velocity.  After
subtraction of the drift, $u^\infty=\tfrac12(-x_2,x_1,0)$.
\end{corollary}

\begin{proof}
With $\rho\equiv 1$ and $\xi\equiv e_3$, the vorticity is
$\omega^\infty\equiv(0,0,1)$.  The two-dimensional Biot--Savart law
recovers $v(x_h)=\tfrac12(-x_2,x_1)+\nabla\phi$ with $\Delta\phi=0$.
Smoothness and the absence of growth faster than linear force
$\nabla\phi$ to be at most a time-dependent constant $c(t)$.
\end{proof}


%% ====================================================================
\section{Proof of the main theorems}\label{sec:proofs}
%% ====================================================================

\begin{proof}[Proof of Theorem~\ref{thm:main}]
Assume $T^*<\infty$.  By Section~\ref{sec:ancient}, the running-max
blow-up extraction produces an ancient element $(u^\infty,p^\infty)$
with $|\omega^\infty(0,0)|=1$ and
$\|\omega^\infty\|_{L^\infty}\le 1$.

Part~(\ref{it:nearfield}) is Theorem~\ref{thm:nearfield}.

Part~(\ref{it:tail}) is Theorem~\ref{thm:ext-tail}: the external tail
vanishes at rate $M_k^{-3/4}$; for the ancient element, passage to the
limit absorbs the external contribution into the near-field, so the full
stretching equals its near-field at every scale.

Part~(\ref{it:coherence}) follows from Corollary~\ref{cor:no-far-tail}.

Part~(\ref{it:direction}) follows from Theorem~\ref{thm:direction}: at
any point $z_1$ with $\rho^\infty(z_1)\ge\eta>0$, the
Serrin--Lipschitz argument provides $\rho^\infty\ge\eta/2$ on a ball
$Q_{\delta_\eta}(z_1)$, the coherence bound converts the weighted energy
into an unweighted energy far below the Struwe threshold, and
Lemma~\ref{lem:struwe-perturbed} yields
$|\nabla\xi^\infty(z_1)|\le
C_S/\delta_\eta=2C_SC_{\mathrm{Ser}}/\eta$.
\end{proof}

\begin{proof}[Proof of Theorem~\ref{thm:conditional}]
Assume $\nabla\xi^\infty\equiv 0$ on $\{\rho^\infty>0\}$.  By
Theorem~\ref{thm:rho-pos}, $\rho^\infty>0$ on all of
$\R^3\times(-\infty,0]$, establishing~(\ref{it:pos}).  By
Theorem~\ref{thm:rho-one}, $\rho^\infty\equiv 1$,
establishing~(\ref{it:one}).  By Corollary~\ref{cor:rigid},
$u^\infty=\frac12(-x_2,x_1,0)+c(t)$,
establishing~(\ref{it:rigid}).
\end{proof}

\begin{proof}[Proof of Corollary~\ref{cor:equiv}]
The forward implication is immediate: if no finite-time blow-up occurs,
no blow-up limit exists.  For the reverse: if blow-up occurs at some
time $T^*<\infty$, then by the conditional classification
(Theorem~\ref{thm:conditional}), the ancient element is the rigid
rotation, contradicting the hypothesis.
\end{proof}


%% ====================================================================
\section{Discussion}\label{sec:conjecture}
%% ====================================================================

\subsection{The direction constancy gap}

Theorem~\ref{thm:main} establishes that the vorticity direction has a
bounded gradient on the high-vorticity set $\{\rho\ge\eta\}$, with the
bound depending only on $\eta$ and universal constants.  As discussed in
Remark~\ref{rem:constancy-obstruction}, the obstruction to upgrading
this to full direction constancy lies in the large-scale behavior of the
$\eps$-regularity argument: the unweighted direction energy on
$\{\rho<\eta\}$ is not controlled by the weighted coherence
bound~(\ref{it:coherence}), and the rescaled drift grows with the
parabolic scale.

Several approaches to closing this gap appear worthy of investigation:
\begin{enumerate}[label=\textup{(\roman*)}, leftmargin=2em, itemsep=4pt]
\item Frequency-localized refinements of the coherence bound that
control $\nabla\xi$ in Morrey or Campanato norms, bypassing the need for
pointwise positivity of $\rho$.
\item A Liouville theorem for the direction equation on the ancient
element, exploiting the unbounded time interval $(-\infty,0]$ and the
parabolic dissipation to propagate small-scale constancy to all scales.
\item Critical-forcing $\eps$-regularity in the
$\mathcal{C}^{3/2}$ Carleson regime for geometric PDEs of the
form~\eqref{eq:direction-eqn}, extending the subcritical theory of
Lemma~\ref{lem:struwe-perturbed} to the borderline case.
\end{enumerate}

\subsection{The rigid rotation conjecture}

Conditional on direction constancy (Theorem~\ref{thm:conditional}), the
blow-up profile is identified as the rigid rotation
$u_{\mathrm{rig}}=\tfrac12(-x_2,x_1,0)$.  This flow has linearly
growing velocity ($|u_{\mathrm{rig}}|=|x_h|/2$), infinite kinetic
energy, and constant vorticity $\omega\equiv e_3$.  It is a smooth,
stationary, ancient solution of the two-dimensional Navier--Stokes
equations (viewed as a three-dimensional flow independent of~$x_3$).

The rigid rotation is \emph{not} excluded by any of the existing
Liouville or regularity theorems:
\begin{itemize}[leftmargin=2em, itemsep=4pt]
\item The KNSS Liouville theorem~\cite{KNSS2009} requires bounded
velocity.
\item The ESS backward uniqueness result~\cite{ESS2003} requires $L^3$
integrability.
\item Type~I blow-up classifications do not apply, as the rigid rotation
is consistent with both Type~I and Type~II blow-up rates.
\end{itemize}

\begin{conjecture}[Exclusion of the rigid rotation]\label{conj:rigid}
For any smooth, divergence-free $u_0\in H^1(\R^3)$ and any running-max
sequence as in Definition~\textup{\ref{def:running-max}}, the ancient
element~$u^\infty$ cannot be the rigid rotation.
\end{conjecture}

Several avenues toward resolving this conjecture appear natural:
\begin{enumerate}[label=\textup{(\roman*)}, leftmargin=2em, itemsep=4pt]
\item Backward uniqueness or Carleman-type estimates at the blow-up
time, exploiting the rigidity of the profile.
\item Quantitative energy-growth transfer: the pre-blow-up solution has
finite energy, while the rigid rotation has infinite energy, and this
mismatch may force quantitative obstructions in the rescaling.
\item Topological constraints on the vorticity direction map near
blow-up, related to the degree theory for maps
$\Sbb^2\to\Sbb^2$.
\item Concentration analysis in the scale-invariant norm $L^{3/2}$ for
the vorticity, exploiting the criticality of this norm under the
Navier--Stokes scaling.
\end{enumerate}


%% ====================================================================
\section*{Acknowledgments}
%% ====================================================================

The author is grateful to the mathematical community for continued
interest in the Navier--Stokes regularity problem, and acknowledges
helpful conversations with colleagues during the preparation of this
work.


%% ====================================================================
%% BIBLIOGRAPHY
%% ====================================================================

\bibliographystyle{amsplain}
\begin{thebibliography}{99}

\bibitem{Aubin1963}
J.-P.\ Aubin,
\emph{Un th\'eor\`eme de compacit\'e},
C.\ R.\ Acad.\ Sci.\ Paris \textbf{256} (1963), 5042--5044.

\bibitem{BKM1984}
J.~T.\ Beale, T.\ Kato, and A.\ Majda,
\emph{Remarks on the breakdown of smooth solutions for the 3-D Euler
equations},
Comm.\ Math.\ Phys.\ \textbf{94} (1984), no.~1, 61--66.
\textsc{doi}:~10.1007/BF01212349.

\bibitem{CKN1982}
L.\ Caffarelli, R.\ Kohn, and L.\ Nirenberg,
\emph{Partial regularity of suitable weak solutions of the Navier--Stokes
equations},
Comm.\ Pure Appl.\ Math.\ \textbf{35} (1982), no.~6, 771--831.
\textsc{doi}:~10.1002/cpa.3160350604.

\bibitem{ConstantinFefferman1993}
P.\ Constantin and C.\ Fefferman,
\emph{Direction of vorticity and the problem of global regularity for
the Navier--Stokes equations},
Indiana Univ.\ Math.\ J.\ \textbf{42} (1993), no.~3, 775--789.

\bibitem{CRW1976}
R.~R.\ Coifman, R.\ Rochberg, and G.\ Weiss,
\emph{Factorization theorems for Hardy spaces in several variables},
Ann.\ of Math.\ (2) \textbf{103} (1976), no.~3, 611--635.
\textsc{doi}:~10.2307/1970954.

\bibitem{ESS2003}
L.\ Escauriaza, G.~A.\ Seregin, and V.\ \v{S}ver\'{a}k,
\emph{$L_{3,\infty}$-solutions of the Navier--Stokes equations and
backward uniqueness},
Russian Math.\ Surveys \textbf{58} (2003), no.~2, 211--250.
\textsc{doi}:~10.1070/RM2003v058n02ABEH000609.

\bibitem{Fefferman2006}
C.~L.\ Fefferman,
\emph{Existence and smoothness of the Navier--Stokes equation},
in \emph{The Millennium Prize Problems},
Clay Math.\ Inst., Amer.\ Math.\ Soc., Providence, RI, 2006,
pp.~57--67.

\bibitem{GKP2016}
I.\ Gallagher, G.~S.\ Koch, and F.\ Planchon,
\emph{Blow-up of critical Besov norms at a potential Navier--Stokes
singularity},
Comm.\ Math.\ Phys.\ \textbf{343} (2016), no.~1, 39--82.
\textsc{doi}:~10.1007/s00220-016-2593-z.

\bibitem{KNSS2009}
G.\ Koch, N.\ Nadirashvili, G.\ Seregin, and V.\ \v{S}ver\'{a}k,
\emph{Liouville theorems for the Navier--Stokes equations and
applications},
Acta Math.\ \textbf{203} (2009), no.~1, 83--105.
\textsc{doi}:~10.1007/s11511-009-0039-6.

\bibitem{Leray1934}
J.\ Leray,
\emph{Sur le mouvement d'un liquide visqueux emplissant l'espace},
Acta Math.\ \textbf{63} (1934), no.~1, 193--248.
\textsc{doi}:~10.1007/BF02547354.

\bibitem{Lieberman}
G.~M.\ Lieberman,
\emph{Second Order Parabolic Differential Equations},
World Scientific, Singapore, 1996.
\textsc{doi}:~10.1142/3302.

\bibitem{LinWang1998}
F.\ Lin and C.\ Wang,
\emph{Energy identity of harmonic map flows from surfaces at finite
singular time},
Calc.\ Var.\ Partial Differential Equations \textbf{6} (1998), no.~4,
369--380.
\textsc{doi}:~10.1007/s005260050097.

\bibitem{Lions1969}
J.-L.\ Lions,
\emph{Quelques m\'ethodes de r\'esolution des probl\`emes aux limites
non lin\'eaires},
Dunod; Gauthier-Villars, Paris, 1969.

\bibitem{Seregin2012}
G.\ Seregin,
\emph{A certain necessary condition of potential blow up for
Navier--Stokes equations},
Comm.\ Math.\ Phys.\ \textbf{312} (2012), no.~3, 833--845.
\textsc{doi}:~10.1007/s00220-012-1484-6.

\bibitem{Serrin1962}
J.\ Serrin,
\emph{On the interior regularity of weak solutions of the Navier--Stokes
equations},
Arch.\ Rational Mech.\ Anal.\ \textbf{9} (1962), 187--195.
\textsc{doi}:~10.1007/BF00253344.

\bibitem{Struwe1988}
M.\ Struwe,
\emph{On the evolution of harmonic maps in higher dimensions},
J.\ Differential Geom.\ \textbf{28} (1988), no.~3, 485--502.

\end{thebibliography}

\end{document}
