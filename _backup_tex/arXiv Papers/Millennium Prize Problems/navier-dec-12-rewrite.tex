\documentclass[12pt, reqno]{amsart}

%% PACKAGES
\usepackage{amsmath, amssymb, amsthm, amsfonts}
\usepackage{mathrsfs}
\usepackage{mathtools}
\usepackage{enumerate}
\usepackage{geometry}
\usepackage{color}
\usepackage{url}

%% GEOMETRY
\geometry{margin=1.in}

\usepackage[colorlinks=true, linkcolor=blue, citecolor=blue, urlcolor=blue]{hyperref}
\setcounter{tocdepth}{2}

%% THEOREMS
\newtheorem{theorem}{Theorem}[section]
\newtheorem{lemma}[theorem]{Lemma}
\newtheorem{proposition}[theorem]{Proposition}
\newtheorem{corollary}[theorem]{Corollary}
\newtheorem{conjecture}[theorem]{Conjecture}
{\color{magenta}\newtheorem{assumption}[theorem]{Assumption}}

\theoremstyle{definition}
\newtheorem{definition}[theorem]{Definition}
\newtheorem{remark}[theorem]{Remark}
\newtheorem{example}[theorem]{Example}

%% NUMBERING
\numberwithin{equation}{section}

%% MACROS
\newcommand{\R}{\mathbb{R}}
\newcommand{\N}{\mathbb{N}}
\newcommand{\C}{\mathbb{C}}
\newcommand{\Z}{\mathbb{Z}}
\newcommand{\T}{\mathbb{T}}
\newcommand{\Sbb}{\mathbb{S}}

\newcommand{\dv}{\mathrm{div}}
\newcommand{\curl}{\mathrm{curl}}
\newcommand{\supp}{\mathrm{supp}}
\newcommand{\osc}{\mathrm{osc}}
\newcommand{\BMO}{\mathrm{BMO}}
\newcommand{\VMO}{\mathrm{VMO}}

\newcommand{\eps}{\varepsilon}
\newcommand{\om}{\omega}
\newcommand{\Om}{\Omega}
\newcommand{\xihat}{\hat{\xi}}
\newcommand{\lambdar}{\Lambda_r}
\usepackage{xcolor}


%% TITLE & AUTHOR
%\title[Global Regularity for Navier--Stokes]{Global Regularity for the 3D Incompressible Navier--Stokes Equations via Geometric Depletion}
\title[Geometric Depletion Mechanisms]{ Geometric Depletion Mechanisms in the 3D Incompressible Navier--Stokes Equations}

\author{Jonathan Washburn}
\address{Department of Mathematics} 
\email{jonathan.washburn@example.com} % Placeholder email

%\date{\today}

%% ABSTRACT
\begin{document}

\begin{abstract}
{\color{magenta}\noindent\textbf{Audit status.} This document currently contains a \emph{conditional} reduction of 3D Navier--Stokes global regularity to several scale-critical inputs about an ancient blow-up profile (here: the running-max/vorticity-normalized ancient element), isolated explicitly as inputs (C)--(E) in Subsection~\ref{subsec:conditional-inputs} (with (B) automatic under running-max normalization). If those inputs are established unconditionally, the argument would yield global regularity.}

{\color{magenta}Assuming the remaining inputs (C),(D),(E) in Subsection~\ref{subsec:conditional-inputs} (with (B) automatic under running-max normalization), we show that smooth, finite-energy solutions to the 3D incompressible Navier--Stokes equations on $\mathbb{R}^3$ exist globally in time.}
The proof proceeds by contradiction, analyzing the geometry of a hypothetical finite-time singularity. We introduce the method of \emph{geometric depletion}, which reduces the analysis to the evolution of the vorticity direction field $\xi = \omega/|\omega|$. 

First, we establish a critical coercivity estimate for the singular integral stretching term, showing that the nonlinear stretching is depleted in the presence of small directional oscillation. Second, we reduce the contradiction step to a Liouville-type rigidity statement for the resulting critical drift--diffusion equation satisfied by $\xi$; under a small Carleson-measure forcing hypothesis, this forces $\xi$ to have constant direction field. This reduction forces the flow to be structurally two-dimensional, for which global regularity is known, thereby contradicting the existence of a singularity.
\end{abstract}

\maketitle

\tableofcontents

\section{Introduction}

{\color{blue}
\subsection{Motivation} The question of global regularity for the 3D incompressible Navier–Stokes equations remains one of the central open problems in mathematical fluid dynamics. Understanding whether finite–time singularities may arise from smooth initial data is crucial both for the analytical structure of the equations and for the predictive reliability of the physical models they describe. The system governs the motion of a viscous, incompressible fluid with constant density and follows from the conservation of linear momentum and mass. The foundational mathematical theory was established by J. Leray~\cite{Leray1934} and E. Hopf~\cite{Hopf1951}, who introduced the notion of weak solutions and established global existence via the fundamental energy inequality. However, the questions of uniqueness and regularity for such weak solutions remain unresolved.

\smallskip



The incompressible Navier--Stokes equations arise from the fundamental principles of 
mass and momentum conservation applied to a viscous fluid treated as a continuum. 
Under the continuum hypothesis, the velocity $u(t,x)$ and pressure $p(t,x)$ are 
well-defined, smoothly varying fields describing, respectively, the instantaneous 
velocity of a fluid parcel and the normal force exerted by the surrounding fluid. 
The condition $\nabla \cdot u = 0$ reflects conservation of mass for a homogeneous, 
incompressible fluid, while the momentum equation expresses Newton’s second law, i.e.
the material acceleration $\frac{D u}{Dt} = \partial_t u + (u \cdot \nabla)u$ is 
balanced by the pressure gradient $-\nabla p$, the viscous diffusion term $\nu \Delta u$ 
arising from internal friction in a Newtonian fluid, and possible external forces $f$. 

In 3D, 
taking the curl of the momentum equation yields the vorticity formulation, in which 
the term $(\omega \cdot \nabla)u$ (with $\omega=\nabla\times u$) describes vortex 
stretching, a mechanism which does not exist in two dimensions and widely regarded as the key 
process responsible for vorticity amplification, energy cascade to smaller scales, 
and the potential formation of singularities. This vortex-stretching mechanism 
encapsulates the central mathematical difficulty of the Navier--Stokes problem, 
at the same time, the essential physical ingredient underlying the onset of 
turbulence in real viscous flows ~\cite{ConstantinFefferman1993,MajdaBertozzi2002}.

{\color{red} Organization of the paper!!!}
}

{\color{blue}
\subsection{The Navier--Stokes Regularity Problem}



Let $T>0$ be an arbitrary finite number representing the time, and $\nu>0$ a positive number representing the kinematic viscosity.  We consider 3D incompressible Navier--Stokes (N-S) equations given by the following system of PDEs:
\begin{equation}\label{eq:NS_domain}
\begin{cases}
\partial_t u + (u \cdot \nabla)u + \nabla p - \nu \Delta u = f,  \\
\nabla \cdot u = 0,
\end{cases}
\end{equation}
where the vector field $u: \R^3 \times [0,T) \to \R^3$ denotes the velocity, and 
$p: \R^3 \times [0,T) \to \R$ denotes the scalar pressure.  

We assume that the external force $f = 0$, but all results can be easily extended to the case of a non-vanishing external force by incorporating $f$ through the Duhamel integral \cite[Proposition~6.1]{Lemarie2016}, under the standard admissibility assumptions on $f$ (e.g. $f \in L^1_{loc}([0,T);L^2(\mathbb{R}^3))$).

We assume the initial data $u(x,0)=u_0(x) \in H^1(\R^3)$ is smooth and divergence-free.
Given such smooth initial data, the fundamental question,  identified as one of the Millennium Prize Problems \cite{Fefferman2006}, is whether such solutions remain smooth for all time $T > 0$, or whether a finite-time singularity can form.



The modern theory of weak solutions to the N--S equations originates 
from the works of J.~Leray~\cite{Leray1934} and E.~Hopf~\cite{Hopf1951}. 
They introduced the notion of what is now called a Leray--Hopf weak solution and 
proved the global-in-time existence of such solutions for any divergence-free 
initial data $u_0 \in L^2(\mathbb{R}^3)$. These solutions satisfy the N--S 
equations in the distributional sense together with the fundamental global energy 
inequality
\begin{equation}\label{eq:energy}
\frac{1}{2} \int_{\mathbb{R}^3} |u(x,t)|^2 \, dx
+ \nu \int_0^t \int_{\mathbb{R}^3} |\nabla u(x,s)|^2 \, dx \, ds
\le \frac{1}{2} \int_{\mathbb{R}^3} |u_0(x)|^2 \, dx
\qquad \forall\, t \ge 0.
\end{equation}
Although global existence is guaranteed, the questions of uniqueness and 
spatial--temporal regularity of Leray--Hopf weak solutions remain open. 
This difficulty is tied to the \emph{supercritical} nature of the nonlinearity 
$(u\cdot\nabla)u$ with respect to the natural dissipation $\nu\Delta u$ under the 
N--S scaling, and motivates the development of refined regularity 
criteria and the introduction of the stronger class of suitable weak solutions.







The N--S equations (\ref{eq:NS_domain}) are invariant under the scaling
\begin{equation}\label{scaling}
    u_\lambda(x,t) = \lambda\, u(\lambda x, \lambda^2 t), 
\qquad
p_\lambda(x,t) = \lambda^2\, p(\lambda x, \lambda^2 t),
\end{equation}
but this transformation maps the energy norm 
$\|u\|_{L^\infty_t L^2_x}$ to $\lambda^{-1/2}\|u\|_{L^\infty_t L^2_x}$, 
making the energy strictly supercritical (too weak to control the nonlinearity).



The underlying physical space is tacitly assumed to be flat, which is the natural assumption for the study of the flow in our
3D Euclidean space. 

M. Kobayashi \cite{Kobayashi} extends the Navier–Stokes equations from flat spaces to manifolds by analyzing the motion of a Newtonian fluid on flow leaves, that is, smooth surfaces in Euclidean three-space that are invariant under the fluid flow. The proposed general equations describing the motion of a Newtonian fluid 
with constant properties on a volume Riemannian manifold $(M,g,\omega)$ are:

\begin{enumerate}
    \item Continuity equation:
    \[
        \operatorname{div}_{\omega} u = 0,
    \]

    \item {N--S equation:}
    \[
        \frac{\partial u}{\partial t} 
        + \nabla_u u
        = -\frac{1}{\rho}\,\operatorname{grad} p
        - \nu\left( \nabla^\ast \nabla u + \mathrm{Ric}(u)
        - \mathcal{L}_{\operatorname{grad}\log \omega} u \right)
        + b.
    \]
\end{enumerate}
Here $\operatorname{div}_{\omega}$ and $\operatorname{grad}$ denote divergence and gradient 
taken with respect to the volume form $\omega$, 
$\nabla^\ast \nabla$ is the Hodge--de\,Rham Laplacian on vector fields, 
$\mathrm{Ric}(u)$ is the Ricci curvature acting on $u$, 
and $\mathcal{L}_{\operatorname{grad}\!\log \omega}\,u$ represents the non--Riemannian 
correction arising from the volume form. It is shown how quantities intrinsic to the manifold, such as curvature and the choice of volume form, fundamentally modify the structure of the equations and the resulting flow behavior.
}


\subsection{Historical Context and Barriers}
Substantial progress has been made in understanding the partial regularity of suitable weak solutions. Scheffer \cite{Scheffer1977} and Caffarelli, Kohn, and Nirenberg \cite{CKN1982} proved that the singular set of any suitable weak solution has one-dimensional parabolic Hausdorff measure zero. Lin \cite{Lin1998} simplified and refined these results. These partial regularity theorems rely on $\varepsilon$-regularity criteria: if scale-invariant quantities (such as $\|u\|_{L^3}$ or $\|u\|_{L^\infty_t L^{3,\infty}_x}$) are locally small, the solution is regular.

Complementing the partial regularity theory are blow-up criteria. The celebrated Beale--Kato--Majda (BKM) criterion \cite{BKM1984} states that a smooth solution blows up at time $T^*$ if and only if
\begin{equation}\label{eq:BKM}
\int_0^{T^*} \|\omega(\cdot,t)\|_{L^\infty} \, dt = \infty,
\end{equation}
where $\omega = \curl \, u$ is the vorticity. Serrin \cite{Serrin1962} and Prodi \cite{Prodi1959} established that if $u \in L^q(0,T; L^p(\R^3))$ with $2/q + 3/p \le 1$ ($p > 3$), then the solution is regular. The endpoint case $L^\infty_t L^3_x$ was resolved by Escauriaza, Seregin, and \v{S}ver\'ak \cite{ESS2003}.

Despite these advances, the "scaling gap" remains. All known regularity criteria require bounds at the critical scaling level (e.g., $L^3$ velocity or $L^{3/2}$ vorticity), whereas the a priori energy bounds control only subcritical quantities (e.g., $L^2$ velocity). Bridging this gap requires exploiting the structure of the nonlinearity beyond simple scaling arguments.

{\color{blue}\subsection{Main Result}
We provide a new geometric decomposition of the nonlinear structure 
that targets the scaling gap and separates the controllable 
geometric terms from the critical singular interaction. This leads to a refined geometric regularity criterion for the 
3D N–S equations, aligned with the critical scaling 
and sensitive to vorticity–direction oscillation.

\begin{theorem}[Conditional Main Theorem (running-max version)]\label{thm:main}
{\color{magenta}
Let $u_0 \in H^1(\R^3)$ be smooth and divergence-free, and let $u$ be the corresponding smooth solution of \eqref{eq:NS_domain} on its maximal interval of existence $[0,T^*)$.
Assume that whenever $T^*<\infty$, the associated running-max/vorticity-normalized ancient element $(u^\infty,p^\infty)$ produced by Lemma~\ref{lem:ancient-limit-runningmax} satisfies the inputs (C),(D),(E) stated in Subsection~\ref{subsec:conditional-inputs}.
(The scale-critical vorticity control (B) is automatic for this normalization; see Lemma~\ref{lem:omega32-runningmax-automatic}.)
Then $T^*=\infty$ and $u$ extends to a unique global smooth solution on $[0,\infty)$.
}
\end{theorem}

{\color{magenta}\begin{remark}[Audit note]
The remainder of this manuscript can be read as a proof of Theorem~\ref{thm:main} \emph{assuming} the remaining inputs (C),(D),(E) in Subsection~\ref{subsec:conditional-inputs} (with (B) automatic under running-max normalization); the deep-audit annotations mark every location where an unproved/non-classical step is currently used.
\end{remark}}

{\color{magenta}\subsection{Conditional inputs (C)--(E) and gap map}\label{subsec:conditional-inputs}
For clarity and referee-checkability, we isolate the \emph{non-classical} steps currently required to turn the manuscript into an unconditional proof.
In this rewrite, the contradiction object is the running-max/vorticity-normalized ancient element extracted from the blow-up sequence (Lemma~\ref{lem:ancient-limit-runningmax}). Under this normalization, the scale-critical vorticity control (B) holds automatically and is recorded below as Lemma~\ref{lem:omega32-runningmax-automatic}.
All later ``gap'' annotations refer back to the items below.

{\color{magenta}\begin{remark}[On (A) in the running-max refactor]
In the original CKN-tangent-flow architecture, a VMO/BMO-smallness hypothesis on $\xi^\infty$ is a natural way to force commutator depletion of the near-field oscillation term.
In the \emph{running-max} rewrite, the ancient element satisfies $\|\omega^\infty\|_{L^\infty}\le 2$ (Lemma~\ref{lem:ancient-limit-runningmax}(iii)), and this bounded-vorticity input already makes the near-field commutator/oscillation term Carleson-small at small scales (Lemma~\ref{lem:nearfield-osc-carleson}).
Accordingly, for the purposes of item (D) below, the near-field commutator/oscillation term does not require any VMO/BMO-smallness input on $\xi^\infty$.
Accordingly, we do \emph{not} treat spatial VMO of $\xi^\infty$ as a separate required hypothesis in this running-max proof architecture.
If a later step truly requires quantitative small oscillation of $\xi^\infty$ at small scales (beyond what follows from bounded vorticity), that requirement will be stated explicitly as part of the forcing input (D) or as a separate hypothesis at the point of use.%
\end{remark}}

{\color{magenta}\begin{example}[Why ``$\xi$ is VMO'' does \emph{not} follow even from smoothness and bounded vorticity]\label{ex:vmo-fails-at-zeros}
The vorticity direction field can fail to have vanishing mean oscillation near points where $\omega=0$, even when $\omega$ is smooth and bounded.
For instance, fix a smooth cutoff $\chi\in C_c^\infty(\R^3)$ with $\chi\equiv 1$ on $B_1(0)$ and define a smooth compactly supported vorticity field
\[
\omega(x):=\chi(x)\,(x_1,x_2,0).
\]
Let $u:=\curl(-\Delta)^{-1}\omega$ be the corresponding smooth divergence-free velocity (Biot--Savart).
On $B_1(0)\setminus\{x_1=x_2=0\}$ one has
\[
\xi(x)=\frac{\omega(x)}{|\omega(x)|}=\frac{(x_1,x_2,0)}{\sqrt{x_1^2+x_2^2}},
\]
so $\xi$ winds once around the circle on each sphere centered at $0$.
In particular, for every $0<r<1$ the average of $\xi$ over $B_r(0)$ vanishes by symmetry, and hence
\[
\frac{1}{|B_r|}\int_{B_r(0)}|\xi(x)-(\xi)_{0,r}|\,dx
=\frac{1}{|B_r|}\int_{B_r(0)}|\xi(x)|\,dx
=1,
\]
so the mean oscillation does \emph{not} tend to $0$ as $r\downarrow0$.
Thus $\xi\notin\VMO$ at $0$ despite $\omega\in C^\infty_c\cap L^\infty$.

\medskip
\noindent\textbf{Related obstruction (critical direction energy).}
In the same example one has $|\nabla\xi(x)|\sim (x_1^2+x_2^2)^{-1/2}$ near the vorticity-zero axis $\{x_1=x_2=0\}$, so
\[
\int_{B_r(0)}|\nabla\xi(x)|^2\,dx=\infty\qquad\text{for every }r>0.
\]
Thus even \emph{finiteness} (let alone smallness) of the unweighted critical direction energy $E(z_0,r)=r^{-3}\iint_{Q_r(z_0)}|\nabla\xi|^2$ is not automatic from smoothness and boundedness of $\omega$ unless one imposes additional structure near $\{\rho=0\}$.

\medskip
\noindent\textbf{Conclusion.}
A ``directional VMO'' statement must either exclude the vorticity-zero set, or be formulated in a weighted/thresholded way (e.g.\ VMO on $\{\rho>\lambda\}$ uniformly in $\lambda$, or smallness of a \emph{weighted} oscillation such as $\rho^{3/2}|\xi-(\xi)_{B_r}|$).
This is one reason we do not treat spatial VMO of $\xi^\infty$ as a standalone unconditional input in the running-max refactor.%
\end{example}}

\begin{lemma}[Scale-critical vorticity control (B), automatic under running-max normalization]\label{lem:omega32-runningmax-automatic}
Let $(u^\infty,p^\infty)$ be the running-max/vorticity-normalized ancient element produced by Lemma~\ref{lem:ancient-limit-runningmax}. Then there exists $K_0<\infty$ such that
\[
\sup_{z_0\in\R^3\times(-\infty,0]}\ \sup_{0<r\le1}\ r^{-2}\iint_{Q_r(z_0)} |\omega^\infty|^{3/2}\,dx\,dt \;\le\; K_0.
\]
\end{lemma}

\begin{proof}
This follows directly from Lemma~\ref{lem:omega32-runningmax} (applied to the running-max rescaling sequence). Equivalently, by Lemma~\ref{lem:ancient-limit-runningmax}(iii) one has $\|\omega^\infty\|_{L^\infty(\R^3\times(-\infty,0])}\le 2$, and hence for any $z_0$ and $0<r\le 1$,
\[
r^{-2}\iint_{Q_r(z_0)} |\omega^\infty|^{3/2}\,dx\,dt
\le r^{-2}\,\|\omega^\infty\|_{L^\infty(Q_r(z_0))}^{3/2}\,|Q_r|
\le C\,2^{3/2},
\]
where $|Q_r|\le C r^5$ for $r\le 1$.
\end{proof}

\begin{assumption}[Directional $\varepsilon$-regularity and Liouville (C)]\label{assump:C-liouville}
Let $\xi$ be an ancient solution of the sphere-valued drift--diffusion equation
\[
\partial_t \xi - \Delta \xi + u\cdot\nabla\xi = |\nabla\xi|^2\xi + H,\qquad |\xi|=1,\qquad H\cdot\xi=0,
\]
on $\R^3\times(-\infty,0]$, where the divergence-free drift $u$ belongs to an admissible Serrin class sufficient to close the Caccioppoli/Campanato scheme (as used in Theorem~\ref{thm:DDE-eps-regularity}).
{\color{magenta}In the running-max rewrite, the ancient element satisfies $\omega^\infty\in L^\infty$, and Lemma~\ref{lem:drift-local-Lp} yields an admissible \emph{local Serrin} drift bound after a Galilean gauge (take $q=\infty$, $p>3$).
Moreover, Lemma~\ref{lem:drift-small-rescaled} shows that after this gauge the \emph{rescaled} drift becomes perturbatively small on $Q_1$ as the physical scale $r\downarrow 0$.
Thus the drift hypothesis is not an additional missing input in this refactor; the remaining missing step is to supply a fully referee-checkable critical $\varepsilon$-regularity theorem in the drift/Carleson forcing setting and to verify the \emph{global} critical-energy smallness hypothesis (e.g.\ $\sup_{z_0,r}E(z_0,r)\le \eps_*^2$) needed to invoke the Liouville mechanism of Theorem~\ref{thm:liouville}.}
and where the tangential forcing is small in the Carleson norm:
\[
\|H\|_{C^{3/2}}\le \delta_*.
\]
Assume moreover that the critical direction energy is globally small on all scales:
\[
\sup_{z_0\in\R^3\times(-\infty,0]}\ \sup_{r>0}\ E(z_0,r)\ \le\ \eps_*^2,
\qquad
E(z_0,r):=r^{-3}\iint_{Q_r(z_0)}|\nabla\xi|^2.
\]
Then $\xi$ is constant in space-time (equivalently, $\nabla\xi\equiv0$ and $\partial_t\xi\equiv0$).
\end{assumption}

\begin{assumption}[Total tangential forcing smallness (D)]\label{assump:D-forcing}
For the running-max/vorticity-normalized ancient element produced by Lemma~\ref{lem:ancient-limit-runningmax}, the tangential forcing
$H=H_{\mathrm{sing}}+H_{\mathrm{geom}}$ in \eqref{eq:direction} is small in the critical Carleson norm at sufficiently small scales.
Concretely, there exists a universal $\delta_*>0$ such that for this ancient element there exists $r_0>0$ with
\[
\sup_{z_0}\ \sup_{0<r\le r_0}\ r^{-2}\iint_{Q_r(z_0)} |H|^{3/2}\,dx\,dt \;\le\; \delta_*^{3/2}.
\]

{\color{magenta}\noindent\textbf{[AI AUDIT / non-circularity warning.]}
Lemma~\ref{lem:hgeom-carleson-from-energy} shows that $H_{\mathrm{geom}}$ is Carleson-small \emph{provided} one already has a uniform scale-invariant bound on $\|\nabla\log\rho\|_{L^2}$ \emph{and} the \emph{global} small direction-energy hypothesis $\sup_{z_0,r}E(z_0,r)\le \eps_*^2$ from (C).
In the present manuscript, Lemma~\ref{lem:log_amplitude} provides only a uniform bound on $\nabla\log(\rho+\varepsilon)$; controlling the $\varepsilon\downarrow 0$ limit across $\{\rho=0\}$ is separated out as Assumption~\ref{assump:D-logamp}.
This is useful for bookkeeping (it shows that, after the Liouville small-energy gate is known, the geometric forcing is automatically negligible at small scales), but it is \emph{not} by itself a proof of (D), because (D) is used to justify applying the $\varepsilon$-regularity/Liouville mechanism in (C).}
\end{assumption}
\textit{Remark.} In this running-max rewrite, the bounded vorticity input $\|\omega^\infty\|_{L^\infty}\le 2$ implies that the \emph{near-field commutator/oscillation term} is Carleson-small at small scales (Lemma~\ref{lem:nearfield-osc-carleson}), and the ``constant-direction remainder'' of the near-field term is also Carleson-small (Remark~\ref{rem:constdir-easy-Linfty}).
Thus, within item (D), the remaining genuine obstruction is \emph{tail smallness} (boundedness is easy but smallness as $r\to0$ is borderline).
For the geometric coupling term $H_{\mathrm{geom}}=2P_\xi((\nabla\log\rho)\cdot\nabla\xi)$, Lemma~\ref{lem:log_amplitude} provides a classical (regularized) log-amplitude Caccioppoli estimate on each cylinder for the running-max ancient element (using bounded vorticity $\Rightarrow$ local smoothness, Lemma~\ref{lem:Linfty-vort-smooth}).
What is \emph{not} automatic is a uniform control of $\nabla\log\rho$ across vorticity zeros (Lemma~\ref{lem:log_amplitude} controls only $\nabla\log(\rho+\varepsilon)$), and we isolate this as a separate quantitative input (Assumption~\ref{assump:D-logamp}).
The current text therefore does not yet provide complete proofs of the remaining implications at the level required for an unconditional proof (in particular: tail smallness, and the log-amplitude/$\varepsilon\downarrow0$ issue encoded in Assumption~\ref{assump:D-logamp}).

{\color{blue}
\begin{lemma}[ODE constraint on the linear mode of $u_3$]\label{lem:linear-mode-ODE}
Assume that for each $t\le 0$ the velocity field $u(\cdot,t)$ of a smooth Navier--Stokes solution has the structure
\[
u_1, u_2 \;\text{independent of } x_3,\qquad u_3(x,t)=a(t)+b(t)\,x_3,
\]
where $a(t),b(t)$ are smooth functions of time. Then:
\begin{enumerate}
\item[(i)] The momentum equation for $u_3$ implies
\[
\dot b + b^2 = 0.
\]
\item[(ii)] The general solution to part (i) is $b(t)=\frac{b_0}{1+b_0 t}$ with $b_0:=b(0)$.
\item[(iii)] For an \emph{ancient} solution (defined on $(-\infty,0]$):
\begin{itemize}
\item If $b_0>0$, the formula $b(t)=\frac{b_0}{1+b_0 t}$ has a singularity at $t=-1/b_0<0$, hence $b_0>0$ is \emph{not allowed};
\item If $b_0\le 0$, the formula is well-defined for all $t\le 0$ (and $b(t)\to 0$ as $t\to-\infty$).
\end{itemize}
In particular, if $b(0)=0$ then $b(t)\equiv 0$ for all $t$.
\end{enumerate}
\end{lemma}

\begin{proof}
\textbf{(i)} With $u_3=a+bx_3$ and $u_h$ independent of $x_3$, the third component of Navier--Stokes reads
\[
\partial_t u_3 + u\cdot\nabla u_3 - \nu \Delta u_3 + \partial_3 p = 0.
\]
Since $\partial_1 u_3=\partial_2 u_3=\partial_{33}u_3=0$, one has $\Delta u_3=0$.
Moreover $u\cdot\nabla u_3 = u_3\,\partial_3 u_3 = (a+bx_3)\,b$.
Therefore
\[
a'(t)+b'(t)\,x_3 + a(t)b(t)+b(t)^2\,x_3 + \partial_3 p = 0,
\]
so $\partial_3 p$ is affine in $x_3$ and in particular
\[
\partial_{33}p(\cdot,t)=-(b'(t)+b(t)^2).
\]
On the other hand, the pressure Poisson equation for incompressible Navier--Stokes,
\[
\Delta p = -\sum_{i,j=1}^3 \partial_i u_j\,\partial_j u_i,
\]
has a right-hand side that is \emph{independent of $x_3$} under the present structural assumptions (all spatial derivatives of $u$ are independent of $x_3$ because $u_h$ is $x_3$-independent and $u_3$ is affine in $x_3$).
Hence $\Delta p$ is independent of $x_3$, which forces $\partial_{33}p$ to be independent of $x_3$ as well.
Comparing with the explicit formula above yields the ODE
\[
b'(t)+b(t)^2=0,
\]
as claimed.

\textbf{(ii)} Separating variables in $\dot b = -b^2$ gives $\int b^{-2}db = -\int dt$, hence $-1/b = -t + C$, i.e.\ $b=\frac{1}{t-C}$. Solving $b(0)=b_0$ gives $C=-1/b_0$, hence $b(t)=\frac{b_0}{1+b_0 t}$.

\textbf{(iii)} The singularity occurs when $1+b_0 t=0$, i.e.\ $t=-1/b_0$. If $b_0>0$, then $-1/b_0<0$, so the solution blows up before $t=0$, ruling out ancient solutions with $b_0>0$.
\end{proof}
}

{\color{blue}\begin{lemma}[Bounded vorticity rules out $b(0)<0$ in the constant-direction regime]\label{lem:E1-b-negative-impossible}
Let $(u,p)$ be a smooth ancient Navier--Stokes solution on $\R^3\times(-\infty,0]$ with constant vorticity direction $\xi\equiv e_3$, so that $\omega=(0,0,\rho)$ with $\rho\ge 0$.
Assume the bound $\|\omega\|_{L^\infty(\R^3\times(-\infty,0])}\le M$ and nontriviality $\rho(0,0)>0$.
If $u_3(x,t)=a(t)+b(t)\,x_3$ with $b$ satisfying $\dot b+b^2=0$ (Lemma~\ref{lem:linear-mode-ODE}), then $b(0)$ cannot be negative.
Consequently, for a constant-direction running-max ancient element one has $b(0)=0$, hence $b(t)\equiv 0$ for all $t\le 0$.
\end{lemma}

\begin{proof}
Write $b_0:=b(0)$. If $b_0>0$, Lemma~\ref{lem:linear-mode-ODE}(iii) rules it out for an ancient solution.
Assume for contradiction that $b_0<0$. Then $b(t)=\frac{b_0}{1+b_0 t}<0$ for all $t\le 0$, and
\[
B(t):=\int_0^t b(s)\,ds=\log(1+b_0 t), \qquad t\le 0.
\]
Since $\xi\equiv e_3$, the third component of the vorticity equation gives the scalar reaction--advection--diffusion equation
\begin{equation}\label{eq:rho-reaction}
\partial_t \rho - \nu\Delta \rho + u\cdot\nabla \rho = b(t)\,\rho
\qquad\text{on }\R^3\times(-\infty,0].
\end{equation}
Define the renormalized amplitude $\widetilde\rho(x,t):=e^{-B(t)}\rho(x,t)$. Because $b$ depends only on $t$, a direct computation using \eqref{eq:rho-reaction} yields
\[
\partial_t \widetilde\rho - \nu\Delta \widetilde\rho + u\cdot\nabla \widetilde\rho = 0.
\]
By the parabolic maximum principle for bounded solutions on $\R^3$, the quantity $\|\widetilde\rho(\cdot,t)\|_{L^\infty(\R^3)}$ is non-increasing forward in time.
Hence for any $t<0$,
\[
\|\rho(\cdot,0)\|_{L^\infty}
=\|\widetilde\rho(\cdot,0)\|_{L^\infty}
\le \|\widetilde\rho(\cdot,t)\|_{L^\infty}
=e^{-B(t)}\|\rho(\cdot,t)\|_{L^\infty}
\le \frac{M}{1+b_0 t}.
\]
Letting $t\to -\infty$ makes the right-hand side tend to $0$, forcing $\|\rho(\cdot,0)\|_{L^\infty}=0$, which contradicts $\rho(0,0)>0$.
Thus $b_0<0$ is impossible.
\end{proof}}

\begin{assumption}[2D classification / Liouville class (E)]\label{assump:E-2d}
Whenever the running-max ancient element $(u^\infty,p^\infty)$ produced by Lemma~\ref{lem:ancient-limit-runningmax} satisfies $\xi^\infty\equiv b_0$ for a constant unit vector $b_0$, the resulting constant-direction ancient flow satisfies the following Liouville-class conditions:
\begin{enumerate}
  \item[(E1)] (\emph{Linear-mode vanishing for $u_3$.})
  After rotating so $b_0=e_3$, the constant-direction structure and Remark~\ref{rem:constdir-uc} imply that $u^\infty_3$ depends only on $x_3$ and $t$.
  Harmonicity $\Delta u^\infty_3=0$ forces $u^\infty_3(x_3,t)=a(t)+b(t)\,x_3$; Lemma~\ref{lem:linear-mode-ODE} shows that $b(0)>0$ is impossible for an ancient solution.
  Lemma~\ref{lem:E1-b-negative-impossible} rules out $b(0)<0$ using the running-max vorticity bound, hence $b(0)=0$ and $u^\infty_3(\cdot,t)$ is spatially constant for each $t\le0$.
  After subtracting a constant Galilean drift one may assume $u^\infty_3\equiv 0$.
  {\color{magenta}\noindent\textbf{[Update.]}
  In the running-max refactor, (E1) is no longer an additional hypothesis: it follows from bounded vorticity once $\xi^\infty$ is constant.}
  \item[(E2)] (\emph{2D ancient Liouville class for the reduced flow.})
  The reduced horizontal velocity field $v=(u^\infty_1,u^\infty_2)$ (which is then a 2D ancient Navier--Stokes solution on $\R^2\times(-\infty,0]$)
  belongs to a classical 2D ancient Liouville class (e.g.\ bounded, or sublinear growth at infinity, or another standard hypothesis sufficient to invoke a known 2D Liouville theorem),
  and therefore $v\equiv 0$, hence $u^\infty\equiv 0$.
\end{enumerate}
\end{assumption}

{\color{magenta}\begin{remark}[What remains open for (E2) in the running-max refactor]\label{rem:E2-open}
By Lemma~\ref{lem:E1-b-negative-impossible}, once $\xi^\infty$ is constant the running-max ancient element satisfies $u^\infty_3\equiv 0$ and is independent of $x_3$, so it reduces to an ancient 2D Navier--Stokes solution $v(x_h,t)$ with scalar vorticity $\rho(x_h,t)$.

The remaining (E) input is therefore \emph{purely global}: one must place $v$ (or $\rho$) in a known 2D Liouville class.
Typical sufficient hypotheses include:
\begin{itemize}
\item bounded velocity $v\in L^\infty(\R^2\times(-\infty,0])$ (as in \cite{KNSS2009});
\item finite global enstrophy $\rho\in L^\infty_tL^2_x(\R^2\times(-\infty,0])$ (then the 2D enstrophy identity forces $\rho$ to be spatially constant);
\item decay of vorticity at spatial infinity (so that Biot--Savart yields bounded or sublinear-growth velocity).
\end{itemize}

{\color{magenta}\noindent\textbf{[AI AUDIT / key gap.]}
Lemma~\ref{lem:ancient-limit-runningmax} provides only \emph{local} compactness and does not, as written, supply any of the global hypotheses above.
Closing (E2) unconditionally therefore requires either:
(i) a new argument transferring a global Liouville-class property from the pre-blow-up solution to the running-max ancient element, or
(ii) a replacement of the final contradiction route (e.g.\ a ``small-data gate'' argument) that avoids needing global 2D classification.}
\end{remark}}

{\color{magenta}\begin{remark}[Linear-mode vanishing (E1) is now automatic]\label{rem:E1-ode-constraint}
In the running-max refactor, once $\xi^\infty$ is known to be constant (from (C)), the linear mode of $u_3$ vanishes automatically:
\begin{itemize}
\item Lemma~\ref{lem:linear-mode-ODE} rules out $b(0)>0$ for an ancient solution.
\item Lemma~\ref{lem:E1-b-negative-impossible} uses the global running-max bound $\|\omega^\infty\|_{L^\infty}\le 2$ to rule out $b(0)<0$ by a maximum-principle argument for the scalar amplitude equation.
\end{itemize}
Therefore $b(0)=0$ and hence $b(t)\equiv 0$ for all $t\le 0$.
\end{remark}}

{\color{blue}\begin{remark}[Biot--Savart structure implies linear growth when $b\neq0$]\label{rem:E1-biotsavart}
For constant-direction vorticity $\omega=(0,0,\rho(x_h))$ with $\rho$ independent of $x_3$, the 3D Biot--Savart formula gives
\[
u_{\mathrm{BS}}(x)=\frac{1}{4\pi}\int_{\R^3}\frac{(x-y)\times\omega(y)}{|x-y|^3}\,dy.
\]
A direct calculation shows $(u_{\mathrm{BS}})_3=0$: the cross product $(x-y)\times(0,0,\rho)$ has zero third component.
Thus \emph{any} nonzero $u_3$ arises from a harmonic, divergence-free correction: $u=u_{\mathrm{BS}}+\nabla\phi$ with $\Delta\phi=0$.

If $u_3=a(t)+b(t)\,x_3$ with $b\neq0$, then $\partial_3\phi=a+b\,x_3$ forces
\[
\phi(x,t)=a(t)\,x_3+\tfrac12\,b(t)\,x_3^2+f(x_h,t),
\qquad \Delta_h f=-b(t).
\]
On $\R^2$, the Poisson equation $\Delta_h f=-b$ has the particular solution $f_{\mathrm{part}}=-\frac{b}{4}|x_h|^2$.
Hence $\phi\sim\tfrac{b}{4}|x_h|^2$ as $|x_h|\to\infty$, and correspondingly
\[
u_h=\nabla_h\phi\;\sim\;\tfrac{b}{2}\,x_h
\qquad\text{as }|x_h|\to\infty,
\]
i.e.\ the horizontal velocity grows \emph{linearly} at spatial infinity.

\smallskip
\noindent\textbf{Why this might force $b=0$ (historical note).}
The observation above shows that any nonzero linear mode $b\neq0$ forces $u_h$ to grow linearly at spatial infinity.
One possible route to rule this out would be to transfer some \emph{global} decay/energy information through the blow-up limit.
In the running-max refactor this is not needed to close (E1): Lemma~\ref{lem:E1-b-negative-impossible} already forces $b(0)=0$ once $\xi^\infty$ is constant.
\end{remark}}

{\color{blue}\begin{lemma}[Local energy growth when $b\neq0$]\label{lem:E1-energy-growth}
In the constant-direction setting with $\omega=(0,0,\rho)$ and $u_3=a(t)+b(t)x_3$ where $b(t)\neq0$, the local $L^2$ energy on a ball $B_R$ satisfies
\[
\int_{B_R}|u(\cdot,t)|^2\,dx\ \gtrsim\ |b(t)|^2\,R^5
\qquad\text{as }R\to\infty.
\]
In particular, the local energy grows like $R^5$, which is faster than $R^3$ (characteristic of bounded velocity fields).
\end{lemma}

\begin{proof}
From Remark~\ref{rem:E1-biotsavart}, the horizontal velocity decomposes as $u_h=(u_{\mathrm{BS}})_h+\nabla_h\phi$ where $\nabla_h\phi\sim\frac{b}{2}x_h$ at large $|x_h|$.
The Biot--Savart component $(u_{\mathrm{BS}})_h$ is bounded (since $\rho\in L^\infty$ and is independent of $x_3$, and the 2D Biot--Savart kernel decays at infinity for such vorticity).
Thus the dominant contribution to $|u_h|^2$ at large $|x_h|$ is $|\nabla_h\phi|^2\sim\frac{b^2}{4}|x_h|^2$.

Integrating over $B_R$:
\[
\int_{B_R}|u_h|^2\,dx\ \gtrsim\ \frac{b^2}{4}\int_{B_R}|x_h|^2\,dx\ \sim\ b^2\,R^2\cdot R^3\ =\ b^2\,R^5.
\]
The vertical component $u_3=a+bx_3$ contributes $\int_{B_R}|u_3|^2\sim a^2 R^3+b^2 R^5$, which is also dominated by $b^2 R^5$ at large $R$.
\end{proof}}

{\color{magenta}\begin{remark}[Energy growth and the running-max limit]\label{rem:E1-energy-vs-compactness}
Lemma~\ref{lem:E1-energy-growth} shows that if $b\neq0$, the local energy grows like $R^5$. This is \emph{not} directly inconsistent with the running-max blow-up compactness, because:
\begin{itemize}
\item The rescaled solutions $u^{(k)}$ converge locally (not globally);
\item The constant $C(R)$ in Lemma~\ref{lem:ancient-limit-runningmax}(i) is allowed to depend on $R$, so $C(R)\sim R^5$ growth is not excluded by the compactness argument alone.
\end{itemize}
To rule out $b\neq0$, one would need either:
\begin{enumerate}
\item A \emph{global} energy bound on the ancient element (not currently available from local blow-up compactness), or
\item An \emph{a priori} constraint on the growth rate of $C(R)$ inherited from specific properties of the pre-blow-up solution.
\end{enumerate}
Neither is currently established. However, in the running-max refactor these are not needed to close (E1): Lemma~\ref{lem:E1-b-negative-impossible} rules out $b(0)<0$ and Lemma~\ref{lem:linear-mode-ODE} rules out $b(0)>0$, hence $b\equiv 0$.
\end{remark}}

{\color{blue}\begin{remark}[Backward-in-time asymptotics when $b_0<0$ (now excluded)]\label{rem:E1-backward-asymptotics}
Lemma~\ref{lem:E1-b-negative-impossible} shows that $b_0<0$ cannot occur for a running-max constant-direction ancient element, because it would contradict the global vorticity bound via the maximum principle.
We keep only the ODE observation (Lemma~\ref{lem:linear-mode-ODE}) as an aside: if one ever works in a blow-up architecture without a uniform $L^\infty$ bound on $\omega$, then $b(t)\to 0$ as $t\to-\infty$ might suggest a backward-asymptotic route.
\end{remark}}

\begin{remark}[Gap map / where the remaining inputs are used]
\begin{itemize}
    \item \textbf{(B)} is not an additional hypothesis in this rewrite: the running-max normalization yields bounded vorticity and hence the needed scale-critical \(L^{3/2}\) vorticity control automatically (Lemma~\ref{lem:omega32-runningmax-automatic}).
    \item \textbf{(C)} is the rigidity step replacing the current non-referee-checkable DDE $\varepsilon$-regularity/Liouville argument (cf.\ Theorem~\ref{thm:DDE-eps-regularity} and Theorem~\ref{thm:liouville}).
    \item \textbf{(D)} is used to guarantee the small-forcing hypothesis needed to apply the directional rigidity step \textbf{(C)} (i.e.\ $\|H\|_{C^{3/2}}\le \delta_*$ for the running-max ancient element at sufficiently small scales).
    In this refactor, the near-field pieces of $H_{\mathrm{sing}}$ are Carleson-small automatically from bounded vorticity, while the remaining forcing obstructions are (i) the far-field/tail term (Assumption~\ref{assump:tail-depletion}) and (ii) the vorticity-zero-set / $\varepsilon\downarrow0$ issue for the log-amplitude (Assumption~\ref{assump:D-logamp}).
    \item \textbf{(E)} is used in the reduction-to-2D and 2D Liouville step (Theorem~\ref{thm:2d_liouville}).
\end{itemize}
\end{remark}
} % end conditional-inputs subsection

\subsection{Constants and Thresholds}\label{subsec:constants}
Throughout, we use universal dimensional constants $C,c>0$ whose value may change from line to line. We introduce the following scale-invariant quantities and thresholds:
\begin{itemize}
    \item The {\it scale-invariant energy} of the direction field $\xi$ on a cylinder $Q_r(z_0)$:
    \[
    E(z_0,r) := r^{-3} \iint_{Q_r(z_0)} |\nabla \xi|^2 \, dx \, dt.
    \]
    \item The {\it critical Carleson norm} of the tangential forcing $H$ in the direction equation at scales $\le r_*$:
    \[
    \|H\|_{C^{3/2}(r_*)} := \sup_{z_0}\ \sup_{0<r\le r_*} r^{-2} \iint_{Q_r(z_0)} |H|^{3/2} \, dx \, dt,
    \qquad (0<r_*\le 1).
    \]
    When $r_*=1$ we write $\|H\|_{C^{3/2}}:=\|H\|_{C^{3/2}(1)}$.
    \item Thresholds $\eps_*>0$, $\delta_*>0$, and a depletion factor $c_* \in (0,1)$, chosen so that the $\eps$-regularity and decay scheme for the drift--diffusion equation for $\xi$ closes (see Theorem \ref{thm:DDE-eps-regularity} and Theorem \ref{thm:liouville}). These thresholds are universal and depend only on Calder\'on--Zygmund constants and whatever quantitative drift bound is assumed in the rigidity input (Assumption~\ref{assump:C-liouville}).
\end{itemize}
{\color{magenta}\noindent\textbf{[AI AUDIT.]}
In the running-max refactor, the ancient element satisfies $\omega^\infty\in L^\infty$, and Lemma~\ref{lem:drift-local-Lp} implies an admissible \emph{local} Serrin drift bound after a Galilean gauge on each cylinder.
What is \emph{not} yet proved is the full critical $\varepsilon$-regularity/Liouville rigidity package for the sphere-valued drift--diffusion equation with this drift/forcing, as well as any global growth/Liouville-class information needed for the final 2D reduction.}
We record that all objects above are invariant under the N--S scaling $x\mapsto \lambda x$, $t\mapsto \lambda^2 t$.}

\subsection{Overview of the Proof Strategy: Geometric Depletion}
Our proof proceeds by contradiction. We assume a finite-time singularity exists and perform a blow-up analysis to extract a nontrivial ancient blow-up profile (here, the running-max/vorticity-normalized ancient element) defined on $\R^3 \times (-\infty, 0]$. This ancient element inherits critical scale-invariant bounds from the blow-up sequence. {\color{magenta}\noindent\textbf{[AI AUDIT.]}
The blow-up/compactness argument in Lemma~\ref{lem:ancient-limit-runningmax} is currently only a proof sketch for the needed uniform cylinder bounds; even granting local compactness/suitability, it does not (as written) provide the strong global/scale-uniform ``critical bounds'' later invoked (e.g. Serrin bounds or uniform VMO/BMO controls). Those additional bounds must be proved or explicitly assumed.}
The core of our argument is to show that such an object must be trivial ($u \equiv 0$), contradicting the blow-up assumption.

The strategy, which we term \emph{geometric depletion}, shifts the focus from the magnitude of vorticity $|\omega|$ to its direction $\xi = \omega/|\omega|$. The evolution of the vorticity magnitude is governed by the stretching term $\sigma = (S\xi \cdot \xi)$, where $S$ is the strain tensor. A singularity requires persistent, strong stretching. However, the direction field $\xi$ satisfies a critical drift--diffusion equation constrained to the sphere $\Sbb^2$:
\begin{equation}\label{eq:direction_intro}
{\color{magenta}\partial_t \xi - \Delta \xi + u \cdot \nabla \xi = |\nabla \xi|^2 \xi + H, \quad |\xi|=1,\quad H\cdot \xi = 0,}
\end{equation}
where $H$ is a forcing term derived from the N--S nonlinearity.

The proof rests on two key innovations that exploit the tension between the "roughness" required for stretching and the "structure" enforced by the direction equation:

\begin{enumerate}
    \item \textbf{Critical Coercivity (Problem 1):} We prove that the stretching term $\sigma$, viewed as a singular integral operator acting on $\omega$, is \emph{depleted} in the near-field if the direction field $\xi$ has small oscillation. Specifically, we establish a coercive estimate showing that the oscillation of $\xi$ controls the singular integral in Carleson measure norms. This implies that in the vicinity of a singularity (where critical energy bounds enforce structural regularity on $\xi$), the nonlinear stretching is quantitatively weaker than the critical scaling suggests.

    \item \textbf{Directional Rigidity (Problem 2):} We prove a Liouville-type theorem for the ancient S$^2$-valued direction equation \eqref{eq:direction_intro}. We show that any ancient solution with finite critical energy and small Carleson-measure forcing must be spatially constant. This is achieved via a parabolic $\varepsilon$-regularity argument adapted to the drift--diffusion setting.
\end{enumerate}

The logic chain concludes as follows: If a singularity occurs, we extract an ancient blow-up profile (here, the running-max/vorticity-normalized ancient element). In this refactor, the bounded-vorticity property of the running-max element already yields depletion of the \emph{near-field} singular forcing at small scales (both the commutator/oscillation term and the constant-direction remainder).
Assuming one can also control the remaining \emph{tail} and \emph{geometric} forcing in the critical Carleson norm, the Directional Rigidity input (C) forces $\xi$ to be a constant vector. A N--S flow with constant vorticity direction is structurally two-dimensional. By known Liouville theorems for 2D ancient solutions (under appropriate global hypotheses), such a flow must vanish. This implies the singularity was spurious.
{\color{magenta}\noindent\textbf{[AI AUDIT.]}
As written, multiple arrows in this chain still require additional hypotheses currently isolated as inputs (C),(D),(E) in Subsection~\ref{subsec:conditional-inputs} (and further technical conditions flagged later).}

\section{Preliminaries and Notation}
{\color{blue}
\subsection{Functional Spaces and Scaling}
We work in Euclidean space $\R^3$. For a point $z_0 = (x_0, t_0) \in \R^3 \times \R$ 
and a radius $r>0$, we define the backward parabolic cylinder
\[
Q_r(z_0) = B_r(x_0) \times (t_0 - r^2,\, t_0),
\]
where $B_r(x_0)$ denotes the open ball of radius $r$ centered at $x_0$. We use standard Lebesgue spaces $L^p(\R^3)$ and parabolic spaces $L^q(0,T; L^p(\R^3))$. 

The vorticity field, defined as $\omega = \nabla \times u$, plays a central role in the analysis. The N--S equations are invariant under the scaling
\begin{equation}\label{scaling2}
u_\lambda(x,t) = \lambda u(\lambda x, \lambda^2 t), \quad p_\lambda(x,t) = \lambda^2 p(\lambda x, \lambda^2 t).
\end{equation}

Under the scaling, the vorticity transforms as $\omega_\lambda(x,t) = \lambda^2 \omega(\lambda x, \lambda^2 t)$. A norm or functional is called \emph{critical} if it is invariant under this transformation.  One of the most important critical norms for the velocity field is 
the scale-invariant quantity $\|u\|_{L^\infty_t L^3_x}$. 
 

 

The Ladyzhenskaya--Prodi--Serrin criterion provides a sufficient condition for global existence: if a smooth solution $u$ belongs to the mixed Lebesgue space
$$u \in L^q(0, T;L^p(\mathbb{R}^3)) \quad \text{such that} \quad \frac{2}{q} + \frac{3}{p} \le 1 \quad \text{for} \quad p \ge 3,$$
then $u$ can be extended after $t = T$, see for example \cite{15,25,27}. A critical advance was the resolution of the endpoint case (where $p=3$), specifically $u \in L^\infty(0, T;L^3(\mathbb{R}^3))$. This result implies the non-existence of self-similar type singularities \cite{23}.

In order to bridge these global criteria with the local analysis of weak solutions, we recall the standard notions of weak and suitable weak solutions.



 




\begin{definition}[Weak Solution]\label{def:weak-solution}
Let $u:Q \to \mathbb{R}^3$ be a measurable function. 
We say that $u$ is a \emph{weak solution} of the N--S equations \textup{(1.1)} 
in the space--time cylinder $Q = \Omega \times (a,b)$ if
\begin{equation}\label{eq:LerayHopfSpaces}
u \in L^\infty\!\left(a,b; L^2(\Omega;\mathbb{R}^3)\right)
\;\cap\;
L^2\!\left(a,b; W^{1,2}(\Omega;\mathbb{R}^3)\right),
\end{equation}
the equation $\operatorname{div} u = 0$ holds in the sense of distributions, and
for all test functions 
\[
\varphi \in C_c^1\!\left((a,b); C_{c,\sigma}^\infty(\Omega;\mathbb{R}^3)\right)
\]
the identity
\begin{equation}\label{eq:weak-formulation}
-\!\!\iint_{Q} u \cdot \partial_t \varphi \, dx\,dt
+ \iint_{Q} \nabla u : \nabla \varphi \, dx\,dt
- \iint_{Q} (u \otimes u) : \nabla \varphi \, dx\,dt = 0
\end{equation}
holds.
\end{definition}

These solutions exist globally in time and possess the global energy inequality in terms of the initial kinetic energy. 
Such solutions are commonly referred to as \emph{Leray--Hopf weak solutions}.

\smallskip

When studying local and partial regularity of the N--S equations, 
a stronger notion of solution is typically used, the class of 
\emph{suitable weak solutions}. Following Scheffer \cite{Scheffer1977} and Caffarelli, Kohn, and Nirenberg \cite{CKN1982}, we work with the class of suitable weak solutions.  
Here we present a version due to Galdi \cite{6}.

\begin{definition}[Suitable Weak Solution]\label{def:suitable}
Let $u:Q \to \mathbb{R}^3$ and $p:Q \to \mathbb{R}$ be measurable.  
The pair $(u,p)$ is called a \emph{suitable weak solution} of the N--S 
equations \textup{(1.1)} in the cylinder $Q = \Omega \times (a,b)$ if:
\begin{align}
u &\in 
L^\infty\!\left(a,b; L^2(\Omega;\mathbb{R}^3)\right)
\;\cap\;
L^2\!\left(a,b; W^{1,2}(\Omega;\mathbb{R}^3)\right), 
\label{eq:suitable-u}
\\[4pt]
p &\in L^{3/2}(Q), 
\label{eq:suitable-p}
\end{align}
the system \textup{(1.1)} is satisfied in the sense of distributions, and the following
\emph{generalized local energy inequality} holds:

For almost every $t \in (a,b)$ and every non-negative test function 
$\phi \in C_c^\infty(Q)$,
\begin{equation}\label{eq:local-energy-ineq}
\begin{aligned}
\int_{\Omega} |u(t)|^2 \phi(t) \, dx
+ 2 \int_{a}^{t} \!\!\int_{\Omega} |\nabla u|^2 \phi \, dx\,ds
\;\le\;
\int_{a}^{t} \!\!\int_{\Omega} 
u^2 (\partial_t \phi + \Delta \phi)
\, dx\,ds 
\\
+ \int_{a}^{t} \!\!\int_{\Omega} \bigl(|u|^2 + 2p\bigr)\, u \cdot \nabla \phi \, dx\,ds .
\end{aligned}
\end{equation}
\end{definition}





While the Ladyzhenskaya–Prodi–Serrin and endpoint criteria provide global regularity conditions, the local counterpart is given by the $\varepsilon$-regularity theory of Caffarelli–Kohn–Nire\-nberg. 

Standard $\varepsilon$-regularity theory \cite{CKN1982, Lin1998} shows that
smallness of certain scale-invariant quantities on a parabolic cylinder forces
regularity. A fundamental example is the Caffarelli--Kohn--Nirenberg
criterion, based on the dimensionless functional
\[
F(r) := r^{-2}\!\iint_{Q_r(z_0)} \big(|u|^{3} + |p|^{3/2}\big)\,dx\,dt .
\]
There exists a universal constant $\varepsilon_{CKN} > 0$ such that if
\[
F(r) < \varepsilon_{CKN},
\]
then $u$ is bounded (and in fact Hölder continuous) on $Q_{r/2}(z_0)$.
This type of estimate constitutes the first prototype of local
regularity criteria for suitable weak solutions.}


\subsection{Blow-up Analysis and Construction of the Running-Max Ancient Element}


{\color{blue} Assume, for contradiction, that the smooth solution develops a finite-time singularity at
$T^* < \infty$. By the Beale–Kato–Majda criterion we know that the vorticity must blow up, so
\[
\limsup_{t \uparrow T^*} \|\omega(\cdot,t)\|_{L^\infty} = \infty.
\]
In order to understand how such a singularity could appear, we rescale the solution near the
points and times where the vorticity is very large, and in this way we obtain a limiting
blow-up profile.



\begin{theorem} [Beale--Kato--Majda (BKM), Euler, \cite{BKM1984}]
Let $u$ be a solution of the incompressible Euler equations ({\color{magenta}i.e.\ \eqref{eq:NS_domain} with $\nu=0$ and $f=0$}), and
suppose there is a time $T^*$ such that the solution cannot be continued in the class $u \in C([0,T]; H^s) \,\cap\, C^1([0,T]; H^{s-1}), \, s \geq 3.$
to $T = T^*$. Assume that $T^*$ is the first such time.
Then
\[
\int_0^{T^*} \|\omega(t)\|_{L^\infty}\, dt = +\infty,
\]
and in particular
\[
\limsup_{t \uparrow T^*} \|\omega(t)\|_{L^\infty} = +\infty.
\]
\end{theorem}


%Lecture notes for Math 256B, Version 2024
%Lenya Ryzhik May 7, 2024
\begin{theorem}[BKM, N-S]\label{thm:BKM-NS}
Let $u_0 \in C^\infty_c(\mathbb{R}^3)$, so that there exists a classical 
solution $u$ to the N-S equations ({\color{magenta}i.e.\ \eqref{eq:NS_domain} with $f=0$ and viscosity $\nu>0$}). 
If for any $T>0$ we have
\begin{equation}\label{eq:BKM-NS-1}
\int_0^T \|\omega(t)\|_{L^\infty}\, dt < +\infty,
\end{equation}
then the smooth solution $u$ exists globally in time.  
If the maximal existence time of the smooth solution is $T < +\infty$, 
then necessarily
\begin{equation}\label{eq:BKM-NS-2}
{\color{magenta}\int_0^{T} \|\omega(s)\|_{L^\infty}\, ds = +\infty.}
\end{equation}
\end{theorem}

\begin{remark}
For the Euler equations the BKM criterion is an equivalence: 
finite--time blow-up occurs if and only if 
$\int_0^{T^*}\|\omega(t)\|_{L^\infty}\,dt=+\infty$. 
For the N-S equations one only has the one--sided continuation 
criterion stated above; the converse implication is not known, nor does it 
hold for weak solutions or suitable weak solutions. 
\end{remark}

The $\varepsilon$--regularity theorem (see Caffarelli--Kohn--Nirenberg \cite{CKN1982})
implies that if no singular point existed at a possible blow\mbox{-}up time $T^{*}$, 
then the solution would remain uniformly bounded in a parabolic neighbourhood of 
the hyperplane $\{t = T^{*}\}$. Combined with the local energy inequality, this 
allows us to extend the solution smoothly past $T^{*}$, contradicting the assumption
that $T^{*}$ is the first blow-up time. F. Lin \cite{Lin1998} later
gave a different proof of this result via a blow-up argument which was expanded upon
and extended by Ladyzhenskaya–Seregin \cite{LG}. The following lemma is a direct consequence of the $\varepsilon$--regularity theory of
Caffarelli–Kohn–Nirenberg (CKN) \cite{CKN1982}.




\begin{remark}[Optional: CKN singular points (not used in the running-max route)]
The running-max/vorticity-normalized construction of the ancient element (Lemmas~\ref{lem:blowup-normalization}--\ref{lem:ancient-limit-runningmax}) does not require a CKN-singular point.
We record the following standard CKN singular-point lemma only to motivate the classical CKN-anchored tangent-flow construction included later for comparison.
\end{remark}

\begin{lemma}\label{lem:singular-point}
Assume that $u$ is a smooth solution of the N-S (\ref{eq:NS_domain}) equations
on $[0,T^*)$ and that $T^*<\infty$ is the first blow-up time.
Then there exists at least one point $x^*\in\R^3$ such that $(x^*,T^*)$ is a singular
point in the sense of CKN.
\end{lemma}


\begin{proof}
Suppose, that no such point exists. Then every $(x,T^*)$ is regular
in the CKN sense. Hence, for each $x\in\R^3$ there exists $r_x>0$ such that 
\[
F(z_0,r) = r^{-2} \iint_{Q_r(z_0)} \bigl(|u|^3 + |p|^{3/2}\bigr)\,dx\,dt
\]
satisfies $F((x,T^*),r_x) < \varepsilon_{\mathrm{CKN}}$.
By the $\varepsilon$-regularity theorem \cite{CKN1982,Lin1998}, this implies that
$u$ is bounded in a smaller parabolic cylinder, there exist constants
$M_x<\infty$ such that
\[
|u(y,s)| \le M_x \quad \text{for all } (y,s)\in
Q_{r_x/2}(x,T^*) = B_{r_x/2}(x)\times(T^*-(r_x/2)^2,T^*].
\]



There exist $R>0$ and consider the compact set $\overline{B_R(0)}\times\{T^*\}$.
Since the balls $B_{r_x/2}(x)$, $x\in\overline{B_R(0)}$, form an open cover of
$\overline{B_R(0)}$, we can extract a finite subcover
\[
\overline{B_R(0)} \subset \bigcup_{i=1}^N B_{r_i/2}(x_i).
\]
Let us define
\[
\delta_R := \min_{1\le i\le N} \frac{r_i^2}{4} > 0,
\qquad
M_R := \max_{1\le i\le N} M_{x_i} < \infty.
\]
Let $(y,s)$ be any point with $|y|\le R$ and $s\in(T^*-\delta_R,T^*]$.
Then there exists $i\in\{1,\dots,N\}$ such that $y\in B_{r_i/2}(x_i)$.
Moreover,  we have
\[
s > T^*-\delta_R \ge T^* - \frac{r_i^2}{4},
\]
so $(y,s)\in Q_{r_i/2}(x_i,T^*)$. Therefore
\[
|u(y,s)| \le M_{x_i} \le M_R.
\]
In other words,
\[
\sup_{|y|\le R,\; s \in (T^*-\delta_R,T^*]} |u(y,s)| \le M_R < \infty.
\]

Thus $u$ is uniformly bounded on $B_R(0)\times(T^*-\delta_R,T^*]$.
Standard local well-posedness and continuation for smooth solutions imply that
$u$ can be smoothly extended beyond $T^*$ on $B_R(0)$.

Since $R>0$ is arbitrary, this shows that $u$ extends smoothly beyond $T^*$ on
all of $\R^3$, contradicting the maximality of $T^*$. Therefore, there exist at least one singular point $(x^*,T^*)$ in the CKN sense.
\end{proof}











\begin{lemma}\label{lem:blowup-normalization}
Let $u_0 \in C_c^\infty(\mathbb{R}^3)$ be divergence-free, and let
$u$ be the unique smooth solution of the N-S equations (\ref{eq:NS_domain})
on its maximal interval of smooth existence $[0,T^*)$. Assume that $T^* < \infty$ is the
first blow-up time.

Then there exist times $t_k \uparrow T^*$, points $x_k \in \mathbb{R}^3$, and scales
$\lambda_k \downarrow 0$ (for instance, $\lambda_k = |\omega(x_k,t_k)|^{-1/2}$) such that,
defining the rescaled velocity fields
\begin{equation}\label{rescaled}
u^{(k)}(y,s)
:=
\lambda_k\, u\!\left(x_k + \lambda_k y,\; t_k + \lambda_k^2 s\right),
\qquad
\;p^{(k)}(y,s)
:=
\lambda_k^2\, p\!\left(x_k + \lambda_k y,\; t_k + \lambda_k^2 s\right),
\qquad
\omega^{(k)} := \curl\, u^{(k)},
\end{equation}
we have the normalization
\[
|\omega^{(k)}(0,0)| = 1 \quad \text{for all } k.
\]
\end{lemma}

\begin{proof}
By the BKM continuation criterion, loss of smoothness at $T^*$ implies that
\[
\limsup_{t \uparrow T^*} \|\omega(\cdot,t)\|_{L^\infty} = +\infty.
\]
Hence we can choose a sequence of times $t_k \uparrow T^*$ such that
\[
M_k := \|\omega(\cdot,t_k)\|_{L^\infty} \to \infty
\quad \text{as } k \to \infty.
\]
{\color{magenta}\noindent\textbf{[AI AUDIT / OPTIONAL STRENGTHENING (running-max times).]}
One may choose the times $t_k$ so that $\|\omega(\cdot,t)\|_{L^\infty}\le \|\omega(\cdot,t_k)\|_{L^\infty}=M_k$ for all $t\le t_k$
(e.g.\ take $t_k$ to be the first hitting time of a level $L_k\uparrow\infty$). This yields uniform backward-in-time $L^\infty$ control for the rescaled vorticities (see below).}
For each $k$, since $\omega(\cdot,t_k)$ is continuous and not identically zero, there exists
a point $x_k \in \mathbb{R}^3$ such that
\[
|\omega(x_k,t_k)| \ge \tfrac{1}{2} M_k.
\]
Let us set $
A_k := |\omega(x_k,t_k)|$, then $A_k \ge \tfrac{1}{2} M_k$, and in particular $A_k \to \infty$ as $k \to \infty$.
Let us define the scaling factors
\[
\lambda_k := A_k^{-1/2}.
\]
Using the rescaling (\ref{rescaled}), by the scaling of the vorticity (\ref{scaling}), we have
\[
\omega^{(k)}(0,0)
= \lambda_k^2\, \omega(x_k,t_k)
= \lambda_k^2 A_k
= 1.
\]
Since $A_k \to \infty$, it follows that $\lambda_k \downarrow 0$.
{\color{magenta}\noindent\textbf{[AI AUDIT / CONSEQUENCE.]}
If the ``running-max'' choice of $t_k$ is made, then for every $s\le 0$ one has $t_k+\lambda_k^2 s\le t_k$ and hence
$\|\omega(\cdot,t_k+\lambda_k^2 s)\|_{L^\infty}\le M_k$.
By scaling and $A_k\ge \tfrac12 M_k$ this gives the uniform bound
\[
\|\omega^{(k)}(\cdot,s)\|_{L^\infty}\le \frac{M_k}{A_k}\le 2
\qquad\text{for all }s\le 0.
\]
In particular, any ancient limit extracted from such a sequence satisfies the scale-critical bound in Lemma~\ref{lem:omega32-runningmax}.}
\end{proof}

\begin{lemma}[Running-max vorticity normalization implies a critical \(L^{3/2}\) bound]\label{lem:omega32-runningmax}
Assume the times $t_k\uparrow T^*$ in Lemma~\ref{lem:blowup-normalization} are chosen as \emph{running maxima} for the vorticity:
\[
\|\omega(\cdot,t)\|_{L^\infty}\le \|\omega(\cdot,t_k)\|_{L^\infty}\qquad\text{for all }t\le t_k.
\]
Then the rescaled vorticities $\omega^{(k)}=\curl u^{(k)}$ satisfy the uniform backward-in-time bound
\[
\|\omega^{(k)}\|_{L^\infty(\R^3\times(-\lambda_k^{-2}t_k,\,0])}\le 2.
\]
In particular, any subsequential weak-$\ast$ limit $\omega^\infty$ of $\omega^{(k)}$ in $L^\infty_{\mathrm{loc}}(\R^3\times(-\infty,0])$
obeys the scale-critical estimate
\[
\sup_{z_0\in\R^3\times(-\infty,0]}\ \sup_{0<r\le1}\ r^{-2}\iint_{Q_r(z_0)} |\omega^\infty|^{3/2}\,dx\,dt
\ \le\ C\,2^{3/2},
\]
where $C>0$ is a universal dimensional constant depending only on the definition of $Q_r$.
\end{lemma}

\begin{proof}
The $L^\infty$ bound on $\omega^{(k)}$ is proved in the audit note at the end of Lemma~\ref{lem:blowup-normalization}.
Passing to a subsequence, we may assume $\omega^{(k)}\rightharpoonup^\ast \omega^\infty$ weak-$\ast$ in $L^\infty_{\mathrm{loc}}$ and hence
$\|\omega^\infty\|_{L^\infty_{\mathrm{loc}}}\le 2$.
Therefore for any $z_0$ and $0<r\le 1$,
\[
r^{-2}\iint_{Q_r(z_0)} |\omega^\infty|^{3/2}\,dx\,dt
\ \le\ r^{-2}\,\|\omega^\infty\|_{L^\infty(Q_r(z_0))}^{3/2}\,|Q_r|
\ \le\ r^{-2}\,(2^{3/2})\,|Q_r|
\ \le\ C\,2^{3/2},
\]
since $|Q_r|\le C\,r^5$ for $r\le 1$.
\end{proof}

\begin{lemma}\label{lem:domain-rescaled}
Let $u^{(k)}$ be the rescaled sequence defined in \eqref{rescaled}.
Then each $u^{(k)}$ is defined on a time interval of the form
\[
s \in \bigl(-\lambda_k^{-2} t_k,\;0\bigr],
\]
and these intervals exhaust $(-\infty,0]$. It means that for every $R>0$ there exists
$k_0(R)$ such that
\[
(-R^2,0] \subset \bigl(-\lambda_k^{-2} t_k,\;0\bigr]
\quad\text{for all } k \ge k_0(R).
\]
\end{lemma}

\begin{proof} 
The original solution $u$ is defined for $0 \le t < T^*$. Since $u^{(k)}$ be the rescaled by (\ref{rescaled}), for $u^{(k)}$ to be
well-defined at time $s$, we need
\[
0 \le t_k + \lambda_k^2 s < T^*.
\]
The upper bound $t_k + \lambda_k^2 s \le t_k$ corresponds exactly to $s \le 0$.
The lower bound $t_k + \lambda_k^2 s \ge 0$ gives
\[
s \ge -\lambda_k^{-2} t_k.
\]
Hence $u^{(k)}$ is defined on $s \in (-\lambda_k^{-2} t_k,0]$.

Since $t_k \uparrow T^*$ and $\lambda_k \downarrow 0$, we have
$\lambda_k^{-2} t_k \to \infty$ as $k\to\infty$. Therefore, for any fixed $R>0$,
we can choose $k_0(R)$ such that $\lambda_k^{-2} t_k > R^2$ for all $k\ge k_0(R)$.
Finally, for  $k\ge k_0(R)$, we obtain
\[
(-R^2,0] \subset (-\lambda_k^{-2} t_k,0].,
\]
which proves the lemma.
\end{proof}

{\color{magenta}\noindent\textbf{[AI AUDIT / NOTATION CLARIFICATION.]}
Lemmas~\ref{lem:blowup-normalization}--\ref{lem:domain-rescaled} construct a \emph{vorticity-normalized} rescaling sequence.
In the running-max variant (choose $t_k$ as running maxima for $\|\omega(\cdot,t)\|_{L^\infty}$), one can extract an ancient limit along this sequence; see Lemma~\ref{lem:ancient-limit-runningmax}.
The CKN-anchored tangent flow of Lemma~\ref{lem:ancient-limit} is retained below for comparison, but the main contradiction chain in this rewrite uses Lemma~\ref{lem:ancient-limit-runningmax}.}

\begin{lemma}[Running-max vorticity-normalized ancient element]\label{lem:ancient-limit-runningmax}
Assume the times $t_k\uparrow T^*$ in Lemma~\ref{lem:blowup-normalization} are chosen as \emph{running maxima} for the vorticity (as in Lemma~\ref{lem:omega32-runningmax}), and let $u^{(k)}$ be the corresponding rescaled sequence \eqref{rescaled}.
Then there exists a subsequence (still denoted by $u^{(k)}$) and a pair $(u^\infty,p^\infty)$ such that:
\begin{enumerate}
\item[(i)] For every $R>0$ and $T>0$,
\[
u^{(k)} \to u^\infty \quad\text{strongly in }
L^p(B_R\times(-T,0)) \quad \text{for all } 1\le p<3,
\]
and
\[
u^{(k)} \rightharpoonup u^\infty
\quad \text{weakly in}\quad
L^3_{\mathrm{loc}}(\R^3\times(-\infty,0)).
\]
Moreover,
\[
p^{(k)} \rightharpoonup p^\infty
\quad\text{weakly in } L^{3/2}_{\mathrm{loc}}(\R^3\times(-\infty,0)).
\]
\item[(ii)] The limit $(u^\infty,p^\infty)$ is a suitable weak solution of the N--S equations on $\R^3\times(-\infty,0)$ and satisfies the local energy inequality on every cylinder $B_R\times(-T,0)$.
\item[(iii)] Writing $\omega^\infty=\curl u^\infty$, one has
\[
|\omega^\infty(0,0)|=1,
\qquad
\|\omega^\infty\|_{L^\infty(\R^3\times(-\infty,0])}\le 2.
\]
In particular, $u^\infty\not\equiv 0$.
\end{enumerate}
We call $u^\infty$ the \emph{running-max ancient element} associated to the blow-up at time $T^*$.
\end{lemma}

{\color{blue}
\begin{proof}
\textbf{[Proof sketch / compactness along the running-max sequence.]}
By Lemma~\ref{lem:domain-rescaled}, for each fixed $R>0$ and $T>0$ the rescaled solutions are well-defined and smooth on $B_R\times(-T,0)$ for all $k$ sufficiently large.

\smallskip
\noindent\textbf{Step 1: Uniform local bounds on cylinders (sketch).}
On each fixed cylinder $Q_R:=B_R\times(-R^2,0)$, one may obtain bounds of the form
\[
\sup_{s\in(-R^2,0)}\int_{B_R}|u^{(k)}(x,s)|^2\,dx
\;+\;\int_{Q_R}|\nabla u^{(k)}|^2\,dx\,ds
\;\le\; C(R),
\]
and the companion bounds $\iint_{Q_R}|u^{(k)}|^3\le C(R)$ and $\|p^{(k)}\|_{L^{3/2}(Q_R)}\le C(R)$, with constants independent of $k$.
These can be derived from the local energy inequality together with standard pressure estimates and the running-max vorticity bound $\|\omega^{(k)}\|_{L^\infty}\le 2$ from Lemma~\ref{lem:omega32-runningmax} (see also the discussion around Lemma~\ref{lem:ancient-limit} for the analogous CKN compactness argument).
{\color{magenta}\noindent\textbf{[AI AUDIT.]}
The uniform cylinder bounds for the vorticity-normalized sequence are treated here as a standard compactness input; a fully referee-checkable derivation should be supplied if this lemma is used as the main contradiction object.}

\smallskip
\noindent\textbf{Step 2: Compactness (Aubin--Lions).}
As in the proof of Lemma~\ref{lem:ancient-limit}, the N--S system yields bounds on $\partial_s u^{(k)}$ in a negative Sobolev space on $Q_R$, and Aubin--Lions gives strong convergence in $L^2(Q_R)$ after passing to a subsequence. Interpolation yields strong convergence in $L^p(Q_R)$ for all $1\le p<3$, and a diagonal subsequence yields (i) on $\R^3\times(-\infty,0)$.
Weak compactness of the pressure in $L^{3/2}_{\mathrm{loc}}$ gives the pressure convergence.

\smallskip
\noindent\textbf{Step 3: Passage to the limit; suitability.}
Passing to the limit in distributions yields that $(u^\infty,p^\infty)$ solves N--S on $\R^3\times(-\infty,0)$ and is suitable, with the local energy inequality inherited by lower semicontinuity, proving (ii).

\smallskip
\noindent\textbf{Step 4: Nontriviality and vorticity bound.}
From Lemma~\ref{lem:blowup-normalization} we have $|\omega^{(k)}(0,0)|=1$ for all $k$. The running-max choice of times yields the uniform backward-in-time bound $\|\omega^{(k)}\|_{L^\infty}\le 2$ (Lemma~\ref{lem:omega32-runningmax}), and hence (after taking a weak-$\ast$ limit) $\|\omega^\infty\|_{L^\infty}\le 2$.
Pointwise nontriviality $|\omega^\infty(0,0)|=1$ follows by standard stability of the normalization under the convergence extracted above, yielding (iii).
\end{proof}
} % end blue proof

\begin{remark}[Optional: CKN-anchored tangent flow (not used in the running-max route)]
The main contradiction chain in this manuscript uses the running-max ancient element of Lemma~\ref{lem:ancient-limit-runningmax}.
For completeness and comparison with the classical partial-regularity framework, we record below the standard CKN-anchored tangent-flow construction at a CKN singular point.
\end{remark}

\begin{lemma}\label{lem:ancient-limit}
Let $u_0\in C_c^\infty(\R^3)$ be divergence-free, let $u$ be the corresponding
smooth solution of the N-S equations \eqref{eq:NS_domain} on its
maximal interval of existence $[0,T^*)$, and assume that $T^*<\infty$ is the
first blow-up time. Let $x^*\in\R^3$ be a CKN-singular point at time $T^*$ as in Lemma~\ref{lem:singular-point}.
Let $r_k\downarrow 0$ be any sequence and define the CKN rescalings
\begin{equation}\label{eq:ckn-rescaled}
\tilde u^{(k)}(y,s):=r_k\,u(x^*+r_k y,\;T^*+r_k^2 s),
\qquad
\tilde p^{(k)}(y,s):=r_k^2\,p(x^*+r_k y,\;T^*+r_k^2 s),
\qquad s<0.
\end{equation}

Then there exists a subsequence (still denoted by $\tilde u^{(k)},\tilde p^{(k)}$) 
and a pair $(u^\infty,p^\infty)$ such that:

\begin{enumerate}

\item[(i)] For every $R>0$ and $T>0$,
\[
\tilde u^{(k)} \to u^\infty \quad\text{strongly in } 
L^p(B_R\times(-T,0)) \quad \text{for all } 1\le p<3,
\]
and
\[
\tilde u^{(k)} \rightharpoonup u^\infty 
\quad \text{weakly in}\quad
L^3_{\mathrm{loc}}(\R^3\times(-\infty,0)).
\]
Moreover,
\[
\tilde p^{(k)} \rightharpoonup p^\infty
\quad\text{weakly in } L^{3/2}_{\mathrm{loc}}(\R^3\times(-\infty,0)).
\]

\item[(ii)]
The limit $(u^\infty,p^\infty)$ is a suitable weak solution of the
N-S equations on $\R^3\times(-\infty,0)$ and satisfies the
local energy inequality on every parabolic cylinder
$B_R\times(-T,0)$.

\item[(iii)] The limit $u^\infty$ is an ancient solution, defined for all $t\le 0$, and it is
non-trivial.  More precisely, there exist $r>0$ and $c>0$ such that
\[
\int_{Q_r(0,0)} |u^\infty(x,t)|^3 \,dx\,dt \;\ge\; c > 0,
\]
where $Q_r(0,0)=B_r(0)\times(-r^2,0)$.
In particular, $u^\infty \not\equiv 0$.
\end{enumerate}

We call $u^\infty$ an \emph{ancient tangent flow} associated to the
blow-up at time $T^*$.
\end{lemma}

{\color{blue}
\begin{proof}
\textbf{[ADDED PROOF / closure of Lemma~\ref{lem:ancient-limit} (compactness + nontriviality).]}
We outline the standard compactness argument for suitable weak solutions, and we make explicit
the missing nontriviality mechanism.

\medskip
\noindent\textbf{Step 1: Uniform local bounds on cylinders.}
Fix $R>0$. For $k$ sufficiently large, the CKN rescalings \eqref{eq:ckn-rescaled} are well-defined on
$Q_R:=B_R\times(-R^2,0)$ since $T^*+r_k^2 s<T^*$ for all $s\in(-R^2,0)$ and $r_k^2R^2<T^*$ for $k$ large.
Since $u$ is smooth on $[0,T^*)$, each rescaled pair $(\tilde u^{(k)},\tilde p^{(k)})$ is smooth on $Q_R$
and in particular is a suitable weak solution there; hence it satisfies the local energy inequality
(cf.\ Definition~\ref{def:suitable}), with constants independent of $k$ after scaling.
Using standard cutoff functions supported in $B_{2R}$, one obtains a bound of the form
\begin{equation}\label{eq:uniform_local_energy_rescaled}
\sup_{s\in(-R^2,0)}\int_{B_R}|\tilde u^{(k)}(x,s)|^2\,dx
\;+\;\int_{Q_R}|\nabla \tilde u^{(k)}|^2\,dx\,ds
\;\le\; C(R),
\end{equation}
where $C(R)$ is independent of $k$.
By interpolation (Ladyzhenskaya + Sobolev) and \eqref{eq:uniform_local_energy_rescaled} we also get
\begin{equation}\label{eq:uniform_L3_rescaled}
\iint_{Q_R}|\tilde u^{(k)}|^3\,dx\,ds \le C(R).
\end{equation}
Finally, the pressure satisfies the standard local estimate (via
$-\Delta \tilde p^{(k)}=\partial_i\partial_j(\tilde u^{(k)}_i\tilde u^{(k)}_j)$ and Calder\'on--Zygmund),
which yields
\begin{equation}\label{eq:uniform_p32_rescaled}
\|\tilde p^{(k)}\|_{L^{3/2}(Q_R)} \le C(R)
\end{equation}
after fixing the additive-in-time constant of the pressure (see, e.g., \cite{CKN1982,Seregin2012}).

\medskip
\noindent\textbf{Step 2: Compactness (Aubin--Lions).}
From the Navier--Stokes system on $Q_R$,
\[
\partial_s \tilde u^{(k)}=\Delta \tilde u^{(k)}-\nabla \tilde p^{(k)}-(\tilde u^{(k)}\cdot\nabla)\tilde u^{(k)},
\]
the bounds \eqref{eq:uniform_local_energy_rescaled}--\eqref{eq:uniform_p32_rescaled} imply
that $\partial_s \tilde u^{(k)}$ is bounded in a negative Sobolev space on $Q_R$
uniformly in $k$ (e.g.\ in $L^{3/2}(-R^2,0;W^{-2,3/2}(B_R))$).
Therefore, by the Aubin--Lions compactness lemma, after passing to a subsequence we have
\[
\tilde u^{(k)}\to u^\infty \quad\text{strongly in }L^2(Q_R).
\]
Combining strong $L^2$ convergence with the uniform $L^3$ bound \eqref{eq:uniform_L3_rescaled}
and interpolation yields strong convergence in $L^p(Q_R)$ for every $1\le p<3$.
Using a diagonal subsequence over $R\in\N$ gives (i).
Similarly, by \eqref{eq:uniform_p32_rescaled} we may extract a subsequence with
$\tilde p^{(k)}\rightharpoonup p^\infty$ weakly in $L^{3/2}_{\mathrm{loc}}$, proving the pressure part of (i).

\medskip
\noindent\textbf{Step 3: Passage to the limit; suitable weak limit.}
The strong convergence of $\tilde u^{(k)}$ in $L^2_{\mathrm{loc}}$ and the weak convergence of $\nabla \tilde u^{(k)}$
in $L^2_{\mathrm{loc}}$ imply $\tilde u^{(k)}\otimes \tilde u^{(k)}\to u^\infty\otimes u^\infty$ in distributions,
so we may pass to the limit in the N--S equations on each $Q_R$.
Lower semicontinuity passes the local energy inequality to the limit, so $(u^\infty,p^\infty)$
is a suitable weak solution on $\R^3\times(-\infty,0)$, proving (ii).

\medskip
\noindent\textbf{Step 4: Nontriviality (how to close (iii) rigorously).}
Nontriviality follows from the CKN-singularity of $(x^*,T^*)$.
By the contrapositive of CKN $\varepsilon$-regularity, there exists a universal $\varepsilon_{\mathrm{CKN}}>0$ such that
for all sufficiently small $r>0$,
\[
r^{-2}\iint_{Q_r(x^*,T^*)}\bigl(|u|^3+|p|^{3/2}\bigr)\,dx\,dt \;\ge\; \varepsilon_{\mathrm{CKN}}.
\]
Taking $r=r_k$ and using the scale invariance of the CKN functional under \eqref{eq:ckn-rescaled} gives
\[
\iint_{Q_1(0,0)}\bigl(|\tilde u^{(k)}|^3+|\tilde p^{(k)}|^{3/2}\bigr)\,dy\,ds \;\ge\; \varepsilon_{\mathrm{CKN}}
\quad\text{for all }k.
\]
Passing to the limit and using lower semicontinuity yields
\[
\iint_{Q_1(0,0)}|u^\infty|^3\,dy\,ds \;\ge\; c_0>0
\]
for a universal $c_0$, proving (iii) (with $r=1$ and $c=c_0$).

\medskip
\noindent\textit{Remark.} If one prefers the vorticity normalization of Lemma~\ref{lem:blowup-normalization} for later
geometric arguments, one can re-center/renormalize the CKN blow-up sequence at a point of large vorticity
inside $Q_1$; the essential point for (iii) is that the construction must preserve a scale-invariant
lower bound (such as the CKN functional), so that triviality of the limit is ruled out.
\end{proof}
} % end added-blue proof
}






\section{The Vorticity Direction Equation}
{\color{blue}
\subsection{Derivation of the Coupled System}

Let $u$ be a sufficiently smooth divergence-free solution of the incompressible N–S equations with unit viscosity and $\omega = \curl\, u$ be the vorticity field. In the region $\{\omega \neq 0\}$,
we decompose the vorticity into its magnitude $\rho = |\omega|$ and its direction
$\xi = \omega/|\omega| \in \mathbb{S}^2$. The vorticity equation  can be written in
vector form as
\begin{equation}
\partial_t \omega + (u \cdot \nabla)\omega - \Delta \omega = (\omega \cdot \nabla)u.
    \end{equation}
Substituting $\omega = \rho \xi$ yields
\[
(\partial_t \rho + u \cdot \nabla \rho - \Delta \rho)\xi
+ \rho (\partial_t \xi + u \cdot \nabla \xi - \Delta \xi)
- 2 (\nabla \rho \cdot \nabla) \xi
= \rho (S\xi),
\]
where $S = \tfrac{1}{2}(\nabla u + (\nabla u)^T)$ is the strain tensor. We take the inner product with $\xi$ to isolate the amplitude equation.Using the identities $|\xi|^2=1$, $\xi \cdot \partial_t \xi = 0$, and $\xi \cdot \Delta \xi = -|\nabla \xi|^2$, we obtain:
\begin{equation}\label{eq:amplitude}
\partial_t \rho + u \cdot \nabla \rho - \Delta \rho = \rho (\sigma - |\nabla \xi|^2),
\end{equation}
where $\sigma = (S\xi \cdot \xi)$ is the vortex stretching scalar.

%%%%

To isolate the evolution of the direction field $\xi$, we apply the
orthogonal projection $P_\xi = I - \xi \otimes \xi$ onto the tangent space
$T_\xi \mathbb{S}^2$.  
Since $P_\xi \xi = 0$, all terms parallel to $\xi$, including the
amplitude component $(\partial_t \rho + u\cdot\nabla\rho - \Delta\rho)\xi$, 
are eliminated after projection. Thus, to derive the direction equation, we project the vorticity decomposition onto
$T_\xi \mathbb{S}^2$, which yields
\[
\rho (\partial_t \xi + u \cdot \nabla \xi - \Delta \xi)
- 2 P_\xi (\nabla \rho \cdot \nabla) \xi
= \rho P_\xi (S\xi).
\]
Dividing by $\rho$ (where $\rho > 0$) we obtain
\begin{equation}\label{eq:direction_intermediate}
\partial_t \xi + u \cdot \nabla \xi - \Delta \xi = P_\xi(S\xi) + 2 P_\xi\bigl( (\nabla \log\rho) \cdot \nabla \xi \bigr).
\end{equation}

{\color{magenta}\noindent\textbf{[AI AUDIT / SIGN-CONSISTENCY FIX.]}
The projection step yields a \emph{tangential} diffusion operator.  Using the identity
$P_\xi(\Delta \xi)=\Delta \xi + |\nabla\xi|^2\xi$ (equivalently $\Delta \xi = P_\xi(\Delta\xi)-|\nabla\xi|^2\xi$),
we may rewrite \eqref{eq:direction_intermediate} in the standard harmonic-map form:}
\begin{equation}\label{eq:direction}
{\color{magenta}\partial_t \xi + u \cdot \nabla \xi - \Delta \xi  = |\nabla\xi|^2\,\xi + H,}
\end{equation}
where the forcing $H$ is given by
\[
H = H_{\mathrm{sing}} + H_{\mathrm{geom}}.
\]
Here, $H_{\mathrm{sing}} = P_\xi (S\xi)$ represents the projection of the vortex stretching term, and $H_{\mathrm{geom}}$ collects the geometric coupling terms:
\begin{equation}\label{hgeom}
{\color{magenta}H_{\mathrm{geom}} = 2 P_\xi \bigl( (\nabla \log \rho) \cdot \nabla \xi \bigr).}
\end{equation}
By construction, the singular term $H_{\mathrm{sing}} = P_\xi(S\xi)$ and the
tangential component of $H_{\mathrm{geom}}$ lie in the tangent space
$T_\xi \mathbb{S}^2$.  
The normal component on the right-hand side of \eqref{eq:direction} is the curvature term $|\nabla\xi|^2\xi$.

{\color{magenta}\begin{remark}[Tangentiality of the geometric coupling term]
Since $|\xi|=1$, one has $\xi\cdot\partial_i\xi=\frac12\partial_i(|\xi|^2)=0$ for each spatial derivative $\partial_i$.
Therefore $(\nabla\log\rho)\cdot\nabla\xi=\sum_i(\partial_i\log\rho)\,\partial_i\xi$ is automatically orthogonal to $\xi$, and hence already lies in $T_\xi\mathbb S^2$.
In particular, the projection in \eqref{hgeom} is redundant:
\[
P_\xi\big((\nabla\log\rho)\cdot\nabla\xi\big)=(\nabla\log\rho)\cdot\nabla\xi.
\]
\end{remark}}










%%%%%%%%%%


 \subsection{The Singular Stretching Term}

The term \( H_{\mathrm{sing}} = P_\xi (S\xi) \) encodes the non‑local nonlinearity 
of the N--S equations. 

{\color{magenta}\noindent\textbf{[AI AUDIT / KERNEL-CONSISTENCY FIX.]}
Strictly speaking, $S$ is a \emph{matrix} field obtained from $\omega$ by a matrix of Calder\'on--Zygmund operators (Riesz transforms).
One convenient way to write Biot--Savart at this level is componentwise:
\[
S_{ij}(x)=\mathrm{p.v.}\int_{\R^3}\mathcal{K}_{ij\ell}(x-y)\,\omega_\ell(y)\,dy,
\]
where $\mathcal{K}$ is a tensor kernel homogeneous of degree $-3$ with cancellation.
Consequently, for each unit vector $e\in\mathbb{S}^2$ there exists a vector-valued Calder\'on--Zygmund kernel $K_e$ (depending linearly on $e$) such that
$(S e)(x)=\mathrm{p.v.}\int_{\R^3} K_e(x-y)\,\omega(y)\,dy$.}

\begin{equation}\label{eq:H_sing_integral}
{\color{magenta}
H_{\mathrm{sing}}(x)
=P_{\xi(x)}\bigl(S(x)\xi(x)\bigr)
=P_{\xi(x)}\left(\mathrm{p.v.}\int_{\R^3} K_{\xi(x)}(x-y)\,\omega(y)\,dy\right)
=P_{\xi(x)}\left(\mathrm{p.v.}\int_{\R^3} K_{\xi(x)}(x-y)\,\rho(y)\xi(y)\,dy\right).
}
\end{equation}

{\color{magenta}\noindent\textbf{[AI AUDIT / USEFUL CLASSICAL IDENTITY.]}}
\begin{lemma}[Biot--Savart identity for vortex stretching]\label{lem:biot-savart-stretching}
Let $u$ be smooth, divergence-free on $\R^3$ at a fixed time, with vorticity $\omega=\curl u$.
Then for each $x\in\R^3$,
\[
(\omega\cdot\nabla)u(x)
=
\frac{1}{4\pi}\,\mathrm{p.v.}\int_{\R^3}
\left(
\frac{\omega(x)\times\omega(y)}{|x-y|^3}
\;+\;3\,\frac{(\omega(x)\cdot(x-y))\,(\omega(y)\times(x-y))}{|x-y|^5}
\right)\,dy.
\]
\end{lemma}

\begin{proof}
This follows by differentiating the Biot--Savart law
$u(x)=\frac{1}{4\pi}\int_{\R^3}\frac{(x-y)\times\omega(y)}{|x-y|^3}\,dy$
in the $\omega(x)$ direction and using the identities
$(\omega(x)\cdot\nabla_x)(x-y)=\omega(x)$ and
$(\omega(x)\cdot\nabla_x)|x-y|^{-3}=-3(\omega(x)\cdot(x-y))|x-y|^{-5}$.
\end{proof}

{\color{magenta}\noindent\textbf{[AI AUDIT / CONSEQUENCE.]}
Writing $\omega=\rho\xi$, the first term in Lemma~\ref{lem:biot-savart-stretching} contains the factor
$\omega(x)\times\omega(y)=\rho(x)\rho(y)\,\xi(x)\times\xi(y)$ and therefore vanishes when directions align.
In particular, since $\xi(x)\times\xi(x)=0$, one may rewrite that part using the direction difference $\xi(y)-\xi(x)$.
The second term requires additional cancellation (e.g.\ via $\nabla\cdot\omega=0$ and/or a refined symmetric representation) and is part of what must be made referee-checkable in the ``near-field commutator'' step.}

{\color{magenta}\noindent\textbf{[AI AUDIT / USEFUL IDENTITY (for $H_{\mathrm{sing}}$).]}}
\begin{lemma}[$(\xi\cdot\nabla)u$ as a singular integral]\label{lem:xi-derivative}
Let $u$ be smooth and divergence-free on $\R^3$ at a fixed time, with vorticity $\omega=\curl u$.
For any $x$ with $\omega(x)\neq 0$, set $\xi(x):=\omega(x)/|\omega(x)|$. Then
\[
(\xi(x)\cdot\nabla)u(x)
=\frac{1}{4\pi}\,\mathrm{p.v.}\int_{\R^3}
\left(
\frac{\xi(x)\times\omega(y)}{|x-y|^3}
\;-\;3\,\frac{(\xi(x)\cdot(x-y))\,((x-y)\times\omega(y))}{|x-y|^5}
\right)\,dy.
\]
\end{lemma}

\begin{proof}
Differentiate the Biot--Savart law
$u(x)=\frac{1}{4\pi}\int_{\R^3}\frac{(x-y)\times\omega(y)}{|x-y|^3}\,dy$
in the (constant) direction $\xi(x)$ at the point $x$.
\end{proof}

{\color{magenta}\noindent\textbf{[AI AUDIT / CONSEQUENCE.]}
Since $\xi\parallel\omega$, the antisymmetric part of $\nabla u$ annihilates $\xi$, so $(\xi\cdot\nabla)u=S\xi$ and hence
$H_{\mathrm{sing}}=P_\xi(S\xi)=P_\xi((\xi\cdot\nabla)u)$.
The first term in Lemma~\ref{lem:xi-derivative} is already tangential and equals $\rho(y)\,\xi(x)\times\xi(y)/|x-y|^3$.
The second term does not display a direction-difference factor directly and is one of the main technical obstacles in turning the schematic commutator step into a complete proof.}

{\color{magenta}\noindent\textbf{[AI AUDIT / EXPLICIT DECOMPOSITION OF $H_{\mathrm{sing}}$.]}
Writing $\omega=\rho\,\xi$ in Lemma~\ref{lem:xi-derivative} yields the decomposition
\[
H_{\mathrm{sing}}(x)=I_{\mathrm{null}}(x)+I_{\mathrm{const}}(x)+I_{\mathrm{osc}}(x),
\]
where (with $r:=x-y$)
\[
I_{\mathrm{null}}(x):=\frac{1}{4\pi}\,\mathrm{p.v.}\int_{\R^3}\frac{\rho(y)\,\xi(x)\times\xi(y)}{|r|^3}\,dy,
\qquad
I_{\mathrm{const}}(x):=-\frac{3}{4\pi}\,\mathrm{p.v.}\int_{\R^3}\frac{(\xi(x)\cdot r)\,\rho(y)\,(r\times\xi(x))}{|r|^5}\,dy,
\]
and
\[
I_{\mathrm{osc}}(x):=-\frac{3}{4\pi}\,P_{\xi(x)}\,\mathrm{p.v.}\int_{\R^3}\frac{(\xi(x)\cdot r)\,\rho(y)\,(r\times(\xi(y)-\xi(x)))}{|r|^5}\,dy.
\]
In particular, $I_{\mathrm{null}}$ vanishes pointwise when $\xi(y)=\xi(x)$, while $I_{\mathrm{const}}$ is a fixed Calder\'on--Zygmund operator on $\rho$ depending only on the frozen direction $\xi(x)$, and equals
$I_{\mathrm{const}}(x)=\xi(x)\times\nabla\bigl((\xi(x)\cdot\nabla)(-\Delta)^{-1}\rho\bigr)(x)$.
If $\xi$ is exactly constant and $\nabla\cdot\omega=0$ (so $\xi\cdot\nabla\rho=0$), then $I_{\mathrm{const}}\equiv 0$ and hence $H_{\mathrm{sing}}\equiv 0$ as required.}

To separate the singular local interaction from the smoother far‑field contribution, 
we fix a (small) radius \( r > 0 \) and decompose the integral into a near‑field 
part and a tail:
\[
H_{\mathrm{sing}} = H_{\mathrm{near}} + H_{\mathrm{tail}},
\]
where
\[
\begin{aligned}
H_{\mathrm{near}}(x) &= P_{\xi(x)}\Bigl( \mathrm{p.v.} \int_{B_r(x)} {\color{magenta}K_{\xi(x)}}(x-y) \rho(y) \xi(y) \, dy \Bigr), \\[2mm]
H_{\mathrm{tail}}(x)  &= P_{\xi(x)}\Bigl( \int_{\mathbb{R}^3 \setminus B_r(x)} {\color{magenta}K_{\xi(x)}}(x-y) \rho(y) \xi(y) \, dy \Bigr).
\end{aligned}
\]
{\color{magenta}\noindent\textbf{[AI AUDIT / TAIL CONTROL (boundedness vs.\ smallness).]}
For fixed $r$, the operator $f\mapsto \int_{\R^3\setminus B_r(x)} K_{\xi(x)}(x-y)f(y)\,dy$ is a standard Calder\'on--Zygmund truncation (up to the frozen-direction dependence).
Thus, from scale-critical $L^{3/2}$ bounds on $\rho=|\omega|$ one can obtain \emph{boundedness} of the tail contribution in the critical Carleson norm.
However, \emph{smallness as $r\to0$ does not follow} from scale-critical control alone; it requires additional input (e.g.\ vanishing-Carleson hypotheses or a separate far-field depletion mechanism).
See \texttt{NS\_Unconditional\_Closures\_A\_to\_E.tex}, \S\texttt{subsec:D-tail}.}
{\color{magenta}\noindent\textbf{[AI AUDIT / NOTATION.]}
Here $K_{\xi(x)}$ denotes the vector-valued Calder\'on--Zygmund kernel appearing in \eqref{eq:H_sing_integral}.
For readability, the dependence on $\xi(x)$ is often suppressed later in the text; any use of CRW/commutator estimates
must account for this dependence.}

{\color{magenta}\noindent\textbf{[AI AUDIT / CPM-STYLE ``FREEZE THE KERNEL'' STEP.]}
The dependence of $K_{\xi(x)}$ on the frozen direction is \emph{linear} in $\xi(x)$ for the Biot--Savart-derived formula in Lemma~\ref{lem:xi-derivative}.
Consequently, for any fixed $a\in S^2$, the difference operator $(T_{\xi(x)}-T_a)$ has kernel bounded by $C|\xi(x)-a|/|x-y|^3$ and is a Calder\'on--Zygmund operator with $L^p$ operator norm $\lesssim |\xi(x)-a|$.
On a small ball where $\xi$ has small mean oscillation (VMO/BMO$_{\le r}$ small), one can choose $a$ to be the local average direction and ``freeze'' the kernel to $T_a$, paying an error controlled by the oscillation of $\xi$.
This is the natural analytic precursor to any referee-checkable CRW commutator estimate in the presence of $x$-dependent frozen kernels.}

The analysis of \( H_{\mathrm{near}} \) is central to our method. A key observation (e.g. see 
\cite{ConstantinFefferman1993}), is that the near‑field term decomposes into:
(i) a \emph{constant-direction} part (obtained by freezing $\xi(y)$ to $\xi(x)$) and
(ii) an \emph{oscillation} part (carrying $\xi(y)-\xi(x)$).
Explicitly, write \( \xi(y) = \xi(x) + (\xi(y) - \xi(x)) \); then
\[
H_{\mathrm{near}}(x) = P_{\xi(x)}\Bigl( 
\int_{B_r(x)} K(x-y)\rho(y)\,\xi(x)\,dy 
+ \mathrm{p.v.} \int_{B_r(x)} K(x-y)\rho(y)\bigl(\xi(y)-\xi(x)\bigr)dy 
\Bigr).
\]
{\color{magenta}\noindent\textbf{[AI AUDIT / STRUCTURE.]}
The cancellation properties of the ``constant-direction'' contribution
$P_{\xi(x)}\!\left(\int_{B_r(x)} K(x-y)\rho(y)\,\xi(x)\,dy\right)$
depend on the \emph{exact} Biot--Savart representation of $P_\xi(S\xi)$.
As discussed in the kernel-consistency note leading to \eqref{eq:H_sing_integral}, the operator involves the contraction with $\xi(x)$ and the projection,
so a referee-checkable depletion argument requires an explicit identity showing that $H_{\mathrm{near}}$ can be rewritten \emph{purely} in terms of the oscillation
$\xi(y)-\xi(x)$ (a true commutator form), so that constant $\xi$ yields $H_{\mathrm{near}}\equiv 0$.
This derivation is not supplied in the current manuscript and must be added (or stated as an explicit hypothesis).}

{\color{magenta}\noindent\textbf{[AI AUDIT / IDENTIFYING THE CONSTANT-DIRECTION REMAINDER.]}
Lemma~\ref{lem:xi-derivative} shows that there is a nontrivial ``constant-direction'' contribution hiding inside the second term:
if one freezes $\xi(y)$ to $\xi(x)$ in that term (i.e.\ replaces $\omega(y)$ by $\rho(y)\,\xi(x)$), then the resulting vector field equals
$\xi(x)\times\nabla((\xi(x)\cdot\nabla)(-\Delta)^{-1}\rho)(x)$ (up to universal constants), which is a fixed Calder\'on--Zygmund operator on $\rho$.
In the \emph{ideal} constant-direction case, $\omega=\rho\,\xi$ with $\xi$ constant and $\nabla\cdot\omega=0$ forces $(\xi\cdot\nabla)\rho=0$, and then
$(\xi\cdot\nabla)(-\Delta)^{-1}\rho\equiv 0$ (Fourier support has $\xi\cdot k=0$), so this term vanishes as it must.
Moreover, using $\nabla\cdot\omega=0$ one has for any fixed $a\in S^2$ the exact identity
$a\cdot\nabla\rho=\nabla\cdot(\rho a-\omega)$, and therefore
\[
a\times\nabla\bigl((a\cdot\nabla)(-\Delta)^{-1}\rho\bigr)
\;=\;a\times\nabla(-\Delta)^{-1}\nabla\cdot(\rho a-\omega).
\]
Taking $a=\xi(x)$ shows that this ``constant-direction'' term can be rewritten as a CZ operator applied to the \emph{direction error} $\rho(\xi(x)-\xi)$.
The remaining issue is to make this cancellation \emph{quantitative} (small in the critical Carleson norm) under the hypotheses available for the running-max ancient element.}

\begin{lemma}[Constant-direction remainder as a CZ operator on the direction error]\label{lem:constdir-remainder}
Let $u$ be smooth and divergence-free on $\R^3$ at a fixed time, with vorticity $\omega=\curl u$. Write $\omega=\rho\,\xi$ on $\{\omega\neq0\}$ and extend $\rho:=|\omega|$ by $0$ on $\{\omega=0\}$. Fix a constant unit vector $a\in\Sbb^2$.
Then, in the sense of distributions on $\R^3$,
\[
a\times\nabla\bigl((a\cdot\nabla)(-\Delta)^{-1}\rho\bigr)
\;=\;a\times\nabla(-\Delta)^{-1}\nabla\cdot(\rho a-\omega).
\]
In particular, since $\rho a-\omega=\rho(a-\xi)$, the left-hand side is a Calder\'on--Zygmund operator applied to the direction error $\rho(a-\xi)$.
\end{lemma}

\begin{proof}
Since $\nabla\cdot\omega=0$, we have $\nabla\cdot(\rho a-\omega)=a\cdot\nabla\rho$ in distributions. Therefore
\[
(a\cdot\nabla)(-\Delta)^{-1}\rho=(-\Delta)^{-1}(a\cdot\nabla\rho)=(-\Delta)^{-1}\nabla\cdot(\rho a-\omega),
\]
and applying $a\times\nabla$ to both sides yields the claim.
\end{proof}

\begin{lemma}[Quantitative consequence: constant-direction term is controlled by a weighted direction error]\label{lem:constdir-weighted-error}
Fix $a\in\Sbb^2$ and define the constant-direction Calder\'on--Zygmund operator on scalars
\[
(T_a f)(x):=a\times\nabla\bigl((a\cdot\nabla)(-\Delta)^{-1}f\bigr)(x).
\]
Then for every $1<p<\infty$ there exists $C_p<\infty$ such that for all vector fields $F:\R^3\to\R^3$,
\[
\|a\times\nabla(-\Delta)^{-1}\nabla\cdot F\|_{L^p(\R^3)}\le C_p\,\|F\|_{L^p(\R^3)}.
\]
In particular, if $\omega=\rho\,\xi$ with $\nabla\cdot\omega=0$, then for each fixed $a\in\Sbb^2$,
\[
\|T_a \rho\|_{L^p(\R^3)} \;=\; \|a\times\nabla(-\Delta)^{-1}\nabla\cdot(\rho(a-\xi))\|_{L^p(\R^3)}
\;\le\; C_p\,\|\rho(a-\xi)\|_{L^p(\R^3)}.
\]
\end{lemma}

\begin{proof}
Each component of $a\times\nabla(-\Delta)^{-1}\nabla\cdot$ is a finite linear combination of Riesz transforms, hence a Calder\'on--Zygmund operator bounded on $L^p$ for $1<p<\infty$.
The final estimate follows from Lemma~\ref{lem:constdir-remainder} with $F=\rho(a-\xi)$.
\end{proof}

{\color{magenta}\noindent\textbf{[AI AUDIT / WHAT THIS IDENTIFIES.]}
Lemma~\ref{lem:constdir-weighted-error} shows that the remaining ``constant-direction'' contribution is quantitatively controlled by the \emph{weighted direction error} $\rho(a-\xi)$.
Thus, in general one needs a mechanism that makes $\rho(\xi-\text{local frozen direction})$ small in a scale-invariant $L^{3/2}$ sense on shrinking cylinders.
\emph{In the running-max setting}, boundedness of $\rho=|\omega^\infty|$ already provides this automatically (Remark~\ref{rem:constdir-easy-Linfty}).}

{\color{magenta}\begin{remark}[Running-max bonus: bounded vorticity makes the constant-direction remainder Carleson-small]\label{rem:constdir-easy-Linfty}
Let $(u^\infty,p^\infty)$ be the running-max ancient element from Lemma~\ref{lem:ancient-limit-runningmax}, and write $\omega^\infty=\rho^\infty\xi^\infty$ on $\{\omega^\infty\neq 0\}$.
Then $\|\rho^\infty\|_{L^\infty(\R^3\times(-\infty,0])}\le 2$ by Lemma~\ref{lem:ancient-limit-runningmax}(iii). Since $|a-\xi^\infty|\le 2$ for any unit vector $a$,
\[
r^{-2}\iint_{Q_r(z_0)} |\rho^\infty(a-\xi^\infty)|^{3/2}
\le (4)^{3/2}\,r^{-2}\,|Q_r|
\le C\,r^{3}\qquad(0<r\le 1),
\]
and hence $\lim_{r_*\to0}\|\rho^\infty(a-\xi^\infty)\|_{C^{3/2}(r_*)}=0$.
Combined with Lemma~\ref{lem:constdir-weighted-error} (with $p=3/2$ and localization to balls), this yields smallness of the constant-direction remainder in the critical Carleson norm at sufficiently small scales.
\end{remark}}

{\color{magenta}\noindent\textbf{[AI AUDIT / NEAR-FIELD REDUCTION (what is now checkable).]}
At the level of the truncated near-field operator, one has the exact algebraic split
\[
H_{\mathrm{near}}(x)=\frac{1}{4\pi}\,\mathcal T_{\xi(x),r}(\rho(\cdot)\xi(x))(x)
\;+\;P_{\xi(x)}\Bigl(\frac{1}{4\pi}\,\mathcal T_{\xi(x),r}\bigl(\rho(\cdot)(\xi(\cdot)-\xi(x))\bigr)(x)\Bigr),
\]
where $\mathcal T_{\xi(x),r}$ denotes the Biot--Savart-derived truncated singular integral in Lemma~\ref{lem:xi-derivative}.
Using $\nabla\cdot\omega=0$, the \emph{full-space} version of the first term is a CZ operator applied to the direction error $\rho(\xi(x)-\xi)$; truncation introduces an explicit tail remainder, so the near-field commutator reduction is now a precise, referee-checkable target.}

{\color{magenta}\noindent\textbf{[AI AUDIT / CHECKABLE COMMUTATOR FORM FOR THE OSCILLATION TERM.]}
Define fixed kernels (for $m,j\in\{1,2,3\}$ and $r:=x-y$)
\[
k_{m,j}(r):=\frac{e_m\times e_j}{|r|^3}-3\,\frac{r_m\,(r\times e_j)}{|r|^5},
\qquad
(T_{m,j,r}f)(x):=\mathrm{p.v.}\int_{B_r(x)} k_{m,j}(x-y)\,f(y)\,dy.
\]
Expanding cross-products shows that for the oscillation piece one has the exact identity
\[
P_{\xi(x)}\Bigl(\frac{1}{4\pi}\,\mathcal T_{\xi(x),r}\bigl(\rho(\cdot)(\xi(\cdot)-\xi(x))\bigr)(x)\Bigr)
\;=\;\frac{1}{4\pi}\,P_{\xi(x)}\sum_{m,j=1}^3 \xi_m(x)\,[T_{m,j,r},\xi_j]\,\rho\,(x),
\]
where $[T,b]f:=T(bf)-b\,Tf$.
Thus, in the CKN-tangent-flow route one can combine a spatial VMO hypothesis for $\xi$ with scale-critical $L^{3/2}$ control of $\rho=|\omega|$ to obtain Carleson smallness of this oscillation term via the Coifman--Rochberg--Weiss theorem.
In the running-max refactor, one has the stronger bound $\rho^\infty\in L^\infty$, and the resulting near-field commutator/oscillation term is Carleson-small at small scales without any VMO input (Lemma~\ref{lem:nearfield-osc-carleson}).}


The dangerous part that can become large is precisely the second term, 
involving the difference \( \xi(y)-\xi(x) \). If the direction field \( \xi \) 
varies slowly (e.g. is Lipschitz with a moderate constant), this term remains 
controllable. Rapid oscillations of \( \xi \), on the other hand, can interact 
with the singular kernel to produce uncontrolled amplification, the mechanism 
that could potentially lead to a finite‑time blow‑up.  

Hence, the geometric regularity criterion can be phrased as follows:  
singular vortex stretching can be tamed provided the vorticity direction does not 
oscillate too violently in regions of intense vorticity.


\subsection{The Geometric Forcing Term}

By analyzing the singular stretching term \( H_{\mathrm{sing}} \), we now turn to 
{\color{magenta}the geometric contributions on the right-hand side of \eqref{eq:direction}.  Geometrically, these arise from the constraint \( |\xi| = 1 \) and 
the coupling between the amplitude \( \rho \) and the direction \( \xi \). They consist 
of two distinct parts:}
\begin{enumerate}
    \item The harmonic map tension term \( |\nabla \xi|^2 \xi \), which is
          normal to the sphere \( \mathbb{S}^2 \). In the equation for \( \xi \), 
          it appears as a Lagrange multiplier such that $|\xi|=1$.
    \item The cross‑term \( 2 P_\xi (\nabla \log \rho \cdot \nabla \xi) \), which 
          is tangential and connects the geometry of the direction field to the 
          gradient of the log‑amplitude \( \log\rho \).
\end{enumerate}

{\color{magenta}Both geometric contributions (the curvature term \( |\nabla \xi|^2 \xi \) and the tangential coupling term \(H_{\mathrm{geom}}\) from \eqref{hgeom}) involve first derivatives and are}
bilinear or quadratic in gradients. Under the  scaling (\ref{scaling}), both terms have the same 
homogeneity as the diffusion term $-\Delta\xi$, placing them at the critical 
dimensional threshold. Unlike the 
nonlocal stretching term \(H_{\mathrm{sing}}\), these geometric contributions are 
purely local and, in analytical practice, can often be controlled through energy 
estimates or interpolation inequalities, provided suitable a priori bounds are 
available on \(\nabla\xi\) and \(\nabla\log\rho\). Nevertheless, their critical 
scaling means that they cannot be treated as negligible error terms in a 
blow-up scenario and must be handled with care in any critical or supercritical 
regularity framework.}



\section{Critical Coercivity of the Stretching Term}

\subsection{Regularity structure of the direction field}
In the original CKN-tangent-flow route, a VMO/BMO-smallness hypothesis on $\xi^\infty$ is a natural way to force commutator depletion of the near-field oscillation term.
In the running-max rewrite, bounded vorticity already yields near-field oscillation depletion for the commutator/oscillation term (Lemma~\ref{lem:nearfield-osc-carleson}).
We therefore do not treat a directional VMO hypothesis as a separate conditional input in this running-max rewrite.
If a later step truly requires quantitative small oscillation of $\xi^\infty$ (beyond bounded vorticity), that requirement should be stated explicitly at the point of use.

\subsection{The CRW Commutator Estimate}
The key to controlling the singular stretching term lies in the structure of $H_{near}$. 
{\color{magenta}\noindent\textbf{[AI AUDIT / STRUCTURE.]}
The ``commutator'' representation below is \emph{schematic} and does not follow from $P_{\xi(x)}\xi(x)=0$ alone,
since the kernel acts before the projection (and the correct Biot--Savart kernel for $S\xi$ depends on $\xi(x)$ as noted in \eqref{eq:H_sing_integral}).
To use CRW rigorously, one must supply a derivation that reduces $H_{near}$ to a Calder\'on--Zygmund commutator with multiplier $\xi$ (or else assume such a representation).}
In the present manuscript, the \emph{oscillation} component of $H_{\mathrm{near}}$ has already been reduced to a finite sum of commutators with \emph{fixed} truncated Calder\'on--Zygmund operators; see the explicit identity in the audit block
\textbf{[CHECKABLE COMMUTATOR FORM FOR THE OSCILLATION TERM]} preceding this subsection (cf.\ \eqref{eq:H_sing_integral} and Lemma~\ref{lem:xi-derivative}).

We now record the classical commutator bound that converts small BMO oscillation of $\xi$ into smallness of these commutator terms.

\begin{lemma}[CRW Commutator Estimate]\label{lem:crw}
Let $T$ be a Calder\'on--Zygmund operator on $\R^3$ and let $T_r$ denote a standard truncation at scale $r>0$
(e.g.\ $T_r f(x)=\mathrm{p.v.}\int_{|x-y|<r}K(x-y)f(y)\,dy$ for a CZ kernel $K$).
Then for every $1<p<\infty$ there exists $C_p<\infty$ (depending only on $p$ and CZ constants of $T$) such that for all $r>0$,
\[
\|[T_r,b]f\|_{L^p(\R^3)}\le C_p\,\|b\|_{\BMO(\R^3)}\,\|f\|_{L^p(\R^3)},
\]
where $[T_r,b]f:=T_r(bf)-b\,T_r f$.
\end{lemma}

\begin{proof}
This is the classical Coifman--Rochberg--Weiss commutator theorem \cite{CRW1976}. The dependence on the truncation scale $r$ is uniform.
\end{proof}

{\color{magenta}\begin{remark}[How \ref{lem:crw} is used here]
Lemma~\ref{lem:crw} is applied to the fixed truncated kernels $T_{m,j,r}$ introduced in the commutator identity
\(
P_{\xi(x)}\sum_{m,j}\xi_m(x)\,[T_{m,j,r},\xi_j]\rho
\)
(see the earlier audit block).
In the running-max setting, $\rho^\infty=|\omega^\infty|$ is \emph{bounded} (Lemma~\ref{lem:ancient-limit-runningmax}(iii)), and $\xi$ is bounded by $1$.
Since $L^\infty\subset\BMO$, the commutator estimate yields a uniform $L^{3/2}$ bound on $H_{\mathrm{near}}^{\mathrm{osc}}$ on each small cylinder, and the parabolic Carleson normalization then forces \emph{smallness as $r\to0$}.
\end{remark}}

\begin{lemma}[Near-field commutator/oscillation term is small in the critical Carleson norm]\label{lem:nearfield-osc-carleson}
Let $(u^\infty,p^\infty)$ be the running-max ancient element of Lemma~\ref{lem:ancient-limit-runningmax}, and write $\omega^\infty=\rho^\infty\xi^\infty$ on $\{\omega^\infty\neq0\}$.
Let $H_{\mathrm{near}}^{\mathrm{osc}}$ denote the oscillation term identified in the commutator identity in the audit block preceding this subsection (i.e.\ the term involving $\rho(\xi-\xi(x))$ and commutators $[T_{m,j,r},\xi_j]\rho$).
Then for every $\varepsilon>0$ there exists $r_0>0$ such that for all $0<r\le r_0$,
\[
\sup_{z_0}\ r^{-2}\iint_{Q_r(z_0)} |H_{\mathrm{near}}^{\mathrm{osc}}|^{3/2}\,dx\,dt\ \le\ \varepsilon.
\]
\end{lemma}

\begin{proof}
Fix $z_0=(x_0,t_0)$ and $0<r\le 1$. For a.e.\ $t\in(t_0-r^2,t_0)$ the commutator identity and Lemma~\ref{lem:crw} (with $p=3/2$) give
\[
\|H_{\mathrm{near}}^{\mathrm{osc}}(\cdot,t)\|_{L^{3/2}(B_r(x_0))}
\ \le\ C\,\|\xi(\cdot,t)\|_{\BMO(\R^3)}\,\|\rho(\cdot,t)\|_{L^{3/2}(B_{2r}(x_0))}.
\]
Since $|\xi|\le 1$, one has $\|\xi(\cdot,t)\|_{\BMO(\R^3)}\le 2$.
Moreover, by Lemma~\ref{lem:ancient-limit-runningmax}(iii) we have $\|\rho^\infty\|_{L^\infty(\R^3\times(-\infty,0])}\le 2$, hence for each $t$,
\(
\|\rho(\cdot,t)\|_{L^{3/2}(B_{2r}(x_0))}\le C\,\|\rho\|_{L^\infty}\,r^2\le C\,r^2.
\)
Raising to the $3/2$ power and integrating in $t$ yields
\[
r^{-2}\iint_{Q_r(z_0)} |H_{\mathrm{near}}^{\mathrm{osc}}|^{3/2}
\ \le\ C\,r^{-2}\int_{t_0-r^2}^{t_0}\Bigl(\|\xi(\cdot,t)\|_{\BMO(\R^3)}\,\|\rho(\cdot,t)\|_{L^{3/2}(B_{2r}(x_0))}\Bigr)^{3/2}\,dt
\ \le\ C\,r^{-2}\int_{t_0-r^2}^{t_0} (r^2)^{3/2}\,dt
\ \le\ C\,r^3.
\]
Choosing $r_0$ so that $C r_0^3\le \varepsilon$ yields the claim.
\end{proof}

\subsection{Tail Control}
For fixed $r>0$, the far-field contribution $H_{\mathrm{tail}}$ is a standard Calder\'on--Zygmund truncation (up to the frozen-direction dependence of the kernel described earlier).
Thus one expects \emph{boundedness} in $L^p$ (uniformly in $r$) via maximal-truncation/Cotlar inequalities, but \emph{not smallness} as $r\to0$ from scale-critical control alone.

\begin{lemma}[Tail boundedness via maximal truncations (no smallness)]\label{lem:tail-bounded}
Let $T$ be a Calder\'on--Zygmund operator on $\R^3$ and let $T_{>r}$ denote a standard truncation
\[
T_{>r}f(x):=\int_{|x-y|>r}K(x-y)\,f(y)\,dy.
\]
Then for every $1<p<\infty$ there exists $C_p$ such that for all $r>0$,
\[
\|T_{>r}f\|_{L^p(\R^3)}\le C_p\,\|f\|_{L^p(\R^3)}.
\]
\end{lemma}

{\color{magenta}\noindent\textbf{[AI AUDIT / CONSEQUENCE.]}
Even with Lemma~\ref{lem:omega32-runningmax-automatic}, Lemma~\ref{lem:tail-bounded} yields only that $H_{\mathrm{tail}}$ is \emph{bounded} in the critical Carleson norm.
Obtaining the \emph{uniform smallness as $r\to0$} demanded by Assumption~\ref{assump:D-forcing} requires additional input (e.g.\ a vanishing-Carleson hypothesis, or a separate far-field depletion mechanism such as the later pressure/tail route).
In particular, while bounded vorticity gives strong local structure, the present manuscript does not supply a uniform mechanism that forces $\sup_{z_0}\sup_{0<r\le r_0} r^{-2}\iint_{Q_r(z_0)}|H_{\mathrm{tail}}|^{3/2}$ to be small for some \emph{single} $r_0>0$ independent of $z_0$.}

\subsection{Theorem: Forcing Depletion}
Combining bounded vorticity, the commutator estimate, and the tail control discussion, we arrive at the first main technical result of this paper.

\begin{theorem}[Forcing Depletion]\label{thm:forcing_depletion}
Let $(u^\infty,p^\infty)$ be the running-max ancient element produced by Lemma~\ref{lem:ancient-limit-runningmax}, and let $\xi^\infty=\omega^\infty/|\omega^\infty|$ on $\{\omega^\infty\neq0\}$.
For any $\varepsilon > 0$, there exists a scale $r_0 > 0$ such that for all $r \le r_0$, the \emph{commutator/oscillation} part of the near-field forcing satisfies the scale-invariant Carleson bound
\[
\sup_{z_0 \in \R^3 \times (-\infty, 0]} r^{-2} \iint_{Q_r(z_0)} |H_{\mathrm{near}}^{\mathrm{osc}}|^{3/2} \, dx \, dt \le \varepsilon.
\]
{\color{magenta}\noindent\textbf{[AI AUDIT.]}
Controlling the full tail $H_{\mathrm{tail}}$ still requires additional input (tail depletion / pressure isotropization), as discussed below and in Section~\ref{sec:pressure}.
For the geometric forcing: Lemma~\ref{lem:log_amplitude} provides a fully classical (regularized) log-amplitude Caccioppoli estimate on each cylinder for the running-max ancient element (bounded vorticity $\Rightarrow$ local smoothness, Lemma~\ref{lem:Linfty-vort-smooth}).
What is \emph{not} automatic is a uniform control of $\nabla\log\rho$ across the vorticity-zero set (since Lemma~\ref{lem:log_amplitude} controls only $\nabla\log(\rho+\varepsilon)$).
Conditional on such a scale-invariant $L^2$ bound on $\nabla\log\rho$, the small direction-energy hypothesis in Assumption~\ref{assump:C-liouville} forces $H_{\mathrm{geom}}$ to be Carleson-small at small scales (Lemma~\ref{lem:hgeom-carleson-from-energy}).
The remaining constant-direction part of $H_{\mathrm{near}}$ is Carleson-small at small scales from bounded vorticity (Remark~\ref{rem:constdir-easy-Linfty}).}
\end{theorem}

\begin{proof}
This is exactly Lemma~\ref{lem:nearfield-osc-carleson}.

\end{proof}

This theorem resolves the "oscillation vs. mass" dilemma. It asserts that in the critical regime, the "mass" (represented by $\rho$) cannot generate critical stretching because it is modulated by the "oscillation" (of $\xi$), which vanishes asymptotically. Thus, the primary driver of potential blow-up is quantitatively depleted.

\section{Control of the Geometric Forcing}

\subsection{Bounds on $\nabla \log \rho$}
We now turn to the geometric term $H_{\mathrm{geom}}$. A crucial component is the gradient of the log-amplitude, $\nabla \log \rho$. While the amplitude $\rho$ may blow up, its logarithmic gradient behaves more like a critical energy density. Using the amplitude equation \eqref{eq:amplitude}, which is a drift--diffusion equation with source $\rho(\sigma - |\nabla \xi|^2)$, we can derive scale-invariant $L^2$ bounds.

\begin{lemma}[Caccioppoli estimate for the regularized log-amplitude]\label{lem:log_amplitude}
Let $(u^\infty,p^\infty)$ be the running-max ancient element from Lemma~\ref{lem:ancient-limit-runningmax}, and write $\omega^\infty=\rho\,\xi$ on $\{\rho>0\}$ with $\rho:=|\omega^\infty|$.
Fix $z_0=(x_0,t_0)$ and $0<r\le 1$.
For $\varepsilon\in(0,1)$ set
\[
h_\varepsilon:=\log(\rho+\varepsilon).
\]
Then there exists a constant $C$ (independent of $z_0,r$ but possibly depending on $\varepsilon$) such that
\[
r^{-3}\iint_{Q_r(z_0)} |\nabla h_\varepsilon|^2 \, dx \, dt
\ \le\ C\Bigl(1+r^{-3}\iint_{Q_{2r}(z_0)}\bigl(|\sigma|+|\nabla \xi|^2\bigr)\,dx\,dt\Bigr)
\ +\ C\,r^{-5}\iint_{Q_{2r}(z_0)}|u-u_{B_{2r}(x_0)}(t)|^2\,dx\,dt.
\]
{\color{magenta}\noindent\textbf{[AI AUDIT / what this does and does not give.]}
This estimate is fully classical once $u^\infty$ is known smooth on $Q_{2r}(z_0)$ (which follows from bounded vorticity via Lemma~\ref{lem:Linfty-vort-smooth}).
It gives a \emph{scale-invariant bound} on $\nabla\log(\rho+\varepsilon)$ on each cylinder, but does \emph{not} automatically yield any uniform bound on $\nabla\log\rho$ as $\varepsilon\downarrow0$ in the presence of vorticity zeros (cf.\ Example~\ref{ex:vmo-fails-at-zeros}, where $\nabla\log\rho$ can fail to be in $L^2$ near $\{\rho=0\}$).
Thus, additional structure is needed to upgrade this into the exact $H_{\mathrm{geom}}$ Carleson-smallness required in (D).}
\end{lemma}

\begin{proof}[Proof (classical, with explicit integration by parts and a Galilean gauge)]
Fix $\varepsilon\in(0,1)$, set $\rho_\varepsilon:=\rho+\varepsilon$ and $h_\varepsilon:=\log(\rho_\varepsilon)$, and choose a standard cutoff $\phi\in C_c^\infty(Q_{2r}(z_0))$ with $\phi\equiv 1$ on $Q_r(z_0)$ and $|\nabla\phi|\lesssim r^{-1}$, $|\partial_t\phi|\lesssim r^{-2}$.
Since $\omega^\infty\in L^\infty$ (Lemma~\ref{lem:ancient-limit-runningmax}(iii)), Lemma~\ref{lem:Linfty-vort-smooth} implies $u^\infty$ (hence $\rho$) is smooth on $Q_{2r}(z_0)$, so all computations below are classical.

Start from the amplitude equation
\(
\partial_t \rho + u\cdot\nabla\rho-\Delta\rho=\rho(\sigma-|\nabla\xi|^2)
\)
and multiply it by $\phi^2(\rho+\varepsilon)^{-1}$.
Integrating by parts in space-time and using $\nabla\cdot u=0$ yields the standard logarithmic Caccioppoli identity.
For completeness we record the key algebraic steps.
Write $\rho_\varepsilon=\rho+\varepsilon$ and $h_\varepsilon=\log\rho_\varepsilon$, so $\partial_t h_\varepsilon=\rho_\varepsilon^{-1}\partial_t\rho$ and $\nabla h_\varepsilon=\rho_\varepsilon^{-1}\nabla\rho$.
Multiplying \eqref{eq:amplitude} by $\phi^2/\rho_\varepsilon$ and integrating over $Q_{2r}(z_0)$ gives
\[
\iint \phi^2\,\partial_t h_\varepsilon
\ +\ \iint \phi^2\,u\cdot\nabla h_\varepsilon
\ -\ \iint \phi^2\,\frac{\Delta\rho}{\rho_\varepsilon}
\ =\ \iint \phi^2\,\frac{\rho}{\rho_\varepsilon}\,(\sigma-|\nabla\xi|^2).
\]
For the diffusion term, an integration by parts in space yields
\[
-\iint \phi^2\,\frac{\Delta\rho}{\rho_\varepsilon}
=\iint \phi^2\,|\nabla h_\varepsilon|^2 + 2\iint \phi\,\nabla h_\varepsilon\cdot\nabla\phi
\ \ge\ \frac12\iint \phi^2\,|\nabla h_\varepsilon|^2 - C\iint |\nabla\phi|^2,
\]
by Young's inequality.
For the time cutoff, integrating by parts in time gives
\[
\iint \phi^2\,\partial_t h_\varepsilon
= \int_{\R^3}\phi^2 h_\varepsilon\Big|_{t=t_0-4r^2}^{t=t_0} - \iint (\partial_t\phi^2)\,h_\varepsilon.
\]
Since $\rho\le \|\omega^\infty\|_{L^\infty}\le 2$ on $Q_{2r}(z_0)$, we have $h_\varepsilon\le \log 3$ pointwise, hence the boundary term at $t=t_0$ is $\le C r^3$.
The remaining term $-\iint (\partial_t\phi^2)\,h_\varepsilon$ is a standard cutoff error.
Controlling it uniformly as $\varepsilon\downarrow0$ across $\{\rho=0\}$ is subtle because $h_\varepsilon$ can take values $\sim \log\varepsilon$ on the zero set; this is part of the ``$\varepsilon\downarrow0$'' issue isolated later as Assumption~\ref{assump:D-logamp}.
For each fixed $\varepsilon>0$ it is harmless and can be bounded in terms of $\iint|\partial_t\phi|$ and the size of $h_\varepsilon$ on the support of $\partial_t\phi$.
Finally, since $0\le \rho/\rho_\varepsilon\le 1$, the right-hand side is bounded by $\iint \phi^2(|\sigma|+|\nabla\xi|^2)$.
Collecting these bounds yields the stated inequality (after absorbing lower-order terms and using the drift estimate below).
\[
\iint_{Q_{2r}} |\nabla h_\varepsilon|^2\phi^2
\ \le\ C\iint_{Q_{2r}} \bigl(|\sigma|+|\nabla\xi|^2\bigr)\phi^2
\ +\ C\iint_{Q_{2r}}\bigl(|\nabla\phi|^2+|\partial_t\phi|\bigr)
\ +\ C\Bigl|\iint_{Q_{2r}} (u\cdot\nabla\phi^2)\,h_\varepsilon\Bigr|.
\]
The last term is the cutoff-error drift contribution.
For each fixed time $t$, $\int_{\R^3} (u(\cdot,t)\cdot\nabla\phi^2(\cdot,t))\,dx=0$ by $\nabla\cdot u=0$ and compact support of $\phi$.
Thus one may subtract the spatial average $(h_\varepsilon)_{B_{2r}(x_0)}(t)$ and write
\[
\int (u\cdot\nabla\phi^2)\,h_\varepsilon
=\int (u\cdot\nabla\phi^2)\,\bigl(h_\varepsilon-(h_\varepsilon)_{B_{2r}}\bigr).
\]
Now apply Cauchy--Schwarz and Poincar\'e on $B_{2r}(x_0)$:
\[
\|h_\varepsilon-(h_\varepsilon)_{B_{2r}}\|_{L^2(B_{2r})}\ \le\ C r\,\|\nabla h_\varepsilon\|_{L^2(B_{2r})}.
\]
Using $|\nabla\phi|\lesssim r^{-1}$ and writing $u=(u-u_{B_{2r}}(t))+u_{B_{2r}}(t)$, the contribution of the constant vector field $u_{B_{2r}}(t)$ is controlled in the same way (it factors out of the spatial integral and only hits the cutoff).
Thus it suffices to bound the $u-u_{B_{2r}}$ contribution, yielding
\[
\Bigl|\iint (u\cdot\nabla\phi^2)\,h_\varepsilon\Bigr|
\le C r^{-1}\|u-u_{B_{2r}}\|_{L^2(Q_{2r})}\,\|h_\varepsilon-(h_\varepsilon)_{B_{2r}}\|_{L^2(Q_{2r})}
\le C \|u-u_{B_{2r}}\|_{L^2(Q_{2r})}\,\|\nabla h_\varepsilon\|_{L^2(Q_{2r})}.
\]
Finally, absorb $\|\nabla h_\varepsilon\|_{L^2}^2$ into the left-hand side with Young's inequality, leaving a contribution $\lesssim \|u-u_{B_{2r}}\|_{L^2(Q_{2r})}^2$.
Dividing by $r^3$ and using $|\nabla\phi|^2+|\partial_t\phi|\lesssim r^{-2}$ completes the claimed scale-invariant inequality.

{\color{magenta}\noindent\textbf{[AI AUDIT / remaining issue in (D-geom).]}
The integration-by-parts/Caccioppoli estimate above is now fully classical and referee-checkable on each fixed cylinder because the running-max ancient element is smooth there (bounded vorticity $\Rightarrow$ local smoothness, Lemma~\ref{lem:Linfty-vort-smooth}).
What remains open is not the cutoff bookkeeping, but whether one has any \emph{uniform} control of $\nabla\log(\rho+\varepsilon)$ as $\varepsilon\downarrow 0$ across the vorticity-zero set (isolated as Assumption~\ref{assump:D-logamp}).}
\end{proof}

{\color{magenta}\noindent\textbf{[AI AUDIT.]}
For the running-max ancient element, bounded vorticity implies local smoothness (Lemma~\ref{lem:Linfty-vort-smooth}), so the amplitude/log-amplitude computations can be carried out classically on each compact cylinder.
Lemma~\ref{lem:log_amplitude} therefore reduces the ``drift absorption'' issue to an explicit cutoff estimate (handled by mean-subtraction, Poincar\'e, and a Galilean gauge).
What remains open for (D) is not the formal integration by parts, but rather:
(i) controlling the behavior of $\nabla\log(\rho+\varepsilon)$ as $\varepsilon\downarrow 0$ in the presence of vorticity zeros (Example~\ref{ex:vmo-fails-at-zeros} shows that $\|\nabla\log(\rho+\varepsilon)\|_{L^2}$ can blow like $\log(1/\varepsilon)$ if $\{\rho=0\}$ has codimension $2$), and
(ii) obtaining a \emph{non-circular} mechanism forcing $H_{\mathrm{geom}}$ to be small in the critical Carleson norm at small scales without simply assuming the global small-energy gate from (C).
In this manuscript we therefore treat ``$\nabla\log\rho$ control across $\{\rho=0\}$'' as part of item (D).}

\subsection{Bilinear Estimates}
The cross-term in the geometric forcing is
$H_{\mathrm{geom}}=2 P_\xi\bigl((\nabla \log\rho)\cdot\nabla\xi\bigr)$ (cf.\ \eqref{hgeom}). Writing $h=\log\rho$ and using $|P_\xi v|\le |v|$,
we have the pointwise bound $|H_{\mathrm{geom}}|\le 2|\nabla h|\,|\nabla\xi|$.
Therefore, by H\"older,
\[
\iint_{Q_r(z_0)} |H_{\mathrm{geom}}|^{3/2}
\le C \iint_{Q_r(z_0)} |\nabla h|^{3/2}|\nabla\xi|^{3/2}
\le C\left(\iint_{Q_r(z_0)} |\nabla h|^2\right)^{3/4}\left(\iint_{Q_r(z_0)} |\nabla\xi|^2\right)^{3/4}.
\]
Thus $H_{\mathrm{geom}}$ is \emph{vanishing-Carleson at small scales} once one has (i) a scale-invariant $L^2$ bound on $\nabla\log\rho$ and (ii) small direction energy on the relevant cylinders, as made precise below.

\begin{lemma}[Geometric forcing becomes Carleson-small from log-amplitude $L^2$ control and small direction energy]\label{lem:hgeom-carleson-from-energy}
Let $h=\log\rho$ and $H_{\mathrm{geom}}$ be defined by \eqref{hgeom}.
Assume there exists $K_h<\infty$ such that for every $z_0$ and every $0<r\le 1$,
\[
r^{-3}\iint_{Q_r(z_0)}|\nabla h|^2\,dx\,dt \le K_h,
\]
and assume the direction energy is globally small in the sense of Assumption~\ref{assump:C-liouville}:
\[
\sup_{z_0}\ \sup_{r>0}\ E(z_0,r)\le \eps_*^2,\qquad E(z_0,r)=r^{-3}\iint_{Q_r(z_0)}|\nabla\xi|^2\,dx\,dt.
\]
Then for every $0<r_0\le 1$,
\[
\sup_{z_0}\ \sup_{0<r\le r_0}\ r^{-2}\iint_{Q_r(z_0)} |H_{\mathrm{geom}}|^{3/2}\,dx\,dt
\le C\,K_h^{3/4}\,\eps_*^{3/2}\,r_0^{5/2}.
\]
In particular, for any $\delta>0$ one may choose $r_0=r_0(\delta,K_h,\eps_*)$ so that $\|H_{\mathrm{geom}}\|_{C^{3/2}(r_0)}\le \delta$.
\end{lemma}

\begin{proof}
Fix $z_0$ and $0<r\le r_0\le 1$. Using $|H_{\mathrm{geom}}|\le 2|\nabla h|\,|\nabla\xi|$ and the estimate above,
\[
\iint_{Q_r(z_0)} |H_{\mathrm{geom}}|^{3/2}
\le C\left(\iint_{Q_r(z_0)} |\nabla h|^2\right)^{3/4}\left(\iint_{Q_r(z_0)} |\nabla\xi|^2\right)^{3/4}.
\]
By the hypotheses, $\iint_{Q_r(z_0)}|\nabla h|^2\le K_h r^3$ and $\iint_{Q_r(z_0)}|\nabla\xi|^2\le \eps_*^2 r^3$, hence
\[
\iint_{Q_r(z_0)} |H_{\mathrm{geom}}|^{3/2}
\le C\,(K_h r^3)^{3/4}\,(\eps_*^2 r^3)^{3/4}
= C\,K_h^{3/4}\,\eps_*^{3/2}\,r^{9/2}.
\]
Multiplying by $r^{-2}$ and using $r\le r_0$ gives
$r^{-2}\iint_{Q_r(z_0)} |H_{\mathrm{geom}}|^{3/2}\le C K_h^{3/4}\eps_*^{3/2} r_0^{5/2}$.
Taking the supremum over $z_0$ and $0<r\le r_0$ yields the claim.
\end{proof}

{\color{magenta}\begin{assumption}[Log-amplitude control across the vorticity-zero set]\label{assump:D-logamp}
For the running-max ancient element, the regularized log-amplitude gradients are uniformly scale-controlled as $\varepsilon\downarrow 0$:
there exists $K_h<\infty$ such that for every $z_0$ and every $0<r\le 1$,
\[
\sup_{0<\varepsilon\le 1}\ r^{-3}\iint_{Q_r(z_0)} |\nabla\log(\rho+\varepsilon)|^2\,dx\,dt \ \le\ K_h,
\]
where $\rho:=|\omega^\infty|$.
\end{assumption}}

{\color{magenta}\begin{remark}[How \ref{assump:D-logamp} closes the geometric forcing part of (D)]
Assumption~\ref{assump:D-logamp} rules out the codimension-$2$ zero-set pathology in Example~\ref{ex:vmo-fails-at-zeros} at the level relevant for the geometric forcing.
Combined with the global small direction-energy hypothesis in Assumption~\ref{assump:C-liouville},
it yields Carleson smallness of the geometric forcing at small scales via Lemma~\ref{lem:hgeom-carleson-from-energy} (interpreting $\nabla\log\rho$ as the limit of $\nabla\log(\rho+\varepsilon)$ when such a limit exists in $L^2$).
In particular, within item (D), the remaining forcing obstruction is then concentrated entirely in the far-field/tail term.%
\end{remark}}

\subsection{Total forcing Carleson norm}
We define the total forcing Carleson norm as
\[
\|H\|_{C^{3/2}(r_*)} = \sup_{z_0}\ \sup_{0<r\le r_*} r^{-2} \iint_{Q_r(z_0)} |H|^{3/2} \, dx \, dt,
\qquad (0<r_*\le 1).
\]
Assumption~\ref{assump:D-forcing} is precisely the assertion that, for the running-max ancient element, $\|H\|_{C^{3/2}(r_0)}\le \delta^*$ for some sufficiently small $r_0$ and universal threshold $\delta^*$.
Theorem~\ref{thm:forcing_depletion} proves that the near-field commutator/oscillation piece of $H_{\mathrm{sing}}$ is Carleson-small at small scales.
Controlling the remaining tail part of $H_{\mathrm{sing}}$ and the geometric term $H_{\mathrm{geom}}$ is the remaining content of item (D) (see Assumption~\ref{assump:tail-depletion} and Lemma~\ref{lem:log_amplitude}).%

This theorem provides the necessary input for the rigidity analysis of the direction equation: the direction field evolves according to a critical heat flow with a forcing term that is quantitatively small in the relevant scale-invariant space.

\section{Carleson Control and Scaling}\label{sec:carleson}

{\color{magenta}\noindent\textbf{[AI AUDIT: major non-classical input.]}
Any use of extension-energy ``Carleson control'' must be made precise. The classical Caffarelli--Silvestre trace theory provides
\emph{boundedness} of a parabolic Carleson functional \(\|\mathcal E[f]\|_{C}\) \emph{provided one already has} a scale-invariant local enstrophy bound
for \(|\nabla f|^2\).  The manuscript does not currently derive such an enstrophy bound for \(f=|\omega|\) from the suitable weak solution framework,
so that step must be proved separately (or isolated as an explicit hypothesis).}

\begin{definition}[Harmonic extension and local extension energy]\label{def:extension-energy}
Let $f:\R^3\to\R$ be locally square-integrable. Let $F:\R^3\times(0,\infty)\to\R$ denote its harmonic extension to the upper half-space:
\[
-\Delta_{x,z}F=0\quad(z>0),\qquad F(\cdot,0)=f(\cdot).
\]
For $x_0\in\R^3$, $r>0$ define the localized extension energy
\[
E_r[f](x_0)\;:=\;\int_{B_r(x_0)}\int_0^r z\,|\nabla_{x,z}F(x,z)|^2\,dz\,dx.
\]
For a space-time function $f(x,t)$ we write $E_r(x_0,t):=E_r[f(\cdot,t)](x_0)$.
We also define the associated \emph{parabolic Carleson functional}
\[
\|\mathcal E[f]\|_{C}\;:=\;\sup_{z_0=(x_0,t_0)}\ \sup_{0<r\le 1}\ r^{-1}\int_{t_0-r^2}^{t_0} E_r(x_0,t)\,dt.
\]
\end{definition}

\begin{proposition}[Time-averaged extension-energy Carleson bound from an enstrophy bound]\label{thm:carleson-control}
Let $f:\R^3\times I\to\R$ be such that $f(\cdot,t)\in H^1_{\mathrm{loc}}(\R^3)$ for a.e.\ $t\in I$.
Assume there exists $K<\infty$ such that for every $z_0=(x_0,t_0)$ and every $0<r\le 1$ with $(t_0-r^2,t_0)\subset I$,
\[
r^{-1}\iint_{Q_r(z_0)}|\nabla_x f(x,t)|^2\,dx\,dt\ \le\ K.
\]
Then the parabolic extension-energy functional in Definition~\ref{def:extension-energy} is finite and obeys
\[
\|\mathcal E[f]\|_{C}\ \le\ C\,K,
\]
where $C=C(3)$ is a universal dimensional constant.
\end{proposition}

\begin{proof}
For each fixed $t$, the harmonic extension characterization of the $\dot H^{1/2}$ seminorm (Caffarelli--Silvestre \cite{CaffarelliSilvestre2007})
and a standard localization/cutoff argument yield a bound of the form
\[
E_r[f(\cdot,t)](x_0)\ \le\ C \int_{B_{2r}(x_0)} |\nabla_x f(x,t)|^2\,dx,
\]
uniformly for $0<r\le1$.
Integrating in time over $(t_0-r^2,t_0)$ gives
\[
\int_{t_0-r^2}^{t_0}E_r(x_0,t)\,dt
\le C\iint_{Q_{2r}(z_0)}|\nabla_x f|^2\,dx\,dt
\le C\,K\,(2r),
\]
and dividing by $r$ yields the desired Carleson bound.
\end{proof}

{\color{magenta}\begin{remark}[What remains to use \ref{thm:carleson-control} with $f=|\omega|$]
To apply Proposition~\ref{thm:carleson-control} with $f=|\omega|$ (or $f=\omega$ componentwise), one must prove a scale-invariant local enstrophy bound of the form
\(
r^{-1}\iint_{Q_r(z_0)}|\nabla_x|\omega||^2\le K
\)
or a comparable bound on $|\nabla\omega|^2$.
Such an estimate is not produced by the CKN tangent-flow compactness alone; it holds under additional hypotheses (e.g.\ Type~I/enstrophy control) but remains a genuine open input in the present manuscript.%
\end{remark}}

\begin{lemma}[Scaling Invariance]\label{thm:carleson-scaling}
Under the N--S scaling $x\mapsto \lambda x$, $t\mapsto \lambda^2 t$, the functional $\|\mathcal E[f]\|_{C}$ in Definition~\ref{def:extension-energy} is scale-invariant.
\end{lemma}

\begin{corollary}[Carleson stability under blow-up limits]\label{cor:carleson-min}
Let $u^{(k)}$ be a blow-up sequence producing a limit $u^\infty$. Then
\[
\|\mathcal{E}^\infty\|_{C} \le \liminf_{k\to\infty} \|\mathcal{E}^{(k)}\|_{C} \le K_*.
\]
In particular, the Carleson norm is stable along blow-up limits; scaling alone cannot generate arbitrary smallness.
\end{corollary}

\begin{proof}
Lower semicontinuity of the Carleson density under local convergence, together with the uniform bound from Theorem \ref{thm:carleson-control}, yields the liminf inequality. Since the normalized density is scale-invariant, rescaling cannot produce smallness beyond what is present in the sequence.
\end{proof}

\section{Pressure Isotropization and Tail Depletion}\label{sec:pressure}

{\color{magenta}\noindent\textbf{[AI AUDIT: major non-classical input.]}
The pressure/strain ``coercivity'' mechanism and the subsequent spherical-harmonic anisotropy defect estimates appear to be new. They are not classical black-box inputs and currently lack detailed hypotheses and derivations sufficient for referee verification.}

To robustly control the far-field contribution of the stretching, one needs a mechanism that prevents highly anisotropic vorticity distributions at large rescaled radii from generating a persistently positive tail coefficient in the stretching kernel.
One natural approach is to quantify how the pressure term penalizes anisotropic strain, and then to relate that penalty to the $\ell=2$ (quadrupolar) anisotropy sector that drives the far-field stretching coefficient.

\subsection{Tail coefficient and anisotropy defect}

\begin{definition}[Tail stretching coefficient and anisotropy defect]\label{def:tail-coeff-aniso}
For $a,b\in\Sbb^2$ and $\widehat w\in\Sbb^2$, define the canonical traceless quadratic (quadrupolar) angular factor
\[
\Phi(a,b,\widehat w)\ :=\ 3\,(a\cdot \widehat w)\,(b\cdot \widehat w)\;-\;(a\cdot b).
\]
Then $\widehat w\mapsto \Phi(a,b,\widehat w)$ has zero spherical mean for each fixed $a,b$ and lies in the $\ell=2$ spherical-harmonic sector.
{\color{magenta}\noindent\textbf{[AI AUDIT / normalization choice.]}
The actual far-field stretching coefficient can be expressed in terms of an $\ell=2$ angular factor; fixing the explicit model above makes $\mathfrak D_{\mathrm{aniso}}$ a fully concrete ``quadrupole moment'' quantity.}
Given a (time-slice) rescaled vorticity profile $\Omega:\{|w|>1\}\to \R$ and a direction $a\in \Sbb^2$, define the tail coefficient
\[
C_{\mathrm{stretch}}(a,\Omega)
:=\int_{|w|>1} \Phi(a,a,\widehat w)\,\frac{\Omega(w)}{|w|^{3}}\,dw.
\]
Define the anisotropy defect by
\[
\mathfrak D_{\mathrm{aniso}}(\Omega):=\sup_{a\in \Sbb^2}\bigl|C_{\mathrm{stretch}}(a,\Omega)\bigr|.
\]
\end{definition}

{\color{magenta}\begin{example}[Tail anisotropy defect is not automatically small]\label{ex:taildefect-not-automatic}
Even for smooth bounded profiles $\Omega$ on $\{|w|>1\}$, the anisotropy defect need not be small.
Fix $a\in\Sbb^2$ such that $\widehat w\mapsto \Phi(a,a,\widehat w)$ is not identically zero (as is the case for the vortex-stretching kernel).
Let $\chi\in C_c^\infty((1,2))$ be nonnegative and not identically zero, and define
\[
\Omega(w):=\chi(|w|)\,\Phi(a,a,\widehat w).
\]
Then
\[
C_{\mathrm{stretch}}(a,\Omega)=\int_{1<|w|<2}\Phi(a,a,\widehat w)^2\,\frac{\chi(|w|)}{|w|^{3}}\,dw\ >\ 0,
\]
and therefore $\mathfrak D_{\mathrm{aniso}}(\Omega)\ge |C_{\mathrm{stretch}}(a,\Omega)|>0$.
In particular, a statement like $\mathfrak D_{\mathrm{aniso}}(\Omega_{z_0,r})\to0$ requires a genuine dynamical/isotropization mechanism; it does not follow from boundedness or smoothness of $\Omega$ alone.%
\end{example}}

\begin{definition}[Exterior rescaled vorticity profile at a basepoint]\label{def:Omega-z0r}
For a space-time basepoint $z_0=(x_0,t_0)$ and a scale $r>0$, we define the exterior rescaled (time-slice) vorticity magnitude profile by
\[
\Omega_{z_0,r}(w):=\rho(x_0+r w,\ t_0)\qquad (|w|>1),
\]
where $\rho(x,t):=|\omega(x,t)|$.
\end{definition}

\begin{theorem}[Pressure Coercivity]\label{thm:pressure-coercivity}
Let $S=\tfrac12(\nabla u + \nabla u^T)$ and $S_{dev}=S - \tfrac13 (\operatorname{tr}S) I$. For any $R>0$ and cutoff $\phi\in C_c^\infty(B_{2R})$ with $\phi\equiv 1$ on $B_R$, one has
\[
\frac12 \frac{d}{dt}\int_{B_R} |S_{dev}|^2 + \frac{\nu}{2} \int_{B_R} |\nabla S_{dev}|^2
\le C \|u\|_{L^3(B_{2R})}^4 \int_{B_R} |S_{dev}|^2 + C R^{-2} \int_{B_{2R}} |S_{dev}|^2.
\]
\end{theorem}

\begin{proof}
For a smooth divergence-free solution, set $A:=\nabla u$.
Differentiating the Navier--Stokes equation gives the matrix evolution
\[
\partial_t A + u\cdot\nabla A - \nu \Delta A = -A^2 - \nabla^2 p.
\]
Let $S=\tfrac12(A+A^T)$ and let $P_{dev}$ denote the traceless (deviatoric) projection. Taking the symmetric part and then the deviatoric part yields
\[
\partial_t S_{dev} + u\cdot\nabla S_{dev} - \nu \Delta S_{dev}
= -P_{dev}\bigl((A^2)^{sym}\bigr) - P_{dev}\bigl((\nabla^2 p)^{sym}\bigr).
\]
Test this equation against $S_{dev}\phi^2$ and integrate over $\R^3$.
The transport term is treated using $\nabla\cdot u=0$ and produces only cutoff errors controlled by $CR^{-2}\int_{B_{2R}}|S_{dev}|^2$.
The diffusion term yields $\nu\int |\nabla S_{dev}|^2\phi^2$ up to cutoff errors of the same type.

For the pressure term, use the Poisson equation $\Delta p = -\nabla\cdot\nabla\cdot(u\otimes u)$ and Calder\'on--Zygmund estimates to obtain
\[
\|\nabla^2 p\|_{L^{3/2}(B_R)} \ \lesssim\ \|u\otimes u\|_{L^{3/2}(B_{2R})}\ \lesssim\ \|u\|_{L^{3}(B_{2R})}^2.
\]
By H\"older and Gagliardo--Nirenberg on $B_R$,
\[
\Bigl|\int_{B_R} S_{dev}:(\nabla^2 p)\Bigr|
\ \le\ \|S_{dev}\|_{L^{3}(B_R)}\,\|\nabla^2 p\|_{L^{3/2}(B_R)}
\ \le\ \frac{\nu}{4}\|\nabla S_{dev}\|_{L^2(B_R)}^2 + C\nu^{-1}\|u\|_{L^3(B_{2R})}^4 \|S_{dev}\|_{L^2(B_R)}^2.
\]
The quadratic term in $A$ is treated similarly (bounding it by $\int |A|\,|S_{dev}|^2\phi^2$ and using the same Sobolev interpolation to absorb a portion of $\|\nabla S_{dev}\|_2^2$), producing a contribution of the same form as above.
Collecting terms and absorbing $\frac{\nu}{4}\|\nabla S_{dev}\|_2^2$ yields the stated inequality.
\end{proof}

{\color{magenta}\begin{lemma}[Anisotropy defect controlled by deviatoric strain (multipole reduction)]\label{lem:defect-vs-strain}
Let $\Omega$ be a rescaled vorticity profile on the annulus $\{|w|>1\}$ and define $\mathfrak D_{\mathrm{aniso}}(\Omega)$ as in Definition~\ref{def:tail-coeff-aniso}.
Then there exists a universal constant $C$ such that, for the associated rescaled strain field $S$ (obtained from $\Omega$ by the Biot--Savart/Riesz-transform representation),
\[
\mathfrak D_{\mathrm{aniso}}(\Omega)^2 \le C \int_{B_1} |S_{dev}(x)|^2\,dx.
\]
\end{lemma}}

{\color{magenta}\begin{proof}[Proof (multipole reduction at the $\ell=2$ level)]
Define the (symmetric, traceless) quadrupole tensor
\[
Q(\Omega)\ :=\ \int_{|w|>1}\Bigl(3\,\widehat w\otimes \widehat w - I\Bigr)\,\frac{\Omega(w)}{|w|^{3}}\,dw.
\]
Then for any unit vector $a\in\Sbb^2$,
\[
C_{\mathrm{stretch}}(a,\Omega)
=\int_{|w|>1}\Phi(a,a,\widehat w)\,\frac{\Omega(w)}{|w|^{3}}\,dw
= a\cdot Q(\Omega)\,a,
\]
and therefore $\mathfrak D_{\mathrm{aniso}}(\Omega)=\sup_{|a|=1}|a\cdot Q(\Omega)a|\le |Q(\Omega)|$.

\smallskip
\noindent
Let $u$ be the velocity field on $B_1$ induced by the exterior vorticity profile (via the Biot--Savart/Riesz-transform representation), and let
$S=\tfrac12(\nabla u + \nabla u^T)$ and $S_{dev}=S-\tfrac13(\operatorname{tr}S)I$.
Since the vorticity is supported in $\{|w|>1\}$, one has $\Delta u=0$ and $\nabla\cdot u=0$ on $B_1$, hence each component of $S_{dev}$ is harmonic on $B_1$.
Moreover, the leading $\ell=2$ term in the multipole expansion of the Biot--Savart kernel shows that $S_{dev}(0)$ is proportional to the quadrupole tensor $Q(\Omega)$; equivalently, there exists a universal constant $c_0\neq 0$ such that
\[
a\cdot S_{dev}(0)\,a\ =\ c_0\,C_{\mathrm{stretch}}(a,\Omega)
\qquad\text{for all }a\in\Sbb^2,
\]
and hence $|Q(\Omega)|\le C\,|S_{dev}(0)|$.

\smallskip
\noindent
Finally, applying the mean-value inequality for harmonic functions componentwise to $S_{dev}$ on $B_1$ gives
\[
|S_{dev}(0)|^2 \le C \int_{B_1} |S_{dev}(x)|^2\,dx.
\]
Combining the previous inequalities yields the claim.

{\color{magenta}\noindent\textbf{[AI AUDIT / remaining input for full referee-checkability.]}}
To make the above fully explicit, one should either (a) write the multipole expansion of $\nabla u$ at $x=0$ in terms of $Q(\Omega)$ directly, or (b) cite a standard Biot--Savart multipole formula identifying the $\ell=2$ coefficient of the interior harmonic strain with the exterior quadrupole moment.
\end{proof}}

{\color{magenta}\begin{assumption}[Tail depletion (minimal form used later)]\label{assump:tail-depletion}
For the running-max ancient element, the far-field/tail part of the stretching forcing is small in the critical Carleson norm at sufficiently small scales:
\[
\forall \varepsilon>0\ \exists r_0>0\ \text{such that}\ \sup_{z_0}\ \sup_{0<r\le r_0}\ r^{-2}\iint_{Q_r(z_0)} |H_{\mathrm{tail}}|^{3/2}\,dx\,dt \le \varepsilon.
\]
\end{assumption}}

{\color{magenta}\begin{remark}[What would be needed to prove Assumption~\ref{assump:tail-depletion}]
One plausible route is:
(i) make the ``tail coefficient'' reduction fully quantitative: relate $H_{\mathrm{tail}}$ on $Q_r(z_0)$ to an explicit $\ell=2$ far-field coefficient of the exterior profile $\Omega_{z_0,r}$ (as in Definition~\ref{def:tail-coeff-aniso}), with controllable remainder terms,
(ii) make Lemma~\ref{lem:defect-vs-strain} fully referee-checkable (multipole reduction),
(iii) prove a \emph{vanishing} small-scale control of $\iint_{Q_1}|S_{dev}|^2$ along the running-max ancient element (uniformly in basepoints), from a pressure-driven isotropization mechanism beyond the classical energy inequality.
At present, step (iii) is not established in this manuscript; it is the core remaining obstruction in the tail part of item (D).

\medskip
\noindent{\color{magenta}\textbf{[AI AUDIT / circularity risk to watch.]}}
The local coercivity estimate in Theorem~\ref{thm:pressure-coercivity} carries a factor $\|u\|_{L^3(B_{2R})}^4$.
To deduce \emph{vanishing} of $\iint_{Q_1}|S_{dev}|^2$ as $R\downarrow 0$ uniformly in basepoints from that inequality, one needs a mechanism that controls this scale-critical $L^3$ factor (e.g.\ via a Galilean gauge plus genuinely uniform smallness, or another monotonicity/compactness input).
Without such a mechanism, ``pressure isotropization $\Rightarrow$ tail depletion'' risks re-introducing assumptions comparable in strength to classical scale-critical regularity hypotheses.%
\end{remark}}

{\color{blue}\begin{remark}[What the running-max bounds give for $S_{dev}$]\label{rem:running-max-strain}
For the running-max ancient element, bounded vorticity $\|\omega^\infty\|_{L^\infty}\le 2$ gives:
\begin{itemize}
\item \textbf{Boundedness of strain:} By Lemma~\ref{lem:Linfty-vort-smooth}, $u^\infty$ is smooth on each compact cylinder, and the velocity gradient $\nabla u^\infty$ is bounded locally. Hence the deviatoric strain $S_{dev}$ is \emph{bounded} on compact sets.
\item \textbf{No automatic smallness:} Bounded vorticity does NOT imply that $\|S_{dev}\|$ becomes small at small scales. In particular, even with $|\omega|\le 2$ everywhere, the strain could remain order-1 as $r\to 0$.
\end{itemize}
The tail depletion hypothesis (Assumption~\ref{assump:tail-depletion}) is precisely the \emph{extra} statement that $S_{dev}$ not only stays bounded but becomes \emph{small} in the sense that its $\ell=2$ multipole moments (captured by the tail coefficient $C_{\mathrm{stretch}}$) vanish at small scales.

\smallskip
\noindent\textbf{Possible mechanisms (speculative):}
\begin{enumerate}
\item \emph{Backward-time decay.} For an ancient solution, as $t\to-\infty$, the flow might approach a ``simpler'' state with smaller $S_{dev}$. This could propagate forward to give small-scale control at any fixed time.
\item \emph{Direction-constancy feedback.} If the Liouville mechanism (C) is already known to force $\xi\to \text{const}$, then $\omega=\rho\xi$ becomes constant-direction, which simplifies the strain structure. However, this creates a circular dependence: (C) needs (D) to get forcing smallness, and (D) might need (C) to get strain vanishing.
\item \emph{Direct energy argument.} If the direction energy $E(z_0,r)$ is globally small (part of the (C) hypothesis), then the vortex stretching mechanism is weak, which might imply that $S_{dev}$ stays close to the ``2D-like'' regime where the tail contribution is small.
\end{enumerate}
At present, none of these is fully developed in the manuscript.
\end{remark}}

{\color{blue}\begin{lemma}[Stretching simplifies in the constant-direction case]\label{lem:constdir-stretching}
In the constant-direction regime with $\omega=(0,0,\rho(x_h,t))$ and $u_3=a(t)+b(t)x_3$, the vortex stretching term simplifies to
\[
S\cdot\omega\ =\ b(t)\,\omega\ =\ (0,0,\,b\rho).
\]
\end{lemma}

\begin{proof}
Since $\omega=(0,0,\rho)$ with $\rho$ independent of $x_3$:
\begin{itemize}
\item The horizontal velocity $u_h=(u_1,u_2)$ is given by the 2D Biot--Savart law applied to $\rho$, and is independent of $x_3$ (Remark~\ref{rem:constdir-uc}).
\item The vertical velocity $u_3=a(t)+b(t)x_3$ depends only on $x_3$ and $t$.
\end{itemize}
The velocity gradient has the structure:
\[
\nabla u\ =\ \begin{pmatrix} \partial_1 u_1 & \partial_2 u_1 & 0 \\ \partial_1 u_2 & \partial_2 u_2 & 0 \\ 0 & 0 & b \end{pmatrix}.
\]
The symmetric part (strain) is
\[
S\ =\ \frac12(\nabla u+\nabla u^T)\ =\ \begin{pmatrix} \partial_1 u_1 & \tfrac12(\partial_1 u_2+\partial_2 u_1) & 0 \\ \tfrac12(\partial_1 u_2+\partial_2 u_1) & \partial_2 u_2 & 0 \\ 0 & 0 & b \end{pmatrix}.
\]
Computing $S\cdot\omega=S\cdot(0,0,\rho)^T$:
\[
(S\cdot\omega)_i\ =\ S_{i3}\,\rho\ =\ \begin{cases} 0 & i=1,2 \\ b\,\rho & i=3 \end{cases}.
\]
Thus $S\cdot\omega=(0,0,b\rho)=b\,\omega$.
\end{proof}}

{\color{magenta}\begin{remark}[Connection between (D) and (E)]\label{rem:D-E-connection}
Lemma~\ref{lem:constdir-stretching} reveals an important connection:
\begin{itemize}
\item In the constant-direction case, if $b=0$ (the conclusion of hypothesis (E1)), then $S\cdot\omega=0$---the vortex stretching \emph{vanishes identically}.
\item With zero stretching, the vorticity equation becomes purely diffusive: $\partial_t\omega=\nu\Delta\omega$.
\item This drastically simplifies the tail depletion problem: with no stretching, there is no far-field contribution from $S\cdot\omega$, and the tail term should vanish.
\end{itemize}
This suggests a potential \textbf{bootstrap}: if (E1) can be established (i.e.\ $b=0$), then (D) becomes significantly easier, which in turn makes (C) easier.
Conversely, if (C) is assumed (direction constancy), then we are in the constant-direction regime, and in the running-max setting Lemma~\ref{lem:E1-b-negative-impossible} forces $b\equiv 0$ automatically.

{\color{magenta}\noindent\textbf{[Update.]}
In the running-max refactor, (E1) is now automatic: Lemma~\ref{lem:E1-b-negative-impossible} (together with Lemma~\ref{lem:linear-mode-ODE}) forces $b\equiv 0$ once $\xi^\infty$ is constant.
Thus, after (C) one is already in the zero-stretching regime, and the remaining (E) obstruction is (E2): placing the reduced 2D flow in a Liouville class (bounded velocity / decay / finite enstrophy) sufficient to invoke a 2D Liouville theorem.}
\end{remark}}

{\color{blue}\begin{lemma}[2D enstrophy evolution in the constant-direction case]\label{lem:constdir-enstrophy}
In the constant-direction setting with $\omega=(0,0,\rho(x_h,t))$ and $u_3=a(t)+b(t)x_3$, for any $R>0$ the localized 2D enstrophy
\[
\Omega_R(t)\ :=\ \int_{|x_h|<R}|\rho(x_h,t)|^2\,dx_h
\]
satisfies
\begin{equation}\label{eq:enstrophy-evol}
\frac{d}{dt}\Omega_R\ \le\ -2\nu\int_{|x_h|<R}|\nabla_h\rho|^2\,dx_h\ +\ b(t)\,\Omega_R\ +\ \Phi_R(t),
\end{equation}
where $\Phi_R(t)$ is a boundary flux term satisfying $|\Phi_R|\le C(R)$ for smooth solutions with bounded vorticity.
\end{lemma}

\begin{proof}
The vorticity equation for $\rho$ is (using Lemma~\ref{lem:constdir-stretching}):
\[
\partial_t\rho + u_h\cdot\nabla_h\rho\ =\ \nu\Delta_h\rho + b\rho.
\]
Multiplying by $2\rho$ and integrating over $\{|x_h|<R\}$:
\[
\frac{d}{dt}\Omega_R\ =\ 2\int\rho\,\partial_t\rho
\ =\ 2\nu\int\rho\,\Delta_h\rho\ -\ 2\int\rho\,u_h\cdot\nabla_h\rho\ +\ 2b\int\rho^2.
\]
The diffusion term gives (via integration by parts):
$2\nu\int_{|x_h|<R}\rho\,\Delta_h\rho = -2\nu\int|\nabla_h\rho|^2 + \text{(boundary terms)}.$
For the advection term we use $\nabla_h\cdot u_h = -\partial_3 u_3 = -b$ (from 3D incompressibility and $u_3=a+bx_3$):
\[
-2\int_{|x_h|<R}\rho\,u_h\cdot\nabla_h\rho
=-\int_{|x_h|<R} u_h\cdot\nabla_h(\rho^2)
= \int_{|x_h|<R}(\nabla_h\cdot u_h)\,\rho^2\ +\ \text{(boundary terms)}
= -b\,\Omega_R\ +\ \text{(boundary terms)}.
\]
The stretching term gives $2b\,\Omega_R$, so the net coefficient on $\Omega_R$ is $b\,\Omega_R$.
Collecting boundary contributions into $\Phi_R$ yields \eqref{eq:enstrophy-evol}.
\end{proof}}

{\color{blue}\begin{remark}[Enstrophy growth backward in time when $b<0$ (now excluded in running-max)]\label{rem:enstrophy-backward}
Ignoring boundary fluxes (valid heuristically for localized vorticity or controlled spatial decay), Lemma~\ref{lem:constdir-enstrophy} gives
\[
\frac{d}{dt}\Omega_R\ \le\ b(t)\,\Omega_R.
\]
For $b(t)<0$ this implies $\Omega_R$ is \emph{decreasing} forward in time.
Going \emph{backward} in time ($\tau=-t\to+\infty$), we have
\[
\frac{d}{d\tau}\Omega_R(\tau)\ \ge\ |b(-\tau)|\,\Omega_R(\tau).
\]
From the ODE solution $b(t)=b_0/(1+b_0 t)$, for large $|\tau|$ with $b_0<0$ one has $|b(-\tau)|\sim 1/\tau$.
Hence $\frac{d}{d\tau}\Omega_R \gtrsim (1/\tau)\Omega_R$, which integrates to
\[
\Omega_R(\tau)\ \gtrsim\ \Omega_R(1)\cdot\tau
\qquad\text{as }\tau\to+\infty.
\]
Thus if $\Omega_R(1)>0$ (nontrivial vorticity at $t=-1$), the enstrophy grows at least like $\tau$ backward in time.

\smallskip
\noindent\textbf{Implications for ancient solutions.}
For the running-max ancient element with $|\omega(0,0)|=1$, we have $\Omega_R(0)>0$ for small $R$.
In the running-max refactor, however, the case $b_0<0$ is excluded outright by Lemma~\ref{lem:E1-b-negative-impossible} using the global vorticity bound. We keep this computation only as intuition about why $b<0$ would create backward growth.
\end{remark}}

{\color{blue}\begin{lemma}[Global 2D enstrophy identity when $b=0$]\label{lem:pure-diffusion-enstrophy}
In the constant-direction case with $b\equiv 0$, the vorticity $\rho(x_h,t)$ satisfies the 2D advection--diffusion equation
\[
\partial_t\rho + u_h\cdot\nabla_h\rho\ =\ \nu\Delta_h\rho,
\]
where $u_h$ is the 2D velocity recovered from $\rho$ via the 2D Biot--Savart law.
Assume that for some $t_1<t_2\le 0$ one has
\[
\rho\in L^\infty\bigl((t_1,t_2);L^2(\R^2)\bigr)\cap L^2\bigl((t_1,t_2);\dot H^1(\R^2)\bigr).
\]
Then the global enstrophy satisfies the exact identity
\[
\|\rho(\cdot,t_2)\|_{L^2(\R^2)}^2
\ +\ 2\nu\int_{t_1}^{t_2}\|\nabla_h\rho(\cdot,t)\|_{L^2(\R^2)}^2\,dt
\ =\ \|\rho(\cdot,t_1)\|_{L^2(\R^2)}^2,
\]
and in particular $t\mapsto \|\rho(\cdot,t)\|_{L^2(\R^2)}$ is non-increasing on $(t_1,t_2)$.
\end{lemma}

\begin{proof}
Multiply the equation by $2\rho$ and integrate over $\R^2$.
The advection term vanishes by divergence-free: $\int_{\R^2}u_h\cdot\nabla_h(\rho^2)\,dx_h=0$.
The diffusion term gives $2\nu\int \rho\,\Delta_h\rho=-2\nu\int|\nabla_h\rho|^2$ by integration by parts.
The stated regularity assumptions justify the integrations by parts and time integration.
\end{proof}}

{\color{magenta}\begin{remark}[Closing (E) via enstrophy: what it would actually require]\label{rem:E-enstrophy-gap}
Lemma~\ref{lem:pure-diffusion-enstrophy} gives the standard 2D enstrophy dissipation identity \emph{provided} the reduced vorticity lies in the global class
$\rho\in L^\infty_tL^2_x\cap L^2_t\dot H^1_x$ on some time interval.

This would be a powerful closing mechanism in the $b=0$ regime, but the running-max ancient element is obtained only with \emph{local} compactness and does not (as written) provide any global $L^2(\R^2)$-type enstrophy control.
Accordingly, the enstrophy identity does not by itself close (E) unless one adds an explicit global hypothesis transferring a suitable vorticity/enstrophy bound from the pre-blow-up solution to the ancient limit.
\end{remark}}

{\color{blue}\begin{remark}[Scaling behavior of the linear mode]\label{rem:E-linear-mode-scaling}
For the running-max ancient element, consider how the linear-in-$x_3$ mode behaves under parabolic rescaling.
If the original pre-blow-up solution has $u_3(x,t)=a(t)+b(t)x_3$ near a point $(x_0,t_0)$, then the rescaled solution with scale $\lambda$ has:
\[
u_3^{(\lambda)}(y,s)\ =\ \lambda\,u_3(x_0+\lambda y,\,t_0+\lambda^2 s)
\ =\ \lambda\,a(t_0+\lambda^2 s)\ +\ \lambda^2\,b(t_0+\lambda^2 s)\,y_3.
\]
The coefficient of $y_3$ in the rescaled velocity is $\lambda^2 b(t_0+\lambda^2 s)$.

\smallskip
\noindent\textbf{Key observation:} For the ancient-element limit ($\lambda\to 0$), the coefficient of $y_3$ is
\[
\lim_{\lambda\to 0}\lambda^2\,b(t_0+\lambda^2 s)\ =\ \lim_{\tau\to 0^+}\tau\,b(t_0+\tau\,s),
\]
where $\tau=\lambda^2$. For this limit to be nonzero (i.e.\ for the ancient element to have $b\neq 0$), the original solution must have $b(t_0+\tau s)\gtrsim \tau^{-1}$ on the rescaling time window.

{\color{magenta}\noindent\textbf{[AI AUDIT / relation to vorticity normalization.]}}
In the running-max construction, $\lambda_k$ is chosen by vorticity normalization ($\lambda_k^2\sim 1/\|\omega(\cdot,t_k)\|_{L^\infty}$), and $b$ is \emph{not} directly controlled by $\omega$.
Thus, turning the heuristic “$\lambda^2 b$ survives the limit only if $b$ blows up like $\lambda^{-2}$” into a contradiction requires an additional estimate relating $|b|=|\partial_3 u_3|$ to the vorticity growth along the running-max sequence.

\smallskip
\noindent\textbf{Implication.}
A nonzero $b$ in the ancient element means that along the blow-up/rescaling window one has $b(t_0+\tau s)\gtrsim \tau^{-1}$.
In the running-max normalization, the parabolic scale satisfies $\tau=\lambda^2\sim 1/\|\omega(\cdot,t_k)\|_{L^\infty}$, so this corresponds to
\[
|b(t_k+\lambda_k^2 s)|\ \gtrsim\ \|\omega(\cdot,t_k)\|_{L^\infty}
\]
on the rescaling window (at least along a subsequence).
This is a strong constraint because $b=\partial_3 u_3$ is a \emph{gradient} component not directly controlled by $\omega$.
Any attempt to rule out $b\neq 0$ in the ancient element must therefore supply an additional mechanism relating $\partial_3 u_3$ to the vorticity growth (or to another scale-critical quantity controlled in the running-max blow-up).
\end{remark}}

\section{The Directional Liouville Theorem}

\subsection{The Critical Drift--Diffusion System}
We have reduced the problem to the analysis of the ancient direction field $\xi^\infty$ satisfying
\begin{equation}\label{eq:DDE}
{\color{magenta}\partial_t \xi - \Delta \xi + u \cdot \nabla \xi = |\nabla \xi|^2 \xi + H, \quad |\xi|=1, \quad H \cdot \xi = 0.}
\end{equation}
{\color{magenta}\noindent\textbf{[AI AUDIT.]}
Unlike the CKN tangent-flow setting, the running-max ancient element satisfies $\omega^\infty\in L^\infty$.
Lemmas~\ref{lem:drift-bmo-from-vorticity}--\ref{lem:drift-local-Lp} then yield an admissible \emph{local Serrin drift bound} after subtracting a ball average (Galilean gauge), so the drift hypothesis needed for the absorption step in the DDE $\varepsilon$-regularity iteration is available in this refactor.
The remaining non-classical content of item (C) is therefore concentrated in: (i) writing the critical drift/Carleson forcing $\varepsilon$-regularity theorem in fully referee-checkable form, and (ii) verifying the global small-energy hypotheses needed for the Liouville step.}
Here, $H$ satisfies the smallness condition $\|H\|_{C^{3/2}} \le \delta^*$.

{\color{magenta}\begin{lemma}[Bounded vorticity gives a uniform $\BMO$ bound on $\nabla u$]\label{lem:drift-bmo-from-vorticity}
Let $u(\cdot,t)$ be divergence-free on $\R^3$ with vorticity $\omega(\cdot,t)=\curl u(\cdot,t)\in L^\infty(\R^3)$.
Then $\nabla u(\cdot,t)\in \BMO(\R^3)$ and
\[
\|\nabla u(\cdot,t)\|_{\BMO(\R^3)}\ \le\ C\,\|\omega(\cdot,t)\|_{L^\infty(\R^3)},
\]
where $C$ is a universal dimensional constant.
In particular, for the running-max ancient element of Lemma~\ref{lem:ancient-limit-runningmax} one has
\(\nabla u^\infty\in L^\infty\big((-\infty,0];\BMO(\R^3)\big)\).
\end{lemma}}

\begin{proof}
This is classical. Since $\nabla\cdot u=0$ and $\omega=\curl u$, one may write
\[
u=\curl(-\Delta)^{-1}\omega,
\]
so each component of $\nabla u$ is a finite linear combination of Riesz transforms applied to components of $\omega$ (a Calder\'on--Zygmund operator).
Calder\'on--Zygmund operators map $L^\infty(\R^3)$ boundedly into $\BMO(\R^3)$; see \cite{Stein1993}.
\end{proof}

{\color{magenta}\begin{remark}[How \ref{lem:drift-bmo-from-vorticity} could reduce the drift gap in (C)]
Lemma~\ref{lem:drift-bmo-from-vorticity} is a classical harmonic-analysis consequence of the Biot--Savart law:
each component of $\nabla u$ is a Calder\'on--Zygmund transform of $\omega$, and CZ operators map $L^\infty$ to $\BMO$.
On a fixed ball, $\BMO$ embeds into $L^p$ for every $1\le p<\infty$ (John--Nirenberg), so bounded vorticity yields strong \emph{local} integrability of $\nabla u$.
Turning this information into the precise drift control needed to close the DDE Caccioppoli/Campanato iteration
(in particular, to absorb the cutoff-error drift terms without assuming a Serrin class) is not supplied here and remains part of item (C).%
\end{remark}}

{\color{magenta}\begin{lemma}[Local Serrin drift from bounded vorticity, modulo a Galilean gauge]\label{lem:drift-local-Lp}
Let $u(\cdot,t)$ be divergence-free on $\R^3$ with vorticity $\omega(\cdot,t)=\curl u(\cdot,t)\in L^\infty(\R^3)$.
Fix $x_0\in\R^3$, a radius $r>0$, and $1\le p<\infty$. Define the spatial average
\[
c_{x_0,r}(t):=\frac{1}{|B_r|}\int_{B_r(x_0)}u(x,t)\,dx.
\]
Then for a.e.\ $t$,
\[
\|u(\cdot,t)-c_{x_0,r}(t)\|_{L^p(B_r(x_0))}\ \le\ C_p\, r^{1+3/p}\,\|\omega(\cdot,t)\|_{L^\infty(\R^3)},
\]
where $C_p$ depends only on $p$ (and dimension).
In particular, if $\omega\in L^\infty(\R^3\times I)$ on a time interval $I$, then $u-c_{x_0,r}\in L^\infty(I;L^p(B_r(x_0)))$ with the same bound.
\end{lemma}}

{\color{magenta}\begin{proof}[Proof sketch]
By Lemma~\ref{lem:drift-bmo-from-vorticity}, $\|\nabla u(\cdot,t)\|_{\BMO}\lesssim \|\omega(\cdot,t)\|_{L^\infty}$.
By John--Nirenberg, $\|\nabla u(\cdot,t)\|_{L^p(B_r(x_0))}\le C_p\,|B_r|^{1/p}\,\|\nabla u(\cdot,t)\|_{\BMO}\le C_p\,r^{3/p}\,\|\omega(\cdot,t)\|_{L^\infty}$.
Poincar\'e's inequality yields $\|u(\cdot,t)-c_{x_0,r}(t)\|_{L^p(B_r(x_0))}\le C\,r\,\|\nabla u(\cdot,t)\|_{L^p(B_r(x_0))}$.
Combining gives the stated bound.
\end{proof}}

{\color{magenta}\begin{lemma}[Bounded vorticity implies local smoothness (via Serrin)]\label{lem:Linfty-vort-smooth}
Let $(u,p)$ be a suitable weak solution of the 3D Navier--Stokes equations on a cylinder $Q_{2r}(z_0)$ and assume $\omega=\curl u\in L^\infty(Q_{2r}(z_0))$.
Then $u$ is smooth on $Q_r(z_0)$.
In particular, the running-max ancient element $(u^\infty,p^\infty)$ from Lemma~\ref{lem:ancient-limit-runningmax} is smooth on every compact cylinder in $\R^3\times(-\infty,0)$.
\end{lemma}}

{\color{magenta}\begin{proof}[Proof sketch]
Fix $p>3$.
By Lemma~\ref{lem:drift-local-Lp}, after subtracting the ball average $c_{x_0,2r}(t)$ (a Galilean gauge), one has
$u-c_{x_0,2r}\in L^\infty\bigl((t_0-(2r)^2,t_0);L^p(B_{2r}(x_0))\bigr)$.
The local Serrin interior regularity criterion (see Serrin \cite{Serrin1962} and standard local-energy refinements for suitable weak solutions) then implies that $u$ is smooth on the smaller cylinder $Q_r(z_0)$.
Since subtracting $c_{x_0,2r}(t)$ is a Galilean change of coordinates, it does not affect regularity.
\end{proof}}

{\color{magenta}\begin{remark}[Why \ref{lem:drift-local-Lp} matters for (C)]
Lemma~\ref{lem:drift-local-Lp} shows that bounded vorticity yields \emph{local} control of the drift $u$ in $L^p$ after subtracting a ball average $c(t)$.
Since adding/subtracting a spatially constant vector field corresponds to a Galilean change of coordinates, such a gauge choice is natural in local parabolic arguments.
In particular, choosing any $p>3$ gives an admissible \emph{local Serrin class} with $q=\infty$ (since $2/q+3/p=3/p<1$).
Thus, for the running-max ancient element (where $\omega^\infty\in L^\infty$), the local Serrin drift hypothesis used in Lemma~\ref{lem:dde-drift-absorb} is available after this Galilean gauge.
What remains open in item (C) is to fully write the Campanato iteration/embedding steps in a referee-checkable way and to verify the global small-energy hypotheses needed for the Liouville step.%
\end{remark}}

{\color{magenta}\begin{lemma}[Galilean-gauged rescaled drift is small at small scales under bounded vorticity]\label{lem:drift-small-rescaled}
Let $u(\cdot,t)$ be divergence-free on $\R^3$ with vorticity $\omega(\cdot,t)=\curl u(\cdot,t)\in L^\infty(\R^3)$.
Fix $x_0\in\R^3$, $r>0$, and $1\le p<\infty$, and set
\[
c_{x_0,r}(t):=\frac{1}{|B_r|}\int_{B_r(x_0)}u(x,t)\,dx.
\]
Define the Galilean-gauged rescaled drift on $Q_1(0,0)$ by
\[
\widetilde u^{(r)}(x,t):=r\Bigl(u(x_0+r x,\ t_0+r^2 t)-c_{x_0,r}(t_0+r^2 t)\Bigr).
\]
Then for a.e.\ $t$,
\[
\|\widetilde u^{(r)}(\cdot,t)\|_{L^p(B_1)}\ \le\ C_p\,r^2\,\|\omega(\cdot,t_0+r^2 t)\|_{L^\infty(\R^3)},
\]
and in particular
\[
\|\widetilde u^{(r)}\|_{L^\infty((-1,0);L^p(B_1))}\ \le\ C_p\,r^2\,\|\omega\|_{L^\infty(\R^3\times(t_0-r^2,t_0))}.
\]
\end{lemma}}

{\color{magenta}\begin{proof}[Proof sketch]
By Lemma~\ref{lem:drift-local-Lp}, for a.e.\ $s\in(t_0-r^2,t_0)$,
\[
\|u(\cdot,s)-c_{x_0,r}(s)\|_{L^p(B_r(x_0))}\ \le\ C_p\,r^{1+3/p}\,\|\omega(\cdot,s)\|_{L^\infty(\R^3)}.
\]
Scaling the $L^p$ norm gives
$\|\widetilde u^{(r)}(\cdot,t)\|_{L^p(B_1)}=r^{1-3/p}\|u(\cdot,t_0+r^2 t)-c_{x_0,r}(t_0+r^2 t)\|_{L^p(B_r(x_0))}$,
which combined with the previous bound yields the claim.
\end{proof}}

{\color{magenta}\begin{remark}[How \ref{lem:drift-small-rescaled} feeds into $\varepsilon$-regularity in the running-max setting]
Lemma~\ref{lem:drift-small-rescaled} provides a genuinely scale-improving drift estimate:
after the natural Galilean gauge and parabolic rescaling to $Q_1$, the drift norm decays like $r^2$.
Thus, for the running-max ancient element (where $\|\omega^\infty\|_{L^\infty}\le 2$), the gauged rescaled drift
$\widetilde u^{(r)}$ can be made arbitrarily small in $L^\infty_tL^p_x(Q_1)$ by choosing $r$ sufficiently small.
This is stronger than mere ``Serrin admissibility'' and supports a perturbative drift-absorption step in the one-step decay estimate (Lemma~\ref{lem:decay}).
The remaining non-classical content in (C) is therefore concentrated in making the Campanato iteration fully referee-checkable in the presence of the geometric nonlinearity $|\nabla\xi|^2\xi$
and critical Carleson forcing (and in verifying whatever small-energy hypotheses are required to start the iteration uniformly in basepoints).%
\end{remark}}

\subsection{Energy Decay Estimates}
To prove rigidity, we establish decay of the scale-invariant energy
\(
E(z_0,r) := r^{-3} \iint_{Q_r(z_0)} |\nabla \xi|^2.
\)
The standard route is a Caccioppoli inequality for $\nabla\xi$, an absorption of the drift term under an admissible Serrin bound, and a Campanato iteration.

\begin{lemma}[Caccioppoli inequality for the DDE (requires $H\in L^2$)]\label{lem:dde-caccioppoli}
Let $\xi$ solve \eqref{eq:DDE} on $Q_r(z_0)$ with $|\xi|=1$ and $H\cdot \xi=0$.
Assume additionally that $H\in L^2(Q_r(z_0))$.
Let $\phi\in C_c^\infty(Q_r(z_0))$ satisfy $\phi\equiv 1$ on $Q_{r/2}(z_0)$ and $|\nabla\phi|\lesssim r^{-1}$, $|\partial_t\phi|\lesssim r^{-2}$.
Then
\[
\iint_{Q_{r/2}(z_0)} |\nabla^2 \xi|^2
\ \le\ C r^{-2} \iint_{Q_r(z_0)} |\nabla \xi|^2
\ +\ C \iint_{Q_r(z_0)} |u|^2\,|\nabla \xi|^2
\ +\ C \iint_{Q_r(z_0)} |H|^2,
\]
where $C$ is a universal constant.
\end{lemma}

\begin{proof}
This is standard. One tests \eqref{eq:DDE} against $-\Delta(\phi^2\xi)$, integrates by parts in space-time, and uses the sphere constraint $|\xi|=1$ (in particular $\xi\cdot\Delta\xi=-|\nabla\xi|^2$) together with $H\cdot\xi=0$ to eliminate normal components. Cutoff terms are controlled using $|\nabla\phi|\lesssim r^{-1}$, $|\partial_t\phi|\lesssim r^{-2}$.
\end{proof}

{\color{magenta}\begin{remark}[Forcing in the critical Carleson regime]\label{rem:forcing-caccioppoli-gap}
Lemma~\ref{lem:dde-caccioppoli} bounds the forcing contribution by Cauchy--Schwarz and therefore requires $H\in L^2$.
In the present manuscript, the forcing hypothesis is instead the \emph{critical} $C^{3/2}$ Carleson smallness from Assumption~\ref{assump:D-forcing}.
As emphasized in Remark~\ref{rem:carleson-not-L2}, $C^{3/2}$ control does not imply $L^2$ control in general.
Thus, to make the $\varepsilon$-regularity/Campanato iteration fully referee-checkable in the intended forcing class, one must replace Lemma~\ref{lem:dde-caccioppoli} by a genuinely $L^{3/2}$-based estimate (or by a Carleson--BMO duality estimate) controlling the forcing term directly at the critical level.%
\end{remark}}

{\color{blue}\begin{remark}[Potential routes to critical forcing $\varepsilon$-regularity]\label{rem:critical-epsreg-routes}
The gap identified in Remark~\ref{rem:forcing-caccioppoli-gap} could potentially be closed by:
\begin{enumerate}
\item \textbf{Morrey-space regularity theory.} The $C^{3/2}$ Carleson condition
\[
\sup_{z_0,r\le 1}\ r^{-2}\iint_{Q_r(z_0)}|H|^{3/2}\le \delta^{3/2}
\]
is equivalent to $H\in \mathcal M^{3/2,2}_{\mathrm{par}}$ (parabolic Morrey space with critical scaling).
There is a well-developed theory of parabolic equations with right-hand sides in Morrey spaces (see Byun--Wang, Krylov, etc.),
but these typically apply to \emph{linear} equations.  For the harmonic-map heat flow, one would need a nonlinear extension.
\emph{Related elliptic} $L^p$-tension regularity results for approximate harmonic maps can be found in Moser~\cite{Moser2015}.
\item \textbf{Struwe-type bubbling analysis.} In the classical harmonic-map heat flow without forcing, Struwe's monotonicity formula gives $\varepsilon$-regularity by controlling the local energy.
With small critical forcing, one could try to perturb Struwe's argument and obtain an energy decay estimate with an additive forcing remainder $+C\delta^2$.
\emph{See} Struwe~\cite{Struwe1988} for the unforced monotonicity framework.
\item \textbf{De Giorgi--Nash--Moser for sphere-valued fields.} Recent work (e.g.\ Gastel--Scheven on $p$-harmonic maps, Diening--Stroffolini--Verde on degenerate equations) develops De Giorgi-type regularity for geometric PDEs.
Adapting these methods to the parabolic setting with small critical forcing is a plausible (but non-trivial) extension.
\item \textbf{Direct $L^{3/2}$ testing.} Instead of testing the DDE with $-\Delta(\phi^2\xi)$, one can test with a power $|\nabla\xi|^{-1/2}\cdot(\text{test function})$ to produce a term naturally at the $L^{3/2}$ level.  This is technically delicate due to degeneracy where $\nabla\xi$ vanishes.
\emph{See} Weber~\cite{Weber2009} for parabolic bootstrapping tools (parabolic Weyl lemma/product estimates) in the harmonic-map heat-flow setting.
\end{enumerate}
We record these as potential strategies; fully implementing any of them goes beyond the current manuscript scope and is isolated in Assumption~\ref{assump:C-epsreg-critical}.
\end{remark}}

{\color{magenta}\begin{remark}[On the Serrin exponents actually used in this manuscript]
While Theorem~\ref{thm:DDE-eps-regularity} is stated for a general admissible Serrin drift class,
the only drift input explicitly quantified in the present $\varepsilon$-regularity scaffolding is the endpoint-in-time form
$u\in L^\infty_tL^p_x$ with $p>3$ (Lemma~\ref{lem:dde-drift-absorb}).
This is sufficient for the running-max refactor because bounded vorticity yields exactly such a local drift bound after a Galilean gauge (Lemma~\ref{lem:drift-local-Lp}),
and the gauged rescaled drift can be made perturbatively small on $Q_1$ (Lemma~\ref{lem:drift-small-rescaled}).
If one wishes to work with a general Serrin pair $(q,p)$ with $2/q+3/p<1$, an additional (referee-checkable) drift absorption lemma in that setting should be supplied.%
\end{remark}}

{\color{magenta}\begin{lemma}[Drift absorption under a local Serrin bound (the $L^\infty_tL^p_x$ case)]\label{lem:dde-drift-absorb}
Assume on $Q_r(z_0)$ that $u$ is divergence-free and belongs to a local Serrin class in the endpoint-in-time form
\[
u\in L^\infty\bigl((t_0-r^2,t_0);L^p(B_r(x_0))\bigr)\qquad\text{for some }p>3.
\]
Then for every $\varepsilon>0$,
\[
\iint_{Q_r(z_0)} |u|^2\,|\nabla \xi|^2
\ \le\ \varepsilon \iint_{Q_r(z_0)} |\nabla^2 \xi|^2
\ +\ C_{\varepsilon,p}\,\|u\|_{L^\infty_tL^p_x(Q_r(z_0))}^2 \iint_{Q_r(z_0)} |\nabla \xi|^2,
\]
where $C_{\varepsilon,p}$ depends only on $\varepsilon,p$ and dimension.
\end{lemma}}

{\color{magenta}\begin{proof}[Proof (sketch with explicit exponents)]
By scaling and translation, it suffices to prove the estimate on $Q_1:=Q_1(0,0)$.
Let $p_*:=\frac{2p}{p-2}\in(2,6)$.
For a.e.\ $t\in(-1,0)$, H\"older in space gives
\[
\int_{B_1}|u|^2|\nabla\xi|^2
\le \|u(\cdot,t)\|_{L^p(B_1)}^2\,\|\nabla\xi(\cdot,t)\|_{L^{p_*}(B_1)}^2.
\]
Taking the supremum in time yields
\[
\iint_{Q_1}|u|^2|\nabla\xi|^2
\le \|u\|_{L^\infty_tL^p_x(Q_1)}^2\ \int_{-1}^0\|\nabla\xi(\cdot,t)\|_{L^{p_*}(B_1)}^2\,dt.
\]
By the (spatial) Gagliardo--Nirenberg--Sobolev inequality applied to $w=\nabla\xi(\cdot,t)$ on $B_1$,
\[
\|w\|_{L^{p_*}(B_1)}\ \le\ C_p\bigl(\|\nabla w\|_{L^2(B_1)}+\|w\|_{L^2(B_1)}\bigr)
\ =\ C_p\bigl(\|\nabla^2\xi(\cdot,t)\|_{L^2(B_1)}+\|\nabla\xi(\cdot,t)\|_{L^2(B_1)}\bigr).
\]
Squaring and integrating in time gives
\[
\int_{-1}^0\|\nabla\xi\|_{L^{p_*}(B_1)}^2
\le C_p\iint_{Q_1}|\nabla^2\xi|^2
\ +\ C_p\iint_{Q_1}|\nabla\xi|^2.
\]
Combining these displays yields
\[
\iint_{Q_1}|u|^2|\nabla\xi|^2
\le C_p\,\|u\|_{L^\infty_tL^p_x(Q_1)}^2\Bigl(\iint_{Q_1}|\nabla^2\xi|^2+\iint_{Q_1}|\nabla\xi|^2\Bigr).
\]
Finally, apply Young's inequality to absorb the $\iint|\nabla^2\xi|^2$ term: for any $\varepsilon>0$,
\[
C_p\,\|u\|_{L^\infty_tL^p_x(Q_1)}^2\iint_{Q_1}|\nabla^2\xi|^2
\le \varepsilon \iint_{Q_1}|\nabla^2\xi|^2
\ +\ C_{\varepsilon,p}\,\|u\|_{L^\infty_tL^p_x(Q_1)}^2\iint_{Q_1}|\nabla\xi|^2,
\]
which proves the claim on $Q_1$, and scaling back gives the general $r$ case.
\end{proof}}

{\color{magenta}\begin{lemma}[Carleson forcing implies an $L^1$ bound]\label{lem:dde-carleson-L1}
If $\|H\|_{C^{3/2}(r)}\le \delta$ on $Q_r(z_0)$ (see \eqref{eq:DDE} and the definition of $\|H\|_{C^{3/2}(r)}$ in Subsection~\ref{subsec:constants}), then
\[
\iint_{Q_r(z_0)} |H| \ \le\ C\,\delta\, r^{3},
\]
where $C$ is universal.
\end{lemma}}

{\color{magenta}\begin{proof}
By H\"older with exponents $(3/2,3)$,
\[
\iint_{Q_r(z_0)} |H|
\le |Q_r|^{1/3}\,\|H\|_{L^{3/2}(Q_r(z_0))}.
\]
The Carleson bound gives $\|H\|_{L^{3/2}(Q_r(z_0))}^{3/2}\le \delta^{3/2} r^2$, hence $\|H\|_{L^{3/2}(Q_r(z_0))}\le \delta\,r^{4/3}$.
Since $|Q_r|\sim r^5$, we have $|Q_r|^{1/3}\sim r^{5/3}$, and therefore $\iint_{Q_r(z_0)}|H|\lesssim r^{5/3}\cdot \delta r^{4/3}=\delta r^3$.
\end{proof}}

{\color{magenta}\begin{remark}[Why one cannot convert $C^{3/2}$ control into $L^2$ control]\label{rem:carleson-not-L2}
The Morrey/Carleson control $\iint_{Q_r}|H|^{3/2}\lesssim \delta^{3/2} r^2$ does \emph{not} in general imply any bound on $\iint_{Q_r}|H|^2$:
on finite-measure sets one has $L^2\hookrightarrow L^{3/2}$ (not the other way around), and $H$ may concentrate on small subsets while keeping the $L^{3/2}$ density bounded.
Accordingly, a fully referee-checkable $\varepsilon$-regularity proof in the critical $C^{3/2}$ forcing regime must estimate the forcing contribution directly at the $L^{3/2}$ level (typically via a Carleson--BMO duality argument or an $L^{3/2}$-based Caccioppoli inequality), rather than converting to $L^2$.%
\end{remark}}

{\color{magenta}\noindent\textbf{[AI AUDIT / consistency.]}
Lemma~\ref{lem:dde-drift-absorb} packages the only place the drift hypothesis enters the $\varepsilon$-regularity iteration.
In the running-max rewrite, bounded vorticity yields a local Serrin drift bound after a Galilean gauge (Lemma~\ref{lem:drift-local-Lp}), so the drift absorption step is available.
The remaining non-classical content is to supply a fully referee-checkable parabolic Sobolev/Campanato iteration in the \emph{critical} drift/Carleson forcing regime and to verify the needed small-energy hypotheses for the ancient element.}

Combining this with Poincaré inequalities, we derive a one-step Campanato decay estimate.

\begin{lemma}[One-Step Energy Decay (conditional on an $L^2$ forcing size)]\label{lem:decay}
There exist constants $\theta \in (0,1)$ and $C > 0$ (depending on the drift bound through Lemma~\ref{lem:dde-drift-absorb}) such that if
\[
E(z_0,r) \le \varepsilon_0
\qquad\text{and}\qquad
H\in L^2(Q_r(z_0)),
\]
then
\[
E(z_0,r/2) \le \theta\,E(z_0,r) + C\,F(z_0,r),
\qquad
F(z_0,r):=r^{-1}\iint_{Q_r(z_0)} |H|^2.
\]
\end{lemma}

\begin{proof}
Scale to $r=1$ and suppress $z_0$ in the notation.
Apply Lemma~\ref{lem:dde-caccioppoli} on $Q_1$ to obtain
\[
\iint_{Q_{1/2}} |\nabla^2 \xi|^2
\ \le\ C \iint_{Q_1} |\nabla \xi|^2
\ +\ C \iint_{Q_1} |u|^2 |\nabla \xi|^2
\ +\ C \iint_{Q_1} |H|^2.
\]
Absorb the drift term using Lemma~\ref{lem:dde-drift-absorb} (choosing $\varepsilon>0$ small enough to absorb a portion of $\|\nabla^2\xi\|_2^2$ into the left-hand side).
This yields an estimate of the form
\[
\iint_{Q_{1/2}} |\nabla^2 \xi|^2 \ \le\ C_1 \iint_{Q_1} |\nabla\xi|^2 + C_2 \iint_{Q_1}|H|^2,
\]
where $C_1$ depends on the drift bound and $C_2$ is universal.
Finally, a standard parabolic Campanato/Poincar\'e step upgrades the Hessian control on $Q_{1/2}$ to decay of the scale-invariant energy on smaller cylinders, giving
$E(1/2)\le \theta E(1)+C\,F(1)$ with $\theta\in(0,1)$, and scaling back yields the stated $r$-version.
\end{proof}
{\color{magenta}\noindent\textbf{[AI AUDIT / forcing mismatch.]}
Lemma~\ref{lem:decay} is the correct one-step decay statement \emph{provided one has} control of the scale-invariant $L^2$ forcing size $F(z_0,r)$.
However, the intended forcing hypothesis in this manuscript is the critical $C^{3/2}$ Carleson smallness from Assumption~\ref{assump:D-forcing}, and (by Remark~\ref{rem:carleson-not-L2}) such Carleson control does not imply any bound on $F(z_0,r)$ in general.
Bridging this gap requires an additional $L^{3/2}$-based Caccioppoli estimate or a Carleson--BMO duality argument as in Remark~\ref{rem:forcing-caccioppoli-gap}.}

\begin{lemma}[Iterating the one-step decay]\label{lem:decay-iterate}
Assume that there exist constants $\theta\in(0,1)$ and $C_0<\infty$ such that for every $0<r\le 1$,
\[
E(z_0,r/2)\ \le\ \theta\,E(z_0,r)\ +\ C_0\,\delta^2,
\]
whenever $E(z_0,r)\le \varepsilon_0$ and $F(z_0,r)\le \delta^2$ (with $F$ as in Lemma~\ref{lem:decay}).
Then there exists $\alpha\in(0,1)$ and $C<\infty$ (depending only on $\theta,C_0$) such that for all dyadic radii $\rho=2^{-k}$ with $k\ge 1$,
\[
E(z_0,\rho)\ \le\ C\,\rho^{2\alpha}\,E(z_0,1)\ +\ C\,\delta^2.
\]
\end{lemma}

\begin{proof}
Fix $z_0$ and write $E_k:=E(z_0,2^{-k})$. The hypothesis gives the recursion
\[
E_{k+1}\le \theta E_k + C_0\delta^2.
\]
Iterating yields
\[
E_k \le \theta^k E_0 + C_0\delta^2\sum_{j=0}^{k-1}\theta^j
\le \theta^k E_0 + \frac{C_0}{1-\theta}\,\delta^2.
\]
Choose $\alpha>0$ such that $\theta=2^{-2\alpha}$, i.e.\ $\alpha:=\tfrac12\log_2(\theta^{-1})$.
Since $\rho=2^{-k}$, we have $\theta^k=(2^{-2\alpha})^k=\rho^{2\alpha}$, hence
\[
E(z_0,\rho)=E_k \le \rho^{2\alpha}E(z_0,1) + \frac{C_0}{1-\theta}\,\delta^2.
\]
Absorb constants into $C$.
\end{proof}

\subsection{Epsilon-Regularity}
\begin{theorem}[DDE $\varepsilon$-Regularity]\label{thm:DDE-eps-regularity}
There exist universal constants $\eps_*>0$, $\delta_*>0$, $\alpha\in(0,1)$, and $C<\infty$ such that, if on $Q_1(z_0)$ the direction equation
\[
\partial_t \xi - \Delta \xi + u \cdot \nabla \xi = |\nabla \xi|^2 \xi + H, \qquad |\xi|=1,\quad H\cdot \xi=0
\]
holds with a divergence-free drift $u$ in an admissible Serrin class (e.g.\ $u\in L^q_tL^p_x$ with $2/q+3/p<1$) and
\[
E(z_0,1)\le \eps_*^2, \qquad \sup_{0<r\le 1}F(z_0,r)\le \delta_*^2,
\]
then for all $\rho\le \tfrac12$,
\[
E(z_0,\rho) \le C \rho^{2\alpha} E(z_0,1)\ +\ C\,\delta_*^2,
\]
and, in particular,
\[
\sup_{Q_{1/2}(z_0)} |\nabla \xi| \le C\,(\eps_*+\delta_*).
\]
\end{theorem}
{\color{magenta}\noindent\textbf{[AI AUDIT: non-classical step.]}
The $\varepsilon$-regularity statement above is written in a \emph{subcritical} forcing class ($L^2$-Carleson control of $H$, i.e.\ $\sup_{r\le 1}F(z_0,r)\le\delta_*^2$).
The intended forcing hypothesis elsewhere in this manuscript is the \emph{critical} $C^{3/2}$ Carleson smallness from Assumption~\ref{assump:D-forcing}.
Upgrading the argument to that critical class requires a different forcing estimate (see Remark~\ref{rem:forcing-caccioppoli-gap}): one cannot pass from $C^{3/2}$ control to the $L^2$ forcing size $F$ (Remark~\ref{rem:carleson-not-L2}).
In the running-max rewrite, the drift admissibility is less mysterious because bounded vorticity provides local drift control modulo Galilean gauge (Lemma~\ref{lem:drift-local-Lp}), but the full critical forcing/iteration still needs to be supplied to count as unconditional.}
{\color{magenta}\begin{assumption}[Critical-forcing $\varepsilon$-regularity for the DDE (missing upgrade)]\label{assump:C-epsreg-critical}
There exist universal constants $\eps_*>0$, $\delta_*>0$, $\alpha\in(0,1)$, $C<\infty$, an exponent $p>3$, and a drift threshold $\eta_*>0$ such that the following holds.
If on $Q_1(z_0)$ the direction equation
\[
\partial_t \xi - \Delta \xi + u \cdot \nabla \xi = |\nabla \xi|^2 \xi + H, \qquad |\xi|=1,\quad H\cdot \xi=0
\]
holds with divergence-free drift $u$ satisfying the small Serrin bound
\[
u\in L^\infty\bigl((t_0-1,t_0);L^p(B_1(x_0))\bigr),
\qquad
\|u\|_{L^\infty_tL^p_x(Q_1(z_0))}\le \eta_*,
\]
and with small initial energy and \emph{critical} forcing size
\[
E(z_0,1)\le \eps_*^2,
\qquad
\sup_{0<r\le 1}\ r^{-2}\iint_{Q_r(z_0)} |H|^{3/2}\,dx\,dt \le \delta_*^{3/2},
\]
then for all $\rho\le \tfrac12$ one has the quantitative decay
\[
E(z_0,\rho) \le C \rho^{2\alpha} E(z_0,1)\ +\ C\,\delta_*^2,
\]
and, in particular, the scale-covariant gradient bound
\[
\sup_{Q_{1/2}(z_0)} |\nabla \xi| \le C\,(\eps_*+\delta_*).
\]
\end{assumption}}
Iterating the one-step decay estimate from Lemma~\ref{lem:decay} on dyadic radii and applying Lemma~\ref{lem:decay-iterate} yields the stated power-law energy decay (including the additive forcing remainder).
The H\"older and gradient bounds then follow by a standard parabolic Campanato/Morrey embedding argument together with interior regularity for the drift--diffusion equation once the scale-invariant energy decays.%

\subsection{Rigidity via Blow-up}
We record a clean (and correct) Liouville mechanism \emph{once a scale-covariant $\varepsilon$-regularity gradient bound is available}.

\begin{theorem}[Directional Liouville from global critical-energy smallness]\label{thm:liouville}
Let $\xi$ be an ancient solution to \eqref{eq:DDE} on $\R^3 \times (-\infty, 0]$.
Assume that the $\varepsilon$-regularity gradient bound of Assumption~\ref{assump:C-epsreg-critical} applies on every cylinder after rescaling, and that there exists $\eps_*>0$ such that
\[
\sup_{z_0\in\R^3\times(-\infty,0]}\ \sup_{r>0}\ E(z_0,r)\ \le\ \eps_*^2.
\]
Then $\xi$ is spatially constant: $\nabla \xi \equiv 0$.
\end{theorem}
{\color{magenta}\noindent\textbf{[AI AUDIT / WHAT REMAINS OPEN FOR (C).]}
Theorem~\ref{thm:liouville} shows that \emph{rigidity follows once one has} (i) a genuinely scale-covariant $\varepsilon$-regularity gradient bound and (ii) \emph{global} smallness of the critical direction energy $E(z_0,r)$ on all scales.
The non-classical content in item (C) is therefore reduced to:
(i) proving a referee-checkable $\varepsilon$-regularity theorem for the sphere-valued drift--diffusion equation with the drift/forcing norms used in this manuscript (see Remark~\ref{rem:critical-epsreg-routes} for potential strategies), and
(ii) verifying the \emph{global} smallness hypothesis $\sup_{z_0,r}E(z_0,r)\le \eps_*^2$ for the running-max ancient element arising in the blow-up argument.
At present, the manuscript does not supply a mechanism forcing $\sup_{z_0,r}E(z_0,r)$ to be small; this should be treated as an additional hypothesis unless/until it is proved.}

{\color{blue}\begin{lemma}[Smoothness gives small direction energy at small scales]\label{lem:smooth-small-energy}
Let $\xi$ be a smooth $\Sbb^2$-valued function on a spacetime cylinder $Q_R(z_0)$ for some $R>0$.
Then for any $r\le R$,
\[
E(z_0,r)\ =\ r^{-3}\iint_{Q_r(z_0)}|\nabla\xi|^2\,dx\,dt
\ \le\ r^2\,\|\nabla\xi\|_{L^\infty(Q_R(z_0))}^2.
\]
In particular, $E(z_0,r)\to 0$ as $r\to 0$.
\end{lemma}

\begin{proof}
Since $|\nabla\xi(x,t)|\le \|\nabla\xi\|_{L^\infty(Q_R)}$ for $(x,t)\in Q_r(z_0)\subset Q_R(z_0)$,
\[
\iint_{Q_r(z_0)}|\nabla\xi|^2\,dx\,dt
\ \le\ \|\nabla\xi\|_{L^\infty}^2\,|Q_r|
\ =\ \|\nabla\xi\|_{L^\infty}^2\,c_3\,r^3\,r^2
\ =\ c_3\,r^5\,\|\nabla\xi\|_{L^\infty}^2,
\]
where $c_3$ is the volume of the unit parabolic cylinder. Dividing by $r^3$ gives the result.
\end{proof}}

{\color{blue}\begin{remark}[On global direction-energy smallness for the running-max ancient element]\label{rem:global-energy-running-max}
The running-max ancient element $u^\infty$ satisfies $\|\omega^\infty\|_{L^\infty}\le 2$ (Lemma~\ref{lem:ancient-limit-runningmax}(iii)).
By Lemma~\ref{lem:Linfty-vort-smooth} (bounded vorticity implies local smoothness), the direction field $\xi^\infty$ is smooth on each compact cylinder.
Smoothness and $|\xi|=1$ imply that $\nabla\xi$ is bounded on each compact cylinder, hence $E(z_0,r)<\infty$ for each fixed $z_0,r$.

\smallskip
\noindent\textbf{Smallness at small scales (non-uniform).}
By Lemma~\ref{lem:smooth-small-energy}, for each fixed basepoint $z_0$ the direction energy satisfies $E(z_0,r)\to 0$ as $r\to 0$.
However, the convergence rate depends on the local gradient bound $\|\nabla\xi\|_{L^\infty(Q_R(z_0))}$, which could vary with $z_0$.

\smallskip
\noindent\textbf{The missing step:} What is \emph{not} automatic is a \emph{uniform} bound $\sup_{z_0,r}E(z_0,r)\le \eps_*^2$ as $z_0$ ranges over $\R^3\times(-\infty,0]$ and $r$ ranges over $(0,\infty)$.
The blow-up compactness that produces $u^\infty$ gives local control but no global uniformity:
\begin{itemize}
\item As $|z_0|\to\infty$, the compactness arguments only give weak convergence, which does not preserve strict smallness of $E$.
\item As $r\to\infty$, the energy $E(z_0,r)$ integrates over larger regions where $\nabla\xi$ may oscillate.
\item As $r\to 0$, small-scale concentration of $\nabla\xi$ could in principle violate smallness.
\end{itemize}
Proving global smallness would require an additional monotonicity or energy-dissipation argument that propagates from the blow-up normalizations (e.g.\ $|\omega(0,0)|=1$) to a uniform bound on $E$.
Absent such an argument, we treat $\sup_{z_0,r}E(z_0,r)\le \eps_*^2$ as part of Assumption~\ref{assump:C-liouville}.

\smallskip
\noindent\textbf{Role of vorticity zeros.}
Near the blow-up normalization point $(0,0)$ where $|\omega(0,0)|=1$, by continuity there is a neighborhood $Q_R(0,0)$ where $|\omega|\ge 1/2$.
On this neighborhood, the direction $\xi=\omega/|\omega|$ is smooth and
\[
|\nabla\xi|\ \le\ C\,|\nabla\omega|/|\omega|\ \le\ 2C\,|\nabla\omega|.
\]
Since $\nabla\omega$ is bounded by Serrin regularity (Lemma~\ref{lem:Linfty-vort-smooth}), the local gradient bound is finite on $Q_R(0,0)$, and Lemma~\ref{lem:smooth-small-energy} gives $E(0,r)\to0$ as $r\to0$ with a quantitative rate.

However, the running-max normalization only guarantees $|\omega(0,0)|=1$; it does \emph{not} prevent vorticity zeros elsewhere.
Near a vorticity zero $\omega(z^*)=0$, the direction $\xi=\omega/|\omega|$ is undefined or has singular gradient ($|\nabla\xi|\to\infty$ as $|\omega|\to0$).
The direction energy $E(z_0,r)$ for basepoints $z_0$ near $z^*$ could therefore be large even for small $r$.
This vorticity-zero issue is the fundamental obstruction to uniform global energy smallness.
\end{remark}}

\begin{proof}
Fix any $z_0=(x_0,t_0)$ and $r>0$. Define the parabolically rescaled fields on $Q_1(0,0)$ by
\[
\xi^{(r)}(x,t):=\xi(x_0+r x,\ t_0+r^2 t),\qquad
u^{(r)}(x,t):=r\,u(x_0+r x,\ t_0+r^2 t),\qquad
H^{(r)}(x,t):=r^2\,H(x_0+r x,\ t_0+r^2 t).
\]
By scale invariance of $E$ we have $E_{\xi^{(r)}}(0,1)=E_\xi(z_0,r)\le \eps_*^2$. By the hypothesis that the $\varepsilon$-regularity bound applies on every rescaled cylinder, Assumption~\ref{assump:C-epsreg-critical} yields
\[
|\nabla\xi^{(r)}(0,0)|\le \sup_{Q_{1/2}(0,0)}|\nabla\xi^{(r)}|\ \le\ C\,(\eps_*+\delta_*).
\]
Undoing the rescaling gives $|\nabla\xi(z_0)|=\frac{1}{r}\,|\nabla\xi^{(r)}(0,0)|\le \frac{C(\eps_*+\delta_*)}{r}$.
Since $r>0$ was arbitrary, letting $r\to\infty$ forces $|\nabla\xi(z_0)|=0$. As $z_0$ was arbitrary, $\nabla\xi\equiv0$.
\end{proof}

\section{Classification and Contradiction}

\subsection{Time-Constancy of the Direction}
Assuming Assumption~\ref{assump:C-liouville} (directional Liouville rigidity), the ancient direction field is constant in space-time:
\[
\xi^\infty(x,t)\equiv b_0\in\Sbb^2.
\]

\subsection{Reduction to 2D Dynamics}
Rotate coordinates so that the constant direction is $b_0=e_3=(0,0,1)$, hence
\[
\omega^\infty=\curl u^\infty=(0,0,\alpha).
\]
The identities $\omega^\infty_1=\omega^\infty_2=0$ together with $\dv u^\infty=0$ imply that, for each fixed time $t$,
\[
\Delta u^\infty_3(\cdot,t)=0 \quad\text{in }\R^3.
\]
{\color{magenta}\noindent\textbf{[AI AUDIT / separation of issues.]}
For \emph{smooth} constant-direction flows $\omega=(0,0,\alpha)$ with $\alpha\not\equiv 0$, the $x_3$-independence of the horizontal velocity $u_h=(u_1,u_2)$ follows from a unique-continuation/analyticity argument (Remark~\ref{rem:constdir-uc}); it is not the remaining hypothesis.
What remains is control of the harmonic component $u_3$: once $\partial_3 u_h\equiv0$, the constraints $\omega_1=\omega_2=0$ imply $\partial_1 u_3=\partial_2 u_3=0$, hence $u_3$ depends only on $x_3$, and the harmonicity above forces $u_3(\cdot,t)$ to be affine in $x_3$.
Lemma~\ref{lem:linear-mode-ODE} below shows that the ancient-solution structure rules out a \emph{positive} linear coefficient but allows a non-positive one.
To reduce to a genuine 2D incompressible flow and apply a 2D ancient Liouville theorem, one imposes a Liouville-class hypothesis forcing $u_3(\cdot,t)$ to be spatially constant (Assumption~\ref{assump:E-2d}); after subtracting a constant Galilean drift one may assume $u_3\equiv 0$.}
In that case the flow reduces to a two-dimensional velocity field in the plane perpendicular to $e_3$:
\[
u^\infty(x,t) = (v_1(x_1, x_2, t), v_2(x_1, x_2, t), 0).
\]
{\color{magenta}\noindent\textbf{[AI AUDIT.]}
The reduction $u^\infty_3\equiv0$ requires an additional Liouville-class hypothesis for the harmonic function $u^\infty_3$ (not provided by Lemma~\ref{lem:ancient-limit-runningmax} as currently written). This is part of Assumption~\ref{assump:E-2d}.}

{\color{magenta}\begin{remark}[Possible unique-continuation shortcut for the $x_3$-independence]\label{rem:constdir-uc}
For a \emph{smooth} constant-direction solution with $\omega=(0,0,\alpha)$, the first two components of the vorticity equation imply
$\alpha\,\partial_3 u_1=0$ and $\alpha\,\partial_3 u_2=0$.
On the open set $\{\alpha\neq0\}$ this forces $\partial_3 u_1=\partial_3 u_2=0$.
Since smooth Navier--Stokes solutions are real-analytic in space for each fixed time $t<0$ (a classical parabolic smoothing fact), either $\alpha(\cdot,t)\equiv0$ or the open set $\{\alpha(\cdot,t)\neq 0\}$ is nonempty.
In the nontrivial case (which holds for the running-max ancient element by normalization), $\partial_3 u_1(\cdot,t)$ and $\partial_3 u_2(\cdot,t)$ are real-analytic functions vanishing on a nonempty open set, hence
\[
\partial_3 u_1(\cdot,t)\equiv 0,\qquad \partial_3 u_2(\cdot,t)\equiv 0 \qquad\text{on }\R^3.
\]
Thus the $x_3$-independence of $u_h$ is automatic in the smooth constant-direction case; the remaining (E) obstruction is the Liouville-class control of the harmonic component $u_3$.%
\end{remark}}

\begin{lemma}[Vanishing Stretching]\label{lem:vanishing_stretching}
If the vorticity direction of a N--S solution is constant in space and time, the vortex stretching term is identically zero.
\end{lemma}

\begin{proof}
Let the direction be constant, $\xi(x,t) \equiv e_3$. Then $\omega = (0, 0, \omega_3)$.
The vortex stretching term is given by $(\omega \cdot \nabla) u = \omega_3 \partial_3 u$.
As discussed in the reduction above, the conclusion that $\partial_3 u\equiv 0$ follows once one assumes the needed Liouville-class hypothesis for the harmonic component $u_3$ (cf.\ Assumption~\ref{assump:E-2d}).
Consequently, $(\omega \cdot \nabla) u = \omega_3 \cdot 0 = 0$.
\end{proof}

{\color{magenta}\begin{remark}[A weaker but unconditional identity from the vorticity equation]\label{rem:constdir-weak-vanish}
Even without any Liouville/growth hypothesis, if $\omega=\rho e_3$ solves the vorticity equation
$\partial_t\omega + u\cdot\nabla\omega - \Delta\omega = \omega\cdot\nabla u$ in the distributional sense, then the first two components imply
\[
0=\rho\,\partial_3 u_1,\qquad 0=\rho\,\partial_3 u_2
\]
in distributions. Thus $\partial_3 u_h=0$ holds on the set $\{\rho\neq 0\}$ (in the a.e.\ sense).
Upgrading this to $\partial_3 u\equiv 0$ globally (hence true 2D dynamics and vanishing stretching everywhere) requires additional global information
about the harmonic component of $u_3$ or a unique-continuation mechanism that is not currently available from Lemma~\ref{lem:ancient-limit-runningmax}. This is part of Assumption~\ref{assump:E-2d}.%
\end{remark}}

\subsection{2D Ancient Liouville Theorem}
If, in addition, the constant-direction running-max ancient element belongs to a 2D Liouville class as isolated in Assumption~\ref{assump:E-2d},
then one may reduce to an ancient 2D Navier--Stokes flow on $\R^2\times(-\infty,0]$ and invoke a classical 2D Liouville theorem.
{\color{magenta}\noindent\textbf{[AI AUDIT.]}
Lemma~\ref{lem:ancient-limit-runningmax} provides local energy and local \(L^3\) control (and bounded vorticity), but does not establish the global Liouville-class hypotheses
(boundedness / global integrability / decay at infinity) needed to apply \cite{KNSS2009}. Those are exactly what is isolated in Assumption~\ref{assump:E-2d}.}

Known Liouville theorems for the 2D N--S equations state that any bounded ancient solution must be constant (essentially due to the monotonicity of enstrophy in 2D).

{\color{blue}\begin{remark}[What is needed for the 2D Liouville step]\label{rem:2d-liouville-requirements}
The classical 2D ancient Liouville theorem (\cite{KNSS2009}) requires \textbf{bounded velocity} $u\in L^\infty(\R^2\times(-\infty,0])$.
The running-max ancient element provides \textbf{bounded vorticity} $\omega\in L^\infty$.
These are \emph{not} equivalent in 2D:

\smallskip
\noindent\textbf{From bounded vorticity to velocity bounds.}
Via the 2D Biot--Savart law, $u=K_{2D}*\omega$ where $K_{2D}(x)=\frac{1}{2\pi}\frac{(-x_2,x_1)}{|x|^2}$.
\begin{itemize}
\item If $\omega\in L^\infty\cap L^p$ for some $p<2$, then $u$ has at most logarithmic growth at infinity (Calder\'on--Zygmund).
\item If $\omega$ is merely $L^\infty$ without decay/integrability, $u$ can have linear or faster growth.
\end{itemize}

\smallskip
\noindent\textbf{What the running-max bounds give.}
From Lemma~\ref{lem:ancient-limit-runningmax}, the 2D reduced flow has:
\begin{itemize}
\item Bounded vorticity: $|\omega^\infty|\le 2$.
\item Local $L^3$ velocity control on each $B_R$.
\end{itemize}
The local $L^3$ control is compatible with polynomial growth as $R\to\infty$, so it does not directly imply bounded velocity.

\smallskip
\noindent\textbf{The gap (E2).}
To apply the KNSS 2D Liouville theorem, one needs either:
\begin{enumerate}
\item \emph{Bounded velocity} as an explicit hypothesis, or
\item \emph{Sublinear growth at spatial infinity} (which is still enough to run the KNSS vorticity integral contradiction; see Lemma~\ref{lem:2d-liouville-sublinear}), or
\item \emph{Finite 2D enstrophy} $\int_{\R^2}|\omega|^2<\infty$ (then Lemma~\ref{lem:pure-diffusion-enstrophy} forces $\omega$ constant), or
\item \emph{Vorticity decay at infinity} so that Biot--Savart yields bounded velocity.
\end{enumerate}
None of these is currently established from the running-max compactness; (E2) isolates this gap.
\end{remark}}

{\color{blue}\begin{lemma}[2D Liouville from bounded vorticity and sublinear growth]\label{lem:2d-liouville-sublinear}
Let $v$ be a smooth ancient solution of the 2D Navier--Stokes equations on $\R^2\times(-\infty,0]$ with bounded vorticity $\alpha=\partial_1 v_2-\partial_2 v_1\in L^\infty(\R^2\times(-\infty,0])$.
Assume there exist constants $C>0$ and $\beta\in[0,1)$ such that
\[
|v(x,t)|\ \le\ C\,(1+|x|^\beta)\qquad\text{for all }(x,t)\in\R^2\times(-\infty,0].
\]
Then $\alpha\equiv 0$ and $v(x,t)=b(t)$ is spatially constant for each $t$.
\end{lemma}

\begin{proof}
This is a direct adaptation of the 2D case in \cite{KNSS2009}.
Let $M_1:=\sup_{\R^2\times(-\infty,0]}\alpha$ and $M_2:=\inf_{\R^2\times(-\infty,0]}\alpha$.
Assume $M_1>0$.
Since $\alpha$ satisfies the scalar vorticity equation $\partial_t\alpha+v\cdot\nabla\alpha-\nu\Delta\alpha=0$ and $v$ is smooth with bounded coefficients on each compact cylinder, the maximum-principle stability lemma (cf.\ Lemma~2.1 in \cite{KNSS2009}) yields arbitrarily large parabolic cylinders
$Q_R=B(\bar x,R)\times(\bar t-R^2,\bar t)$ on which $\alpha\ge M_1/2$.
Hence
\[
\iint_{Q_R}\alpha\,dx\,dt\ \ge\ c\,M_1\,R^4.
\]
On the other hand, by Stokes' theorem in space for each fixed time and the growth bound on $v$,
\[
\int_{B(\bar x,R)}\alpha(x,t)\,dx
=\int_{\partial B(\bar x,R)} v(x,t)\cdot \tau\,ds
\le C\,R\,\sup_{\partial B(\bar x,R)}|v(\cdot,t)|
\le C\,R\,(1+R^\beta),
\]
where $\tau$ is the unit tangent and $|\partial B|\sim R$.
Integrating in time over an interval of length $R^2$ gives
\[
\iint_{Q_R}\alpha\,dx\,dt\ \le\ C\,R^3\,(1+R^\beta)=o(R^4)\qquad(R\to\infty),
\]
since $\beta<1$.
This contradicts the lower bound $cM_1R^4$ for large $R$. Therefore $M_1\le 0$.
The same argument applied to $-\alpha$ shows $M_2\ge 0$.
Hence $\alpha\equiv 0$.
With $\curl v=0$ and $\dv v=0$ in $\R^2$, each $v(\cdot,t)$ is harmonic.
The sublinear growth assumption forces $v(\cdot,t)$ to be constant in $x$ for each $t$, i.e.\ $v(x,t)=b(t)$.
\end{proof}}

\begin{theorem}[2D Ancient Liouville]\label{thm:2d_liouville}
Let $u$ be a bounded ancient solution to the 2D N--S equations on $\R^2 \times (-\infty, 0]$. Then $u$ is a constant (specifically $u \equiv 0$ for finite energy).
\end{theorem}

\begin{proof}
\textbf{Step 1: Enstrophy identity.}
For a bounded ancient 2D solution $u=(v_1,v_2)$ on $\R^2\times(-\infty,0]$, the scalar vorticity
$\alpha=\partial_1 v_2-\partial_2 v_1$ satisfies
\[
\partial_t \alpha + v \cdot \nabla \alpha = \nu \Delta \alpha.
\]
Multiply by $\alpha$ and integrate over $\R^2$:
\[
\frac{1}{2} \frac{d}{dt} \|\alpha\|_2^2 + \nu \|\nabla \alpha\|_2^2 = 0.
\]
This implies $\|\alpha(t)\|_2$ is non-increasing.
\textbf{Step 2: Liouville conclusion.}
We rely on the Liouville theorem for bounded ancient 2D flows (Koch--Nadirashvili--Seregin--\v{S}ver\'{a}k \cite{KNSS2009}), which concludes that any bounded ancient 2D Navier--Stokes solution is constant.
\end{proof}

\subsection{The Final Contradiction}
Assuming Assumptions~\ref{assump:C-liouville} and~\ref{assump:E-2d}, the constant-direction running-max ancient element belongs to a 2D Liouville class,
so the classical 2D Liouville theorem (e.g.\ \cite{KNSS2009}) forces $u^\infty\equiv 0$ and hence $\omega^\infty\equiv 0$.

However, Lemma~\ref{lem:ancient-limit-runningmax}(iii) guarantees that the running-max ancient element is \emph{non-trivial}: $|\omega^\infty(0,0)|=1$, hence $\omega^\infty\not\equiv 0$ and therefore $u^\infty\not\equiv 0$.
This contradicts $u^\infty \equiv 0$.

Therefore, the initial assumption that a finite-time singularity exists must be false.

\bibliographystyle{amsplain}
\begin{thebibliography}{10}

\bibitem{BKM1984}
J.~T. Beale, T.~Kato, and A.~Majda, \emph{Remarks on the breakdown of smooth solutions for the 3-{D} {E}uler equations}, Comm. Math. Phys. \textbf{94} (1984), no.~1, 61--66.

{\color{blue} \bibitem{CFM1996} C. Fefferman, A. J. Majda,  {\it Geometric constraints on potentially,} Communications in Partial Differential Equations, 21(3–4) (1996)., 559–571. https://doi.org/10.1080/03605309608821197}

%\bibitem{CFM1996}
%P.~Constantin, C.~Fefferman, and A.~Majda, \emph{Geometric constraints on potentially singular solutions for the 3-{D} {E}uler equations}, Comm. Partial Differential Equations \textbf{21} (1996), no.~3-4, 559--571.

\bibitem{CKN1982}
L.~Caffarelli, R.~Kohn, and L.~Nirenberg, \emph{Partial regularity of suitable weak solutions of the {N}avier-{S}tokes equations}, Comm. Pure Appl. Math. \textbf{35} (1982), no.~6, 771--831.

%\bibitem{ESS2003}
%L.~Escauriaza, G.~Seregin, and V.~{\v{S}}ver{\'a}k, \emph{{$L_{3,\infty}$}-solutions of {N}avier-{S}tokes equations and backward uniqueness}, Uspekhi Mat. Nauk \textbf{58} (2003), no.~2(350), 3--44.

\bibitem{ESS2003}
{\color{blue}L.~Escauriaza, G.~A.~Seregin, and V.~\v{S}ver\'{a}k,
\emph{$L_{3,\infty}$-solutions of the N--S equations and backward uniqueness},
Uspekhi Mat.\ Nauk \textbf{58} (2(350)) (2003), 3--44;
translation in Russian Math.\ Surveys \textbf{58} (2003), no.~2, 211--250.}


\bibitem{Fefferman2006} {\color{blue}
C. L. Fefferman, Existence and smoothness of the Navier-Stokes
equation. In J.A. Carlson, A. Jaffe, A. Wiles, Clay Mathematics Institute,
and American Mathematical Society, editors, \emph{The Millennium
Prize Problems}, 57–67. American Mathematical Society, 2006.}

\bibitem{Hopf1951}
E.~Hopf, \emph{{\"U}ber die {A}nfangswertaufgabe f{\"u}r die hydrodynamischen {G}rundgleichungen}, Math. Nachr. \textbf{4} (1951), 213--231.

\bibitem{CRW1976}
{\color{blue} R.~R. Coifman, R.~Rochberg, and G.~Weiss, \emph{Factorization theorems for Hardy spaces in several variables}, Ann. of Math. (2) \textbf{103} (1976), no.~3, 611--635.}
%%izbrisana zadnja recenica


\bibitem{CaffarelliSilvestre2007} {\color{blue}
L.~Caffarelli and L.~Silvestre, \emph{An extension problem related to the fractional Laplacian}, Comm. Partial Differential Equations \textbf{32} (2007), no.8, 1245--1260.} 

%pisalo no.7-9, a treba 8

\bibitem{KNSS2009}
G.~Koch, N.~Nadirashvili, G.~Seregin, and V.~{\v{S}}ver{\'a}k, \emph{Liouville theorems for the {N}avier-{S}tokes equations and applications}, Acta Math. \textbf{203} (2009), no.~1, 83--105.




\bibitem{KochTataru2001}
H.~Koch and D.~Tataru, \emph{Well-posedness for the {N}avier-{S}tokes equations}, Adv. Math. \textbf{157} (2001), no.~1, 22--35.




\bibitem{Leray1934}
J.~Leray, \emph{Sur le mouvement d'un liquide visqueux emplissant l'espace}, Acta Math. \textbf{63} (1934), no.~1, 193--248.

{\color{blue}\bibitem{Lemarie2016}
P.-G.~Lemari\'e\mbox{-}Rieusset,
\emph{The Navier--Stokes Problem in the 21st Century},
Chapman \& Hall/CRC, Boca Raton, FL, 2016.}

\bibitem{Lin1998}
F.~Lin, \emph{A new proof of the {C}affarelli-{K}ohn-{N}irenberg theorem}, Comm. Pure Appl. Math. \textbf{51} (1998), no.~3, 241--257.

\bibitem{Prodi1959}
G.~Prodi, \emph{Un teorema di unicit{\`a} per le equazioni di {N}avier-{S}tokes}, Ann. Mat. Pura Appl. (4) \textbf{48} (1959), 173--182.

 

\bibitem{Scheffer1977}
V.~Scheffer, \emph{Hausdorff measure and the {N}avier-{S}tokes equations}, Comm. Math. Phys. \textbf{55} (1977), no.~2, 97--112.

\bibitem{Seregin2012}
G.~Seregin, \emph{A certain necessary condition of potential blow up for {N}avier-{S}tokes equations}, Comm. Math. Phys. \textbf{312} (2012), no.~3, 833--845.

 

\bibitem{Serrin1962}
J.~Serrin, \emph{On the interior regularity of weak solutions of the {N}avier-{S}tokes equations}, Arch. Rational Mech. Anal. \textbf{9} (1) (1962), 187--195.


\bibitem{ConstantinFefferman1993}
{\color{blue}P.~Constantin and C.~Fefferman,
\emph{Direction of vorticity and the problem of global regularity for the Navier--Stokes equations},
Indiana Univ.\ Math.\ J.\ \textbf{42} (1993), no.~3, 775--789.}

\bibitem{MajdaBertozzi2002}
{\color{blue}A.~J.~Majda and A.~L.~Bertozzi,
\emph{Vorticity and Incompressible Flow},
Cambridge Texts in Applied Mathematics, Cambridge Univ.\ Press, 2002.}

\bibitem{Stein1993}
E.~M.~Stein, \emph{Harmonic Analysis: Real-Variable Methods, Orthogonality, and Oscillatory Integrals}, Princeton Mathematical Series, vol.~43, Princeton University Press, 1993.

\bibitem{Moser2015}
R.~Moser, \emph{An $L^p$ regularity theory for harmonic maps}, Trans.\ Amer.\ Math.\ Soc.\ \textbf{367} (2015), no.~1, 1--34.

\bibitem{Struwe1988}
M.~Struwe, \emph{On the evolution of harmonic maps in higher dimensions}, J.\ Differential Geom.\ \textbf{28} (1988), no.~3, 485--502.

\bibitem{Weber2009}
J.~Weber, \emph{A product estimate, the parabolic Weyl lemma and applications}, preprint (2009), arXiv:0910.2739.

\bibitem{15} {\color{blue}
Y. Giga, {\it Solutions for semilinear parabolic equations in Lp and regularity of weak solutions of the Navier-Stokes system,} J. Differential Equations, 62(1986), 186-212.}

\bibitem{25} {\color{blue}
 J. Serrin, {\it The initial value problem for the Navier-stokes equations, in Nonlinear problems(R. E. Langer Ed.),} pp.69-98, Univ. of Wisconsin Press, Madison, 1963.}

\bibitem{27} {\color{blue}
M. Struwe, {\it On partial regularity results for the Navier-Stokes equations,} Comm. Pure
Appl. Math., 41(1988), 437-458.}

\bibitem{23} {\color{blue}
J.~Ne\v{c}as, M.~R{u}\v{z}i\v{c}ka and V.~\v{S}ver\'ak,
{\it On Leray's self-similar solutions of the Navier--Stokes equations},
\newblock {\em Acta Mathematica}, {\bf 176} (1996), 283--294.}

\bibitem{6} {\color{blue}
G. P. Galdi, {\it An Introduction to the Navier-Stokes Initial-Boundary Value
Problem}, In: Fundamental Directions in Mathematical Fluid Mechanics.
Basel: Birkhäuser, 2000, pages 1–70.}

\bibitem{Aubin1963} {\color{blue}
J.-P.\ Aubin,
\emph{Un théorème de compacité},
C.\ R.\ Acad.\ Sci.\ Paris \textbf{256} (1963), 5042--5044.}

\bibitem{Lions1969} {\color{blue}
J.-L.\ Lions,
\emph{Quelques méthodes de résolution des problèmes aux limites non linéaires},
Dunod; Gauthier-Villars, Paris, 1969.}

\bibitem{Kobayashi} {\color{blue}
M. Kobayashi. {\it On the Navier-Stokes equations on manifolds with curvature,} J. Eng. Math. (2008) 60:55–68.}

\bibitem{LG} {\color{blue}
O. A. Ladyzhenskaya and G. A. Seregin, {\it On partial regularity of suitable weak
solutions to the three-dimensional Navier–Stokes equations,} J. Math. Fluid Mech. 1
(1999), 356-387.}

\end{thebibliography}

\end{document}
