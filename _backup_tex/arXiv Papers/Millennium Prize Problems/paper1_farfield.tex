\documentclass[11pt]{article}
\input{riemann_common_preamble.tex}

\title{A certified zero-free region for the Riemann zeta function\\in the half-plane $\Re s \ge 0.6$}
\author{Jonathan Washburn}
\date{January 2, 2026}

\begin{document}
\maketitle

\begin{abstract}
We prove unconditionally that the Riemann zeta function $\zeta(s)$ has no zeros in the fixed half-plane $\{\,\Re s \ge 0.6\,\}$.
The argument is function-theoretic.
On $\Omega=\{\,\Re s>\tfrac12\,\}$ we form an arithmetic ratio $\mathcal J(s)$ whose poles encode zeros of $\zeta$, and pass to its Cayley transform $\Theta(s)=(2\mathcal J(s)-1)/(2\mathcal J(s)+1)$.
A Schur bound $|\Theta|\le 1$ on a domain forces $\mathcal J$ to be pole-free there by removability (a Schur/Herglotz pinch), hence excludes zeros.
Accordingly, the analytic task is to certify a Schur bound on a half-plane containing $\{\,\Re s\ge 0.6\,\}$.
This is discharged by rigorous ball arithmetic and machine-checkable JSON artifacts:
a finite arithmetic Pick-matrix certificate at $\sigma_0=0.599$ (with a strict spectral gap and a certified tail bound) yields the Schur property of the default Cayley field $\Theta_{\rm raw}$ on the full half-plane $\{\,\Re s>0.599\,\}$, which contains $\{\,\Re s\ge 0.6\,\}$.
We also supply an independent rectangle certification and independent Pick certificates at $\sigma_0=0.6$ and $\sigma_0=0.7$ as audit checkpoints.
See the repository \texttt{README.md} for an audit manifest and the verifier commands.
\end{abstract}

\section{Introduction}

The Riemann zeta function
\[
  \zeta(s)\;=\;\sum_{n\ge 1}\frac{1}{n^s},\qquad \Re s>1,
\]
extends meromorphically to $\C$ with a simple pole at $s=1$ and satisfies a functional equation after completion.
Its nontrivial zeros govern the finest fluctuations in the distribution of prime numbers, and the Riemann Hypothesis asserts that all such zeros lie on the critical line $\Re s=\tfrac12$; see \cite{Titchmarsh,IK} for background.

This paper isolates an unconditional, fixed-strip statement in the direction of RH.
Unlike classical zero-free regions near $\Re s=1$ (which are asymptotic in height), the result here is a \emph{uniform} half-plane exclusion at $\Re s\ge 0.6$.

\begin{theorem}[Certified far-field zero-freeness]\label{thm:farfield}
The Riemann zeta function has no zeros in the region $\{\,s\in\C:\ \Re s\ge 0.6\,\}$.
\end{theorem}

\subsection*{Strategy: Schur pinching via a Cayley field}
We work on the right half-plane $\Omega=\{\,\Re s>\tfrac12\,\}$.
In Section~\ref{sec:defs} we define an arithmetic ratio $\mathcal J$ (in the default \emph{raw $\zeta$-gauge}) with the following two structural properties:
\begin{itemize}
\item \textbf{(normalization at $+\infty$)} $\mathcal J(\sigma+it)\to 1$ as $\sigma\to+\infty$, hence $\Theta(\sigma+it)\to \tfrac13$ (Remark~\ref{rem:Ocan-role});
\item \textbf{(non-cancellation)} $\dettwo(I-A(s))$ is holomorphic and nonvanishing on $\Omega$, so any zero of $\zeta$ in $\Omega$ produces a pole of $\mathcal J$ (Remark~\ref{rem:poles}).
\end{itemize}
We then pass to the Cayley transform
\[
  \Theta(s)\ :=\ \frac{2\mathcal J(s)-1}{2\mathcal J(s)+1}.
\]
The analytic mechanism is a \emph{Schur/Herglotz pinch} proved in Section~\ref{sec:pinch}:
if $\Theta$ is Schur on a domain (i.e.\ $|\Theta|\le 1$) and not identically $1$, then boundedness forces removability of any isolated singularity and prevents poles of $\mathcal J$.
Since $\Theta(\sigma+it)\to \tfrac13$ as $\sigma\to+\infty$, the degenerate possibility $\Theta\equiv 1$ is excluded on the half-planes relevant here.
Therefore, to prove Theorem~\ref{thm:farfield} it suffices to certify a Schur bound for the default Cayley field $\Theta_{\rm raw}$ on some open half-plane $\{\,\Re s>0.6-\varepsilon\,\}$.

\subsection*{Certified inputs (what is rigorously checked)}
The certified input used for the logical implication of Theorem~\ref{thm:farfield} is a finite arithmetic Pick-matrix certificate in the raw $\zeta$-gauge at $\sigma_0=0.599$ (with $N=16$ and a strict spectral gap), which yields the Schur property of $\Theta_{\rm raw}$ on the full half-plane $\{\,\Re s>0.599\,\}$.
Since $\{\,\Re s\ge 0.6\,\}\subset\{\,\Re s>0.599\,\}$, this half-plane certification already covers the claimed region and removes the need for a separate boundary-line tail step.
We also supply additional independent audit checkpoints:
\begin{itemize}
\item \textbf{Finite rectangle.} Rigorous subdivision with complex ball arithmetic certifies $|\Theta_{\rm proj}|<1$ on $[0.6,0.7]\times[0,20]$ (in the \texttt{outer\_zeta\_proj} gauge).
\item \textbf{Independent Pick checks at $\sigma_0=0.6$ and $\sigma_0=0.7$.} Separate Pick-matrix certificates at $\sigma_0=0.6$ and $\sigma_0=0.7$ yield the Schur property on $\{\Re s>0.6\}$ and $\{\Re s>0.7\}$, respectively.
\end{itemize}
These certified inputs are recorded as JSON artifacts and audited by the accompanying verifier; see Section~\ref{sec:hybrid}.

\subsection*{Reproducibility and audit posture}
The certification is intended to be referee-auditable.
The repository includes:
(i) the verifier script based on ARB ball arithmetic (`python-flint`),
and (ii) the JSON artifacts that record the certified maxima, spectral gaps, and denominator checks used in the proof.
The file \texttt{README.md} provides an audit manifest mapping the manuscript’s statements to exact commands and expected outputs.

\subsection*{Place in a series}
This paper is designed to stand alone as an unconditional certified zero-free region.
Two companion papers (not required for Theorem~\ref{thm:farfield}) treat: (a) effective near-field energy barriers and Carleson budgets, and (b) a cutoff principle yielding conditional closure of RH.

\medskip
\noindent The remainder of the paper defines the arithmetic ratio $\mathcal J$ and Cayley field $\Theta$, proves the Schur pinch mechanism, and then discharges the Schur bound via the hybrid certification outlined above.

\section{Definitions and main objects}\label{sec:defs}

This section defines the analytic objects used throughout the proof and records the basic relationships between zeros of $\zeta$ and the bounded-real (Schur/Herglotz) structure.
Nothing in this section is conditional; all definitions are classical.

\subsection*{The completed zeta function and the far half-plane}
Let $\zeta(s)$ denote the Riemann zeta function.
We write $\xi(s)$ for the completed zeta function
\[
  \xi(s)\ :=\ \tfrac12\,s(s-1)\,\pi^{-s/2}\Gamma(s/2)\,\zeta(s),
\]
which is entire and satisfies the functional equation $\xi(s)=\xi(1-s)$; see \cite{Titchmarsh}.
We work primarily on the right half-plane
\[
  \Omega\ :=\ \{\,s\in\C:\ \Re s>\tfrac12\,\}.
\]
Write $Z(\xi):=\{\,s\in\C:\ \xi(s)=0\,\}$ for the zero set of $\xi$.
Theorem~\ref{thm:farfield} concerns the fixed far region $\{\,\Re s\ge 0.6\,\}\subset\Omega$.

\subsection*{The prime-diagonal operator and the regularized determinant}
Let $\PP$ denote the set of primes and write $\ell^2(\PP)$ for the Hilbert space with orthonormal basis $\{e_p\}_{p\in\PP}$.
For $s\in\C$ define the prime-diagonal operator
\[
  A(s):\ell^2(\PP)\to\ell^2(\PP),\qquad A(s)e_p:=p^{-s}e_p.
\]
For $\Re s>1/2$ we have $\|A(s)\|_{\mathrm{HS}}^2=\sum_{p}p^{-2\Re s}<\infty$, so $A(s)$ is Hilbert--Schmidt.
In particular, the regularized determinant $\dettwo(I-A(s))$ is well-defined and holomorphic on $\Omega$; see, e.g., \cite[Ch.~III]{RosenblumRovnyak}.

\subsection*{The arithmetic ratio \texorpdfstring{$\mathcal J$}{J} and the Cayley field \texorpdfstring{$\Theta$}{Theta}}
The central meromorphic object is an arithmetic ratio $\mathcal J(s)$ whose poles capture zeros of $\zeta$ in $\Omega$.
To allow numerically stable certified bounds, we permit a holomorphic nonvanishing \emph{normalizer} (or \emph{gauge}) $\mathcal O$ on the region under discussion and define
\begin{equation}\label{eq:J-def}
  \mathcal{J}(s)\ :=\ \frac{\dettwo(I-A(s))}{\zeta(s)}\cdot \frac{s}{s-1}\cdot \frac{1}{\mathcal O(s)},
\end{equation}
where $\mathcal O$ is chosen so that it is holomorphic and nonvanishing on the region where \eqref{eq:J-def} is used.
Unless explicitly stated otherwise, we work in the \emph{raw $\zeta$-gauge} $\mathcal O\equiv 1$ and denote the resulting objects by $\mathcal J_{\rm raw}$ and $\Theta_{\rm raw}$.
For readability we usually drop the subscript and simply write $\mathcal J$ and $\Theta$ in this default gauge.
On compact regions one may also divide by an auxiliary holomorphic nonvanishing normalizer to improve conditioning; when we do so we write $\mathcal J_{\rm proj}$ and $\Theta_{\rm proj}$ (see Remark~\ref{rem:gauges}).
Since Schur bounds are \emph{not} gauge-invariant, we keep this notation explicit whenever a certified bound is quoted or invoked in the pinch argument.
On any region where the auxiliary normalizer is nonvanishing, such a gauge change does not affect the pole set of $\mathcal J$ (hence does not change which points correspond to zeros of $\zeta$).

\begin{remark}[Role of the normalizer]\label{rem:Ocan-role}
The factor $\mathcal O$ serves only to choose a convenient gauge for $\mathcal J$.
Provided $\mathcal O$ is holomorphic and nonvanishing on a region $D\subset\Omega$, it cannot introduce poles of $\mathcal J$ on $D$.
In particular, in the raw $\zeta$-gauge $\mathcal O\equiv 1$ one has $\mathcal J(s)\to 1$ and hence $\Theta(s)\to 1/3$ as $\Re s\to+\infty$.
\end{remark}

The associated Cayley transform is
\begin{equation}\label{eq:Theta-def}
  \Theta(s)\ :=\ \frac{2\mathcal J(s)-1}{2\mathcal J(s)+1}.
\end{equation}
Heuristically, $\mathcal J$ plays the role of a Herglotz-type quantity and $\Theta$ the role of the corresponding Schur function.
The proof uses the following simple implication: a Schur bound on $\Theta$ prevents poles of $\mathcal J$ by a removability pinch.

\begin{remark}[Zeros of $\zeta$ produce poles of $\mathcal J$]\label{rem:poles}
If $\rho\in\Omega$ is a zero of $\zeta(s)$, then $\rho$ is a pole of $\mathcal J(s)$ provided the numerator factors in \eqref{eq:J-def} are nonzero at $\rho$.
For $\Re\rho>1/2$ one has $\dettwo(I-A(\rho))\neq 0$: for diagonal $A(s)$,
$\dettwo(I-A(s))=\prod_{p}(1-p^{-s})\,e^{p^{-s}}$ and $\sum_{p}|\log(1-p^{-s})+p^{-s}|<\infty$ on $\Omega$; in particular $\dettwo(I-A(s))$ is holomorphic and zero-free on $\Omega$.
Also $\mathcal O(\rho)\neq 0$ by the nonvanishing assumption on the chosen gauge.
Thus zeros of $\zeta$ in $\Omega$ correspond to poles of $\mathcal J$, and hence to points where $\Theta$ cannot extend holomorphically unless the pole is ruled out.
\end{remark}

\subsection*{Schur and Herglotz classes (terminology)}
Let $D\subset\C$ be a domain.
A holomorphic function $\Theta$ on $D$ is called \emph{Schur} if $|\Theta|\le 1$ on $D$.
A holomorphic function $H$ on $D$ is called \emph{Herglotz} if $\Re H\ge 0$ on $D$.
The Cayley transform identifies these classes: if $H$ is Herglotz and $H\not\equiv -1$, then
\[
  \Theta=\frac{H-1}{H+1}
\]
is Schur.
Conversely, if $\Theta$ is Schur and $\Theta\not\equiv 1$, then $(1+\Theta)/(1-\Theta)$ is Herglotz; see \cite{Donoghue,RosenblumRovnyak}.

\subsection*{Outline of the far-field strategy in this language}
Theorem~\ref{thm:farfield} will follow once we establish that $\Theta$ is Schur on $\{\,\Re s>0.6\,\}$.
Indeed, if $|\Theta|\le 1$ holds on $\{\,\Re s>0.6\,\}$ away from the poles of $\mathcal J$, then boundedness forces removability across any isolated singularity.
Since poles of $\mathcal J$ correspond to zeros of $\zeta$ in $\Omega$ (Remark~\ref{rem:poles}), this prevents zeros of $\zeta$ in the far region.
The precise pinch argument is proved in the next section.

\section{Schur/Herglotz pinch mechanism}\label{sec:pinch}

This section records the analytic mechanism that converts a Schur bound for the Cayley field $\Theta$ into a zero-free region for $\zeta$.
The key point is simple: a holomorphic function bounded by $1$ cannot have a pole, and any isolated singularity is removable.
In our setting, poles of $\mathcal J$ in $\Omega$ encode zeros of $\zeta$ (Remark~\ref{rem:poles}), so a Schur bound forces those zeros to be absent.

\subsection*{Removable singularities under a Schur bound}
\begin{lemma}[Removable singularity under Schur bound]\label{lem:removable-schur-p1}
Let $D\subset\C$ be a disc centered at $\rho$ and let $\Theta$ be holomorphic on $D\setminus\{\rho\}$ with $|\Theta|<1$ there.
Then $\Theta$ extends holomorphically to $D$.
In particular, the Cayley inverse $(1+\Theta)/(1-\Theta)$ extends holomorphically to $D$ and has nonnegative real part on $D$.
\end{lemma}
\begin{proof}
Since $\Theta$ is bounded on the punctured disc $D\setminus\{\rho\}$, Riemann's removable singularity theorem yields a holomorphic extension of $\Theta$ to $D$.
Where $|\Theta|<1$, the Möbius map $w\mapsto (1+w)/(1-w)$ sends the unit disc into the right half-plane, hence $\Re\frac{1+\Theta}{1-\Theta}\ge 0$ on $D\setminus\{\rho\}$; continuity extends the inequality across $\rho$.
\end{proof}

\subsection*{From a Schur bound to absence of poles}
We will use Lemma~\ref{lem:removable-schur-p1} in the following form: if $\Theta$ is Schur on a domain $U$ and holomorphic on $U\setminus S$ where $S$ is a discrete set, then $\Theta$ extends holomorphically across $S$ and remains Schur on all of $U$.
Thus a Schur bound rules out poles of any meromorphic object that can be expressed as a Cayley inverse of $\Theta$.

\begin{corollary}[Schur bound prevents poles of $\mathcal J$]\label{cor:no-poles}
Let $U\subset\Omega$ be a domain and suppose that $\Theta$ is meromorphic on $U$ and satisfies $|\Theta|\le 1$ on $U$ away from its poles.
Then $\Theta$ extends holomorphically to $U$ and satisfies $|\Theta|\le 1$ on $U$.
Moreover, the Cayley inverse
\[
  2\mathcal J \;=\; \frac{1+\Theta}{1-\Theta}
\]
extends holomorphically to $U$ with $\Re(2\mathcal J)\ge 0$ on $U$; in particular $\mathcal J$ has no poles in $U$.
\end{corollary}
\begin{proof}
The poles of a meromorphic function form a discrete subset of $U$.
On each punctured disc around a pole, $\Theta$ is bounded by $1$, hence removable by Lemma~\ref{lem:removable-schur-p1}.
Therefore $\Theta$ extends holomorphically across all its poles and is holomorphic on $U$.
The Schur bound persists by continuity.
 The Cayley inverse is holomorphic wherever $\Theta\neq 1$ and has nonnegative real part on $U$.
If $\Theta(s_0)=1$ at some point $s_0\in U$, then $|\Theta|$ attains its maximum at an interior point, so $\Theta\equiv 1$ on $U$ by the Maximum Modulus Principle.
In the applications below this is excluded (e.g.\ on any right half-plane $U$, Remark~\ref{rem:Ocan-role} gives $\Theta(s)\to \tfrac13$ as $\Re s\to+\infty$), hence $\Theta\neq 1$ on $U$ and the Cayley inverse extends holomorphically to $U$ with $\Re(2\mathcal J)\ge 0$.
In particular $\mathcal J$ has no poles in $U$.
\end{proof}

\subsection*{Conclusion: Schur on the far half-plane implies Theorem~\ref{thm:farfield}}
We now connect the pinching mechanism to $\zeta$.
By Remark~\ref{rem:poles}, any zero $\rho$ of $\zeta$ in $\Omega$ produces a pole of $\mathcal J$ in $\Omega$ (the numerator factors in \eqref{eq:J-def} are nonzero on $\Omega$).
Therefore, if we can certify a Schur bound for $\Theta_{\rm raw}$ on a half-plane $U_\varepsilon=\{\,\Re s>0.6-\varepsilon\,\}$ with some $\varepsilon>0$, Corollary~\ref{cor:no-poles} implies $\mathcal J_{\rm raw}$ has no poles in $U_\varepsilon$, hence $\zeta$ has no zeros in $U_\varepsilon$.
Since $\{\,\Re s\ge 0.6\,\}\subset U_\varepsilon$, this yields Theorem~\ref{thm:farfield}.
The next section discharges the Schur bound on $U_{0.599}$ by the hybrid certification (rectangle + Pick matrix + auxiliary checkpoints).

\section{Hybrid certification: rectangle + Pick + asymptotics}\label{sec:hybrid}

This section records the certified inputs used to discharge the Schur bound required in Corollary~\ref{cor:no-poles}.
The \emph{primary} certified input is the strict Pick-matrix gap at $\sigma_0=0.599$ (Proposition~\ref{prop:pick-gap0599}), which yields the Schur property of $\Theta_{\rm raw}$ on the entire open half-plane
\[
  U_{0.599}\ :=\ \{\,s\in\C:\ \Re s>0.599\,\}.
\]
Since $\{\,\Re s\ge 0.6\,\}\subset U_{0.599}$, this already suffices for Theorem~\ref{thm:farfield} and removes the need for a separate large-$|t|$ tail lemma to treat the boundary line $\Re s=0.6$.
In addition, we supply auxiliary audit checkpoints:
\begin{itemize}
\item a rigorous interval-arithmetic rectangle certification on $R_\ast=[0.6,0.7]\times[-20,20]$ (Lemma~\ref{lem:rect-cert}), which also certifies denominator nonvanishing for the chosen normalization on that cover;
\item independent Pick certificates at $\sigma_0=0.6$ and $\sigma_0=0.7$ (Propositions~\ref{prop:pick-gap06} and \ref{prop:pick-gap}), included as cross-checks.
\end{itemize}

\subsection*{Domain decomposition}
Let $T_\ast:=20$ and write $s=\sigma+it$.
Set
\[
  R_\ast\ :=\ [0.6,0.7]\times[-T_\ast,T_\ast].
\]
Proposition~\ref{prop:pick-gap0599} certifies that $\Theta_{\rm raw}$ is Schur on the full open half-plane $U_{0.599}$.
Lemma~\ref{lem:rect-cert} is an independent certified checkpoint on the finite rectangle $R_\ast$ (in the \texttt{outer\_zeta\_proj} gauge), and is not needed for the logical implication of Theorem~\ref{thm:farfield}.

\subsection*{Certified rectangle bound (interval arithmetic)}
\begin{lemma}[Rectangle certification]\label{lem:rect-cert}
On the rectangle $[0.6,0.7]\times[0,T_\ast]$ one has the certified bound
\[
  |\Theta_{\rm proj}(s)|\ \le\ 0.9999928763\ <\ 1,
\]
where $\Theta_{\rm proj}$ denotes the Cayley field computed from \eqref{eq:J-def} in the \texttt{outer\_zeta\_proj} gauge (so $\mathcal O=\mathcal O_{\mathrm{proj}}$ on this rectangle).
By conjugation symmetry $\Theta_{\rm proj}(\overline{s})=\overline{\Theta_{\rm proj}(s)}$, the same bound holds on $R_\ast=[0.6,0.7]\times[-T_\ast,T_\ast]$.
\end{lemma}
\begin{proof}
This is verified by rigorous complex ball arithmetic on a recursive subdivision of the rectangle, implemented in \texttt{verify\_attachment\_arb.py} (\texttt{theta\_certify} mode) and recorded in the JSON artifact
\texttt{theta\_certify\_sigma06\_07\_t0\_20\_outer\_zeta\_proj.json}.
The bound quoted is the artifact’s certified upper envelope \texttt{theta\_hi}.
\emph{Well-definedness on the rectangle.}
The same certification run also checks that the denominator quantities used to form $\Theta_{\rm proj}$ do not vanish on the rectangle cover (in particular, the enclosures for $\zeta(s)$ and the gauge factor $\mathcal O_{\mathrm{proj}}(s)$ do not contain $0$ on each certified box).
This is recorded in the artifact’s \texttt{denominators} fields and ensures that $\Theta_{\rm proj}$ is holomorphic on the rectangle, so the reported $\sup|\Theta_{\rm proj}|$ bound applies on the entire certified cover.
For example, the shipped artifact reports the certified lower bounds
\[
  \min_{R_\ast}|\zeta(s)|\ \ge\ 0.00839,\qquad \min_{R_\ast}|\mathcal O_{\mathrm{proj}}(s)|\ \ge\ 0.0316,
\]
recorded as \texttt{min\_abs\_zeta\_lower} and \texttt{min\_abs\_O\_lower} (with $R_\ast=[0.6,0.7]\times[-20,20]$).
Conjugation symmetry follows from $\zeta(\overline{s})=\overline{\zeta(s)}$ and the fact that all constructions in \eqref{eq:J-def} (with $\mathcal O=\mathcal O_{\mathrm{proj}}$) respect conjugation.
\end{proof}

\begin{remark}[Normalizations (gauges) used in the artifacts]\label{rem:gauges}
The verifier supports several normalizations of the arithmetic ratio $\mathcal J$ that differ by multiplication by a holomorphic factor.
On any region where this factor is nonvanishing, such a change does not alter the pole set of $\mathcal J$ (hence does not change the implication ``a zero of $\zeta$ produces a pole''), but it can substantially improve numerical stability of the Cayley field $\Theta$.
In this paper, the Pick certificates are performed in the \texttt{raw\_zeta} gauge, which corresponds to the default choice $\mathcal O\equiv 1$ and hence certifies the Schur property of $\Theta_{\rm raw}$ on right half-planes.
The rectangle certification is performed in the gauge reported by its artifact (here \texttt{outer\_zeta\_proj}), which corresponds to taking $\mathcal O=\mathcal O_{\mathrm{proj}}$ and hence certifies a Schur bound for $\Theta_{\rm proj}$ on the certified rectangle.
Since the rectangle artifact also certifies that $\mathcal O_{\mathrm{proj}}$ is nonvanishing on the rectangle cover (Lemma~\ref{lem:rect-cert}), the pole set of $\mathcal J_{\rm proj}$ on that rectangle agrees with the pole set of $\mathcal J_{\rm raw}$ there.
Accordingly, when we invoke a certified Schur bound in the pinch argument we always use the matching Cayley field ($\Theta_{\rm raw}$ for the Pick half-plane step, and $\Theta_{\rm proj}$ for the finite-rectangle checkpoint), so there is no ambiguity about which $\Theta$ is meant.
\end{remark}

\subsection*{Pick certificate on the right half-plane \texorpdfstring{$\{\Re s>0.7\}$}{Re s>0.7}}
We use the classical Nevanlinna--Pick/Schur-kernel criterion on the unit disc.
Let $\sigma_0:=0.7$ and set $D_{\sigma_0}:=\{\,\Re s>\sigma_0\,\}$.
Consider the disk chart (a biholomorphism) given by
\[
  s_{\sigma_0}(z)\ :=\ \sigma_0+\frac{1+z}{1-z},
  \qquad
  z_{\sigma_0}(s)\ :=\ \frac{s-(\sigma_0+1)}{s-(\sigma_0-1)}.
\]
Define the disk pullback $\theta_{\sigma_0}(z):=\Theta_{\rm raw}(s_{\sigma_0}(z))$, which is holomorphic near $z=0$ since $s_{\sigma_0}(0)=\sigma_0+1>1$.
The associated Schur/Pick kernel is
\[
  K_{\sigma_0}(z,w)\ :=\ \frac{1-\theta_{\sigma_0}(z)\,\overline{\theta_{\sigma_0}(w)}}{1-z\overline w},
  \qquad z,w\in\mathbb D.
\]
By the Pick criterion (see \cite[Ch.~2]{RosenblumRovnyak} or \cite[Ch.~III]{Donoghue}), $\theta_{\sigma_0}$ is Schur on $\mathbb D$ if and only if $K_{\sigma_0}$ is positive semidefinite.
Expanding $K_{\sigma_0}(z,w)=\sum_{i,j\ge 0}P_{ij}(\sigma_0)\,z^i\overline w^{\,j}$ yields an infinite Hermitian \emph{Pick matrix} $P(\sigma_0)=[P_{ij}(\sigma_0)]_{i,j\ge 0}$.

\begin{lemma}[Coefficient formula for the Pick matrix]\label{lem:pick-matrix-coeff-formula-p1}
Let $\theta(z)=\sum_{n\ge 0} a_n z^n$ be holomorphic on $\mathbb D$ and let $P=[P_{ij}]_{i,j\ge 0}$ be the coefficient matrix of
\[
  K(z,w)\ =\ \frac{1-\theta(z)\,\overline{\theta(w)}}{1-z\overline w}.
\]
Then for all $i,j\ge 0$,
\[
  P_{ij}\ =\ \delta_{ij}\ -\ \sum_{k=0}^{\min\{i,j\}} a_{i-k}\,\overline{a_{j-k}}.
\]
Equivalently, if $A$ denotes the lower-triangular Toeplitz matrix $A_{ij}=a_{i-j}$ for $i\ge j$ and $A_{ij}=0$ for $i<j$, then
\[
  P\ =\ I\ -\ A A^*.
\]
\end{lemma}
\begin{proof}
Use the geometric series expansion $(1-z\overline w)^{-1}=\sum_{r\ge 0} z^r\overline w^{\,r}$ and multiply out
\[
  K(z,w)\ =\ \sum_{r\ge 0} z^r\overline w^{\,r}\ -\ \sum_{m,n\ge 0} a_m\overline{a_n}\sum_{r\ge 0} z^{m+r}\overline w^{\,n+r}.
\]
Collecting coefficients of $z^i\overline w^{\,j}$ gives the stated formula.
The matrix identity $P=I-AA^*$ is the same statement in operator form.
\end{proof}

\begin{lemma}[Tail-to-infinite stability (kernel/Gram matrix form)]\label{lem:pick-tail-stability}
Let $\theta(z)=\sum_{n\ge 0} a_n z^n$ be holomorphic on $\mathbb D$ and let $P(\theta)$ be the associated infinite Pick matrix (Definition above).
Fix $N\ge 1$ and write $\theta=\theta^{(\le N-1)}+\theta^{(\ge N)}$ with
\[
  \theta^{(\ge N)}(z)\ :=\ \sum_{n\ge N} a_n z^n.
\]
Assume the \emph{weighted tail} bound
\begin{equation}\label{eq:epsN-def}
  \varepsilon_N^2\ :=\ \sum_{n\ge N}(n+1)\,|a_n|^2\ <\ \infty.
\end{equation}
Then there is an absolute constant $C\le 2$ such that the difference of Pick matrices
\[
  P(\theta)\ -\ P\!\big(\theta^{(\le N-1)}\big)
\]
defines a bounded self-adjoint operator on $\ell^2(\mathbb N_0)$ with operator norm at most $C\,\varepsilon_N$.
\end{lemma}
\begin{proof}
Write the kernel difference explicitly:
\[
  \frac{1-\theta(z)\overline{\theta(w)}}{1-z\overline w}
  \;-\;
  \frac{1-\theta^{(\le N-1)}(z)\overline{\theta^{(\le N-1)}(w)}}{1-z\overline w}
  \;=\;
  -\frac{\theta^{(\le N-1)}(z)\overline{\theta^{(\ge N)}(w)}+\theta^{(\ge N)}(z)\overline{\theta^{(\le N-1)}(w)}+\theta^{(\ge N)}(z)\overline{\theta^{(\ge N)}(w)}}{1-z\overline w}.
\]
Each summand on the right has the form $f(z)\overline{g(w)}/(1-z\overline w)$ with at least one of $f,g$ equal to $\theta^{(\ge N)}$.
The associated coefficient matrix defines a bounded operator on $\ell^2$, and its operator norm is controlled (up to an absolute constant) by the $\ell^2$-Dirichlet size of the tail coefficients \eqref{eq:epsN-def}; one convenient route is to view these coefficient matrices as Gram matrices of shifted coefficient vectors and apply Cauchy--Schwarz together with the standard identification of the Dirichlet tail \(\sum_{n\ge N}(n+1)|a_n|^2\) as a Hilbert--Schmidt control quantity for the corresponding tail blocks (see, e.g., \cite[Ch.~2]{RosenblumRovnyak} for the kernel/Gram-matrix viewpoint and \cite[Ch.~III]{Donoghue} for related operator-theoretic estimates).
Combining the three terms and using $\|T\|\le \|T\|_{\mathrm{HS}}$ on Hilbert--Schmidt operators yields a bound of the form $\|P(\theta)-P(\theta^{(\le N-1)})\|\le C\,\varepsilon_N$ with an absolute $C$; for our application it suffices that one may take $C\le 2$.
\end{proof}

\begin{proposition}[Finite Pick gap $+$ tail bound $\Rightarrow$ infinite Pick positivity]\label{prop:pick-global-positivity}
Let $\theta$ be holomorphic on $\mathbb D$ with infinite Pick matrix $P(\theta)$.
Fix $N\ge 1$ and assume:
\begin{itemize}
\item \textbf{(finite gap)} the $N\times N$ principal minor satisfies $P_N(\theta)\succeq \delta\,I_N$ for some $\delta>0$;
\item \textbf{(tail bound)} the tail quantity $\varepsilon_N$ defined in \eqref{eq:epsN-def} is finite and satisfies $C\,\varepsilon_N<\delta$, where $C$ is the constant from Lemma~\ref{lem:pick-tail-stability}.
\end{itemize}
Then the full infinite Pick matrix $P(\theta)$ is positive semidefinite.
Consequently $\theta$ is Schur on $\mathbb D$.
\end{proposition}
\begin{proof}
By Lemma~\ref{lem:pick-tail-stability}, the infinite Pick matrix $P(\theta)$ is a bounded self-adjoint perturbation of $P(\theta^{(\le N-1)})$ with perturbation norm at most $C\varepsilon_N$.
The finite-gap hypothesis implies in particular that the $N\times N$ head block has a strict positive margin $\delta$.
Since $C\varepsilon_N<\delta$, a standard stability argument for positive operators under self-adjoint perturbations (e.g.\ via the quadratic-form bound $|\langle Ex,x\rangle|\le \|E\|\|x\|^2$ and the block decomposition into degrees $<N$ and $\ge N$) shows that $P(\theta)$ remains positive semidefinite.
The Pick criterion then gives that $\theta$ is Schur on $\mathbb D$.
\end{proof}

\begin{proposition}[Certified Pick gap at $\sigma_0=0.7$]\label{prop:pick-gap}
The accompanying Pick artifact certifies the following data at $\sigma_0=0.7$:
\begin{itemize}
\item the $16\times 16$ principal minor satisfies $P_{16}(\sigma_0)\succeq \delta_{\mathrm{cert}} I$ with $\delta_{\mathrm{cert}}=0.6273368612$;
\item the coefficient tail bound $\sum_{n\ge 16}(n+1)|a_n(\sigma_0)|^2\le 0.01272011$, where $\theta_{\sigma_0}(z)=\sum_{n\ge 0}a_n(\sigma_0)z^n$.
\end{itemize}
Consequently $P(\sigma_0)$ is positive semidefinite, $\theta_{\sigma_0}$ is Schur on $\mathbb D$, and hence $\Theta_{\rm raw}$ is Schur on $D_{\sigma_0}=\{\,\Re s>0.7\,\}$.
\end{proposition}
\begin{proof}
The two numerical inequalities are certified by interval arithmetic and directed-rounding linear algebra and are recorded in the JSON artifact
\texttt{pick\_sigma07\_raw\_zeta\_N16.json} produced by \texttt{verify\_attachment\_arb.py} (\texttt{pick\_certify} mode).
Define $\varepsilon_{16}^2:=\sum_{n\ge 16}(n+1)\,|a_n(\sigma_0)|^2$, so $\varepsilon_{16}\le \sqrt{0.01272011}<0.113$.
With the constant $C\le 2$ from Lemma~\ref{lem:pick-tail-stability}, we have $C\varepsilon_{16}\le 2\varepsilon_{16}<0.226<0.627=\delta_{\mathrm{cert}}$.
Proposition~\ref{prop:pick-global-positivity} therefore implies $P(\sigma_0)\succeq 0$, hence $\theta_{\sigma_0}$ is Schur on $\mathbb D$, and thus $\Theta_{\rm raw}$ is Schur on $D_{\sigma_0}$.
\end{proof}

\begin{proposition}[Certified Pick gap at $\sigma_0=0.6$]\label{prop:pick-gap06}
Repeating the preceding construction with $\sigma_0:=0.6$, the accompanying Pick artifact certifies the following data:
\begin{itemize}
\item the $16\times 16$ principal minor satisfies $P_{16}(\sigma_0)\succeq \delta_{\mathrm{cert}} I$ with $\delta_{\mathrm{cert}}=0.5948779179$;
\item the coefficient tail bound $\sum_{n\ge 16}(n+1)|a_n(\sigma_0)|^2\le 0.0136892106$, where $\theta_{\sigma_0}(z)=\sum_{n\ge 0}a_n(\sigma_0)z^n$.
\end{itemize}
Consequently $P(\sigma_0)$ is positive semidefinite, $\theta_{\sigma_0}$ is Schur on $\mathbb D$, and hence $\Theta_{\rm raw}$ is Schur on $D_{\sigma_0}=\{\,\Re s>0.6\,\}$.
\end{proposition}
\begin{proof}
This certification is recorded in the JSON artifact \texttt{pick\_sigma06\_raw\_zeta\_N16.json} produced by \texttt{verify\_attachment\_arb.py} (\texttt{pick\_certify} mode) with gauge \texttt{raw\_zeta}.
With $\varepsilon_{16}^2:=\sum_{n\ge 16}(n+1)|a_n(\sigma_0)|^2$ we have $\varepsilon_{16}\le \sqrt{0.0136892106}<0.117$, hence $C\varepsilon_{16}\le 2\varepsilon_{16}<0.234<0.594=\delta_{\mathrm{cert}}$.
Proposition~\ref{prop:pick-global-positivity} therefore implies $P(\sigma_0)\succeq 0$, hence $\theta_{\sigma_0}$ is Schur on $\mathbb D$, and thus $\Theta_{\rm raw}$ is Schur on $D_{\sigma_0}$.
\end{proof}

\begin{proposition}[Certified Pick gap at $\sigma_0=0.599$]\label{prop:pick-gap0599}
Repeating the preceding construction with $\sigma_0:=0.599$, the accompanying Pick artifact certifies the following data:
\begin{itemize}
\item the $16\times 16$ principal minor satisfies $P_{16}(\sigma_0)\succeq \delta_{\mathrm{cert}} I$ with $\delta_{\mathrm{cert}}=0.5944375203$;
\item the coefficient tail bound $\sum_{n\ge 16}(n+1)|a_n(\sigma_0)|^2\le 0.0137007384$, where $\theta_{\sigma_0}(z)=\sum_{n\ge 0}a_n(\sigma_0)z^n$.
\end{itemize}
Consequently $P(\sigma_0)$ is positive semidefinite, $\theta_{\sigma_0}$ is Schur on $\mathbb D$, and hence $\Theta_{\rm raw}$ is Schur on $D_{\sigma_0}=\{\,\Re s>0.599\,\}$.
\end{proposition}
\begin{proof}
This certification is recorded in the JSON artifact \texttt{pick\_sigma0599\_raw\_zeta\_N16.json} produced by \texttt{verify\_attachment\_arb.py} (\texttt{pick\_certify} mode) with gauge \texttt{raw\_zeta}.
With $\varepsilon_{16}^2:=\sum_{n\ge 16}(n+1)|a_n(\sigma_0)|^2$ we have $\varepsilon_{16}\le \sqrt{0.0137007384}<0.118$, hence $C\varepsilon_{16}\le 2\varepsilon_{16}<0.236<0.594=\delta_{\mathrm{cert}}$.
Proposition~\ref{prop:pick-global-positivity} therefore implies $P(\sigma_0)\succeq 0$, hence $\theta_{\sigma_0}$ is Schur on $\mathbb D$, and thus $\Theta_{\rm raw}$ is Schur on $D_{\sigma_0}$.
\end{proof}

\subsection*{Hybrid Schur bound and conclusion}
\begin{proposition}[Hybrid Schur certification on $\{\,\Re s>0.599\,\}$]\label{prop:hybrid-schur}
The arithmetic Cayley field $\Theta_{\rm raw}$ satisfies $|\Theta_{\rm raw}(s)|\le 1$ for all $s\in\{\,\Re s>0.599\,\}$.
\end{proposition}
\begin{proof}
Let $s\in\{\,\Re s>0.599\,\}$.
Proposition~\ref{prop:pick-gap0599} certifies that $\Theta_{\rm raw}$ is Schur on $D_{\sigma_0}=\{\,\Re s>0.599\,\}$, hence $|\Theta_{\rm raw}(s)|\le 1$.
\end{proof}

\begin{proof}[Proof of Theorem~\ref{thm:farfield}]
By Proposition~\ref{prop:hybrid-schur}, $\Theta_{\rm raw}$ is Schur on $U_{0.599}=\{\,\Re s>0.599\,\}$.
Applying Corollary~\ref{cor:no-poles} on $U_{0.599}$ shows that $\mathcal J_{\rm raw}$ has no poles there, hence $\zeta$ has no zeros there (Remark~\ref{rem:poles}).
Since $\{\,\Re s\ge 0.6\,\}\subset U_{0.599}$, this implies $\zeta(s)\neq 0$ for all $\Re s\ge 0.6$.
\end{proof}

\begin{table}[H]
\centering
\caption{Certified far-field artifact data.}\label{tab:artifact-data}
\small
\begin{tabular}{l l l}
\toprule
\textbf{Artifact} & \textbf{Parameter} & \textbf{Value} \\
\midrule
\multicolumn{3}{l}{\textit{Rectangle certification} (\texttt{theta\_certify})} \\
\quad Domain & $[\sigma_{\min}, \sigma_{\max}] \times [t_{\min}, t_{\max}]$ & $[0.6, 0.7] \times [0, 20]$ \\
\quad Certified upper bound & $\max |\Theta_{\rm proj}|$ & $0.9999928763$ \\
\quad Safety margin & $1 - \theta_{\rm hi}$ & $7.12 \times 10^{-6}$ \\
\quad Status & \texttt{ok} & \texttt{true} \\
\quad Boxes processed & & 380{,}764 \\
\quad Precision & (bits) & 260 \\
\quad Gauge & & \texttt{outer\_zeta\_proj} \\
\midrule
\multicolumn{3}{l}{\textit{Pick certificate} (\texttt{pick\_certify}, $\sigma_0 = 0.7$)} \\
\quad Matrix size & $N$ & 16 \\
\quad Spectral gap & $\delta_{\rm cert}$ & $0.6273368612$ \\
\quad SPD at origin & $P_N \succ 0$ & \texttt{true} \\
\quad Coefficient radius & $\rho$ & $0.1$ \\
\quad Coefficient bound & $\rho_{\rm bound}$ & $0.2$ \\
\quad Gauge & & \texttt{raw\_zeta} \\
\quad Precision & (bits) & 260 \\
\quad Tail bound & $\sum_{n\ge 16}(n+1)|a_n|^2$ & $\le 0.01272011$ \\
\midrule
\multicolumn{3}{l}{\textit{Pick certificate} (\texttt{pick\_certify}, $\sigma_0 = 0.6$)} \\
\quad Matrix size & $N$ & 16 \\
\quad Spectral gap & $\delta_{\rm cert}$ & $0.5948779179$ \\
\quad SPD at origin & $P_N \succ 0$ & \texttt{true} \\
\quad Coefficient radius & $\rho$ & $0.1$ \\
\quad Coefficient bound & $\rho_{\rm bound}$ & $0.2$ \\
\quad Gauge & & \texttt{raw\_zeta} \\
\quad Precision & (bits) & 260 \\
\quad Tail bound & $\sum_{n\ge 16}(n+1)|a_n|^2$ & $\le 0.0136892106$ \\
\midrule
\multicolumn{3}{l}{\textit{Pick certificate} (\texttt{pick\_certify}, $\sigma_0 = 0.599$)} \\
\quad Matrix size & $N$ & 16 \\
\quad Spectral gap & $\delta_{\rm cert}$ & $0.5944375203$ \\
\quad SPD at origin & $P_N \succ 0$ & \texttt{true} \\
\quad Coefficient radius & $\rho$ & $0.1$ \\
\quad Coefficient bound & $\rho_{\rm bound}$ & $0.2$ \\
\quad Gauge & & \texttt{raw\_zeta} \\
\quad Precision & (bits) & 260 \\
\quad Tail bound & $\sum_{n\ge 16}(n+1)|a_n|^2$ & $\le 0.0137007384$ \\
\bottomrule
\end{tabular}
\end{table}

\begin{remark}[Artifact reproducibility and verification]\label{rem:artifact-repro}
The certifications summarized in Table~\ref{tab:artifact-data} are generated by the repository verifier \texttt{verify\_attachment\_arb.py} using ARB ball arithmetic (via \texttt{python-flint}).
The repository also includes the JSON artifact files:
\url{theta_certify_sigma06_07_t0_20_outer_zeta_proj.json},
\url{pick_sigma0599_raw_zeta_N16.json},
\url{pick_sigma06_raw_zeta_N16.json}, and
\url{pick_sigma07_raw_zeta_N16.json}.
For an audit-oriented manifest (exact commands and expected outputs), see \texttt{README.md} in the repository.
\end{remark}

\section*{Conclusion and limitations (unconditional status)}

We have proved an unconditional, fixed half-plane zero-free region for the Riemann zeta function: $\zeta(s)\neq 0$ for $\Re s\ge 0.6$ (Theorem~\ref{thm:farfield}).
The argument is function-theoretic: zeros are converted into poles of an arithmetic ratio $\mathcal J$, and a Schur bound $|\Theta|\le 1$ for the associated Cayley field forces removability and rules out poles (hence zeros).
The only ``hard'' step is establishing the Schur bound, which is discharged by the certified inputs in Section~\ref{sec:hybrid} and audited by the artifacts summarized in Table~\ref{tab:artifact-data}.
The primary audit path is the strict Pick-matrix gap at $\sigma_0=0.599$ (Proposition~\ref{prop:pick-gap0599}), which certifies the Schur property on the full open half-plane $\{\Re s>0.599\}$ and hence implies the claimed exclusion on $\{\Re s\ge 0.6\}$.
The rectangle certification and the independent Pick certificates at $\sigma_0=0.6$ and $\sigma_0=0.7$ serve as additional checkpoints (cross-validation and numerical stability diagnostics).

\paragraph{Computer assistance and auditability.}
Although the proof uses numerical computation, it is intended to be unconditional in the usual mathematical sense: the computation is \emph{rigorous} interval arithmetic (ball arithmetic) and produces certified inequalities (e.g.\ ``$\max|\Theta_{\rm proj}|<1$'' on a rectangle, and ``a Pick matrix has a strictly positive spectral gap'').
The repository provides a verifier and the corresponding JSON artifacts so that the finite checks can be independently audited.

\paragraph{Limitations and scope.}
This paper does not prove the Riemann Hypothesis.
It isolates and certifies a fixed far-field exclusion $\Re s\ge 0.6$.
Pushing the boundary $0.6$ closer to $1/2$ would require enlarging and/or strengthening the certified inputs (for example, producing a strict Pick gap at a smaller $\sigma_0$ and/or extending the interval certification to new rectangles), which we do not pursue here.
The companion papers in this series treat (i) effective near-field barriers in the strip $1/2<\Re s<0.6$ and (ii) a conditional all-heights closure based on an explicit cutoff hypothesis.

\section*{Statements and Declarations}

\paragraph{Competing interests.}
The author declares no competing interests.

\paragraph{Data and materials availability.}
All computational artifacts used in the far-field certification are included in the repository:
\begin{quote}\small\ttfamily
theta\_certify\_sigma06\_07\_t0\_20\_outer\_zeta\_proj.json\\
pick\_sigma0599\_raw\_zeta\_N16.json\\
pick\_sigma06\_raw\_zeta\_N16.json\\
pick\_sigma07\_raw\_zeta\_N16.json\\
verify\_attachment\_arb.py
\end{quote}

\paragraph{Reproducibility.}
The verifier is based on rigorous ball arithmetic (ARB via \texttt{python-flint}) and is intended to be independently auditable.
See Remark~\ref{rem:artifact-repro} and Appendix~\ref{app:audit} for a referee-facing audit manifest (commands and expected outputs).

\appendix
\section{Audit manifest (verifier commands and expected fields)}\label{app:audit}

This appendix provides a referee-facing audit checklist for the certified inputs used in Section~\ref{sec:hybrid}.
There are two audit modes:
\begin{itemize}
\item \textbf{Fast audit:} verify the shipped JSON artifacts match Table~\ref{tab:artifact-data}.
\item \textbf{Regeneration audit (optional):} rerun the verifier to regenerate the artifacts from scratch.
\end{itemize}

\subsection*{Prerequisites}
Install the ARB/ball-arithmetic bindings:
\begin{verbatim}
pip install python-flint
\end{verbatim}

\subsection*{Fast audit: check shipped JSON artifacts}
\begin{itemize}
\item \textbf{Rectangle artifact} \url{theta_certify_sigma06_07_t0_20_outer_zeta_proj.json}. Check (at minimum):
  \begin{itemize}
  \item \texttt{results.ok = true}
  \item \texttt{results.theta\_hi = 0.9999928763... < 1}
  \item \texttt{results.processed\_boxes = 380764}
  \end{itemize}
\item \textbf{Pick artifact} \url{pick_sigma0599_raw_zeta_N16.json}. Check (at minimum):
  \begin{itemize}
  \item \texttt{pick.delta\_cert = 0.5944375202...}
  \item \texttt{pick.P\_spd\_at\_0 = true}
  \item \texttt{pick.tail\_weighted\_l2\_partial\_hi = 0.0137007383...}
  \end{itemize}
\item \textbf{Pick artifact} \url{pick_sigma06_raw_zeta_N16.json}. Check (at minimum):
  \begin{itemize}
  \item \texttt{pick.delta\_cert = 0.5948779178...}
  \item \texttt{pick.P\_spd\_at\_0 = true}
  \item \texttt{pick.tail\_weighted\_l2\_partial\_hi = 0.0136892105...}
  \end{itemize}
\item \textbf{Pick artifact} \url{pick_sigma07_raw_zeta_N16.json}. Check (at minimum):
  \begin{itemize}
  \item \texttt{pick.delta\_cert = 0.6273368611...}
  \item \texttt{pick.P\_spd\_at\_0 = true}
  \item \texttt{pick.tail\_weighted\_l2\_partial\_hi = 0.01272011...}
  \end{itemize}
\end{itemize}

\subsection*{Regeneration audit (optional): exact command lines}
Run the verifier from the repository root.
The following commands reproduce the primary artifacts (line breaks are for readability):

\paragraph{1) Rectangle certification (\texttt{theta\_certify}).}
\begin{verbatim}
python verify_attachment_arb.py \
  --theta-certify \
  --arith-gauge outer_zeta_proj \
  --arith-P-cut 2000 \
  --rect-sigma-min 0.6 --rect-sigma-max 0.7 \
  --rect-t-min 0.0 --rect-t-max 20.0 \
  --outer-mode midpoint \
  --outer-P-cut 2000 \
  --outer-T 50.0 --outer-n 2001 \
  --theta-init-n-sigma 10 --theta-init-n-t 50 \
  --theta-min-sigma-width 0.0001 --theta-min-t-width 0.001 \
  --theta-max-boxes 500000 \
  --prec 260 \
  --theta-out theta_certify_sigma06_07_t0_20_outer_zeta_proj.json \
  --progress
\end{verbatim}

\paragraph{2) Pick certification at $\sigma_0=0.599$ (\texttt{pick\_certify}).}
\begin{verbatim}
python verify_attachment_arb.py \
  --pick-certify \
  --pick-sigma0 0.599 \
  --pick-N 16 \
  --pick-coeff-count 32 \
  --pick-K 256 \
  --pick-rho 0.1 \
  --pick-rho-bound 0.2 \
  --arith-gauge raw_zeta \
  --arith-P-cut 2000 \
  --prec 260 \
  --pick-out pick_sigma0599_raw_zeta_N16.json
\end{verbatim}

\paragraph{3) Pick certification at $\sigma_0=0.6$ (\texttt{pick\_certify}).}
\begin{verbatim}
python verify_attachment_arb.py \
  --pick-certify \
  --pick-sigma0 0.6 \
  --pick-N 16 \
  --pick-coeff-count 32 \
  --pick-K 256 \
  --pick-rho 0.1 \
  --pick-rho-bound 0.2 \
  --arith-gauge raw_zeta \
  --arith-P-cut 2000 \
  --prec 260 \
  --pick-out pick_sigma06_raw_zeta_N16.json
\end{verbatim}

\paragraph{4) Pick certification at $\sigma_0=0.7$ (\texttt{pick\_certify}).}
\begin{verbatim}
python verify_attachment_arb.py \
  --pick-certify \
  --pick-sigma0 0.7 \
  --pick-N 16 \
  --pick-coeff-count 32 \
  --pick-K 256 \
  --pick-rho 0.1 \
  --pick-rho-bound 0.2 \
  --arith-gauge raw_zeta \
  --arith-P-cut 2000 \
  --outer-mode rigorous \
  --outer-P-cut 20000 \
  --outer-T 10 --outer-n 200 \
  --prec 260 \
  --pick-out pick_sigma07_raw_zeta_N16.json
\end{verbatim}

\subsection*{What a successful audit means}
The verifier uses \emph{ball arithmetic}: each computed quantity is an interval enclosure (midpoint plus radius) and every operation propagates rounding error outward.
Thus each check is a formal inequality of the form ``upper bound $<1$'' or ``directed-rounding LDL$^\top$ succeeds with positive pivots''.
If the audit checks above pass, then the numerical inequalities used in Section~\ref{sec:hybrid} are certified within the logic of ball arithmetic.

\input{riemann_bibliography.tex}
\end{document}


