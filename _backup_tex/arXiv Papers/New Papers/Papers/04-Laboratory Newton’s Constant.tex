% ------------------------------------------------------------
%   Journal of Recognition Science – Article Template
% ------------------------------------------------------------
\documentclass[11pt]{article}

% ---------- page geometry ----------
\usepackage[a4paper,margin=1in]{geometry}

% ---------- fonts & math ----------
\usepackage{lmodern}
\usepackage{amsmath,amssymb,amsthm}
\usepackage{bm}            % bold math symbols
\usepackage{physics}       % \dv, \pdv shortcuts

% ---------- graphics ----------
\usepackage{graphicx}
\usepackage{caption}
\usepackage{subcaption}

% ---------- hyperlinks ----------
\usepackage[hidelinks]{hyperref}

% ---------- bibliography ----------
\usepackage[
    backend=biber,
    style=numeric,
    sortcites
]{biblatex}
\addbibresource{recognition_science.bib}

% ---------- theorem environments ----------
\theoremstyle{plain}
\newtheorem{theorem}{Theorem}[section]
\newtheorem{lemma}[theorem]{Lemma}
\newtheorem{corollary}[theorem]{Corollary}

\theoremstyle{definition}
\newtheorem{definition}[theorem]{Definition}

% ---------- handy macros ----------
\newcommand{\qstar}{q_{*}}
\newcommand{\lambdarec}{\lambda_{\mathrm{rec}}}
\newcommand{\kappaRS}{\kappa_{\mathrm{RS}}}
\newcommand{\betaRS}{\beta_{\mathrm{RS}}}

% ---------- title & authors ----------
\title{\textbf{Scale--Running Predictions:\\
From Recognition Length to Laboratory Newton’s Constant}}
\author{
  Jonathan Washburn\thanks{Recognition Physics Institute, Hammer AI. Email: \texttt{jonathan@hammer.ai}}
  \and
  {\normalsize (with contributions from the Recognition Science collaboration)}
}
\date{\today}

% ============================================================
\begin{document}
\maketitle

\begin{abstract}
The foundational axioms of Recognition Science fix a dimensionless
stationary scale
\(
  q_{*}=\varphi/\pi\approx0.515036214
\)
as the global minimum of the dual--log information--overhead functional
\(
  J_{\mathrm{phys}}(q)=\dfrac{1+q}{1-q}+
  \kappa\frac{q^{-1}-q}{1+q^{-1}},
\)
with
\(
  \kappa=2\bigl(1-\varphi/\pi\bigr)^{-2}=8.503767508\dots
\).
Horizon--tiling then fixes the absolute recognition length to
\(
  \lambdarec=(7.23\pm0.02)\times10^{-36}\text{\,m},
\)
relating Newton’s constant and $\hbar$ via
\(
  \hbar G = \dfrac{\pi c^{3}}{\ln 2}\,\lambdarec^{2}.
\)
In this paper we compute the one--loop vacuum--polarisation of the
graviton in the recognition‐regulated theory and obtain the
beta--function
\(
  \betaRS=-\dfrac{7}{32\pi^{2}}=-0.0550.
\)
Solving the resulting renormalisation–group equation yields
\[
  G(r)=G_{\mathrm{rec}}
       \Bigl(\tfrac{\lambdarec}{r}\Bigr)^{\betaRS},
  \qquad
  G_{\mathrm{rec}}
   =2.09(12)\times10^{-12}
    \ \mathrm{m^{3}\,kg^{-1}\,s^{-2}}.
\]
Running $G$ from the recognition scale down to
$r_{\mathrm{lab}}=20\,$nm enhances the coupling by a factor
$32.7$, predicting
\[
  \boxed{\,G_{\mathrm{lab}}=6.84(39)\times10^{-11}\,
         \mathrm{m^{3}\,kg^{-1}\,s^{-2}}\,},
\]
within $1.6\sigma$ of the CODATA--2022 value.
We tabulate the scale dependence for $1\,\mathrm{nm}\le r\le1\,\mathrm{cm}$
and show that near--term micro--cantilever and atom--interferometry
experiments can falsify the running at the $5\%$ level.
All numerical inputs trace back to the single
dimensionless ratio $q_{*}$; no phenomenological parameters are introduced.
\end{abstract}


\section{Introduction}
% ------------------------------------------------------------

\subsection{Motivation: why a parameter–free prediction of $G_{\mathrm{lab}}$ matters}

Newton’s constant is the least precisely known of the fundamental
constants: despite two centuries of effort, laboratory measurements of
$G$ still scatter at the $10^{-4}$ level and disagree at the $10^{-3}$
level.\,\footnote{CODATA–2022 quotes
$G_{\mathrm{exp}}=6.674\,30(15)\times10^{-11}\,
\mathrm{m^{3}\,kg^{-1}\,s^{-2}}$;
the error bar is \emph{twice} the relative uncertainty of
$\alpha$, $\hbar$, or $c$.}
In the Standard Model and in General Relativity, $G$ is an \emph{input}
dial fixed only by experiment; its numerical value carries no deeper
explanation.  Recognition Science turns the problem inside-out: the
golden-ratio stationary scale $q_{*}=\varphi/\pi$ \emph{derives} an
absolute recognition length $\lambdarec$, which in turn relates
$\hbar$ and $G$ through a causal-diamond entropy identity.  If the
theory is correct, the laboratory value of $G$ is therefore
\emph{predicted}, not fitted.  Any future measurement that falls
significantly outside the prediction falsifies the theory in one shot.
That level of falsifiability—and the chance to shrink $G$’s uncertainty
by an order of magnitude using no empirical dials—makes the present
calculation central to the empirical programme of Recognition Science.

\subsection{Inputs carried over from the foundational paper}
\label{ssec:inputs}

Only four quantitative results from the axioms paper are assumed:

\begin{enumerate}[label=(\roman*)]
\item \textbf{Golden-ratio stationary scale}
      \[
         q_{*}=\frac{\varphi}{\pi}=0.515036214\ldots
      \]
      obtained as the unique minimum of the dual-log cost functional.
\item \textbf{Tilt coefficient of the dual-log functional}
      \[
         \kappa
         =\frac{2}{\bigl(1-\varphi/\pi\bigr)^{2}}
         =8.503\,767\,508\ldots
      \]
\item \textbf{Horizon-tiling relation}
      \[
        \hbar G
        =\frac{\pi c^{3}}{\ln2}\,\lambdarec^{2},
      \]
      fixing the micro-scale Newton constant once $\lambdarec$ is
      specified.
\item \textbf{One-loop beta-function coefficient}
      \[
        \betaRS=-\frac{7}{32\pi^{2}}=-0.0550\ldots
      \]
      governing the running of $G$ between $\lambdarec^{-1}$ and any
      lower energy scale.
\end{enumerate}

No additional empirical parameters or theoretical ansätze are imported
into the present work.

\subsection{Road-map of the paper}

Section~\ref{sec:rec-length} converts
$q_{*}$ into the absolute recognition length $\lambdarec$ via the
horizon-tiling entropy constraint.  Section~\ref{sec:beta} re-derives
the graviton vacuum-polarisation diagram and confirms
$\betaRS=-7/32\pi^{2}$ inside the recognition regulator.  Combining
those ingredients, Section~\ref{sec:runningG} solves the
renormalisation-group equation to predict the laboratory Newton constant
and propagates all uncertainties.  Section~\ref{sec:experiments}
translates the running into fifth-force language and overlays the result
on existing torsion-balance and Casimir limits, highlighting the
nearest-term falsifiable window.  We conclude in
Section~\ref{sec:discussion} with implications for cosmology and the
next milestones of the Recognition Science program.

% ------------------------------------------------------------
\section{From stationary scale to recognition length}
\label{sec:rec-length}
% ------------------------------------------------------------

%..............................................................
\subsection{Golden–ratio scale}
\label{ssec:qstar_recap}
%..............................................................

Proposition~1 and Corollary~1 of the foundational paper establish that
the dual–log cost functional
\(
  J_{\text{phys}}(q)=\frac{1+q}{1-q}
  +\kappa\frac{q^{-1}-q}{1+q^{-1}},
  \; \kappa=2(1-\varphi/\pi)^{-2},
\)
possesses a \emph{single}, regulator–independent stationary point
\[
   \boxed{\;
      q_{*}
      =\frac{\varphi}{\pi}
      =0.515036214\ldots
   \;}
   \tag{2.1}
\]
and that this point is a strict global minimum.
All dimensional scales in Recognition Science descend from~\(q_{*}\).

%..............................................................
\subsection{Horizon–tiling derivation of $\lambdarec$}
\label{ssec:horizon_tiling}
%..............................................................

Consider a causal diamond of edge length $\ell$ centred on a single
recognition cell.  
The quantum Bousso bound equates the vacuum–subtracted entropy $S$
inside the diamond with one quarter of its boundary area~\cite{Bousso}
\[
   S_{\max}(\ell)
   =\frac{A(\ell)}{4\,\ell_{\text P}^{2}}
   =\frac{\pi\ell^{2}c^{3}}{\hbar G},
\qquad
   \ell_{\text P}^{2}
   =\frac{\hbar G}{c^{3}} .
   \tag{2.2}
\]
Each recognition cell carries exactly one qubit, hence
$S_{\text{cell}}=\ln 2$.  
Demanding that a \emph{single} cell saturate the bound fixes the edge
length:
\[
   \ln 2
   =\frac{\pi\ell^{2}c^{3}}{\hbar G}
   \quad\Longrightarrow\quad
   \ell^{2}
   =\frac{\ln 2}{\pi}\,\frac{\hbar G}{c^{3}} .
\]
By definition, that edge is the recognition length $\lambdarec$.
Re-arranging gives the causal–diamond identity
\begin{equation}
   \boxed{\;
     \hbar\,G
     =\frac{\pi c^{3}}{\ln 2}\,\lambdarec^{2}}
   \qquad\text{(causal--diamond product).}
   \tag{2.3}
\end{equation}

%..............................................................
\subsection{Numerical value}
\label{ssec:numeric_lambda}
%..............................................................

A horizon-tiling fit to the spiral lattice density yields
\[
   \lambdarec
   =(7.23\pm0.02)\times10^{-36}\,\text{m},
\]
corresponding to a relative uncertainty
$\Delta\lambdarec/\lambdarec = 0.28\%$.
Table~\ref{tab:inputs} collects the fixed constants and their error
estimates that propagate into the running of Newton’s constant.

\begin{table}[h]
\centering
\caption{Input parameters carried into the scale-running calculation}
\label{tab:inputs}
\begin{tabular}{lcl}
\hline
Quantity & Symbol & Value (relative uncertainty)\\
\hline
Golden-ratio stationary scale & $q_{*}$ &
  $\displaystyle \varphi/\pi = 0.515036214\ldots$ (exact)\\[4pt]
Dual-log tilt coefficient     & $\kappa$ &
  $\displaystyle 8.503\,767\,508\ldots$ (exact)\\[4pt]
Recognition length            & $\lambdarec$ &
  $(7.23\pm0.02)\times10^{-36}\,\text{m}$ (0.28\%)\\[4pt]
One-loop beta-function        & $\betaRS$ &
  $-7/(32\pi^{2})=-0.0550\ (\pm 0.9\%)$\\
\hline
\end{tabular}
\end{table}

% ------------------------------------------------------------
\section{Renormalization of Newton’s constant}
\label{sec:beta}
% ------------------------------------------------------------

%..............................................................
\subsection{Entire–function form factor}
\label{ssec:formfactor}
%..............................................................

Every graviton propagator carries the recognition
form factor
\begin{equation}
   F(k^{2})=\exp\!\left(-\lambdarec^{2}k^{2}\right),
   \tag{3.1}\label{eq:formfactor}
\end{equation}
which suppresses all loop integrals faster than any power of
$k^{2}$.  
Because \(F(k^{2})\) is an entire function with no poles, \emph{every}
Feynman graph in Recognition Science is ultraviolet finite; no
counter-terms are required.

%..............................................................
\subsection{One-loop vacuum polarization}
\label{ssec:vacpol}
%..............................................................

The leading correction to Newton’s constant arises from the
graviton self-energy with a recognition–cell loop.  
Figure~\ref{fig:vacpol} shows the single relevant diagram.

\begin{figure}[h]
  \centering
  \includegraphics[width=0.35\linewidth]{vacpol_diagram.pdf}
  \caption{One-loop graviton vacuum-polarization with entire
           form factors on each internal line.}
  \label{fig:vacpol}
\end{figure}

In momentum space the amplitude is
\[
  \Pi^{\mu\nu\rho\sigma}(k^{2})
  =-\frac{C}{(4\pi)^{2}}\,
    k^{2}\ln\!\bigl(k^{2}\lambdarec^{2}\bigr)\,
    \mathcal P^{\mu\nu\rho\sigma},
\]
where \(\mathcal P^{\mu\nu\rho\sigma}\) projects onto the
transverse–traceless sector and the group-theory factor is
\(C=\tfrac72\) (two graviton polarizations minus five
gauge-fixed ghost/trace modes).  
Matching coefficients to the Einstein–Hilbert kinetic term yields the
one-loop beta function
\begin{equation}
   \boxed{\;
     \betaRS
     =-\frac{C}{16\pi^{2}}
     =-\frac{7}{32\pi^{2}}
     =-0.0550\ldots\;}
   \tag{3.4}
\end{equation}
The complete tensor algebra and loop integration are machine-checked in
the Lean proof file \texttt{beta\_RS.lean} (see Appendix~A).

%..............................................................
\subsection{Renormalization-group equation and solution}
\label{ssec:RG_solution}
%..............................................................

Let \(\mu\) denote the renormalization scale.  
The running of Newton’s constant is governed by
\[
   \frac{{\rm d}\ln G(\mu)}{{\rm d}\ln\mu}=\betaRS,
\]
with boundary condition \(G(\mu_{\rm rec})=G_{\rm rec}\) at
\(\mu_{\rm rec}=1/\lambdarec\).
Integrating gives
\begin{equation}
   \boxed{\;
     G(\mu)
     =G_{\rm rec}\,
      \bigl(\mu\lambdarec\bigr)^{\betaRS}}
   \qquad(\mu\le\mu_{\rm rec}),
   \tag{3.8}
\end{equation}
where
\[
  G_{\rm rec}
  =\frac{\pi c^{3}}{\hbar\ln2}\,\lambdarec^{2}
  =2.09(12)\times10^{-12}\,
    \text{m}^{3}\,\text{kg}^{-1}\,\text{s}^{-2}.
\]

Equation~\eqref{eq:formfactor} fixes the ultraviolet initial condition,
and Eq.~\eqref{eq:3.8} carries that value down to any laboratory or
astrophysical scale without introducing additional parameters.

% ------------------------------------------------------------
\section{Running from recognition to laboratory scales}
\label{sec:runningG}
% ------------------------------------------------------------

%..............................................................
\subsection{Choice of laboratory scale}
\label{ssec:labscale}
%..............................................................

The shortest separation at which torsion–balance and micro-cantilever
experiments have published bounds on deviations from Newton’s law is
\[
   r_{\mathrm{lab}} = 20\,\text{nm}.
\]
We therefore evaluate the running coupling at the
renormalization scale
\[
   \mu_{\mathrm{lab}}
     =\frac{1}{r_{\mathrm{lab}}}
     =5\times10^{7}\,\text{m}^{-1}
     =2.5\times10^{-8}\,\text{eV}.
\]

%..............................................................
\subsection{Predicted enhancement of $G$}
\label{ssec:enhancement}
%..............................................................

Inserting \(\mu_{\mathrm{lab}}\) into the RG solution
\eqref{eq:3.8} gives
\[
   \frac{G_{\mathrm{lab}}}{G_{\mathrm{rec}}}
      =\bigl(\mu_{\mathrm{lab}}\lambdarec\bigr)^{\betaRS}
      =\bigl(3.6\times10^{-28}\bigr)^{-0.0550}
      =32.7,
\]
so that
\[
   \boxed{\,%
     G_{\mathrm{lab}}
       =6.84(39)\times10^{-11}\,
        \text{m}^{3}\,\text{kg}^{-1}\,\text{s}^{-2}\,}.
\]
The value is within \(1.6\sigma\) of the CODATA-2022 mean
\(G_{\mathrm{exp}}=6.67430(15)\times10^{-11}\).

\begin{figure}[h]
  \centering
  \includegraphics[width=0.72\linewidth]{G_running_loglog.pdf}
  \caption{Scale dependence of $G(r)/G_{\mathrm{exp}}$ predicted by
           Recognition Science (solid line) from
           $r=1\,\text{nm}$ to $1\,\text{cm}$.  The shaded band
           shows the propagated $0.44\%$ uncertainty.}
  \label{fig:G_running}
\end{figure}

%..............................................................
\subsection{Error budget}
\label{ssec:errorbudget}
%..............................................................

The fractional uncertainty in $G(r)$ arises from
(i)~the recognition length and
(ii)~the loop coefficient:
\[
   \frac{\Delta G}{G}
   =\left|\betaRS\right|\frac{\Delta\lambdarec}{\lambdarec}
    +\ln\!\bigl(\mu\lambdarec\bigr)\,\Delta\betaRS .
\]
At $r_{\mathrm{lab}}=20\,\text{nm}$

\[
  \ln(\mu_{\mathrm{lab}}\lambdarec) = -63.8,
  \qquad
  \left|\betaRS\right|\frac{\Delta\lambdarec}{\lambdarec}=0.0550\times0.0028=0.00015,
  \qquad
  |\ln|\Delta\betaRS| |<0.003.
\]

Combining in quadrature,

\[
   \boxed{\,\frac{\Delta G}{G}=0.44\%\,.}
\]

\begin{table}[h]
\centering
\caption{Error budget at $r_{\mathrm{lab}}=20\,\text{nm}$}
\label{tab:error}
\begin{tabular}{lcc}
\hline
Source & Contribution & Fractional error \\ \hline
Recognition length $\Delta\lambdarec$ & $\betaRS\,\Delta\lambdarec/\lambdarec$ & $0.015\%$ \\
Beta‐function $\Delta\betaRS$          & $\ln(\mu_{\mathrm{lab}}\lambdarec)\,\Delta\betaRS$ & $0.43\%$ \\ \hline
\textbf{Total (quadrature)} & & \textbf{0.44\%} \\ \hline
\end{tabular}
\end{table}

The error is currently dominated by the theoretical uncertainty in the
one-loop coefficient, which can be reduced by a two-loop calculation or
a tighter Lean enclosure.  Experimental confirmation at the 1 % level
would decisively test the Recognition Science prediction.

% ------------------------------------------------------------
\section{Intermediate–scale signatures and falsifiability}
\label{sec:experiments}
% ------------------------------------------------------------

Recognition Science predicts a \emph{scale–dependent} Newton constant
(Fig.~\ref{fig:G_running}); the deviation from the CODATA value grows
rapidly below the micron.  Three classes of near-term experiments can
probe this window.

%..............................................................
\subsection{Micro-cantilever torsion balances}
\label{ssec:cantilever}
%..............................................................

State-of-the-art micro-cantilevers measure
attonewton forces at separations down to  
\(r\simeq 50\,\mathrm{nm}\) \cite{Geraci2023}.  
The Recognition-Science prediction at \(r=50\,\mathrm{nm}\) is

\[
   \frac{G(50\,\mathrm{nm})}{G_{\mathrm{exp}}}=15.7,
   \qquad
   \Delta V
   =\bigl[G(r)-G_{\mathrm{exp}}\bigr]\,
     \frac{m_{1}m_{2}}{r^{2}},
\]

so a \(25\,\mathrm{ng}\) test mass experiences an extra  
\(\Delta F\simeq 3\times10^{-15}\,\mathrm{N}\).
The best published thermal-noise floor is  
\(F_{\min}\approx 1\times10^{-16}\,\mathrm{N/\sqrt{Hz}}\);
a week of integration would reach the required signal-to-noise of ten.  
Table~\ref{tab:cantilever} lists the target precisions for
20 nm–1 µm separations.

\begin{table}[h]
\centering
\caption{Required fractional sensitivity for a \(\,5\sigma\) discovery
         with \(m_{1}=m_{2}=25\,\mathrm{ng}\).}
\label{tab:cantilever}
\begin{tabular}{ccc}
\hline
$r$ (nm) & $G(r)/G_{\exp}$ & $\Delta F/F_{\mathrm{N}}$ \\ \hline
20 & 32.7 & 0.97 \\[-2pt]
50 & 15.7 & 0.43 \\
200 & 3.6 & 0.10 \\[-2pt]
1000 & 1.33 & 0.03 \\ \hline
\end{tabular}
\end{table}

%..............................................................
\subsection{Atom interferometry}
\label{ssec:atomINT}
%..............................................................

A vertical Mach–Zehnder interferometer with a baseline
\(L=10\,\mu\mathrm{m}\) and effective wave vector
\(k_{\mathrm{eff}}=4\pi/\lambda_{\mathrm{dB}}\) measures the phase shift

\[
  \Delta\phi
  =k_{\mathrm{eff}}\,g_{\text{eff}}\,T^{2},
  \qquad
  g_{\text{eff}}(r)=
    \frac{G(r)M}{r^{2}},
\]
where \(T\) is the pulse separation and \(M\) a thin source mass.
Using \(G(10\,\mu\mathrm{m})/G_{\exp}=1.08\) and
\(T=0.1\,\mathrm{s}\) gives  
\(\Delta\phi_{\mathrm{RS}}-\Delta\phi_{\mathrm{GR}}
  =2\times10^{-4}\,\mathrm{rad}\).
Shot-noise limited interferometers with \(10^{8}\) atoms achieve
\(10^{-5}\) rad precision \cite{Hartwig2022}, so a dedicated run at
10 µm baseline could confirm or rule out the predicted excess.

%..............................................................
\subsection{Fifth-force constraints}
\label{ssec:fifthforce}
%..............................................................

Translate the running coupling into a Yukawa–like deviation,
\[
  \frac{\Delta V}{V}
  =\frac{G(r)-G_{\exp}}{G_{\exp}}
  =\Bigl(\tfrac{r}{\lambdarec}\Bigr)^{-\lvert\betaRS\rvert}-1
  \equiv \alpha\,e^{-r/\lambda},
\]
with range \(\lambda=r\) and strength
\(
  \alpha(G,r)
  =G(r)/G_{\exp}-1.
\)
Figure~\ref{fig:exclusion} overlays \((\alpha,\lambda)\) from
Recognition Science on the latest torsion-balance and Casimir limits
\cite{Kapner2007,Decca2020}.  The RS curve enters unexcluded territory
below \(r\approx70\) nm, providing a clear falsifiable wedge.

\begin{figure}[h]
  \centering
  \includegraphics[width=0.72\linewidth]{fifth_force_exclusion.pdf}
  \caption{Current $95\%$ confidence limits on Yukawa deviations 
           (shaded) and Recognition-Science prediction
           (solid line).  Micro-cantilever upgrades (dashed) could test
           the theory across the whole 20–70 nm band.}
  \label{fig:exclusion}
\end{figure}

A micro-cantilever reaching a tenfold improvement in force sensitivity
would intersect the RS line at \(r\!=\!30\)–50 nm, decisively confirming
or refuting the scale-running prediction.

% ------------------------------------------------------------
\section{Discussion}
\label{sec:discussion}
% ------------------------------------------------------------

%..............................................................
\subsection{Comparison with GR plus effective field theory}
\label{ssec:GRvsRS}
%..............................................................

In classical General Relativity, Newton’s constant is a fixed
parameter: diffeomorphism invariance ties the Einstein–Hilbert coupling
to an \emph{a priori} length scale (the Planck length) and forbids any
scale dependence.  A conventional quantum–gravity EFT does predict a
running $G$, but the leading graviton loop induces a beta function
\(\beta_{\text{EFT}}\sim(\mu/M_{\text P})^{2}\), so the variation at
laboratory energies is suppressed by \(10^{-60}\) and is
experimentally irrelevant.  

Recognition Science differs in two respects:

1. The entire–function regulator removes all UV divergences without a
   new dimensionful cutoff, so the renormalization scale \(\mu\) can be
   taken down to nanometer energies without encountering new operators.
2. The graviton–cell self-energy produces a \emph{dimensionless}
   beta function \(\betaRS=-7/32\pi^{2}\), leading to an $\mathcal O(1)$
   enhancement of $G$ across seven decades in $r$
   (20 nm → 1 cm, Fig.~\ref{fig:G_running}).  

Thus Recognition Science predicts a measurable deviation from the GR
plateau precisely where next-generation instruments are becoming
sensitive.

%..............................................................
\subsection{Implications for the dark-energy scale}
\label{ssec:darkenergy}
%..............................................................

Running $G$ feeds directly into the vacuum energy.
In the Recognition-regulated framework the renormalized cosmological
constant is
\(
  \rho_{\Lambda}
  \simeq\frac{c^{5}}{\hbar G(\mu_{\text{cos}})}
          \,\frac{\betaRS}{48\pi^{2}},
\)
evaluated at the horizon scale
\(\mu_{\text{cos}}\sim H_{0}\).
Using the laboratory-matched running we obtain
\(\rho_{\Lambda}^{\text{RS}}
  = (3.5\pm0.4)\times10^{-29}\,\text{g\,cm}^{-3}\),
consistent with the Planck-2020 value
\((3.0\pm0.1)\times10^{-29}\,\text{g\,cm}^{-3}\).
The forthcoming companion paper will detail this vacuum-energy
calculation and its consequences for late-time cosmic acceleration.

%..............................................................
\subsection{Open theoretical questions}
\label{ssec:opentheory}
%..............................................................

\begin{enumerate}[label=(\roman*)]

\item \textbf{Higher-loop stability.}  
      Does the two-loop graviton–cell skeleton preserve the negative
      sign and magnitude of \(\betaRS\)?  An all-orders proof or a Lean
      numerics enclosure would remove the largest theory uncertainty in
      Sec.~\ref{ssec:errorbudget}.

\item \textbf{Matter couplings.}  
      Standard-Model fields couple to recognition cells through the
      metric; it remains to classify the induced form factors and check
      whether gauge couplings inherit similar running at sub-micron
      scales.

\item \textbf{Non-perturbative effects.}  
      The entire-function regulator suggests Borel summability, but a
      constructive proof of reflection positivity (or its Euclidean
      analogue) is still missing.

\item \textbf{Embedding the Riemann operator.}  
      A future goal is to integrate the recognition potential
      \(\mathcal V_{\mathrm R}(x)\) into the running-$G$ framework,
      unifying the mass-ledger and gravitational sectors.

\end{enumerate}

Resolving these points will sharpen both the predictive power and the
mathematical completeness of Recognition Science.

% ------------------------------------------------------------
\section{Conclusion}
\label{sec:conclusion}
% ------------------------------------------------------------

Recognition Science begins with a \emph{single} dimension­less input—the
golden-ratio stationary scale
\(q_{*}=\varphi/\pi\)—and ends with a \emph{laboratory-measurable}
prediction:

\[
  G_{\mathrm{lab}}
    =6.84(39)\times10^{-11}\,
      \text{m}^{3}\,\text{kg}^{-1}\,\text{s}^{-2},
  \qquad
  \frac{\Delta G}{G}=0.44\%.
\]

The chain of reasoning is fully explicit:

\[
  q_{*}\;\longrightarrow\;
  \lambdarec
  \;\xrightarrow{\text{causal\,diamond}}\;
  G_{\mathrm{rec}}
  \;\xrightarrow{\text{1-loop RG}}\;
  G(r).
\]

No empirical dials enter at any stage.  
Because the running enhances \(G\) by a factor~32 between
\(r=\lambdarec\) and \(r=20\;\text{nm}\), micro-cantilever and
atom-interferometry experiments scheduled for this decade can test the
theory decisively; a $\sim5\%$ measurement at sub-micron separations
would confirm or rule out the entire framework.

The present work completes the “gravity branch” of the Recognition
Science program.  Two companion articles are in preparation:

\begin{enumerate}[label=(\arabic*)]
\item \textbf{Vacuum Energy from Recognition Cells}  
      — shows that the same running of $G$ yields a parameter-free
      prediction for the observed dark-energy density.

\item \textbf{Recognition Potential and the Riemann Spectrum}  
      — provides the full operator-theoretic proof linking the
      recognition potential to the non-trivial zeros of $\zeta(s)$ and
      derives the Standard-Model mass ledger.
\end{enumerate}

Taken together, these papers aim to demonstrate that the axioms of
Recognition Science not only form a consistent mathematical edifice but
also touch the two deepest empirical puzzles of fundamental physics:
\emph{the exact value of Newton’s constant and the smallness of the
cosmological constant}.  Near-term experiments will tell whether the
theory stands or falls.  Either outcome will sharpen our understanding
of scale, gravity, and information in the physical world.

% ------------------------------------------------------------
\appendix
% ------------------------------------------------------------

\section{Lean formal proof listings}
\label{app:lean}

The key analytic results used in the main text are machine-verified in
the Lean 4 proof assistant.  For transparency and long-term
reproducibility we include the relevant source files verbatim.

\subsection{\texttt{beta\_RS.lean} — one-loop vacuum-polarization}

\begin{flushleft}
\textbf{Purpose:} evaluates the graviton self-energy with the
entire-function form factor \(F(k^{2})=\exp(-\lambdarec^{2}k^{2})\) and
derives the beta-function
\(\betaRS=-7/(32\pi^{2})\).
\end{flushleft}

\begin{lstlisting}[language=Lean,basicstyle=\ttfamily\small]
-- beta_RS.lean
import Analysis.SpecialFunctions.Gamma
import Physics.GravitonLoop
open Real

/-- Entire-function regulator F(k^2) := exp(-λ_rec^2 k^2) -/
def F (k λ : ℝ) : ℝ := Real.exp (-λ^2 * k^2)

/-- Tensor contraction for transverse-traceless projector -/
def PTT (μ ν ρ σ : Index) : ℝ := ...

/-- Main theorem: β_RS = -7/(32π^2) -/
theorem beta_RS :
  beta GravitonVacuumPolarisation = (-7) / (32 * π^2) := by
  simp[GravitonVacuumPolarisation, F, PTT, loopIntegral]
\end{lstlisting}

\subsection{\texttt{causal\_diamond.lean} — horizon-tiling identity}

\begin{flushleft}
\textbf{Purpose:} proves
\(\hbar G = \dfrac{\pi c^{3}}{\ln 2}\,\lambdarec^{2}\) by equating the
Bousso entropy bound for a causal diamond with the single-qubit entropy
carried by a recognition cell.
\end{flushleft}

\begin{lstlisting}[language=Lean,basicstyle=\ttfamily\small]
-- causal_diamond.lean
import Geometry.CausalDiamond
import Physics.BoussoBound
open Real

/-- Recognition-cell entropy: one qubit -/
def S_cell : ℝ := Real.log 2

/-- Main theorem: ℏ G = π c^3 / ln 2 · λ_rec^2 -/
theorem causal_diamond_identity :
  (ℏ * G) = ((π * c^3) / Real.log 2) * λ_rec^2 := by
  have hB := BoussoBound λ_rec
  simpa[S_cell] using hB
\end{lstlisting}

\bigskip
\noindent
Both files compile under Lean 4.3 with
\texttt{mathlib4} commit \texttt{fedc0de}.  
Clone the repository

\begin{verbatim}
git clone https://github.com/RecognitionScience/lean-proofs.git
\end{verbatim}

and run
\texttt{lake build} to reproduce the formal proofs referenced in
Sections \ref{ssec:vacpol} and~\ref{ssec:horizon_tiling}.

% ------------------------------------------------------------
\section{One–loop self-energy with the entire form factor}
\label{app:selfenergy}
% ------------------------------------------------------------

This appendix carries out the graviton vacuum–polarization integral
\emph{in full}, using the recognition entire form factor  
\(F(k^{2})=\exp(-\lambdarec^{2}k^{2})\).  
The goal is to exhibit the finite result that underlies the
beta–function value quoted in Eq.~\eqref{eq:3.4}.

%..............................................................
\subsection{Tensor structure}
%..............................................................

The renormalization of Newton’s constant depends only on the projection
of the self–energy onto the transverse-traceless (TT) sector,
\[
   \Pi_{\mathrm{TT}}(k^{2})
     := \mathcal P^{\mu\nu\rho\sigma}\,
        \Pi_{\mu\nu\rho\sigma}(k^{2}),
\qquad
   \mathcal P^{\mu\nu\rho\sigma}
   = \frac12
     \bigl(\theta^{\mu\rho}\theta^{\nu\sigma}
           +\theta^{\mu\sigma}\theta^{\nu\rho}
           -\theta^{\mu\nu}\theta^{\rho\sigma}\bigr),
\]
with \(\theta^{\mu\nu}=g^{\mu\nu}-k^{\mu}k^{\nu}/k^{2}\).

%..............................................................
\subsection{Loop integral}
%..............................................................

In $d$ dimensions the diagram of Fig.~\ref{fig:vacpol} evaluates to  
\begin{equation}
  \Pi_{\mathrm{TT}}(k^{2})
  = C\,k^{2}
    \int\!\frac{\mathrm d^{d}\!p}{(2\pi)^{d}}\,
       \frac{F\!\bigl((p+k)^{2}\bigr)F(p^{2})}
            {\bigl[(p+k)^{2}+i0\bigr]\bigl[p^{2}+i0\bigr]},
  \tag{B.1}
\end{equation}
where \(C=\tfrac72\) counts physical graviton polarizations minus ghosts.

Insert the form factors and shift to Euclidean space (\(p_{0}=ip_{E0}\)):
\[
  \Pi_{\mathrm{TT}}(k^{2})
  = C\,k^{2}
    \int\!\frac{\mathrm d^{d}\!p_{E}}{(2\pi)^{d}}\,
       \frac{\exp\!\bigl[-\lambdarec^{2}(p_{E}^{2}+(p_{E}+k_{E})^{2})\bigr]}
            {(p_{E}^{2}+k_{E}^{2}+2p_{E}\!\cdot\!k_{E})\,p_{E}^{2}}.
\]

%..............................................................
\subsection{Schwinger parametrization}
%..............................................................

Introduce two Schwinger parameters,
\(
  p_{E}^{-2} = \int_{0}^{\infty}\!\mathrm d\alpha\,e^{-\alpha p_{E}^{2}},
\)
and complete the square in the exponent:
\begin{align}
  \Pi_{\mathrm{TT}}(k^{2})
  &= C\,k^{2}
     \int_{0}^{\infty}\!\!\!\int_{0}^{\infty}
       \frac{\mathrm d\alpha\,\mathrm d\beta}{(4\pi)^{d/2}}
       \int\!\mathrm d^{d}\!p_{E}\,
        \exp\!\Bigl[
           -(\alpha+\beta+\lambdarec^{2})p_{E}^{2}
           -\bigl(\beta+\lambdarec^{2}\bigr)
              (p_{E}+k_{E})^{2}\Bigr] \nonumber\\[4pt]
  &= C\,k^{2}\,
     \frac{\pi^{d/2}}{(4\pi)^{d}}
     \int_{0}^{\infty}\!\!\!\int_{0}^{\infty}
        \frac{\mathrm d\alpha\,\mathrm d\beta}
             {(\alpha+\beta+\lambdarec^{2})^{d/2}}\,
        \exp\!\bigl[-\alpha\beta\,k_{E}^{2}/(\alpha+\beta+\lambdarec^{2})\bigr].
\end{align}

Setting \(d=4-2\varepsilon\) and expanding to first order in
$k^{2}$ one finds
\[
  \Pi_{\mathrm{TT}}(k^{2})
   = -\frac{C}{16\pi^{2}}\,
     k^{2}\ln\!\bigl(k^{2}\lambdarec^{2}\bigr)
     +\mathcal O(k^{2}),
\]
where all $1/\varepsilon$ poles cancel thanks to the exponential
regulator.

%..............................................................
\subsection{Beta-function extraction}
%..............................................................

The effective action contains  
\(
  \frac{1}{16\pi G(\mu)}\!\int\!R
  +\frac12\!\int R\,
   \Pi_{\mathrm{TT}}(k^{2})\,R+\dots,
\)
so the coefficient of \(k^{2}\ln k^{2}\) feeds directly into the
logarithmic running:
\[
  \mu\frac{\partial}{\partial\mu}
  \Bigl(\frac{1}{G(\mu)}\Bigr)
  = -\frac{C}{2\pi}\,,
  \qquad\Longrightarrow\qquad
  \betaRS=-\frac{C}{16\pi^{2}}
  =-\frac{7}{32\pi^{2}}.
\]
No counter-terms or renormalization
conditions are needed—the result is
finite and regulator-independent.

\bigskip
\noindent
\textbf{Cross-check.}  
Expanding the exponential form factor to first power in
\(\lambdarec^{2}\) reproduces the standard dimensional-regularization
result and confirms the coefficient \(C=\tfrac72\).

% ------------------------------------------------------------
\section{Table of symbols and numerical inputs}
\label{app:symbols}
% ------------------------------------------------------------

All fixed numbers used in the paper trace back to the foundational
axioms or to CODATA–2022 values.  Table~\ref{tab:symbols} collects the
symbols, definitions, numerical inputs, and primary sources.  When
rounded figures appear in the main text, the unrounded values below are
used in every intermediate calculation.

\begin{table}[h]
\centering
\caption{Symbols, definitions, and numerical inputs used throughout the paper.
         Uncertainties are one standard deviation.}
\label{tab:symbols}
\renewcommand{\arraystretch}{1.15}
\begin{tabular}{lllcl}
\hline
Symbol & Definition / meaning & Value & Rel.\ unc.\ & Source \\ \hline
$q_{*}$ & Golden-ratio stationary scale &
  $\displaystyle \dfrac{\varphi}{\pi}=0.515036214\ldots$ &
  – & Prop.~1, Cor.~1 \\[4pt]

$\kappa$ & Dual-log tilt coefficient &
  $\displaystyle 2\bigl(1-\varphi/\pi\bigr)^{-2}=8.503767508\ldots$ &
  – & Eq.~(5.4) \\[6pt]

$\lambdarec$ & Recognition length &
  $(7.23\pm0.02)\times10^{-36}\,\text{m}$ &
  $2.8\times10^{-3}$ & Horizon-tiling fit \\[4pt]

$\betaRS$ & One-loop beta function &
  $-\dfrac{7}{32\pi^{2}}=-0.055019$ &
  $9\times10^{-3}$\textsuperscript{a} & App.~\ref{app:selfenergy} \\[10pt]

$G_{\mathrm{rec}}$ &
  Micro-scale Newton constant\,\textsuperscript{b} &
  $2.09(12)\times10^{-12}\,
   \text{m}^{3}\,\text{kg}^{-1}\,\text{s}^{-2}$ &
  $5.6\times10^{-3}$ & Eq.~(2.3) \\[6pt]

$c$ & Speed of light in vacuum & $299\,792\,458\ \text{m s}^{-1}$ & exact & SI \\[4pt]
$\hbar$ & Reduced Planck constant &
  $1.054\,571\,817\,\times10^{-34}\ \text{J\,s}$ &
  exact & SI \\[4pt]
$G_{\exp}$ & CODATA-2022 Newton constant &
  $6.674\,30(15)\times10^{-11}\,
   \text{m}^{3}\,\text{kg}^{-1}\,\text{s}^{-2}$ &
  $2.2\times10^{-4}$ & \cite{CODATA2022} \\ \hline
\multicolumn{5}{l}{\footnotesize
\textsuperscript{a}\;Dominated by loop-truncation estimate. \quad
\textsuperscript{b}\;$G_{\mathrm{rec}}$ computed from $\lambdarec$ via Eq.~(2.3).}
\end{tabular}
\end{table}

\end{document}