\documentclass[11pt,a4paper]{article}
\usepackage[margin=1in]{geometry}
\usepackage[T1]{fontenc}
\usepackage{lmodern}
\usepackage{microtype}
\usepackage{amsmath,amssymb,amsthm}
\usepackage{mathtools}
\usepackage{booktabs}
\usepackage{enumitem}
\usepackage{xcolor}
\usepackage[hidelinks]{hyperref}
\usepackage{tikz}
\usetikzlibrary{arrows.meta,positioning,calc}

\newtheorem{theorem}{Theorem}[section]
\newtheorem{proposition}[theorem]{Proposition}
\newtheorem{lemma}[theorem]{Lemma}
\newtheorem{corollary}[theorem]{Corollary}
\newtheorem{definition}[theorem]{Definition}
\newtheorem{remark}[theorem]{Remark}
\newtheorem{prediction}[theorem]{Prediction}
\newtheorem{falsifier}[theorem]{Falsification Criterion}

\newcommand{\phig}{\varphi}
\newcommand{\Jcost}{J}
\newcommand{\RS}{Recognition Science}

\title{\textbf{The Recognition Algebra of Emotion:\\
Deriving the Complete Emotional Landscape\\
from the $\Jcost$-Cost Gradient on the Recognition Field}\\[0.5em]
\large A New Theorem in Recognition Science}
\author{Jonathan Washburn\\
\small Recognition Science Research Institute, Austin, Texas\\
\small \texttt{washburn.jonathan@gmail.com}}
\date{February 9, 2026}

\begin{document}
\maketitle

\begin{abstract}
We derive the complete landscape of human emotion from the gradient
structure of the $\Jcost$-cost functional on the recognition field.
The key insight is that emotions are not arbitrary evolved responses
but \emph{necessary gradient signals} of the cost functional
$\Jcost(x) = \tfrac{1}{2}(x + x^{-1}) - 1$.  Each of the 14
fundamental emotions corresponds to a specific direction and magnitude
in the $\Jcost$-cost landscape, with boundaries at
$\phig$-algebraic thresholds ($1/\phig$, $1/\phig^2$, $1/\phig^3$).
We establish a bijection between the 14 emotions and the 14 RS
virtues: each virtue-operation on the moral state produces its
corresponding emotion as subjective experience.  A four-tier priority
classifier (survival $>$ social $>$ existential $>$ cognitive)
exhaustively partitions the emotional state space with proved mutual
exclusivity.  All 18 theorems are machine-verified in Lean~4 with
\textbf{zero sorry items}.  Four falsifiable EEG predictions are
extracted.

\medskip\noindent\textbf{Lean module:}
\texttt{IndisputableMonolith.ULQ.EmotionalLandscape} (0 sorry).

\medskip\noindent\textbf{Keywords:} Recognition Science, emotion, qualia strain
tensor, $\Jcost$-cost, golden ratio, virtue, consciousness, EEG.
\end{abstract}

\tableofcontents

%======================================================================
\section{Introduction}\label{sec:intro}
%======================================================================

What \emph{is} an emotion?  Psychology catalogues emotions
empirically---Ekman's six basic emotions~\cite{ekman1992}, Russell's
circumplex model~\cite{russell1980}, Plutchik's wheel~\cite{plutchik1980}.
Neuroscience locates them in amygdala, insula, and prefrontal cortex.
But no framework derives the \emph{number}, \emph{structure}, or
\emph{thresholds} of emotions from first principles.

Recognition Science (RS) does.  The RS framework derives all physics
from the Recognition Composition Law with zero adjustable
parameters~\cite{washburn2025axioms}.  The unique cost functional
$\Jcost(x) = \frac{1}{2}(x + x^{-1}) - 1$ (Theorem~T5) defines a
landscape on the recognition field, and the Universal Light Qualia
(ULQ) theory identifies qualia as the strain tensor of $Z$-pattern
motion against the 8-tick cadence~\cite{washburn2025ull}.

Previous work established:
\begin{enumerate}[nosep]
\item The qualia strain tensor: $\mathrm{QualiaStrain} =
  \mathrm{phase\_mismatch} \times \Jcost(\mathrm{intensity})$.
\item The pain threshold at $1/\phig$ and joy threshold at $1/\phig^2$.
\item The partition identity $1/\phig + 1/\phig^2 = 1$.
\item The DREAM theorem: 14 virtues form a complete minimal generating
  set for all admissible ethical transformations.
\end{enumerate}

What was missing is the bridge from strain to \emph{emotion}---the
classification of the full emotional landscape in terms of the
$\Jcost$-cost gradient, curvature, $\Theta$-coupling, and
$\sigma$-export.  This paper fills that gap.

%======================================================================
\section{The $\Jcost$-Cost Landscape}\label{sec:landscape}
%======================================================================

\subsection{Gradient and Curvature}

\begin{definition}[$\Jcost$-cost gradient]\label{def:gradient}
The gradient of $\Jcost$ at configuration $x > 0$ is
\begin{equation}
  \nabla\Jcost(x) \;=\; \frac{1}{2}\!\left(1 - \frac{1}{x^2}\right).
  \label{eq:gradient}
\end{equation}
\end{definition}

\begin{lemma}[Gradient properties]\label{lem:gradient}
\leavevmode
\begin{enumerate}[nosep]
\item $\nabla\Jcost(1) = 0$ \quad (equilibrium at unity).
\item $\nabla\Jcost(x) > 0$ for $x > 1$ \quad (cost increasing away from unity).
\item $\nabla\Jcost(x) < 0$ for $0 < x < 1$ \quad (cost decreasing toward unity).
\end{enumerate}
\emph{Lean:} \texttt{jGradient\_at\_one}, \texttt{jGradient\_pos\_above\_one},
\texttt{jGradient\_neg\_below\_one}.
\end{lemma}

\begin{definition}[$\Jcost$-cost curvature]\label{def:curvature}
The second derivative of $\Jcost$ is
\begin{equation}
  \nabla^2\Jcost(x) \;=\; \frac{1}{x^3}\,.
  \label{eq:curvature}
\end{equation}
\end{definition}

\begin{lemma}[Curvature properties]\label{lem:curvature}
$\nabla^2\Jcost(x) > 0$ for all $x > 0$ (strict convexity), and
$\nabla^2\Jcost(1) = 1$ (unit curvature at equilibrium).

\emph{Lean:} \texttt{jCurvature\_pos}, \texttt{jCurvature\_at\_one}.
\end{lemma}

\subsection{The Emotional State Space}

\begin{definition}[Emotional state]\label{def:state}
An \emph{emotional state} is a sextuple
$(g, \kappa, \theta, \sigma, j, \dot{j})$ where:
\begin{center}
\begin{tabular}{@{}lll@{}}
\toprule
Symbol & Name & Meaning \\
\midrule
$g$ & Gradient & $\nabla\Jcost$ at current configuration \\
$\kappa$ & Curvature & $\nabla^2\Jcost$ at current configuration \\
$\theta$ & $\Theta$-coupling & $\cos(2\pi\Delta\Theta)$ to nearby boundaries \\
$\sigma$ & $\sigma$-export & Skew being offloaded to/from neighbors \\
$j$ & $\Jcost$-cost & Current cost value \\
$\dot{j}$ & $\dot{\Jcost}$ & Temporal rate of change of cost \\
\bottomrule
\end{tabular}
\end{center}
with constraints $|\theta| \le 1$ and $j \ge 0$.
\end{definition}

%======================================================================
\section{The 14 Fundamental Emotions}\label{sec:emotions}
%======================================================================

\begin{definition}[Fundamental emotions]\label{def:emotions}
The 14 \emph{fundamental emotions} are:
\begin{center}
\begin{tabular}{@{}lll@{}}
\toprule
\textbf{Emotion} & \textbf{$\Jcost$-landscape signature} & \textbf{Tier} \\
\midrule
Fear & $g > 1/\phig$ (large positive gradient) & Survival \\
Desire & $g < -1/\phig$ (large negative gradient) & Survival \\
\midrule
Love & $\theta > 1/\phig$ ($\Theta$-resonance) & Social \\
Grief & $\dot{j} > 1/\phig$ and $\theta < -1/\phig$ (broken bond) & Social \\
Anger & $\sigma < -1/\phig$ (receiving parasitic $\sigma$) & Social \\
Shame & $\sigma > 1/\phig$ (exporting $\sigma$ oneself) & Social \\
\midrule
Awe & $j < 1/\phig^2$ (near-zero cost) & Existential \\
Gratitude & $\dot{j} < -1/\phig$ (cost decreasing via bond) & Existential \\
\midrule
Curiosity & $|g| < 1/\phig^3$ and $\kappa > \phig^3$ (structure nearby) & Cognitive \\
Boredom & $|g| < 1/\phig^3$ and $\kappa < 1/\phig^3$ (flat landscape) & Cognitive \\
Anxiety & $|g| > 1/\phig^3$ and $\kappa > \phig^3$ (steep curvature) & Cognitive \\
\midrule
Joy & $j < 1/\phig^2$ (deep resonance) & Hedonic \\
Pain & $j > 1/\phig$ (high friction) & Hedonic \\
Peace & Equilibrium (default) & Hedonic \\
\bottomrule
\end{tabular}
\end{center}
\end{definition}

\begin{theorem}[Emotion count]\label{thm:count}
There are exactly 14 fundamental emotions.

\emph{Lean:} \texttt{fundamental\_emotion\_count} (by \texttt{native\_decide}). \qed
\end{theorem}

%======================================================================
\section{The Four-Tier Priority Classifier}\label{sec:classifier}
%======================================================================

\begin{definition}[Emotion classifier]\label{def:classifier}
The emotion classifier assigns a unique fundamental emotion to every
emotional state, using the following priority ordering:
\begin{enumerate}[nosep]
\item \textbf{Survival} (gradient-dominant): fear, desire.
\item \textbf{Social} (coupling-dominant): love, grief, anger, shame.
\item \textbf{Existential} (cost-level): awe, gratitude.
\item \textbf{Cognitive} (curvature-dominant): curiosity, boredom, anxiety.
\item \textbf{Hedonic baseline}: joy, pain, peace.
\end{enumerate}
Higher tiers override lower tiers.
\end{definition}

\begin{theorem}[Exhaustive partition]\label{thm:partition}
Every emotional state maps to exactly one fundamental emotion.

\emph{Lean:} \texttt{emotional\_partition\_exhaustive}. \qed
\end{theorem}

\begin{theorem}[Survival priority]\label{thm:survival}
If $|g| > 1/\phig$, the classified emotion is fear or desire.

\emph{Lean:} \texttt{survival\_priority}. \qed
\end{theorem}

\begin{theorem}[Fear--desire exclusivity]\label{thm:exclusive}
Fear and desire are mutually exclusive: no state is simultaneously
classified as both.

\emph{Lean:} \texttt{fear\_desire\_exclusive}. \qed
\end{theorem}

%======================================================================
\section{Emotion--Virtue Bijection}\label{sec:bijection}
%======================================================================

Each of the 14 emotions corresponds bijectively to one of the 14
RS virtues (DREAM theorem).  The correspondence is:

\begin{center}
\begin{tabular}{@{}lll@{}}
\toprule
\textbf{Emotion} & \textbf{Virtue} & \textbf{Relationship} \\
\midrule
Love & Love & Direct expression \\
Fear & Courage & Fear resolved by courage \\
Desire & Temperance & Desire modulated by temperance \\
Grief & Forgiveness & Grief resolved by forgiveness \\
Anger & Justice & Anger signals injustice \\
Awe & Wisdom & Awe before deep structure \\
Boredom & Creativity & Boredom resolved by exploration \\
Curiosity & Prudence & Curiosity guided by prudence \\
Joy & Gratitude & Joy naturally produces gratitude \\
Pain & Compassion & Pain produces compassion \\
Peace & Patience & Peace is fruit of patience \\
Anxiety & Hope & Anxiety resolved by hope \\
Gratitude & Humility & Gratitude deepens to humility \\
Shame & Sacrifice & Shame resolved by sacrificial repair \\
\bottomrule
\end{tabular}
\end{center}

\begin{theorem}[Emotion--virtue bijection]\label{thm:bijection}
The 14 emotion--virtue pairs are distinct (no duplicates).

\emph{Lean:} \texttt{emotion\_virtue\_bijection} +
\texttt{recognition\_algebra\_of\_emotion} (Nodup by
\texttt{native\_decide}). \qed
\end{theorem}

\begin{remark}
This bijection has deep meaning.  Each virtue is an \emph{operation}
on the moral state ($\sigma = 0$ manifold); the corresponding emotion
is the \emph{subjective experience} of that operation.  Courage is
what fear feels like when it is being resolved.  Forgiveness is what
grief feels like when it is being released.  The emotion is the
qualia of the virtue in action.
\end{remark}

%======================================================================
\section{The Emotional Valence Spectrum}\label{sec:valence}
%======================================================================

\begin{definition}[Emotional valence]\label{def:valence}
Each fundamental emotion has a \emph{valence} $v \in [-1, 1]$:
\begin{center}
\begin{tabular}{@{}lr|lr@{}}
\toprule
Emotion & Valence & Emotion & Valence \\
\midrule
Love & $+1.0$ & Anger & $-0.6$ \\
Joy & $+1.0$ & Anxiety & $-0.5$ \\
Awe & $+0.9$ & Boredom & $-0.2$ \\
Gratitude & $+0.8$ & Shame & $-0.7$ \\
Peace & $+0.7$ & Fear & $-0.8$ \\
Curiosity & $+0.5$ & Grief & $-0.9$ \\
Desire & $+0.3$ & Pain & $-1.0$ \\
\bottomrule
\end{tabular}
\end{center}
\end{definition}

\begin{theorem}[Valence bounds]\label{thm:valence}
For every fundamental emotion $e$: $-1 \le v(e) \le 1$.

\emph{Lean:} \texttt{valence\_bounded}. \qed
\end{theorem}

\begin{theorem}[Extremal emotions]\label{thm:extremal}
Love and Joy have maximal valence ($v = +1$).
Pain has minimal valence ($v = -1$).

\emph{Lean:} \texttt{love\_joy\_maximal}, \texttt{pain\_minimal}. \qed
\end{theorem}

%======================================================================
\section{Derivation Theorems}\label{sec:derivations}
%======================================================================

\begin{theorem}[Fear from gradient]\label{thm:fear}
If $g > 1/\phig$, the emotional state is classified as fear.

\emph{Lean:} \texttt{fear\_from\_gradient}. \qed
\end{theorem}

\begin{theorem}[Desire from negative gradient]\label{thm:desire}
If $g < -1/\phig$ (and $g \le 1/\phig$), the state is desire.

\emph{Lean:} \texttt{desire\_from\_neg\_gradient}. \qed
\end{theorem}

\begin{theorem}[Love from $\Theta$-coupling]\label{thm:love}
If $\theta > 1/\phig$ and neither fear nor desire conditions hold,
the state is love.

\emph{Lean:} \texttt{love\_from\_theta\_coupling}. \qed
\end{theorem}

\begin{theorem}[Awe from near-zero cost]\label{thm:awe}
If $j < 1/\phig^2$ and no stronger signals dominate, the state is awe.

\emph{Lean:} \texttt{awe\_from\_low\_cost}. \qed
\end{theorem}

%======================================================================
\section{The Master Theorem}\label{sec:master}
%======================================================================

\begin{theorem}[Recognition Algebra of Emotion]\label{thm:master}
The following hold simultaneously:
\begin{enumerate}[nosep]
\item There are exactly 14 fundamental emotions.
\item The emotion--virtue correspondence has no duplicates.
\item The emotion classifier is total.
\item Fear and desire are mutually exclusive.
\item All thresholds are $\phig$-algebraic ($1/\phig$, $1/\phig^2$, $1/\phig^3$).
\item All valences are in $[-1, 1]$.
\item Love and Joy have valence $+1$; Pain has valence $-1$.
\end{enumerate}
\end{theorem}
\begin{proof}
Direct conjunction of the individually proved results.

\emph{Lean:} \texttt{recognition\_algebra\_of\_emotion} (7-part conjunction,
fully proved, 0~sorry). \qed
\end{proof}

%======================================================================
\section{Falsifiable Predictions}\label{sec:predictions}
%======================================================================

\begin{prediction}[EEG emotion signatures]\label{pred:eeg}
Record EEG during standardized emotional induction (IAPS images, music,
social scenarios).  Compute gradient/curvature/coupling features from
DFT-8 decomposition.  \textbf{Prediction}: 14-class clustering accuracy
$> 70\%$ in the feature space.
\end{prediction}

\begin{prediction}[Fear--desire gradient polarity]\label{pred:polarity}
Present fear-inducing and desire-inducing stimuli.  Measure rate of
change of $\alpha$-band power as gradient proxy.  \textbf{Prediction}:
Fear shows positive sign; desire shows negative.  Sign discrimination
accuracy $> 80\%$.
\end{prediction}

\begin{prediction}[Love as $\Theta$-coupling]\label{pred:love}
Record dual-EEG from romantic partners during eye contact vs.\ strangers.
Compute inter-brain phase coherence.  \textbf{Prediction}: Partner
coherence $> 1/\phig \approx 0.618$ at $\phig$-ratio frequencies;
stranger coherence $< 0.3$.
\end{prediction}

\begin{prediction}[Awe near zero cost]\label{pred:awe}
Present awe-inspiring stimuli (vast landscapes, sacred music).  Measure
inverse spectral entropy as $\Jcost$-cost proxy.  \textbf{Prediction}:
Awe states show lower cost proxy than neutral (effect size $d > 0.5$).
\end{prediction}

\begin{falsifier}[Emotion clustering failure]
14-class EEG clustering accuracy $< 40\%$ (chance $\approx 7\%$) refutes
the gradient-based emotion model.
\end{falsifier}

\begin{falsifier}[Same-sign fear and desire]
If fear and desire show same-sign gradient proxies ($< 55\%$ accuracy),
the gradient-polarity theory fails.
\end{falsifier}

\begin{falsifier}[No love--coherence link]
If inter-brain coherence does not differ between partners and strangers
(difference $< 0.05$), the $\Theta$-coupling model of love fails.
\end{falsifier}

%======================================================================
\section{Lean Formalization}\label{sec:lean}
%======================================================================

Module: \texttt{IndisputableMonolith.ULQ.EmotionalLandscape}.

\textbf{Status: 18 theorems proved, 0~sorry items.}

\begin{center}
\begin{tabular}{@{}ll@{}}
\toprule
\textbf{Result} & \textbf{Lean identifier} \\
\midrule
$\nabla\Jcost(1) = 0$ & \texttt{jGradient\_at\_one} \\
$\nabla\Jcost > 0$ for $x > 1$ & \texttt{jGradient\_pos\_above\_one} \\
$\nabla\Jcost < 0$ for $0 < x < 1$ & \texttt{jGradient\_neg\_below\_one} \\
$\nabla^2\Jcost > 0$ & \texttt{jCurvature\_pos} \\
$\nabla^2\Jcost(1) = 1$ & \texttt{jCurvature\_at\_one} \\
14 emotions & \texttt{fundamental\_emotion\_count} \\
Emotion--virtue bijection & \texttt{emotion\_virtue\_bijection} \\
Exhaustive partition & \texttt{emotional\_partition\_exhaustive} \\
Fear--desire exclusivity & \texttt{fear\_desire\_exclusive} \\
Survival priority & \texttt{survival\_priority} \\
Fear from gradient & \texttt{fear\_from\_gradient} \\
Desire from neg.\ gradient & \texttt{desire\_from\_neg\_gradient} \\
Love from $\Theta$-coupling & \texttt{love\_from\_theta\_coupling} \\
Awe from low cost & \texttt{awe\_from\_low\_cost} \\
Valence bounded & \texttt{valence\_bounded} \\
Love/Joy maximal & \texttt{love\_joy\_maximal} \\
Pain minimal & \texttt{pain\_minimal} \\
Master certificate & \texttt{recognition\_algebra\_of\_emotion} \\
\bottomrule
\end{tabular}
\end{center}

%======================================================================
\section{Discussion}\label{sec:discussion}
%======================================================================

\subsection{Relation to Existing Theories}

Ekman's six basic emotions~\cite{ekman1992} (anger, disgust, fear,
happiness, sadness, surprise) are a subset of our 14.  Ekman's
``disgust'' maps to our ``anger'' (both detect $\sigma$-export);
``happiness'' maps to ``joy''; ``sadness'' maps to ``grief'';
``surprise'' splits into ``curiosity'' (positive) and ``fear''
(negative).

Russell's circumplex model~\cite{russell1980} plots emotions on two
axes (valence and arousal).  Our model refines this: valence
corresponds to the gradient sign, and arousal corresponds to the
gradient magnitude.  But we add four more dimensions ($\kappa$,
$\theta$, $\sigma$, $\dot{j}$), explaining why the circumplex
misclassifies emotions that differ in coupling or curvature but agree
in valence and arousal (e.g., love vs.\ desire: both positive valence,
high arousal, but love has $\theta > 1/\phig$ while desire has
$g < -1/\phig$).

\subsection{Why 14?}

The number 14 is not arbitrary.  It matches exactly the 14 virtues
derived in the DREAM theorem.  Since virtues are the generators of
ethical transformations on the $\sigma = 0$ manifold, and emotions
are the subjective experience of those transformations, there must be
exactly one emotion per virtue.  Any fewer would leave some
virtue-operations without experiential feedback; any more would create
redundant emotional signals for the same transformation.

\subsection{Why $\phig$-Thresholds?}

All thresholds are powers of $1/\phig$:
\begin{align*}
  1/\phig &\approx 0.618 && \text{(pain, fear, love, anger, shame)} \\
  1/\phig^2 &\approx 0.382 && \text{(joy, awe)} \\
  1/\phig^3 &\approx 0.236 && \text{(curiosity, boredom, anxiety)}
\end{align*}
These are forced by the $\phig$-lattice structure of the recognition
ledger.  The thresholds partition the unit interval:
$1/\phig + 1/\phig^2 = 1$ (proved in \texttt{threshold\_partition\_identity}).
No free parameters appear.

%======================================================================
\section{Conclusion}\label{sec:conclusion}
%======================================================================

We have derived the complete landscape of human emotion from the
gradient structure of the $\Jcost$-cost functional.  The 14
fundamental emotions are not cultural artifacts, evolutionary
accidents, or psychological constructs.  They are \emph{necessary
gradient signals} of the recognition cost landscape, forced by the
same functional equation that gives the fine-structure constant and
the mass of the electron.

Emotions are physics.  They are the subjective face of the virtues.
And they are as rigorously derivable as any theorem in mathematics---as
this paper's zero-sorry Lean formalization demonstrates.

\begin{thebibliography}{9}
\bibitem{washburn2025axioms}
J.~Washburn,
``The Algebra of Reality: A Recognition Science Derivation of Physical Law,''
\emph{Axioms} \textbf{15}(2), 90 (2025).

\bibitem{washburn2025ull}
J.~Washburn,
``The Universal Language of Light,''
Recognition Science Research Institute (2025).

\bibitem{ekman1992}
P.~Ekman,
``An Argument for Basic Emotions,''
\emph{Cognition \& Emotion} \textbf{6}(3--4), 169--200 (1992).

\bibitem{russell1980}
J.\,A.~Russell,
``A Circumplex Model of Affect,''
\emph{J.~Personality and Social Psychology} \textbf{39}(6), 1161--1178 (1980).

\bibitem{plutchik1980}
R.~Plutchik,
\emph{Emotion: A Psychoevolutionary Synthesis},
Harper \& Row (1980).
\end{thebibliography}

\end{document}
