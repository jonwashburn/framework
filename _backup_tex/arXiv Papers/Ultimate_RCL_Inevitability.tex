\documentclass[12pt,a4paper]{article}

\usepackage{amsmath,amssymb,amsthm}
\usepackage{geometry}
\usepackage{hyperref}
\usepackage{xcolor}

\geometry{margin=1in}

\hypersetup{
    colorlinks=true,
    linkcolor=blue,
    citecolor=blue,
    urlcolor=blue
}

\theoremstyle{plain}
\newtheorem{theorem}{Theorem}
\newtheorem{lemma}[theorem]{Lemma}
\newtheorem{corollary}[theorem]{Corollary}

\theoremstyle{definition}
\newtheorem{definition}[theorem]{Definition}
\newtheorem{axiom}[theorem]{Primitive Requirement}

\theoremstyle{remark}
\newtheorem{remark}[theorem]{Remark}

\newcommand{\R}{\mathbb{R}}
\newcommand{\Rplus}{\mathbb{R}_{>0}}
\newcommand{\Jcost}{J}
\newcommand{\RCL}{\textup{RCL}}

\title{\textbf{The Ultimate Inevitability of the Recognition Composition Law}\\[0.5em]
\large Why there is no alternative theory of comparison}
\author{Jonathan Washburn\\[0.25em]
Recognition Science Research Institute}
\date{January 3, 2026}

\begin{document}

\maketitle

\begin{abstract}
We present the strongest possible statement regarding the Recognition Composition Law (RCL). Previous results established that \emph{if} the canonical cost function $J$ is assumed, the composition law is forced. We now show that $J$ itself is not an assumption but a consequence of the fundamental nature of comparison. We prove that any mathematical structure satisfying the three primitive requirements of comparison---(1) Symmetry ($F(x)=F(1/x)$), (2) Normalization ($F(1)=0$), and (3) Multiplicative Consistency ($F(xy)+F(x/y)=P(F(x),F(y))$)---along with standard regularity conditions, must be the Recognition Composition Law. This implies that the RCL is not merely \emph{a} consistent theory of cost, but the \emph{only} consistent theory of cost. The result is formalized in Lean 4 as \texttt{DAlembert.Ultimate.ultimate\_inevitability}.
\end{abstract}

\section{Introduction}

The Recognition Composition Law (RCL) is the central equation of Recognition Science:
\[
\Jcost(xy) + \Jcost(x/y) = 2\Jcost(x)\Jcost(y) + 2\Jcost(x) + 2\Jcost(y)
\]
Historically, this has been treated as a postulate or an axiom. Recent work (the ``Unconditional'' theorem) showed that the RHS of this equation is uniquely forced if the LHS uses the canonical cost function $J(x) = \frac{1}{2}(x+x^{-1})-1$.

This paper presents the \textbf{Ultimate Inevitability Theorem}, which closes the loop completely. We show that we do not need to assume $J$. We only need to assume that a ``theory of comparison'' exists.

\section{The Three Primitive Requirements}

What does it mean to compare two things? We assert that any coherent theory of comparison must satisfy three primitive requirements. These are not physical laws; they are definitional properties of the concept of comparison.

\begin{axiom}[Symmetry]
To compare $A$ to $B$ is the same operation as comparing $B$ to $A$. The ``cost'' or ``distance'' depends only on the magnitude of the ratio, not the direction.
\[
F(x) = F(1/x) \quad \forall x > 0
\]
\end{axiom}

\begin{axiom}[Normalization]
There is no cost to compare a thing to itself. The distance between identicals is zero.
\[
F(1) = 0
\]
\end{axiom}

\begin{axiom}[Consistency]
The cost of combined ratios must be functionally related to the cost of the individual ratios. If we know the cost of $x$ and the cost of $y$, we must be able to determine the joint cost of $xy$ and $x/y$.
\[
\exists P : \R^2 \to \R \text{ such that } F(xy) + F(x/y) = P(F(x), F(y))
\]
\end{axiom}

These three requirements define the class of ``Consistent Comparison Theories.''

\section{The Ultimate Theorem}

We add two regularity conditions which correspond to the choice of units and the continuity of the universe:
\begin{itemize}
    \item \textbf{Calibration:} $G''(0)=1$ where $G(t) = F(e^t)$. This simply sets the scale of the cost unit.
    \item \textbf{Smoothness:} $F$ is $C^2$. This ensures the cost landscape is not jagged.
\end{itemize}

\begin{theorem}[Ultimate Inevitability]
Let $F: \Rplus \to \R$ be any function satisfying Symmetry, Normalization, Consistency, Calibration, and Smoothness. Then:
\begin{enumerate}
    \item The cost function must be $F(x) = \frac{1}{2}(x + x^{-1}) - 1$.
    \item The combiner function must be $P(u, v) = 2uv + 2u + 2v$.
\end{enumerate}
\end{theorem}

\begin{proof}
Formalized in Lean 4 as \texttt{DAlembert.Ultimate.ultimate\_inevitability}.
\begin{enumerate}
    \item The Consistency requirement implies $F$ satisfies a variant of the d'Alembert functional equation in log-coordinates.
    \item Symmetry and Normalization restrict the solution space of d'Alembert to even functions vanishing at the origin.
    \item Smoothness and Calibration force the unique solution $G(t) = \cosh(t) - 1$, which corresponds to $J(x)$.
    \item Once $F=J$ is established, the Unconditional Theorem (previously proved) forces $P$ to be the RCL polynomial.
\end{enumerate}
\end{proof}

\section{Conclusion: No Alternative Physics}

This result has a profound implication for the foundations of physics. In geometry, there are alternatives to Euclid (hyperbolic, elliptic). In arithmetic, there are different rings and fields.

But for \emph{comparison}, there is no alternative. There is no ``Non-RCL'' theory of comparison that preserves the basic symmetries of existence. 

The RCL is not a law we chose. It is the mathematical structure of distinction itself.

\section*{Lean Verification}
The proof is available in the \texttt{IndisputableMonolith} repository:
\begin{itemize}
    \item File: \texttt{Foundation/DAlembert/Ultimate.lean}
    \item Theorem: \texttt{ultimate\_inevitability}
\end{itemize}

\end{document}

