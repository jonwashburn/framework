\documentclass[11pt]{article}

\usepackage[utf8]{inputenc}
\usepackage[T1]{fontenc}
\usepackage[margin=1.25in]{geometry}
\usepackage{parskip}
\usepackage{hyperref}

\hypersetup{
    colorlinks=true,
    linkcolor=black,
    urlcolor=blue
}

% Title formatting
\title{\vspace{-1cm}\textbf{Why Gravity Exists}\\[0.5em]
\large A Simple Explanation from First Principles}

\author{Jonathan Washburn\\
\small Recognition Physics Institute\\
\small \href{mailto:jon@recognitionphysics.org}{jon@recognitionphysics.org}}

\date{December 2025}

\begin{document}

\maketitle

\vspace{1em}

\begin{abstract}
\noindent Physics tells us \textit{how} gravity behaves---masses attract, light bends, time slows near heavy objects. But physics has never explained \textit{why} gravity exists at all. This essay offers an answer. Gravity is what happens when reality has to process itself, and some regions require more processing than others. The consequences---time dilation, curved spacetime, falling objects---all follow from this single insight.
\end{abstract}

\vspace{2em}

\section*{Part 1: The Processing Problem}

In the film \textit{Interstellar}, astronauts land on Miller's planet, which orbits close to a massive black hole called Gargantua. They spend what feels like a few hours on the surface. When they return to their ship in higher orbit, twenty-three years have passed for their crewmate who stayed behind.

This is not science fiction. This is real physics, confirmed by countless experiments. Clocks on GPS satellites tick faster than clocks on Earth's surface---by enough that engineers must correct for it, or your phone's map would drift by kilometers per day. Time genuinely runs slower near mass.

Einstein's general relativity describes this effect with precision. Mass curves spacetime, and time dilation is part of that curvature. The mathematics works beautifully. But here is the question no physics textbook answers: \textit{why} does mass curve spacetime? What is actually happening near that black hole that makes time slow down?

The standard answer is that mass just \textit{does} curve spacetime---it's a brute fact about the universe. But this is not an explanation. It's a description wearing the mask of an explanation.

\subsection*{Reality Must Process Itself}

Consider what it means for reality to exist from moment to moment. Every particle must maintain its identity. Every interaction must be tracked. Every quantum state must evolve according to definite rules. This is not nothing. This is \textit{work}---something like computation or bookkeeping must be happening to keep the universe consistent.

This idea might seem strange, but take it seriously for a moment. If reality requires processing to maintain itself, then we can ask: is this processing capacity finite or infinite?

If infinite, then every region of space could be updated instantaneously and perfectly, with no constraints whatsoever. But we know this isn't true. Light has a maximum speed. Information cannot travel faster than $c$. There are fundamental limits built into the structure of reality.

So processing capacity is finite. And if it's finite, it must be allocated somehow. Some regions might require more processing than others.

\subsection*{What Is Mass?}

Now ask: what makes one region of space different from another? The answer is \textit{complexity}. A cubic meter of empty space is simple---there's little to track. A cubic meter containing a hydrogen atom is more complex---there's a proton, an electron, their interaction, their quantum states. A cubic meter inside a neutron star is enormously complex---trillions of particles, immense energies, intricate quantum correlations.

Mass, in this view, is not a mysterious intrinsic property. Mass is a measure of how much structure is present---how much \textit{bookkeeping} a region requires to maintain itself.

A stone has more mass than a feather not because it contains some invisible ``mass-stuff,'' but because it embodies more structural complexity. More particles. More interactions. More to keep track of.

\subsection*{Processing Load and Refresh Rate}

Here is the key insight: if a region contains more complexity, it requires more processing to maintain. And if processing capacity is finite, regions of high complexity will experience a kind of \textit{slowdown}.

Think of a computer running multiple programs. When one program demands more CPU cycles, other programs slow down---or that program itself takes longer to complete each operation. There is only so much processing to go around.

Now apply this to reality. A region of high mass---high complexity---demands more processing to maintain all those particles, all those interactions, all those quantum states. With finite capacity, the ``refresh rate'' of that region must slow down.

From inside that region, nothing seems different. Your thoughts, your heartbeat, your clocks---all are part of the same local processing, so they all slow down together. You notice nothing.

But from outside, looking in? That region is updating more slowly. Time, as measured by any process in that region, runs slower than time in a less complex region.

\textit{This is time dilation.}

Miller's planet doesn't experience slow time because of some mystical influence from Gargantua. It experiences slow time because it exists in a region of extreme processing load---near a concentration of mass so dense that the local reality requires enormous resources to maintain. The refresh rate slows. Time dilates. Twenty-three years pass outside while hours pass within.

\subsection*{Mass Doesn't Cause Time Dilation---Mass \textit{Is} Processing Load}

It is important to understand what is \textit{not} being claimed here. We are not saying that mass reaches out and ``does something'' to time through some mechanism. That would just push the mystery back one step.

Rather, mass and time dilation are two aspects of the same underlying reality. Mass \textit{is} processing load. Time dilation \textit{is} what high processing load looks like. They are not cause and effect---they are the same phenomenon seen from different angles.

When you ask ``why does mass slow down time?'' you are asking the wrong question. It's like asking ``why does the front of the car move forward when the back of the car moves forward?'' They are not separate things. There is just the car, moving.

There is just reality, processing itself. Where that processing is dense, we call it mass. The slowdown of that processing, we call time dilation. Same thing.

This is why general relativity's equations work so well without ever explaining \textit{why}. The equations describe the relationship between mass-energy and spacetime curvature---which is to say, between processing load and refresh rate. The equations capture the pattern perfectly. What they don't do is tell you what the pattern \textit{is}. 

Now we can say: the pattern is processing. The pattern is the finite capacity of reality to maintain itself, and the necessary consequences of that finitude.

\newpage

\section*{Part 2: The Gradient Extends Outward}

There is a tempting but wrong way to picture what we have said so far. You might imagine a star as a dense ball of matter with a kind of ``processing boundary'' at its surface---a shell where all the computational work happens, with ordinary empty space beyond.

This is not how it works. The processing load is not contained at the surface like the skin of a balloon. It \textit{radiates outward} as a gradient, falling off with distance but never fully disappearing.

\subsection*{Every Point Knows About the Star}

Think carefully about what it means for a star to exist. The star is not isolated. It interacts with everything around it. Light leaves it. Gravitational influence extends from it. Particles from the solar wind stream away from it. Any object passing nearby must have its trajectory affected.

This means that every point in space near the star must, in some sense, \textit{know about} the star. The local physics at each point must account for the star's presence---its location, its mass, its influence. This ``knowing'' is not metaphorical. It is encoded in the structure of spacetime itself, in the local values of fields, in the information content of that region.

The closer you are to the star, the more of your local reality is devoted to tracking it. At the star's surface, almost everything is about the star---the density is extreme, the interactions are overwhelming, the information content is enormous. A million kilometers away, the star is still the dominant influence, but less so. A billion kilometers away, the star is one gravitational influence among others. At the edge of the solar system, the star is a minor consideration. In intergalactic space, it is negligible.

But it is never \textit{zero}. Even at vast distances, some tiny fraction of the local information budget is allocated to that distant star. The star's existence is written, however faintly, into the fabric of spacetime everywhere.

\subsection*{The Shape of the Gradient}

This gives us a picture of gravity that is radically different from the textbook image of a ``force'' reaching out from a mass. There is no force reaching out. There is only a landscape of processing density, with peaks at masses and a smooth falloff in all directions.

Near the Sun, the processing density is high. Spacetime there is ``thick'' with information---every cubic meter must track enormous numbers of particles, intense fields, complex quantum states. The refresh rate slows accordingly.

As you move outward, the processing density drops. There is less to track. The fields are weaker. The particle density is lower. The refresh rate increases, approaching (but never quite reaching) the baseline rate of empty space far from any mass.

This gradient has a specific shape. In three-dimensional space, the ``information about the star'' spreads outward over spherical surfaces. The surface area of a sphere grows with the square of the radius. So the information density per unit area falls off as one over the radius squared.

This is why gravity follows an inverse-square law in the Newtonian limit. It is not a mysterious coincidence. It is geometry. The processing load of a spherical mass spreads over spherical surfaces, and spherical surfaces grow as the square of distance. The mathematics could not be otherwise.

\subsection*{Spacetime Curvature Is the Gradient}

Now we can say precisely what ``curved spacetime'' means in this picture.

Spacetime curvature is not some abstract mathematical property layered on top of space. Spacetime curvature \textit{is} the processing gradient. The two descriptions are identical---different languages for the same reality.

When physicists say that mass curves spacetime, they are saying that mass creates a processing-density gradient. When they write down the metric tensor of general relativity, they are describing the shape of that gradient. When they calculate geodesics---the paths that free-falling objects follow---they are finding the trajectories that navigate the gradient with minimum total cost.

Einstein's field equations, in this view, are not arbitrary rules that nature happens to follow. They are the mathematical description of how processing load distributes itself given the presence of mass-energy. The equations \textit{must} take the form they do, because they are describing the necessary relationship between information density and refresh rate.

This is why general relativity works so extraordinarily well. It is not an approximation. It is not a model that happens to fit the data. It is a precise description of the processing landscape, and that description is essentially complete for classical phenomena.

\subsection*{The Sun's Gravity Is Not ``At'' the Sun}

Consider what this means for how we think about the Sun's gravity.

In the old Newtonian picture, the Sun sits at the center of the solar system and somehow ``reaches out'' to pull on the planets. The gravity is imagined as a kind of invisible tether connecting the Sun to each planet, with the Sun doing the pulling.

In the relativistic picture, this is replaced by curved spacetime. The Sun curves the space around it, and planets follow the curves. This is better, but still suggests that the curving is something the Sun ``does'' to the space ``around'' it---as if the Sun is here, and the curved space is there, and they are separate things.

In the processing picture, even this separation dissolves. The Sun is not separate from the space around it. The Sun \textit{is} a peak in the processing landscape. The ``curved space'' around it is not something the Sun creates---it is the Sun, extended. The gradient is not the Sun's \textit{effect}; it is part of what the Sun \textit{is}.

When you stand on Earth and feel the Sun's gravity (or rather, see its effects on Earth's orbit), you are not experiencing an influence that traveled from the Sun to here. You are experiencing the local processing density at Earth's location---a processing density that is elevated because this region of space must account for the Sun's existence.

The Sun's gravity is not ``at'' the Sun. The Sun's gravity is everywhere the Sun's information has spread---which is everywhere, to varying degrees.

\subsection*{Why the Gradient Matters}

This might seem like philosophical hair-splitting. What difference does it make whether we think of gravity as a force, as curved spacetime, or as a processing gradient?

The difference becomes clear when we ask: why do things \textit{move} toward mass? Why does the apple fall? Why does light bend toward the Sun rather than away from it?

If gravity is a force, then things move toward mass because the force pulls them. But this just restates the mystery---\textit{why} does the force pull? What is doing the pulling?

If gravity is curved spacetime, then things move toward mass because spacetime is curved that way. But this also restates the mystery---\textit{why} is spacetime curved? And why does curving make things move inward rather than outward?

But if gravity is a processing gradient, then we can finally answer these questions. Things move toward mass because of what it means to exist as a coherent pattern across a gradient. Things move toward slower-refreshing regions because that is the only way to maintain coherence.

This is what Part 3 will explain.

\newpage

\section*{Part 3: Why Things Fall}

We now have all the pieces. Mass is processing load. The processing load creates a gradient that extends outward, falling off with distance. Time runs slower where processing is denser.

But we have not yet answered the most basic question of all: why does the apple fall?

It is not enough to say that spacetime is curved, or that there is a gradient. A ball sitting on a hillside does not roll down just because the hill is sloped---you need to explain \textit{why} the slope makes it roll. Similarly, we need to explain why a processing gradient makes things move toward the denser region.

The answer is simple, but it requires careful thought: \textbf{things fall because falling is the only way to stay whole.}

\subsection*{You Are Not a Point}

The key insight is that you are not a dimensionless point. You are an extended object. You have a top and a bottom, a left and a right, a front and a back. Even the smallest particle has some spatial extent---a wavelength, a region of influence, a quantum spread.

This means that when you exist in a processing gradient, \textit{different parts of you experience different refresh rates.}

Stand on the surface of the Earth. Your feet are closer to the Earth's center than your head. The processing density at your feet is slightly higher than at your head. Therefore, time at your feet runs slightly slower than time at your head.

This is not a metaphor. It is measurable. Atomic clocks at different altitudes tick at detectably different rates. Your feet really do age more slowly than your head, by a tiny but nonzero amount.

Now ask: what does this mean for you as a coherent pattern? What does it mean for something to \textit{exist} across a region where different parts are refreshing at different rates?

\subsection*{The Asymmetry of Refresh}

Imagine you are a pattern---a structured arrangement of matter and energy that persists through time. At each moment, you must update yourself. Your atoms must maintain their configurations. Your molecules must preserve their bonds. Your cells must continue their processes. All of this requires the local reality to refresh, to compute the next state from the current state.

Now imagine that the bottom of you is refreshing more slowly than the top of you.

In the time it takes your head to complete one update cycle, your feet have completed slightly \textit{less} than one cycle. There is a mismatch. The top of you is ``pulling ahead'' of the bottom of you in some sense.

For you to remain a coherent, unified pattern, this mismatch must be reconciled. The different parts of you must stay synchronized despite experiencing different refresh rates.

How is this reconciliation achieved?

\subsection*{Falling as Synchronization}

Here is the answer: \textbf{you fall.}

When you move toward the slower-refreshing region (downward, toward the mass), you are doing something very specific. You are moving the faster-refreshing parts of yourself into the regime where they will refresh more slowly. You are equalizing the refresh rates across your spatial extent.

Think of it this way. If your head is refreshing faster than your feet, then your head is accumulating a kind of ``update debt'' relative to your feet. The way to pay off this debt is to move your head into a region where it will refresh more slowly---which means moving downward.

Falling is not something that happens \textit{to} you because of an external force. Falling is something you \textit{do} in order to maintain internal coherence. It is the natural response of any extended pattern to a refresh gradient.

If you did not fall---if you somehow remained fixed in place---then the refresh mismatch would accumulate. The top of you would increasingly desynchronize from the bottom of you. You would experience internal stress, a kind of tearing. For ordinary matter in ordinary gravitational fields, this stress is far too small to notice. But the tendency is there: the gradient wants to pull you apart, and falling is how you avoid being pulled apart.

\textit{Falling is what staying whole looks like in a refresh gradient.}

\subsection*{Why Light Bends}

The same logic explains why light bends toward massive objects.

Light is often described as a particle (a photon) or as a wave. In either description, light has spatial extent. A photon is not a dimensionless point---it has a wavelength, a spatial spread. A light wave has wavefronts that extend across space.

When light passes near a massive object like the Sun, it traverses a region where the processing density varies across its spatial extent. The part of the wavefront closer to the Sun is in a slower-refreshing region than the part farther from the Sun.

What happens to a wavefront when one side refreshes slower than the other?

It pivots. It turns toward the slower side.

Think of a marching band crossing a muddy patch. If the left side of the line hits the mud first, those marchers slow down while the right side continues at normal speed. The line pivots---it turns toward the slower side. No one is ``pulling'' the band leftward. The turn happens automatically because of the speed differential across the line.

Light bending around the Sun is exactly the same phenomenon. The inner edge of the light beam (closer to the Sun) refreshes more slowly than the outer edge. The wavefront pivots toward the Sun. The light bends inward.

Again, nothing is pulling the light. There is no force acting on the photons. The bending is a necessary consequence of the refresh differential across the light's spatial extent. It is what coherent propagation looks like in a processing gradient.

\subsection*{Geodesics: Paths of Least Resistance}

Physicists describe the paths of falling objects and bending light as ``geodesics''---the straightest possible paths through curved spacetime. This language is mathematically precise but can obscure the physical meaning.

In the processing picture, geodesics have a clear interpretation: they are the paths that maintain coherence with minimum internal stress.

Any path through a processing gradient will involve some refresh differential across the traveling object's extent. But some paths are worse than others. A path that fights the gradient---that tries to maintain constant altitude, for instance---requires constant expenditure of energy to avoid falling. This is what we call ``standing still'' in a gravitational field, and it is not natural; it requires a force (like the ground pushing up on your feet).

A geodesic is the path of surrender. It is the path you follow when you stop fighting the gradient and simply let coherence-maintenance do its work. When you fall freely, you are following a geodesic. When light bends around the Sun, it is following a geodesic. In both cases, the path is determined by the requirement of staying whole while traversing the gradient.

This is why gravity feels like nothing when you are in free fall. Astronauts in orbit are not ``weightless'' because gravity has stopped acting on them---they are deep in Earth's gravitational field. They feel weightless because they are following a geodesic, falling freely, experiencing no internal stress from the refresh gradient. The gradient is still there. But they are moving with it rather than against it, so they feel nothing.

\subsection*{Attraction Without Force}

We can now answer the question we began with. Why do things fall? Why does light bend toward mass? Why do massive objects ``attract'' each other?

The answer: there is no attraction. There is no force reaching out from mass to pull on other objects. There is no mysterious action-at-a-distance.

There is only the processing gradient, and the requirement that patterns maintain coherence across that gradient. The gradient creates a refresh differential across any extended object. The object responds by moving in the direction that minimizes that differential---which is toward the slower-refreshing region, which is toward the mass.

What we call ``gravitational attraction'' is actually \textit{coherence-seeking behavior} in a refresh landscape. Objects do not fall because they are pulled. Objects fall because falling is the only way to remain objects.

This is why Einstein said that gravity is not a force. In general relativity, a freely falling object is not accelerating---it is following the natural, unforced path through spacetime. The mathematics of general relativity encodes exactly what we have described: curved spacetime is the processing gradient, geodesics are coherence-maintaining paths, and ``gravity'' is just geometry.

What we have added is the \textit{why}. Why is there a gradient? Because mass is processing load, and processing load must be distributed. Why do objects follow geodesics? Because extended patterns must maintain coherence, and geodesics are the paths that achieve this with minimum stress.

\subsection*{The Interstellar Scene, Revisited}

Return to Miller's planet one last time.

The astronauts descend to the surface, where time runs desperately slow because Gargantua's mass creates an extreme processing gradient. When they return to their ship, twenty-three years have passed outside.

Now you understand why.

Gargantua is not ``pulling'' on time. Gargantua \textit{is} a peak in the processing landscape---an enormous concentration of structure that demands enormous computational resources to maintain. Near Gargantua, the refresh rate of reality slows to a crawl.

Miller's planet, deep in that gradient, exists in a region where the universe is processing as hard as it can just to maintain the local physics. Time there runs at a fraction of its normal rate---not because time is being ``affected'' by gravity, but because time \textit{is} the refresh rate, and the refresh rate is determined by processing load.

The astronauts on the surface age slowly because \textit{they} are refreshing slowly. The astronaut in orbit ages quickly because \textit{he} is in a region of lower processing density. Both are experiencing time normally from their own perspectives. The difference only becomes apparent when they reunite and compare notes.

And if you asked why Miller's planet doesn't fall into Gargantua---or rather, why it orbits instead of falling straight in---the answer is the same logic applied to circular motion. The planet is following a geodesic, a coherence-maintaining path through the gradient. That path happens to be an orbit rather than a straight plunge, because of the planet's initial motion. But orbiting is falling, in the deepest sense. The planet is perpetually falling toward Gargantua, perpetually surrendering to the gradient, perpetually maintaining its coherence by moving with the processing landscape rather than against it.

\subsection*{Conclusion: Gravity Is Existence}

Gravity is not a force. Gravity is not an interaction. Gravity is not something that happens between masses.

Gravity is what existence looks like when the processing that maintains reality is distributed unevenly. Gravity is the geometry of that distribution. Gravity is the behavior of patterns maintaining themselves across that geometry.

Mass curves spacetime because mass is processing load, and processing load determines refresh rate, and the variation in refresh rate across space is what we call curvature.

Objects fall because they are extended, and extended patterns in a refresh gradient must move toward slower regions to maintain coherence.

Light bends because wavefronts pivot when one side refreshes slower than the other.

Orbits are stable because they are geodesics---paths of perpetual free fall, perpetual surrender to the gradient, perpetual coherence without stress.

The equations of general relativity are not arbitrary rules. They are the precise mathematical description of processing-density gradients and the behavior of patterns within them. Einstein discovered the equations; now we understand what they mean.

Gravity is not mysterious. Gravity is inevitable. Any reality that processes itself, with finite capacity unevenly distributed, will exhibit exactly this behavior. Gravity is not a feature added to the universe. Gravity is what it means for a universe to exist at all.

\vspace{3em}

\begin{center}
*\quad *\quad *
\end{center}

\vspace{1em}

\noindent \textit{This essay presents the conceptual core of the processing interpretation of gravity. The mathematical formalism, derivations, and experimental predictions are developed in the technical papers of Recognition Science. The key insight---that gravity emerges from finite processing capacity distributed according to mass---requires no equations to understand, only careful thought about what it means for reality to maintain itself from moment to moment.}

\end{document}
