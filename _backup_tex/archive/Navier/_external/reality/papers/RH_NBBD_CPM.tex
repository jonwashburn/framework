% RH_NBBD_CPM.tex
% Riemann Hypothesis via NB/BD + Recognition Science (CPM + Eight‑Phase Neutrality)
% Version 0.2

\documentclass[11pt]{article}
\usepackage{amsmath,amssymb,amsthm,mathtools}
\usepackage{bm}
\usepackage{enumitem}
\usepackage[hidelinks]{hyperref}
\usepackage{geometry}
\geometry{margin=1in}
\usepackage{microtype}

\newcommand{\C}{\mathbb{C}}
\newcommand{\R}{\mathbb{R}}
\newcommand{\N}{\mathbb{N}}
\newcommand{\zetaR}{\zeta}
\newcommand{\xiR}{\xi}
\newcommand{\ip}[2]{\left\langle #1,\,#2 \right\rangle}
\newcommand{\norm}[1]{\left\lVert #1 \right\rVert}
\newcommand{\abs}[1]{\left\lvert #1 \right\rvert}

\title{Riemann Hypothesis via Nyman--Beurling/B\'aez--Duarte and Recognition Science:\\
CPM Energy, Group Averaging, and a No Physical Gap Principle}
\author{Proof Track v0.2}
\date{\today}

\begin{document}
\maketitle

\begin{abstract}
We record a concrete proof track for the Riemann Hypothesis (RH) that marries the Nyman--Beurling/B\'aez--Duarte (NB/BD) criterion with Recognition Science (RS). The primary backbone works directly in $L^2(0,1)$ (avoiding Hardy--Dirichlet boundary machinery initially). The route passes through: (i) the classical NB/BD subspace $M\subset L^2(0,1)$ with RH $\Leftrightarrow \mathrm{dist}(1,M)=0$, (ii) an RS embedding and an order--8 unitary $L$ (``tick''), (iii) a convex CPM energy $E$ that is $L$--invariant so that \emph{group averaging} lowers energy and forces minimizers to lie in the $L$--fixed space, and (iv) a variational functional $F(f)=E(f)+\lambda\|P_M^\perp f\|_H^2$ with local strict convexity near $1$ yielding a \emph{No Physical Gap} lemma: local minimizers in the physical manifold $M_{\rm phys}$ lie in the NB/BD subspace $M$. Together with NB/BD equivalence, this forces $1\in M$ and hence RH. We separate ``standard classical'' inputs, RS proof obligations (unitary invariance + averaging), and analytic lemmas to be formalized. Optional operator/flank routes (balance operator, band energy pinch) are kept as nonessential adjuncts.
\end{abstract}

\section*{Global Conventions}
\begin{itemize}[leftmargin=2em]
  \item $\C$ complex numbers, $\R$ reals, $\N=\{1,2,3,\dots\}$.
  \item $s=\sigma+it\in\C$, $\sigma=\Re(s)$, $t=\Im(s)$.
  \item $\zetaR(s)$ Riemann zeta; $\xiR(s)$ completed zeta.
  \item ``Classical standard'': importable from mathlib or standard analytic NT.
  \item RS background (pattern layer, $\texttt{gradeNat}$, $\texttt{tick}$, CPM $J$) per existing RS foundations.
\end{itemize}

\section{NB/BD Backbone on $L^2(0,1)$ (classical)}
\textbf{Working space.} For the primary track, let $H\coloneqq L^2(0,1)$ with the standard inner product. (A Hardy--Dirichlet presentation and boundary equivalence can be added later as an equivalent viewpoint, but is \emph{not} needed for the proof chain here.)

\medskip
\noindent\textbf{NB/BD subspace.} Let $M\subset H$ be the classical Nyman--Beurling/B\'aez--Duarte subspace (defined via standard NB/BD generators on $(0,1)$; details omitted here as we import the equivalence). Define the defect $d\coloneqq \mathrm{dist}_H(1,M)$.

\medskip
\noindent\textbf{NB/BD equivalence.} RH $\Longleftrightarrow d=0 \Longleftrightarrow 1\in \overline{M}$ (import as a classical theorem).

\section{Recognition Science Embedding}
\textbf{Pattern layer.} $\mathcal P$ a countable free monoid of patterns with multiplicative $\texttt{gradeNat}:\mathcal P\to \N$. Irreducibles $\leftrightarrow$ primes (RS structure).

\medskip
\noindent\textbf{Hilbert identification.} $\ell^2(\mathcal P)\to \ell^2(\N)$ by grouping equal grades and normalizing by multiplicities $m_n$; optionally map further into the working space $H=L^2(0,1)$ via a standard identification. This provides an RS context but is not logically required for the NB/BD backbone.

\medskip
\noindent\textbf{Tick, unitary $L$, and finite-group averaging.} A unitary $\texttt{tick}$ on $\ell^2(\mathcal P)$ with $\texttt{tick}^8=\mathrm{Id}$ and $\texttt{gradeNat}\circ \texttt{tick}=\texttt{gradeNat}$ induces (via the identification) a unitary $L$ on $H$ with $L^8=\mathrm{Id}$. Let $\omega_k=e^{2\pi i k/8}$ and define projections
\[
  P_k \coloneqq \tfrac{1}{8}\sum_{m=0}^{7}\omega_k^{-m} L^m,\qquad H_k\coloneqq \mathrm{ran}(P_k),\qquad H=\bigoplus_{k=0}^7 H_k.
\]

\medskip
\noindent\textbf{Group averaging lemma (RS proof obligation).} Assume or prove $E(Lf)=E(f)$ for all $f\in H$ (see Section~\ref{sec:cpm}). Since $E$ is convex and $L$--invariant, the finite-group average $\bar f\coloneqq \tfrac18\sum_{m=0}^7 L^m f$ satisfies $E(\bar f)\le E(f)$. Hence any $E$--minimizer is $L$--fixed (no need to postulate per-phase equality).

\section{CPM Energy on the Critical Line}
\textbf{Scalar cost (CPM).} $J:(0,\infty)\to[0,\infty)$, $J(x)=\tfrac12(x+1/x)-1$, satisfies $J(1)=J'(1)=0$, $J''(1)=1$; hence local convexity: $\exists\,\delta,c_0>0$ with $J(x)\ge \tfrac{c_0}{2}(x-1)^2$ for $\abs{x-1}\le \delta$.

\medskip
\noindent\textbf{Energy.}\label{sec:cpm} Let $H$ denote the working Hilbert space (here $H=L^2(0,1)$). Define
\[
  E(f)\coloneqq \int J\!\big(\abs{f(x)}\big)\, w(x)\,dx\in [0,\infty],
\]
for an admissible weight $w$ comparable to $1$ on $(0,1)$. (In a Hardy--Dirichlet variant, one instead uses boundary values on the critical line; this is optional and not needed for the present track.)

\medskip
\noindent\textbf{Local coercivity (analytic).} For each $\varepsilon>0$ there exists a neighborhood $U$ of $1$ in $H$, and $c(\varepsilon)>0$, such that $f\in U$ implies $\abs{f-1}\le \varepsilon$ a.e.\ and $E(f)\ge c(\varepsilon)\,\norm{f-1}_H^2$.

\medskip
\noindent\textbf{Averaging consequence.} If $E(Lf)=E(f)$ for unitary $L$ with $L^8=\mathrm{Id}$, convexity implies $E(\bar f)\le E(f)$ for $\bar f=\tfrac18\sum_{m=0}^7 L^m f$. Any minimizer is $L$--fixed.

\section{Physical Cone and Law of Existence (minimal, provable)}
Define $C\subset H$ to be the \emph{closed, convex, $L$--invariant cone generated by NB/BD approximants with finite $E$}. In particular, the following hold:
\begin{enumerate}[leftmargin=2em, label=(LOE\arabic*)]
  \item $1\in C$;
  \item NB/BD generators with finite $E$ and their finite linear combinations belong to $C$;
  \item $C$ is the $H$--closure of such finite constructions and is $L$--invariant.
\end{enumerate}

\section{Mass Constraint and Variational Functional}
Let $m(f)\coloneqq \Re\,\ip{f}{1}_H$ and define $M_{\rm phys}\coloneqq \{f\in C\,:\, m(f)=m(1)\}$. Let $P_M$ be the orthogonal projection onto $M$ and $P_M^\perp\coloneqq \mathrm{Id}-P_M$.

\medskip
\noindent\textbf{Functional.} For small $\lambda>0$ define
\[
  F(f)\coloneqq E(f)+\lambda\,\norm{P_M^\perp f}_H^2,\qquad f\in C.
\]

\medskip
\noindent\textbf{Local strict convexity (analytic).} There exists a neighborhood $U$ of $1$ in $C$ and $\lambda_0>0$ such that for $0<\lambda\le \lambda_0$, $F$ is finite, Fr\'echet differentiable on $U$, and strictly convex on $U\cap C$.

\section{No Physical Gap Principle}
\textbf{Lemma (No Physical Gap).} Assume: (A1) NB/BD equivalence; (A2) $E$ is $L$--invariant and convex so group averaging lowers $E$ and minimizers are $L$--fixed; (A3) minimal LOE (the above definition of $C$); (A4) local strict convexity of $F$. If $f^\star\in M_{\rm phys}$ is a local minimizer of $F$ in $M_{\rm phys}$, write $f^\star=m+h$ with $m\in M$, $h\in M^\perp$. Then $h=0$. Hence any local $F$--minimizer in $M_{\rm phys}$ lies in $M$.

\medskip
\noindent\textit{Idea.} Euler--Lagrange at $f^\star$ under admissible variations in $M_{\rm phys}\cap C$ gives $E'(f^\star)[\delta f]+2\lambda\,\Re\,\ip{h}{\delta h}=0$. Choosing variations tangent to $M$ yields stationarity for $E$ along $M\cap M_{\rm phys}$. Strict convexity then forces $h=0$, otherwise moving along $-h$ decreases $F$ while staying in $M_{\rm phys}\cap C$.

\section{Final Theorem (RH from RS + Analysis)}
\textbf{Theorem.} Under (A1)--(A4), RH holds.

\medskip
\noindent\textit{Proof sketch.} If RH fails, NB/BD gives $d=\text{dist}_H(1,M)>0$ and $r\coloneqq 1-P_M1\in M^\perp$, $\norm{r}=d$. For small $\lambda$, local strict convexity gives a unique local minimizer $f^\star$ near $1$ in $U\cap M_{\rm phys}$. By No Physical Gap, $f^\star\in M$. Strict convexity on $U\cap M$ forces $f^\star=1$. Hence $1\in M$ and $d=0$, contradiction. Therefore RH.

\section*{Lean Implementation Checklist}
\begin{enumerate}[leftmargin=2em]
  \item NB/BD on $L^2(0,1)$: define $M$ and import (or state) RH $\Leftrightarrow \mathrm{dist}(1,M)=0$.
  \item RS $\to L^2(0,1)$ bridge (optional): $\ell^2(\mathcal P)\to \ell^2(\N)$ grade isometry; standard map into $L^2(0,1)$.
  \item Tick/L and averaging: define $L$ with $L^8=\mathrm{Id}$; prove $E(Lf)=E(f)$; deduce averaging lowers $E$ and minimizers are $L$--fixed.
  \item CPM energy: prove scalar $J$ local convexity; define $E$ on $L^2(0,1)$; prove local coercivity near $1$.
  \item Cone and manifold: define $C$ as closed, convex, $L$--invariant cone generated by NB/BD functions with finite $E$; define $m$, $M_{\rm phys}$.
  \item Variational step: define $F$; prove local strict convexity near $1$; prove No Physical Gap.
  \item Conclude RH from NB/BD and No Physical Gap.
\end{enumerate}

\section*{Appendix A: Lean Modules to Import (from RS repo)}
The new repository should import (or depend on) the following modules.

\medskip
\noindent\textbf{Required for this track:}
\begin{itemize}[leftmargin=2em]
  \item \texttt{IndisputableMonolith.Cost}
  \item \texttt{IndisputableMonolith.Patterns}
  \item \texttt{IndisputableMonolith.RH.NymanBeurling}
  \item \texttt{IndisputableMonolith.RH.NB\_CPM\_Bridge}
  \item \texttt{IndisputableMonolith.RH.BaezDuarte}
  \item \texttt{IndisputableMonolith.RH.RS.Core}
  \item \texttt{IndisputableMonolith.RH.RS.Spec}
  \item \texttt{IndisputableMonolith.RH.RS.Framework}
  \item \texttt{IndisputableMonolith.RH.RS.UDExplicit}
  \item \texttt{IndisputableMonolith.RH.RS.Anchors}
  \item \texttt{IndisputableMonolith.RH.RS.Bands}
  \item \texttt{IndisputableMonolith.RH.RS.ClosureShim}
  \item \texttt{IndisputableMonolith.RH.RS.Witness.Core}
  \item \texttt{IndisputableMonolith.RH.RS.Universe}
\end{itemize}

\noindent\textbf{Optional flank (not needed for NB/BD+CPM chain):}
\begin{itemize}[leftmargin=2em]
  \item \texttt{IndisputableMonolith.RH.Primes.PatternCorrespondence}
  \item \texttt{IndisputableMonolith.RH.BalanceOperator}
  \item \texttt{IndisputableMonolith.RH.BandEnergyPinch}
  \item \texttt{IndisputableMonolith.Constants}
  \item \texttt{IndisputableMonolith.Verification}
  \item \texttt{IndisputableMonolith.Measurement.C2ABridge}
  \item \texttt{IndisputableMonolith.Measurement.TwoBranchGeodesic}
  \item \texttt{IndisputableMonolith.Quantum}
\end{itemize}
These cover cost/$J$, core RS framework and witnesses, and the NB/BD scaffolding. The optional items include operator/flank modules that are interesting but not required for the present proof chain.

\section*{Appendix B: External Reference}
Prior boundary product--certificate work and related notes are available in the repository \emph{riemann--side}:
\begin{center}
\url{https://github.com/jonwashburn/riemann-side}
\end{center}
This path ultimately did not close RH but provides reusable infrastructure and context.

\end{document}

