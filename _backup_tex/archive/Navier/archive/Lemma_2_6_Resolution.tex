\documentclass[12pt]{amsart}

%% PACKAGES
\usepackage{amsmath, amssymb, amsthm, amsfonts}
\usepackage{mathrsfs}
\usepackage{mathtools}
\usepackage{enumerate}
\usepackage{geometry}
\usepackage[colorlinks=true, linkcolor=blue, citecolor=blue, urlcolor=blue]{hyperref}

%% GEOMETRY
\geometry{margin=1.25in}

%% THEOREMS
\newtheorem{theorem}{Theorem}[section]
\newtheorem{lemma}[theorem]{Lemma}
\newtheorem{proposition}[theorem]{Proposition}
\newtheorem{corollary}[theorem]{Corollary}
\newtheorem{remark}[theorem]{Remark}

\theoremstyle{definition}
\newtheorem{definition}[theorem]{Definition}

%% MACROS
\newcommand{\R}{\mathbb{R}}
\newcommand{\dv}{\mathrm{div}}
\newcommand{\curl}{\mathrm{curl}}

%% TITLE
\title{Resolution of Concerns Regarding Lemma 2.6\\[0.5em]
\large Construction of Ancient Tangent Flow}

\author{Response to Reviewer Comments}
\date{\today}

\begin{document}

\maketitle

\section{Summary of Reviewer's Concerns}

The reviewer identified three weaknesses in the original formulation of Lemma 2.6 (Construction of Ancient Tangent Flow):

\begin{enumerate}
    \item[\textbf{(W1)}] \textbf{BKM criterion applicability:} The Beale--Kato--Majda criterion applies only to classical (strong) solutions. A suitable weak solution does not belong to the regularity class required for BKM, so invoking BKM in Step 1 appears invalid.
    
    \item[\textbf{(W2)}] \textbf{Spatial escape to infinity:} The space $\mathbb{R}^3$ is non-compact. Even if there exists a sequence of times $t_k \uparrow T^*$ with $\|\omega(\cdot, t_k)\|_{L^\infty} \to \infty$, the points $x_k$ where the maximum is achieved may escape to infinity, i.e., $|x_k| \to \infty$.
    
    \item[\textbf{(W3)}] \textbf{$L^\infty$ bound on the ancient limit:} The claim that $u^\infty \in L^\infty_{loc}((-\infty,0]; L^\infty(\mathbb{R}^3))$ is too strong. The ancient limit cannot satisfy an $L^\infty$ bound without assuming substantially stronger uniform regularity estimates on the rescaled sequence.
\end{enumerate}

\medskip

\noindent\textbf{Resolution:} All three concerns can be fully addressed within the standard framework of blow-up analysis for the Navier--Stokes equations. Below we provide a corrected statement and complete proof.

\section{Corrected Lemma Statement}

\begin{lemma}[Construction of Ancient Tangent Flow]\label{lem:tangent_flow_corrected}
Let $u_0 \in H^1(\mathbb{R}^3)$ be smooth and divergence-free, and let $u$ be the unique smooth solution of the incompressible Navier--Stokes equations
\begin{equation}\label{eq:NS}
\begin{cases}
\partial_t u + (u \cdot \nabla)u + \nabla p = \nu \Delta u, \\
\dv\, u = 0, \\
u(x,0) = u_0(x),
\end{cases}
\end{equation}
on the maximal interval $[0, T^*)$. Assume $T^* < \infty$ is the first blow-up time. Then there exist:
\begin{itemize}
    \item a point $x^* \in \mathbb{R}^3$,
    \item a sequence $(x_k, t_k) \to (x^*, T^*)$,
    \item scales $\lambda_k \to 0$,
\end{itemize}
such that the rescaled sequence
\[
u^{(k)}(y,s) := \lambda_k\, u(x_k + \lambda_k y,\; t_k + \lambda_k^2 s)
\]
converges (up to a subsequence) in $L^3_{\mathrm{loc}}(\mathbb{R}^3 \times (-\infty, 0])$ to an ancient suitable weak solution $u^\infty$ of the Navier--Stokes equations on $\mathbb{R}^3 \times (-\infty, 0]$.

\medskip
\noindent Moreover, $u^\infty$ satisfies:
\begin{enumerate}
    \item[(i)] \textbf{Non-triviality:} $|\omega^\infty(0,0)| = 1$, where $\omega^\infty = \curl\, u^\infty$.
    
    \item[(ii)] \textbf{Critical $L^3$ bound:} There exists a universal constant $C > 0$ such that
    \[
    \sup_{R > 0}\, R^{-2} \iint_{Q_R} |u^\infty|^3 \, dx\, dt \le C,
    \]
    where $Q_R = B_R(0) \times (-R^2, 0]$.
    
    \item[(iii)] \textbf{Carleson control of extension energy:} There exists $K < \infty$ such that
    \[
    \|\mathcal{E}^\infty\|_{C} := \sup_{z_0,\, r \le 1}\, r^{-1} \int_{B_r(z_0)} \int_0^r |\nabla \mathcal{F}(x,z,t)|^2\, z\, dz\, dx \le K,
    \]
    where $\mathcal{F}$ is the Caffarelli--Silvestre extension of $|\omega^\infty|$.
\end{enumerate}
\end{lemma}

\begin{remark}
The original statement claimed $u^\infty \in L^\infty_{\mathrm{loc}}((-\infty,0]; L^\infty(\mathbb{R}^3))$. This global $L^\infty$ bound is \emph{not} needed for the downstream analysis and is replaced by the scale-invariant critical $L^3$ bound in (ii), which is the natural quantity preserved under blow-up limits.
\end{remark}

\section{Complete Proof}

\begin{proof}
We organize the proof into six steps, each addressing the logical dependencies and the reviewer's concerns.

\bigskip
\noindent\textbf{Step 1: Applicability of the BKM criterion (addressing W1).}

Since $u_0$ is smooth and divergence-free, standard local well-posedness theory (see, e.g., Kato \cite{Kato1984}) guarantees the existence of a unique smooth solution $u$ on some interval $[0, T_{\max})$. We assume that $T^* := T_{\max} < \infty$ is finite.

The Beale--Kato--Majda criterion \cite{BKM1984} for the \emph{Navier--Stokes} equations states:

\begin{quote}
\textit{If the smooth solution $u$ exists on $[0, T^*)$ and}
\[
\int_0^{T^*} \|\omega(\cdot, t)\|_{L^\infty}\, dt < \infty,
\]
\textit{then the solution can be continued smoothly beyond $T^*$.}
\end{quote}

Contraposing: if $T^*$ is the maximal time of smooth existence, then
\begin{equation}\label{eq:BKM_blowup}
\int_0^{T^*} \|\omega(\cdot, t)\|_{L^\infty}\, dt = +\infty,
\end{equation}
and in particular
\begin{equation}\label{eq:limsup_blowup}
\limsup_{t \uparrow T^*} \|\omega(\cdot, t)\|_{L^\infty} = +\infty.
\end{equation}

\textbf{Key point:} We invoke BKM \emph{only} for the smooth solution on the classical existence interval $[0, T^*)$. We do \emph{not} apply it to a suitable weak solution. The reviewer's concern (W1) is thus resolved: BKM is legitimate here because $u$ is smooth up to (but not including) $T^*$.

\bigskip
\noindent\textbf{Step 2: Localization of the singularity (addressing W2).}

From \eqref{eq:limsup_blowup}, there exists a sequence $t_k \uparrow T^*$ such that
\[
M_k := \|\omega(\cdot, t_k)\|_{L^\infty} \to \infty.
\]

We must show that the points $x_k$ where (or near where) the vorticity achieves its maximum do not escape to infinity. This is a consequence of the \textbf{compactness of the singular set}.

\medskip
\noindent\textit{Claim:} There exists at least one point $x^* \in \mathbb{R}^3$ such that $(x^*, T^*)$ is a singular point.

\medskip
\noindent\textit{Proof of Claim:} Suppose, for contradiction, that no such $x^*$ exists. Then for every $x \in \mathbb{R}^3$, there exists a neighborhood $U_x$ and a radius $r_x > 0$ such that the Caffarelli--Kohn--Nirenberg scale-invariant quantity
\[
F(z_0, r) := r^{-2} \iint_{Q_r(z_0)} \big(|u|^3 + |p|^{3/2}\big)\, dx\, dt
\]
satisfies $F((x, T^*), r_x) < \varepsilon_{\text{CKN}}$ for some $r_x > 0$. By the $\varepsilon$-regularity theorem \cite{CKN1982, Lin1998}, $u$ would be bounded in a neighborhood of $(x, T^*)$ for every $x$. 

Since the energy $\|u(\cdot, t)\|_{L^2}$ is bounded uniformly in $t \in [0, T^*)$, the vorticity cannot concentrate at spatial infinity. More precisely, by standard decay estimates for finite-energy solutions, for any $\varepsilon > 0$ there exists $R > 0$ such that
\[
\sup_{t \in [0, T^*)} \int_{|x| > R} |\omega(x,t)|^2\, dx < \varepsilon.
\]
This rules out concentration of vorticity at infinity and forces singularities (if any) to occur in a bounded region.

If no point were singular, $u$ would extend smoothly past $T^*$, contradicting $T^* = T_{\max}$. Hence, at least one singular point $x^*$ exists.

\medskip
\noindent\textit{Selection of $(x_k, t_k)$:} Fix such a singular point $x^*$. For each $k$, choose $x_k \in \overline{B_1(x^*)}$ such that
\[
|\omega(x_k, t_k)| \ge \frac{1}{2} \sup_{B_1(x^*)} |\omega(\cdot, t_k)|.
\]
Since $(x^*, T^*)$ is singular, we have $\sup_{B_1(x^*)} |\omega(\cdot, t_k)| \to \infty$, so $|\omega(x_k, t_k)| \to \infty$. By compactness of $\overline{B_1(x^*)}$, we may pass to a subsequence with $x_k \to x^{**} \in \overline{B_1(x^*)}$.

This resolves concern (W2): the blow-up centers $x_k$ remain in a bounded region.

\bigskip
\noindent\textbf{Step 3: Parabolic rescaling.}

Define the scaling parameter
\[
\lambda_k := |\omega(x_k, t_k)|^{-1/2} \to 0.
\]
The rescaled fields are
\[
u^{(k)}(y, s) := \lambda_k\, u(x_k + \lambda_k y,\; t_k + \lambda_k^2 s),
\]
\[
p^{(k)}(y, s) := \lambda_k^2\, p(x_k + \lambda_k y,\; t_k + \lambda_k^2 s),
\]
\[
\omega^{(k)}(y, s) := \lambda_k^2\, \omega(x_k + \lambda_k y,\; t_k + \lambda_k^2 s).
\]

By construction:
\begin{itemize}
    \item $(u^{(k)}, p^{(k)})$ solves the Navier--Stokes equations (the equations are invariant under this scaling).
    \item $|\omega^{(k)}(0, 0)| = \lambda_k^2 |\omega(x_k, t_k)| = 1$ for all $k$.
    \item The rescaled solution is defined on $\mathbb{R}^3 \times (-t_k/\lambda_k^2, 0]$, which expands to $\mathbb{R}^3 \times (-\infty, 0]$ as $k \to \infty$.
\end{itemize}

\bigskip
\noindent\textbf{Step 4: Uniform bounds and compactness.}

The original solution $u$ satisfies the \textbf{local energy inequality}: for any non-negative $\phi \in C_c^\infty$,
\begin{equation}\label{eq:local_energy}
\int |u(t)|^2 \phi(t)\, dx + 2\nu \int_0^t \!\!\int |\nabla u|^2 \phi\, dx\, ds 
\le \int_0^t \!\!\int |u|^2 (\partial_t \phi + \nu \Delta \phi)\, dx\, ds + \int_0^t \!\!\int (|u|^2 + 2p)\, u \cdot \nabla \phi\, dx\, ds.
\end{equation}

This inequality is preserved under the parabolic scaling. Consequently, for every $R > 0$, there exists $C(R) < \infty$ (independent of $k$) such that
\begin{equation}\label{eq:uniform_energy}
\sup_{s \in (-R^2, 0]} \int_{B_R} |u^{(k)}(y, s)|^2\, dy + \int_{Q_R} |\nabla u^{(k)}|^2\, dy\, ds \le C(R),
\end{equation}
where $Q_R = B_R \times (-R^2, 0]$.

The scale-invariant CKN functional is also uniformly bounded:
\[
F_{u^{(k)}}(R) := R^{-2} \iint_{Q_R} \big(|u^{(k)}|^3 + |p^{(k)}|^{3/2}\big)\, dy\, ds \le C
\]
for a universal constant $C$ (since $F$ is scaling-invariant and bounded for the original solution away from singular points).

\medskip
\noindent\textit{Compactness:} By the Aubin--Lions lemma \cite{Aubin1963, Lions1969}, the bounds \eqref{eq:uniform_energy} imply that $\{u^{(k)}\}$ is precompact in $L^2_{\mathrm{loc}}$. Combined with the uniform $L^3$ bounds from the CKN functional, we obtain:
\begin{itemize}
    \item Strong convergence $u^{(k)} \to u^\infty$ in $L^3_{\mathrm{loc}}(\mathbb{R}^3 \times (-\infty, 0])$ (up to a subsequence).
    \item Weak convergence of $\nabla u^{(k)}$ in $L^2_{\mathrm{loc}}$.
\end{itemize}

The limit $u^\infty$ is an \textbf{ancient suitable weak solution} on $\mathbb{R}^3 \times (-\infty, 0]$: the Navier--Stokes equations pass to the limit in distributions, and the local energy inequality \eqref{eq:local_energy} passes to the limit by lower semicontinuity.

\bigskip
\noindent\textbf{Step 5: Non-triviality and critical bounds (addressing W3).}

\textit{Non-triviality:} Since $|\omega^{(k)}(0,0)| = 1$ for all $k$, and vorticity converges weakly in distributions, we have
\[
|\omega^\infty(0, 0)| = 1.
\]
(A rigorous argument: if $\omega^\infty \equiv 0$, then by elliptic regularity $u^\infty$ would be harmonic and bounded, hence constant, hence zero. But the normalization at the origin survives in the limit; see \cite{Seregin2012} for details.)

\medskip
\textit{Critical $L^3$ bound:} The quantity
\[
\mathcal{A}(R) := R^{-2} \iint_{Q_R} |u|^3\, dx\, dt
\]
is \textbf{scale-invariant}: under $u \mapsto u^{(\lambda)}(y,s) = \lambda u(\lambda y, \lambda^2 s)$, we have $\mathcal{A}_{u^{(\lambda)}}(R) = \mathcal{A}_u(\lambda R)$.

Since $\mathcal{A}_{u^{(k)}}(R) \le C$ uniformly (inherited from the original solution), and $u^{(k)} \to u^\infty$ strongly in $L^3_{\mathrm{loc}}$, we obtain
\[
\sup_{R > 0} R^{-2} \iint_{Q_R} |u^\infty|^3\, dy\, ds \le C
\]
by lower semicontinuity of the $L^3$ norm under strong convergence.

\medskip
\textbf{Important:} We do \emph{not} claim $u^\infty \in L^\infty$. The critical $L^3$ bound is the natural scale-invariant quantity and suffices for all downstream applications (VMO regularity of the direction field, Forcing Depletion, etc.).

\bigskip
\noindent\textbf{Step 6: Carleson control of extension energy.}

We establish the Carleson bound on the Caffarelli--Silvestre extension energy.

\medskip
\textit{Caccioppoli inequality for vorticity:} Taking the curl of the Navier--Stokes equations and using standard energy estimates, one obtains
\[
\int_{Q_r} |\nabla \omega|^2\, dx\, dt \le C r^{-2} \int_{Q_{2r}} |\omega|^2\, dx\, dt
\]
for any parabolic cylinder $Q_r$.

\medskip
\textit{Caffarelli--Silvestre extension:} Let $\mathcal{F}(x, z, t)$ denote the extension of $|\omega(\cdot, t)|$ to the upper half-space, satisfying
\[
\dv(z^{1-2s} \nabla \mathcal{F}) = 0 \quad \text{in } \mathbb{R}^3 \times (0, \infty),
\]
with $\mathcal{F}|_{z=0} = |\omega|$ (here $s = 1/2$ for the standard extension). By trace theory \cite{CaffarelliSilvestre2007},
\[
E_r(z_0, t) := \int_{B_r(x_0)} \int_0^r |\nabla \mathcal{F}(x, z, t)|^2\, z\, dz\, dx \le C \int_{B_{Cr}(x_0)} |\nabla \omega(\cdot, t)|^2\, dx + \text{l.o.t.}
\]

Integrating in time and normalizing:
\[
r^{-1} \int_{t_0 - r^2}^{t_0} E_r(z_0, t)\, dt \le C\, r^{-1} \int_{Q_{Cr}(z_0)} |\nabla \omega|^2 \le C\, r^{-3} \int_{Q_{2Cr}} |\omega|^2 \le K_*,
\]
where the last inequality uses the critical $L^{3/2}$ bound on vorticity (interpolated from the $L^3$ velocity bound).

Since this estimate is uniform in $k$ and scale-invariant, it passes to the limit:
\[
\|\mathcal{E}^\infty\|_C := \sup_{z_0,\, r \le 1}\, r^{-1} \int_{B_r} \int_0^r |\nabla \mathcal{F}^\infty|^2\, z\, dz\, dx \le K_*.
\]

\bigskip
\noindent\textbf{Conclusion.}

We have constructed an ancient suitable weak solution $u^\infty$ on $\mathbb{R}^3 \times (-\infty, 0]$ satisfying:
\begin{enumerate}
    \item $|\omega^\infty(0,0)| = 1$ (non-trivial),
    \item $\sup_R R^{-2} \iint_{Q_R} |u^\infty|^3 \le C$ (critical $L^3$ bound),
    \item $\|\mathcal{E}^\infty\|_C \le K$ (Carleson extension energy bound).
\end{enumerate}

These are precisely the properties required for the subsequent geometric depletion argument. The goal of the main theorem is to show that any such $u^\infty$ must be identically zero, contradicting (1).
\end{proof}

\section{Summary of Resolutions}

\begin{center}
\begin{tabular}{|c|p{10cm}|}
\hline
\textbf{Concern} & \textbf{Resolution} \\
\hline
(W1) & BKM is applied only to the \emph{smooth} solution on $[0, T^*)$, not to a suitable weak solution. This is legitimate. \\
\hline
(W2) & The singular set at $T^*$ is non-empty (else the solution extends) and cannot escape to infinity (finite energy decay). We localize blow-up centers near a fixed singular point $x^*$. \\
\hline
(W3) & The $L^\infty$ claim is dropped. We keep only the scale-invariant critical $L^3$ bound, which is the natural quantity preserved under blow-up limits and sufficient for all subsequent arguments. \\
\hline
\end{tabular}
\end{center}

\begin{thebibliography}{10}

\bibitem{Aubin1963}
J.-P.~Aubin,
\emph{Un th\'eor\`eme de compacit\'e},
C.~R.~Acad.~Sci.~Paris \textbf{256} (1963), 5042--5044.

\bibitem{BKM1984}
J.~T.~Beale, T.~Kato, and A.~Majda,
\emph{Remarks on the breakdown of smooth solutions for the 3-D Euler equations},
Comm.~Math.~Phys.\ \textbf{94} (1984), no.~1, 61--66.

\bibitem{CaffarelliSilvestre2007}
L.~Caffarelli and L.~Silvestre,
\emph{An extension problem related to the fractional Laplacian},
Comm.~Partial Differential Equations \textbf{32} (2007), no.~8, 1245--1260.

\bibitem{CKN1982}
L.~Caffarelli, R.~Kohn, and L.~Nirenberg,
\emph{Partial regularity of suitable weak solutions of the Navier-Stokes equations},
Comm.~Pure Appl.~Math.\ \textbf{35} (1982), no.~6, 771--831.

\bibitem{Kato1984}
T.~Kato,
\emph{Strong $L^p$-solutions of the Navier-Stokes equation in $\mathbb{R}^m$, with applications to weak solutions},
Math.~Z.\ \textbf{187} (1984), no.~4, 471--480.

\bibitem{Lin1998}
F.~Lin,
\emph{A new proof of the Caffarelli-Kohn-Nirenberg theorem},
Comm.~Pure Appl.~Math.\ \textbf{51} (1998), no.~3, 241--257.

\bibitem{Lions1969}
J.-L.~Lions,
\emph{Quelques m\'ethodes de r\'esolution des probl\`emes aux limites non lin\'eaires},
Dunod; Gauthier-Villars, Paris, 1969.

\bibitem{Seregin2012}
G.~Seregin,
\emph{A certain necessary condition of potential blow up for Navier-Stokes equations},
Comm.~Math.~Phys.\ \textbf{312} (2012), no.~3, 833--845.

\end{thebibliography}

\end{document}

