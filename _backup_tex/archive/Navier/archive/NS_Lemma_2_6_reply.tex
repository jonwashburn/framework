\documentclass[11pt]{article}

\usepackage[margin=1in]{geometry}
\usepackage{amsmath,amssymb}
\usepackage[colorlinks=true,linkcolor=blue,citecolor=blue,urlcolor=blue]{hyperref}

\setlength{\parindent}{0pt}
\setlength{\parskip}{0.5em}

\begin{document}

\begin{center}
{\Large Response on NS draft and Lemma 2.6 (Ancient Tangent Flow)}\\
\vspace{0.25em}
{\small \today}
\end{center}

\textbf{To:} Milan Zlatanovi\'c\\
\textbf{Cc:} Prof.\ Elshad Allahyarov\\
\textbf{Re:} ``Geometric Depletion Mechanisms in the 3D Incompressible Navier--Stokes Equations'' (Overleaf project: \href{https://www.overleaf.com/3989828692dqrcmngchgvs#eba565}{link})

\vspace{0.5em}

Dear Milan,

Thank you for the careful pass and for restructuring Lemma~2.6. I agree with your assessment: the \emph{idea} of Lemma~2.6 is right, but the original presentation was too compressed for a referee-level check. Your decomposition into four lemmas
\emph{(singular point existence; normalization; domain exhaustion; ancient limit)}
is exactly the right way to make the logic verifiable.

\textbf{Layperson subtext (what your comments are really asking):}
the ``tangent flow'' step is the bridge from the classical smooth solution (where we can do pointwise/vorticity arguments) to the weak/limit object (where we can only pass properties that are stable under compactness). So the question is not whether one can \emph{define} a rescaling, but whether one can (i) choose centers/scales that do not drift to infinity, (ii) obtain enough uniform bounds to extract a subsequence, and (iii) guarantee the limit is \emph{nontrivial} in a way that survives convergence. These are exactly the three places where blow-up arguments can silently fail if not spelled out.

\textbf{My answers / how the model closes the proof:}

\begin{enumerate}
  \item \textbf{Your Lemma ``Existence of a singular point'' is the correct anchor.}
  Once we know there exists at least one CKN-singular point $(x^*,T^*)$, we should \emph{anchor the blow-up sequence to that point}. This is what prevents the ``escape to infinity'' issue: we choose all centers $x_k$ inside a fixed ball around $x^*$ and choose $t_k \uparrow T^*$.

  \item \textbf{Normalization: prefer a normalization that passes to the limit robustly.}
  There are two common normalizations:
  \begin{itemize}
    \item \emph{Vorticity normalization} $|\omega^{(k)}(0,0)|=1$ is intuitive, but to pass it to the limit one needs enough compactness to control $\omega^{(k)}$ (or an indirect argument that prevents $u^\infty\equiv 0$).
    \item \emph{CKN-functional normalization} is more robust: because $(x^*,T^*)$ is singular, there exists a sequence $r_k\downarrow 0$ such that
    \[
      r_k^{-2}\iint_{Q_{r_k}(x^*,T^*)}\bigl(|u|^3+|p|^{3/2}\bigr)\,dx\,dt \ge \varepsilon_{\rm CKN}.
    \]
    If we set $\lambda_k:=r_k$ and center at $(x^*,T^*)$, then the rescaled sequence satisfies the \emph{scale-1 lower bound} on $Q_1$ automatically, and this lower bound survives to the limit by lower semicontinuity. This gives nontriviality of $u^\infty$ in a way that is hard to lose.
  \end{itemize}
  Practically: your current vorticity-based normalization lemma is fine, but I recommend (for referee-proof clarity) adding one short remark that we \emph{can also} normalize via the CKN functional at the singular point, and that this is the cleanest route to nontriviality of the limit.

  \item \textbf{Domain exhaustion under rescaling: your Lemma is correct.}
  The key estimate is $\lambda_k^{-2}t_k\to\infty$, so $(-R^2,0]\subset(-\lambda_k^{-2}t_k,0]$ for large $k$.

  \item \textbf{Compactness / existence of the ancient limit: what needs to be made explicit.}
  To close the final lemma rigorously, we should state the precise compactness input (a standard theorem for suitable weak solutions):
  from the local energy inequality we get uniform bounds on each cylinder $Q_R$:
  \[
    u^{(k)}\ \text{bounded in}\ L^\infty_tL^2_x(Q_R)\cap L^2_tH^1_x(Q_R),\qquad
    p^{(k)}\ \text{bounded in}\ L^{3/2}(Q_R),
  \]
  and then Aubin--Lions gives strong compactness in $L^q_{\rm loc}$ for $q<3$ (and strong $L^3_{\rm loc}$ away from $s=0$ is standard in blow-up literature, as you wrote).
  This yields a subsequence converging to an ancient suitable weak solution $(u^\infty,p^\infty)$ on $\mathbb{R}^3\times(-\infty,0]$.

  \item \textbf{Nontriviality of the limit: the missing closure step.}
  The cleanest closure is:
  \begin{enumerate}
    \item Choose the rescaling so that a scale-invariant quantity is \emph{uniformly bounded below} (e.g.\ the CKN functional on $Q_1$) along the sequence.
    \item Pass to the limit using lower semicontinuity/strong $L^3_{\rm loc}$ convergence to obtain a positive lower bound for $u^\infty$ on some $Q_r$, which implies $u^\infty\not\equiv 0$.
  \end{enumerate}
  This is exactly what your statement ``$\int_{Q_r}|u^\infty|^3\ge c$'' encodes; the only remaining work is to tie it explicitly to the singular-point lower bound before rescaling.

  \item \textbf{One technical simplification: remove any global $L^\infty$ claim for $u^\infty$.}
  I agree with your earlier concern in the old draft: asserting $u^\infty\in L^\infty$ is not justified from local energy bounds alone and is not needed for the geometric depletion mechanism. The right inherited quantities are local suitable-weak bounds and scale-invariant $L^3$/$L^{3/2}$ controls.
\end{enumerate}

\textbf{Physical-motivation paragraph (suggested insertion):}
I agree it would help the exposition to add a short physics paragraph near the start of the blow-up section: the blow-up procedure is a ``microscope''/renormalization idea---we zoom into regions where vorticity becomes large and ask whether the rescaled flow has a nontrivial limiting profile. The geometric depletion method then shows that any such limiting profile must have essentially constant vorticity direction (hence becomes 2D), and 2D dynamics cannot sustain a 3D blow-up mechanism. This connects the analysis to the physical intuition that vortex stretching requires \emph{misalignment} of vorticity direction, and that strong alignment kills stretching.

\vspace{0.5em}
\textbf{Bottom line:} your four-lemma decomposition is the right structure. To ``close'' Lemma~2.6 cleanly, I suggest anchoring the blow-up at a CKN singular point and (optionally) normalizing via the CKN functional to make nontriviality of the limit immediate by semicontinuity. With those minor clarifications, the lemma becomes fully referee-checkable.

\vspace{0.5em}
Thank you again---please proceed with preparing the Recognition Geometry manuscript while we finalize this lemma chain; I’m happy for you to keep polishing the blue-verified parts in parallel.

\vspace{0.5em}
Sincerely,\\
Jonathan Washburn

\end{document}


