\documentclass[11pt]{amsart}

\usepackage{amsmath,amssymb,amsthm}
\usepackage[margin=1in]{geometry}
\usepackage[colorlinks=true,linkcolor=blue,citecolor=blue,urlcolor=blue]{hyperref}

\newtheorem{theorem}{Theorem}
\newtheorem{lemma}{Lemma}
\newtheorem{remark}{Remark}

\newcommand{\R}{\mathbb{R}}
\newcommand{\C}{\mathbb{C}}
\newcommand{\Z}{\mathbb{Z}}

\begin{document}

\title[Calibration deficit vs.\ K\"ahler-angle defect]{Calibration deficit controls squared K\"ahler-angle defect}
\author{Jonathan Washburn}
\date{\today}
\begin{abstract}
We record a pointwise (and hence global) inequality comparing the Wirtinger/calibration mass deficit
to an $L^2$-type \emph{squared} K\"ahler-angle defect. We also note that no uniform inequality can hold
with the square-root (unsquared) $L^2$-defect, since the mass deficit is generically quadratic in the angle.
\end{abstract}

\maketitle

\section{Setup}
Let $(X^{2n},g,J,\omega)$ be a K\"ahler manifold and fix an integer $1\le p\le n$.
Let
\[
\psi:=\frac{\omega^p}{p!}.
\]
Then $\psi$ is closed and has comass $1$; in particular, for any oriented unit simple $2p$-vector $\xi$,
one has $|\psi(\xi)|\le 1$ (Wirtinger inequality), with equality iff $\xi$ spans a $J$-invariant (complex) $p$-plane
with the calibration orientation.

Let $T$ be an integral $2p$-cycle in $X$ (so $\partial T=0$). Its mass is
\[
\mathrm{Mass}(T)=\int 1\,d|T|.
\]
Since $d\psi=0$ and $\partial T=0$, the pairing $T(\psi)$ depends only on the homology class $[T]$.
Moreover, by comass$=1$,
\[
T(\psi)=\int \psi(\vec T)\,d|T| \le \int 1\,d|T|=\mathrm{Mass}(T).
\]
Thus the \emph{calibration deficit} is
\[
\mathrm{Mass}(T)-T(\psi)=\int \bigl(1-\psi(\vec T)\bigr)\,d|T|.
\]

\section{K\"ahler angles and the squared angle defect}
At $|T|$-a.e.\ point $x$, the approximate tangent plane of $T$ is an oriented $2p$-plane $\xi=\vec T(x)$.
Its (unoriented) K\"ahler angles $\theta_1(\xi),\dots,\theta_p(\xi)\in[0,\pi/2]$ are defined so that
\[
|\psi(\xi)|=\prod_{i=1}^p \cos\theta_i(\xi).
\]
A natural ``$L^2$-type squared'' K\"ahler-angle defect is the pointwise quantity
\[
d(\xi):=\sum_{i=1}^p \sin^2\theta_i(\xi),
\]
and the associated global defect functional is
\[
D(T):=\int d(\vec T)\,d|T|=\int \left(\sum_{i=1}^p \sin^2\theta_i(\vec T)\right)\,d|T|.
\]

\begin{theorem}[Deficit controls squared K\"ahler-angle defect]\label{thm:deficit-controls-angle}
For any integral $2p$-cycle $T$ as above,
\[
\mathrm{Mass}(T)-T(\psi)\ \ge\ c_p\, D(T),
\]
where one may take $c_p=\frac12$ for $p=1$ and $c_p=\frac1p$ for $p\ge 2$ (in particular, $c_p=\frac{1}{2p}$ works for all $p\ge 1$).
\end{theorem}

\begin{proof}
It suffices to prove the pointwise inequality
\[
1-\psi(\xi)\ \ge\ c_p\,\sum_{i=1}^p \sin^2\theta_i(\xi)
\qquad\text{for every oriented unit simple $2p$-vector $\xi$},
\]
and then integrate over $|T|$.

Since $1-\psi(\xi)\ge 1-|\psi(\xi)|$, it is enough to control $1-|\psi(\xi)|$.
Write $x_i:=\cos\theta_i(\xi)\in[0,1]$, so that $|\psi(\xi)|=\prod_{i=1}^p x_i$ and $\sin^2\theta_i=1-x_i^2$.

\smallskip
\noindent\emph{Case $p=1$.}
We need $1-x_1\ge \frac12(1-x_1^2)$ for $x_1\in[0,1]$, which is immediate from
$1-x_1=\frac{1-x_1^2}{1+x_1}\ge \frac12(1-x_1^2)$.

\smallskip
\noindent\emph{Case $p\ge 2$.}
By AM--GM applied to the nonnegative numbers $x_1^2,\dots,x_p^2$,
\[
\prod_{i=1}^p x_i^2 \le \left(\frac{1}{p}\sum_{i=1}^p x_i^2\right)^p.
\]
Taking square-roots gives
\[
\prod_{i=1}^p x_i \le \left(\frac{1}{p}\sum_{i=1}^p x_i^2\right)^{p/2}.
\]
Since $0\le \frac{1}{p}\sum x_i^2\le 1$ and $\frac{p}{2}\ge 1$, we have $a^{p/2}\le a$ for $a\in[0,1]$,
hence
\[
\prod_{i=1}^p x_i \le \frac{1}{p}\sum_{i=1}^p x_i^2.
\]
Therefore,
\[
1-|\psi(\xi)| = 1-\prod_{i=1}^p x_i \ge 1-\frac{1}{p}\sum_{i=1}^p x_i^2
=\frac{1}{p}\sum_{i=1}^p (1-x_i^2)
=\frac{1}{p}\sum_{i=1}^p \sin^2\theta_i(\xi).
\]
This proves the pointwise inequality with $c_p=1/p$ for $p\ge 2$.

\smallskip
Integrating over $|T|$ yields the claimed inequality.
\end{proof}

\begin{lemma}[No uniform inequality for the square-root $L^2$ defect]\label{lem:no-root}
There is no universal constant $c>0$ such that
\[
\mathrm{Mass}(T)-T(\psi)\ \ge\ c\,\sqrt{D(T)}
\]
holds for all integral $2p$-cycles $T$.
\end{lemma}

\begin{proof}
Work on a flat complex torus $X=\C^n/\Z^{2n}$ with its standard K\"ahler form.
Fix a smooth flat $2p$-dimensional subtorus $T_0$ calibrated by $\psi$, and for $\varepsilon>0$ let $T_\varepsilon$
be the same underlying subtorus with its tangent planes rotated so that all K\"ahler angles equal $\varepsilon$.
Then $D(T_\varepsilon)\simeq \mathrm{Mass}(T_0)\cdot p\,\varepsilon^2$ while
$\mathrm{Mass}(T_\varepsilon)-T_\varepsilon(\psi)\simeq \mathrm{Mass}(T_0)\cdot C\,\varepsilon^2$
for small $\varepsilon$.
Thus $(\mathrm{Mass}(T_\varepsilon)-T_\varepsilon(\psi))/\sqrt{D(T_\varepsilon)} \simeq C'\varepsilon\to 0$ as $\varepsilon\to 0$,
contradicting any uniform lower bound by a fixed $c>0$.
\end{proof}

\begin{remark}[On the vanishing set of $D(T)$]
The functional $D(T)=\int\sum_i\sin^2\theta_i\,d|T|$ vanishes iff the approximate tangent planes are $J$-invariant
$|T|$-a.e., but it does \emph{not} encode the calibration orientation.
To conclude that $T$ is a complex analytic cycle, one typically needs a positivity/calibration condition,
e.g.\ $\psi(\vec T)=1$ $|T|$-a.e.\ (or, in complex language, that $T$ is a positive closed $(p,p)$-current with integer multiplicities).
\end{remark}

\end{document}


