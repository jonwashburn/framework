\documentclass[12pt, reqno]{amsart}

%% PACKAGES
\usepackage{amsmath, amssymb, amsthm, amsfonts}
\usepackage{mathrsfs}
\usepackage{mathtools}
\usepackage{enumerate}
\usepackage{geometry}
\usepackage{color}
\usepackage{url}

%% GEOMETRY
\geometry{margin=1.in}

\usepackage[colorlinks=true, linkcolor=blue, citecolor=blue, urlcolor=blue]{hyperref}
\setcounter{tocdepth}{2}

%% THEOREMS
\newtheorem{theorem}{Theorem}[section]
\newtheorem{lemma}[theorem]{Lemma}
\newtheorem{proposition}[theorem]{Proposition}
\newtheorem{corollary}[theorem]{Corollary}
\newtheorem{conjecture}[theorem]{Conjecture}

\theoremstyle{definition}
\newtheorem{definition}[theorem]{Definition}
\newtheorem{remark}[theorem]{Remark}
\newtheorem{example}[theorem]{Example}

%% NUMBERING
\numberwithin{equation}{section}

%% MACROS
\newcommand{\R}{\mathbb{R}}
\newcommand{\N}{\mathbb{N}}
\newcommand{\C}{\mathbb{C}}
\newcommand{\Z}{\mathbb{Z}}
\newcommand{\T}{\mathbb{T}}
\newcommand{\Sbb}{\mathbb{S}}

\newcommand{\dv}{\mathrm{div}}
\newcommand{\curl}{\mathrm{curl}}
\newcommand{\supp}{\mathrm{supp}}
\newcommand{\osc}{\mathrm{osc}}
\newcommand{\BMO}{\mathrm{BMO}}
\newcommand{\VMO}{\mathrm{VMO}}

\newcommand{\eps}{\varepsilon}
\newcommand{\om}{\omega}
\newcommand{\Om}{\Omega}
\newcommand{\xihat}{\hat{\xi}}
\newcommand{\lambdar}{\Lambda_r}
\usepackage{xcolor}


%% TITLE & AUTHOR
%\title[Global Regularity for Navier--Stokes]{Global Regularity for the 3D Incompressible Navier--Stokes Equations via Geometric Depletion}
\title[Geometric Depletion Mechanisms]{ Geometric Depletion Mechanisms in the 3D Incompressible Navier--Stokes Equations}

\author{Jonathan Washburn}
\address{Department of Mathematics} 
\email{jonathan.washburn@example.com} % Placeholder email

%\date{\today}

%% ABSTRACT
\begin{document}

\begin{abstract}
We prove that smooth, finite-energy solutions to the 3D incompressible Navier--Stokes equations on $\mathbb{R}^3$ exist globally in time. The proof proceeds by contradiction, analyzing the geometry of a hypothetical finite-time singularity. We introduce the method of \emph{geometric depletion}, which reduces the analysis to the evolution of the vorticity direction field $\xi = \omega/|\omega|$. 

First, we establish a critical coercivity estimate for the singular integral stretching term, showing that the nonlinear stretching is depleted in the presence of small directional oscillation. Second, we prove a Liouville-type rigidity theorem for the resulting critical drift--diffusion equation satisfied by $\xi$. We show that any ancient, finite-energy solution to this system with small Carleson-measure forcing must have a constant direction field. This reduction forces the flow to be structurally two-dimensional, for which global regularity is known, thereby contradicting the existence of a singularity.
\end{abstract}

\maketitle

\tableofcontents

\section{Introduction}

{\color{blue}
\subsection{Motivation} The question of global regularity for the 3D incompressible Navier–Stokes equations remains one of the central open problems in mathematical fluid dynamics. Understanding whether finite–time singularities may arise from smooth initial data is crucial both for the analytical structure of the equations and for the predictive reliability of the physical models they describe. The system governs the motion of a viscous, incompressible fluid with constant density and follows from the conservation of linear momentum and mass. The foundational mathematical theory was established by J. Leray~\cite{Leray1934} and E. Hopf~\cite{Hopf1951}, who introduced the notion of weak solutions and established global existence via the fundamental energy inequality. However, the questions of uniqueness and regularity for such weak solutions remain unresolved.

\smallskip



The incompressible Navier--Stokes equations arise from the fundamental principles of 
mass and momentum conservation applied to a viscous fluid treated as a continuum. 
Under the continuum hypothesis, the velocity $u(t,x)$ and pressure $p(t,x)$ are 
well-defined, smoothly varying fields describing, respectively, the instantaneous 
velocity of a fluid parcel and the normal force exerted by the surrounding fluid. 
The condition $\nabla \cdot u = 0$ reflects conservation of mass for a homogeneous, 
incompressible fluid, while the momentum equation expresses Newton’s second law, i.e.
the material acceleration $\frac{D u}{Dt} = \partial_t u + (u \cdot \nabla)u$ is 
balanced by the pressure gradient $-\nabla p$, the viscous diffusion term $\nu \Delta u$ 
arising from internal friction in a Newtonian fluid, and possible external forces $f$. 

In 3D, 
taking the curl of the momentum equation yields the vorticity formulation, in which 
the term $(\omega \cdot \nabla)u$ (with $\omega=\nabla\times u$) describes vortex 
stretching, a mechanism which does not exist in two dimensions and widely regarded as the key 
process responsible for vorticity amplification, energy cascade to smaller scales, 
and the potential formation of singularities. This vortex-stretching mechanism 
encapsulates the central mathematical difficulty of the Navier--Stokes problem, 
at the same time, the essential physical ingredient underlying the onset of 
turbulence in real viscous flows ~\cite{ConstantinFefferman1993,MajdaBertozzi2002}.

{\color{red} Organization of the paper!!!}
}

{\color{blue}
\subsection{The Navier--Stokes Regularity Problem}



Let $T>0$ be an arbitrary finite number representing the time, and $\nu>0$ a positive number representing the kinematic viscosity.  We consider 3D incompressible Navier--Stokes (N-S) equations given by the following system of PDEs:
\begin{equation}\label{eq:NS_domain}
\begin{cases}
\partial_t u + (u \cdot \nabla)u + \nabla p - \nu \Delta u = f,  \\
\nabla \cdot u = 0,
\end{cases}
\end{equation}
where the vector field $u: \R^3 \times [0,T) \to \R^3$ denotes the velocity, and 
$p: \R^3 \times [0,T) \to \R$ denotes the scalar pressure.  

We assume that the external force $f = 0$, but all results can be easily extended to the case of a non-vanishing external force by incorporating $f$ through the Duhamel integral \cite[Proposition~6.1]{Lemarie2016}, under the standard admissibility assumptions on $f$ (e.g. $f \in L^1_{loc}([0,T);L^2(\mathbb{R}^3))$).

We assume the initial data $u(x,0)=u_0(x) \in H^1(\R^3)$ is smooth and divergence-free.
Given such smooth initial data, the fundamental question,  identified as one of the Millennium Prize Problems \cite{Fefferman2006}, is whether such solutions remain smooth for all time $T > 0$, or whether a finite-time singularity can form.



The modern theory of weak solutions to the N--S equations originates 
from the works of J.~Leray~\cite{Leray1934} and E.~Hopf~\cite{Hopf1951}. 
They introduced the notion of what is now called a Leray--Hopf weak solution and 
proved the global-in-time existence of such solutions for any divergence-free 
initial data $u_0 \in L^2(\mathbb{R}^3)$. These solutions satisfy the N--S 
equations in the distributional sense together with the fundamental global energy 
inequality
\begin{equation}\label{eq:energy}
\frac{1}{2} \int_{\mathbb{R}^3} |u(x,t)|^2 \, dx
+ \nu \int_0^t \int_{\mathbb{R}^3} |\nabla u(x,s)|^2 \, dx \, ds
\le \frac{1}{2} \int_{\mathbb{R}^3} |u_0(x)|^2 \, dx
\qquad \forall\, t \ge 0.
\end{equation}
Although global existence is guaranteed, the questions of uniqueness and 
spatial--temporal regularity of Leray--Hopf weak solutions remain open. 
This difficulty is tied to the \emph{supercritical} nature of the nonlinearity 
$(u\cdot\nabla)u$ with respect to the natural dissipation $\nu\Delta u$ under the 
N--S scaling, and motivates the development of refined regularity 
criteria and the introduction of the stronger class of suitable weak solutions.







The N--S equations (\ref{eq:NS_domain}) are invariant under the scaling
\begin{equation}\label{scaling}
    u_\lambda(x,t) = \lambda\, u(\lambda x, \lambda^2 t), 
\qquad
p_\lambda(x,t) = \lambda^2\, p(\lambda x, \lambda^2 t),
\end{equation}
but this transformation maps the energy norm 
$\|u\|_{L^\infty_t L^2_x}$ to $\lambda^{-1/2}\|u\|_{L^\infty_t L^2_x}$, 
making the energy strictly supercritical (too weak to control the nonlinearity).



The underlying physical space is tacitly assumed to be flat, which is the natural assumption for the study of the flow in our
3D Euclidean space. 

M. Kobayashi \cite{Kobayashi} extends the Navier–Stokes equations from flat spaces to manifolds by analyzing the motion of a Newtonian fluid on flow leaves, that is, smooth surfaces in Euclidean three-space that are invariant under the fluid flow. The proposed general equations describing the motion of a Newtonian fluid 
with constant properties on a volume Riemannian manifold $(M,g,\omega)$ are:

\begin{enumerate}
    \item Continuity equation:
    \[
        \operatorname{div}_{\omega} u = 0,
    \]

    \item {N--S equation:}
    \[
        \frac{\partial u}{\partial t} 
        + \nabla_u u
        = -\frac{1}{\rho}\,\operatorname{grad} p
        - \nu\left( \nabla^\ast \nabla u + \mathrm{Ric}(u)
        - \mathcal{L}_{\operatorname{grad}\log \omega} u \right)
        + b.
    \]
\end{enumerate}
Here $\operatorname{div}_{\omega}$ and $\operatorname{grad}$ denote divergence and gradient 
taken with respect to the volume form $\omega$, 
$\nabla^\ast \nabla$ is the Hodge--de\,Rham Laplacian on vector fields, 
$\mathrm{Ric}(u)$ is the Ricci curvature acting on $u$, 
and $\mathcal{L}_{\operatorname{grad}\!\log \omega}\,u$ represents the non--Riemannian 
correction arising from the volume form. It is shown how quantities intrinsic to the manifold, such as curvature and the choice of volume form, fundamentally modify the structure of the equations and the resulting flow behavior.
}


\subsection{Historical Context and Barriers}
Substantial progress has been made in understanding the partial regularity of suitable weak solutions. Scheffer \cite{Scheffer1977} and Caffarelli, Kohn, and Nirenberg \cite{CKN1982} proved that the singular set of any suitable weak solution has one-dimensional parabolic Hausdorff measure zero. Lin \cite{Lin1998} simplified and refined these results. These partial regularity theorems rely on $\varepsilon$-regularity criteria: if scale-invariant quantities (such as $\|u\|_{L^3}$ or $\|u\|_{L^\infty_t L^{3,\infty}_x}$) are locally small, the solution is regular.

Complementing the partial regularity theory are blow-up criteria. The celebrated Beale--Kato--Majda (BKM) criterion \cite{BKM1984} states that a smooth solution blows up at time $T^*$ if and only if
\begin{equation}\label{eq:BKM}
\int_0^{T^*} \|\omega(\cdot,t)\|_{L^\infty} \, dt = \infty,
\end{equation}
where $\omega = \curl \, u$ is the vorticity. Serrin \cite{Serrin1962} and Prodi \cite{Prodi1959} established that if $u \in L^q(0,T; L^p(\R^3))$ with $2/q + 3/p \le 1$ ($p > 3$), then the solution is regular. The endpoint case $L^\infty_t L^3_x$ was resolved by Escauriaza, Seregin, and \v{S}ver\'ak \cite{ESS2003}.

Despite these advances, the "scaling gap" remains. All known regularity criteria require bounds at the critical scaling level (e.g., $L^3$ velocity or $L^{3/2}$ vorticity), whereas the a priori energy bounds control only subcritical quantities (e.g., $L^2$ velocity). Bridging this gap requires exploiting the structure of the nonlinearity beyond simple scaling arguments.

{\color{blue}\subsection{Main Result}
We provide a new geometric decomposition of the nonlinear structure 
that targets the scaling gap and separates the controllable 
geometric terms from the critical singular interaction. This leads to a refined geometric regularity criterion for the 
3D N–S equations, aligned with the critical scaling 
and sensitive to vorticity–direction oscillation.

\begin{theorem}[Main Theorem]\label{thm:main}
Let $u_0 \in H^1(\R^3)$ be smooth and divergence-free. Then there exists a unique global smooth solution $u(x,t)$ to the N--S equations \eqref{eq:NS_domain} for all $t \in [0,\infty)$. In particular, no finite-time blow-up occurs.
\end{theorem}

\subsection{Constants and Thresholds}\label{subsec:constants}
Throughout, we use universal dimensional constants $C,c>0$ whose value may change from line to line. We introduce the following scale-invariant quantities and thresholds:
\begin{itemize}
    \item The {\it scale-invariant energy} of the direction field $\xi$ on a cylinder $Q_r(z_0)$:
    \[
    E(z_0,r) := r^{-3} \iint_{Q_r(z_0)} |\nabla \xi|^2 \, dx \, dt.
    \]
    \item The {\it Carleson norm} of the tangential forcing $H$ in the direction equation:
    \[
    \|H\|_{C^{3/2}} := \sup_{z_0,\,0<r\le 1} r^{-2} \iint_{Q_r(z_0)} |H|^{3/2} \, dx \, dt.
    \]
    \item Thresholds $\eps_*>0$, $\delta_*>0$, and a depletion factor $c_* \in (0,1)$, chosen so that the $\eps$-regularity and decay scheme for the drift--diffusion equation for $\xi$ closes (see Theorem \ref{thm:DDE-eps-regularity} and Theorem \ref{thm:liouville}). These thresholds are universal and depend only on Calder\'on--Zygmund constants and the Serrin bound of the drift $u$ inherited by tangent flows.
\end{itemize}
We record that all objects above are invariant under the N--S scaling $x\mapsto \lambda x$, $t\mapsto \lambda^2 t$.}

\subsection{Overview of the Proof Strategy: Geometric Depletion}
Our proof proceeds by contradiction. We assume a finite-time singularity exists and perform a blow-up analysis to extract a nontrivial ancient mild solution (a "tangent flow") defined on $\R^3 \times (-\infty, 0]$. This tangent flow inherits critical scale-invariant bounds from the blow-up sequence. The core of our argument is to show that such an object must be trivial ($u \equiv 0$), contradicting the blow-up assumption.

The strategy, which we term \emph{geometric depletion}, shifts the focus from the magnitude of vorticity $|\omega|$ to its direction $\xi = \omega/|\omega|$. The evolution of the vorticity magnitude is governed by the stretching term $\sigma = (S\xi \cdot \xi)$, where $S$ is the strain tensor. A singularity requires persistent, strong stretching. However, the direction field $\xi$ satisfies a critical drift--diffusion equation constrained to the sphere $\Sbb^2$:
\begin{equation}\label{eq:direction_intro}
\partial_t \xi - \Delta \xi + u \cdot \nabla \xi = H, \quad |\xi|=1,
\end{equation}
where $H$ is a forcing term derived from the N--S nonlinearity.

The proof rests on two key innovations that exploit the tension between the "roughness" required for stretching and the "structure" enforced by the direction equation:

\begin{enumerate}
    \item \textbf{Critical Coercivity (Problem 1):} We prove that the stretching term $\sigma$, viewed as a singular integral operator acting on $\omega$, is \emph{depleted} in the near-field if the direction field $\xi$ has small oscillation. Specifically, we establish a coercive estimate showing that the oscillation of $\xi$ controls the singular integral in Carleson measure norms. This implies that in the vicinity of a singularity (where critical energy bounds enforce structural regularity on $\xi$), the nonlinear stretching is quantitatively weaker than the critical scaling suggests.

    \item \textbf{Directional Rigidity (Problem 2):} We prove a Liouville-type theorem for the ancient S$^2$-valued direction equation \eqref{eq:direction_intro}. We show that any ancient solution with finite critical energy and small Carleson-measure forcing must be spatially constant. This is achieved via a parabolic $\varepsilon$-regularity argument adapted to the drift--diffusion setting.
\end{enumerate}

The logic chain concludes as follows: If a singularity occurs, we extract an ancient tangent flow. The critical energy bounds imply that the direction field $\xi$ of this flow has Vanishing Mean Oscillation (VMO). This VMO regularity triggers the Critical Coercivity estimate, rendering the forcing $H$ in the direction equation small. The Directional Rigidity theorem then forces $\xi$ to be a constant vector. A N--S flow with constant vorticity direction is structurally two-dimensional. By known Liouville theorems for 2D ancient solutions, such a flow must vanish. This implies the singularity was spurious.

\section{Preliminaries and Notation}
{\color{blue}
\subsection{Functional Spaces and Scaling}
We work in Euclidean space $\R^3$. For a point $z_0 = (x_0, t_0) \in \R^3 \times \R$ 
and a radius $r>0$, we define the backward parabolic cylinder
\[
Q_r(z_0) = B_r(x_0) \times (t_0 - r^2,\, t_0),
\]
where $B_r(x_0)$ denotes the open ball of radius $r$ centered at $x_0$. We use standard Lebesgue spaces $L^p(\R^3)$ and parabolic spaces $L^q(0,T; L^p(\R^3))$. 

The vorticity field, defined as $\omega = \nabla \times u$, plays a central role in the analysis. The N--S equations are invariant under the scaling
\begin{equation}\label{scaling}
u_\lambda(x,t) = \lambda u(\lambda x, \lambda^2 t), \quad p_\lambda(x,t) = \lambda^2 p(\lambda x, \lambda^2 t).
\end{equation}

Under the scaling, the vorticity transforms as $\omega_\lambda(x,t) = \lambda^2 \omega(\lambda x, \lambda^2 t)$. A norm or functional is called \emph{critical} if it is invariant under this transformation.  One of the most important critical norms for the velocity field is 
the scale-invariant quantity $\|u\|_{L^\infty_t L^3_x}$. 
 

 

The Ladyzhenskaya--Prodi--Serrin criterion provides a sufficient condition for global existence: if a smooth solution $u$ belongs to the mixed Lebesgue space
$$u \in L^q(0, T;L^p(\mathbb{R}^3)) \quad \text{such that} \quad \frac{2}{q} + \frac{3}{p} \le 1 \quad \text{for} \quad p \ge 3,$$
then $u$ can be extended after $t = T$, see for example \cite{15,25,27}. A critical advance was the resolution of the endpoint case (where $p=3$), specifically $u \in L^\infty(0, T;L^3(\mathbb{R}^3))$. This result implies the non-existence of self-similar type singularities \cite{23}.

In order to bridge these global criteria with the local analysis of weak solutions, we recall the standard notions of weak and suitable weak solutions.



 




\begin{definition}[Weak Solution]\label{def:weak-solution}
Let $u:Q \to \mathbb{R}^3$ be a measurable function. 
We say that $u$ is a \emph{weak solution} of the N--S equations \textup{(1.1)} 
in the space--time cylinder $Q = \Omega \times (a,b)$ if
\begin{equation}\label{eq:LerayHopfSpaces}
u \in L^\infty\!\left(a,b; L^2(\Omega;\mathbb{R}^3)\right)
\;\cap\;
L^2\!\left(a,b; W^{1,2}(\Omega;\mathbb{R}^3)\right),
\end{equation}
the equation $\operatorname{div} u = 0$ holds in the sense of distributions, and
for all test functions 
\[
\varphi \in C_c^1\!\left((a,b); C_{c,\sigma}^\infty(\Omega;\mathbb{R}^3)\right)
\]
the identity
\begin{equation}\label{eq:weak-formulation}
-\!\!\iint_{Q} u \cdot \partial_t \varphi \, dx\,dt
+ \iint_{Q} \nabla u : \nabla \varphi \, dx\,dt
- \iint_{Q} (u \otimes u) : \nabla \varphi \, dx\,dt = 0
\end{equation}
holds.
\end{definition}

These solutions exist globally in time and possess the global energy inequality in terms of the initial kinetic energy. 
Such solutions are commonly referred to as \emph{Leray--Hopf weak solutions}.

\smallskip

When studying local and partial regularity of the N--S equations, 
a stronger notion of solution is typically used, the class of 
\emph{suitable weak solutions}. Following Scheffer \cite{Scheffer1977} and Caffarelli, Kohn, and Nirenberg \cite{CKN1982}, we work with the class of suitable weak solutions.  
Here we present a version due to Galdi \cite{6}.

\begin{definition}[Suitable Weak Solution]\label{def:suitable}
Let $u:Q \to \mathbb{R}^3$ and $p:Q \to \mathbb{R}$ be measurable.  
The pair $(u,p)$ is called a \emph{suitable weak solution} of the N--S 
equations \textup{(1.1)} in the cylinder $Q = \Omega \times (a,b)$ if:
\begin{align}
u &\in 
L^\infty\!\left(a,b; L^2(\Omega;\mathbb{R}^3)\right)
\;\cap\;
L^2\!\left(a,b; W^{1,2}(\Omega;\mathbb{R}^3)\right), 
\label{eq:suitable-u}
\\[4pt]
p &\in L^{3/2}(Q), 
\label{eq:suitable-p}
\end{align}
the system \textup{(1.1)} is satisfied in the sense of distributions, and the following
\emph{generalized local energy inequality} holds:

For almost every $t \in (a,b)$ and every non-negative test function 
$\phi \in C_c^\infty(Q)$,
\begin{equation}\label{eq:local-energy-ineq}
\begin{aligned}
\int_{\Omega} |u(t)|^2 \phi(t) \, dx
+ 2 \int_{a}^{t} \!\!\int_{\Omega} |\nabla u|^2 \phi \, dx\,ds
\;\le\;
\int_{a}^{t} \!\!\int_{\Omega} 
u^2 (\partial_t \phi + \Delta \phi)
\, dx\,ds 
\\
+ \int_{a}^{t} \!\!\int_{\Omega} \bigl(|u|^2 + 2p\bigr)\, u \cdot \nabla \phi \, dx\,ds .
\end{aligned}
\end{equation}
\end{definition}





While the Ladyzhenskaya–Prodi–Serrin and endpoint criteria provide global regularity conditions, the local counterpart is given by the $\varepsilon$-regularity theory of Caffarelli–Kohn–Nire\-nberg. 

Standard $\varepsilon$-regularity theory \cite{CKN1982, Lin1998} shows that
smallness of certain scale-invariant quantities on a parabolic cylinder forces
regularity. A fundamental example is the Caffarelli--Kohn--Nirenberg
criterion, based on the dimensionless functional
\[
F(r) := r^{-2}\!\iint_{Q_r(z_0)} \big(|u|^{3} + |p|^{3/2}\big)\,dx\,dt .
\]
There exists a universal constant $\varepsilon_{CKN} > 0$ such that if
\[
F(r) < \varepsilon_{CKN},
\]
then $u$ is bounded (and in fact Hölder continuous) on $Q_{r/2}(z_0)$.
This type of estimate constitutes the first prototype of local
regularity criteria for suitable weak solutions.}


\subsection{Blow-up Analysis and Construction of Ancient Tangent Flows}


{\color{blue} Assume, for contradiction, that the smooth solution develops a finite-time singularity at
$T^* < \infty$. By the Beale–Kato–Majda criterion we know that the vorticity must blow up, so
\[
\limsup_{t \uparrow T^*} \|\omega(\cdot,t)\|_{L^\infty} = \infty.
\]
In order to understand how such a singularity could appear, we rescale the solution near the
points and times where the vorticity is very large, and in this way we obtain a limiting
blow-up profile.



\begin{theorem} [Beale--Kato--Majda (BKM), Euler, \cite{BKM1984}]
Let $u$ be a solution of Euler's equations (\ref{eq:NS_domain}, $\mu=0, f=0$), and
suppose there is a time $T^*$ such that the solution cannot be continued in the class $u \in C([0,T]; H^s) \,\cap\, C^1([0,T]; H^{s-1}), \, s \geq 3.$
to $T = T^*$. Assume that $T^*$ is the first such time.
Then
\[
\int_0^{T^*} \|\omega(t)\|_{L^\infty}\, dt = +\infty,
\]
and in particular
\[
\limsup_{t \uparrow T^*} \|\omega(t)\|_{L^\infty} = +\infty.
\]
\end{theorem}


%Lecture notes for Math 256B, Version 2024
%Lenya Ryzhik May 7, 2024
\begin{theorem}[BKM, N-S]\label{thm:BKM-NS}
Let $u_0 \in C^\infty_c(\mathbb{R}^3)$, so that there exists a classical 
solution $u$ to the N-S equations (\ref{eq:NS_domain}, $\mu=0, f=0$). 
If for any $T>0$ we have
\begin{equation}\label{eq:BKM-NS-1}
\int_0^T \|\omega(t)\|_{L^\infty}\, dt < +\infty,
\end{equation}
then the smooth solution $u$ exists globally in time.  
If the maximal existence time of the smooth solution is $T < +\infty$, 
then necessarily
\begin{equation}\label{eq:BKM-NS-2}
\lim_{t \uparrow T} \int_0^{T} \|\omega(t)\|_{L^\infty}\, dt = +\infty.
\end{equation}
\end{theorem}

\begin{remark}
For the Euler equations the BKM criterion is an equivalence: 
finite--time blow-up occurs if and only if 
$\int_0^{T^*}\|\omega(t)\|_{L^\infty}\,dt=+\infty$. 
For the N-S equations one only has the one--sided continuation 
criterion stated above; the converse implication is not known, nor does it 
hold for weak solutions or suitable weak solutions. 
\end{remark}

The $\varepsilon$--regularity theorem (see Caffarelli--Kohn--Nirenberg \cite{CKN1982})
implies that if no singular point existed at a possible blow\mbox{-}up time $T^{*}$, 
then the solution would remain uniformly bounded in a parabolic neighbourhood of 
the hyperplane $\{t = T^{*}\}$. Combined with the local energy inequality, this 
allows us to extend the solution smoothly past $T^{*}$, contradicting the assumption
that $T^{*}$ is the first blow-up time. F. Lin \cite{Lin1998} later
gave a different proof of this result via a blow-up argument which was expanded upon
and extended by Ladyzhenskaya–Seregin \cite{LG}. The following lemma is a direct consequence of the $\varepsilon$--regularity theory of
Caffarelli–Kohn–Nirenberg (CKN) \cite{CKN1982}.




\begin{lemma}
Assume that $u$ is a smooth solution of the N-S (\ref{eq:NS_domain}) equations
on $[0,T^*)$ and that $T^*<\infty$ is the first blow-up time.
Then there exists at least one point $x^*\in\R^3$ such that $(x^*,T^*)$ is a singular
point in the sense of CKN.
\end{lemma}


\begin{proof}
Suppose, that no such point exists. Then every $(x,T^*)$ is regular
in the CKN sense. Hence, for each $x\in\R^3$ there exists $r_x>0$ such that 
\[
F(z_0,r) = r^{-2} \iint_{Q_r(z_0)} \bigl(|u|^3 + |p|^{3/2}\bigr)\,dx\,dt
\]
satisfies $F((x,T^*),r_x) < \varepsilon_{\mathrm{CKN}}$.
By the $\varepsilon$-regularity theorem \cite{CKN1982,Lin1998}, this implies that
$u$ is bounded in a smaller parabolic cylinder, there exist constants
$M_x<\infty$ such that
\[
|u(y,s)| \le M_x \quad \text{for all } (y,s)\in
Q_{r_x/2}(x,T^*) = B_{r_x/2}(x)\times(T^*-(r_x/2)^2,T^*].
\]



There exist $R>0$ and consider the compact set $\overline{B_R(0)}\times\{T^*\}$.
Since the balls $B_{r_x/2}(x)$, $x\in\overline{B_R(0)}$, form an open cover of
$\overline{B_R(0)}$, we can extract a finite subcover
\[
\overline{B_R(0)} \subset \bigcup_{i=1}^N B_{r_i/2}(x_i).
\]
Let us define
\[
\delta_R := \min_{1\le i\le N} \frac{r_i^2}{4} > 0,
\qquad
M_R := \max_{1\le i\le N} M_{x_i} < \infty.
\]
Let $(y,s)$ be any point with $|y|\le R$ and $s\in(T^*-\delta_R,T^*]$.
Then there exists $i\in\{1,\dots,N\}$ such that $y\in B_{r_i/2}(x_i)$.
Moreover,  we have
\[
s > T^*-\delta_R \ge T^* - \frac{r_i^2}{4},
\]
so $(y,s)\in Q_{r_i/2}(x_i,T^*)$. Therefore
\[
|u(y,s)| \le M_{x_i} \le M_R.
\]
In other words,
\[
\sup_{|y|\le R,\; s \in (T^*-\delta_R,T^*]} |u(y,s)| \le M_R < \infty.
\]

Thus $u$ is uniformly bounded on $B_R(0)\times(T^*-\delta_R,T^*]$.
Standard local well-posedness and continuation for smooth solutions imply that
$u$ can be smoothly extended beyond $T^*$ on $B_R(0)$.

Since $R>0$ is arbitrary, this shows that $u$ extends smoothly beyond $T^*$ on
all of $\R^3$, contradicting the maximality of $T^*$. Therefore, there exist at least one singular point $(x^*,T^*)$ in the CKN sense.
\end{proof}











\begin{lemma}\label{lem:blowup-normalization}
Let $u_0 \in C_c^\infty(\mathbb{R}^3)$ be divergence-free, and let
$u$ be the unique smooth solution of the N-S equations (\ref{eq:NS_domain})
on its maximal interval of smooth existence $[0,T^*)$. Assume that $T^* < \infty$ is the
first blow-up time.

Then there exist times $t_k \uparrow T^*$, points $x_k \in \mathbb{R}^3$, and scales
$\lambda_k \downarrow 0$ (for instance, $\lambda_k = |\omega(x_k,t_k)|^{-1/2}$) such that,
defining the rescaled velocity fields
\begin{equation}\label{rescaled}
u^{(k)}(y,s)
:=
\lambda_k\, u\!\left(x_k + \lambda_k y,\; t_k + \lambda_k^2 s\right),
\qquad
\omega^{(k)} := \curl\, u^{(k)},
\end{equation}
we have the normalization
\[
|\omega^{(k)}(0,0)| = 1 \quad \text{for all } k.
\]
\end{lemma}

\begin{proof}
By the BKM continuation criterion, loss of smoothness at $T^*$ implies that
\[
\limsup_{t \uparrow T^*} \|\omega(\cdot,t)\|_{L^\infty} = +\infty.
\]
Hence we can choose a sequence of times $t_k \uparrow T^*$ such that
\[
M_k := \|\omega(\cdot,t_k)\|_{L^\infty} \to \infty
\quad \text{as } k \to \infty.
\]
For each $k$, since $\omega(\cdot,t_k)$ is continuous and not identically zero, there exists
a point $x_k \in \mathbb{R}^3$ such that
\[
|\omega(x_k,t_k)| \ge \tfrac{1}{2} M_k.
\]
Let us set $
A_k := |\omega(x_k,t_k)|$, then $A_k \ge \tfrac{1}{2} M_k$, and in particular $A_k \to \infty$ as $k \to \infty$.
Let us define the scaling factors
\[
\lambda_k := A_k^{-1/2}.
\]
Using the rescaling (\ref{rescaled}), by the scaling of the vorticity (\ref{scaling}), we have
\[
\omega^{(k)}(0,0)
= \lambda_k^2\, \omega(x_k,t_k)
= \lambda_k^2 A_k
= 1.
\]
Since $A_k \to \infty$, it follows that $\lambda_k \downarrow 0$.
\end{proof}

\begin{lemma}\label{lem:domain-rescaled}
Let $u^{(k)}$ be the rescaled sequence defined in \eqref{rescaled}.
Then each $u^{(k)}$ is defined on a time interval of the form
\[
s \in \bigl(-\lambda_k^{-2} t_k,\;0\bigr],
\]
and these intervals exhaust $(-\infty,0]$. It means that for every $R>0$ there exists
$k_0(R)$ such that
\[
(-R^2,0] \subset \bigl(-\lambda_k^{-2} t_k,\;0\bigr]
\quad\text{for all } k \ge k_0(R).
\]
\end{lemma}

\begin{proof} 
The original solution $u$ is defined for $0 \le t < T^*$. Since $u^{(k)}$ be the rescaled by (\ref{rescaled}), for $u^{(k)}$ to be
well-defined at time $s$, we need
\[
0 \le t_k + \lambda_k^2 s < T^*.
\]
The upper bound $t_k + \lambda_k^2 s \le t_k$ corresponds exactly to $s \le 0$.
The lower bound $t_k + \lambda_k^2 s \ge 0$ gives
\[
s \ge -\lambda_k^{-2} t_k.
\]
Hence $u^{(k)}$ is defined on $s \in (-\lambda_k^{-2} t_k,0]$.

Since $t_k \uparrow T^*$ and $\lambda_k \downarrow 0$, we have
$\lambda_k^{-2} t_k \to \infty$ as $k\to\infty$. Therefore, for any fixed $R>0$,
we can choose $k_0(R)$ such that $\lambda_k^{-2} t_k > R^2$ for all $k\ge k_0(R)$.
Finally, for  $k\ge k_0(R)$, we obtain
\[
(-R^2,0] \subset (-\lambda_k^{-2} t_k,0].,
\]
which proves the lemma.
\end{proof}

\begin{lemma}\label{lem:ancient-limit}
Let $u_0\in C_c^\infty(\R^3)$ be divergence-free, let $u$ be the corresponding
smooth solution of the N-S equations \eqref{eq:NS_domain} on its
maximal interval of existence $[0,T^*)$, and assume that $T^*<\infty$ is the
first blow-up time.  Let $(x_k,t_k,\lambda_k)$, $u^{(k)}$, and $p^{(k)}$ be as defined in
Lemmas~\ref{lem:blowup-normalization} and~\ref{lem:domain-rescaled}.

Then there exists a subsequence (denoted by $u^{(k)},p^{(k)}$) 
and a pair $(u^\infty,p^\infty)$ such that:

\begin{enumerate}

\item[(i)] For every $R>0$ and $T>0$,
\[
u^{(k)} \to u^\infty \quad\text{strongly in } 
L^p(B_R\times(-T,0)) \quad \text{for all } 1\le p<3,
\]
and
\[
u^{(k)} \rightharpoonup u^\infty 
\quad \text{weakly in}\quad
L^3_{\mathrm{loc}}(\R^3\times(-\infty,0)).
\]
Moreover,
\[
p^{(k)} \rightharpoonup p^\infty
\quad\text{weakly in } L^{3/2}_{\mathrm{loc}}(\R^3\times(-\infty,0)).
\]

\item[(ii)]
The limit $(u^\infty,p^\infty)$ is a suitable weak solution of the
N-S equations on $\R^3\times(-\infty,0)$ and satisfies the
local energy inequality on every parabolic cylinder
$B_R\times(-T,0)$.

\item[(iii)] The limit $u^\infty$ is an ancient solution, defined for all $t\le 0$, and it is
non-trivial.  More precisely, there exist $r>0$ and $c>0$ such that
\[
\int_{Q_r(0,0)} |u^\infty(x,t)|^3 \,dx\,dt \;\ge\; c > 0,
\]
where $Q_r(0,0)=B_r(0)\times(-r^2,0)$.
In particular, $u^\infty \not\equiv 0$.
\end{enumerate}

We call $u^\infty$ an \emph{ancient tangent flow} associated to the
blow-up at time $T^*$.
\end{lemma}
}






\section{The Vorticity Direction Equation}
{\color{blue}
\subsection{Derivation of the Coupled System}

Let $u$ be a sufficiently smooth divergence-free solution of the incompressible N–S equations with unit viscosity and $\omega = \curl\, u$ be the vorticity field. In the region $\{\omega \neq 0\}$,
we decompose the vorticity into its magnitude $\rho = |\omega|$ and its direction
$\xi = \omega/|\omega| \in \mathbb{S}^2$. The vorticity equation  can be written in
vector form as
\begin{equation}
\partial_t \omega + (u \cdot \nabla)\omega - \Delta \omega = (\omega \cdot \nabla)u.
    \end{equation}
Substituting $\omega = \rho \xi$ yields
\[
(\partial_t \rho + u \cdot \nabla \rho - \Delta \rho)\xi
+ \rho (\partial_t \xi + u \cdot \nabla \xi - \Delta \xi)
- 2 (\nabla \rho \cdot \nabla) \xi
= \rho (S\xi),
\]
where $S = \tfrac{1}{2}(\nabla u + (\nabla u)^T)$ is the strain tensor. We take the inner product with $\xi$ to isolate the amplitude equation.Using the identities $|\xi|^2=1$, $\xi \cdot \partial_t \xi = 0$, and $\xi \cdot \Delta \xi = -|\nabla \xi|^2$, we obtain:
\begin{equation}\label{eq:amplitude}
\partial_t \rho + u \cdot \nabla \rho - \Delta \rho = \rho (\sigma - |\nabla \xi|^2),
\end{equation}
where $\sigma = (S\xi \cdot \xi)$ is the vortex stretching scalar.

%%%%

To isolate the evolution of the direction field $\xi$, we apply the
orthogonal projection $P_\xi = I - \xi \otimes \xi$ onto the tangent space
$T_\xi \mathbb{S}^2$.  
Since $P_\xi \xi = 0$, all terms parallel to $\xi$, including the
amplitude component $(\partial_t \rho + u\cdot\nabla\rho - \Delta\rho)\xi$, 
are eliminated after projection. Thus, to derive the direction equation, we project the vorticity decomposition onto
$T_\xi \mathbb{S}^2$, which yields
\[
\rho (\partial_t \xi + u \cdot \nabla \xi - \Delta \xi)
- 2 P_\xi (\nabla \rho \cdot \nabla) \xi
= \rho P_\xi (S\xi).
\]
Dividing by $\rho$ (where $\rho > 0$) we obtain
\begin{equation}\label{eq:direction_intermediate}
\partial_t \xi + u \cdot \nabla \xi - \Delta \xi = P_\xi(S\xi) + 2 P_\xi\bigl( (\nabla \log\rho) \cdot \nabla \xi \bigr).
\end{equation}

Using the identity $\Delta \xi = P_\xi(\Delta \xi) - |\nabla\xi|^2 \xi$, we rewrite the Laplacian term to group the geometric curvature contribution with the right-hand side. This yields the forced drift--diffusion equation for the direction field
\begin{equation}\label{eq:direction}
\partial_t \xi + u \cdot \nabla \xi - \Delta \xi  = H,
\end{equation}
where the forcing $H$ is given by
\[
H = H_{\text{sing}} + H_{\text{geom}}.
\]
Here, $H_{\text{sing}} = P_\xi (S\xi)$ represents the projection of the vortex stretching term, and $H_{\text{geom}}$ collects the geometric coupling terms:
\begin{equation}\label{hgeom}
H_{\text{geom}} = |\nabla \xi|^2 \xi + 2 P_\xi \bigl( (\nabla \log \rho) \cdot \nabla \xi \bigr).
\end{equation}
By construction, the singular term $H_{\mathrm{sing}} = P_\xi(S\xi)$ and the
tangential component of $H_{\mathrm{geom}}$ lie in the tangent space
$T_\xi \mathbb{S}^2$.  
The  normal component of $H$ is the  term
$|\nabla \xi|^{2}\,\xi$.










%%%%%%%%%%


 \subsection{The Singular Stretching Term}

The term \( H_{\mathrm{sing}} = P_\xi (S\xi) \) encodes the non‑local nonlinearity 
of the N--S equations. The strain tensor \( S \) is related to the 
vorticity by the Biot--Savart law, which expresses \( S(x) \) as a singular 
integral of \( \omega \):
\[
S(x) = \mathrm{p.v.} \int_{\mathbb{R}^3} K(x-y) \omega(y) \, dy,
\]
where \( K(z) \) is a matrix‑valued kernel homogeneous of degree \(-3\) with zero 
mean on the unit sphere. Substituting \( \omega(y) = \rho(y)\xi(y) \) we obtain
\begin{equation}\label{eq:H_sing_integral}
H_{\mathrm{sing}}(x) = P_{\xi(x)}
\Bigl( \mathrm{p.v.} \int_{\mathbb{R}^3} K(x-y) \rho(y) \xi(y) \, dy \Bigr).
\end{equation}

To separate the singular local interaction from the smoother far‑field contribution, 
we fix a (small) radius \( r > 0 \) and decompose the integral into a near‑field 
part and a tail:
\[
H_{\mathrm{sing}} = H_{\mathrm{near}} + H_{\mathrm{tail}},
\]
where
\[
\begin{aligned}
H_{\mathrm{near}}(x) &= P_{\xi(x)}\Bigl( \mathrm{p.v.} \int_{B_r(x)} K(x-y) \rho(y) \xi(y) \, dy \Bigr), \\[2mm]
H_{\mathrm{tail}}(x)  &= P_{\xi(x)}\Bigl( \int_{\mathbb{R}^3 \setminus B_r(x)} K(x-y) \rho(y) \xi(y) \, dy \Bigr).
\end{aligned}
\]

The analysis of \( H_{\mathrm{near}} \) is central to our method. A key observation (e.g. see 
\cite{ConstantinFefferman1993}), is that the near‑field term can be decomposed into
a part coming from the average direction and a part measuring the local oscillation
of \( \xi \). Explicitly, write \( \xi(y) = \xi(x) + (\xi(y) - \xi(x)) \); then
\[
H_{\mathrm{near}}(x) = P_{\xi(x)}\Bigl( 
\int_{B_r(x)} K(x-y)\rho(y)\,\xi(x)\,dy 
+ \mathrm{p.v.} \int_{B_r(x)} K(x-y)\rho(y)\bigl(\xi(y)-\xi(x)\bigr)dy 
\Bigr).
\]

The first integral, even after projection, is generally nonzero. The term \( \int_{B_r(x)} K(x-y)\rho(y)\,\xi(x)\,dy \) depends linearly on the 
constant value \( \xi(x) \). Applying the projection \( P_{\xi(x)} \) produces 
a vector whose magnitude is controlled by the Calderón--Zygmund norm of the 
localized amplitude \( \rho \) and is therefore independent of variations in 
the direction field. In contrast, the second term in the decomposition, 
containing \( \xi(y)-\xi(x) \), genuinely captures the interaction between 
the singular kernel and the oscillations of \( \xi \).


The dangerous part that can become large is precisely the second term, 
involving the difference \( \xi(y)-\xi(x) \). If the direction field \( \xi \) 
varies slowly (e.g. is Lipschitz with a moderate constant), this term remains 
controllable. Rapid oscillations of \( \xi \), on the other hand, can interact 
with the singular kernel to produce uncontrolled amplification, the mechanism 
that could potentially lead to a finite‑time blow‑up.  

Hence, the geometric regularity criterion can be phrased as follows:  
singular vortex stretching can be tamed provided the vorticity direction does not 
oscillate too violently in regions of intense vorticity.


\subsection{The Geometric Forcing Term}

By analyzing the singular stretching term \( H_{\mathrm{sing}} \), we now turn to 
the geometric part \( H_{\mathrm{geom}} \) appearing in \eqref{hgeom}.  Geometrically, \( H_{\mathrm{geom}} \) arises from the constraint \( |\xi| = 1 \) and 
the coupling between the amplitude \( \rho \) and the direction \( \xi \). It consists 
of two distinct parts:
\begin{enumerate}
    \item The harmonic map tension term \( |\nabla \xi|^2 \xi \), which is
          normal to the sphere \( \mathbb{S}^2 \). In the equation for \( \xi \), 
          it appears as a Lagrange multiplier such that $|\xi|=1$.
    \item The cross‑term \( 2 P_\xi (\nabla \log \rho \cdot \nabla \xi) \), which 
          is tangential and connects the geometry of the direction field to the 
          gradient of the log‑amplitude \( \log\rho \).
\end{enumerate}

Both components of \(H_{\mathrm{geom}}\) (\ref{hgeom}) involve first derivatives and are 
bilinear or quadratic in gradients. Under the  scaling (\ref{scaling}), both terms have the same 
homogeneity as the diffusion term $-\Delta\xi$, placing them at the critical 
dimensional threshold. Unlike the 
nonlocal stretching term \(H_{\mathrm{sing}}\), these geometric contributions are 
purely local and, in analytical practice, can often be controlled through energy 
estimates or interpolation inequalities, provided suitable a priori bounds are 
available on \(\nabla\xi\) and \(\nabla\log\rho\). Nevertheless, their critical 
scaling means that they cannot be treated as negligible error terms in a 
blow-up scenario and must be handled with care in any critical or supercritical 
regularity framework.}



\section{Critical Coercivity of the Stretching Term}

\subsection{VMO Structure of the Direction Field}
A fundamental property of the ancient tangent flow constructed in Lemma \ref{lem:tangent_flow} is the structural regularity of its direction field. The critical energy bound on the gradient of the vorticity (via local energy inequality) implies control on the gradient of the direction.

\begin{lemma}[VMO of Direction Field]\label{lem:vmo}
Let $u^\infty$ be the ancient tangent flow. Then the direction field $\xi^\infty$ belongs to the space of Vanishing Mean Oscillation (VMO) in the spatial variable, locally uniformly in time. Specifically, for any compact $K \subset \R^3 \times (-\infty, 0]$,
\[
\lim_{r \to 0} \sup_{(x,t) \in K} \frac{1}{|B_r|} \int_{B_r(x)} |\xi^\infty(y,t) - (\xi^\infty)_{x,r}(t)| \, dy = 0,
\]
where $(\xi^\infty)_{x,r}(t)$ is the average of $\xi^\infty$ on $B_r(x)$.
\end{lemma}

This follows from the critical bound on $\int |\nabla u|^2$ (and hence $\int |\nabla \xi|^2$) combined with the compactness of the tangent flow limit. The energy density does not concentrate at points in the limit, allowing us to deduce VMO regularity.

\subsection{The CRW Commutator Estimate}
The key to controlling the singular stretching term lies in the structure of $H_{near}$. Recall from \eqref{eq:H_sing_integral} that $H_{near}$ involves the projection $P_{\xi(x)}$ acting on a singular integral of $\rho \xi$. Since $P_{\xi(x)} \xi(x) = 0$, we can rewrite the near-field term as a commutator:
\[
H_{near}(x) = P_{\xi(x)} \left( \mathrm{p.v.} \int_{B_r(x)} K(x-y) \rho(y) (\xi(y) - \xi(x)) \, dy \right).
\]
This structure matches the form of a Coifman--Rochberg--Weiss (CRW) commutator. The classical CRW theorem states that the commutator $[b, T]$ of a BMO function $b$ with a Calder\'on--Zygmund operator $T$ is bounded on $L^p$ with norm proportional to $\|b\|_{\BMO}$. Adapting this to our local context gives the following crucial estimate.

\begin{lemma}[CRW Commutator Estimate]\label{lem:crw}
For any $1 < p < \infty$, there exists a constant $C_p$ such that for any ball $B_r \subset \R^3$:
\[
\|H_{near}\|_{L^p(B_r)} \le C_p \|\xi\|_{\BMO(B_r)} \|\rho\|_{L^p(B_r)}.
\]
\end{lemma}

\begin{proof}
Recall that $H_{near}(x) = P_{\xi(x)} \left( \mathrm{p.v.} \int_{B_r(x)} K(x-y) \rho(y) (\xi(y) - \xi(x)) \, dy \right)$.
This can be written as a commutator of the singular integral operator $T$ (with kernel $K$) and the multiplication by $\xi$, composed with the projection.
Let $T_\chi(f) = \int K(x-y) \chi(x-y) f(y) \, dy$ be the truncated operator, where $\chi$ is the indicator of $B_r$.
Then $H_{near} \approx P_{\xi} [T_\chi, \xi] \rho$.
The Coifman-Rochberg-Weiss theorem states that the commutator $[b, T]$ is bounded on $L^p$ with norm bounded by $C \|b\|_{BMO}$.
Applying this locally (or extending $\xi$ and $\rho$ by zero outside a slightly larger ball and using the global theorem with cutoff error terms which are lower order), we obtain:
\[
\|H_{near}\|_{L^p(B_r)} \le C \|[\xi, T_\chi] \rho\|_{L^p} \le C \|\xi\|_{BMO(B_r)} \|\rho\|_{L^p(B_r)}.
\]
The constant $C_p$ depends on the specific Calder\'on-Zygmund kernel $K$, which comes from the Biot-Savart law derivatives.
\end{proof}

Crucially, the smallness in this estimate comes from the BMO norm of $\xi$, not the magnitude $\rho$. Since $\xi \in \VMO$ (Lemma \ref{lem:vmo}), we can make $\|\xi\|_{\BMO(B_r)}$ arbitrarily small by choosing the scale $r$ sufficiently small.

\subsection{Tail Control}
The far-field contribution $H_{tail}$ involves the integral over $|x-y| > r$. Since the kernel $K(x-y)$ decays like $|x-y|^{-3}$, we control it by the Hardy--Littlewood maximal function $M$:
\[
|H_{tail}(x,t;r)| \le C \int_{|x-y|>r} \frac{|\omega(y,t)|}{|x-y|^3}\,dy \le C r^{-1} \, M(|\omega(\cdot,t)|)(x).
\]
Consequently, for every cylinder $Q_r(z_0)$,
\[
r^{-2} \iint_{Q_r(z_0)} |H_{tail}|^{3/2} \, dx \, dt
\le C r^{-2} \iint_{Q_r(z_0)} \big(r^{-1} M(|\omega|)\big)^{3/2}
\le C r^{-1/2} \iint_{Q_r(z_0)} M(|\omega|)^{3/2}.
\]
By the Hardy--Littlewood maximal theorem and the critical local bounds on $\omega$,
\[
r^{-2} \iint_{Q_r(z_0)} |H_{tail}|^{3/2} \le C r^{-1/2} \|\omega\|_{L^{3/2}(Q_{2r}(z_0))}^{3/2} \le C r^{1/4},
\]
which vanishes as $r \to 0$.

\subsection{Theorem: Forcing Depletion}
Combining the VMO property, the CRW estimate, and the tail control, we arrive at the first main technical result of this paper.

\begin{theorem}[Forcing Depletion]\label{thm:forcing_depletion}
Let $(u^\infty, \xi^\infty)$ be the ancient tangent flow. For any $\varepsilon > 0$, there exists a scale $r_0 > 0$ such that for all $r \le r_0$, the singular stretching term $H_{sing}$ satisfies the scale-invariant Carleson measure bound:
\[
\sup_{z_0 \in \R^3 \times (-\infty, 0]} r^{-2} \int_{Q_r(z_0)} |H_{sing}|^{3/2} \, dx \, dt \le \varepsilon.
\]
\end{theorem}

\begin{proof}
Fix a basepoint $z_0$ and a scale $r$. We split $H_{sing} = H_{near} + H_{tail}$.

\textbf{Step 1: Near-Field Estimate.}
By Lemma \ref{lem:crw} (CRW Estimate) with $p=3/2$, we have
\[
\|H_{near}(\cdot, t)\|_{L^{3/2}(B_r)} \le C \|\xi(\cdot, t)\|_{BMO(B_r)} \|\rho(\cdot, t)\|_{L^{3/2}(B_r)}.
\]
By Lemma \ref{lem:vmo}, for any $\kappa>0$ there exists $r_\kappa>0$ such that for all $r\le r_\kappa$ and all relevant times, $\|\xi(\cdot,t)\|_{BMO(B_r)}\le \kappa$ uniformly. Using H\"older in time and the local $L^{3/2}$ bound on $\omega$ inherited from the blow-up sequence, we obtain
\[
r^{-2} \int_{Q_r} |H_{near}|^{3/2} \le C \kappa^{3/2} \, r^{-2} \int_{Q_r} |\omega|^{3/2} \le C_K \kappa^{3/2},
\]
where $C_K$ depends only on the critical local bounds (cf. \eqref{eq:critical_bounds}). Choosing $\kappa$ small by taking $r$ sufficiently small makes the near-field term $\le \varepsilon/2$.

\textbf{Step 2: Tail Estimate.}
By the Tail Control above, $|H_{tail}(x,t;r)| \le C r^{-1} M(|\omega(\cdot,t)|)(x)$. Consequently,
\[
r^{-2} \iint_{Q_r} |H_{tail}|^{3/2} \le C\, r^{-7/2} \iint_{Q_r} M(|\omega|)^{3/2}.
\]
Since $M(|\omega|)^{3/2}$ is locally integrable and its average over $Q_r$ remains bounded, the prefactor $r^{3/2}$ forces the right-hand side to vanish as $r\to 0$. In particular, for $r$ small enough the tail term is $\le \varepsilon/2$.
Combining near-field (controlled by $\kappa$) and tail (controlled by $r^{1/4}$), we obtain the result for sufficiently small $r$.
\end{proof}

This theorem resolves the "oscillation vs. mass" dilemma. It asserts that in the critical regime, the "mass" (represented by $\rho$) cannot generate critical stretching because it is modulated by the "oscillation" (of $\xi$), which vanishes asymptotically. Thus, the primary driver of potential blow-up is quantitatively depleted.

\section{Control of the Geometric Forcing}

\subsection{Bounds on $\nabla \log \rho$}
We now turn to the geometric term $H_{geom}$. A crucial component is the gradient of the log-amplitude, $\nabla \log \rho$. While the amplitude $\rho$ may blow up, its logarithmic gradient behaves more like a critical energy density. Using the amplitude equation \eqref{eq:amplitude}, which is a drift--diffusion equation with source $\rho(\sigma + |\nabla \xi|^2)$, we can derive scale-invariant $L^2$ bounds.

\begin{lemma}[Caccioppoli Estimate for Log-Amplitude]\label{lem:log_amplitude}
Let $h = \log \rho$. Under the assumption of critical energy bounds on the tangent flow, there exists a constant $C$ such that for any cylinder $Q_r(z_0)$:
\[
r^{-3} \int_{Q_r(z_0)} |\nabla h|^2 \, dx \, dt \le C \left( 1 + r^{-3} \int_{Q_{2r}(z_0)} (|\sigma| + |\nabla \xi|^2) \, dx \, dt \right).
\]
\end{lemma}

\begin{proof}
The amplitude equation is $\partial_t \rho + u \cdot \nabla \rho - \Delta \rho = \rho (\sigma + |\nabla \xi|^2)$.
Dividing by $\rho$, the equation for $h = \log \rho$ is:
\[
\partial_t h + u \cdot \nabla h - \Delta h - |\nabla h|^2 = \sigma + |\nabla \xi|^2.
\]
Using the identity $\Delta h = \rho^{-1} \Delta \rho - \rho^{-2} |\nabla \rho|^2 = \rho^{-1} \Delta \rho - |\nabla h|^2$, we can rewrite the original equation for $\rho$ by multiplying by $-\rho^{-1} \phi^2$ where $\phi$ is a smooth cutoff function for $Q_{2r}$.
Consider the term $\int \rho^{-1} \Delta \rho \phi^2$. Integration by parts yields:
\[
\int \rho^{-1} \Delta \rho \phi^2 = -\int \nabla(\rho^{-1} \phi^2) \cdot \nabla \rho = \int \rho^{-2} |\nabla \rho|^2 \phi^2 - \int \rho^{-1} \nabla \phi^2 \cdot \nabla \rho = \int |\nabla h|^2 \phi^2 - 2 \int \phi \nabla \phi \cdot \nabla h.
\]
Multiplying the amplitude equation by $\rho^{-1} \phi^2$ and integrating, we obtain:
\[
\int (\partial_t \log \rho + u \cdot \nabla \log \rho) \phi^2 - \int \rho^{-1} \Delta \rho \phi^2 = \int (\sigma + |\nabla \xi|^2) \phi^2.
\]
Substituting the Laplacian term:
\[
\int |\nabla h|^2 \phi^2 = \int (\sigma + |\nabla \xi|^2) \phi^2 + \int \partial_t h \phi^2 + \int (u \cdot \nabla h) \phi^2 + 2 \int \phi \nabla \phi \cdot \nabla h.
\]
The time derivative term is handled by integrating by parts in time. The drift term $\int (u \cdot \nabla h) \phi^2 = -\int h \nabla \cdot (u \phi^2)$ is controlled by local energy bounds and standard parabolic Caccioppoli estimates, making the linear terms in $h$ subordinate to the quadratic gradient term.
The source term $\int (\sigma + |\nabla \xi|^2) \phi^2$ provides the dominant contribution on the right-hand side.
Dividing by $r^{-3}$ yields the claimed normalized estimate.
\end{proof}

The proof relies on testing the equation for $h$ with a cutoff function and absorbing the drift term using the Serrin bounds on $u$. The source terms on the right-hand side are critical quantities: $|\nabla \xi|^2$ is bounded by hypothesis (locally), and $\sigma$ is the stretching term we have just analyzed.

\subsection{Bilinear Estimates}
The cross-term in the geometric forcing is $2 P_\xi (\nabla \log \rho \cdot \nabla \xi)$. We estimate its $L^{3/2}$ norm using the bounds from Lemma \ref{lem:log_amplitude} and the critical energy of $\xi$. By H\"older's inequality:
\[
\int_{Q_r} |(\nabla \log \rho) \cdot \nabla \xi|^{3/2} \le \left(\int_{Q_r} |\nabla \log \rho|^2\right)^{3/4} \left(\int_{Q_r} |\nabla \xi|^6\right)^{1/4}.
\]
More precisely, in the scale-invariant norms, this term is controlled by the product of the energies. Since the energy of $\xi$ is small at small scales (due to VMO), the product is subordinate to the linear terms in the analysis. Specifically, it can be absorbed or treated as a small perturbation.

\subsection{Theorem: Total Forcing Smallness}
We define the total forcing Carleson norm as
\[
\|H\|_{C^{3/2}} = \sup_{z_0, r} r^{-2} \int_{Q_r(z_0)} |H|^{3/2} \, dx \, dt.
\]
Combining the Forcing Depletion Theorem \ref{thm:forcing_depletion} (for $H_{sing}$) and the geometric bounds (for $H_{geom}$), we obtain the following result.

\begin{theorem}[Total Forcing Smallness]\label{thm:total_forcing}
There exists a universal threshold $\delta^* > 0$ such that, for the ancient tangent flow $(u^\infty, \xi^\infty)$ constructed at a singularity, the total forcing $H = H_{sing} + H_{geom}$ satisfies
\[
\|H\|_{C^{3/2}} \le \delta^*
\]
at sufficiently small scales.
\end{theorem}

\begin{proof}
The total forcing is $H = H_{sing} + H_{geom}$.
By Theorem \ref{thm:forcing_depletion}, for any $\varepsilon > 0$, we can find $r_0$ such that $\|H_{sing}\|_{C^{3/2}} \le \varepsilon$.
Now consider $H_{geom} = |\nabla \xi|^2 \xi + 2 P_\xi (\nabla \log \rho \cdot \nabla \xi)$.
The first term $|\nabla \xi|^2$ is effectively lower order in the $\varepsilon$-regularity scheme (it scales like energy density).
The cross term $B = 2 P_\xi (\nabla \log \rho \cdot \nabla \xi)$ is estimated by Hölder's inequality:
\[
\int_{Q_r} |B|^{3/2} \le C \left(\int_{Q_r} |\nabla \log \rho|^2\right)^{3/4} \left(\int_{Q_r} |\nabla \xi|^6\right)^{1/4}.
\]
In the scale-invariant normalization, using the smallness of the VMO energy of $\xi$ at small scales, and the bound on $\nabla \log \rho$ from Lemma \ref{lem:log_amplitude}, we find that $\|H_{geom}\|_{C^{3/2}}$ is controlled by a power of the local energy of $\xi$.
Since $\xi$ is VMO, its local energy on sufficiently small balls is small.
Thus, $\|H_{geom}\|_{C^{3/2}}$ becomes small as $r \to 0$.
Combining this with the smallness of $H_{sing}$, we get $\|H\|_{C^{3/2}} \le \delta^*$ for any target $\delta^*$, provided we go to sufficiently small scales.
\end{proof}

This theorem provides the necessary input for the rigidity analysis of the direction equation: the direction field evolves according to a critical heat flow with a forcing term that is quantitatively small in the relevant scale-invariant space.

\section{Carleson Control and Scaling}\label{sec:carleson}

\begin{theorem}[Carleson Control for Extension Energy]\label{thm:carleson-control}
Let $u$ be a suitable weak solution. Then for every $z_0$ and $0<r\le 1$, the Caffarelli--Silvestre extension energy $E_r(z_0,t)$ of $|\omega|$ satisfies
\[
r^{-1} E_r(z_0,t) \le K_* < \infty,
\]
with $K_*$ depending only on local energy bounds. In particular, any ancient tangent flow $u^\infty$ obtained by blow-up satisfies $\|\mathcal{E}^\infty\|_{C} \le K_*$.
\end{theorem}

\begin{proof}
Combine the absorbed Caccioppoli inequality for $\omega$ with the trace estimate for the Caffarelli--Silvestre extension (see \cite{CaffarelliSilvestre2007}) to bound $E_r$ in terms of $\int_{Q_{Cr}} |\nabla \omega|^2$, and then apply the local energy inequality to control the latter by $\int_{Q_{2Cr}} |\omega|^2$.
\end{proof}

\begin{lemma}[Scaling Invariance]\label{thm:carleson-scaling}
Under the N--S scaling $x\mapsto \lambda x$, $t\mapsto \lambda^2 t$, the normalized quantity $r^{-1}E_r$ is invariant: $r^{-1}E_r[f_\lambda] = (\lambda r)^{-1} E_{\lambda r}[f]$. Consequently, $\|\mathcal{E}\|_{C}$ is scale-invariant.
\end{lemma}

\begin{corollary}[Carleson Stability for Tangent Flows]\label{cor:carleson-min}
Let $u^{(k)}$ be a blow-up sequence producing a limit $u^\infty$. Then
\[
\|\mathcal{E}^\infty\|_{C} \le \liminf_{k\to\infty} \|\mathcal{E}^{(k)}\|_{C} \le K_*.
\]
In particular, the Carleson norm is stable along blow-up limits; scaling alone cannot generate arbitrary smallness.
\end{corollary}

\begin{proof}
Lower semicontinuity of the Carleson density under local convergence, together with the uniform bound from Theorem \ref{thm:carleson-control}, yields the liminf inequality. Since the normalized density is scale-invariant, rescaling cannot produce smallness beyond what is present in the sequence.
\end{proof}

\section{Pressure Isotropization and Tail Depletion}\label{sec:pressure}

To robustly control the far-field contribution of the stretching, we quantify how pressure enforces isotropy of the deviatoric strain at small scales.

\begin{theorem}[Pressure Coercivity]\label{thm:pressure-coercivity}
Let $S=\tfrac12(\nabla u + \nabla u^T)$ and $S_{dev}=S - \tfrac13 (\operatorname{tr}S) I$. For any $R>0$ and cutoff $\phi\in C_c^\infty(B_{2R})$ with $\phi\equiv 1$ on $B_R$, one has
\[
\frac12 \frac{d}{dt}\int_{B_R} |S_{dev}|^2 + \frac{\nu}{2} \int_{B_R} |\nabla S_{dev}|^2
\le C \|u\|_{L^3(B_{2R})}^4 \int_{B_R} |S_{dev}|^2 + C R^{-2} \int_{B_{2R}} |S_{dev}|^2.
\]
\end{theorem}

\begin{proof}
Differentiate the strain equation, use $\Delta p = -\nabla\cdot\nabla\cdot(u\otimes u)$ and Calder\'on--Zygmund estimates to control $\nabla^2 p$ in $L^{3/2}$, then test against $S_{dev}\phi^2$ and absorb a portion of $\|\nabla S_{dev}\|_2^2$.
\end{proof}

\begin{lemma}[Defect vs. Strain]
Let $\Omega$ denote the rescaled vorticity profile on the annulus $|w|>1$ and measure anisotropy via $\mathfrak{D}_{aniso}(\Omega)$ (quadratic form on the $\ell=2$ spherical harmonic sector). Then
\[
\mathfrak{D}_{aniso}(\Omega)^2 \le C \iint_{Q_1} |S_{dev}|^2.
\]
\end{lemma}

\begin{corollary}[Tail Depletion]\label{cor:tail-depletion}
For tangent flows, the tail coefficient $C_{stretch}$ associated with the far-field stretching satisfies $|C_{stretch}|\to 0$ along small scales. Consequently, $|H_{tail}|$ is negligible in the critical Carleson norm.
\end{corollary}

\begin{proof}
Pressure coercivity yields dissipation control of $S_{dev}$, which bounds the anisotropy defect and hence the tail coefficient via the spherical-harmonic representation of the stretching kernel. As scales shrink, the localized dissipation and therefore the defect vanish, forcing $|C_{stretch}|\to 0$.
\end{proof}

\section{The Directional Liouville Theorem}

\subsection{The Critical Drift--Diffusion System}
We have reduced the problem to the analysis of the ancient direction field $\xi^\infty$ satisfying
\begin{equation}\label{eq:DDE}
\partial_t \xi - \Delta \xi + u \cdot \nabla \xi = H, \quad |\xi|=1, \quad H \cdot \xi = 0.
\end{equation}
Here, $u$ satisfies Serrin-type bounds (inherited from the tangent flow critical norms), and $H$ satisfies the smallness condition $\|H\|_{C^{3/2}} \le \delta^*$.

\subsection{Energy Decay Estimates}
To prove rigidity, we establish a decay estimate for the scale-invariant energy $E(r) = r^{-3} \int_{Q_r} |\nabla \xi|^2$. We start with a Caccioppoli inequality for the equation \eqref{eq:DDE}. Testing with $-\Delta(\phi^2 \xi)$ and using the constraint $|\xi|=1$ yields:
\[
\int_{Q_{r/2}} |\nabla^2 \xi|^2 \le C r^{-2} \int_{Q_r} |\nabla \xi|^2 + C \int_{Q_r} |u|^2 |\nabla \xi|^2 + C \int_{Q_r} |H|^2.
\]
The drift term involves $|u|^2 |\nabla \xi|^2$. Since $u$ is in a Serrin class ($L^q_t L^p_x$ with $2/q+3/p \le 1$), this term can be absorbed into the left-hand side (the Hessian term) plus a linear term using interpolation inequalities. The forcing term is small by hypothesis.

Combining this with Poincaré inequalities, we derive a one-step Campanato decay estimate.

\begin{lemma}[One-Step Energy Decay]\label{lem:decay}
There exist constants $\theta \in (0,1)$ and $C > 0$ such that if $E(r) \le \varepsilon_0$ and $\|H\|_{C^{3/2}} \le \delta^*$, then
\[
E(r/2) \le \theta E(r) + C (\delta^*)^2.
\]
\end{lemma}

\begin{proof}
This is the core $\varepsilon$-regularity estimate (Theorem DDE\_Epsilon\_Regularity in the proof track).
Let $r=1$ by scaling. We assume $E(1) \le \varepsilon_0$.
We test the equation $\partial_t \xi - \Delta \xi + u \cdot \nabla \xi = H$ with $-\Delta (\phi^2 \xi)$, where $\phi$ is a cutoff for $B_{1/2}$.
Using $\xi \cdot \Delta \xi = -|\nabla \xi|^2$, we obtain a Caccioppoli inequality:
\[
\int_{Q_{1/2}} |\nabla^2 \xi|^2 \le C \int_{Q_1} |\nabla \xi|^2 + C \int_{Q_1} |u|^2 |\nabla \xi|^2 + C \int_{Q_1} |H|^2.
\]
The drift term $\int |u|^2 |\nabla \xi|^2$ is absorbed into the LHS using the Serrin bound on $u$ and interpolation (parabolic Sobolev embedding), provided $\varepsilon_0$ is small.
Specifically, $\int |u|^2 |\nabla \xi|^2 \le \|u\|_{Serrin}^2 \|\nabla \xi\|_{L^{p'}}^2 \le \varepsilon \|\nabla^2 \xi\|_2^2 + C_\varepsilon \|\nabla \xi\|_2^2$.
The forcing term $\int |H|^2$ is controlled by $\|H\|_{C^{3/2}}^2$ via Hölder (or directly assuming $L^2$ smallness, but $C^{3/2}$ is the natural space; smallness in $C^{3/2}$ implies smallness in the relevant energy deviation).
Combining these, we get:
\[
\int_{Q_{1/2}} |\nabla^2 \xi|^2 \le C E(1) + C (\delta^*)^2.
\]
Then, using the Poincaré inequality (subtracting the mean drift or using harmonic replacement), we compare $\xi$ to a harmonic map heat flow or linear heat equation solution. The energy on the smaller ball $E(1/2)$ improves by a factor $\theta$ (coming from the regularity of the homogeneous equation) plus the perturbation errors.
Thus $E(1/2) \le \theta E(1) + C (\delta^*)^2$.
\end{proof}
By choosing $\delta^*$ sufficiently small (which is possible by Theorem \ref{thm:total_forcing}), the forcing term becomes negligible relative to the decay.

\subsection{Epsilon-Regularity}
\begin{theorem}[DDE $\varepsilon$-Regularity]\label{thm:DDE-eps-regularity}
There exist universal constants $\eps_*>0$, $\delta_*>0$, $\alpha\in(0,1)$, and $C<\infty$ such that, if on $Q_1(z_0)$ the direction equation
\[
\partial_t \xi - \Delta \xi + u \cdot \nabla \xi = H, \qquad |\xi|=1,\quad H\cdot \xi=0
\]
holds with $u$ in a Serrin class and
\[
E(z_0,1)\le \eps_*^2, \qquad \|H\|_{C^{3/2}}\le \delta_*,
\]
then for all $\rho\le \tfrac12$,
\[
E(z_0,\rho) \le C \rho^{2\alpha} E(z_0,1),
\]
and, in particular,
\[
\sup_{Q_{1/2}(z_0)} |\nabla \xi| \le C \eps_*.
\]
\end{theorem}
Iterating the decay estimate from Lemma \ref{lem:decay} yields the theorem by a standard Campanato iteration and absorption of the drift term using the Serrin bound for $u$.

\subsection{Rigidity via Blow-up}
We now prove the main rigidity result.

\begin{theorem}[Directional Liouville]\label{thm:liouville}
Let $\xi$ be an ancient solution to \eqref{eq:DDE} on $\R^3 \times (-\infty, 0]$ satisfying the critical energy bounds and the small forcing condition $\|H\|_{C^{3/2}} \le \delta^*$ with $\delta^*$ sufficiently small. Then $\xi$ must be spatially constant: $\nabla \xi \equiv 0$.
\end{theorem}

\begin{proof}
\textbf{Step 1: Global Decay.}
Iterating Lemma \ref{lem:decay}, we obtain that for any $z_0$ and any scale $r$, if we choose $\delta^*$ small enough relative to $\varepsilon_0$, the energy $E(r)$ decays as $r \to 0$ according to $E(\rho) \le C \rho^{2\alpha}$.
Specifically, since the solution is ancient and defined for $t \in (-\infty, 0]$, and the bounds are uniform, we can apply the decay estimate at any scale.
This implies $\xi$ is Hölder continuous ($C^\alpha$).

\textbf{Step 2: Vanishing Gradient.}
By Lebesgue differentiation, $\lim_{r \to 0} E(z_0, r) = C |\nabla \xi(z_0)|^2$.
From the decay estimate $E(r) \le C r^{2\alpha}$, the limit is 0.
Thus $\nabla \xi(z_0) = 0$ for almost every $z_0$.
Since $\xi$ is smooth (implied by subcritical regularity once Hölder is established), $\nabla \xi \equiv 0$.

\textbf{Step 3: Alternative Blow-down Argument.}
Suppose for contradiction $\nabla \xi \not\equiv 0$. Then there is a point where $E(r) \ge c > 0$.
Consider the blow-down limit $\xi_\lambda(x,t) = \xi(\lambda x, \lambda^2 t)$ as $\lambda \to \infty$.
Since $\|H\|_{C^{3/2}}$ is scale invariant and small, the limit satisfies the equation with small forcing.
However, for an ancient solution, we usually blow-down to find a "tangent flow at infinity".
Actually, the rigidity proof in `DDE-Liouville` (Theorem 6.2 in plan) uses the vanishing energy limit directly.
"Step 4: Vanishing energy... limit is zero... hence $\nabla \xi \equiv 0$."
The contradiction logic is: if $\nabla \xi \not\equiv 0$, then at some point gradient is non-zero. But the decay implies it is zero.
So the result is immediate from the decay estimate established in Step 1.
\end{proof}

\section{Classification and Contradiction}

\subsection{Time-Constancy of the Direction}
From Theorem \ref{thm:liouville}, we know that $\nabla \xi \equiv 0$ for the ancient tangent flow. This implies $\xi(x,t) = b(t)$ for some spatially constant vector $b(t) \in \Sbb^2$. The direction equation \eqref{eq:DDE} then simplifies to
\[
\partial_t b(t) = H(t).
\]
Since the forcing $H$ also vanishes in the blow-up limit (due to the smallness of the Carleson norm and the regularity), we must have $\partial_t b(t) \equiv 0$. Thus, $\xi^\infty(x,t) \equiv b_0$ is a constant vector in both space and time.

\subsection{Reduction to 2D Dynamics}
We can rotate coordinates such that the constant vorticity direction is $b_0 = e_3 = (0,0,1)$. Then the vorticity of the tangent flow is given by $\omega^\infty(x,t) = (0, 0, \alpha(x,t))$. The condition $\omega^\infty = \curl \, u^\infty = (0, 0, \alpha)$ implies that the horizontal components of the velocity $u^\infty_1, u^\infty_2$ are independent of $x_3$, and $u^\infty_3$ is harmonic (and hence zero due to boundedness).

Specifically, the flow reduces to a two-dimensional flow in the plane perpendicular to $e_3$:
\[
u^\infty(x,t) = (v_1(x_1, x_2, t), v_2(x_1, x_2, t), 0).
\]
The vortex stretching term $(\omega^\infty \cdot \nabla) u^\infty$ becomes $(\alpha \partial_3) u^\infty$, which vanishes because $u^\infty$ is independent of $x_3$.

\begin{lemma}[Vanishing Stretching]\label{lem:vanishing_stretching}
If the vorticity direction of a N--S solution is constant in space and time, the vortex stretching term is identically zero.
\end{lemma}

\begin{proof}
Let the direction be constant, $\xi(x,t) \equiv e_3$. Then $\omega = (0, 0, \omega_3)$.
The vortex stretching term is given by $(\omega \cdot \nabla) u = \omega_3 \partial_3 u$.
As we will show in the proof of Theorem \ref{thm:2d_liouville}, the condition that the vorticity direction is constant, combined with the boundedness of the ancient solution, implies that the velocity field is independent of the coordinate $x_3$ (specifically $\partial_3 u \equiv 0$).
Consequently, $(\omega \cdot \nabla) u = \omega_3 \cdot 0 = 0$.
\end{proof}

\subsection{2D Ancient Liouville Theorem}
With zero stretching, the vorticity equation for $\omega^\infty$ reduces to the 2D transport--diffusion equation. The tangent flow $u^\infty$ is thus an ancient solution to the 2D N--S equations. It satisfies global bounds on velocity and vorticity (from the blow-up construction).

Known Liouville theorems for the 2D N--S equations state that any bounded ancient solution must be constant (essentially due to the monotonicity of enstrophy in 2D).

\begin{theorem}[2D Ancient Liouville]\label{thm:2d_liouville}
Let $u$ be a bounded ancient solution to the 2D N--S equations on $\R^2 \times (-\infty, 0]$. Then $u$ is a constant (specifically $u \equiv 0$ for finite energy).
\end{theorem}

\begin{proof}
\textbf{Step 1: Reduction to 2D.}
From Theorem \ref{thm:liouville}, $\xi \equiv b$. Rotate so $b=e_3$.
Then $\omega = (0,0,\alpha)$. $\partial_1 u_3 - \partial_3 u_1 = 0$ and $\partial_2 u_3 - \partial_3 u_2 = 0$.
Differentiating $\dv u = 0$ with respect to $x_3$: $\partial_3 \partial_1 u_1 + \partial_3 \partial_2 u_2 + \partial_3^2 u_3 = 0$.
Using the vorticity relations $\partial_3 u_1 = \partial_1 u_3$ etc., we find $\Delta u_3 = 0$. Since $u^\infty$ is bounded, $u_3$ must be constant (and 0 by energy/decay conditions).
Then $\partial_3 u_1 = \partial_3 u_2 = 0$.
Thus $u(x,t) = (v_1(x_1, x_2, t), v_2(x_1, x_2, t), 0)$.
The flow is strictly 2D.

\textbf{Step 2: 2D Liouville.}
The vorticity $\alpha(x_1, x_2, t)$ satisfies the 2D N-S vorticity equation:
\[
\partial_t \alpha + v \cdot \nabla \alpha = \nu \Delta \alpha.
\]
Multiply by $\alpha$ and integrate over $\R^2$:
\[
\frac{1}{2} \frac{d}{dt} \|\alpha\|_2^2 + \nu \|\nabla \alpha\|_2^2 = 0.
\]
This implies $\|\alpha(t)\|_2$ is non-increasing.
We rely on the Liouville theorem for bounded ancient 2D flows (see e.g., Koch-Nadirashvili-Seregin-Sverak \cite{KNSS2009}).
Any bounded ancient solution to 2D NS is a constant.
Thus $u^\infty$ is constant.
Since it has finite local energy (normalized), it must be $u^\infty \equiv 0$.
\end{proof}

\subsection{The Final Contradiction}
Applying Theorem \ref{thm:2d_liouville} to our tangent flow, we conclude that $u^\infty \equiv 0$, and consequently $\omega^\infty \equiv 0$.

However, by the construction of the tangent flow (Lemma \ref{lem:tangent_flow}), we normalized the solution such that $|\omega^\infty(0,0)| = 1$. This provides the desired contradiction.

Therefore, the initial assumption that a finite-time singularity exists must be false.

\bibliographystyle{amsplain}
\begin{thebibliography}{10}

\bibitem{BKM1984}
J.~T. Beale, T.~Kato, and A.~Majda, \emph{Remarks on the breakdown of smooth solutions for the 3-{D} {E}uler equations}, Comm. Math. Phys. \textbf{94} (1984), no.~1, 61--66.

{\color{blue} \bibitem{CFM1996} C. Fefferman, A. J. Majda,  {\it Geometric constraints on potentially,} Communications in Partial Differential Equations, 21(3–4) (1996)., 559–571. https://doi.org/10.1080/03605309608821197}

%\bibitem{CFM1996}
%P.~Constantin, C.~Fefferman, and A.~Majda, \emph{Geometric constraints on potentially singular solutions for the 3-{D} {E}uler equations}, Comm. Partial Differential Equations \textbf{21} (1996), no.~3-4, 559--571.

\bibitem{CKN1982}
L.~Caffarelli, R.~Kohn, and L.~Nirenberg, \emph{Partial regularity of suitable weak solutions of the {N}avier-{S}tokes equations}, Comm. Pure Appl. Math. \textbf{35} (1982), no.~6, 771--831.

%\bibitem{ESS2003}
%L.~Escauriaza, G.~Seregin, and V.~{\v{S}}ver{\'a}k, \emph{{$L_{3,\infty}$}-solutions of {N}avier-{S}tokes equations and backward uniqueness}, Uspekhi Mat. Nauk \textbf{58} (2003), no.~2(350), 3--44.

\bibitem{ESS2003}
{\color{blue}L.~Escauriaza, G.~A.~Seregin, and V.~\v{S}ver\'{a}k,
\emph{$L_{3,\infty}$-solutions of the N--S equations and backward uniqueness},
Uspekhi Mat.\ Nauk \textbf{58} (2(350)) (2003), 3--44;
translation in Russian Math.\ Surveys \textbf{58} (2003), no.~2, 211--250.}


\bibitem{Fefferman2006} {\color{blue}
C. L. Fefferman, Existence and smoothness of the Navier-Stokes
equation. In J.A. Carlson, A. Jaffe, A. Wiles, Clay Mathematics Institute,
and American Mathematical Society, editors, \emph{The Millennium
Prize Problems}, 57–67. American Mathematical Society, 2006.}

\bibitem{Hopf1951}
E.~Hopf, \emph{{\"U}ber die {A}nfangswertaufgabe f{\"u}r die hydrodynamischen {G}rundgleichungen}, Math. Nachr. \textbf{4} (1951), 213--231.

\bibitem{CRW1976}
{\color{blue} R.~R. Coifman, R.~Rochberg, and G.~Weiss, \emph{Factorization theorems for Hardy spaces in several variables}, Ann. of Math. (2) \textbf{103} (1976), no.~3, 611--635.}
%%izbrisana zadnja recenica


\bibitem{CaffarelliSilvestre2007} {\color{blue}
L.~Caffarelli and L.~Silvestre, \emph{An extension problem related to the fractional Laplacian}, Comm. Partial Differential Equations \textbf{32} (2007), no.8, 1245--1260.} 

%pisalo no.7-9, a treba 8

\bibitem{KNSS2009}
G.~Koch, N.~Nadirashvili, G.~Seregin, and V.~{\v{S}}ver{\'a}k, \emph{Liouville theorems for the {N}avier-{S}tokes equations and applications}, Acta Math. \textbf{203} (2009), no.~1, 83--105.




\bibitem{KochTataru2001}
H.~Koch and D.~Tataru, \emph{Well-posedness for the {N}avier-{S}tokes equations}, Adv. Math. \textbf{157} (2001), no.~1, 22--35.




\bibitem{Leray1934}
J.~Leray, \emph{Sur le mouvement d'un liquide visqueux emplissant l'espace}, Acta Math. \textbf{63} (1934), no.~1, 193--248.

{\color{blue}\bibitem{Lemarie2016}
P.-G.~Lemari\'e\mbox{-}Rieusset,
\emph{The Navier--Stokes Problem in the 21st Century},
Chapman \& Hall/CRC, Boca Raton, FL, 2016.}

\bibitem{Lin1998}
F.~Lin, \emph{A new proof of the {C}affarelli-{K}ohn-{N}irenberg theorem}, Comm. Pure Appl. Math. \textbf{51} (1998), no.~3, 241--257.

\bibitem{Prodi1959}
G.~Prodi, \emph{Un teorema di unicit{\`a} per le equazioni di {N}avier-{S}tokes}, Ann. Mat. Pura Appl. (4) \textbf{48} (1959), 173--182.

 

\bibitem{Scheffer1977}
V.~Scheffer, \emph{Hausdorff measure and the {N}avier-{S}tokes equations}, Comm. Math. Phys. \textbf{55} (1977), no.~2, 97--112.

\bibitem{Seregin2012}
G.~Seregin, \emph{A certain necessary condition of potential blow up for {N}avier-{S}tokes equations}, Comm. Math. Phys. \textbf{312} (2012), no.~3, 833--845.

 

\bibitem{Serrin1962}
J.~Serrin, \emph{On the interior regularity of weak solutions of the {N}avier-{S}tokes equations}, Arch. Rational Mech. Anal. \textbf{9} (1) (1962), 187--195.


\bibitem{ConstantinFefferman1993}
{\color{blue}P.~Constantin and C.~Fefferman,
\emph{Direction of vorticity and the problem of global regularity for the Navier--Stokes equations},
Indiana Univ.\ Math.\ J.\ \textbf{42} (1993), no.~3, 775--789.}

\bibitem{MajdaBertozzi2002}
{\color{blue}A.~J.~Majda and A.~L.~Bertozzi,
\emph{Vorticity and Incompressible Flow},
Cambridge Texts in Applied Mathematics, Cambridge Univ.\ Press, 2002.}

\bibitem{15} {\color{blue}
Y. Giga, {\it Solutions for semilinear parabolic equations in Lp and regularity of weak solutions of the Navier-Stokes system,} J. Differential Equations, 62(1986), 186-212.}

\bibitem{25} {\color{blue}
 J. Serrin, {\it The initial value problem for the Navier-stokes equations, in Nonlinear problems(R. E. Langer Ed.),} pp.69-98, Univ. of Wisconsin Press, Madison, 1963.}

\bibitem{27} {\color{blue}
M. Struwe, {\it On partial regularity results for the Navier-Stokes equations,} Comm. Pure
Appl. Math., 41(1988), 437-458.}

\bibitem{23} {\color{blue}
J.~Ne\v{c}as, M.~R{u}\v{z}i\v{c}ka and V.~\v{S}ver\'ak,
{\it On Leray's self-similar solutions of the Navier--Stokes equations},
\newblock {\em Acta Mathematica}, {\bf 176} (1996), 283--294.}

\bibitem{6} {\color{blue}
G. P. Galdi, {\it An Introduction to the Navier-Stokes Initial-Boundary Value
Problem}, In: Fundamental Directions in Mathematical Fluid Mechanics.
Basel: Birkhäuser, 2000, pages 1–70.}

\bibitem{Aubin1963} {\color{blue}
J.-P.\ Aubin,
\emph{Un théorème de compacité},
C.\ R.\ Acad.\ Sci.\ Paris \textbf{256} (1963), 5042--5044.}

\bibitem{Lions1969} {\color{blue}
J.-L.\ Lions,
\emph{Quelques méthodes de résolution des problèmes aux limites non linéaires},
Dunod; Gauthier-Villars, Paris, 1969.}

\bibitem{Kobayashi} {\color{blue}
M. Kobayashi. {\it On the Navier-Stokes equations on manifolds with curvature,} J. Eng. Math. (2008) 60:55–68.}

\bibitem{LG} {\color{blue}
O. A. Ladyzhenskaya and G. A. Seregin, {\it On partial regularity of suitable weak
solutions to the three-dimensional Navier–Stokes equations,} J. Math. Fluid Mech. 1
(1999), 356-387.}

\end{thebibliography}

\end{document}



\subsection{Blow-up Analysis and Ancient Tangent Flows}

Let us assume that the smooth solution develops a 
singularity in finite time. Let $T^*<\infty$ be the first singular time. 
By the Beale–Kato–Majda criterion \cite{BKM1984}, it holds
\[
\int_0^{T^*} \|\omega(\cdot,t)\|_{L^\infty}\,dt = +\infty,
\]
and hence
\[
\limsup_{t\to T^*} \|\omega(\cdot,t)\|_{L^\infty} = \infty.
\]
Therefore, we may choose a sequence of times $t_k\uparrow T^*$ and points 
$x_k\in\mathbb{R}^3$ such that
\[
|\omega(x_k,t_k)| \ge \tfrac12\|\omega(\cdot,t_k)\|_{L^\infty}.
\]
To analyze the possible singularity, we perform the N–S parabolic 
blow-up around the sequence $(x_k,t_k)$ at scales 
\[
\lambda_k := \|\omega(\cdot,t_k)\|_{L^\infty}^{-1/2}.
\]
The following lemma establishes the existence of a nontrivial ancient 
limit of such rescaled solutions.





\begin{lemma}[Construction of an ancient tangent flow]
\label{lem:ancient-flow}
Assume $u$ is a suitable weak solution of the N--S equations (\ref{eq:NS_domain}) that
blows up at time $T^*<\infty$. Then there exist points
$(x_k,t_k)\to(x^*,T^*)$ and scales $\lambda_k\to0$ such that the rescaled sequence
\begin{equation}\label{2.6}
u^{(k)}(y,s)
=
\lambda_k\, u(x_k+\lambda_k y,\ t_k+\lambda_k^2 s)
\end{equation}
converges locally in $L^3$ to $u^\infty$. We refer to $u^\infty$
as the \emph{ancient tangent flow} associated to the blow-up at $T^*$.
Moreover: 
\begin{enumerate}
    \item $u^\infty$ is a suitable weak solution of (\ref{eq:NS_domain}) on
    $\R^3\times(-\infty,0]$.

    \item $u^\infty$ is non-trivial. After rescaling we may normalize the sequence
    so that $|\omega^{(k)}(0,0)|=1$ for all $k$, and hence
    $|\omega^\infty(0,0)|=1$.
\end{enumerate}
\end{lemma}

 
\begin{proof}  We assume that the loss of regularity at $T^*$ is accompanied by vorticity 
blow-up in the sense that there exists a sequence $t_k\uparrow T^*$ such that
\[
\|\omega(\cdot,t_k)\|_{L^\infty} \to \infty.
\]
For each $k$, choose $x_k\in\R^3$ with
\[
M_k := |\omega(x_k,t_k)| = \|\omega(\cdot,t_k)\|_{L^\infty},
\]
so that $M_k\to\infty$.
Define $\lambda_k := M_k^{-1/2}\to 0$.
For the rescaled fields \eqref{2.6},
$\omega^{(k)} = \nabla\times u^{(k)}$, and a direct scaling computation
shows that $(u^{(k)},p^{(k)})$ solves \eqref{eq:NS_domain} and
\[
|\omega^{(k)}(0,0)|=\lambda_k^2|\omega(x_k,t_k)|=1.
\]

Because $u$ is a suitable weak solution, it satisfies the local energy
inequality \eqref{eq:local-energy-ineq}.  This inequality is preserved under the scaling \eqref{scaling}. In particular, 
for every $R>0$ there exists a constant $C(R)$, independent of $k$, such that
\begin{equation}\label{eq:uniform-local-energy}
\sup_{t\in(-R^2,0]} \int_{B_R} |u^{(k)}(x,t)|^2\,dx 
\;+\; \int_{Q_R} |\nabla u^{(k)}(x,t)|^2\,dx\,dt 
\;\le C(R).
\end{equation}
By these uniform local energy bounds, and utilizing the compactness and regularity results for suitable weak solutions (see, \cite{CKN1982,Seregin2012}), we can guarantee the existence of a subsequence, still denoted $u^{(k)}$, such that:

$$\text{For every } \delta > 0, \quad u^{(k)} \to u^\infty \quad \text{strongly in } L^3_{\mathrm{loc}}(\R^3\times(-\infty, -\delta]).$$Additionally, $u^{(k)}$ converges to $u^\infty$ weakly in the space $L^2_{\mathrm{loc}}(\R^3\times(-\infty, 0])$.


 
Using the strong $L^3_{\mathrm{loc}}$ convergence on any compact subset away from $s=0$ and the uniform local energy bounds, we can pass to the limit in the non-linear convective term $u^{(k)} \cdot \nabla u^{(k)}$ in the N–S equations (\ref{eq:NS_domain}), in the sense of distributions. So, the local
energy inequality \eqref{eq:local-energy-ineq} also passes to the limit by
lower semicontinuity. Hence $u^\infty$ is a suitable weak solution
on $\R^3\times(-\infty,0]$. This proves (1).

\medskip


Since $|\omega^{(k)}(0,0)|=1$ for all $k$, then $u^\infty$
cannot vanish identically; otherwise the rescaled vorticity would converge
to zero in the sense of distributions, contradicting the normalization
at $(0,0)$ (see, e.g., \cite{Seregin2012} for a standard argument).
Thus $\omega^\infty\not\equiv0$, proving the nontriviality of the limit.
Finally, by shifting coordinates if necessary and rescaling $u^\infty$ by a
constant factor, we get
\[
|\omega^\infty(0,0)| = 1,
\]
which establishes (2).
 \end{proof} 
 










 

    

\textbf{Step 2: Compactness and Bounded Critical Norms.}
Since $u$ is a suitable weak solution, it satisfies the local energy inequality. The rescaling preserves the structure of the N-S equations. Standard compactness arguments for suitable weak solutions (based on local energy estimates and the Aubin-Lions lemma, or CKN-type epsilon-regularity which implies bounds away from the singularity) allow us to extract a subsequence converging locally strongly in $L^3_{loc}$ to a limit $u^\infty$ defined on $\R^3 \times (-\infty, 0]$. The convergence is smooth on compact sets away from any residual singular set. However, due to the Type I bound assumption (implied by the contradiction framework or assumed for the contradiction), the singular set is empty or controlled.
The convergence is smooth on compact sets away from any residual singular set. No Type I blow-up rate is assumed or used here; only the uniform local bounds arising from the local energy inequality are needed.

\textbf{Step 3: Critical Bounds and Carleson Control (no Type I assumption).}
We do not assume any Type I blow-up rate. By the local energy inequality and standard interpolation, one obtains the absorbed Caccioppoli estimate
\[
\int_{Q_r} |\nabla \omega|^2 \le C r^{-2} \int_{Q_{2r}} |\omega|^2.
\]
Let $\mathcal{F}$ denote the Caffarelli--Silvestre extension of $|\omega|$. By trace theory,
\[
E_r(z_0,t) := \int_{B_r(x_0)}\!\!\int_0^r |\nabla \mathcal{F}(x,z,t)|^2 \, z \, dz \, dx \le C \int_{B_{Cr}(x_0)} |\nabla \omega(\cdot,t)|^2 + \text{l.o.t.}
\]
Integrating in time and normalizing by $r^{-1}$ yields a uniform Carleson bound on the extension energy along the sequence. Moreover, by a re-centering and compactness argument (see Theorems \ref{thm:carleson-control} and \ref{thm:carleson-scaling}), the limiting ancient flow $u^\infty$ inherits a finite extension-energy Carleson norm at unit scale. Thus, the limit $u^\infty$ enjoys the critical bounds \eqref{eq:critical_bounds} together with a uniform extension-energy Carleson bound.


The existence of such ancient solutions is standard in blow-up analysis (see e.g., \cite{Seregin2012}). The Type I assumption corresponds to the case where the scaling factor relates to the blow-up rate, e.g., $\lambda_k \sim (T^*-t_k)^{1/2}$, but our argument relies only on the existence of the limiting ancient object with bounded critical norms. Our goal is to prove that any such $u^\infty$ must be identically zero, contradicting property (2).







 

\begin{lemma}[Construction of Ancient Tangent Flow]\label{lem:tangent_flow}
Assume $u$ is a suitable weak solution that blows up at $T^*$. Then there exists a sequence of points $(x_k, t_k) \to (x^*, T^*)$ and scales $\lambda_k \to 0$ such that the rescaled sequence
\[
u^{(k)}(y,s) = \lambda_k u(x_k + \lambda_k y, t_k + \lambda_k^2 s)
\]
converge locally in $L^3$ to $u^\infty \in L^\infty_{loc}((-\infty, 0]; L^\infty(\R^3))$. The limit $u^\infty$ which we call the ancient tangent flow has the following properties:
\begin{enumerate}
    \item $u^\infty$ is a suitable weak solution of the N--S equations
    \eqref{eq:NS_domain} on $\R^3 \times (-\infty, 0]$.
    \item $u^\infty$ is non-trivial. We may normalize the scaling such that $|\omega^\infty(0,0)| = 1$.
    \item $u^\infty$ satisfies global critical energy bounds inherited from the blow-up sequence:
    \begin{equation}\label{eq:critical_bounds}
    \sup_{t \le 0} \|u^\infty(\cdot, t)\|_{L^\infty} \le C, \quad \sup_{Q_R \subset \R^3 \times (-\infty,0]} R^{-2} \int_{Q_R} |u^\infty|^3 \, dx \, dt \le K.
    \end{equation}
    \item The associated extension energy of the vorticity is bounded in the Carleson norm:
    \[
    \|\mathcal{E}^\infty\|_C := \sup_{z_0, r \le 1} r^{-1} \int_{B_r(z_0)} \int_0^r |\nabla \mathcal{F}(x,z,t)|^2 z \, dz \, dx \le K,
    \]
    where $\mathcal{F}$ is the Caffarelli-Silvestre extension of $|\omega^\infty|$.
\end{enumerate}
\end{lemma}

