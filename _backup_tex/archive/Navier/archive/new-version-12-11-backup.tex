\documentclass[12pt, reqno]{amsart}

%% PACKAGES
\usepackage{amsmath, amssymb, amsthm, amsfonts}
\usepackage{mathrsfs}
\usepackage{mathtools}
\usepackage{enumerate}
\usepackage{geometry}
\usepackage{color}
\usepackage{url}

%% GEOMETRY
\geometry{margin=1.in}

\usepackage[colorlinks=true, linkcolor=blue, citecolor=blue, urlcolor=blue]{hyperref}
\setcounter{tocdepth}{2}

%% THEOREMS
\newtheorem{theorem}{Theorem}[section]
\newtheorem{lemma}[theorem]{Lemma}
\newtheorem{proposition}[theorem]{Proposition}
\newtheorem{corollary}[theorem]{Corollary}
\newtheorem{conjecture}[theorem]{Conjecture}
{\color{magenta}\newtheorem{assumption}[theorem]{Assumption}}

\theoremstyle{definition}
\newtheorem{definition}[theorem]{Definition}
\newtheorem{remark}[theorem]{Remark}
\newtheorem{example}[theorem]{Example}

%% NUMBERING
\numberwithin{equation}{section}

%% MACROS
\newcommand{\R}{\mathbb{R}}
\newcommand{\N}{\mathbb{N}}
\newcommand{\C}{\mathbb{C}}
\newcommand{\Z}{\mathbb{Z}}
\newcommand{\T}{\mathbb{T}}
\newcommand{\Sbb}{\mathbb{S}}

\newcommand{\dv}{\mathrm{div}}
\newcommand{\curl}{\mathrm{curl}}
\newcommand{\supp}{\mathrm{supp}}
\newcommand{\osc}{\mathrm{osc}}
\newcommand{\BMO}{\mathrm{BMO}}
\newcommand{\VMO}{\mathrm{VMO}}

\newcommand{\eps}{\varepsilon}
\newcommand{\om}{\omega}
\newcommand{\Om}{\Omega}
\newcommand{\xihat}{\hat{\xi}}
\newcommand{\lambdar}{\Lambda_r}
\usepackage{xcolor}


%% TITLE & AUTHOR
%\title[Global Regularity for Navier--Stokes]{Global Regularity for the 3D Incompressible Navier--Stokes Equations via Geometric Depletion}
\title[Geometric Depletion Mechanisms]{ Geometric Depletion Mechanisms in the 3D Incompressible Navier--Stokes Equations}

\author{Jonathan Washburn}
\address{Department of Mathematics} 
\email{jonathan.washburn@example.com} % Placeholder email

%\date{\today}

%% ABSTRACT
\begin{document}

\begin{abstract}
{\color{magenta}\noindent\textbf{Audit status.} This document currently contains a \emph{conditional} reduction of 3D Navier--Stokes global regularity to several scale-critical inputs about ancient tangent flows, isolated explicitly as Assumptions~(A)--(E) in Subsection~\ref{subsec:conditional-inputs}. If those inputs are established unconditionally, the argument would yield global regularity.}

{\color{magenta}Assuming Assumptions~(A)--(E), we show that smooth, finite-energy solutions to the 3D incompressible Navier--Stokes equations on $\mathbb{R}^3$ exist globally in time.}
The proof proceeds by contradiction, analyzing the geometry of a hypothetical finite-time singularity. We introduce the method of \emph{geometric depletion}, which reduces the analysis to the evolution of the vorticity direction field $\xi = \omega/|\omega|$. 

First, we establish a critical coercivity estimate for the singular integral stretching term, showing that the nonlinear stretching is depleted in the presence of small directional oscillation. Second, we reduce the contradiction step to a Liouville-type rigidity statement for the resulting critical drift--diffusion equation satisfied by $\xi$; under a small Carleson-measure forcing hypothesis, this forces $\xi$ to have constant direction field. This reduction forces the flow to be structurally two-dimensional, for which global regularity is known, thereby contradicting the existence of a singularity.
\end{abstract}

\maketitle

\tableofcontents

\section{Introduction}

{\color{blue}
\subsection{Motivation} The question of global regularity for the 3D incompressible Navier–Stokes equations remains one of the central open problems in mathematical fluid dynamics. Understanding whether finite–time singularities may arise from smooth initial data is crucial both for the analytical structure of the equations and for the predictive reliability of the physical models they describe. The system governs the motion of a viscous, incompressible fluid with constant density and follows from the conservation of linear momentum and mass. The foundational mathematical theory was established by J. Leray~\cite{Leray1934} and E. Hopf~\cite{Hopf1951}, who introduced the notion of weak solutions and established global existence via the fundamental energy inequality. However, the questions of uniqueness and regularity for such weak solutions remain unresolved.

\smallskip



The incompressible Navier--Stokes equations arise from the fundamental principles of 
mass and momentum conservation applied to a viscous fluid treated as a continuum. 
Under the continuum hypothesis, the velocity $u(t,x)$ and pressure $p(t,x)$ are 
well-defined, smoothly varying fields describing, respectively, the instantaneous 
velocity of a fluid parcel and the normal force exerted by the surrounding fluid. 
The condition $\nabla \cdot u = 0$ reflects conservation of mass for a homogeneous, 
incompressible fluid, while the momentum equation expresses Newton’s second law, i.e.
the material acceleration $\frac{D u}{Dt} = \partial_t u + (u \cdot \nabla)u$ is 
balanced by the pressure gradient $-\nabla p$, the viscous diffusion term $\nu \Delta u$ 
arising from internal friction in a Newtonian fluid, and possible external forces $f$. 

In 3D, 
taking the curl of the momentum equation yields the vorticity formulation, in which 
the term $(\omega \cdot \nabla)u$ (with $\omega=\nabla\times u$) describes vortex 
stretching, a mechanism which does not exist in two dimensions and widely regarded as the key 
process responsible for vorticity amplification, energy cascade to smaller scales, 
and the potential formation of singularities. This vortex-stretching mechanism 
encapsulates the central mathematical difficulty of the Navier--Stokes problem, 
at the same time, the essential physical ingredient underlying the onset of 
turbulence in real viscous flows ~\cite{ConstantinFefferman1993,MajdaBertozzi2002}.

{\color{red} Organization of the paper!!!}
}

{\color{blue}
\subsection{The Navier--Stokes Regularity Problem}



Let $T>0$ be an arbitrary finite number representing the time, and $\nu>0$ a positive number representing the kinematic viscosity.  We consider 3D incompressible Navier--Stokes (N-S) equations given by the following system of PDEs:
\begin{equation}\label{eq:NS_domain}
\begin{cases}
\partial_t u + (u \cdot \nabla)u + \nabla p - \nu \Delta u = f,  \\
\nabla \cdot u = 0,
\end{cases}
\end{equation}
where the vector field $u: \R^3 \times [0,T) \to \R^3$ denotes the velocity, and 
$p: \R^3 \times [0,T) \to \R$ denotes the scalar pressure.  

We assume that the external force $f = 0$, but all results can be easily extended to the case of a non-vanishing external force by incorporating $f$ through the Duhamel integral \cite[Proposition~6.1]{Lemarie2016}, under the standard admissibility assumptions on $f$ (e.g. $f \in L^1_{loc}([0,T);L^2(\mathbb{R}^3))$).

We assume the initial data $u(x,0)=u_0(x) \in H^1(\R^3)$ is smooth and divergence-free.
Given such smooth initial data, the fundamental question,  identified as one of the Millennium Prize Problems \cite{Fefferman2006}, is whether such solutions remain smooth for all time $T > 0$, or whether a finite-time singularity can form.



The modern theory of weak solutions to the N--S equations originates 
from the works of J.~Leray~\cite{Leray1934} and E.~Hopf~\cite{Hopf1951}. 
They introduced the notion of what is now called a Leray--Hopf weak solution and 
proved the global-in-time existence of such solutions for any divergence-free 
initial data $u_0 \in L^2(\mathbb{R}^3)$. These solutions satisfy the N--S 
equations in the distributional sense together with the fundamental global energy 
inequality
\begin{equation}\label{eq:energy}
\frac{1}{2} \int_{\mathbb{R}^3} |u(x,t)|^2 \, dx
+ \nu \int_0^t \int_{\mathbb{R}^3} |\nabla u(x,s)|^2 \, dx \, ds
\le \frac{1}{2} \int_{\mathbb{R}^3} |u_0(x)|^2 \, dx
\qquad \forall\, t \ge 0.
\end{equation}
Although global existence is guaranteed, the questions of uniqueness and 
spatial--temporal regularity of Leray--Hopf weak solutions remain open. 
This difficulty is tied to the \emph{supercritical} nature of the nonlinearity 
$(u\cdot\nabla)u$ with respect to the natural dissipation $\nu\Delta u$ under the 
N--S scaling, and motivates the development of refined regularity 
criteria and the introduction of the stronger class of suitable weak solutions.







The N--S equations (\ref{eq:NS_domain}) are invariant under the scaling
\begin{equation}\label{scaling}
    u_\lambda(x,t) = \lambda\, u(\lambda x, \lambda^2 t), 
\qquad
p_\lambda(x,t) = \lambda^2\, p(\lambda x, \lambda^2 t),
\end{equation}
but this transformation maps the energy norm 
$\|u\|_{L^\infty_t L^2_x}$ to $\lambda^{-1/2}\|u\|_{L^\infty_t L^2_x}$, 
making the energy strictly supercritical (too weak to control the nonlinearity).



The underlying physical space is tacitly assumed to be flat, which is the natural assumption for the study of the flow in our
3D Euclidean space. 

M. Kobayashi \cite{Kobayashi} extends the Navier–Stokes equations from flat spaces to manifolds by analyzing the motion of a Newtonian fluid on flow leaves, that is, smooth surfaces in Euclidean three-space that are invariant under the fluid flow. The proposed general equations describing the motion of a Newtonian fluid 
with constant properties on a volume Riemannian manifold $(M,g,\omega)$ are:

\begin{enumerate}
    \item Continuity equation:
    \[
        \operatorname{div}_{\omega} u = 0,
    \]

    \item {N--S equation:}
    \[
        \frac{\partial u}{\partial t} 
        + \nabla_u u
        = -\frac{1}{\rho}\,\operatorname{grad} p
        - \nu\left( \nabla^\ast \nabla u + \mathrm{Ric}(u)
        - \mathcal{L}_{\operatorname{grad}\log \omega} u \right)
        + b.
    \]
\end{enumerate}
Here $\operatorname{div}_{\omega}$ and $\operatorname{grad}$ denote divergence and gradient 
taken with respect to the volume form $\omega$, 
$\nabla^\ast \nabla$ is the Hodge--de\,Rham Laplacian on vector fields, 
$\mathrm{Ric}(u)$ is the Ricci curvature acting on $u$, 
and $\mathcal{L}_{\operatorname{grad}\!\log \omega}\,u$ represents the non--Riemannian 
correction arising from the volume form. It is shown how quantities intrinsic to the manifold, such as curvature and the choice of volume form, fundamentally modify the structure of the equations and the resulting flow behavior.
}


\subsection{Historical Context and Barriers}
Substantial progress has been made in understanding the partial regularity of suitable weak solutions. Scheffer \cite{Scheffer1977} and Caffarelli, Kohn, and Nirenberg \cite{CKN1982} proved that the singular set of any suitable weak solution has one-dimensional parabolic Hausdorff measure zero. Lin \cite{Lin1998} simplified and refined these results. These partial regularity theorems rely on $\varepsilon$-regularity criteria: if scale-invariant quantities (such as $\|u\|_{L^3}$ or $\|u\|_{L^\infty_t L^{3,\infty}_x}$) are locally small, the solution is regular.

Complementing the partial regularity theory are blow-up criteria. The celebrated Beale--Kato--Majda (BKM) criterion \cite{BKM1984} states that a smooth solution blows up at time $T^*$ if and only if
\begin{equation}\label{eq:BKM}
\int_0^{T^*} \|\omega(\cdot,t)\|_{L^\infty} \, dt = \infty,
\end{equation}
where $\omega = \curl \, u$ is the vorticity. Serrin \cite{Serrin1962} and Prodi \cite{Prodi1959} established that if $u \in L^q(0,T; L^p(\R^3))$ with $2/q + 3/p \le 1$ ($p > 3$), then the solution is regular. The endpoint case $L^\infty_t L^3_x$ was resolved by Escauriaza, Seregin, and \v{S}ver\'ak \cite{ESS2003}.

Despite these advances, the "scaling gap" remains. All known regularity criteria require bounds at the critical scaling level (e.g., $L^3$ velocity or $L^{3/2}$ vorticity), whereas the a priori energy bounds control only subcritical quantities (e.g., $L^2$ velocity). Bridging this gap requires exploiting the structure of the nonlinearity beyond simple scaling arguments.

{\color{blue}\subsection{Main Result}
We provide a new geometric decomposition of the nonlinear structure 
that targets the scaling gap and separates the controllable 
geometric terms from the critical singular interaction. This leads to a refined geometric regularity criterion for the 
3D N–S equations, aligned with the critical scaling 
and sensitive to vorticity–direction oscillation.

\begin{theorem}[Conditional Main Theorem]\label{thm:main}
{\color{magenta}
Let $u_0 \in H^1(\R^3)$ be smooth and divergence-free, and let $u$ be the corresponding smooth solution of \eqref{eq:NS_domain} on its maximal interval of existence $[0,T^*)$.
Assume that whenever $T^*<\infty$ and $(x^*,T^*)$ is a CKN-singular point, the associated ancient tangent flow $(u^\infty,p^\infty)$ produced by Lemma~\ref{lem:ancient-limit} satisfies Assumptions~(A)--(E) stated in Subsection~\ref{subsec:conditional-inputs}.
Then $T^*=\infty$ and $u$ extends to a unique global smooth solution on $[0,\infty)$.
}
\end{theorem}

{\color{magenta}\begin{remark}[Audit note]
The remainder of this manuscript can be read as a proof of Theorem~\ref{thm:main} \emph{assuming} the inputs (A)--(E); the deep-audit annotations mark every location where an unproved/non-classical step is currently used.
\end{remark}}

{\color{magenta}\subsection{Conditional inputs (A)--(E) and gap map}\label{subsec:conditional-inputs}
For clarity and referee-checkability, we isolate the \emph{non-classical} steps currently required to turn the manuscript into an unconditional proof.
All later ``gap'' annotations refer back to the items below.

\begin{assumption}[Directional VMO for tangent flows (A)]\label{assump:A-vmo}
Let $(u^\infty,p^\infty)$ be an ancient tangent flow produced by Lemma~\ref{lem:ancient-limit}, with vorticity direction $\xi^\infty=\omega^\infty/|\omega^\infty|$ on $\{\omega^\infty\neq0\}$.
Then $\xi^\infty$ belongs to $\VMO$ in the spatial variable, locally uniformly in time, i.e. for every compact $K\subset\R^3\times(-\infty,0]$,
\[
\lim_{r\to0}\ \sup_{(x,t)\in K}\ \frac{1}{|B_r|}\int_{B_r(x)}\Big|\xi^\infty(y,t)-(\xi^\infty)_{x,r}(t)\Big|\,dy \;=\;0.
\]
\end{assumption}

\begin{assumption}[Scale-critical vorticity control (B)]\label{assump:B-omega32}
There exists $K_0<\infty$ such that every ancient tangent flow $(u^\infty,p^\infty)$ produced by Lemma~\ref{lem:ancient-limit} satisfies the scale-invariant local bound
\[
\sup_{z_0\in\R^3\times(-\infty,0]}\ \sup_{0<r\le1}\ r^{-2}\iint_{Q_r(z_0)} |\omega^\infty|^{3/2}\,dx\,dt \;\le\; K_0.
\]
\end{assumption}
{\color{magenta}\begin{remark}[Audit note: partial progress toward (B)]
If one uses the optional ``running-max'' time choice in Lemma~\ref{lem:blowup-normalization}, then the corresponding vorticity-normalized rescalings satisfy a uniform backward-in-time $L^\infty$ bound on $\omega^{(k)}$, and hence any ancient limit extracted from that \emph{running-max} sequence obeys a scale-critical \(L^{3/2}\) bound (see Lemma~\ref{lem:omega32-runningmax}).
However, the manuscript still needs a bridge showing that the \emph{CKN-anchored} ancient tangent flow of Lemma~\ref{lem:ancient-limit} satisfies such a uniform, scale-critical bound.
\end{remark}}

\begin{assumption}[Directional $\varepsilon$-regularity and Liouville (C)]\label{assump:C-liouville}
Let $\xi$ be an ancient solution of the sphere-valued drift--diffusion equation
\[
\partial_t \xi - \Delta \xi + u\cdot\nabla\xi = |\nabla\xi|^2\xi + H,\qquad |\xi|=1,\qquad H\cdot\xi=0,
\]
on $\R^3\times(-\infty,0]$, where the drift $u$ belongs to a scale-critical class sufficient to close the Caccioppoli/Campanato scheme (as used in Theorem~\ref{thm:DDE-eps-regularity}),
and where the tangential forcing is small in the Carleson norm:
\[
\|H\|_{C^{3/2}}\le \delta_*.
\]
Then $\xi$ is constant in space-time (equivalently, $\nabla\xi\equiv0$ and $\partial_t\xi\equiv0$).
\end{assumption}

\begin{assumption}[Total tangential forcing smallness (D)]\label{assump:D-forcing}
For every ancient tangent flow produced at a CKN-singular point, the tangential forcing $H=H_{\mathrm{sing}}+H_{\mathrm{geom}}$ in \eqref{eq:direction} is small in the critical Carleson norm at sufficiently small scales.
Concretely, there exists a universal $\delta_*>0$ such that for each such tangent flow there exists $r_0>0$ with
\[
\sup_{z_0}\ \sup_{0<r\le r_0}\ r^{-2}\iint_{Q_r(z_0)} |H|^{3/2}\,dx\,dt \;\le\; \delta_*^{3/2}.
\]
\end{assumption}
\textit{Remark.} In the manuscript, smallness of the \emph{near-field commutator/oscillation part} of $H_{\mathrm{sing}}$ is intended to follow from CRW+VMO plus scale-critical vorticity control (cf.\ Theorem~\ref{thm:forcing_depletion}); controlling the remaining constant-direction and tail contributions requires additional input (tail depletion / pressure isotropization). Smallness of $H_{\mathrm{geom}}$ is intended to follow from log-amplitude and direction regularity estimates (cf.\ Lemma~\ref{lem:log_amplitude}). The current text does not yet provide complete proofs of these implications at the level required for an unconditional proof.

\begin{assumption}[2D classification / Liouville class (E)]\label{assump:E-2d}
Whenever an ancient tangent flow $(u^\infty,p^\infty)$ satisfies $\xi^\infty\equiv b_0$ for a constant unit vector $b_0$, the corresponding velocity field $u^\infty$ belongs to a 2D Liouville class (e.g.\ is a bounded ancient solution on $\R^2\times(-\infty,0]$ after reduction) so that the 2D ancient Liouville theorem applies and forces $u^\infty\equiv 0$.
\end{assumption}

\begin{remark}[Gap map / where (A)--(E) are used]
\begin{itemize}
    \item \textbf{(A)} is used to justify smallness of the BMO seminorm of $\xi^\infty$ on small balls (Lemma~\ref{lem:vmo}) and hence the CRW commutator smallness in the near-field forcing estimate.
    \item \textbf{(B)} is used in the near-field and tail bounds for $H_{\mathrm{sing}}$ (H\"older in time and maximal-function control) whenever scale-invariant \(L^{3/2}\) bounds on $\omega$ are invoked. {\color{magenta}A scale-critical bound holds for any running-max vorticity-normalized ancient limit (Lemma~\ref{lem:omega32-runningmax}); bridging this to the CKN-anchored tangent flow remains open.}
    \item \textbf{(C)} is the rigidity step replacing the current non-referee-checkable DDE $\varepsilon$-regularity/Liouville argument (cf.\ Theorem~\ref{thm:DDE-eps-regularity} and Theorem~\ref{thm:liouville}).
    \item \textbf{(D)} is used to guarantee the small-forcing hypothesis needed to apply the directional rigidity step \textbf{(C)} (i.e.\ $\|H\|_{C^{3/2}}\le \delta_*$ for the tangent flow at sufficiently small scales).
    \item \textbf{(E)} is used in the reduction-to-2D and 2D Liouville step (Theorem~\ref{thm:2d_liouville}).
\end{itemize}
\end{remark}
} % end conditional-inputs subsection

\subsection{Constants and Thresholds}\label{subsec:constants}
Throughout, we use universal dimensional constants $C,c>0$ whose value may change from line to line. We introduce the following scale-invariant quantities and thresholds:
\begin{itemize}
    \item The {\it scale-invariant energy} of the direction field $\xi$ on a cylinder $Q_r(z_0)$:
    \[
    E(z_0,r) := r^{-3} \iint_{Q_r(z_0)} |\nabla \xi|^2 \, dx \, dt.
    \]
    \item The {\it critical Carleson norm} of the tangential forcing $H$ in the direction equation at scales $\le r_*$:
    \[
    \|H\|_{C^{3/2}(r_*)} := \sup_{z_0}\ \sup_{0<r\le r_*} r^{-2} \iint_{Q_r(z_0)} |H|^{3/2} \, dx \, dt,
    \qquad (0<r_*\le 1).
    \]
    When $r_*=1$ we write $\|H\|_{C^{3/2}}:=\|H\|_{C^{3/2}(1)}$.
    \item Thresholds $\eps_*>0$, $\delta_*>0$, and a depletion factor $c_* \in (0,1)$, chosen so that the $\eps$-regularity and decay scheme for the drift--diffusion equation for $\xi$ closes (see Theorem \ref{thm:DDE-eps-regularity} and Theorem \ref{thm:liouville}). These thresholds are universal and depend only on Calder\'on--Zygmund constants and the Serrin bound of the drift $u$ inherited by tangent flows.
\end{itemize}
{\color{magenta}\noindent\textbf{[AI AUDIT.]}
The claimed ``Serrin bound of the drift $u$ inherited by tangent flows'' is not derived from the blow-up construction in Lemma~\ref{lem:ancient-limit}. If a Serrin bound is required for later steps, it must be proved as part of the argument or stated explicitly as an additional hypothesis.}
We record that all objects above are invariant under the N--S scaling $x\mapsto \lambda x$, $t\mapsto \lambda^2 t$.}

\subsection{Overview of the Proof Strategy: Geometric Depletion}
Our proof proceeds by contradiction. We assume a finite-time singularity exists and perform a blow-up analysis to extract a nontrivial ancient mild solution (a "tangent flow") defined on $\R^3 \times (-\infty, 0]$. This tangent flow inherits critical scale-invariant bounds from the blow-up sequence. {\color{magenta}\noindent\textbf{[AI AUDIT.]}
The blow-up/compactness argument in Lemma~\ref{lem:ancient-limit} establishes local energy and local \(L^3\) bounds on fixed cylinders, but it does not (as written) provide the strong global/scale-uniform ``critical bounds'' later invoked (e.g. Serrin bounds, uniform BMO/VMO controls, or scale-uniform \(L^{3/2}\) control of \(\omega\)). Those additional bounds must be proved or explicitly assumed.}
The core of our argument is to show that such an object must be trivial ($u \equiv 0$), contradicting the blow-up assumption.

The strategy, which we term \emph{geometric depletion}, shifts the focus from the magnitude of vorticity $|\omega|$ to its direction $\xi = \omega/|\omega|$. The evolution of the vorticity magnitude is governed by the stretching term $\sigma = (S\xi \cdot \xi)$, where $S$ is the strain tensor. A singularity requires persistent, strong stretching. However, the direction field $\xi$ satisfies a critical drift--diffusion equation constrained to the sphere $\Sbb^2$:
\begin{equation}\label{eq:direction_intro}
{\color{magenta}\partial_t \xi - \Delta \xi + u \cdot \nabla \xi = |\nabla \xi|^2 \xi + H, \quad |\xi|=1,\quad H\cdot \xi = 0,}
\end{equation}
where $H$ is a forcing term derived from the N--S nonlinearity.

The proof rests on two key innovations that exploit the tension between the "roughness" required for stretching and the "structure" enforced by the direction equation:

\begin{enumerate}
    \item \textbf{Critical Coercivity (Problem 1):} We prove that the stretching term $\sigma$, viewed as a singular integral operator acting on $\omega$, is \emph{depleted} in the near-field if the direction field $\xi$ has small oscillation. Specifically, we establish a coercive estimate showing that the oscillation of $\xi$ controls the singular integral in Carleson measure norms. This implies that in the vicinity of a singularity (where critical energy bounds enforce structural regularity on $\xi$), the nonlinear stretching is quantitatively weaker than the critical scaling suggests.

    \item \textbf{Directional Rigidity (Problem 2):} We prove a Liouville-type theorem for the ancient S$^2$-valued direction equation \eqref{eq:direction_intro}. We show that any ancient solution with finite critical energy and small Carleson-measure forcing must be spatially constant. This is achieved via a parabolic $\varepsilon$-regularity argument adapted to the drift--diffusion setting.
\end{enumerate}

The logic chain concludes as follows: If a singularity occurs, we extract an ancient tangent flow. The critical energy bounds imply that the direction field $\xi$ of this flow has Vanishing Mean Oscillation (VMO). This VMO regularity triggers the Critical Coercivity estimate, rendering the forcing $H$ in the direction equation small. The Directional Rigidity theorem then forces $\xi$ to be a constant vector. A N--S flow with constant vorticity direction is structurally two-dimensional. By known Liouville theorems for 2D ancient solutions, such a flow must vanish. This implies the singularity was spurious.
{\color{magenta}\noindent\textbf{[AI AUDIT.]}
As written, multiple arrows in this chain require additional hypotheses currently isolated as Assumptions~(A)--(E) in Subsection~\ref{subsec:conditional-inputs} (and further technical conditions flagged later).}

\section{Preliminaries and Notation}
{\color{blue}
\subsection{Functional Spaces and Scaling}
We work in Euclidean space $\R^3$. For a point $z_0 = (x_0, t_0) \in \R^3 \times \R$ 
and a radius $r>0$, we define the backward parabolic cylinder
\[
Q_r(z_0) = B_r(x_0) \times (t_0 - r^2,\, t_0),
\]
where $B_r(x_0)$ denotes the open ball of radius $r$ centered at $x_0$. We use standard Lebesgue spaces $L^p(\R^3)$ and parabolic spaces $L^q(0,T; L^p(\R^3))$. 

The vorticity field, defined as $\omega = \nabla \times u$, plays a central role in the analysis. The N--S equations are invariant under the scaling
\begin{equation}\label{scaling2}
u_\lambda(x,t) = \lambda u(\lambda x, \lambda^2 t), \quad p_\lambda(x,t) = \lambda^2 p(\lambda x, \lambda^2 t).
\end{equation}

Under the scaling, the vorticity transforms as $\omega_\lambda(x,t) = \lambda^2 \omega(\lambda x, \lambda^2 t)$. A norm or functional is called \emph{critical} if it is invariant under this transformation.  One of the most important critical norms for the velocity field is 
the scale-invariant quantity $\|u\|_{L^\infty_t L^3_x}$. 
 

 

The Ladyzhenskaya--Prodi--Serrin criterion provides a sufficient condition for global existence: if a smooth solution $u$ belongs to the mixed Lebesgue space
$$u \in L^q(0, T;L^p(\mathbb{R}^3)) \quad \text{such that} \quad \frac{2}{q} + \frac{3}{p} \le 1 \quad \text{for} \quad p \ge 3,$$
then $u$ can be extended after $t = T$, see for example \cite{15,25,27}. A critical advance was the resolution of the endpoint case (where $p=3$), specifically $u \in L^\infty(0, T;L^3(\mathbb{R}^3))$. This result implies the non-existence of self-similar type singularities \cite{23}.

In order to bridge these global criteria with the local analysis of weak solutions, we recall the standard notions of weak and suitable weak solutions.



 




\begin{definition}[Weak Solution]\label{def:weak-solution}
Let $u:Q \to \mathbb{R}^3$ be a measurable function. 
We say that $u$ is a \emph{weak solution} of the N--S equations \textup{(1.1)} 
in the space--time cylinder $Q = \Omega \times (a,b)$ if
\begin{equation}\label{eq:LerayHopfSpaces}
u \in L^\infty\!\left(a,b; L^2(\Omega;\mathbb{R}^3)\right)
\;\cap\;
L^2\!\left(a,b; W^{1,2}(\Omega;\mathbb{R}^3)\right),
\end{equation}
the equation $\operatorname{div} u = 0$ holds in the sense of distributions, and
for all test functions 
\[
\varphi \in C_c^1\!\left((a,b); C_{c,\sigma}^\infty(\Omega;\mathbb{R}^3)\right)
\]
the identity
\begin{equation}\label{eq:weak-formulation}
-\!\!\iint_{Q} u \cdot \partial_t \varphi \, dx\,dt
+ \iint_{Q} \nabla u : \nabla \varphi \, dx\,dt
- \iint_{Q} (u \otimes u) : \nabla \varphi \, dx\,dt = 0
\end{equation}
holds.
\end{definition}

These solutions exist globally in time and possess the global energy inequality in terms of the initial kinetic energy. 
Such solutions are commonly referred to as \emph{Leray--Hopf weak solutions}.

\smallskip

When studying local and partial regularity of the N--S equations, 
a stronger notion of solution is typically used, the class of 
\emph{suitable weak solutions}. Following Scheffer \cite{Scheffer1977} and Caffarelli, Kohn, and Nirenberg \cite{CKN1982}, we work with the class of suitable weak solutions.  
Here we present a version due to Galdi \cite{6}.

\begin{definition}[Suitable Weak Solution]\label{def:suitable}
Let $u:Q \to \mathbb{R}^3$ and $p:Q \to \mathbb{R}$ be measurable.  
The pair $(u,p)$ is called a \emph{suitable weak solution} of the N--S 
equations \textup{(1.1)} in the cylinder $Q = \Omega \times (a,b)$ if:
\begin{align}
u &\in 
L^\infty\!\left(a,b; L^2(\Omega;\mathbb{R}^3)\right)
\;\cap\;
L^2\!\left(a,b; W^{1,2}(\Omega;\mathbb{R}^3)\right), 
\label{eq:suitable-u}
\\[4pt]
p &\in L^{3/2}(Q), 
\label{eq:suitable-p}
\end{align}
the system \textup{(1.1)} is satisfied in the sense of distributions, and the following
\emph{generalized local energy inequality} holds:

For almost every $t \in (a,b)$ and every non-negative test function 
$\phi \in C_c^\infty(Q)$,
\begin{equation}\label{eq:local-energy-ineq}
\begin{aligned}
\int_{\Omega} |u(t)|^2 \phi(t) \, dx
+ 2 \int_{a}^{t} \!\!\int_{\Omega} |\nabla u|^2 \phi \, dx\,ds
\;\le\;
\int_{a}^{t} \!\!\int_{\Omega} 
u^2 (\partial_t \phi + \Delta \phi)
\, dx\,ds 
\\
+ \int_{a}^{t} \!\!\int_{\Omega} \bigl(|u|^2 + 2p\bigr)\, u \cdot \nabla \phi \, dx\,ds .
\end{aligned}
\end{equation}
\end{definition}





While the Ladyzhenskaya–Prodi–Serrin and endpoint criteria provide global regularity conditions, the local counterpart is given by the $\varepsilon$-regularity theory of Caffarelli–Kohn–Nire\-nberg. 

Standard $\varepsilon$-regularity theory \cite{CKN1982, Lin1998} shows that
smallness of certain scale-invariant quantities on a parabolic cylinder forces
regularity. A fundamental example is the Caffarelli--Kohn--Nirenberg
criterion, based on the dimensionless functional
\[
F(r) := r^{-2}\!\iint_{Q_r(z_0)} \big(|u|^{3} + |p|^{3/2}\big)\,dx\,dt .
\]
There exists a universal constant $\varepsilon_{CKN} > 0$ such that if
\[
F(r) < \varepsilon_{CKN},
\]
then $u$ is bounded (and in fact Hölder continuous) on $Q_{r/2}(z_0)$.
This type of estimate constitutes the first prototype of local
regularity criteria for suitable weak solutions.}


\subsection{Blow-up Analysis and Construction of Ancient Tangent Flows}


{\color{blue} Assume, for contradiction, that the smooth solution develops a finite-time singularity at
$T^* < \infty$. By the Beale–Kato–Majda criterion we know that the vorticity must blow up, so
\[
\limsup_{t \uparrow T^*} \|\omega(\cdot,t)\|_{L^\infty} = \infty.
\]
In order to understand how such a singularity could appear, we rescale the solution near the
points and times where the vorticity is very large, and in this way we obtain a limiting
blow-up profile.



\begin{theorem} [Beale--Kato--Majda (BKM), Euler, \cite{BKM1984}]
Let $u$ be a solution of the incompressible Euler equations ({\color{magenta}i.e.\ \eqref{eq:NS_domain} with $\nu=0$ and $f=0$}), and
suppose there is a time $T^*$ such that the solution cannot be continued in the class $u \in C([0,T]; H^s) \,\cap\, C^1([0,T]; H^{s-1}), \, s \geq 3.$
to $T = T^*$. Assume that $T^*$ is the first such time.
Then
\[
\int_0^{T^*} \|\omega(t)\|_{L^\infty}\, dt = +\infty,
\]
and in particular
\[
\limsup_{t \uparrow T^*} \|\omega(t)\|_{L^\infty} = +\infty.
\]
\end{theorem}


%Lecture notes for Math 256B, Version 2024
%Lenya Ryzhik May 7, 2024
\begin{theorem}[BKM, N-S]\label{thm:BKM-NS}
Let $u_0 \in C^\infty_c(\mathbb{R}^3)$, so that there exists a classical 
solution $u$ to the N-S equations ({\color{magenta}i.e.\ \eqref{eq:NS_domain} with $f=0$ and viscosity $\nu>0$}). 
If for any $T>0$ we have
\begin{equation}\label{eq:BKM-NS-1}
\int_0^T \|\omega(t)\|_{L^\infty}\, dt < +\infty,
\end{equation}
then the smooth solution $u$ exists globally in time.  
If the maximal existence time of the smooth solution is $T < +\infty$, 
then necessarily
\begin{equation}\label{eq:BKM-NS-2}
{\color{magenta}\int_0^{T} \|\omega(s)\|_{L^\infty}\, ds = +\infty.}
\end{equation}
\end{theorem}

\begin{remark}
For the Euler equations the BKM criterion is an equivalence: 
finite--time blow-up occurs if and only if 
$\int_0^{T^*}\|\omega(t)\|_{L^\infty}\,dt=+\infty$. 
For the N-S equations one only has the one--sided continuation 
criterion stated above; the converse implication is not known, nor does it 
hold for weak solutions or suitable weak solutions. 
\end{remark}

The $\varepsilon$--regularity theorem (see Caffarelli--Kohn--Nirenberg \cite{CKN1982})
implies that if no singular point existed at a possible blow\mbox{-}up time $T^{*}$, 
then the solution would remain uniformly bounded in a parabolic neighbourhood of 
the hyperplane $\{t = T^{*}\}$. Combined with the local energy inequality, this 
allows us to extend the solution smoothly past $T^{*}$, contradicting the assumption
that $T^{*}$ is the first blow-up time. F. Lin \cite{Lin1998} later
gave a different proof of this result via a blow-up argument which was expanded upon
and extended by Ladyzhenskaya–Seregin \cite{LG}. The following lemma is a direct consequence of the $\varepsilon$--regularity theory of
Caffarelli–Kohn–Nirenberg (CKN) \cite{CKN1982}.




\begin{lemma}\label{lem:singular-point}
Assume that $u$ is a smooth solution of the N-S (\ref{eq:NS_domain}) equations
on $[0,T^*)$ and that $T^*<\infty$ is the first blow-up time.
Then there exists at least one point $x^*\in\R^3$ such that $(x^*,T^*)$ is a singular
point in the sense of CKN.
\end{lemma}


\begin{proof}
Suppose, that no such point exists. Then every $(x,T^*)$ is regular
in the CKN sense. Hence, for each $x\in\R^3$ there exists $r_x>0$ such that 
\[
F(z_0,r) = r^{-2} \iint_{Q_r(z_0)} \bigl(|u|^3 + |p|^{3/2}\bigr)\,dx\,dt
\]
satisfies $F((x,T^*),r_x) < \varepsilon_{\mathrm{CKN}}$.
By the $\varepsilon$-regularity theorem \cite{CKN1982,Lin1998}, this implies that
$u$ is bounded in a smaller parabolic cylinder, there exist constants
$M_x<\infty$ such that
\[
|u(y,s)| \le M_x \quad \text{for all } (y,s)\in
Q_{r_x/2}(x,T^*) = B_{r_x/2}(x)\times(T^*-(r_x/2)^2,T^*].
\]



There exist $R>0$ and consider the compact set $\overline{B_R(0)}\times\{T^*\}$.
Since the balls $B_{r_x/2}(x)$, $x\in\overline{B_R(0)}$, form an open cover of
$\overline{B_R(0)}$, we can extract a finite subcover
\[
\overline{B_R(0)} \subset \bigcup_{i=1}^N B_{r_i/2}(x_i).
\]
Let us define
\[
\delta_R := \min_{1\le i\le N} \frac{r_i^2}{4} > 0,
\qquad
M_R := \max_{1\le i\le N} M_{x_i} < \infty.
\]
Let $(y,s)$ be any point with $|y|\le R$ and $s\in(T^*-\delta_R,T^*]$.
Then there exists $i\in\{1,\dots,N\}$ such that $y\in B_{r_i/2}(x_i)$.
Moreover,  we have
\[
s > T^*-\delta_R \ge T^* - \frac{r_i^2}{4},
\]
so $(y,s)\in Q_{r_i/2}(x_i,T^*)$. Therefore
\[
|u(y,s)| \le M_{x_i} \le M_R.
\]
In other words,
\[
\sup_{|y|\le R,\; s \in (T^*-\delta_R,T^*]} |u(y,s)| \le M_R < \infty.
\]

Thus $u$ is uniformly bounded on $B_R(0)\times(T^*-\delta_R,T^*]$.
Standard local well-posedness and continuation for smooth solutions imply that
$u$ can be smoothly extended beyond $T^*$ on $B_R(0)$.

Since $R>0$ is arbitrary, this shows that $u$ extends smoothly beyond $T^*$ on
all of $\R^3$, contradicting the maximality of $T^*$. Therefore, there exist at least one singular point $(x^*,T^*)$ in the CKN sense.
\end{proof}











\begin{lemma}\label{lem:blowup-normalization}
Let $u_0 \in C_c^\infty(\mathbb{R}^3)$ be divergence-free, and let
$u$ be the unique smooth solution of the N-S equations (\ref{eq:NS_domain})
on its maximal interval of smooth existence $[0,T^*)$. Assume that $T^* < \infty$ is the
first blow-up time.

Then there exist times $t_k \uparrow T^*$, points $x_k \in \mathbb{R}^3$, and scales
$\lambda_k \downarrow 0$ (for instance, $\lambda_k = |\omega(x_k,t_k)|^{-1/2}$) such that,
defining the rescaled velocity fields
\begin{equation}\label{rescaled}
u^{(k)}(y,s)
:=
\lambda_k\, u\!\left(x_k + \lambda_k y,\; t_k + \lambda_k^2 s\right),
\qquad
\omega^{(k)} := \curl\, u^{(k)},
\end{equation}
we have the normalization
\[
|\omega^{(k)}(0,0)| = 1 \quad \text{for all } k.
\]
\end{lemma}

\begin{proof}
By the BKM continuation criterion, loss of smoothness at $T^*$ implies that
\[
\limsup_{t \uparrow T^*} \|\omega(\cdot,t)\|_{L^\infty} = +\infty.
\]
Hence we can choose a sequence of times $t_k \uparrow T^*$ such that
\[
M_k := \|\omega(\cdot,t_k)\|_{L^\infty} \to \infty
\quad \text{as } k \to \infty.
\]
{\color{magenta}\noindent\textbf{[AI AUDIT / OPTIONAL STRENGTHENING (running-max times).]}
One may choose the times $t_k$ so that $\|\omega(\cdot,t)\|_{L^\infty}\le \|\omega(\cdot,t_k)\|_{L^\infty}=M_k$ for all $t\le t_k$
(e.g.\ take $t_k$ to be the first hitting time of a level $L_k\uparrow\infty$). This yields uniform backward-in-time $L^\infty$ control for the rescaled vorticities (see below).}
For each $k$, since $\omega(\cdot,t_k)$ is continuous and not identically zero, there exists
a point $x_k \in \mathbb{R}^3$ such that
\[
|\omega(x_k,t_k)| \ge \tfrac{1}{2} M_k.
\]
Let us set $
A_k := |\omega(x_k,t_k)|$, then $A_k \ge \tfrac{1}{2} M_k$, and in particular $A_k \to \infty$ as $k \to \infty$.
Let us define the scaling factors
\[
\lambda_k := A_k^{-1/2}.
\]
Using the rescaling (\ref{rescaled}), by the scaling of the vorticity (\ref{scaling}), we have
\[
\omega^{(k)}(0,0)
= \lambda_k^2\, \omega(x_k,t_k)
= \lambda_k^2 A_k
= 1.
\]
Since $A_k \to \infty$, it follows that $\lambda_k \downarrow 0$.
{\color{magenta}\noindent\textbf{[AI AUDIT / CONSEQUENCE.]}
If the ``running-max'' choice of $t_k$ is made, then for every $s\le 0$ one has $t_k+\lambda_k^2 s\le t_k$ and hence
$\|\omega(\cdot,t_k+\lambda_k^2 s)\|_{L^\infty}\le M_k$.
By scaling and $A_k\ge \tfrac12 M_k$ this gives the uniform bound
\[
\|\omega^{(k)}(\cdot,s)\|_{L^\infty}\le \frac{M_k}{A_k}\le 2
\qquad\text{for all }s\le 0.
\]
In particular, any ancient limit extracted from such a sequence satisfies the scale-critical bound in Lemma~\ref{lem:omega32-runningmax}.}
\end{proof}

\begin{lemma}[Running-max vorticity normalization implies a critical \(L^{3/2}\) bound]\label{lem:omega32-runningmax}
Assume the times $t_k\uparrow T^*$ in Lemma~\ref{lem:blowup-normalization} are chosen as \emph{running maxima} for the vorticity:
\[
\|\omega(\cdot,t)\|_{L^\infty}\le \|\omega(\cdot,t_k)\|_{L^\infty}\qquad\text{for all }t\le t_k.
\]
Then the rescaled vorticities $\omega^{(k)}=\curl u^{(k)}$ satisfy the uniform backward-in-time bound
\[
\|\omega^{(k)}\|_{L^\infty(\R^3\times(-\lambda_k^{-2}t_k,\,0])}\le 2.
\]
In particular, any subsequential weak-$\ast$ limit $\omega^\infty$ of $\omega^{(k)}$ in $L^\infty_{\mathrm{loc}}(\R^3\times(-\infty,0])$
obeys the scale-critical estimate
\[
\sup_{z_0\in\R^3\times(-\infty,0]}\ \sup_{0<r\le1}\ r^{-2}\iint_{Q_r(z_0)} |\omega^\infty|^{3/2}\,dx\,dt
\ \le\ C\,2^{3/2},
\]
where $C>0$ is a universal dimensional constant depending only on the definition of $Q_r$.
\end{lemma}

\begin{proof}
The $L^\infty$ bound on $\omega^{(k)}$ is proved in the audit note at the end of Lemma~\ref{lem:blowup-normalization}.
Passing to a subsequence, we may assume $\omega^{(k)}\rightharpoonup^\ast \omega^\infty$ weak-$\ast$ in $L^\infty_{\mathrm{loc}}$ and hence
$\|\omega^\infty\|_{L^\infty_{\mathrm{loc}}}\le 2$.
Therefore for any $z_0$ and $0<r\le 1$,
\[
r^{-2}\iint_{Q_r(z_0)} |\omega^\infty|^{3/2}\,dx\,dt
\ \le\ r^{-2}\,\|\omega^\infty\|_{L^\infty(Q_r(z_0))}^{3/2}\,|Q_r|
\ \le\ r^{-2}\,(2^{3/2})\,|Q_r|
\ \le\ C\,2^{3/2},
\]
since $|Q_r|\le C\,r^5$ for $r\le 1$.
\end{proof}

\begin{lemma}\label{lem:domain-rescaled}
Let $u^{(k)}$ be the rescaled sequence defined in \eqref{rescaled}.
Then each $u^{(k)}$ is defined on a time interval of the form
\[
s \in \bigl(-\lambda_k^{-2} t_k,\;0\bigr],
\]
and these intervals exhaust $(-\infty,0]$. It means that for every $R>0$ there exists
$k_0(R)$ such that
\[
(-R^2,0] \subset \bigl(-\lambda_k^{-2} t_k,\;0\bigr]
\quad\text{for all } k \ge k_0(R).
\]
\end{lemma}

\begin{proof} 
The original solution $u$ is defined for $0 \le t < T^*$. Since $u^{(k)}$ be the rescaled by (\ref{rescaled}), for $u^{(k)}$ to be
well-defined at time $s$, we need
\[
0 \le t_k + \lambda_k^2 s < T^*.
\]
The upper bound $t_k + \lambda_k^2 s \le t_k$ corresponds exactly to $s \le 0$.
The lower bound $t_k + \lambda_k^2 s \ge 0$ gives
\[
s \ge -\lambda_k^{-2} t_k.
\]
Hence $u^{(k)}$ is defined on $s \in (-\lambda_k^{-2} t_k,0]$.

Since $t_k \uparrow T^*$ and $\lambda_k \downarrow 0$, we have
$\lambda_k^{-2} t_k \to \infty$ as $k\to\infty$. Therefore, for any fixed $R>0$,
we can choose $k_0(R)$ such that $\lambda_k^{-2} t_k > R^2$ for all $k\ge k_0(R)$.
Finally, for  $k\ge k_0(R)$, we obtain
\[
(-R^2,0] \subset (-\lambda_k^{-2} t_k,0].,
\]
which proves the lemma.
\end{proof}

{\color{magenta}\noindent\textbf{[AI AUDIT / NOTATION CLARIFICATION.]}
Lemmas~\ref{lem:blowup-normalization}--\ref{lem:domain-rescaled} construct a \emph{vorticity-normalized} rescaling sequence.
For the existence and nontriviality of an ancient \emph{tangent flow} in the CKN sense, we will instead use the standard CKN blow-up
about a CKN-singular point $(x^*,T^*)$, defined in Lemma~\ref{lem:ancient-limit} below.}

\begin{lemma}\label{lem:ancient-limit}
Let $u_0\in C_c^\infty(\R^3)$ be divergence-free, let $u$ be the corresponding
smooth solution of the N-S equations \eqref{eq:NS_domain} on its
maximal interval of existence $[0,T^*)$, and assume that $T^*<\infty$ is the
first blow-up time. Let $x^*\in\R^3$ be a CKN-singular point at time $T^*$ as in Lemma~\ref{lem:singular-point}.
Let $r_k\downarrow 0$ be any sequence and define the CKN rescalings
\begin{equation}\label{eq:ckn-rescaled}
\tilde u^{(k)}(y,s):=r_k\,u(x^*+r_k y,\;T^*+r_k^2 s),
\qquad
\tilde p^{(k)}(y,s):=r_k^2\,p(x^*+r_k y,\;T^*+r_k^2 s),
\qquad s<0.
\end{equation}

Then there exists a subsequence (still denoted by $\tilde u^{(k)},\tilde p^{(k)}$) 
and a pair $(u^\infty,p^\infty)$ such that:

\begin{enumerate}

\item[(i)] For every $R>0$ and $T>0$,
\[
\tilde u^{(k)} \to u^\infty \quad\text{strongly in } 
L^p(B_R\times(-T,0)) \quad \text{for all } 1\le p<3,
\]
and
\[
\tilde u^{(k)} \rightharpoonup u^\infty 
\quad \text{weakly in}\quad
L^3_{\mathrm{loc}}(\R^3\times(-\infty,0)).
\]
Moreover,
\[
\tilde p^{(k)} \rightharpoonup p^\infty
\quad\text{weakly in } L^{3/2}_{\mathrm{loc}}(\R^3\times(-\infty,0)).
\]

\item[(ii)]
The limit $(u^\infty,p^\infty)$ is a suitable weak solution of the
N-S equations on $\R^3\times(-\infty,0)$ and satisfies the
local energy inequality on every parabolic cylinder
$B_R\times(-T,0)$.

\item[(iii)] The limit $u^\infty$ is an ancient solution, defined for all $t\le 0$, and it is
non-trivial.  More precisely, there exist $r>0$ and $c>0$ such that
\[
\int_{Q_r(0,0)} |u^\infty(x,t)|^3 \,dx\,dt \;\ge\; c > 0,
\]
where $Q_r(0,0)=B_r(0)\times(-r^2,0)$.
In particular, $u^\infty \not\equiv 0$.
\end{enumerate}

We call $u^\infty$ an \emph{ancient tangent flow} associated to the
blow-up at time $T^*$.
\end{lemma}

{\color{blue}
\begin{proof}
\textbf{[ADDED PROOF / closure of Lemma~\ref{lem:ancient-limit} (compactness + nontriviality).]}
We outline the standard compactness argument for suitable weak solutions, and we make explicit
the missing nontriviality mechanism.

\medskip
\noindent\textbf{Step 1: Uniform local bounds on cylinders.}
Fix $R>0$. For $k$ sufficiently large, the CKN rescalings \eqref{eq:ckn-rescaled} are well-defined on
$Q_R:=B_R\times(-R^2,0)$ since $T^*+r_k^2 s<T^*$ for all $s\in(-R^2,0)$ and $r_k^2R^2<T^*$ for $k$ large.
Since $u$ is smooth on $[0,T^*)$, each rescaled pair $(\tilde u^{(k)},\tilde p^{(k)})$ is smooth on $Q_R$
and in particular is a suitable weak solution there; hence it satisfies the local energy inequality
(cf.\ Definition~\ref{def:suitable}), with constants independent of $k$ after scaling.
Using standard cutoff functions supported in $B_{2R}$, one obtains a bound of the form
\begin{equation}\label{eq:uniform_local_energy_rescaled}
\sup_{s\in(-R^2,0)}\int_{B_R}|\tilde u^{(k)}(x,s)|^2\,dx
\;+\;\int_{Q_R}|\nabla \tilde u^{(k)}|^2\,dx\,ds
\;\le\; C(R),
\end{equation}
where $C(R)$ is independent of $k$.
By interpolation (Ladyzhenskaya + Sobolev) and \eqref{eq:uniform_local_energy_rescaled} we also get
\begin{equation}\label{eq:uniform_L3_rescaled}
\iint_{Q_R}|\tilde u^{(k)}|^3\,dx\,ds \le C(R).
\end{equation}
Finally, the pressure satisfies the standard local estimate (via
$-\Delta \tilde p^{(k)}=\partial_i\partial_j(\tilde u^{(k)}_i\tilde u^{(k)}_j)$ and Calder\'on--Zygmund),
which yields
\begin{equation}\label{eq:uniform_p32_rescaled}
\|\tilde p^{(k)}\|_{L^{3/2}(Q_R)} \le C(R)
\end{equation}
after fixing the additive-in-time constant of the pressure (see, e.g., \cite{CKN1982,Seregin2012}).

\medskip
\noindent\textbf{Step 2: Compactness (Aubin--Lions).}
From the Navier--Stokes system on $Q_R$,
\[
\partial_s \tilde u^{(k)}=\Delta \tilde u^{(k)}-\nabla \tilde p^{(k)}-(\tilde u^{(k)}\cdot\nabla)\tilde u^{(k)},
\]
the bounds \eqref{eq:uniform_local_energy_rescaled}--\eqref{eq:uniform_p32_rescaled} imply
that $\partial_s \tilde u^{(k)}$ is bounded in a negative Sobolev space on $Q_R$
uniformly in $k$ (e.g.\ in $L^{3/2}(-R^2,0;W^{-2,3/2}(B_R))$).
Therefore, by the Aubin--Lions compactness lemma, after passing to a subsequence we have
\[
\tilde u^{(k)}\to u^\infty \quad\text{strongly in }L^2(Q_R).
\]
Combining strong $L^2$ convergence with the uniform $L^3$ bound \eqref{eq:uniform_L3_rescaled}
and interpolation yields strong convergence in $L^p(Q_R)$ for every $1\le p<3$.
Using a diagonal subsequence over $R\in\N$ gives (i).
Similarly, by \eqref{eq:uniform_p32_rescaled} we may extract a subsequence with
$\tilde p^{(k)}\rightharpoonup p^\infty$ weakly in $L^{3/2}_{\mathrm{loc}}$, proving the pressure part of (i).

\medskip
\noindent\textbf{Step 3: Passage to the limit; suitable weak limit.}
The strong convergence of $\tilde u^{(k)}$ in $L^2_{\mathrm{loc}}$ and the weak convergence of $\nabla \tilde u^{(k)}$
in $L^2_{\mathrm{loc}}$ imply $\tilde u^{(k)}\otimes \tilde u^{(k)}\to u^\infty\otimes u^\infty$ in distributions,
so we may pass to the limit in the N--S equations on each $Q_R$.
Lower semicontinuity passes the local energy inequality to the limit, so $(u^\infty,p^\infty)$
is a suitable weak solution on $\R^3\times(-\infty,0)$, proving (ii).

\medskip
\noindent\textbf{Step 4: Nontriviality (how to close (iii) rigorously).}
Nontriviality follows from the CKN-singularity of $(x^*,T^*)$.
By the contrapositive of CKN $\varepsilon$-regularity, there exists a universal $\varepsilon_{\mathrm{CKN}}>0$ such that
for all sufficiently small $r>0$,
\[
r^{-2}\iint_{Q_r(x^*,T^*)}\bigl(|u|^3+|p|^{3/2}\bigr)\,dx\,dt \;\ge\; \varepsilon_{\mathrm{CKN}}.
\]
Taking $r=r_k$ and using the scale invariance of the CKN functional under \eqref{eq:ckn-rescaled} gives
\[
\iint_{Q_1(0,0)}\bigl(|\tilde u^{(k)}|^3+|\tilde p^{(k)}|^{3/2}\bigr)\,dy\,ds \;\ge\; \varepsilon_{\mathrm{CKN}}
\quad\text{for all }k.
\]
Passing to the limit and using lower semicontinuity yields
\[
\iint_{Q_1(0,0)}|u^\infty|^3\,dy\,ds \;\ge\; c_0>0
\]
for a universal $c_0$, proving (iii) (with $r=1$ and $c=c_0$).

\medskip
\noindent\textit{Remark.} If one prefers the vorticity normalization of Lemma~\ref{lem:blowup-normalization} for later
geometric arguments, one can re-center/renormalize the CKN blow-up sequence at a point of large vorticity
inside $Q_1$; the essential point for (iii) is that the construction must preserve a scale-invariant
lower bound (such as the CKN functional), so that triviality of the limit is ruled out.
\end{proof}
} % end added-blue proof
}






\section{The Vorticity Direction Equation}
{\color{blue}
\subsection{Derivation of the Coupled System}

Let $u$ be a sufficiently smooth divergence-free solution of the incompressible N–S equations with unit viscosity and $\omega = \curl\, u$ be the vorticity field. In the region $\{\omega \neq 0\}$,
we decompose the vorticity into its magnitude $\rho = |\omega|$ and its direction
$\xi = \omega/|\omega| \in \mathbb{S}^2$. The vorticity equation  can be written in
vector form as
\begin{equation}
\partial_t \omega + (u \cdot \nabla)\omega - \Delta \omega = (\omega \cdot \nabla)u.
    \end{equation}
Substituting $\omega = \rho \xi$ yields
\[
(\partial_t \rho + u \cdot \nabla \rho - \Delta \rho)\xi
+ \rho (\partial_t \xi + u \cdot \nabla \xi - \Delta \xi)
- 2 (\nabla \rho \cdot \nabla) \xi
= \rho (S\xi),
\]
where $S = \tfrac{1}{2}(\nabla u + (\nabla u)^T)$ is the strain tensor. We take the inner product with $\xi$ to isolate the amplitude equation.Using the identities $|\xi|^2=1$, $\xi \cdot \partial_t \xi = 0$, and $\xi \cdot \Delta \xi = -|\nabla \xi|^2$, we obtain:
\begin{equation}\label{eq:amplitude}
\partial_t \rho + u \cdot \nabla \rho - \Delta \rho = \rho (\sigma - |\nabla \xi|^2),
\end{equation}
where $\sigma = (S\xi \cdot \xi)$ is the vortex stretching scalar.

%%%%

To isolate the evolution of the direction field $\xi$, we apply the
orthogonal projection $P_\xi = I - \xi \otimes \xi$ onto the tangent space
$T_\xi \mathbb{S}^2$.  
Since $P_\xi \xi = 0$, all terms parallel to $\xi$, including the
amplitude component $(\partial_t \rho + u\cdot\nabla\rho - \Delta\rho)\xi$, 
are eliminated after projection. Thus, to derive the direction equation, we project the vorticity decomposition onto
$T_\xi \mathbb{S}^2$, which yields
\[
\rho (\partial_t \xi + u \cdot \nabla \xi - \Delta \xi)
- 2 P_\xi (\nabla \rho \cdot \nabla) \xi
= \rho P_\xi (S\xi).
\]
Dividing by $\rho$ (where $\rho > 0$) we obtain
\begin{equation}\label{eq:direction_intermediate}
\partial_t \xi + u \cdot \nabla \xi - \Delta \xi = P_\xi(S\xi) + 2 P_\xi\bigl( (\nabla \log\rho) \cdot \nabla \xi \bigr).
\end{equation}

{\color{magenta}\noindent\textbf{[AI AUDIT / SIGN-CONSISTENCY FIX.]}
The projection step yields a \emph{tangential} diffusion operator.  Using the identity
$P_\xi(\Delta \xi)=\Delta \xi + |\nabla\xi|^2\xi$ (equivalently $\Delta \xi = P_\xi(\Delta\xi)-|\nabla\xi|^2\xi$),
we may rewrite \eqref{eq:direction_intermediate} in the standard harmonic-map form:}
\begin{equation}\label{eq:direction}
{\color{magenta}\partial_t \xi + u \cdot \nabla \xi - \Delta \xi  = |\nabla\xi|^2\,\xi + H,}
\end{equation}
where the forcing $H$ is given by
\[
H = H_{\mathrm{sing}} + H_{\mathrm{geom}}.
\]
Here, $H_{\mathrm{sing}} = P_\xi (S\xi)$ represents the projection of the vortex stretching term, and $H_{\mathrm{geom}}$ collects the geometric coupling terms:
\begin{equation}\label{hgeom}
{\color{magenta}H_{\mathrm{geom}} = 2 P_\xi \bigl( (\nabla \log \rho) \cdot \nabla \xi \bigr).}
\end{equation}
By construction, the singular term $H_{\mathrm{sing}} = P_\xi(S\xi)$ and the
tangential component of $H_{\mathrm{geom}}$ lie in the tangent space
$T_\xi \mathbb{S}^2$.  
The normal component on the right-hand side of \eqref{eq:direction} is the curvature term $|\nabla\xi|^2\xi$.










%%%%%%%%%%


 \subsection{The Singular Stretching Term}

The term \( H_{\mathrm{sing}} = P_\xi (S\xi) \) encodes the non‑local nonlinearity 
of the N--S equations. 

{\color{magenta}\noindent\textbf{[AI AUDIT / KERNEL-CONSISTENCY FIX.]}
Strictly speaking, $S$ is a \emph{matrix} field obtained from $\omega$ by a matrix of Calder\'on--Zygmund operators (Riesz transforms).
One convenient way to write Biot--Savart at this level is componentwise:
\[
S_{ij}(x)=\mathrm{p.v.}\int_{\R^3}\mathcal{K}_{ij\ell}(x-y)\,\omega_\ell(y)\,dy,
\]
where $\mathcal{K}$ is a tensor kernel homogeneous of degree $-3$ with cancellation.
Consequently, for each unit vector $e\in\mathbb{S}^2$ there exists a vector-valued Calder\'on--Zygmund kernel $K_e$ (depending linearly on $e$) such that
$(S e)(x)=\mathrm{p.v.}\int_{\R^3} K_e(x-y)\,\omega(y)\,dy$.}

\begin{equation}\label{eq:H_sing_integral}
{\color{magenta}
H_{\mathrm{sing}}(x)
=P_{\xi(x)}\bigl(S(x)\xi(x)\bigr)
=P_{\xi(x)}\left(\mathrm{p.v.}\int_{\R^3} K_{\xi(x)}(x-y)\,\omega(y)\,dy\right)
=P_{\xi(x)}\left(\mathrm{p.v.}\int_{\R^3} K_{\xi(x)}(x-y)\,\rho(y)\xi(y)\,dy\right).
}
\end{equation}

{\color{magenta}\noindent\textbf{[AI AUDIT / USEFUL CLASSICAL IDENTITY.]}}
\begin{lemma}[Biot--Savart identity for vortex stretching]\label{lem:biot-savart-stretching}
Let $u$ be smooth, divergence-free on $\R^3$ at a fixed time, with vorticity $\omega=\curl u$.
Then for each $x\in\R^3$,
\[
(\omega\cdot\nabla)u(x)
=
\frac{1}{4\pi}\,\mathrm{p.v.}\int_{\R^3}
\left(
\frac{\omega(x)\times\omega(y)}{|x-y|^3}
\;+\;3\,\frac{(\omega(x)\cdot(x-y))\,(\omega(y)\times(x-y))}{|x-y|^5}
\right)\,dy.
\]
\end{lemma}

\begin{proof}
This follows by differentiating the Biot--Savart law
$u(x)=\frac{1}{4\pi}\int_{\R^3}\frac{(x-y)\times\omega(y)}{|x-y|^3}\,dy$
in the $\omega(x)$ direction and using the identities
$(\omega(x)\cdot\nabla_x)(x-y)=\omega(x)$ and
$(\omega(x)\cdot\nabla_x)|x-y|^{-3}=-3(\omega(x)\cdot(x-y))|x-y|^{-5}$.
\end{proof}

{\color{magenta}\noindent\textbf{[AI AUDIT / CONSEQUENCE.]}
Writing $\omega=\rho\xi$, the first term in Lemma~\ref{lem:biot-savart-stretching} contains the factor
$\omega(x)\times\omega(y)=\rho(x)\rho(y)\,\xi(x)\times\xi(y)$ and therefore vanishes when directions align.
In particular, since $\xi(x)\times\xi(x)=0$, one may rewrite that part using the direction difference $\xi(y)-\xi(x)$.
The second term requires additional cancellation (e.g.\ via $\nabla\cdot\omega=0$ and/or a refined symmetric representation) and is part of what must be made referee-checkable in the ``near-field commutator'' step.}

{\color{magenta}\noindent\textbf{[AI AUDIT / USEFUL IDENTITY (for $H_{\mathrm{sing}}$).]}}
\begin{lemma}[$(\xi\cdot\nabla)u$ as a singular integral]\label{lem:xi-derivative}
Let $u$ be smooth and divergence-free on $\R^3$ at a fixed time, with vorticity $\omega=\curl u$.
For any $x$ with $\omega(x)\neq 0$, set $\xi(x):=\omega(x)/|\omega(x)|$. Then
\[
(\xi(x)\cdot\nabla)u(x)
=\frac{1}{4\pi}\,\mathrm{p.v.}\int_{\R^3}
\left(
\frac{\xi(x)\times\omega(y)}{|x-y|^3}
\;-\;3\,\frac{(\xi(x)\cdot(x-y))\,((x-y)\times\omega(y))}{|x-y|^5}
\right)\,dy.
\]
\end{lemma}

\begin{proof}
Differentiate the Biot--Savart law
$u(x)=\frac{1}{4\pi}\int_{\R^3}\frac{(x-y)\times\omega(y)}{|x-y|^3}\,dy$
in the (constant) direction $\xi(x)$ at the point $x$.
\end{proof}

{\color{magenta}\noindent\textbf{[AI AUDIT / CONSEQUENCE.]}
Since $\xi\parallel\omega$, the antisymmetric part of $\nabla u$ annihilates $\xi$, so $(\xi\cdot\nabla)u=S\xi$ and hence
$H_{\mathrm{sing}}=P_\xi(S\xi)=P_\xi((\xi\cdot\nabla)u)$.
The first term in Lemma~\ref{lem:xi-derivative} is already tangential and equals $\rho(y)\,\xi(x)\times\xi(y)/|x-y|^3$.
The second term does not display a direction-difference factor directly and is one of the main technical obstacles in turning the schematic commutator step into a complete proof.}

{\color{magenta}\noindent\textbf{[AI AUDIT / EXPLICIT DECOMPOSITION OF $H_{\mathrm{sing}}$.]}
Writing $\omega=\rho\,\xi$ in Lemma~\ref{lem:xi-derivative} yields the decomposition
\[
H_{\mathrm{sing}}(x)=I_{\mathrm{null}}(x)+I_{\mathrm{const}}(x)+I_{\mathrm{osc}}(x),
\]
where (with $r:=x-y$)
\[
I_{\mathrm{null}}(x):=\frac{1}{4\pi}\,\mathrm{p.v.}\int_{\R^3}\frac{\rho(y)\,\xi(x)\times\xi(y)}{|r|^3}\,dy,
\qquad
I_{\mathrm{const}}(x):=-\frac{3}{4\pi}\,\mathrm{p.v.}\int_{\R^3}\frac{(\xi(x)\cdot r)\,\rho(y)\,(r\times\xi(x))}{|r|^5}\,dy,
\]
and
\[
I_{\mathrm{osc}}(x):=-\frac{3}{4\pi}\,P_{\xi(x)}\,\mathrm{p.v.}\int_{\R^3}\frac{(\xi(x)\cdot r)\,\rho(y)\,(r\times(\xi(y)-\xi(x)))}{|r|^5}\,dy.
\]
In particular, $I_{\mathrm{null}}$ vanishes pointwise when $\xi(y)=\xi(x)$, while $I_{\mathrm{const}}$ is a fixed Calder\'on--Zygmund operator on $\rho$ depending only on the frozen direction $\xi(x)$, and equals
$I_{\mathrm{const}}(x)=\xi(x)\times\nabla\bigl((\xi(x)\cdot\nabla)(-\Delta)^{-1}\rho\bigr)(x)$.
If $\xi$ is exactly constant and $\nabla\cdot\omega=0$ (so $\xi\cdot\nabla\rho=0$), then $I_{\mathrm{const}}\equiv 0$ and hence $H_{\mathrm{sing}}\equiv 0$ as required.}

To separate the singular local interaction from the smoother far‑field contribution, 
we fix a (small) radius \( r > 0 \) and decompose the integral into a near‑field 
part and a tail:
\[
H_{\mathrm{sing}} = H_{\mathrm{near}} + H_{\mathrm{tail}},
\]
where
\[
\begin{aligned}
H_{\mathrm{near}}(x) &= P_{\xi(x)}\Bigl( \mathrm{p.v.} \int_{B_r(x)} {\color{magenta}K_{\xi(x)}}(x-y) \rho(y) \xi(y) \, dy \Bigr), \\[2mm]
H_{\mathrm{tail}}(x)  &= P_{\xi(x)}\Bigl( \int_{\mathbb{R}^3 \setminus B_r(x)} {\color{magenta}K_{\xi(x)}}(x-y) \rho(y) \xi(y) \, dy \Bigr).
\end{aligned}
\]
{\color{magenta}\noindent\textbf{[AI AUDIT / TAIL CONTROL (boundedness vs.\ smallness).]}
For fixed $r$, the operator $f\mapsto \int_{\R^3\setminus B_r(x)} K_{\xi(x)}(x-y)f(y)\,dy$ is a standard Calder\'on--Zygmund truncation (up to the frozen-direction dependence).
Thus, from scale-critical $L^{3/2}$ bounds on $\rho=|\omega|$ one can obtain \emph{boundedness} of the tail contribution in the critical Carleson norm.
However, \emph{smallness as $r\to0$ does not follow} from scale-critical control alone; it requires additional input (e.g.\ vanishing-Carleson hypotheses or a separate far-field depletion mechanism).
See \texttt{NS\_Unconditional\_Closures\_A\_to\_E.tex}, \S\texttt{subsec:D-tail}.}
{\color{magenta}\noindent\textbf{[AI AUDIT / NOTATION.]}
Here $K_{\xi(x)}$ denotes the vector-valued Calder\'on--Zygmund kernel appearing in \eqref{eq:H_sing_integral}.
For readability, the dependence on $\xi(x)$ is often suppressed later in the text; any use of CRW/commutator estimates
must account for this dependence.}

{\color{magenta}\noindent\textbf{[AI AUDIT / CPM-STYLE ``FREEZE THE KERNEL'' STEP.]}
The dependence of $K_{\xi(x)}$ on the frozen direction is \emph{linear} in $\xi(x)$ for the Biot--Savart-derived formula in Lemma~\ref{lem:xi-derivative}.
Consequently, for any fixed $a\in S^2$, the difference operator $(T_{\xi(x)}-T_a)$ has kernel bounded by $C|\xi(x)-a|/|x-y|^3$ and is a Calder\'on--Zygmund operator with $L^p$ operator norm $\lesssim |\xi(x)-a|$.
On a small ball where $\xi$ has small mean oscillation (VMO/BMO$_{\le r}$ small), one can choose $a$ to be the local average direction and ``freeze'' the kernel to $T_a$, paying an error controlled by the oscillation of $\xi$.
This is the natural analytic precursor to any referee-checkable CRW commutator estimate in the presence of $x$-dependent frozen kernels.}

The analysis of \( H_{\mathrm{near}} \) is central to our method. A key observation (e.g. see 
\cite{ConstantinFefferman1993}), is that the near‑field term decomposes into:
(i) a \emph{constant-direction} part (obtained by freezing $\xi(y)$ to $\xi(x)$) and
(ii) an \emph{oscillation} part (carrying $\xi(y)-\xi(x)$).
Explicitly, write \( \xi(y) = \xi(x) + (\xi(y) - \xi(x)) \); then
\[
H_{\mathrm{near}}(x) = P_{\xi(x)}\Bigl( 
\int_{B_r(x)} K(x-y)\rho(y)\,\xi(x)\,dy 
+ \mathrm{p.v.} \int_{B_r(x)} K(x-y)\rho(y)\bigl(\xi(y)-\xi(x)\bigr)dy 
\Bigr).
\]
{\color{magenta}\noindent\textbf{[AI AUDIT / STRUCTURE.]}
The cancellation properties of the ``constant-direction'' contribution
$P_{\xi(x)}\!\left(\int_{B_r(x)} K(x-y)\rho(y)\,\xi(x)\,dy\right)$
depend on the \emph{exact} Biot--Savart representation of $P_\xi(S\xi)$.
As discussed in the kernel-consistency note leading to \eqref{eq:H_sing_integral}, the operator involves the contraction with $\xi(x)$ and the projection,
so a referee-checkable depletion argument requires an explicit identity showing that $H_{\mathrm{near}}$ can be rewritten \emph{purely} in terms of the oscillation
$\xi(y)-\xi(x)$ (a true commutator form), so that constant $\xi$ yields $H_{\mathrm{near}}\equiv 0$.
This derivation is not supplied in the current manuscript and must be added (or stated as an explicit hypothesis).}

{\color{magenta}\noindent\textbf{[AI AUDIT / IDENTIFYING THE CONSTANT-DIRECTION REMAINDER.]}
Lemma~\ref{lem:xi-derivative} shows that there is a nontrivial ``constant-direction'' contribution hiding inside the second term:
if one freezes $\xi(y)$ to $\xi(x)$ in that term (i.e.\ replaces $\omega(y)$ by $\rho(y)\,\xi(x)$), then the resulting vector field equals
$\xi(x)\times\nabla((\xi(x)\cdot\nabla)(-\Delta)^{-1}\rho)(x)$ (up to universal constants), which is a fixed Calder\'on--Zygmund operator on $\rho$.
In the \emph{ideal} constant-direction case, $\omega=\rho\,\xi$ with $\xi$ constant and $\nabla\cdot\omega=0$ forces $(\xi\cdot\nabla)\rho=0$, and then
$(\xi\cdot\nabla)(-\Delta)^{-1}\rho\equiv 0$ (Fourier support has $\xi\cdot k=0$), so this term vanishes as it must.
Moreover, using $\nabla\cdot\omega=0$ one has for any fixed $a\in S^2$ the exact identity
$a\cdot\nabla\rho=\nabla\cdot(\rho a-\omega)$, and therefore
\[
a\times\nabla\bigl((a\cdot\nabla)(-\Delta)^{-1}\rho\bigr)
\;=\;a\times\nabla(-\Delta)^{-1}\nabla\cdot(\rho a-\omega).
\]
Taking $a=\xi(x)$ shows that this ``constant-direction'' term can be rewritten as a CZ operator applied to the \emph{direction error} $\rho(\xi(x)-\xi)$.
The remaining issue is to make this cancellation \emph{quantitative} (small in the critical Carleson norm) under the hypotheses available for tangent flows.}

\begin{lemma}[Constant-direction remainder as a CZ operator on the direction error]\label{lem:constdir-remainder}
Let $u$ be smooth and divergence-free on $\R^3$ at a fixed time, with vorticity $\omega=\curl u$. Write $\omega=\rho\,\xi$ on $\{\omega\neq0\}$ and extend $\rho:=|\omega|$ by $0$ on $\{\omega=0\}$. Fix a constant unit vector $a\in\Sbb^2$.
Then, in the sense of distributions on $\R^3$,
\[
a\times\nabla\bigl((a\cdot\nabla)(-\Delta)^{-1}\rho\bigr)
\;=\;a\times\nabla(-\Delta)^{-1}\nabla\cdot(\rho a-\omega).
\]
In particular, since $\rho a-\omega=\rho(a-\xi)$, the left-hand side is a Calder\'on--Zygmund operator applied to the direction error $\rho(a-\xi)$.
\end{lemma}

\begin{proof}
Since $\nabla\cdot\omega=0$, we have $\nabla\cdot(\rho a-\omega)=a\cdot\nabla\rho$ in distributions. Therefore
\[
(a\cdot\nabla)(-\Delta)^{-1}\rho=(-\Delta)^{-1}(a\cdot\nabla\rho)=(-\Delta)^{-1}\nabla\cdot(\rho a-\omega),
\]
and applying $a\times\nabla$ to both sides yields the claim.
\end{proof}

\begin{lemma}[Quantitative consequence: constant-direction term is controlled by a weighted direction error]\label{lem:constdir-weighted-error}
Fix $a\in\Sbb^2$ and define the constant-direction Calder\'on--Zygmund operator on scalars
\[
(T_a f)(x):=a\times\nabla\bigl((a\cdot\nabla)(-\Delta)^{-1}f\bigr)(x).
\]
Then for every $1<p<\infty$ there exists $C_p<\infty$ such that for all vector fields $F:\R^3\to\R^3$,
\[
\|a\times\nabla(-\Delta)^{-1}\nabla\cdot F\|_{L^p(\R^3)}\le C_p\,\|F\|_{L^p(\R^3)}.
\]
In particular, if $\omega=\rho\,\xi$ with $\nabla\cdot\omega=0$, then for each fixed $a\in\Sbb^2$,
\[
\|T_a \rho\|_{L^p(\R^3)} \;=\; \|a\times\nabla(-\Delta)^{-1}\nabla\cdot(\rho(a-\xi))\|_{L^p(\R^3)}
\;\le\; C_p\,\|\rho(a-\xi)\|_{L^p(\R^3)}.
\]
\end{lemma}

\begin{proof}
Each component of $a\times\nabla(-\Delta)^{-1}\nabla\cdot$ is a finite linear combination of Riesz transforms, hence a Calder\'on--Zygmund operator bounded on $L^p$ for $1<p<\infty$.
The final estimate follows from Lemma~\ref{lem:constdir-remainder} with $F=\rho(a-\xi)$.
\end{proof}

{\color{magenta}\noindent\textbf{[AI AUDIT / WHAT THIS IDENTIFIES.]}
Lemma~\ref{lem:constdir-weighted-error} shows that the remaining ``constant-direction'' contribution is quantitatively controlled by the \emph{weighted direction error} $\rho(a-\xi)$.
Thus, beyond VMO of $\xi$ and scale-critical control of $\rho$, closing the near-field constant-direction term requires a mechanism that makes $\rho(\xi-\text{local frozen direction})$ small in a scale-invariant $L^{3/2}$ sense on shrinking cylinders.}

{\color{magenta}\begin{remark}[When the constant-direction remainder becomes easy: bounded vorticity]\label{rem:constdir-easy-Linfty}
If, in addition, one has a local $L^\infty$ bound $\|\rho\|_{L^\infty(Q_{2r_*}(z_0))}\le M$ (for example, this holds for any \emph{running-max}
vorticity-normalized ancient limit; see Lemma~\ref{lem:omega32-runningmax}), then the weighted direction error is automatically small at small scales:
for any choice of unit vector $a$,
\[
r^{-2}\iint_{Q_r(z_0)} |\rho(a-\xi)|^{3/2}
\le (2M)^{3/2}\,r^{-2}\,|Q_r|
\le C\,(2M)^{3/2}\,r^{3},
\qquad (0<r\le r_*),
\]
so $\|\rho(a-\xi)\|_{C^{3/2}(r_*)}\lesssim M^{3/2}r_*^{3}\to0$ as $r_*\to0$.
Combined with Lemma~\ref{lem:constdir-weighted-error} (with $p=3/2$ and localization), this yields smallness of the constant-direction remainder in the critical Carleson norm on sufficiently small scales.
Thus, the main obstruction for this piece in the CKN-tangent-flow framework is precisely the lack of a bounded-vorticity (or comparable equi-integrability) property.%
\end{remark}}

{\color{magenta}\noindent\textbf{[AI AUDIT / NEAR-FIELD REDUCTION (what is now checkable).]}
At the level of the truncated near-field operator, one has the exact algebraic split
\[
H_{\mathrm{near}}(x)=\frac{1}{4\pi}\,\mathcal T_{\xi(x),r}(\rho(\cdot)\xi(x))(x)
\;+\;P_{\xi(x)}\Bigl(\frac{1}{4\pi}\,\mathcal T_{\xi(x),r}\bigl(\rho(\cdot)(\xi(\cdot)-\xi(x))\bigr)(x)\Bigr),
\]
where $\mathcal T_{\xi(x),r}$ denotes the Biot--Savart-derived truncated singular integral in Lemma~\ref{lem:xi-derivative}.
Using $\nabla\cdot\omega=0$, the \emph{full-space} version of the first term is a CZ operator applied to the direction error $\rho(\xi(x)-\xi)$; truncation introduces an explicit tail remainder, so the near-field commutator reduction is now a precise, referee-checkable target.}

{\color{magenta}\noindent\textbf{[AI AUDIT / CHECKABLE COMMUTATOR FORM FOR THE OSCILLATION TERM.]}
Define fixed kernels (for $m,j\in\{1,2,3\}$ and $r:=x-y$)
\[
k_{m,j}(r):=\frac{e_m\times e_j}{|r|^3}-3\,\frac{r_m\,(r\times e_j)}{|r|^5},
\qquad
(T_{m,j,r}f)(x):=\mathrm{p.v.}\int_{B_r(x)} k_{m,j}(x-y)\,f(y)\,dy.
\]
Expanding cross-products shows that for the oscillation piece one has the exact identity
\[
P_{\xi(x)}\Bigl(\frac{1}{4\pi}\,\mathcal T_{\xi(x),r}\bigl(\rho(\cdot)(\xi(\cdot)-\xi(x))\bigr)(x)\Bigr)
\;=\;\frac{1}{4\pi}\,P_{\xi(x)}\sum_{m,j=1}^3 \xi_m(x)\,[T_{m,j,r},\xi_j]\,\rho\,(x),
\]
where $[T,b]f:=T(bf)-b\,Tf$.
Thus, assuming spatial VMO for $\xi$ and a scale-critical $L^{3/2}$ control of $\rho=|\omega|$, the Coifman--Rochberg--Weiss theorem yields smallness of this oscillation term in the critical Carleson norm at small scales (see \texttt{NS\_Unconditional\_Closures\_A\_to\_E.tex}, Lemma~\texttt{lem:D-nearfield-carleson}).}


The dangerous part that can become large is precisely the second term, 
involving the difference \( \xi(y)-\xi(x) \). If the direction field \( \xi \) 
varies slowly (e.g. is Lipschitz with a moderate constant), this term remains 
controllable. Rapid oscillations of \( \xi \), on the other hand, can interact 
with the singular kernel to produce uncontrolled amplification, the mechanism 
that could potentially lead to a finite‑time blow‑up.  

Hence, the geometric regularity criterion can be phrased as follows:  
singular vortex stretching can be tamed provided the vorticity direction does not 
oscillate too violently in regions of intense vorticity.


\subsection{The Geometric Forcing Term}

By analyzing the singular stretching term \( H_{\mathrm{sing}} \), we now turn to 
{\color{magenta}the geometric contributions on the right-hand side of \eqref{eq:direction}.  Geometrically, these arise from the constraint \( |\xi| = 1 \) and 
the coupling between the amplitude \( \rho \) and the direction \( \xi \). They consist 
of two distinct parts:}
\begin{enumerate}
    \item The harmonic map tension term \( |\nabla \xi|^2 \xi \), which is
          normal to the sphere \( \mathbb{S}^2 \). In the equation for \( \xi \), 
          it appears as a Lagrange multiplier such that $|\xi|=1$.
    \item The cross‑term \( 2 P_\xi (\nabla \log \rho \cdot \nabla \xi) \), which 
          is tangential and connects the geometry of the direction field to the 
          gradient of the log‑amplitude \( \log\rho \).
\end{enumerate}

{\color{magenta}Both geometric contributions (the curvature term \( |\nabla \xi|^2 \xi \) and the tangential coupling term \(H_{\mathrm{geom}}\) from \eqref{hgeom}) involve first derivatives and are}
bilinear or quadratic in gradients. Under the  scaling (\ref{scaling}), both terms have the same 
homogeneity as the diffusion term $-\Delta\xi$, placing them at the critical 
dimensional threshold. Unlike the 
nonlocal stretching term \(H_{\mathrm{sing}}\), these geometric contributions are 
purely local and, in analytical practice, can often be controlled through energy 
estimates or interpolation inequalities, provided suitable a priori bounds are 
available on \(\nabla\xi\) and \(\nabla\log\rho\). Nevertheless, their critical 
scaling means that they cannot be treated as negligible error terms in a 
blow-up scenario and must be handled with care in any critical or supercritical 
regularity framework.}



\section{Critical Coercivity of the Stretching Term}

\subsection{VMO Structure of the Direction Field}
A fundamental property of the ancient tangent flow constructed in Lemma \ref{lem:ancient-limit} is the structural regularity of its direction field. The critical energy bound on the gradient of the vorticity (via local energy inequality) implies control on the gradient of the direction.

\begin{lemma}[VMO of Direction Field]\label{lem:vmo}
Let $u^\infty$ be the ancient tangent flow. Then the direction field $\xi^\infty$ belongs to the space of Vanishing Mean Oscillation (VMO) in the spatial variable, locally uniformly in time. Specifically, for any compact $K \subset \R^3 \times (-\infty, 0]$,
\[
\lim_{r \to 0} \sup_{(x,t) \in K} \frac{1}{|B_r|} \int_{B_r(x)} |\xi^\infty(y,t) - (\xi^\infty)_{x,r}(t)| \, dy = 0,
\]
where $(\xi^\infty)_{x,r}(t)$ is the average of $\xi^\infty$ on $B_r(x)$.
\end{lemma}

{\color{magenta}\noindent\textbf{[AI AUDIT: major non-classical input.]}
As written, Lemma~\ref{lem:vmo} does \emph{not} follow from the local energy inequality/compactness alone; it is a genuine additional regularity claim about $\xi^\infty$ (uniform VMO in space, locally uniformly in time).
To keep the proof unconditional, this step needs either (i) a complete proof from hypotheses already established for the tangent flow, or (ii) to be stated explicitly as an extra hypothesis/geometric assumption. In this audit, it is isolated as Assumption~\ref{assump:A-vmo}.}

\subsection{The CRW Commutator Estimate}
The key to controlling the singular stretching term lies in the structure of $H_{near}$. 
{\color{magenta}\noindent\textbf{[AI AUDIT / STRUCTURE.]}
The ``commutator'' representation below is \emph{schematic} and does not follow from $P_{\xi(x)}\xi(x)=0$ alone,
since the kernel acts before the projection (and the correct Biot--Savart kernel for $S\xi$ depends on $\xi(x)$ as noted in \eqref{eq:H_sing_integral}).
To use CRW rigorously, one must supply a derivation that reduces $H_{near}$ to a Calder\'on--Zygmund commutator with multiplier $\xi$ (or else assume such a representation).}
In the present manuscript, the \emph{oscillation} component of $H_{\mathrm{near}}$ has already been reduced to a finite sum of commutators with \emph{fixed} truncated Calder\'on--Zygmund operators; see the explicit identity in the audit block
\textbf{[CHECKABLE COMMUTATOR FORM FOR THE OSCILLATION TERM]} preceding this subsection (cf.\ \eqref{eq:H_sing_integral} and Lemma~\ref{lem:xi-derivative}).

We now record the classical commutator bound that converts small BMO oscillation of $\xi$ into smallness of these commutator terms.

\begin{lemma}[CRW Commutator Estimate]\label{lem:crw}
Let $T$ be a Calder\'on--Zygmund operator on $\R^3$ and let $T_r$ denote a standard truncation at scale $r>0$
(e.g.\ $T_r f(x)=\mathrm{p.v.}\int_{|x-y|<r}K(x-y)f(y)\,dy$ for a CZ kernel $K$).
Then for every $1<p<\infty$ there exists $C_p<\infty$ (depending only on $p$ and CZ constants of $T$) such that for all $r>0$,
\[
\|[T_r,b]f\|_{L^p(\R^3)}\le C_p\,\|b\|_{\BMO(\R^3)}\,\|f\|_{L^p(\R^3)},
\]
where $[T_r,b]f:=T_r(bf)-b\,T_r f$.
\end{lemma}

\begin{proof}
This is the classical Coifman--Rochberg--Weiss commutator theorem \cite{CRW1976}. The dependence on the truncation scale $r$ is uniform.
\end{proof}

{\color{magenta}\begin{remark}[How \ref{lem:crw} is used here]
Lemma~\ref{lem:crw} is applied to the fixed truncated kernels $T_{m,j,r}$ introduced in the commutator identity
\(
P_{\xi(x)}\sum_{m,j}\xi_m(x)\,[T_{m,j,r},\xi_j]\rho
\)
(see the earlier audit block).
Under Assumption~\ref{assump:A-vmo}, the local BMO seminorm of $\xi(\cdot,t)$ on balls of radius $\le r$ is small uniformly for $t$ in compact time intervals.
Combined with Assumption~\ref{assump:B-omega32}, this yields smallness of the \emph{oscillation/commutator piece} of $H_{\mathrm{near}}$ in the critical Carleson norm at sufficiently small scales.
\end{remark}}

\begin{lemma}[Near-field commutator/oscillation term is small in the critical Carleson norm]\label{lem:nearfield-osc-carleson}
Assume Assumptions~\ref{assump:A-vmo} and~\ref{assump:B-omega32}. Let $H_{\mathrm{near}}^{\mathrm{osc}}$ denote the oscillation term
identified in the commutator identity in the audit block preceding this subsection (i.e.\ the term involving $\rho(\xi-\xi(x))$ and commutators $[T_{m,j,r},\xi_j]\rho$).
Then for every $\varepsilon>0$ there exists $r_0>0$ such that for all $0<r\le r_0$,
\[
\sup_{z_0}\ r^{-2}\iint_{Q_r(z_0)} |H_{\mathrm{near}}^{\mathrm{osc}}|^{3/2}\,dx\,dt\ \le\ \varepsilon.
\]
\end{lemma}

\begin{proof}
Fix $z_0=(x_0,t_0)$ and $0<r\le 1$. For a.e.\ $t\in(t_0-r^2,t_0)$ the commutator identity and Lemma~\ref{lem:crw} (with $p=3/2$) give
\[
\|H_{\mathrm{near}}^{\mathrm{osc}}(\cdot,t)\|_{L^{3/2}(B_r(x_0))}
\ \le\ C\,\|\xi(\cdot,t)\|_{\BMO(B_{2r}(x_0))}\,\|\rho(\cdot,t)\|_{L^{3/2}(B_{2r}(x_0))}.
\]
Raising to the $3/2$ power, integrating in $t$, and using Assumption~\ref{assump:B-omega32} yields
\[
r^{-2}\iint_{Q_r(z_0)} |H_{\mathrm{near}}^{\mathrm{osc}}|^{3/2}
\ \le\ C\,\Bigl(\sup_{t\in(t_0-r^2,t_0)}\|\xi(\cdot,t)\|_{\BMO(B_{2r}(x_0))}\Bigr)^{3/2}\,K_0.
\]
Assumption~\ref{assump:A-vmo} implies the BMO seminorm factor tends to $0$ uniformly as $r\to0$ on compact cylinders, hence the right-hand side can be made $\le\varepsilon$ by choosing $r\le r_0(\varepsilon)$.
\end{proof}

\subsection{Tail Control}
For fixed $r>0$, the far-field contribution $H_{\mathrm{tail}}$ is a standard Calder\'on--Zygmund truncation (up to the frozen-direction dependence of the kernel described earlier).
Thus one expects \emph{boundedness} in $L^p$ (uniformly in $r$) via maximal-truncation/Cotlar inequalities, but \emph{not smallness} as $r\to0$ from scale-critical control alone.

\begin{lemma}[Tail boundedness via maximal truncations (no smallness)]\label{lem:tail-bounded}
Let $T$ be a Calder\'on--Zygmund operator on $\R^3$ and let $T_{>r}$ denote a standard truncation
\[
T_{>r}f(x):=\int_{|x-y|>r}K(x-y)\,f(y)\,dy.
\]
Then for every $1<p<\infty$ there exists $C_p$ such that for all $r>0$,
\[
\|T_{>r}f\|_{L^p(\R^3)}\le C_p\,\|f\|_{L^p(\R^3)}.
\]
\end{lemma}

{\color{magenta}\noindent\textbf{[AI AUDIT / CONSEQUENCE.]}
Even granting Assumption~\ref{assump:B-omega32}, Lemma~\ref{lem:tail-bounded} yields only that $H_{\mathrm{tail}}$ is \emph{bounded} in the critical Carleson norm.
Obtaining \emph{smallness as $r\to0$} requires additional input (e.g.\ a vanishing-Carleson hypothesis, or a separate far-field depletion mechanism such as the later pressure/tail route).}

\subsection{Theorem: Forcing Depletion}
Combining the VMO property, the CRW estimate, and the tail control, we arrive at the first main technical result of this paper.

\begin{theorem}[Forcing Depletion]\label{thm:forcing_depletion}
Let $(u^\infty, \xi^\infty)$ be the ancient tangent flow. {\color{magenta}Assume Assumptions~\ref{assump:A-vmo} and~\ref{assump:B-omega32}.}
For any $\varepsilon > 0$, there exists a scale $r_0 > 0$ such that for all $r \le r_0$, the \emph{commutator/oscillation} part of the near-field forcing satisfies the scale-invariant Carleson bound
\[
\sup_{z_0 \in \R^3 \times (-\infty, 0]} r^{-2} \iint_{Q_r(z_0)} |H_{\mathrm{near}}^{\mathrm{osc}}|^{3/2} \, dx \, dt \le \varepsilon.
\]
{\color{magenta}\noindent\textbf{[AI AUDIT.]}
Controlling the remaining constant-direction part of $H_{\mathrm{near}}$ and the full tail $H_{\mathrm{tail}}$ requires additional input (tail depletion / pressure isotropization), as discussed below and in Section~\ref{sec:pressure}.}
\end{theorem}

\begin{proof}
This is exactly Lemma~\ref{lem:nearfield-osc-carleson}.

\end{proof}

This theorem resolves the "oscillation vs. mass" dilemma. It asserts that in the critical regime, the "mass" (represented by $\rho$) cannot generate critical stretching because it is modulated by the "oscillation" (of $\xi$), which vanishes asymptotically. Thus, the primary driver of potential blow-up is quantitatively depleted.

\section{Control of the Geometric Forcing}

\subsection{Bounds on $\nabla \log \rho$}
We now turn to the geometric term $H_{\mathrm{geom}}$. A crucial component is the gradient of the log-amplitude, $\nabla \log \rho$. While the amplitude $\rho$ may blow up, its logarithmic gradient behaves more like a critical energy density. Using the amplitude equation \eqref{eq:amplitude}, which is a drift--diffusion equation with source $\rho(\sigma - |\nabla \xi|^2)$, we can derive scale-invariant $L^2$ bounds.

\begin{lemma}[Caccioppoli Estimate for Log-Amplitude]\label{lem:log_amplitude}
Let $h = \log \rho$. Under the assumption of critical energy bounds on the tangent flow, there exists a constant $C$ such that for any cylinder $Q_r(z_0)$:
\[
r^{-3} \int_{Q_r(z_0)} |\nabla h|^2 \, dx \, dt \le C \left( 1 + r^{-3} \int_{Q_{2r}(z_0)} (|\sigma| + |\nabla \xi|^2) \, dx \, dt \right).
\]
\end{lemma}

\begin{proof}
The amplitude equation is $\partial_t \rho + u \cdot \nabla \rho - \Delta \rho = \rho (\sigma - |\nabla \xi|^2)$.
Dividing by $\rho$, the equation for $h = \log \rho$ is:
\[
\partial_t h + u \cdot \nabla h - \Delta h - |\nabla h|^2 = \sigma - |\nabla \xi|^2.
\]
Using the identity $\Delta h = \rho^{-1} \Delta \rho - \rho^{-2} |\nabla \rho|^2 = \rho^{-1} \Delta \rho - |\nabla h|^2$, we can rewrite the original equation for $\rho$ by multiplying by $-\rho^{-1} \phi^2$ where $\phi$ is a smooth cutoff function for $Q_{2r}$.
Consider the term $\int \rho^{-1} \Delta \rho \phi^2$. Integration by parts yields:
\[
\int \rho^{-1} \Delta \rho \phi^2 = -\int \nabla(\rho^{-1} \phi^2) \cdot \nabla \rho = \int \rho^{-2} |\nabla \rho|^2 \phi^2 - \int \rho^{-1} \nabla \phi^2 \cdot \nabla \rho = \int |\nabla h|^2 \phi^2 - 2 \int \phi \nabla \phi \cdot \nabla h.
\]
Multiplying the amplitude equation by $\rho^{-1} \phi^2$ and integrating, we obtain:
\[
\int (\partial_t \log \rho + u \cdot \nabla \log \rho) \phi^2 - \int \rho^{-1} \Delta \rho \phi^2 = \int (\sigma - |\nabla \xi|^2) \phi^2.
\]
Substituting the Laplacian term:
\[
\int |\nabla h|^2 \phi^2 = \int (\sigma - |\nabla \xi|^2) \phi^2 + \int \partial_t h \phi^2 + \int (u \cdot \nabla h) \phi^2 + 2 \int \phi \nabla \phi \cdot \nabla h.
\]
The time derivative term is handled by integrating by parts in time. The drift term $\int (u \cdot \nabla h) \phi^2 = -\int h \nabla \cdot (u \phi^2)$ is controlled by local energy bounds and standard parabolic Caccioppoli estimates, making the linear terms in $h$ subordinate to the quadratic gradient term.
The source term $\int (\sigma - |\nabla \xi|^2) \phi^2$ provides the dominant contribution on the right-hand side.
Dividing by $r^{-3}$ yields the claimed normalized estimate.
\end{proof}

{\color{magenta}\noindent\textbf{[AI AUDIT.]}
This paragraph invokes ``Serrin bounds on $u$'' to absorb the drift term, but such a Serrin bound is not established for the tangent flow in the current blow-up/compactness framework. As written, this absorption is another non-classical gap (needs proof or an explicit hypothesis).}
The proof relies on testing the equation for $h$ with a cutoff function and absorbing the drift term using the Serrin bounds on $u$. The source terms on the right-hand side are critical quantities: $|\nabla \xi|^2$ is bounded by hypothesis (locally), and $\sigma$ is the stretching term we have just analyzed.

\subsection{Bilinear Estimates}
The cross-term in the geometric forcing is $2 P_\xi (\nabla \log \rho \cdot \nabla \xi)$. We estimate its $L^{3/2}$ norm using the bounds from Lemma \ref{lem:log_amplitude} and the critical energy of $\xi$. By H\"older's inequality:
\[
\int_{Q_r} |(\nabla \log \rho) \cdot \nabla \xi|^{3/2} \le \left(\int_{Q_r} |\nabla \log \rho|^2\right)^{3/4} \left(\int_{Q_r} |\nabla \xi|^6\right)^{1/4}.
\]
More precisely, in the scale-invariant norms, this term is controlled by the product of the energies. Since the energy of $\xi$ is small at small scales (due to VMO), the product is subordinate to the linear terms in the analysis. Specifically, it can be absorbed or treated as a small perturbation.

\subsection{Theorem: Total Forcing Smallness}
We define the total forcing Carleson norm as
\[
\|H\|_{C^{3/2}(r_*)} = \sup_{z_0}\ \sup_{0<r\le r_*} r^{-2} \iint_{Q_r(z_0)} |H|^{3/2} \, dx \, dt,
\qquad (0<r_*\le 1).
\]
Combining the Forcing Depletion Theorem \ref{thm:forcing_depletion} (near-field commutator depletion for $H_{\mathrm{sing}}$), the geometric bounds (for $H_{\mathrm{geom}}$), and an additional tail depletion input, we obtain the following result.

\begin{theorem}[Total Forcing Smallness]\label{thm:total_forcing}
There exists a universal threshold $\delta^* > 0$ such that, for the ancient tangent flow $(u^\infty, \xi^\infty)$ constructed at a singularity, the total forcing $H = H_{\mathrm{sing}} + H_{\mathrm{geom}}$ satisfies
\[
\|H\|_{C^{3/2}(r_0)} \le \delta^*
\]
for some sufficiently small scale $r_0\in(0,1]$.
\end{theorem}
{\color{magenta}\noindent\textbf{[AI AUDIT.]}
This theorem is intended to establish Assumption~\ref{assump:D-forcing}. The proof below currently relies on additional unproved implications (notably involving VMO $\Rightarrow$ small energy at small scales and control of $\nabla\log\rho$), which are flagged in-line.}

\begin{proof}
The total forcing is $H = H_{\mathrm{sing}} + H_{\mathrm{geom}}$.
By Theorem \ref{thm:forcing_depletion}, for any $\varepsilon > 0$, we can find $r_0$ such that the near-field commutator/oscillation part of $H_{\mathrm{sing}}$ is $\le\varepsilon$ in the critical Carleson norm.  Controlling the remaining constant-direction and tail contributions is part of Assumption~\ref{assump:D-forcing}.
{\color{magenta}\noindent\textbf{[AI AUDIT / consistency fix.]}
Recall from \eqref{eq:direction}--\eqref{hgeom} that the curvature term $|\nabla\xi|^2\xi$ is part of the equation structure and is \emph{not} included in the tangential forcing $H$. Thus the geometric forcing term here is the tangential coupling
\[
H_{\mathrm{geom}}=2P_\xi\big((\nabla\log\rho)\cdot\nabla\xi\big).
\]}
The term $B := H_{\mathrm{geom}} = 2 P_\xi (\nabla \log \rho \cdot \nabla \xi)$ is estimated by Hölder's inequality:
\[
\int_{Q_r} |B|^{3/2} \le C \left(\int_{Q_r} |\nabla \log \rho|^2\right)^{3/4} \left(\int_{Q_r} |\nabla \xi|^6\right)^{1/4}.
\]
In the scale-invariant normalization, using the smallness of the VMO energy of $\xi$ at small scales, and the bound on $\nabla \log \rho$ from Lemma \ref{lem:log_amplitude}, we find that $\|H_{\mathrm{geom}}\|_{C^{3/2}}$ is controlled by a power of the local energy of $\xi$.
Since $\xi$ is VMO, its local energy on sufficiently small balls is small.
Thus, $\|H_{\mathrm{geom}}\|_{C^{3/2}}$ becomes small as $r \to 0$.
Combining this with the smallness of $H_{\mathrm{sing}}$, we get $\|H\|_{C^{3/2}} \le \delta^*$ for any target $\delta^*$, provided we go to sufficiently small scales.
\end{proof}

{\color{magenta}\noindent\textbf{[AI AUDIT.]}
The argument above uses ``$\xi$ is VMO $\Rightarrow$ its local energy on sufficiently small balls is small'' and then upgrades this to a scale-uniform Carleson smallness statement. Neither implication is currently justified in the manuscript; this is a key non-classical step that must be proved (or recast as an explicit hypothesis).}

This theorem provides the necessary input for the rigidity analysis of the direction equation: the direction field evolves according to a critical heat flow with a forcing term that is quantitatively small in the relevant scale-invariant space.

\section{Carleson Control and Scaling}\label{sec:carleson}

{\color{magenta}\noindent\textbf{[AI AUDIT: major non-classical input.]}
Any use of extension-energy ``Carleson control'' must be made precise. The classical Caffarelli--Silvestre trace theory provides
\emph{boundedness} of a parabolic Carleson functional \(\|\mathcal E[f]\|_{C}\) \emph{provided one already has} a scale-invariant local enstrophy bound
for \(|\nabla f|^2\).  The manuscript does not currently derive such an enstrophy bound for \(f=|\omega|\) from the suitable weak solution framework,
so that step must be proved separately (or isolated as an explicit hypothesis).}

\begin{definition}[Harmonic extension and local extension energy]\label{def:extension-energy}
Let $f:\R^3\to\R$ be locally square-integrable. Let $F:\R^3\times(0,\infty)\to\R$ denote its harmonic extension to the upper half-space:
\[
-\Delta_{x,z}F=0\quad(z>0),\qquad F(\cdot,0)=f(\cdot).
\]
For $x_0\in\R^3$, $r>0$ define the localized extension energy
\[
E_r[f](x_0)\;:=\;\int_{B_r(x_0)}\int_0^r z\,|\nabla_{x,z}F(x,z)|^2\,dz\,dx.
\]
For a space-time function $f(x,t)$ we write $E_r(x_0,t):=E_r[f(\cdot,t)](x_0)$.
We also define the associated \emph{parabolic Carleson functional}
\[
\|\mathcal E[f]\|_{C}\;:=\;\sup_{z_0=(x_0,t_0)}\ \sup_{0<r\le 1}\ r^{-1}\int_{t_0-r^2}^{t_0} E_r(x_0,t)\,dt.
\]
\end{definition}

\begin{proposition}[Time-averaged extension-energy Carleson bound from an enstrophy bound]\label{thm:carleson-control}
Let $f:\R^3\times I\to\R$ be such that $f(\cdot,t)\in H^1_{\mathrm{loc}}(\R^3)$ for a.e.\ $t\in I$.
Assume there exists $K<\infty$ such that for every $z_0=(x_0,t_0)$ and every $0<r\le 1$ with $(t_0-r^2,t_0)\subset I$,
\[
r^{-1}\iint_{Q_r(z_0)}|\nabla_x f(x,t)|^2\,dx\,dt\ \le\ K.
\]
Then the parabolic extension-energy functional in Definition~\ref{def:extension-energy} is finite and obeys
\[
\|\mathcal E[f]\|_{C}\ \le\ C\,K,
\]
where $C=C(3)$ is a universal dimensional constant.
\end{proposition}

\begin{proof}
For each fixed $t$, the harmonic extension characterization of the $\dot H^{1/2}$ seminorm (Caffarelli--Silvestre \cite{CaffarelliSilvestre2007})
and a standard localization/cutoff argument yield a bound of the form
\[
E_r[f(\cdot,t)](x_0)\ \le\ C \int_{B_{2r}(x_0)} |\nabla_x f(x,t)|^2\,dx,
\]
uniformly for $0<r\le1$.
Integrating in time over $(t_0-r^2,t_0)$ gives
\[
\int_{t_0-r^2}^{t_0}E_r(x_0,t)\,dt
\le C\iint_{Q_{2r}(z_0)}|\nabla_x f|^2\,dx\,dt
\le C\,K\,(2r),
\]
and dividing by $r$ yields the desired Carleson bound.
\end{proof}

{\color{magenta}\begin{remark}[What remains to use \ref{thm:carleson-control} with $f=|\omega|$]
To apply Proposition~\ref{thm:carleson-control} with $f=|\omega|$ (or $f=\omega$ componentwise), one must prove a scale-invariant local enstrophy bound of the form
\(
r^{-1}\iint_{Q_r(z_0)}|\nabla_x|\omega||^2\le K
\)
or a comparable bound on $|\nabla\omega|^2$.
Such an estimate is not produced by the CKN tangent-flow compactness alone; it holds under additional hypotheses (e.g.\ Type~I/enstrophy control) but remains a genuine open input in the present manuscript.%
\end{remark}}

\begin{lemma}[Scaling Invariance]\label{thm:carleson-scaling}
Under the N--S scaling $x\mapsto \lambda x$, $t\mapsto \lambda^2 t$, the functional $\|\mathcal E[f]\|_{C}$ in Definition~\ref{def:extension-energy} is scale-invariant.
\end{lemma}

\begin{corollary}[Carleson Stability for Tangent Flows]\label{cor:carleson-min}
Let $u^{(k)}$ be a blow-up sequence producing a limit $u^\infty$. Then
\[
\|\mathcal{E}^\infty\|_{C} \le \liminf_{k\to\infty} \|\mathcal{E}^{(k)}\|_{C} \le K_*.
\]
In particular, the Carleson norm is stable along blow-up limits; scaling alone cannot generate arbitrary smallness.
\end{corollary}

\begin{proof}
Lower semicontinuity of the Carleson density under local convergence, together with the uniform bound from Theorem \ref{thm:carleson-control}, yields the liminf inequality. Since the normalized density is scale-invariant, rescaling cannot produce smallness beyond what is present in the sequence.
\end{proof}

\section{Pressure Isotropization and Tail Depletion}\label{sec:pressure}

{\color{magenta}\noindent\textbf{[AI AUDIT: major non-classical input.]}
The pressure/strain ``coercivity'' mechanism and the subsequent spherical-harmonic anisotropy defect estimates appear to be new. They are not classical black-box inputs and currently lack detailed hypotheses and derivations sufficient for referee verification.}

To robustly control the far-field contribution of the stretching, we quantify how pressure enforces isotropy of the deviatoric strain at small scales.

\begin{theorem}[Pressure Coercivity]\label{thm:pressure-coercivity}
Let $S=\tfrac12(\nabla u + \nabla u^T)$ and $S_{dev}=S - \tfrac13 (\operatorname{tr}S) I$. For any $R>0$ and cutoff $\phi\in C_c^\infty(B_{2R})$ with $\phi\equiv 1$ on $B_R$, one has
\[
\frac12 \frac{d}{dt}\int_{B_R} |S_{dev}|^2 + \frac{\nu}{2} \int_{B_R} |\nabla S_{dev}|^2
\le C \|u\|_{L^3(B_{2R})}^4 \int_{B_R} |S_{dev}|^2 + C R^{-2} \int_{B_{2R}} |S_{dev}|^2.
\]
\end{theorem}

\begin{proof}
Differentiate the strain equation, use $\Delta p = -\nabla\cdot\nabla\cdot(u\otimes u)$ and Calder\'on--Zygmund estimates to control $\nabla^2 p$ in $L^{3/2}$, then test against $S_{dev}\phi^2$ and absorb a portion of $\|\nabla S_{dev}\|_2^2$.
\end{proof}

\begin{lemma}[Defect vs. Strain]
Let $\Omega$ denote the rescaled vorticity profile on the annulus $|w|>1$ and measure anisotropy via $\mathfrak{D}_{aniso}(\Omega)$ (quadratic form on the $\ell=2$ spherical harmonic sector). Then
\[
\mathfrak{D}_{aniso}(\Omega)^2 \le C \iint_{Q_1} |S_{dev}|^2.
\]
\end{lemma}

\begin{corollary}[Tail Depletion]\label{cor:tail-depletion}
For tangent flows, the tail coefficient $C_{stretch}$ associated with the far-field stretching satisfies $|C_{stretch}|\to 0$ along small scales. Consequently, $|H_{tail}|$ is negligible in the critical Carleson norm.
\end{corollary}

\begin{proof}
Pressure coercivity yields dissipation control of $S_{dev}$, which bounds the anisotropy defect and hence the tail coefficient via the spherical-harmonic representation of the stretching kernel. As scales shrink, the localized dissipation and therefore the defect vanish, forcing $|C_{stretch}|\to 0$.
\end{proof}

\section{The Directional Liouville Theorem}

\subsection{The Critical Drift--Diffusion System}
We have reduced the problem to the analysis of the ancient direction field $\xi^\infty$ satisfying
\begin{equation}\label{eq:DDE}
{\color{magenta}\partial_t \xi - \Delta \xi + u \cdot \nabla \xi = |\nabla \xi|^2 \xi + H, \quad |\xi|=1, \quad H \cdot \xi = 0.}
\end{equation}
{\color{magenta}\noindent\textbf{[AI AUDIT.]}
The statement ``$u$ satisfies Serrin-type bounds (inherited from the tangent flow critical norms)'' is not justified by the blow-up construction in Lemma~\ref{lem:ancient-limit} as currently written; the tangent-flow compactness yields local energy/\(L^3\) bounds, but not a Serrin bound. This is another non-classical gap that must be proved or assumed.}
Here, $H$ satisfies the smallness condition $\|H\|_{C^{3/2}} \le \delta^*$.

\subsection{Energy Decay Estimates}
To prove rigidity, we establish a decay estimate for the scale-invariant energy $E(r) = r^{-3} \int_{Q_r} |\nabla \xi|^2$. We start with a Caccioppoli inequality for the equation \eqref{eq:DDE}. Testing with $-\Delta(\phi^2 \xi)$ and using the constraint $|\xi|=1$ yields:
\[
\int_{Q_{r/2}} |\nabla^2 \xi|^2 \le C r^{-2} \int_{Q_r} |\nabla \xi|^2 + C \int_{Q_r} |u|^2 |\nabla \xi|^2 + C \int_{Q_r} |H|^2.
\]
The drift term involves $|u|^2 |\nabla \xi|^2$. Since $u$ is in a Serrin class ($L^q_t L^p_x$ with $2/q+3/p \le 1$), this term can be absorbed into the left-hand side (the Hessian term) plus a linear term using interpolation inequalities. The forcing term is small by hypothesis.

Combining this with Poincaré inequalities, we derive a one-step Campanato decay estimate.

\begin{lemma}[One-Step Energy Decay]\label{lem:decay}
There exist constants $\theta \in (0,1)$ and $C > 0$ such that if $E(r) \le \varepsilon_0$ and $\|H\|_{C^{3/2}} \le \delta^*$, then
\[
E(r/2) \le \theta E(r) + C (\delta^*)^2.
\]
\end{lemma}

\begin{proof}
This is the core $\varepsilon$-regularity estimate (Theorem DDE\_Epsilon\_Regularity in the proof track).
Let $r=1$ by scaling. We assume $E(1) \le \varepsilon_0$.
{\color{magenta}We test the equation $\partial_t \xi - \Delta \xi + u \cdot \nabla \xi = |\nabla \xi|^2 \xi + H$ with $-\Delta (\phi^2 \xi)$, where $\phi$ is a cutoff for $B_{1/2}$.}
Using $\xi \cdot \Delta \xi = -|\nabla \xi|^2$, we obtain a Caccioppoli inequality:
\[
\int_{Q_{1/2}} |\nabla^2 \xi|^2 \le C \int_{Q_1} |\nabla \xi|^2 + C \int_{Q_1} |u|^2 |\nabla \xi|^2 + C \int_{Q_1} |H|^2.
\]
The drift term $\int |u|^2 |\nabla \xi|^2$ is absorbed into the LHS using the Serrin bound on $u$ and interpolation (parabolic Sobolev embedding), provided $\varepsilon_0$ is small.
Specifically, $\int |u|^2 |\nabla \xi|^2 \le \|u\|_{Serrin}^2 \|\nabla \xi\|_{L^{p'}}^2 \le \varepsilon \|\nabla^2 \xi\|_2^2 + C_\varepsilon \|\nabla \xi\|_2^2$.
The forcing term $\int |H|^2$ is controlled by $\|H\|_{C^{3/2}}^2$ via Hölder (or directly assuming $L^2$ smallness, but $C^{3/2}$ is the natural space; smallness in $C^{3/2}$ implies smallness in the relevant energy deviation).
Combining these, we get:
\[
\int_{Q_{1/2}} |\nabla^2 \xi|^2 \le C E(1) + C (\delta^*)^2.
\]
Then, using the Poincaré inequality (subtracting the mean drift or using harmonic replacement), we compare $\xi$ to a harmonic map heat flow or linear heat equation solution. The energy on the smaller ball $E(1/2)$ improves by a factor $\theta$ (coming from the regularity of the homogeneous equation) plus the perturbation errors.
Thus $E(1/2) \le \theta E(1) + C (\delta^*)^2$.
\end{proof}
By choosing $\delta^*$ sufficiently small (which is possible by Theorem \ref{thm:total_forcing}), the forcing term becomes negligible relative to the decay.

\subsection{Epsilon-Regularity}
\begin{theorem}[DDE $\varepsilon$-Regularity]\label{thm:DDE-eps-regularity}
There exist universal constants $\eps_*>0$, $\delta_*>0$, $\alpha\in(0,1)$, and $C<\infty$ such that, if on $Q_1(z_0)$ the direction equation
\[
\partial_t \xi - \Delta \xi + u \cdot \nabla \xi = H, \qquad |\xi|=1,\quad H\cdot \xi=0
\]
holds with $u$ in a Serrin class and
\[
E(z_0,1)\le \eps_*^2, \qquad \|H\|_{C^{3/2}}\le \delta_*,
\]
then for all $\rho\le \tfrac12$,
\[
E(z_0,\rho) \le C \rho^{2\alpha} E(z_0,1),
\]
and, in particular,
\[
\sup_{Q_{1/2}(z_0)} |\nabla \xi| \le C \eps_*.
\]
\end{theorem}
{\color{magenta}\noindent\textbf{[AI AUDIT: non-classical step.]}
The DDE $\varepsilon$-regularity statement above is not a standard published theorem in this exact setting (sphere-valued drift--diffusion with Serrin drift and Carleson forcing). As written it relies on an external ``proof track'' rather than a complete argument; this must be fully supplied to count as unconditional.}
Iterating the decay estimate from Lemma \ref{lem:decay} yields the theorem by a standard Campanato iteration and absorption of the drift term using the Serrin bound for $u$.

\subsection{Rigidity via Blow-up}
We now prove the main rigidity result.

\begin{theorem}[Directional Liouville]\label{thm:liouville}
Let $\xi$ be an ancient solution to \eqref{eq:DDE} on $\R^3 \times (-\infty, 0]$ satisfying the critical energy bounds and the small forcing condition $\|H\|_{C^{3/2}} \le \delta^*$ with $\delta^*$ sufficiently small. Then $\xi$ must be spatially constant: $\nabla \xi \equiv 0$.
\end{theorem}
{\color{magenta}\noindent\textbf{[AI AUDIT.]}
This theorem is intended to provide Assumption~\ref{assump:C-liouville}. At present, only the $\varepsilon$-regularity (Hölder regularity) part is justified from Lemma~\ref{lem:decay} / Theorem~\ref{thm:DDE-eps-regularity}; the global rigidity/constancy conclusion remains a missing non-classical input.}

\begin{proof}
\textbf{What is currently justified.}
Under the smallness hypotheses (small $E$ on a unit cylinder and small forcing in the critical norm), iterating Lemma~\ref{lem:decay} yields the $\varepsilon$-regularity estimate of Theorem~\ref{thm:DDE-eps-regularity}, and hence local $C^\alpha$ (Campanato) regularity of $\xi$.

\medskip
\noindent\textbf{What remains open.}
The decay of the scale-invariant quantity $E(r)=r^{-3}\iint_{Q_r}|\nabla\xi|^2$ as $r\to0$ does \emph{not} imply $\nabla\xi\equiv0$ (see the audit note preceding this proof in earlier versions).
Upgrading $\varepsilon$-regularity to a genuine \emph{Liouville rigidity} statement requires an additional global rigidity mechanism (or extra hypotheses placing $\xi$ in a suitable Liouville class).
This is one of the key non-classical closure items and is isolated as Assumption~\ref{assump:C-liouville}.
\end{proof}

\section{Classification and Contradiction}

\subsection{Time-Constancy of the Direction}
Assuming Assumption~\ref{assump:C-liouville} (directional Liouville rigidity), the ancient direction field is constant in space-time:
\[
\xi^\infty(x,t)\equiv b_0\in\Sbb^2.
\]

\subsection{Reduction to 2D Dynamics}
Rotate coordinates so that the constant direction is $b_0=e_3=(0,0,1)$, hence
\[
\omega^\infty=\curl u^\infty=(0,0,\alpha).
\]
The identities $\omega^\infty_1=\omega^\infty_2=0$ together with $\dv u^\infty=0$ imply that, for each fixed time $t$,
\[
\Delta u^\infty_3(\cdot,t)=0 \quad\text{in }\R^3.
\]
If, in addition, $u^\infty_3(\cdot,t)$ lies in a harmonic Liouville class (e.g.\ is bounded or belongs to $L^p(\R^3)$ for some $1\le p<\infty$),
then $u^\infty_3(\cdot,t)$ is spatially constant; after subtracting a constant Galilean drift one may assume $u^\infty_3\equiv 0$.
In that case $\omega^\infty_1=\omega^\infty_2=0$ gives $\partial_3 u^\infty_1=\partial_3 u^\infty_2=0$, so the flow reduces to a two-dimensional velocity field in the plane perpendicular to $e_3$:
\[
u^\infty(x,t) = (v_1(x_1, x_2, t), v_2(x_1, x_2, t), 0).
\]
{\color{magenta}\noindent\textbf{[AI AUDIT.]}
The reduction $u^\infty_3\equiv0$ (and hence $\partial_3 u^\infty\equiv0$) requires an additional Liouville-class hypothesis for the harmonic function $u^\infty_3$ (and for the ensuing $x_3$-independence), which is not provided by Lemma~\ref{lem:ancient-limit}. This is part of Assumption~\ref{assump:E-2d}.}

\begin{lemma}[Vanishing Stretching]\label{lem:vanishing_stretching}
If the vorticity direction of a N--S solution is constant in space and time, the vortex stretching term is identically zero.
\end{lemma}

\begin{proof}
Let the direction be constant, $\xi(x,t) \equiv e_3$. Then $\omega = (0, 0, \omega_3)$.
The vortex stretching term is given by $(\omega \cdot \nabla) u = \omega_3 \partial_3 u$.
As discussed in the reduction above, the conclusion that $\partial_3 u\equiv 0$ follows once one assumes the needed Liouville-class hypothesis for the harmonic component $u_3$ (cf.\ Assumption~\ref{assump:E-2d}).
Consequently, $(\omega \cdot \nabla) u = \omega_3 \cdot 0 = 0$.
\end{proof}

{\color{magenta}\begin{remark}[A weaker but unconditional identity from the vorticity equation]\label{rem:constdir-weak-vanish}
Even without any Liouville/growth hypothesis, if $\omega=\rho e_3$ solves the vorticity equation
$\partial_t\omega + u\cdot\nabla\omega - \Delta\omega = \omega\cdot\nabla u$ in the distributional sense, then the first two components imply
\[
0=\rho\,\partial_3 u_1,\qquad 0=\rho\,\partial_3 u_2
\]
in distributions. Thus $\partial_3 u_h=0$ holds on the set $\{\rho\neq 0\}$ (in the a.e.\ sense).
Upgrading this to $\partial_3 u\equiv 0$ globally (hence true 2D dynamics and vanishing stretching everywhere) requires additional global information
about the harmonic component of $u_3$ or a unique-continuation mechanism that is not currently available from Lemma~\ref{lem:ancient-limit}. This is part of Assumption~\ref{assump:E-2d}.%
\end{remark}}

\subsection{2D Ancient Liouville Theorem}
If, in addition, the constant-direction tangent flow belongs to a 2D Liouville class as isolated in Assumption~\ref{assump:E-2d},
then one may reduce to an ancient 2D Navier--Stokes flow on $\R^2\times(-\infty,0]$ and invoke a classical 2D Liouville theorem.
{\color{magenta}\noindent\textbf{[AI AUDIT.]}
Lemma~\ref{lem:ancient-limit} provides local energy and local \(L^3\) control, but does not establish the global Liouville-class hypotheses
(boundedness / global integrability / decay at infinity) needed to apply \cite{KNSS2009}. Those are exactly what is isolated in Assumption~\ref{assump:E-2d}.}

Known Liouville theorems for the 2D N--S equations state that any bounded ancient solution must be constant (essentially due to the monotonicity of enstrophy in 2D).

\begin{theorem}[2D Ancient Liouville]\label{thm:2d_liouville}
Let $u$ be a bounded ancient solution to the 2D N--S equations on $\R^2 \times (-\infty, 0]$. Then $u$ is a constant (specifically $u \equiv 0$ for finite energy).
\end{theorem}

\begin{proof}
\textbf{Step 1: Enstrophy identity.}
For a bounded ancient 2D solution $u=(v_1,v_2)$ on $\R^2\times(-\infty,0]$, the scalar vorticity
$\alpha=\partial_1 v_2-\partial_2 v_1$ satisfies
\[
\partial_t \alpha + v \cdot \nabla \alpha = \nu \Delta \alpha.
\]
Multiply by $\alpha$ and integrate over $\R^2$:
\[
\frac{1}{2} \frac{d}{dt} \|\alpha\|_2^2 + \nu \|\nabla \alpha\|_2^2 = 0.
\]
This implies $\|\alpha(t)\|_2$ is non-increasing.
\textbf{Step 2: Liouville conclusion.}
We rely on the Liouville theorem for bounded ancient 2D flows (Koch--Nadirashvili--Seregin--\v{S}ver\'{a}k \cite{KNSS2009}), which concludes that any bounded ancient 2D Navier--Stokes solution is constant.
\end{proof}

\subsection{The Final Contradiction}
Assuming Assumptions~\ref{assump:C-liouville} and~\ref{assump:E-2d}, the constant-direction tangent flow belongs to a 2D Liouville class,
so the classical 2D Liouville theorem (e.g.\ \cite{KNSS2009}) forces $u^\infty\equiv 0$ and hence $\omega^\infty\equiv 0$.

However, Lemma~\ref{lem:ancient-limit}(iii) guarantees that the ancient tangent flow is \emph{non-trivial}: there exist $r>0$ and $c>0$ such that
\[
\int_{Q_r(0,0)} |u^\infty|^3 \,dx\,dt \ge c>0.
\]
This contradicts $u^\infty \equiv 0$.

Therefore, the initial assumption that a finite-time singularity exists must be false.

\bibliographystyle{amsplain}
\begin{thebibliography}{10}

\bibitem{BKM1984}
J.~T. Beale, T.~Kato, and A.~Majda, \emph{Remarks on the breakdown of smooth solutions for the 3-{D} {E}uler equations}, Comm. Math. Phys. \textbf{94} (1984), no.~1, 61--66.

{\color{blue} \bibitem{CFM1996} C. Fefferman, A. J. Majda,  {\it Geometric constraints on potentially,} Communications in Partial Differential Equations, 21(3–4) (1996)., 559–571. https://doi.org/10.1080/03605309608821197}

%\bibitem{CFM1996}
%P.~Constantin, C.~Fefferman, and A.~Majda, \emph{Geometric constraints on potentially singular solutions for the 3-{D} {E}uler equations}, Comm. Partial Differential Equations \textbf{21} (1996), no.~3-4, 559--571.

\bibitem{CKN1982}
L.~Caffarelli, R.~Kohn, and L.~Nirenberg, \emph{Partial regularity of suitable weak solutions of the {N}avier-{S}tokes equations}, Comm. Pure Appl. Math. \textbf{35} (1982), no.~6, 771--831.

%\bibitem{ESS2003}
%L.~Escauriaza, G.~Seregin, and V.~{\v{S}}ver{\'a}k, \emph{{$L_{3,\infty}$}-solutions of {N}avier-{S}tokes equations and backward uniqueness}, Uspekhi Mat. Nauk \textbf{58} (2003), no.~2(350), 3--44.

\bibitem{ESS2003}
{\color{blue}L.~Escauriaza, G.~A.~Seregin, and V.~\v{S}ver\'{a}k,
\emph{$L_{3,\infty}$-solutions of the N--S equations and backward uniqueness},
Uspekhi Mat.\ Nauk \textbf{58} (2(350)) (2003), 3--44;
translation in Russian Math.\ Surveys \textbf{58} (2003), no.~2, 211--250.}


\bibitem{Fefferman2006} {\color{blue}
C. L. Fefferman, Existence and smoothness of the Navier-Stokes
equation. In J.A. Carlson, A. Jaffe, A. Wiles, Clay Mathematics Institute,
and American Mathematical Society, editors, \emph{The Millennium
Prize Problems}, 57–67. American Mathematical Society, 2006.}

\bibitem{Hopf1951}
E.~Hopf, \emph{{\"U}ber die {A}nfangswertaufgabe f{\"u}r die hydrodynamischen {G}rundgleichungen}, Math. Nachr. \textbf{4} (1951), 213--231.

\bibitem{CRW1976}
{\color{blue} R.~R. Coifman, R.~Rochberg, and G.~Weiss, \emph{Factorization theorems for Hardy spaces in several variables}, Ann. of Math. (2) \textbf{103} (1976), no.~3, 611--635.}
%%izbrisana zadnja recenica


\bibitem{CaffarelliSilvestre2007} {\color{blue}
L.~Caffarelli and L.~Silvestre, \emph{An extension problem related to the fractional Laplacian}, Comm. Partial Differential Equations \textbf{32} (2007), no.8, 1245--1260.} 

%pisalo no.7-9, a treba 8

\bibitem{KNSS2009}
G.~Koch, N.~Nadirashvili, G.~Seregin, and V.~{\v{S}}ver{\'a}k, \emph{Liouville theorems for the {N}avier-{S}tokes equations and applications}, Acta Math. \textbf{203} (2009), no.~1, 83--105.




\bibitem{KochTataru2001}
H.~Koch and D.~Tataru, \emph{Well-posedness for the {N}avier-{S}tokes equations}, Adv. Math. \textbf{157} (2001), no.~1, 22--35.




\bibitem{Leray1934}
J.~Leray, \emph{Sur le mouvement d'un liquide visqueux emplissant l'espace}, Acta Math. \textbf{63} (1934), no.~1, 193--248.

{\color{blue}\bibitem{Lemarie2016}
P.-G.~Lemari\'e\mbox{-}Rieusset,
\emph{The Navier--Stokes Problem in the 21st Century},
Chapman \& Hall/CRC, Boca Raton, FL, 2016.}

\bibitem{Lin1998}
F.~Lin, \emph{A new proof of the {C}affarelli-{K}ohn-{N}irenberg theorem}, Comm. Pure Appl. Math. \textbf{51} (1998), no.~3, 241--257.

\bibitem{Prodi1959}
G.~Prodi, \emph{Un teorema di unicit{\`a} per le equazioni di {N}avier-{S}tokes}, Ann. Mat. Pura Appl. (4) \textbf{48} (1959), 173--182.

 

\bibitem{Scheffer1977}
V.~Scheffer, \emph{Hausdorff measure and the {N}avier-{S}tokes equations}, Comm. Math. Phys. \textbf{55} (1977), no.~2, 97--112.

\bibitem{Seregin2012}
G.~Seregin, \emph{A certain necessary condition of potential blow up for {N}avier-{S}tokes equations}, Comm. Math. Phys. \textbf{312} (2012), no.~3, 833--845.

 

\bibitem{Serrin1962}
J.~Serrin, \emph{On the interior regularity of weak solutions of the {N}avier-{S}tokes equations}, Arch. Rational Mech. Anal. \textbf{9} (1) (1962), 187--195.


\bibitem{ConstantinFefferman1993}
{\color{blue}P.~Constantin and C.~Fefferman,
\emph{Direction of vorticity and the problem of global regularity for the Navier--Stokes equations},
Indiana Univ.\ Math.\ J.\ \textbf{42} (1993), no.~3, 775--789.}

\bibitem{MajdaBertozzi2002}
{\color{blue}A.~J.~Majda and A.~L.~Bertozzi,
\emph{Vorticity and Incompressible Flow},
Cambridge Texts in Applied Mathematics, Cambridge Univ.\ Press, 2002.}

\bibitem{15} {\color{blue}
Y. Giga, {\it Solutions for semilinear parabolic equations in Lp and regularity of weak solutions of the Navier-Stokes system,} J. Differential Equations, 62(1986), 186-212.}

\bibitem{25} {\color{blue}
 J. Serrin, {\it The initial value problem for the Navier-stokes equations, in Nonlinear problems(R. E. Langer Ed.),} pp.69-98, Univ. of Wisconsin Press, Madison, 1963.}

\bibitem{27} {\color{blue}
M. Struwe, {\it On partial regularity results for the Navier-Stokes equations,} Comm. Pure
Appl. Math., 41(1988), 437-458.}

\bibitem{23} {\color{blue}
J.~Ne\v{c}as, M.~R{u}\v{z}i\v{c}ka and V.~\v{S}ver\'ak,
{\it On Leray's self-similar solutions of the Navier--Stokes equations},
\newblock {\em Acta Mathematica}, {\bf 176} (1996), 283--294.}

\bibitem{6} {\color{blue}
G. P. Galdi, {\it An Introduction to the Navier-Stokes Initial-Boundary Value
Problem}, In: Fundamental Directions in Mathematical Fluid Mechanics.
Basel: Birkhäuser, 2000, pages 1–70.}

\bibitem{Aubin1963} {\color{blue}
J.-P.\ Aubin,
\emph{Un théorème de compacité},
C.\ R.\ Acad.\ Sci.\ Paris \textbf{256} (1963), 5042--5044.}

\bibitem{Lions1969} {\color{blue}
J.-L.\ Lions,
\emph{Quelques méthodes de résolution des problèmes aux limites non linéaires},
Dunod; Gauthier-Villars, Paris, 1969.}

\bibitem{Kobayashi} {\color{blue}
M. Kobayashi. {\it On the Navier-Stokes equations on manifolds with curvature,} J. Eng. Math. (2008) 60:55–68.}

\bibitem{LG} {\color{blue}
O. A. Ladyzhenskaya and G. A. Seregin, {\it On partial regularity of suitable weak
solutions to the three-dimensional Navier–Stokes equations,} J. Math. Fluid Mech. 1
(1999), 356-387.}

\end{thebibliography}

\end{document}
