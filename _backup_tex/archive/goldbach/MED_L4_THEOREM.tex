\documentclass[11pt]{article}
\usepackage[utf8]{inputenc}
\usepackage[T1]{fontenc}
\usepackage{amsmath,amssymb,amsthm}
\usepackage{geometry}
\usepackage{hyperref}
\geometry{margin=1in}

\newtheorem{theorem}{Theorem}
\newtheorem{conjecture}[theorem]{Conjecture}
\newtheorem{lemma}[theorem]{Lemma}
\theoremstyle{definition}
\newtheorem{definition}[theorem]{Definition}
\theoremstyle{remark}
\newtheorem{remark}[theorem]{Remark}

\title{The Medium-Arc $L^4$ Saving Conjecture\\[0.5em]
\large A Sufficient Condition for Goldbach via the Circle Method}
\author{Recognition Physics Institute}
\date{December 2025}

\begin{document}
\maketitle

\begin{abstract}
We formulate a precise conjecture about exponential sums over primes on medium arcs in the circle method. This conjecture, if true, would imply that every sufficiently large even integer is a sum of two primes. We state the conjecture, explain its context, and outline a potential proof strategy based on dispersion methods.
\end{abstract}

\section{The Conjecture}

Let $N \geq e^{100}$ be a large integer. Define the exponential sum over primes
\[
S(\alpha) = \sum_{n \leq 2N} \Lambda(n) \, e(\alpha n) \, \eta\!\left(\frac{n}{N}\right),
\]
where $\Lambda$ is the von Mangoldt function, $e(x) = e^{2\pi i x}$, and $\eta \in C_c^\infty((0,2))$ is a smooth cutoff with $\eta \equiv 1$ on $[\tfrac14, \tfrac74]$.

Let $\chi_8$ be the primitive real Dirichlet character modulo 8, defined by
\[
\chi_8(n) = \begin{cases}
0, & n \equiv 0, 2, 4, 6 \pmod{8}, \\
+1, & n \equiv 1, 7 \pmod{8}, \\
-1, & n \equiv 3, 5 \pmod{8}.
\end{cases}
\]
Define the twisted sum $S_{\chi_8}(\alpha) = \sum_{n \leq 2N} \Lambda(n) \chi_8(n) e(\alpha n) \eta(n/N)$.

\begin{definition}[Arc Decomposition]
Fix parameters
\[
Q = \frac{N^{1/2}}{(\log N)^4}, \qquad Q' = \frac{N^{2/3}}{(\log N)^6}.
\]
The \emph{medium arcs} are
\[
\mathfrak{M}_{\mathrm{med}} = \bigcup_{Q < q \leq Q'} \bigcup_{\substack{a \bmod q \\ (a,q)=1}} \left\{ \alpha : \left|\alpha - \frac{a}{q}\right| \leq \frac{Q'}{qN} \right\} \setminus \mathfrak{M},
\]
where $\mathfrak{M}$ is the union of major arcs (those with $q \leq Q$).
\end{definition}

\begin{conjecture}[MED-L4: Medium-Arc $L^4$ Saving]\label{conj:med-l4}
There exist absolute constants $C_{\mathrm{disp}} > 0$ and $\delta_{\mathrm{med}} > 0$ such that for all $N \geq e^{100}$,
\begin{equation}\label{eq:med-l4}
\boxed{
\int_{\mathfrak{M}_{\mathrm{med}}} \left( |S(\alpha)|^4 + |S_{\chi_8}(\alpha)|^4 \right) d\alpha \leq C_{\mathrm{disp}} \cdot N^2 \cdot (\log N)^{4 - \delta_{\mathrm{med}}}.
}
\end{equation}
Moreover, one may take $\delta_{\mathrm{med}} \geq 10^{-3}$ and $C_{\mathrm{disp}} \leq 10^3$.
\end{conjecture}

\begin{remark}
The trivial bound for this integral is $O(N^2 (\log N)^4)$. The conjecture asserts a \emph{logarithmic power saving}: the exponent drops from $4$ to $4 - \delta_{\mathrm{med}}$. Any positive $\delta_{\mathrm{med}} > 0$ suffices for the application to Goldbach.
\end{remark}

\section{Consequence for Goldbach}

\begin{theorem}[Conditional Goldbach]
Assume Conjecture~\ref{conj:med-l4} holds. Then there exists an explicit $N_0$ (computable from $C_{\mathrm{disp}}$ and $\delta_{\mathrm{med}}$) such that every even integer $2m \geq N_0$ is a sum of two primes.
\end{theorem}

\begin{proof}[Proof sketch]
Combine the conjecture with:
\begin{enumerate}
\item Standard major-arc analysis giving a main term $(c_8(2m) + o(1)) \mathfrak{S}(2m) N / \log^2 N$ with $\mathfrak{S}(2m) \geq 2C_2 > 0$;
\item Deep-minor arc bounds $\int_{\mathfrak{m}_{\mathrm{deep}}} |S|^2 \ll N/(\log N)^{10}$;
\item A coercivity lemma: $R_8(2m;N) \geq \text{main} - C \cdot \mathcal{D}_{\mathrm{med}}^{1/2} - \epsilon_{\mathrm{deep}}$.
\end{enumerate}
For $N \geq N_0$, the main term dominates, giving $R_8(2m;N) > 0$ for all $m \leq N$.
\end{proof}

\section{Proof Strategy for the Conjecture}

\subsection{Vaughan Decomposition}
With $U = V = N^{1/3}$, Vaughan's identity gives
\[
S(\alpha) = S_{\mathrm{I}}(\alpha) + S_{\mathrm{II}}(\alpha) + R(\alpha),
\]
where $S_{\mathrm{I}}, S_{\mathrm{II}}$ are bilinear forms with divisor-bounded coefficients.

\subsection{Bilinear Analysis on Medium Arcs}
On $\alpha = a/q + \beta$ with $Q < q \leq Q'$ and $|\beta| \leq Q'/(qN)$:
\[
\mathcal{B}(\alpha) = \sum_{m \sim M} A_m \sum_{n \sim N/M} B_n \, e\!\left(\frac{a \cdot mn}{q}\right) e(\beta \cdot mn).
\]

\subsection{Key Lemmas}

\begin{lemma}[Local $L^4$ on Short Arcs]
For $|c_x| \leq C$ and $B \leq 1$,
\[
\int_{|\beta| \leq B} \left| \sum_x c_x e(\beta x) \right|^4 d\beta \leq 2B \cdot \left( \sum_x |c_x|^2 \right)^2.
\]
\end{lemma}

\begin{lemma}[Large Sieve]
\[
\sum_{q \leq Q'} \sum_{\substack{a \bmod q \\ (a,q)=1}} \left| \sum_{n \leq X} a_n e\!\left(\frac{an}{q}\right) \right|^2 \leq (X + Q'^2) \sum_{n \leq X} |a_n|^2.
\]
\end{lemma}

\subsection{Where the Saving Comes From}
The saving arises from:
\begin{itemize}
\item Arc width $Q'/(qN)$ shrinking with $q$;
\item Balanced bilinear ranges when $M \approx N^{1/2}$;
\item Large sieve cancellation in the $q$-summation.
\end{itemize}
The computation $\delta(N) = c \cdot \log(Q'/Q)/\log N \approx c/6 > 0$ suggests the saving is real.

\subsection{The Gap}
Making this rigorous requires:
\begin{enumerate}
\item Explicit tracking of all constants through Vaughan's identity;
\item Careful completion to additive characters mod $q$;
\item Aggregating the large sieve bounds over all $q \in (Q, Q']$ and dyadic $M$.
\end{enumerate}

\section{Related Literature}

\begin{itemize}
\item \textbf{Vaughan (1997)}: The Hardy--Littlewood Method. [Vaughan identity, Ch.~3]
\item \textbf{Montgomery--Vaughan (2007)}: Multiplicative Number Theory I. [Large sieve, Ch.~7]
\item \textbf{Deshouillers--Iwaniec (1982)}: Kloosterman sums and Fourier coefficients. [Dispersion method]
\item \textbf{Duke--Friedlander--Iwaniec (1997)}: Bilinear forms with Kloosterman sums.
\item \textbf{Iwaniec--Kowalski (2004)}: Analytic Number Theory. [Ch.~16: Dispersion]
\end{itemize}

These references provide the \emph{techniques} but do not prove Conjecture~\ref{conj:med-l4} for this specific arc geometry and with explicit constants.

\section{Status}

\begin{center}
\begin{tabular}{|l|l|}
\hline
\textbf{Aspect} & \textbf{Status} \\
\hline
Precise statement & Complete \\
Proof strategy & Outlined \\
Key lemmas & Classical (provable) \\
Explicit constants & Conjectured \\
Full proof & \textbf{Open} \\
\hline
\end{tabular}
\end{center}

\bigskip

\noindent\textbf{Proving Conjecture~\ref{conj:med-l4} would establish Goldbach's conjecture for all sufficiently large even integers.}

\end{document}

