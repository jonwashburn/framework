% Framework Exclusivity + Verification Method — Family A Patent (Method, System, Medium)
\documentclass[12pt]{article}
\usepackage[a4paper,margin=1in]{geometry}
\usepackage{amsmath,amssymb}
\usepackage{enumitem}
\usepackage{hyperref}
\usepackage{microtype}
\usepackage{xcolor}

\title{Framework Exclusivity and Verification Method\\Family A Patent (Method, System, Medium)}
\author{Jonathan Washburn}
\date{\today}

\begin{document}
\maketitle

\begin{center}
{\large Camera-Ready Patent Draft}\\[4pt]
{\normalsize Machine-Implemented Definitional Equivalence for Zero-Parameter Frameworks}
\end{center}

\section*{Field}
This disclosure relates to computer-implemented verification of physical frameworks. It provides a method, system, and non-transitory medium for establishing \emph{definitional equivalence} between any candidate \emph{zero-parameter} framework that derives observables and a reference Recognition Computing architecture, by executing machine-checkable proofs and issuing a \emph{certificate of equivalence} that gates claims or operation.

\section*{Background}
Scientific and engineering frameworks are often compared informally or empirically, leading to ambiguity about equivalence, provenance, and scope of claims. Conventional verification pipelines (testing, model checking) do not address \emph{parameter-free} derivations of observables, nor do they provide a \emph{definitional} proof of equivalence tied to a certificate that can gate claims or operation. A formal, machine-executable method is needed to (i) recognize zero-parameter provenance, (ii) establish definitional equivalence to a canonical stack, and (iii) produce a standard certificate for enforcement and audit.

\section*{Summary}
Disclosed is a machine-implemented verification method that:
\begin{enumerate}[leftmargin=*]
  \item formalizes a candidate physics framework $\mathcal{F}$; 
  \item verifies that $\mathcal{F}$ is \emph{zero-parameter} and \emph{derives observables} under declared axioms and invariants;
  \item constructs interpretation maps and proof obligations for equivalence to a reference Recognition Computing architecture;
  \item executes a definitional-equivalence proof using a proof engine (e.g., Lean 4 or equivalent);
  \item issues a \emph{certificate of equivalence} with proof hashes, tolerance bands, and metadata; and
  \item gates claims or operation based on certificate status (pass/fail).
\end{enumerate}
The method yields technical effects: unambiguous, reproducible equivalence judgments; reduced disputes; audit-ready provenance; and automation-ready gating of usage and claims.

\section*{Definitions}
\noindent\textbf{Physics framework ($\mathcal{F}$).} A formal structure comprising a state space, observables, transition/evolution rules, and derivation machinery under declared axioms.

\noindent\textbf{Zero-parameter.} $\mathcal{F}$ admits no tunable parameters in deriving observables; all outputs are fixed by internal axioms/invariants up to isomorphism.

\noindent\textbf{Derives observables.} $\mathcal{F}$ computes physically meaningful, testable observables (e.g., dimensionless invariants, display quantities) from first principles.

\noindent\textbf{Recognition Computing reference ($\mathcal{R}$).} The five-layer architecture comprising recognition ratios, axiomatized J-ledger, $\varphi$-commensurate neutral timing, certificate gating, and absolute-layer selection.

\noindent\textbf{Interpretation maps.} Bi-interpretability data: structure-preserving mappings between signatures, categories, or typed modules of $\mathcal{F}$ and $\mathcal{R}$ used to construct equivalence proofs.

\noindent\textbf{Certificate of equivalence.} A signed, hash-addressed artifact containing proof summaries, invariants, tolerance bands, and verdicts sufficient to gate claims or operation.

\noindent\textbf{Compliance API.} Interfaces to submit frameworks, retrieve proof status, and attach/gate certificates.

\section*{Claims}
\setlist[enumerate]{leftmargin=*,labelsep=0.5em,itemsep=0.6em,topsep=0.4em}

\subsection*{Independent Claims}
\begin{enumerate}
\item \textbf{(Method)} A computer-implemented method of verifying a physics framework, comprising:
\begin{enumerate}[label=(\alph*)]
  \item receiving a candidate framework specification $\mathcal{F}$ and formalizing $\mathcal{F}$ into a machine-checkable representation;
  \item verifying that $\mathcal{F}$ is \emph{zero-parameter} and \emph{derives observables} under declared axioms and invariants;
  \item constructing interpretation maps and proof obligations for definitional equivalence between $\mathcal{F}$ and a reference Recognition Computing architecture $\mathcal{R}$;
  \item executing a definitional-equivalence proof with a proof engine to obtain a proof verdict and proof hash set;
  \item generating a \emph{certificate of equivalence} including the verdict, proof hashes, invariants used, tolerance bands, timestamps, and framework identifiers; and
  \item gating claims or operation of $\mathcal{F}$ based on the certificate.
\end{enumerate}

\item \textbf{(System)} A verification system comprising one or more processors, memory, and a proof engine configured to: (i) formalize a candidate framework $\mathcal{F}$; (ii) verify zero-parameter and derives-observables properties; (iii) construct interpretation maps to a reference architecture $\mathcal{R}$; (iv) execute definitional-equivalence proofs; (v) generate a certificate of equivalence; and (vi) gate claims or operation on certificate status.

\item \textbf{(Non-transitory medium)} A non-transitory computer-readable medium storing instructions that, when executed by one or more processors, cause a system to perform the method of claim 1.
\end{enumerate}

\subsection*{Dependent Claims}
\begin{enumerate}[resume]
\item (Method) wherein verifying zero-parameter includes static analysis to detect absence of tunable knobs, priors, or external anchors in derivations of observables.
\item (Method) wherein derives-observables includes construction or recovery of dimensionless invariants or display quantities from internal axioms.
\item (Method) wherein interpretation maps comprise functors or typed morphisms between module signatures of $\mathcal{F}$ and $\mathcal{R}$.
\item (Method) wherein proof obligations include preservation and reflection of invariants, bi-interpretability, and normalization to canonical forms.
\item (Method) further comprising executing normalization to a standard basis (e.g., units-quotient, J-ledger form) before equivalence.
\item (Method) wherein the proof engine is Lean 4 or an equivalent proof assistant executing machine-checkable proofs and emitting proof hashes.
\item (Method) wherein the certificate comprises: framework identifiers, signature hashes, proof hashes, invariants used, thresholds, verdict, and issuance metadata.
\item (Method) further comprising registering the certificate in a transparency registry and exposing retrieval endpoints.
\item (System) further comprising APIs to submit frameworks, retrieve proof status, download certificates, and attach certificates to dependent claims or operation policies.
\item (Method) further comprising a \emph{replay} mode that re-executes proofs from the certificate to validate equivalence deterministically.
\item (Method) further comprising \emph{challenge} workflows that accept third-party counterexamples and re-open proof obligations automatically.
\item (Method) wherein gating claims includes denying advertising or legal assertions of novelty unless a valid certificate of non-equivalence is present.
\item (Method) wherein gating operation includes denying device or simulator operation unless a valid certificate of equivalence or compatibility is present.
\item (System) further comprising audit logging, signatures, and remote attestation of the proof engine and environment.
\item (Method) wherein proof obligations include constraints particular to $\mathcal{R}$ (recognition ratios, J-ledger axioms, $\varphi$-timing constraints, certificate gating, absolute-layer selection).
\end{enumerate}

\section*{Brief Description of the Drawings}
\begin{enumerate}[leftmargin=*]
  \item Fig. 1: Verification pipeline (ingest $\mathcal{F}$, formalize, static checks, interpretations, proof, certificate, gating).
  \item Fig. 2: Interpretation maps and bi-interpretability proof obligations.
  \item Fig. 3: Certificate schema and registry integration.
  \item Fig. 4: Compliance API and gating flows for claims and operation.
\end{enumerate}

\section*{Detailed Description}
\subsection*{Formalization and Static Checks}
The system ingests a candidate $\mathcal{F}$ via a specification format (typed signatures, axioms, modules) and compiles it to a proof-engine representation. Static checks establish that derivations of observables do not depend on tunable parameters or external anchors (zero-parameter) and that $\mathcal{F}$ derives observables (dimensionless invariants, display quantities) from internal axioms.

\subsection*{Interpretation Maps and Proof Obligations}
Interpretation maps relate $\mathcal{F}$ to the reference $\mathcal{R}$. Obligations include preservation of invariants, reflection of structure, and existence of bi-interpretability or categorical equivalence. Normalization steps map $\mathcal{F}$ into canonical forms (e.g., units quotient, ledger form) to simplify proofs.

\subsection*{Definitional Equivalence Execution}
The proof engine executes obligations and records proof artifacts (hashes, theorem references, steps). A verdict (equivalent / not equivalent / inapplicable) is produced. Failure modes include discovery of hidden parameters, lack of observables, or unresolvable mismatches to $\mathcal{R}$.

\subsection*{Certificate of Equivalence and Gating}
Upon success, a certificate is generated with proof hashes, invariants, thresholds, and timestamps. A compliance API publishes the certificate and enables downstream gating: claims (e.g., assertion of a zero-parameter alternative) must attach a certificate of non-equivalence; operation (e.g., device or simulator modes) must attach valid certificates of equivalence/compatibility.

\subsection*{Security and Anti-Evasion}
The system uses signature-based attestation of proof-engine binaries and environments. Certificates are signed and content-addressed. Replay mode verifies determinism. Challenge mode permits third-party evidence to re-open obligations, ensuring robustness.

\section*{Technical Effects}
The invention provides: (i) unambiguous, reproducible equivalence decisions; (ii) reduction of disputes and litigation through standardized certificates; (iii) automation-ready gatekeeping of claims and operation; and (iv) improved scientific integrity via zero-parameter provenance checks.

\section*{Enablement and Best Mode}
Implementations include: (i) a Lean 4 proof backend; (ii) formalization schemas for frameworks; (iii) normalization libraries (units quotient, J-ledger); (iv) certificate schemas and a transparency registry; (v) a compliance API and client SDKs. Benchmarks quantify throughput (proof time), determinism, and robustness to model complexity. 

\section*{Alternative Embodiments}
Other proof assistants may be used. Certificates may be on-device, cloud, or hybrid. Registry may be public or permissioned. Equivalence may target subsets (e.g., recognition+ledger only) with appropriately scoped certificates.

\section*{Legal Notice}
All embodiments, features, and steps described herein may be combined in any operable permutation. Where means-plus-function is implicated, claims are intended to cover corresponding structures and their equivalents.

\end{document}


