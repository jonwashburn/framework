\documentclass[11pt]{article}

\usepackage[margin=1in]{geometry}
\usepackage{amsmath,amssymb,amsthm}
\usepackage{booktabs}
\usepackage{hyperref}
\hypersetup{colorlinks=true, linkcolor=blue, citecolor=blue, urlcolor=blue}

\newcommand{\vphi}{\varphi}
\newcommand{\ML}{M\!/\!L}

\title{Technical Memo:\\
The Mass-to-Light Ratio in CPM-Gravity\\[0.3em]
\large Why M/L = $\vphi$ and Its Connection to the Kernel Constants}

\author{Recognition Physics Institute\\
\texttt{jon@recognitionphysics.org}}

\date{\today}

\begin{document}
\maketitle

\section*{Purpose}

This memo documents the correspondence between the stellar mass-to-light ratio ($\ML$) and the CPM-Gravity kernel constants. This material is \textbf{not currently in the CPM-Gravity paper} and may be added to strengthen the zero-parameter claim.

\section*{Executive Summary}

\begin{center}
\fbox{\parbox{0.9\textwidth}{
\textbf{Key Result:} Both the kernel enhancement $w(r)$ and the stellar mass-to-light ratio $\ML$ are locked to the same $\vphi$-ladder. Neither is a free parameter---both are derived from the same mathematical structure.
}}
\end{center}

\section{The Gap in the Current Paper}

The CPM-Gravity paper currently:
\begin{itemize}
  \item[$\checkmark$] Derives kernel constants $\alpha = \frac{1}{2}(1 - \vphi^{-1})$ and $C = \vphi^{-3/2}$ from Recognition Geometry
  \item[$\checkmark$] Derives coercivity constant $c = 49/162$ from CPM structure
  \item[$\checkmark$] States the rotation curve equation $v^2 = w(r) \times v^2_{\text{baryon}}$
  \item[$\times$] \textbf{Does not mention} that $\ML = \vphi$ is also derived
  \item[$\times$] \textbf{Does not connect} $\ML$ to the same $\vphi$-ladder as the kernel
\end{itemize}

A reviewer may ask: ``You claim zero per-galaxy parameters, but don't you still fit $\ML$ per galaxy?'' The answer is \textbf{no}---and the paper should say so.

\section{The Rotation Curve Equation}

In CPM-Gravity (ILG), the model velocity is:
\begin{equation}
v_{\text{model}}^2(r) = w(r) \times v_{\text{baryon}}^2(r)
\end{equation}
where $w(r) \geq 1$ is the kernel enhancement (replacing dark matter).

The baryonic velocity depends on the mass-to-light ratio:
\begin{equation}
v_{\text{baryon}}^2 = v_{\text{gas}}^2 + \left(\sqrt{\ML} \cdot v_{\text{disk}}\right)^2 + v_{\text{bulge}}^2
\end{equation}

In standard analyses (MOND, $\Lambda$CDM), $\ML$ is a \textbf{free parameter} fitted per galaxy. In CPM-Gravity, it is \textbf{derived}.

\section{The M/L Derivation}

The stellar mass-to-light ratio is derived from J-cost minimization on the recognition ledger. Three independent strategies converge to the same result:

\subsection*{Strategy 1: Stellar Assembly (Recognition Cost Weighting)}

Stars form where recognition cost is minimized. The cost differential between photon emission and mass storage determines the equilibrium:
\[
\ML = \exp\left(\frac{\Delta\delta}{J_{\text{bit}}}\right) = \vphi^n
\]
where $J_{\text{bit}} = \ln\vphi$ is the fundamental information unit.

\subsection*{Strategy 2: $\vphi$-Tier Nucleosynthesis}

Nuclear densities and photon fluxes occupy discrete $\vphi$-tiers:
\[
\ML = \frac{\vphi^{n_{\text{nuclear}}}}{\vphi^{n_{\text{photon}}}} = \vphi^{\Delta n}
\]

\subsection*{Strategy 3: Geometric Observability Limits}

Observability constraints ($\lambda_{\text{rec}}$, $\tau_0$, $E_{\text{coh}}$) combined with J-minimization force $\ML$ onto the $\vphi$-ladder.

\subsection*{Result}

All three strategies yield:
\begin{equation}
\boxed{\ML = \vphi \approx 1.618 \text{ solar units (characteristic value)}}
\end{equation}

Valid range: $\ML \in \{\vphi^n : n \in \{0, 1, 2, 3\}\} = \{1, 1.618, 2.618, 4.236\}$

This matches observed stellar $\ML \in [0.5, 5]$ solar units.

\section{The $\vphi$-Ladder Correspondence}

\begin{center}
\begin{tabular}{lll}
\toprule
\textbf{Quantity} & \textbf{Value} & \textbf{$\vphi$-Connection} \\
\midrule
Kernel exponent $\alpha$ & $\frac{1}{2}(1 - \vphi^{-1}) \approx 0.191$ & Direct from $\vphi$ \\
Kernel amplitude $C$ & $\vphi^{-3/2} \approx 0.486$ & Power of $\vphi$ \\
Mass-to-light $\ML$ & $\vphi \approx 1.618$ & Power of $\vphi$ \\
Coercivity slack $c$ & $49/162 \approx 0.302$ & From 8-tick ($\varepsilon = 1/8$) \\
\bottomrule
\end{tabular}
\end{center}

\textbf{Key insight:} The kernel constants and $\ML$ are not independent. They emerge from the same underlying structure---the golden ratio $\vphi$ and its self-similar scaling.

\section{Implications for the Paper}

\subsection*{Strengthens the Zero-Parameter Claim}

The paper currently states ``no per-galaxy tuning.'' Adding the $\ML$ derivation makes this precise: \textbf{neither} the kernel \textbf{nor} the mass-to-light ratio is fitted per galaxy.

\subsection*{Closes a Potential Objection}

Without this material, a reviewer could legitimately ask: ``Your rotation curve fits still require choosing $\ML$---isn't that a free parameter?''

With this material, the answer is clear: $\ML = \vphi$ is derived from the same framework that derives the kernel. Choosing a different $\ML$ per galaxy would violate the coercive projection law.

\subsection*{Shows the Deep Connection}

The correspondence reveals that CPM-Gravity is not just a phenomenological modification to gravity. The same $\vphi$-structure that controls:
\begin{itemize}
  \item How gravity is enhanced at large scales (kernel $w$)
  \item How much mass stars contain per unit luminosity ($\ML$)
\end{itemize}
is a single, unified mathematical architecture.

\section{Suggested Text for the Paper}

The following could be added as a subsection in Section 2 or Section 6:

\begin{quote}
\textbf{Mass-to-Light Ratio: Derived, Not Fitted}

The baryonic velocity entering the rotation curve equation depends on the stellar mass-to-light ratio $\ML$:
\[
v^2_{\text{baryon}} = v^2_{\text{gas}} + (\sqrt{\ML} \cdot v_{\text{disk}})^2 + v^2_{\text{bulge}}
\]

In standard analyses, $\ML$ is a free parameter fitted per galaxy. Under the coercive projection law, $\ML$ is \emph{derived} from the same $\vphi$-structure that fixes the kernel constants:
\[
\ML = \vphi \approx 1.618 \text{ solar units (characteristic)}
\]

This follows from J-cost minimization on the recognition ledger. The valid range $\ML \in \{1, \vphi, \vphi^2, \vphi^3\} \approx \{1, 1.6, 2.6, 4.2\}$ matches observed stellar populations.

\textbf{Key point}: Both the kernel enhancement $w(r)$ and the mass-to-light ratio $\ML$ are locked to the $\vphi$-ladder. Per-galaxy fitting of either quantity violates the coercive projection law.
\end{quote}

\section{Lean Verification}

The $\ML$ derivation is machine-verified in Lean 4:

\begin{center}
\begin{tabular}{ll}
\toprule
\textbf{File} & \textbf{Key Theorems} \\
\midrule
\texttt{Astrophysics/MassToLight.lean} & \texttt{ml\_derivation\_complete} \\
\texttt{Astrophysics/StellarAssembly.lean} & \texttt{ml\_from\_cost\_minimization} \\
\texttt{Astrophysics/NucleosynthesisTiers.lean} & \texttt{nucleosynthesis\_ml\_agrees} \\
\texttt{Astrophysics/ObservabilityLimits.lean} & \texttt{geometric\_ml\_agrees} \\
\texttt{URCGenerators/MassToLightCert.lean} & \texttt{MassToLightCert.verified} \\
\bottomrule
\end{tabular}
\end{center}

All three derivation strategies are proven to agree on $\ML = \vphi$.

\section{Falsifiability}

The $\ML = \vphi$ prediction is falsifiable:

\begin{itemize}
  \item If observed stellar $\ML$ values systematically deviate from the $\vphi$-ladder $\{1, 1.6, 2.6, 4.2\}$, the theory is falsified.
  \item If rotation curve fits \emph{require} per-galaxy $\ML$ tuning to achieve acceptable residuals, the coercive projection law is violated.
  \item The $\ML$ and kernel constants are \emph{locked together}---you cannot adjust one without breaking the other.
\end{itemize}

\section*{Summary}

\begin{enumerate}
  \item The CPM-Gravity paper should state that $\ML = \vphi$ is \textbf{derived}, not fitted.
  \item This strengthens the zero-parameter claim and closes a potential reviewer objection.
  \item The correspondence between $\ML$ and the kernel constants reveals the unified $\vphi$-architecture underlying CPM-Gravity.
  \item The material is machine-verified in Lean and falsifiable against observations.
\end{enumerate}

\vspace{1em}
\hrule
\vspace{0.5em}
\noindent\textit{This memo prepared for Brett's CPM-Gravity paper. Contact: jon@recognitionphysics.org}

\end{document}



