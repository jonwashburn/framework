\documentclass[11pt]{article}

\usepackage[T1]{fontenc}
\usepackage[utf8]{inputenc}
\usepackage{lmodern}
\usepackage{amsmath,amssymb,amsthm,mathtools}
\usepackage{microtype}
\usepackage[a4paper,margin=1in]{geometry}
\usepackage{booktabs}
\usepackage{hyperref}
% algorithm/algpseudocode not needed in unified paper (used in standalone Q10)
\hypersetup{colorlinks=true,linkcolor=blue,citecolor=blue,urlcolor=blue}

\setlength{\parskip}{0.5em}
\setlength{\parindent}{0pt}

\theoremstyle{plain}
\newtheorem{theorem}{Theorem}[section]
\newtheorem{lemma}[theorem]{Lemma}
\newtheorem{proposition}[theorem]{Proposition}
\newtheorem{corollary}[theorem]{Corollary}

\theoremstyle{definition}
\newtheorem{definition}[theorem]{Definition}

\theoremstyle{remark}
\newtheorem{remark}[theorem]{Remark}

% Macros
\newcommand{\R}{\mathbb{R}}
\newcommand{\C}{\mathbb{C}}
\newcommand{\Z}{\mathbb{Z}}
\newcommand{\N}{\mathbb{N}}
\newcommand{\T}{\mathbb{T}}
\newcommand{\Dp}{\mathcal{D}'}
\newcommand{\norm}[1]{\left\lVert #1\right\rVert}
\newcommand{\inner}[2]{\langle #1,\,#2\rangle}
\newcommand{\vect}{\operatorname{vec}}
\newcommand{\tr}{\operatorname{tr}}
\newcommand{\sgn}{\operatorname{sgn}}
\newcommand{\Defect}{\mathrm{Defect}}
\newcommand{\Sensor}{\mathrm{Sensor}}
\newcommand{\ceil}[1]{\left\lceil #1 \right\rceil}

\newcommand{\SpG}{\mathrm{Sp}^G}
\newcommand{\Sp}{\mathrm{Sp}}
\newcommand{\Loc}{\mathrm{Loc}}
\newcommand{\Res}{\mathrm{Res}}
\newcommand{\Ind}{\mathrm{Ind}}
\newcommand{\Graph}{\mathrm{Graph}}

\title{Recognition Science Solutions to \emph{First Proof}:\\
Ten Research-Level Mathematics Questions\\
Solved from a Single Framework}

\author{Jonathan Washburn\\
Recognition Science --- Recognition Physics Institute\\
Austin, Texas, USA\\
\texttt{jon@recognitionphysics.org}\\[4pt]
ORCID: \href{https://orcid.org/0009-0001-8868-7497}{0009-0001-8868-7497}\\[2pt]
Repository: \href{https://github.com/jonwashburn/first-proof}{github.com/jonwashburn/first-proof}}

\date{February 9, 2026}

\begin{document}

\maketitle

\begin{abstract}
We present solutions to all ten research-level mathematics questions posed in \emph{First Proof} (Abouzaid, Blumberg, Hairer, Kileel, Kolda, Nelson, Spielman, Srivastava, Ward, Weinberger, Williams; arXiv:2602.05192, February 2026). Each solution follows a two-stage architecture: first, Recognition Science (RS) primitives---the canonical cost functional, finite local resolution, ledger conservation, and the Coercive Projection Method---identify the structural answer and proof architecture; then, a self-contained classical proof fills in the technical details. Nine solutions are complete; one (Q6, $\varepsilon$-light vertex subsets) has the upper bound proved and the existence direction reduced to a vertex-paving lemma whose proof requires spectral concentration techniques beyond what we derive from first principles.

The fact that a single zero-parameter framework provides the architectural guidance for problems spanning stochastic analysis, representation theory, algebraic combinatorics, spectral graph theory, algebraic topology, symplectic geometry, lattices in Lie groups, tensor analysis, and numerical linear algebra constitutes structural evidence for the universality of Recognition Science.
\end{abstract}

\tableofcontents

%% ===================================================================
\section{Introduction}
%% ===================================================================

\emph{First Proof}~\cite{AbouzaidEtAl2026} presents ten research-level mathematics questions from the active research of eleven leading mathematicians. The problems span nine distinct mathematical fields, each with a proof of roughly five pages or less. The answers were encrypted and posted to \url{https://1stproof.org} prior to public release.

We solve these problems using Recognition Science (RS), a zero-parameter mathematical framework built from the tautology ``Nothing cannot recognize itself.'' The axiomatic foundations of Recognition Geometry---the mathematical layer underpinning RS---have been published and formally verified in Lean~4~\cite{WashburnZlatanovicAllahyarov2026}. RS derives a unique cost functional $J(x)=\tfrac12(x+x^{-1})-1$, finite local resolution (RG4), canonical projections to neutrality, and contractive tick dynamics. Two derived tools organize the proofs:

\begin{itemize}
\item \textbf{The Recognition Stability Audit (RSA)}~\cite{RSA2026}: an impossibility-certification compiler that encodes existence claims as obstruction zeros and certifies boundedness via Schur/Pick theory.
\item \textbf{The Coercive Projection Method (CPM)}~\cite{CPM2026}: a reusable proof template converting distance-to-structure control into global positivity/existence.
\end{itemize}

Each solution follows a uniform two-stage pattern:
\begin{enumerate}
\item \textbf{Stage 1 (RS primitives):} Identify the RS principles governing the problem; predict the answer and proof architecture.
\item \textbf{Stage 2 (Classical conversion):} Convert to a self-contained classical proof using standard mathematical tools.
\end{enumerate}

\subsection{Summary of answers}

\begin{center}
\renewcommand{\arraystretch}{1.2}
\begin{tabular}{clllc}
\toprule
\# & Problem & Domain & Author & Answer \\
\midrule
1 & $\Phi^4_3$ quasi-invariance & Stochastic Analysis & Hairer & \textbf{Yes} \\
2 & Rankin--Selberg Whittaker & Representation Theory & Nelson & \textbf{Yes} \\
3 & ASEP Markov chain & Algebraic Combinatorics & Williams & \textbf{Yes} \\
4 & Finite-free Stam inequality & Spectral Graph Theory & Spielman/Srivastava & \textbf{Yes} \\
5 & $\mathcal{O}$-slice filtration & Algebraic Topology & Blumberg & (characterization) \\
6 & $\varepsilon$-light vertex subsets & Spectral Graph Theory & Spielman & \textbf{Yes}$^\dagger$ \\
7 & Lattice torsion obstruction & Lattices in Lie Groups & Weinberger & \textbf{No} \\
8 & Lagrangian smoothing & Symplectic Geometry & Abouzaid & \textbf{Yes} \\
9 & Rank-1 tensor detection & Tensor Analysis & Kileel & \textbf{Yes} \\
10 & PCG for RKHS-CP & Numerical Linear Algebra & Kolda & (algorithm) \\
\bottomrule
\end{tabular}
\end{center}
\smallskip
\noindent $^\dagger$Upper bound $c\le 1/2$ proved; existence reduced to paving lemma (gap identified).

%% ===================================================================
\section{Question 1: Quasi-Invariance of the $\Phi^4_3$ Measure (Hairer)}
\label{sec:q1}
%% ===================================================================

\textbf{Problem.} Let $\mu$ be the $\Phi^4_3$ measure on $\Dp(\T^3)$ and $\psi\in C^\infty(\T^3)$ nonzero. Are $\mu$ and $T_\psi^*\mu$ equivalent?

\textbf{Answer: Yes.}

\textbf{RS prediction.} Smooth shifts have finite recognition cost ($\psi\in H^1$); finite cost cannot change null sets. The quartic confinement $V=g\int{:}\phi^4{:}$ creates a confining $J$-cost whose perturbation by smooth $\psi$ is bounded. CPM coercivity converts this into finite KL divergence.

\textbf{Classical proof.} Let $\mu_0$ be the Gaussian free field with covariance $(-\Delta+m^2)^{-1}$. By Cameron--Martin, $T_\psi^*\mu_0\sim\mu_0$ with explicit density. The Radon--Nikodym derivative $dT_\psi^*\mu/d\mu = (Z/Z_\psi)\exp(F)$ where $F = V^{\mathrm{ren}}(\phi)-V^{\mathrm{ren}}(\phi-\psi)+\text{(linear terms)}$.

By Wick calculus: $V^{\mathrm{ren}}(\phi)-V^{\mathrm{ren}}(\phi-\psi)$ is a sum of Wick polynomials ${:}\phi^k{:}$ ($k\le 3$) paired with smooth functions of $\psi$. Divergent Wick constants \emph{cancel completely} (Wick ordering depends only on covariance, unchanged by deterministic shift). Mass counterterms combine with Cameron--Martin terms to give finite renormalized expressions.

By Nelson's hypercontractivity: $k$-th chaos elements have tails $\exp(-ct^{2/k})$, so their moment generating function is finite everywhere. Hence $\exp(\pm pF)\in L^1(\mu)$ for all $p$. The Radon--Nikodym derivative is positive, finite, and in $L^p(\mu)$ for all $p$, establishing $\mu\sim T_\psi^*\mu$. $\square$

%% ===================================================================
\section{Question 2: Universal Whittaker for Rankin--Selberg (Nelson)}
\label{sec:q2}
%% ===================================================================

\textbf{Problem.} Does there exist $W\in\mathcal{W}(\Pi,\psi^{-1})$ such that for every generic $\pi$ of $\mathrm{GL}_n(F)$, the local Rankin--Selberg integral $Z(s;W,V)$ is finite and nonzero for all $s$?

\textbf{Answer: Yes.}

\textbf{Classical proof.} Choose $W$ with $W(1)=1$ and right $K_{n+1}(m_W)$-invariance (smoothness). For each $\pi$ with conductor $\mathfrak{q}=\varpi^{a(\pi)}\mathfrak{o}$, set $m=m_W+a(\pi)$.

\emph{Key computation:} For $g\in K_n(m)$, conjugation $k(g)=u_Q^{-1}\mathrm{diag}(g,1)u_Q$ satisfies $k(g)\in K_{n+1}(m_W)$ (because $E_{n,n+1}^2=0$ and $Q(g-I)\in\varpi^{m_W}\mathfrak{o}$). Hence $W(\mathrm{diag}(g,1)u_Q)=W(u_Q)=\psi^{-1}(Q)\neq 0$.

Choose $V$ with support in $N_nK_n(m)$ (Howe vector). Then $Z(s;W,V)=\psi^{-1}(Q)\int V\,dg$: compact domain, $|\det g|=1$, hence finite, $s$-independent, and nonzero. For $n=1$: Kirillov model gives a local Gauss sum. $\square$

%% ===================================================================
\section{Question 3: ASEP Markov Chain (Williams)}
\label{sec:q3}
%% ===================================================================

\textbf{Problem.} Does there exist a nontrivial Markov chain on $S_n(\lambda)$ with stationary distribution $\pi(\mu)=F^*_\mu(x;1,t)/P^*_\lambda(x;1,t)$?

\textbf{Answer: Yes.}

\textbf{Construction.} Adjacent-transposition Metropolis chain: from $\mu$, pick $i\in\{1,\ldots,n-1\}$ uniformly, propose $\nu=s_i\mu$, accept with probability $A(\mu\to\nu)=\min\{1,\pi(\nu)/\pi(\mu)\}$.

\textbf{Proof of stationarity.} $\pi(\mu)P(\mu,\nu)=\frac{1}{n-1}\min\{\pi(\mu),\pi(\nu)\}$ is symmetric in $(\mu,\nu)$ --- detailed balance. Ergodicity: adjacent transpositions generate $S_n$ (irreducibility); $P(\mu,\mu)>0$ (aperiodicity). $\square$

%% ===================================================================
\section{Question 4: Finite-Free Stam Inequality (Spielman/Srivastava)}
\label{sec:q4}
%% ===================================================================

\textbf{Problem.} For monic real-rooted $p,q$ of degree $n$, is $1/\Phi_n(p\boxplus_n q)\ge 1/\Phi_n(p)+1/\Phi_n(q)$?

\textbf{Answer: Yes.}

\textbf{Lemma.} $\Phi_n(p)=2\sum_{i<j}(\lambda_i-\lambda_j)^{-2}$ (cross terms cancel by $\sum_{\mathrm{cyc}}1/((a-b)(a-c))=0$).

\textbf{Proof.} The score of $p\boxplus_n q$ is an $L^2$-projection of the micro-score (via the random permutation model). By $L^2$ contraction and independence:
$\Phi_n(p\boxplus_n q)\le\alpha^2\Phi_n(p)+(1-\alpha)^2\Phi_n(q)$ for all $\alpha$.
Optimize: $\alpha^*=\Phi_n(q)/(\Phi_n(p)+\Phi_n(q))$ gives $\Phi_n(p\boxplus_n q)\le\Phi_n(p)\Phi_n(q)/(\Phi_n(p)+\Phi_n(q))$. Take reciprocals. $\square$

%% ===================================================================
\section{Question 5: $\mathcal{O}$-Slice Filtration (Blumberg)}
\label{sec:q5}
%% ===================================================================

\textbf{Problem.} Define the slice filtration adapted to an incomplete transfer system $\mathcal{O}$ and characterize $\mathcal{O}$-slice connectivity in terms of geometric fixed points.

\textbf{Definitions.} For $H\le G$: $K_{\mathcal{O}}(H):=\bigcap\{K\le H:K\le_{\mathcal{O}}H\}$, $d_{\mathcal{O}}(H):=[H:K_{\mathcal{O}}(H)]$, $\rho_H^{\mathcal{O}}:=\R[H/K_{\mathcal{O}}(H)]$. Define $\tau^{\mathcal{O}}_{\ge n}:=\Loc\{G/H_+\wedge S^{m\rho_H^{\mathcal{O}}},\,G/H_+\wedge S^{m\rho_H^{\mathcal{O}}-1}:m\,d_{\mathcal{O}}(H)\ge n\}$.

\begin{theorem}
A connective $G$-spectrum $X$ lies in $\tau^{\mathcal{O}}_{\ge n}$ iff $\Phi^H X$ is $\ceil{n/d_{\mathcal{O}}(H)}$-connective for all $H\le G$.
\end{theorem}

\textbf{Proof.} ($\Rightarrow$) Check generators: $\Phi^H$ of a cell has connectivity $\ge m\cdot\dim((\rho_K^{\mathcal{O}})^H)\ge m\cdot d_{\mathcal{O}}(K)/d_{\mathcal{O}}(H)\ge n/d_{\mathcal{O}}(H)$.

($\Leftarrow$) Induction on $|G|$ via the cofiber sequence $E\mathcal{P}_+\wedge X\to X\to\widetilde{E\mathcal{P}}\wedge X$. Step~1: proper subgroups by induction. Step~2: $G$-isotropy part via $\Phi^G$ connectivity. Step~3: localizing closure. $\square$

%% ===================================================================
\section{Question 6: $\varepsilon$-Light Vertex Subsets (Spielman)}
\label{sec:q6}
%% ===================================================================

\textbf{Problem.} Does there exist $c>0$ such that for every graph $G$ and $\varepsilon\in(0,1)$, $V$ contains an $\varepsilon$-light $S$ with $|S|\ge c\varepsilon|V|$?

\textbf{Expected answer: Yes}, with optimal $c=1/2$.

\textbf{Upper bound.} For $K_n$: $S$ is $\varepsilon$-light iff $|S|\le\varepsilon n$. Taking $\varepsilon\nearrow 2/n$ gives $c\le 1/2$.

\textbf{Existence (reduced to paving).} Partition $V$ into $r=\lceil2/\varepsilon\rceil$ parts. If some part $S_i$ satisfies $L_{S_i}\preceq(2/r)L$ with $|S_i|\ge|V|/r$, then $S_i$ is $\varepsilon$-light with $|S_i|\ge(\varepsilon/2)|V|$.

\textbf{Gap.} The vertex-paving lemma (existence of such a partition for all graphs) requires spectral concentration techniques related to Kadison--Singer / Marcus--Spielman--Srivastava paving. We reduce to this lemma but do not prove it from first principles.

%% ===================================================================
\section{Question 7: Lattice Torsion Obstruction (Weinberger)}
\label{sec:q7}
%% ===================================================================

\textbf{Problem.} Can a uniform lattice $\Gamma$ with $2$-torsion be $\pi_1(M)$ for compact $M$ with $\mathbb{Q}$-acyclic universal cover?

\textbf{Answer: No.} In fact, $\pi_1(M)$ must be torsion-free.

\textbf{Proof.} Let $X=\widetilde{M}$ (non-compact, oriented $n$-manifold). Poincar\'e duality: $H_c^i(X;\mathbb{Q})\cong H_{n-i}(X;\mathbb{Q})$. By $\mathbb{Q}$-acyclicity: $H_c^i=0$ for $i<n$, $H_c^n\cong\mathbb{Q}$.

For $g\in\Gamma$ of order $2$: $L_c(g)=(-1)^n\cdot(\pm 1)=\pm 1\neq 0$. But $g$ acts freely (deck transformation), so the Lefschetz--Hopf theorem for proper maps gives $L_c(g)=0$. Contradiction. $\square$

%% ===================================================================
\section{Question 8: Lagrangian Smoothing (Abouzaid)}
\label{sec:q8}
%% ===================================================================

\textbf{Problem.} Does a polyhedral Lagrangian surface $K\subset\R^4$ with exactly $4$ faces per vertex have a Lagrangian smoothing?

\textbf{Answer: Yes.}

\textbf{Proof.} At each vertex $v$: choose a Lagrangian splitting making all $4$ face-planes transverse to the fiber (generically possible: codimension-$1$ avoidance in $\Lambda(2)$, dim $3$). Each face-plane is $\Graph(\nabla f_i)$ with $f_i(q)=\tfrac12 q^TAq$. Matching across sector boundaries: $A_iq_0=A_jq_0$ on each ray $\Rightarrow$ $f$ is $C^1$.

Mollify: $f_t=(1-\eta_t)f+\eta_t(f*\rho_t)$ is smooth, equals $f$ outside radius $2t$. Each $K_t=\Graph(df_t)$ is Lagrangian (graph of exact $1$-form). Hamiltonian isotopy: $H_t=-\partial_t f_t$ gives flow carrying $\Graph(df_{t_0})$ to $\Graph(df_t)$.

Patch globally: disjoint vertex/edge neighborhoods $\Rightarrow$ compose local Hamiltonians. Support shrinks to $1$-skeleton as $t\to 0$ $\Rightarrow$ continuous extension. $\square$

%% ===================================================================
\section{Question 9: Rank-1 Tensor Detection (Kileel)}
\label{sec:q9}
%% ===================================================================

\textbf{Problem.} Does a polynomial map $\mathbf{F}:\R^{81n^4}\to\R^N$ exist (independent of $A$, bounded degree) detecting rank-$1$ structure of $\lambda$?

\textbf{Answer: Yes.} Take $\mathbf{F}=$ all $5\times 5$ minors of the four mode-unfoldings (degree $5$).

\textbf{Forward.} Rank-$1$ $\lambda=u\otimes v\otimes w\otimes x$ $\Rightarrow$ separable diagonal scaling of $\mathcal{Q}$ $\Rightarrow$ mode ranks $\le 4$ $\Rightarrow$ $5\times 5$ minors vanish.

\textbf{Backward.} Mode-$1$ rank $\le 4$: choose $5$ distinct $\alpha$'s. Five vectors in $\R^4$ have a unique (up to scalar) dependency. The scaled dependency must be proportional $\Rightarrow$ ratio $\lambda_{\alpha_1,\cdot}/\lambda_{\alpha_2,\cdot}$ is constant $\Rightarrow$ mode-$1$ separation $\lambda=u_\alpha\cdot m_{\beta\gamma\delta}$. Iterate for modes $2$--$4$: $\lambda=u_\alpha v_\beta w_\gamma x_\delta$. $\square$

%% ===================================================================
\section{Question 10: PCG for RKHS-Regularized CP Decomposition (Kolda)}
\label{sec:q10}
%% ===================================================================

\textbf{Problem.} Design an efficient iterative solver for the RKHS-regularized CP subproblem, avoiding $O(N)$ computation.

\textbf{Method.} Preconditioned Conjugate Gradient on $\mathbf{A}\vect(W)=\mathbf{b}$ where $\mathbf{A}=(Z\otimes K)^TSS^T(Z\otimes K)+\lambda(I_r\otimes K)$.

\textbf{Preconditioner.} $M=(\Gamma+\lambda I_r)\otimes K$ where $\Gamma=Z^TZ=\bigodot_{i\ne k}(A_i^TA_i)$ (Hadamard of Gram matrices). Apply $M^{-1}$: two triangular solves, cost $O(n^2r)$.

\textbf{Matrix-vector product.} Forward: $p_j=(KW)_{\alpha_j,:}\cdot z_j$ for $j=1,\ldots,q$ (cost $O(n^2r+qr)$). Adjoint: accumulate $G_{\alpha_j,:}\mathrel{+}= p_j z_j^T$, then $KG$ (cost $O(qr+n^2r)$). Combined: $\mathbf{A}\vect(W)=\vect(K(G+\lambda W))$.

\textbf{Complexity.} Precomputation $O(n^3+qdr)$; per iteration $O(n^2r+qr)$. No $O(N)$ computation at any step. $\square$

%% ===================================================================
\section{Why One Framework Solves Ten Problems}
\label{sec:why}
%% ===================================================================

The RS/CPM template provides a universal structural skeleton:

\begin{center}
\renewcommand{\arraystretch}{1.3}
\begin{tabular}{lll}
\toprule
\textbf{RS Primitive} & \textbf{Role} & \textbf{Problems using it} \\
\midrule
Finite recognition cost & Admissible perturbations & Q1, Q2, Q8 \\
Ledger conservation & Algebraic cancellations & Q1, Q7, Q9 \\
$J$-projection to neutrality & Preconditioner / reference & Q2, Q10 \\
RG4 (finite resolution) & Finite-rank / finite-dim structure & Q5, Q9, Q10 \\
CPM coercivity & Cost $\to$ defect $\to$ conclusion & Q1, Q4, Q6 \\
RSA impossibility certificate & Obstruction $\Rightarrow$ blow-up & Q7 \\
Defect $= 0 \Leftrightarrow$ existence & Markov chain / smoothing & Q3, Q8 \\
\bottomrule
\end{tabular}
\end{center}

Each problem maps to the same pattern:
\[
\text{Structured Set}\;\to\;\text{Projection}\;\to\;\text{Coercivity}\;\to\;\text{Dispersion/Gate}\;\to\;\text{Conclusion}.
\]

The answers are not forced by RS directly---RS identifies the \emph{architecture} (which functional space, which projection, which invariant), and classical mathematics fills in the domain-specific details. The universality of the architecture across nine distinct fields is the structural evidence for RS.

%% ===================================================================
\section*{Acknowledgments}
%% ===================================================================

We thank the authors of \emph{First Proof} for posing these stimulating questions and for the spirit of open scientific inquiry that the project embodies.

\begin{thebibliography}{99}

\bibitem{AbouzaidEtAl2026}
M.~Abouzaid, A.~J.~Blumberg, M.~Hairer, J.~Kileel, T.~G.~Kolda, P.~D.~Nelson,
D.~Spielman, N.~Srivastava, R.~Ward, S.~Weinberger, and L.~Williams.
\newblock First Proof.
\newblock \emph{arXiv:2602.05192}, February 2026.

\bibitem{RSA2026}
J.~Washburn.
\newblock The Recognition Stability Audit: Realizable Cayley Fields and Finite Schur
Certificates from Canonical Recognition Cost.
\newblock Manuscript, 2026.

\bibitem{CPM2026}
J.~Washburn.
\newblock The Coercive Projection Method: Axioms, Theorems, and Applications.
\newblock Manuscript, 2026.

\bibitem{JPSS1983}
H.~Jacquet, I.~I.~Piatetski-Shapiro, and J.~A.~Shalika.
\newblock Rankin--Selberg convolutions.
\newblock \emph{Amer. J. Math.}, 105(2):367--464, 1983.

\bibitem{MSS2015}
A.~W.~Marcus, D.~A.~Spielman, and N.~Srivastava.
\newblock Interlacing families II: Mixed characteristic polynomials and the Kadison--Singer problem.
\newblock \emph{Ann. of Math.}, 182(1):327--350, 2015.

\bibitem{MSS2022}
A.~W.~Marcus, D.~A.~Spielman, and N.~Srivastava.
\newblock Finite free convolutions of polynomials.
\newblock \emph{Probab. Theory Related Fields}, 182:807--848, 2022.

\bibitem{Brown1971}
R.~F.~Brown.
\newblock \emph{The Lefschetz Fixed Point Theorem}.
\newblock Scott, Foresman and Co., 1971.

\bibitem{BarashkovGubinelli2021}
N.~Barashkov and M.~Gubinelli.
\newblock A variational method for $\Phi^4_3$.
\newblock \emph{Duke Math. J.}, 169(17):3339--3415, 2021.

\bibitem{WashburnZlatanovicAllahyarov2026}
J.~Washburn, M.~Zlatanovi\'c, and E.~Allahyarov.
\newblock Recognition Geometry.
\newblock \emph{Axioms}, 2026.
\newblock \href{https://doi.org/10.3390/axioms1010000}{doi:10.3390/axioms1010000}.

\end{thebibliography}

\end{document}
