\documentclass[11pt]{article}

\usepackage{amsmath,amssymb,amsthm}
\usepackage[a4paper,margin=1in]{geometry}
\usepackage{hyperref}
\hypersetup{colorlinks=true,linkcolor=blue,citecolor=blue,urlcolor=blue}

\setlength{\parskip}{0.5em}
\setlength{\parindent}{0pt}

\theoremstyle{plain}
\newtheorem{theorem}{Theorem}
\newtheorem{lemma}[theorem]{Lemma}
\newtheorem{proposition}[theorem]{Proposition}

\theoremstyle{remark}
\newtheorem{remark}[theorem]{Remark}

\newcommand{\C}{\mathbf{C}}

\title{Solution to First Proof, Question~2:\\
Existence of a Universal Whittaker Function\\
for Local Rankin--Selberg Integrals\\[6pt]
\large Via Recognition Science Primitives and Classical Conversion}

\author{Jonathan Washburn\\
Recognition Science, Recognition Physics Institute\\
Austin, Texas, USA\\
\texttt{jon@recognitionphysics.org}}

\date{February 9, 2026}

\begin{document}

\maketitle

\begin{abstract}
We prove the existence of a single Whittaker function $W\in\mathcal{W}(\Pi,\psi^{-1})$ such that for every generic irreducible admissible $\pi$ of $\mathrm{GL}_n(F)$, the local Rankin--Selberg integral $Z(s;W,V)$ is finite and nonzero for all $s\in\C$, for some appropriately chosen $V\in\mathcal{W}(\pi,\psi)$. The key mechanism: choose $W$ right-invariant under a deep congruence subgroup with $W(1)=1$, then for each $\pi$ gate $V$ to have support in $N_nK_n(m)$ where $m=m_W+a(\pi)$ absorbs the conductor. On this compact domain, $W(\mathrm{diag}(g,1)u_Q)$ is constant and $|\det g|=1$, making the integral $s$-independent, finite, and nonzero.
\end{abstract}

\tableofcontents

%% ===================================================================
\section{The Question (Nelson)}
%% ===================================================================

Let $F$ be a non-archimedean local field with ring of integers $\mathfrak{o}$, uniformizer $\varpi$.
Let $N_r\subset \mathrm{GL}_r(F)$ be the standard upper-triangular unipotent subgroup.
Fix a nontrivial additive character $\psi:F\to \C^\times$ of conductor $\mathfrak{o}$.
Let $\Pi$ be a generic irreducible admissible representation of $\mathrm{GL}_{n+1}(F)$ realized in its $\psi^{-1}$-Whittaker model $\mathcal{W}(\Pi,\psi^{-1})$.

Must there exist $W\in \mathcal{W}(\Pi,\psi^{-1})$ with the following property?
For every generic irreducible admissible $\pi$ of $\mathrm{GL}_n(F)$ with conductor ideal $\mathfrak{q}$ and $Q$ generating $\mathfrak{q}^{-1}$, setting $u_Q := I_{n+1}+Q\,E_{n,n+1}$, there exists $V\in \mathcal{W}(\pi,\psi)$ such that
\[
Z(s;W,V)
:=\int_{N_n\backslash \mathrm{GL}_n(F)} W(\mathrm{diag}(g,1)\,u_Q)\,V(g)\,|\det g|^{\,s-\frac12}\,dg
\]
is finite and nonzero for all $s\in \C$.

\medskip
\textbf{Answer: Yes.}

%% ===================================================================
\section{Stage 1: Recognition Science Framing}
%% ===================================================================

The RS/RSA framing packages the Rankin--Selberg integrand as a composite recognizer (RG6):
\begin{itemize}
\item Configuration space: $\mathcal{C}:=(N_n\backslash \mathrm{GL}_n(F))\times \C$ with coordinates $(g,s)$.
\item Recognizers: $R_W(g,s):=W(\mathrm{diag}(g,1)u_Q)\,|\det g|^{s-1/2}$ and $R_V(g):=V(g)$.
\item Composite: $R=R_W\otimes R_V$; the integral $Z(s)$ is the aggregate recognized mass.
\end{itemize}

The RSA audit strategy:
\begin{enumerate}
\item \textbf{LOCK}: Pick $W$ right-invariant under a deep congruence subgroup, with $W(1)=1$.
\item \textbf{PINCH}: For each $\pi$, choose $V$ supported on a compact ``resolution cell'' $N_nK_n(m)$ where $W(\mathrm{diag}(g,1)u_Q)$ is constant and $|\det g|=1$.
\item \textbf{DECIDE}: On that cell the integrand is a nonzero constant times a positive integral $\Rightarrow$ finite and nonzero for all $s$.
\end{enumerate}

%% ===================================================================
\section{Stage 2: Classical Proof}
%% ===================================================================

\subsection{Step 1: Choose $W$ with $W(1)=1$ and deep right-invariance}

Pick any nonzero $W_0\in \mathcal{W}(\Pi,\psi^{-1})$. Since $\Pi$ is smooth, choose $g_0\in \mathrm{GL}_{n+1}(F)$ with $W_0(g_0)\neq 0$ and set $W(g):=W_0(g g_0)$.
Then $W(1)=W_0(g_0)\neq 0$; rescale so $W(1)=1$.

Because $W$ is smooth (locally constant), there exists $m_W\ge 1$ such that $W$ is right-invariant under the principal congruence subgroup
\[
K_{n+1}(m_W):=I_{n+1}+\varpi^{m_W} M_{n+1}(\mathfrak{o}).
\]

\subsection{Step 2: Reduce $W(\mathrm{diag}(g,1)u_Q)$ to a constant on a compact domain}

Let $\pi$ have conductor ideal $\mathfrak{q}=\varpi^{a(\pi)}\mathfrak{o}$, so $v(Q)=-a(\pi)$.
Set $m:=m_W+a(\pi)$ and $K_n(m):=I_n+\varpi^{m} M_n(\mathfrak{o})$.

\begin{lemma}\label{lem:conjugation}
For $g\in K_n(m)$, define $k(g):=u_Q^{-1}\,\mathrm{diag}(g,1)\,u_Q$. Then $k(g)\in K_{n+1}(m_W)$.
\end{lemma}

\begin{proof}
Using $u_Q=I_{n+1}+Q E_{n,n+1}$ and $E_{n,n+1}^2=0$:
\[
k(g) = (I-QE_{n,n+1})\,\mathrm{diag}(g,1)\,(I+QE_{n,n+1}).
\]
The only deviation from $I_{n+1}$ in the last column comes from $Q\cdot(g_{i,n}-\delta_{in})$ for $i\le n$.
Since $g\in K_n(m)$, we have $g_{i,n}-\delta_{in}\in \varpi^m\mathfrak{o}$, so
$Q(g_{i,n}-\delta_{in})\in \varpi^{m-a(\pi)}\mathfrak{o}=\varpi^{m_W}\mathfrak{o}$.
All other entries of $k(g)-I_{n+1}$ lie in $\varpi^m\mathfrak{o}\subset\varpi^{m_W}\mathfrak{o}$.
Hence $k(g)\in K_{n+1}(m_W)$.
\end{proof}

Therefore, for $g\in K_n(m)$:
\[
W(\mathrm{diag}(g,1)u_Q)
= W\bigl(u_Q\,k(g)\bigr)
= W(u_Q),
\]
by right $K_{n+1}(m_W)$-invariance.

Since $u_Q=I_{n+1}+QE_{n,n+1}\in N_{n+1}$, the Whittaker property gives:
\[
W(u_Q)=\psi^{-1}(Q)\,W(1)=\psi^{-1}(Q)\neq 0.
\]

\subsection{Step 3: Choose $V$ for $\pi$ that gates the integral to $K_n(m)$}

\begin{lemma}[Howe vector / compact gating]\label{lem:howe}
Let $n\ge 2$ and $\pi$ be generic irreducible admissible. For every $m\ge 1$ there exists $V_m\in \mathcal{W}(\pi,\psi)$ with $\mathrm{supp}(V_m)\subset N_n K_n(m)$ and $V_m(1)\neq 0$.
\end{lemma}

This is standard in the local theory of Whittaker models (Jacquet--Piatetski-Shapiro--Shalika; see also the theory of new vectors / Howe vectors for $\mathrm{GL}_n$).

Fix $V:=V_m$ for $m=m_W+a(\pi)$.

\subsection{Step 4: Evaluate the integral}

Since $\mathrm{supp}(V)\subset N_n K_n(m)$:
\[
Z(s;W,V)
=\int_{N_n\backslash N_n K_n(m)}
W(\mathrm{diag}(g,1)u_Q)\,V(g)\,|\det g|^{\,s-\frac12}\,dg.
\]
On $K_n(m)\subset\mathrm{GL}_n(\mathfrak{o})$: $|\det g|=1$. By Step~2: $W(\mathrm{diag}(g,1)u_Q)=\psi^{-1}(Q)$ (constant). Therefore:
\[
Z(s;W,V)=\psi^{-1}(Q)\int_{N_n\backslash N_n K_n(m)} V(g)\,dg.
\]

This is:
\begin{itemize}
\item \textbf{Finite}: integral over a compact set of a locally constant function.
\item \textbf{Independent of $s$}: the factor $|\det g|^{s-1/2}=1$ on the support.
\item \textbf{Nonzero}: $\psi^{-1}(Q)\neq 0$ and $\int V\,dg\neq 0$ (since $V(1)\neq 0$ and the domain has positive Haar measure).
\end{itemize}

\subsection{Case $n=1$}

When $n=1$, $\mathrm{GL}_1(F)=F^\times$ and $\mathcal{W}(\pi,\psi)$ is one-dimensional ($\pi$ is a character $\chi$ of $F^\times$). Choose $W\in\mathcal{W}(\Pi,\psi^{-1})$ (now a generic $\mathrm{GL}_2(F)$ representation) so that $a\mapsto W(\mathrm{diag}(a,1))$ is compactly supported on $\mathfrak{o}^\times$ with $W(I_2)=1$.

Using the factorization $\mathrm{diag}(a,1)u_Q=\bigl(\begin{smallmatrix}1&aQ\\0&1\end{smallmatrix}\bigr)\mathrm{diag}(a,1)$ and Whittaker equivariance:
\[
W(\mathrm{diag}(a,1)u_Q)=\psi^{-1}(aQ)\,W(\mathrm{diag}(a,1)).
\]
The integral becomes
\[
Z(s;W,\chi)=\int_{\mathfrak{o}^\times} \psi^{-1}(aQ)\,W(\mathrm{diag}(a,1))\,\chi(a)\,|a|^{s-1/2}\,da.
\]
On $\mathfrak{o}^\times$: $|a|=1$, so this is $s$-independent, finite (compact domain), and a local Gauss sum that is nonzero for the appropriate choice of $W$.

%% ===================================================================
\section{Verification Notes}
%% ===================================================================

\begin{remark}[Steps verified]
\begin{enumerate}
\item \emph{$W(1)=1$ achievable}: smooth + right-translation. \checkmark
\item \emph{Right-invariance}: smooth $\Rightarrow$ locally constant $\Rightarrow$ $K_{n+1}(m_W)$-invariant. \checkmark
\item \emph{Conjugation bound}: $E_{n,n+1}^2=0$; error $Q(g-I)$ has valuation $\geq m_W$. \checkmark
\item \emph{$W(u_Q)\neq 0$}: $u_Q\in N_{n+1}$; Whittaker property gives $\psi^{-1}(Q)\in\C^\times$. \checkmark
\item \emph{Compact gating}: Howe vectors for generic $\pi$ with $n\geq 2$ (standard). \checkmark
\item \emph{$|\det g|=1$ on $K_n(m)$}: $K_n(m)\subset\mathrm{GL}_n(\mathfrak{o})$. \checkmark
\item \emph{Nonvanishing}: $\psi^{-1}(Q)\neq 0$ and $\int V>0$. \checkmark
\item \emph{$s$-independence}: $|\det g|^{s-1/2}=1$ on support. \checkmark
\end{enumerate}
\end{remark}

\begin{remark}[The crucial mechanism]
The interplay between the conductor $a(\pi)$ and the invariance level $m_W$ is the heart of the proof. Setting $m = m_W + a(\pi)$ ensures that conjugation by $u_Q$ (which introduces errors of size $Q \sim \varpi^{-a(\pi)}$) is compensated by the deeper congruence level of $g$ (errors of size $\varpi^m$), giving a net precision of $\varpi^{m_W}$---exactly the right-invariance level of $W$.
\end{remark}

\begin{thebibliography}{9}
\bibitem{AbouzaidEtAl2026}
M.~Abouzaid et al.
\newblock First Proof.
\newblock \emph{arXiv:2602.05192}, February 2026.

\bibitem{JPSS1983}
H.~Jacquet, I.~I.~Piatetski-Shapiro, and J.~A.~Shalika.
\newblock Rankin--Selberg convolutions.
\newblock \emph{Amer. J. Math.}, 105(2):367--464, 1983.
\end{thebibliography}

\end{document}
