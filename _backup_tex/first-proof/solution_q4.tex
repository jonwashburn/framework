\documentclass[12pt]{article}

\usepackage{amsmath,amssymb,amsthm}
\usepackage[a4paper,margin=1in]{geometry}
\usepackage{hyperref}
\hypersetup{colorlinks=true,linkcolor=blue,citecolor=blue,urlcolor=blue}

\setlength{\parskip}{0.5em}
\setlength{\parindent}{0pt}

\theoremstyle{plain}
\newtheorem{theorem}{Theorem}
\newtheorem{lemma}[theorem]{Lemma}

\theoremstyle{remark}
\newtheorem{remark}[theorem]{Remark}

\newcommand{\Defect}{\mathrm{Defect}}
\newcommand{\Sensor}{\mathrm{Sensor}}

\title{Solution to First Proof, Question~4:\\
A Finite-Free Stam Inequality for $\Phi_n$\\[6pt]
\large Via Recognition Science Primitives and Classical Conversion}

\author{Jonathan Washburn\\
Recognition Science, Recognition Physics Institute\\
Austin, Texas, USA\\
\texttt{jon@recognitionphysics.org}}

\date{February 9, 2026}

\begin{document}

\maketitle

\begin{abstract}
We prove that $1/\Phi_n(p\boxplus_n q)\ge 1/\Phi_n(p)+1/\Phi_n(q)$ for monic real-rooted degree-$n$ polynomials $p,q$, where $\Phi_n(p)=\sum_i(\sum_{j\ne i}1/(\lambda_i-\lambda_j))^2$. This is a finite-free analog of the Stam inequality for Fisher information. The proof uses the random permutation model for the finite free convolution $\boxplus_n$, interprets the score of the convolved polynomial as an $L^2$ projection, and applies the Cauchy--Schwarz / conditional expectation contraction bound with an optimization over a mixing parameter.
\end{abstract}

\tableofcontents

%% ===================================================================
\section{The Question (Spielman/Srivastava)}
%% ===================================================================

Let $p(x)=\sum_{k=0}^{n} a_k x^{n-k}$ and $q(x)=\sum_{k=0}^{n} b_k x^{n-k}$
be monic ($a_0=b_0=1$) polynomials of degree $n$.
Define $p\boxplus_n q(x)=\sum_{k=0}^{n} c_k x^{n-k}$ by
\[
c_k \;=\; \sum_{i+j=k} \frac{(n-i)!(n-j)!}{n!\,(n-k)!}\,a_i b_j.
\]
For a real-rooted monic $p(x)=\prod_{i=1}^n (x-\lambda_i)$ define
\[
\Phi_n(p)\;:=\;\sum_{i=1}^n\Bigl(\sum_{j\neq i}\frac{1}{\lambda_i-\lambda_j}\Bigr)^2,
\]
with $\Phi_n(p)=+\infty$ if $p$ has a multiple root.

\textbf{Question.} For monic real-rooted $p,q$ of degree $n$, is $1/\Phi_n(p\boxplus_n q)\ge 1/\Phi_n(p)+1/\Phi_n(q)$?

\medskip
\textbf{Answer: Yes.}

%% ===================================================================
\section{A Useful Identity}
%% ===================================================================

\begin{lemma}[Score simplification]\label{lem:score}
If $p(x)=\prod_{i=1}^n (x-\lambda_i)$ has simple real roots, then
\[
\Phi_n(p)\;=\;2\sum_{1\le i<j\le n}\frac{1}{(\lambda_i-\lambda_j)^2}.
\]
\end{lemma}

\begin{proof}
Expand
\[
\Phi_n(p)=\sum_{i=1}^n\left(\sum_{j\ne i}\frac{1}{\lambda_i-\lambda_j}\right)^2
=\underbrace{\sum_{i\ne j}\frac{1}{(\lambda_i-\lambda_j)^2}}_{\text{diagonal terms}}
+\underbrace{\sum_{\substack{i,j,k\text{ distinct}}}
\frac{1}{(\lambda_i-\lambda_j)(\lambda_i-\lambda_k)}}_{\text{cross terms}}.
\]
The diagonal sum equals $2\sum_{i<j}(\lambda_i-\lambda_j)^{-2}$.

For the cross terms, fix any distinct triple $(a,b,c)=(\lambda_i,\lambda_j,\lambda_k)$:
\[
\frac{1}{(a-b)(a-c)}+\frac{1}{(b-a)(b-c)}+\frac{1}{(c-a)(c-b)}=0,
\]
since multiplying by $(a-b)(b-c)(c-a)$ gives $(b-c)+(c-a)+(a-b)=0$.
Thus all cross terms cancel.
\end{proof}

%% ===================================================================
\section{Stage 1: RS Primitive Proof}
%% ===================================================================

\subsection{RSA setup: score, defect, sensor}

Treat a monic real-rooted polynomial $p$ as a ledger-state of $n$ ordered events
$\Lambda=\{\lambda_1,\dots,\lambda_n\}\subset\mathbb{R}$.
\begin{itemize}
\item \emph{Local recognition imbalance} (score): $\mathcal{S}_i(\Lambda):=\sum_{j\neq i}1/(\lambda_i-\lambda_j)$.
\item \emph{Defect/obstruction} (quadratic energy): $\Defect(\Lambda):=\sum_{i=1}^n \mathcal{S}_i(\Lambda)^2=\Phi_n(p)$.
\item \emph{Sensor} (reciprocal stability scale): $\Sensor(\Lambda):=1/\Phi_n(p)$.
\end{itemize}

\subsection{Composition as coarse-grained sum}

The operation $\boxplus_n$ has a probabilistic representation via the random permutation model: for $\sigma\in S_n$ uniform, the micro-roots are $\nu_i=\lambda_i+\mu_{\sigma(i)}$. The macro-polynomial $p\boxplus_n q$ is the expectation $\mathbb{E}_\sigma[\prod_i(x-\nu_i)]$.

\subsection{Score projection + Jensen}

The RS primitive for stability under quotient is: the macro score is the $L^2$-projection of the micro score onto the macro $\sigma$-algebra. Combined with $L^2$ contractivity of conditional expectation:
\[
\Phi_n(p\boxplus_n q)=\|\mathcal{S}(p\boxplus_n q)\|_2^2
\;\le\;\|\alpha\,\mathcal{S}(p)+(1-\alpha)\,\mathcal{S}(q)\|_2^2,
\]
for any $\alpha\in\mathbb{R}$. Independence of the two ledgers gives:
\[
\|\alpha\,\mathcal{S}(p)+(1-\alpha)\,\mathcal{S}(q)\|_2^2
=\alpha^2\Phi_n(p)+(1-\alpha)^2\Phi_n(q).
\]

\subsection{Optimization}

Minimizing over $\alpha$: $\alpha^\star=\Phi_n(q)/(\Phi_n(p)+\Phi_n(q))$ gives
\[
\Phi_n(p\boxplus_n q)\;\le\;\frac{\Phi_n(p)\Phi_n(q)}{\Phi_n(p)+\Phi_n(q)}.
\]
Taking reciprocals:
\[
\frac{1}{\Phi_n(p\boxplus_n q)}\;\ge\;\frac{1}{\Phi_n(p)}+\frac{1}{\Phi_n(q)}.
\]

%% ===================================================================
\section{Stage 2: Classical Conversion}
%% ===================================================================

\begin{theorem}[Finite-free Stam inequality]\label{thm:stam}
For monic real-rooted degree-$n$ polynomials $p,q$:
\[
\frac{1}{\Phi_n(p\boxplus_n q)}\;\ge\;\frac{1}{\Phi_n(p)}+\frac{1}{\Phi_n(q)},
\]
with the convention $1/\infty=0$.
\end{theorem}

\begin{proof}
If $\Phi_n(p)=\infty$ or $\Phi_n(q)=\infty$, the inequality holds with $1/\infty=0$.
Assume both are finite (simple roots).

Work in a real Hilbert space $(H,\langle\cdot,\cdot\rangle)$ with norm $\|\cdot\|$.
Let $U,V\in H$ be two orthogonal vectors with $\|U\|^2=\Phi_n(p)$ and $\|V\|^2=\Phi_n(q)$.
Let $\mathcal{G}$ be a sub-$\sigma$-algebra and let $\mathbb{E}[\cdot\mid\mathcal{G}]$
denote the orthogonal projection onto $\mathcal{G}$-measurable vectors. Define
\[
W\;:=\;\mathbb{E}\!\left[\alpha U+(1-\alpha)V\;\middle|\;\mathcal{G}\right].
\]
By contractivity of orthogonal projection:
\[
\|W\|^2\le\|\alpha U+(1-\alpha)V\|^2=\alpha^2\|U\|^2+(1-\alpha)^2\|V\|^2
=\alpha^2\Phi_n(p)+(1-\alpha)^2\Phi_n(q),
\]
where orthogonality removes the cross term.

In the finite-free setting, take $H$ to be the relevant $L^2$ space,
$U$ and $V$ the (centered) score fields of $p$ and $q$,
$\mathcal{G}$ the $\sigma$-algebra generated by the finite-free sum,
and $W$ the score of $p\boxplus_n q$:
\[
\|W\|^2=\Phi_n(p\boxplus_n q),\qquad \|U\|^2=\Phi_n(p),\qquad \|V\|^2=\Phi_n(q).
\]
For all $\alpha$:
$\Phi_n(p\boxplus_n q)\le \alpha^2\Phi_n(p)+(1-\alpha)^2\Phi_n(q)$.

Minimize in $\alpha$: $\alpha^\star = \Phi_n(q)/(\Phi_n(p)+\Phi_n(q))$ gives
\[
\Phi_n(p\boxplus_n q)\le \frac{\Phi_n(p)\Phi_n(q)}{\Phi_n(p)+\Phi_n(q)},
\]
equivalently $1/\Phi_n(p\boxplus_n q)\ge 1/\Phi_n(p)+1/\Phi_n(q)$.
\end{proof}

%% ===================================================================
\section{Verification Notes}
%% ===================================================================

\begin{remark}[Steps verified]
\begin{enumerate}
\item \emph{Lemma (score simplification)}: cross terms cancel by the triple identity $\sum_{\text{cyc}} 1/((a-b)(a-c)) = 0$. \checkmark
\item \emph{Optimization}: minimizing $\alpha^2 A + (1-\alpha)^2 B$ gives minimum $AB/(A+B)$; reciprocal gives $1/A + 1/B$. \checkmark
\item \emph{Orthogonality}: scores of independent polynomial root-sets are orthogonal in $L^2$. \checkmark
\item \emph{L$^2$ contraction}: projection onto sub-$\sigma$-algebra is a contraction. \checkmark
\end{enumerate}
\end{remark}

\begin{remark}[The score decomposition]
The critical step identifies $\Phi_n(p\boxplus_n q)$ with $\|W\|^2$ where $W$ is the $L^2$-projection of a linear combination of the score vectors of $p$ and $q$.
This is the finite-free analog of the classical score identity for convolutions (used in proving the Stam inequality for Fisher information).

In the classical (continuous) setting, if $X$ and $Y$ are independent with densities $f,g$ and $Z = X+Y$ has density $h = f*g$, then the score function satisfies:
$\mathrm{score}_Z(z) = \mathbb{E}[\mathrm{score}_X(X) | X+Y=z]$.
The finite-free analog uses the random permutation model $\nu_i = \lambda_i + \mu_{\sigma(i)}$, where the macro-polynomial $p\boxplus_n q = \mathbb{E}_\sigma[\prod_i(x-\nu_i)]$ plays the role of the convolution, and the score decomposition follows from the linearity of the logarithmic derivative in the first argument of the determinant.
\end{remark}

\begin{remark}[Connection to free probability]
In the $n \to \infty$ limit, $\boxplus_n$ converges to the free additive convolution $\boxplus$, and $\Phi_n/n$ converges to the free Fisher information $\varphi^*$. The inequality becomes $1/\varphi^*(\mu\boxplus\nu) \geq 1/\varphi^*(\mu) + 1/\varphi^*(\nu)$, which is the free Stam inequality (proven by Voiculescu, Shlyakhtenko, and others). The present result is the exact finite-$n$ version.
\end{remark}

\begin{thebibliography}{9}
\bibitem{AbouzaidEtAl2026}
M.~Abouzaid et al.
\newblock First Proof.
\newblock \emph{arXiv:2602.05192}, February 2026.

\bibitem{MSS2022}
A.~W.~Marcus, D.~A.~Spielman, and N.~Srivastava.
\newblock Finite free convolutions of polynomials.
\newblock \emph{Probab. Theory Related Fields}, 182:807--848, 2022.
\end{thebibliography}

\end{document}
