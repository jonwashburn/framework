\documentclass[11pt]{article}

\usepackage{amsmath,amssymb,amsthm}
\usepackage[a4paper,margin=1in]{geometry}
\usepackage{hyperref}
\hypersetup{colorlinks=true,linkcolor=blue,citecolor=blue,urlcolor=blue}

\setlength{\parskip}{0.5em}
\setlength{\parindent}{0pt}

\theoremstyle{plain}
\newtheorem{theorem}{Theorem}
\newtheorem{lemma}[theorem]{Lemma}

\theoremstyle{remark}
\newtheorem{remark}[theorem]{Remark}

\title{Solution to First Proof, Question~7:\\
Lattice Fundamental Groups with\\
Rationally Acyclic Universal Cover\\[6pt]
\large Via Recognition Science Primitives and Classical Conversion}

\author{Jonathan Washburn\\
Recognition Science, Recognition Physics Institute\\
Austin, Texas, USA\\
\texttt{jon@recognitionphysics.org}}

\date{February 9, 2026}

\begin{document}

\maketitle

\begin{abstract}
We prove that if $M$ is a compact manifold without boundary whose universal cover is $\mathbb{Q}$-acyclic, then $\pi_1(M)$ is torsion-free. In particular, a uniform lattice $\Gamma$ in a real semi-simple group cannot simultaneously contain $2$-torsion and serve as the fundamental group of such a manifold. The proof uses Poincar\'e duality for non-compact manifolds to compute the compactly-supported Lefschetz number of a torsion element, obtaining $\pm 1 \neq 0$, which contradicts the Lefschetz fixed-point theorem for the fixed-point-free deck transformation.
\end{abstract}

\tableofcontents

%% ===================================================================
\section{The Question (Weinberger)}
%% ===================================================================

Suppose that $\Gamma$ is a uniform lattice in a real semi-simple group, and that $\Gamma$ contains some $2$-torsion. Is it possible for $\Gamma$ to be the fundamental group of a compact manifold without boundary whose universal cover is acyclic over the rational numbers $\mathbb{Q}$?

\medskip
\textbf{Answer: No.}

\begin{theorem}\label{thm:main}
If $M$ is a compact manifold without boundary and its universal cover
$\widetilde M$ is $\mathbb{Q}$-acyclic (i.e.\ $H_i(\widetilde M;\mathbb{Q})=0$ for all $i>0$), then $\pi_1(M)$ is torsion-free.
Hence a group $\Gamma$ containing $2$-torsion cannot arise as such a fundamental group.
\end{theorem}

%% ===================================================================
\section{Stage 1: RS Primitive Proof (Defect/Obstruction Style)}
%% ===================================================================

Recognition Science uses the pattern ``state exists $\Leftrightarrow$ defect $=0$'' as a primitive organizing move.
We encode the feasibility question by a vanishing obstruction.

Assume for contradiction that there is a compact $n$-manifold $M$ with $\pi_1(M)=\Gamma$ and universal cover
$X:=\widetilde M$ satisfying $H_i(X;\mathbb{Q})=0$ for all $i>0$.

Let $g\in \Gamma$ be an element of order $2$.
As a deck transformation, $g$ acts freely on $X$ (no fixed points).

\medskip
\noindent\textbf{RS-style obstruction/defect.}
Define the (rational) \emph{fixed-point obstruction} (``defect'') of $g$ on $X$ to be the compact-support Lefschetz number
\[
\Delta(g;X)\;:=\;\sum_{i=0}^n (-1)^i\,\mathrm{tr}\!\left(g^\ast\big|_{H_c^i(X;\mathbb{Q})}\right)\;\in\;\mathbb{Q}.
\]
This is an \emph{impossibility certificate} in the RSA sense: if the action is fixed-point-free, then the fixed-point index
sum is empty, hence must evaluate to $0$, i.e.\ a realizable ``no-fixed-point'' state forces $\Delta(g;X)=0$.

\medskip
\noindent\textbf{Compute the defect from the RS-given invariant (acyclicity).}
Because $X$ is an oriented $n$-manifold (universal covers are orientable) and $H_\ast(X;\mathbb{Q})$ is trivial above degree $0$,
Poincar\'e duality for noncompact manifolds gives
\[
H_c^i(X;\mathbb{Q}) \cong H_{n-i}(X;\mathbb{Q}),
\]
so $H_c^i(X;\mathbb{Q})=0$ for $i<n$ and $H_c^n(X;\mathbb{Q})\cong H_0(X;\mathbb{Q})\cong \mathbb{Q}$.
Therefore
\[
\Delta(g;X)=(-1)^n\,\mathrm{tr}\!\left(g^\ast\big|_{H_c^n(X;\mathbb{Q})}\right).
\]
On the $1$-dimensional space $H_c^n(X;\mathbb{Q})\cong \mathbb{Q}$, the induced map $g^\ast$ is multiplication by $\pm 1$
(orientation character), hence $\Delta(g;X)=\pm 1\neq 0$.

\medskip
\noindent\textbf{Contradiction.}
A fixed-point-free (deck) involution forces $\Delta(g;X)=0$, but $\mathbb{Q}$-acyclicity forces $\Delta(g;X)=\pm 1$.
Thus such a $g$ cannot exist, so $\Gamma$ cannot contain $2$-torsion (or indeed any torsion).

%% ===================================================================
\section{Stage 2: Classical Proof}
%% ===================================================================

\begin{proof}[Proof of Theorem~\ref{thm:main}]
Assume $M$ is a compact $n$-manifold without boundary with $\pi_1(M)=\Gamma$ and $X:=\widetilde M$ satisfying
$H_i(X;\mathbb{Q})=0$ for $i>0$.

Let $g\in \Gamma$ have finite order (in particular order $2$ as in the problem). As a deck transformation,
$g:X\to X$ is a fixed-point-free homeomorphism.

\medskip\noindent
\emph{Step 1: Compute $H_c^*(X;\mathbb{Q})$.}
Since $X$ is a connected oriented $n$-manifold without boundary (the universal cover of an $n$-manifold is an $n$-manifold; universal covers are orientable), Poincar\'e duality identifies compactly supported cohomology with homology:
\[
H_c^i(X;\mathbb{Q})\cong H_{n-i}(X;\mathbb{Q}).
\]
The $\mathbb{Q}$-acyclicity hypothesis gives $H_{k}(X;\mathbb{Q})=0$ for $k>0$ and $H_0(X;\mathbb{Q})\cong \mathbb{Q}$ (connectedness), hence
\[
H_c^i(X;\mathbb{Q})=0 \text{ for } i<n,\qquad H_c^n(X;\mathbb{Q})\cong \mathbb{Q}.
\]
(Consistency check: $H_c^0(X;\mathbb{Q}) = 0$ for a non-compact connected space, and indeed $H_n(X;\mathbb{Q}) = 0$ for a non-compact manifold.)

\medskip\noindent
\emph{Step 2: Compute the compactly-supported Lefschetz number.}
Define
\[
L_c(g):=\sum_{i=0}^n (-1)^i\,\mathrm{tr}\!\left(g^\ast\big|_{H_c^i(X;\mathbb{Q})}\right).
\]
From Step~1, the only nonzero contribution is from $i = n$:
\[
L_c(g)=(-1)^n\,\mathrm{tr}\!\left(g^\ast\big|_{H_c^n(X;\mathbb{Q})}\right).
\]
Because $H_c^n(X;\mathbb{Q})$ is one-dimensional, $g^\ast$ acts by $\varepsilon(g) = \pm 1$ (the orientation character of $g$), so
\[
L_c(g)=(-1)^n\,\varepsilon(g) = \pm 1\neq 0.
\]

\medskip\noindent
\emph{Step 3: Apply the Lefschetz fixed-point theorem.}
The deck transformation $g: X \to X$ is a homeomorphism, hence a proper map. By the Lefschetz--Hopf fixed-point theorem for proper self-maps of locally compact ANRs (manifolds are ANRs): if $g$ has no fixed points, then $L_c(g) = 0$.

But $g$ is a nontrivial deck transformation, so it acts freely (no fixed points) on $X$, giving $L_c(g) = 0$.

This contradicts $L_c(g) = \pm 1 \neq 0$.

\medskip\noindent
\emph{Step 4: Conclusion.}
No finite-order element $g \in \Gamma$ can exist. In particular, $\Gamma$ is torsion-free, so a group containing $2$-torsion cannot be the fundamental group of such a manifold.
\end{proof}

%% ===================================================================
\section{Verification Notes}
%% ===================================================================

\begin{remark}[Stronger conclusion]
The proof shows that $\pi_1(M)$ must be \emph{entirely torsion-free}, not just free of $2$-torsion. The argument works for any finite-order element $g$ (not just involutions), because $g^*$ on $H_c^n(X;\mathbb{Q}) \cong \mathbb{Q}$ acts by a root of unity in $\mathbb{Q}$, hence by $\pm 1$.
\end{remark}

\begin{remark}[Why the lattice hypothesis is not used]
The proof does not actually use the hypothesis that $\Gamma$ is a uniform lattice in a semi-simple group. It applies to \emph{any} group $\Gamma$ that is the fundamental group of a compact manifold with $\mathbb{Q}$-acyclic universal cover. The lattice hypothesis in the problem provides context (such groups naturally arise in geometry) but is not logically needed.
\end{remark}

\begin{remark}[Why $\mathbb{Q}$-acyclicity, not $\mathbb{Z}/2$-acyclicity]
Smith theory gives stronger conclusions from $\mathbb{Z}/p$-acyclicity (the fixed-point set is also $\mathbb{Z}/p$-acyclic). But the problem specifies $\mathbb{Q}$-acyclicity, which is a weaker hypothesis. The Lefschetz number argument works with $\mathbb{Q}$ coefficients, where Poincar\'e duality gives the clean computation $H_c^* = \mathbb{Q}$ in top degree.
\end{remark}

\begin{remark}[Lefschetz theorem for proper maps]
The proof uses the Lefschetz--Hopf theorem for proper self-maps of locally compact ANRs: if $f: X \to X$ is proper with no fixed points, then $L_c(f) = 0$. For manifolds, this is well-established; see Dold, \emph{Lectures on Algebraic Topology}, Chapter VIII, or Brown, \emph{The Lefschetz Fixed Point Theorem} (1971). The key input is that the local fixed-point index at each fixed point contributes to $L_c$, and with no fixed points, the sum is empty.
\end{remark}

\begin{remark}[Steps verified]
\begin{enumerate}
\item \emph{$X$ is non-compact}: $\Gamma$ is infinite (lattice in semi-simple group), so the universal cover is non-compact. \checkmark
\item \emph{Poincar\'e duality}: $X$ is an oriented $n$-manifold without boundary; PD gives $H_c^i \cong H_{n-i}$. \checkmark
\item \emph{Cohomology vanishing}: $H_c^i = 0$ for $i < n$; $H_c^n \cong \mathbb{Q}$. Consistent with $H_c^0 = 0$ for non-compact connected $X$. \checkmark
\item \emph{Lefschetz computation}: $L_c(g) = (-1)^n \varepsilon(g) = \pm 1 \neq 0$. \checkmark
\item \emph{Lefschetz theorem}: deck transformation is proper + fixed-point-free $\Rightarrow$ $L_c = 0$. \checkmark
\item \emph{Contradiction}: $\pm 1 \neq 0$. \checkmark
\end{enumerate}
\end{remark}

\begin{thebibliography}{9}
\bibitem{AbouzaidEtAl2026}
M.~Abouzaid et al.
\newblock First Proof.
\newblock \emph{arXiv:2602.05192}, February 2026.

\bibitem{Brown1971}
R.~F.~Brown.
\newblock \emph{The Lefschetz Fixed Point Theorem}.
\newblock Scott, Foresman and Co., 1971.

\bibitem{Dold1972}
A.~Dold.
\newblock \emph{Lectures on Algebraic Topology}.
\newblock Springer-Verlag, 1972.
\end{thebibliography}

\end{document}
