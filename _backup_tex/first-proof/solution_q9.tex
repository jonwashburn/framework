\documentclass[11pt]{article}

\usepackage{amsmath,amssymb,amsthm}
\usepackage[a4paper,margin=1in]{geometry}
\usepackage{hyperref}
\hypersetup{colorlinks=true,linkcolor=blue,citecolor=blue,urlcolor=blue}

\setlength{\parskip}{0.5em}
\setlength{\parindent}{0pt}

\theoremstyle{plain}
\newtheorem{theorem}{Theorem}
\newtheorem{lemma}[theorem]{Lemma}
\newtheorem{proposition}[theorem]{Proposition}
\newtheorem{corollary}[theorem]{Corollary}

\theoremstyle{definition}
\newtheorem{definition}[theorem]{Definition}

\theoremstyle{remark}
\newtheorem{remark}[theorem]{Remark}

\newcommand{\R}{\mathbb{R}}

\title{Solution to First Proof, Question~9:\\
Algebraic Relations on Determinantal Tensors\\
and Rank-One Scaling Detection\\[6pt]
\large Via Recognition Science Primitives and Classical Conversion}

\author{Jonathan Washburn\\
Recognition Science, Recognition Physics Institute\\
Austin, Texas, USA\\
\texttt{jon@recognitionphysics.org}}

\date{February 9, 2026}

\begin{document}

\maketitle

\begin{abstract}
We prove that a polynomial map $\mathbf{F}:\R^{81n^4}\to\R^N$ exists satisfying the three stated properties. The answer is \textbf{yes}. The map $\mathbf{F}$ is the collection of all $5\times 5$ minors of the four mode-unfoldings of the input tensor (after a fixed reshaping). The degree is $5$, independent of both $n$ and $A$. The key geometric fact is that the determinantal tensors $Q$ have mode-ranks exactly $4$ (because the row-vectors live in $\R^4$), and rank-one diagonal scaling preserves mode-ranks. The converse---that mode-rank $\leq 4$ forces rank-one $\lambda$---follows by an iterative separation argument using the uniqueness (up to proportion) of linear dependencies among $5$ vectors in $\R^4$.
\end{abstract}

\tableofcontents

%% ===================================================================
\section{The Question (Kileel)}
%% ===================================================================

Fix $n\ge 5$ and Zariski-generic matrices $A^{(1)},\dots,A^{(n)}\in\R^{3\times 4}$.
For each $(\alpha,\beta,\gamma,\delta)\in[n]^4$, define a tensor
$Q^{(\alpha\beta\gamma\delta)}\in\R^{3\times 3\times 3\times 3}$ by
\[
Q^{(\alpha\beta\gamma\delta)}_{ijk\ell}
\;:=\;
\det\!\begin{bmatrix}
A^{(\alpha)}_{i,*}\\
A^{(\beta)}_{j,*}\\
A^{(\gamma)}_{k,*}\\
A^{(\delta)}_{\ell,*}
\end{bmatrix}
\qquad (i,j,k,\ell\in[3]).
\]
Does there exist a polynomial map $\mathbf{F}:\R^{81n^4}\to\R^N$ satisfying the three properties stated in the problem?

\medskip
\textbf{Answer: Yes.}

%% ===================================================================
\section{Stage 1: Recognition Science Formulation}
%% ===================================================================

In RG2 language (Recognition Geometry):
\begin{itemize}
\item Configuration space: $\mathcal{C}:=\R^{81n^4}$ (the raw observed data array).
\item Event space: $\mathcal{E}:=\R^N$.
\item Candidate state $S$: ``$\lambda$ factorizes off the diagonal as $u\otimes v\otimes w\otimes x$''.
\item Defect functional (RS-style): a map $\Delta_S:\mathcal{C}\to \mathcal{E}$ whose zeros encode $S$.
\end{itemize}

The RSA mindset: choose a \emph{finite} defect encoding whose coordinates have \emph{bounded complexity}
(degree not growing with $n$) and which is \emph{parameter-free} (no dependence on hidden $A$).
The ambient geometric fact: all determinants live in an underlying
$4$-dimensional latent space (the row-vectors live in $\R^4$), so \emph{rank-$4$} is the
intrinsic finite-resolution scale (RG4). The corresponding finite certificate is: \emph{all $5\times 5$ minors vanish.}

%% ===================================================================
\section{Stage 2: Classical Construction and Proof}
%% ===================================================================

\subsection{Step 1: Reshape into a $(3n)^4$ tensor}

Identify the input family
$T^{(\alpha\beta\gamma\delta)}:=\lambda_{\alpha\beta\gamma\delta}Q^{(\alpha\beta\gamma\delta)}$
with a single order-$4$ tensor $\mathcal{T}\in\R^{(3n)\times(3n)\times(3n)\times(3n)}$
by bundling the pair-index $(\alpha,i)$ into one index in $[3n]$:
\[
p=3(\alpha-1)+i,\quad q=3(\beta-1)+j,\quad r=3(\gamma-1)+k,\quad s=3(\delta-1)+\ell,
\]
setting $\mathcal{T}_{pqrs} := T^{(\alpha\beta\gamma\delta)}_{ijk\ell}$.
This is only a fixed permutation/reshaping of coordinates.

\subsection{Step 2: Define the four mode-unfoldings}

For $m\in\{1,2,3,4\}$, let $U_m(\mathcal{T})$ be the standard mode-$m$ matricization:
\begin{align*}
U_1(\mathcal{T}) &\in \R^{(3n)\times(3n)^3}, &
U_1(\mathcal{T})_{p,(q,r,s)} &:= \mathcal{T}_{pqrs},\\
U_2(\mathcal{T}) &\in \R^{(3n)\times(3n)^3}, &
U_2(\mathcal{T})_{q,(p,r,s)} &:= \mathcal{T}_{pqrs},
\end{align*}
and similarly for $U_3, U_4$.

\subsection{Step 3: Define $\mathbf{F}$ as all $5\times 5$ minors}

\begin{definition}[The polynomial map]
Define
\[
\mathbf{F}(\mathcal{T})
:=
\Big(
\det\big(U_m(\mathcal{T})[I,J]\big)
\Big)_{\substack{m\in\{1,2,3,4\}\\ I\subset[3n],\,J\subset[(3n)^3]\\ |I|=|J|=5}}
\in\R^N,
\]
where $U_m(\mathcal{T})[I,J]$ denotes the $5\times 5$ submatrix with row-set $I$ and column-set $J$.
\end{definition}

\paragraph{Property checks (items 1--2).}
\begin{itemize}
\item $\mathbf{F}$ depends only on the coordinates of $\mathcal{T}$: independent of $A$. \checkmark
\item Each coordinate is a $5\times 5$ determinant, hence degree $5$ in $\mathcal{T}$, independent of $n$. \checkmark
\end{itemize}

\subsection{Key geometric fact: mode-rank $4$}

Write the row vectors of $A^{(\alpha)}$ as $r_{\alpha,i}\in\R^4$ ($i\in[3]$).
Then
\[
\mathcal{Q}_{pqrs}
=
\det(r_{\alpha(p),i(p)},\;r_{\beta(q),j(q)},\;r_{\gamma(r),k(r)},\;r_{\delta(s),\ell(s)}).
\]
The determinant is linear in the first argument, so the mode-1 unfolding factors:
\[
U_1(\mathcal{Q})_{p,\mathbf{c}} = \sum_{a=1}^4 [r_{\alpha(p),i(p)}]_a \cdot C_a(\mathbf{c}),
\]
where $C_a(\mathbf{c})$ is the cofactor of position $a$ in the first row.
Hence $U_1(\mathcal{Q}) = R\cdot C$ with $R\in\R^{3n\times 4}$, $C\in\R^{4\times(3n)^3}$, and $\mathrm{rank}(U_1(\mathcal{Q}))\le 4$.
For Zariski-generic $A$, the rank is exactly $4$.

\begin{remark}[Diagonal blocks vanish]
For $\alpha=\beta=\gamma=\delta$: $Q^{(\alpha\alpha\alpha\alpha)}_{ijk\ell}=\det(r_{\alpha,i},r_{\alpha,j},r_{\alpha,k},r_{\alpha,\ell})$.
Since $i,j,k,\ell\in[3]$ and we need $4$ entries, at least two indices coincide, giving repeated rows.
Hence $Q^{(\alpha\alpha\alpha\alpha)}=0$ for all inner indices.
\end{remark}

\subsection{Forward direction ($\Rightarrow$): rank-1 implies $\mathbf{F}=0$}

Assume $\lambda_{\alpha\beta\gamma\delta}=u_\alpha v_\beta w_\gamma x_\delta$ for all non-identical tuples
(with $\lambda=0$ on the diagonal, matching $Q=0$).

Define lifted vectors $\widehat{u}_{3(\alpha-1)+i}=u_\alpha$, and similarly for $\widehat{v},\widehat{w},\widehat{x}$.
Then for every entry:
\[
\mathcal{T}_{pqrs}
=
\widehat{u}_p\,\widehat{v}_q\,\widehat{w}_r\,\widehat{x}_s \,\mathcal{Q}_{pqrs}.
\]
This is a separable (mode-wise) diagonal scaling of $\mathcal{Q}$:
\[
U_1(\mathcal{T}) = D_{\widehat{u}} \cdot U_1(\mathcal{Q}) \cdot D_{\widehat{v}\otimes\widehat{w}\otimes\widehat{x}},
\]
where $D_\cdot$ denotes diagonal matrices. Diagonal scaling does not change rank, so
$\mathrm{rank}(U_1(\mathcal{T}))\le 4$ and all $5\times 5$ minors vanish. The same holds for modes 2--4.
Hence $\mathbf{F}(\mathcal{T})=0$. \qed

\subsection{Backward direction ($\Leftarrow$): $\mathbf{F}=0$ implies rank-1}

Assume $\mathbf{F}(\mathcal{T})=0$. Then $\mathrm{rank}(U_m(\mathcal{T}))\le 4$ for each mode $m$.

\begin{lemma}[Diagonal rigidity]\label{lem:rigidity}
Let $B\in\R^{4\times M}$ have full row rank $4$ and assume \emph{every} $4\times 4$ minor of $B$ is nonzero.
If $D\in(\R^*)^{M\times M}$ is diagonal and there exists $G\in\mathrm{GL}_4$ with $G\,B = B\,D$,
then $D=cI$ for some $c\in\R^*$.
\end{lemma}

\begin{proof}
Pick any $4$-subset $I\subset[M]$ and take determinants of the $4\times 4$ submatrix $B_I$:
\[
\det(G)\det(B_I) = \det(G B_I) = \det(B_I D_I) = \det(B_I)\prod_{i\in I} d_i.
\]
Since $\det(B_I)\neq 0$, we get $\prod_{i\in I} d_i=\det(G)$ for every $|I|=4$.
Comparing two subsets sharing three indices forces all $d_i$ equal, hence $D=cI$.
\end{proof}

\paragraph{Iterative mode separation.}
The rank-$\le 4$ condition on $U_1(\mathcal{T})$ is used as follows.
Choose $5$ distinct matrix indices $\alpha_1,\ldots,\alpha_5\in[n]$ and fix a row index $i_0\in[3]$.
The $5$ vectors $r_{\alpha_m,i_0}\in\R^4$ satisfy a unique (up to scalar) linear dependency
$\sum_{m=1}^5 d_m r_{\alpha_m,i_0}=0$ with $d_m\neq 0$ (Zariski-genericity: any $4$ are independent).

By multilinearity of the determinant, $\sum_m d_m\, U_1(\mathcal{Q})_{(\alpha_m,i_0),\mathbf{c}}=0$ for all columns $\mathbf{c}$.

Since $\mathrm{rank}(U_1(\mathcal{T}))\le 4$, the $5$ scaled rows
$U_1(\mathcal{T})_{(\alpha_m,i_0),\mathbf{c}} = \lambda_{\alpha_m,\beta(\mathbf{c}),\gamma(\mathbf{c}),\delta(\mathbf{c})}\cdot U_1(\mathcal{Q})_{(\alpha_m,i_0),\mathbf{c}}$
also satisfy a dependency $\sum_m c_m\, U_1(\mathcal{T})_{(\alpha_m,i_0),\mathbf{c}}=0$.

For generic $A$, the kernel of the $5\times(3n)^3$ submatrix of $U_1(\mathcal{Q})$ is $1$-dimensional.
The scaled dependency must be proportional to the unscaled one:
\[
c_m\,\lambda_{\alpha_m,\beta,\gamma,\delta} = \eta(\beta,\gamma,\delta)\cdot d_m
\qquad\text{for some function }\eta.
\]
For any two indices $m_1,m_2$ with $c_{m_1},d_{m_1}\neq 0$:
\[
\frac{\lambda_{\alpha_{m_1},\beta,\gamma,\delta}}{\lambda_{\alpha_{m_2},\beta,\gamma,\delta}}
=
\frac{c_{m_2}\,d_{m_1}}{c_{m_1}\,d_{m_2}}
= \text{constant (independent of }\beta,\gamma,\delta\text{)}.
\]
Since this holds for all pairs from $\{\alpha_1,\ldots,\alpha_5\}$ and all such $5$-tuples exhaust $[n]$,
we conclude: there exist $u\in(\R^*)^n$ and $m\in(\R^*)^{n\times n\times n}$ with
\begin{equation}\label{eq:mode1}
\lambda_{\alpha\beta\gamma\delta}=u_\alpha\,m_{\beta\gamma\delta}
\qquad\text{for all non-identical }(\alpha,\beta,\gamma,\delta).
\end{equation}

Applying the identical argument to $U_2(\mathcal{T})$:
\begin{equation}\label{eq:mode2}
\lambda_{\alpha\beta\gamma\delta}=v_\beta\,n_{\alpha\gamma\delta}.
\end{equation}
Combining \eqref{eq:mode1} and \eqref{eq:mode2}: fix $\alpha_0$ and write
$u_{\alpha_0}m_{\beta\gamma\delta}=v_\beta n_{\alpha_0\gamma\delta}$,
giving $m_{\beta\gamma\delta}=v_\beta\,t_{\gamma\delta}$ with $t_{\gamma\delta}:=n_{\alpha_0\gamma\delta}/u_{\alpha_0}$.
Hence $\lambda_{\alpha\beta\gamma\delta}=u_\alpha v_\beta\,t_{\gamma\delta}$.

Applying the argument to $U_3(\mathcal{T})$ and $U_4(\mathcal{T})$ forces
$t_{\gamma\delta}=w_\gamma x_\delta$. Therefore
\[
\lambda_{\alpha\beta\gamma\delta}=u_\alpha v_\beta w_\gamma x_\delta
\qquad\text{for all non-identical }(\alpha,\beta,\gamma,\delta),
\]
completing the proof.

%% ===================================================================
\section{Verification Notes}
%% ===================================================================

\begin{remark}[Steps verified]
\begin{enumerate}
\item \emph{Reshaping}: fixed bijection, no content change. \checkmark
\item \emph{Mode rank $\leq 4$}: determinant is multilinear $\Rightarrow$ each unfolding factors through $\R^4$. \checkmark
\item \emph{Diagonal blocks vanish}: pigeonhole ($4$ indices from $[3]$ $\Rightarrow$ repeated row $\Rightarrow$ det $= 0$). \checkmark
\item \emph{Forward}: separable diagonal scaling preserves rank. \checkmark
\item \emph{Diagonal rigidity lemma}: comparing $4$-subsets sharing $3$ elements forces all $d_i$ equal. Requires $M \geq 5$; here $M = (3n)^3 \gg 5$. \checkmark
\item \emph{Backward, proportionality}: $5$ vectors in $\R^4$ have unique (up to scalar) dependency for generic $A$; scaled dependency must be proportional $\Rightarrow$ ratio $\lambda_{\alpha_1}/\lambda_{\alpha_2}$ is constant in remaining indices. \checkmark
\item \emph{Iterative separation}: mode-1 $\Rightarrow$ $\lambda = u_\alpha m$; mode-2 $\Rightarrow$ $\lambda = v_\beta n$; combine $\Rightarrow$ $\lambda = u_\alpha v_\beta t$; modes 3,4 $\Rightarrow$ $t = w_\gamma x_\delta$. \checkmark
\item \emph{Degree bound}: $5 \times 5$ determinant = degree $5$, independent of $n$. \checkmark
\end{enumerate}
\end{remark}

\begin{remark}[Why $n \geq 5$]
We need $5$ distinct matrix indices to form the $5$-vector dependency argument. With $n \geq 5$, this is always possible.
\end{remark}

\begin{thebibliography}{9}
\bibitem{AbouzaidEtAl2026}
M.~Abouzaid et al.
\newblock First Proof.
\newblock \emph{arXiv:2602.05192}, February 2026.
\end{thebibliography}

\end{document}
