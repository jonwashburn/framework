\documentclass[11pt]{article}

% --- Preamble ---------------------------------------------------------------
\usepackage[margin=1in]{geometry}
\usepackage{microtype}
\usepackage{amsmath,amssymb,mathtools}
\usepackage{booktabs,longtable}
\usepackage{xcolor}
\usepackage{hyperref}
\usepackage{graphicx}

\hypersetup{
  colorlinks=true,
  linkcolor=blue,
  urlcolor=blue,
  citecolor=blue
}

% --- Notation --------------------------------------------------------------
\newcommand{\phiG}{\varphi}
\newcommand{\tauzero}{\tau_{0}}
\newcommand{\eps}{\varepsilon}

% --- Metadata --------------------------------------------------------------
\newcommand{\DocTitle}{Discrete Scale Invariance on a $\phiG$-Ladder}
\newcommand{\DocSubtitle}{Cross-Domain Coincidences Between Particle Masses and Biophysical Gate Times}
\newcommand{\DocVersion}{0.1}
\newcommand{\DocDate}{2025-12-18}

\title{\DocTitle\\\large \DocSubtitle}
\author{Reality Science Team (Draft)}
\date{\DocDate\ (\DocVersion)}

\begin{document}
\maketitle

\begin{abstract}
We investigate a cross-domain alignment consistent with \emph{Discrete Scale Invariance} (DSI): a proposed protein-folding ``molecular gate'' timescale near $65$--$70$ ps and the tau lepton are both associated (within the Recognition Science model) with rung $19$ on a $\phiG$-scaling ladder. We provide a \emph{claim-hygiene} separation between (A) \textbf{structural identities} provable within a model and (B) \textbf{empirical matches} requiring datasets, uncertainty, preregistered procedures, and multiple-comparisons controls. We define a preregisterable rung assignment rule on log-time/log-mass spaces, specify null models, and lay out falsifiable predictions including ``jamming'' experiments targeting the rung-$19$ band. This paper is designed to stand alone as an evidence-focused study: it does not require accepting any broader metaphysical interpretation.
\end{abstract}

\tableofcontents
\newpage

% ===========================================================================
\section{Claim Hygiene (Anti-Numerology)}

To keep this work publishable and falsifiable, we enforce a strict separation:

\begin{itemize}
  \item \textbf{Structural (model-level)}: identities that hold \emph{exactly} within a declared formal system (e.g., $\phiG$-power relations between \emph{structural} masses). These can be machine-verified.
  \item \textbf{Empirical (data-level)}: matches to measured quantities in the world (particle masses, relaxation times, vibrational periods). These require explicit datasets, uncertainties, preregistered procedures, and multiple-comparisons corrections.
\end{itemize}

This paper focuses on the empirical side while referencing structural identities as supporting context.

% ===========================================================================
\section{Background: Discrete Scale Invariance}

\subsection{DSI and log-periodicity}
In systems with continuous scale invariance, observables exhibit power laws. In DSI, invariance holds only at discrete scale factors (e.g., $\lambda$), leading to \emph{log-periodic} corrections \cite{sornette_1998,sornette_2002}.

\subsection{Why $\phiG$?}
We examine $\phiG = (1+\sqrt{5})/2 \approx 1.618$ as a candidate discrete scale factor. This is motivated by internal structure/closure arguments in the broader Recognition Science program; however, in this evidence paper, $\phiG$ is treated as a \emph{fixed hypothesis} and evaluated against data with preregistered methods.

% ===========================================================================
\section{The $\phiG$-Ladder and Rung Assignment}

\subsection{Time ladder}
We define a time ladder:
\[
\tau_n = \tauzero \, \phiG^n,\qquad n\in\mathbb{Z}.
\]
Given a time measurement $t>0$, define the rung assignment
\[
n^\star(t) = \mathrm{round}\!\left(\frac{\log(t/\tauzero)}{\log \phiG}\right),
\]
and define the log-space residual
\[
\eps(t) = \log(t/\tauzero) - n^\star(t)\log \phiG.
\]
A preregistered tolerance threshold can be defined via $|\eps(t)| \le \eps_{\max}$, or equivalently a relative error bound $|\exp(\eps)-1|$.

\subsection{Mass ladder (structural vs empirical)}
We distinguish:
\begin{itemize}
  \item \textbf{Structural mass ladder (model)}: a formula producing dimensionless or internal-unit masses $m_{\mathrm{struct}}$ with exact $\phiG$-power relations between generations.
  \item \textbf{Empirical masses (data)}: PDG lepton masses in MeV \cite{pdg_2024}.
\end{itemize}
Bridging structural and empirical masses generally requires a declared unit map and (possibly) a small correction term (``residue/transport''). This bridge must be preregistered before being tuned.

\section{The "Octave Map" (Rosetta Stone)}
A motivating hypothesis in the Recognition Science program is that certain \emph{anchor phenomena} (spectroscopic modes, gating limits, and mass scales) may cluster near shared rung indices across domains (``octaves''). This section records an \emph{Octave Map} as a compact hypothesis ledger: it is intended to be tested, and it includes a mix of (i) empirically measured quantities and (ii) protocol-defined targets or conjectural mappings.

\textbf{Important:} Table~\ref{tab:octave_map} is \emph{not} itself statistical evidence. Where an entry is not present in the preregistered scoring set, it is explicitly labeled as a \emph{target / hypothesis} rather than an observed match.

\begin{table}[h]
\centering
\caption{The Octave Map: Cross-Domain Alignment of Fundamental Resonances}
\label{tab:octave_map}
\begin{tabular}{@{}r p{0.28\linewidth} p{0.40\linewidth} p{0.22\linewidth}@{}}
\toprule
\textbf{Rung} ($n$) & \textbf{Physics (mass-side anchor)} & \textbf{Time/biophysics (time-side anchor)} & \textbf{Status / notes} \\
\midrule
2  & Electron mass (PDG; model labels rung 2) & Water HOH bend period $\approx 20$ fs (Table~\ref{tab:paper1_time_dataset}) & Observed (time); rung label on mass-side is structural/model \\
4  & --- & Water libration period $\approx 50$ fs (Table~\ref{tab:paper1_time_dataset}) & Observed (time) \\
13 & Muon mass (PDG; model labels rung 13) & Target: $\tau_{13}=\tau_0\phi^{{13}}\approx 3.82$ ps (primary measurement TBD) & Hypothesis target (not in scoring unless promoted) \\
19 & Tau mass (PDG; model labels rung 19) & Protocol target: ``molecular gate'' $\tau_{\mathrm{gate}}\approx 65$--$70$ ps (App.~\ref{app:prereg}; protocol file) & Protocol-defined hypothesis; excluded from scoring until measured \\
45 & --- & Target: $\tau_{45}\approx 18.6\,\mu$s (``coherence limit'') & Hypothesis target; not used for scoring in v0 \\
53 & --- & Exploratory: neural spike width $\sim$ms (highly variable process) & Exploratory only; not a fundamental clock \\
\bottomrule
\end{tabular}
\end{table}

\noindent\textbf{Note:} Mass-side rung indices are part of the structural model labeling; time-side rung indices are computed from $\tau_0$ and $\phiG$ (and thus inherit any uncertainty or provenance issues in $\tau_0$). Paper~1 treats the Octave Map as a hypothesis ledger and evaluates $\phiG$ against null models using the preregistered scoring set.

% ===========================================================================
\section{Datasets (Clocks vs Processes)}

\subsection{Particle physics}
We use PDG lepton masses for electron, muon, and tau \cite{pdg_2024}. We also include selected PDG particle \emph{lifetimes} as time-domain observables (e.g., $\tau_\mu$, $\tau_\tau$, and meson lifetimes) \cite{pdg_2024}. These are high-confidence measurements.

\subsection{Biophysical timescales}
This draft references several candidate timescales:
\begin{itemize}
  \item Water vibrational bands (OH stretch, HOH bend) and ultrafast frequency fluctuations \cite{larsen_woutersen_2004,imoto_xantheas_saito_2013}.
  \item Water libration dynamics (mid-IR / pump-probe studies) \cite{amir_gallot_hache_2004}.
  \item Water hydrogen-bond kinetics \cite{luzar_chandler_1996} and reorientation mechanisms \cite{laage_hynes_2006}.
  \item Bulk water dielectric relaxation \cite{kaatze_2015_dielectric_water}.
  \item Hydration/peptide dielectric modes on $\sim$10 ps and $\sim$100 ps timescales \cite{sasisanker_weingartner_2008}.
  \item Review-level hierarchy of protein internal-motion timescales (fs to ms) \cite{henzler_wildman_kern_2007} (used for context only; excluded from preregistered scoring unless promoted to a primary-measurement dataset).
  \item Primary folding kinetics time constants in the ns--$\mu$s regime from hydrogen-exchange/NMR and temperature-jump studies \cite{meisner_sosnick_2004,mayor_fersht_2000}.
  \item Action-potential durations (order ms; cell-type dependent) \cite{jana_2023_action_potential}.
\end{itemize}

\textbf{Important:} some items in the CSV are still \emph{candidates} (marked \texttt{include=false}) until tied to primary literature and/or internal datasets with uncertainty bounds. The plan requires a \emph{curated table} with citations and measurement methods before any significance claims are made.

\subsection{Curated time dataset (v0; auto-generated)}
Table~\ref{tab:paper1_time_dataset} is auto-generated from \texttt{docs/paper1\_times\_dataset.csv} by \texttt{docs/paper1\_rung\_assignment.py}. Only entries marked \texttt{include=true} are shown; excluded entries remain in the CSV but are not used for scoring until promoted with citations and uncertainty bounds.

\begin{center}
\label{tab:paper1_time_dataset}
% Auto-generated by paper1_rung_assignment.py
\begin{longtable}{@{}p{0.14\linewidth}p{0.18\linewidth}r r r r r p{0.16\linewidth}@{}}
\toprule
ID & Observable & $t$ (s) & $\sigma_t$ (s) & $n^\star$ & $\hat{\tau}_{n^\star}$ (s) & $|\epsilon|$ & Citation \\
\midrule
\endhead
water\_oh\_stretch\_period & water\_oh\_stretch\_period & 9.800e-15 & 1.000e-15 & 1 & 1.186e-14 & 1.908e-01 & \cite{imoto_xantheas_saito_2013} \\
water\_hoh\_bend\_period & water\_hoh\_bend\_period & 2.020e-14 & 2.000e-15 & 2 & 1.919e-14 & 5.128e-02 & \cite{larsen_woutersen_2004} \\
water\_libration\_period & water\_libration\_period & 5.000e-14 & 1.500e-14 & 4 & 5.024e-14 & 4.800e-03 & \cite{amir_gallot_hache_2004} \\
water\_hbond\_kinetics & water\_hbond\_kinetics\_time & 1.000e-12 & 3.000e-13 & 10 & 9.015e-13 & 1.037e-01 & \cite{luzar_chandler_1996} \\
water\_reorientation & water\_reorientation\_time & 2.500e-12 & 5.000e-13 & 12 & 2.360e-12 & 5.753e-02 & \cite{laage_hynes_2006} \\
water\_dielectric\_relaxation & water\_dielectric\_relaxation\_time & 8.200e-12 & 1.000e-12 & 15 & 9.998e-12 & 1.983e-01 & \cite{kaatze_2015_dielectric_water} \\
peptide\_dielectric\_fast\_mode & peptide\_dielectric\_fast\_mode & 1.000e-11 & 2.000e-12 & 15 & 9.998e-12 & 1.875e-04 & \cite{sasisanker_weingartner_2008} \\
peptide\_dielectric\_slow\_mode & peptide\_dielectric\_slow\_mode & 1.000e-10 & 3.000e-11 & 20 & 1.109e-10 & 1.033e-01 & \cite{sasisanker_weingartner_2008} \\
pdg\_tau\_lifetime & tau\_lepton\_lifetime & 2.903e-13 & 1.000e-15 & 8 & 3.444e-13 & 1.708e-01 & \cite{pdg_2024} \\
pdg\_muon\_lifetime & muon\_lifetime & 2.197e-06 & 2.000e-09 & 41 & 2.714e-06 & 2.113e-01 & \cite{pdg_2024} \\
pdg\_charged\_pion\_lifetime & pi\_plus\_lifetime & 2.603e-08 & 5.000e-11 & 31 & 2.207e-08 & 1.653e-01 & \cite{pdg_2024} \\
pdg\_kaon\_charged\_lifetime & K\_plus\_lifetime & 1.238e-08 & 5.000e-11 & 30 & 1.364e-08 & 9.674e-02 & \cite{pdg_2024} \\
pdg\_kaon\_short\_lifetime & K\_short\_lifetime & 8.950e-11 & 5.000e-13 & 20 & 1.109e-10 & 2.142e-01 & \cite{pdg_2024} \\
pdg\_kaon\_long\_lifetime & K\_long\_lifetime & 5.116e-08 & 2.000e-10 & 33 & 5.777e-08 & 1.215e-01 & \cite{pdg_2024} \\
\bottomrule
\end{longtable}

\end{center}

% ===========================================================================
\section{Results (Draft-Level; Pending Prereg + Curated Data)}

\subsection{Tau--Gate coincidence (structural claim + empirical hook)}
Within the Lean formalization of the Recognition Science structural model, the tau rung is set to $19$, and a ``molecular gate rung'' is also set to $19$ (this is a \emph{definition-level} equality in the current model artifact). This is a \textbf{structural identity}, not yet an empirical result.

The empirical question for this paper is: \emph{does an independently measured biophysical gate timescale near $\tau_{19}$ exist, with uncertainty, and does it robustly survive multiple-comparisons controls?}

\subsection{Lepton ratio sanity check (why correction terms matter)}
A naive comparison of PDG lepton ratios to $\phiG$ powers yields gaps at the few-percent level, e.g.
\[
\frac{m_\mu}{m_e}\approx 206.8 \quad \text{vs}\quad \phiG^{11}\approx 199.0,
\]
so any ``exact match'' claim must be supported by a preregistered correction term and error accounting.

\subsection{Lepton mass-ratio rungs (base-free; auto-generated)}
Because the rung assignment for masses depends on a declared base mass scale, we also report a \emph{base-free} diagnostic: rungs computed from \emph{ratios} $m_2/m_1$, i.e.\ the nearest integer $k$ such that $m_2/m_1 \approx \phiG^k$.
Table~\ref{tab:paper1_mass_ratios} is auto-generated from \texttt{docs/paper1\_particle\_masses.csv} by \texttt{docs/paper1\_mass\_ratios.py}.

\begin{center}
\label{tab:paper1_mass_ratios}
\IfFileExists{paper1_mass_ratios.tex}{
  % Auto-generated by paper1_mass_ratios.py
\begin{center}
\begin{tabular}{@{}lrrrrr@{}}
\toprule
Ratio & observed & $k$ (nearest) & $\phi^k$ & $|\epsilon|$ & rel. err. \\
\midrule
muon/electron & 206.768283 & 11 & 199.005025 & 3.827e-02 & 3.901\% \\
tau/muon & 16.817029 & 6 & 17.944272 & 6.488e-02 & 6.282\% \\
tau/electron & 3477.228280 & 17 & 3571.000280 & 2.661e-02 & 2.626\% \\
\bottomrule
\end{tabular}
\end{center}
\vspace{0.25em}
{\small Mass values from \cite{pdg_2024}.}

}{
  \fbox{\parbox{0.9\linewidth}{\small Missing \texttt{paper1\_mass\_ratios.tex}. Run \texttt{cd docs \&\& ./build\_resonance\_artifacts.sh} to regenerate.}}
}
\end{center}

% ===========================================================================
\section{Statistical Plan}

\subsection{Search space declaration}
The statistical meaning of ``coincidence'' depends critically on the declared search space:
\begin{itemize}
  \item Which domains are included (masses only? times only? both?)?
  \item Which candidate observables are allowed per domain?
  \item Which rung range $[n_{\min},n_{\max}]$ is considered?
  \item What tolerance $|\eps|\le \eps_{\max}$ defines a hit?
\end{itemize}
All four must be preregistered.

\subsection{Null models}
We will evaluate at least three null models:
\begin{enumerate}
  \item \textbf{Uniform log-space} over declared ranges.
  \item \textbf{Empirical priors} learned from curated distributions (e.g., known biochemical relaxation times).
  \item \textbf{Permutation/scramble tests} preserving within-domain clustering.
\end{enumerate}

\subsection{Multiple comparisons}
We will report Bonferroni and FDR-adjusted p-values.

\subsection{Multiple-comparisons adjustments across candidate $\lambda$ (auto-generated)}
If more than one candidate scale factor is considered, p-values must be corrected for that family. Table~\ref{tab:paper1_multiplicity} reports raw and adjusted p-values (Bonferroni and BH-FDR) across preregistered $\lambda$ candidates, under each null model. It is auto-generated by \texttt{docs/paper1\_multiplicity.py}.

\begin{center}
\label{tab:paper1_multiplicity}
\IfFileExists{paper1_multiplicity.tex}{
  % Auto-generated by paper1_multiplicity.py
\begin{center}
\begin{tabular}{@{}l l r r r r@{}}
\toprule
Null model & candidate & $\lambda$ & raw $p$ & Bonferroni & BH-FDR $q$ \\
\midrule
bootstrap\_jitter & 2 & 2.000000 & 0.568 & 1.000 & 0.599 \\
bootstrap\_jitter & e & 2.718282 & 0.474 & 1.000 & 0.599 \\
bootstrap\_jitter & phi & 1.618034 & 0.487 & 1.000 & 0.599 \\
bootstrap\_jitter & sqrt2 & 1.414214 & 0.599 & 1.000 & 0.599 \\
log\_normal & 2 & 2.000000 & 0.502 & 1.000 & 0.666 \\
log\_normal & e & 2.718282 & 0.154 & 0.618 & 0.618 \\
log\_normal & phi & 1.618034 & 0.318 & 1.000 & 0.636 \\
log\_normal & sqrt2 & 1.414214 & 0.666 & 1.000 & 0.666 \\
log\_uniform & 2 & 2.000000 & 0.616 & 1.000 & 0.616 \\
log\_uniform & e & 2.718282 & 0.235 & 0.942 & 0.616 \\
log\_uniform & phi & 1.618034 & 0.518 & 1.000 & 0.616 \\
log\_uniform & sqrt2 & 1.414214 & 0.590 & 1.000 & 0.616 \\
\bottomrule
\end{tabular}
\end{center}

}{
  \fbox{\parbox{0.9\linewidth}{\small Missing \texttt{paper1\_multiplicity.tex}. Run \texttt{cd docs \&\& ./build\_resonance\_artifacts.sh} to regenerate.}}
}
\end{center}

% ===========================================================================
\section{Scale-Factor Controls ($\\lambda$ Sweep) and Null Models (v0)}

\subsection{Why $\\lambda$ controls matter}
If a ladder model is evaluated without comparing alternative scale factors, it is difficult to know whether $\phiG$ is genuinely special or merely one of many plausible discrete scalings. We therefore include:
\begin{itemize}
  \item a fixed-candidate comparison ($\phiG$, $2$, $e$, $\sqrt{2}$),
  \item a best-on-grid baseline (look-elsewhere control),
  \item null-model Monte Carlo for a preregistered score.
\end{itemize}

\subsection{Important note on correlated rows (independence-group weighting)}
Even with a primary-measurement-only scoring set, the dataset can contain \emph{correlated clusters} (e.g., multiple water spectroscopic times extracted from closely related experiments, or multiple PDG lifetimes within a particle-family block). To avoid inflating effective sample size, the scoring function used in \texttt{docs/paper1\_analysis.py} weights each \texttt{independence\_group} to contribute total weight 1 (see Appendix~\ref{app:prereg}). This makes the $\lambda$ sweep less sensitive to how many rows are added within a correlated cluster.

\subsection{What the current v0 tables do (and do not) show}
Because the included dataset is still evolving, the tables in this section should be read as a \emph{pipeline demonstration}. The preregistered scoring set is defined by \texttt{docs/paper1\_prereg.json} (\texttt{search\_space.time\_ids}) and excludes review-level ``timescale-bin'' rows (kept separately as \texttt{exploratory\_time\_ids}). In particular:
\begin{itemize}
  \item If $\lambda=\phiG$ is not competitive against other candidates or against best-on-grid baselines on the preregistered dataset, that is \emph{evidence against} the $\phiG$-ladder hypothesis as stated.
  \item The ``best\_grid'' row is a look-elsewhere control: it will nearly always outperform fixed candidates because it is optimized.
\end{itemize}

\subsection{Auto-generated $\\lambda$ sweep table}
Table~\ref{tab:paper1_lambda_sweep} is auto-generated by \texttt{docs/paper1\_analysis.py} from the curated dataset.

\begin{center}
\label{tab:paper1_lambda_sweep}
% Auto-generated by paper1_analysis.py
\begin{center}
\begin{tabular}{@{}lrrrr@{}}
\toprule
Candidate & $\lambda$ & SSE $(\sum \epsilon^2)$ & hits ($|\epsilon|\le\epsilon_{\max}$) & note \\
\midrule
phi & 1.618034 & 1.329583e-01 & 2.417 & $\lambda=\phi$ \\
2 & 2.000000 & 3.004532e-01 & 2.000 &  \\
e & 2.718282 & 4.592663e-01 & 2.917 &  \\
sqrt2 & 1.414214 & 7.376119e-02 & 4.417 &  \\
best\_grid & 1.215608 & 1.096467e-02 & 7.000 & best on grid \\
\bottomrule
\end{tabular}
\end{center}

\end{center}

\begin{figure}[t]
  \centering
  \includegraphics[width=0.92\linewidth]{paper1_lambda_curve.pdf}
  \caption{Score as a function of scale factor $\lambda$ on a preregisterable grid (v0). Lower is better (log y-axis). The dashed line marks $\lambda=\phi$.}
  \label{fig:paper1_lambda_curve}
\end{figure}

\subsection{Auto-generated null-model table}
Table~\ref{tab:paper1_null_models} reports Monte Carlo p-values for the observed $\mathrm{SSE}_\phi$ under three null models (log-uniform, log-normal, bootstrap+jitter). These are \emph{v0 placeholders} until preregistration freezes the exact ranges, tolerance, and sample sizes.

\begin{center}
\label{tab:paper1_null_models}
% Auto-generated by paper1_analysis.py
\begin{center}
\begin{tabular}{@{}lrrrrr@{}}
\toprule
Null model & reps & $\mathrm{SSE}_{\phi}$ & p-value & null mean & [min,max] \\
\midrule
log\_uniform & 2000 & 1.330e-01 & 0.518 & 1.335e-01 & [4.254e-02,2.490e-01] \\
log\_normal & 2000 & 1.330e-01 & 0.325 & 1.512e-01 & [3.229e-02,2.871e-01] \\
bootstrap\_jitter & 2000 & 1.330e-01 & 0.486 & 1.350e-01 & [4.131e-02,2.593e-01] \\
\bottomrule
\end{tabular}
\end{center}

\end{center}

\begin{figure}[t]
  \centering
  \includegraphics[width=0.92\linewidth]{paper1_null_hist.pdf}
  \caption{Monte Carlo null distributions for $\mathrm{SSE}_\phi$ under three null models (v0). The dashed line marks the observed $\mathrm{SSE}_\phi$ from the included dataset rows.}
  \label{fig:paper1_null_hist}
\end{figure}

\subsection{Robustness sweeps (auto-generated)}
To satisfy the plan’s robustness requirement, we generate a preregistered sweep over small perturbations of $\tauzero$ and over multiple tolerance thresholds $\eps_{\max}$. Table~\ref{tab:paper1_robustness} is auto-generated from \texttt{docs/paper1\_prereg.json} by \texttt{docs/paper1\_robustness.py}.

\begin{center}
\label{tab:paper1_robustness}
\IfFileExists{paper1_robustness.tex}{
  % Auto-generated by paper1_robustness.py
\begin{center}
\begin{tabular}{@{}rrrrrrr@{}}
\toprule
$\tau_0$ mult. & $\eps_{\max}$ & $\mathrm{SSE}_{\phi}$ & hits$_\phi$ & best $\lambda$ & best SSE & best hits \\
\midrule
0.980 & 0.050 & 1.199e-01 & 0.833 & 1.216258 & 1.108e-02 & 5.333 \\
0.980 & 0.100 & 1.199e-01 & 2.917 & 1.216258 & 1.108e-02 & 7.000 \\
0.980 & 0.150 & 1.199e-01 & 4.167 & 1.216258 & 1.108e-02 & 7.000 \\
0.990 & 0.050 & 1.257e-01 & 0.833 & 1.215608 & 1.107e-02 & 6.167 \\
0.990 & 0.100 & 1.257e-01 & 2.917 & 1.215608 & 1.107e-02 & 7.000 \\
0.990 & 0.150 & 1.257e-01 & 4.167 & 1.215608 & 1.107e-02 & 7.000 \\
1.000 & 0.050 & 1.330e-01 & 0.833 & 1.215608 & 1.096e-02 & 6.167 \\
1.000 & 0.100 & 1.330e-01 & 2.417 & 1.215608 & 1.096e-02 & 7.000 \\
1.000 & 0.150 & 1.330e-01 & 4.167 & 1.215608 & 1.096e-02 & 7.000 \\
1.010 & 0.050 & 1.415e-01 & 2.167 & 1.214957 & 1.068e-02 & 5.917 \\
1.010 & 0.100 & 1.415e-01 & 3.167 & 1.214957 & 1.068e-02 & 7.000 \\
1.010 & 0.150 & 1.415e-01 & 4.167 & 1.214957 & 1.068e-02 & 7.000 \\
1.020 & 0.050 & 1.513e-01 & 2.167 & 1.214957 & 1.048e-02 & 6.167 \\
1.020 & 0.100 & 1.513e-01 & 3.167 & 1.214957 & 1.048e-02 & 7.000 \\
1.020 & 0.150 & 1.513e-01 & 4.417 & 1.214957 & 1.048e-02 & 7.000 \\
\bottomrule
\end{tabular}
\end{center}

}{
  \fbox{\parbox{0.9\linewidth}{\small Missing \texttt{paper1\_robustness.tex}. Run \texttt{cd docs \&\& ./build\_resonance\_artifacts.sh} to regenerate.}}
}
\end{center}

% ===========================================================================
\section{Predictions and Falsifiers (Pre-registration Targets)}

\subsection{Rung-19 ``jamming'' experiment}
If a rung-19 gate is real and functional, driving the system near the rung-19 band (and/or nearby beat frequencies between adjacent rungs) should measurably alter folding kinetics. A preregistered protocol must specify:
\begin{itemize}
  \item Frequency targets and bandwidth.
  \item Power/field constraints and safety gates.
  \item Metrics (folding time distribution, success rate, RMSD).
  \item Negative controls (nearby non-target frequencies; sham conditions).
\end{itemize}

\paragraph{Preregistered frequency targets (auto-generated).}
Table~\ref{tab:paper1_jamming_targets} lists the stimulus frequencies implied by preregistered rungs (targets) and off-rung controls. It is auto-generated from \texttt{docs/paper1\_prereg.json} by \texttt{docs/paper1\_jamming\_targets.py}.

\begin{center}
\label{tab:paper1_jamming_targets}
\IfFileExists{paper1_jamming_targets.tex}{
  % Auto-generated by paper1_jamming_targets.py
\begin{center}
\begin{tabular}{@{}lrrrllp{0.28\linewidth}@{}}
\toprule
kind & rung $n$ & $\tau_n$ (s) & $f_n$ (Hz) & $f_n$ & unit & note \\
\midrule
target & 19.0 & 6.853e-11 & 1.459e+10 & 14.593 & GHz & on-rung target \\
control & 18.5 & 5.387e-11 & 1.856e+10 & 18.562 & GHz & off-rung control (n-0.5) \\
control & 19.5 & 8.717e-11 & 1.147e+10 & 11.472 & GHz & off-rung control (n+0.5) \\
\addlinespace
target & 45.0 & 1.860e-05 & 5.376e+04 & 53.759 & kHz & on-rung target \\
control & 44.5 & 1.462e-05 & 6.838e+04 & 68.383 & kHz & off-rung control (n-0.5) \\
control & 45.5 & 2.366e-05 & 4.226e+04 & 42.263 & kHz & off-rung control (n+0.5) \\
\addlinespace
target & 53.0 & 8.739e-04 & 1.144e+03 & 1.144 & kHz & on-rung target \\
control & 52.5 & 6.870e-04 & 1.456e+03 & 1.456 & kHz & off-rung control (n-0.5) \\
control & 53.5 & 1.112e-03 & 8.996e+02 & 899.616 & Hz & off-rung control (n+0.5) \\
\bottomrule
\end{tabular}
\end{center}

}{
  \fbox{\parbox{0.9\linewidth}{\small Missing \texttt{paper1\_jamming\_targets.tex}. Run \texttt{cd docs \&\& ./build\_resonance\_artifacts.sh} to regenerate.}}
}
\end{center}

\subsection{Out-of-sample rung coincidences}
We will preregister 2--5 additional rung predictions derived \emph{before} checking the corresponding data.

\subsection{Prediction registry (auto-generated)}
To operationalize preregistration, we maintain a machine-readable prediction registry with explicit falsifiers. Table~\ref{tab:paper1_prediction_registry} is auto-generated from \texttt{docs/paper1\_prediction\_registry.csv} by \texttt{docs/paper1\_prediction\_registry.py} and includes only entries marked \texttt{include=true}.

\begin{center}
\label{tab:paper1_prediction_registry}
\IfFileExists{paper1_prediction_registry.tex}{
  % Auto-generated by paper1_prediction_registry.py\n\begin{longtable}{@{}p{0.14\linewidth}p{0.10\linewidth}r p{0.14\linewidth}p{0.10\linewidth}p{0.26\linewidth}p{0.22\linewidth}@{}}\n\toprule\nID & kind & rung & quantity & predicted & method (+ controls) & falsifier \\\n\midrule\n\endhead\njamming\_14\_6\_GHz\_r19 & prediction & 19 & jamming\_frequency & 14.6 GHz & Apply controlled RF/microwave stimulation near 14.6 GHz (rung 19) during folding and measure kinetics + success metrics (fold time distribution, native-contact satisfaction, RMSD). Predefine dose/power accounting, temperature monitoring, and blinding. \newline controls: Sham stimulation; off-rung controls n-0.5 and n+0.5 from Table~\ref{tab:paper1\_jamming\_targets}; thermal-control condition with matched bulk heating; randomized-frequency control within declared band excluding targets. & Primary falsifier: no statistically significant difference vs sham and off-rung controls under preregistered effect-size threshold (e.g. >=10% shift in median folding time or >=0.2 SD change) with multiple-comparisons correction; or observed effects fully explained by measured temperature rise / denaturation. \\\nbiophase\_tau\_gate\_protocol & protocol & 19 & tau\_gate\_extraction & 65 ps & Time-resolved IR spectroscopy near 724 cm$^{-1}$; define X(t) from a preregistered band window and extract tau\_gate as the 1/e decay of the autocorrelation envelope; see docs/paper1\_biophase\_gate\_protocol.md. \newline controls: Off-band windows; time-shuffle control; sham/thermal control for any stimulation condition. & Falsifier: no stable tau\_gate fit above instrument-response floor, or tau\_gate indistinguishable from off-band controls, or non-reproducible across preregistered replications/analysts. \\\n\bottomrule\n\end{longtable}\n
}{
  \fbox{\parbox{0.9\linewidth}{\small Missing \texttt{paper1\_prediction\_registry.tex}. Run \texttt{cd docs \&\& ./build\_resonance\_artifacts.sh} to regenerate.}}
}
\end{center}

\subsection{Operational definition for the BIOPHASE ``molecular gate''}
\label{sec:biophase_gate_protocol}
The \(\sim 65\)–\(70\,\mathrm{ps}\) ``molecular gate'' claim is only meaningful if attached to an explicit, preregisterable measurement definition and negative controls. We therefore treat it as a protocol-bound hypothesis until a frozen dataset is attached.

\begin{itemize}
  \item Protocol (internal): \texttt{docs/paper1\_biophase\_gate\_protocol.md}.
  \item Internal reference for the BIOPHASE spec (including \(\tau_{\mathrm{gate}}\approx 65\,\mathrm{ps}\)): \cite{washburn_light_consciousness_theorem_2025}.
\end{itemize}

% ===========================================================================
\section{Discussion}

\subsection{What would be impressive}
\begin{itemize}
  \item Prospective predictions of new rung coincidences, validated independently.
  \item A reproducible statistical pipeline showing significance under conservative nulls.
  \item A positive ``jamming'' result with clean negative controls.
\end{itemize}

\subsection{What would falsify}
\begin{itemize}
  \item Failure to predict new coincidences out-of-sample.
  \item Sensitivity that collapses under small changes in $\tauzero$ or tolerance.
  \item Experimental results indistinguishable from controls under preregistered analysis.
\end{itemize}

\subsection{Current status (v0 prereg scoring set)}
On the current preregistered scoring set (Table~\ref{tab:paper1_time_dataset}), the auto-generated tables show:
\begin{itemize}
  \item $\mathrm{SSE}_{\phi}\approx 1.33\times 10^{-1}$ for $\lambda=\phiG$ (Table~\ref{tab:paper1_lambda_sweep}).
  \item Under three null models, $\lambda=\phiG$ is \emph{not} statistically distinguishable from chance ($p\approx 0.33$--$0.52$; Table~\ref{tab:paper1_null_models}).
  \item A best-on-grid $\lambda\approx 1.216$ achieves substantially lower SSE (look-elsewhere control); $\sqrt{2}$ also outperforms $\phiG$ on this metric (Table~\ref{tab:paper1_lambda_sweep}).
\end{itemize}
\textbf{Interpretation.} The present evidence is \emph{null/negative} for $\phiG$ being special under this scoring function and dataset. This is the correct outcome to report if the hypothesis is not supported.

% ===========================================================================
\section{Conclusion}
This draft formalizes an evidence-first program for evaluating a $\phiG$-ladder DSI hypothesis across domains with preregistration, explicit search spaces, and conservative null models. On the current preregistered scoring set, $\lambda=\phiG$ does not outperform null models and is not competitive with other candidate scale factors.

The Octave Map (Table~\ref{tab:octave_map}) is therefore best treated as a \emph{hypothesis registry}: it identifies specific cross-domain targets (notably the rung-19 ``molecular gate'') that require primary measurements with uncertainty bounds and negative controls before they can be treated as empirical evidence.

The central next steps are:
\begin{enumerate}
  \item \textbf{Measure or falsify} the proposed rung-19 gate using the preregistered operational protocol (Appendix~\ref{app:prereg} and the linked protocol file).
  \item \textbf{Retain honesty}: if $\phiG$ remains indistinguishable from chance under preregistered updates, report that as evidence against the hypothesis.
  \item \textbf{Run definitive tests}: execute preregistered ``jamming'' targets with rigorous negative controls.
\end{enumerate}

\appendix
\section{Preregistration Template (to be frozen before promotion to ``evidence'')}
\label{app:prereg}

This appendix is a \emph{template}. Before any ``significance'' claims, the following items must be frozen (timestamped) and made public (or at minimum committed in-repo with an immutable hash).

\subsection{Constants and rung assignment}
\begin{itemize}
  \item Base tick: $\tauzero =$ \texttt{7.33e-15 s}. Provenance: (to be filled) \texttt{[source + derivation]}. Operationally, the artifact scripts read this (and other preregisterable settings) from \texttt{docs/paper1\_prereg.json}.
  \item Ladder: $\tau_n = \tauzero \phiG^n$, $n\in\mathbb{Z}$.
  \item Rung assignment: $n^\star(t) = \mathrm{round}\!\left(\log(t/\tauzero)/\log\phiG\right)$.
  \item Residual: $\eps(t) = \log(t/\tauzero) - n^\star(t)\log\phiG$.
\end{itemize}

\subsection{Dataset inclusion rules}
\begin{itemize}
  \item Inclusion list: the exact set of candidate observables (rows) permitted for analysis, with citations and uncertainty bounds.
  \item Machine-readable freeze: the analysis search space is frozen by explicit IDs in \texttt{docs/paper1\_prereg.json} under \texttt{search\_space.time\_ids} and \texttt{search\_space.mass\_ids}.
  \item Independence grouping: the exact meaning of \texttt{independence\_group} (what correlations are assumed within a group).
  \item Exclusion policy: objective criteria for \texttt{include=false} (e.g., missing primary citation, missing operational definition, unclear uncertainty).
\end{itemize}

\subsection{Score function (as implemented in \texttt{docs/paper1\_analysis.py})}
We preregister the primary score as weighted SSE:
\[
\mathrm{SSE}_\lambda = \sum_i w_i \, \eps_\lambda(t_i)^2,
\]
where weights are defined so that each independence group contributes total weight $1$ (i.e., $w_i = 1/|G(i)|$).

We preregister a secondary ``hit'' score:
\[
\mathrm{hits}_\lambda = \sum_i w_i \, \mathbf{1}\{|\eps_\lambda(t_i)| \le \eps_{\max}\}.
\]

\subsection{Tolerance and search space}
\begin{itemize}
  \item Residual tolerance: $\eps_{\max}=$ \texttt{0.10} (log-space), or a declared alternative (recorded in \texttt{docs/paper1\_prereg.json}).
  \item Rung range: declare $[n_{\min}, n_{\max}]$ or equivalent time-range bounds.
  \item Scale-factor candidates: fixed set (e.g., $\phiG$, $2$, $e$, $\sqrt{2}$).
  \item Look-elsewhere control: if a grid search over $\lambda$ is used, preregister \texttt{LAMBDA\_GRID\_MIN}, \texttt{LAMBDA\_GRID\_MAX}, \texttt{LAMBDA\_GRID\_N}.
\end{itemize}

\subsection{Null models and Monte Carlo settings}
\begin{itemize}
  \item Null models: log-uniform, log-normal, bootstrap+jitter (group-preserving).
  \item Log-uniform range: use preregistered time bounds \texttt{null\_ranges.time\_seconds.\{min,max\}} from \texttt{docs/paper1\_prereg.json} (do not silently use sample min/max).
  \item Monte Carlo reps and seed: \texttt{MC\_REPS=2000}, \texttt{MC\_SEED=1337} (or updated values, but frozen in \texttt{docs/paper1\_prereg.json}).
  \item Primary p-value: $p=\Pr(\mathrm{SSE}_{\phi}\_\mathrm{null}\le \mathrm{SSE}_{\phi}\_\mathrm{obs})$ (lower is better).
\end{itemize}

\subsection{Multiple comparisons}
If multiple datasets, domains, rung-ranges, or alternative calibrations are explored, we preregister the correction (Bonferroni and/or FDR) and the full family of hypotheses.

\section{Reproducibility (artifact-only build loop)}
\label{app:repro}

To regenerate tables and figures without compiling the full PDFs:

\begin{verbatim}
cd docs
./build_resonance_artifacts.sh
\end{verbatim}

This regenerates:
\begin{itemize}
  \item \texttt{paper1\_rung\_results.tex} from \texttt{paper1\_times\_dataset.csv}
  \item \texttt{paper1\_full\_rung\_table.tex} from \texttt{paper1\_prereg.json}
  \item \texttt{paper1\_mass\_ratios.tex} from \texttt{paper1\_particle\_masses.csv}
  \item \texttt{paper1\_robustness.tex} from \texttt{paper1\_prereg.json} and \texttt{paper1\_times\_dataset.csv}
  \item \texttt{paper1\_jamming\_targets.tex} from \texttt{paper1\_prereg.json}
  \item \texttt{paper1\_multiplicity.tex} from \texttt{paper1\_prereg.json} and \texttt{paper1\_times\_dataset.csv}
  \item \texttt{paper1\_prediction\_registry.tex} from \texttt{paper1\_prediction\_registry.csv}
  \item \texttt{paper1\_lambda\_sweep.tex}, \texttt{paper1\_null\_models.tex} from \texttt{paper1\_analysis.py}
  \item \texttt{paper1\_lambda\_curve.pdf}, \texttt{paper1\_null\_hist.pdf} from \texttt{paper1\_plots.py}
\end{itemize}

\section{Full rung table (Supplement)}
\label{app:rungs}

For completeness and to make all implied target scales explicit, we include a full rung table over a preregistered range. Table~\ref{tab:paper1_full_rung_table} is auto-generated from \texttt{docs/paper1\_prereg.json} by \texttt{docs/paper1\_full\_rung\_table.py}.

\begin{center}
\label{tab:paper1_full_rung_table}
\IfFileExists{paper1_full_rung_table.tex}{
  % Auto-generated by paper1_full_rung_table.py
\begin{longtable}{@{}r r l r l@{}}
\toprule
$n$ & $\tau_n$ (s) & $\tau_n$ (display) & $f_n$ (Hz) & $f_n$ (display) \\
\midrule
\endhead
0 & 7.330e-15 & 7.330\,fs & 1.364e+14 & 136.426\,THz \\
1 & 1.186e-14 & 11.860\,fs & 8.432e+13 & 84.316\,THz \\
2 & 1.919e-14 & 19.190\,fs & 5.211e+13 & 52.110\,THz \\
3 & 3.105e-14 & 31.050\,fs & 3.221e+13 & 32.206\,THz \\
4 & 5.024e-14 & 50.241\,fs & 1.990e+13 & 19.904\,THz \\
5 & 8.129e-14 & 81.291\,fs & 1.230e+13 & 12.301\,THz \\
6 & 1.315e-13 & 131.532\,fs & 7.603e+12 & 7.603\,THz \\
7 & 2.128e-13 & 212.822\,fs & 4.699e+12 & 4.699\,THz \\
8 & 3.444e-13 & 344.354\,fs & 2.904e+12 & 2.904\,THz \\
9 & 5.572e-13 & 557.176\,fs & 1.795e+12 & 1.795\,THz \\
10 & 9.015e-13 & 901.530\,fs & 1.109e+12 & 1.109\,THz \\
11 & 1.459e-12 & 1.459\,ps & 6.855e+11 & 685.539\,GHz \\
12 & 2.360e-12 & 2.360\,ps & 4.237e+11 & 423.686\,GHz \\
13 & 3.819e-12 & 3.819\,ps & 2.619e+11 & 261.852\,GHz \\
14 & 6.179e-12 & 6.179\,ps & 1.618e+11 & 161.834\,GHz \\
15 & 9.998e-12 & 9.998\,ps & 1.000e+11 & 100.019\,GHz \\
16 & 1.618e-11 & 16.177\,ps & 6.181e+10 & 61.815\,GHz \\
17 & 2.618e-11 & 26.175\,ps & 3.820e+10 & 38.204\,GHz \\
18 & 4.235e-11 & 42.353\,ps & 2.361e+10 & 23.611\,GHz \\
19 & 6.853e-11 & 68.528\,ps & 1.459e+10 & 14.593\,GHz \\
20 & 1.109e-10 & 110.881\,ps & 9.019e+09 & 9.019\,GHz \\
21 & 1.794e-10 & 179.409\,ps & 5.574e+09 & 5.574\,GHz \\
22 & 2.903e-10 & 290.290\,ps & 3.445e+09 & 3.445\,GHz \\
23 & 4.697e-10 & 469.699\,ps & 2.129e+09 & 2.129\,GHz \\
24 & 7.600e-10 & 759.989\,ps & 1.316e+09 & 1.316\,GHz \\
25 & 1.230e-09 & 1.230\,ns & 8.132e+08 & 813.214\,MHz \\
26 & 1.990e-09 & 1.990\,ns & 5.026e+08 & 502.594\,MHz \\
27 & 3.219e-09 & 3.219\,ns & 3.106e+08 & 310.620\,MHz \\
28 & 5.209e-09 & 5.209\,ns & 1.920e+08 & 191.974\,MHz \\
29 & 8.428e-09 & 8.428\,ns & 1.186e+08 & 118.646\,MHz \\
30 & 1.364e-08 & 13.637\,ns & 7.333e+07 & 73.327\,MHz \\
31 & 2.207e-08 & 22.066\,ns & 4.532e+07 & 45.319\,MHz \\
32 & 3.570e-08 & 35.703\,ns & 2.801e+07 & 28.009\,MHz \\
33 & 5.777e-08 & 57.769\,ns & 1.731e+07 & 17.310\,MHz \\
34 & 9.347e-08 & 93.472\,ns & 1.070e+07 & 10.698\,MHz \\
35 & 1.512e-07 & 151.242\,ns & 6.612e+06 & 6.612\,MHz \\
36 & 2.447e-07 & 244.714\,ns & 4.086e+06 & 4.086\,MHz \\
37 & 3.960e-07 & 395.956\,ns & 2.526e+06 & 2.526\,MHz \\
38 & 6.407e-07 & 640.670\,ns & 1.561e+06 & 1.561\,MHz \\
39 & 1.037e-06 & 1.037\,us & 9.647e+05 & 964.668\,kHz \\
40 & 1.677e-06 & 1.677\,us & 5.962e+05 & 596.198\,kHz \\
41 & 2.714e-06 & 2.714\,us & 3.685e+05 & 368.471\,kHz \\
42 & 4.391e-06 & 4.391\,us & 2.277e+05 & 227.727\,kHz \\
43 & 7.105e-06 & 7.105\,us & 1.407e+05 & 140.743\,kHz \\
44 & 1.150e-05 & 11.496\,us & 8.698e+04 & 86.984\,kHz \\
45 & 1.860e-05 & 18.601\,us & 5.376e+04 & 53.759\,kHz \\
46 & 3.010e-05 & 30.098\,us & 3.322e+04 & 33.225\,kHz \\
47 & 4.870e-05 & 48.699\,us & 2.053e+04 & 20.534\,kHz \\
48 & 7.880e-05 & 78.797\,us & 1.269e+04 & 12.691\,kHz \\
49 & 1.275e-04 & 127.497\,us & 7.843e+03 & 7.843\,kHz \\
50 & 2.063e-04 & 206.294\,us & 4.847e+03 & 4.847\,kHz \\
51 & 3.338e-04 & 333.790\,us & 2.996e+03 & 2.996\,kHz \\
52 & 5.401e-04 & 540.084\,us & 1.852e+03 & 1.852\,kHz \\
53 & 8.739e-04 & 873.874\,us & 1.144e+03 & 1.144\,kHz \\
54 & 1.414e-03 & 1.414\,ms & 7.072e+02 & 707.235\,Hz \\
55 & 2.288e-03 & 2.288\,ms & 4.371e+02 & 437.095\,Hz \\
56 & 3.702e-03 & 3.702\,ms & 2.701e+02 & 270.140\,Hz \\
57 & 5.990e-03 & 5.990\,ms & 1.670e+02 & 166.955\,Hz \\
58 & 9.691e-03 & 9.691\,ms & 1.032e+02 & 103.184\,Hz \\
59 & 1.568e-02 & 15.681\,ms & 6.377e+01 & 63.771\,Hz \\
60 & 2.537e-02 & 25.372\,ms & 3.941e+01 & 39.413\,Hz \\
\bottomrule
\end{longtable}

}{
  \fbox{\parbox{0.9\linewidth}{\small Missing \texttt{paper1\_full\_rung\_table.tex}. Run \texttt{cd docs \&\& ./build\_resonance\_artifacts.sh} to regenerate.}}
}
\end{center}

\bibliographystyle{unsrt}
\bibliography{RESONANCE_PAPERS}

\end{document}


