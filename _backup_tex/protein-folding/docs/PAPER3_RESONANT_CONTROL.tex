\documentclass[11pt]{article}

% --- Preamble ---------------------------------------------------------------
\usepackage[margin=1in]{geometry}
\usepackage{microtype}
\usepackage{amsmath,amssymb,mathtools}
\usepackage{booktabs,longtable}
\usepackage{xcolor}
\usepackage{hyperref}
\usepackage{graphicx}

\hypersetup{
  colorlinks=true,
  linkcolor=blue,
  urlcolor=blue,
  citecolor=blue
}

% --- Notation ---------------------------------------------------------------
\newcommand{\J}{\mathcal{J}}
\newcommand{\phiG}{\varphi}

% --- Metadata --------------------------------------------------------------
\newcommand{\DocTitle}{Resonant Control}
\newcommand{\DocSubtitle}{Real-Time Dissonance Feedback as a General Optimization Primitive}
\newcommand{\DocVersion}{0.1}
\newcommand{\DocDate}{2025-12-18}

\title{\DocTitle\\\large \DocSubtitle}
\author{Reality Science Team (Draft)}
\date{\DocDate\ (\DocVersion)}

\begin{document}
\maketitle

\begin{abstract}
We introduce \emph{Resonant Control}, an engineering pattern for optimization in which a global objective (or constraint violation field) is mapped to an \emph{auditory} signal, and the solver is biased to ``tune'' toward consonance. We motivate the approach using protein folding as a canonical rugged, partially observed search problem and define a concrete sonification protocol that maps structure and constraint signals to pitch, detuning, and roughness. We present the \emph{Marco Polo} algorithm: (Marco) attribute dissonance to the worst local contributor; (Polo) apply a targeted perturbation to reduce that contributor under a fixed compute budget. This draft is intentionally claim-hygienic: it specifies protocol, ablations, and falsifiers, but does not assert performance improvements without logged benchmark runs. The intended contribution is a reproducible control primitive and evaluation harness that can be tested on proteins, constraint satisfaction, and learning systems.
\end{abstract}

\tableofcontents
\newpage

% ===========================================================================
\section{Reader Contract (Engineering Framing)}

This paper stands or falls on measurable improvements under controlled conditions:
\begin{itemize}
  \item \textbf{No mysticism}: audio is treated as an engineered observation channel for a compressed global error signal.
  \item \textbf{No cherry-picking}: all claims of ``speedup'' require preregistered metrics, seeds, and compute budgets.
  \item \textbf{Safety boundary}: no biological irradiation/``jamming'' claims are made here; those belong to Paper~1’s preregistered experimental registry.
\end{itemize}

% ===========================================================================
\section{Background}

\subsection{Optimization on rugged landscapes}
Many scientific search problems exhibit high-dimensional rugged landscapes with abundant local minima (protein folding is a canonical example). Standard methods (simulated annealing, MCMC variants, heuristic local search) frequently struggle when the objective provides only weak global guidance.

\subsection{Sonification: from visualization to control}
Sonification has a long history as a \emph{visualization} aid \cite{sonification_handbook_2011}. Resonant Control treats sonification as a \emph{control interface}: the audio stream is not only human-interpretable but also provides a scalar/attribution signal that can bias an algorithmic optimizer.

\subsection{Consonance, roughness, and critical bands}
We adopt a psychoacoustic notion of consonance in which near-frequency components within critical bandwidths generate roughness/dissonance \cite{plomp_levelt_1965}. In practice, a computable proxy for roughness is used as an objective-shaped signal.

% ===========================================================================
\section{System Overview}

\subsection{Architecture (dataflow)}
At each optimization step:
\[
  \text{state} \;\to\; \text{features} \;\to\; \text{audio parameters} \;\to\; \text{roughness metric} \;\to\; \text{bias/selection}.
\]
The design goal is \emph{replayability}: given a logged trajectory and mapping version, the audio and roughness metric are regenerated exactly.

\subsection{Determinism and logging}
All evaluations must log:
\begin{itemize}
  \item RNG seeds and acceptance decisions,
  \item per-step objective terms (e.g., strain/contact components),
  \item the derived audio event stream (MIDI-like), and
  \item the derived roughness metric over time.
\end{itemize}

% ===========================================================================
\section{Sonification Protocol (Bio-Audio Mapping)}

This section specifies a mapping that is \emph{ablation-ready}: each component can be toggled independently for controlled comparisons.

\subsection{Inputs}
For a protein folding state \(S\), define:
\begin{itemize}
  \item backbone angles \((\phi_i,\psi_i)\) per residue \(i\),
  \item a constraint/strain decomposition \(\J(S)=\sum_k \J_k(S)\),
  \item optional secondary-structure labels (helix/sheet/coil).
\end{itemize}

\subsection{Pitch mapping}
We require a deterministic mapping from geometry to frequencies. Define a wrapped angle feature
\[
  \theta_i \;:=\; \mathrm{wrap}(\phi_i+\psi_i)\in[-\pi,\pi),
  \qquad
  u_i \;:=\; \frac{\theta_i+\pi}{2\pi}\in[0,1).
\]
Fix an integer rung range \([r_{\min},r_{\max}]\subset\mathbb{Z}\) and define the pitch rung:
\[
  r_i \;:=\; \mathrm{round}\!\bigl(r_{\min} + u_i\,(r_{\max}-r_{\min})\bigr)\in\mathbb{Z}.
\]
We support two \emph{ablation-ready} pitch modes:
\begin{itemize}
  \item \textbf{ET12 (control)}: \(f_i = f_0\cdot 2^{r_i/12}\) (equal temperament).
  \item \textbf{$\phi$-rung (hypothesis)}: \(f_i = f_0\cdot \phiG^{r_i}\) (geometric ladder).
\end{itemize}
All parameters (including \(f_0,r_{\min},r_{\max}\) and mode) must be frozen in a machine-readable mapping artifact before reporting results; we use \texttt{docs/paper3\_sonification\_mapping\_v0.json} as the v0 spec.

\subsection{Detuning mapping}
Map constraint violation to detuning:
\[
  \Delta f_i \;=\; g\big(\text{local violation at } i\big),
\]
where \(g\) is monotone and capped to prevent pathological audio.

\paragraph{Concrete v0 choice (cents, bounded).}
Let \(s_i\ge 0\) be a \emph{local strain proxy} (e.g., a per-residue attribution of \(\J\)). Define a detune value in cents:
\[
  \Delta c_i \;:=\; c_{\max}\,\tanh\!\Bigl(\frac{s_i}{s_0}\Bigr),
\]
and apply it multiplicatively:
\[
  f_i' \;=\; f_i\cdot 2^{\Delta c_i/1200}.
\]
The purpose of detuning is \emph{not} to be musically perfect; it is to create a smoothly varying dissonance signal proportional to constraint violation while remaining bounded.

\subsection{Timbre and spatialization}
Use timbre to encode coarse structure class and stereo position to encode residue index (left-to-right).

\paragraph{Polyphony cap (required).}
To avoid \(\mathcal{O}(N^2)\) auditory clutter for long sequences, we cap polyphony at \(K\) notes by selecting the top-\(K\) residues under a declared selection rule (e.g., top-\(K\) by \(s_i\)). The cap \(K\) and selection rule are part of the frozen mapping artifact.

% ===========================================================================
\section{Consonance as Objective}

\subsection{Roughness metric}
Define a computable roughness proxy \(R(\text{audio}(S))\) inspired by critical-band interference \cite{plomp_levelt_1965} (and common pairwise roughness approximations \cite{sethares_1993}). The core hypothesis is engineering-only:
\begin{quote}
If the audio mapping preserves enough structure, then \(R\) can serve as a \emph{global} error proxy correlated with folding quality (e.g., RMSD/contact satisfaction).
\end{quote}

\paragraph{Pairwise roughness proxy (v0).}
For a set of active notes \(\{(f_i',a_i)\}_{i=1}^K\) (frequency and amplitude), define
\[
  R \;:=\; \sum_{1\le i<j\le K} a_i a_j\,\Bigl(e^{-\alpha x_{ij}} - e^{-\beta x_{ij}}\Bigr),
  \qquad
  x_{ij} := \frac{|f_i'-f_j'|}{B(\tfrac{f_i'+f_j'}{2})},
\]
where \(B(\cdot)\) is a critical-bandwidth proxy and \(\alpha,\beta>0\) are fixed constants. This yields a bounded, differentiable ``roughness'' score that increases when frequency components crowd within critical bands.

\subsection{Empirical validation task}
Given logged trajectories, test correlation between \(R\) and conventional folding metrics (RMSD/contact satisfaction) under preregistered evaluation.

% ===========================================================================
\section{The Marco Polo Algorithm}

\subsection{Marco: dissonance attribution}
Compute a per-residue contribution \(r_i\) to overall roughness \(R\). Under a pairwise roughness proxy, a natural attribution is:
\[
  r_i \;:=\; \sum_{j\ne i} a_i a_j\,\Bigl(e^{-\alpha x_{ij}} - e^{-\beta x_{ij}}\Bigr),
\]
so that \(R = \tfrac12\sum_i r_i\). Any alternative attribution (e.g., via ablations of individual notes or gradient-free sensitivity) must be declared and fixed for the benchmark suite.

\subsection{Polo: targeted perturbation}
Select \(i^\star=\arg\max_i r_i\) and apply a targeted perturbation (e.g., proposal distribution biased toward residue \(i^\star\)), subject to the same acceptance mechanism and compute budget as baseline.

\subsection{Baselines and ablations (required)}
\begin{itemize}
  \item baseline optimizer (no audio),
  \item audio mapping only (no attribution),
  \item attribution only (no audio shaping),
  \item audio + Marco Polo,
  \item anti-Marco control: random residue targeting with matched compute.
\end{itemize}

% ===========================================================================
\section{Experimental Design (Benchmarks and Fairness)}

\subsection{Benchmarks}
The v2 plan targets a small benchmark suite (e.g., Trp-cage, Villin HP35, BBA5) with sufficient seeds (e.g., \(20\) per condition) and a fixed compute budget.

\subsection{Primary endpoint}
Primary endpoint (preregistered): median time-to-native at a declared RMSD threshold under fixed budget; report effect sizes and confidence intervals.

\subsection{Failure conditions}
Resonant Control is falsified (as an improvement) if it fails to outperform baselines under preregistered budgets and ablations, or if any apparent gain disappears under anti-Marco controls.

% ===========================================================================
\section{Reproducibility Package}

This draft requires a reproducibility bundle before any performance claim:
\begin{itemize}
  \item code version + mapping version (hash),
  \item benchmark manifest and seeds,
  \item logs sufficient to regenerate audio and plots,
  \item one-command reproduction instructions.
\end{itemize}

\appendix
\section{Supplement: Mapping Specification Checklist}
To make the mapping independently reproducible, freeze:
\begin{itemize}
  \item angle-to-pitch quantization rule,
  \item violation-to-detuning function \(g\) and caps,
  \item roughness metric definition,
  \item MIDI schema + synthesizer/rendering parameters (sample rate, normalization).
\end{itemize}

\bibliographystyle{unsrt}
\bibliography{RESONANCE_PAPERS}

\end{document}



