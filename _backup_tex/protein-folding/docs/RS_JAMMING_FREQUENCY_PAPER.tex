\documentclass[twocolumn,10pt,a4paper]{article}

% Packages (keep minimal for broad TeX compatibility)
\usepackage[utf8]{inputenc}
\usepackage[T1]{fontenc}
\usepackage{amsmath, amssymb, amsfonts}
\usepackage{graphicx}
\usepackage{hyperref}
\usepackage{booktabs}
\usepackage{geometry}
\usepackage{microtype}

% Manual definitions for compatibility (avoid siunitx dependency)
\newcommand{\angstrom}{\text{\normalfont\AA}}
\newcommand{\SI}[2]{#1\,\text{#2}}
\newcommand{\code}[1]{\texttt{#1}}

% Geometry settings
\geometry{top=2cm, bottom=2cm, left=1.5cm, right=1.5cm}

% ---------------------------------------------------------------------------
% Title and metadata
% ---------------------------------------------------------------------------
\title{\textbf{A Testable Prediction: \SI{14.653}{GHz} Microwave ``Jamming'' of Protein Folding}\\
\textit{A falsifiable dynamics experiment derived from RS Bio-Clocking}}

\author{
Jonathan Washburn\\
Recognition Science Research Institute\\
Austin, Texas
}

\date{\today}

\begin{document}
\maketitle

\begin{abstract}
Modern protein structure predictors achieve extraordinary accuracy, but mechanistic questions remain open: what physical process sets the \emph{timescale} of folding and how does the system traverse the global topology without exhaustive search? Recognition Science (RS) proposes that folding is not purely continuous diffusion on an energy landscape, but a discrete, clocked process implemented by a ``hydration gearbox'' that frequency-divides an atomic tick into biological timescales via a golden-ratio ladder.

From RS Bio-Clocking, biological timescales follow \(\tau(N)=\tau_0 \phi^{N}\), where \(\phi\) is the golden ratio and \(\tau_0\) is the RS ``atomic tick'' \cite{Washburn2026RSThoryBioClocking}. RS identifies a ``molecular gate'' relevant to folding at rung \(N=19\), yielding \(\tau_{19} \approx \SI{68.248}{ps}\) and a corresponding frequency \(f_{\mathrm{jam}} = 1/\tau_{19} \approx \SI{14.653}{GHz}\) (E41 calculator \cite{Washburn2026E41JammingCalc}). The central prediction is that irradiation near \(f_{\mathrm{jam}}\) will induce ``clock slip'' in the hydration gearbox and measurably reduce folding rate or destabilize the folded state, \emph{beyond what is explained by bulk heating}.

This paper derives the \SI{14.653}{GHz} target, enumerates harmonics and off-rung controls, and specifies a falsifiable experimental protocol (frequency sweep with strict thermal control, matched off-resonance controls, and an isotope shift test in D\(_2\)O). A positive result would support a discrete-time folding mechanism; a null result would falsify the RS jamming prediction under the tested conditions.
\end{abstract}

\section{Introduction}
Protein folding is often modeled as stochastic motion on a continuous free-energy landscape: the chain samples conformations via thermal fluctuations and relaxes toward minima, with kinetics governed by diffusion, friction, and barrier crossing. Within this view, perturbations such as temperature, denaturants, viscosity changes, or mutations modulate folding rates broadly and smoothly; sharp, frequency-specific arrest of folding is not a standard prediction. Recent advances in structure prediction (e.g., AlphaFold) have greatly improved the \emph{prediction} problem \cite{Jumper2021}, but do not by themselves settle the question of dynamical mechanism.

Recognition Science motivates a different class of dynamics claims. In RS, the physical substrate is a discrete ``ledger'' with an 8-tick correction cycle, and biological timescales arise from frequency division through a structured water mechanism (the ``hydration gearbox''). This framework makes a strong promise: if folding is clocked, then there should exist \emph{specific} experimentally accessible frequencies at which the clock can be perturbed.

This paper isolates one such prediction: a microwave-band jamming frequency near \SI{14.653}{GHz}. The goal is not to reinterpret all of folding, but to present a clean, falsifiable experiment that distinguishes a clocked mechanism from a purely continuous landscape picture. Importantly, because water absorbs microwaves and heating is a dominant confound in this band, the protocol is designed around controls that explicitly separate resonance effects from thermal effects.

\section{Theory: RS Bio-Clocking and the molecular gate}
\subsection{Bio-Clocking theorem}
RS Bio-Clocking posits that a hierarchy of biological timescales is generated by powers of the golden ratio:
\begin{equation}
\tau(N) = \tau_0 \, \phi^{N},
\end{equation}
where \(\phi = (1+\sqrt{5})/2 \approx 1.6180339887\) and \(\tau_0\) is the RS atomic tick. In the current implementation, \(\tau_0\) is mapped to SI seconds as:
\begin{equation}
\tau_0 \approx \SI{7.300e-15}{s}.
\end{equation}
The RS motivation and supporting derivations for this mapping are documented in the project theory notes \cite{Washburn2026RSThoryBioClocking}.

\subsection{Rung 19 ``molecular gate''}
RS identifies rung \(N=19\) as a ``molecular gate'' timescale relevant to protein folding steps. Substituting \(N=19\) gives:
\begin{equation}
\tau_{19} = \tau_0 \phi^{19} \approx \SI{6.82477e-11}{s} = \SI{68.248}{ps}.
\end{equation}
We define the primary jamming frequency as the reciprocal:
\begin{equation}
f_{\mathrm{jam}} = \frac{1}{\tau_{19}} \approx \SI{1.46525e10}{Hz} = \SI{14.653}{GHz}.
\end{equation}
These values are computed by the E41 script \cite{Washburn2026E41JammingCalc} and have a corresponding Lean formalization scaffold \cite{Washburn2026LeanJamming}.

\subsection{Related frequencies and controls}
A clocked mechanism generally admits harmonics and sub-harmonics. The RS literature emphasizes an 8-tick structure, motivating an 8-fold sub-harmonic:
\begin{equation}
f_{\mathrm{jam}}/8 \approx \SI{1.832}{GHz}.
\end{equation}
We also consider:
\begin{itemize}
    \item \textbf{Second harmonic}: \(2f_{\mathrm{jam}} \approx \SI{29.305}{GHz}\).
    \item \textbf{Nyquist alternative}: \(f_{\mathrm{jam}}/2 \approx \SI{7.326}{GHz}\) (if the relevant perturbation is half-cycle sampling).
    \item \textbf{Off-rung controls}: rung offsets \(N \pm 0.5\), which shift frequency by \(\phi^{\pm 1/2}\). With \(\phi^{1/2}\approx 1.272\),
    \begin{align}
        f_{18.5} &= f_{\mathrm{jam}}\phi^{1/2} \approx \SI{18.638}{GHz},\\
        f_{19.5} &= f_{\mathrm{jam}}/\phi^{1/2} \approx \SI{11.519}{GHz}.
    \end{align}
\end{itemize}

\begin{table}[t]
\centering
\small
\caption{Primary target, harmonics, and off-rung controls derived from \(\tau_0=\SI{7.300e-15}{s}\) and \(\phi\).}
\label{tab:freqs}
\begin{tabular}{@{}lll@{}}
\toprule
\textbf{Kind} & \textbf{Frequency} & \textbf{Note} \\
\midrule
Target & \SI{14.653}{GHz} & \(f_{\mathrm{jam}} = 1/\tau_{19}\) \\
Sub-harmonic & \SI{1.832}{GHz} & \(f_{\mathrm{jam}}/8\) \\
Harmonic & \SI{29.305}{GHz} & \(2 f_{\mathrm{jam}}\) \\
Alt. (Nyquist) & \SI{7.326}{GHz} & \(f_{\mathrm{jam}}/2\) \\
Control & \SI{11.519}{GHz} & off-rung \(N=19.5\) \\
Control & \SI{18.638}{GHz} & off-rung \(N=18.5\) \\
\bottomrule
\end{tabular}
\end{table}

\section{Experimental prediction}
\subsection{Primary claim}
\textbf{Prediction P1 (jamming):} irradiation near \(\SI{14.653}{GHz}\) produces a measurable decrease in folding rate \(k_{\mathrm{fold}}\) and/or destabilization of the folded state, \emph{not explainable by bulk temperature increase} under controlled conditions.

\subsection{Isotope shift (secondary claim)}
\textbf{Prediction P2 (isotope shift):} repeating the sweep in D\(_2\)O shifts the resonance. A conservative mass-scaling hypothesis uses the molar mass ratio:
\begin{equation}
f_{\mathrm{D2O}} \approx \frac{f_{\mathrm{H2O}}}{\sqrt{M_{\mathrm{D2O}}/M_{\mathrm{H2O}}}} \approx \frac{\SI{14.653}{GHz}}{\sqrt{20.03/18.02}} \approx \SI{13.897}{GHz}.
\end{equation}
RS sub-models that attribute the relevant inertia to H\(\rightarrow\)D substitution at the bond level would predict a larger shift (\(\sim 1/\sqrt{2}\)), so the D\(_2\)O measurement is treated as an informative discriminator rather than a fixed numeric claim.

\section{Experimental protocol (falsifiable design)}
This protocol is a LaTeX rendering of \code{docs/JAMMING\_PROTOCOL.md} \cite{Washburn2026JammingProtocol} with additional control logic.

\subsection{Materials and readouts}
\textbf{Proteins:} choose at least one fast folder (e.g., Trp-cage / 1L2Y) and optionally a larger test protein (e.g., ubiquitin / 1UBQ) to test generality.

\textbf{Solvent:} phosphate buffer at pH 7.0. Repeat the key sweep in D\(_2\)O buffer for isotope-shift testing.

\textbf{Readout options:}
\begin{itemize}
    \item \textbf{CD spectroscopy} (222 nm) to monitor helicity/folding fraction.
    \item \textbf{Intrinsic fluorescence} (tryptophan) for fast kinetic monitoring (especially for Trp-cage).
\end{itemize}

\subsection{Microwave delivery and thermal control}
The dominant confound at Ku-band frequencies is heating. Therefore:
\begin{itemize}
    \item Use active temperature control (Peltier or circulating bath) and log temperature continuously.
    \item Prefer pulsed irradiation (duty-cycled) if continuous-wave power cannot be made non-thermal.
    \item Use matched-power controls: for each frequency, adjust incident power so that measured bulk temperature and/or absorbed power is matched across conditions. This is essential to separate resonance from dielectric heating.
\end{itemize}

\subsection{Procedure}
\paragraph{Step 1: Baseline.} Measure \(k_{\mathrm{fold}}\) and \(k_{\mathrm{unfold}}\) (or equilibrium folded fraction) at fixed temperature with microwaves off.

\paragraph{Step 2: Frequency sweep near the target.} At fixed temperature (e.g., \(25\,^{\circ}\mathrm{C}\)), sweep \(\SI{14.0}{GHz}\) to \(\SI{15.2}{GHz}\) in \(\SI{0.05}{GHz}\) steps. At each frequency, record the folding observable after an equilibration period or via kinetic trace (depending on assay).

\paragraph{Step 3: Off-resonance controls.} Repeat the same procedure at off-rung controls (Table \ref{tab:freqs}), e.g., \SI{11.519}{GHz} and \SI{18.638}{GHz}. If an effect is seen broadly across frequencies, it is likely heating; a clocked resonance should show a localized peak/dip.

\paragraph{Step 4: Power dependence.} At the strongest-effect frequency (if any), vary power/duty cycle while maintaining temperature. Measure whether the effect has a threshold or scales smoothly.

\paragraph{Step 5: D\(_2\)O shift.} Repeat the frequency sweep in D\(_2\)O buffer. A shifted peak strengthens the ``clocked hydration'' interpretation.

\subsection{Success and falsification criteria}
\textbf{Pass:} a sharp, reproducible dip in \(k_{\mathrm{fold}}\) or stability localized near \(\SI{14.653 \pm 0.1}{GHz}\), inconsistent with broad water heating curves, and ideally exhibiting an isotope shift in D\(_2\)O.

\textbf{Fail:} no measurable effect under adequate field strength; or only broad effects that track temperature/absorbed power without localized resonance features.

\section{Discussion}
\subsection{What a positive result would imply}
A localized resonance in folding dynamics at a predicted frequency would be difficult to explain under a purely continuous landscape model without introducing an additional resonant degree of freedom. It would support at least one of the following:
\begin{itemize}
    \item folding dynamics couple to a structured-water/hydration-shell mode with a narrow frequency response, or
    \item folding proceeds via discrete-time gating steps that can be perturbed by external periodic forcing.
\end{itemize}
Either interpretation would motivate a new class of ``non-chemical'' folding perturbation experiments and would justify deeper investigation of RS dynamic claims (8-tick correction cycles and hydration gearbox downmixing).

\subsection{What a null result would imply}
A null result does not prove that folding is purely continuous, but it would falsify the RS jamming prediction \emph{under the tested conditions}. The most common failure modes would be:
\begin{itemize}
    \item insufficient field strength at the sample (coupling too weak),
    \item thermal noise or heating dominates and masks any narrow resonance,
    \item the relevant mechanism exists but at a different rung or depends on protein context.
\end{itemize}
These failure modes are why the protocol emphasizes power calibration, matched heating controls, and an isotope-shift test.

\section*{Data and code availability}
All derivations and protocols are included in the repository:
\begin{itemize}
    \item \code{run\_e41\_jamming\_calc.py} (E41 calculator; \(\tau_{19}\) and \(\SI{14.653}{GHz}\))
    \item \code{docs/JAMMING\_PROTOCOL.md} (protocol)
    \item \code{lean\_formalization/ProteinFolding/Derivations/D9\_JammingFrequency.lean} (formal derivation scaffold \cite{Washburn2026LeanJamming})
    \item \code{Recognition-Science-Full-Theory.txt} (Bio-Clocking + hydration gearbox background \cite{Washburn2026RSThoryBioClocking})
\end{itemize}

\bibliographystyle{plain}
\bibliography{references}

\end{document}


