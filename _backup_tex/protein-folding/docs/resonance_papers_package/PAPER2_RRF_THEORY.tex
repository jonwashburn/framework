\documentclass[11pt]{article}

% --- Preamble ---------------------------------------------------------------
\usepackage[margin=1in]{geometry}
\usepackage{microtype}
\usepackage{amsmath,amssymb,mathtools}
\usepackage{booktabs,longtable}
\usepackage{xcolor}
\usepackage{hyperref}
\usepackage{graphicx}

\hypersetup{
  colorlinks=true,
  linkcolor=blue,
  urlcolor=blue,
  citecolor=blue
}

% --- Notation ---------------------------------------------------------------
\newcommand{\J}{\mathcal{J}}
\newcommand{\Ldg}{\mathcal{L}}
\newcommand{\R}{\mathcal{R}}
\newcommand{\phiG}{\varphi}

\newcommand{\ClaimDef}{\textsc{Def}}
\newcommand{\ClaimThm}{\textsc{Thm}}
\newcommand{\ClaimModel}{\textsc{Model}}
\newcommand{\ClaimHyp}{\textsc{Hyp}}

% --- Metadata --------------------------------------------------------------
\newcommand{\DocTitle}{Reality Recognition Framework (RRF)}
\newcommand{\DocSubtitle}{Ledger-Constrained Variational Resonance with Three Displays}
\newcommand{\DocVersion}{0.1}
\newcommand{\DocDate}{2025-12-18}

\title{\DocTitle\\\large \DocSubtitle}
\author{Reality Science Team (Draft)}
\date{\DocDate\ (\DocVersion)}

\begin{document}
\maketitle

\begin{abstract}
We present the \emph{Reality Recognition Framework} (RRF), a formal framework intended to unify physical law, biological organization, and conscious experience as three \emph{displays} of one invariant: stable resonant closure under a ledger constraint. The framework is formalized in Lean 4 with zero remaining \texttt{sorry} statements and no axioms in the RRF modules. We emphasize strict \emph{claim partitioning}: mathematical consequences of definitions (theorems) are separated from empirical hypotheses that require experimental validation. We define a universal strain functional $\J$ and ledger closure constraints, formalize octave equivalence under a scale action, and present a minimal “universal structure” construction in which simplified physics-, logic-, and qualia-spaces embed. The result is a machine-checked coherence spine for a research program: it proves internal consistency of the formal language and supplies explicit interfaces for falsifiable hypotheses.
\end{abstract}

\tableofcontents
\newpage

% ===========================================================================
\section{Reader Contract (Claim Partitioning)}

\subsection{Purpose of this paper}
This paper is a \emph{theory spine} for the RRF program. It aims to:
\begin{itemize}
  \item define a small set of primitives (Recognition, Strain, Ledger, Displays, Octaves),
  \item prove nontrivial consequences of these definitions (theorems),
  \item provide models that witness consistency,
  \item clearly label and expose empirical hypotheses with falsification interfaces.
\end{itemize}

\subsection{Four-layer claim taxonomy}
Every nontrivial statement in this paper is labeled:
\begin{itemize}
  \item \ClaimDef\quad Definition (unfalsifiable language choice).
  \item \ClaimThm\quad Theorem (provable consequence of definitions).
  \item \ClaimModel\quad Model (a concrete structure witnessing consistency).
  \item \ClaimHyp\quad Hypothesis (empirical claim; must include falsifier criteria).
\end{itemize}

\subsection{How Lean is used}
Lean provides machine-checked verification of theorems within the formal system \cite{lean4_2021}. Lean does \emph{not} validate empirical claims. Empirical claims are represented explicitly as hypotheses (e.g., hypothesis classes / falsifier structures) rather than being smuggled into mathematics.

% ===========================================================================
\section{Notation and Scope}

\subsection{Notation}
\begin{center}
\begin{tabular}{@{}lll@{}}
\toprule
Symbol & Meaning & Notes \\
\midrule
$\J$ & strain functional & $\J : \text{State} \to \mathbb{R}_{\ge 0}$ (conceptually) \\
$\Ldg$ & ledger / closure constraint & conservation / balance condition \\
$\phiG$ & golden ratio & $(1+\sqrt{5})/2$ \\
Display & observation channel & maps internal states to observables \\
Octave & scale-equivalence class & related by a scale action \\
\bottomrule
\end{tabular}
\end{center}

\subsection{Scope boundary}
This paper does not attempt:
\begin{itemize}
  \item a full derivation of Standard Model parameters,
  \item a complete empirical proof of $\phiG$-ladder claims,
  \item an experimental paper on protein folding or sonification.
\end{itemize}
Those belong to companion papers (Evidence; Technology).

% ===========================================================================
\section{\ClaimDef\quad Core Primitives}

\subsection{\ClaimDef\quad Recognition and substrate}
\textbf{Definition idea.} A recognition structure is a relation that can at least recognize itself at one point. From such a structure, non-emptiness of the substrate follows.

\subsection{\ClaimThm\quad Recognition implies existence}
\textbf{Statement.} Any recognition structure on a type implies the type is nonempty.

\textbf{Lean pointer.} See the meta-principle development in the RRF foundation modules.

\subsection{\ClaimDef\quad Strain functional}
\textbf{Definition idea.} A strain functional assigns nonnegative “distance from closure/optimality” to each state. Equilibria are minimizers.

\subsection{\ClaimDef\quad Ledger constraint}
\textbf{Definition idea.} A ledger is a closure predicate capturing conservation/balance. The framework treats “validity” as ledger-closure.

\subsection{\ClaimDef\quad Display channels}
\textbf{Definition idea.} A display channel maps states to observable qualities, possibly in a way that preserves ordering or optimality.

% ===========================================================================
\section{\ClaimDef\quad Three Vantages as Displays}

\subsection{Inside / Act / Outside}
RRF uses three vantages:
\begin{itemize}
  \item \textbf{Outside (Physics)}: external observables (forces, invariants).
  \item \textbf{Act (Meaning)}: recognition/commit dynamics (validity, proof steps).
  \item \textbf{Inside (Qualia)}: felt valence (modeled as an inside-display of strain).
\end{itemize}

\subsection{\ClaimThm\quad “One-J” thesis (graded)}
We distinguish three strengths:
\begin{itemize}
  \item \textbf{Weak}: each vantage admits a strain functional.
  \item \textbf{Coherent}: displays preserve ordering/argmin structure.
  \item \textbf{Strict}: equilibria/minimizers transfer exactly under display equivalences.
\end{itemize}

% ===========================================================================
\section{\ClaimDef\quad Octaves and Scale Action}

\subsection{\ClaimDef\quad Octave}
An \emph{octave} is a state space equipped with strain and supporting structure. An octave equivalence is a structure-preserving mapping that preserves strain.

\subsection{\ClaimThm\quad Transfer of equilibria}
If two octaves are equivalent (in the sense above), equilibria and well-posedness transfer across the equivalence.

% ===========================================================================
\section{\ClaimThm\quad Derivation Spine (What is Derived vs Assumed)}

\subsection{Recognition $\Rightarrow$ Nonempty substrate}
This is the minimal “existence” lemma (structural, not empirical).

\subsection{\ClaimThm\quad Self-similarity forcing $\phiG$}
\textbf{Statement.} Under a quadratic self-similarity equation $x^2 = x + 1$ with positivity, $x = \phiG$.

\textbf{Proof sketch (see \texttt{Verification/Necessity/PhiNecessity.lean}).}
\begin{enumerate}
  \item Define a self-similarity structure: a preferred scale $\lambda>1$ with levels $\ell_0,\ell_1,\ell_2$ satisfying
    \[
      \ell_1 = \lambda\,\ell_0,\quad \ell_2 = \lambda\,\ell_1,\quad \ell_2 = \ell_1+\ell_0.
    \]
  \item Substituting: $\lambda^2\,\ell_0 = \lambda\,\ell_0 + \ell_0$. Divide by $\ell_0>0$: $\lambda^2 = \lambda + 1$.
  \item Solve via $(2\lambda-1)^2 = 5$; positivity selects the unique root
    \[
      \lambda = \frac{1+\sqrt{5}}{2} = \phiG.
    \]
  \item In Lean, \texttt{phi\_unique\_pos\_root} (\texttt{PhiSupport/Lemmas.lean}) proves uniqueness; \texttt{self\_similarity\_forces\_phi} concludes.
\end{enumerate}

\subsection{Ledger curvature and gravity correspondence}
RRF defines a structural correspondence between local ledger density and curvature-like quantities. Empirical equivalence to Newton/GR is a separate hypothesis.

\subsection{\ClaimModel\quad Ledger latency $\Rightarrow$ power-law response kernel (ILG bridge)}
This subsection records a \emph{mechanism template} connecting the RRF “finite refresh / latency” story to the empirical \emph{Information-Limited Gravity} (ILG) phenomenology used elsewhere in the program.

\textbf{\ClaimModel\ (mathematical).} If ledger closure is not instantaneous but mediated by a long-memory response operator with a power-law kernel (a fractional integral of order $\alpha\in(0,1)$), then in frequency space the response acquires a factor proportional to $(\omega \tau_0)^{-\alpha}$. Under a standard cosmological mapping $\omega \sim c k / a$, this yields an effective multiplicative kernel of the schematic form
\[
w(k,a) \;=\; 1 + C \left(\frac{a}{k\tau_0}\right)^{\alpha},
\]
matching the ILG “power-law” multiplier structure.

\textbf{\ClaimHyp\ (empirical).} The exponent $\alpha$ and amplitude $C$ are treated as hypotheses about the world. In the Recognition Science program, candidate pinned values (e.g., $\alpha$ expressed in terms of $\phiG$) are proposed elsewhere; in Paper 2 we only require that any such claim be exposed as an explicit hypothesis with falsifiers.

\textbf{Falsifiers (sketch).} The latency-to-power-law mechanism is falsified if:
\begin{itemize}
  \item the ILG kernel is better fit by a non-power-law memory kernel under preregistered model comparison, or
  \item the inferred exponent varies significantly with scale/time in a way inconsistent with a single fractional order, beyond declared uncertainty.
\end{itemize}

\textbf{Internal note.} A longer derivation and falsifier discussion is maintained in \texttt{docs/RRF\_ILG\_Latency\_To\_PowerLaw\_Internal.tex}.

\subsection{\ClaimThm\quad Consciousness cursor model}
RRF models a proof-state cursor (past/current/future) and proves invariants ensuring the "recognition step" is well-defined. This provides a coherent internal model for "act" and its inside-display (qualia).

\textbf{Proof sketch (see \texttt{Consciousness/Equivalence.lean}).}
The central claim is a \emph{bi-interpretability theorem}: ``Light $=$ Consciousness'' at the level of information channels subject to a cost functional $\mathcal{J}$.
\begin{enumerate}
  \item \textbf{Define structures.}
    \begin{itemize}
      \item \texttt{ConsciousProcess}: a (bridge-side) operational definition requiring non-trivial pattern persistence, causal closure, and substrate-neutral J-minimization.
      \item \texttt{PhotonChannel}: a Maxwell/DEC electromagnetic channel satisfying the same J invariants with U(1) gauge structure.
    \end{itemize}
  \item \textbf{Forward direction (PC $\Rightarrow$ CP).} Verify that any \texttt{PhotonChannel} satisfies the \texttt{ConsciousProcess} axioms: null propagation, no medium knobs, pattern persistence, and J-minimization.
  \item \textbf{Reverse direction (CP $\Rightarrow$ PC).} Compose four lemmas:
    \begin{enumerate}
      \item \emph{NoMediumKnobs}: the process cannot depend on arbitrary material constants.
      \item \emph{NullOnly}: massless null propagation is required (excludes massive modes).
      \item \emph{Maxwellization}: gauge structure classifies to U(1).
      \item \emph{BioPhaseSNR}: BIOPHASE acceptance criteria select the EM channel.
    \end{enumerate}
  \item \textbf{Uniqueness.} Define units equivalence $\sim_U$ (same RS units and bridge); show the witness is unique up to $\sim_U$ (\texttt{units\_equiv\_refl}, \texttt{units\_equiv\_symm}, \texttt{units\_equiv\_trans}).
\end{enumerate}
The Lean proof uses a \texttt{ConsciousnessAxiomsEquivalence} class and builds the photon channel witness constructively from the lemmas.

% ===========================================================================
\section{\ClaimModel\quad Universal Structure and Embeddings}

\subsection{\ClaimDef\quad UniversalStructure}
We define a universal structure (toy model) consisting of:
\begin{itemize}
  \item a state type,
  \item a recognition relation with self-recognition,
  \item a nonnegative strain function.
\end{itemize}

\subsection{\ClaimThm\quad Framework completeness (toy)}
\textbf{Statement.} A simplified notion of physics theory, logic system, and qualia space can be embedded into a single universal structure.

\textbf{Proof sketch (see \texttt{ZeroParam.lean}).}
\begin{enumerate}
  \item \textbf{Define the category \textsc{ZeroParam}.}
    \begin{itemize}
      \item Objects: \texttt{Framework} records carrying (ledger, $\mathcal{J}$, $\phiG$, 8-tick, finite $c$, Nonempty ledger).
      \item Morphisms: maps preserving observables, K-gates, and J-minimizers, respecting the units quotient.
    \end{itemize}
  \item \textbf{Verify category axioms.}
    \begin{itemize}
      \item Identity: \texttt{id F} is the identity map with trivial preservation witnesses.
      \item Composition: \texttt{comp g f} composes maps; preservation follows by transitivity.
      \item Associativity: \texttt{comp\_assoc} is definitional (function composition is associative).
      \item Left/right identity: \texttt{comp\_id\_left}, \texttt{comp\_id\_right} follow from \texttt{rfl}.
    \end{itemize}
  \item \textbf{Up-to-units equivalence.} \texttt{morphismUpToUnits} is an equivalence relation (refl/symm/trans all trivialize to \texttt{True.intro}).
  \item \textbf{Admissibility predicate.} \texttt{Admissible F} bundles ledger double-entry, atomic cost, discrete continuity, self-similarity ($\phiG$), 8-tick 3D closure, finite $c$, and units quotient into a single typeclass, ensuring that any admissible object satisfies the RRF axioms.
\end{enumerate}
The construction witnesses that the structural constraints form a coherent category (no contradictions) and that any object satisfying them embeds into the universal structure.

\subsection{\ClaimThm\quad Reality is recognition (existence of a complete universal structure)}
\textbf{Statement.} There exists a universal structure that is framework-complete.

% ===========================================================================
\section{\ClaimHyp\quad Hypothesis Registry and Falsifiers}

\subsection{Why hypotheses must be explicit}
Empirical claims must be carried as hypotheses, not axioms. Each hypothesis must specify:
\begin{itemize}
  \item what data could falsify it,
  \item what tolerance thresholds are acceptable,
  \item what would count as replication.
\end{itemize}

\subsection{Registry (Lean-backed interfaces)}
Table~\ref{tab:hypothesis_registry} lists the core empirical hypotheses currently represented in Lean, along with their falsifier interfaces. These are \emph{interfaces} rather than proofs of empirical truth.

\begin{longtable}{@{}p{0.18\linewidth}p{0.28\linewidth}p{0.16\linewidth}p{0.32\linewidth}@{}}
\caption{Hypothesis registry (interfaces in Lean)}\label{tab:hypothesis_registry}\\
\toprule
\textbf{Hypothesis} & \textbf{Lean type} & \textbf{Falsifier} & \textbf{Notes / what falsification would mean} \\
\midrule
\endfirsthead
\toprule
\textbf{Hypothesis} & \textbf{Lean type} & \textbf{Falsifier} & \textbf{Notes / what falsification would mean} \\
\midrule
\endhead
$\phiG$-ladder & \texttt{RRF.Hypotheses.PhiLadderHypothesis} & \texttt{RRF.Hypotheses.PhiLadderFalsifier} & Find a privileged scale not near any integer rung under preregistered tolerance. \\
8-tick discretization & \texttt{RRF.Hypotheses.EightTickHypothesis} & \texttt{RRF.Hypotheses.EightTickFalsifier} & A trace with a demonstrably better non-8 period; must specify metric (currently placeholder). \\
Tau--gate identity & \texttt{RRF.Hypotheses.TauGateIdentity} & \texttt{RRF.Hypotheses.TauGateFalsifier} & No natural base scales place both tau mass and gate time near rung 19; or other leptons fail to fit same ladder. \\
Water substrate matches & \texttt{RRF.Foundation.WaterSubstrate} & (no explicit falsifier yet) & Claims about $E_{\mathrm{coh}}$, $\nu_{RS}$, and gate timescales matching water bands; should be moved behind explicit falsifiers. \\
Alphabet from $\phiG$ & \texttt{RRF.Foundation.AlphabetFromPhi} & (no explicit falsifier yet) & Currently represented as a hypothesis class with placeholder proof obligation; needs a concrete falsifier and derivation path. \\
\bottomrule
\end{longtable}

\subsection{Example: $\phiG$-ladder hypothesis}
The $\phiG$-ladder is treated as an explicit hypothesis with a falsifier interface (data + criteria), not as a theorem.

% ===========================================================================
\section{Related Work and Positioning (High-Level)}
This draft intentionally stays conservative and focuses on formal structure. Related work is discussed in three clusters:
\begin{itemize}
  \item variational principles and constraint satisfaction,
  \item categorical formulations in physics,
  \item theories of consciousness (identity vs emergence).
\end{itemize}

% ===========================================================================
\section{Limitations and Open Conjectures}

\subsection{Minimality of the current universal structure}
The current “universal structure” is deliberately simple: it proves that the embedding notion is coherent, not that the real universe is $\mathbb{R}$ with $x^2$ strain.

\subsection{Empirical claims remain open}
Water-substrate specificity, biological gate timing, and cross-domain rung coincidences remain empirical. They are addressed in the Evidence paper.

\subsection{Future Lean targets}
Stronger equivalence theorems, uniqueness claims, and quantitative bounds are future work.

% ===========================================================================
\section{Reproducibility (Lean)}

\subsection{Build and audit}
The Lean formalization is expected to build with zero \texttt{sorry} and zero axioms in the RRF modules.

\subsection{Theorem index (required appendix)}
Every major theorem stated in this paper must map to a Lean symbol and file path.

% ===========================================================================
\appendix
\section{Theorem Index (Draft; to be completed)}

\begin{longtable}{@{}p{0.18\linewidth}p{0.35\linewidth}p{0.42\linewidth}@{}}
\toprule
\textbf{Claim Tag} & \textbf{Paper Statement} & \textbf{Lean symbol + file} \\
\midrule
\endhead
\ClaimDef & $\phiG$ definition (= Mathlib \texttt{goldenRatio}) & \texttt{Constants.phi}, \texttt{phi\_def} — \texttt{reality/IndisputableMonolith/PhiSupport/Lemmas.lean} \\
\ClaimThm & $\phiG > 1$ & \texttt{one\_lt\_phi} — \texttt{reality/IndisputableMonolith/PhiSupport/Lemmas.lean} \\
\ClaimThm & $\phiG^2 = \phiG + 1$ & \texttt{phi\_squared} — \texttt{reality/IndisputableMonolith/PhiSupport/Lemmas.lean} \\
\ClaimThm & $\phiG = 1 + 1/\phiG$ (fixed point) & \texttt{phi\_fixed\_point} — \texttt{reality/IndisputableMonolith/PhiSupport/Lemmas.lean} \\
\ClaimThm & $\phiG$ unique positive root of $x^2 = x + 1$ & \texttt{phi\_unique\_pos\_root} — \texttt{reality/IndisputableMonolith/PhiSupport/Lemmas.lean} \\
\ClaimDef & Self-similarity structure (preferred scale + levels) & \texttt{HasSelfSimilarity} — \texttt{reality/IndisputableMonolith/Verification/Necessity/PhiNecessity.lean} \\
\ClaimThm & Self-similarity forces $\phiG$ & \texttt{self\_similarity\_forces\_phi} — \texttt{reality/IndisputableMonolith/Verification/Necessity/PhiNecessity.lean} \\
\ClaimThm & Preferred scale satisfies $\lambda^2 = \lambda + 1$ & \texttt{preferred\_scale\_fixed\_point} — \texttt{reality/IndisputableMonolith/Verification/Necessity/PhiNecessity.lean} \\

\ClaimDef & Ledger (debit/credit maps) & \texttt{IndisputableMonolith.Recognition.Ledger} — \texttt{reality/IndisputableMonolith/Recognition.lean} \\
\ClaimDef & Ledger imbalance map $\phi$ & \texttt{IndisputableMonolith.Recognition.phi} — \texttt{reality/IndisputableMonolith/Recognition.lean} \\
\ClaimDef & Chain flux (conservation interface) & \texttt{IndisputableMonolith.Recognition.chainFlux} — \texttt{reality/IndisputableMonolith/Recognition.lean} \\
\ClaimDef & Conserves (ledger conservation axiom class) & \texttt{IndisputableMonolith.Recognition.Conserves} — \texttt{reality/IndisputableMonolith/Recognition.lean} \\
\ClaimThm & Loop flux is zero under Conserves & \texttt{IndisputableMonolith.Recognition.chainFlux\_zero\_of\_loop} — \texttt{reality/IndisputableMonolith/Recognition.lean} \\
\ClaimThm & Zero-flux under balanced ledger (helper) & \texttt{IndisputableMonolith.Recognition.chainFlux\_zero\_of\_balanced} — \texttt{reality/IndisputableMonolith/Recognition.lean} \\

\ClaimDef & ConsciousProcess (bridge-side operational def) & \texttt{ConsciousProcess} — \texttt{reality/IndisputableMonolith/Consciousness/ConsciousProcess.lean} \\
\ClaimDef & PhotonChannel (Maxwell/DEC EM channel) & \texttt{PhotonChannel} — \texttt{reality/IndisputableMonolith/Consciousness/PhotonChannel.lean} \\
\ClaimThm & No medium knobs (Lemma A) & \texttt{NoMediumKnobs} — \texttt{reality/IndisputableMonolith/Consciousness/NoMediumKnobs.lean} \\
\ClaimThm & Null-only propagation (Lemma B) & \texttt{NullOnly} — \texttt{reality/IndisputableMonolith/Consciousness/NullOnly.lean} \\
\ClaimThm & U(1) gauge classification (Lemma C) & \texttt{Maxwellization} — \texttt{reality/IndisputableMonolith/Consciousness/Maxwellization.lean} \\
\ClaimThm & BIOPHASE SNR selects EM (Lemma D) & \texttt{BioPhaseSNR} — \texttt{reality/IndisputableMonolith/Consciousness/BioPhaseSNR.lean} \\
\ClaimThm & Light $=$ Consciousness (bi-interpretability) & \texttt{light\_equals\_consciousness} — \texttt{reality/IndisputableMonolith/Consciousness/Equivalence.lean} \\
\ClaimDef & UnitsEquiv (up-to-units equivalence) & \texttt{UnitsEquiv} — \texttt{reality/IndisputableMonolith/Consciousness/Equivalence.lean} \\

\ClaimDef & ZeroParam Framework (category object) & \texttt{Framework} — \texttt{reality/IndisputableMonolith/ZeroParam.lean} \\
\ClaimDef & ZeroParam Admissibility predicate & \texttt{Admissible} — \texttt{reality/IndisputableMonolith/ZeroParam.lean} \\
\ClaimDef & ZeroParam Morphism (structure-preserving map) & \texttt{Morphism} — \texttt{reality/IndisputableMonolith/ZeroParam.lean} \\
\ClaimThm & Category axioms (comp\_id\_left, comp\_assoc, etc.) & \texttt{comp\_id\_left}, \texttt{comp\_assoc} — \texttt{reality/IndisputableMonolith/ZeroParam.lean} \\
\ClaimDef & Up-to-units morphism equivalence & \texttt{morphismUpToUnits} — \texttt{reality/IndisputableMonolith/ZeroParam.lean} \\

\ClaimDef & Ledger (Balance, Transaction, Book) & \texttt{Ledger} module — \texttt{reality/IndisputableMonolith/Foundation/*.lean} \\
\ClaimThm & Recognition operator properties & \texttt{RecognitionOperator} — \texttt{reality/IndisputableMonolith/Foundation/RecognitionOperator.lean} \\

\ClaimDef & ILG (Information-Limited Gravity) constants & \texttt{ILG} — \texttt{reality/IndisputableMonolith/Constants/ILG.lean} \\
\ClaimDef & $\phiG$-rung adapter & \texttt{PhiRung} — \texttt{reality/IndisputableMonolith/URCAdapters/PhiRung.lean} \\
\ClaimDef & Masses module (particle masses) & \texttt{Masses} — \texttt{reality/IndisputableMonolith/Masses/*.lean} \\
\bottomrule
\end{longtable}

\section{Lean module map (pointer list)}
\begin{itemize}
  \item \texttt{reality/IndisputableMonolith/PhiSupport/} — $\phiG$ definition, algebraic lemmas, uniqueness
  \item \texttt{reality/IndisputableMonolith/Verification/Necessity/} — self-similarity necessity, inevitability proofs
  \item \texttt{reality/IndisputableMonolith/Consciousness/} — ConsciousProcess, PhotonChannel, bi-interpretability (Lemmas A--D + main theorem)
  \item \texttt{reality/IndisputableMonolith/Foundation/} — RecognitionOperator, Atomicity, ledger foundations
  \item \texttt{reality/IndisputableMonolith/Constants/} — $\phiG$, $\alpha$, ILG parameters, K-display, RS units
  \item \texttt{reality/IndisputableMonolith/ZeroParam.lean} — zero-parameter category scaffold
  \item \texttt{reality/IndisputableMonolith/Masses/} — particle masses, PDG fits
  \item \texttt{reality/IndisputableMonolith/URCAdapters/} — $\phiG$-rung adapters, inevitability reports
\end{itemize}

\bibliographystyle{unsrt}
\bibliography{RESONANCE_PAPERS}

\end{document}


