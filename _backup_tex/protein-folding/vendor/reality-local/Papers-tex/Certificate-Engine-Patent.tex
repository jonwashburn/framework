% Certificate Engine — Family G Patent (Method, System, Medium)
\documentclass[12pt]{article}
\usepackage[a4paper,margin=1in]{geometry}
\usepackage{amsmath,amssymb}
\usepackage{enumitem}
\usepackage{hyperref}
\usepackage{microtype}
\usepackage{xcolor}

\title{Certificate Engine (Formal Audit)\\Family G Patent (Method, System, Medium)}
\author{Jonathan Washburn}
\date{\today}

\begin{document}
\maketitle

\begin{center}
{\large Camera-Ready Patent Draft}\\[4pt]
{\normalsize Lean Proof Execution; Hash-Addressed Certificate Bundles; Operation/Ingestion Gating on Passing Proofs}
\end{center}

\section*{Field}
This disclosure relates to formal verification and audit for calibration, measurement, and control systems. It provides a method, system, and non-transitory medium for executing \emph{machine-checkable proofs} of invariants (route identities, neutrality, units-quotient factorization, gate separation, policy), generating \emph{hash-addressed certificate bundles}, and \emph{gating} device/simulator operation and data ingestion based on passing proofs.

\section*{Background}
Calibration and measurement acceptance commonly rely on statistical tests or manual review without formal provenance. Prior methods do not: (i) execute \emph{machine-checkable proofs} of physical invariants as part of acceptance; (ii) bundle proof artifacts in \emph{hash-addressed, signed certificates} tied to runs; or (iii) \emph{gate operation or data ingestion} deterministically based on proof verdicts. Such integration is needed to achieve reproducible, auditable, and enforceable acceptance workflows.

\section*{Summary}
Disclosed is a \emph{Certificate Engine} that:
\begin{enumerate}[leftmargin=*]
  \item executes \textbf{machine-checkable proofs} (Lean 4 or equivalent) for invariants: route identities, neutrality, units-quotient factorization, gate separation, and policy compliance;
  \item collects proof results, invariant values, thresholds, run metadata, and environment attestations into a \textbf{hash-addressed certificate bundle};
  \item \textbf{gates} device or simulator operation and data ingestion based on certificate status: only \emph{passing} certificates authorize execution or acceptance; and
  \item optionally publishes certificate headers to a \textbf{transparency registry} for third-party audit and revocation management.
\end{enumerate}

\section*{Definitions}
\noindent\textbf{Machine-checkable proof.} A formal proof executed by a proof assistant (e.g., Lean 4, Coq, Isabelle) that verifies an invariant or identity holds, emitting proof artifacts and hashes.

\noindent\textbf{Invariants.} Examples: route identities ($\tau_{\mathrm{rec}}/\tau_0=\lambda_{\mathrm{kin}}/\ell_0$), neutrality ($\sum_{i=1}^{8}\delta_i=0$), units-quotient factorization, gate separation (Planck vs IR non-mixing), policy constraints.

\noindent\textbf{Hash-addressed certificate bundle.} A signed archive containing proof hashes, invariant values, thresholds, timestamps, run IDs, device metadata, and acceptance verdicts, content-addressed by cryptographic hash (e.g., SHA-256).

\noindent\textbf{Gating.} A control policy that denies device operation, simulator execution, or data ingestion unless a valid certificate with passing proofs is present.

\noindent\textbf{Transparency registry.} A public or permissioned ledger of certificate headers (hashes, issuers, timestamps, profile IDs) supporting third-party verification and revocation.

\noindent\textbf{Compliance API.} Interfaces to request proofs, retrieve certificate status, attach certificates to runs, and query the registry.

\section*{Claims}
\setlist[enumerate]{leftmargin=*,labelsep=0.5em,itemsep=0.6em,topsep=0.4em}

\subsection*{Independent Claims}
\begin{enumerate}
\item \textbf{(Method)} A computer-implemented method for proof-backed calibration audit, comprising:
\begin{enumerate}[label=(\alph*)]
  \item executing machine-checkable proofs for a plurality of invariants including route identities, neutrality, units-quotient factorization, and gate separation, using a proof engine;
  \item collecting proof results, invariant values, thresholds, run identifiers, device metadata, and timestamps into a hash-addressed certificate bundle;
  \item signing the certificate bundle with a cryptographic signature and content-addressing by hash;
  \item gating device operation or simulator execution and data ingestion based on certificate status, authorizing only if proofs pass; and
  \item optionally publishing certificate headers to a transparency registry for audit and revocation.
\end{enumerate}

\item \textbf{(System)} A certificate engine comprising one or more processors and memory configured to: (i) execute machine-checkable proofs of invariants; (ii) bundle proof results, metadata, and verdicts into hash-addressed certificates; (iii) sign certificates cryptographically; (iv) gate operation/ingestion on passing certificates; and (v) optionally publish to a transparency registry.

\item \textbf{(Non-transitory medium)} A non-transitory computer-readable medium storing instructions that, when executed, cause a system to perform the method of claim 1.
\end{enumerate}

\subsection*{Dependent Claims}
\begin{enumerate}[resume]
\item (Method) wherein the proof engine is Lean 4, Coq, Isabelle/HOL, or an equivalent interactive theorem prover producing verifiable proof artifacts.
\item (Method) wherein invariants include: (i) route identities $\tau_{\mathrm{rec}}/\tau_0=\lambda_{\mathrm{kin}}/\ell_0$; (ii) neutrality $\sum_{i=1}^{8}\delta_i=0$ over mirrored blocks; (iii) units-quotient factorization correctness; and (iv) gate separation (layer non-mixing).
\item (Method) wherein the certificate bundle comprises a JSON or binary schema with fields: profile version, run ID, invariant values with uncertainties, proof hashes, threshold settings, acceptance verdict, timestamps, device/environment metadata, and signatures.
\item (Method) wherein content-addressing uses SHA-256 or equivalent cryptographic hash; certificates are immutable once signed.
\item (Method) wherein gating comprises denying execution, rejecting data writes, or blocking downstream pipelines unless a valid certificate is present and all proofs pass.
\item (System) further comprising a compliance API exposing: request proof execution, retrieve certificate by hash or run ID, attach certificate to operation, query registry, and revoke/supersede certificates.
\item (Method) wherein certificate headers include hash, issuer identifier, profile ID, issuance timestamp, and expiry or revocation status, published to a transparency registry for third-party verification.
\item (Method) further comprising on-device certificate signing with a hardware security module (HSM) or secure enclave for tamper-evident provenance.
\item (Method) wherein replay mode re-executes proofs from a certificate to validate determinism and detect tampering.
\item (System) wherein failures to prove an invariant trigger diagnostic export with counterexample traces and policy-suggested remediation.
\item (Method) further comprising certificate chaining where run certificates reference upstream calibration certificates, forming an auditable provenance chain.
\item (Method) wherein revocation comprises publishing signed revocation notices to the registry and denying gating on revoked certificates.
\item (System) wherein the transparency registry is implemented as a distributed ledger, cloud database, or federated service with append-only guarantees and query APIs.
\item (Method) wherein proof execution is triggered per-run, per-block, or on-demand, with certificates cached and reused within declared validity windows.
\item (Method) further comprising certificate profiles (e.g., RB-CORE, RB-TIMING, RB-AUDIT) specifying required invariant subsets and thresholds.
\end{enumerate}

\section*{Brief Description of the Drawings}
\begin{enumerate}[leftmargin=*]
  \item Fig. 1: Certificate engine pipeline (proof execution, bundling, signing, gating).
  \item Fig. 2: Certificate bundle schema with proof hashes, invariants, and metadata.
  \item Fig. 3: Gating flow for device operation and data ingestion.
  \item Fig. 4: Transparency registry integration and revocation.
\end{enumerate}

\section*{Detailed Description}
\subsection*{Proof Execution}
The engine invokes a proof assistant with formalized invariants (route identities, neutrality, units quotient, gate separation, policy). Proofs are executed to obtain verdicts (pass/fail/error) and artifacts (proof terms, hashes). Failures produce counterexamples and diagnostics.

\subsection*{Certificate Bundling}
Proof results, invariant values, thresholds, run IDs, device metadata, timestamps, and acceptance verdicts are collected into a structured bundle (JSON or binary). The bundle is signed with a private key (optionally HSM-backed) and content-addressed by cryptographic hash, making it immutable.

\subsection*{Gating Logic}
Device controllers, simulator workflows, or data pipelines query the certificate engine before operation. If a valid certificate with all proofs passing is present, execution proceeds; otherwise the system fails closed (deny operation, reject data write, block pipeline). Gating policies may require specific proof subsets or profile compliance.

\subsection*{Transparency Registry}
Certificate headers (hash, issuer, profile, timestamp) are optionally published to a public or permissioned registry. Third parties can verify issuance and check revocation status. Payload privacy is preserved; only headers are public.

\subsection*{Compliance and Revocation}
A compliance API enables: proof requests, certificate retrieval, attachment to runs, and registry queries. Revocation publishes signed notices invalidating certificates (e.g., supersession, discovered error); gating logic denies revoked certificates.

\subsection*{Certificate Chaining and Provenance}
Run certificates may reference upstream calibration or acceptance certificates, forming a directed acyclic graph (DAG) of provenance. Chains enable end-to-end audit from raw data to final outputs, supporting regulatory and model-risk governance.

\section*{Technical Effects}
The engine yields: (i) reproducible, auditable acceptance via formal proofs; (ii) deterministic gating with fail-closed safety; (iii) provenance and traceability through signed, immutable certificates; (iv) reduced dispute via machine-verifiable evidence; and (v) regulatory compliance support (FDA, aerospace, model-risk governance).

\section*{Enablement and Best Mode}
Reference implementations include: (i) Lean 4 proof backend with invariant libraries (route identities, neutrality, units-quotient, gate separation); (ii) certificate bundling and signing modules (JSON schema, cryptographic primitives); (iii) gating logic for devices/simulators; (iv) transparency registry (distributed ledger or cloud database); and (v) compliance APIs (REST or gRPC). Benchmarks report proof latency, certificate size, and gating overhead.

\section*{Alternative Embodiments}
Other proof assistants may be used (Coq, Isabelle). Certificates may be on-device, cloud, or hybrid. Registry may be centralized, federated, or blockchain-based. Gating may be enforced at firmware, OS, or application layers. Certificate profiles may be domain-specific (metrology, fusion, manufacturing, AI).

\section*{Legal Notice}
All embodiments, features, and steps described herein may be combined in any operable permutation. Where means-plus-function is implicated, claims are intended to cover corresponding structures and their equivalents.

\end{document}

