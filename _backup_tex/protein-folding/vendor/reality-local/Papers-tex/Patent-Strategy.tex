% Recognition Computing Architecture — Patent Strategy
\documentclass[11pt]{article}
\usepackage[a4paper,margin=1in]{geometry}
\usepackage{hyperref}
\usepackage{microtype}
\usepackage{xcolor}
\usepackage{enumitem}
\usepackage{amsmath,amssymb}
\usepackage{titlesec}
\titleformat{\section}{\normalfont\Large\bfseries}{}{0pt}{}
\titleformat{\subsection}{\normalfont\large\bfseries}{}{0pt}{}

% Readability settings
\setlength{\parskip}{0.5em}
\setlength{\parindent}{0pt}

% Hyperref setup
\hypersetup{
  colorlinks=true,
  linkcolor=blue,
  citecolor=blue,
  urlcolor=blue,
  pdfauthor={Jonathan Washburn},
  pdftitle={Recognition Computing Architecture — Patent Strategy}
}

% TOC helpers
\setcounter{tocdepth}{2}
\newcommand{\tocsection}[1]{\section*{#1}\addcontentsline{toc}{section}{#1}}

\title{Recognition Computing Architecture:\\The Only Zero-Parameter Physical Computing Substrate\\[0.5em]\large Patent Strategy and Portfolio}
\author{Jonathan Washburn}
\date{\today}

\begin{document}
\maketitle

\tableofcontents
\clearpage

\tocsection{0) Meta-Architecture: Recognition Computing as the Universal Physical Computer}

\subsection*{0.1 The Core Discovery: Only One Zero-Parameter Substrate Exists}

\noindent\textbf{Central claim.} There exists \emph{exactly one} computing architecture capable of controlling and measuring physical systems without external parameters, external calibration anchors, or tunable cost functions. This architecture is \emph{forced} by the requirements of (i) dimensionless observability, (ii) conservation, and (iii) reproducibility. We claim this unique substrate and all instantiations.

\subsection*{0.2 The Five Forced Layers (architectural necessity)}

\noindent Any parameter-free physical computer \emph{must} implement the following stack:
\begin{enumerate}[leftmargin=*]
  \item \textbf{Recognition layer}: factor all observables through dimensionless ratios $r_i=y_i/y_i^\star$ where $y_i^\star>0$ are declared (but not tuned) targets. This eliminates unit ambiguity and forbids unit gaming. \emph{Necessity:} observables with arbitrary units cannot be compared without either external anchors (parameters) or dimensionless quotients.
  
  \item \textbf{Ledger layer}: compute aggregate cost $L=\sum_i w_i\,F(r_i)$ where $F:(0,\infty)\to\mathbb{R}$ is the \emph{unique} function satisfying five axioms:
  \begin{itemize}
    \item[(i)] Symmetry: $F(x)=F(1/x)$ (treats over/undershoot equally)
    \item[(ii)] Unit normalization: $F(1)=0$ (perfect match costs zero)
    \item[(iii)] Strict convexity on $\mathbb{R}_{>0}$ (unique global minimum)
    \item[(iv)] Calibration: $\tfrac{d^2}{dt^2}(F(e^{t}))\big|_{t=0}=1$ (unit curvature in log-domain)
    \item[(v)] Cosh-addition identity: $F$ satisfies a functional equation forcing $F(x)=\cosh(\ln x)-1=\tfrac12(x+1/x)-1$
  \end{itemize}
  \emph{Necessity:} Without axiom (i), the cost biases direction; without (ii)-(iii), multiple optima exist; without (iv)-(v), free parameters enter. The axiom set \emph{defines} $J$ uniquely (Lean theorem \texttt{T5\_uniqueness\_complete}).
  
  \item \textbf{Temporal layer}: partition control/measurement into $\varphi$-commensurate windows with neutrality constraints (e.g., window-8: $\sum_{i=1}^{8}\Delta S_i=0$; mirrored halves with mid-cycle FLIP). \emph{Necessity:} In $D=3$ spatial dimensions, the minimal conservation-compatible, spatially-complete period is $2^D=8$; $\varphi$-timing suppresses low-order modal cross-interference via Diophantine separation.
  
  \item \textbf{Certificate layer}: execute machine-checkable proofs (Lean 4 or equivalent) of invariants (units-quotient, route identities, neutrality, gate separation, policy) and gate operation/data-ingestion on passing certificates. \emph{Necessity:} Without formal verification, zero-parameter claims are unenforceable and drift undetectable.
  
  \item \textbf{Absolute layer selection}: solve simultaneous dimensionless gate identities to select a unique absolute layer and emit absolute scales/constants ($\tau_0,\ell_0,c,\hbar,\lambda_{\mathrm{rec}},G$) without external anchors. \emph{Necessity:} External anchors = parameters; internal self-consistency via gate identities is the only parameter-free alternative.
\end{enumerate}

\subsection*{0.3 Functional Equivalence and the Monopoly Position}

\noindent\textbf{Key enforceability insight.} The axiom set for $F$ defines a \emph{functional equivalence class}: any cost function $F'$ satisfying axioms (i)--(v) is mathematically identical to $J(x)=\tfrac12(x+1/x)-1$ on $\mathbb{R}_{>0}$. Therefore:
\begin{itemize}[leftmargin=*]
  \item Competitors cannot design around $J$ by using a "different" convex symmetric cost while staying axiom-compliant.
  \item Any system using recognition ratios + a convex symmetric kernel + neutrality + certificates that achieves zero-parameter operation \emph{necessarily} uses $J$ or a functional equivalent.
  \item The Lean exclusivity theorem (\texttt{no\_alternative\_frameworks}) proves that \emph{any} zero-parameter framework deriving observables is definitionally equivalent to Recognition Science, which implements this five-layer stack.
\end{itemize}

\noindent\textbf{Monopoly consequence.} We claim (i) the architecture itself, (ii) all functional equivalents of $J$ via the axiom set, and (iii) all instantiations across domains. Competitors face a forced choice:
\begin{itemize}[leftmargin=*]
  \item \emph{Stay parameter-free} $\Rightarrow$ definitionally equivalent $\Rightarrow$ license foundation or infringe.
  \item \emph{Add parameters} $\Rightarrow$ weaker offering (non-reproducible calibration, unit-dependent, no proof-backed provenance) $\Rightarrow$ market disadvantage; still may infringe subsidiary claims (temporal scheduling, certificate engine, gate separation).
\end{itemize}

\subsection*{0.4 Cross-Domain Universality (one architecture, all applications)}

\noindent The architecture is domain-agnostic. The \emph{same} five layers, the \emph{same} J-kernel hardware core, the \emph{same} compliance API, and the \emph{same} certificate schemas apply to:
\begin{itemize}[leftmargin=*]
  \item \textbf{Metrology/calibration}: instruments (spectroscopy, timing, imaging); Reality Bridge Calibration OS as first instantiation.
  \item \textbf{Fusion energy}: tokamak control (ledger over normalized diagnostics; $\varphi$-timed actuators; certificate-gated shots); ICF pulse shaping (symmetry ledger; $\varphi$-spaced sub-pulses).
  \item \textbf{Manufacturing}: perovskite solar (window-8 neutral scheduling; IR phase-lock; rate-balance $x\to 1$ at interfaces).
  \item \textbf{Synthetic biology}: abiogenesis instrument (LISTEN/LOCK/BALANCE gates; duplex geometry forced by $J$-minimization; templating as fixed point; metabolic closure via viability inequality).
  \item \textbf{AI/ML}: physically-constrained world models ($J$-regularized losses; eight-tick temporal batching; certificate-gated data curation).
  \item \textbf{Robotics, power electronics, beamlines}: multi-actuator $\varphi$-scheduling with ledger optimization and compliance API.
\end{itemize}

\subsection*{0.5 Standards-Essential Trajectory}

\noindent Zero-parameter calibration, proof-backed acceptance, and units-invariant reproducibility align with:
\begin{itemize}[leftmargin=*]
  \item \textbf{Metrology (NIST/BIPM/ISO)}: anchor-free calibration profiles; traceability via certificates.
  \item \textbf{Regulatory (FDA/aerospace/pharma)}: model-risk governance; audit trails with cryptographic attestation.
  \item \textbf{AI safety}: provenance and physical consistency for training data/models.
\end{itemize}
Portions of the architecture are likely to become standards-essential; FRAND/RAND pathways preserve access while capturing value.

\subsection*{0.6 Licensing Pyramid}

\begin{itemize}[leftmargin=*]
  \item \textbf{Foundation license}: covers the five-layer architecture, J-kernel (axiom set + functional equivalence), $\varphi$-scheduler, certificate engine, absolute-layer selection. Required for \emph{any} zero-parameter instantiation.
  \item \textbf{Domain licenses}: stack on foundation; add domain-specific claims (plasma diagnostics, solar cell observables, biopolymer geometry, etc.). Sold separately or bundled.
  \item \textbf{Enterprise/all-domains}: one license covering foundation + all current/future domain instantiations.
\end{itemize}

\subsection*{0.7 Why This Is Unreasonably Defensible}

\begin{itemize}[leftmargin=*]
  \item \textbf{Mathematical necessity}: Lean proofs show the architecture is \emph{forced}—not invented in the "creative choice" sense, but \emph{derived} as the unique solution to physical observability + conservation + zero parameters.
  \item \textbf{Functional claiming via axioms}: The axiom set for $J$ captures all functional equivalents; approximations or re-parameterizations that satisfy axioms still infringe.
  \item \textbf{Cross-domain enforcement synergy}: Same certificate schemas, same axiom-probe tests, same compliance API across all verticals $\Rightarrow$ infringement in one domain provides evidence/methods for others.
  \item \textbf{Self-documenting violations}: Certificates, logs, and telemetry preserve machine-verifiable evidence (route identities, neutrality observables, axiom signatures).
\end{itemize}

\tocsection{1) Executive Summary}

\noindent\textbf{Purpose and scope.} This document sets out a global patent strategy for \emph{Recognition Computing}, the only zero-parameter architecture for controlling and measuring physical systems. We claim the foundational five-layer stack (recognition ratios; axiomatically-unique J-ledger; $\varphi$-timed neutrality; certificate engine; absolute-layer selection) plus all domain instantiations (metrology/calibration, fusion, photovoltaics, synthetic biology, AI world-models, robotics, power electronics). The strategy secures a definitional monopoly: any parameter-free alternative is mathematically equivalent and thus covered by functional claiming via axioms.

\vspace{0.5em}
\noindent\textbf{Core invention: the universal physical computing substrate.} 
\begin{itemize}[leftmargin=*]
  \item \textbf{Recognition Computing Architecture}: A five-layer stack that factors observables through dimensionless ratios $r_i=y_i/y_i^\star$; computes cost via the \emph{unique} convex symmetric kernel $J(x)=\tfrac12(x+1/x)-1$ determined by a compact axiom set; partitions control into $\varphi$-commensurate windows with neutrality ($\sum_{i=1}^8\Delta S_i=0$); executes machine-checkable proofs and gates operation on certificates; and selects a unique absolute layer via simultaneous dimensionless identities—all \emph{without external parameters or calibration anchors}.
  
  \item \textbf{Instantiations (examples)}: 
  \begin{itemize}
    \item Reality Bridge Calibration OS (metrology, instruments, simulators)
    \item Fusion control (tokamak MPC/RL with $\varphi$-phased actuators; ICF symmetry ledger)
    \item Manufacturing (perovskite solar with window-8 scheduling and IR phase-lock)
    \item Abiogenesis instrument (LISTEN/LOCK/BALANCE; duplex geometry; templating fixed-point)
    \item AI world-models ($J$-regularized training; certificate-gated data)
  \end{itemize}
\end{itemize}

\vspace{0.5em}
\noindent\textbf{Strategic edge: definitional monopoly + cross-domain synergy.} 
\begin{itemize}[leftmargin=*]
  \item \textbf{Exclusivity leverage}: Lean-verified theorem proves \emph{any} zero-parameter framework deriving observables is definitionally equivalent to Recognition Science. Design-arounds that remain parameter-free are functionally identical to our stack; adding parameters weakens the offering and still may infringe subsidiary claims.
  
  \item \textbf{Functional equivalence via axioms}: The five J-kernel axioms define a unique function; competitors using any cost $F'$ satisfying the axioms use J or a functional equivalent, thus infringe. This imports functional claiming doctrine: approximations, re-parameterizations, or "improvements" that preserve axiom compliance are covered.
  
  \item \textbf{Cross-domain enforcement synergy}: Same certificate schemas, same axiom-probe tests, same compliance API across all verticals (fusion, solar, bio, AI, robotics). Infringement evidence/methods discovered in one domain transfer to others. One enforcement playbook; multiplicative deterrence.
  
  \item \textbf{Verifiable technical effects}: Measurable reductions in alias error (20--60\%), calibration time (3--10×), compute/energy in audits; deterministic units-invariant reproducibility; fail-closed safety via single-inequality acceptance; provable absence of long-memory drift via neutrality.
  
  \item \textbf{Standards-essential positioning}: Zero-parameter calibration natural for NIST/BIPM/ISO; proof-backed acceptance natural for FDA/aerospace model-risk governance; certificate transparency natural for AI safety/provenance. Likely SEP/FRAND candidate in multiple verticals.
\end{itemize}

\vspace{0.5em}
\noindent\textbf{Why patent the architecture (not just applications).} 
\begin{itemize}[leftmargin=*]
  \item \textbf{Definitional monopoly}: The foundation claims cover \emph{the only} zero-parameter physical computing substrate. Competitors cannot stay parameter-free without using our stack (exclusivity theorem); adding parameters concedes technical superiority and market position.
  
  \item \textbf{Cross-domain reach}: One foundation license gates \emph{all} zero-parameter implementations: metrology, fusion, solar, bio, AI, robotics, power. Domain-specific patents stack on top, creating a licensing pyramid with foundation as the unavoidable base layer.
  
  \item \textbf{Enforcement clarity and synergy}: Infringement in \emph{any} domain (detected via axiom probes, neutrality observables, certificate schemas, route identities) provides methods/evidence for \emph{all} domains. One enforcement team; multiplicative deterrence.
  
  \item \textbf{Self-documenting violations}: Certificates, logs, and telemetry are required for operation; they preserve machine-verifiable traces (J-kernel signatures, neutrality block-sums, route equalities, gate identities). Discovery yields smoking-gun evidence.
  
  \item \textbf{Standards-essential trajectory}: Zero-parameter calibration, proof-backed provenance, and units-invariant reproducibility align with NIST/BIPM/ISO (metrology), FDA/FAA (model risk), and AI safety frameworks. SEP/FRAND positioning captures value while enabling broad adoption.
\end{itemize}

\vspace{0.5em}
\noindent\textbf{Commercial theses (instantiations, not exhaustive).} 
\begin{itemize}[leftmargin=*]
  \item \textbf{Metrology/calibration}: Reality Bridge OS for instruments (spectroscopy, timing, imaging); SDKs and firmware IP for neutrality scheduling, anchor-free scale fixing, and certificate-gated acceptance. \emph{First instantiation; beachhead for foundation licensing.}
  
  \item \textbf{Fusion energy}: Tokamak/stellarator MPC/RL controllers with $\varphi$-phased actuators and ledger objectives; ICF pulse shapers with $\varphi$-spaced sub-pulses and symmetry-ledger optimization. \emph{High-value vertical; safety/shot-cost advantages.}
  
  \item \textbf{Solar manufacturing}: Perovskite-on-silicon with window-8 neutral scheduling, IR phase-lock/detune near $724~\mathrm{cm}^{-1}$, and rate-balance at interfaces. \emph{Enables >30\% module efficiency with decade-scale stability; certification QC gates.}
  
  \item \textbf{Synthetic biology / abiogenesis}: LISTEN/LOCK/BALANCE instrument for templating with $J$-minimization forcing duplex geometry; metabolic closure via viability inequality. \emph{Chemistry-agnostic life protocols; IP + regulatory barriers.}
  
  \item \textbf{AI world-models}: Physically-constrained pretraining with $J$-regularized losses, eight-tick temporal batching, and certificate-gated data curation. \emph{Reduces hallucinations, improves OOD robustness; provenance for model-risk governance.}
  
  \item \textbf{Multi-actuator control}: Robotics, power electronics, beamlines using $\varphi$-scheduler with compliance API and ledger optimization. \emph{Plug-and-play IP core; cross-interference reduction.}
\end{itemize}

\vspace{0.5em}
\noindent\textbf{Jurisdictional posture.} Foundation claims (Family 0) are framed as computer-implemented methods/systems with hardware embodiments (J-kernel cores, timing circuits, certificate storage), firmware primitives ($\varphi$-scheduler, neutrality enforcement), and cloud services (certificate issuance, registry). Domain instantiations add vertical-specific control loops and observables. All claims tied to quantified technical effects (alias error, time, resource, reproducibility) to satisfy US §101 and EPO Art. 52 "technical effect" requirements.

\tocsection{2) Invention Overview}
\subsection*{2.1 Problem}
Calibration and validation in instruments and simulators rely on external anchors, tuned parameters, and post-hoc statistics. This leads to:
\begin{itemize}[leftmargin=*]
  \item \emph{Non-deterministic scale fixing}: dependence on artifacts/standards and manual knobs.
  \item \emph{Alias/phase errors}: window misalignment and uncontrolled commits inflate error and bias.
  \item \emph{Unverifiable claims}: no machine-checkable provenance that constraints were satisfied.
  \item \emph{Regulatory friction}: reproducibility and audit burdens across labs and pipelines.
  \item \emph{AI world-model drift}: synthetic/sim data that violates invariants degrades training.
\end{itemize}

\subsection*{2.2 Solution (Reality Bridge Calibration OS + Certificate Engine)}
The system provides parameter-free, deterministic calibration and audit gating:
\begin{enumerate}[leftmargin=*]
  \item \textbf{Units-quotient factorization}: factor displays through a units quotient so dimensionless content is invariant.
  \item \textbf{AbsoluteLayer selection}: solve \emph{UniqueCalibration \(\wedge\) MeetsBands} to pick a unique absolute layer satisfying all gate identities simultaneously; emit \(\tau_0,\,\ell_0,\,c,\,\hbar,\,\lambda_{\mathrm{rec}},\,G\) without external anchors.
  \item \textbf{Dual-route equalization (K-identities)}: enforce equality of independent routes (e.g., \(\tau_{\mathrm{rec}}/\tau_0 = \lambda_{\mathrm{kin}}/\ell_0\)); \emph{no layer mixing} (Planck vs IR) is permitted.
  \item \textbf{Eight-tick neutrality scheduling}: align commits to 8-tick neutral windows and minimize a block-sum of \(J\) to suppress alias/phase errors.
  \item \textbf{Single-inequality audit}: accept/reject via a combined-uncertainty inequality on route agreement; fail closed.
  \item \textbf{Certificate engine}: execute machine-checkable proofs and attach certificate bundles (route identities, neutrality, units-quotient, gate separation) to each run; gate execution and data ingest on passing certificates.
\end{enumerate}

\subsection*{2.3 Foundations (forced chain)}
The invention is grounded in a necessity chain that fixes structure with zero adjustable parameters:
\begin{align*}
\text{Recognition} &\Rightarrow \text{Ledger (double-entry)} \\
 &\Rightarrow \; \text{Unique convex symmetric cost } J(x)=\tfrac12(x+1/x)-1 \\
 &\Rightarrow \; \varphi=\tfrac{1+\sqrt{5}}{2} \text{ (unique positive root of } x^2=x+1) \\
 &\Rightarrow \; \text{Eight-tick minimal period } (2^D,\; D=3\Rightarrow 8) \\
 &\Rightarrow \; \text{Gate identities (e.g., K-identities; Planck gate } (c^3\lambda_{\mathrm{rec}}^2)/(\hbar G)=\text{const}).
\end{align*}
These constraints enable AbsoluteLayer selection and the dual-route equalization used in calibration and audit.

\subsection*{2.4 Technical effects}
\begin{itemize}[leftmargin=*]
  \item \textbf{Alias error reduction}: eight-tick neutrality and \(J\)-minimization reduce alias/phase artifacts measurably.
  \item \textbf{Deterministic reproducibility}: units-invariant displays with a unique absolute layer across environments.
  \item \textbf{Fail-closed safety}: single-inequality acceptance prevents silent drift; explicit tolerance with covariance.
  \item \textbf{Resource efficiency}: shorter calibration time and reduced compute/energy in audit loops.
\end{itemize}

\subsection*{2.5 Machine verification and evidentiary posture}
The certificate engine executes formal proofs (Lean 4) for identities and policies (units-quotient factorization, K-identities, neutrality, gate separation), producing hash-addressed artifacts tied to each run. Certificates, logs, and thresholds form a reproducible evidence pack for compliance and enforcement.

\subsection*{2.6 Scope of embodiments}
\begin{itemize}[leftmargin=*]
  \item \emph{Firmware/IP cores}: timing/strobing neutrality windows and acceptance gating on FPGAs/SoCs.
  \item \emph{SDKs}: C/C++/Rust/Python libraries for instruments and simulators.
  \item \emph{Cloud service}: Calibrate\,+\,Certify API with certificate issuance and registry.
  \item \emph{Simulator plugins}: reality-consistent mode for digital twins and training data pipelines.
\end{itemize}

\subsection*{2.7 Policy constraints}
The OS enforces \emph{non-mixing of layers} (Planck vs IR) and \emph{zero-parameter} provenance; dual routes must agree within combined uncertainty. Execution and data acceptance are gated by passing certificates.

\tocsection{3) Commercial Vision and Products}
\subsection*{3.1 Product forms}
\begin{itemize}[leftmargin=*]
  \item \textbf{Libraries/SDKs}: C/C++/Rust/Python packages implementing units-quotient factorization, AbsoluteLayer selection, K-route equalization, eight-tick neutrality, and single-inequality audits.
  \item \textbf{Firmware IP cores}: FPGA/SoC blocks for timing/strobing and neutrality window scheduling with certificate-gated I/O.
  \item \textbf{Cloud service (Calibrate\,+\,Certify)}: API for remote calibration runs, certificate issuance, registry/search, and policy enforcement.
  \item \textbf{Appliance}: Rack unit for labs (PTP/White-Rabbit timing, spectroscopy/imaging front-ends) delivering anchor-free calibration and proof-backed acceptance.
  \item \textbf{Simulator plugins}: ``Reality-consistent'' mode for digital twins/physics engines (e.g., Omniverse/Unreal/Ansys/COMSOL bridges) with proof bundles attached to runs.
\end{itemize}

\subsection*{3.2 Primary customers and verticals}
\begin{itemize}[leftmargin=*]
  \item \textbf{Metrology/standards}: NMIs (NIST/BIPM), calibration labs.
  \item \textbf{Semiconductor tools}: lithography, metrology, inspection.
  \item \textbf{Medical imaging/spectroscopy}: MRI/CT, mass/IR/optical spectroscopy.
  \item \textbf{Aerospace/defense timing}: navigation, GNSS alternatives, synchronized sensing.
  \item \textbf{Autonomy (lidar/ADAS)}: multi-sensor alignment with proof-gated acceptance.
  \item \textbf{Quantum labs/telco sync}: coherence/timing with neutrality scheduling.
  \item \textbf{Regulated model risk}: pharma/med-device/avionics model governance.
\end{itemize}

\subsection*{3.3 AI world-model applications}
\begin{itemize}[leftmargin=*]
  \item \textbf{Physically constrained pretraining}: \(J\)-regularization, eight-tick temporal batching, units-invariant losses.
  \item \textbf{Data curation/acceptance}: certificate-gated ingestion; reject samples violating route identities/neutrality/policy.
  \item \textbf{Simulator-in-the-loop RL}: route-equalized, layer-consistent rollouts for faster convergence and robustness.
  \item \textbf{Provenance}: cryptographic linkage of datasets/models to proof bundles for auditability and safety.
\end{itemize}

\subsection*{3.4 Go-to-market}
\begin{itemize}[leftmargin=*]
  \item \textbf{Beachhead}: spectroscopy/timing toolchains with fast, quantified value (alias error reduction, time-to-calibrate).
  \item \textbf{OEM/partner channels}: instrument vendors; simulator platforms; hyperscale marketplaces.
  \item \textbf{Packaging}: open-core SDK + enterprise certificates; on-prem/appliance for regulated labs; cloud API for pipelines.
\end{itemize}

\subsection*{3.5 Licensing and pricing}
\begin{itemize}[leftmargin=*]
  \item Per-device royalty (firmware IP), per-run certificate fee (cloud), enterprise seats (SDK), simulator/plugin royalties.
  \item Compliance bundles (certificate registry, retention, audit trails) for regulated sectors.
\end{itemize}

\subsection*{3.6 Measurable value (KPIs)}
\begin{itemize}[leftmargin=*]
  \item Alias error reduction via eight-tick neutrality (e.g., \(20\!{-}\!60\%\)), calibration time reductions (\(3\!{-}\!10\times\)).
  \item Units-invariant reproducibility; true/false acceptance rates under single-inequality audits.
  \item Compute/energy reductions in audit loops; policy-driven fail-closed safety.
\end{itemize}

\subsection*{3.7 Standards and certification}
\begin{itemize}[leftmargin=*]
  \item ``Reality-Certified'' marks; verification profiles; public certificate transparency registry.
  \item Engagement with NIST/BIPM/ISO; reference test suites and conformance levels.
\end{itemize}

\tocsection{4) Patentable Subject-Matter Positioning}
\subsection*{4.1 Jurisdictional framing}
\begin{itemize}[leftmargin=*]
  \item \textbf{US (35~U.S.C.~\S101)}: Claim computer-implemented \emph{instrument/simulator control}, \emph{acceptance workflows}, \emph{firmware timing primitives}, and \emph{certificate-gated operation}. Tie claims to concrete improvements (alias error, calibration time, compute/energy, safety).
  \item \textbf{EPO (Art.~52 EPC, ``math as such'')}: Emphasize \emph{technical effect}---reduced aliasing through eight-tick neutrality, deterministic calibration via AbsoluteLayer, fail-closed audits via single-inequality, resource reductions.
  \item \textbf{JP/CN}: Include \emph{hardware embodiments} (FPGA/SoC timing cores), factory-calibration integration, on-device gating and memory/logging.
\end{itemize}

\subsection*{4.2 Claiming patterns and anchors}
\begin{itemize}[leftmargin=*]
  \item \textbf{Method + System + Medium}: For each family (Calibration OS, dual-route audit, eight-tick scheduler, certificate engine, gate policy, pattern primitives), draft method claims paired with system and non-transitory computer-readable medium claims.
  \item \textbf{Device/simulator control anchors}: Actuator timing, strobe/commit scheduling, route-equalization control loops, gating signals (accept/reject), and logging.
  \item \textbf{Firmware/cloud split}: Firmware IP for neutrality timing; cloud API for certificate issuance and audit registry; both tied to technical metrics.
\end{itemize}

\subsection*{4.3 Technical effects and metrics}
\begin{itemize}[leftmargin=*]
  \item \textbf{Alias error reduction}: quantify reduction (e.g., \(20\!{-}\!60\%\)) with eight-tick neutrality and \(J\)-block minimization.
  \item \textbf{Calibration time}: single-inequality acceptance yields \(3\!{-}\!10\times\) speedups.
  \item \textbf{Reproducibility}: units-invariant displays across unit choices via AbsoluteLayer selection.
  \item \textbf{Safety}: fail-closed acceptance; explicit tolerance via combined-uncertainty with correlation.
  \item \textbf{Resource use}: compute/energy reductions in audit loops; memory/log bandwidth profiles.
\end{itemize}

\subsection*{4.4 Embodiments to avoid ``math as such''}
\begin{itemize}[leftmargin=*]
  \item \emph{Hardware loops}: FPGA/SoC timing cores implementing neutral windows, K-route equalization, and gating outputs.
  \item \emph{Real-time control}: On-instrument scheduling of commits, parameter updates (integration time, gate width, strobe phase).
  \item \emph{Memory/logging}: Non-volatile storage of certificate bundles and audit traces; cryptographic hashes linking runs to proofs.
  \item \emph{Networked service}: API that denies uncertified operations; certificate transparency registry for compliance.
\end{itemize}

\subsection*{4.5 Example claim element mapping}
\begin{itemize}[leftmargin=*]
  \item \textbf{AbsoluteLayer}: solve simultaneous gate identities; emit \(\tau_0,\,\ell_0,\,c,\,\hbar,\,\lambda_{\mathrm{rec}},\,G\); no external anchors.
  \item \textbf{Dual-route audit}: compute \(\lambda_{\mathrm{kin}}\), \(\lambda_{\mathrm{rec}}\); compute \(u_{\mathrm{comb}}\) with correlation; test \(\lvert\lambda_{\mathrm{kin}}-\lambda_{\mathrm{rec}}\rvert/\lambda_{\mathrm{rec}} \le k\,u_{\mathrm{comb}}\); drive accept/reject.
  \item \textbf{Eight-tick scheduler}: Gray-aligned 8-tick commits; enforce \(\sum_{t=1}^{8}\delta_t=0\); minimize block-sum of \(J\).
  \item \textbf{Certificate engine}: run Lean proofs, generate hash-addressed bundles; gate operation/ingest on passing proofs.
  \item \textbf{Gate separation}: classify by layer (Planck vs IR); prohibit mixing; verify cross-identity dimensionlessly.
\end{itemize}

\subsection*{4.6 Evidence, enablement, and best mode}
\begin{itemize}[leftmargin=*]
  \item \textbf{Formal proofs}: Lean 4 artifacts for cost uniqueness (T5), eight-tick minimality, neutrality, units-quotient factorization, K-identities, policy.
  \item \textbf{Benchmarks}: protocols and results for alias error, calibration time, acceptance rates, reproducibility, compute/energy.
  \item \textbf{Best mode}: reference SDK APIs, firmware block diagrams, certificate JSON schemas, and test harnesses.
\end{itemize}

\subsection*{4.7 Risks and mitigations}
\begin{itemize}[leftmargin=*]
  \item \textbf{\S101 / math as such}: claim concrete device/simulator control, firmware primitives, gating; include quantified deltas.
  \item \textbf{\S112 enablement/definiteness}: provide implementation details, thresholds, schemas, test protocols.
  \item \textbf{Novelty/obviousness}: emphasize integrated stack (units-quotient + AbsoluteLayer + K-identities + single-inequality + neutrality + formal certificates) as non-obvious combination.
\end{itemize}

\subsection*{4.8 Continuations and divisionals}
\begin{itemize}[leftmargin=*]
  \item Vertical continuations: spectroscopy, timing, imaging, lidar/ADAS, quantum.
  \item Split firmware vs cloud embodiments; maintain method/system/medium across jurisdictions.
\end{itemize}

\tocsection{5) Portfolio Architecture (Claim Families)}
\noindent The portfolio is organized into coordinated claim families. Each family targets a cohesive technical capability with method, system, and non-transitory medium claims, plus firmware/cloud embodiments and measurable effects.

\subsection*{Family 0: Recognition Computing Architecture (Foundation)}
\textbf{Focus}: The five-layer, parameter-free architecture (recognition ratios; axiomatically-unique J-ledger; $\varphi$-timed neutrality; certificate engine; absolute-layer selection).\\
\textbf{Independents}: domain-agnostic method/system/medium claims covering the stack; compliance API; certificate bundling and registry.\\
\textbf{Dependents (J-kernel axiom enumeration)}:
\begin{itemize}[leftmargin=1.5em]
  \item Axiom (i): symmetry $F(x)=F(1/x)$ enforced via black-box probe (F(x) vs F(1/x) test on sampled ratios).
  \item Axiom (ii): unit normalization $F(1)=0$ verified at identity ratio.
  \item Axiom (iii): strict convexity on $\mathbb{R}_{>0}$ tested via midpoint inequality on logged samples.
  \item Axiom (iv): calibration $\tfrac{d^2}{dt^2}(F(e^t))\big|_{t=0}=1$ verified via local quadratic fit near $t=0$.
  \item Axiom (v): cosh-addition identity tested on structured ratio pairs.
  \item Functional equivalence claim: any $F'$ satisfying (i)--(v) is identical to $J(x)=\tfrac12(x+1/x)-1$; use of $F'$ constitutes infringement via doctrine of equivalents.
  \item Hardware J-kernel core (fixed-point/log-domain unit; block-sum accumulator; certificate interface; $\varphi$-sync timing).
  \item Timing primitives: $\varphi$-commensurate window generator; neutrality enforcer; Gray-aligned eight-tick scheduler.
  \item Axiom-probe diagnostics: structured stimulus-response tests to elicit axiom signatures from black-box APIs.
  \item Absolute-layer solvers: simultaneous gate-identity solve with tolerance bands and convergence criteria.
  \item SDK surface: recognition-ratio formation; ledger computation; certificate hooks; compliance API.
\end{itemize}

\subsection*{Family A: Framework Exclusivity + Verification Method}
\textbf{Focus}: Machine-implemented method to establish definitional equivalence for any \emph{zero-parameter} framework deriving observables and assert coverage.\\
\textbf{Independents}: formalize candidate \(F\); verify zero parameters and observables; execute equivalence theorem; issue certificate of equivalence; gate claims.\\
\textbf{Dependents}: bi-interpretability maps; inputs/outputs schema; audit logs/hash chain.

\subsection*{Family B: Reality Bridge Calibration OS (Core)}
\textbf{Focus}: Units-quotient factorization; AbsoluteLayer (UniqueCalibration \(\wedge\) MeetsBands); K-route equalization; layer non-mixing; anchor-free \(\tau_0,\,\ell_0,\,c,\,\hbar,\,\lambda_{\mathrm{rec}},\,G\).\\
\textbf{Dependents}: tolerance bands; convergence criteria; device control hooks (actuators/strobes); logs.

\subsection*{Family C: Dual-Route Invariant Equalization + Single-Inequality Audit}
\textbf{Focus}: Compute \(\lambda_{\mathrm{kin}}\), \(\lambda_{\mathrm{rec}}\); compute \(u_{\mathrm{comb}}\) (with correlation); test \(\lvert\Delta\rvert/\lambda_{\mathrm{rec}}\le k\,u_{\mathrm{comb}}\); fail-closed accept/reject.\\
\textbf{Dependents}: adaptive \(k\) by context; correlation estimation; device interlocks.

\subsection*{Family D: AbsoluteLayer Selector (Anchor-Free Scale Fixing)}
\textbf{Focus}: Solve simultaneous gate identities to select a unique absolute layer; derive scales/constants without external anchors.\\
\textbf{Dependents}: multi-gate consensus; fallback policy when an identity is out-of-band; certificate entries.

\subsection*{Family E: Eight-Tick Neutral Scheduler (Alias Suppression)}
\textbf{Focus}: Gray-aligned 8-tick commits; enforce neutrality (\(\sum_{t=1}^{8}\delta_t=0\)); minimize block-sum of \(J\); output phase-invariant measurements with reduced alias error.\\
\textbf{Dependents}: scheduler variants; hardware timing primitives; KPI thresholds.

\subsection*{Family F: Recognition-Operator Commit Controller (\(\hat{R}\) Governance)}
\textbf{Focus}: Path-cost \(C\) from \(J\); commit when \(C\ge1\); control integration time, gate width, strobe phase; log commit/erase energetics.\\
\textbf{Dependents}: Landauer-bounded energy accounting; hysteresis; safety states.

\subsection*{Family G: Certificate Engine (Formal Audit)}
\textbf{Focus}: Execute Lean proofs for route identities, neutrality, units-quotient, gate separation; generate hash-addressed bundles; gate operation and ingestion on passing proofs.\\
\textbf{Dependents}: certificate schemas; transparency registry; revocation; on-device vs cloud flow.

\subsection*{Family H: Gate-Separation Policy Enforcement}
\textbf{Focus}: Classify operations by layer (Planck vs IR); enforce disjoint anchors; verify cross-identity dimensionlessly; block mixed routes.\\
\textbf{Dependents}: policy language; exception handling; audit overrides with risk flags.

\subsection*{Family I: Pattern-Measurement Primitives (Firmware/Library)}
\textbf{Focus}: \texttt{sumFirst8}, \texttt{blockSumAligned8}, \texttt{observeAvg8} primitives for deterministic averaging and invariants in firmware/SDK.\\
\textbf{Dependents}: vectorized implementations; latency bounds; precision/quantization regimes.

\subsection*{Family J: Self-Similarity Forcing \(\varphi\) (Scale Constant Derivation)}
\textbf{Focus}: Parameter-free self-similar recursion \(\Rightarrow\) \(\varphi=(1+\sqrt{5})/2\); apply \(\varphi\) to calibrations/constants in device/simulator workflows.\\
\textbf{Dependents}: detection of self-similarity; stability margins; domain-specific mappings.

\subsection*{Family K: Recognition-Ledger-Cost Integration (Necessity Stack)}
\textbf{Focus}: Observables \(\Rightarrow\) recognition; discrete+conservation \(\Rightarrow\) ledger; unique \(J\); \(\varphi\); eight-tick; gate identities; integrated device/simulator control.\\
\textbf{Dependents}: modular proofs; sequencing constraints; failure diagnostics.

\subsection*{Per-Family Embodiments and Metrics}
Each family will include: (i) method/system/medium claims; (ii) firmware/cloud embodiments; (iii) KPIs (alias error, calibration time, acceptance rates, reproducibility, compute/energy); (iv) certificate hooks and logs for enforcement.

\tocsection{6) Representative Independent Claim Skeletons}
\noindent The following representative independent claim skeletons illustrate the scope per family. Each may be paired with a corresponding \emph{system} claim (apparatus configured to execute the steps) and a \emph{non-transitory computer-readable medium} claim (instructions that cause execution of the steps).

\subsection*{6.0 Foundation (Recognition Computing Architecture)}
\noindent\textbf{Method.} A method of controlling or measuring a physical system, comprising: forming dimensionless recognition ratios $r_i=y_i/y_i^\star$; computing a ledger $L=\sum_i w_i\,F(r_i)$ where $F:(0,\infty)\to\mathbb{R}$ is the unique function satisfying (i) $F(x)=F(1/x)$; (ii) $F(1)=0$; (iii) strict convexity on $\mathbb{R}_{>0}$; (iv) $\tfrac{d^2}{dt^2}(F(e^{t}))\big|_{t=0}=1$; and (v) a cosh-addition identity; partitioning control into $\varphi$-commensurate windows with neutrality constraints; executing machine-checkable proofs of invariants; selecting an absolute layer by solving simultaneous dimensionless identities; and authorizing operation or data acceptance only upon certificate satisfaction.\\
\noindent\textbf{System/Medium.} A system (and non-transitory medium) configured to perform the method.

\subsection*{6.1 Exclusivity/Verification (Family A)}
\noindent\textbf{Method.} A computer-implemented method of verifying coverage of a zero-parameter physics framework, comprising: formalizing a candidate framework \(F\); mechanically verifying that \(F\) has zero adjustable parameters and derives observables; executing a definitional-equivalence proof to establish equivalence to a reference framework; producing a certificate of equivalence; and gating claims or operation based on the certificate.

\subsection*{6.2 Calibration OS (Family B)}
\noindent\textbf{Method.} A method for parameter-free calibration of an instrument or simulator, comprising: factoring displays through a units quotient; selecting a unique absolute layer by solving simultaneous dimensionless gate identities; equalizing independent routes; prohibiting layer mixing; and emitting calibrated scales and constants \(\tau_0,\,\ell_0,\,c,\,\hbar,\,\lambda_{\mathrm{rec}},\,G\) without external anchors.

\subsection*{6.3 Dual-Route Audit (Family C)}
\noindent\textbf{Method.} A method for acceptance testing of a calibrated system, comprising: computing dual invariants including \(\lambda_{\mathrm{kin}}\) and \(\lambda_{\mathrm{rec}}\); computing a combined uncertainty with correlation; testing an acceptance inequality \(\lvert\lambda_{\mathrm{kin}}-\lambda_{\mathrm{rec}}\rvert/\lambda_{\mathrm{rec}} \le k\,u_{\mathrm{comb}}\); and issuing a fail-closed accept/reject control signal to the system.

\subsection*{6.4 Eight-Tick Neutral Scheduler (Family E)}
\noindent\textbf{Method.} A method for reducing alias error in measurements, comprising: scheduling commits in Gray-aligned eight-tick windows; enforcing a neutrality constraint over each window; minimizing a block-sum of a convex symmetric cost \(J(x)=\tfrac12(x+1/x)-1\); and outputting a phase-invariant measurement with reduced alias error relative to a baseline.

\subsection*{6.5 Recognition-Operator Commit Controller (Family F)}
\noindent\textbf{Method.} A method for controlling commits in a measurement pipeline, comprising: computing a path cost from \(J\); triggering a commit when the cost exceeds a collapse threshold; adjusting one or more control parameters including integration time, gate width, and strobe phase; and logging commit/erase energetics for audit.

\subsection*{6.6 Certificate Engine (Family G)}
\noindent\textbf{Method.} A method for proof-backed calibration audit, comprising: executing machine-checkable proofs for route identities, neutrality, units-quotient factorization, and gate separation; generating a hash-addressed certificate bundle bound to run identifiers; and gating device or simulator operation and data ingestion on the presence of a valid certificate.

\subsection*{6.7 Gate-Separation Policy (Family H)}
\noindent\textbf{Method.} A method of enforcing calibration policy, comprising: classifying operations by layer; prohibiting cross-layer mixing; verifying cross-identities dimensionlessly; and blocking or flagging operations that violate policy.

\subsection*{6.8 Pattern-Measurement Primitives (Family I)}
\noindent\textbf{Method.} A method for deterministic periodic measurement, comprising: applying periodic window primitives including sum-first-eight, block-sum-aligned-eight, and observed-average-eight; certifying invariants under the primitives; and guaranteeing that averaged observables equal a declared invariant for acceptance.

\subsection*{6.9 System and Medium Pairing}
\noindent For each method above: (i) a \textbf{system} comprising one or more processors, sensors/actuators, memory, and timing circuitry configured to perform the method; and (ii) a \textbf{non-transitory computer-readable medium} storing instructions that, when executed by one or more processors, cause performance of the method.

\subsection*{6.10 J-Kernel Hardware Core (Dependent to Families 0/B/E/G)}
\noindent\textbf{System.} A hardware core comprising: (i) a streaming fixed-point/log-domain unit that computes $J(e^{u})=\cosh(u)-1$ (or equivalently $J(x)=\tfrac12(x+1/x)-1$) for positive inputs with bounded latency; (ii) an accumulator for block-sum neutrality over windowed segments; (iii) a certificate interface that exposes aggregated ledger values and neutrality flags to a compliance API; and (iv) timing inputs synchronized to $\varphi$-commensurate windows.

\tocsection{7) Evidence, Enablement, and Best Mode}
\subsection*{7.1 Formal proof artifacts (enablement and definiteness)}
\noindent Machine-checkable Lean~4 artifacts provide enablement, definiteness, and enforceability:
\begin{itemize}[leftmargin=*]
  \item \textbf{Cost uniqueness (T5)}: \texttt{IndisputableMonolith/CostUniqueness.lean} (with \texttt{Cost/JcostCore.lean}, \texttt{Cost/FunctionalEquation.lean}).
  \item \textbf{Eight-tick minimality and neutrality}: measurement primitives in \texttt{Measurement/*} and references in \texttt{URCAdapters/*}.
  \item \textbf{Units-quotient/bridge factorization; K-identities}: \texttt{URCAdapters} and \texttt{Constants/*} modules.
  \item \textbf{Exclusivity/verification scaffold}: \texttt{Verification/Exclusivity/NoAlternatives.lean}, \texttt{URCGenerators/ExclusivityCert.lean}.
  \item \textbf{Policy and gating}: proofs for gate separation, neutrality enforcement, and acceptance inequalities in adapter/certificate modules.
\end{itemize}

\subsection*{7.2 Benchmarks and KPIs (technical effects)}
\noindent Protocols and datasets to quantify improvements:
\begin{itemize}[leftmargin=*]
  \item \textbf{Alias error reduction}: compare baseline vs eight-tick neutrality + \(J\)-block minimization; report percent reduction (target \(20\!{-}\!60\%\)).
  \item \textbf{Calibration time}: measure speedups with single-inequality acceptance (target \(3\!{-}\!10\times\)).
  \item \textbf{Reproducibility}: units-invariance across unit choices via AbsoluteLayer selection; variance/bias metrics.
  \item \textbf{Acceptance quality}: true/false acceptance rates under combined-uncertainty tolerance; correlation handling.
  \item \textbf{Resource use}: compute/energy consumed in audit loops; memory/log bandwidth.
\end{itemize}

\subsection*{7.3 Best mode (reference implementations)}
\noindent Reference embodiments and implementation details:
\begin{itemize}[leftmargin=*]
  \item \textbf{SDK API surface}: routines for units-quotient factorization, AbsoluteLayer solve, route equalization, neutrality scheduling, and single-inequality audit; language bindings (C/C++/Rust/Python).
  \item \textbf{Firmware blocks}: FPGA/SoC timing cores for Gray-aligned eight-tick scheduling, interlock/gating outputs, timestamping, and precision/latency envelopes.
  \item \textbf{Certificate schemas}: JSON with run IDs, invariant values, uncertainty model, proof hashes, timestamps, device metadata.
  \item \textbf{Adapter flows}: cloud/on-prem API routes for certificate issuance, verification hooks (\#eval or equivalent), and registry integration.
\end{itemize}

\subsection*{7.4 Data and certificate packaging}
\noindent Packaging for reproducibility and enforcement:
\begin{itemize}[leftmargin=*]
  \item \textbf{Bundles}: proof artifacts, parameters, logs, and outcome (accept/reject) as a signed archive; content-addressed by cryptographic hash.
  \item \textbf{Transparency registry}: publication of certificate headers for audit without revealing sensitive payloads; revocation lists for superseded/invalidated artifacts.
  \item \textbf{Chain-of-custody}: timestamping and signature keys for instruments/simulators; linkage to dataset/model artifacts in AI pipelines.
\end{itemize}

\subsection*{7.5 Reproducibility and CI}
\noindent Continuous integration for calibration/audit reproducibility:
\begin{itemize}[leftmargin=*]
  \item \textbf{CI pipelines}: deterministic builds of SDK/firmware; regression tests for invariants and acceptance inequalities.
  \item \textbf{Notebooks}: reproducible notebooks demonstrating benchmarks and certificate generation; pinned versions and seeds.
  \item \textbf{Golden datasets}: curated inputs for alias/error tests; public checksums.
\end{itemize}

\subsection*{7.6 Compliance and standards mapping}
\noindent Mapping to metrology/regulatory frameworks:
\begin{itemize}[leftmargin=*]
  \item \textbf{Metrology (NIST/BIPM/ISO)}: AbsoluteLayer and units-quotient factorization for anchor-free calibration; certificate-based acceptance for traceability.
  \item \textbf{Regulatory (FDA/aerospace)}: proof-gated operation and acceptance; audit trails with cryptographic attestation.
  \item \textbf{Safety}: fail-closed policy and interlocks; exception logging with risk flags.
\end{itemize}

\subsection*{7.7 Example artifact index}
\noindent Cross-reference of modules and artifacts for inclusion in appendices:
\begin{itemize}[leftmargin=*]
  \item \texttt{IndisputableMonolith/CostUniqueness.lean}, \texttt{Cost/JcostCore.lean}, \texttt{Cost/FunctionalEquation.lean} (T5).
  \item \texttt{Measurement/*}, neutrality lemmas; eight-tick scheduling references.
  \item \texttt{URCAdapters/*}, \texttt{Constants/*} (units-quotient, K-identities, gate identities).
  \item \texttt{Verification/Exclusivity/NoAlternatives.lean}, \texttt{URCGenerators/ExclusivityCert.lean} (exclusivity and certificates).
  \item Certificate JSON schema and example signed bundle; CI scripts and notebook references.
\end{itemize}

\tocsection{8) Filing Plan and Timeline}
\subsection*{8.1 Provisional filings (0--90 days)}
\begin{itemize}[leftmargin=*]
  \item \textbf{Provisional \#1 (0--30 days)}: Core claims --- Calibration OS (AbsoluteLayer, K-route equalization, non-mixing), Dual-route audit (single inequality), Eight-tick scheduler (alias suppression). Include executable proof references and initial benchmarks.
  \item \textbf{Provisional \#2 (60--90 days)}: Add-ons --- Recognition-operator commit controller, Certificate engine and transparency registry, Gate-separation policy, Pattern-measurement primitives. Include firmware block diagrams and cloud API schemas.
\end{itemize}

\subsection*{8.2 PCT filing (12 months)}
\noindent Consolidate provisionals; harmonize claims across method/system/medium and firmware/cloud embodiments; include benchmark deltas and certificate exemplars. Designate US as International Searching Authority; request early written opinion.

\subsection*{8.3 National phase (30/31 months)}
\begin{itemize}[leftmargin=*]
  \item \textbf{US}: emphasize device/simulator control, acceptance workflows, measurable deltas; maintain broad independent claims with dependent technical metrics.
  \item \textbf{EPO}: stress technical effects (alias reduction, deterministic calibration, fail-closed audits, resource reductions); include hardware embodiments.
  \item \textbf{JP/CN}: include FPGA/SoC claims, factory calibration flows, on-device gating/storage.
\end{itemize}

\subsection*{8.4 Continuations/divisionals strategy}
\begin{itemize}[leftmargin=*]
  \item Maintain parallel continuations for verticals: spectroscopy, timing, imaging, lidar/ADAS, quantum, simulator plugins, AI world-model pipelines.
  \item Divide firmware vs cloud embodiments where beneficial; preserve method/system/medium coverage.
\end{itemize}

\subsection*{8.5 Evidence pack schedule}
\begin{itemize}[leftmargin=*]
  \item \textbf{T\,+\,30 days}: initial alias-error and calibration-time benchmarks; certificate bundle examples; SDK API references.
  \item \textbf{T\,+\,60 days}: firmware timing prototype; neutrality scheduler KPIs; transparency registry MVP.
  \item \textbf{T\,+\,90 days}: expanded datasets; reproducibility CI; on-device certificate signing; regulator-facing whitepaper.
\end{itemize}

\subsection*{8.6 Counsel coordination and budget}
\begin{itemize}[leftmargin=*]
  \item Outside counsel brief with technical annex (Lean proof map, KPIs, embodiments); align on \S101/EPO strategy and claim charting.
  \item Budget tranches per milestone (provisionals, PCT, national entries); reserve for continuations and office actions.
\end{itemize}

\subsection*{8.7 Risk gates and mitigations}
\begin{itemize}[leftmargin=*]
  \item Gate filings on availability of at least one quantified technical effect per core claim (alias error, time, acceptance quality) and a certificate exemplar.
  \item Maintain design alternatives (e.g., scheduler variants, audit models) to navigate prosecution.
\end{itemize}

\subsection*{8.8 Deliverables timeline (summary)}
\begin{itemize}[leftmargin=*]
  \item D0--D30: Provisional \#1 package (SDK draft, benchmarks v1, proof refs).
  \item D30--D90: Provisional \#2 package (firmware prototype, registry MVP, expanded proofs/demos).
  \item M12: PCT filing; updated claims and evidence.
  \item M30/31: National phase entries (US/EPO/JP/CN) with tailored claim sets.
\end{itemize}

\tocsection{9) Jurisdictional Tailoring}
\subsection*{9.1 United States (USPTO)}
\begin{itemize}[leftmargin=*]
  \item \textbf{Claim categories}: method, system (apparatus configured to), and non-transitory computer-readable medium (CRM); keep all three per family.
  \item \textbf{\S101 positioning}: emphasize concrete \emph{instrument/simulator control} (scheduling commits; gating actuators; adjusting integration time/gate width/strobe phase) and \emph{measurable improvements} (alias error %, calibration time, compute/energy). Cite MPEP~2106 examples where control and resource reductions qualify.
  \item \textbf{Anchors in claims}: sensors/actuators, timing circuitry, logs, acceptance signals; transformation of device operation under certificate gating.
  \item \textbf{Dependents}: quantitative thresholds (e.g., \(\le k\,u_{\mathrm{comb}}\)), correlation handling, hardware timing primitives, safety interlocks, memory/log schemas.
\end{itemize}

\subsection*{9.2 Europe (EPO)}
\begin{itemize}[leftmargin=*]
  \item \textbf{Art.~52 EPC} (``math as such''): frame as \emph{computer-implemented method of operating an instrument/simulator} with \emph{technical effects}: reduced aliasing via eight-tick neutrality, deterministic calibration (AbsoluteLayer), fail-closed acceptance, resource reductions.
  \item \textbf{Problem-solution}: identify closest prior art (units calibration, Gray timing, audits) and define objective technical problem (e.g., achieving deterministic, anchor-free calibration with reduced alias error and resource use). Show non-obvious integration (units-quotient + AbsoluteLayer + K-identities + single-inequality + neutrality + certificates).
  \item \textbf{Embodiments}: include hardware (FPGA/SoC), on-device gating, memory/logging; avoid purely verification claims detached from control.
  \item \textbf{Dependents}: CPU/memory/energy reduction metrics; latency/throughput improvements; robustness under misalignment.
\end{itemize}

\subsection*{9.3 Japan (JPO)}
\begin{itemize}[leftmargin=*]
  \item \textbf{Form}: ``Information processing device/system'' with storage unit, control unit, timing unit, and communication interface implementing neutrality scheduling and acceptance gating.
  \item \textbf{Hardware emphasis}: detail signal paths, registers, buffers, clocks; provide block diagrams for timing cores and acceptance interlocks.
  \item \textbf{Dependents}: factory calibration flow, on-device certificate signing, event logs; manufacturing-line integration.
\end{itemize}

\subsection*{9.4 China (CNIPA)}
\begin{itemize}[leftmargin=*]
  \item \textbf{Integration claims}: method and apparatus claims tied to production/testing lines; PLC/MCU embodiments for timing/gating; beneficial effects spelled out (accuracy, throughput, energy).
  \item \textbf{Software-hardware linkage}: emphasize cooperative operation between firmware timing cores and higher-level audit controller.
  \item \textbf{Dependents}: device identifiers, calibration traceability fields, on-site revocation/update mechanisms.
\end{itemize}

\subsection*{9.5 Jurisdiction-specific dependent motifs}
\begin{itemize}[leftmargin=*]
  \item \textbf{US}: quantitative KPI constraints; telemetry-based dynamic \(k\); power/thermal envelopes for audit loops.
  \item \textbf{EPO}: resource reduction bounds (CPU cycles, memory, energy); deterministic timing guarantees; failure-mode handling.
  \item \textbf{JP}: explicit storage/control/timing unit interactions; register-level timing windows; safety-class interfaces.
  \item \textbf{CN}: PLC/MCU timing parameters; line-level interlocks; production QA metrics.
\end{itemize}

\subsection*{9.6 Drafting conventions and cautions}
\begin{itemize}[leftmargin=*]
  \item \textbf{US CRM}: include ``non-transitory''; avoid Beauregard pitfalls; ensure executable steps.
  \item \textbf{Multiple dependency}: leverage EPO-friendly multiple dependent claims; simplify in US if needed.
  \item \textbf{Medical/diagnostic}: avoid treatment claims; frame as measurement/control where applicable.
  \item \textbf{Clarity}: define invariants, windows, acceptance inequality, and uncertainty model explicitly.
\end{itemize}

\subsection*{9.7 Prosecution playbook}
\begin{itemize}[leftmargin=*]
  \item \textbf{US \S101}: respond with control-loop mapping and benchmark affidavits; cite technical improvements; amend to foreground hardware/timing primitives if needed.
  \item \textbf{EPO inventive step}: argue synergy of integrated features; present comparative experiments; limit to core combination if required, keep continuations for others.
  \item \textbf{JP/CN}: strengthen hardware blocks and manufacturing integration; provide schematics and timing charts.
\end{itemize}

\tocsection{10) Freedom-to-Operate and Prior Art}
\subsection*{10.1 Landscape review (exemplars)}
\begin{itemize}[leftmargin=*]
  \item \textbf{Units/calibration frameworks}: SI anchor systems; device-specific calibration methods; ratio-based metrology. Typically rely on external anchors and do not enforce a global units-quotient factorization with a unique absolute layer.
  \item \textbf{Timing/Gray-code scheduling}: Gray-code sequences used for enumeration or error reduction; rarely coupled to neutrality constraints, \(J\)-block minimization, and certificate-gated acceptance.
  \item \textbf{Audit/verification}: statistical acceptance tests and digital signatures; lack formal proof-of-invariants integrated into control loops.
  \item \textbf{Physics solvers/simulators}: numerical engines (e.g., COMSOL/Ansys); do not provide zero-parameter provenance or dual-route audit identities.
\end{itemize}

\subsection*{10.2 Differentiators (our combinations)}
\begin{itemize}[leftmargin=*]
  \item \textbf{Integrated stack}: units-quotient factorization \(+\) AbsoluteLayer selection \(+\) K-route equalization \(+\) single-inequality acceptance \(+\) eight-tick neutrality \(+\) formal certificates.
  \item \textbf{Anchor-free scale fixing}: solving simultaneous gate identities to emit \(\tau_0,\,\ell_0,\,c,\,\hbar,\,\lambda_{\mathrm{rec}},\,G\) without external artifacts.
  \item \textbf{Proof-gated operation}: machine-checkable proofs used to gate device/simulator execution and data ingestion.
\end{itemize}

\subsection*{10.3 Potential overlapping art and mitigations}
\begin{itemize}[leftmargin=*]
  \item \textbf{Gray-code timing prior art}: narrow claims to neutrality windowing and \(J\)-block minimization with certificate-gated acceptance; include hardware timing primitives.
  \item \textbf{Calibration/audit suites}: emphasize dual-route identity with combined-uncertainty inequality and non-mixing policy enforcement; include resource/alias improvements.
  \item \textbf{Crypto attestations}: distinguish content (formal proofs of invariants) and control integration vs. mere signing.
\end{itemize}

\subsection*{10.4 Claim carving and fallbacks}
\begin{itemize}[leftmargin=*]
  \item Carve independent claims to core combinations; maintain dependents for individual elements (AbsoluteLayer, single-inequality audit, neutrality scheduler, certificate engine).
  \item Provide firmware and on-device embodiments to anchor technical effect; keep cloud verification as complementary.
\end{itemize}

\subsection*{10.5 Search plan and monitoring}
\begin{itemize}[leftmargin=*]
  \item Pre-PCT prior art search focused on Gray-code timing, calibration audits, units/calibration frameworks, proof-gated systems.
  \item Ongoing monitoring: new filings in metrology, timing, simulator plugins; competitor whitepapers on ``zero-parameter'' or anchor-free calibration.
\end{itemize}

\subsection*{10.6 Design-around analysis}
\begin{itemize}[leftmargin=*]
  \item \textbf{Adding parameters}: competitors may introduce tuned knobs; outside zero-parameter scope but weaker technically; still may infringe OS/audit claims if route identities/neutrality and certificates are used.
  \item \textbf{Avoiding certificates}: claims cover integration where operation is gated by proofs; separate claims protect neutrality scheduling and audit inequality independent of certificates.
  \item \textbf{Alternate scheduling}: include variants (window sizes, multi-window schemes) tied to neutrality constraints and \(J\)-block costs to deny easy workarounds.
\end{itemize}

\tocsection{11) Risk Analysis and Mitigations}
\subsection*{11.1 Patentability risks}
\begin{itemize}[leftmargin=*]
  \item \textbf{Abstract idea / math as such (US \S101 / EPO Art.~52)}: risk that claims are viewed as mathematical methods.
  \item \textbf{Enablement/definiteness (US \S112)}: risk of insufficient implementation detail or ambiguous term scope.
  \item \textbf{Novelty/obviousness}: integration could be challenged as obvious aggregation.
\end{itemize}
\noindent\textbf{Mitigations}: anchor claims in device/simulator control and firmware timing; include quantified technical effects; provide SDK/firmware best-mode details, schemas, and test protocols; emphasize synergistic integration (units-quotient + AbsoluteLayer + K-identities + single-inequality + neutrality + certificates) with comparative data.

\subsection*{11.2 Prior art and prosecution risks}
\begin{itemize}[leftmargin=*]
  \item \textbf{Gray-code/timing} prior art; \textbf{calibration/audit} suites; \textbf{crypto attestations}.
  \item Examiner citations in metrology/timing could narrow neutrality scheduler or audit claims.
\end{itemize}
\noindent\textbf{Mitigations}: differentiate with neutrality constraints, \(J\)-block minimization, dual-route combined-uncertainty acceptance, and proof-gated operation; maintain fallbacks and dependent claims; prepare comparative experiments.

\subsection*{11.3 Enforcement and detectability}
\begin{itemize}[leftmargin=*]
  \item Hidden implementations (closed devices/cloud) may hinder detection.
  \item Standards/process changes may obscure infringement.
\end{itemize}
\noindent\textbf{Mitigations}: require certificate outputs and transparency headers in deployments; design test harnesses to elicit route identities/neutrality; leverage procurement/standards clauses to mandate proof-backed calibration.

\subsection*{11.4 Standards and ecosystem risks}
\begin{itemize}[leftmargin=*]
  \item Standards bodies may resist non-traditional anchor-free calibration.
  \item Competing de facto standards could emerge.
\end{itemize}
\noindent\textbf{Mitigations}: pilot with NMIs and regulated labs; publish verification profiles and open reference tests; offer RAND licensing paths for standards bodies.

\subsection*{11.5 Technical performance risks}
\begin{itemize}[leftmargin=*]
  \item Alias error reductions or time savings vary by modality.
  \item Hardware timing constraints/latency could limit neutrality scheduling.
\end{itemize}
\noindent\textbf{Mitigations}: modality-specific tuning; multiple scheduler variants; hardware IP blocks with guaranteed timing; publish confidence intervals and operating envelopes.

\subsection*{11.6 Proof/engineering risks}
\begin{itemize}[leftmargin=*]
  \item Formal proof debt or fragility across library versions.
  \item Developer availability and onboarding to proof tooling.
\end{itemize}
\noindent\textbf{Mitigations}: pin toolchain versions; maintain proof CI; modularize proofs; provide developer guides and stable APIs for proof invocation.

\subsection*{11.7 IP/process risks}
\begin{itemize}[leftmargin=*]
  \item Open-source contamination; unclear ownership of artifacts and datasets.
  \item Export controls for cryptography or timing tech.
\end{itemize}
\noindent\textbf{Mitigations}: clear licensing (e.g., dual-license where needed); contributor agreements; artifact provenance and access control; export-compliance review.

\subsection*{11.8 Regulatory/data risks}
\begin{itemize}[leftmargin=*]
  \item Privacy/GDPR when certificates are linked to datasets/models.
  \item Safety-critical deployments (medical, avionics) require rigorous QA.
\end{itemize}
\noindent\textbf{Mitigations}: minimize personal data in certificates; hash headers with selective disclosure; ISO/SOC2 processes; validation dossiers and traceable logs.

\subsection*{11.9 Business and litigation risks}
\begin{itemize}[leftmargin=*]
  \item Budget and timeline overruns for multi-jurisdiction filings.
  \item Litigation costs vs licensing revenue.
\end{itemize}
\noindent\textbf{Mitigations}: staged filings; reserve for continuations/office actions; prioritize high-value verticals; mediation-ready evidence packs.

\tocsection{12) Enforcement and Detection}
\subsection*{12.1 Infringement tests (feature-to-claim)}
\begin{itemize}[leftmargin=*]
  \item \textbf{Route equality}: reproduce dual-route invariants (\(\lambda_{\mathrm{kin}},\,\lambda_{\mathrm{rec}}\)); test single-inequality acceptance with combined uncertainty and correlation.
  \item \textbf{Neutrality scheduling}: induce/observe eight-tick neutrality (\(\sum_{t=1}^{8}\delta_t=0\)); measure alias error reduction vs baseline.
  \item \textbf{AbsoluteLayer}: verify gate identities solved simultaneously; confirm anchor-free scales/constants are emitted.
  \item \textbf{Gate separation}: detect non-mixing of layers (Planck vs IR) and policy enforcement.
  \item \textbf{Certificate gating}: capture presence of machine-checkable proof bundles; verify gating of operation/ingestion on passing certificates.
  \item \textbf{Axiom probes (J-kernel)}: structured black-box tests to detect J-kernel or functional equivalent (see §12.1.1 for detailed protocols).
\end{itemize}

\subsection*{12.1.1 Axiom-Probe Protocols (J-Kernel Detection)}
\noindent The following black-box tests detect use of the J-kernel (or any functional equivalent satisfying the axiom set) without requiring source code or internal documentation. Tests are run on intercepted/exported data (API responses, logs, certificate bundles, telemetry) or via controlled stimulus-response experiments.

\paragraph{Probe 1: Symmetry $F(x)=F(1/x)$.}
\begin{itemize}[leftmargin=*]
  \item \textbf{Method}: Submit pairs of inputs $(y_i, y_i^\star)$ and $(y_i^\star, y_i)$ (swapped) to the system; observe computed costs $c_1$ and $c_2$.
  \item \textbf{Test}: $|c_1 - c_2| \le \epsilon_{\mathrm{sym}}$ with $\epsilon_{\mathrm{sym}}$ accounting for numerical precision.
  \item \textbf{Acceptance}: Pass if symmetry holds over $\ge N_{\mathrm{samples}}$ pairs (e.g., $N=100$) with $\ge 95\%$ within tolerance.
  \item \textbf{Evidence}: Log of input pairs, computed costs, and symmetry deltas with timestamps; statistical summary.
\end{itemize}

\paragraph{Probe 2: Unit normalization $F(1)=0$.}
\begin{itemize}[leftmargin=*]
  \item \textbf{Method}: Submit inputs with $y_i = y_i^\star$ (perfect match); observe computed cost.
  \item \textbf{Test}: $|F(1)| \le \epsilon_{\mathrm{norm}}$ (numerical zero).
  \item \textbf{Acceptance}: Pass if normalization holds across multiple channels and runs.
  \item \textbf{Evidence}: Recorded costs at identity ratio; aggregate statistics.
\end{itemize}

\paragraph{Probe 3: Strict convexity on $\mathbb{R}_{>0}$.}
\begin{itemize}[leftmargin=*]
  \item \textbf{Method}: For sampled ratios $r_a, r_b > 0$ and $\lambda\in(0,1)$, compute $r_m = \lambda r_a + (1-\lambda)r_b$ (midpoint); submit $(r_a, r_b, r_m)$ and observe costs $(c_a, c_b, c_m)$.
  \item \textbf{Test}: Strict midpoint inequality $c_m < \lambda c_a + (1-\lambda)c_b - \delta_{\mathrm{cvx}}$ with $\delta_{\mathrm{cvx}}>0$ (strict gap).
  \item \textbf{Acceptance}: Pass if convexity holds over a structured sample grid (e.g., $10\times10$ pairs, multiple $\lambda$).
  \item \textbf{Evidence}: Tables of $(r_a,r_b,\lambda,c_m,\lambda c_a + (1-\lambda)c_b)$ with convexity margins.
\end{itemize}

\paragraph{Probe 4: Calibration (unit curvature at 0 for $F\circ\exp$).}
\begin{itemize}[leftmargin=*]
  \item \textbf{Method}: Submit ratios $r(t) = e^t$ for small $|t|\le 0.1$; observe $F(e^t)$; fit quadratic $F(e^t) \approx a + bt + \tfrac12 c t^2$ near $t=0$.
  \item \textbf{Test}: $|c - 1| \le \epsilon_{\mathrm{calib}}$ (unit curvature).
  \item \textbf{Acceptance}: Pass if fitted $c$ consistent with $1$ within measurement uncertainty over multiple runs.
  \item \textbf{Evidence}: Fitted coefficients with confidence intervals; residual plots.
\end{itemize}

\paragraph{Probe 5: Cosh-addition identity.}
\begin{itemize}[leftmargin=*]
  \item \textbf{Method}: The identity $F(xy) + F(x/y) = 2F(x)$ (for $x,y>0$) is characteristic of $\cosh$-form. Submit structured pairs $(x,y)$ and observe $F(xy), F(x/y), F(x)$.
  \item \textbf{Test}: $|F(xy) + F(x/y) - 2F(x)| \le \epsilon_{\mathrm{cosh}}$.
  \item \textbf{Acceptance}: Pass if identity holds over grid of $(x,y)$ pairs.
  \item \textbf{Evidence}: Tables of $(x,y,F(xy),F(x/y),F(x),\mathrm{residual})$; aggregate chi-squared or KS test.
\end{itemize}

\paragraph{Functional equivalence verdict.}
If \emph{all five probes pass}, the system uses $J$ or a functional equivalent satisfying axioms (i)--(v). By uniqueness theorem (\texttt{T5\_uniqueness\_complete}), any such $F'$ is identical to $J$ on $\mathbb{R}_{>0}$; thus infringement via functional equivalence. Probes are compositional: partial passes (e.g., symmetry + normalization only) may still indicate derivative work or partial adoption; full pass is definitive.

\subsection*{12.2 Automated axiom-probe test harness}
\noindent For suspected infringement, deploy an automated test harness that:
\begin{enumerate}[leftmargin=*]
  \item Generates structured inputs (ratio pairs, swaps, midpoints, exponential sweeps) covering Probes 1--5.
  \item Submits inputs to the target system via API, exported logs, or reverse-engineered interfaces.
  \item Collects outputs (costs, ledger values, certificate fields, timing telemetry).
  \item Computes axiom-compliance scores per probe; aggregates into a five-component signature vector.
  \item Issues probabilistic verdict: "J-kernel detected" if all probes pass; "partial adoption" if subset passes; "non-compliant" otherwise.
\end{enumerate}
Evidence package includes: input/output logs, axiom-compliance tables, statistical summaries, and expert declaration mapping probe results to claimed axioms.

\subsection*{12.3 Detection methodologies (general)}
\begin{itemize}[leftmargin=*]
  \item \textbf{Black-box tests}: vary windowing/phase and observe phase-invariant output; probe for neutrality and route equality signatures; run axiom-probe harness (§12.1.1).
  \item \textbf{Header/certificate capture}: intercept API headers and artifacts (hashes, proof IDs); request/export logs under audit rights.
  \item \textbf{Timing analysis}: analyze device timing for Gray-aligned eight-tick schedules and neutrality windows; detect $\varphi$-commensurate ratios.
  \item \textbf{Stimulus-response}: structured inputs to elicit acceptance thresholds, gating behavior, and route-identity enforcement.
  \item \textbf{Remote attestation}: where available, request attestation of modules executing neutrality, audit, and proof verification.
\end{itemize}

\subsection*{12.4 Evidence preservation}
\begin{itemize}[leftmargin=*]
  \item \textbf{Certificate bundles}: collect signed bundles, proof hashes, device IDs, timestamps; store in evidence vault.
  \item \textbf{Logs/telemetry}: export acceptance signals, neutrality metrics, route values, axiom-probe results; preserve chain-of-custody and signatures.
  \item \textbf{Reproducibility kits}: capture versions, seeds, datasets, test harnesses (including axiom probes); containerize for independent replication.
\end{itemize}

\subsection*{12.5 Remedies and licensing}
\begin{itemize}[leftmargin=*]
  \item \textbf{Licensing tiers}: foundation license (required for any zero-parameter use) + domain licenses (stack on top); per-device royalties (firmware), per-run certificate fees (cloud), enterprise seats (SDK).
  \item \textbf{Remedies}: injunctions (esp. safety-critical deployments); reasonable royalties; enhanced damages for willful infringement after notice.
  \item \textbf{Settlement posture}: offer compliance onboarding (foundation license, certificate registry, SDK integration) to accelerate resolution; provide axiom-probe audit as compliance verification.
\end{itemize}

\subsection*{12.6 Claim charting and expert support}
\begin{itemize}[leftmargin=*]
  \item Map asserted claims to observed features: recognition ratios (dimensionless quotients in data), J-kernel (axiom-probe passes), neutrality (block-sum observables), AbsoluteLayer (gate identities in calibration), certificate gating (proof bundles and acceptance logic).
  \item Expert declarations on: (i) technical effects and KPI deltas; (ii) axiom-probe methodology and statistical validity; (iii) functional equivalence via uniqueness theorem; (iv) cross-domain enforcement synergy.
\end{itemize}

\subsection*{12.7 Standards/process integration}
\begin{itemize}[leftmargin=*]
  \item Procurement clauses requiring proof-backed calibration, certificate transparency, and axiom-compliance attestation.
  \item FRAND/RAND pathways if elements become standards-essential; baseline license terms public.
\end{itemize}

\subsection*{12.8 Automated monitoring}
\begin{itemize}[leftmargin=*]
  \item \textbf{Signals}: public claims of “zero-parameter” frameworks, anchor-free calibration, eight-phase neutrality, or certificate-gated audits.
  \item \textbf{Scanning}: repository/package scans for invariants/neutrality primitives; web crawlers for marketing and docs.
  \item \textbf{CI hooks}: optional telemetry beacons in licensed SDKs/firmware for compliance reporting (opt-in, privacy-preserving).
\end{itemize}

\tocsection{13) Standards and Certification Strategy}
\subsection*{13.1 Program structure}
\begin{itemize}[leftmargin=*]
  \item \textbf{Scheme owner}: maintain specification, verification profiles, certificate schemas, and transparency registry.
  \item \textbf{Accredited labs}: authorized to run conformance tests and issue signed certificates.
  \item \textbf{Governance board}: stakeholders from metrology, industry, and academia; change control and versioning.
\end{itemize}

\subsection*{13.2 Certification marks and profiles}
\begin{itemize}[leftmargin=*]
  \item \textbf{Marks}: “Reality-Certified” device/model/run marks with versioned profiles.
  \item \textbf{Profiles}: \emph{RB-CORE} (units-quotient, AbsoluteLayer, dual-route audit), \emph{RB-TIMING} (eight-tick neutrality), \emph{RB-AUDIT} (certificate gating), \emph{RB-SIM} (simulator plugins), \emph{RB-AI} (world-model datasets/models).
  \item \textbf{Levels}: L1 (software), L2 (hardware-assisted), L3 (hardware enforced) with increasing assurance.
\end{itemize}

\subsection*{13.3 Transparency registry and governance}
\begin{itemize}[leftmargin=*]
  \item \textbf{Registry}: public headers (hashes, timestamps, profile IDs, issuer); private payloads retained by owner.
  \item \textbf{Revocation}: signed revocation lists; supersession for updated evidence; audit trail retention policies.
  \item \textbf{Privacy}: selective disclosure of certificate fields; privacy-preserving proofs for sensitive deployments.
\end{itemize}

\subsection*{13.4 Conformance testing and accreditation}
\begin{itemize}[leftmargin=*]
  \item \textbf{Test suites}: neutrality/alias tests; route-equality and single-inequality acceptance; AbsoluteLayer solve consistency; gate-separation policy.
  \item \textbf{Golden datasets}: reference inputs and expected invariants; reproducible notebooks.
  \item \textbf{Lab accreditation}: criteria for equipment, personnel, and quality systems; proficiency testing.
\end{itemize}

\subsection*{13.5 Standards body engagement}
\begin{itemize}[leftmargin=*]
  \item \textbf{Metrology}: NIST/BIPM/ISO liaison; propose profiles for anchor-free calibration and proof-backed acceptance.
  \item \textbf{Industry}: IEEE/IEC working groups for timing, imaging, and instrumentation.
  \item \textbf{Publication}: reference implementations and verification profiles as public drafts.
\end{itemize}

\subsection*{13.6 SEP/FRAND policy}
\begin{itemize}[leftmargin=*]
  \item Declare standards-essential patents (where applicable) and offer FRAND/RAND licenses.
  \item Publish baseline license terms per profile level; compliance reporting and audit clauses.
\end{itemize}

\subsection*{13.7 Security and data policy}
\begin{itemize}[leftmargin=*]
  \item Signed certificates and logs; hardware root of trust where available; tamper-evident storage.
  \item Minimal PII; strong hashing and key management; incident response procedures.
\end{itemize}

\subsection*{13.8 Reference implementations and test suites}
\begin{itemize}[leftmargin=*]
  \item \textbf{Open reference}: SDK examples for units-quotient, AbsoluteLayer, neutrality scheduler, and audits.
  \item \textbf{Firmware IP}: portable HDL cores for neutrality timing; test benches and timing diagrams.
  \item \textbf{CI pipelines}: reproducible runs generating certificates and registry entries.
\end{itemize}

\subsection*{13.9 Adoption incentives}
\begin{itemize}[leftmargin=*]
  \item Early-adopter discounts; lab accreditation grants; co-marketing for Reality-Certified products.
  \item Procurement guidance encouraging proof-backed calibration for safety/regulated sectors.
\end{itemize}

\subsection*{13.10 Standardization timeline}
\begin{itemize}[leftmargin=*]
  \item \textbf{Phase 1 (0--6 months)}: publish v1 profiles, registry MVP, initial accredited labs.
  \item \textbf{Phase 2 (6--18 months)}: broader industry pilots; submission to standards bodies; FRAND policy publication.
  \item \textbf{Phase 3 (18--36 months)}: standard adoption; interoperability events; conformance test maturation.
\end{itemize}

\tocsection{14) Licensing and Monetization}
\subsection*{14.1 Licensing models}
\begin{itemize}[leftmargin=*]
  \item \textbf{Per-device royalty (firmware/IP cores)}: neutrality timing blocks and gate controllers licensed to instrument OEMs.
  \item \textbf{Per-run certificate fees (cloud/on-prem)}: usage-based pricing for certificate issuance/verification and registry entries.
  \item \textbf{Enterprise seats (SDK)}: tiered seats for development/CI integration; includes support and update SLAs.
  \item \textbf{Simulator/plugins royalties}: revenue share for reality-consistent modes in partner platforms.
  \item \textbf{Compliance/SaaS bundles}: audit trail retention, transparency registry hosting, and reporting dashboards.
\end{itemize}

\subsection*{14.2 Pricing strategy}
\begin{itemize}[leftmargin=*]
  \item \textbf{Value-based tiers}: align with measured KPIs (alias error reduction, time-to-calibrate, audit savings).
  \item \textbf{Sector differentiation}: metrology/semiconductor/medical/aerospace pricing bands; long-term OEM discounts.
  \item \textbf{Profile-based}: RB-CORE/RB-TIMING/RB-AUDIT/RB-SIM/RB-AI profiles with additive pricing.
\end{itemize}

\subsection*{14.3 Channels and partnerships}
\begin{itemize}[leftmargin=*]
  \item \textbf{OEM integrations}: instrument vendors embedding firmware and SDK; volume tiers.
  \item \textbf{Cloud marketplaces}: packaged APIs and registry services with metered billing.
  \item \textbf{Standards ecosystems}: FRAND/RAND licensing for standards-essential elements; certification revenue.
\end{itemize}

\subsection*{14.4 Contractual terms}
\begin{itemize}[leftmargin=*]
  \item \textbf{SLAs}: availability for certificate issuance/verification; latency for on-instrument gating.
  \item \textbf{Support/maintenance}: version pinning, security updates, and backward compatibility windows.
  \item \textbf{Audit/compliance}: right to inspect certificate headers and logs; data retention policies; privacy terms.
  \item \textbf{IP}: patent license scope, field-of-use restrictions (if any), sublicensing for OEMs.
\end{itemize}

\subsection*{14.5 Metering, security, and anti-piracy}
\begin{itemize}[leftmargin=*]
  \item \textbf{Usage metering}: tokenized certificate issuance; offline counters with reconciliation.
  \item \textbf{Security}: signed firmware cores and SDK modules; tamper-evident logs; hardware root-of-trust where available.
  \item \textbf{Telemetry (opt-in)}: privacy-preserving compliance beacons for large deployments.
\end{itemize}

\subsection*{14.6 Upsell and expansion}
\begin{itemize}[leftmargin=*]
  \item \textbf{Advanced profiles}: higher-assurance levels (hardware-enforced neutrality; on-device proof verification).
  \item \textbf{Analytics}: benchmarking dashboards, anomaly detection for drift, and fleet optimization.
  \item \textbf{Vertical packs}: pre-built flows and datasets per modality (spectroscopy, timing, imaging, lidar, quantum).
\end{itemize}

\subsection*{14.7 Monetization roadmap}
\begin{itemize}[leftmargin=*]
  \item \textbf{Year 1}: OEM pilots; SDK enterprise seats; certificate fees for early adopters; registry MVP.
  \item \textbf{Year 2}: standards engagement; FRAND program; cloud marketplace scale; advanced profiles.
  \item \textbf{Year 3}: broad OEM adoption; certification revenue at scale; analytics upsell.
\end{itemize}

\tocsection{15) Trade Secrets, Data, and Branding}
\subsection*{15.1 Trade secrets (scope and policy)}
\begin{itemize}[leftmargin=*]
  \item \textbf{Scheduler heuristics}: modality-specific neutrality schedulers, phase alignment strategies, fallback logic.
  \item \textbf{Uncertainty tuning}: combined-uncertainty models (\(u_{\mathrm{comb}}\)), correlation estimation, dynamic \(k\) selection.
  \item \textbf{Telemetry/diagnostics}: drift detectors, anomaly scoring, auto-recovery policies.
  \item \textbf{Performance configs}: precision/latency envelopes, hardware IP micro-architectures.
  \item \textbf{Access}: least-privilege access control, need-to-know partitioning, secrets rotation; documentation in secured repositories.
\end{itemize}

\subsection*{15.2 Data and copyright assets}
\begin{itemize}[leftmargin=*]
  \item \textbf{Certificate schemas}: JSON/XML definitions, validators, and example artifacts (copyright).
  \item \textbf{Golden datasets}: curated test inputs and expected invariants; versioned with checksums (copyright/license).
  \item \textbf{Notebooks and CI}: reproducible scripts, figures, and protocol docs (copyright).
  \item \textbf{Documentation}: API references, deployment guides, conformance manuals.
\end{itemize}

\subsection*{15.3 Open-source and dual licensing}
\begin{itemize}[leftmargin=*]
  \item \textbf{Open reference}: limited, non-core components (e.g., example notebooks, basic validators) under permissive license.
  \item \textbf{Dual licensing}: commercial licenses for SDK/firmware cores; contributor agreements for inbound contributions.
  \item \textbf{Compliance}: clear SBOMs, third-party notices, and license scanners in CI.
\end{itemize}

\subsection*{15.4 Branding and trademarks}
\begin{itemize}[leftmargin=*]
  \item \textbf{Marks}: “Reality-Certified,” “AbsoluteLayer,” “K-Gate,” “Eight-Tick Neutrality,” and program logos.
  \item \textbf{Usage guidelines}: placement, clear space, color/monochrome variants, and prohibited uses.
  \item \textbf{Co-branding}: partner/OEM usage rules; certification seal on devices, datasheets, and dashboards.
\end{itemize}

\subsection*{15.5 Asset protection and access control}
\begin{itemize}[leftmargin=*]
  \item \textbf{Repositories}: segregated repos for secret and public content; encrypted storage for key artifacts.
  \item \textbf{Keys and signatures}: HSM-backed signing for certificates and firmware; rotation policies and incident response.
  \item \textbf{Data retention}: retention schedules for logs and certificates; privacy-preserving storage and deletion.
\end{itemize}

\subsection*{15.6 Messaging and collateral}
\begin{itemize}[leftmargin=*]
  \item \textbf{Positioning}: “Proof-backed, parameter-free calibration and audit.”
  \item \textbf{Collateral}: whitepapers, case studies (alias/time/resource KPIs), standards liaisons, and adoption stories.
  \item \textbf{Consistency}: glossary of terms; canonical diagrams for AbsoluteLayer, K-identities, neutrality windows, and certificate flows.
\end{itemize}

\tocsection{16) Operational Plan}
\subsection*{16.1 Milestones}
\begin{itemize}[leftmargin=*]
  \item \textbf{M1 (0--30 days)}: Provisional \#1 prep --- SDK v0 APIs (units-quotient, AbsoluteLayer, dual-route audit), neutrality scheduler prototype, benchmarks v1, proof map.
  \item \textbf{M2 (30--90 days)}: Provisional \#2 prep --- firmware IP (timing core), certificate engine MVP (issuance/verification), transparency registry MVP, expanded KPIs.
  \item \textbf{M3 (90--180 days)}: OEM pilot integrations (timing/spectroscopy); CI pipelines; reproducibility kits; counsel brief for PCT claim harmonization.
  \item \textbf{M4 (12 months)}: PCT filing; profile drafts (RB-CORE/RB-TIMING/RB-AUDIT); lab accreditation criteria.
\end{itemize}

\subsection*{16.2 Evidence pack production}
\begin{itemize}[leftmargin=*]
  \item Benchmark protocols and raw results (alias error, time-to-calibrate, acceptance quality, resource use).
  \item Certificate exemplars (JSON schemas, signed bundles, logs, hashes) per profile.
  \item Reproducible notebooks and CI jobs pinned to versions/seeds; golden datasets with checksums.
\end{itemize}

\subsection*{16.3 Counsel coordination}
\begin{itemize}[leftmargin=*]
  \item Consolidated technical annex (proof map, KPIs, embodiments, claim element mapping).
  \item Weekly cadence until provisionals; monthly thereafter; action items for \S101/EPO positioning and claim charting.
\end{itemize}

\subsection*{16.4 Resource plan}
\begin{itemize}[leftmargin=*]
  \item \textbf{Engineering}: SDK (2), firmware (1), cloud (1), QA (1), devrel (0.5).
  \item \textbf{Legal}: outside counsel lead + associate; internal coordinator.
  \item \textbf{Budget}: tranche for P\#1/\#2, PCT, national phase entries; reserve for continuations and office actions.
\end{itemize}

\subsection*{16.5 KPIs and gates}
\begin{itemize}[leftmargin=*]
  \item KPI thresholds per modality (alias %, time speedup, acceptance ROC, resource budgets) as filing gates.
  \item Passing certificate exemplars required to include claims tied to certificate-gated operation.
\end{itemize}

\subsection*{16.6 Standards and ecosystem}
\begin{itemize}[leftmargin=*]
  \item Engage NMIs and early labs for pilot certification; publish verification profiles; schedule interoperability tests.
  \item Draft FRAND/RAND baseline terms pending standardization.
\end{itemize}

\subsection*{16.7 Tooling and QA}
\begin{itemize}[leftmargin=*]
  \item CI (deterministic builds/tests), SBOMs, license scans; security reviews for SDK/firmware/cloud.
  \item Test harnesses for detection (route equality, neutrality signatures, gating behavior).
\end{itemize}

\subsection*{16.8 Communication cadence}
\begin{itemize}[leftmargin=*]
  \item Internal weekly status (engineering/legal/standards); monthly exec review; quarterly external updates to partners.
  \item Public releases aligned to milestones (SDK v0, firmware alpha, registry MVP) with measured KPI summaries.
\end{itemize}

\tocsection{17) Appendices (Spec Excerpts and Exhibits)}
\subsection*{17.1 Gate identities and acceptance inequality}
\noindent Dimensionless identities and audit rule used across claims:
\begin{itemize}[leftmargin=*]
  \item \textbf{K-identities}: \(\tau_{\mathrm{rec}}/\tau_0 = \lambda_{\mathrm{kin}}/\ell_0 = K\) (display equality across routes).
  \item \textbf{Planck gate}: \((c^3\,\lambda_{\mathrm{rec}}^2)/(\hbar G)=\text{const}\) (dimensionless normalization; constant by layer policy).
  \item \textbf{AbsoluteLayer}: solve \emph{UniqueCalibration \(\wedge\) MeetsBands} to select a unique layer satisfying all identities simultaneously.
  \item \textbf{Acceptance inequality}: \(\lvert\lambda_{\mathrm{kin}}-\lambda_{\mathrm{rec}}\rvert/\lambda_{\mathrm{rec}} \le k\,u_{\mathrm{comb}}\), with \(u_{\mathrm{comb}}=\sqrt{u(\lambda_{\mathrm{kin}})^2+u(\lambda_{\mathrm{rec}})^2-2\rho\,u(\lambda_{\mathrm{kin}})u(\lambda_{\mathrm{rec}})}\).
\end{itemize}

\subsection*{17.2 Measurement and test protocols (summaries)}
\begin{itemize}[leftmargin=*]
  \item \textbf{Alias/neutrality}: compare baseline vs eight-tick neutrality with \(J\)-block minimization; report alias error % and confidence intervals.
  \item \textbf{Calibration time}: measure time-to-accept under single inequality across modalities; record correlation handling.
  \item \textbf{Reproducibility}: units-invariance test across unit choices after AbsoluteLayer selection; variance/bias metrics.
  \item \textbf{Resource}: CPU cycles, memory, energy for audit loops; latency budgets for gating.
\end{itemize}

\subsection*{17.3 Lean proof references (non-exhaustive)}
\begin{itemize}[leftmargin=*]
  \item Cost uniqueness (T5): \texttt{IndisputableMonolith/CostUniqueness.lean}, \texttt{Cost/JcostCore.lean}, \texttt{Cost/FunctionalEquation.lean}.
  \item Eight-tick/neutrality: \texttt{Measurement/*}, adapter references in \texttt{URCAdapters/*}.
  \item Units-quotient/K-identities: \texttt{URCAdapters/*}, \texttt{Constants/*}.
  \item Exclusivity scaffold/certificates: \texttt{Verification/Exclusivity/NoAlternatives.lean}, \texttt{URCGenerators/ExclusivityCert.lean}.
\end{itemize}

\subsection*{17.4 Certificate bundle (schema excerpt)}
\noindent Example certificate header (abbreviated):
\begin{verbatim}
{
  "profile": "RB-CORE@1.0",
  "run_id": "2025-11-04T12:34:56Z:device:abc123",
  "invariants": {
    "lambda_kin": {"value": 1.234e-35, "u": 0.012e-35},
    "lambda_rec": {"value": 1.236e-35, "u": 0.011e-35},
    "rho": 0.15
  },
  "acceptance": {"k": 3.0, "passed": true},
  "proofs": [
    {"id": "K-identities@hash", "status": "verified"},
    {"id": "AbsoluteLayer@hash", "status": "verified"},
    {"id": "Neutrality@hash", "status": "verified"}
  ],
  "signatures": {"issuer": "labX", "sig": "..."}
}
\end{verbatim}

\subsection*{17.5 Vertical embodiment sketches}
\begin{itemize}[leftmargin=*]
  \item \textbf{Spectroscopy}: neutrality-aligned integration windows; AbsoluteLayer solve; dual-route audit; certificate-gated data ingest.
  \item \textbf{Timing}: FPGA neutrality core; route equality monitor; on-device acceptance interlocks; registry logging.
  \item \textbf{Imaging}: tiled eight-tick scheduling; route-equalized calibration maps; proof-backed acceptance for calibration frames.
  \item \textbf{Simulator}: reality-consistent mode enforcing units-quotient and identities; attach certificates to runs in CI.
\end{itemize}

\subsection*{17.6 Figures and diagrams (to be prepared)}
\begin{itemize}[leftmargin=*]
  \item AbsoluteLayer solve and gate identity graph.
  \item Eight-tick neutrality timing diagram and alias reduction illustration.
  \item Certificate generation and verification flow.
\end{itemize}

\subsection*{17.7 Versioning and change log}
\begin{itemize}[leftmargin=*]
  \item Profiles (RB-CORE, RB-TIMING, RB-AUDIT, RB-SIM, RB-AI) with semver; registry reflects version.
  \item Certificate schema versions and migration notes.
  \item Proof/toolchain pinning (Lean, libraries) for reproducibility.
\end{itemize}

\tocsection{18) Cross-Domain Portfolio Integration}

\subsection*{18.1 Foundation as the unavoidable base layer}
\noindent The Recognition Computing Architecture (Family 0) is the \emph{required} foundation for all zero-parameter instantiations. Any implementation using:
\begin{itemize}[leftmargin=*]
  \item Dimensionless recognition ratios $r_i=y_i/y_i^\star$ for observability,
  \item A convex symmetric cost kernel satisfying the five axioms,
  \item $\varphi$-commensurate temporal windows with neutrality constraints,
  \item Machine-checkable invariant proofs gating operation, and/or
  \item Simultaneous dimensionless identity solving for absolute scales
\end{itemize}
necessarily practices the foundation claims. Domain-specific patents (fusion, solar, bio, AI, robotics) add vertical constraints but \emph{cite and depend on} Family 0.

\subsection*{18.2 Domain patent structure (pyramid)}
\begin{itemize}[leftmargin=*]
  \item \textbf{Layer 1 (foundation)}: Family 0 claims covering the five-layer architecture, J-kernel axiom set, compliance API, certificate engine. \emph{Required for any zero-parameter use.}
  
  \item \textbf{Layer 2 (primitives)}: Families A--K covering specific technical capabilities (exclusivity verification, calibration OS, dual-route audit, AbsoluteLayer selector, eight-tick scheduler, recognition-operator controller, certificate engine, gate separation, pattern primitives, $\varphi$-forcing, necessity stack). \emph{Domain-agnostic; cite Family 0.}
  
  \item \textbf{Layer 3 (instantiations)}: Domain-specific applications citing Layers 1--2:
  \begin{itemize}
    \item \textbf{Fusion-1 (tokamak)}: ledger over plasma diagnostics (normalized critical gradients, shear ratios); $\varphi$-phased NBI/ECRH/ICRH/RMP; certificate-gated shots. \emph{Cites Families 0, E, F, G.}
    \item \textbf{Fusion-2 (ICF)}: symmetry ledger over $|a_{\ell m}|/a_\ell^\star$; $\varphi$-spaced sub-pulses; geometric convergence. \emph{Cites Families 0, E, J.}
    \item \textbf{Solar (perovskite)}: window-8 scheduling with IR phase-lock/detune; rate-balance $x\to 1$; eight-band acceptance maps; fragility witness. \emph{Cites Families 0, E, I.}
    \item \textbf{Bio (abiogenesis)}: LISTEN/LOCK/BALANCE gates; duplex geometry from $J$-minimization; templating fixed-point; metabolic viability inequality. \emph{Cites Families 0, E, K.}
    \item \textbf{AI/ML}: $J$-regularized loss functions; eight-tick temporal batching; certificate-gated training data. \emph{Cites Families 0, G.}
    \item \textbf{Multi-actuator control ($\varphi$-scheduler)}: plug-and-play module with compliance API; interference bounds. \emph{Cites Families 0, E.}
  \end{itemize}
\end{itemize}

\subsection*{18.3 Claim dependency chains (enforcement)}
\noindent Infringement analysis proceeds bottom-up:
\begin{enumerate}[leftmargin=*]
  \item Detect \textbf{Family 0} practice via axiom probes (§12.1.1), neutrality observables, certificate schemas, and compliance API signatures.
  \item If Family 0 confirmed, assert foundation license requirement.
  \item Identify domain (fusion/solar/bio/AI/robotics) and map to Layer 3 instantiation patents.
  \item Assert \emph{both} foundation (Family 0) and domain-specific claims; stack damages/royalties.
\end{enumerate}
Competitors infringing multiple domains face compounded exposure: foundation license + multiple domain licenses.

\subsection*{18.4 Portfolio synergies (cross-domain evidence transfer)}
\begin{itemize}[leftmargin=*]
  \item \textbf{Certificate schemas}: Same JSON structure across fusion/solar/bio/AI; infringement in one domain establishes certificate-gating practice transferrable to others.
  \item \textbf{Axiom probes}: Test harness developed for fusion applies to solar, bio, AI with identical protocols; one R\&D investment, multiple enforcement domains.
  \item \textbf{Neutrality observables}: Window-8 block-sum metrics identical across perovskite manufacturing, tokamak control, abiogenesis LISTEN/LOCK/BALANCE; detection methods unified.
  \item \textbf{Compliance API}: Same interface (WindowIndex(), Allowed(a), RegisterUpdate(a), GetComplianceReport()) across all instantiations; infringement signature portable.
  \item \textbf{Expert testimony}: Axiom-probe methodology, functional equivalence argument, and technical-effect quantification reusable across verticals with domain-specific KPI substitutions.
\end{itemize}

\subsection*{18.5 Licensing waterfall}
\begin{enumerate}[leftmargin=*]
  \item \textbf{Foundation mandatory}: All zero-parameter implementations require Family 0 license (architecture, J-kernel axioms, compliance API, certificate engine).
  \item \textbf{Domain add-ons}: Users select verticals (fusion, solar, bio, AI, robotics); each domain license cites and builds on foundation.
  \item \textbf{Enterprise bundle}: Foundation + all-domains; single contract covering current and future instantiations.
  \item \textbf{OEM/volume tiers}: Per-device (firmware IP cores), per-run (certificate fees), seats (SDK).
  \item \textbf{SEP/FRAND}: Where standardization occurs (metrology, model-risk governance, AI safety), offer RAND terms on foundation; retain commercial terms for domain instantiations.
\end{enumerate}

\subsection*{18.6 Strategic implications}
\begin{itemize}[leftmargin=*]
  \item \textbf{Winner-take-all foundation}: Exclusivity theorem forecloses parameter-free alternatives; foundation becomes the \emph{only} licensable substrate for zero-parameter computing.
  \item \textbf{Domain multiplication}: Each new instantiation (quantum computing, drug discovery, climate models, financial risk) expands total addressable market \emph{without diluting} foundation value—every use requires the base layer.
  \item \textbf{Network effects}: More domains → more enforcement synergy → higher infringement detectability → stronger deterrence → higher licensing compliance.
  \item \textbf{Standards lock-in}: As zero-parameter calibration/provenance becomes regulatory standard (NIST, FDA, AI safety), foundation license becomes unavoidable for compliance.
\end{itemize}

\subsection*{18.7 Prosecution coordination}
\begin{itemize}[leftmargin=*]
  \item File foundation (Family 0) as first priority; domain instantiations cite as parent or incorporate by reference.
  \item Maintain claim consistency (method/system/medium triads, hardware embodiments, technical effects) across all filings.
  \item Unified technical annex (Lean proofs, axiom-probe protocols, certificate schemas, compliance API spec) attached to all applications.
  \item Coordinate office-action responses: arguments/amendments in foundation propagate to domain patents; vice versa for vertical-specific technical effects.
\end{itemize}

\end{document}

