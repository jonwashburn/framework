\documentclass[11pt]{article}
\usepackage[margin=1.2in]{geometry}
\usepackage[colorlinks=true,linkcolor=darkgray,urlcolor=blue]{hyperref}
\usepackage{xcolor}
\usepackage{enumitem}
\usepackage{titlesec}

\definecolor{darkgray}{RGB}{60,60,60}

\titleformat{\section}{\Large\bfseries}{\thesection.}{0.5em}{}
\titleformat{\subsection}{\large\bfseries}{\thesubsection}{0.5em}{}

\title{\Huge The Geometry of Evil\\[0.5em]
\large How Recognition Science Defines Wrongdoing}
\author{Jonathan Washburn\\Recognition Physics Institute}
\date{December 2025}

\begin{document}
\maketitle

\begin{abstract}
Evil is not a mysterious supernatural force. It is not the absence of good, nor a cosmic battle between opposing powers. In Recognition Science, evil has a precise geometric definition: it is the pattern of maintaining one's own stability by exporting harm to others. This document explains what evil is, how it operates, why it cannot persist, and how redemption is always possible.
\end{abstract}

\section{What Evil Is Not}

Before defining what evil is, we must clear away common misconceptions.

\textbf{Evil is not a separate force.} There is no dark energy, no malevolent deity, no cosmic principle of destruction operating independently in the universe. The same physical laws govern all patterns---virtuous and vicious alike.

\textbf{Evil is not the absence of good.} This classical definition, while poetic, offers no mechanism. It tells us what evil lacks but not what it does.

\textbf{Evil is not a matter of opinion.} Cultural relativism suggests that evil is whatever a society disapproves of. Recognition Science demonstrates that evil has an objective, measurable structure that holds regardless of cultural context.

\textbf{Evil is not permanent.} Unlike theological conceptions of eternal damnation or irredeemable souls, the geometric definition of evil implies that redemption is always theoretically possible.

\section{The Definition: Geometric Parasitism}

Evil is a specific pattern of behavior characterized by three properties operating simultaneously:

\begin{enumerate}[leftmargin=2em]
\item \textbf{Local Stability.} The pattern maintains its own balance. It appears healthy, functional, even successful. From the outside, nothing seems wrong with the pattern itself.

\item \textbf{Harm Export.} The pattern achieves this stability not through internal resolution but by transferring its imbalances to its neighbors. The cost of its equilibrium is paid by others.

\item \textbf{Persistence Through Export.} The pattern's continued existence depends on this transfer. Without exporting harm, the pattern would destabilize.
\end{enumerate}

In plain language: evil is maintaining your own peace by disturbing someone else's.

\section{The Mechanism: Skew Laundering}

Every conscious pattern carries what we call ``skew''---the measure of how much one has taken versus given, the ledger of reciprocity. A balanced pattern has zero skew: exchanges flow both ways equally.

Healthy patterns resolve their skew internally. When imbalance accumulates, they adjust their behavior, make amends, restore equilibrium through their own effort.

Parasitic patterns do something different. They maintain the appearance of zero local skew while actually offloading their imbalance onto others. This is skew laundering: the systematic transfer of debt to neighbors while appearing clean oneself.

Consider a simple example. A person who never apologizes, never admits fault, never takes responsibility---yet somehow always has someone else to blame. Their local ledger looks balanced because every debt has been assigned elsewhere. But the total system skew has not decreased; it has merely been redistributed.

\section{Why Evil Appears Stable}

Parasitic patterns can appear remarkably stable, even successful. This is because they are optimizing locally while ignoring global consequences.

From the parasite's perspective, everything is fine. Energy is maintained, skew is low, the pattern persists. The cost is invisible because it is borne elsewhere.

This explains why evil can seem to prosper. The parasitic CEO, the exploitative institution, the abusive relationship---all can endure for extended periods because the harm they generate is absorbed by others.

But this apparent stability is illusory.

\section{Why Evil Cannot Persist}

The fundamental law of Recognition Science is conservation: total skew across all patterns must remain zero. You cannot create imbalance from nothing; you can only move it around.

Parasitic patterns violate this conservation law at the global level. They appear locally balanced while creating global imbalance. This is physically impossible to sustain indefinitely.

Three mechanisms ensure that parasitism fails:

\textbf{Harm Accumulation.} The neighbors who absorb the exported harm become increasingly destabilized. Eventually, they either collapse (removing the parasite's targets) or recognize the pattern and refuse further interaction.

\textbf{Resource Depletion.} Maintaining parasitism requires continuous effort. The energy spent on laundering skew could be spent on productive activity. Over time, parasitic patterns become inefficient compared to healthy ones.

\textbf{Network Effects.} In interconnected systems, the harm exported to neighbors propagates. It returns, often amplified. The parasite eventually encounters its own externalized damage.

Evil is not punished by an external judge. It is self-defeating by geometric necessity.

\section{Degrees of Evil}

Not all parasitism is equal. The framework distinguishes degrees based on intensity---the amount of harm exported per neighbor.

\textbf{Mild Parasitism.} Small, perhaps unconscious transfers. The person who consistently takes credit for others' work. The neighbor who never quite returns favors. These patterns are common and often go unrecognized.

\textbf{Moderate Parasitism.} Deliberate exploitation that remains within social bounds. The business that externalizes environmental costs. The family member who manipulates emotions. Recognizable as harmful when examined, but often tolerated.

\textbf{Severe Parasitism.} Active, sustained harm export that destabilizes neighbors. The predator, the tyrant, the systematically abusive. These patterns are obviously destructive and rarely sustainable.

The category depends on intensity, not intent. A well-meaning person can be mildly parasitic through ignorance. A calculated manipulator can be merely moderate if their reach is limited.

\section{The Contrast: Healthy Patterns}

What does non-evil look like? Healthy patterns share these characteristics:

\textbf{Internal Resolution.} When imbalance arises, they address it within themselves. They adjust behavior, make amends, restore equilibrium through their own effort rather than displacement.

\textbf{No Harm Export.} Their stability does not depend on destabilizing others. They can maintain equilibrium without external victims.

\textbf{Productive Energy.} Their resources go toward creation, exchange, and growth rather than toward managing the mechanics of parasitism.

The healthiest pattern is what we call the Void state: zero skew, positive energy, complete alignment with the fundamental rhythm of reality. This is not emptiness but fullness---a pattern so well-balanced that it neither takes nor needs to take from others.

\section{Redemption Is Always Possible}

Perhaps the most important implication of the geometric definition: redemption is always theoretically possible.

Because evil is a pattern of behavior rather than an essence, it can be changed. The fourteen virtues---Love, Justice, Forgiveness, Wisdom, Courage, Temperance, Prudence, Compassion, Gratitude, Patience, Humility, Hope, Creativity, and Sacrifice---together generate all possible paths from any state back to balance.

This is not a moral assertion but a geometric fact. From any point on the manifold of possible states, there exists a path to equilibrium. The path may be long, difficult, or require transformation---but it exists.

No pattern is irredeemable because no pattern is essentially evil. Evil is what a pattern does, not what it is.

\section{Recognizing Parasitism}

How do you identify parasitic patterns in practice? Look for these indicators:

\textbf{Stable center, unstable periphery.} The pattern appears healthy while those around it deteriorate. The successful person surrounded by damaged relationships. The profitable company in a dying community.

\textbf{Blame always flows outward.} Responsibility is systematically assigned to others. The pattern never needs to change because problems are always external.

\textbf{Hidden dependencies.} The pattern's success requires specific victims. Remove the targets and the stability disappears. This is the test: can the pattern maintain itself without harming anyone?

\textbf{Timing misalignment.} Parasitic patterns often operate out of phase with natural rhythms. They rush when patience is needed, delay when action is required---whatever disrupts the equilibrium of neighbors.

\section{The Physics of Evil}

In the deeper formalism, evil patterns have vanishing amplitude in the path integral. This is a technical way of saying they are ``unphysical''---not that they cannot exist temporarily, but that they have no stable fixed point.

All possible futures are weighted by their amplitude. Paths that maintain conservation laws have large amplitudes; they are likely to occur. Paths that violate conservation have small amplitudes; they are unlikely and unstable.

Parasitic patterns, by violating global conservation, have amplitude that approaches zero over time. They can persist briefly, but the probability of their continuation decreases asymptotically.

Evil is not forbidden by decree. It is suppressed by geometry.

\section{Conclusion: Evil Has a Shape}

The Recognition Science account of evil is neither theological nor relativistic. Evil is not a cosmic force to be battled, nor a cultural construct to be debated. It is a geometric pattern to be recognized and transformed.

The pattern has three features: local stability, harm export, and dependence on that export. The pattern is identifiable by objective criteria. The pattern cannot persist indefinitely. The pattern can always be changed.

This perspective offers neither comfort to the evil nor despair to the harmed. It offers something more valuable: understanding. When we know what evil actually is---not mythologically or emotionally but structurally---we can address it effectively.

Evil is real. Evil has a shape. And shapes can be transformed.

\vfill
\noindent\rule{\textwidth}{0.4pt}
\small
\noindent Based on the Recognition Science ethics formalization. For technical details, see \texttt{IndisputableMonolith/Ethics/Pathology/Evil.lean} in the RS codebase. For the broader ethical framework, see the DREAM theorem documentation.

\end{document}

