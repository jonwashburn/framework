\pdfoutput=1
\documentclass[11pt,a4paper]{article}

% Packages
\usepackage[T1]{fontenc}
\usepackage{lmodern}
\usepackage{microtype}
\usepackage[margin=1in]{geometry}
\usepackage{enumitem}
\usepackage{booktabs}
\usepackage{amsmath,amssymb,amsthm}
\usepackage{mathtools}
\usepackage{hyperref}
\hypersetup{hidelinks}
% Prevent revision-color macros from polluting PDF bookmarks
\pdfstringdefDisableCommands{\def\rev#1{#1}}
\usepackage{graphicx}
\usepackage{xcolor}
\newcommand{\editamir}[1]{#1}
% Prevent revision-color macros from polluting PDF bookmarks
\pdfstringdefDisableCommands{\def\editamir#1{#1}}

% Typography
\setlength{\parindent}{0pt}
\setlength{\parskip}{0.6em}
\setlist[itemize]{leftmargin=*,noitemsep,topsep=0.3em}
\setlist[enumerate]{leftmargin=*,noitemsep,topsep=0.3em}

% Theorem environments
\theoremstyle{plain}
\newtheorem{theorem}{Theorem}[section]
\newtheorem{lemma}[theorem]{Lemma}
\newtheorem{proposition}[theorem]{Proposition}
\newtheorem{corollary}[theorem]{Corollary}

\theoremstyle{definition}
\newtheorem{definition}[theorem]{Definition}
\newtheorem{example}[theorem]{Example}
\newtheorem{remark}[theorem]{Remark}

\theoremstyle{remark}
\newtheorem*{notation}{Notation}

% Custom commands
\newcommand{\R}{\mathbb{R}}
\newcommand{\N}{\mathbb{N}}
\newcommand{\Z}{\mathbb{Z}}
\newcommand{\Q}{\mathbb{Q}}
\newcommand{\CC}{\mathbb{C}}
\newcommand{\Jcost}{J}
\newcommand{\Sym}{\mathcal{S}}
\newcommand{\Obj}{\mathcal{O}}
\newcommand{\RefStruct}{\mathcal{R}}
\newcommand{\Mean}{\mathrm{Mean}}
\newcommand{\rev}[1]{#1}
\newcommand{\ph}{\varphi}

\title{A Cost-Minimization Theory of Reference: \\[0.3em]
Aboutness from Balance and Compression}

\author{%
	Jonathan Washburn\thanks{\raggedright
		Recognition Physics Research Institute, Austin, Texas, USA.
		Email: \texttt{jon@recognitionphysics.org}.}%
	\and
	Amir Rahnamai Barghi\thanks{\raggedright
		Corresponding author.
		Recognition Physics Research Institute, Austin, Texas, USA.
		Email: \texttt{arahnamab@gmail.com}.}%
}

\date{}

\begin{document}

\maketitle

\begin{abstract}
We present an axiomatic and checkable model of \emph{reference} in which a \emph{symbol} $s\in S$ and a \emph{candidate referent} $o\in O$ are compared through positive scale maps $\iota_S:S\to\mathbb R_{>0}$ and $\iota_O:O\to\mathbb R_{>0}$. The reference cost is defined by a fixed mismatch penalty $\Jcost:(0,\infty)\to[0,\infty)$ applied to the ratio of scales. In this paper we work with the canonical choice
\[
\Jcost(x)=\tfrac12(x+x^{-1})-1,\qquad x>0,
\]
which is symmetric in $x\leftrightarrow x^{-1}$ and satisfies a multiplicative d'Alembert identity. Under the axioms recorded in Definition~\ref{def:cost-axioms}, this form is in fact forced up to a single parameter (Appendix~\ref{app:dalembert}); after a harmless rescaling of the scale maps one may take the normalization above.

Given $\iota_S$ and $\iota_O$, we define the \emph{meaning set} of $s$ to be the set of minimizers of $o\mapsto \Jcost\!\big(\iota_S(s)/\iota_O(o)\big)$ over $O$. Under explicit attainment/closure hypotheses on the feasible scale set $\iota_O(O)$, we establish: (i) existence of meanings; (ii) for finite object dictionaries, a closed-form decision geometry in which meaning regions are separated by geometric-mean boundaries, yielding stability away from the boundaries; (iii) exact compositionality for product symbol/object models under separability assumptions; and (iv) an optimal mediation principle for sequential reference, with an explicit mediator that weakly decreases total cost. The paper is purely mathematical: all claims are conditional on the stated axioms and do not assert that any particular cognitive or linguistic system instantiates the model.
\end{abstract}

\medskip\noindent\textbf{2020 Mathematics Subject Classification.} Primary 39B52, 49J40; Secondary 26A51, 90C25, 94A17.

\medskip\noindent\textbf{Key words and phrases.} aboutness; reference as optimization; canonical reciprocal cost; mismatch function; compositionality.

\tableofcontents

%==============================================================================
\section{Introduction}\label{sec:introduction}


This paper develops a compact mathematical framework in which \emph{aboutness} is \emph{defined} by an optimization rule.  We start with two sets:

\begin{itemize}
  \item a \emph{symbol space} $S$ (words, codes, internal states, messages, \dots),
  \item an \emph{object space} $O$ (candidate referents, concepts, states of affairs, \dots).
\end{itemize}

Each space is equipped with a positive \emph{scale} map
\(\iota_S:S\to\mathbb R_{>0}\) and \(\iota_O:O\to\mathbb R_{>0}\), interpreted as an intrinsic ``size/complexity'' in a common currency.  Fix a cost functional \(\Jcost:(0,\infty)\to[0,\infty)\) with the properties stated in Section~\ref{sec:Jcost} (symmetry under inversion, strict convexity, and a unique minimum at $1$).  We then define a \emph{ratio-induced reference cost}
\begin{equation}\label{eq:intro-ratio}
  c(s,o):=\Jcost\!\left(\frac{\iota_S(s)}{\iota_O(o)}\right),\qquad (s,o)\in S\times O.
\end{equation}

\paragraph{Meaning as minimization.}
The \emph{meaning set} of a symbol $s$ is the set of objects achieving minimal cost:
\[
  \mathrm{Mean}(s):=\operatorname*{arg\,min}_{o\in O}\;c(s,o).
\]
Equivalently, $o\in \mathrm{Mean}(s)$ iff $c(s,o)\le c(s,o')$ for all $o'\in O$ (Definition~\ref{def:meaning}).  Ties are allowed: meaning is set-valued unless uniqueness is proved under additional hypotheses.

\paragraph{Interpretive content (and its limits).}
Because $\Jcost$ is minimized at $1$, low reference cost forces \emph{scale matching}: a symbol can only refer cheaply to objects whose scale is close to its own.  This yields an explicit, checkable constraint on admissible reference patterns.  The framework is deliberately \emph{axiomatic}: the scale maps and the chosen $\Jcost$ are inputs.  The mathematical results below are unconditional \emph{within this model}, but the manuscript does \emph{not} claim that any particular empirical system realizes the axioms without separate validation.

\subsection{A toy example: three-object dictionary}
Let $O=\{o_1,o_2,o_3\}$ with scales $y_i:=\iota_O(o_i)$ satisfying $0<y_1<y_2<y_3$.  For a symbol $s$ with scale $x:=\iota_S(s)$, the meaning rule compares the three costs $\Jcost(x/y_i)$.  For the explicit functional \eqref{eq:Jcost}, the boundary between preferring $o_1$ and $o_2$ occurs at the \emph{geometric mean} $\sqrt{y_1y_2}$, and similarly between $o_2$ and $o_3$ at $\sqrt{y_2y_3}$ (Theorem~\ref{thm:geom-boundaries}).  Thus the model induces a piecewise-constant semantic partition of the positive line in the symbol ratio $x$, with stability away from the boundary points.

\subsection{Relation to prior work}
Classical analyses of reference emphasize logical form and truth conditions (e.g.\ Frege and Russell) \cite{frege1892,russell1905}.  The symbol-grounding literature highlights that purely formal symbol manipulation does not by itself determine what symbols are about \cite{harnad1990}.  The present work does not attempt to resolve these debates empirically.  Instead, it isolates a mathematically tractable \emph{selection principle}: aboutness is determined by minimizing an explicit mismatch cost.

\subsection{Contributions and what is proved}
Within the ratio-induced model \eqref{eq:intro-ratio} (and the explicit choice \eqref{eq:Jcost} used throughout), we establish the following structural facts under clearly stated hypotheses:

\begin{itemize}
  \item \textbf{Existence.} If the feasible scale set $\iota_O(O)\subset\mathbb R_{>0}$ is nonempty and closed and if the minimum is attained (as made precise in Theorem~\ref{thm:meaning-exists}), then every symbol admits at least one meaning.

  \item \textbf{Finite-dictionary decision geometry.} For finite ordered dictionaries, decision boundaries are given by geometric means of adjacent object scales, and meanings are locally stable away from these boundaries (Theorem~\ref{thm:geom-boundaries} and Corollary~\ref{cor:stability-away}).

  \item \textbf{Compositionality.} For product symbol/object spaces with separable scales, meaning factorizes componentwise (Theorem~\ref{thm:comp}).

  \item \textbf{Mediation.} For sequential reference through an intermediate representation, there is an explicit optimal mediator ratio that weakly decreases the total mismatch cost (Theorem~\ref{thm:seq-mediator} and Corollary~\ref{cor:mediation-reduces}).
\end{itemize}

\subsection{Organization}
Section~\ref{sec:Jcost} states the axioms for $\Jcost$ and fixes the explicit mismatch functional \eqref{eq:Jcost}.  Section~\ref{sec:costed-spaces} defines costed spaces, ratio-induced reference, and the meaning relation.  Section~\ref{sec:main-theorems} contains the principal theorems, followed by compositionality (Section~\ref{sec:compositionality}), extensions, and examples.


\section{The mismatch functional $\Jcost$}
\label{sec:Jcost}
%==============================================================================


This section fixes the scalar mismatch functional $\Jcost:(0,\infty)\to[0,\infty)$ used throughout to compare symbol and object scales via the ratio-induced cost \eqref{eq:intro-ratio}.  The role of $\Jcost$ here is purely mathematical: it is an explicit penalty for scale mismatch, and no physical, cognitive, or linguistic interpretation is assumed.

\subsection{Standard properties and canonicity}

\editamir{The conditions below are recorded as a compact axiom package for the mismatch penalty. They encode inversion symmetry, strict convexity, and a multiplicative compatibility under scale multiplication. After a log change of variables, the compatibility axiom becomes d'Alembert's functional equation, so the resulting class of penalties is classical. We include a tailored derivation in Appendix~\ref{app:dalembert} to keep the manuscript self-contained and to emphasize that the axioms are used only as mathematical assumptions, not as a claim of novelty.}


\begin{definition}[Cost Functional Axioms]\label{def:cost-axioms}
A \emph{mismatch functional} is a function $\Jcost:(0,\infty)\to[0,\infty)$ satisfying:
\begin{enumerate}
    \item \textbf{Normalization}: $\Jcost(1) = 0$.
    \item \textbf{Inversion symmetry}: $\Jcost(x) = \Jcost(x^{-1})$ for all $x > 0$.
    \item \textbf{Non-negativity}: $\Jcost(x) \ge 0$ for all $x > 0$.
    \item \textbf{Strict convexity}: $\Jcost$ is strictly convex on $(0,\infty)$.
    \item \textbf{Multiplicative d'Alembert identity}: for all $x,y>0$,
    \begin{equation}\label{eq:dalembert}
        \Jcost(xy) + \Jcost(x/y) = 2\Jcost(x) + 2\Jcost(y) + 2\Jcost(x)\Jcost(y).
    \end{equation}
\end{enumerate}
\end{definition}

\begin{lemma}[Uniqueness of the zero-cost point]\label{lem:Jcost-zero}
If $\Jcost$ satisfies Definition~\ref{def:cost-axioms}, then $\Jcost(x)=0$ implies $x=1$.
\end{lemma}

\begin{proof}
By (3) and (1), $\Jcost$ attains its minimum value $0$ at $x=1$.  By strict convexity (4), the minimizer is unique. Hence $\Jcost(x)=0$ forces $x=1$.
\end{proof}

\subsection{The explicit choice used in this paper}

\begin{definition}[The functional fixed below]\label{def:Jcost}
In the remainder of this paper we fix the explicit functional
\begin{equation}\label{eq:Jcost}
\Jcost(x)=\tfrac12(x+x^{-1})-1=\tfrac{(x-1)^2}{2x}\qquad(x>0).
\end{equation}
\end{definition}

\begin{proposition}[Verification of the axioms]\label{prop:Jcost-axioms}
The function \eqref{eq:Jcost} satisfies Definition~\ref{def:cost-axioms}.
\end{proposition}

\begin{proof}
Normalization, inversion symmetry, and non-negativity are immediate from \eqref{eq:Jcost}.  Differentiating $\Jcost(x)=\tfrac12(x+x^{-1})-1$ gives
\[
\Jcost'(x)=\tfrac12-\tfrac{1}{2x^2},\qquad \Jcost''(x)=\tfrac{1}{x^3}>0\quad(x>0),
\]
so $\Jcost$ is strictly convex on $(0,\infty)$.  For (5), set $C(x)=1+\Jcost(x)=\tfrac12(x+x^{-1})$.  Then
\[
C(xy)+C(x/y)=\tfrac12\Big(xy+\tfrac{1}{xy}+\tfrac{x}{y}+\tfrac{y}{x}\Big)=\tfrac12\big(x+\tfrac1x\big)\big(y+\tfrac1y\big)=2C(x)C(y),
\]
which is equivalent to \eqref{eq:dalembert} after substituting $C=1+\Jcost$ and expanding.
\end{proof}

\begin{proposition}[Classical characterization of $\Jcost$]\label{prop:Jcost-characterization}
\editamir{Assume $\Jcost:(0,\infty)\to[0,\infty)$ satisfies Definition~\ref{def:cost-axioms}. Then there exists a constant $a>0$ such that for all $x>0$,
\[
  \Jcost(x)=\cosh(a\log x)-1=\tfrac12\big(x^{a}+x^{-a}\big)-1.
\]
Moreover, if we replace the scale maps by $\tilde\iota_S:=\iota_S^{a}$ and $\tilde\iota_O:=\iota_O^{a}$, then the ratio-induced model with parameter $a$ becomes the same model written with parameter $1$. Consequently, one may take $a=1$ without loss of generality at the level of the induced reference costs.}
\end{proposition}

\begin{proof}
\editamir{See Appendix~\ref{app:dalembert}.}
\end{proof}

\begin{example}[Small-mismatch regime]
For $|u|\ll 1$ one has
\[
\Jcost(1+u)=\frac{u^2}{2}+O(u^3),
\]
so near balance the mismatch cost behaves like a quadratic penalty in the relative deviation.
\end{example}

\begin{remark}
\editamir{The later results use only the explicit mismatch penalty \eqref{eq:Jcost} (equivalently, the axiom package in Definition~\ref{def:cost-axioms} which it satisfies). The characterization Proposition~\ref{prop:Jcost-characterization} is included to record canonicity; it is not presented as a new functional-equation result. Any additional interpretation (physical, cognitive, or linguistic) would require hypotheses beyond those stated here.}
\end{remark}


\section{Costed Spaces and Reference Structures}
\label{sec:costed-spaces}
%==============================================================================


We now formalize the axioms of the model introduced in Section~\ref{sec:introduction}.  Throughout, the mismatch functional $\Jcost$ is fixed as in Section~\ref{sec:Jcost}.  The intent is to make precise which pieces of data are inputs (symbol/object spaces and their scale maps) and which pieces are derived (reference costs and meaning).

\subsection{Costed spaces}

\begin{definition}[Costed space]
Fix a mismatch functional $\Jcost:(0,\infty)\to[0,\infty)$ (Section~\ref{sec:Jcost}).  A \emph{costed space} is a triple $(C,J_C,\iota_C)$ consisting of:
\begin{itemize}
\item a set $C$ of configurations,
\item a map $\iota_C:C\to\R_{>0}$ called the \emph{scale map},
\item a cost function $J_C:C\to\R_{\ge 0}$ satisfying $J_C(c)=\Jcost(\iota_C(c))$ for all $c\in C$.
\end{itemize}
Equivalently, once $\iota_C$ is fixed, $J_C$ is determined by $\Jcost$; we retain $J_C$ in the notation since later statements compare symbol costs and object costs directly.
\end{definition}

\begin{notation}
We write $\Sym=(S,J_S,\iota_S)$ for a symbol costed space and $\Obj=(O,J_O,\iota_O)$ for an object costed space.
\end{notation}

\begin{example}[Ratio space]
The canonical example is $C=\R_{>0}$ with $\iota_C=\mathrm{id}$ and $J_C=\Jcost$.
\end{example}

\begin{example}[Near-balanced configurations]
For $\epsilon>0$ let $C_\epsilon:=\{x\in\R_{>0}:|x-1|<\epsilon\}$.  Then every $c\in C_\epsilon$ satisfies $J_C(c)=\Jcost(c)<\Jcost(1+\epsilon)$.
\end{example}

\subsection{Reference structures}

\begin{definition}[Reference structure]\label{def:ref-struct}
A reference structure from $\Sym$ to $\Obj$ is a function
\begin{equation}
    c_\RefStruct : S \times O \to \R_{\ge 0},
\end{equation}
called the \emph{reference cost}.  It assigns to each pair $(s,o)$ the cost of using $s$ to refer to $o$.
\end{definition}

\begin{definition}[Ratio-induced reference]
Given scale maps $\iota_S$ and $\iota_O$, the \emph{ratio-induced} reference structure is defined by
\begin{equation}\label{eq:ratio-ref}
    c_\RefStruct^\Jcost(s,o):=\Jcost\!\Big(\frac{\iota_S(s)}{\iota_O(o)}\Big).
\end{equation}
This is the cost used in the Introduction (Eq.~\ref{eq:intro-ratio}).
\end{definition}

\begin{definition}[Admissible reference structure]\label{def:admissible-ref-struct}
A reference structure $\RefStruct$ from $\Sym$ to $\Obj$ is called \emph{admissible} (with respect to $\Jcost$ and the scale maps $\iota_S,\iota_O$) if it is ratio-induced, i.e.
\begin{equation}
c_\RefStruct(s,o)=\Jcost\!\Big(\frac{\iota_S(s)}{\iota_O(o)}\Big)\qquad\forall (s,o)\in S\times O.
\end{equation}
Unless stated otherwise, we work with admissible reference structures.
\end{definition}

\begin{proposition}[Inversion symmetry of the reference cost]\label{prop:ref-sym}
If $\RefStruct$ is admissible, then for all $(s,o)\in S\times O$ one has
\[
c_\RefStruct(s,o)=\Jcost\!\Big(\frac{\iota_S(s)}{\iota_O(o)}\Big)=\Jcost\!\Big(\frac{\iota_O(o)}{\iota_S(s)}\Big).
\]
\end{proposition}

\begin{proof}
Immediate from admissibility and inversion symmetry $\Jcost(x)=\Jcost(x^{-1})$ (Definition~\ref{def:cost-axioms}(2)).
\end{proof}

\subsection{Meaning and symbols}

\begin{definition}[Meaning]\label{def:meaning}
Let $\RefStruct$ be a reference structure from $\Sym$ to $\Obj$.  A symbol $s\in S$ \emph{means} an object $o\in O$, written $\Mean_\RefStruct(s,o)$, if $o$ minimizes the reference cost among all objects:
\begin{equation}
    \Mean_\RefStruct(s,o)\iff \forall o'\in O,\quad c_\RefStruct(s,o)\le c_\RefStruct(s,o').
\end{equation}
\end{definition}

For each $s\in S$ we write
\[
    \Mean_\RefStruct(s):=\{o\in O:\Mean_\RefStruct(s,o)\}
\]
for the (possibly multi-valued) meaning set.  If $\RefStruct$ is admissible, then equivalently
\[
    \Mean_\RefStruct(s)=\operatorname*{arg\,min}_{o\in O}\ \Jcost\!\Big(\frac{\iota_S(s)}{\iota_O(o)}\Big).
\]

\begin{definition}[Symbol]\label{def:symbol}
Let $\RefStruct$ be a reference structure from $\Sym$ to $\Obj$.  A configuration $s\in S$ is a \emph{symbol} for an object $o\in O$ (relative to $\RefStruct$) if:
\begin{enumerate}
\item \textbf{Reference:} $\Mean_\RefStruct(s,o)$.
\item \textbf{Compression:} $J_S(s) < J_O(o)$.
\end{enumerate}
\end{definition}

The compression requirement is a modeling assumption: it enforces that symbols are lower-cost encodings than their referents in the common currency induced by $\Jcost$.  No empirical interpretation is asserted; the condition is simply part of the definition used in later results.


%==============================================================================
\section{Main Theorems}
\label{sec:main-theorems}
%==============================================================================


This section collects the main mathematical consequences of the ratio-induced reference model.  Throughout we fix the explicit mismatch functional
\begin{equation}\label{eq:Jcost-explicit-reminder}
\Jcost(x)=\frac{(x-1)^2}{2x}=\tfrac12\bigl(x+x^{-1}\bigr)-1\qquad (x>0)
\end{equation}
(which was verified to satisfy the axioms in Section~\ref{sec:Jcost}), and we assume the reference structure is \emph{admissible} (Definition~\ref{def:admissible-ref-struct}):
\begin{equation}\label{eq:main-admissible}
  c_{\RefStruct}(s,o)=\Jcost\!\left(\frac{\iota_S(s)}{\iota_O(o)}\right).
\end{equation}
Thus, for each $s\in S$, the meaning set $\Mean_{\RefStruct}(s)$ is the set of minimizers of $o\mapsto \Jcost(\iota_S(s)/\iota_O(o))$.

\subsection{Sublevel geometry of the explicit mismatch cost}

\begin{lemma}[Sublevel intervals]\label{lem:sublevel-interval}
Assume $\Jcost$ is given by \eqref{eq:Jcost} (equivalently \eqref{eq:Jcost-explicit-reminder}). For each $\epsilon>0$, the sublevel set
\[
L_\epsilon:=\{x\in \R_{>0}:\ \Jcost(x)\le \epsilon\}
\]
coincides with the closed interval $[a_\epsilon,b_\epsilon]$, where
\[
  b_\epsilon:=(1+\epsilon)+\sqrt{\epsilon(2+\epsilon)},\qquad
  a_\epsilon:=(1+\epsilon)-\sqrt{\epsilon(2+\epsilon)}=\frac{1}{b_\epsilon}.
\]
\end{lemma}

\begin{proof}
Using $\Jcost(x)=\frac{(x-1)^2}{2x}$, the inequality $\Jcost(x)\le \epsilon$ is equivalent (after multiplying by $2x>0$) to
\[
(x-1)^2\le 2\epsilon x
\ \Longleftrightarrow\ 
 x^2-2(1+\epsilon)x+1\le 0.
\]
The quadratic has discriminant $\Delta=4\epsilon(2+\epsilon)$ and roots
$x_\pm=(1+\epsilon)\pm\sqrt{\epsilon(2+\epsilon)}$.  Since it opens upward, the inequality holds exactly for $x\in[x_-,x_+]$.  Set $a_\epsilon:=x_-$ and $b_\epsilon:=x_+$.  Then $a_\epsilon b_\epsilon=(1+\epsilon)^2-\epsilon(2+\epsilon)=1$, so $a_\epsilon=1/b_\epsilon$.
\end{proof}

\subsection{Meaning constraints from a balanced baseline}

\begin{theorem}[Scale window for meanings of low-cost symbols]\label{thm:low-cost-meaning}
Assume $1\in Y:=\iota_O(O)$ and choose $o_0\in O$ with $\iota_O(o_0)=1$.  Let $s\in S$ and let $o\in \Mean_{\RefStruct}(s)$.
Then
\begin{equation}\label{eq:mean-cost-bound}
  c_{\RefStruct}(s,o)\le c_{\RefStruct}(s,o_0)=\Jcost(\iota_S(s))=J_S(s).
\end{equation}
In particular, for every $\epsilon>0$, if $J_S(s)\le \epsilon$ then
\begin{equation}\label{eq:ratio-window}
  \frac{\iota_S(s)}{\iota_O(o)}\in [a_\epsilon,b_\epsilon]
\end{equation}
and hence
\begin{equation}\label{eq:object-scale-window}
  \frac{\iota_S(s)}{b_\epsilon}\le \iota_O(o)\le \frac{\iota_S(s)}{a_\epsilon},
\end{equation}
where $[a_\epsilon,b_\epsilon]$ is as in Lemma~\ref{lem:sublevel-interval}.
\end{theorem}

\begin{proof}
Since $o\in\Mean_{\RefStruct}(s)$, by definition $c_{\RefStruct}(s,o)\le c_{\RefStruct}(s,o_0)$.  By admissibility \eqref{eq:main-admissible} and $\iota_O(o_0)=1$,
$c_{\RefStruct}(s,o_0)=\Jcost(\iota_S(s))=J_S(s)$, which gives \eqref{eq:mean-cost-bound}.  If $J_S(s)\le\epsilon$, then \eqref{eq:mean-cost-bound} implies $\Jcost(\iota_S(s)/\iota_O(o))\le\epsilon$, hence \eqref{eq:ratio-window} by Lemma~\ref{lem:sublevel-interval}.  Rearranging yields \eqref{eq:object-scale-window}.
\end{proof}

\begin{corollary}[Near-balanced symbols force near-balanced meanings]\label{cor:near-balanced}
Under the hypotheses of Theorem~\ref{thm:low-cost-meaning}, if $J_S(s)\le\epsilon$ and $o\in\Mean_{\RefStruct}(s)$, then
\[
\iota_O(o)\in \Bigl[\frac{1}{b_\epsilon^2},\ b_\epsilon^2\Bigr].
\]
In particular, as $\epsilon\downarrow 0$, any meaning of an $\epsilon$-cheap symbol must satisfy $\iota_O(o)\to 1$.
\end{corollary}

\begin{proof}
From $J_S(s)=\Jcost(\iota_S(s))\le\epsilon$ and Lemma~\ref{lem:sublevel-interval} we have $\iota_S(s)\in[a_\epsilon,b_\epsilon]$.  Combining this with \eqref{eq:object-scale-window} and $a_\epsilon=1/b_\epsilon$ gives the stated bounds.
\end{proof}

\subsection{Existence of meanings under attainment hypotheses}

\begin{lemma}[Coercivity of $\Jcost$]\label{lem:Jcost-coercive}
Assume $\Jcost$ is given by \eqref{eq:Jcost}. Then $\Jcost(x)\to\infty$ as $x\to 0^+$ and as $x\to\infty$.  In particular, for each $M\ge 0$ the sublevel set $\{x\in\R_{>0}:\ \Jcost(x)\le M\}$ is compact in $\R$.
\end{lemma}

\begin{proof}
From \eqref{eq:Jcost-explicit-reminder}, $\Jcost(x)=\tfrac12(x+x^{-1})-1$.  As $x\to\infty$ the term $\tfrac12x$ dominates, and as $x\to0^+$ the term $\tfrac12x^{-1}$ dominates, so in both limits $\Jcost(x)\to\infty$.  If $\Jcost(x)\le M$ then $x+x^{-1}\le 2(M+1)$, hence both $x$ and $x^{-1}$ are bounded; the sublevel set is therefore closed and bounded away from $0$ and $\infty$, hence compact.
\end{proof}

\begin{theorem}[Existence of meanings for ratio-induced reference]\label{thm:meaning-exists}
Assume $\RefStruct$ is admissible (Definition~\ref{def:admissible-ref-struct}) and that $\Jcost$ is given by \eqref{eq:Jcost}.  Let $Y:=\iota_O(O)\subset\R_{>0}$ be nonempty and closed.
Then for every $s\in S$ there exists $o\in O$ such that $\Mean_{\RefStruct}(s,o)$ (equivalently, $\Mean_{\RefStruct}(s)\neq\emptyset$).
Moreover, if $x:=\iota_S(s)\in Y$, then any $o\in O$ with $\iota_O(o)=x$ is a meaning and satisfies $c_{\RefStruct}(s,o)=0$.
\end{theorem}

\begin{proof}
Fix $s$ and set $x:=\iota_S(s)$. Consider $f:Y\to\R_{\ge 0}$ defined by $f(y):=\Jcost(x/y)$.  The map $f$ is continuous.  By Lemma~\ref{lem:Jcost-coercive}, $f(y)\to\infty$ as $y\to 0^+$ or $y\to\infty$, so the infimum of $f$ over $Y$ is achieved on a compact sublevel set.
Concretely, choose a minimizing sequence $y_n\in Y$ with $f(y_n)\downarrow \inf_Y f$.  Coercivity implies $(y_n)$ is bounded away from $0$ and $\infty$, hence has a convergent subsequence; since $Y$ is closed, the limit $y_*\in Y$, and continuity gives $f(y_*)=\inf_Y f$.  Choose $o\in O$ with $\iota_O(o)=y_*$.  Then $c_{\RefStruct}(s,o)=f(y_*)\le f(\iota_O(o'))=c_{\RefStruct}(s,o')$ for all $o'\in O$, i.e. $\Mean_{\RefStruct}(s,o)$.
If $x\in Y$, take $y_*=x$; then $\Jcost(x/x)=\Jcost(1)=0$, so any $o$ with $\iota_O(o)=x$ is a meaning with zero reference cost.
\end{proof}

\begin{remark}
If $Y=\iota_O(O)$ is not closed, the minimum need not be attained; in that case $\Mean_{\RefStruct}(s)$ may be empty even though the infimum exists.
\end{remark}

\subsection{A simple total-cost benchmark}

\begin{theorem}[Balanced reference minimizes the intrinsic+reference sum]\label{thm:optimal-ref}
Assume admissible reference \eqref{eq:main-admissible} and intrinsic costs $J_S(s)=\Jcost(\iota_S(s))$, $J_O(o)=\Jcost(\iota_O(o))$.  Define
\[
C(s,o):=J_S(s)+J_O(o)+c_{\RefStruct}(s,o).
\]
Then $C(s,o)\ge 0$ for all $(s,o)\in S\times O$, and
\[
C(s,o)=0 \iff \iota_S(s)=1\ \text{and}\ \iota_O(o)=1.
\]
In particular, if there exist $s_0\in S$ and $o_0\in O$ with $\iota_S(s_0)=\iota_O(o_0)=1$, then $(s_0,o_0)$ is a global minimizer of $C$ over $S\times O$.
\end{theorem}

\begin{proof}
Each term in $C$ is nonnegative, hence $C\ge 0$.  If $C(s,o)=0$, then all three terms vanish; by Lemma~\ref{lem:Jcost-zero} this forces $\iota_S(s)=\iota_O(o)=1$.  The converse is immediate from $\Jcost(1)=0$.
\end{proof}

\subsection{A backbone window for near-balanced symbol classes}

\begin{definition}[Referential capacity]\label{def:capacity}
Given a reference structure $\RefStruct$ from $\Sym$ to $\Obj$, define the \emph{referential capacity} to be
\[
\mathrm{Cap}(\Sym,\Obj;\RefStruct)
:=\bigl|\{o\in O:\ \exists s\in S\ \text{with}\ o\in\Mean_{\RefStruct}(s)\}\bigr|.
\]
(If $O$ is infinite, this cardinality may be infinite.)
\end{definition}

\begin{theorem}[Backbone window for near-balanced symbols]\label{thm:backbone}
Let $\Sym_\delta=(S_\delta,J_\delta,\iota_\delta)$ be the near-balanced ratio space
\[
S_\delta:=\{x\in\R_{>0}:\ |x-1|<\delta\},\qquad \iota_\delta=\mathrm{id},\qquad J_\delta=\Jcost|_{S_\delta}.
\]
Let $\Obj=(O,J_O,\iota_O)$ be a costed space such that $Y:=\iota_O(O)\subset\R_{>0}$ is nonempty, closed, and contains $1$.  Assume $\RefStruct$ is admissible and $\Jcost$ is given by \eqref{eq:Jcost}.

Set $\epsilon_\delta:=\Jcost(1+\delta)$ and let $[a_{\epsilon_\delta},b_{\epsilon_\delta}]$ be as in Lemma~\ref{lem:sublevel-interval}.  Define the window
\[
I_\delta
:=\left[\frac{1-\delta}{\,b_{\epsilon_\delta}\,},\ \frac{1+\delta}{\,a_{\epsilon_\delta}\,}\right].
\]
Then:
\begin{enumerate}
\item For every $s\in S_\delta$ the meaning set $\Mean_{\RefStruct}(s)$ is nonempty.
\item If $s\in S_\delta$ and $o\in\Mean_{\RefStruct}(s)$, then $\iota_O(o)\in I_\delta$.
Equivalently, if $\iota_O(o)\notin I_\delta$, then no $s\in S_\delta$ can mean $o$ under admissible reference.
\end{enumerate}
In particular,
\[
\mathrm{Cap}(\Sym_\delta,\Obj;\RefStruct)\le \bigl|\{o\in O:\ \iota_O(o)\in I_\delta\}\bigr|.
\]
\end{theorem}

\begin{proof}
(1) is a direct application of Theorem~\ref{thm:meaning-exists} to the closed nonempty set $Y$.

For (2), fix $s\in S_\delta$ and write $x:=\iota_\delta(s)\in(1-\delta,1+\delta)$.  Let $o\in\Mean_{\RefStruct}(s)$ and choose $o_0\in O$ with $\iota_O(o_0)=1$ (possible since $1\in Y$).  By Theorem~\ref{thm:low-cost-meaning},
\[
\Jcost\!\left(\frac{x}{\iota_O(o)}\right)=c_{\RefStruct}(s,o)\le c_{\RefStruct}(s,o_0)=\Jcost(x)\le \Jcost(1+\delta)=\epsilon_\delta.
\]
Applying Lemma~\ref{lem:sublevel-interval} gives $x/\iota_O(o)\in[a_{\epsilon_\delta},b_{\epsilon_\delta}]$, hence
\[
\frac{x}{b_{\epsilon_\delta}}\le \iota_O(o)\le \frac{x}{a_{\epsilon_\delta}}.
\]
Using $x\in[1-\delta,1+\delta]$ yields $\iota_O(o)\in I_\delta$.

For the capacity bound, any object counted in $\mathrm{Cap}(\Sym_\delta,\Obj;\RefStruct)$ lies in $\Mean_{\RefStruct}(s)$ for some $s\in S_\delta$, hence satisfies $\iota_O(o)\in I_\delta$ by (2).
\end{proof}

\begin{corollary}[Local effectiveness]\label{cor:effectiveness-local}
Assume an admissible (ratio-induced) reference structure $\RefStruct$ and the hypotheses of Theorem~\ref{thm:backbone}. If $s\in S_\delta$ and $o\in\Mean_{\RefStruct}(s)$, then $\iota_O(o)\in I_\delta$.
\end{corollary}

\begin{proof}
Immediate from Theorem~\ref{thm:backbone}(2).
\end{proof}

\begin{remark}\label{rem:effectiveness-comment}
In this direct admissible model, restricting attention to the near-balanced class $S_\delta$ forces all meanings to lie in the fixed scale window $I_\delta$. Any extension of the framework that permits meanings outside $I_\delta$ must, in particular, enlarge the available symbol class and/or modify the reference mechanism (e.g. via product composition or sequential mediation as in Section~\ref{sec:compositionality}).
\end{remark}


%==============================================================================

\section{Compositionality}\label{sec:compositionality}
%==============================================================================


This section records two elementary composition mechanisms for reference costs:
\emph{(i)} product composition (independent coordinates) and \emph{(ii)} sequential mediation through an intermediate space.
Both are purely variational constructions: they introduce no semantic primitive beyond the cost function(s).

\subsection{Product reference and coordinatewise meaning}

\begin{definition}[Product reference]\label{def:product-ref}
Let $\RefStruct_1$ be a reference structure from a symbol set $S_1$ to an object set $O_1$, and let $\RefStruct_2$ be a reference structure from a symbol set $S_2$ to an object set $O_2$.
Write their costs as $c_{\RefStruct_i}$. The \emph{product reference structure} $\RefStruct_1\otimes\RefStruct_2: S_1\times S_2\to O_1\times O_2$ is defined by
\begin{equation}\label{eq:product-cost}
  c_{\RefStruct_1\otimes\RefStruct_2}\big((s_1,s_2),(o_1,o_2)\big)
  :=c_{\RefStruct_1}(s_1,o_1)+c_{\RefStruct_2}(s_2,o_2).
\end{equation}
\end{definition}

\begin{theorem}[Compositionality of product meaning]\label{thm:comp}
For any reference structures $\RefStruct_1,\RefStruct_2$ and their product $\RefStruct_1\otimes\RefStruct_2$,
for all $(s_1,s_2)\in S_1\times S_2$ and $(o_1,o_2)\in O_1\times O_2$ one has
\[
\Mean_{\RefStruct_1\otimes\RefStruct_2}\big((s_1,s_2),(o_1,o_2)\big)
\iff
\Mean_{\RefStruct_1}(s_1,o_1)\ \text{and}\ \Mean_{\RefStruct_2}(s_2,o_2).
\]
Equivalently,
\[
\Mean_{\RefStruct_1\otimes\RefStruct_2}(s_1,s_2)
=\Mean_{\RefStruct_1}(s_1)\times\Mean_{\RefStruct_2}(s_2).
\]
\end{theorem}

\begin{proof}
Fix $(s_1,s_2)$ and define $f_1(o_1):=c_{\RefStruct_1}(s_1,o_1)$ on $O_1$ and $f_2(o_2):=c_{\RefStruct_2}(s_2,o_2)$ on $O_2$.
By \eqref{eq:product-cost}, the product cost equals $f_1(o_1)+f_2(o_2)$.

If $\Mean_{\RefStruct_1}(s_1,o_1)$ and $\Mean_{\RefStruct_2}(s_2,o_2)$, then for all $(o_1',o_2')$ we have
$f_1(o_1)\le f_1(o_1')$ and $f_2(o_2)\le f_2(o_2')$, hence
$f_1(o_1)+f_2(o_2)\le f_1(o_1')+f_2(o_2')$, which is exactly $\Mean_{\RefStruct_1\otimes\RefStruct_2}\big((s_1,s_2),(o_1,o_2)\big)$.

Conversely, if $(o_1,o_2)$ minimizes $f_1+f_2$ on $O_1\times O_2$, then for any $o_1'\in O_1$,
\[
  f_1(o_1)+f_2(o_2)\le f_1(o_1')+f_2(o_2),
\]
so $f_1(o_1)\le f_1(o_1')$, i.e. $\Mean_{\RefStruct_1}(s_1,o_1)$.
The same argument with $o_2'$ yields $\Mean_{\RefStruct_2}(s_2,o_2)$.
\end{proof}

\begin{corollary}[Existence of product meanings under the explicit mismatch cost]\label{cor:product-meaning-exists}
Assume the explicit mismatch cost \eqref{eq:Jcost-explicit-reminder} and admissible reference on each component.
If, for $i=1,2$, the object ratio set $Y_{O_i}:=\iota_{O_i}(O_i)\subset\mathbb{R}_{>0}$ is nonempty and closed, then
for every $(s_1,s_2)\in S_1\times S_2$ the product meaning set $\Mean_{\RefStruct_1\otimes \RefStruct_2}(s_1,s_2)$ is nonempty.
\end{corollary}

\begin{proof}
Under the stated hypotheses, Theorem~\ref{thm:meaning-exists} implies $\Mean_{\RefStruct_i}(s_i)\neq\emptyset$ for each $i$.
Pick $o_i\in\Mean_{\RefStruct_i}(s_i)$.
Then Theorem~\ref{thm:comp} yields $(o_1,o_2)\in\Mean_{\RefStruct_1\otimes\RefStruct_2}(s_1,s_2)$.
\end{proof}

\subsection{Sequential mediation}

\begin{definition}[Sequential reference]\label{def:sequential-ref}
Let $\RefStruct_1 : \Sym \to \mathcal{M}$ and $\RefStruct_2 : \mathcal{M} \to \Obj$ be reference structures.
Their \emph{sequential composition} $\RefStruct_2 \circ \RefStruct_1 : \Sym \to \Obj$ is defined by the infimal convolution
\begin{equation}\label{eq:sequential-cost}
    c_{\RefStruct_2 \circ \RefStruct_1}(s, o) = \inf_{m \in \mathcal{M}} \left[ c_{\RefStruct_1}(s, m) + c_{\RefStruct_2}(m, o) \right].
\end{equation}
A \emph{mediator} $m$ is \emph{optimal} for $(s,o)$ if it attains the infimum in \eqref{eq:sequential-cost}.
\end{definition}

\begin{theorem}[Geometric-mean mediator for the explicit mismatch cost]\label{thm:seq-mediator}
Assume the explicit mismatch functional \eqref{eq:Jcost-explicit-reminder} and admissible reference for
$\RefStruct_1: \Sym\to\mathcal{M}$ and $\RefStruct_2: \mathcal{M}\to\Obj$ with scale maps $\iota_S,\iota_M,\iota_O$.
Fix $s\in S$ and $o\in O$ and set $a:=\iota_S(s)$ and $c:=\iota_O(o)$.
Let $Y_M:=\iota_M(\mathcal{M})\subset\mathbb{R}_{>0}$.
If $Y_M$ is closed and contains $b_*:=\sqrt{ac}$, then the infimum in \eqref{eq:sequential-cost} is attained by some $m_*\in\mathcal{M}$ with $\iota_M(m_*)=b_*$, and
\[
  c_{\RefStruct_2 \circ \RefStruct_1}(s,o)=\Jcost\!\left(\frac{a}{b_*}\right)+\Jcost\!\left(\frac{b_*}{c}\right)
  =2\,\Jcost\!\left(\sqrt{\frac{a}{c}}\right).
\]
Moreover, at the level of mediator ratios $b\in(0,\infty)$ this minimizer is unique.
\end{theorem}

\begin{proof}
Under admissibility, the objective depends on $m$ only through $b:=\iota_M(m)\in Y_M$, namely
\[
F(b):=\Jcost\!\left(\frac{a}{b}\right)+\Jcost\!\left(\frac{b}{c}\right).
\]
Using the explicit form \eqref{eq:Jcost}, $\Jcost(x)=\tfrac12(x+x^{-1})-1$, we obtain
\[
F(b)=\tfrac12\left(\frac{a}{b}+\frac{b}{a}+\frac{b}{c}+\frac{c}{b}\right)-2.
\]
A direct computation gives
\[
F'(b)=\tfrac12\left(-\frac{a+c}{b^2}+\frac{1}{a}+\frac{1}{c}\right),
\qquad
F''(b)=\frac{a+c}{b^3}>0,
\]
so $F$ is strictly convex on $(0,\infty)$ and therefore has at most one minimizer.
The unique critical point solves $F'(b)=0$, equivalently $b^2=ac$, hence $b=b_*:=\sqrt{ac}$.
By hypothesis $b_*\in Y_M$ and closedness of $Y_M$, the infimum over $Y_M$ is attained at $b_*$, realized by some $m_*\in\mathcal{M}$ with $\iota_M(m_*)=b_*$.
Substituting $b_*=\sqrt{ac}$ yields
$\Jcost(a/b_*)=\Jcost(\sqrt{a/c})=\Jcost(b_*/c)$ and the stated formula.
\end{proof}

\begin{corollary}[Mediation can strictly reduce mismatch]\label{cor:mediation-reduces}
For every $x>0$ one has
\[
2\,\Jcost(\sqrt{x})\le \Jcost(x),
\]
with equality if and only if $x=1$.
Consequently, if a direct admissible reference $\RefStruct:\Sym\to\Obj$ is available (built from the same $\Jcost$ and scale maps), then
\[
  c_{\RefStruct_2\circ\RefStruct_1}(s,o)\le c_{\RefStruct}(s,o),
\]
with equality if and only if $\iota_S(s)=\iota_O(o)$.
\end{corollary}

\begin{proof}
Let $t:=\sqrt{x}>0$. Using \eqref{eq:Jcost}, a direct calculation gives
\[
\Jcost(t^2)-2\,\Jcost(t)=\frac12\Bigl((t-1)^2+\bigl(t^{-1}-1\bigr)^2\Bigr)\ge 0,
\]
with equality if and only if $t=1$, i.e. $x=1$.
The final inequality follows by substituting $x=\iota_S(s)/\iota_O(o)$ for the direct admissible cost.
\end{proof}


%==============================================================================
%==============================================================================
\section{Extensions: Multi-dimensional Scales and Robustness}\label{sec:extensions}
%==============================================================================


The core framework above uses a single positive scale coordinate $\iota(\cdot)\in\R_{>0}$.  In some applications one may want a 
finite list of independent scale coordinates (for instance, a symbol might carry multiple features, each measured in the same 
``cost currency'' through $\Jcost$).  This section records a minimal extension of the model to $d$ coordinates and a simple robustness 
lemma for finite dictionaries.

\subsection{Multi-dimensional costed spaces}

\begin{definition}[Multi-dimensional costed space]\label{def:md-costed-space}
Let $d\in\N$.  A \emph{$d$-dimensional costed space} is a triple $(C,J_C,\iota_C)$ where
\begin{itemize}
\item $C$ is a set,
\item $\iota_C:C\to(\R_{>0})^d$ is a scale map, and
\item $J_C:C\to\R_{\ge 0}$ is the induced (separable) cost
\[
  J_C(c):=\sum_{i=1}^d \Jcost\big(\iota_C(c)_i\big),\qquad c\in C.
\]
\end{itemize}
\end{definition}

\begin{definition}[Multi-dimensional admissible reference]\label{def:md-admissible-ref}
Let $(S,J_S,\iota_S)$ and $(O,J_O,\iota_O)$ be $d$-dimensional costed spaces.  A reference structure $\RefStruct$ from $S$ to $O$ is
\emph{multi-dimensionally admissible} if its reference cost is the coordinatewise ratio cost
\begin{equation}\label{eq:md-admissible}
  c_{\RefStruct}(s,o)
  =\sum_{i=1}^d \Jcost\!\left(\frac{\iota_S(s)_i}{\iota_O(o)_i}\right),\qquad (s,o)\in S\times O.
\end{equation}
\end{definition}

\begin{corollary}[Coordinatewise meaning for product models]\label{thm:md-product-meaning}
Assume $S=\prod_{i=1}^d S_i$ and $O=\prod_{i=1}^d O_i$ and that the scale maps factor coordinatewise:
$\iota_S(s)_i=\iota_{S_i}(s_i)$ and $\iota_O(o)_i=\iota_{O_i}(o_i)$.  If $\RefStruct$ is multi-dimensionally admissible, then
\[
(o_1,\dots,o_d)\in\Mean_{\RefStruct}(s_1,\dots,s_d)
\quad\Longleftrightarrow\quad
\forall i,\; o_i\in\Mean_{\RefStruct_i}(s_i),
\]
where $\RefStruct_i$ denotes the induced one-dimensional admissible reference on $(S_i,O_i)$.
\end{corollary}

\begin{proof}
By \eqref{eq:md-admissible} the cost is a separable sum of $d$ nonnegative terms, each depending only on $(s_i,o_i)$.  Thus minimizing over
$O=\prod_i O_i$ is equivalent to minimizing each summand over its coordinate; this is the same argument as in Theorem~\ref{thm:comp}.
\end{proof}

\subsection{Log-space geometry for the explicit mismatch cost}

In this subsection we specialize to the explicit mismatch functional
\begin{equation}\label{eq:Jcost-explicit-reminder-6}
\Jcost(x)=\tfrac12(x+x^{-1})-1\qquad (x>0),
\end{equation}
already used in Sections~\ref{sec:Jcost}--\ref{sec:compositionality}.

\begin{lemma}[Log-coordinate form]\label{lem:log-cosh}
For all $t\in\R$ one has $\Jcost(e^t)=\cosh(t)-1$.
\end{lemma}

\begin{proof}
Immediate from \eqref{eq:Jcost-explicit-reminder-6}: $\Jcost(e^t)=\tfrac12(e^t+e^{-t})-1=\cosh(t)-1$.
\end{proof}

\begin{proposition}[Quadratic regime with explicit remainder]\label{prop:quadratic-bound}
For all $t\in\R$,
\[
0\le \Jcost(e^t)-\frac{t^2}{2}\le \frac{t^4}{24}\,\cosh(|t|).
\]
In particular, for $|t|\le 1$,
\[
\frac{t^2}{2}\le \Jcost(e^t)\le \frac{t^2}{2}+\frac{\cosh(1)}{24}\,t^4.
\]
\end{proposition}

\begin{proof}
By Lemma~\ref{lem:log-cosh} it suffices to estimate $\cosh(t)-1-\tfrac12 t^2$.  Taylor's theorem at $0$ with remainder gives
\[
\cosh(t)=1+\frac{t^2}{2}+\frac{t^4}{24}\cosh(\xi)
\]
for some $\xi$ between $0$ and $t$.  Since $\cosh$ is even and increasing on $\R_{\ge 0}$, one has $\cosh(\xi)\le \cosh(|t|)$, yielding the upper bound.
Nonnegativity follows since $\cosh(\xi)>0$.
\end{proof}

\begin{corollary}[Local Euclidean geometry in log-ratio]\label{cor:local-euclid}
For the explicit mismatch cost \eqref{eq:Jcost-explicit-reminder-6}, set $x:=\iota_S(s)$ and $y:=\iota_O(o)$.  If $|\log(x/y)|\le 1$, then
\[
\frac12\big(\log(x/y)\big)^2\le c_{\RefStruct}(s,o)\le \frac12\big(\log(x/y)\big)^2+\frac{\cosh(1)}{24}\big(\log(x/y)\big)^4.
\]
Thus, in the small-mismatch regime, meanings behave like nearest neighbors in the log-ratio metric.
\end{corollary}

\begin{proof}
For admissible reference, $c_{\RefStruct}(s,o)=\Jcost(x/y)$ with $x:=\iota_S(s)$ and $y:=\iota_O(o)$.
Write $t:=\log(x/y)$. Then $x/y=e^{t}$ and $|t|\le 1$ by hypothesis.
Apply Proposition~\ref{prop:quadratic-bound} to obtain
\(\tfrac12 t^2\le \Jcost(e^{t})\le \tfrac12 t^2+\tfrac{\cosh(1)}{24}t^4\), and substitute $t=\log(x/y)$.
\end{proof}




\subsection{Margin stability for finite dictionaries}

\begin{definition}[Decision margin]\label{def:margin}
Fix a symbol $s\in S$ and a finite object dictionary $O=\{o_1,\dots,o_N\}$.  Write $C_k:=c_{\RefStruct}(s,o_k)$ and let
$M:=\min_{1\le k\le N} C_k$.  The \emph{decision margin} at $s$ is
\[
\Delta(s):=\min\{C_k-M:\ 1\le k\le N,\ C_k>M\}\in[0,\infty],
\]
with the convention $\Delta(s)=\infty$ if all $C_k$ are equal.
\end{definition}

\begin{proposition}[Robustness under bounded perturbations]\label{prop:margin-stability}
In the setting of Definition~\ref{def:margin}, suppose the costs $C_k$ are perturbed to numbers $\widetilde C_k$ satisfying
\[
\max_{1\le k\le N}|\widetilde C_k-C_k|\le \eta.
\]
If $\Delta(s)>2\eta$, then the set of minimizers is unchanged:
\[
\{k: C_k=\min_j C_j\}=\{k: \widetilde C_k=\min_j \widetilde C_j\}.
\]
\end{proposition}

\begin{proof}
Let $I:=\{k: C_k=M\}$ be the (nonempty) set of original minimizers.  For $k\in I$ one has $\widetilde C_k\le M+\eta$.
If $k\notin I$, then $C_k\ge M+\Delta(s)$ by definition of $\Delta(s)$, hence $\widetilde C_k\ge M+\Delta(s)-\eta$.
If $\Delta(s)>2\eta$ then $M+\Delta(s)-\eta> M+\eta$, so every perturbed minimizer must lie in $I$ and conversely every $k\in I$ remains minimal.
\end{proof}

\subsection{Existence (and optional uniqueness) in $d$ dimensions}

\begin{theorem}[Existence of meanings for multi-dimensional admissible reference]\label{thm:md-meaning-exists}
Let $d\in\N$ and let $(S,J_S,\iota_S)$ and $(O,J_O,\iota_O)$ be $d$-dimensional costed spaces.
Assume $\RefStruct$ is multi-dimensionally admissible in the sense of Definition~\ref{def:md-admissible-ref}.
Let $Y:=\iota_O(O)\subset (\R_{>0})^d$ be nonempty and closed.
Then for every $s\in S$ the meaning set $\Mean_{\RefStruct}(s)$ is nonempty.
Moreover, if $x:=\iota_S(s)$ lies in $Y$, then any $o\in O$ with $\iota_O(o)=x$ is a meaning and satisfies $c_{\RefStruct}(s,o)=0$.
\end{theorem}

\begin{proof}
Fix $s\in S$ and write $x:=\iota_S(s)\in (\R_{>0})^d$.  Consider the continuous objective on $Y$,
\[
F_x(y):=\sum_{i=1}^d \Jcost\!\left(\frac{x_i}{y_i}\right),\qquad y=(y_1,\dots,y_d)\in Y.
\]
By Lemma~\ref{lem:Jcost-coercive}, for each $M\ge 0$ the one-dimensional sublevel set $K_M:=\{z>0:\ \Jcost(z)\le M\}$ is compact.
Hence there exist $0<a_M\le 1\le b_M<\infty$ such that $K_M\subset[a_M,b_M]$.
If $F_x(y)\le M$ then each term satisfies $\Jcost(x_i/y_i)\le M$, so $x_i/y_i\in K_M\subset[a_M,b_M]$, i.e.
\[
  \frac{x_i}{b_M}\le y_i\le \frac{x_i}{a_M}\qquad (i=1,\dots,d).
\]
Therefore the sublevel set $\{y\in Y: F_x(y)\le M\}$ is closed and contained in the bounded box $\prod_i [x_i/b_M,\ x_i/a_M]$,
so it is compact (Heine--Borel).  Thus $F_x$ attains its minimum on $Y$ at some $y_*\in Y$.
Choose $o\in O$ with $\iota_O(o)=y_*$; then $o\in\Mean_{\RefStruct}(s)$ by \eqref{eq:md-admissible}.

If $x\in Y$, then $F_x(x)=\sum_i \Jcost(1)=0$.  Since each term is nonnegative, $0$ is the global minimum, so any $o$ with $\iota_O(o)=x$ is a meaning.
\end{proof}

\begin{definition}[Log-image and log-convexity]\label{def:log-dictionary}
For $Y\subset(\R_{>0})^d$ define
\[
\log Y:=\{(\log y_1,\dots,\log y_d):\ y\in Y\}\subset\R^d.
\]
We call $Y$ \emph{log-convex} if $\log Y$ is convex.
\end{definition}

\begin{theorem}[Uniqueness and continuity for log-convex dictionaries]\label{thm:md-unique-continuity}
Assume the explicit mismatch cost \eqref{eq:Jcost-explicit-reminder-6} and the hypotheses of Theorem~\ref{thm:md-meaning-exists}.
If $U:=\log Y\subset\R^d$ is closed and convex, then the minimizer $y_*(x)\in Y$ of $F_x$ is unique.
Equivalently, the meaning set $\Mean_{\RefStruct}(s)$ equals the fiber $\{o\in O:\ \iota_O(o)=y_*(\iota_S(s))\}$.
Moreover, the optimizer is continuous in log-coordinates: the map $t\mapsto u_*(t)$ is continuous, where $t:=\log x$ and $u_*(t):=\log y_*(e^t)\in U$.
\end{theorem}

\begin{proof}
Let $t:=\log x\in\R^d$ and write $u:=\log y\in U$.  By Lemma~\ref{lem:log-cosh},
\[
F_x(y)=\sum_{i=1}^d (\cosh(t_i-u_i)-1)=:G_t(u).
\]
For each $i$, the map $u_i\mapsto \cosh(t_i-u_i)-1$ is strictly convex, hence $G_t$ is strictly convex on $\R^d$.
Restricting to the convex set $U$ preserves strict convexity, so $G_t$ has at most one minimizer on $U$; existence follows from Theorem~\ref{thm:md-meaning-exists}.
Thus the optimizer $u_*(t)$ is unique, and so is $y_*(x)=e^{u_*(\log x)}$.

For continuity, let $t_n\to t$ and set $u_n:=u_*(t_n)\in U$.  Fix $u_0\in U$.
Since $u_n$ minimizes $G_{t_n}$ on $U$, one has $G_{t_n}(u_n)\le G_{t_n}(u_0)$.
The right-hand side is bounded because $(t,u)\mapsto G_t(u)$ is continuous and $t_n\to t$.
As in the proof of Theorem~\ref{thm:md-meaning-exists}, boundedness of $G_{t_n}(u_n)$ implies boundedness of $\{u_n\}$ in $\R^d$.
Passing to a convergent subsequence (still denoted $u_n$) with limit $\bar u\in U$ (closedness), continuity gives
$G_t(\bar u)=\lim_n G_{t_n}(u_n)\le \lim_n G_{t_n}(u)=G_t(u)$ for all $u\in U$.
Hence $\bar u$ minimizes $G_t$ on $U$, and by uniqueness $\bar u=u_*(t)$.
Therefore every subsequence has the same limit, so $u_n\to u_*(t)$ and continuity holds.
\end{proof}


\section{Worked Examples}\label{sec:examples}
%==============================================================================


This section gives explicit computations in simple settings. The purpose is not to add new axioms, but to make the definition of meaning
\(\Mean_{\RefStruct}(s)=\operatorname*{arg\,min}_{o\in O}\,\Jcost(\iota_S(s)/\iota_O(o))\)
concrete and to illustrate the decision-geometry proved earlier.

\subsection{Continuous ratio model}

\begin{proposition}[Meaning in the continuous ratio model]\label{prop:continuous-model}
Let $S=O=\R_{>0}$ with $\iota_S=\iota_O=\mathrm{id}$ and intrinsic costs $J_S=J_O=\Jcost$.  Let $\RefStruct$ be admissible (Definition~\ref{def:admissible-ref-struct}), so that
\[
  c_{\RefStruct}(s,o)=\Jcost\!\left(\frac{s}{o}\right).
\]
Then for every $s\in\R_{>0}$ there exists a \emph{unique} meaning, namely $\Mean_{\RefStruct}(s)=\{s\}$, and the minimum reference cost equals $0$.
\end{proposition}

\begin{proof}
By Lemma~\ref{lem:Jcost-zero}, one has $\Jcost(x)\ge 0$ for all $x>0$ with equality if and only if $x=1$.  Hence $c_{\RefStruct}(s,o)=\Jcost(s/o)\ge 0$ with equality if and only if $s/o=1$, i.e.\ $o=s$.  Therefore $o=s$ is the unique minimizer and the minimum cost is $0$.
\end{proof}

\subsection{Finite dictionaries and boundary points}

\begin{example}[Finite object dictionary]\label{ex:finite-dictionary}
Let $O=\{o_1,\dots,o_n\}$ be finite, set $y_i:=\iota_O(o_i)$, and keep $S=\R_{>0}$ with $\iota_S=\mathrm{id}$.  Under admissible reference, for a given symbol $s$ with ratio $x:=\iota_S(s)$ the meaning set is
\[
  \Mean_{\RefStruct}(s)=\Bigl\{o_i:\ \Jcost\!\left(\frac{x}{y_i}\right)=\min_{1\le j\le n}\Jcost\!\left(\frac{x}{y_j}\right)\Bigr\}.
\]
In general, boundary points (where the meaning set is not a singleton) occur when two or more of the values $\Jcost(x/y_i)$ tie.
\end{example}

\subsection{Geometric-mean boundaries for the explicit mismatch cost}

\begin{theorem}[Geometric-mean decision boundaries for the explicit mismatch cost]\label{thm:geom-boundaries}
Assume the explicit mismatch functional \eqref{eq:Jcost} and admissible (ratio-induced) reference
$c_{\RefStruct}(s,o)=\Jcost(\iota_S(s)/\iota_O(o))$.
Let $O=\{o_1,\dots,o_N\}$ be a finite object set such that the ratios $y_i:=\iota_O(o_i)$ are pairwise distinct and ordered $0<y_1<\cdots<y_N$.
For $x:=\iota_S(s)\in\R_{>0}$ define the boundary points
\[
  m_i:=\sqrt{y_i y_{i+1}}\qquad (i=1,\dots,N-1),
\]
and set $m_0:=0$, $m_N:=+\infty$.
Then:
\begin{itemize}
\item If $m_{k-1}<x<m_k$ for some $k\in\{1,\dots,N\}$, then $o_k$ is the \emph{unique} meaning of $s$.
\item If $x=m_k$ for some $k\in\{1,\dots,N-1\}$, then $s$ has \emph{exactly two} meanings, namely $o_k$ and $o_{k+1}$.
\end{itemize}
Equivalently, the map $x\mapsto \operatorname*{arg\,min}_{i}\Jcost(x/y_i)$ is piecewise constant on the open intervals $(m_{k-1},m_k)$.
\end{theorem}

\begin{proof}
Using \eqref{eq:Jcost} one computes, for each $i$,
\[
  c_{\RefStruct}(s,o_i)=\Jcost\!\left(\frac{x}{y_i}\right)=\frac{\bigl(\frac{x}{y_i}-1\bigr)^2}{2(x/y_i)}=\frac{(x-y_i)^2}{2x y_i}.
\]
Fix $i\in\{1,\dots,N-1\}$ and define the adjacent difference
\[
  \Delta_i(x):=c_{\RefStruct}(s,o_{i+1})-c_{\RefStruct}(s,o_i).
\]
Multiplying by $2x>0$ and simplifying gives
\[
  2x\,\Delta_i(x)=(y_{i+1}-y_i)\Bigl(1-\frac{x^2}{y_i y_{i+1}}\Bigr).
\]
Hence $\Delta_i(x)=0$ if and only if $x^2=y_i y_{i+1}$, i.e.\ $x=m_i$.  Moreover,
\(\Delta_i(x)>0\) when $x<m_i$ and \(\Delta_i(x)<0\) when $x>m_i$.
Therefore:
\begin{itemize}
\item if $x<m_i$ then $c_{\RefStruct}(s,o_i)<c_{\RefStruct}(s,o_{i+1})$ (so the adjacent comparison favors $o_i$),
\item if $x>m_i$ then $c_{\RefStruct}(s,o_{i+1})<c_{\RefStruct}(s,o_i)$ (so it favors $o_{i+1}$).
\end{itemize}
Fix $k\in\{1,\dots,N\}$ such that $m_{k-1}<x<m_k$.  For every $i\le k-1$ we have $x>m_i$, hence $\Delta_i(x)<0$, so
\(c_{\RefStruct}(s,o_{i+1})<c_{\RefStruct}(s,o_i)\).  Iterating these strict inequalities yields
\(c_{\RefStruct}(s,o_k)<c_{\RefStruct}(s,o_i)\) for all $i<k$.  For every $i\ge k$ we have $x<m_i$, hence $\Delta_i(x)>0$, so
\(c_{\RefStruct}(s,o_{i+1})>c_{\RefStruct}(s,o_i)\).  Iterating yields
\(c_{\RefStruct}(s,o_k)<c_{\RefStruct}(s,o_j)\) for all $j>k$.  Therefore $o_k$ is the unique minimizer.

If $x=m_k$ for some $k\in\{1,\dots,N-1\}$, then for every $i<k$ we still have $x>m_i$ and the costs strictly decrease up to index $k$, while for every $i\ge k+1$ we have $x<m_i$ and the costs strictly increase from index $k+1$ onward.  At $i=k$ one has $\Delta_k(m_k)=0$, i.e.\ $c_{\RefStruct}(s,o_k)=c_{\RefStruct}(s,o_{k+1})$.  Hence the argmin consists of exactly two meanings, $\{o_k,o_{k+1}\}$.
\end{proof}

\begin{corollary}[Stability away from boundaries]\label{cor:stability-away}
Under the hypotheses of Theorem~\ref{thm:geom-boundaries}, if $m_{k-1}<x<m_k$ then there exists $\delta>0$ such that every $x'$ with $|x'-x|<\delta$ satisfies $m_{k-1}<x'<m_k$ and hence has the same unique meaning $o_k$.
\end{corollary}

\begin{proof}
Since $(m_{k-1},m_k)$ is open and contains $x$, choose $\delta:=\min\{x-m_{k-1},\,m_k-x\}/2>0$.  Then $|x'-x|<\delta$ implies $x'\in(m_{k-1},m_k)$, and the conclusion follows from Theorem~\ref{thm:geom-boundaries}.
\end{proof}

\subsection{Numerical micro-example (three-object dictionary)}

Take $O=\{o_1,o_2,o_3\}$ with ratios $y_1=\tfrac14<y_2=1<y_3=4$, and keep $S=\R_{>0}$ with $\iota_S=\mathrm{id}$.  The boundary points are
$m_1=\sqrt{y_1y_2}=\tfrac12$ and $m_2=\sqrt{y_2y_3}=2$.
Thus a symbol with ratio $x$ means $o_1$ for $0<x<\tfrac12$, means $o_2$ for $\tfrac12<x<2$, and means $o_3$ for $x>2$ (with ties at the boundary points).

\begin{center}
\begin{tabular}{|c|c|c|c|c|}
\hline
$x$ & $c_{\RefStruct}(s,o_1)$ & $c_{\RefStruct}(s,o_2)$ & $c_{\RefStruct}(s,o_3)$ & meaning(s) \\
\hline
$\tfrac{3}{10}$ & $\tfrac{1}{60}$ & $\tfrac{49}{60}$ & $\tfrac{1369}{240}$ & $o_1$ \\
$\tfrac{3}{2}$  & $\tfrac{25}{12}$ & $\tfrac{1}{12}$ & $\tfrac{25}{48}$ & $o_2$ \\
$3$             & $\tfrac{121}{24}$ & $\tfrac{2}{3}$ & $\tfrac{1}{24}$ & $o_3$ \\
\hline
\end{tabular}
\end{center}

\begin{example}[Mediation can sharply reduce cost in a toy case]\label{ex:mediation-toy}
Let $a:=\iota_S(s)=4$ and $c:=\iota_O(o)=\tfrac14$, so the direct admissible reference cost is
$\Jcost(a/c)=\Jcost(16)=\tfrac{225}{32}$.
If the mediator space contains a configuration $m$ with ratio $b_*:=\sqrt{ac}=1$, then Theorem~\ref{thm:seq-mediator} gives an optimal sequential cost
\[
  c_{\RefStruct_2\circ\RefStruct_1}(s,o)=2\,\Jcost\!\left(\sqrt{\frac{a}{c}}\right)=2\,\Jcost(4)=\frac{9}{4},
\]
which is strictly smaller, in accordance with Corollary~\ref{cor:mediation-reduces}.
\end{example}


\section{Applications}\label{sec:applications}
%==============================================================================

\editamir{This section collects immediate, checkable consequences of the formal development. Each statement below follows from earlier definitions and theorems, and no external or empirical claim is being made. The meaning rule is an optimization rule (Definition~\ref{def:meaning}) driven by the canonical mismatch penalty \Jcost (Definition~\ref{def:Jcost}); the axiomatic characterization of \Jcost is classical and recorded for completeness in Appendix~\ref{app:dalembert}.}

\subsection{Symbol grounding as a criterion}

\editamir{We treat ``grounding'' as an internal consistency condition in this model: a token $s$ is grounded for an object $o$ when (i) $o$ is a meaning of $s$ (Definition~\ref{def:meaning}) and (ii) the symbol condition $J_S(s)<J_O(o)$ holds (Definition~\ref{def:symbol}).}

\begin{corollary}[Grounding criterion under admissible reference]\label{cor:grounding}
Fix an admissible reference structure $\RefStruct$ (Definition~\ref{def:admissible-ref-struct}). Then, for $s\in S$ and $o\in O$,
\[
(s,o)\text{ is a symbol (Definition~\ref{def:symbol})}\quad\Longleftrightarrow\quad
\Mean_{\RefStruct}(s,o)\ \text{ and }\ J_S(s)<J_O(o).
\]
\end{corollary}

\begin{proof}
This is immediate from Definition~\ref{def:symbol} and Definition~\ref{def:meaning}.
\end{proof}

\begin{corollary}[Grounding rule for finite object dictionaries]\label{cor:grounding-dict}
Assume the finite-dictionary hypotheses of Theorem~\ref{thm:geom-boundaries}. Then the meaning map $x\mapsto \Mean(s,\cdot)$ is piecewise constant in the symbol ratio $x=\iota_S(s)$, with decision boundaries at geometric means $m_i=\sqrt{y_i y_{i+1}}$. In particular, away from the boundaries the meaning is stable under small perturbations (Corollary~\ref{cor:stability-away}).
\end{corollary}

\begin{proof}
Immediate from Theorem~\ref{thm:geom-boundaries} and Corollary~\ref{cor:stability-away}.
\end{proof}

\subsection{Mathematical effectiveness via low-cost primitives}

\editamir{The next corollary records a purely internal ``near-balance'' restriction: if a symbol has small intrinsic cost, then any of its meanings must lie in the corresponding low-mismatch window determined by the sublevel sets of \Jcost.}

\begin{corollary}[Near-balance restricts possible referents]\label{cor:near-balance-window}
Assume $\RefStruct$ is admissible and that the hypotheses of Theorem~\ref{thm:backbone} hold. If $s\in S$ satisfies $J_S(s)\le \epsilon$, and if $o$ is a meaning of $s$, then
\[
\Jcost\!\left(\frac{\iota_S(s)}{\iota_O(o)}\right)\le \epsilon,
\]
so $\iota_O(o)$ must lie in the corresponding bounded sublevel window determined by $\epsilon$ (as in Theorem~\ref{thm:backbone}).
\end{corollary}

\begin{proof}
This is a direct restatement of Theorem~\ref{thm:backbone}.
\end{proof}

\begin{remark}[Compositional ``range expansion'' (model-dependent)]\label{rem:range-expansion}
In a continuous ratio model where ratios can be realized densely (e.g.\ $S=O=\mathbb R_{>0}$ with $\iota=\mathrm{id}$ as in Proposition~\ref{prop:continuous-model}), large mismatches can be decomposed into many small mismatches: write a target ratio $r=e^{t}$ as a product $r=(e^{t/k})^k$. Since $\Jcost(e^{u})=\cosh(u)-1\to 0$ as $u\to 0$, choosing $k$ large makes each primitive step low-cost. Coupled with the compositionality results (Theorem~\ref{thm:comp}) and optimal mediation (Corollary~\ref{cor:mediation-reduces}), this shows that, in the continuous ratio model, large ratios can be factored into many small-ratio steps, each incurring small mismatch cost. This is an interpretive program; empirical relevance depends on what ratios are actually realizable in the intended application domain.
\end{remark}

\subsection{Information-theoretic interpretation}

\editamir{Although our framework is stated in intrinsic-cost terms, the canonical mismatch penalty admits a simple log-ratio form. We record the identity as a proposition; any further links to coding/learning are interpretive and not used in the proofs.}

\begin{proposition}[Log-ratio form of the canonical mismatch cost]\label{prop:logratio}
For $x>0$ write $x=e^{t}$. Then the canonical cost satisfies
\[
\Jcost(x)=\Jcost(e^{t})=\cosh(t)-1.
\]
In particular, $\Jcost$ is a convex even function of the log-ratio $t=\log x$ and vanishes exactly at $t=0$.
\end{proposition}

\begin{proof}
Substitute $x=e^{t}$ into $\Jcost(x)=\tfrac12(x+x^{-1})-1$ (Definition~\ref{def:Jcost}).
\end{proof}

\begin{remark}[Coding/learning viewpoint]\label{rem:coding-view}
In coding theory and learning, one often selects representations by minimizing a tradeoff between description length and distortion (e.g.\ Shannon \cite{shannon1948} and MDL \cite{rissanen1978}). Our framework instantiates a specific distortion---$\Jcost(\iota_S(s)/\iota_O(o))$---that is symmetric in under-/over-shooting and naturally expressed in log-scale (Proposition~\ref{prop:logratio}). This suggests interpreting meanings as ``best matches'' under a fixed mismatch penalty, with compression enforced by the symbol condition $J_S(s)<J_O(o)$.
\end{remark}

%==============================================================================
\section{Related Work and Positioning}\label{sec:related}
%==============================================================================

\editamir{This section positions the manuscript relative to standard themes in semantics and information theory. We do \emph{not} present the mismatch penalty as novel: the axiom package in Definition~\ref{def:cost-axioms} is a convenient specification whose solutions are classical (Appendix~\ref{app:dalembert}). The contribution of the paper is instead the explicit \emph{optimization semantics} (Definition~\ref{def:meaning}) and the structural theorems derived from it (existence, stability geometry, compositionality, and mediation).}


\paragraph{Symbol grounding and operational meaning rules.}
The symbol grounding problem concerns how tokens acquire meaning without a homunculus \cite{harnad1990}.  The present work is compatible with grounding motivations, but it is formulated as a \emph{mathematical model}: the meaning of $s$ is \emph{defined} as an argmin under an explicit cost.  Any interpretation as a cognitive mechanism requires extra hypotheses beyond those stated.

\paragraph{Compression principles.}
The general idea that effective representations trade off succinctness and fidelity is classical in information theory (Shannon \cite{shannon1948}) and in algorithmic notions of complexity \cite{kolmogorov1965}; MDL makes this tradeoff concrete in model selection \cite{rissanen1978}.  Our setup uses a different primitive: a ratio map $\iota$ into $\R_{>0}$ and a fixed mismatch penalty $\Jcost$, with compression enforced by the symbol condition $J_S(s)<J_O(o)$.  Within this model, reference and compositional behavior become theorem-level consequences.

\paragraph{What is mathematically concrete here.}
Two examples of explicit structure are: (i) for finite object dictionaries under the canonical mismatch penalty, decision boundaries occur at geometric means (Theorem~\ref{thm:geom-boundaries}) and meanings are locally stable away from them (Corollary~\ref{cor:stability-away}); (ii) for sequential mediation, the optimal intermediate ratio is explicit (Theorem~\ref{thm:seq-mediator}) and strictly improves over direct reference when the mediator set contains the balance point (Corollary~\ref{cor:mediation-reduces}).

\paragraph{Interpretation layer.}
Sections~\ref{sec:applications} and~\ref{sec:discussion} illustrate how the proved statements can be read once a modeling choice for $\iota$ is fixed.  These illustrations are optional: removing them does not affect the correctness of the theorems.


\section{Discussion}\label{sec:discussion}

%==============================================================================

\editamir{This section clarifies scope: which statements are proved inside the model and which statements are interpretation. It also records limitations and concrete mathematical extensions.}


\subsection{What is proved vs. what is modeled}
The core mathematical content consists of the definitions and theorems in Sections~\ref{sec:Jcost}--\ref{sec:examples}.  In particular, meaning is defined by optimization (Definition~\ref{def:meaning}); existence is conditional on an attainment hypothesis (Theorem~\ref{thm:meaning-exists}); and explicit geometry, stability, compositionality, and mediation statements follow for admissible reference structures and the canonical mismatch penalty (Theorems~\ref{thm:geom-boundaries}, \ref{thm:comp}, \ref{thm:seq-mediator}).

By contrast, any claim that a given real-world domain \emph{does} admit a scale map $\iota$ with the required properties, or that agents \emph{compute} meaning by solving the optimization problem, is an interpretation and is outside the theorem-level scope of this manuscript.

\subsection{Comparison to classical semantic viewpoints (brief)}
In many semantic theories, reference is treated as primitive, descriptivist, or truth-conditional.  Our work does not attempt to refute those approaches.  It proposes a different, explicit \emph{model}: reference emerges from minimizing an intrinsic mismatch penalty under a compression constraint.
\begin{itemize}
    \item \textbf{Frege-style}: reference is primitive. \editamir{Here it is derived \emph{within the model} from cost minimization.}
    \item \textbf{Russell-style}: reference is mediated by descriptions. \editamir{Here ``description'' is replaced by intrinsic cost comparison and a mismatch penalty.}
    \item \textbf{Possible-worlds}: semantic value is a truth-condition across worlds. \editamir{We do not assume modal structure; we assume scale maps and a penalty.}
\end{itemize}

Our framework aligns with information-theoretic viewpoints that treat representation as efficient coding (e.g.\ Shannon \cite{shannon1948}), but differs in that distortion is fixed to the canonical mismatch penalty $\Jcost$ and the meaning rule is set-valued (argmin), making the induced decision geometry explicit.

\subsection{Limitations}

\begin{enumerate}
    \item \textbf{Ratio embedding}: Our framework requires configurations to embed into $\R_{>0}$ via a ratio map. Not all semantic domains naturally admit such embeddings.
    
    \item \textbf{Single penalty}: We work with the canonical mismatch penalty $\Jcost$. Alternative penalties may be appropriate in domains where under- and over-shooting are not symmetric.
    
    \item \textbf{Static analysis}: The theory is synchronic. Incorporating learning or time-evolution requires additional structure (e.g., dynamics for $\iota$ or for admissible reference classes).
\end{enumerate}

\subsection{Future Directions}

\begin{enumerate}
    \item \textbf{Broader admissible reference.} Classify reference structures beyond the ratio-induced form (Definition~\ref{def:admissible-ref-struct}) for which analogues of the stability and compositionality theorems remain true.
    \item \textbf{Multi-dimensional ratios.} Extend the decision-geometry and boundary descriptions to $\iota:C\to(\R_{>0})^d$ with non-separable penalties, and quantify how coupling between coordinates affects stability margins.
    \item \textbf{Learning the scale map.} Given data of successful/unsuccessful references, formulate and analyze estimation procedures for $\iota$ (and admissible reference parameters) that preserve the proved invariances.
\end{enumerate}


%==============================================================================
\section{Conclusion}
%==============================================================================

\editamir{We have developed a mathematical \emph{model} of reference grounded in cost minimization. The theorem-level contributions are internal to the stated axioms and hypotheses.}

We summarize the main points:

\begin{enumerate}
    \item \textbf{Reference as compression}: Symbols are low-cost encodings of high-cost objects.
    
    \item \textbf{Canonical mismatch geometry}: The canonical penalty $\Jcost(x) = \frac{1}{2}(x + x^{-1}) - 1$ yields explicit decision boundaries and stability regions for finite dictionaries (Theorem~\ref{thm:geom-boundaries}).
    
    \item \textbf{Universal backbone}: \rev{Near-balanced configurations provide a provable backbone window around balance under admissible reference (Theorem~\ref{thm:backbone}). Global descriptive reach is obtained by composing many such low-cost primitives (Section~\ref{sec:compositionality}).}
    
    \item \textbf{Compositionality}: Reference structures compose via products and sequences.
\end{enumerate}

\editamir{The framework connects a simple optimization semantics with explicit geometric and compositional structure. Any application to a specific empirical domain requires specifying an appropriate scale map $\iota$ and verifying that the admissibility assumptions reasonably match that domain.}

%==============================================================================
\section*{Acknowledgments}
%==============================================================================

\editamir{We thank colleagues and readers for helpful discussions and feedback on earlier drafts.}

%==============================================================================
\bibliographystyle{plain}

% (The separation of theorem-level content and interpretation is recorded in Section~\ref{sec:discussion}.)



\appendix

\section{Classical characterization of the mismatch penalty}\label{app:dalembert}

\editamir{We prove Proposition~\ref{prop:Jcost-characterization}. The underlying functional-equation step is classical; see, for example, Acz\'{e}l \cite{aczel1966} or Kuczma \cite{kuczma2009}. We include the argument here to keep the manuscript self-contained and to clarify that the mismatch penalty is not introduced as a new object.}

\begin{proof}[Proof of Proposition~\ref{prop:Jcost-characterization}]
Let $\Jcost$ satisfy Definition~\ref{def:cost-axioms}. Define
\[
  C(x):=1+\Jcost(x)\qquad(x>0).
\]
Then \eqref{eq:dalembert} is equivalent to the multiplicative identity
\begin{equation}\label{eq:mult-dalembert-C}
  C(xy)+C(x/y)=2C(x)C(y)\qquad(x,y>0).
\end{equation}
By strict convexity, $\Jcost$ (hence $C$) is convex on any compact subinterval of $(0,\infty)$ and therefore continuous on $(0,\infty)$.

Now set $f:\mathbb R\to\mathbb R$ by $f(t):=C(e^{t})$. Since $e^{t}>0$, $f$ is well-defined and continuous. Substituting $x=e^{t}$ and $y=e^{s}$ into \eqref{eq:mult-dalembert-C} yields d'Alembert's functional equation
\begin{equation}\label{eq:dalembert-additive}
  f(t+s)+f(t-s)=2f(t)f(s)\qquad(t,s\in\mathbb R).
\end{equation}
Moreover, $f(0)=C(1)=1$ and $f(t)\ge 1$ for all $t$ by non-negativity of $\Jcost$.

The continuous solutions of \eqref{eq:dalembert-additive} with $f(0)=1$ are known to be $f(t)\equiv 1$, $f(t)=\cos(at)$, or $f(t)=\cosh(at)$ for some $a\ge 0$ \cite[Ch.~2]{aczel1966}; see also \cite[Ch.~13]{kuczma2009}. The constraint $f(t)\ge 1$ rules out the cosine family unless $a=0$, and strict convexity rules out the constant solution. Hence there exists $a>0$ such that $f(t)=\cosh(at)$ for all $t$.

Undoing the change of variables gives
\[
  C(x)=f(\log x)=\cosh(a\log x),\qquad x>0,
\]
and therefore
\[
  \Jcost(x)=C(x)-1=\cosh(a\log x)-1=\tfrac12\big(x^{a}+x^{-a}\big)-1.
\]
Finally, note that
\[
\cosh(a\log(\iota_S/\iota_O)) -1 = \cosh\big(\log((\iota_S)^{a}/(\iota_O)^{a})\big)-1,
\]
so replacing $\iota_S,\iota_O$ by $\tilde\iota_S:=\iota_S^{a}$ and $\tilde\iota_O:=\iota_O^{a}$ absorbs the parameter $a$ into the scale maps and produces the normalized choice $a=1$ at the level of ratio-induced reference costs.
\end{proof}


\begin{thebibliography}{99}

\bibitem{frege1892}
G. Frege.
\newblock \"Uber Sinn und Bedeutung.
\newblock {\em Zeitschrift f\"ur Philosophie und philosophische Kritik}, 100:25--50, 1892.

\bibitem{russell1905}
B. Russell.
\newblock On denoting.
\newblock {\em Mind}, 14(56):479--493, 1905.

\bibitem{wigner1960}
E. Wigner.
\newblock The unreasonable effectiveness of mathematics in the natural sciences.
\newblock {\em Communications on Pure and Applied Mathematics}, 13(1):1--14, 1960.

\bibitem{harnad1990}
S. Harnad.
\newblock The symbol grounding problem.
\newblock {\em Physica D: Nonlinear Phenomena}, 42(1-3):335--346, 1990.

\bibitem{shannon1948}
C.E. Shannon.
\newblock A mathematical theory of communication.
\newblock {\em Bell System Technical Journal}, 27(3):379--423; 27(4):623--656, 1948.
\bibitem{kolmogorov1965}
A.N. Kolmogorov.
\newblock Three approaches to the quantitative definition of information.
\newblock {\em Problems of Information Transmission}, 1(1):1--7, 1965.

\bibitem{rissanen1978}
\rev{J. Rissanen.
\newblock Modeling by shortest data description.
\newblock {\em Automatica}, 14(5):465--471, 1978.}

\bibitem{aczel1966}
J. Acz\'{e}l.
\newblock Lectures on Functional Equations and Their Applications.
\newblock Academic Press, 1966.

\bibitem{kuczma2009}
M. Kuczma.
\newblock An Introduction to the Theory of Functional Equations and Inequalities: Cauchy's Equation and Jensen's Inequality.
\newblock 2nd edition, Birkh\"{a}user, 2009.
\end{thebibliography}

\end{document}