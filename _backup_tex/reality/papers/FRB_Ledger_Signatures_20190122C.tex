\documentclass[11pt]{article}

\usepackage[margin=1in]{geometry}
\usepackage[T1]{fontenc}
\usepackage{lmodern}
\usepackage{microtype}
\usepackage{amsmath, amssymb}
\usepackage{hyperref}
\usepackage{enumitem}

\title{Eight-Component Quasi-Periodic Train and a Candidate $\sqrt{5}$ Damping Ratio in FRB 20190122C}
\author{Jonathan Washburn\\
Recognition Science Research Institute, Austin, Texas, USA}
\date{Draft v0.2 --- 2026-01-26}

\begin{document}
\maketitle

\begin{abstract}
Using the published component-level pulse table for FRB 20190122C (Xiao et al., arXiv:2601.03950), we
measure an eight-component millisecond pulse train with quasi-period $P = 0.933 \pm 0.063$ ms. Fitting a
post-peak exponential envelope yields $\tau = 2.179 \pm 0.052$ ms, giving $\tau/P = 2.334 \pm 0.167$,
consistent with $\sqrt{5} = 2.236$ within uncertainties. We compare an unconstrained exponential envelope
to a constrained model with $\tau$ fixed to $\sqrt{5}\,P$. The constrained model is not rejected
(likelihood-ratio $p = 0.068$), though the unconstrained model is weakly preferred as expected for an
additional free parameter. Both envelope fits exhibit large $\chi^2$ values, indicating that a single-$\tau$
exponential is an imperfect phenomenological model for these component amplitudes; we therefore treat
model-comparison $p$-values as heuristic. We also report secondary ratios (e.g., $A_4/A_5$) as exploratory
observations and provide an illustrative Monte Carlo null model to calibrate coincidence rates under
simple assumptions (not a formal $p$-value). Definitive conclusions require replication on independent
high-component events (e.g., FRB 20201014B) and validation from raw baseband/intensity data products.
\end{abstract}

\section{Introduction}
Fast Radio Bursts (FRBs) are millisecond-duration radio transients of unknown origin. A subset show
sub-burst structure and, in rare cases, quasi-periodic oscillations (QPO-like morphology) on millisecond
timescales. FRB 20190122C is notable for a reported eight-component morphology separated by $\sim$1 ms with
an overall damped envelope (Xiao et al., arXiv:2601.03950).

Motivation: if FRB sub-burst trains are produced by an underlying oscillator or clock-like process, the
component count and damping ratio $\tau/P$ become key invariants to compare across physical models. This
draft documents a fully reproducible measurement of these quantities from the published component table
and provides a minimal statistical test of the constraint $\tau=\sqrt{5}\,P$, which arises as a
parameter-free prediction in the Recognition Science (RS) framework.

This manuscript is intentionally split into: (A) empirical measurements from published data tables, and
(B) an interpretation layer (RS) clearly separated from (A).

\section{Data}
We use Extended Data Table 1 from Xiao et al. (arXiv:2601.03950), transcribed into
\texttt{data/FRB\_20190122C\_pulses.csv}.
Columns: \texttt{tick, time\_ms, amplitude, amplitude\_err, width\_ms}.

This analysis does not use raw baseband voltages; it is therefore a secondary analysis of published fit
products. Replication from baseband/intensity data is a priority.

\section{Methods}
\subsection{Period estimation}
We estimate the quasi-period $P$ by linear regression:
\[
t_n = t_0 + nP
\]
where $n$ is the component index (0..7) and $t_n$ is the reported arrival time (ms).
We report $P_{\mathrm{fit}}$ (slope) and $P_{\mathrm{mean}}$ (mean adjacent interval) as a descriptive statistic.

\subsection{Exponential envelope fit}
We fit a damped envelope from the peak onward using:
\[
A(t) = A_0\,e^{-t/\tau}
\]
where $t$ is time since the peak component. The fit is weighted by the reported amplitude uncertainties.

\subsection{RS constrained model test (nested model comparison)}
We compare:
\begin{itemize}[leftmargin=*]
  \item Free model: $A(t) = A_0 e^{-t/\tau}$ (parameters: $A_0, \tau$)
  \item RS constrained: $A(t) = A_0 e^{-t/(\sqrt{5}\,P)}$ (parameter: $A_0$; $\tau$ fixed)
\end{itemize}

We compute $\chi^2$ for both models on the same post-peak points, then evaluate $\Delta\chi^2$ and report
$\Delta$AIC and $\Delta$BIC.

\paragraph{Important caveat.}
In this event, the phenomenological envelope fit yields very large $\chi^2$ values ($\chi^2/\mathrm{dof} \gg 1$),
suggesting underestimated uncertainties and/or model misspecification. Therefore, likelihood-ratio $p$-values
should be interpreted qualitatively (as a plausibility check), not as definitive statistical significance.

\subsection{Secondary ratio checks}
We compute $\tau/P$ with propagated uncertainty and $A_4/A_5$ (using tick indices 4 and 5) with propagated uncertainty.

\section{Results (FRB 20190122C)}
All results below are generated by \texttt{scripts/frb\_model\_comparison.py}, which writes
\texttt{data/FRB\_20190122C\_model\_comparison.json}.

\subsection{Eight-component morphology}
The event exhibits exactly $N=8$ components in the published decomposition.

\subsection{Period and damping}
From the published component times:
\[
P_{\mathrm{fit}} = 0.933452 \pm 0.063195~\mathrm{ms}
\]
and $P_{\mathrm{mean}} = 0.952857$ ms (std over intervals: 0.283885 ms).

From a weighted post-peak exponential fit:
\[
\tau_{\mathrm{free}} = 2.179045 \pm 0.051607~\mathrm{ms}
\]
Ratio:
\[
\tau/P_{\mathrm{fit}} = 2.334393 \pm 0.167429,\qquad \sqrt{5}=2.236068
\]
This is consistent within $\sim 0.59\sigma$ under Gaussian error propagation.

\subsection{RS constrained vs free model comparison}
Using $\tau_{\mathrm{RS}}=\sqrt{5}\,P_{\mathrm{fit}}$ gives $\tau_{\mathrm{RS}}=2.087263$ ms.
Goodness-of-fit (post-peak, 6 points):
\[
\chi^2_{\mathrm{free}}(\mathrm{dof}=4)=132.721,\qquad
\chi^2_{\mathrm{RS}}(\mathrm{dof}=5)=136.062
\]
\[
\Delta\chi^2 = 3.341 \quad (\mathrm{1~dof}) \quad \Rightarrow\quad p \approx 0.0676
\]
Information criteria:
\[
\Delta\mathrm{AIC}(\mathrm{RS}-\mathrm{free})=+1.341,\qquad
\Delta\mathrm{BIC}(\mathrm{RS}-\mathrm{free})=+1.549
\]

Interpretation: the RS constraint $\tau=\sqrt{5}\,P$ is plausible given these data and this envelope model,
but the unconstrained fit is (weakly) preferred. Note the very large $\chi^2$ values, indicating that a
single-$\tau$ exponential does not fully describe the component amplitudes and that quoted $p$-values should
be treated as heuristic.

\subsection{Secondary ratios (exploratory)}
Amplitude ratio (ticks 4 and 5):
\[
A_4/A_5 = 2.277283 \pm 0.367061,\qquad \sqrt{5}=2.236068
\]

\subsection{Illustrative null model Monte Carlo (not a formal p-value)}
We assess whether the joint observation of $\tau/P \approx \sqrt{5}$ and $A_4/A_5 \approx \sqrt{5}$ could arise
by chance under a toy null model where these quantities are independent and have no special relationship to
$\sqrt{5}$ (see \texttt{scripts/frb\_null\_model\_mc.py}, output \texttt{data/FRB\_20190122C\_null\_model\_mc.json}).
This exercise is intended to calibrate plausibility, not to produce a definitive statistical significance level.

\section{Population context and replication targets}
Microsecond morphology analyses (Curtin et al., arXiv:2411.02870 source tables) suggest that high-component
bursts are rare. Within that dataset, only one additional eight-component burst was identified:
FRB 20201014B (repeater FRB 20200202A), $N_{\mathrm{components}}=8$.

A key replication test is whether FRB 20201014B exhibits a comparable constrained ratio $\tau/P \approx \sqrt{5}$
when analyzed with the same pipeline. Replication on independent events is essential to control a posteriori
pattern risk and to establish generality.

\section{Discussion (interpretation layer)}
\subsection{Conventional interpretations}
An eight-component damped millisecond train can arise from several physical processes (e.g., magnetospheric
oscillations, emission windowing, propagation effects). A complete astrophysical model should jointly explain:
component count distribution, period stability, envelope damping, frequency dependence and scattering, and
repetition statistics.

\subsection{Recognition Science hypothesis (clearly labeled)}
In RS, discrete ``ledger'' update structure forces an 8-step cycle and a $\phi$-geometry that implies
$\sqrt{5} = \phi + 1/\phi$. Under this hypothesis, rare FRB events may transiently expose this structure,
producing: $N=8$ component trains, $\tau/P$ clustering near $\sqrt{5}$, and additional $\phi$-related scaling
in secondary observables.

This manuscript does not claim RS is established by one event; it documents a concrete, falsifiable set of
predictions and a replication program.

\subsection{Limitations and threats to validity}
\begin{itemize}[leftmargin=*]
  \item Measurements are derived from a published component table (fit products), not direct re-processing of raw voltages.
  \item Component identification and indexing can be analysis-dependent; alternative decompositions could shift amplitudes/times.
  \item The exponential envelope is a phenomenological simplification; the large $\chi^2$ indicates unmodeled structure.
  \item The toy Monte Carlo null model is not a population model and does not account for selection effects.
  \item Because FRB 20190122C is a singled-out rare event, independent replication (e.g., FRB 20201014B) is the most important next test.
\end{itemize}

\section{Reproducibility / artifacts}
Repository artifacts used:
\begin{itemize}[leftmargin=*]
  \item \texttt{data/FRB\_20190122C\_pulses.csv}
  \item \texttt{scripts/frb\_rs\_analysis.py} (general RS checks; corrected to fit $\tau$ in ms)
  \item \texttt{scripts/frb\_model\_comparison.py} (nested model comparison + uncertainty propagation)
  \item \texttt{scripts/frb\_null\_model\_mc.py} (null model Monte Carlo, 100k simulations)
  \item \texttt{data/FRB\_20190122C\_rs\_analysis.json}
  \item \texttt{data/FRB\_20190122C\_model\_comparison.json}
  \item \texttt{data/FRB\_20190122C\_null\_model\_mc.json}
  \item \texttt{data/FRB\_RS\_Analysis\_Report.txt}
\end{itemize}

Minimal reproduction commands:
\begin{verbatim}
python3 scripts/frb_model_comparison.py
python3 scripts/frb_null_model_mc.py
\end{verbatim}

\section{References}
\begin{itemize}[leftmargin=*]
  \item Xiao, S., Jiang, J., \& Li, D. (2025). arXiv:2601.03950. ``Evidence for a Damped Millisecond Quasi-Periodic Structure in FRB 20190122C.'' (Title as listed on arXiv.)
  \item Curtin, A. P., et al. (2024). arXiv:2411.02870. ``Morphology of 35 Repeating Fast Radio Burst Sources at Microsecond Time Scales with CHIME/FRB.''
  \item CHIME/FRB Collaboration et al. (2023). CHIME/FRB Baseband Catalog 1. Data citation DOI: 23.0029.
\end{itemize}

\section*{Acknowledgments}
This research used the facilities of the Canadian Astronomy Data Centre operated by the National Research
Council of Canada with the support of the Canadian Space Agency.

\end{document}

