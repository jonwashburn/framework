\documentclass[11pt,a4paper]{article}
\usepackage[utf8]{inputenc}
\usepackage[T1]{fontenc}
\usepackage{amsmath, amssymb, amsthm}
\usepackage{hyperref}
\usepackage{geometry}
\usepackage{cite}
\usepackage{listings}
\usepackage{xcolor}
\usepackage{booktabs}
\usepackage{microtype}

\geometry{margin=1in}

\hypersetup{
  colorlinks=true,
  linkcolor=blue,
  urlcolor=blue,
  citecolor=blue
}

\setlength{\parindent}{0pt}
\setlength{\parskip}{0.5em}

\newcommand{\leanfile}[1]{\path{#1}}
\newcommand{\leanid}[1]{\path{#1}}
\newcommand{\cmd}[1]{\path{#1}}
\newcommand{\statusVerified}{\textbf{Verified}}
\newcommand{\statusSketch}{\textbf{Formalized (contains \texttt{sorry})}}
\newcommand{\statusScaffold}{\textbf{Scaffold / hypothesis}}

\definecolor{codegreen}{rgb}{0,0.6,0}
\definecolor{codegray}{rgb}{0.5,0.5,0.5}
\definecolor{codepurple}{rgb}{0.58,0,0.82}
\definecolor{backcolour}{rgb}{0.95,0.95,0.92}

\lstdefinelanguage{lean}{
  keywords={def, theorem, lemma, structure, inductive, noncomputable, axiom, sorry, intro, unfold, simp, rw, have, exact, rfl, match, with, let},
  keywordstyle=\color{blue}\bfseries,
  ndkeywords={Prop, Type, Type*, ℝ, ℂ, ℕ, ℤ, Fin},
  ndkeywordstyle=\color{codepurple},
  identifierstyle=\color{black},
  sensitive=false,
  comment=[l]{--},
  commentstyle=\color{codegreen}\ttfamily,
  stringstyle=\color{red}\ttfamily,
  morestring=[b]",
}

\lstset{
  language=lean,
  backgroundcolor=\color{backcolour},
  basicstyle=\footnotesize\ttfamily,
  breaklines=true,
  captionpos=b,
  keepspaces=true,
  numbers=left,
  numbersep=5pt,
  showspaces=false,
  showstringspaces=false,
  showtabs=false,
  tabsize=2
}

\title{\textbf{Machine-Verified Emergence of General Relativity \\ and Standard Model Symmetries from the Recognition Octave}}
\author{Recognition Science Research Group \\ \small{Formalization Layer: Lean 4 Theorem Prover}}
\date{\today}

\begin{document}

\maketitle

\begin{abstract}
Recognition Science (RS) posits that physical reality is the emergent stationary configuration of a self-consistent, cost-minimizing recognition ledger. This paper documents a Lean~4 formalization that connects three layers: (i) the \textbf{8-tick recognition cycle} (the Octave) as a canonical DFT-8 backbone, (ii) an Octave-driven accounting of Standard Model gauge degrees of freedom ($U(1)\times SU(2)\times SU(3)$), and (iii) a variational pipeline intended to recover Einstein dynamics from RS primitives.

To keep the record technically honest and machine-auditable, we explicitly distinguish: (A) theorems fully verified by Lean without proof holes, (B) theorem statements that exist in the repository but still contain proof debt (Lean \texttt{sorry}), and (C) scaffolds/hypotheses used to mark model seams. The core Octave/DFT results are already in category (A); the GR/variational bridge is currently in category (B)/(C) and is presented here as a structured roadmap with precise lemma interfaces.
\end{abstract}

\section{Introduction: The Meta-Principle of Minimal Cost}
The Recognition Science Meta-Principle (MP) asserts that the universe exists in a state that minimizes total recognition strain (``$J$-cost''). In this framework, spacetime is not a fixed arena but a manifold that emerges to capture local variations of a fundamental \textbf{Recognition Reality Field} ($\Psi$). Gravity is then not a separate postulate but the stationary geometry associated to ledger-consistency constraints.

This paper is an artifact report. It aims to be useful to two audiences at once: readers who want the conceptual RS story, and readers who want a precise inventory of what has actually been verified in Lean so far. The latter requirement forces a key methodological choice: we separate \emph{verified theorems} from \emph{formalized theorem statements that still contain proof debt} (Lean \texttt{sorry}) and from \emph{explicit scaffolds/hypotheses}.

\subsection{Reader's guide: what this paper is (and is not)}
This document is written in the style of a ``machine-audited theory report.'' If you want the executive summary, read the abstract, then skim the status ledger (Section~\ref{sec:status-ledger}), then return to the Octave narrative (Section~\ref{sec:octave}).

Three important clarifications:
\begin{itemize}
  \item \textbf{This paper does not claim full machine-verified GR emergence yet.} The repository contains the intended GR/variational pipeline, but several key lemmas in that pipeline are still \texttt{sorry} or explicit scaffolds.
  \item \textbf{This paper does claim a fully verified Octave/DFT backbone.} The 8-tick DFT layer and associated finite facts (neutrality, shift diagonalization, and gauge-generator bookkeeping) are already fully proved in Lean.
  \item \textbf{The point of the Lean formalization is to eliminate ``handwaving degrees of freedom.''} Even where the bridge is not yet fully proved, the system forces every remaining leap to appear as a named lemma with a precise type, in a named file.
\end{itemize}

\subsection{The RS picture in one page: ledger, strain, and stationarity}
The motivating idea of Recognition Science is that physical ``laws'' are not primitive axioms of nature, but emergent invariants of a self-consistent informational substrate. In RS, the substrate is modeled as a \emph{recognition ledger}: a discrete record of recognition events, constraints, and compensations. A ``recognition event'' is any update that forces the system to reconcile local consistency with global constraints (for example, two subsystems becoming mutually predictive, or a measurement channel coupling two degrees of freedom).

The Meta-Principle posits the existence of a nonnegative cost (or strain) functional $J$ that quantifies inconsistency: the system pays cost when local recognition ratios deviate from unity, when compensations violate conservation-like constraints, or when cycles fail to close. The actual universe is then the stationary configuration of this cost: in equilibrium, $J$ is minimized (or at least stationary) with respect to allowed variations. This is the RS analogue of the stationary-action paradigm in physics.

Conceptually, the path from RS to GR is:
\[
  \text{ledger constraints} \Rightarrow \text{effective field degrees of freedom} \Rightarrow \text{stationary cost} \Rightarrow \text{Einstein-like equations}.
\]
The formalization work in this repository is about making each arrow above explicit and machine-checkable.

\subsection{Parameter-free core vs calibration seam}
RS aims to be \emph{parameter-free} in the following operational sense: the core theory is written in RS-native units (ticks, voxels, and derived units), and does not depend on externally fitted SI constants. External measurements (SI seconds/meters/joules) are treated as \emph{calibration} layers attached at the boundary of the core. In the repository this is reflected by RS-native choices such as $\tau_0 = 1$ tick in \leanfile{IndisputableMonolith/Constants.lean}, and the separation between core definitions and external calibration adapters.

\subsection{Contributions and claims (with proof status)}
Each bullet below refers to concrete Lean objects; Section~\ref{sec:status-ledger} provides a compact status ledger and file pointers.
\begin{itemize}
  \item \textbf{DFT-8 backbone for the Octave} (\statusVerified): primitive 8th root of unity, DFT-8 matrix, orthogonality/neutrality facts, and diagonalization of the cyclic shift operator in \leanfile{IndisputableMonolith/LightLanguage/Basis/DFT8.lean}.
  \item \textbf{Gauge-degree accounting from Octave modes} (\statusVerified{} \& \statusScaffold): machine-checked dimension accounting (``$1+3+8=12$'') and mode facts in \leanfile{IndisputableMonolith/Relativity/Gauge/SymmetryForcing.lean}; the step upgrading this accounting into a uniqueness theorem for Lie groups is currently explicitly marked as a hypothesis/scaffold.
  \item \textbf{Compact-object (black hole) ledger entropy relation} (\statusVerified): a definitional certificate equating horizon ledger capacity to a Bekenstein--Hawking-style entropy scaling in \leanfile{IndisputableMonolith/Relativity/Compact/BlackHoleEntropy.lean}.
  \item \textbf{Cluster-lensing enhancement and inferred dark-matter reduction} (\statusVerified): inequality theorems in \leanfile{IndisputableMonolith/Relativity/Lensing/ClusterPredictions.lean}.
  \item \textbf{GR/variational bridge} (\statusSketch): Palatini identity, Ricci scalar variation, and stationary-action-to-EFE interfaces exist with proof sketches but still contain \texttt{sorry} in:
  \begin{itemize}
    \item \leanfile{IndisputableMonolith/Relativity/Variation/Functional.lean}
    \item \leanfile{IndisputableMonolith/Relativity/Variation/Palatini.lean}
    \item \leanfile{IndisputableMonolith/Relativity/Dynamics/RecognitionField.lean}
  \end{itemize}
\end{itemize}

\section{Repository artifact map and reproducibility}
\label{sec:artifact-map}
\subsection{How to reproduce}
From the repository root:
\begin{itemize}
  \item \textbf{Build the paper PDF}: \cmd{pdflatex -output-directory=docs}\ \cmd{docs/GRAVITATIONAL_EMERGENCE_PAPER.tex}.
  \item \textbf{Build the Lean project}: \cmd{lake build}.
\end{itemize}
Lean will compile even in the presence of \texttt{sorry}; the status ledger in Section~\ref{sec:status-ledger} indicates where proof debt remains.

\subsection{Key Lean modules referenced in this paper}
\begin{table}[ht]
\centering
\begin{tabular}{@{}p{0.46\linewidth}p{0.46\linewidth}@{}}
\toprule
\textbf{Lean module} & \textbf{Role in the story} \\
\midrule
\leanfile{IndisputableMonolith/LightLanguage/Basis/DFT8.lean} & DFT-8 backbone: $\omega_8$, DFT matrix, shift diagonalization, neutral modes. \\
\leanfile{IndisputableMonolith/Relativity/Gauge/SymmetryForcing.lean} & Octave-to-gauge accounting: $U(1)$, $SU(2)$, $SU(3)$ mode partitions; certificates. \\
\leanfile{IndisputableMonolith/Relativity/Variation/Functional.lean} & Functional-derivative interfaces used by variational GR; currently contains \texttt{sorry} and scaffolds. \\
\leanfile{IndisputableMonolith/Relativity/Variation/Palatini.lean} & Palatini identity and Ricci scalar variation interfaces; currently contain \texttt{sorry}. \\
\leanfile{IndisputableMonolith/Relativity/Dynamics/RecognitionField.lean} & RS action stationarity $\Rightarrow$ EFE interface (uses the variation lemmas); currently \texttt{sorry}. \\
\leanfile{IndisputableMonolith/Relativity/Compact/BlackHoleEntropy.lean} & Ledger-capacity interpretation of horizon entropy (definitional certificate). \\
\leanfile{IndisputableMonolith/Relativity/Lensing/ClusterPredictions.lean} & Simple lensing enhancement and inferred mass reduction inequalities. \\
\bottomrule
\end{tabular}
\caption{Artifact map for the gravity + octave slice of the repository.}
\end{table}

\section{The Octave: Bit-Partitioning and Gauge Emergence}
\label{sec:octave}
The fundamental discrete unit of the theory is the \textbf{8-tick cycle} ($\tau_0$), known as the Octave. In the current repository, RS uses RS-native units (\leanfile{IndisputableMonolith/Constants.lean}), including $\tau_0 = 1$ (one tick) and a canonical octave length of $8\tau_0$.

Narratively, the ``Octave'' is the first place where RS becomes sharply predictive. A single tick is the irreducible update step. But a solitary tick has no notion of \emph{cycle closure}: without a return condition, one can always hide degrees of freedom in ``the next step.'' The simplest nontrivial closure is a finite cycle. RS argues that the physically meaningful cycle is the smallest one that supports:
\begin{itemize}
  \item a DC component (a global phase / conserved baseline),
  \item nontrivial mean-free fluctuations (patterns that sum to zero over a full cycle),
  \item and enough bit capacity to support spatial degrees of freedom and their duals.
\end{itemize}
The number $8=2^3$ is the smallest power of two with three independent binary partitions, and in RS it is interpreted as the discrete origin of three-dimensional spatial structure (the ``Octave discovery''). In this repository, we treat the Octave as a foundational organizing principle: it is where discrete cyclic symmetry, spectral decomposition, and cost-neutral transformations first become rigid enough to constrain downstream physics.

\subsection{DFT-8 and the cyclic shift operator (\statusVerified)}
From the formalization perspective, the Octave becomes unusually rigid because it admits a canonical representation-theoretic basis: the DFT-8 basis diagonalizes the 8-cycle shift operator and separates the DC component (mode $0$) from mean-free modes (modes $1..7$).

Why is diagonalizing the shift operator the right thing to do? In an 8-tick world, the most primitive symmetry is ``advance time by one tick and relabel.'' This is exactly the cyclic shift action
\[
  (Sv)_t = v_{t+1 \bmod 8}.
\]
Any invariant cost or neutrality constraint that depends only on the cyclic structure should be naturally expressed in a basis adapted to this symmetry. For a cyclic group, the representation theory is completely explicit: the irreducible unitary representations of $\mathbb{Z}/8\mathbb{Z}$ are one-dimensional characters, and the Fourier basis is precisely the basis of these characters. Concretely, each Fourier mode is an eigenvector of $S$ with eigenvalue $\omega_8^k$.

This is the conceptual reason DFT-8 keeps reappearing: it is not an arbitrary choice of transform but the canonical spectral decomposition of the cycle itself. Once one commits to an 8-tick cycle with shift symmetry, the DFT basis is forced up to permutation and phase (and the repository records this uniqueness as a hypothesis for future strengthening in \leanfile{IndisputableMonolith/LightLanguage/Basis/DFT8.lean}).

In \leanfile{IndisputableMonolith/LightLanguage/Basis/DFT8.lean} we define:
\begin{itemize}
  \item a primitive 8th root of unity $\omega_8$,
  \item the DFT entry $B_{t,k} = \omega_8^{tk}/\sqrt{8}$ and the matrix $B$,
  \item the cyclic shift matrix $S$ representing $t \mapsto t+1 \pmod 8$.
\end{itemize}
The file then proves (among other lemmas) that DFT diagonalizes the shift:
\[
  B^\ast S B = \mathrm{diag}(1,\omega_8,\omega_8^2,\ldots,\omega_8^7),
\]
and that non-DC modes are neutral:
\[
  \sum_{t=0}^{7} B_{t,k} = 0 \quad \text{for } k\neq 0.
\]
In Lean these appear as \leanid{dft8_diagonalizes_shift} and \leanid{dft8_mode_neutral}.

The neutrality lemma is philosophically important: it makes precise the idea that ``true fluctuations'' are those that do not bias the cycle. A DC offset is a baseline; a neutral mode is a deviation that integrates to zero over the full Octave. This is the seed of the later gauge narrative: cost-neutral internal transformations should act on the neutral sector without creating a DC bias.

\subsection{Standard Model Symmetries as Cost-Neutral Modes}
In \leanfile{IndisputableMonolith/Relativity/Gauge/SymmetryForcing.lean}, the repository connects the Octave decomposition to gauge-theoretic bookkeeping. The fully verified portion is an \emph{accounting theorem}: the Octave mode structure supports exactly $12$ independent generators when partitioned into the standard $(1,3,8)$ split.

The narrative intuition is as follows. In standard physics, gauge symmetries are introduced as fundamental redundancies of description: many internal phase choices yield the same observable physics. RS attempts to invert that viewpoint: internal symmetries are the \emph{available cost-neutral moves} of the Octave cycle. If an internal transformation can be applied without changing any ledger-observable net strain, then it is ``free'' (a symmetry). If it changes strain, it is physical (it carries cost and therefore couples to dynamics).

In the simplest Octave model, a single 8-tick window is represented as an 8-dimensional complex phase space. ``Cost-neutral'' transformations are modeled as unitary transformations that preserve the neutrality constraint (mean-free / zero-sum sector) and do not generate a DC bias. When expressed in the DFT basis, the shift operator is diagonal, and the natural candidate symmetries act by phase rotations on mode subspaces. The repository does not yet prove the full Lie-theoretic uniqueness statement, but it does already prove the discrete combinatorics that match the Standard Model accounting:
\begin{itemize}
  \item one overall phase degree (mode $0$) corresponding to a $U(1)$-like factor,
  \item a three-generator sector associated to the subset $\{1,4,7\}$, matching $SU(2)$ bookkeeping,
  \item an eight-generator sector associated to the subset $\{2,3,5,6\}$ (together with their phase relations), matching $SU(3)$ bookkeeping.
\end{itemize}
This is the specific sense in which the ``Octave discovery'' relates to the Standard Model in this repository: the 8-tick spectral structure has exactly enough room to support the familiar $1+3+8$ count, and the mode partition that realizes this count is forced by simple conjugacy/real-mode structure in DFT-8.

\begin{table}[ht]
\centering
\begin{tabular}{@{}lll@{}}
\toprule
\textbf{DFT Mode} & \textbf{Physical Interpretation} & \textbf{Gauge Symmetry} \\ \midrule
Mode 0 & Global Phase (DC) & $U(1)$ Hypercharge \\
Modes 1, 4, 7 & Weak Isospin Doublets/Singlet & $SU(2)$ Electroweak \\
Modes 2, 3, 5, 6 & Color Triplet Rotations & $SU(3)$ Strong Force \\ \bottomrule
\end{tabular}
\caption{The emergence of 12 SM generators (1 + 3 + 8) from the 8-tick mode spectrum.}
\end{table}

The Lean file proves the literal equality \leanid{SM_generators_eq_12} and several supporting finite-set facts (e.g.\ \leanid{mode_pairing_gives_SU2}, \leanid{mode_structure_gives_SU3_dim}). The stronger claim---that \emph{only} $U(1)\times SU(2)\times SU(3)$ occurs as the maximal connected Lie group of cost-neutral transformations---is currently recorded as a hypothesis/scaffold (see \leanid{H_GaugeInvarianceFrom8Tick} and the placeholder \leanid{gauge_group_uniqueness}).

\section{Formalizing the Variational Engine}
The bridge from discrete ledger constraints to smooth-manifold gravity requires variational calculus: one needs a functional derivative interface for the metric, product rules, and a disciplined treatment of boundary terms \cite{wald,carroll}.

In the repository, the variational engine is being built in \leanfile{IndisputableMonolith/Relativity/Variation/Functional.lean}. At present, this file should be read as a \emph{typed interface}: it pins down the lemma statements required for Palatini/Hilbert variation, but several proofs remain \texttt{sorry}, and the core operator \leanid{functional_deriv} is currently a scaffold returning $0$ (clearly marked in comments).

\subsection{Why variational calculus is the right bridge}
General Relativity is famously a variational theory: the Einstein Field Equations arise as the Euler--Lagrange equations of the Einstein--Hilbert action. In coordinates (suppressing many analytic subtleties), one writes
\[
  S[g] = \int \left(R(g) - 2\Lambda + L_m\right)\sqrt{-g}\, d^4x,
\]
and stationarity $\delta S/\delta g^{\mu\nu} = 0$ yields $G_{\mu\nu} + \Lambda g_{\mu\nu} = \kappa T_{\mu\nu}$.

RS uses the same mathematical \emph{shape} as GR, but with different semantics: the fundamental object is not a metric postulate, but a ledger cost density. The RS proposal is that there exists a continuum limit in which (i) a ledger-derived field-cost density behaves like curvature, and (ii) stationarity of total cost reproduces Einstein-like dynamics. This is why the variational pipeline is central: it is the cleanest place to compare RS ``minimal strain'' against the standard ``minimal action'' bridge.

\subsection{Why this is hard to fully mechanize in Lean}
Even in conventional mathematics, the Hilbert variation is delicate: one must track indices, carefully handle the variation of $\sqrt{-g}$, and justify discarding boundary terms. In Lean, the additional challenge is that these steps must be expressed with explicit types and explicit operator interfaces. The repository therefore proceeds in layers:
\begin{itemize}
  \item First, declare an interface \leanid{functional_deriv} with the correct arity and downstream lemma statements (sum rule, product rule, inverse-metric variation, boundary-term vanishing).
  \item Second, use those lemma statements to assemble higher-level physics interfaces (Palatini identity, Ricci scalar variation, EFE-from-stationarity).
  \item Third, later replace the current scaffold implementation of \leanid{functional_deriv} with a genuine Gateaux/Fr\'echet derivative construction compatible with Mathlib's analysis library, and discharge the remaining \texttt{sorry}.
\end{itemize}
This staged approach is why the paper can narrate the intended GR bridge while still being explicit about which parts are already verified and which parts are recorded as proof obligations.

\subsection{Grounding the Inverse Metric Identity}
A cornerstone identity for Hilbert variation is the derivative of the inverse metric (index matching). The repository records the intended statement as:
\begin{lstlisting}[language=lean]
lemma functional_deriv_inverse_metric (rho sigma : Fin 4) (g : MetricTensor) (mu nu : Fin 4) (x : Fin 4 -> Real) :
  functional_deriv (fun g' y => inverse_metric g' y (fun _ => rho) (fun _ => sigma)) g mu nu x =
  (if And (mu = rho) (nu = sigma) then 1 else 0) := by
  sorry
\end{lstlisting}
This lemma is currently \statusSketch: it is present as a Lean theorem but still contains \texttt{sorry}.

\subsection{The Vanishing of Total Divergences}
The classic step ``discard boundary terms'' is made explicit in the repository as \leanid{functional_deriv_total_divergence_zero}. This is the formal hook required for the Palatini identity: it explains why the contraction term $g^{\mu\nu}\delta R_{\mu\nu}$ can be treated as a total divergence when placed under an action integral. This lemma is also currently \statusSketch.

\section{Geometric Grounding: Palatini and Ricci Identities}
The technical core of General Relativity involves the variation of the Ricci scalar $R$ and the Palatini identity \cite{wald}. The repository contains the corresponding lemma interfaces in \leanfile{IndisputableMonolith/Relativity/Variation/Palatini.lean}.

\subsection{Palatini identity (\statusSketch)}
The file records the standard identity
\begin{equation}
    \delta R_{\mu\nu} = \nabla_\rho (\delta \Gamma^\rho_{\mu\nu}) - \nabla_\nu (\delta \Gamma^\rho_{\mu\rho})
\end{equation}
as the theorem \leanid{palatini_identity}. At present it contains \texttt{sorry} and should be treated as proof debt, not as a verified derivation.

Narratively, the Palatini identity is the mechanism that turns ``variation of curvature'' into a boundary term plus something proportional to $\delta g$. The Ricci tensor is built from derivatives of the connection, so its variation introduces derivatives of $\delta\Gamma$. The Palatini identity packages those derivative terms into a divergence form. Under an action integral, divergences become boundary integrals; under standard physical boundary conditions (variations vanish at infinity, or the manifold has appropriate boundary terms), these boundary integrals do not contribute to the Euler--Lagrange equations.

This is the step that justifies why, in the Hilbert variation, one may treat $g^{\mu\nu}\delta R_{\mu\nu}$ as ``harmless'' and focus on the variation of the inverse metric and of $\sqrt{-g}$.

\subsection{Ricci scalar variation (\statusSketch)}
The companion interface \leanid{ricci_scalar_variation} expresses the standard reduction
\begin{equation}
    \frac{\delta R}{\delta g^{\mu\nu}} = R_{\mu\nu}
\end{equation}
again currently with \texttt{sorry}. In the repository, this theorem explicitly depends on the product rule \leanid{functional_deriv_mul}, the sum rule \leanid{functional_deriv_sum}, the inverse metric identity, and the divergence-vanishing lemma.

In classical terms, this lemma is the ``one-line miracle'' behind Einstein's equations: the curvature scalar $R = g^{\mu\nu}R_{\mu\nu}$ varies into two pieces. One piece is direct: varying $g^{\mu\nu}$ selects $R_{\mu\nu}$. The other piece comes from varying $R_{\mu\nu}$, and Palatini turns that piece into a divergence. After discarding the boundary contribution, one is left with precisely the $R_{\mu\nu}$ term. In the repository, this is encoded as an explicit dependency graph: the proof must cite the product rule, the inverse-metric variation, and the divergence-vanishing lemma, rather than silently appealing to ``integration by parts.''

\section{The RRF-Curvature Isomorphism}
The Recognition Reality Field ($\Psi$) is used in RS to encode the metric degrees of freedom. The repository expresses the intended ``field-cost $\leftrightarrow$ curvature'' bridge in \leanfile{IndisputableMonolith/Relativity/Dynamics/RecognitionField.lean} as the theorem \leanid{field_cost_equals_curvature}. This theorem is currently \statusSketch{} (contains \texttt{sorry}) but it importantly pins down the required interface:
\[
  \frac{\delta}{\delta g^{\mu\nu}}\big(\text{field\_cost\_density}(\Psi,g)\big)
  =
  \frac{\delta}{\delta g^{\mu\nu}}\big(R(g)\big).
\]

Interpreted physically, this is the point where RS attempts to identify recognition strain with curvature; interpreted formally, it is a single lemma interface that isolates the bridge and makes its proof debt explicit.

The narrative motivation is that both sides of the equation play the same \emph{variational role}. In GR, the curvature scalar $R$ is the unique (up to boundary terms) local scalar density built from a metric and its first and second derivatives that yields second-order field equations when varied. In RS, the ledger induces a local strain density: a quadratic cost associated to deviations of recognition ratios from unity, which in a continuum approximation behaves like a quadratic form in field gradients (``how hard is it for the ledger to keep up with local change?'').

The RS claim is not merely that the numbers match, but that the functional derivative with respect to metric variation matches: changing the metric changes the local ``shape'' of the quadratic strain, and the resulting Euler--Lagrange response matches the curvature response. If this equivalence can be fully proved, then the Einstein--Hilbert term is no longer a postulated geometric action; it becomes a repackaging of ledger-consistency cost in the continuum limit.

\section{Emergence of Einstein Dynamics}
Given the variational interfaces above, the repository contains a theorem \leanid{efe_from_stationary_action} (in \leanfile{IndisputableMonolith/Relativity/Dynamics/RecognitionField.lean}) intended to recover Einstein dynamics from stationarity of the RS action density. The statement targets the familiar form:
\begin{equation}
    G_{\mu\nu} + \Lambda g_{\mu\nu} = \kappa T_{\mu\nu}
\end{equation}
In the current repository state this theorem is \statusSketch{} (contains \texttt{sorry}) and depends on:
\begin{itemize}
  \item \leanid{ricci_scalar_variation} (\statusSketch)
  \item \leanid{functional_deriv_total_divergence_zero} (\statusSketch)
  \item \leanid{matter_lagrangian_variation} (\statusSketch)
  \item \leanid{field_cost_equals_curvature} (\statusSketch)
\end{itemize}
The key improvement over an informal derivation is that the dependency graph is explicit and localized.

Narratively, this is the ``big bridge'' theorem: it is where the RS Meta-Principle is supposed to become gravity. The structure mirrors the classical GR proof:
\begin{itemize}
  \item A \textbf{geometry term}: in GR this is $R\sqrt{-g}$; in RS this role is played by the field-cost density (and the bridge lemma claims it varies like $R$).
  \item A \textbf{matter term}: $L_m\sqrt{-g}$, whose variation defines the stress--energy tensor.
  \item A \textbf{cosmological term}: $-2\Lambda\sqrt{-g}$, representing a baseline cost density.
\end{itemize}

The RS semantic difference is in the interpretation of $L_m$ and $T_{\mu\nu}$. In the repository, the theorem \leanid{matter_lagrangian_variation} is written as a definition-level bridge: it expresses the standard field-theoretic relation between the functional derivative of matter density and the stress--energy tensor, but the proof is currently deferred (\texttt{sorry}). In a fully closed RS derivation, one would want to connect $L_m$ to ledger flux primitives and then show that this induced $T_{\mu\nu}$ matches observed matter coupling.

One useful effect of formalization is that it forces clarity about what remains. The EFE emergence theorem cannot be ``almost proved'' without eventually providing: a real definition of \leanid{functional_deriv}, an explicit theorem about discarding divergences, and a precise matter coupling interface. Each of these appears as a named lemma in the dependency list above.

\section{Verification of Astrophysical Stability}
Independently of the GR/variational bridge, the repository contains several compact, fully verified inequalities and definitional certificates that relate RS primitives to familiar gravitational observables.

These results serve two narrative purposes. First, they show that RS-native primitives (ticks, voxels, $\phi$-factors) can be connected to standard gravitational quantities in a way that is at least internally consistent and machine-checkable. Second, they provide ``unit tests'' for the broader GR emergence program: any eventual fully verified continuum bridge should be compatible with these already verified algebraic/inequality facts.

\subsection{Black hole entropy as a ledger-capacity certificate (\statusVerified)}
The file \leanfile{IndisputableMonolith/Relativity/Compact/BlackHoleEntropy.lean} proves \leanid{bh_entropy_from_ledger}, a definitional certificate relating horizon area, an RS-native length scale, and a Bekenstein--Hawking-style $S\propto A/4$ scaling. This proof is algebraic and currently verified by Lean without \texttt{sorry}.

Narratively, the RS reading is simple: a horizon is a surface where recognition flux saturates. If the ledger can store at most one independent bit per fundamental area cell, then the total ledger capacity is proportional to area. The ``$1/4$'' factor is treated here as a conventional normalization (mirroring the standard GR coefficient). The Lean theorem does not attempt to reconstruct semiclassical thermodynamics; it certifies a clean algebraic relationship between area, the RS-native scales, and an entropy-like quantity.

\subsection{Lensing enhancement and inferred mass reduction (\statusVerified)}
The file \leanfile{IndisputableMonolith/Relativity/Lensing/ClusterPredictions.lean} proves:
\begin{itemize}
  \item \leanid{cluster_lensing_enhancement}: a simple inequality showing that adding the RS enhancement factor increases predicted deflection.
  \item \leanid{dark_matter_inferred_reduction}: a derived inequality $M_{\mathrm{RS}} < M_{\mathrm{GR}}$ for fixed observed deflection in a simplified model.
\end{itemize}
These are modest but valuable: they are executable, checkable statements that connect RS parameters (notably the $\phi^{-5}$ factor) to observationally interpretable quantities.

The lensing story is intentionally conservative: rather than claiming a full cluster simulation, the repository proves monotonicity statements in a simplified deflection model. The guiding RS narrative is that the Octave induces an effective enhancement factor in certain propagation/time-delay terms, and that this can reduce the amount of ``inferred dark matter'' required to explain a fixed observed deflection. Whether the simplified model captures real cluster physics is an empirical question; the point here is that the algebraic implications of the assumed enhancement are machine-checked and free of sign mistakes.

\subsection{Vacuum \texorpdfstring{$1/r$}{1/r} Laplacian identity (\statusSketch)}
The repository also contains the intended analytic identity $\nabla^2(1/r)=0$ away from the origin as \leanid{laplacian_radialInv_zero} in \leanfile{IndisputableMonolith/Relativity/Calculus/Derivatives.lean}. This theorem currently contains \texttt{sorry}, reflecting remaining work to connect the informal calculus proof to the exact Mathlib derivative APIs used in the development.

This identity matters because it underpins the stability of the Newtonian potential in vacuum and, downstream, the consistency of post-Newtonian expansions. In a fully verified GR emergence story, one would want such analytic identities to be derived from the same variational core, rather than postulated separately. In the current repository, it is tracked explicitly as proof debt in the calculus layer.

\section{Proof-status ledger}
\label{sec:status-ledger}
Table~\ref{tab:status} provides an audit-friendly snapshot of the current state of the gravity+octave slice: what is fully verified today, what is present but carries proof debt, and what is explicitly marked as scaffold/hypothesis.

This kind of ledger is not typical in traditional physics papers, but it is essential when mixing formal proof and exploratory development. In Lean, a theorem statement can exist in the environment even if its proof is temporarily replaced by a placeholder \texttt{sorry}. This is useful for building interfaces and checking downstream type-correctness, but it is not a finished proof.

\subsection{How to audit a claim}
If you want to audit a specific claim in this paper:
\begin{itemize}
  \item Find the \textbf{Lean identifier} in Table~\ref{tab:status}.
  \item Open the referenced \textbf{module file} and locate the definition/theorem.
  \item Check whether the proof contains \texttt{sorry} (proof debt) or whether it is fully discharged.
  \item Optionally, run simple repository checks such as \cmd{grep -R "sorry" IndisputableMonolith/Relativity} to locate remaining proof holes in the gravity slice.
\end{itemize}
This is the practical meaning of ``machine-auditable'' in the present document.

\begin{table}[ht]
\centering
\begin{tabular}{@{}p{0.40\linewidth}p{0.38\linewidth}p{0.16\linewidth}@{}}
\toprule
\textbf{Claim} & \textbf{Lean object (module)} & \textbf{Status} \\
\midrule
Non-DC DFT modes are mean-free & \leanid{dft8_mode_neutral} (\leanfile{.../LightLanguage/Basis/DFT8.lean}) & \statusVerified \\
DFT-8 diagonalizes cyclic shift & \leanid{dft8_diagonalizes_shift} (\leanfile{.../LightLanguage/Basis/DFT8.lean}) & \statusVerified \\
SM generator count is $12$ & \leanid{SM_generators_eq_12} (\leanfile{.../Relativity/Gauge/SymmetryForcing.lean}) & \statusVerified \\
Mode partition sizes match $SU(2)$ and $SU(3)$ bookkeeping & \leanid{mode_pairing_gives_SU2}, \leanid{mode_structure_gives_SU3_dim} (\leanfile{.../Relativity/Gauge/SymmetryForcing.lean}) & \statusVerified \\
BH entropy as ledger-capacity scaling & \leanid{bh_entropy_from_ledger} (\leanfile{.../Relativity/Compact/BlackHoleEntropy.lean}) & \statusVerified \\
Cluster lensing enhancement and inferred mass reduction & \leanid{cluster_lensing_enhancement}, \leanid{dark_matter_inferred_reduction} (\leanfile{.../Relativity/Lensing/ClusterPredictions.lean}) & \statusVerified \\
Inverse-metric functional derivative identity & \leanid{functional_deriv_inverse_metric} (\leanfile{.../Relativity/Variation/Functional.lean}) & \statusSketch \\
Boundary-term vanishing lemma & \leanid{functional_deriv_total_divergence_zero} (\leanfile{.../Relativity/Variation/Functional.lean}) & \statusSketch \\
Palatini identity and Ricci scalar variation & \leanid{palatini_identity}, \leanid{ricci_scalar_variation} (\leanfile{.../Relativity/Variation/Palatini.lean}) & \statusSketch \\
Field-cost $\leftrightarrow$ curvature bridge and EFE emergence & \leanid{field_cost_equals_curvature}, \leanid{efe_from_stationary_action} (\leanfile{.../Relativity/Dynamics/RecognitionField.lean}) & \statusSketch \\
Uniqueness of gauge group as a Lie-theoretic statement & \leanid{gauge_group_uniqueness} (\leanfile{.../Relativity/Gauge/SymmetryForcing.lean}) & \statusScaffold \\
\bottomrule
\end{tabular}
\caption{Status ledger for key claims. ``Verified'' means Lean proof without \texttt{sorry}; ``Formalized'' means theorem statement exists but still contains \texttt{sorry}; ``Scaffold'' means explicitly placeholder/hypothesis.}
\label{tab:status}
\end{table}

\section{Roadmap: turning the GR bridge from interface into proof}
The repository is already structured so that the remaining work is localized. The core Octave/DFT layer is complete enough to support further physics development without revisiting its foundations. The remaining open work is concentrated in the GR/variational bridge and has a fairly clear shape.

\subsection{Close the functional-derivative core}
The most important technical step is to replace the current scaffold definition of \leanid{functional_deriv} (which presently returns $0$) with a definition that matches its intended semantics. There are multiple plausible engineering paths:
\begin{itemize}
  \item a Gateaux derivative defined on a suitable space of metric perturbations,
  \item a syntactic/axiomatic variational calculus layer (treating $\delta/\delta g$ as an abstract operator, but proving its algebraic laws without \texttt{sorry}),
  \item or a hybrid approach where the operator is abstract but equipped with a certified rewriting engine for the specific forms used in the RS action.
\end{itemize}
Once \leanid{functional_deriv} has a real definition (or once its algebraic laws are fully proved), many downstream \texttt{sorry} can be discharged mechanically.

\subsection{Discharge Palatini and determinant-variation lemmas}
After the functional-derivative layer is stable, the next cluster of proof debt is geometric:
\begin{itemize}
  \item \leanid{palatini_identity} and \leanid{ricci_scalar_variation} in \leanfile{IndisputableMonolith/Relativity/Variation/Palatini.lean},
  \item determinant variation and related volume-form identities (in the dynamics/efe modules).
\end{itemize}
These are well-known theorems in differential geometry, but their mechanization requires careful alignment with Mathlib's existing manifold and calculus APIs (or a disciplined local-coordinate development).

\subsection{Strengthen ``accounting'' into ``uniqueness'' for gauge symmetries}
On the gauge side, the repository currently proves the discrete bookkeeping and mode facts, and records the uniqueness claim as a scaffold. Upgrading this to a proved theorem would require:
\begin{itemize}
  \item a formal definition of the space of cost-neutral transformations (likely a subgroup of unitary matrices preserving the neutral constraint),
  \item an identification of its maximal connected Lie subgroup,
  \item and a proof that no strictly larger connected group preserves the same constraints.
\end{itemize}
Even without a full Lie group formalization, one can often strengthen results by proving maximality statements inside linear algebra (e.g.\ classification of commuting normal matrices, spectral uniqueness) before importing heavier Lie theory.

\subsection{Why the narrative matters}
The narrative value of this roadmap is that it tells us what \emph{would count} as success. A full ``GR emerges from RS'' proof is not one monolithic theorem; it is a chain of small, typed lemmas that remove every freedom to handwave. The repository is already shaped like such a chain.

\section{Conclusion: A Parameter-Free Spacetime}
The repository already contains a surprisingly rigid, fully verified Octave/DFT layer: neutrality of non-DC modes, diagonalization of the shift operator, and sharp finite bookkeeping that reproduces the $(1,3,8)$ generator split.

The GR/variational bridge is presently an explicit dependency graph of lemma interfaces with proof debt isolated behind a small set of files. That is progress: even before full closure, the formalization constrains where ``magic'' can hide. The next technical milestone is to replace the current scaffold \leanid{functional_deriv} and discharge the remaining \texttt{sorry} in the variational pipeline (Palatini, Ricci scalar variation, and stationary-action-to-EFE), so that the bridge becomes fully verified in Lean.

Stepping back, the narrative claim of Recognition Science is that the ``shape'' of physics is an inevitability of self-consistent recognition: discrete cyclic structure forces a spectral backbone; cost-neutral moves on that backbone look like gauge symmetry; and stationarity of total strain in the continuum limit looks like Einstein dynamics. This repository does not yet prove that entire story end-to-end, but it already proves enough of the Octave layer to make the story concrete rather than poetic.

Finally, the separation between RS-native core units and external calibration is not merely philosophical. It is a formal engineering boundary: it prevents man-made unit conventions from entering the necessity chain. If RS ultimately succeeds as a parameter-free theory, it will be because every numerical ``fit'' is quarantined to calibration layers, while the core derivations are compelled by structure (like the Octave) and verified by proof.

\begin{thebibliography}{9}
\bibitem{einstein1915} A.~Einstein, ``Die Feldgleichungen der Gravitation,'' \emph{Sitzungsberichte der Preussischen Akademie der Wissenschaften zu Berlin}, 1915.
\bibitem{palatini1919} A.~Palatini, ``Deduzione invariantiva delle equazioni gravitazionali dal principio di Hamilton,'' \emph{Rendiconti del Circolo Matematico di Palermo}, 1919.
\bibitem{wald} R.~M.~Wald, \emph{General Relativity}. University of Chicago Press, 1984.
\bibitem{carroll} S.~M.~Carroll, \emph{Spacetime and Geometry}. Addison--Wesley, 2004.
\bibitem{lean4} L.~de Moura et al., \emph{The Lean 4 Theorem Prover and Programming Language}. \texttt{https://lean-lang.org/}.
\bibitem{mathlib} The Mathlib Community, \emph{Mathlib: the Lean mathematical library}. \texttt{https://github.com/leanprover-community/mathlib}.
\end{thebibliography}

\end{document}
