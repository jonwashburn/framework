\documentclass[11pt]{article}

% Packages
\usepackage[utf8]{inputenc}
\usepackage[T1]{fontenc}
\usepackage{geometry}
\usepackage{hyperref}
\usepackage{amsmath,amssymb}
\usepackage{graphicx}
\usepackage{booktabs}
\usepackage{xcolor}
\usepackage{enumitem}
\usepackage{array}

% Geometry
\geometry{margin=1in}

% Hyperref setup
\hypersetup{
  colorlinks=true,
  linkcolor=blue,
  urlcolor=blue
}

% Title
\title{\textbf{Project Nautilus — Power Scaling Model}\\(Target: Powering Los Angeles)}
\author{Project Nautilus Planning Team}
\date{February 1, 2026}

\begin{document}

\maketitle

\begin{abstract}
This memo models the deployment requirements to power the City of Los Angeles using Nautilus Metric Engine technology. We analyze the target load (LADWP peak demand), define three scaling unit classes (Distributed, Substation, and Grid-Scale), and calculate the required unit counts and physical footprints. The model assumes the ``Vacuum Pump'' generator mode described in NM-2 and the 10 kW performance envelope defined in the Power Testbed MVP spec.
\end{abstract}

\section{The Target: Los Angeles Power Demand}

Based on Los Angeles Department of Water and Power (LADWP) historical data:

\begin{itemize}
    \item \textbf{Peak Demand}: $\approx 6.5 \text{ GW}$ (6,500 MW).
    \item \textbf{Base Load}: $\approx 3.0 \text{ GW}$ (typical nighttime/minimum).
    \item \textbf{Annual Energy}: $\approx 22,000 \text{ GWh}$.
    \item \textbf{Customers}: $\approx 1.5 \text{ million}$ meters.
\end{itemize}

\noindent \textbf{Design Goal:} To fully ``power Los Angeles'' independently of external imports, the Nautilus fleet must reliably deliver \textbf{6.5 GW continuous} capacity (plus reserve margin).

\section{Unit Classes \& Scaling Assumptions}

We define three classes of Nautilus Generator Units based on the MVP architecture.

\subsection{Class A: The ``Appliance'' (Distributed)}
\begin{itemize}
    \item \textbf{Rating}: 10 kW (Continuous).
    \item \textbf{Basis}: Direct production version of the MVP Testbed.
    \item \textbf{Form Factor}: Standard server rack ($600\text{mm} \times 1000\text{mm} \times 42\text{U}$) or appliance (washing machine size).
    \item \textbf{Deployment}: Behind-the-meter (residential/small commercial).
\end{itemize}

\subsection{Class B: The ``Block'' (Substation)}
\begin{itemize}
    \item \textbf{Rating}: 1 MW (Continuous).
    \item \textbf{Basis}: Scaled array (e.g., 100 $\times$ 10 kW cores or a larger single core).
    \item \textbf{Form Factor}: 20 ft Shipping Container ($6\text{m} \times 2.4\text{m}$).
    \item \textbf{Deployment}: Neighborhood distribution / commercial parks.
\end{itemize}

\subsection{Class C: The ``Plant'' (Grid-Scale)}
\begin{itemize}
    \item \textbf{Rating}: 100 MW (Continuous).
    \item \textbf{Basis}: Integrated facility with massive core geometry.
    \item \textbf{Form Factor}: Warehouse / Industrial Hall ($\approx 2,000 \text{ m}^2$).
    \item \textbf{Deployment}: Replaces existing gas/thermal plants; ties into high-voltage transmission.
\end{itemize}

\section{Deployment Scenarios}

\subsection{Scenario 1: Fully Distributed (The ``Appliance'' Swarm)}
Every home and business generates its own power.
\begin{itemize}
    \item \textbf{Target}: 6.5 GW.
    \item \textbf{Unit Capacity}: 10 kW.
    \item \textbf{Required Units}: $\frac{6.5 \times 10^9}{10 \times 10^3} = \mathbf{650,000 \text{ units}}$.
    \item \textbf{Feasibility}: High unit count, but aligns with 1.5M customers. Roughly 1 unit per 2-3 households.
    \item \textbf{Pros}: Grid resilience, zero transmission losses.
    \item \textbf{Cons}: Maintenance logistics for 650k devices.
\end{itemize}

\subsection{Scenario 2: Neighborhood Microgrids (The ``Block'' Network)}
Containerized units placed at substations or large buildings.
\begin{itemize}
    \item \textbf{Target}: 6.5 GW.
    \item \textbf{Unit Capacity}: 1 MW.
    \item \textbf{Required Units}: $\frac{6,500 \text{ MW}}{1 \text{ MW}} = \mathbf{6,500 \text{ units}}$.
    \item \textbf{Feasibility}: Very high. 6,500 containers scattered across LA (parking lots, substations, roofs).
    \item \textbf{Pros}: Balance of centralization and resilience.
\end{itemize}

\subsection{Scenario 3: Centralized Replacement (The ``Plant'' Fleet)}
Direct replacement of Scattergood, Haynes, and Harbor generating stations.
\begin{itemize}
    \item \textbf{Target}: 6.5 GW.
    \item \textbf{Unit Capacity}: 100 MW.
    \item \textbf{Required Units}: $\frac{6,500 \text{ MW}}{100 \text{ MW}} = \mathbf{65 \text{ units}}$.
    \item \textbf{Feasibility}: Extremely compact. 65 units could fit within the footprint of \emph{one} existing gas plant site.
    \item \textbf{Pros}: Easiest integration with existing transmission infrastructure.
\end{itemize}

\section{Physical Footprint Comparison}

Assuming the ``Vacuum Pump'' energy density is high (no fuel storage, just machinery):

\begin{table}[h]
\centering
\begin{tabular}{@{}lcccc@{}}
\toprule
\textbf{Technology} & \textbf{Capacity} & \textbf{Area (approx)} & \textbf{Units for LA} & \textbf{Total Area} \\ \midrule
Solar PV & 6.5 GW & 20,000 m$^2$/MW & N/A & 130 km$^2$ \\
Natural Gas & 6.5 GW & 500 m$^2$/MW & 10 plants & 3.2 km$^2$ \\
\textbf{Nautilus (Class C)} & \textbf{6.5 GW} & \textbf{20 m$^2$/MW} & \textbf{65 units} & \textbf{0.13 km$^2$} \\ \bottomrule
\end{tabular}
\caption{Land use estimation. Nautilus area assumes machinery footprint only (no fuel yard).}
\end{table}

\section{Conclusion}

To power Los Angeles (6.5 GW peak), the Nautilus project would need to deploy:
\begin{itemize}
    \item \textbf{650,000} household units (10 kW), OR
    \item \textbf{6,500} containerized microgrid units (1 MW), OR
    \item \textbf{65} industrial-scale units (100 MW).
\end{itemize}

\noindent The \textbf{10 kW MVP Testbed} directly validates the ``brick'' for Scenario 1. If the scaling law $P \propto f \cdot \sigma_{\nabla}$ holds (NM-2), larger units (Scenario 3) may offer superior efficiency and are the likely long-term path for utility-scale adoption.

\end{document}
