\documentclass[11pt]{article}

\usepackage[margin=1in]{geometry}
\usepackage{amsmath,amssymb}
\usepackage{microtype}
\usepackage{xcolor}
\usepackage{hyperref}

\hypersetup{
    colorlinks=true,
    linkcolor=blue,
    citecolor=blue,
    urlcolor=blue
}

% --- RS notation (keep lightweight; no extra packages) ---
\newcommand{\phiG}{\varphi}
\newcommand{\Ecoh}{E_{\mathrm{coh}}}
\newcommand{\epscoh}{\epsilon_{\mathrm{coh}}}
\newcommand{\tauZero}{\tau_0}
\newcommand{\muStar}{\mu_\star}
\newcommand{\lambdaRec}{\lambda_{\mathrm{rec}}}
\newcommand{\tauRec}{\tau_{\mathrm{rec}}}

\title{\textbf{Internal Memo: Mass from Light}\\[0.3em]
\large Coherence Energy, the Eight-Tick Clock, and the Ladder Spectrum}

\author{Jonathan Washburn\\
Recognition Physics Institute\\
Austin, Texas, USA\\
\texttt{jon@recognitionphysics.org}}

\date{\today}

\begin{document}
\maketitle

\section{Executive Summary}

\paragraph{Two distinct meanings of ``mass from light.''}
This memo uses the phrase \emph{mass from light} in two layered senses:
\begin{itemize}
  \item \textbf{Standard relativity (the ``true derivation''):}
  light (photons) is individually massless, but \emph{any isolated system of light} can have \emph{nonzero invariant mass} whenever its total four-momentum is timelike.
  Equivalently, \emph{confined radiation} contributes an inertial/gravitational mass increment \(\Delta M=E_{\mathrm{rad}}/c^2\).

  \item \textbf{RS modeling statement (how RS uses the phrase):}
  RS organizes particle \(mc^2\) values as a \(\phiG\)-ladder built on a coherence-energy unit \(\Ecoh\).
  In that sense, the mass spectrum is ``made of'' a light-linked energy quantum times discrete rungs.
\end{itemize}

\paragraph{Refereeing note (what was wrong before, and what is fixed here).}
The earlier draft of this memo correctly stated the RS mass-law formula but (i) pointed to a non-existent file name (\texttt{Source-Super-rrf.txt}), (ii) used stale repo paths (\texttt{reality/IndisputableMonolith/...}), and (iii) implicitly mixed the IR-gate identity with a Planck-gate display formula.
This revision corrects those pointers, separates the layers explicitly, and adds the standard SR/GR derivation of how light energy yields invariant mass.

\section{Repository Ground Truth (Definitions vs. Bridge Displays)}

\paragraph{Golden ratio.}
\(\phiG\) denotes the golden ratio
\[
  \phiG := (1+\sqrt{5})/2.
\]
In Lean this is \path{IndisputableMonolith.Constants.phi} (in \path{IndisputableMonolith/Constants.lean}).

\paragraph{Coherence factor in the model layer (dimensionless).}
In the Lean \emph{model layer}, the coherence quantity is defined as a \emph{dimensionless} factor
\[
  \epscoh := \phiG^{-5}.
\]
This is \path{Anchor.E_coh} in \path{IndisputableMonolith/Masses/Anchor.lean}.
When papers/memos quote ``\(\Ecoh\approx 0.09017\,\mathrm{eV}\)'', they are using the display convention
\(\Ecoh := \epscoh\,\mathrm{eV}\) (a choice of units, not a Lean theorem).

\paragraph{Display numerics (IR scale).}
Under the common convention \(\Ecoh := \epscoh\,\mathrm{eV}\), one has
\(\Ecoh\approx 0.09017\,\mathrm{eV}\), corresponding to an infrared photon scale
\(\lambda\approx 13.8\,\mu\mathrm{m}\) (equivalently \(\tilde\nu\approx 724\,\mathrm{cm^{-1}}\)).
This IR-scale interpretation appears explicitly in the super-source (see \path{book/papers/txt/Source-Super.txt}, e.g.\ the ``eight\_beat\_IR'' notes).

\paragraph{Eight-tick minimality (proved witness).}
The eight-tick claim exists in Lean as a witness proposition
\texttt{RH.RS.eightTickWitness} and is proven via
\path{Patterns.period_exactly_8}; see
\path{IndisputableMonolith/RH/RS/Spec.lean}.
A corresponding certificate wrapper exists as \path{URCGenerators.EightTickMinimalCert}
in \path{IndisputableMonolith/URCGenerators/CoreCerts.lean}.

\paragraph{IR gate (stated in the super-source).}
The project super-source records an IR gate identity
\[
  \hbar = \Ecoh\,\tauZero
  \quad\text{(IR gate statement)}.
\]
See \path{book/papers/txt/Source-Super.txt} (entry \path{IR_GATE}, field \path{hbar_identity}).
\emph{Referee clarification:} this identity is \textbf{dimensionally meaningful only after you specify what physical units \(\Ecoh\) carries}.
In the Lean model layer \(\epscoh\) is dimensionless; converting it into Joules/eV is part of the bridge-to-SI story.

\paragraph{\(2\pi\) bookkeeping (avoiding a common confusion).}
Recall \(E=\hbar\omega\) and \(T=2\pi/\omega\), hence \(E=2\pi\hbar/T\).
Thus an identity of the form \(\hbar=E\,\tau\) naturally corresponds to \(\tau=1/\omega\) (a time-per-radian), whereas formulas written with \(2\pi\hbar/\tau\) treat \(\tau\) as a cycle period.

\paragraph{Planck-gate display (Lean bridge helper).}
Separately, the Lean \emph{bridge display} includes a Planck-side construction
\[
  \lambdaRec(B) := \sqrt{\frac{\hbar G}{\pi c^3}},\qquad
  \tauRec(B) := \frac{\lambdaRec(B)}{c},
\]
and then defines an \emph{energy-scale display}
\[
  \Ecoh^{\mathrm{disp}}(B)
  := \phiG^{-5}\,\frac{2\pi\hbar}{\tauRec(B)}.
\]
This is implemented as \path{IndisputableMonolith.Bridge.DataExt.tick_tau0} and
\path{IndisputableMonolith.Bridge.DataExt.E_coh} in
\path{IndisputableMonolith/Bridge/DataExt.lean}.
\emph{Referee note:} this is a Planck-gate display expression (it uses \(G,\hbar,c\)); it should not be silently identified with the IR-gate statement without an explicit ``no-mixing'' justification.

\section{The True Derivation: How Light Contributes to Mass}

\subsection{Invariant mass from four-momentum (SR)}

\paragraph{Definition.}
For any isolated system in special relativity, define total energy \(E\) and total momentum \(\mathbf p\) in a given inertial frame.
The invariant mass \(M\) of the \emph{whole system} is
\begin{equation}
  M^2c^4 = E^2 - (pc)^2,\qquad p:=\lVert\mathbf p\rVert.
  \label{eq:invariant-mass}
\end{equation}
Equivalently, if \(P^\mu=(E/c,\mathbf p)\), then \(M^2c^2=P^\mu P_\mu\).

\paragraph{Key point.}
A single photon satisfies \(E=pc\), so \eqref{eq:invariant-mass} gives \(M=0\).
But a \emph{collection} of photons can have \(M>0\) whenever the vector momenta fail to sum to a null total.

\subsection{Two-photon example (why ``light can make mass'' without contradiction)}

Take two photons each of energy \(E_\gamma\) with an angle \(\theta\) between their momentum vectors.
Then \(E_{\rm tot}=2E_\gamma\) and
\(p_{\rm tot}=\lVert\mathbf p_1+\mathbf p_2\rVert = 2(E_\gamma/c)\cos(\theta/2)\).
Plugging into \eqref{eq:invariant-mass}:
\[
  M^2c^4 = (2E_\gamma)^2 - \bigl(2E_\gamma\cos(\theta/2)\bigr)^2
  = 4E_\gamma^2\sin^2(\theta/2),
\]
so
\begin{equation}
  M = \frac{2E_\gamma}{c^2}\,\sin(\theta/2).
  \label{eq:two-photon-mass}
\end{equation}
Special cases:
\(\theta=0\Rightarrow M=0\) (collinear photons behave like one photon);
\(\theta=\pi\Rightarrow M=2E_\gamma/c^2\) (head-on pair has a rest frame).

\subsection{``Box of light'' (confined radiation has inertia)}

Consider a perfectly reflecting cavity that contains radiation of total energy \(E_{\rm rad}\) in the cavity rest frame.
By symmetry the total momentum vanishes in that rest frame, so \eqref{eq:invariant-mass} gives a system mass increment
\begin{equation}
  \Delta M = \frac{E_{\rm rad}}{c^2}.
  \label{eq:box-of-light}
\end{equation}
This is the cleanest statement of ``mass from light'' in standard physics:
\emph{energy stored as radiation contributes to the rest mass of the composite system}.

\paragraph{Why confinement matters.}
If the light is not confined, there is generally no rest frame (the total four-momentum can be null).
Confinement (or any arrangement that yields a timelike total four-momentum) is what turns massless quanta into a massive composite.

\subsection{Einstein's two-pulse argument (why \texorpdfstring{\(E=mc^2\)}{E=mc2} follows from SR + conservation)}

Consider a body of rest mass \(M\) that, in its rest frame, emits two photons of equal energy \(L/2\) in opposite directions.
In that rest frame the net momentum of the light is zero, so the body does not recoil; its rest mass changes to \(M'\).

Now view the same emission in a frame where the body moves at speed \(v\) along the emission axis, with
\(\beta:=v/c\) and \(\gamma:=1/\sqrt{1-\beta^2}\).
Photon energies transform as \(E'=\gamma E(1\pm\beta)\), so the forward/back photon energies are
\[
  E'_\pm = \gamma\,\frac{L}{2}\,(1\pm\beta),
\]
hence the light carries net momentum
\[
  p'_{\rm light}=\frac{E'_+-E'_-}{c}=\gamma\,\frac{L\beta}{c}.
\]
Since the body does not recoil in its rest frame, it continues at the same speed \(v\) in this moving frame; its momentum therefore changes only because its mass changed:
\[
  \gamma M v - \gamma M' v = p'_{\rm light}.
\]
Canceling \(\gamma v\) gives
\[
  (M-M')c^2 = L.
\]
Thus emitting (or absorbing) light energy \(L\) changes inertial mass by \(L/c^2\). This is the classic ``mass from light'' derivation.

\subsection{GR viewpoint (stress--energy, pressure, and the ``Tolman paradox'')}\label{sec:gr-view}

General relativity replaces ``mass sources gravity'' with ``stress--energy sources curvature.''
Radiation has pressure as well as energy density, so naively it can look like ``pressure gravitates too.''
For a \emph{closed, static} system (radiation + container + stresses), those stress contributions are not optional: the wall stresses required to confine radiation contribute with opposite sign in the relevant global mass formula, and the net gravitational mass reduces to the same total-energy statement \eqref{eq:box-of-light}.
Operationally: a cavity full of light weighs more by \(E_{\rm rad}/c^2\).

\paragraph{Connection back to RS language.}
Equations \eqref{eq:box-of-light} and \((M-M')c^2=L\) are the mainstream content behind the slogan ``mass from light'': \\emph{stable, bound field energy contributes to rest mass}.
RS's additional claim is not this equivalence, but the specific discrete \(\phiG\)-ladder organization of the stabilized (coherent) energy spectrum.

\section{How RS Uses ``Mass from Light'' (Ladder Spectrum)}

\paragraph{Mass in energy units.}
In RS mass formulas, ``mass'' is often treated in particle-physics units where one quotes \(mc^2\) in eV/GeV.
With that convention, the phrase ``mass from light'' becomes literal: the mass scale is built from a light-linked energy unit.

\paragraph{Canonical spectrum statement (super-source).}
The project super-source summarizes the mass spectrum as
\begin{equation}
  m_{\mathrm{pole},i}
  =
  B_i\,\Ecoh\,\phiG^{\,r_i + f^{\mathrm{Rec}}(Z_i) + f^{\mathrm{RG}}_i},
  \qquad
  r_i\in\mathbb Z,\quad
  B_i\in\{2^k:k\in\mathbb Z\}.
  \label{eq:mass-law}
\end{equation}
See \texttt{book/papers/txt/Source-Super.txt} (block \texttt{@SPECTRA} \(\to\) \texttt{MASS\_LAW}).

\subsection{The three ingredients}

\paragraph{(1) Sector prefactor \(B_i\).}
\(B_i\) is a sector-global power-of-two prefactor (leptons vs up-quarks vs down-quarks vs EW), implemented in Lean as \texttt{Anchor.B\_pow} in \texttt{IndisputableMonolith/Masses/Anchor.lean}.

\paragraph{(2) Rung integer \(r_i\).}
\(r_i\) is an integer rung.
Example rung tables are implemented in Lean as \texttt{Integers.r\_lepton}, \texttt{Integers.r\_up}, \texttt{Integers.r\_down}, \texttt{Integers.r\_boson} in \texttt{IndisputableMonolith/Masses/Anchor.lean}.

\paragraph{(3) Recognition residue \(f^{\mathrm{Rec}}(Z)\) and RG residue \(f^{\mathrm{RG}}_i\).}
In the mass manuscripts, the spectrum separates a closed-form recognition-side residue from a small Standard-Model transport correction.
A canonical closed form used across the repo is
\begin{equation}
  f^{\mathrm{Rec}}(Z)
  =
  \frac{1}{\ln\phiG}\,\ln\!\Bigl(1+\frac{Z}{\phiG}\Bigr),
  \label{eq:rec-residue}
\end{equation}
with an RG transport term \(f^{\mathrm{RG}}_i\) used only for like-for-like comparison across schemes/scales.
(See \texttt{Papers-tex/Masses-Paper1-Single-Anchor-updated.txt} for the framework-separated statement.)

\subsection{The integer charge map \texorpdfstring{\(Z\)}{Z}}

RS encodes a charge-derived integer \(Z\) (``word charge'') using the scaled charge \(\tilde Q:=6Q\in\mathbb Z\):
\[
  Z
  =
  \begin{cases}
  \tilde Q^2+\tilde Q^4, & \text{charged leptons},\\[2pt]
  4+\tilde Q^2+\tilde Q^4, & \text{quarks},\\[2pt]
  0, & \text{Dirac neutrinos (in the stated model)}.
  \end{cases}
\]
This is implemented as \texttt{ChargeIndex.Z} in \texttt{IndisputableMonolith/Masses/Anchor.lean}.

\section{Single-Anchor Evaluation Point (\texorpdfstring{\(\muStar\)}{mu-star})}

For Standard-Model running-mass residues, the project uses a common anchor scale
\[
  \muStar = 182.201~\mathrm{GeV},
  \qquad
  \lambda = \ln\phiG,
  \qquad
  \kappa = \phiG,
\]
as recorded in \texttt{book/papers/txt/Source-Super.txt} (block \texttt{@SM\_MASSES}).

\section{Where This Lives in the Repo (Pointers)}

\begin{itemize}
  \item \textbf{Mass constants and integer maps (Lean, model layer):}
  \texttt{IndisputableMonolith/Masses/Anchor.lean}.

  \item \textbf{Eight-tick witness and certificates (Lean):}
  \texttt{IndisputableMonolith/RH/RS/Spec.lean} and
  \texttt{IndisputableMonolith/URCGenerators/CoreCerts.lean}.

  \item \textbf{Bridge display helpers (Lean):}
  \texttt{IndisputableMonolith/Bridge/Data.lean} and
  \texttt{IndisputableMonolith/Bridge/DataExt.lean}.

  \item \textbf{Single-anchor mass manuscript (framework-separated):}
  \texttt{Papers-tex/Masses-Paper1-Single-Anchor-updated.txt}.

  \item \textbf{One-file narrative summary (mass law, IR gate, SM anchor):}
  \texttt{book/papers/txt/Source-Super.txt}.
\end{itemize}

\end{document}
