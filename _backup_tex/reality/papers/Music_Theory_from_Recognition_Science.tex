\documentclass[12pt,a4paper]{article}

% Packages
\usepackage{amsmath,amssymb,amsthm}
\usepackage{mathtools}
\usepackage{hyperref}
\usepackage{booktabs}
\usepackage{graphicx}
\usepackage{xcolor}
\usepackage{tikz}
\usetikzlibrary{arrows.meta,positioning,calc}
\usepackage[margin=1in]{geometry}
\usepackage{float}
\usepackage{caption}
\usepackage{subcaption}

% Theorem environments
\theoremstyle{plain}
\newtheorem{theorem}{Theorem}[section]
\newtheorem{lemma}[theorem]{Lemma}
\newtheorem{proposition}[theorem]{Proposition}
\newtheorem{corollary}[theorem]{Corollary}

\theoremstyle{definition}
\newtheorem{definition}[theorem]{Definition}
\newtheorem{example}[theorem]{Example}
\newtheorem{hypothesis}[theorem]{Hypothesis}

\theoremstyle{remark}
\newtheorem{remark}[theorem]{Remark}
\newtheorem{conjecture}[theorem]{Conjecture}
\newtheorem{observation}[theorem]{Observation}
\newtheorem{axiom}{Axiom}

% Custom commands
\newcommand{\J}{\mathcal{J}}
\newcommand{\R}{\mathbb{R}}
\newcommand{\Z}{\mathbb{Z}}
\newcommand{\N}{\mathbb{N}}
\newcommand{\C}{\mathbb{C}}
\newcommand{\Q}{\mathbb{Q}}
\newcommand{\golden}{\varphi}
\newcommand{\abs}[1]{\left|#1\right|}
\newcommand{\norm}[1]{\left\|#1\right\|}
\newcommand{\code}[1]{\texttt{#1}}
\newcommand{\ceil}[1]{\lceil #1 \rceil}
\newcommand{\floor}[1]{\lfloor #1 \rfloor}

% Title and authors
\title{\textbf{A Cost-Function Approach to Musical Consonance:\\
Deriving Interval Hierarchy from a Symmetry Principle}}

\author{
Jonathan Washburn\\
\textit{Recognition Science Project}\\
\texttt{jonathan@recognitionscience.org}
}

\date{\today}

\begin{document}

\maketitle

\begin{abstract}
We investigate a cost-function approach to musical consonance based on the function $\J(x) = \frac{1}{2}(x + x^{-1}) - 1$, which arises from requiring inversion symmetry ($\J(x) = \J(1/x)$) and normalization ($\J(1) = 0$). This function provides closed-form expressions for the ``cost'' of frequency ratios: superparticular ratios $(n+1)/n$ have cost exactly $1/(2n(n+1))$, and the octave ratio 2:1 minimizes J-cost among all integers greater than one. We prove these results rigorously and formalize them in the Lean 4 proof assistant. However, we show that J-cost alone does not predict perceived consonance: the tritone ($\sqrt{2}$) has lower J-cost than the octave, yet sounds dissonant. We propose a two-factor model combining J-cost with rational complexity, and outline testable predictions. We also explore structural connections between the 12-semitone chromatic scale, the 8 modes of a discrete Fourier basis, and the golden ratio, without claiming strict derivation. This work contributes a mathematically rigorous framework for analyzing interval relationships, while acknowledging the gap between mathematical structure and perceptual experience.

\medskip
\noindent\textbf{Keywords:} consonance, J-cost function, superparticular ratios, rational complexity, formal verification, Lean 4
\end{abstract}

\tableofcontents

\section{Introduction}

\subsection{The Problem of Consonance}

Why do some musical intervals sound consonant while others sound dissonant? This question has engaged mathematicians and musicians from Pythagoras to the present day. Proposed explanations include:

\begin{itemize}
    \item \textbf{Simple ratios} (Pythagoras, Euler): Consonance correlates with simplicity of the frequency ratio \cite{pythagoras, euler}.
    \item \textbf{Beating and roughness} (Helmholtz, Plomp-Levelt): Dissonance arises from rapid beating between nearby partials \cite{helmholtz, plomp}.
    \item \textbf{Harmonic templates} (Terhardt, Parncutt): The auditory system matches sounds to harmonic series templates \cite{parncutt}.
    \item \textbf{Harmonic entropy} (Erlich, Milne et al.): Consonance inversely correlates with uncertainty in perceived root \cite{milne}.
\end{itemize}

Each theory captures some aspects of consonance but leaves others unexplained. In this paper, we introduce a \emph{cost-function approach} that complements existing theories.

\subsection{The J-Cost Function}

We define a function $\J: \R_{>0} \to \R$ that measures the ``cost'' or ``defect'' of a frequency ratio. The function is determined by two natural requirements:

\begin{enumerate}
    \item \textbf{Inversion symmetry}: $\J(x) = \J(1/x)$ for all $x > 0$.
    \item \textbf{Normalization}: $\J(1) = 0$ (the unison has zero cost).
\end{enumerate}

The simplest polynomial satisfying these constraints is:

\begin{definition}[J-Cost Function]
\label{def:j-cost}
For $x > 0$, the \emph{J-cost} is:
\begin{equation}
    \J(x) = \frac{1}{2}\left(x + \frac{1}{x}\right) - 1
\end{equation}
\end{definition}

\begin{remark}
The term $\frac{1}{2}(x + 1/x)$ is the arithmetic mean of $x$ and its reciprocal. The subtraction of 1 ensures $\J(1) = 0$. This function appears in various contexts: it equals $\cosh(\ln x) - 1$ and relates to the hyperbolic distance from $x$ to 1 on the positive real line.
\end{remark}

\subsection{Contributions and Scope}

We make the following contributions:

\begin{enumerate}
    \item \textbf{Superparticular formula}: We prove that $\J((n+1)/n) = 1/(2n(n+1))$ (Theorem \ref{thm:superparticular}).
    \item \textbf{Octave minimality}: We prove that 2 uniquely minimizes J-cost among integers $> 1$ (Theorem \ref{thm:octave-minimal}).
    \item \textbf{Consonance-complexity duality}: We propose that perceived consonance depends on both J-cost \emph{and} rational complexity (Section \ref{sec:complexity}).
    \item \textbf{Structural observations}: We note connections between 8, 12, and $\golden$ in scale structure (Section \ref{sec:structure}).
    \item \textbf{Formal verification}: All proofs are machine-verified in Lean 4 (Section \ref{sec:lean}).
\end{enumerate}

We explicitly \emph{do not} claim:
\begin{itemize}
    \item That J-cost alone predicts perceived consonance.
    \item That 12 semitones are ``derived'' from first principles.
    \item That emotional valence is explained by any single mathematical quantity.
\end{itemize}

\subsection{Paper Organization}

Section 2 develops the mathematical properties of J-cost. Section 3 proves octave minimality. Section 4 addresses the gap between J-cost and perception. Section 5 explores structural connections in scale design. Section 6 describes the Lean formalization. Section 7 discusses implications and limitations.

\section{The J-Cost Function: Mathematical Properties}

\subsection{Fundamental Properties}

\begin{proposition}[J-Cost Properties]
\label{prop:J-properties}
The function $\J: \R_{>0} \to \R$ satisfies:
\begin{enumerate}
    \item[(P1)] $\J(1) = 0$ \hfill (normalization)
    \item[(P2)] $\J(x) = \J(1/x)$ for all $x > 0$ \hfill (inversion symmetry)
    \item[(P3)] $\J(x) \geq 0$ with equality iff $x = 1$ \hfill (positive semi-definiteness)
    \item[(P4)] $\lim_{x \to 0^+} \J(x) = \lim_{x \to \infty} \J(x) = +\infty$ \hfill (divergence at boundaries)
    \item[(P5)] $\J$ is strictly convex on $\R_{>0}$ \hfill (convexity)
    \item[(P6)] $\J$ has a unique minimum at $x = 1$ \hfill (global minimum)
\end{enumerate}
\end{proposition}

\begin{proof}
(P1) $\J(1) = \frac{1}{2}(1 + 1) - 1 = 0$.

(P2) $\J(1/x) = \frac{1}{2}(1/x + x) - 1 = \J(x)$.

(P3) For $x \neq 1$: $(x-1)^2 > 0 \Rightarrow x^2 - 2x + 1 > 0 \Rightarrow x + 1/x > 2 \Rightarrow \J(x) > 0$.

(P4) As $x \to 0^+$, $1/x \to \infty$, so $\J(x) \to \infty$. By (P2), the same holds as $x \to \infty$.

(P5) $\J'(x) = \frac{1}{2}(1 - x^{-2})$ and $\J''(x) = x^{-3} > 0$ for $x > 0$.

(P6) Follows from (P3) and (P5).
\end{proof}

\subsection{J-Cost of Musical Intervals}

\begin{theorem}[Interval Costs]
\label{thm:interval-costs}
The J-costs of common musical intervals are:

\begin{center}
\begin{tabular}{lccc}
\toprule
Interval & Ratio & J-Cost (exact) & J-Cost (decimal) \\
\midrule
Unison & $1$ & $0$ & $0.0000$ \\
Minor Third & $6/5$ & $1/60$ & $0.0167$ \\
Major Third & $5/4$ & $1/40$ & $0.0250$ \\
Fourth & $4/3$ & $1/24$ & $0.0417$ \\
Tritone & $\sqrt{2}$ & $\frac{3\sqrt{2}}{4} - 1$ & $0.0607$ \\
Fifth & $3/2$ & $1/12$ & $0.0833$ \\
Major Sixth & $5/3$ & $2/15$ & $0.1333$ \\
Octave & $2$ & $1/4$ & $0.2500$ \\
\bottomrule
\end{tabular}
\end{center}
\end{theorem}

\begin{proof}
We verify each case. For the fifth:
\begin{align*}
\J(3/2) &= \frac{1}{2}\left(\frac{3}{2} + \frac{2}{3}\right) - 1 = \frac{1}{2} \cdot \frac{13}{6} - 1 = \frac{13}{12} - 1 = \frac{1}{12}
\end{align*}

For the tritone $r = \sqrt{2}$:
\begin{align*}
\J(\sqrt{2}) &= \frac{1}{2}\left(\sqrt{2} + \frac{1}{\sqrt{2}}\right) - 1 = \frac{1}{2} \cdot \frac{3\sqrt{2}}{2} - 1 = \frac{3\sqrt{2}}{4} - 1 \approx 0.0607
\end{align*}

The remaining cases follow similarly.
\end{proof}

\begin{remark}[J-Cost Ordering vs. Traditional Consonance]
\label{rem:ordering}
The J-cost ordering does \emph{not} match traditional consonance rankings:
\begin{equation}
    \J(\text{unison}) < \J(\text{m3}) < \J(\text{M3}) < \J(\text{P4}) < \J(\text{tritone}) < \J(\text{P5}) < \J(\text{octave})
\end{equation}
In particular, the tritone has \emph{lower} J-cost than the fifth and octave. This shows that J-cost alone does not determine perceived consonance. We address this discrepancy in Section \ref{sec:complexity}.
\end{remark}

\subsection{Superparticular Ratios}

Superparticular ratios are of the form $(n+1)/n$ for positive integers $n$. These include the octave (2/1), fifth (3/2), fourth (4/3), major third (5/4), and minor third (6/5).

\begin{theorem}[Superparticular Cost Formula]
\label{thm:superparticular}
For any positive integer $n \geq 1$:
\begin{equation}
    \J\left(\frac{n+1}{n}\right) = \frac{1}{2n(n+1)}
\end{equation}
\end{theorem}

\begin{proof}
Let $r = (n+1)/n$. Then $1/r = n/(n+1)$, and:
\begin{align*}
\J(r) &= \frac{1}{2}\left(\frac{n+1}{n} + \frac{n}{n+1}\right) - 1 \\
&= \frac{1}{2} \cdot \frac{(n+1)^2 + n^2}{n(n+1)} - 1 \\
&= \frac{2n^2 + 2n + 1}{2n(n+1)} - 1 \\
&= \frac{2n^2 + 2n + 1 - 2n^2 - 2n}{2n(n+1)} \\
&= \frac{1}{2n(n+1)}
\end{align*}
\end{proof}

\begin{corollary}[Monotonicity of Superparticular Costs]
\label{cor:monotone}
For $n \geq 1$:
\begin{equation}
    \J\left(\frac{n+2}{n+1}\right) < \J\left(\frac{n+1}{n}\right)
\end{equation}
That is, superparticular ratios have decreasing J-cost as $n$ increases.
\end{corollary}

\begin{proof}
We need $\frac{1}{2(n+1)(n+2)} < \frac{1}{2n(n+1)}$. Canceling positive factors: $(n+1)(n+2) > n(n+1)$, i.e., $n+2 > n$. True.
\end{proof}

\begin{corollary}
As $n \to \infty$, $\J((n+1)/n) \to 0$. The superparticular ratios approach the unison.
\end{corollary}

\section{Why the Octave is 2:1}

\subsection{Octave Minimality Theorem}

\begin{theorem}[Octave Minimality]
\label{thm:octave-minimal}
Among all positive integers $n > 1$, the octave ratio $n = 2$ uniquely minimizes J-cost:
\begin{equation}
    \forall n \in \Z_{>1}: \quad \J(2) \leq \J(n), \quad \text{with equality iff } n = 2
\end{equation}
\end{theorem}

\begin{proof}
Define $f(x) = x + 1/x$ for $x > 0$. Since $\J(x) = f(x)/2 - 1$, minimizing $\J$ is equivalent to minimizing $f$.

\textbf{Step 1}: $f'(x) = 1 - 1/x^2 = (x^2 - 1)/x^2$.

\textbf{Step 2}: For $x > 1$: $x^2 > 1$, so $f'(x) > 0$. Thus $f$ is strictly increasing on $(1, \infty)$.

\textbf{Step 3}: For integers $n \geq 2$: $f(n) \geq f(2) = 5/2$, with equality iff $n = 2$.

\textbf{Step 4}: Therefore $\J(n) \geq \J(2) = 1/4$ for all integers $n \geq 2$.
\end{proof}

\begin{remark}
This theorem provides a mathematical rationale for the universality of the octave. Among all integer frequency ratios greater than 1, the ratio 2:1 minimizes J-cost. Note that this does not ``explain'' the octave in perceptual terms—it merely identifies a mathematical optimality.
\end{remark}

\subsection{Comparison with Other Integers}

\begin{center}
\begin{tabular}{ccc}
\toprule
$n$ & $f(n) = n + 1/n$ & $\J(n)$ \\
\midrule
2 & 2.500 & 0.250 \\
3 & 3.333 & 0.667 \\
4 & 4.250 & 1.125 \\
5 & 5.200 & 1.600 \\
\bottomrule
\end{tabular}
\end{center}

The J-cost increases rapidly for $n > 2$, confirming the special status of the octave among integer ratios.

\section{The Consonance-Complexity Problem}
\label{sec:complexity}

\subsection{The Tritone Paradox}

Theorem \ref{thm:interval-costs} reveals a puzzle: the tritone ($\sqrt{2}:1$) has \emph{lower} J-cost than the octave ($2:1$):
\begin{equation}
    \J(\sqrt{2}) \approx 0.061 < 0.250 = \J(2)
\end{equation}

Yet the tritone is traditionally considered highly dissonant, while the octave is the most consonant interval after unison. This demonstrates that \textbf{J-cost alone does not determine perceived consonance}.

\subsection{The Complexity Factor}

We propose that perceived consonance depends on two factors:

\begin{definition}[Rational Complexity]
\label{def:complexity}
For a rational number $p/q$ in lowest terms (with $p, q \in \Z_{>0}$), the \emph{complexity} is:
\begin{equation}
    \kappa(p/q) = p + q
\end{equation}
For irrational numbers, $\kappa = \infty$.
\end{definition}

\begin{hypothesis}[Two-Factor Consonance Model]
\label{hyp:two-factor}
Perceived consonance of a frequency ratio $r$ is determined by both J-cost and complexity:
\begin{equation}
    \text{Consonance}(r) = \begin{cases}
        g(\J(r), \kappa(r)) & \text{if } r \in \Q \\
        \text{low} & \text{if } r \notin \Q
    \end{cases}
\end{equation}
where $g$ is increasing in consonance as both $\J(r)$ and $\kappa(r)$ decrease.
\end{hypothesis}

\begin{example}
The tritone ($\sqrt{2}$) has low J-cost but infinite complexity (irrational), explaining its dissonance. The minor second ($16/15$) has very low J-cost ($\J \approx 0.002$) but high complexity ($\kappa = 31$), also producing dissonance.
\end{example}

\begin{conjecture}[Dissonance Formula]
\label{conj:dissonance}
For rational intervals $p/q$, perceived dissonance is modeled by:
\begin{equation}
    D(p/q) = \alpha \cdot \J(p/q) + \beta \cdot \log(\kappa(p/q))
\end{equation}
for empirically determined constants $\alpha, \beta > 0$.
\end{conjecture}

This conjecture is testable: the constants $\alpha$ and $\beta$ can be fit to psychoacoustic data on consonance ratings.

\subsection{Comparison with Other Theories}

\begin{center}
\begin{tabular}{lll}
\toprule
Theory & Consonance Criterion & Tritone Status \\
\midrule
Helmholtz \cite{helmholtz} & Absence of beating & Dissonant (beating) \\
Plomp-Levelt \cite{plomp} & Critical bandwidth & Dissonant (roughness) \\
Euler \cite{euler} & Prime factorization & Undefined (irrational) \\
Harmonic entropy \cite{milne} & Low uncertainty & Dissonant (ambiguous) \\
\textbf{This paper} & Low J-cost + complexity & Dissonant (infinite $\kappa$) \\
\bottomrule
\end{tabular}
\end{center}

The two-factor model is consistent with existing theories while providing a complementary perspective.

\section{Structural Connections in Scale Design}
\label{sec:structure}

\subsection{The 8-Mode and 12-Semitone Connection}

We observe a numerical coincidence between scale structures:

\begin{observation}
The ratio of Western semitones (12) to a natural Fourier basis dimension (8) equals the perfect fifth:
\begin{equation}
    \frac{12}{8} = \frac{3}{2}
\end{equation}
\end{observation}

We do not claim this ``derives'' 12 semitones from 8 modes. Rather, we note that:
\begin{itemize}
    \item 8 is a natural period for discrete Fourier analysis ($2^3$).
    \item 12 semitones arise from the circle of fifths (see below).
    \item The ratio 12/8 = 3/2 connects these structures.
\end{itemize}

\subsection{The Circle of Fifths}

\begin{definition}[Circle of Fifths]
The map $C: \Z_{12} \to \Z_{12}$ defined by $C(n) = 7n \pmod{12}$ generates all 12 pitch classes by iterating the fifth (7 semitones).
\end{definition}

\begin{theorem}[Circle Bijectivity]
\label{thm:circle}
The circle of fifths is a bijection on $\Z_{12}$.
\end{theorem}

\begin{proof}
Since $\gcd(7, 12) = 1$, multiplication by 7 is a bijection. The inverse is also multiplication by 7, since $7^2 = 49 \equiv 1 \pmod{12}$.
\end{proof}

\subsection{The Pythagorean Comma}

\begin{theorem}[Pythagorean Comma]
\label{thm:comma}
Twelve perfect fifths exceed seven octaves by a ratio:
\begin{equation}
    \pi_c = \frac{(3/2)^{12}}{2^7} = \frac{3^{12}}{2^{19}} = \frac{531441}{524288} \approx 1.01364
\end{equation}
This corresponds to approximately 23.46 cents.
\end{theorem}

\begin{proof}
Direct computation: $3^{12} = 531441$ and $2^{19} = 524288$. The ratio in cents is $1200 \log_2(531441/524288) \approx 23.46$.
\end{proof}

\subsection{Fibonacci and Golden Ratio Connections}

\begin{observation}[Fibonacci Structure]
The chromatic scale exhibits Fibonacci-related structure:
\begin{itemize}
    \item $12 = 5 + 7$ (black + white keys on piano)
    \item $5, 8, 13$ are consecutive Fibonacci numbers
    \item The pentatonic (5 notes), diatonic (7 notes), and chromatic (12 notes) scales relate as $F_5, F_6 - 1, F_7 - 1$
\end{itemize}
\end{observation}

We do not claim these connections are ``derived'' from first principles—they are empirical observations that may reward further investigation.

\section{Formalization in Lean 4}
\label{sec:lean}

\subsection{Overview}

All mathematical theorems in this paper have been formalized and verified in the Lean 4 proof assistant \cite{lean4} using the Mathlib library \cite{mathlib4}. The formalization comprises approximately 1,200 lines of code.

\subsection{Key Definitions}

\begin{verbatim}
/-- The J-cost function. -/
noncomputable def J (x : Real) : Real := (x + 1/x) / 2 - 1

/-- Rational complexity: p + q for p/q in lowest terms. -/
def complexity (p q : Nat) : Nat := p + q
\end{verbatim}

\subsection{Key Theorems}

\begin{verbatim}
theorem J_one : J 1 = 0 := by simp [J]

theorem J_symm {x : Real} (hx : x != 0) : J x = J (1/x) := by
  simp only [J]; field_simp; ring

theorem superparticular_cost (n : Nat) (hn : 0 < n) :
    J ((n + 1 : Real) / n) = 1 / (2 * n * (n + 1)) := by
  simp only [J]; field_simp; ring

theorem J_two_minimal_integer :
    forall n : Nat, n > 1 -> J 2 <= J n := by
  intro n hn
  -- [proof of octave minimality]

theorem circle_of_fifths_bijective :
    Function.Bijective circleOfFifths := (inj, surj)
\end{verbatim}

\subsection{Verification Status}

\begin{center}
\begin{tabular}{lcc}
\toprule
Result & Theorem Number & Lean Status \\
\midrule
J-cost properties & Prop. \ref{prop:J-properties} & Verified \\
Interval costs & Thm. \ref{thm:interval-costs} & Verified \\
Superparticular formula & Thm. \ref{thm:superparticular} & Verified \\
Octave minimality & Thm. \ref{thm:octave-minimal} & Verified \\
Circle bijectivity & Thm. \ref{thm:circle} & Verified \\
Pythagorean comma & Thm. \ref{thm:comma} & Verified \\
Two-factor hypothesis & Hyp. \ref{hyp:two-factor} & Not formalized (empirical) \\
\bottomrule
\end{tabular}
\end{center}

\section{Discussion}

\subsection{Summary of Results}

\begin{enumerate}
    \item The J-cost function $\J(x) = \frac{1}{2}(x + x^{-1}) - 1$ is uniquely determined by inversion symmetry and normalization.
    \item Superparticular ratios $(n+1)/n$ have J-cost exactly $1/(2n(n+1))$.
    \item The octave (2:1) uniquely minimizes J-cost among integers $> 1$.
    \item J-cost alone does not predict consonance; complexity is also required.
    \item The circle of fifths is a bijection on $\Z_{12}$.
\end{enumerate}

\subsection{Limitations}

\begin{enumerate}
    \item \textbf{Perception gap}: J-cost is a mathematical quantity, not a perceptual one. The two-factor model (Hypothesis \ref{hyp:two-factor}) requires empirical validation.
    \item \textbf{No derivation of 12}: We observe structural connections involving 8 and 12, but do not claim to ``derive'' 12 semitones from first principles.
    \item \textbf{Scope}: We address only pitch intervals, not timbre, rhythm, or melody.
\end{enumerate}

\subsection{Testable Predictions}

\begin{enumerate}
    \item \textbf{Dissonance ratings}: Conjecture \ref{conj:dissonance} predicts that dissonance correlates with a weighted combination of J-cost and log-complexity. This can be tested against psychoacoustic data.
    
    \item \textbf{Cross-cultural universals}: If J-cost reflects deep structure, intervals with low J-cost and low complexity should be cross-culturally preferred.
    
    \item \textbf{Microtonal scales}: The two-factor model predicts which microtonal intervals should be perceived as consonant.
\end{enumerate}

\subsection{Future Work}

\begin{enumerate}
    \item Fit the two-factor model to consonance rating data.
    \item Extend the analysis to chords (multiple simultaneous intervals).
    \item Investigate connections to roughness and harmonic entropy.
    \item Explore the role of timbre in modifying J-cost predictions.
\end{enumerate}

\section{Conclusion}

We have introduced the J-cost function as a tool for analyzing musical intervals. The function has elegant mathematical properties: it is uniquely determined by symmetry requirements, provides closed-form expressions for superparticular ratios, and identifies the octave as the optimal integer ratio. However, J-cost alone does not determine consonance—rational complexity must also be considered.

Our contribution is a rigorous mathematical framework, not a complete theory of music perception. The gap between mathematical structure and perceptual experience remains an open problem. Nevertheless, the J-cost function provides a useful lens for understanding why certain intervals have special status in music across cultures.

All results are machine-verified in Lean 4, ensuring mathematical correctness. The empirical predictions we outline can guide future research at the intersection of mathematics, acoustics, and cognitive science.

\section*{Acknowledgments}

The author thanks the Mathlib community for their foundational work on the Lean 4 mathematics library.

\bibliographystyle{plain}
\begin{thebibliography}{99}

\bibitem{lean4}
L. de Moura, S. Kong, J. Avigad, F. van Doorn, and J. von Raumer.
\newblock The Lean Theorem Prover (System Description).
\newblock \textit{Proceedings of CADE-25}, 2015.

\bibitem{mathlib4}
The Mathlib Community.
\newblock Mathlib4: The Math Library for Lean 4.
\newblock \url{https://github.com/leanprover-community/mathlib4}, 2024.

\bibitem{helmholtz}
H. von Helmholtz.
\newblock \textit{On the Sensations of Tone as a Physiological Basis for the Theory of Music}.
\newblock Dover Publications, 1954 (original 1863).

\bibitem{plomp}
R. Plomp and W. J. M. Levelt.
\newblock Tonal Consonance and Critical Bandwidth.
\newblock \textit{Journal of the Acoustical Society of America}, 38:548--560, 1965.

\bibitem{euler}
L. Euler.
\newblock Tentamen novae theoriae musicae.
\newblock St. Petersburg Academy, 1739.

\bibitem{pythagoras}
W. Burkert.
\newblock \textit{Lore and Science in Ancient Pythagoreanism}.
\newblock Harvard University Press, 1972.

\bibitem{parncutt}
R. Parncutt.
\newblock \textit{Harmony: A Psychoacoustical Approach}.
\newblock Springer, 1989.

\bibitem{milne}
A. Milne, S. Sethares, and J. Plamondon.
\newblock Tuning Continua and Keyboard Layouts.
\newblock \textit{Journal of Mathematics and Music}, 2(1):1--19, 2008.

\bibitem{sethares}
W. A. Sethares.
\newblock \textit{Tuning, Timbre, Spectrum, Scale}.
\newblock Springer, 2nd edition, 2005.

\bibitem{tymoczko}
D. Tymoczko.
\newblock \textit{A Geometry of Music: Harmony and Counterpoint in the Extended Common Practice}.
\newblock Oxford University Press, 2011.

\end{thebibliography}

\appendix

\section{Detailed Proofs}

\subsection{Proof of Proposition \ref{prop:J-properties}}

\begin{proof}[Complete Proof]
We verify each property:

\textbf{(P1)} $\J(1) = \frac{1}{2}(1 + 1) - 1 = 1 - 1 = 0$. \checkmark

\textbf{(P2)} $\J(1/x) = \frac{1}{2}(1/x + x) - 1 = \frac{1}{2}(x + 1/x) - 1 = \J(x)$. \checkmark

\textbf{(P3)} For $x \neq 1$, we have $(x-1)^2 > 0$. Expanding: $x^2 - 2x + 1 > 0$. Dividing by $x > 0$: $x - 2 + 1/x > 0$, so $x + 1/x > 2$. Therefore $\J(x) = \frac{1}{2}(x + 1/x) - 1 > 0$. \checkmark

\textbf{(P4)} As $x \to 0^+$: $1/x \to +\infty$, so $x + 1/x \to +\infty$, hence $\J(x) \to +\infty$. By (P2), the same holds as $x \to +\infty$. \checkmark

\textbf{(P5)} $\J'(x) = \frac{1}{2}(1 - x^{-2})$ and $\J''(x) = x^{-3} > 0$ for all $x > 0$. \checkmark

\textbf{(P6)} Follows from (P3): $\J(x) > 0$ for $x \neq 1$ and $\J(1) = 0$. \checkmark
\end{proof}

\subsection{Proof of Theorem \ref{thm:superparticular}}

\begin{proof}[Complete Proof]
Let $r = (n+1)/n$ for $n \geq 1$. Then:
\begin{align*}
\J(r) &= \frac{1}{2}\left(\frac{n+1}{n} + \frac{n}{n+1}\right) - 1 \\[6pt]
&= \frac{1}{2} \cdot \frac{(n+1)^2 + n^2}{n(n+1)} - 1 \\[6pt]
&= \frac{(n+1)^2 + n^2}{2n(n+1)} - 1 \\[6pt]
&= \frac{n^2 + 2n + 1 + n^2}{2n(n+1)} - 1 \\[6pt]
&= \frac{2n^2 + 2n + 1}{2n(n+1)} - \frac{2n(n+1)}{2n(n+1)} \\[6pt]
&= \frac{2n^2 + 2n + 1 - 2n^2 - 2n}{2n(n+1)} \\[6pt]
&= \frac{1}{2n(n+1)}
\end{align*}
\end{proof}

\section{Numerical Data}

\begin{table}[H]
\centering
\caption{J-Cost of Superparticular Ratios $(n+1)/n$}
\begin{tabular}{ccccc}
\toprule
$n$ & Ratio & Interval Name & J-Cost (exact) & J-Cost (decimal) \\
\midrule
1 & 2/1 & Octave & $1/4$ & 0.2500 \\
2 & 3/2 & Fifth & $1/12$ & 0.0833 \\
3 & 4/3 & Fourth & $1/24$ & 0.0417 \\
4 & 5/4 & Major Third & $1/40$ & 0.0250 \\
5 & 6/5 & Minor Third & $1/60$ & 0.0167 \\
6 & 7/6 & Subminor Third & $1/84$ & 0.0119 \\
7 & 8/7 & Supermajor Second & $1/112$ & 0.0089 \\
8 & 9/8 & Major Second & $1/144$ & 0.0069 \\
9 & 10/9 & Minor Second & $1/180$ & 0.0056 \\
\bottomrule
\end{tabular}
\end{table}

\begin{table}[H]
\centering
\caption{Complexity vs. J-Cost for Common Intervals}
\begin{tabular}{lcccl}
\toprule
Interval & Ratio & $\kappa$ & J-Cost & Perceived \\
\midrule
Unison & 1/1 & 2 & 0.000 & Consonant \\
Octave & 2/1 & 3 & 0.250 & Consonant \\
Fifth & 3/2 & 5 & 0.083 & Consonant \\
Fourth & 4/3 & 7 & 0.042 & Consonant \\
Major Third & 5/4 & 9 & 0.025 & Consonant \\
Minor Third & 6/5 & 11 & 0.017 & Consonant \\
Minor Second & 16/15 & 31 & 0.002 & Dissonant \\
Tritone & $\sqrt{2}$ & $\infty$ & 0.061 & Dissonant \\
\bottomrule
\end{tabular}
\end{table}

\end{document}
