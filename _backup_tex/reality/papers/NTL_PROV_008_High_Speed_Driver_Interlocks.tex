\documentclass[11pt]{article}

% Keep packages minimal for TeX Live "basic" installs.
\usepackage[utf8]{inputenc}
\usepackage[T1]{fontenc}
\usepackage{geometry}
\usepackage{hyperref}
\usepackage{amsmath,amssymb}
\usepackage{graphicx}
\usepackage{booktabs}
\usepackage{xcolor}
\usepackage{enumitem}
\usepackage{array}

\geometry{margin=1in}
\hypersetup{
  colorlinks=true,
  linkcolor=blue,
  urlcolor=blue
}

% ---------------------------------------------------------------------------
% Convenience macros (avoid Unicode Greek in text; use LaTeX math symbols)
% ---------------------------------------------------------------------------
\newcommand{\R}{\mathbb{R}}
\newcommand{\Z}{\mathbb{Z}}
\newcommand{\N}{\mathbb{N}}

\newcommand{\PatentTitle}{High-Speed Multi-Phase Driver Interfaces with Per-Phase Sensing, Interlocks, and Fail-Safe Detuning for Solid-State Virtual Rotors}
\newcommand{\Docket}{NTL-PROV-008}
\newcommand{\Inventors}{[Inventor Names]}
\newcommand{\Assignee}{[Assignee / Organization]}
\newcommand{\FilingDate}{February 1, 2026}

\begin{document}

\begin{center}
{\LARGE \textbf{\PatentTitle}}\\[0.75em]
{\large \textbf{Docket:} \Docket}\\[0.25em]
{\large \textbf{Inventors:} \Inventors}\\[0.25em]
{\large \textbf{Assignee:} \Assignee}\\[0.25em]
{\large \textbf{Date:} \FilingDate}\\[0.75em]
\end{center}

\vspace{-0.5em}
\hrule
\vspace{0.75em}

% ===========================================================================
% ABSTRACT (PATENT)
% ===========================================================================
\section*{Abstract}

Disclosed are apparatus, systems, methods, and non-transitory computer-readable media for driving multi-phase electromagnetic arrays (including solid-state virtual rotors) using a high-speed driver interface that integrates per-phase sensing, fault interlocks, and fail-safe behaviors. In various embodiments, a driver subsystem comprises \(C\) independently addressable channels configured to energize electromagnetic elements according to a commutation schedule. Each channel includes (i) a controllable switching stage, (ii) per-phase sensing of at least one electrical quantity (e.g., current, voltage, temperature proxy), and (iii) interlocks that enforce constraints including overcurrent, overvoltage, overtemperature, timing/jitter violations, and loss-of-load conditions.

The disclosure further provides a multi-layer control surface separating high-level schedule intent from low-level waveform realization, and provides a hardware/firmware architecture that supports deterministic replay and tamper-evident configuration. In one embodiment, when a fault is detected, the driver transitions to a safe state that includes at least one of: detuning/phase-slip injection, switching to a neutral safe kernel, insertion of a dump load/crowbar, or full shutdown with a latched fault state. The disclosed driver interface enables reproducible operation at high switching rates while bounding risk and enabling rigorous metrology.

% ===========================================================================
% TECHNICAL FIELD
% ===========================================================================
\section*{Technical Field}

The present disclosure relates to power electronics and control interfaces for multi-phase electromagnetic systems, and more particularly to high-speed multi-channel driver interfaces with per-phase sensing, safety interlocks, deterministic replay, and fail-safe detuning or shutdown behaviors for commutated coil arrays and virtual rotor systems.

% ===========================================================================
% BACKGROUND
% ===========================================================================
\section*{Background}

Multi-channel commutation systems can synthesize rotating magnetic fields by energizing multiple electromagnetic elements according to a schedule. At high switching rates and power levels, these systems face failure modes including overcurrent, overtemperature, EMI-induced mis-triggering, timing skew, controller faults, and unsafe behavior under load disconnect.

Conventional drivers often provide only coarse aggregate sensing (e.g., total supply current) and may not detect per-phase faults, resulting in damage or unsafe behavior. Additionally, experiments requiring high reproducibility benefit from deterministic replay and from configuration integrity guarantees.

Accordingly, there is a need for an integrated driver interface that provides per-phase sensing, fast interlocks, deterministic replay, and fail-safe mitigation actions appropriate for high-speed virtual rotor operation.

% ===========================================================================
% SUMMARY
% ===========================================================================
\section*{Summary}

This disclosure provides a driver interface for multi-phase commutation systems. In one aspect, each channel includes a switching stage and per-phase sensing and is controlled by a controller that executes commutation schedules. In another aspect, the driver provides a structured control surface separating schedule-level parameters (phase order, frequency, duty) from hardware-level parameters (gate timing, rise/fall shaping, current limits).

In another aspect, the driver includes interlocks that monitor per-phase current, voltage, and temperature (or proxies) and trigger mitigation actions including detuning, safe neutral patterns, dump-load insertion, or shutdown.

In another aspect, the driver stores configuration artifacts and supports deterministic replay and integrity checks.

% ===========================================================================
% BRIEF DESCRIPTION OF DRAWINGS
% ===========================================================================
\section*{Brief Description of the Drawings}

Drawings may be provided later. For purposes of this specification:
\begin{itemize}[leftmargin=*]
  \item \textbf{FIG. 1} depicts a multi-channel driver subsystem connected to a virtual rotor array.
  \item \textbf{FIG. 2} depicts a per-channel block including switching stage, current sensing, temperature sensing, and interlocks.
  \item \textbf{FIG. 3} depicts a control surface separating schedule intent from waveform realization.
  \item \textbf{FIG. 4} depicts a safety state machine including nominal, detune, and shutdown states.
  \item \textbf{FIG. 5} depicts deterministic replay packaging and configuration integrity checks.
\end{itemize}

% ===========================================================================
% DEFINITIONS
% ===========================================================================
\section*{Definitions and Notation}

Unless otherwise indicated:
\begin{itemize}[leftmargin=*]
  \item \(C\in\N\) is the number of driver channels/phases.
  \item A \emph{channel} refers to a controllable drive path from a power stage to an electromagnetic element or phase group.
  \item A \emph{schedule} refers to a discrete-time commutation intent, such as phase order and dwell timing.
  \item A \emph{waveform realization} refers to hardware-level signals that implement the schedule (e.g., gate drive signals).
  \item An \emph{interlock} refers to a hardware and/or firmware condition that prevents unsafe operation by forcing a safe state.
  \item A \emph{safe state} refers to a state in which drive is disabled or detuned and risk is bounded.
  \item A \emph{detune action} refers to deliberate destruction of resonance/ordering by modifying phase timing or sequence.
\end{itemize}

% ===========================================================================
% DETAILED DESCRIPTION
% ===========================================================================
\section*{Detailed Description}

\subsection*{1. System Architecture}

In one embodiment, a virtual rotor system comprises:
\begin{itemize}[leftmargin=*]
  \item a multi-phase electromagnetic array (coil elements or phase groups);
  \item a driver subsystem comprising \(C\) channels;
  \item a controller producing commutation schedules;
  \item sensors and interlocks providing safety enforcement.
\end{itemize}

\subsection*{2. Driver Channel Block}

In one embodiment, each channel comprises:
\begin{itemize}[leftmargin=*]
  \item a switching stage (e.g., MOSFET half-bridge, full-bridge, or other topology);
  \item a gate driver with controlled rise/fall time and dead-time insertion;
  \item a current sense element (e.g., shunt, Hall, or current transformer);
  \item a voltage sense element (e.g., divider, isolated measurement);
  \item a temperature measurement element (sensor or proxy);
  \item a fast interlock path that can disable the channel independent of the main controller.
\end{itemize}

\subsection*{3. Per-Phase Sensing}

\paragraph{3.1 Current sensing.}
In one embodiment, per-channel current \(I_c(t)\) is measured and compared to thresholds:
\[
|I_c(t)| \le I_{\max,c}.
\]
Thresholds may be static or schedule-dependent (e.g., different phases may permit different peaks).

\paragraph{3.2 Voltage sensing.}
In one embodiment, per-channel voltage \(V_c(t)\) is measured and compared to thresholds:
\[
|V_c(t)| \le V_{\max,c}.
\]

\paragraph{3.3 Temperature sensing.}
In one embodiment, per-channel temperature \(T_c(t)\) (or proxy) is measured and compared to thresholds:
\[
T_c(t) \le T_{\max,c}.
\]

\paragraph{3.4 Derived metrics.}
In one embodiment, the system computes derived metrics such as:
\begin{itemize}[leftmargin=*]
  \item instantaneous power proxy \(P_c(t)=V_c(t)I_c(t)\),
  \item RMS current over a window,
  \item duty-cycle and switching-loss proxies.
\end{itemize}

\subsection*{4. Timing Integrity and Interlocks}

In one embodiment, the driver integrates timing integrity monitoring, including:
\begin{itemize}[leftmargin=*]
  \item detection of missing pulses, extra pulses, or out-of-order phases,
  \item detection of jitter exceeding a configured bound,
  \item detection of skew between channels exceeding a bound.
\end{itemize}

If timing integrity is violated, the interlock transitions the system to a safe state.

\subsection*{5. Control Surface (Schedule Intent vs Waveform Realization)}

In one embodiment, the driver exposes two layers:
\begin{itemize}[leftmargin=*]
  \item \textbf{Schedule intent layer:} phase order, dwell time, direction, amplitude profile.
  \item \textbf{Waveform realization layer:} gate timing, dead-time, edge shaping, current limiting, PWM carrier details.
\end{itemize}

Separating these layers allows consistent experiments across hardware revisions while protecting the safety and integrity of low-level implementations.

\subsection*{6. Fault Detection and Safe-State Machine}

\paragraph{6.1 Fault classes.}
Faults include (non-limiting):
\begin{itemize}[leftmargin=*]
  \item overcurrent, overvoltage, overtemperature,
  \item timing faults (jitter/skew/out-of-order),
  \item supply rail faults (undervoltage/overvoltage),
  \item load disconnect / open circuit on a phase,
  \item sensor faults (stuck-at, out-of-range),
  \item watchdog timeout.
\end{itemize}

\paragraph{6.2 Safety state machine.}
In one embodiment, the driver implements a state machine:
\begin{itemize}[leftmargin=*]
  \item \textbf{NOMINAL:} schedule executes as commanded.
  \item \textbf{DETUNE:} driver injects detune timing/sequence or switches to a safe neutral kernel.
  \item \textbf{SHUTDOWN:} all switching disabled; fault latched until reset.
\end{itemize}

\paragraph{6.3 Mitigation actions.}
Mitigation actions include:
\begin{itemize}[leftmargin=*]
  \item per-channel disable (selective shedding),
  \item global disable (hard stop),
  \item detune/phase slip injection (soft stop),
  \item dump-load insertion or crowbar on output bus,
  \item amplitude/duty reduction and ramp-down.
\end{itemize}

\subsection*{7. Deterministic Replay and Integrity}

In one embodiment, a run is recorded as a bundle containing:
\begin{itemize}[leftmargin=*]
  \item schedule intent data (phase order, dwell, direction, setpoints),
  \item waveform realization parameters (gate timing, dead-time, limits),
  \item sensor logs (per-phase I/V/T),
  \item interlock events and state transitions,
  \item hashes and optional signatures for tamper evidence.
\end{itemize}

\subsection*{8. Example Embodiments (Non-Limiting)}

\paragraph{Embodiment A: per-phase current-limited driver.}
Each phase has independent current sensing and a fast comparator that disables the phase if \(I_c>I_{\max,c}\) within a configured blanking time to avoid false triggers.

\paragraph{Embodiment B: jitter-gated high-speed commutation.}
The driver measures commutation event timing and triggers DETUNE if RMS jitter exceeds a threshold for more than a persistence interval.

\paragraph{Embodiment C: load disconnect protection.}
The driver detects a phase open circuit (e.g., current below threshold while commanded on) and triggers a dump load and global detune to prevent overshoot.

% ===========================================================================
% CLAIMS (DRAFT / PROVISIONAL-STYLE)
% ===========================================================================
\section*{Claims (Draft)}

\textbf{Note:} The following claims are an initial, non-limiting claim set intended to preserve multiple fallback positions. Final claim strategy should be reviewed by counsel.

\subsection*{Independent Claims}

\begin{enumerate}[leftmargin=*]
  \item \textbf{(System)} A multi-channel driver system for a phased electromagnetic array, the system comprising: a plurality of driver channels configured to energize electromagnetic elements; per-channel sensing circuitry configured to measure at least one of current, voltage, or temperature for each driver channel; and an interlock subsystem configured to transition the driver system to a safe state when a sensed quantity violates a threshold.

  \item \textbf{(Method)} A method of operating a solid-state virtual rotor, the method comprising: energizing a plurality of electromagnetic elements according to a commutation schedule; sensing per-phase electrical quantities during energization; detecting at least one fault condition based on the sensed quantities; and, responsive to detecting the fault condition, performing a mitigation action comprising at least one of detuning the commutation schedule, switching to a neutral safe schedule, inserting a dump load, or disabling switching.

  \item \textbf{(Non-transitory medium)} A non-transitory computer-readable medium storing instructions that, when executed by one or more processors, cause the one or more processors to: receive schedule intent parameters; translate the schedule intent parameters into waveform realization parameters including at least one of gate timing or dead-time; and enforce per-channel interlock thresholds based on measured per-channel current, voltage, or temperature.
\end{enumerate}

\subsection*{Dependent Claims (Examples; Non-Limiting)}

\begin{enumerate}[leftmargin=*]
  \setcounter{enumi}{3}
  \item The system of claim 1, wherein the per-channel sensing circuitry comprises a current shunt and a comparator configured to disable a channel within a bounded response time.
  \item The system of claim 1, wherein the interlock subsystem is configured to detect a timing integrity fault comprising at least one of excessive jitter, excessive skew, missing pulses, or out-of-order phases.
  \item The method of claim 2, wherein detuning comprises injecting a phase slip in the commutation schedule.
  \item The method of claim 2, wherein inserting the dump load comprises switching a resistive load onto a DC bus to absorb energy.
  \item The non-transitory medium of claim 3, further comprising storing a deterministic replay bundle including schedule intent, waveform realization parameters, and sensor logs.
  \item The system of claim 1, wherein the safe state comprises a shutdown state that latches until manual reset.
  \item The system of claim 1, further comprising a watchdog configured to force the safe state upon controller timeout.
  \item The system of claim 1, wherein the driver channels comprise half-bridge or full-bridge stages.
\end{enumerate}

% ===========================================================================
% FALLBACK POSITIONS / ADDITIONAL EMBODIMENTS
% ===========================================================================
\section*{Additional Embodiments and Fallback Positions (Non-Limiting)}

\begin{itemize}[leftmargin=*]
  \item Sensing may be analog, digital, or mixed-signal and may include isolated sensing for high-voltage stages.
  \item Interlocks may be implemented purely in hardware, purely in firmware, or as a hybrid with a hardware fast path and firmware supervisory path.
  \item Safe states may include selective shedding of channels, reduced duty operation, or substitution of a verified safe commutation kernel.
  \item Thresholds may be schedule-dependent and may be parameterized per phase group.
  \item The driver may include EMI mitigation features such as slew-rate control, spread-spectrum modulation, and shield/return-path enforcement.
\end{itemize}

\vspace{1em}
\hrule
\vspace{0.75em}
\noindent \textbf{End of Specification (Draft)}

\end{document}

