\documentclass[11pt]{article}

\usepackage[margin=1in]{geometry}
\usepackage{amsmath, amssymb, amsthm}
\usepackage{booktabs}
\usepackage{hyperref}
\usepackage{enumitem}

\hypersetup{
  colorlinks=true,
  linkcolor=blue,
  urlcolor=blue,
  citecolor=blue
}

\newtheorem{theorem}{Theorem}
\newtheorem{lemma}{Lemma}
\newtheorem{definition}{Definition}
\newtheorem{proposition}{Proposition}

\title{P0-A2 Ionization Energy Sawtooth\\(Mathematical Derivation + NIST Validation Tables)}
\author{Recognition Science Derivation Campaign}
\date{2026-01-17}

\begin{document}
\maketitle

\begin{abstract}
This document rewrites the P0-A2 ionization ``sawtooth'' result as a full mathematical derivation.
The repository’s Lean scaffold defines an \emph{ionization proxy} $P(Z)$ from the period boundary maps
and proves that, within each period, $P(Z)$ increases from the first element (alkali) to the closure
endpoint (noble gas), and then resets after each closure. We then reproduce the preregistered validator
tables comparing these ordering predictions to NIST first ionization energies (PASS 3/3 hard tests).
\end{abstract}

\section{Claim (P0-A2)}
Let $I_1(Z)$ denote the first ionization energy (in eV) for atomic number $Z$.
Empirically, $I_1$ exhibits a characteristic sawtooth pattern across periods.
In this repository, the fit-free content is an \textbf{ordering prediction}:
\begin{itemize}[leftmargin=*]
  \item Within each period, the ionization proxy increases from the alkali metal to the noble gas.
  \item Across a period boundary, the proxy resets downward at the next alkali.
\end{itemize}
The validator compares these ordering predictions to NIST data for $Z\le 86$.

\section{Definitions}
\label{sec:defs}
We reuse the closure maps $\mathrm{prev}(Z)$ and $\mathrm{next}(Z)$ from P0-A0.
Define the derived period length and valence position
\[
  \mathrm{periodLen}(Z) := \mathrm{next}(Z)-\mathrm{prev}(Z),
\qquad
  \mathrm{valence}(Z) := Z-\mathrm{prev}(Z).
\]

\begin{definition}[Ionization proxy]
Define the ionization proxy by
\[
  P(Z) := \mathrm{valence}(Z).
\]
In the Lean file this is \texttt{ionizationProxy Z := valenceElectrons Z}.
\end{definition}

\begin{definition}[Normalized and $\phi$-scaled display seams]
The implementation also defines (for display) a normalized value
\[
  \mathrm{norm}(Z) := \begin{cases}
  0 & \mathrm{periodLen}(Z)=0,\\
  \frac{P(Z)}{\mathrm{periodLen}(Z)} & \mathrm{periodLen}(Z)>0,
  \end{cases}
\]
and a $\phi$-scaled quantity $\mathrm{scaled}(Z)=\phi^{2\cdot \mathrm{period}(Z)}\cdot \mathrm{norm}(Z)$.
The ordering theorems below are proved at the proxy level $P(Z)$ (fit-free).
\end{definition}

\section{Derivations (proxy-level ordering)}
\label{sec:theorems}

\subsection{Alkali metals have minimal proxy within a period}
\begin{proposition}
If $Z$ is the first element after a closure, i.e.\ $Z=\mathrm{prev}(Z)+1$, then $P(Z)=1$.
\end{proposition}
\begin{proof}
By definition,
\[
P(Z)=Z-\mathrm{prev}(Z)=(\mathrm{prev}(Z)+1)-\mathrm{prev}(Z)=1.
\]
\end{proof}

\subsection{Noble gases have maximal proxy within a period}
\begin{proposition}
If $Z$ is a closure endpoint (a noble gas in the chemistry scaffold), i.e.\ $\mathrm{next}(Z)=Z$, then
\[
  P(Z)=\mathrm{periodLen}(Z).
\]
\end{proposition}
\begin{proof}
If $\mathrm{next}(Z)=Z$, then
\[
\mathrm{periodLen}(Z)=\mathrm{next}(Z)-\mathrm{prev}(Z)=Z-\mathrm{prev}(Z)=P(Z).
\]
\end{proof}

\subsection{Monotonicity within a fixed period}
\begin{theorem}
If $Z_1<Z_2$ and $\mathrm{prev}(Z_1)=\mathrm{prev}(Z_2)$ (so no boundary is crossed), then $P(Z_1)<P(Z_2)$.
\end{theorem}
\begin{proof}
Let $c:=\mathrm{prev}(Z_1)=\mathrm{prev}(Z_2)$. Then
\[
P(Z_1)=Z_1-c < Z_2-c = P(Z_2).
\]
\end{proof}

\subsection{Sawtooth reset across a boundary}
\begin{theorem}
If $Z_{\mathrm{noble}}$ is a closure endpoint and $Z_{\mathrm{alkali}}=Z_{\mathrm{noble}}+1$, then
\[
  P(Z_{\mathrm{alkali}}) < P(Z_{\mathrm{noble}}).
\]
\end{theorem}
\begin{proof}
Because $Z_{\mathrm{alkali}}$ is the first element of the next period, $\mathrm{prev}(Z_{\mathrm{alkali}})=Z_{\mathrm{noble}}$.
Hence $P(Z_{\mathrm{alkali}})=1$ by the first proposition. Meanwhile $P(Z_{\mathrm{noble}})=\mathrm{periodLen}(Z_{\mathrm{noble}})$
by the previous proposition, and in the scaffold $\mathrm{periodLen}(Z_{\mathrm{noble}})\ge 2$, so $1<P(Z_{\mathrm{noble}})$.
\end{proof}

\section{Validation against NIST first ionization energies}
The preregistered validator \url{scripts/analysis/chem_ionization_energy_compare.py}
writes \url{artifacts/chem_ionization_energy_comparison.json}. The committed artifact reports
\textbf{PASS (3/3 hard tests)} for $Z\le 86$.

\subsection{Hard test 1: alkali minimum within period (main group)}
\begin{center}
\begin{tabular}{@{}r l r l l@{}}
\toprule
Alkali $Z$ & Element & $I_1$ (eV) & Period & Pass \\
\midrule
3  & Li & 5.392 & 3--10  & true \\
11 & Na & 5.139 & 11--18 & true \\
19 & K  & 4.341 & 19--36 & true \\
37 & Rb & 4.177 & 37--54 & true \\
55 & Cs & 3.894 & 55--86 & true \\
\bottomrule
\end{tabular}
\end{center}

\subsection{Hard test 2: noble maximum within period}
\begin{center}
\begin{tabular}{@{}r l r l l@{}}
\toprule
Noble $Z$ & Element & $I_1$ (eV) & Period & Pass \\
\midrule
2  & He & 24.587 & 1--2   & true \\
10 & Ne & 21.565 & 3--10  & true \\
18 & Ar & 15.760 & 11--18 & true \\
36 & Kr & 14.000 & 19--36 & true \\
54 & Xe & 12.130 & 37--54 & true \\
86 & Rn & 10.749 & 55--86 & true \\
\bottomrule
\end{tabular}
\end{center}

\subsection{Hard test 3: sawtooth reset across boundary}
\begin{center}
\begin{tabular}{@{}r l r@{\qquad}r l r@{\qquad}l@{}}
\toprule
Alkali $Z$ & Element & $I_1$ (eV) &
Prev. noble $Z$ & Element & $I_1$ (eV) &
Pass \\
\midrule
3  & Li & 5.392 &
2  & He & 24.587 & true \\
11 & Na & 5.139 &
10 & Ne & 21.565 & true \\
19 & K  & 4.341 &
18 & Ar & 15.760 & true \\
37 & Rb & 4.177 &
36 & Kr & 14.000 & true \\
55 & Cs & 3.894 &
54 & Xe & 12.130 & true \\
\bottomrule
\end{tabular}
\end{center}

\subsection{Note on ``proxy ordering'' violations}
The validator additionally counts local anomalies where the strictly increasing proxy does not match
strict increase in the NIST values (e.g.\ Be $\to$ B, N $\to$ O). These are recorded as informational
and are not treated as a hard falsifier in the current prereg.

\section{Repo cross-references}
Lean modules:
\begin{itemize}[leftmargin=*]
  \item \texttt{IndisputableMonolith/Chemistry/IonizationEnergy.lean}
  \item \texttt{IndisputableMonolith/Chemistry/PeriodicTable.lean} (period/valence primitives)
\end{itemize}

Key proved theorems (names as exported in the module):
\begin{itemize}[leftmargin=*]
  \item \texttt{alkali\_min\_ionization}: if \texttt{valenceElectrons Z = 1} then \texttt{ionizationProxy Z = 1}
  \item \texttt{noble\_max\_ionization}: if \texttt{isNobleGas Z} then \texttt{ionizationProxy Z = periodLength Z}
  \item \texttt{ionization\_monotone\_within\_period}: same-period implies strict increase in proxy
  \item \texttt{sawtooth\_reset}: if \texttt{Zalkali = Znoble + 1} then \texttt{proxy(Zalkali) < proxy(Znoble)}
\end{itemize}

\paragraph{Artifact reference.}
Validation output: \url{artifacts/chem_ionization_energy_comparison.json}.

\end{document}

