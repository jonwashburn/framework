\documentclass[11pt]{article}

\usepackage{amsmath,amssymb,amsthm}

\title{Corner-Exit Slivers for Calibrated Sheet Constructions:\\ Deterministic Face Incidence and Uniform Boundary Control}
\author{
Jonathan Washburn\\
Recognition Science\\
Recognition Physics Institute\\
\texttt{jon@recognitionphysics.org}\\
Austin, Texas, USA
}
\date{}

% --- theorem environments ---
\newtheorem{theorem}{Theorem}
\newtheorem{lemma}{Lemma}
\newtheorem{proposition}{Proposition}
\newtheorem{corollary}{Corollary}
\theoremstyle{definition}
\newtheorem{definition}{Definition}
\newtheorem{remark}{Remark}

% --- basic macros ---
\newcommand{\C}{\mathbb{C}}
\newcommand{\R}{\mathbb{R}}
\newcommand{\Hh}{\mathcal{H}}
\newcommand{\dist}{\operatorname{dist}}
\newcommand{\Mass}{\operatorname{Mass}}

% Recognition Geometry notation (kept consistent with Recognition Geometry)
\newcommand{\config}{\mathcal{C}}
\newcommand{\events}{\mathcal{E}}

\begin{document}
\maketitle

\begin{abstract}
We introduce corner-exit slivers: local calibrated template pieces inside a cube whose footprint is a uniformly fat simplex meeting only a prescribed set of boundary faces. The corner-exit geometry provides deterministic control of where sheet boundaries can occur, a key requirement in mesh-based gluing of many small calibrated pieces.

In Euclidean $\C^n$, we construct a $(2p-1)$-parameter translation family of complex $(n-p)$-plane templates whose intersection with a cube has identical corner-exit footprint geometry. We prove two robust properties: (i) face incidence is stable under sufficiently small $C^1$ graph perturbations, so a realized sheet intersects a cube face if and only if that face is designated by the template; (ii) the boundary mass on each designated face is comparable to a fixed scale $v^{(k-1)/k}$, where $v$ is the interior $k$-volume of the sliver.

Combining this geometry with Bergman-scale holomorphic manufacturing yields holomorphic corner-exit slivers in projective K\"ahler manifolds with an $L^1$-type interface estimate controlling the total boundary-face mass by $O(v^{(k-1)/k})$. We further build robust corner-exit template families for finite nets of calibrated directions, providing a uniform dictionary of geometric building blocks for later global coherence and gluing arguments.
\end{abstract}

\section{Introduction}

When one assembles many local sheet pieces on a cubical mesh, the only place the assembled object can fail to be closed is along mesh interfaces: codimension-one faces of cubes. If the sheet pieces are arbitrary, their boundary traces on cube faces can be geometrically complicated and hard to match. Corner-exit slivers are designed to remove that difficulty.

A corner-exit sliver is built from a local \emph{template} whose footprint inside a cube is a uniformly fat simplex placed near a chosen cube vertex. The defining inequalities force the footprint to touch only a prescribed set of ``near'' faces (those incident to that vertex) and to stay a definite distance from all other faces. This has two consequences:

\smallskip
\noindent\textbf{(G1) Deterministic face incidence.}
Any sufficiently small $C^1$ graph perturbation of the template intersects exactly the same set of cube faces.

\smallskip
\noindent\textbf{(G2) Uniform boundary control.}
Each designated face slice has $(k-1)$-mass comparable to the $k$-volume of the footprint to the power $(k-1)/k$.

\smallskip
These two outputs are exactly what one needs later for global matching and gluing: (G1) makes the boundary \emph{combinatorial}, and (G2) makes it \emph{quantitatively summable}.

\begin{remark}[Recognition Geometry framing (optional)]
Let $\config$ be a configuration space of templates (for example, translation parameters $t$ in a fixed admissible box), and let $\events$ be the finite event space of face-incidence patterns (subsets of cube faces). The map
\[
R:\config\to\events,\qquad R(t)=\{\text{faces hit by the footprint }E(t)\}
\]
is a recognizer in the sense of Recognition Geometry. A corner-exit construction produces a large region of $\config$ on which $R$ is constant (one resolution cell), and the stability theorem below says that small $C^1$ perturbations remain in the same cell. In short: corner-exit slivers turn ``which faces are hit'' into a robust finite-resolution observable.
\end{remark}

\section{Corner-exit templates and slivers in cubes}

We begin in Euclidean space. Let $d\ge 2$ and let
\[
Q=[0,h]^d\subset \R^d
\]
be the standard cube of side length $h>0$. A \emph{codimension-one face} of $Q$ is any set of the form $\{x_i=0\}\cap Q$ or $\{x_i=h\}\cap Q$.

\begin{definition}[Corner-exit simplex footprint]
Fix an integer $1\le k<d$ and a vertex $v$ of $Q$. Let $P\subset\R^d$ be an affine $k$-plane and set
\[
E:=P\cap Q.
\]
We say that $E$ is a \emph{corner-exit simplex footprint at $v$} if:
\begin{enumerate}
\item $E$ is a (nondegenerate) $k$-simplex with one vertex at $v$;
\item there exist \emph{distinct} codimension-one faces $F_0,\dots,F_k$ of $Q$, each incident to $v$, such that the $k+1$ facets of $E$ are exactly the sets $E\cap F_i$ $(i=0,\dots,k)$;
\item $E$ meets no other codimension-one faces of $Q$.
\end{enumerate}
The faces $F_0,\dots,F_k$ are called the \emph{designated exit faces}.
\end{definition}

\begin{definition}[Quantitative fatness for simplices]
Let $E\subset\R^d$ be a $k$-simplex, and let $\Pi=\mathrm{aff}(E)$ be its affine span. Fix $\Lambda\ge 1$.
We say $E$ is \emph{$\Lambda$-fat} if there exists an affine isomorphism $A:\Pi\to \R^k$ such that
\[
\|DA\|\le \Lambda,\qquad \|(DA)^{-1}\|\le \Lambda,
\]
and $A(E)$ is a standard $k$-simplex of some scale $s>0$.
\end{definition}

\begin{definition}[Corner-exit sliver (geometric form)]
Let $E=P\cap Q$ be a corner-exit simplex footprint with designated exit faces $F_0,\dots,F_k$ and gap
\[
\delta := \min\{\dist(E,F):\ F \text{ a codimension-one face of }Q,\ F\notin\{F_0,\dots,F_k\}\}.
\]
A smooth oriented $k$-submanifold $Y\subset\R^d$ is a \emph{corner-exit sliver over $E$ in $Q$} if $Y\cap Q$ is the image of a $C^1$ embedding
$\Phi:E\to \R^d$ such that:
\begin{enumerate}
\item (\emph{small slope}) $\Phi$ is a $C^1$ graph over $E$ with slope $\le \varepsilon$, meaning $\|D\Phi-\mathrm{Id}\|_{C^0(E)}\le C\,\varepsilon$ in the coordinates of $P$;
\item (\emph{small displacement}) $\sup_{x\in E}|\Phi(x)-x|<\delta/2$.
\end{enumerate}
\end{definition}

The purpose of these definitions is that they isolate what later gluing arguments actually use: the face incidence pattern and the size of boundary traces.

\section{An explicit complex corner-exit translation template in $\C^n$}

We now give an explicit construction of corner-exit templates in $\C^n$ with a $(2p-1)$-parameter translation family and identical footprints.

Fix integers $n\ge 1$ and $1\le p\le n$, and set
\[
k := 2(n-p),\qquad d:=2n.
\]
Identify $\C^n\cong \R^{2n}$ and write $\C^n=\C^{n-p}\times\C^p$ with coordinates $z=(u,w)$, where
$u=(u_1,\dots,u_{n-p})$ and $w=(w_1,\dots,w_p)$.

\begin{lemma}[A concrete complex corner-exit translation template in a cube]
Let $Q=[0,h]^{2n}\subset \R^{2n}\cong \C^n$ be the coordinate cube with vertex $0$.
Fix a constant $0<c_0<1$ and choose a scale $s>0$ with $s\le c_0 h/100$.

Define a complex $(n-p)$-plane $P\subset\C^n$ as the graph of the complex-linear map $A:\C^{n-p}\to\C^p$ given by
\[
w_1=-(1-i)\sum_{j=1}^{n-p}u_j,\qquad w_2=\cdots=w_p=0.
\]
For translation parameters $t=(t_1,\dots,t_p)\in\C^p$, write
\[
P_t:=\{(u,Au+t): u\in\C^{n-p}\}.
\]
Assume $t$ satisfies the interior-margin bounds
\[
\Re t_1=s,\qquad 2s\le \Im t_1\le 3s,\qquad
2s\le \Re t_j,\Im t_j\le 3s\ \ (2\le j\le p).
\]
Then $E(t):=P_t\cap Q$ has the following properties:
\begin{enumerate}
\item (\emph{Corner-exit simplex footprint}) $E(t)$ is a $k$-simplex contained in the ball $B(0,c_0 h)$.
\item (\emph{Fixed designated exit faces}) The $k+1$ facets of $E(t)$ lie on the $k+1$ coordinate faces
\[
F_{\Re u_j=0},\quad F_{\Im u_j=0}\ (1\le j\le n-p),
\qquad\text{and}\qquad F_{\Re w_1=0},
\]
and $E(t)$ meets no other codimension-one faces of $Q$.
\item (\emph{Uniform fatness and equal footprint geometry}) Throughout the admissible parameter box (with fixed $\Re t_1=s$),
the footprints are identical up to translation in transverse directions. In particular, $\Hh^k(E(t))$ and each facet measure
$\Hh^{k-1}(E(t)\cap F)$ are independent of $t$. Moreover, $E(t)$ is $\Lambda$-fat for a constant $\Lambda$ depending only on $(n,p)$.
\end{enumerate}
Finally, the admissible parameter box has real dimension $2p-1$, so for any separation scale $\eta>0$ one can choose an ordered $\eta$-separated list
$(t_a)_{a\ge 1}$ in that box. All footprints $E(t_a)$ then have exactly equal $k$-volume and exactly equal facet measures.
\end{lemma}

\begin{proof}
Write $u_j=x_j+i y_j$ with $x_j=\Re u_j$ and $y_j=\Im u_j$.
On $P_t$ one computes
\[
\Re w_1 = \Re t_1 + \Re\Bigl(-(1-i)\sum_{j=1}^{n-p}u_j\Bigr)
= s - \sum_{j=1}^{n-p}(x_j+y_j),
\]
and
\[
\Im w_1 = \Im t_1 + \Im\Bigl(-(1-i)\sum_{j=1}^{n-p}u_j\Bigr)
= \Im t_1 + \sum_{j=1}^{n-p}(x_j-y_j).
\]
The cube constraints on $w_2,\dots,w_p$ are automatic because $w_j\equiv t_j$ and the margin assumptions place each $t_j$ strictly inside $(0,h)+i(0,h)$.

Consider the region cut out by $x_j\ge 0$, $y_j\ge 0$, and $\sum_j(x_j+y_j)\le s$.
On this region one has $\bigl|\sum_j(x_j-y_j)\bigr|\le \sum_j(x_j+y_j)\le s$, hence
\[
\Im w_1 \in [\Im t_1-s,\ \Im t_1+s]\subset [s,4s]\subset (0,h),
\]
so the faces $\{\Im w_1=0\}$ and $\{\Im w_1=h\}$ are avoided. Also $\Re w_1\in[0,s]\subset(0,h)$ avoids the face $\{\Re w_1=h\}$.
Finally $x_j,y_j\le s\ll h$ avoids the far faces $\{\Re u_j=h\}$ and $\{\Im u_j=h\}$.

Consequently, $E(t)=P_t\cap Q$ is cut out on $P_t$ exactly by the inequalities
\[
x_j\ge 0,\quad y_j\ge 0\quad (1\le j\le n-p),
\qquad\text{and}\qquad \Re w_1\ge 0,
\]
which is equivalent (on $P_t$) to $\sum_j(x_j+y_j)\le s$ together with the nonnegativity of the $k=2(n-p)$ real coordinates
$(x_1,y_1,\dots,x_{n-p},y_{n-p})$. This is the standard $k$-simplex of scale $s$ embedded affinely in $\R^{2n}$,
so $E(t)$ is a $k$-simplex and its facets lie exactly on the faces listed, proving (1) and (2).
Since $s\le c_0h/100$, the whole simplex lies in a ball $B(0,c_0h)$.

The defining inequalities on $(x_j,y_j)$ do not depend on $t$ once $\Re t_1=s$ is fixed, so $\Hh^k(E(t))$ and all facet measures are independent of $t$.
Fatness follows because $E(t)$ is the image of the standard simplex under the fixed affine embedding $u\mapsto (u,Au+t)$;
the distortion is controlled by the condition number of the fixed linear map $(\Id,A)$, hence by a constant depending only on $(n,p)$.

The dimension count is immediate: $\Re t_1$ is fixed, $\Im t_1$ varies in an interval (1 real parameter),
and for each $j=2,\dots,p$ both $\Re t_j$ and $\Im t_j$ vary (2 real parameters each), giving $1+2(p-1)=2p-1$.
\end{proof}

\section{Two quantitative lemmas: graph distortion and simplex facet scaling}

We now isolate two quantitative facts used repeatedly.

\begin{lemma}[Small-slope graph distortion for volumes and facet measures]
Let $k\ge 1$ and let $E\subset\R^k$ be measurable. Let $G:E\to\R^q$ be $C^1$ with
\[
\|DG\|_{L^\infty(E)}\le \varepsilon,\qquad 0<\varepsilon\le 1.
\]
Define the graph map $\Gamma:E\to\R^{k+q}$ by $\Gamma(x)=(x,G(x))$. Then
\[
1 \ \le\ J_k\Gamma(x)=\sqrt{\det\bigl(I_k+(DG(x))^T DG(x)\bigr)}\ \le\ (1+\varepsilon^2)^{k/2}\ \le\ 1+k\varepsilon^2,
\]
and consequently
\[
\Hh^k(E)\ \le\ \Hh^k(\Gamma(E))\ \le\ (1+k\varepsilon^2)\,\Hh^k(E).
\]
Moreover, if $F\subset E$ is contained in a $C^1$ hypersurface in $\R^k$, then the same estimate holds in dimension $(k-1)$:
\[
\Hh^{k-1}(F)\ \le\ \Hh^{k-1}(\Gamma(F))\ \le\ (1+(k-1)\varepsilon^2)\,\Hh^{k-1}(F).
\]
\end{lemma}

\begin{proof}
The matrix $A:=(DG)^T DG$ is positive semidefinite and every eigenvalue of $A$ is at most $\|DG\|^2\le \varepsilon^2$.
Hence $\det(I_k+A)\le (1+\varepsilon^2)^k$ and so $J_k\Gamma\le (1+\varepsilon^2)^{k/2}$. The inequality
$(1+\varepsilon^2)^{k/2}\le 1+k\varepsilon^2$ holds for $0\le\varepsilon\le 1$ by convexity of $t\mapsto (1+t)^{k/2}$.
The volume inequalities follow from integrating the Jacobian over $E$.

For the $(k-1)$-dimensional bound, restrict $\Gamma$ to a $(k-1)$-dimensional tangent subspace:
the corresponding Jacobian is controlled by the same eigenvalue bound, with $k$ replaced by $k-1$.
\end{proof}

\begin{lemma}[Facet mass comparable to $v^{(k-1)/k}$ for fat simplices]
Fix $d\ge 2$, $1\le k<d$, and $\Lambda\ge 1$.
Let $E\subset\R^d$ be a $\Lambda$-fat $k$-simplex and write $v_E:=\Hh^k(E)$.
Let $\sigma_0,\dots,\sigma_k$ be the $(k-1)$-facets of $E$, and set $a_i:=\Hh^{k-1}(\sigma_i)$.
Then there exist constants $0<c_\star\le C_\star<\infty$ depending only on $(k,\Lambda)$ such that for every $i=0,\dots,k$,
\[
c_\star\, v_E^{(k-1)/k}\ \le\ a_i\ \le\ C_\star\, v_E^{(k-1)/k}.
\]
\end{lemma}

\begin{proof}
By $\Lambda$-fatness there is an affine isomorphism $A:\mathrm{aff}(E)\to\R^k$ with operator norms of $DA$ and $(DA)^{-1}$
bounded by $\Lambda$, mapping $E$ to a standard simplex $\Delta_s$ of some scale $s>0$.
Since $A$ is affine, its $k$-Jacobian and $(k-1)$-Jacobian are constant on $\mathrm{aff}(E)$ and are bounded above and below
by constants depending only on $(k,\Lambda)$. Therefore
\[
v_E=\Hh^k(E)\simeq_{k,\Lambda}\Hh^k(\Delta_s)\simeq_k s^k,
\qquad
a_i=\Hh^{k-1}(\sigma_i)\simeq_{k,\Lambda}\Hh^{k-1}(\text{facet of }\Delta_s)\simeq_k s^{k-1}.
\]
Eliminating $s$ yields $a_i\simeq_{k,\Lambda} v_E^{(k-1)/k}$ uniformly in $i$.
\end{proof}

\section{Stability under $C^1$ perturbations: deterministic face incidence and boundary control}

We now prove the two main outputs (G1)--(G2) in a general Euclidean form.

\begin{proposition}[Corner-exit footprint geometry for small-slope graphs]
Fix $d\ge 2$ and $1\le k<d$. Let $Q=[0,h]^d\subset\R^d$ and let $E=P\cap Q$ be a corner-exit simplex footprint
with designated exit faces $F_0,\dots,F_k$ and gap $\delta>0$ to all non-designated faces.
Assume $E$ is $\Lambda$-fat.

Let $Y\subset\R^d$ be a smooth oriented $k$-submanifold such that $Y\cap Q=\Phi(E)$ where $\Phi:E\to\R^d$ is a $C^1$ graph over $E$
with slope at most $\varepsilon$ and displacement $\sup_{x\in E}|\Phi(x)-x|<\delta/2$.

Then:
\begin{enumerate}
\item[\textnormal{(G1)}] (\emph{Face incidence is deterministic.})
For any codimension-one face $F$ of $Q$,
\[
Y\cap F\neq\emptyset\quad\Longleftrightarrow\quad F\in\{F_0,\dots,F_k\}.
\]
\item[\textnormal{(G2)}] (\emph{Per-face boundary mass is uniformly controlled.})
For each $i=0,\dots,k$, the intersection $Y\cap F_i$ is a smooth oriented $(k-1)$-submanifold and
\[
\Hh^{k-1}(Y\cap F_i)=\bigl(1+O_k(\varepsilon^2)\bigr)\,\Hh^{k-1}(E\cap F_i)\ \simeq_{k,\Lambda}\ v_E^{(k-1)/k},
\]
where $v_E=\Hh^k(E)$ and the implied constants depend only on $(k,\Lambda)$.
\end{enumerate}
\end{proposition}

\begin{proof}
\textnormal{(G1)} Let $F$ be any codimension-one face not in $\{F_0,\dots,F_k\}$.
By definition of $\delta$, $\dist(E,F)\ge \delta$.
If $y\in Y\cap F$, then $y=\Phi(x)$ for some $x\in E$, hence
\[
\delta\ \le\ \dist(x,F)\ \le\ |x-\Phi(x)|\ <\ \delta/2,
\]
a contradiction. Therefore $Y\cap F=\emptyset$ for all non-designated faces.

Conversely, if $F=F_i$ is designated, then $E\cap F_i$ is a nonempty facet of $E$.
Because $\Phi$ is $C^1$-close to the identity on $E$ and maps $E$ into $Q$, the image of this facet must meet $F_i$.
(Geometrically: the facet is the locus where a defining inequality of $E$ becomes equality; a small $C^1$ perturbation preserves that boundary contact.)
Thus $Y\cap F_i\neq\emptyset$.

\textnormal{(G2)} The intersection $Y\cap F_i$ is a small-slope graph over the facet $E\cap F_i$,
so the $(k-1)$-dimensional graph distortion lemma gives
\[
\Hh^{k-1}(Y\cap F_i)=\bigl(1+O_k(\varepsilon^2)\bigr)\,\Hh^{k-1}(E\cap F_i).
\]
Since $E$ is $\Lambda$-fat, the facet scaling lemma yields $\Hh^{k-1}(E\cap F_i)\simeq_{k,\Lambda} v_E^{(k-1)/k}$,
giving the final bound.
\end{proof}

\begin{remark}[Currents language (what later gluing uses)]
If $Y$ is a smooth oriented $k$-manifold without boundary, the integration current $[Y]$ satisfies
$\partial([Y]\llcorner Q)=[Y]\llcorner \partial Q$. In that case,
$\Mass(\partial([Y]\llcorner Q)\llcorner F_i)=\Hh^{k-1}(Y\cap F_i)$.
Thus (G2) can be read as a quantitative per-face boundary mass estimate for the restricted current $[Y]\llcorner Q$.
\end{remark}

\section{Boundary-face $L^1$ interface control}

We now package (G2) into an $L^1$-type estimate that is robust under taking many slivers.

\begin{proposition}[$L^1$ interface mass control on boundary faces]
Fix $d\ge 2$ and $1\le k<d$. Let $Q=[0,h]^d$ and let $Y^{(1)},\dots,Y^{(N)}\subset\R^d$ be smooth oriented $k$-submanifolds
such that for each $a$ the piece $Y^{(a)}\cap Q$ is a corner-exit sliver over a $\Lambda$-fat corner-exit simplex footprint $E_a$
(with some designated exit faces depending on $a$), with slope at most $\varepsilon\le 1$.

Write $v_a:=\Hh^k(E_a)$. Then there exists a constant $C=C(k,\Lambda)$ such that
\[
\sum_{a=1}^N \sum_{F\subset \partial Q\ \text{\rm codim-1}} \Hh^{k-1}\!\bigl(Y^{(a)}\cap F\bigr)
\ \le\ C\,(1+O_k(\varepsilon^2))\ \sum_{a=1}^N v_a^{(k-1)/k}.
\]
In particular, if one measures boundary by the mass of $\partial([Y^{(a)}]\llcorner Q)$, then
\[
\sum_{a=1}^N \Mass\bigl(\partial([Y^{(a)}]\llcorner Q)\bigr)\ \le\ C\,(1+O_k(\varepsilon^2))\ \sum_{a=1}^N v_a^{(k-1)/k}.
\]
\end{proposition}

\begin{proof}
Each footprint $E_a$ has exactly $k+1$ designated exit faces, and meets no other codimension-one faces.
By (G2), on each designated face the $(k-1)$-measure of the boundary slice is $\simeq_{k,\Lambda} v_a^{(k-1)/k}$
up to the common $1+O_k(\varepsilon^2)$ factor coming from the graph distortion.
Summing the $k+1$ face contributions per sliver and then summing over $a$ yields the stated bound.
\end{proof}

\begin{corollary}[No-heavy-tail uniformity for the explicit complex template family]
In the explicit $\C^n$ construction of the corner-exit translation template, all footprints $E(t)$ in the admissible parameter box
have identical $k$-volume and identical facet measures. Therefore an ordered separated list $(t_a)$ produces a family of corner-exit templates
with exactly equal per-piece boundary scales (no heavy tails along the order).
If a realized family $(Y^{(a)})$ consists of small-slope graphs over these footprints with a common slope bound $\varepsilon$,
then all per-piece boundary masses differ only by a common $(1+O(\varepsilon^2))$ factor.
\end{corollary}

\section{Uniform corner-exit template dictionaries for direction nets}

For later global constructions, one often fixes a finite net of calibrated directions and wants a uniform corner-exit template family
associated to each direction label.

\begin{proposition}[Uniform corner-exit templates for a finite calibrated direction net]
Fix $(n,p)$ and work locally in holomorphic charts in a K\"ahler manifold so that cubes are defined in holomorphic normal coordinates.
Let $\{\Pi_1,\dots,\Pi_M\}$ be a finite set of complex $(n-p)$-planes in $\C^n$ (a finite direction net).
Then there exist constants $c_0\in(0,1)$ and $\Lambda\ge 1$, depending only on $(n,p)$, with the following property:

For each $j\in\{1,\dots,M\}$ there exists a choice of unitary holomorphic coordinates in which $\Pi_j$ is represented as a fixed reference plane,
and in those coordinates the explicit complex corner-exit translation template produces a $(2p-1)$-parameter family of calibrated planes
whose cube intersections are $\Lambda$-fat corner-exit simplex footprints with the same designated-face pattern and the same footprint scale.
All constants ($\Lambda$, the slope-to-face-incidence tolerance, and the per-face mass comparability constants) are uniform in $j$.
\end{proposition}

\begin{proof}
All complex $(n-p)$-planes in $\C^n$ are related by unitary transformations, and unitary transformations preserve calibration,
Euclidean norms, and the quantitative fatness constants of simplices.
Because the net is finite, one may choose for each $\Pi_j$ a unitary coordinate frame adapted to it, and then apply the explicit template
construction in that frame with the same fixed linear map $A$.
Uniformity of the constants follows because the explicit template constants depend only on $(n,p)$ and the chosen scale separation $s\ll h$,
not on $j$.
\end{proof}

\section{Interface with Bergman-scale holomorphic manufacturing}

The Euclidean results above are purely geometric. To obtain \emph{holomorphic} corner-exit slivers in a projective K\"ahler manifold,
one needs a separate analytic input: a local manufacturing result guaranteeing that, on a Bergman-scale cube, a holomorphic complete intersection
can be made into a small-slope graph over a prescribed affine complex plane template.

We record the interface as a clean corollary: if holomorphic manufacturing produces a small-slope graph over a corner-exit footprint,
then the corner-exit conclusions follow automatically from the geometry already proved.

\begin{corollary}[Holomorphic corner-exit slivers inherit deterministic face incidence and boundary control]
Let $X$ be a smooth complex projective manifold with K\"ahler form $\omega$ arising as the curvature of an ample line bundle.
Fix $p$ and let $k=2(n-p)$.

Fix a holomorphic chart identifying a neighborhood with a cube $Q=[0,h]^{2n}\subset\C^n$.
Let $P_t$ be any plane in the explicit complex corner-exit translation template family (with footprint scale $s\ll h$) and let
$E(t)=P_t\cap Q$ be the corresponding corner-exit simplex footprint.

Suppose a holomorphic complete intersection $Y\subset X$ (cut out by $p$ holomorphic sections of a large tensor power)
satisfies that $Y\cap Q$ is a single $C^1$ graph over $E(t)$ with slope at most $\varepsilon$ and displacement $<\delta/2$,
where $\delta$ is the gap from $E(t)$ to all non-designated cube faces.

Then $Y\cap Q$ is a corner-exit sliver and satisfies (G1)--(G2) from the stability proposition:
it intersects a cube face if and only if that face is designated by the template, and each designated face slice has
boundary mass $\simeq v^{(k-1)/k}$ where $v=\Hh^k(E(t))$.
Consequently, any finite family of such holomorphic corner-exit slivers in $Q$ satisfies the $L^1$ interface estimate of the previous section.
\end{corollary}

\begin{proof}
The hypotheses are exactly those of the stability proposition (with $d=2n$).
Holomorphicity is used only to ensure smoothness of $Y$ and the absence of boundary inside $Q$;
the face-incidence and boundary-mass conclusions are geometric consequences of being a small-slope graph over a corner-exit footprint.
\end{proof}

\section{Discussion}

Corner-exit slivers are structurally stronger than generic ``graph over a plane'' pieces for one specific reason:
they promote boundary behavior from geometry to combinatorics.
A corner-exit footprint comes with a \emph{finite} face-incidence pattern that is stable under perturbations,
and with a uniform boundary scale that depends only on the footprint volume.
This makes it possible to run later matching and gluing arguments on a mesh without needing to solve a geometric boundary-tracing problem anew in every cell.

\end{document}