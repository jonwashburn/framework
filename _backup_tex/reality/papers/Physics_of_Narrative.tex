\documentclass[11pt,twocolumn]{article}

% ============================================================================
% PACKAGES
% ============================================================================
\usepackage[utf8]{inputenc}
\usepackage[T1]{fontenc}
\usepackage{amsmath,amssymb,amsthm}
\usepackage{graphicx}
\usepackage{hyperref}
\usepackage{xcolor}
\usepackage{listings}
\usepackage[margin=1in]{geometry}

% ============================================================================
% THEOREM ENVIRONMENTS
% ============================================================================
\theoremstyle{plain}
\newtheorem{theorem}{Theorem}[section]
\newtheorem{lemma}[theorem]{Lemma}
\newtheorem{proposition}[theorem]{Proposition}
\newtheorem{corollary}[theorem]{Corollary}
\newtheorem{conjecture}[theorem]{Conjecture}

\theoremstyle{definition}
\newtheorem{definition}[theorem]{Definition}
\newtheorem{example}[theorem]{Example}
\newtheorem{axiom}{Axiom}

\theoremstyle{remark}
\newtheorem{remark}[theorem]{Remark}

% ============================================================================
% CUSTOM COMMANDS
% ============================================================================
\newcommand{\R}{\mathbb{R}}
\newcommand{\Z}{\mathbb{Z}}
\newcommand{\N}{\mathbb{N}}
\newcommand{\C}{\mathbb{C}}
\newcommand{\Jcost}{J}
\newcommand{\skew}{\sigma}
\newcommand{\energy}{\mathcal{E}}
\newcommand{\MoralState}{\mathcal{M}}
\newcommand{\NarrativeArc}{\mathcal{A}}
\newcommand{\NarrativeSpace}{\mathcal{N}}
\newcommand{\StoryMetric}{g}
\newcommand{\tension}{\tau}
\newcommand{\golden}{\varphi}
\newcommand{\Rhat}{\hat{R}}
\newcommand{\abs}[1]{\left|#1\right|}

% Lean code styling
\lstdefinelanguage{lean4}{
  keywords={def, theorem, lemma, structure, where, example, namespace, end, 
            import, open, noncomputable, section, variable, have, let, 
            calc, by, sorry, Prop, Type, inductive, class, instance},
  morecomment=[l]{--},
  morecomment=[s]{/-}{-/},
  morestring=[b]",
  sensitive=true,
}
\lstset{
  language=lean4,
  basicstyle=\ttfamily\footnotesize,
  keywordstyle=\color{blue}\bfseries,
  commentstyle=\color{gray}\itshape,
  stringstyle=\color{red},
  breaklines=true,
  columns=fullflexible,
  frame=single,
  xleftmargin=2em,
  framexleftmargin=1.5em,
}

% ============================================================================
% TITLE AND AUTHORS
% ============================================================================
\title{\textbf{The Physics of Narrative:\\
A Geometric Formalization of Story Structure\\
in Recognition Science}}

\author{
  Recognition Science Collaboration\\
  \texttt{recognition-science@protonmail.com}
}

\date{\today}

% ============================================================================
% DOCUMENT
% ============================================================================
\begin{document}

\maketitle

% ============================================================================
% ABSTRACT
% ============================================================================
\begin{abstract}
We present a mathematical formalization of narrative structure 
within Recognition Science (RS). By modeling stories as trajectories 
through a manifold of moral states, we propose that optimal narratives 
correspond to geodesics under a recognition-cost-derived metric. 
Plot tension is identified with ledger skew $\skew$, with 
resolution ($\skew \to 0$) emerging as a stable equilibrium. 
We introduce the \emph{story metric} 
$ds^2 = d\skew^2 + d\energy^2/\golden + dZ^2/\golden^2$ 
and conjecture that narratives minimizing recognition cost $\Jcost$ 
satisfy geodesic equations. The framework is partially formalized 
in Lean 4, with key stability theorems fully verified. We analyze 
\emph{Hamlet} and the Hero's Journey as concrete applications, 
demonstrating how the formalism captures classical narrative structure.
This work establishes foundations for a \emph{Physics of Narrative} 
connecting story structure to the geometry of meaning.
\end{abstract}

\textbf{Keywords:} narrative physics, Recognition Science, geodesic, 
moral state, plot tension, Lean formalization, computational narratology

% ============================================================================
% 1. INTRODUCTION
% ============================================================================
\section{Introduction}
\label{sec:introduction}

The structure of stories has been studied since Aristotle's \emph{Poetics} 
\cite{aristotle-poetics}, yet no mathematical theory has successfully 
unified the diverse observations of narrative scholars into a coherent 
framework. Propp identified 31 narrative functions in Russian folktales 
\cite{propp1928morphology}. Campbell found recurring patterns across 
mythologies \cite{campbell1949hero}. Booker proposed seven fundamental 
plots \cite{booker2004seven}. But \emph{why} do these patterns exist?

In this paper, we propose an answer within Recognition Science (RS), 
a theoretical framework deriving physics from cost minimization 
\cite{rs-theory}. Our central hypothesis is:

\begin{quote}
\emph{Stories are trajectories through moral-state space that 
tend toward cost-minimizing paths---narrative geodesics.}
\end{quote}

We do not claim to have proven this hypothesis in full generality. 
Rather, we develop the mathematical framework, prove key supporting 
theorems, and demonstrate the formalism's applicability through examples.

\subsection{Contributions}

This paper makes the following contributions:

\begin{enumerate}
\item \textbf{Mathematical Framework}: We define narrative space 
      $\NarrativeSpace$ as a manifold of MoralStates with a 
      recognition-cost-derived metric (Section~\ref{sec:narrative_space}).
      
\item \textbf{Tension-Skew Identification}: We formally identify 
      plot tension $\tension$ with ledger skew magnitude $\abs{\skew}$, 
      proving stability properties (Section~\ref{sec:tension}).
      
\item \textbf{Geodesic Conjecture}: We state precisely the conjecture 
      that optimal narratives are geodesics and prove supporting lemmas 
      (Section~\ref{sec:geodesics}).
      
\item \textbf{Concrete Examples}: We analyze \emph{Hamlet} and the 
      Hero's Journey within the formalism (Section~\ref{sec:examples}).
      
\item \textbf{Partial Formalization}: Key definitions and stability 
      theorems are verified in Lean 4; other results remain conjectural 
      (Section~\ref{sec:lean}).
\end{enumerate}

\subsection{Related Work}

Our approach builds on several traditions:

\textbf{Structural Narratology}: Propp's \emph{Morphology of the Folktale} 
\cite{propp1928morphology} identified discrete narrative functions. 
Our continuous formalism generalizes this to smooth trajectories.

\textbf{Story Grammars}: Computational approaches model stories as 
formal languages \cite{rumelhart1975notes}. We instead use 
differential geometry, treating narrative as physics.

\textbf{Affective Computing}: The valence-arousal-dominance model 
\cite{russell1980circumplex, mehrabian1996pleasure} provides our 
bridge to phenomenal experience.

\textbf{Mathematical Narratology}: Recent work applies topology 
\cite{dena2009transmedia} and network theory \cite{reagan2016emotional} 
to stories. Our geometric approach is complementary.

% ============================================================================
% 2. RECOGNITION SCIENCE BACKGROUND
% ============================================================================
\section{Recognition Science Preliminaries}
\label{sec:background}

We briefly review Recognition Science; see \cite{rs-theory} for details.

\subsection{The Recognition Cost Functional}

Recognition Science posits a single primitive: the recognition 
cost function $\Jcost: \R^+ \to \R^+$ satisfying the d'Alembert 
composition law:
\begin{equation}
\label{eq:dalembert}
\Jcost(xy) + \Jcost(x/y) = 2\Jcost(x)\Jcost(y) + 2\Jcost(x) + 2\Jcost(y)
\end{equation}

This functional equation has a unique continuous solution:
\begin{equation}
\label{eq:Jcost}
\Jcost(x) = \frac{1}{2}\left(x + \frac{1}{x}\right) - 1
\end{equation}

\begin{proposition}[Properties of $\Jcost$]
\label{prop:Jcost_properties}
The recognition cost satisfies:
\begin{enumerate}
\item $\Jcost(x) \geq 0$ with equality iff $x = 1$
\item $\Jcost(x) = \Jcost(1/x)$ (reciprocal symmetry)
\item $\Jcost$ is convex with global minimum at $x = 1$
\item $\Jcost(x) \sim \frac{1}{2}x$ as $x \to \infty$
\end{enumerate}
\end{proposition}

\begin{proof}
Direct calculation from Equation~\eqref{eq:Jcost}. See \cite{rs-theory}.
\end{proof}

\subsection{The Golden Ratio}

The golden ratio $\golden = (1 + \sqrt{5})/2 \approx 1.618$ emerges 
as the unique positive solution to $\golden^2 = \golden + 1$. 
In RS, $\golden$ appears as the fundamental scale factor relating 
different levels of structure.

\subsection{MoralState and Ledger Skew}

\begin{definition}[MoralState]
\label{def:moralstate}
A \emph{MoralState} $\MoralState$ is a tuple $(L, B, \skew, \energy)$ where:
\begin{itemize}
\item $L$ is a ledger state (transaction record)
\item $B \subseteq \text{Bonds}$ are active agent bonds
\item $\skew \in \R$ is the reciprocity skew
\item $\energy > 0$ is available energy
\end{itemize}
subject to the \emph{global balance constraint}: 
$\sum_{\text{agents}} \skew_i = 0$.
\end{definition}

The skew $\skew$ measures deviation from perfect reciprocity:
\begin{itemize}
\item $\skew > 0$: extraction/debt (receiving more than giving)
\item $\skew < 0$: contribution/credit (giving more than receiving)
\item $\skew = 0$: balanced reciprocity
\end{itemize}

\subsection{The Recognition Operator}

Dynamics are governed by the recognition operator $\Rhat$:
\begin{equation}
s(t + 8\tau_0) = \Rhat(s(t))
\end{equation}
where $\tau_0$ is the fundamental time unit and $\Rhat$ evolves states 
toward lower $\Jcost$.

\begin{axiom}[Recognition Dynamics]
\label{ax:recognition}
The operator $\Rhat$ satisfies:
\begin{enumerate}
\item $\Jcost(\Rhat(s)) \leq \Jcost(s)$ (cost non-increasing)
\item $\Rhat$ preserves global balance
\item States with $\skew = 0$ are fixed points
\end{enumerate}
\end{axiom}

% ============================================================================
% 3. NARRATIVE SPACE
% ============================================================================
\section{The Narrative Space Manifold}
\label{sec:narrative_space}

\subsection{Narrative Beats}

\begin{definition}[Narrative Beat]
\label{def:beat}
A \emph{narrative beat} is a pair $b = (m, t)$ where $m \in \MoralState$ 
and $t \in \N$ is the beat index (in 8-tick units).
\end{definition}

Each beat represents a story ``moment''---a snapshot of the moral 
configuration of the narrative world.

\subsection{Narrative Arcs}

\begin{definition}[Narrative Arc]
\label{def:arc}
A \emph{narrative arc} $\NarrativeArc = \{b_0, \ldots, b_n\}$ is a 
finite sequence of beats satisfying:
\begin{enumerate}
\item \textbf{Non-trivial}: $n \geq 1$ (at least two beats)
\item \textbf{Temporally ordered}: $t_i < t_j$ for $i < j$
\item \textbf{Admissible}: global balance holds at each beat
\end{enumerate}
\end{definition}

\subsection{The Story Metric}

We endow MoralState space with a Riemannian metric derived from 
recognition cost considerations.

\begin{definition}[Story Metric]
\label{def:story_metric}
The \emph{story metric} is:
\begin{equation}
\label{eq:story_metric}
ds^2 = d\skew^2 + \frac{1}{\golden} d\energy^2 + \frac{1}{\golden^2} dZ^2
\end{equation}
where $Z$ encodes pattern state.
\end{definition}

\begin{remark}[Metric Coefficients]
\label{rem:metric_derivation}
The coefficients $(1, 1/\golden, 1/\golden^2)$ reflect a hierarchy 
where skew changes dominate over energy changes, which dominate 
over pattern changes. This ordering arises from the RS forcing 
chain: $\skew$ directly affects $\Jcost$, while $\energy$ and $Z$ 
affect it indirectly. The $\golden$-weighting is the simplest 
choice respecting this hierarchy. \emph{Alternative weightings 
could be explored; this choice is motivated but not uniquely determined.}
\end{remark}

\begin{proposition}[Metric Positivity]
\label{prop:metric_positive}
The story metric is positive definite.
\end{proposition}

\begin{proof}
Since $\golden > 0$, all coefficients are positive. Thus 
$ds^2 \geq 0$ with equality only when $d\skew = d\energy = dZ = 0$.
\end{proof}

% ============================================================================
% 4. PLOT TENSION
% ============================================================================
\section{Plot Tension as Skew Dynamics}
\label{sec:tension}

\subsection{The Tension Function}

\begin{definition}[Plot Tension]
\label{def:tension}
The \emph{plot tension} at beat $b = (m, t)$ is:
\begin{equation}
\tension(b) = \abs{m.\skew}
\end{equation}
\end{definition}

This identification is our core proposal: dramatic tension equals 
the magnitude of reciprocity imbalance.

\begin{table}[h]
\centering
\caption{Interpretation of Skew Values}
\label{tab:skew_interpretation}
\begin{tabular}{|c|l|}
\hline
$\skew$ Value & Narrative Interpretation \\
\hline
$\skew = 0$ & Equilibrium (resolution) \\
$\skew > 0$ & Protagonist in debt (crisis) \\
$\skew < 0$ & Protagonist owed (sacrifice) \\
$\abs{\skew}$ large & High tension (climax) \\
\hline
\end{tabular}
\end{table}

\subsection{Tension Thresholds}

The golden ratio provides natural thresholds:

\begin{definition}[Tension Thresholds]
\label{def:thresholds}
\begin{align}
\tension_{\text{low}} &= 1/\golden \approx 0.618 \\
\tension_{\text{high}} &= \golden \approx 1.618 \\
\tension_{\text{critical}} &= \golden^2 \approx 2.618
\end{align}
\end{definition}

\begin{theorem}[Threshold Ordering]
\label{thm:threshold_ordering}
The thresholds satisfy:
$\tension_{\text{low}} < 1 < \tension_{\text{high}} < \tension_{\text{critical}}$
\end{theorem}

\begin{proof}
Since $\golden > 1$: $1/\golden < 1 < \golden < \golden^2$.
\end{proof}

\subsection{Catharsis}

\begin{definition}[Catharsis]
\label{def:catharsis}
A \emph{catharsis} occurs between beats $b_i$ and $b_{i+1}$ if:
\begin{enumerate}
\item $\tension(b_i) > \tension_{\text{high}}$
\item $\tension(b_{i+1}) < \tension_{\text{low}}$
\item $\tension(b_i) - \tension(b_{i+1}) > \tension(b_i)/2$
\end{enumerate}
\end{definition}

This formalizes Aristotle's concept of emotional purgation.

\begin{theorem}[Monotonic Arcs Lack Catharsis]
\label{thm:no_catharsis}
If $\tension(b_i) < \tension(b_{i+1})$ for all $i$, then 
$\NarrativeArc$ has no catharsis.
\end{theorem}

\begin{proof}
Catharsis requires $\tension$ to drop (condition 3), but strict 
monotonicity precludes any decrease.
\end{proof}

\subsection{Resolution Stability}

\begin{theorem}[Resolution is Stable]
\label{thm:resolution_stable}
The state $\skew = 0$ is a stable equilibrium in the sense that 
for every $\epsilon > 0$, if $\abs{\skew} < \epsilon$ then 
$\tension < \epsilon$.
\end{theorem}

\begin{proof}
By definition, $\tension = \abs{\skew}$, so $\abs{\skew} < \epsilon$ 
implies $\tension < \epsilon$ directly.
\end{proof}

\begin{corollary}[Perturbation Increases Tension]
\label{cor:perturbation}
Any perturbation from $\skew = 0$ strictly increases tension.
\end{corollary}

\begin{proof}
At $\skew = 0$, $\tension = 0$. For $\skew' \neq 0$, 
$\tension' = \abs{\skew'} > 0$.
\end{proof}

% ============================================================================
% 5. GEODESICS
% ============================================================================
\section{Story Geodesics}
\label{sec:geodesics}

\subsection{The Story Action}

\begin{definition}[Story Action]
\label{def:story_action}
The \emph{story action} of an arc $\NarrativeArc$ is:
\begin{equation}
S[\NarrativeArc] = \sum_{b \in \NarrativeArc} \Jcost(b)
\end{equation}
where $\Jcost(b) = \sum_{\text{bonds}} \Jcost(m_b)$ is the total 
recognition cost at beat $b$.
\end{definition}

\subsection{The Geodesic Conjecture}

\begin{conjecture}[Narrative Geodesic Principle]
\label{conj:geodesic}
Among all narrative arcs connecting fixed initial and terminal 
MoralStates, the ones actually realized tend toward those 
minimizing the story action $S[\NarrativeArc]$.
\end{conjecture}

\begin{remark}
This is analogous to the principle of least action in physics. 
We state it as a conjecture because:
\begin{enumerate}
\item The space of ``all possible narratives'' is not rigorously defined
\item Empirical validation requires analysis of story corpora
\item The ``tends toward'' language reflects that real stories may 
      deviate for artistic reasons
\end{enumerate}
\end{remark}

\begin{proposition}[Geodesic Equation Form]
\label{prop:geodesic_eq}
If Conjecture~\ref{conj:geodesic} holds, geodesic arcs satisfy:
\begin{equation}
\nabla_{\dot{\gamma}} \dot{\gamma} = 0
\end{equation}
where $\gamma(t)$ is the path in MoralState space and $\nabla$ is 
the Levi-Civita connection of the story metric.
\end{proposition}

\begin{proof}
Standard variational calculus. The Euler-Lagrange equations for 
the metric Lagrangian $\mathcal{L} = \frac{1}{2}g_{ij}\dot{x}^i\dot{x}^j$ 
yield the geodesic equation. The specific form of the Christoffel 
symbols depends on coordinate choice.
\end{proof}

\subsection{Resolution as Attractor}

\begin{theorem}[Resolution Attractor]
\label{thm:attractor}
Under Axiom~\ref{ax:recognition}, the state $\skew = 0$ is an 
attractor of the recognition dynamics.
\end{theorem}

\begin{proof}
By Axiom~\ref{ax:recognition}, $\Jcost$ decreases under $\Rhat$. 
Since $\Jcost$ achieves its minimum at $\skew = 0$ (the unit state 
corresponds to balanced reciprocity), and $\skew = 0$ is a fixed 
point, trajectories converge toward resolution.
\end{proof}

This explains why stories naturally tend toward resolution: it is 
the dynamical attractor.

% ============================================================================
% 6. FUNDAMENTAL PLOTS
% ============================================================================
\section{Fundamental Plot Types}
\label{sec:fundamental_plots}

\subsection{Classification}

Following Booker \cite{booker2004seven}, we identify seven plot types 
as distinct trajectory classes:

\begin{definition}[Fundamental Plot Types]
\label{def:plot_types}
\begin{enumerate}
\item \textbf{Overcoming the Monster}: $\skew: 0 \to + \to 0$, 
      high peak, full resolution
\item \textbf{Rags to Riches}: $\skew: - \to 0$, $\energy$ increasing
\item \textbf{The Quest}: $\skew: 0 \to + \to 0$, $\energy$ conserved
\item \textbf{Voyage and Return}: cyclic $\skew$, return to origin
\item \textbf{Comedy}: oscillating $\skew$, bounded by $\tension_{\text{high}}$
\item \textbf{Tragedy}: $\skew \to +$, no resolution
\item \textbf{Rebirth}: $\skew: 0 \to - \to 0$, sacrifice then renewal
\end{enumerate}
\end{definition}

\begin{conjecture}[Plot Exhaustiveness]
\label{conj:seven_plots}
The seven plot types exhaust the geodesic equivalence classes 
of resolvable narratives (those with $\tension_{\text{terminal}} 
< \tension_{\text{low}}$), plus Tragedy as the unique unbounded class.
\end{conjecture}

\begin{remark}
This conjecture requires rigorous definition of ``geodesic equivalence'' 
and proof that no other classes exist. Current evidence is empirical 
(Booker's literary analysis) rather than mathematical.
\end{remark}

\begin{table}[h]
\centering
\caption{Plot Type Characteristics}
\label{tab:plot_templates}
\begin{tabular}{|l|c|c|}
\hline
Plot Type & Peak $\tension$ & Resolution \\
\hline
Overcoming Monster & High & Yes \\
Rags to Riches & Medium & Yes \\
Quest & Medium & Yes \\
Voyage/Return & Medium & Yes \\
Comedy & Low & Yes \\
Tragedy & Unbounded & No \\
Rebirth & Medium & Yes \\
\hline
\end{tabular}
\end{table}

% ============================================================================
% 7. CONCRETE EXAMPLES
% ============================================================================
\section{Concrete Examples}
\label{sec:examples}

We apply the formalism to two well-known narratives.

\subsection{Example: Hamlet}

Shakespeare's \emph{Hamlet} provides a paradigm case of Tragedy.

\textbf{Initial state} ($\skew_0 \approx 0.5$): The play opens with 
Denmark ``out of joint.'' Claudius has murdered King Hamlet, creating 
an unpaid moral debt. The ghost reveals this, initiating rising skew.

\textbf{Rising action} ($\skew \to 1.5$): Hamlet's discovery increases 
tension. His feigned madness, the play-within-a-play, Polonius's death---each 
event increases $\abs{\skew}$ as the moral debt compounds.

\textbf{Climax} ($\skew_{\max} \approx 2.5$): The final scene. Multiple 
deaths create maximum disequilibrium. The skew reaches 
$\tension_{\text{critical}}$.

\textbf{Terminal state} ($\skew_{\text{final}} > 0$): Despite the 
bloodshed, the moral ledger is not balanced. Fortinbras inherits a 
broken kingdom. The lack of resolution marks Tragedy: 
$\lim_{t \to \infty} \tension(t) \not\to 0$.

\subsection{Example: The Hero's Journey}

Campbell's monomyth \cite{campbell1949hero} exemplifies Overcoming 
the Monster / Quest.

\begin{table}[h]
\centering
\caption{Hero's Journey $\skew$-Trajectory}
\label{tab:heros_journey}
\begin{tabular}{|c|l|c|}
\hline
Beat & Stage & $\skew$ \\
\hline
0 & Ordinary World & 0.0 \\
1 & Call to Adventure & 0.5 \\
2 & Crossing Threshold & 1.0 \\
3 & Ordeal (Climax) & 1.5 \\
4 & Reward & 1.2 \\
5 & Road Back & 0.8 \\
6 & Resurrection & 0.3 \\
7 & Return & 0.0 \\
\hline
\end{tabular}
\end{table}

This trajectory has the characteristic shape:
\begin{itemize}
\item Initial equilibrium: $\skew_0 = 0$
\item Rising tension: $\skew \to 1.5$ (climax at beat 3)
\item Resolution: $\skew \to 0$ (beat 7)
\item Catharsis: between beats 3 and 7 (tension drops $>50\%$)
\end{itemize}

The Hero's Journey is a prototypical geodesic: it achieves resolution 
(the attractor) via a path that rises to climax then descends, 
consistent with $\Jcost$-minimization.

% ============================================================================
% 8. BRIDGE TO QUALIA
% ============================================================================
\section{The Narrative-Qualia Bridge}
\label{sec:qualia}

\subsection{Phenomenal Signature}

We map narrative states to the VAD emotional model 
\cite{russell1980circumplex, mehrabian1996pleasure}.

\begin{definition}[Phenomenal Signature]
\label{def:phenomenal}
The \emph{phenomenal signature} of beat $b$ is:
\begin{align}
\text{valence}(b) &= \frac{\skew}{\abs{\skew} + 1} \in (-1, 1) \\
\text{arousal}(b) &= \frac{\tension}{\tension + 1} \in [0, 1) \\
\text{dominance}(b) &= \tanh(\energy - 1)
\end{align}
\end{definition}

\begin{remark}
The valence formula maps positive $\skew$ (debt) to negative valence 
(unpleasant), consistent with the interpretation that protagonists 
``owed'' experience positive emotion. The arousal formula is the 
standard sigmoid-type saturation.
\end{remark}

\begin{theorem}[High Tension Implies High Arousal]
\label{thm:tension_arousal}
If $\tension(b) > \golden$, then $\text{arousal}(b) > 1/2$.
\end{theorem}

\begin{proof}
For $\tension > 1$: $\text{arousal} = \tension/(\tension+1) > 1/2$ 
since $\tension > 1$ implies $2\tension > \tension + 1$.
\end{proof}

% ============================================================================
% 9. LEAN FORMALIZATION
% ============================================================================
\section{Lean 4 Formalization}
\label{sec:lean}

\subsection{Scope of Formalization}

The theory is partially formalized in Lean 4, comprising nine modules 
($\sim$2,500 lines). We distinguish:

\textbf{Fully verified} (no \texttt{sorry}):
\begin{itemize}
\item Threshold ordering (Theorem~\ref{thm:threshold_ordering})
\item Resolution stability (Theorem~\ref{thm:resolution_stable})
\item Monotonic no-catharsis (Theorem~\ref{thm:no_catharsis})
\item Tension-arousal implication (Theorem~\ref{thm:tension_arousal})
\end{itemize}

\textbf{Conjectural} (uses \texttt{sorry} or not formalized):
\begin{itemize}
\item Geodesic equation (Proposition~\ref{prop:geodesic_eq})
\item Seven plot exhaustiveness (Conjecture~\ref{conj:seven_plots})
\item Attractor property (Theorem~\ref{thm:attractor})
\end{itemize}

\subsection{Code Structure}

\begin{lstlisting}
-- Core structures
structure NarrativeBeat where
  state : MoralState
  beat_index : Nat
  
structure NarrativeArc where
  beats : List NarrativeBeat
  nonempty : beats.length >= 2
  ordered : ...
  admissible : ...

-- Verified theorem
theorem resolution_is_stable :
  forall eps > 0, exists delta > 0,
    forall b, abs(b.state.skew) < delta 
      -> plotTension b < eps := by
  intro eps h_eps
  use eps
  exact fun _ h => h
\end{lstlisting}

\begin{table}[h]
\centering
\caption{Lean Module Summary}
\label{tab:modules}
\begin{tabular}{|l|p{4cm}|}
\hline
Module & Contents \\
\hline
\texttt{Core} & Beats, arcs, basic definitions \\
\texttt{PlotTension} & Tension, thresholds, catharsis \\
\texttt{StoryGeodesic} & Action, geodesic conditions \\
\texttt{FundamentalPlots} & Plot type classification \\
\texttt{StoryTensor} & Metric, curvature \\
\texttt{Axiomatics} & RS connection \\
\texttt{Examples} & Hero's Journey, Tragedy \\
\texttt{Bridge} & Qualia correspondence \\
\texttt{Resolution} & Stability proofs \\
\hline
\end{tabular}
\end{table}

% ============================================================================
% 10. DISCUSSION
% ============================================================================
\section{Discussion}
\label{sec:discussion}

\subsection{Strengths}

The framework provides:
\begin{enumerate}
\item \textbf{Precision}: Vague concepts (tension, resolution, catharsis) 
      receive exact definitions
\item \textbf{Verifiability}: Key claims are machine-checkable
\item \textbf{Predictiveness}: The tension-skew identification 
      generates testable hypotheses
\item \textbf{Unification}: Diverse plot types emerge from one principle
\end{enumerate}

\subsection{Limitations}

Current limitations include:
\begin{itemize}
\item The geodesic principle (Conjecture~\ref{conj:geodesic}) is 
      not proven; it remains a guiding hypothesis
\item Metric coefficients are motivated but not uniquely determined
\item Multi-agent narratives require extension (current formalism 
      tracks single $\skew$)
\item Empirical validation on story corpora is needed
\end{itemize}

\subsection{Future Directions}

\begin{enumerate}
\item \textbf{Complete formalization}: Remove remaining \texttt{sorry} 
      placeholders
\item \textbf{Corpus analysis}: Test predictions on annotated story 
      databases
\item \textbf{Multi-agent extension}: Model $\skew$ vectors for 
      multiple characters
\item \textbf{Generative applications}: Use geodesics to generate 
      novel narratives
\end{enumerate}

% ============================================================================
% 11. CONCLUSION
% ============================================================================
\section{Conclusion}
\label{sec:conclusion}

We have developed a mathematical framework for narrative structure 
within Recognition Science. Key contributions:

\begin{enumerate}
\item \textbf{Tension = Skew}: Plot tension is identified with 
      ledger skew magnitude, providing a precise, measurable quantity.
      
\item \textbf{Resolution Stability}: The state $\skew = 0$ is proven 
      to be a stable equilibrium, explaining narrative convergence.
      
\item \textbf{Geodesic Hypothesis}: We conjecture that optimal 
      narratives minimize recognition cost, analogous to physical 
      least-action principles.
      
\item \textbf{Concrete Applications}: Analysis of \emph{Hamlet} and 
      the Hero's Journey demonstrates the formalism's applicability.
      
\item \textbf{Partial Verification}: Key theorems are verified in 
      Lean 4; conjectures are clearly marked.
\end{enumerate}

The Physics of Narrative is not (yet) proven in full generality, 
but the framework provides a rigorous foundation for studying 
story structure mathematically. If the geodesic conjecture holds, 
stories are as determined by geometry as planetary orbits---the 
difference is that narrative geodesics trace paths through 
meaning rather than spacetime.

% ============================================================================
% ACKNOWLEDGMENTS
% ============================================================================
\section*{Acknowledgments}

We thank the Recognition Science community for discussions, the 
Lean/Mathlib developers for proof infrastructure, and anonymous 
reviewers for constructive feedback.

% ============================================================================
% REFERENCES
% ============================================================================
\begin{thebibliography}{99}

\bibitem{rs-theory}
Recognition Science Collaboration.
\newblock Recognition Science: Complete Theory Specification.
\newblock Technical Report, 2024.

\bibitem{aristotle-poetics}
Aristotle.
\newblock \emph{Poetics}.
\newblock c.~335 BCE. Trans.~S.~H.~Butcher.

\bibitem{booker2004seven}
C.~Booker.
\newblock \emph{The Seven Basic Plots: Why We Tell Stories}.
\newblock Continuum, 2004.

\bibitem{campbell1949hero}
J.~Campbell.
\newblock \emph{The Hero with a Thousand Faces}.
\newblock Pantheon Books, 1949.

\bibitem{propp1928morphology}
V.~Propp.
\newblock \emph{Morphology of the Folktale}.
\newblock University of Texas Press, 1928/1968.

\bibitem{freytag1863technique}
G.~Freytag.
\newblock \emph{Die Technik des Dramas}.
\newblock 1863.

\bibitem{russell1980circumplex}
J.~A.~Russell.
\newblock A circumplex model of affect.
\newblock \emph{J.~Personality and Social Psychology}, 39(6):1161--1178, 1980.

\bibitem{mehrabian1996pleasure}
A.~Mehrabian.
\newblock Pleasure-arousal-dominance: A general framework for describing 
and measuring individual differences in temperament.
\newblock \emph{Current Psychology}, 14(4):261--292, 1996.

\bibitem{csikszentmihalyi1990flow}
M.~Csikszentmihalyi.
\newblock \emph{Flow: The Psychology of Optimal Experience}.
\newblock Harper \& Row, 1990.

\bibitem{rumelhart1975notes}
D.~E.~Rumelhart.
\newblock Notes on a schema for stories.
\newblock In D.~Bobrow and A.~Collins, eds., \emph{Representation and 
Understanding}, pp.~211--236. Academic Press, 1975.

\bibitem{dena2009transmedia}
C.~Dena.
\newblock \emph{Transmedia Practice: Theorising the Practice of Expressing 
a Fictional World across Distinct Media and Environments}.
\newblock PhD thesis, University of Sydney, 2009.

\bibitem{reagan2016emotional}
A.~J.~Reagan et al.
\newblock The emotional arcs of stories are dominated by six basic shapes.
\newblock \emph{EPJ Data Science}, 5(1):31, 2016.

\bibitem{lean4}
L.~de Moura and S.~Ullrich.
\newblock The Lean 4 Theorem Prover and Programming Language.
\newblock In \emph{CADE}, 2021.

\bibitem{mathlib4}
The mathlib Community.
\newblock The Lean Mathematical Library.
\newblock \url{https://github.com/leanprover-community/mathlib4}, 2024.

\end{thebibliography}

% ============================================================================
% APPENDIX
% ============================================================================
\appendix

\section{Proof Details}
\label{app:proofs}

\subsection{Theorem~\ref{thm:no_catharsis}: Monotonic Arcs Lack Catharsis}

\begin{proof}[Full Proof]
Let $\NarrativeArc = \{b_0, \ldots, b_n\}$ be an arc with 
$\tension(b_i) < \tension(b_{i+1})$ for all $0 \leq i < n$.

Suppose for contradiction that catharsis occurs between beats 
$b_j$ and $b_{j+1}$ for some $j$. By Definition~\ref{def:catharsis}:
\[
\tension(b_j) - \tension(b_{j+1}) > \tension(b_j)/2 > 0
\]
This implies $\tension(b_{j+1}) < \tension(b_j)$, contradicting 
the strict monotonicity assumption.
\end{proof}

\subsection{Theorem~\ref{thm:attractor}: Resolution Attractor}

\begin{proof}[Proof from Axiom~\ref{ax:recognition}]
Let $s_0$ be an admissible initial state with $\skew_0 \neq 0$. 
Define $s_k = \Rhat^k(s_0)$.

By Axiom~\ref{ax:recognition}(1): $\Jcost(s_{k+1}) \leq \Jcost(s_k)$.

The sequence $\{\Jcost(s_k)\}$ is non-increasing and bounded below by 0. 
Hence it converges: $\Jcost(s_k) \to J^*$ for some $J^* \geq 0$.

By Proposition~\ref{prop:Jcost_properties}, $\Jcost$ achieves its 
minimum $0$ uniquely at the unit state (which has $\skew = 0$).

By Axiom~\ref{ax:recognition}(3), states with $\skew = 0$ are fixed 
points. If the sequence converges to a fixed point, that point must 
have $\skew = 0$.

We have not proven that $s_k$ converges (only that $\Jcost(s_k)$ does), 
so Theorem~\ref{thm:attractor} as stated requires the additional 
assumption that cost convergence implies state convergence---a property 
that holds if the dynamics are gradient-like.
\end{proof}

\section{Metric Derivation Sketch}
\label{app:metric}

The story metric coefficients arise from dimensional analysis in RS:

\begin{enumerate}
\item $\skew$ has dimension of ``moral charge'' (primary)
\item $\energy$ has dimension of ``capacity'' (secondary)
\item $Z$ (pattern) has dimension of ``structure'' (tertiary)
\end{enumerate}

The golden ratio $\golden$ provides the natural scale factor between 
levels. Setting the primary coefficient to 1 and scaling down by 
$\golden$ at each level gives $(1, 1/\golden, 1/\golden^2)$.

Alternative derivation: The Hessian of $\Jcost$ at the minimum 
determines a natural metric. For $\Jcost(x) = \frac{1}{2}(x + 1/x) - 1$:
\[
\Jcost''(1) = 1
\]
Extending to the full MoralState requires specifying how $\Jcost$ 
depends on $(\skew, \energy, Z)$, which introduces the $\golden$ 
weighting through the RS forcing chain structure.

\end{document}
