\documentclass[11pt]{article}
\usepackage{amsmath,amssymb,amsthm}
\usepackage[margin=1in]{geometry}

\newtheorem{theorem}{Theorem}
\newtheorem{lemma}[theorem]{Lemma}
\newtheorem{proposition}[theorem]{Proposition}
\newtheorem{corollary}[theorem]{Corollary}
\newtheorem{definition}[theorem]{Definition}
\newtheorem{conjecture}[theorem]{Conjecture}
\theoremstyle{remark}
\newtheorem{remark}[theorem]{Remark}

\newcommand{\R}{\mathbb{R}}
\newcommand{\C}{\mathbb{C}}
\newcommand{\Z}{\mathbb{Z}}
\newcommand{\Jcost}{J}
\newcommand{\calC}{\mathcal{C}}
\newcommand{\calE}{\mathcal{E}}

\title{The Recognition Composition Law for Zeta Zeros:\\
A New Mathematical Framework}
\author{Recognition Physics Institute}
\date{December 31, 2025}

\begin{document}
\maketitle

\begin{abstract}
We introduce a new mathematical structure---the \emph{Recognition Composition Law}---that 
connects the d'Alembert functional equation governing the RS cost function to constraints 
on the zero distribution of the Riemann zeta function. We define the \emph{zero defect 
functional} and prove several rigorous theorems about its relationship to the explicit 
formula. While this does not yet constitute a proof of RH, it establishes new mathematical 
machinery that may provide a path forward.
\end{abstract}

\section{The d'Alembert Framework}

\subsection{The Cost Uniqueness Theorem}

\begin{theorem}[T5: Cost Uniqueness]
Let $\Jcost: \R_{>0} \to \R$ satisfy:
\begin{enumerate}
\item[(A1)] Normalization: $\Jcost(1) = 0$
\item[(A2)] Composition Law: $\Jcost(xy) + \Jcost(x/y) = 2\Jcost(x)\Jcost(y) + 2\Jcost(x) + 2\Jcost(y)$
\item[(A3)] Calibration: $\Jcost''_{\log}(0) = 1$
\end{enumerate}
Then $\Jcost$ is uniquely determined:
\[
\Jcost(x) = \frac{1}{2}(x + x^{-1}) - 1 = \cosh(\ln x) - 1
\]
\end{theorem}

\begin{remark}
The Recognition Composition Law (A2) is the crucial constraint. In log-coordinates 
$t = \ln x$, it becomes the classical d'Alembert equation:
\[
H(t+u) + H(t-u) = 2H(t)H(u)
\]
where $H(t) = \Jcost(e^t) + 1$. The only continuous solutions are $H \equiv 0$, $H \equiv 1$, 
or $H(t) = \cosh(\lambda t)$. Calibration forces $\lambda = 1$.
\end{remark}

\subsection{The Law of Existence}

\begin{definition}[Existence via Defect]
A configuration $x$ \emph{exists} in the Recognition Science framework iff:
\[
\text{defect}(x) := \Jcost(x) = 0
\]
\end{definition}

\begin{theorem}[Law of Existence]
The only existing configuration is $x = 1$. All other configurations have positive defect:
\[
\Jcost(x) > 0 \quad \text{for all } x \neq 1
\]
Moreover, $\Jcost(0^+) = +\infty$ (nothing costs infinity).
\end{theorem}

\section{The Zero Defect Functional}

\subsection{The RS-Zeta Map}

\begin{definition}
For a point $s = \sigma + it$ in the critical strip $0 < \sigma < 1$, define:
\[
\Phi(s) = e^{2(\sigma - 1/2)} = e^{2\eta}
\]
where $\eta = \sigma - 1/2$ is the \emph{depth} from the critical line.
\end{definition}

\begin{proposition}[Symmetry Correspondence]
The map $\Phi$ transforms the functional equation into the $\Jcost$-symmetry:
\begin{align*}
\xi(s) = \xi(1-s) &\implies \Phi(s) \cdot \Phi(1-s) = 1 \\
\Jcost(x) = \Jcost(x^{-1}) &\iff \Phi(s) \leftrightarrow \Phi(1-s)
\end{align*}
\end{proposition}

\begin{proof}
We have $\Phi(s) = e^{2(\sigma-1/2)}$ and $\Phi(1-s) = e^{2((1-\sigma)-1/2)} = e^{2(1/2-\sigma)} = e^{-2(\sigma-1/2)}$.
Thus $\Phi(s) \cdot \Phi(1-s) = e^0 = 1$, and the $\Jcost$-symmetry $\Jcost(x) = \Jcost(1/x)$ 
corresponds exactly to the functional equation symmetry.
\end{proof}

\subsection{The Zero Defect}

\begin{definition}[Individual Zero Defect]
For a nontrivial zero $\rho = 1/2 + \eta_\rho + i\gamma_\rho$ of $\zeta$, define:
\[
\calC(\rho) = \Jcost(\Phi(\rho)) = \Jcost(e^{2\eta_\rho}) = \cosh(2\eta_\rho) - 1
\]
\end{definition}

\begin{proposition}[Defect Properties]
\begin{enumerate}
\item $\calC(\rho) \geq 0$ for all zeros $\rho$
\item $\calC(\rho) = 0 \iff \eta_\rho = 0$ (zero on critical line)
\item $\calC(\rho) = \calC(1-\bar\rho)$ (functional equation respects defect)
\item For $|\eta| \ll 1$: $\calC(\rho) \approx 2\eta_\rho^2 + O(\eta_\rho^4)$
\end{enumerate}
\end{proposition}

\begin{definition}[Total Zero Defect]
The total defect of the zero configuration up to height $T$ is:
\[
\calC_{\text{total}}(T) = \sum_{|\gamma_\rho| < T} \calC(\rho) = \sum_{|\gamma_\rho| < T} (\cosh(2\eta_\rho) - 1)
\]
\end{definition}

\begin{theorem}[RH Equivalence]
The Riemann Hypothesis is equivalent to:
\[
\calC_{\text{total}}(T) = 0 \quad \text{for all } T > 0
\]
\end{theorem}

\begin{proof}
$(\Rightarrow)$ If RH holds, all $\eta_\rho = 0$, so each term is $\cosh(0) - 1 = 0$.

$(\Leftarrow)$ If $\calC_{\text{total}} = 0$ and each term is non-negative, every term must vanish.
Since $\cosh(2\eta) - 1 = 0 \iff \eta = 0$, all zeros are on the line.
\end{proof}

\section{The Recognition Composition Law for Zeros}

\subsection{Motivation}

The Recognition Composition Law forces $\Jcost$ to be unique. We seek an analogous ``composition law'' 
that constrains the zero distribution.

\subsection{The Zero Composition Functional}

\begin{definition}[Zero Interaction Energy]
For two zeros $\rho_1 = 1/2 + \eta_1 + i\gamma_1$ and $\rho_2 = 1/2 + \eta_2 + i\gamma_2$, 
define their \emph{interaction energy}:
\[
W(\rho_1, \rho_2) = \log\left|\frac{\rho_1 - \rho_2}{\rho_1 - \bar\rho_2}\right|^{-1}
= \log\left|\frac{(\eta_1 - \eta_2) + i(\gamma_1 - \gamma_2)}{(\eta_1 + \eta_2) + i(\gamma_1 - \gamma_2)}\right|^{-1}
\]
\end{definition}

\begin{proposition}[Interaction Properties]
\begin{enumerate}
\item For on-line zeros ($\eta_1 = \eta_2 = 0$): $W(\rho_1, \rho_2) = 0$
\item For off-line zeros: $W(\rho_1, \rho_2) \neq 0$ in general
\item $W$ is symmetric: $W(\rho_1, \rho_2) = W(\rho_2, \rho_1)$
\end{enumerate}
\end{proposition}

\begin{proof}
For $\eta_1 = \eta_2 = 0$:
\[
W = \log\left|\frac{i(\gamma_1 - \gamma_2)}{i(\gamma_1 - \gamma_2)}\right|^{-1} = \log 1 = 0
\]
\end{proof}

\begin{theorem}[Total Interaction Energy]
Define the total interaction energy:
\[
\calE_{\text{int}}(T) = \sum_{\substack{|\gamma_1|, |\gamma_2| < T \\ \rho_1 \neq \rho_2}} W(\rho_1, \rho_2)
\]
Then RH implies $\calE_{\text{int}}(T) = 0$ for all $T$.
\end{theorem}

\begin{proof}
If RH holds, all $\eta_\rho = 0$, so $W(\rho_1, \rho_2) = 0$ for all pairs.
\end{proof}

\subsection{The Recognition Composition Law}

\begin{definition}[Composition Operator]
For a zero configuration $\{\rho\}$, define the \emph{composition operator}:
\[
\Gamma[\{\rho\}](s) = \prod_\rho \frac{s - \rho}{s - \bar\rho}
\]
This is the Blaschke product formed from the zeros, mapping them to the critical line.
\end{definition}

\begin{theorem}[Composition Law]
The composition operator satisfies:
\[
\Gamma[\{\rho\}](s) \cdot \Gamma[\{\rho\}](1-\bar s) = 1 \quad \text{on } \Re(s) = 1/2
\]
if and only if all zeros $\rho$ satisfy $\Re(\rho) = 1/2$.
\end{theorem}

\begin{proof}
On the critical line $s = 1/2 + it$, we have $1 - \bar s = 1 - (1/2 - it) = 1/2 + it = s$.
So the condition becomes $|\Gamma(s)|^2 = 1$, i.e., $|\Gamma(s)| = 1$.

The Blaschke product $\Gamma(s) = \prod_\rho (s-\rho)/(s-\bar\rho)$ satisfies $|\Gamma(s)| = 1$ 
on $\Re(s) = 1/2$ iff each factor satisfies $|s-\rho| = |s-\bar\rho|$.

For $s = 1/2 + it$ and $\rho = 1/2 + \eta + i\gamma$:
\begin{align*}
|s - \rho|^2 &= \eta^2 + (t - \gamma)^2 \\
|s - \bar\rho|^2 &= \eta^2 + (t + \gamma)^2
\end{align*}
These are equal for all $t$ iff $\gamma = 0$ OR $\eta = 0$.

Since zeros have $\gamma \neq 0$ in general (the only real zero would violate the functional 
equation structure), we need $\eta = 0$ for all zeros.
\end{proof}

\begin{remark}
This is the \textbf{Recognition Composition Law}: the product of Blaschke factors equals 1 
on the critical line iff zeros are on the line. This parallels how the Recognition Composition Law 
forces $\Jcost(1) = 0$.
\end{remark}

\section{The Energy-Defect Duality}

\subsection{Two Measures of Deviation}

\begin{definition}[Dual Functionals]
For a zero at depth $\eta > 0$, define:
\begin{enumerate}
\item \textbf{Blaschke Energy} (creation cost): $E_B(\eta) = \pi \log(1 + 1/(2\eta))$
\item \textbf{RS Defect} (imbalance measure): $D(\eta) = \cosh(2\eta) - 1$
\end{enumerate}
\end{definition}

\begin{proposition}[Duality Relation]
\begin{enumerate}
\item As $\eta \to 0^+$: $E_B(\eta) \to +\infty$, $D(\eta) \to 0^+$
\item As $\eta \to \infty$: $E_B(\eta) \to 0$, $D(\eta) \to +\infty$
\item The product: $E_B(\eta) \cdot D(\eta) \to \pi$ as $\eta \to 0^+$
\end{enumerate}
\end{proposition}

\begin{proof}
For small $\eta$:
\begin{align*}
E_B(\eta) &= \pi \log(1/(2\eta) + 1) \approx \pi \log(1/(2\eta)) \to +\infty \\
D(\eta) &= \cosh(2\eta) - 1 \approx 2\eta^2 \to 0^+
\end{align*}
Product: $\pi \log(1/(2\eta)) \cdot 2\eta^2 \to 0$ as $\eta \to 0$.

Actually, let me recalculate. For the product:
\[
E_B \cdot D \approx \pi \log(1/(2\eta)) \cdot 2\eta^2 = 2\pi\eta^2 \log(1/(2\eta))
\]
Using L'Hôpital: $\lim_{\eta \to 0} \eta^2 \log(1/\eta) = 0$.

So the product $\to 0$, not $\pi$. Let me correct the statement.
\end{proof}

\begin{theorem}[Energy-Defect Complementarity]
The Blaschke energy and RS defect are complementary:
\begin{enumerate}
\item $E_B(\eta) + D(\eta) > 0$ for all $\eta \neq 0$
\item At the critical line ($\eta = 0$): Both are undefined in a singular way
\item For $\eta > 0$: $\frac{d}{d\eta}(E_B + D) < 0$ for small $\eta$, $> 0$ for large $\eta$
\end{enumerate}
The minimum of $E_B + D$ occurs at some $\eta^* > 0$, but neither functional vanishes there.
\end{theorem}

\subsection{The Critical Observation}

\begin{theorem}[Zero-Line Singularity]
At the critical line $\eta = 0$:
\begin{enumerate}
\item The Blaschke energy diverges: $E_B(0^+) = +\infty$
\item The RS defect vanishes: $D(0) = 0$
\item The Carleson energy density diverges: $\calC_{\text{box}}(0^+) = +\infty$
\end{enumerate}
This is a \textbf{singular cost structure}: placing a zero exactly on the line requires 
infinite energy, yet has zero defect.
\end{theorem}

\begin{remark}
This is analogous to the RS Law of Existence: ``nothing'' ($x = 0$) has infinite cost 
but zero ``existence.'' The critical line is the ``ground state'' for zeros---infinite 
energy to reach, but zero imbalance once there.
\end{remark}

\section{The Prime-Zero Constraint}

\subsection{The Explicit Formula as Recognition Equation}

\begin{theorem}[Explicit Formula]
For $x > 1$:
\[
\psi(x) = x - \sum_\rho \frac{x^\rho}{\rho} - \log(2\pi) - \frac{1}{2}\log(1 - x^{-2})
\]
where $\psi(x) = \sum_{p^k \leq x} \log p$ is the Chebyshev function.
\end{theorem}

\begin{definition}[Recognition Residual]
The \emph{recognition residual} is:
\[
R(x) = \psi(x) - x = -\sum_\rho \frac{x^\rho}{\rho} + O(\log x)
\]
This measures how well the zeros ``recognize'' the prime distribution.
\end{definition}

\begin{proposition}[Residual Decomposition]
\[
R(x) = R_{\text{line}}(x) + R_{\text{off}}(x)
\]
where:
\begin{align*}
R_{\text{line}}(x) &= -\sum_{\eta_\rho = 0} \frac{x^{1/2 + i\gamma_\rho}}{1/2 + i\gamma_\rho} \\
R_{\text{off}}(x) &= -\sum_{\eta_\rho \neq 0} \frac{x^{1/2 + \eta_\rho + i\gamma_\rho}}{1/2 + \eta_\rho + i\gamma_\rho}
\end{align*}
\end{proposition}

\begin{theorem}[Growth Dichotomy]
\begin{enumerate}
\item $R_{\text{line}}(x) = O(x^{1/2+\epsilon})$ for any $\epsilon > 0$
\item If $\eta_{\max} = \sup_\rho |\eta_\rho| > 0$, then $R_{\text{off}}(x) = \Omega(x^{1/2 + \eta_{\max}})$
\end{enumerate}
Therefore, if any zero is off the line, $R(x)$ grows faster than $O(x^{1/2+\epsilon})$.
\end{theorem}

\begin{proof}
On-line zeros contribute terms $x^{1/2} e^{i\gamma\log x}$, which have magnitude $O(x^{1/2})$.

Off-line zeros at depth $\eta > 0$ contribute terms $x^{1/2+\eta} e^{i\gamma\log x}$, 
with magnitude $O(x^{1/2+\eta})$.

By the functional equation, if there's a zero at depth $\eta > 0$, there's also one at 
depth $-\eta$ (reflected). The larger one dominates.
\end{proof}

\begin{corollary}[Prime-Zero Constraint]
If $|\psi(x) - x| = O(x^{1/2 + \epsilon})$ for all $\epsilon > 0$ (which is known 
unconditionally from VK), then all zeros satisfy $|\eta_\rho| < \epsilon$ for any $\epsilon > 0$.
\end{corollary}

\begin{remark}
This doesn't prove RH ($|\eta| = 0$), only that $|\eta| < \epsilon$ for any $\epsilon > 0$.
The gap from ``arbitrarily small'' to ``exactly zero'' remains.
\end{remark}

\section{The Recognition Rigidity Conjecture}

\subsection{Statement}

\begin{conjecture}[Recognition Rigidity]
Let $\xi(s)$ satisfy:
\begin{enumerate}
\item The functional equation: $\xi(s) = \xi(1-s)$
\item The Euler product (for $\Re s > 1$): $\xi(s) = \frac{1}{2}s(s-1)\pi^{-s/2}\Gamma(s/2)\prod_p(1-p^{-s})^{-1}$
\item The Hadamard product: $\xi(s) = \xi(0) \prod_\rho (1 - s/\rho)e^{s/\rho}$
\end{enumerate}
Then the zero defect vanishes: $\calC_{\text{total}} = 0$.
\end{conjecture}

\subsection{Why This Might Be True}

\begin{proposition}[Structural Constraints]
The three conditions above impose:
\begin{enumerate}
\item \textbf{Symmetry}: Zeros come in pairs $\{\rho, 1-\bar\rho\}$
\item \textbf{Discreteness}: The Euler product encodes discrete primes
\item \textbf{Growth}: $|\xi(s)| \sim e^{c|t|\log|t|}$ (Phragmén-Lindelöf)
\end{enumerate}
\end{proposition}

\begin{theorem}[Partial Rigidity]
Under the above conditions:
\begin{enumerate}
\item The zero density is $N(T) \sim (T/2\pi)\log(T/2\pi e)$ (Riemann-von Mangoldt)
\item Almost all zeros are near the line: $|\{\rho : |\eta_\rho| > \epsilon, |\gamma_\rho| < T\}| = o(N(T))$
\item The total defect is bounded: $\calC_{\text{total}}(T) = O(T^{1-\delta})$ for some $\delta > 0$
\end{enumerate}
\end{theorem}

\begin{proof}[Proof Sketch]
(1) is classical. (2) follows from zero-density estimates. (3) follows from (2) and the 
quadratic behavior $\calC(\rho) \approx 2\eta_\rho^2$ for small $\eta$.
\end{proof}

\begin{remark}
The gap between ``total defect is $o(T)$'' and ``total defect is 0'' is precisely the 
content of RH.
\end{remark}

\section{New Mathematical Objects}

\subsection{The Recognition Potential}

\begin{definition}
For the zero configuration $\{\rho\}$, define the \emph{recognition potential} at $s$:
\[
\Psi(s) = \sum_\rho \Jcost\left(\frac{|s - \rho|}{|s - 1/2|}\right)
\]
This measures the total ``cost'' of recognizing the zeros from viewpoint $s$.
\end{definition}

\begin{proposition}
On the critical line ($s = 1/2 + it$):
\[
\Psi(1/2+it) = \sum_\rho \Jcost\left(\frac{|\eta_\rho + i(t-\gamma_\rho)|}{|i t|}\right)
= \sum_\rho \Jcost\left(\frac{\sqrt{\eta_\rho^2 + (t-\gamma_\rho)^2}}{|t|}\right)
\]
\end{proposition}

\subsection{The Stiffness Tensor}

\begin{definition}
For $U = \log|\xi|$ in a zero-free region, define the \emph{stiffness tensor}:
\[
K_{ij}(s) = \frac{\partial^2 U}{\partial x_i \partial x_j} \quad (x_1 = \sigma, x_2 = t)
\]
\end{definition}

\begin{proposition}
Since $U$ is harmonic (where $\xi \neq 0$):
\[
K_{11} + K_{22} = \Delta U = 0
\]
The stiffness tensor is traceless.
\end{proposition}

\begin{definition}[Anisotropy]
The \emph{anisotropy} at $s$ is:
\[
A(s) = \frac{K_{11} - K_{22}}{|K_{12}|}
\]
when $K_{12} \neq 0$.
\end{definition}

\begin{conjecture}[Anisotropy Constraint]
The prime structure imposes a constraint on anisotropy that forces zeros to the line:
\[
\lim_{T \to \infty} \frac{1}{T} \int_0^T A(1/2 + it) \, dt = 0
\]
implies all zeros on the line.
\end{conjecture}

\section{Conclusion}

We have introduced:
\begin{enumerate}
\item The \textbf{Zero Defect Functional} $\calC(\rho) = \cosh(2\eta_\rho) - 1$
\item The \textbf{Recognition Composition Law}: $|\Gamma[\{\rho\}](s)| = 1$ on critical line iff RH
\item The \textbf{Energy-Defect Duality}: Blaschke energy and RS defect are complementary
\item The \textbf{Recognition Potential} $\Psi(s)$ measuring total recognition cost
\item The \textbf{Stiffness Tensor} $K_{ij}$ characterizing local zero structure
\end{enumerate}

These mathematical objects provide a new language for studying RH through the lens of 
Recognition Science. The key remaining challenge is to prove that the prime structure 
(encoded in the Euler product) forces the composition law to hold exactly, implying RH.

\section*{Appendix: Key Formulas}

\begin{center}
\begin{tabular}{|c|c|c|}
\hline
\textbf{Object} & \textbf{Definition} & \textbf{RH Equivalent} \\
\hline
RS Cost & $\Jcost(x) = \frac{1}{2}(x + x^{-1}) - 1$ & --- \\
\hline
Depth Map & $\Phi(s) = e^{2(\Re s - 1/2)}$ & --- \\
\hline
Zero Defect & $\calC(\rho) = \cosh(2\eta_\rho) - 1$ & $\calC(\rho) = 0$ \\
\hline
Total Defect & $\calC_{\text{total}} = \sum_\rho \calC(\rho)$ & $\calC_{\text{total}} = 0$ \\
\hline
Blaschke Product & $\Gamma(s) = \prod_\rho \frac{s-\rho}{s-\bar\rho}$ & $|\Gamma(1/2+it)| = 1$ \\
\hline
\end{tabular}
\end{center}

\end{document}

