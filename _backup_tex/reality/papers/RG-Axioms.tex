%  LaTeX support: latex@mdpi.com 
%  For support, please attach all files needed for compiling as well as the log file, and specify your operating system, LaTeX version, and LaTeX editor.

%=================================================================
\documentclass[journal,article,submit,pdftex,moreauthors]{Definitions/mdpi}  
%\documentclass[preprints,article,submit,pdftex,moreauthors]{Definitions/mdpi} 
% For posting an early version of this manuscript as a preprint, you may use "preprints" as the journal. Changing "submit" to "accept" before posting will remove line numbers.

\newtheoremstyle{axiomstyle}
  {}{}                 % razmaci
  {\itshape}           % telo
  {}                   % indent
  {\bfseries}          % naslov
  {.}                  % tačka iza broja
  {0.5em}              % razmak
  {\thmname{#1}~\thmnumber{#2}\thmnote{ (#3)}} % OVO PRIKAZUJE (RG0...)
  
\theoremstyle{axiomstyle}
\newtheorem{axiom}{Axiom}
%\newtheorem{remark}{Remark}[section]
%--------------------
% Class Options:
%--------------------
%----------
% journal
%----------
% Choose between the following MDPI journals:
% accountaudit, acoustics, actuators, addictions, adhesives, adolescents, aerobiology, aerospace, agriculture, agriengineering, agrochemicals, agronomy, ai, aichem, aieng, aimater, aimed, aipa, air, aisens, algorithms, allergies, alloys, amh, analog, analytica, analytics, anatomia, anesthres, animals, antibiotics, antibodies, antioxidants, applbiosci, appliedchem, appliedmath, appliedphys, applmech, applmicrobiol, applnano, applsci, aquacj, architecture, arm, arthropoda, asc, asi, astronautics, astronomy, atmosphere, atoms, audiolres, automation, axioms, bacteria, batteries, bdcc, beverages, biochem, bioengineering, biologics, biology, biomass, biomechanics, biomed, biomedicines, biomedinformatics, biomimetics, biomolecules, biophysica, bioresourbioprod, biosensors, biosphere, biotech, birds, blockchains, bloods, blsf, brainsci, breath, buildings, cancers, carbon, cardio, cardiogenetics, cardiovascmed, catalysts, cells, ceramics, challenges, chemengineering, chemistry, chemosensors, chemproc, children, chips, cimb, civileng, cleantechnol, climate, clinbioenerg, clinpract, clockssleep, cmd, cmtr, coasts, coatings, colloids, colorants, commodities, complexities, complications, compounds, computation, computers, condensedmatter, conservation, constrmater, cosmetics, covid, crops, cryo, cryptography, crystals, csmf, ctn, culture, curroncol, cyber, dairy, data, ddc, dentistry, dermato, dermatopathology, designs, devices, dhi, diabetology, diagnostics, dietetics, digital, disabilities, diseases, diversity, dna, drones, dynamics, earth, ebj, ecm, ecologies, edm, eesp, electricity, electrochem, electronicmat, electronics, encyclopedia, endocrines, energies, eng, engproc, entomology, entropy, environments, environremediat, epidemiologia, epigenomes, esa, est, fermentation, fibers, fintech, fire, fishes, fluids, foods, forecasting, forensicsci, forests, fossstud, foundations, fractalfract, fuels, future, futureinternet, futurepharmacol, futurephys, futuretransp, galaxies, gases, gastroent, gastrointestdisord, gastronomy, gels, genes, geographies, geohazards, geomatics, geometry, geosciences, geotechnics, geriatrics, germs, glacies, grasses, green, greenhealth, gucdd, hardware, hazardousmatters, healthcare, hearts, hemato, hematolrep, hep, heritage, higheredu, highthroughput, horticulturae, hospitals, hydrobiology, hydrogen, hydrology, hydropower, hygiene, idr, iic, ijem, ijerph, ijgi, ijmd, ijms, ijns, ijom, ijpb, ijt, ijtm, ijtpp, ime, immuno, informatics, information, infrastructures, inorganics, insects, instruments, inventions, iot, j, jaestheticmed, jal, jcdd, jcm, jcp, jcrm, jcs, jcto, jdad, jdb, jdream, jemr, jeta, jfb, jfmk, jgbg, jgg, jimaging, jlpea, jmahp, jmmp, jmms, jmp, jmse, jne, jnt, jof, joi, joitmc, joma, jop, joptm, jor, jox, jpbi, jphytomed, jpm, jsan, jtaer, jvd, jzbg, kidneydial, kinasesphosphatases, knowledge, labmed, laboratories, lae, land, life, lights, limnolrev, lipidology, liquids, livers, logics, logistics, lubricants, lymphatics, machines, macromol, magnetism, magnetochemistry, make, marinedrugs, materials, materproc, mathematics, mca, measurements, medicina, medicines, medsci, membranes, merits, metabolites, metals, meteorology, methane, metrics, metrology, micro, microarrays, microbiolres, microelectronics, micromachines, microorganisms, microplastics, microwave, minerals, mining, mmphys, modelling, molbank, molecules, mps, msf, mti, multimedia, muscles, nanoenergyadv, nanomanufacturing, nanomaterials, ncrna, ndt, network, neuroglia, neuroimaging, neurolint, neurosci, nitrogen, notspecified, nursrep, nutraceuticals, nutrients, obesities, occuphealth, oceans, ohbm, onco, optics, oral, organics, organoids, osteology, oxygen, pandemics, parasites, parasitologia, particles, pathogens, pathophysiology, pediatrrep, pets, pharmaceuticals, pharmaceutics, pharmacoepidemiology, pharmacy, philosophies, photochem, photonics, phycology, physchem, physics, physiologia, plants, plasma, platforms, pollutants, polymers, polysaccharides, populations, poultry, powders, precipitation, precisoncol, preprints, proceedings, processes, prosthesis, proteomes, psf, psychiatryint, psychoactives, purification, quantumrep, quaternary, qubs, radiation, rdt, reactions, realestate, receptors, recycling, regeneration, remotesensing, reports, reprodmed, resources, rheumato, rjpm, robotics, rsee, ruminants, safety, sci, scipharm, sclerosis, seeds, sensors, separations, sexes, shi, signals, sinusitis, siuj, skins, smartcities, sna, societies, software, soilsystems, solar, solids, spectroscj, sports, standards, stats, std, stratsediment, stresses, surfaces, surgeries, suschem, sustainability, symmetry, synbio, systems, tae, targets, taxonomy, technologies, telecom, test, textiles, thalassrep, therapeutics, thermo, timespace, tomography, toxics, toxins, tph, transplantology, transportation, traumacare, traumas, tri, tropicalmed, universe, urbansci, uro, vaccines, vehicles, venereology, vetsci, vibration, virtualworlds, viruses, vision, waste, water, welding, wem, wevj, wild, wind, women, world, zoonoticdis

%---------
% article
%---------
% The default type of manuscript is "article", but can be replaced by: 
% abstract, addendum, article, benchmark, book, bookreview, briefcommunication, briefreport, casereport, changes, clinicopathologicalchallenge, comment, commentary, communication, conceptpaper, conferenceproceedings, correction, conferencereport, creative, datadescriptor, discussion, entry, expressionofconcern, extendedabstract, editorial, essay, erratum, fieldguide, hypothesis, interestingimages, letter, meetingreport, monograph, newbookreceived, obituary, opinion, proceedingpaper, projectreport, reply, retraction, review, perspective, protocol, shortnote, studyprotocol, supfile, systematicreview, technicalnote, viewpoint, guidelines, registeredreport, tutorial,  giantsinurology, urologyaroundtheworld
% supfile = supplementary materials

%----------
% submit
%----------
% The class option "submit" will be changed to "accept" by the Editorial Office when the paper is accepted. This will only make changes to the frontpage (e.g., the logo of the journal will get visible), the headings, and the copyright information. Also, line numbering will be removed. Journal info and pagination for accepted papers will also be assigned by the Editorial Office.

%------------------
% moreauthors
%------------------
% If there is only one author the class option oneauthor should be used. Otherwise use the class option moreauthors.

%---------
% pdftex
%---------
% The option pdftex is for use with pdfLaTeX. Remove "pdftex" for (1) compiling with LaTeX & dvi2pdf (if eps figures are used) or for (2) compiling with XeLaTeX.

%=================================================================
% MDPI internal commands - do not modify
\firstpage{1} 
\makeatletter 
\setcounter{page}{\@firstpage} 
\makeatother
\pubvolume{1}
\issuenum{1}
\articlenumber{0}
\pubyear{2026}
\copyrightyear{2025}
%\externaleditor{Firstname Lastname} % More than 1 editor, please add `` and '' before the last editor name
\datereceived{ } 
\daterevised{ } % Comment out if no revised date
\dateaccepted{ } 
\datepublished{ } 
%\datecorrected{} % For corrected papers: "Corrected: XXX" date in the original paper.
%\dateretracted{} % For retracted papers: "Retracted: XXX" date in the original paper.
%\doinum{} % Used for some special journals, like molbank
%\pdfoutput=1 % Uncommented for upload to arXiv.org
%\CorrStatement{yes}  % For updates
%\longauthorlist{yes} % For many authors that exceed the left citation part
%\IsAssociation{yes} % For association journals

%=================================================================
% Add packages and commands here. The following packages are loaded in our class file: fontenc, inputenc, calc, indentfirst, fancyhdr, graphicx, epstopdf, lastpage, ifthen, float, amsmath, amssymb, lineno, setspace, enumitem, mathpazo, booktabs, titlesec, etoolbox, tabto, xcolor, colortbl, soul, multirow, microtype, tikz, totcount, changepage, attrib, upgreek, array, tabularx, pbox, ragged2e, tocloft, marginnote, marginfix, enotez, amsthm, natbib, hyperref, cleveref, scrextend, url, geometry, newfloat, caption, draftwatermark, seqsplit
% cleveref: load \crefname definitions after \begin{document}

%=================================================================
% Please use the following mathematics environments: Theorem, Lemma, Corollary, Proposition, Characterization, Property, Problem, Example, ExamplesandDefinitions, Hypothesis, Remark, Definition, Notation, Assumption
%% For proofs, please use the proof environment (the amsthm package is loaded by the MDPI class).

%=================================================================
% Full title of the paper (Capitalized)
\Title{Recognition Geometry}

% Author Orchid ID: enter ID or remove command
\newcommand{\orcidauthorA}{https://orcid.org/0000-0002-0318-1092} % Add \orcidA{} behind the author's name
\newcommand{\orcidauthorB}{https://orcid.org/0000-0001-7212-4713} % Add \orcidB{} behind the author's name



% Authors, for the paper (add full first names)
\Author{Jonathan Washburn$^{1}$, Milan Zlatanovi\'c $^{2,*}$\orcidA{} and Elshad Allahyarov$^{3}$\orcidB{} }

%\longauthorlist{yes}

% MDPI internal command: Authors, for metadata in PDF
\AuthorNames{Jonathan Washburn, Milan Zlatanovi\'c and Elshad Allahyarov}

% Affiliations / Addresses (Add [1] after \address if there is only one affiliation.)
\address{%
$^{1}$ \quad Recognition Physics Institute Austin, Texas, USA; jon@recognitionphysics.org\\
$^{2}$ \quad Faculty of Science and Mathematics, University of Ni\v s,  Serbia; zlatmilan@yahoo.com\\
$^{3}$ \quad Recognition Physics Institute, Austin, TX, USA;  
Institut für Theoretische Physik II: Weiche Materie, Heinrich-Heine-Universität Düsseldorf, Germany;  
Theoretical Department, Joint Institute for High Temperatures, Russian Academy of Sciences (IVTAN), Moscow, Russia;  
Department of Physics, Case Western Reserve University, Cleveland, OH, USA; elshad.allakhyarov@case.edu}

% Contact information of the corresponding author
\corres{Correspondence: zlatmilan@yahoo.com}

% Current address and/or shared authorship
%\firstnote{Current address: Affiliation.}  
% Current address should not be the same as any items in the Affiliation section.

%\secondnote{These authors contributed equally to this work.}
% The commands \thirdnote{} till \eighthnote{} are available for further notes.

%\simplesumm{} % Simple summary

%\conference{} % An extended version of a conference paper

% Abstract (Do not insert blank lines, i.e. \\) 
\abstract{  We introduce Recognition Geometry (RG), an axiomatic framework in which
geometric structure is not assumed a priori, but it is derived.
The starting point of the theory is a configuration space together with
recognizers that map configurations to observable events.
Observational indistinguishability induces an equivalence relation,
and the observable space is obtained as a recognition quotient.
Locality is introduced through a neighborhood system, without assuming
any metric or topological structure.
A finite local resolution axiom RG3 formalizes the fact that any observer can
distinguish only finitely many outcomes in a local region. 
Comparative recognizers allow us to define order-type
relations based on operational comparison.
Under additional assumptions, quantitative notions of distinguishability
can be introduced in the form of recognition distances, defined as
pseudometrics. 
Several examples are provided, including threshold recognizers on
$\mathbb{R}^n$, discrete lattice models, quantum spin measurements, and an
example motivated by Recognition Science.
In the last part, we discuss the composition of recognizers.
A significant part of the axiomatic  framework  and the main constructions have been formalized in the Lean~4 proof assistant.}

% Keywords
\keyword{Recognition geometry, configuration spaces, event spaces, recognizer, quotient spaces,  resolution cells,  recognition distances} 

% The fields PACS, MSC, and JEL may be left empty or commented out if not applicable
%\PACS{J0101}
\MSC{51A05, 54A05, 03B30, 81P15, 18B99}
%\JEL{}

%%%%%%%%%%%%%%%%%%%%%%%%%%%%%%%%%%%%%%%%%%
% Only for the journal Diversity
%\LSID{\url{http://}}

%%%%%%%%%%%%%%%%%%%%%%%%%%%%%%%%%%%%%%%%%%
% Only for the journal Applied Sciences
%\featuredapplication{Authors are encouraged to provide a concise description of the specific application or a potential application of the work. This section is not mandatory.}
%%%%%%%%%%%%%%%%%%%%%%%%%%%%%%%%%%%%%%%%%%

%%%%%%%%%%%%%%%%%%%%%%%%%%%%%%%%%%%%%%%%%%
% Only for the journal Data
%\dataset{DOI number or link to the deposited data set if the data set is published separately. If the data set shall be published as a supplement to this paper, this field will be filled by the journal editors. In this case, please submit the data set as a supplement.}
%\datasetlicense{License under which the data set is made available (CC0, CC-BY, CC-BY-SA, CC-BY-NC, etc.)}

%%%%%%%%%%%%%%%%%%%%%%%%%%%%%%%%%%%%%%%%%%
% Only for the journal BioTech, Fishes, Neuroimaging and Toxins
%\keycontribution{The breakthroughs or highlights of the manuscript. Authors can write one or two sentences to describe the most important part of the paper.}

%%%%%%%%%%%%%%%%%%%%%%%%%%%%%%%%%%%%%%%%%%
% Only for the journal Encyclopedia
%\encyclopediadef{For entry manuscripts only: please provide a brief overview of the entry title instead of an abstract.}

%%%%%%%%%%%%%%%%%%%%%%%%%%%%%%%%%%%%%%%%%%
% Different journals have different requirements. Please check the specific journal guidelines in the "Instructions for Authors" on the journal's official website.
%\addhighlights{yes}
%\renewcommand{\addhighlights}{%
%
%\noindent The goal is to increase the discoverability and readability of the article via search engines and other scholars. Highlights should not be a copy of the abstract, but a simple text allowing the reader to quickly and simplified find out what the article is about and what can be cited from it. Each of these parts should be devoted up to 2~bullet points.\vspace{3pt}\\
%\textbf{What are the main findings?}
% \begin{itemize}[labelsep=2.5mm,topsep=-3pt]
% \item First bullet.
% \item Second bullet.
% \end{itemize}\vspace{3pt}
%\textbf{What are the implications of the main findings?}
% \begin{itemize}[labelsep=2.5mm,topsep=-3pt]
% \item First bullet.
% \item Second bullet.
% \end{itemize}
%}



%%%%%%%%%%%%%%%%%%%%%%%%%%%%%%%%%%%%%%%%%%
\begin{document}

%%%%%%%%%%%%%%%%%%%%%%%%%%%%%%%%%%%%%%%%%%
 


\newcommand{\config}{\mathcal{C}}
\newcommand{\configR}{\mathcal{C}_R}

\section{Motivation and Introduction}


In geometry, from Euclid’s space to Riemann’s manifolds, the usual approach is to begin with a set of points equipped with some structure.
Objects (points, lines, planes, etc.) are then located in the space, and one studies how they interact and what can be measured. Measurement is usually modeled as a function assigning an observable value
to a pre-existing state, usually written in the form \(f(x) \in \mathbb{R}.
\)
In this classical viewpoint, the existence of the state $x$ is taken to be
ontologically prior to the measurement $f(x)$. 


In the formulation of mathematical physics, one begins with a space or a spacetime manifold $M$, equipped with a topology $T$, a
differential structure $A$, and sometimes a metric tensor $g$.
Observables and measurements are then defined as functions on this space. While most of these points are not directly
observable, the use of a topology and a metric provides a precise
and flexible language for expressing locality, smoothness, and distance, which
has proven extremely effective in physical modeling. In this sense, the continuum
should be understood primarily as a mathematical idealization rather than as an ontological claim. Experimental limitations are typically incorporated
later, either through approximations or effective descriptions, without denying
the practical success of the underlying continuous framework.

This way of thinking dates back to Euclidean geometry, where the foundation was established through axioms regarding points, lines, and planes. Later, with Descartes, geometry became identified with $\mathbb{R}^n$, giving a coordinate‑based algebraic formulation. In the 19th century, non‑Euclidean ideas appeared in the work of Gauss, Lobachevsky, and Bolyai \cite{Zlat}. They still relied on the same foundation except for the fifth Euclidean postulate. This concept further evolved into the manifold framework \cite{Lee} used in General Relativity, where space-time is modeled as a smooth 4‑dimensional continuum \cite{Penrose}. Even in Quantum Mechanics, the underlying Hilbert space is again a continuous structure built over the field of complex numbers. In this sense, the assumption of a pre‑existing continuous substrate appears almost everywhere in modern theoretical physics.

Despite this classical picture, the operational foundations of quantum theory have long emphasized the primacy of measurement over state. Von Neumann's axiomatization of quantum mechanics \cite{vonNeumann} placed measurement postulates on equal footing with unitary evolution, while the Wheeler-Zurek anthology \cite{WheelerZurek} documented decades of debate over whether quantum states exist independently of observation. More recently, operational approaches \cite{Hardy} and quantum Bayesian (QBist) interpretations \cite{Fuchs} have argued that quantum theory is fundamentally a calculus of expectations about measurement outcomes, not a description of an observer-independent reality. In mathematical physics, the C*-algebraic formulation \cite{Bratteli} constructs quantum observables without presupposing a Hilbert space, instead deriving the space from the algebra of measurements. Category-theoretic approaches \cite{Abramsky,Coecke} similarly privilege processes (morphisms) over states (objects), emphasizing the relational structure of physical theories. Information-geometric methods \cite{Jaynes,Amari1985} treat probability distributions as the fundamental objects, with the manifold structure emerging from distinguishability measures between distributions. In topology, the point-free approach via locales \cite{Johnstone} and the categorical treatment of quotient spaces \cite{Adamek} demonstrate that spatial structure can be recovered from purely relational or logical primitives. These diverse threads suggest a common theme: \emph{the geometric structure of physical theory may be derivative rather than fundamental.}





The paper is structured as follows.
In \S\,2 we develop the axiomatic foundations of Recognition Geometry.
We introduce the primitive notion of a configuration space equipped with a
locality structure (\S\,2.1) and define recognizers as nontrivial maps to event
spaces (\S\,2.3).
The indistinguishability relation is defined in \S\,2.4, which leads to the
construction of resolution cells and the recognition quotient
(\S\,2.5--\S\,2.7). Theorem~\ref{thm:observable-injective},
shows that the induced observable map
$\overline{R}:\mathcal{C}_R\to\mathcal{E}$ is injective, meaning that distinct
observable states produce distinct events, and no hidden structure remains in
the quotient. We give several examples following the concept: threshold recognizers,
discrete lattices, quantum spin systems, and an instantiation from
Recognition Science, which illustrate the abstract constructions.
In (\S\,3) we develop more advanced structures.
We introduce the composition of recognizers, finite local resolution,
and comparative recognizers.
We also show how order-type relations arise from comparative recognition
and how recognition distances can be constructed under additional
assumptions.
 

The main idea of Recognition Geometry (RG): a fundamental inversion of the usual viewpoint, where recognition is taken as primitive, and space with its geometric structure is derived from it.  

\section{Axioms and Basic Structure}
\label{geomod}


In this section, we begin by specifying the basic axioms and primitive sets that define the underlying structure of the model. We assume that the ordinary set theory is consistent. 

\subsection{Configuration and event spaces}

The starting point of the model consists of two primitive objects: a set of states and a set of observable outcomes. We postulate the existence of a set $\mathcal C$ of configurations. A configuration $c \in \mathcal C$ represents a
complete, precise specification of the state of the system. 

 
\begin{axiom}[RG0: Nonempty Configuration Space]\label{ax1}\label{RG0}
There exists a nonempty set $\mathcal{C}$, called {\rm the configuration space.}
\end{axiom}

We explicitly do not assume that $\mathcal C$ carries any topological, metric, or algebraic
structure.  The set $\mathcal C$ may consist of vectors, graphs, labels, combinatorial objects, etc. 
Intuitively, recognizers (introduced in \S\,2.3) map configurations to {\it events}. An \emph{event} is an observable outcome: 
a pointer reading, a detector click, a boolean value, or a distinctive pattern, etc.


\begin{axiom}[RG1: Event Space]\label{ax2}
There exists a set $\mathcal E$ with at least two distinct elements,
called {\rm the event space.}
\end{axiom} 

We do not impose any algebraic, { metric} or topological structure on $\mathcal E$.
All relevant structure is induced by recognizers through their action. 

By assumption $|\mathcal E|\ge2$ in Axiom~\ref{ax2} we  exclude the trivial case. So, if a recognizer   outputs the same event, it  provides no information and induces no geometry. No cardinality restriction is imposed on $\mathcal C$. The case $|\mathcal C| = 1$ corresponds to a degenerate but admissible model, in which recognizers cannot make any nontrivial distinctions. While we do not assume a topology on $\mathcal C$, we require a notion of locality. 


Intuitively, geometry will arise not from the points themselves, but from the way
observable outcomes are produced and distinguished by measurements.




\smallskip

\noindent\textbf{Physical motivation.}
In physical experiments, measurements are inherently \emph{local} operations.
A thermometer records the temperature at its location, a Geiger counter responds
to radiation within its detection volume, and a telescope observes only the light
that reaches the instrument. Even in quantum mechanics, measurements are
localized to the region where the apparatus interacts with the system
\cite{Busch}.

This empirical fact that recognizers have a limited domain of sensitivity,
suggests that any mathematical framework should encode a notion of
``local accessibility'' among configurations.
At the same time, we wish to avoid assuming a pre-existing topological or metric
structure on $\mathcal{C}$, since such structure is intended to emerge from
recognition rather than be postulated in advance.

We therefore introduce locality in a minimal way by postulating a
\emph{neighborhood system} $\mathcal{N}$ as a primitive structure.
The neighborhoods specify which configurations are locally accessible to
measurements, without presupposing distance, continuity, or geometry.
This leads to the following axiom.
\begin{Remark}
The locality structure $\mathcal{N}$ is postulated as primitive data, specifying which configurations are considered "locally accessible" from any given configuration. While the physical motivation appeals to spatial locality, the mathematical framework treats $\mathcal{N}$ as an abstract accessibility relation that need not presuppose metric or geometric structure. In applications, $\mathcal{N}$ is typically derived from physical constraints (detector range, interaction locality, causal structure), but within the axiomatic framework it is a given structure, analogous to how a manifold's atlas is specified rather than derived.
\end{Remark} 


For each configuration $c \in \mathcal{C}$ we associate a family of subsets
$\mathcal{N}(c)$, called the \emph{neighborhoods of $c$}.


\begin{axiom}[RG2: Local Configuration Space]\label{ax:RG2}
A \emph{Local Configuration Space} is a configuration space equipped with, 
for each configuration $c \in \mathcal{C}$, a nonempty collection 
$\mathcal{N}(c) \subseteq \mathcal{P}(\mathcal{C})$ of subsets of $\mathcal{C}$, 
called the \emph{neighborhoods of $c$}, satisfying:

\begin{enumerate}
    \item[(i)]  {Reflexivity:} 
        \( \forall c \in \mathcal{C},\ \forall U \in \mathcal{N}(c),\ c \in U. \)
    \item[(ii)]  {Intersection closure:} 
        \( \forall c \in \mathcal{C},\ \forall U,V \in \mathcal{N}(c),\ 
        \exists W \in \mathcal{N}(c)\ \text{such that}\ W \subseteq U \cap V. \)
    \item[(iii)] {Local refinement:}
        \( \forall c \in \mathcal{C},\ \forall U \in \mathcal{N}(c),\ \forall c' \in U,\ 
        \exists V \in \mathcal{N}(c')\ \text{such that}\ V \subseteq U. \)
\end{enumerate}
\end{axiom}

Intuitively, for each $c \in \mathcal{C}$, the family $\mathcal{N}(c)$ specifies
which subsets of configurations are considered {locally accessible} from $c$.
Thus, every configuration is contained in each of its neighborhoods (reflexivity); any two local neighborhoods can be refined to a common, smaller one (intersection closure); and any point inside a neighborhood has its own neighborhood contained in the original one (local refinement).

Notice that $\mathcal{N}$ is not itself a neighborhood system in the
topological sense, since the  \emph{monotonicity}  is not assumed 
(if $U \in \mathcal N(c)$ and $V \subseteq U$,
it does \emph{not} follow that $V \in \mathcal N(c)$). Consequently, $\mathcal{N}$ does not specify a topology directly, but it does generate a canonical topology on $\mathcal{C}$, which we will explain in the following section.  


\subsection{Topology Generated by the Locality Structure}

Although $\mathcal N$ is not a neighborhood system in the
topological sense, it nevertheless generates
a canonical topology on $\mathcal C$.

\begin{Definition}
\label{def:generated-topology}
Let $\mathcal N$ be a locality structure on $\mathcal C$.
A set $U \subseteq \mathcal C$ is  \emph{open} if and only if
for every $c \in U$, there exists $V \in \mathcal N(c)$ such that
$V \subseteq U$.
The collection of all open sets is denoted by $\tau_{\mathcal N}$.
\end{Definition}

 
\begin{Proposition}
$\tau_{\mathcal N}$ is a topology on $\mathcal C$.
\end{Proposition}

\begin{proof}
Clearly, the empty set is open. The whole space $\mathcal C$ is open since
for each $c \in \mathcal C$, any $V \in \mathcal N(c)$ satisfies $V \subseteq \mathcal C$.
Arbitrary unions of open sets are open, and finite intersections of open sets are also open. This follows directly from Definition \ref{def:generated-topology}
and Axiom~\ref{ax:RG2}(ii).
\end{proof}

\begin{Remark}
Definition~\ref{def:generated-topology} provides the canonical way to generate a topology from the locality structure $\mathcal{N}$. The construction declares a set open if it is locally a neighborhood: for every point in the set, the set contains a neighborhood of that point. This is a standard method for generating a topology from a neighborhood system (see \cite{Munkres}, Chapter 2). Because $\mathcal{N}(c)$ is not assumed to be monotone, we do not claim that every set in $\mathcal{N}(c)$ is a topological neighborhood of $c$ in $\tau_{\mathcal{N}}$. Rather, $\tau_{\mathcal{N}}$ is the natural topology used in this paper: openness is defined by local containment of some $\mathcal{N}$-neighborhood.
\end{Remark}



\subsection{Recognition maps}\label{rec}

In this section, we introduce the central object of the theory, the \emph{recognizer}.

\begin{Definition}[Recognition triple]\label{def:recognition_triple}
A \emph{recognition triple} is an ordered triple $(\mathcal{C}, \mathcal{E}, \Sigma)$ where:
\begin{itemize}
    \item $\mathcal{C}$ is a nonempty set (configuration space, Axiom~\ref{ax1}),
    \item $\mathcal{E}$ is a set with $|\mathcal{E}| \geq 2$ (event space, Axiom~\ref{ax2}),
    \item $\Sigma$ is a nonempty set of functions $R: \mathcal{C} \to \mathcal{E}$ 
          such that $|\operatorname{Im}(R)| \geq 2$ for each $R \in \Sigma$.
\end{itemize}
Elements of $\Sigma$ are called \emph{recognizers}.
\end{Definition}


The condition $|\operatorname{Im}(R)| \geq 2$ ensures that every recognizer distinguishes at least two different configurations in $\mathcal{C}$.
Constant functions convey no information and are therefore excluded.


This paper treats recognizers as total, deterministic functions. 
Several natural generalizations exist but require substantial modifications:

\smallskip\noindent
\textbf{Partial recognizers.} 
If a recognizer $R$ is only defined on a domain $\text{dom}(R) \subseteq \mathcal{C}$, 
the quotient construction (Section~\ref{quotient}) applies 
only to $\text{dom}(R)$. This models detectors with finite range but introduces 
complications in defining global geometric structures.

\smallskip\noindent
\textbf{Stochastic recognizers.} 
  A stochastic recognizer 
$R: \mathcal{C} \to \Delta(\mathcal{E})$ assigns a probability measure on $\mathcal{E}$ 
to each configuration. Indistinguishability must then be defined via a metric 
on $\Delta(\mathcal{E})$ (e.g., total variation distance). This connects to POVMs 
in quantum theory \cite{Busch}, but requires additional measure-theoretic structure 
not assumed in this paper.

\begin{Definition}[Fiber]\label{def:fiber}
Let $R: \mathcal{C} \to \mathcal{E}$ be a recognizer.
For $e \in \operatorname{Im}(R)$, the \emph{fiber over $e$} is the set
\[
R^{-1}(e) := \{\, c \in \mathcal{C} \mid R(c) = e \,\}.
\]
\end{Definition}

\begin{Remark}
The collection of fibers $\{\, R^{-1}(e) : e \in \operatorname{Im}(R) \,\}$ 
forms a partition of $\mathcal{C}$, i.e. each configuration belongs to exactly one fiber.
\end{Remark}


Recognizers induce a very important relation on $\mathcal{C}$ by the following definition.  
\begin{Definition}[Indistinguishability]\label{Indistinguishability}
Given a recognizer $R: \mathcal{C} \to \mathcal{E}$, we say that two configurations $c_1$ and $c_2$ are \emph{indistinguishable} with respect to $R$, and write 
\[
c_1 \sim_R c_2
\quad\mbox{if}
\quad
R(c_1) = R(c_2).
\]
\end{Definition}
Consequently, the relation ${\sim_R}$ is an equivalence relation on $\mathcal{C}$,
since it is defined by equality in $\mathcal{E}$.




\subsection{The Quotient Space}

The equivalence relation ${\sim_R}$ induces a quotient space.

\begin{Definition}[Quotient Space]\label{def:quotient}
Given a recognizer $R: \mathcal{C} \to \mathcal{E}$. 
The \emph{quotient space} is
\[
\mathcal{C}/{\sim_R} 
\;:=\; 
\bigl\{\, [c]_R : c \in \mathcal{C} \,\bigr\}
\quad\text{where}\quad
[c]_R = \{\, c' \in \mathcal{C} : c' \sim_R c \,\}.
\]
\end{Definition}

\begin{Remark}
The quotient $\mathcal{C}/{\sim_R}$ is in natural bijection with $\operatorname{Im}(R)$ 
with respect to the map $[c]_R \mapsto R(c)$.
Each equivalence class $[c]_R$ is a fiber $R^{-1}(e)$ for some $e \in \operatorname{Im}(R)$.
\end{Remark}


 
\begin{Example}[Threshold Recognizers]\label{ex:threshold}
Let $\mathcal{C} = \mathbb{R}^n$ be the configuration space and  
$\mathcal E = \{0,1\}$ be the event space.  
Let us define $\Sigma$ as the family of threshold recognizers, i.e., the set of functions $\mathcal{C}\to\mathcal{E}$ such that 
\[
f_{v,t}(x) = 
\begin{cases}
1, & \text{if } x \cdot v > t,\\[4pt]
0, & \text{otherwise},
\end{cases}
\]
where $\cdot$ denotes the standard Euclidean inner product on $\mathbb{R}^n$, $v \in \mathbb{R}^n$ and $t \in \mathbb{R}$.

Each recognizer $f_{v,t}$ divides $\mathbb{R}^n$ into two half-spaces, one
``recognized'' (event $1$) and one ``not recognized'' (event $0$).  
The family $\Sigma = \{ f_{v,t} : v \in \mathbb{R}^n,\, t \in \mathbb{R} \}$ 
thus induces a geometric structure:  two points are considered indistinguishable with respect to the recognizers if and only if they lie in the same collection of half-spaces.
\end{Example}

\begin{Remark}
In this paper, indistinguishability is defined relative to a single recognizer $R:\mathcal{C}\to\mathcal{E}$. For a family of recognizers $\Sigma$, a standard way to package their joint observational content is the product map
\[
R_{\Sigma}:\mathcal{C}\to \mathcal{E}^{\Sigma},\qquad
R_{\Sigma}(c)=(R(c))_{R\in\Sigma},
\]
and then to form the quotient $\mathcal{C}/\sim_{R_{\Sigma}}$, which identifies configurations that agree on every recognizer in $\Sigma$. This construction is the (possibly infinite) analog of the composite recognizer discussed in \S\,\ref{sec:advanced}: combining recognizers refines the quotient by intersecting their resolution cells. For the full family of threshold tests on $\mathbb{R}^n$, $R_{\Sigma}$ separates points (any two distinct points are separated by some half-space), so the resulting indistinguishability coincides with equality.
\end{Remark}






\begin{Example}[Discrete Lattice]\label{ex:discrete}
Let $\mathcal{C} = \mathbb{Z}^3$ be the configuration space and $\mathcal{E} = \{0,1\}$ the event space. 
Let us define the recognizer $R:\mathcal{C} \to \mathcal{E}$ by
\[
R(x,y,z) = (x+y+z) \bmod 2.
\] 
This recognizer divides the integer lattice into two classes: points with an even sum of coordinates, and points with an odd sum of coordinates.
Therefore, the quotient $\mathcal{C}_R$ has exactly 2 points. Here $\mathcal{C}_R$ denotes the quotient of $\mathcal{C}$ by the equivalence relation induced by $R$ (see Definition~\ref{def:quotient}). 
This example shows that RG can be applied naturally to discrete spaces, without any continuity assumption.
\end{Example}

\begin{Example}[Quantum Spin]\label{ex:quantum}
Let $\mathcal{C} = S^2$ be the Bloch sphere (the space of pure quantum spin-$\tfrac{1}{2}$ states) and $\mathcal{E} = \{+1, -1\}$. Given a unit vector $\mathbf{n} \in S^2$, the spin measurement along direction $\mathbf{n}$ is operationally defined via the Stern-Gerlach apparatus oriented along $\mathbf{n}$. For a fixed choice (e.g., the laboratory $z$-axis), the recognizer $R_z: S^2 \to \{+1,-1\}$ partitions the Bloch sphere into two regions corresponding to the two measurement outcomes. In this example, the manifold structure of $S^2$ (as the space of normalized spinors $\mathbb{C}^2/\mathbb{C}^*$) is assumed a priori; the recognizers partition this given space rather than constructing it. 
Adding the $x$-component recognizer $R_x$ refines the partition into four regions, and adding $R_y$ further refines it. 
However, no finite family of binary recognizers can distinguish all pairs 
of distinct points on $S^2$; the quotient remains finite and discrete. 
This illustrates the fundamental limitation imposed by finite-resolution measurements \cite{Busch}. 
\end{Example}

\begin{Example}[Recognition Science Instantiation]\label{ex:RS}
In the framework of Recognition Science, let $\mathcal{C} = \mathcal{L}$ be the space of all ledger states (the complete ontological record of all entities and their properties), and let $\mathcal{E} = \mathbb{R}^3$. 
Define the position recognizer $R_{\text{pos}}: \mathcal{L} \to \mathbb{R}^3$ that extracts the spatial coordinates of a given entity from the ledger. 
{The recognition quotient is then $\mathcal{L}/{\sim_{R_{\text{pos}}}} \cong \operatorname{Im}(R_{\text{pos}}) \subseteq \mathbb{R}^3$ (by Proposition~\ref{prop:quotient-image-iso}). Points in the quotient are equivalence classes of ledger states indistinguishable with respect to position. In this construction, the 3-dimensional Euclidean structure is present in the event space $\mathcal{E} = \mathbb{R}^3$; the quotient inherits this structure via the isomorphism to $\operatorname{Im}(R_{\text{pos}})$. This example illustrates the RG framework applied to the Recognition Science paradigm \cite{RQM}, showing how the mathematical formalism relates ontological states (ledger) to observable spatial structure.}
\end{Example}


The four examples above illustrate key structural features of RG:

\begin{enumerate}
\item	\textbf{Configuration space structure:} The framework applies equally to discrete spaces (Example~\ref{ex:discrete}: $\mathbb{Z}^3$), continuous manifolds (Example~\ref{ex:threshold}: $\mathbb{R}^n$; Example~\ref{ex:quantum}: $S^2$), and abstract state spaces (Example~\ref{ex:RS}: the Ledger $\mathcal{L}$). No topology, metric, or smooth structure is required a priori. 
\item	\textbf{Event space cardinality:} Event spaces may be finite (Examples~\ref{ex:discrete}, \ref{ex:quantum}: binary outcomes) or continuous (Example~\ref{ex:RS}: $\mathbb{R}^3$). The quotient $\mathcal{C}_R$ is always isomorphic to $\operatorname{Im}(R)$, so the ``size'' of observable space is determined entirely by the recognizer.
\item	\textbf{Refinement and composition:} Example~\ref{ex:quantum} demonstrates that multiple recognizers ($R_z$, $R_x$, $R_y$) refine the partition. Adding recognizers never coarsens the quotient---it can only distinguish states that were previously indistinguishable. However, Example~\ref{ex:quantum} also shows a fundamental limitation: no finite family of binary recognizers can recover the full continuous structure of $S^2$. The quotient remains finite and discrete.
\item {\textbf{Construction of observable space:}} Example~\ref{ex:RS} embodies the core philosophy of RG. {The quotient $\mathcal{L}/{\sim_R}$ provides a formal construction of observable space from the ledger and recognizer. The quotient's structure (in this case, isomorphic to a subset of $\mathbb{R}^3$) is inherited from the event space by Proposition~\ref{prop:quotient-image-iso}. The conceptual contribution is the inversion of the usual order: rather than assuming physical space exists and defining measurements as functions on it, we take measurements (recognizers) as primitive and construct the observable space as the quotient. The geometric structure of the observable space depends on both the configuration space $\mathcal{C}$ and the target event space $\mathcal{E}$.}
\end{enumerate}

\subsection{Formalizing the Recognition Structure}
In this section, we formalize the recognition elements introduced in the Definition~\ref{def:recognition_triple}, making explicit the
structural assumptions underlying locality, recognition, etc.


\begin{Definition}[Locality Structure]\label{def:locality-structure}
A \emph{locality structure} on the configuration space $\mathcal C$ is a map
\[
\mathcal N : \mathcal C \to \mathcal P(\mathcal P(\mathcal C)),
\]
satisfying Axiom~\ref{ax:RG2}. For each configuration $c\in\mathcal C$, the family
$\mathcal N(c)$ specifies which subsets of $\mathcal C$ are regarded as locally
accessible from $c$.
\end{Definition}

\begin{Definition}[Recognition Structure]\label{def:rec-structure}
A \emph{Recognition Structure} on a pair $(\mathcal C,\mathcal E)$ is a tuple
\[
\mathcal S = (\mathcal N, \Sigma),
\]
consisting of:
\begin{enumerate}
    \item a locality structure $\mathcal N$ on $\mathcal C$
    (Definition~\ref{def:locality-structure});
    \item a nonempty set $\Sigma$ of functions $R:\mathcal C\to\mathcal E$, called
    \emph{recognizers}, such that
    \[
    |\operatorname{Im}(R)| \ge 2 \quad \text{for all } R \in \Sigma .
    \]
\end{enumerate}
In this way, a recognition structure specifies both what is observable
(via the recognizers in $\Sigma$) and what is locally accessible
(via the locality structure $\mathcal N$).
\end{Definition}
To summarize, Definitions~\ref{def:recognition_triple} and \ref{def:rec-structure} provide

\begin{Definition}[Recognition Triple (formalized)]\label{def:rec-triple-formal}
A \emph{Recognition Triple} is a tuple $(\mathcal C,\mathcal E,\mathcal S)$ where:
\begin{itemize}
    \item $\mathcal C$ is a nonempty configuration space (Axiom~\ref{ax1});
    \item $\mathcal E$ is an event space with $|\mathcal E|\ge 2$
    (Axiom~\ref{ax2});
    \item $\mathcal S=(\mathcal N,\Sigma)$ is a recognition structure on
    $(\mathcal C,\mathcal E)$ in the sense of
    Definition~\ref{def:rec-structure}.
\end{itemize}
For notational convenience, we may also denote a Recognition Triple by 
$(\mathcal C, \mathcal E, \mathcal N, \Sigma)$.
\end{Definition}

\begin{Remark}
The locality structure $\mathcal N$ is global data on the configuration space
$\mathcal C$ and is independent of the choice of recognizer. Although
$\mathcal N$ does not enter the purely set-theoretic definition of the quotient
$\mathcal C_R$ associated with a single recognizer $R\in\Sigma$, it becomes
essential when addressing questions of continuity, regularity, or induced
topology on $\mathcal C_R$. The role of $\mathcal{N}$ is to specify which configurations are "locally accessible" to measurements, independent of which specific recognizer is applied. This structure is part of the physical setup (e.g. which regions an instrument can access) rather than a property of individual observables.

When multiple recognizers act on the same configuration space, as in
Example~\ref{ex:quantum}, they are treated as elements of the same set $\Sigma$
within a single recognition structure and share the same locality structure
$\mathcal N$. Operations relating to different recognizers rely on this common
structure.
\end{Remark}

\begin{Remark} 
From a categorical viewpoint \cite{MacLane}, the quotient construction can be
understood as defining a functor when appropriate morphisms are specified on
configuration spaces and observable spaces. The injectivity of the induced map
$\overline{R}$ (Theorem~\ref{thm:observable-injective}) ensures that the quotient
$\mathcal{C}_R$ faithfully represents observable distinctions. A full
categorical treatment, including functoriality and universal properties, is
deferred to future work.
\end{Remark}


\subsection{Recognition quotient}\label{quotient}

We now arrive at the first major structural object of RG: the observable space
obtained by identifying indistinguishable configurations.

\smallskip

\noindent{\it Construction.}
Given a recognizer $R : \mathcal C \to \mathcal E$, the indistinguishability
relation $\sim_R$ on $\mathcal C$ ($
c_1 \sim_R c_2   \Longleftrightarrow   R(c_1)=R(c_2).
$)
This relation partitions $\mathcal C$ into \emph{resolution cells}, i.e.,
equivalence classes $[c]_R$. The \emph{recognition quotient} is the quotient of
the configuration space by this partition.

\begin{Definition}[Recognition Quotient]\label{def:recognition-quotient}
The \emph{recognition quotient} associated with the recognizer $R$ is the
quotient set
\[
\mathcal C_R \;=\; \mathcal C / {\sim_R}.
\]
\end{Definition}

We denote by
\[
\pi : \mathcal C \longrightarrow \mathcal C_R,
\qquad
\pi(c) = [c]_R,
\]
the canonical projection that maps each configuration to its resolution cell.

\begin{Remark}
The quotient space $\mathcal C_R$ represents the space of observationally
distinguishable states: two configurations have the same image in
$\mathcal C_R$ if and only if the recognizer assigns them the same event. Thus
$\mathcal C_R$ captures precisely the \emph{observable geometry} determined by
$R$.
\end{Remark}

Since $R(c)$ is constant on each resolution cell $[c]_R$, the recognizer
$R : \mathcal C \to \mathcal E$ descends to a well-defined map on the quotient.
We denote the induced observable map by
\[
\overline{R} : \mathcal C_R \longrightarrow \mathcal E,
\qquad
\overline{R}([c]_R) := R(c).
\]
This map is well defined because if $[c_1]_R = [c_2]_R$, then $c_1 \sim_R c_2$
and hence $R(c_1)=R(c_2)$.

\begin{Theorem} 
\label{thm:observable-injective}
The induced map
\(
\overline{R} : \mathcal C_R \to \mathcal E
\)
is injective. 
\end{Theorem}

\begin{proof}
Let $q_1, q_2 \in \mathcal C_R$ satisfy
\(
\overline{R}(q_1)=\overline{R}(q_2).
\)
Let us choose representatives $c_1, c_2 \in \mathcal C$ such that
$q_1 = [c_1]_R$ and $q_2 = [c_2]_R$. Then
\[
R(c_1)=\overline{R}(q_1)=\overline{R}(q_2)=R(c_2),
\]
hence $c_1 \sim_R c_2$ by Definition \ref{Indistinguishability}.
Therefore $[c_1]_R=[c_2]_R$, and consequently $q_1=q_2$.
\end{proof}

\begin{Proposition} 
\label{prop:quotient-image-iso}
The recognition quotient $\mathcal C_R$ is naturally isomorphic to the image
$\operatorname{Im}(R)$ with respect to the map
\[
\Phi : \mathcal C_R \longrightarrow \operatorname{Im}(R),
\qquad
\Phi([c]_R)=R(c).
\]
\end{Proposition}

\begin{proof}
The map $\Phi$ is well-defined since $R$ is constant on equivalence classes.
It is surjective by definition of $\operatorname{Im}(R)$ and injective by
Theorem~\ref{thm:observable-injective}. Hence $\Phi$ is a bijection.
\end{proof}

The following corollary follows immediately from the previous results.

\begin{Corollary}\label{cor:no-hidden}
In the recognition quotient $\mathcal C_R$, observable states are completely
and uniquely determined by the events they produce. {As a set, $\mathcal{C}_R$ carries no distinctions beyond those induced by $R$ (equivalently, $\mathcal{C}_R \cong \operatorname{Im}(R)$). If additional structure (e.g., a topology) is supplied or induced via the locality structure $\mathcal{N}$, then $\mathcal{C}_R$ may carry further structure not determined by $R$ alone.}
\end{Corollary}

\begin{proof}
By Proposition~\ref{prop:quotient-image-iso}, the quotient $\mathcal C_R$ is
isomorphic to $\operatorname{Im}(R)$. Thus, distinct observable
states correspond to distinct events, and no further distinctions exist within
$\mathcal C_R$ beyond those encoded by $R$.
\end{proof}

%%%
\begin{Remark}
Corollary~\ref{cor:no-hidden} admits a clear epistemic interpretation. Relative
to a fixed recognizer $R$, all observable information about a configuration is
exhausted by the corresponding event. In this sense, the quotient
$\mathcal C_R$ represents the effective state space accessible to an observer
using $R$, and no finer distinctions are observable within this framework.{(Here ``no finer distinctions'' is understood in the sense of identification of configurations by $\sim_R$; additional, independently postulated structure on $\mathcal{C}$ may still induce nontrivial topological properties on $\mathcal{C}_R$.)}
\end{Remark}
%%%%%%


{The recognition quotient construction is mathematically equivalent to several well-known structures in differential geometry, probability theory, and physics. The conceptual contribution lies in the reinterpretation: taking recognizers (measurements) as the primitive objects that determine the quotient space, rather than assuming a space and then studying partitions on it. Key connections include:}


\begin{Example}[Orbit spaces] In Lie theory, a group $G$ acting on a manifold $M$ partitions $M$ into orbits, and the orbit space $M/G$ is the quotient by the equivalence relation $x \sim y \iff \exists g \in G : g \cdot x = y$. Recognition quotients are similar, with the recognizer $R$ playing the role of the ``observable'' that is constant on orbits. {Both constructions study quotients by equivalence relations. The interpretive difference is that RG emphasizes the recognizer (measurement) as the primitive object that induces the partition, whereas orbit space theory typically begins with the group action and derives the quotient. Mathematically, if $R : M \to \mathcal{E}$ is constant on $G$-orbits, then $M/G$ maps naturally into $\mathcal{C}_R$.}\end{Example}

\begin{Example}[Level sets] For a smooth submersion $f : M \to \mathbb{R}$, the level sets $f^{-1}(c)$ form a foliation of $M$.
For any recognizer $R : \mathcal C \to \mathcal E$, the resolution cells
$[c]_R = R^{-1}(\{R(c)\})$ are precisely the level sets (fibers) of $R$.
When $R$ satisfies appropriate regularity assumptions, these level sets may carry
additional geometric structure analogous to a foliation. {The quotient space $\mathcal{C}_R$ identifies level sets to points and is isomorphic to the image $\operatorname{Im}(R)$ (Proposition~\ref{prop:quotient-image-iso}). Both classical differential geometry and RG study the same mathematical structure; the difference lies in which object is taken as primary (the manifold $M$ with its foliation, versus the quotient $\mathcal{C}_R$ as observable space).}\end{Example}

\begin{Example}[Measurable partitions.] In ergodic theory and probability \cite{Chentsov}, measurable partitions are used to define conditional expectations and factors of dynamical systems. The recognition quotient $\mathcal{C}_R$ is the factor algebra corresponding to the partition induced by $R$. Our framework extends this to the non-probabilistic, purely geometric setting, emphasizing the quotient space itself rather than the $\sigma$-algebra structure.\end{Example}

\begin{Example}[Relational quantum mechanics.] Rovelli's relational interpretation \cite{RQM} asserts that quantum states are relative to observers. RG formalizes this: the ``state relative to observer $R$'' is precisely the equivalence class $[c]_R$ in the quotient. Different recognizers (observers) induce different quotients (relative realities), unified by the underlying configuration space $\mathcal{C}$. The framework emphasizes the \emph{primacy of recognizers} as the foundational objects: the observable space $\mathcal{C}_R$ is constructed as the quotient induced by the recognizer $R$, rather than being assumed a priori. This reinterpretation connects naturally to operational and measurement-based approaches in quantum theory and provides a formal setting for studying how geometric structure relates to observational capabilities.\end{Example} 

In the following theorem, we present the universal property of the recognition quotient.

\begin{Theorem}\label{thm:universal}
Let $R : \mathcal{C} \to \mathcal{E}$ be a recognizer, and let $\pi : \mathcal{C} \to \mathcal{C}_R$ be the canonical projection. Then for any set $X$ and any function $f : \mathcal{C} \to X$ that is constant on resolution cells (i.e., whenever $c_1 \sim_R c_2$, we have $f(c_1) = f(c_2)$), there exists a unique function $\overline{f} : \mathcal{C}_R \to X$ such that 
\[
f = \overline{f} \circ \pi.
\]
\end{Theorem}

\begin{proof}
\textit{Existence:} Let us define $\overline{f} : \mathcal{C}_R \to X$ by $\overline{f}([c]_R) := f(c)$. We must verify this is well-defined. Suppose $[c_1]_R = [c_2]_R$. Then $c_1 \sim_R c_2$, and consequently $f(c_1) = f(c_2)$. Hence, $\overline{f}$ is independent of the choice of representative and is well-defined. By construction, $\overline{f}(\pi(c)) = \overline{f}([c]_R) = f(c)$, so $f = \overline{f} \circ \pi$.

\smallskip\noindent
\textit{Uniqueness:} Let $g : \mathcal{C}_R \to X$ also satisfies $f = g \circ \pi$. Then for any $q \in \mathcal{C}_R$, choose a representative $c \in \mathcal{C}$ with $q = [c]_R = \pi(c)$. We have
\[
g(q) = g(\pi(c)) = f(c) = \overline{f}(\pi(c)) = \overline{f}(q).
\]
So, $g = \overline{f}$, which completes the proof.
\end{proof}




 Theorem~\ref{thm:universal} is the standard universal property of quotient spaces (see \cite{MacLane}, \cite{Adamek}).
The universal property characterizes the recognition quotient $\mathcal{C}_R$ up to unique isomorphism. It states that $\mathcal{C}_R$ is the ``finest'' or ``most refined'' quotient through which $R$ factors: any other quotient on which $R$ is well-defined must factor through $\mathcal{C}_R$. In categorical terms, the pair $(\mathcal{C}_R, \pi)$ is the 
\emph{coequalizer} of the kernel pair of $R$, that is, 
the universal object that identifies precisely those 
configurations recognized as equivalent by $R$.



\subsection{The Quotient Topology}
 

Let $\tau := \tau_{\mathcal N}$ denote the topology on $\mathcal C$
generated by the locality structure $\mathcal N$
(Definition~\ref{def:generated-topology}).
We now show that this structure descends naturally to the recognition quotient, endowing $\mathcal{C}_R$ with the structure of a topological space.

%In other words, we show that the topology on $\mathcal C$ can be 
%transferred to the quotient $\mathcal{C}_R$ in such a way that a set 
%$U \subseteq \mathcal{C}_R$ is open if and only if its preimage 
%$\pi^{-1}(U) \subseteq \mathcal{C}$ is open.

\begin{Definition}[Quotient Topology on $\mathcal{C}_R$]\label{def:quotient-topology}
Let $\tau$ be the topology on $\mathcal{C}$ generated by the neighborhood system $\mathcal{N}$. 
The \emph{quotient topology} $\tau_R$ on $\mathcal{C}_R$ is defined as follows: a subset $U \subseteq \mathcal{C}_R$ is open if and only if its preimage under the canonical projection is open in $\mathcal{C}$:
\[
U \in \tau_R \quad\Longleftrightarrow\quad \pi^{-1}(U) \in \tau.
\]
\end{Definition}



\begin{Proposition}\label{prop:quotient-topology}
The quotient topology $\tau_R$ is a topology on $\mathcal{C}_R$, and the canonical projection $\pi : (\mathcal{C}, \tau) \to (\mathcal{C}_R, \tau_R)$ is continuous and surjective.
\end{Proposition}

\begin{proof}
We verify the axioms of a topology. 

\smallskip\noindent
(i) $\emptyset \in \tau_R$ because $\pi^{-1}(\emptyset) = \emptyset \in \tau$. Similarly, $\mathcal{C}_R \in \tau_R$ because $\pi^{-1}(\mathcal{C}_R) = \mathcal{C} \in \tau$.

\smallskip\noindent
(ii) Let $\{U_i\}_{i \in I}$ be a family of open sets in $\tau_R$. Then each $\pi^{-1}(U_i)$ is open in $\tau$, and 
\[
\pi^{-1}\left(\bigcup_{i \in I} U_i\right) = \bigcup_{i \in I} \pi^{-1}(U_i) \in \tau.
\]
Hence $\bigcup_{i \in I} U_i \in \tau_R$.

\smallskip\noindent
(iii) Let $U, V \in \tau_R$. Then $\pi^{-1}(U), \pi^{-1}(V) \in \tau$, and 
\[
\pi^{-1}(U \cap V) = \pi^{-1}(U) \cap \pi^{-1}(V) \in \tau.
\]
Hence $U \cap V \in \tau_R$.

\smallskip\noindent
Continuity of $\pi$ follows from Definition \ref{def:quotient-topology}, i.e. if $U \in \tau_R$, then $\pi^{-1}(U) \in \tau$. Surjectivity holds because every equivalence class $[c]_R \in \mathcal{C}_R$ is the image of $c \in \mathcal{C}$ under $\pi$.
\end{proof}

{ Recall that a map between topological spaces is continuous if the preimage
of every open set is open.}

\begin{Proposition}\label{prop:final-topology}
The quotient topology $\tau_R$ is the \emph{final topology} (also called the \emph{coinduced topology}) with respect to $\pi$:
it is the finest topology on $\mathcal C_R$ that makes $\pi$ continuous.
\end{Proposition}

\begin{proof}
Let $\tau'$ be any topology on $\mathcal C_R$ such that
$
\pi : (\mathcal C,\tau) \to (\mathcal C_R,\tau')
$
is continuous. Then for every $U\in\tau'$, continuity implies
\[
\pi^{-1}(U)\in\tau.
\]
By Definition~\ref{def:quotient-topology}, this is equivalent to $U\in\tau_R$.
Hence $\tau'\subseteq\tau_R$, showing that $\tau_R$ contains every topology on
$\mathcal C_R$ that makes $\pi$ continuous, and is therefore the finest (largest)
such topology.
\end{proof}




\begin{Remark}
The quotient topology ensures that the observable space $\mathcal{C}_R$ inherits a natural topological structure from the configuration space. Open sets in $\mathcal{C}_R$ are precisely those subsets 
$U \subseteq \mathcal{C}_R$ whose preimage $\pi^{-1}(U)$, the union of all resolution cells in $U$,  is open in 
$\mathcal{C}$. This topology encodes which observable 
states are "nearby" in a manner consistent with the 
locality structure on configurations. Intuitively, observable states $[c_1]_R$ and $[c_2]_R$ are topologically close in $\mathcal{C}_R$ if the corresponding resolution cells (equivalence classes) in $\mathcal{C}$ are close in the sense that their union forms an open set.
\end{Remark}

{  

Having endowed $\mathcal C$ with the topology $\tau_{\mathcal N}$ generated by the locality structure (Definition~\ref{def:generated-topology}), we can naturally define when a recognizer is continuous.

 

\begin{Definition}[Continuous Recognizer]\label{def:continuous-recognizer}
A recognizer $R:\mathcal C\to\mathcal E$ is called continuous if,
for every open set $U\subseteq\mathcal E$,
the preimage $R^{-1}(U)$ is open in $\mathcal C$.
\end{Definition}

\begin{Proposition}\label{prop:continuity-on-quotient}
Let $\tau_R$ be the quotient topology on $\mathcal C_R$ induced by the canonical
projection $\pi:(\mathcal C,\tau)\to\mathcal C_R$, and let $\tau_{\mathcal E}$ be a
topology on $\mathcal E$.
If the recognizer $R:(\mathcal C,\tau)\to(\mathcal E,\tau_{\mathcal E})$ is continuous,
then the induced map
\[
\overline{R}:(\mathcal C_R,\tau_R)\to(\mathcal E,\tau_{\mathcal E})
\]
is continuous.
\end{Proposition}


\begin{proof}
By definition of the quotient topology $\tau_R$, a map
$\overline{R}:\mathcal C_R\to\mathcal E$ is continuous if and only if
the composition $\overline{R}\circ\pi:\mathcal C\to\mathcal E$ is continuous.
Since $\overline{R}\circ\pi = R$ and $R$ is continuous by assumption,
it follows that $\overline{R}$ is continuous.
\end{proof}
} 




  
\section{Advanced Structure}\label{sec:advanced}

 
In this section, we study more advanced recognition structures, including composition of recognizers, finite resolution, comparative recognizers, and metric structures induced by recognition. Their study requires additional axioms and technical tools.



\subsection{Composition of Recognizers}

Physical measurement rarely involves a single isolated observation. For example, we can observe position {and} momentum, color {and} shape, etc. This combination of measurements is formalized as the composition of recognizers.


\begin{Definition}[Composite Recognizer]
    Given two recognizers $R_1:\mathcal{C} \to \mathcal{E}_1$ and $R_2: \mathcal{C}\to \mathcal{E}_2$, their \textit{composition} is the recognizer $R_1 \otimes R_2: \mathcal{C} \to \mathcal{E}_1\times \mathcal{E}_2$ defined by:
    \[ (R_1 \otimes R_2)(c) = (R_1(c), R_2(c)) \]
\end{Definition}

From the definition, it is clear that the composite $R_1 \otimes R_2$ is a recognizer: its image satisfies
\[
\mathrm{Im}(R_1 \otimes R_2) \subseteq \mathrm{Im}(R_1) \times \mathrm{Im}(R_2),
\]
and is nontrivial, since there exist configurations $c_1,c_2\in\mathcal C$ with
either $R_1(c_1)\neq R_1(c_2)$ or $R_2(c_1)\neq R_2(c_2)$, implying
$(R_1\otimes R_2)(c_1)\neq (R_1\otimes R_2)(c_2)$.

 

Composition increases distinguishing power. If two configurations are distinguishable by $R_1$ or by $R_2$, they are also distinguishable by the composite recognizer $R_1 \otimes R_2$.

\begin{Theorem}[Composite Indistinguishability]
    \[ c_1 \sim_{R_1 \otimes R_2} c_2 \iff (c_1 \sim_{R_1} c_2) \land (c_1 \sim_{R_2} c_2) \]
\end{Theorem}

 \begin{proof}
By definition, $c_1 \sim_{R_1 \otimes R_2} c_2$ means $(R_1 \otimes R_2)(c_1) = (R_1 \otimes R_2)(c_2)$, i.e., $(R_1(c_1), R_2(c_1)) = (R_1(c_2), R_2(c_2))$. The last equality holds if and only if both coordinates are equal: $R_1(c_1)=R_1(c_2)$ and $R_2(c_1)=R_2(c_2)$, which is equivalent to $c_1 \sim_{R_1} c_2$ and $c_1 \sim_{R_2} c_2$.
\end{proof}

As an immediate consequence, the equivalence classes of the composite recognizer are given by intersections of the equivalence classes of its components.

\begin{Corollary} 
For any $c \in \mathcal{C}$,
\[ [c]_{R_1 \otimes R_2} = [c]_{R_1} \cap [c]_{R_2} \]
\end{Corollary}
\begin{proof}
$c' \in [c]_{R_1 \otimes R_2}$ $\iff$ $c' \sim_{R_1 \otimes R_2} c$ $\iff$ $c' \sim_{R_1} c$ and $c' \sim_{R_2} c$ $\iff$ $c' \in [c]_{R_1}$ and $c' \in [c]_{R_2}$ $\iff$ $c' \in [c]_{R_1} \cap [c]_{R_2}$.
\end{proof}

Further, the classes of the composite recognizer naturally map onto the classes of the individual recognizers via the canonical projections $\pi_1$ and $\pi_2$. More precisely, it holds the following theorem.

\begin{Theorem} \label{thm:refinement}
    The recognition quotient of the composite refines the quotients of its components. There exist surjective canonical maps:
    \[ \pi_1: \config_{R_1 \otimes R_2} \twoheadrightarrow \config_{R_1} \quad \text{and} \quad \pi_2: \config_{R_1 \otimes R_2} \twoheadrightarrow \config_{R_2} \]  
defined by $\pi_1([c]_{R_1 \otimes R_2}) = [c]_{R_1}$ and 
$\pi_2([c]_{R_1 \otimes R_2}) = [c]_{R_2}$.
\end{Theorem}

 \begin{proof}
We must show that $\pi_1$ and $\pi_2$ are well-defined and surjective. 

  If $[c]_{R_1 \otimes R_2} = [c']_{R_1 \otimes R_2}$, then $c' \in [c]_{R_1 \otimes R_2} = [c]_{R_1} \cap [c]_{R_2}$, so $c' \in [c]_{R_1}$, hence $[c']_{R_1} = [c]_{R_1}$. Thus $\pi_1$ is well-defined (and similarly for $\pi_2$).

 For any $[c]_{R_1} \in \mathcal{C}_{R_1}$, we have $\pi_1([c]_{R_1 \otimes R_2}) = [c]_{R_1}$, so $\pi_1$ is surjective (and similarly for $\pi_2$).
\end{proof}
 

This theorem formalizes the intuition that ``more measurement yields more geometry.'' As we add recognizers, the quotient space unfolds, revealing more detail.
  



\subsection{Symmetries and Gauge Equivalence}

Transformations are mappings of a space that change the position or state of configurations while keeping certain properties unchanged. Geometry studies what is preserved and what is distinguished under these transformations.

 

\begin{Definition}[Recognition-Preserving Map]\label{recprez}
    A transformation $T: \config \to \config$ is \textit{recognition-preserving} for $R$ if it preserves all events, i.e.
    \[ \forall c \in \config,\, R(T(c)) = R(c) \]
\end{Definition}

 

\begin{Proposition}\label{31}
    Recognition-preserving maps are closed under composition and contain the identity. Consequently, they form a monoid.
\end{Proposition}

\begin{proof}
Let $T_1, T_2$ be recognition-preserving for $R$. Then for any $c \in \mathcal{C}$,
\[ R((T_1 \circ T_2)(c)) = R(T_1(T_2(c))) = R(T_2(c)) = R(c), \]
so $T_1 \circ T_2$ is recognition-preserving. The identity map $\mathrm{id}_{\mathcal{C}}$ 
clearly satisfies $R(\mathrm{id}(c)) = R(c)$. Associativity comes from function composition.
\end{proof}


%In particular, the bijective elements of this monoid are called recognition automorphisms. 
\begin{Definition} \label{def:aut}
A \textit{recognition automorphism} is a bijective recognition-preserving map. 
The collection of all recognition automorphisms for $R$ is denoted 
$\mathrm{Aut}_R(\mathcal{C})$.
\end{Definition}

 

\begin{Proposition}
$\mathrm{Aut}_R(\mathcal{C})$ forms a group under composition.
\end{Proposition}
\begin{proof}
By the Proposition \ref{31}, $\mathrm{Aut}_R(\mathcal{C})$ is closed under 
composition and contains the identity. 

 
If $T \in \mathrm{Aut}_R(\mathcal{C})$, then $T$ is bijective, so $T^{-1}$ exists. 
For any $c \in \mathcal{C}$, let $c' = T^{-1}(c)$, so $T(c') = c$. Then
\[ R(T^{-1}(c)) = R(c') = R(T(c')) = R(c), \]
where the second equality uses that $T$ is recognition-preserving. 
Thus $T^{-1} \in \mathrm{Aut}_R(\mathcal{C})$. Associativity comes from function composition. 
\end{proof}

\begin{Theorem} 
    If $T$ is recognition-preserving, then $c_1 \sim_R c_2$ implies $T(c_1) \sim_R T(c_2)$.
\end{Theorem}
\begin{proof}
If $c_1 \sim_R c_2$, then $R(c_1) = R(c_2)$. Since $T$ is recognition-preserving,
\[ R(T(c_1)) = R(c_1) = R(c_2) = R(T(c_2)), \]
hence $T(c_1) \sim_R T(c_2)$. 
\end{proof}



In physics, a \emph{gauge transformation} is a change in the mathematical description of a system that does not affect any observable quantities, i.e. physical observables are invariant (map-preserving) under such transformations. Recognition Geometry makes this idea precise through the concept of gauge equivalence. 

\begin{Definition} 
\label{def:gauge_equivalence}
Two configurations $c_1, c_2 \in \config$ are said to be \textit{gauge equivalent}, denoted $c_1 \sim_{\text{gauge}} c_2$, if there exists a recognition automorphism $T \in \mathrm{Aut}_R(\mathcal C)$ such that
\[
T(c_1) = c_2.
\]
\end{Definition}


So, gauge equivalence defines an equivalence relation on $\config$, partitioning the configuration space into orbits under the action of $\mathrm{Aut}_R(\config)$.


\begin{Theorem} 
\label{thm:gauge_indistinguishable}
If two configurations are gauge equivalent, then they are observationally indistinguishable by Definition \ref{Indistinguishability} i.e.
\[
c_1 \sim_{\text{gauge}} c_2 \implies c_1 \sim_R c_2.
\]
\end{Theorem}

\begin{proof}
Let $T \in \mathrm{Aut}_R(\config)$ such that $T(c_1) = c_2$. By Definition (\ref{def:aut}), $R(T(c)) = R(c)$ for all $c \in \config$. Therefore, $R(c_1) = R(T(c_1)) = R(c_2)$ and consequently $c_1 \sim_R c_2$.
\end{proof}



%%%%%%%%%%%%%%%%%%
\begin{Remark}
With the definition given above (where $\mathrm{Aut}_R(\mathcal{C})$ denotes \emph{all} bijective maps $T$ satisfying $R\circ T = R$), indistinguishability and gauge equivalence coincide: if $c_1 \sim_R c_2$ then there exists a recognition automorphism mapping $c_1$ to $c_2$ (one may permute points inside the fiber $R^{-1}(R(c_1))$ while fixing all other fibers). In physical applications, however, the intended ``gauge group'' is typically a distinguished subgroup $G \subseteq \mathrm{Aut}_R(\mathcal{C})$ singled out by additional structure (e.g., dynamics, locality constraints, smoothness); then $G$-gauge equivalence can be strictly stronger than $\sim_R$. For example, let $\mathcal{C}=\{a,b,c\}$, let $R(a)=R(b)=0$ and $R(c)=1$, and take $G=\{\mathrm{id}\}$; then $a\sim_R b$ but $a$ and $b$ are not $G$-gauge equivalent.
\end{Remark} 
%%%%%%%%%%%%%%%%%%%%




\subsection{Finite Local Resolution} 
  


We now introduce the axiom that distinguishes RG from classical continuum
geometry (such as $\mathbb{R}^n$ and differentiable manifolds) and establishes
a fundamental connection to finite observational resolution. The Finite Resolution Axiom says that, locally, a recognizer can distinguish only finitely many states, while in classical geometry infinite precision is assumed.

\begin{axiom}[RG3: Finite Local Resolution]
\label{axiom:rg4}
For every configuration $c \in \config$ and recognizer $R$, there exists a
neighborhood $U \in \mathcal{N}(c)$ such that the image $R(U)$ is a finite set, i.e.$|R(U)| < \infty$.
\end{axiom}


\begin{Remark}
Axiom~\ref{axiom:rg4} means that a recognizer cannot distinguish infinitely many
different outcomes inside a single local region of the configuration space.
\end{Remark}



Axiom~\ref{axiom:rg4} has a simple consequence: if a local neighborhood is infinite, but the recognizer has finite resolution there, then $R$ cannot be injective on that neighborhood.

\begin{Theorem}[Local Non-Injectivity]
\label{thm:no-injection}
Let $c \in \config$ and let $U \in \mathcal{N}(c)$ be a neighborhood satisfying
Axiom~\ref{axiom:rg4}. If $U$ is an infinite set (i.e. contains infinitely many configurations), then the
restriction
\[
R|_U : U \to \mathcal{E}
\]
is not injective.
\end{Theorem}

\begin{proof}
Suppose, for contradiction, that $R|_U$ is injective. Then $|U| \le |R(U)|$.
By Axiom~\ref{axiom:rg4}, the set $R(U)$ is finite, whereas by assumption
$U$ is infinite. This is a contradiction. Hence $R|_U$ cannot be injective.
\end{proof}

\begin{Remark}
If an infinite neighborhood is mapped to a finite set of observable outcomes,
then different configurations must belong to the same equivalence class.
In continuum-based models, where local neighborhoods are typically infinite,
this leads to observable resolution cells rather than mathematical points.
In this sense, finite resolution gives a geometric explanation of effective
discretization: distinct configurations become observationally
indistinguishable due to limited resolution.
\end{Remark}


%%%%%%%%%%%%
%\section{Connectivity and Local Regularity}

%\subsection{Axiom RG5: Local Regularity}
%\begin{axiomenv}[RG5: Local Regularity]
%\label{axiom:rg5}
%A recognizer $R$ is \textit{locally regular} if for every configuration $c \in \mathcal{C}$,
%there exists a neighborhood $U \in \mathcal{N}(c)$ such that:
%\begin{enumerate}
  %  \item $[c]_R \cap U \neq \emptyset$ (the neighborhood actually meets the resolution cell),
%    \item $[c]_R \cap U$ is \emph{path-connected} in the subspace topology of $U$.
%\end{enumerate}
%\end{axiomenv}
%%Recall that a subset $S \subseteq U$ is \emph{path-connected} if for every pair of points 
%$x, y \in S$, there exists a continuous map $\gamma: [0,1] \to U$ (continuous with respect 
%to the topology %$\tau_{\mathcal{N}}$) such that:
%\[
%\gamma(0) = x, \quad \gamma(1) = y, \quad \text{and} \quad \gamma(t) \in S \text{ for all } t \in [0,1].
%\]
%This means one can travel from $x$ to $y$ along a continuous path that stays entirely 
%within the resolution cell $[c]_R$. The axiom excludes pathological cases where a 
%resolution cell is locally disconnected or fractal-like (e.g., Cantor dust) within 
%arbitrarily small neighborhoods.
%\end{remark}
%\begin{example}[Good vs. Bad Cases]
%\begin{itemize}
%    \item \textbf{Satisfies RG5}: A recognizer whose resolution cells are smooth 
 %         hypersurfaces in $\mathbb{R}^n$. Any small neighborhood intersecting such 
  %        a hypersurface meets it in a path-connected patch.
    
   % \item \textbf{Violates RG5}: A recognizer that partitions $\mathbb{R}^2$ into 
 %         the rational and irrational points (based on, say, $x$-coordinate being 
  %        rational). Each resolution cell is dense but totally disconnected—within 
  %        any neighborhood, the cell consists of isolated points, hence is not 
   %       path-connected.
%\end{itemize}
%\end{example}
%%%%%%%%%%%%%
\subsection{Comparative Recognizers}


In standard geometry, a distance $d(x,y)$ is usually given.
In RG, distance is not given in advance.
Instead, it is derived from a weaker structure, called
\emph{comparative recognition}.

A standard recognizer $R:\mathcal C\to\mathcal E$ assigns an event to a single
configuration and induces the observable space $\mathcal C_R$ by a quotient
construction.
In contrast, a comparative recognizer assigns an event to a pair of
configurations.
This allows us to compare configurations directly, after which notions of order and distance can be introduced.





\begin{axiom}[RG4: Comparative Recognizers]\label{ax:RG4}
A \emph{comparative recognizer} is a map
\[
\mathrm{Comp}_R : \mathcal C \times \mathcal C \longrightarrow \mathcal E
\]
such that:
\begin{enumerate}
    \item $\mathrm{Comp}_R(c,c)=e_{\mathrm{eq}}$ for all
    $c\in\mathcal C$, where $e_{\mathrm{eq}}\in\mathcal E$ is a distinguished
\emph{recognition equality event};
    \item $\mathrm{Comp}_R$ is nontrivial, i.e. there exist
    $c_1,c_2\in\mathcal C$ with
    $\mathrm{Comp}_R(c_1,c_2)\neq e_{\mathrm{eq}}$.
\end{enumerate}
\end{axiom}


As an immediate consequence of Axiom~\ref{ax:RG4}, no symmetry, transitivity,
or numerical structure is assumed for $\mathrm{Comp}_R$.
In particular, it is not required that
$\mathrm{Comp}_R(c_1,c_2)=\mathrm{Comp}_R(c_2,c_1)$.


Additional regularity conditions may be imposed later, if needed.
Comparative recognizers can be used to describe physical devices whose
output depends on comparison rather than on absolute measurement.
Typical examples include balance scales (``is object $A$ heavier than object $B$?''),
interferometers measuring relative phase, or devices comparing arrival
times of signals.




\subsection{Emergence of Order}

Given a comparative recognizer $\mathrm{Comp}_R$, some events can be
understood as indicating an order-type relation.
Let $\mathcal E_{>}\subseteq\mathcal E$ be a chosen subset of events,
interpreted as ``strictly greater than'' outcomes.

We define two binary relations:
\begin{itemize}
    \item 
    $
    c_1 \prec c_2 
    \ \Longleftrightarrow\ 
    \mathrm{Comp}_R(c_1,c_2)\in\mathcal E_{>}
    $
    \item 
    $
    c_1 \preceq c_2 
    \ \Longleftrightarrow\ 
    (c_1 \prec c_2)\ \text{or}\ (c_1 = c_2 \text{ in } \mathcal C).
    $
\end{itemize}

In general, the relations obtained in this way do not need to be
partial or total order relations.
They only reflect order-like comparison information that is accessible
at the level of recognition.

\begin{Remark}
The choice of $\mathcal E_{>}$ and its interpretation are part of the
physical modeling.
The relations $\prec$ and $\preceq$ inherit their properties directly
from the behavior of $\mathrm{Comp}_R$ on the chosen subset
$\mathcal E_{>}$.
The equality $c_1 = c_2$ refers to identity in the configuration space
$\mathcal C$, and not to operational indistinguishability.
\end{Remark}

In this way, comparative recognizers provide order-type information
based on operational comparison, before any notion of distance is
introduced.



\subsection{Emergence of Recognition Distance}

Under additional assumptions, comparative recognizers can be used to define quantitative notions of distinguishability. The basic idea is that distance is not assumed in advance, but arises as a measure of how difficult it is to distinguish configurations using available measurements.

\begin{Definition}[Recognition Distance]
A \emph{recognition distance} is a pseudometric
\[
d:\mathcal{C} \times \mathcal{C} \longrightarrow \mathbb{R}_{\ge 0}
\]
on the configuration space.
The distance $d$ is called \emph{operationally grounded} if it is constructed
from a family of comparative recognizers, in a way that reflects the
operational effort required to distinguish configurations.
\end{Definition}

 

In particular, operational grounding requires that $d(c_1,c_2)=0$
whenever all available comparative recognizers $\mathrm{Comp}_{R_i}$ satisfy
$\mathrm{Comp}_{R_i}(c_1,c_2) \in \mathcal{E}_{\mathrm{indist}}$,
where $\mathcal{E}_{\mathrm{indist}} \subseteq \mathcal{E}$ is a chosen
subset of events, possibly depending on the recognizer, interpreted as
indicating operational indistinguishability.



The precise construction of recognition distance depends on the choice of
comparative recognizers and on additional assumptions, and is not fixed
at the axiomatic level.



\begin{Remark}
We use a pseudometric since different configurations may have zero recognition distance if the
available recognizers cannot distinguish them.
This reflects finite resolution and limited observational power.
\end{Remark}

\begin{Remark}
Although a comparative recognizer itself need not be symmetric, in
physical realizations the resulting distance is typically symmetric.
This may follow from symmetry of the measuring device, from averaging
over both $\mathrm{Comp}_R(c_1,c_2)$ and $\mathrm{Comp}_R(c_2,c_1)$, or
from other symmetrization procedures applied to comparison outcomes.
\end{Remark}

\begin{Example}[Discrete recognition distance]\label{ex:disc-dist}
Let $\mathfrak{R}=\{\mathrm{Comp}_{R_i}\}_{i\in I}$ be a family of comparative
recognizers on $\mathcal{C}$. For each $i\in I$, let
$\mathcal{E}^{(i)}_{\mathrm{indist}}\subseteq \mathcal{E}$
be a subset of events interpreted as indicating operational
indistinguishability for $\mathrm{Comp}_{R_i}$. Let us define a relation $\approx$ on $\mathcal{C}$ by
\[
c_1 \approx c_2
\quad \Longleftrightarrow \quad
\mathrm{Comp}_{R_i}(c_1,c_2)\in \mathcal{E}^{(i)}_{\mathrm{indist}}
\ \text{for all } i\in I .
\]

Assume that $\approx$ is transitive. (reflexivity holds if
$\mathrm{Comp}_{R_i}(c,c)\in \mathcal{E}^{(i)}_{\mathrm{indist}}$ for all $c$
and all $i$, and symmetry holds if the family $\mathfrak{R}$ is closed under
reversal, i.e., whenever $\mathrm{Comp}_R\in\mathfrak{R}$ then also
$\mathrm{Comp}_{R^{\mathrm{rev}}}\in\mathfrak{R}$, where
$\mathrm{Comp}_{R^{\mathrm{rev}}}(c_1,c_2):=\mathrm{Comp}_R(c_2,c_1)$.)
If we define
\[
d(c_1,c_2)=
\begin{cases}
0, & \text{if } c_1 \approx c_2,\\
1, & \text{otherwise.}
\end{cases}
\]
then $d$ is a pseudometric on $\mathcal{C}$.
Moreover, $d$ is operationally grounded: $d(c_1,c_2)=0$ when none of the
available comparative recognizers can operationally distinguish $c_1$ from $c_2$,
and $d(c_1,c_2)=1$ as soon as at least one recognizer produces an outcome
indicating distinguishability.
\end{Example}

\begin{proof}
Clearly $d(c,c)=0$. For symmetry, assume $\mathfrak{R}$ is closed under reversal.
If $d(c_1,c_2)=0$, then $\mathrm{Comp}_{R_i}(c_1,c_2)\in\mathcal{E}^{(i)}_{\mathrm{indist}}$
for all $i$, hence also $\mathrm{Comp}_{R_i}(c_2,c_1)\in\mathcal{E}^{(i)}_{\mathrm{indist}}$
(using the reversed recognizers), so $d(c_2,c_1)=0$.
If $d(c_1,c_2)=1$, then for some $i$ we have
$\mathrm{Comp}_{R_i}(c_1,c_2)\notin\mathcal{E}^{(i)}_{\mathrm{indist}}$, hence
$\mathrm{Comp}_{R_i}(c_2,c_1)\notin\mathcal{E}^{(i)}_{\mathrm{indist}}$, so
$d(c_2,c_1)=1$.

For the triangle inequality, if $d(c_1,c_2)=0$ and $d(c_2,c_3)=0$, then
$c_1\approx c_2$ and $c_2\approx c_3$, hence $c_1\approx c_3$ by transitivity,
so $d(c_1,c_3)=0$. If $d(c_1,c_3)=1$, then at least one of $d(c_1,c_2)$ or
$d(c_2,c_3)$ must be $1$, otherwise both would be $0$ and transitivity   give
$d(c_1,c_3)=0$.
\end{proof}


A recognition distance measures the minimal comparative effort required
to distinguish two configurations.
In other words, it formalizes the idea of ``how hard it is to tell them apart''
as a quantitative pseudometric, for example analogous to graph distances,
where edges represent elementary distinction steps.
Concrete constructions depend on the chosen family of recognizers and
will be discussed in further work.


\begin{Example}[Balance‑scale recognizer]
A balance scale defines a comparative recognizer $\mathrm{Comp}_R(m_1,m_2)$
with event space $\mathcal{E}=\{e_{\mathrm{eq}}, e_{>}, e_{<}\}$, indicating
equal mass, left heavier, or right heavier, respectively.
Choosing $\mathcal{E}_{>}=\{e_{>}\}$ induces a binary comparison relation
on masses. If the associated indistinguishability relation (with 
$\mathcal{E}_{\mathrm{indist}} = \{e_{\mathrm{eq}}\}$) is transitive, the
construction in Example~\ref{ex:disc-dist} yields a recognition distance
that distinguishes masses in a discrete way.
\end{Example}



Every comparative recognizer $\mathrm{Comp}_R$ induces a family of standard
recognizers $R_c(x) := \mathrm{Comp}_R(c,x)$, parametrized by a reference
configuration $c$. Conversely, any standard recognizer $R$ lifts to a
comparative one via
$\mathrm{Comp}_R(c_1,c_2) = e_{\mathrm{eq}}$ if and only if $R(c_1)=R(c_2)$.

\smallskip

Comparative recognizers complete the conceptual inversion of RG: distance,
and with it geometric structure, emerges as the \emph{operational cost}
of distinguishing configurations, not as an independently given primitive.
Geometry appears only after recognition, as a secondary structure induced
by what can be operationally distinguished.

\section{Lean formalization}
An important part of the axiomatic framework presented in this paper has been formalized in the proof assistant Lean~4~\cite{Lean4}. The Lean development is intended as a \emph{claims-hygiene} layer: it forces explicit definitions, prevents hidden assumptions, and makes it easy for collaborators to check exactly which statements are proved and which are model assumptions.

At the time of writing, the formalization includes the core primitives (configuration and event spaces, locality structures, recognizers, and indistinguishability), the recognition quotient and its universal mapping property (lifting functions constant on resolution cells), composition/refinement of recognizers, finite local resolution and the corresponding non-injectivity obstruction, and symmetry/gauge constructions. A bridge module also records how Recognition Science data (ledger states, an 8-tick finite-resolution hypothesis, and a J-cost comparative functional) instantiates the abstract RG structures at the level needed for the quotient and metric constructions.

For a collaborator-facing map from the paper to Lean module and theorem names, we provide a short companion note: \texttt{docs/RG\_Lean\_Formalization\_Summary.pdf}.

%\section{Charts and Dimension}

%We endow $\quotientspace$ with the topology inherited from the projection $\pi: \config \twoheadrightarrow \quotientspace$ and use it to formulate atlas and manifold notions.

%\subsection{Quotient Topology}\label{subsec:quot-topo}

%\begin{definition}[Recognition quotient topology]
   % The \emph{recognition quotient topology} $\mathcal{T}_R$ on $\quotientspace$ is the final topology with respect to $\pi$: a subset $U \subseteq \quotientspace$ is open iff $\pi^{-1}(U)$ is open in $\config$ (with the locality structure viewed as a topology generated by neighborhoods).
%\end{definition}

%\begin{proposition}\label{prop:quot-top-final}
   % The projection $\pi: (\config, \mathcal{T}) \to (\quotientspace, \mathcal{T}_R)$ is continuous and is the universal continuous map rendering $R$ constant on fibers: any map $f: \config \to X$ that is constant on resolution cells factors uniquely through a continuous map $\tilde f: \quotientspace \to X$.
%\end{proposition}

%\begin{proof}
 %   Immediate from the definition of the final topology; uniqueness of $\tilde f$ follows because $\pi$ is surjective.
%\end{proof}
% Henceforth we regard $\quotientspace$ as a topological space with topology $\mathcal{T}_R$.

%\subsection{Recognition Charts}

%\begin{definition}[Recognition Chart]
 %   A chart $(\phi, U)$ is a map $\phi: U \to \mathbb{R}^n$ from a neighborhood $U \subset C$ such that:
  %  \begin{enumerate}
   %     \item $\phi$ respects indistinguishability: $c_1 \sim_R c_2 \implies \phi(c_1) = \phi(c_2)$
     %   \item $\phi$ is injective on resolution cells: $\phi(c_1) = \phi(c_2) \implies c_1 \sim_R c_2$
    %\end{enumerate}
%\end{definition}

 

%This defines a local coordinate system for the \textit{quotient space} $quotientspace$.

%\subsection{Recognition Atlases and Compatibility}

%\begin{definition}[Recognition Atlas]
 %   A \emph{Recognition Atlas} of dimension $n$ on $U \subseteq C$ is a family of charts $\{(\phi_i, U_i)\}_{i \in I}$ with $U \subseteq \bigcup_i U_i$, such that the induced maps
  %  \[
   %     \tilde{\phi}_i : \pi(U_i) \longrightarrow \phi_i(U_i) \subseteq \mathbb{R}^n
    %\]
    %are homeomorphisms (with $\pi(U_i)$ inheriting $\mathcal{T}_R$) and the transition maps
    %\[
     %   \tilde{\phi}_{j} \circ \tilde{\phi}_{i}^{-1} : \tilde{\phi}_i(\pi(U_i \cap U_j)) \to \tilde{\phi}_j(\pi(U_i \cap U_j))
    %\]
    %are smooth.
%\end{definition}

%\begin{remark}
 %   Under RG5 (local regularity) fibers align with neighborhoods, so $\pi(U_i)$ is open in $quotientspace$ and the induced maps $\tilde{\phi}_i$ are continuous bijections with continuous inverses.
%\end{remark}

 

%\begin{theorem}[Recognition Manifold Theorem]\label{thm:recognition-manifold}
 %   Let $(C, E, R)$ satisfy RG0--RG5. Assume:
 %   \begin{enumerate}
 %       \item The quotient $(quotientspace, \mathcal{T}_R)$ is Hausdorff and second countable.
  %      \item There exists a recognition atlas of dimension $n$ whose induced charts $\tilde{\phi}_i$ are homeomorphisms onto open subsets of $\mathbb{R}^n$ with smooth transition maps.
 %   \end{enumerate}
 %   Then $(\quotientspace, \mathcal{T}_R)$ is a smooth $n$-manifold. Moreover, the projection $\pi: \config \to \quotientspace$ is a submersion onto this manifold.
%\end{theorem}

%\begin{proof}
%    Conditions (1) and (2) match the standard characterization of smooth manifolds (see Lee~\cite{Lee}). The $\tilde{\phi}_i$ provide a smooth atlas on quotientspace, whose Hausdorff and second-countable properties ensure compatibility with the definition of smooth manifold. For the final assertion, RG5 guarantees $\pi$ has constant-rank fibers locally, so $\pi$ is a smooth submersion with respect to the constructed structure.
%\end{proof}

%\subsection{Recognition Dimension}

%\begin{definition}[Recognition Dimension]
 %   The recognition dimension at a point is the integer $n$ such that a recognition chart to $\mathbb{R}^n$ exists.
%\end{definition}

%This gives a purely operational definition of dimension:
%\begin{quote}
 %   \textbf{Dimension is the minimum number of independent recognizers needed to distinguish all local configurations.}
%\end{quote}

%If spacetime is 4-dimensional, it is because exactly 4 independent measurements (e.g., $x, y, z, t$) are required to resolve an event.

%\begin{theorem}[Fundamental Obstruction to Charts]\label{thm:no-chart}
%    If a neighborhood has finite resolution (RG4) but contains infinitely many configurations, \textbf{no recognition chart exists} on that neighborhood.
%\end{theorem}

%This theorem highlights the tension between the discrete reality of RG4 and the continuous approximation of manifolds. Manifolds are only an emergent approximation valid when the number of resolution cells is large enough to be treated as a continuum.

%\begin{theorem}[Local Dimension Uniqueness]\label{thm:dimension-uniqueness}
 %   Let $U \subseteq$ quotientspace admit two smooth atlas structures of dimensions $n$ and $m$ derived from recognizers satisfying the hypotheses of \ref{thm:recognition-manifold}. If both atlases are minimal (no subatlas of smaller dimension exists) and their inclusion maps coincide on $U$, then $n = m$.
%\end{theorem}

%\begin{proof}
 %   Assume $n < m$. The identity map $U \to U$ is a smooth, injective, open map from an $n$-manifold to an $m$-manifold. By invariance of domain (Munkres~\cite{Munkres}) such a map can exist only if $n = m$. The case $m < n$ is identical, so $n = m$.
%\end{proof}

 




%%%

\section{Conclusion}


In this paper, we presented the basic framework of Recognition Geometry (RG),
an axiomatic approach in which the observable space is not assumed in advance,
but is obtained from recognition processes.
In RG, space appears as a quotient structure induced by recognition maps.

The main points of the paper can be summarized as follows.

\begin{enumerate}
\item
We reverse the usual geometric viewpoint by taking recognition as the
primitive notion and deriving space from it, instead of starting with a
given space and defining measurements on it.

\item
We introduced a minimal axiomatic system: a nonempty configuration space,
an event space, a locality structure, and nontrivial recognizers, together with
the induced indistinguishability relation. From this, we constructed the
recognition quotient $\mathcal C_R=\mathcal C/{\sim_R}$ (resolution cells) and
the induced observable map $\overline R:\mathcal C_R\to\mathcal E$. We also described how the locality
structure generates a topology on $\mathcal C$ and induces the quotient topology
on $\mathcal C_R$.
 

\item
We described the recognition triple $(\mathcal{C}, \mathcal{E}, \mathcal{S})$
with $\mathcal{S}=(\mathcal{N},\Sigma)$ and its role in constructing the
observable space.


\item
Several examples were given to illustrate the framework, including threshold
recognizers on $\mathbb{R}^n$, discrete lattice recognizers, quantum spin
measurements, and examples from Recognition Science, where physical space
emerges as a quotient structure.

\item
We developed comparative recognizers and used them to define
order-type relations and recognition distances as pseudometrics derived
from operational distinguishability.

\end{enumerate}

The main message of RG is that space is not given in advance, but is obtained
through recognition.



%%%%%%%%%%%%%%%%%%%%%%%







 













%%%%%%%%%%%%%%%%%%%%%%%%%%%%%%%%%%%%%%%%%%
\authorcontributions{Conceptualization, J.W.; methodology, , M.Z. and E.A.; software, J.W.; validation, J.W., M.Z. and E.A.; formal analysis, J.W.; investigation, J.W., M.Z. and E.A.; writing—original draft preparation, J.W., M.Z. and E.A.; 
writing—review and editing, M.Z. and E.A.; 
visualization, J.W.; supervision, J.W.; 
project administration, M.Z.; funding acquisition, J.W.
All authors have read and agreed to the published version of the manuscript.}





\institutionalreview{Not applicable.}

\informedconsent{Not applicable.}

\dataavailability{Data are contained within the article.} 

%\durcstatement{Current research is limited to the [please insert a specific academic field, e.g., XXX], which is beneficial [share benefits and/or primary use] and does not pose a threat to public health or national security. Authors acknowledge the dual-use potential of the research involving xxx and confirm that all necessary precautions have been taken to prevent potential misuse. As an ethical responsibility, authors strictly adhere to relevant national and international laws about DURC. Authors advocate for responsible deployment, ethical considerations, regulatory compliance, and transparent reporting to mitigate misuse risks and foster beneficial outcomes.}

% Only for journal Nursing Reports
%\publicinvolvement{Please describe how the public (patients, consumers, carers) were involved in the research. Consider reporting against the GRIPP2 (Guidance for Reporting Involvement of Patients and the Public) checklist. If the public were not involved in any aspect of the research add: ``No public involvement in any aspect of this research''.}
%
%% Only for journal Nursing Reports
%\guidelinesstandards{Please add a statement indicating which reporting guideline was used when drafting the report. For example, ``This manuscript was drafted against the XXX (the full name of reporting guidelines and citation) for XXX (type of research) research''. A complete list of reporting guidelines can be accessed via the equator network: \url{https://www.equator-network.org/}.}
%
%% Only for journal Nursing Reports
%\useofartificialintelligence{Please describe in detail any and all uses of artificial intelligence (AI) or AI-assisted tools used in the preparation of the manuscript. This may include, but is not limited to, language translation, language editing and grammar, or generating text. Alternatively, please state that “AI or AI-assisted tools were not used in drafting any aspect of this manuscript”.}



\conflictsofinterest{``The authors declare no conflicts of interest.'' } 

 %
%% Optional%%%%%%%%%%%%%%%%%%%%%%%
%\isPreprints{} % If the paper is ``preprints'', please uncomment this parenthesis.
%\printendnotes[custom] % Un-comment to print a list of endnotes

\reftitle{References}

% Please provide the correct journal abbreviation (e.g. according to the “List of Title Word Abbreviations” http://www.issn.org/services/online-services/access-to-the-ltwa/).
% Citations and References in Supplementary files are permitted provided that they also appear in the reference list here. 

%=====================================
% References, variant A: external bibliography
%=====================================
% \bibliography{your_external_BibTeX_file}

%=====================================
% References, variant B: internal bibliography
%=====================================

% ACS format
\begin{thebibliography}{999}

\bibitem{Abramsky} Abramsky, S., Coecke, B. \textit{A Categorical Semantics of Quantum Protocols}. In: Proceedings of LICS 2004, IEEE Computer Society, 2004.

\bibitem{Adamek} Ad\'amek, J., Herrlich, H., Strecker, G.E. \textit{Abstract and Concrete Categories: The Joy of Cats}. Wiley, 1990.

\bibitem{Amari1985} Amari, S. \textit{Differential-Geometrical Methods in Statistics}. Lecture Notes in Statistics 28, Springer, 1985.

\bibitem{InfoGeom} Amari, S. \textit{Information Geometry and Its Applications}. Springer, 2016.

\bibitem{Bratteli} Bratteli, O., Robinson, D.W. \textit{Operator Algebras and Quantum Statistical Mechanics}. Vol. 1, 2nd ed., Springer, 1987.

\bibitem{Busch} Busch, P., Lahti, P., Pellonpää, J.-P., Ylinen, K. \textit{Quantum Measurement}. Springer, 2016.

\bibitem{Chentsov} Chentsov, N.N. \textit{Statistical Decision Rules and Optimal Inference}. American Mathematical Society, 1982.

\bibitem{Coecke} Coecke, B. \textit{Quantum Picturalism}. Contemp. Phys. 51, 59--83, 2010.

\bibitem{NCG} Connes, A. \textit{Noncommutative Geometry}. Academic Press, 1994.

\bibitem{Topos} Döring, A., Isham, C., \textit{A Topos Foundation for Theories of Physics}, J.Math.Phys. 49 (2008) 053515.

\bibitem{Frieden} Frieden, B.R. \textit{Physics from Fisher Information}. Cambridge University Press, 1998.

\bibitem{Fuchs} Fuchs, C.A., Schack, R. \textit{Quantum-Bayesian Coherence}. Rev. Mod. Phys. 85, 1693--1715, 2013.

\bibitem{Hardy} Hardy, L. \textit{Quantum Theory From Five Reasonable Axioms}. arXiv:quant-ph/0101012, 2001.

\bibitem{Jaynes} Jaynes, E.T. \textit{Probability Theory: The Logic of Science}. Cambridge University Press, 2003.

\bibitem{Johnstone} Johnstone, P.T. \textit{Stone Spaces}. Cambridge Studies in Advanced Mathematics 3, Cambridge University Press, 1982.

\bibitem{Lawvere} Lawvere, F.W., Schanuel, S.H. \textit{Conceptual Mathematics: A First Introduction to Categories}. Cambridge University Press, 2009.

\bibitem{Lee} Lee, J.M. \textit{Introduction to Smooth Manifolds}. 3rd ed., Springer, 2013.

\bibitem{Lean4} Moura, L.d., Ullrich, S. (2021), {\it The Lean 4 Theorem Prover and Programming Language,} In: Platzer, A., Sutcliffe, G. (eds) Automated Deduction – CADE 28. CADE 2021., Lecture Notes in Computer Science(), vol 12699. Springer, Cham.

\bibitem{Lurie} Lurie, J. \textit{Higher Topos Theory}. Princeton University Press, 2009.

\bibitem{MacLane} Mac Lane, S. \textit{Categories for the Working Mathematician}. Springer, 2nd ed., 1998.

\bibitem{Munkres} Munkres, J. \textit{Topology}. 2nd ed., Prentice Hall, 2000.

\bibitem{Penrose} Penrose, R. \textit{The Road to Reality}. Jonathan Cape, 2004.

\bibitem{RQM} Rovelli, C. \textit{Relational Quantum Mechanics}. Int. J. Theor. Phys. 35, 1996.

\bibitem{LQG} Rovelli, C. \textit{Quantum Gravity}. Cambridge University Press, 2004.

\bibitem{Sorkin} Sorkin, R.D. \textit{Causal Sets: Discrete Gravity}. In \textit{Lectures on Quantum Gravity}, Springer, 2005.

\bibitem{Verlinde} Verlinde, E. \textit{On the Origin of Gravity and the Laws of Newton}. JHEP 04, 2011.

\bibitem{vonNeumann} von Neumann, J. \textit{Mathematical Foundations of Quantum Mechanics}. Princeton University Press, 1955.

\bibitem{WheelerZurek} Wheeler, J.A., Zurek, W.H. (eds.) \textit{Quantum Theory and Measurement}. Princeton University Press, 1983. 

\bibitem{Wolfram} Wolfram, S. \textit{A New Kind of Science}. Wolfram Media, 2002.

\bibitem{Zlat} Stankovi\' c, M., Zlatanovi\' c, M. 
\textit{Non-Euclidean geomety}. Faculty of Sciences and Mathematics, Ni\v s, 2016.























 

\end{thebibliography}


% If authors have biography, please use the format below
%\section*{Short Biography of Authors}
%\bio
%{\raisebox{-0.35cm}{\includegraphics[width=3.5cm,height=5.3cm,clip,keepaspectratio]{Definitions/author1.pdf}}}
%{\textbf{Firstname Lastname} Biography of first author}
%
%\bio
%{\raisebox{-0.35cm}{\includegraphics[width=3.5cm,height=5.3cm,clip,keepaspectratio]{Definitions/author2.jpg}}}
%{\textbf{Firstname Lastname} Biography of second author}

% For the MDPI journals use author-date citation, please follow the formatting guidelines on http://www.mdpi.com/authors/references
% To cite two works by the same author: \citeauthor{ref-journal-1a} (\citeyear{ref-journal-1a}, \citeyear{ref-journal-1b}). This produces: Whittaker (1967, 1975)
% To cite two works by the same author with specific pages: \citeauthor{ref-journal-3a} (\citeyear{ref-journal-3a}, p. 328; \citeyear{ref-journal-3b}, p.475). This produces: Wong (1999, p. 328; 2000, p. 475)

%%%%%%%%%%%%%%%%%%%%%%%%%%%%%%%%%%%%%%%%%%
%% for journal Sci
%\reviewreports{\\
%Reviewer 1 comments and authors’ response\\
%Reviewer 2 comments and authors’ response\\
%Reviewer 3 comments and authors’ response
%}
%%%%%%%%%%%%%%%%%%%%%%%%%%%%%%%%%%%%%%%%%%
\PublishersNote{}
%\isPreprints{} % If the paper is ``preprints'', please uncomment this parenthesis.
\end{document}

