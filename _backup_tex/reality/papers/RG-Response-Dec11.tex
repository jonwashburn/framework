\documentclass[11pt]{amsart}
\usepackage[colorlinks=true,linkcolor=blue,citecolor=blue]{hyperref}
\usepackage{amsfonts,amsmath,amsthm,amssymb}
\usepackage[margin=1.2in]{geometry}

\newtheorem{theorem}{Theorem}[section]
\newtheorem{definition}{Definition}[section]
\newtheorem{example}{Example}[section]
\newtheorem{remark}{Remark}[section]

\title{Response to Recognition Geometry Comments}
\author{Jonathan Washburn}
\date{December 11, 2025}

\begin{document}
\maketitle

\section{Administrative Points}

\subsection{Lean Code Placement}
\textbf{Comment}: Move all Lean code to an appendix.

\textbf{Response}: Agreed. The appendix will be organized as:
\begin{itemize}
    \item[A.] Core definitions
    \item[B.] Quotient construction
    \item[C.] Composition and refinement
    \item[D.] Finite resolution theorems
    \item[E.] RS Bridge
\end{itemize}

\subsection{References for Section 2}
\textbf{Comment}: Section 2 needs references.

\textbf{Response}: Suggested citations by subsection:

\begin{center}
\begin{tabular}{|l|l|}
\hline
\textbf{Topic} & \textbf{Reference} \\
\hline
Configuration spaces & Penrose \cite{Penrose} \\
Topological structure & Munkres \cite{Munkres} \\
Smooth manifolds & Lee \cite{Lee} \\
Measurement theory & Busch et al.\ \cite{Busch} \\
Categorical perspective & Mac Lane \cite{MacLane} \\
Relational interpretation & Rovelli \cite{RQM} \\
\hline
\end{tabular}
\end{center}

\subsection{Additional Examples}
\textbf{Comment}: Example 2.1 is important. More examples welcome.

\textbf{Response}: Three additional examples recommended:

\begin{example}[Discrete Lattice]
$\mathcal{C} = \mathbb{Z}^3$, $\mathcal{E} = \{0,1\}$, $R(x,y,z) = (x+y+z) \mod 2$. 
Quotient has 2 points. Shows RG applies without continuity.
\end{example}

\begin{example}[Quantum Spin]
$\mathcal{C} = S^2$ (Bloch sphere), $\mathcal{E} = \{\pm 1\}$. The $z$-spin recognizer partitions into hemispheres. Adding $R_x, R_y$ refines but never recovers full $S^2$---illustrating finite resolution.
\end{example}

\begin{example}[RS Instantiation]
$\mathcal{C} = \mathcal{L}$ (ledger), $\mathcal{E} = \mathbb{R}^3$. Position recognizer: $R_{\text{pos}}(\ell) = (x,y,z)$. Physical space \emph{is} $\mathcal{L}/\!\sim_{R}$.
\end{example}

\subsection{RG3 as Definition}
\textbf{Comment}: Indistinguishability is a definition, not an axiom.

\textbf{Response}: Correct. The relation $c_1 \sim_R c_2 \Leftrightarrow R(c_1) = R(c_2)$ inherits equivalence properties from equality. The Lean code treats it as \texttt{def}, not \texttt{axiom}. Paper should list RG0--2, RG4--7 as axioms; RG3 as definition.

%----------------------------------------------------------------------
\section{Conceptual Questions}
%----------------------------------------------------------------------

\subsection{Total vs.\ Partial Recognizers}
\textbf{Comment}: Should recognizers be total? Physical detectors have range limits.

\textbf{Response}: Keep total recognizers for the foundational paper. The key points:

\begin{itemize}
    \item \textbf{Partial recognizers}: $R : \text{dom}(R) \subseteq \mathcal{C} \to \mathcal{E}$. Quotient restricted to domain. Straightforward extension.
    
    \item \textbf{Stochastic recognizers}: $R : \mathcal{C} \to \mathcal{P}(\mathcal{E})$. Connects to POVMs. More radical---requires distributional equivalence.
    
    \item \textbf{RS context}: The 8-tick cycle ensures RS recognizers are always total. Finite resolution (RG4) handles detector limits.
\end{itemize}

Add a remark noting these extensions for future work.

\subsection{Formalizing $\Sigma$}
\textbf{Comment}: The structure $\Sigma$ is not formalized. Without this, the Recognition Triple remains metaphorical.

\textbf{Response}: Valid concern. Here is a precise formalization:

\begin{definition}[Recognition Structure]
A \emph{Recognition Structure} on $(\mathcal{C}, \mathcal{E})$ is a tuple $\Sigma = (\mathcal{N}, R)$ where:
\begin{enumerate}
    \item $\mathcal{N} : \mathcal{C} \to \mathcal{P}(\mathcal{P}(\mathcal{C}))$ is a locality structure (RG1)
    \item $R : \mathcal{C} \to \mathcal{E}$ is a recognizer satisfying finite resolution (RG4)
\end{enumerate}
\end{definition}

\begin{definition}[Recognition Triple]
A \emph{Recognition Triple} is $(\mathcal{C}, \mathcal{E}, \Sigma)$ where $\mathcal{C}$ is nonempty (RG0), $|\mathcal{E}| \geq 2$, and $\Sigma$ is a recognition structure.
\end{definition}

The Lean formalization already captures this:
\begin{verbatim}
structure RecognitionGeometry (C E : Type*) where
  locality : LocalConfigSpace C
  recognizer : Recognizer C E
  finite_resolution : HasFiniteResolution locality recognizer
\end{verbatim}

The general case (families of recognizers) is recovered via composition (RG6).

\textbf{Recommendation}: Add Section 2.5 with this definition.

%----------------------------------------------------------------------
\section{Summary of Changes}
%----------------------------------------------------------------------

\begin{enumerate}
    \item Move Lean code to Appendix A--E
    \item Add citations per table above
    \item Include 3 additional examples (discrete, quantum, RS)
    \item Confirm RG3 labeled as Definition
    \item Add remark on partial/stochastic extensions
    \item Add Section 2.5 formalizing $\Sigma$
\end{enumerate}

%----------------------------------------------------------------------
\begin{thebibliography}{99}

\bibitem{Munkres} Munkres, J. \textit{Topology}. Prentice Hall, 2000.

\bibitem{Lee} Lee, J.M. \textit{Introduction to Smooth Manifolds}. Springer, 2013.

\bibitem{Penrose} Penrose, R. \textit{The Road to Reality}. Jonathan Cape, 2004.

\bibitem{Busch} Busch, P. et al. \textit{Quantum Measurement}. Springer, 2016.

\bibitem{MacLane} Mac Lane, S. \textit{Categories for the Working Mathematician}. Springer, 1998.

\bibitem{RQM} Rovelli, C. Relational Quantum Mechanics. \textit{Int.\ J.\ Theor.\ Phys.}\ 35, 1996.

\end{thebibliography}

\end{document}
