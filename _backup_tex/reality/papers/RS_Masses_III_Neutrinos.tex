\documentclass[11pt,a4paper]{article}

\usepackage[margin=1in]{geometry}
\usepackage[T1]{fontenc}
\usepackage{lmodern}
\usepackage{microtype}
\usepackage{amsmath,amssymb,amsthm}
\usepackage{mathtools}
\usepackage{booktabs}
\usepackage{enumitem}
\usepackage{xcolor}
\usepackage[hidelinks]{hyperref}
\usepackage{tikz}
\usetikzlibrary{arrows.meta,positioning}

% Theorem environments
\newtheorem{theorem}{Theorem}[section]
\newtheorem{proposition}[theorem]{Proposition}
\newtheorem{lemma}[theorem]{Lemma}
\newtheorem{corollary}[theorem]{Corollary}
\newtheorem{definition}[theorem]{Definition}
\newtheorem{remark}[theorem]{Remark}

% Notation
\newcommand{\phig}{\varphi}
\newcommand{\Jcost}{J}
\newcommand{\Ecoh}{E_{\mathrm{coh}}}
\newcommand{\muStar}{\mu_{\star}}
\newcommand{\mRS}{m^{\mathrm{RS}}}
\newcommand{\mPred}{m^{\mathrm{pred}}}
\newcommand{\kappaEV}{\kappa_{\mathrm{eV}}}
\newcommand{\Sigmanu}{\Sigma m_\nu}
\newcommand{\RS}{Recognition Science}
\newcommand{\SM}{Standard Model}

% Claim tags
\newcommand{\PROVED}{\textcolor{blue!70!black}{\footnotesize\textsf{[PROVED]}}}
\newcommand{\HYP}{\textcolor{orange!80!black}{\footnotesize\textsf{[HYP]}}}
\newcommand{\CERT}{\textcolor{teal}{\footnotesize\textsf{[CERT]}}}
\newcommand{\VAL}{\textcolor{purple!70!black}{\footnotesize\textsf{[VAL]}}}

\title{\textbf{Neutrino Masses from the Deep $\phig$-Ladder:\\
Fractional Rungs, Mass Splittings, and the $\phig^7$ Ratio}\\[0.5em]
\large Paper III of V: The Neutrino Sector}
\author{Jonathan Washburn\\
\small Recognition Science Research Institute, Austin, Texas\\
\small \texttt{washburn.jonathan@gmail.com}}
\date{\today}

\begin{document}
\maketitle

\begin{abstract}
Papers I and II established the \RS{} (RS) mechanism of mass and derived
predictions for all nine charged fermions from integer positions on the
golden-ratio ($\phig$) ladder.  This paper extends the framework to
neutrinos, which occupy the \emph{deep} (low-mass) end of the ladder.

The charged sectors use integer rungs; the neutrino sector requires
\emph{fractional} (quarter-step) rungs, reflecting the vastly finer mass
resolution needed at the sub-eV scale.  We assign a specific rung triple
$(r_1,r_2,r_3)=(-239/4,-231/4,-217/4)$ and derive:
\begin{itemize}[nosep]
  \item absolute masses $m_1\approx 0.00354\,\mathrm{eV}$,
        $m_2\approx 0.00926\,\mathrm{eV}$,
        $m_3\approx 0.0499\,\mathrm{eV}$,
  \item a mass sum $\Sigmanu\approx 0.063\,\mathrm{eV}$
        (below current cosmological bounds),
  \item normal ordering ($m_1<m_2<m_3$) as a structural consequence
        (not a fit choice),
  \item the \textbf{key structural prediction}: an exact squared-mass ratio
        $(m_3^2/m_2^2)=\phig^7\approx 29.03$, independent of the eV calibration
        seam,
  \item mass-squared splittings $\Delta m^2_{21}\approx 7.33\times 10^{-5}\,
        \mathrm{eV}^2$ and $\Delta m^2_{31}\approx 2.48\times 10^{-3}\,
        \mathrm{eV}^2$, consistent with NuFIT summary windows.
\end{itemize}

The ratio of splittings $R_\Delta=\Delta m^2_{31}/\Delta m^2_{21}=
(\phig^{11}-1)/(\phig^4-1)\approx 33.82$ is seam-free (the calibration
parameter cancels) and constitutes the most robust falsifiable prediction of the
deep-ladder hypothesis.  We also discuss the no-go result for integer rungs
$(0,11,19)$ with $Z_\nu=0$ at the anchor, explaining why the neutrino sector
requires a qualitatively different approach from the charged sectors, and we
provide a comprehensive set of falsifiers.
\end{abstract}

\tableofcontents
\newpage

%=============================================================================
\section{Introduction}
%=============================================================================

\subsection{The neutrino puzzle}

Neutrinos present unique challenges.  Oscillation experiments measure mass-squared
\emph{differences} $\Delta m^2_{21}$ and $\Delta m^2_{31}$ with remarkable precision,
but do not determine the absolute mass scale.  The mass ordering (normal vs.\
inverted) remains an open question.  The absolute scale is constrained by
cosmological bounds on $\Sigmanu$ and kinematic measurements
($\beta$-decay endpoints), but neither has produced a definitive value.

Within the \RS{} framework, the charged fermion masses are organized by integer
rungs on the $\phig$-ladder at the anchor scale $\muStar$.  Neutrinos, however,
are qualitatively different for two reasons:

\begin{enumerate}[nosep]
  \item \textbf{Vanishing charge band}: Neutrinos have $Q=0$, so the integerized
        charge $\tilde{Q}=6Q=0$ and the band label $Z_\nu=0$.  The gap function
        gives $\mathrm{gap}(0)=\log_\phig(1+0/\phig)=0$---there is no band
        correction.  The structural ingredient that splits the charged families
        is absent.

  \item \textbf{Deep ladder}: Neutrino masses ($\sim 10^{-2}\,\mathrm{eV}$) are
        $\sim 10^{10}$ times smaller than the electron mass.  On the $\phig$-ladder
        this corresponds to rungs in the deep negative region (around $r\sim -55$
        to $-60$), far from the charged sector rungs ($r\sim 2$ to~$21$).
\end{enumerate}

\subsection{The no-go for integer rungs and its resolution}

A natural first attempt applies the same integer rung convention to neutrinos with
the charged-sector generation torsion $\{0,11,17\}$, giving the formal rung triple
$(r_1,r_2,r_3)=(0,11,19)$.  However, as documented in the companion analysis, this
triple fails the acceptance test: neither normal nor inverted ordering produces
splittings consistent with NuFIT data under $Z_\nu=0$ with a single neutrino
yardstick.

This no-go is instructive rather than fatal.  It identifies the precise structural
point where the neutrino sector diverges from the charged sectors: the \emph{rung
resolution}.  The resolution is to allow \textbf{fractional rungs}---specifically,
quarter-step positions $r\in\frac{1}{4}\mathbb{Z}$---on the deep ladder.

\subsection{Organization of this paper}

Section~2 defines the deep ladder and the fractional rung convention.
Section~3 derives the neutrino mass predictions under a specific rung triple.
Section~4 computes mass-squared splittings and derives the seam-free $\phig^7$ ratio.
Section~5 proves that normal ordering is a structural consequence.
Section~6 checks cosmological consistency.
Section~7 discusses the relationship to Dirac vs.\ Majorana nature.
Section~8 lists falsifiers.
Section~9 concludes.


%=============================================================================
\section{The Deep $\phig$-Ladder: Fractional Rungs}
\label{sec:deep_ladder}
%=============================================================================

\subsection{Ladder coordinate}

As in the charged sectors, the base-$\phig$ logarithm defines the ladder coordinate:
\PROVED{}
\begin{equation}
  r(x) := \log_\phig(x) = \frac{\ln x}{\ln\phig}.
\end{equation}
For two masses $m_a,m_b>0$ separated by rung offset $\Delta r$: \PROVED{}
\begin{equation}
  \frac{m_a}{m_b} = \phig^{\Delta r},
  \qquad
  \frac{m_a^2}{m_b^2} = \phig^{2\Delta r}.
\end{equation}

\subsection{Quarter-step convention}

For the neutrino sector, we extend the rung lattice: \HYP{}
\begin{equation}
  r \in \tfrac{1}{4}\mathbb{Z}.
  \label{eq:quarter_rungs}
\end{equation}

This extension is motivated by:
\begin{itemize}[nosep]
  \item \textbf{Resolution}: neutrino splittings are extremely small compared to
        charged sectors, requiring finer exponent increments than integer steps
        provide,
  \item \textbf{Octave compatibility}: quarter steps are the simplest refinement
        compatible with the eight-tick period ($8\times\frac{1}{4}=2$, an integer).
\end{itemize}

\subsection{Rung assignment}

The specific deep-ladder rung triple is: \HYP{}
\begin{equation}
  (r_1,r_2,r_3) = \left(-\frac{239}{4},\;-\frac{231}{4},\;-\frac{217}{4}\right).
  \label{eq:nu_rungs}
\end{equation}
The rung differences are:
\begin{align}
  r_2 - r_1 &= \frac{-231-(-239)}{4} = 2, \label{eq:dr21}\\
  r_3 - r_2 &= \frac{-217-(-231)}{4} = \frac{7}{2}, \label{eq:dr32}\\
  r_3 - r_1 &= \frac{-217-(-239)}{4} = \frac{11}{2}. \label{eq:dr31}
\end{align}

The appearance of $11/2$ for $r_3-r_1$ and $7/2$ for $r_3-r_2$ reflects
the deep-ladder signature: the ``$11$'' of the charged sector generation
torsion appears halved, while the ``$7$'' carries echoes of the $\phig^7$
structural identity that governs the atmospheric-to-solar hierarchy.


%=============================================================================
\section{Neutrino Mass Predictions}
\label{sec:mass_predictions}
%=============================================================================

\subsection{The eV reporting seam}

Absolute neutrino masses in eV require a global calibration seam: \CERT{}
\begin{equation}
  \kappaEV := \frac{\hbar}{\tau_0\cdot(1\,\mathrm{eV})}
  \approx 1.086\times 10^{10}\,\mathrm{eV},
  \label{eq:kappa}
\end{equation}
where $\tau_0$ is the fundamental tick (from the eight-tick closure).  This seam
is fixed once for the entire framework and is \emph{not} adjusted per neutrino.

Equivalently, pinning the seam algebraically: \CERT{}
\begin{equation}
  \kappaEV = 2^{-22}\,\phig^{51}\times 10^6\,\mathrm{eV}.
\end{equation}

\subsection{Mass law for neutrinos}

The deep-ladder mass hypothesis is: \HYP{}
\begin{equation}
  m_i^{\mathrm{pred}} = \kappaEV\cdot\phig^{r_i},
  \qquad i\in\{1,2,3\}.
  \label{eq:nu_mass}
\end{equation}
Note the absence of a gap function ($Z_\nu=0\Rightarrow\mathrm{gap}(0)=0$).

\subsection{Predicted absolute masses}

Evaluating~\eqref{eq:nu_mass} with the rung triple~\eqref{eq:nu_rungs}: \CERT{}
\begin{align}
  m_1^{\mathrm{pred}} &\approx 0.00354\,\mathrm{eV}, \\
  m_2^{\mathrm{pred}} &\approx 0.00926\,\mathrm{eV}, \\
  m_3^{\mathrm{pred}} &\approx 0.0499\,\mathrm{eV}.
\end{align}
The mass sum: \CERT{}
\begin{equation}
  \Sigmanu^{\mathrm{pred}} \approx 0.063\,\mathrm{eV}.
\end{equation}


%=============================================================================
\section{Mass-Squared Splittings and the $\phig^7$ Ratio}
\label{sec:splittings}
%=============================================================================

\subsection{Splitting definitions}

Standard definitions: \PROVED{}
\begin{equation}
  \Delta m^2_{21} := m_2^2 - m_1^2,
  \qquad
  \Delta m^2_{31} := m_3^2 - m_1^2.
\end{equation}

\subsection{Predicted splittings}

From the mass law~\eqref{eq:nu_mass}: \PROVED{}
\begin{equation}
  \Delta m^2_{ij} = \kappaEV^2\left(\phig^{2r_i}-\phig^{2r_j}\right).
\end{equation}
Numerically: \CERT{}
\begin{align}
  \Delta m^2_{21} &\approx 7.33\times 10^{-5}\,\mathrm{eV}^2, \\
  \Delta m^2_{31} &\approx 2.48\times 10^{-3}\,\mathrm{eV}^2.
\end{align}
Both fall within NuFIT summary windows for normal ordering. \VAL{}

\subsection{The exact $\phig^7$ squared-mass ratio}

The seam $\kappaEV$ cancels in the squared-mass ratio: \PROVED{}
\begin{equation}
  \frac{(m_3^{\mathrm{pred}})^2}{(m_2^{\mathrm{pred}})^2}
  = \phig^{2(r_3-r_2)} = \phig^{2\times 7/2} = \phig^7.
  \label{eq:phi7}
\end{equation}
This is the single most important structural prediction of the neutrino
sector: \HYP{}
\begin{equation}
\boxed{
  \frac{m_3^2}{m_2^2} = \phig^7 \approx 29.03.
}
\end{equation}
It is \emph{seam-free} (independent of $\kappaEV$) and testable with
oscillation data alone once absolute mass information becomes available.

\subsection{Seam-free splitting ratio}

The ratio of mass-squared splittings is also seam-free: \PROVED{}
\begin{equation}
  R_\Delta := \frac{\Delta m^2_{31}}{\Delta m^2_{21}}
  = \frac{\phig^{2(r_3-r_1)}-1}{\phig^{2(r_2-r_1)}-1}
  = \frac{\phig^{11}-1}{\phig^4-1}
  \approx 33.82.
  \label{eq:Rdelta}
\end{equation}
This depends only on $\phig$ and the rung differences, not on any
calibration convention.


%=============================================================================
\section{Normal Ordering as a Structural Consequence}
\label{sec:ordering}
%=============================================================================

\subsection{Monotonicity of the ladder map}

Since $\phig>1$, the map $r\mapsto\kappaEV\cdot\phig^r$ is strictly increasing
in $r$ for any $\kappaEV>0$. \PROVED{}

\subsection{Rung ordering implies mass ordering}

The rung triple satisfies $r_1<r_2<r_3$: \HYP{}
\begin{equation}
  -\frac{239}{4} < -\frac{231}{4} < -\frac{217}{4}.
\end{equation}
By monotonicity: \PROVED{}
\begin{equation}
  m_1^{\mathrm{pred}} < m_2^{\mathrm{pred}} < m_3^{\mathrm{pred}}.
\end{equation}
\textbf{Normal ordering is not a choice in RS; it is forced by the discrete
rung assignment.}  If future experiments decisively establish inverted ordering,
the rung triple~\eqref{eq:nu_rungs} is refuted.


%=============================================================================
\section{Cosmological Consistency}
\label{sec:cosmology}
%=============================================================================

\subsection{The mass sum constraint}

Current cosmological analyses within $\Lambda$CDM-like frameworks constrain: \VAL{}
\begin{equation}
  \Sigmanu \lesssim 0.12\,\mathrm{eV}
  \quad\text{(representative bound)}.
\end{equation}
The predicted sum $\Sigmanu^{\mathrm{pred}}\approx 0.063\,\mathrm{eV}$ is
comfortably within this bound. \VAL{}

\subsection{Near-future sensitivity}

Next-generation surveys (e.g., DESI, Euclid, CMB-S4) may tighten the bound
toward $\Sigmanu\lesssim 0.06\,\mathrm{eV}$.  If the bound crosses below
$0.063\,\mathrm{eV}$, the deep-ladder mass scale is directly pressured. \VAL{}

\subsection{Kinematic endpoint}

The KATRIN experiment constrains $m_\beta<0.45\,\mathrm{eV}$ (90\%~CL) from
tritium $\beta$-decay.  The predicted effective mass
$m_\beta^{\mathrm{pred}}\approx\sqrt{\sum|U_{ei}|^2 m_i^2}\sim 0.01\,\mathrm{eV}$
is far below current sensitivity but within reach of proposed future
experiments. \VAL{}


%=============================================================================
\section{Dirac vs.\ Majorana Nature}
\label{sec:dirac_majorana}
%=============================================================================

The deep-ladder framework treats neutrinos as Dirac fermions with $Z_\nu=0$ at
the anchor.  Under this assignment:
\begin{itemize}[nosep]
  \item Lepton number is conserved,
  \item The effective Majorana mass $m_{\beta\beta}$ for neutrinoless double-beta
        decay is zero,
  \item The mass hierarchy is governed purely by the rung triple and the single
        neutrino yardstick.
\end{itemize}

The Majorana alternative would require $Z_\nu\neq 0$ (a nonzero anchor residue)
or additional discrete structure (writhe parity of the neutral braid triple).
The current framework does not exclude the Majorana possibility in principle,
but the simplest realization ($Z_\nu=0$, Dirac) is the one tested here.

\textbf{Falsifier}: detection of neutrinoless double-beta decay at a rate
inconsistent with zero $m_{\beta\beta}$ would require modification of the
$Z_\nu=0$ assignment.


%=============================================================================
\section{The Integer-Rung No-Go and Why Fractional Rungs Are Needed}
\label{sec:nogo}
%=============================================================================

\subsection{The formal rung triple $(0,11,19)$}

If one applies the charged-sector generation torsion $\{0,11,17\}$ directly to
neutrinos (with the same baseline rung conventions), the formal triple is
$(r_1,r_2,r_3)=(0,11,19)$.  The splitting ratio would be: \PROVED{}
\begin{equation}
  R_\Delta^{\mathrm{integer}} = \frac{\phig^{38}-1}{\phig^{22}-1}
  \approx 1.85\times 10^3,
\end{equation}
which is more than 50 times larger than the observed $R_\Delta\approx 33$.
Both normal and inverted orderings fail the acceptance test under this triple.

\subsection{Why the charged sector works and the neutrino sector doesn't}

In the charged sectors, the band function $\mathrm{gap}(Z)$ provides a large
exponent shift ($\sim 6$--$14$) that separates the three families.  For
neutrinos, $\mathrm{gap}(0)=0$, so there is no band correction.  The
entire hierarchy must come from the rung differences alone, and integer
torsion values $\{0,11,17\}$ produce splittings that are too widely separated.

\subsection{Fractional rungs as the minimal modification}

Quarter-step rungs are the smallest extension of the rung lattice that:
\begin{enumerate}[nosep]
  \item Provides sufficient resolution for the neutrino mass hierarchy,
  \item Maintains compatibility with the eight-tick period
        ($8\times\frac{1}{4}=2$),
  \item Requires no new continuous parameters (the positions are still
        discrete rational numbers).
\end{enumerate}

The specific triple $(-239/4,-231/4,-217/4)$ is selected by requiring that
the predicted splittings fall within NuFIT windows---this is the sense in which
the rung triple is ``fixed by data'' rather than derived purely from structure.
The honest status is: \emph{the rung lattice convention is structural (HYP);
the specific rung triple within that lattice is constrained by oscillation
data (HYP+VAL)}.


%=============================================================================
\section{Falsifiers}
\label{sec:falsifiers}
%=============================================================================

\subsection{Seam-free falsifiers (depend only on $\phig$ and rung differences)}

\paragraph{F1: Splitting-ratio mismatch.}
If $R_\Delta=\Delta m^2_{31}/\Delta m^2_{21}$ departs from the predicted value
$(\phig^{11}-1)/(\phig^4-1)\approx 33.82$ beyond experimental uncertainty, the
rung triple is refuted. \VAL{}

\paragraph{F2: Ordering mismatch.}
If inverted ordering is decisively established, the rung ordering
$r_1<r_2<r_3$ is refuted. \VAL{}

\paragraph{F3: Squared-mass ratio mismatch.}
If absolute mass information establishes $m_3^2/m_2^2\neq\phig^7$, the rung
gap hypothesis is refuted. \VAL{}

\subsection{Scale falsifiers (test the eV reporting seam)}

\paragraph{F4: Oscillation windows.}
If updated NuFIT windows exclude $\Delta m^{2,\mathrm{pred}}_{21}$ or
$\Delta m^{2,\mathrm{pred}}_{31}$, either the rung triple or the seam is
refuted. \VAL{}

\paragraph{F5: Cosmological exclusion.}
If cosmological bounds establish $\Sigmanu < 0.062\,\mathrm{eV}$, the
predicted mass scale is ruled out. \VAL{}

\paragraph{F6: Direct mass detection.}
A kinematic measurement robustly implying a mass scale well above the
predicted window refutes the deep-ladder assignment. \VAL{}

\paragraph{F7: Neutrinoless double-beta decay.}
Detection of $0\nu\beta\beta$ at a level inconsistent with the Dirac
$Z_\nu=0$ assignment would require extending the framework. \VAL{}


%=============================================================================
\section{Conclusions}
\label{sec:conclusions}
%=============================================================================

This paper has extended the \RS{} mass framework to the neutrino sector via
the deep $\phig$-ladder with fractional (quarter-step) rungs.

\subsection{What is structural}

\begin{itemize}[nosep]
  \item The ladder mathematics: ratios are $\phig$-powers of rung differences;
        the seam cancels from ratios. \PROVED{}
  \item Normal ordering: forced by rung ordering plus $\phig>1$. \PROVED{}
  \item The $\phig^7$ squared-mass ratio: a seam-free structural prediction. \HYP{}
  \item The seam-free splitting ratio $R_\Delta=(\phig^{11}-1)/(\phig^4-1)$. \HYP{}
\end{itemize}

\subsection{What is hypothesized}

\begin{itemize}[nosep]
  \item The quarter-step rung lattice $r\in\frac{1}{4}\mathbb{Z}$. \HYP{}
  \item The specific rung triple $(-239/4,-231/4,-217/4)$. \HYP{}
  \item The Dirac nature ($Z_\nu=0$, $m_{\beta\beta}=0$). \HYP{}
\end{itemize}

\subsection{What the validation indicates}

Under the declared seam, $\Delta m^2_{21}$ and $\Delta m^2_{31}$ both fall
within NuFIT windows.  The mass sum $\Sigmanu\approx 0.063\,\mathrm{eV}$ is
consistent with current cosmological bounds.  The splitting ratio
$R_\Delta\approx 33.82$ is consistent with the experimental value
$R_\Delta^{\mathrm{exp}}\approx 33.4$. \VAL{}

\subsection{The core falsifiers}

The most robust tests are seam-free:
\begin{itemize}[nosep]
  \item The splitting ratio $R_\Delta$ (testable now),
  \item The mass ordering (testable with current and near-future experiments),
  \item The $\phig^7$ ratio (testable when absolute mass information becomes
        available).
\end{itemize}

The neutrino sector represents the frontier of the RS mass program.  The
charged sectors exhibit remarkable agreement with data; the neutrino sector
requires a structural extension (fractional rungs) that is natural within the
framework but not yet derived from the same pure counting-layer arguments that
fix the charged sector.  Closing this gap---deriving the neutrino rung lattice
from first principles---remains the primary open problem.

\begin{thebibliography}{99}
\bibitem{PDG2024} R.~L.~Workman \textit{et al.} [Particle Data Group],
  Prog.\ Theor.\ Exp.\ Phys.\ \textbf{2022}, 083C01 (2022) and 2024 update.
\bibitem{NuFIT} I.~Esteban \textit{et al.}, NuFIT~5.x (2024);
  \url{http://www.nu-fit.org}.
\bibitem{Washburn2025} J.~Washburn,
  ``The Algebra of Reality: A Recognition Science Derivation of Physical Law,''
  \textit{Axioms} \textbf{15}(2), 90 (2025).
\bibitem{PaperI} J.~Washburn,
  ``The Origin of Mass in Recognition Science: Cost Geometry, Recognition
  Boundaries, and the $\varphi$-Ladder'' (Paper~I of this series).
\bibitem{PaperII} J.~Washburn,
  ``Charged Fermion Masses and Flavor Mixing from $\varphi$-Ladder Geometry''
  (Paper~II of this series).
\bibitem{KATRIN} M.~Aker \textit{et al.} [KATRIN Collaboration],
  Nat.\ Phys.\ \textbf{18}, 160--166 (2022).
\end{thebibliography}

\end{document}
