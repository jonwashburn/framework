\documentclass[11pt,a4paper]{article}

\usepackage[margin=1in]{geometry}
\usepackage[T1]{fontenc}
\usepackage{lmodern}
\usepackage{microtype}
\usepackage{amsmath,amssymb,amsthm}
\usepackage{mathtools}
\usepackage{booktabs}
\usepackage{enumitem}
\usepackage{xcolor}
\usepackage[hidelinks]{hyperref}
\usepackage{tikz}
\usetikzlibrary{arrows.meta,positioning}

% Theorem environments
\newtheorem{theorem}{Theorem}[section]
\newtheorem{proposition}[theorem]{Proposition}
\newtheorem{lemma}[theorem]{Lemma}
\newtheorem{corollary}[theorem]{Corollary}
\newtheorem{definition}[theorem]{Definition}
\newtheorem{remark}[theorem]{Remark}

% Notation
\newcommand{\phig}{\varphi}
\newcommand{\Jcost}{J}
\newcommand{\Ecoh}{E_{\mathrm{coh}}}
\newcommand{\muStar}{\mu_{\star}}
\newcommand{\mRS}{m^{\mathrm{RS}}}
\newcommand{\mPred}{m^{\mathrm{pred}}}
\newcommand{\mdata}{m^{\mathrm{data}}}
\newcommand{\fRG}{f^{\mathrm{RG}}}
\newcommand{\fRec}{f^{\mathrm{Rec}}}
\newcommand{\tildeQ}{\tilde{Q}}
\newcommand{\Zidx}{Z}
\newcommand{\RS}{Recognition Science}
\newcommand{\SM}{Standard Model}

% Claim tags
\newcommand{\PROVED}{\textcolor{blue!70!black}{\footnotesize\textsf{[PROVED]}}}
\newcommand{\HYP}{\textcolor{orange!80!black}{\footnotesize\textsf{[HYP]}}}
\newcommand{\CERT}{\textcolor{teal}{\footnotesize\textsf{[CERT]}}}
\newcommand{\VAL}{\textcolor{purple!70!black}{\footnotesize\textsf{[VAL]}}}

\title{\textbf{Charged Fermion Masses and Flavor Mixing\\
from $\phig$-Ladder Geometry}\\[0.5em]
\large Paper II of V: Predictions and Phenomenology}
\author{Jonathan Washburn\\
\small Recognition Science Research Institute, Austin, Texas\\
\small \texttt{washburn.jonathan@gmail.com}}
\date{\today}

\begin{document}
\maketitle

\begin{abstract}
Building on the mechanism developed in Paper~I, this paper presents the
phenomenological predictions of the \RS{} (RS) mass framework for all
nine charged fermions, the CKM quark mixing matrix, and the PMNS leptonic
mixing angles.  The entire charged fermion spectrum is organized at a single
anchor scale $\muStar=182.201\,\mathrm{GeV}$ by three ingredients:
sector-global yardsticks derived from cube combinatorics ($D=3$), integer
rungs on the golden-ratio ($\phig$) ladder with generation torsion
$\tau_g\in\{0,11,17\}$, and a closed-form charge-to-band map
$\mathrm{gap}(Z)=\log_\phig(1+Z/\phig)$ with $Z\in\{24,276,1332\}$.
No per-species fitting is permitted.

For charged leptons, an absolute mass prediction chain
(electron break $\delta_e$ plus generation steps $S_{e\to\mu}$,
$S_{\mu\to\tau}$) reproduces the electron, muon, and tau masses to
sub-part-per-million agreement with PDG values.

For CKM mixing, the framework predicts
$|V_{cb}|=1/24$ (vertex--edge slot normalization),
$|V_{us}|=\phig^{-3}-\tfrac{3}{2}\alpha$ (golden projection plus
radiative correction), and $|V_{ub}|=\alpha/2$ (fine-structure suppression),
all within PDG uncertainties.

For PMNS mixing, it predicts $\sin^2\theta_{13}=\phig^{-8}$ (octave-forced),
$\sin^2\theta_{12}=\phig^{-2}-10\alpha$, and
$\sin^2\theta_{23}=\frac{1}{2}+6\alpha$ (upper octant), consistent with
NuFIT at current precision.

All integer coefficients trace to cube counts ($V\!=\!8$, $E\!=\!12$,
$F\!=\!6$, $S\!=\!24$) and the crystallographic constant $W\!=\!17$.
The only shared small parameter is $\alpha=1/137.036$.
Explicit ablations and falsifiers are provided throughout.
\end{abstract}

\tableofcontents
\newpage

%=============================================================================
\section{Introduction}
%=============================================================================

Paper~I established that in \RS{}, mass is a geometric coordinate on a
$\phig$-ladder forced by the cost functional $\Jcost(x)=\frac{1}{2}(x+x^{-1})-1$.
This paper converts that mechanism into concrete predictions and validates them
against the experimental record.

\subsection{The prediction pipeline}

The pipeline from RS structure to experimental comparison has three stages:
\begin{enumerate}[nosep]
  \item \textbf{Structural layer} (model):  sector yardsticks, integer rungs,
        charge-band map $\Rightarrow$ $\mRS(i;\muStar)$.
  \item \textbf{Transport layer} (bookkeeping):  SM renormalization-group running
        $\Rightarrow$ $\fRG_i(\muStar,\mu_{\mathrm{target}})$.
  \item \textbf{Validation layer} (comparison):  predicted values vs.\ PDG/NuFIT data.
\end{enumerate}

The structural layer contains zero adjustable parameters.  The transport layer uses
standard SM physics (4-loop QCD, 2-loop QED) as bookkeeping---it is never conflated
with the structural band coordinate.

\subsection{Claim hygiene}

Every equation in this paper is tagged with one of four labels:
\begin{itemize}[nosep]
  \item \PROVED: structural derivation with no per-species fitting,
  \item \CERT: declared convention or certified numerical value,
  \item \HYP: modeling hypothesis (falsifiable),
  \item \VAL: validation comparison against external data.
\end{itemize}


%=============================================================================
\section{The Single-Anchor Mass Law}
\label{sec:mass_law}
%=============================================================================

\subsection{Anchor scale}

All predictions are stated at a single common anchor scale: \CERT{}
\begin{equation}
  \muStar = 182.201\,\mathrm{GeV}.
\end{equation}
This value is determined by a mass-free PMS/BLM stationarity condition (the SM
anomalous dimension $\gamma_m(\mu)$ vanishes at $\muStar$ for species-independent
QCD/QED kernels) and is not fitted to any fermion mass.

\subsection{The mass law at the anchor}

At $\muStar$, each charged fermion $i$ is assigned: \HYP{}
\begin{equation}
\boxed{
  \mRS(i;\muStar)
  \;=\;
  A_{\mathrm{sector}(i)}\;\phig^{\,r_i - 8 + \mathrm{gap}(\Zidx_i)},
}
  \label{eq:mass_law}
\end{equation}
where:
\begin{itemize}[nosep]
  \item $A_{\mathrm{sector}}=2^{B_{\mathrm{pow}}}\cdot\Ecoh\cdot\phig^{r_0}$ is the
        sector yardstick (Table~\ref{tab:yardsticks}), \PROVED{}
  \item $r_i\in\mathbb{Z}$ is the integer rung (Table~\ref{tab:rungs}), \PROVED{}
  \item $-8$ is the octave reference (eight-tick coordinate origin), \HYP{}
  \item $\mathrm{gap}(Z)=\log_\phig(1+Z/\phig)$ is the band function, \HYP{}
  \item $\Zidx_i$ is the charge-derived integer (Section~\ref{sec:Z_map}).  \HYP{}
\end{itemize}

\begin{table}[t]
\centering
\caption{Sector yardstick exponents derived from counting layer.}
\label{tab:yardsticks}
\begin{tabular}{lrrll}
\toprule
Sector      & $B_{\mathrm{pow}}$ & $r_0$ & Formula for $B_{\mathrm{pow}}$ & Formula for $r_0$ \\
\midrule
Lepton      & $-22$ & $62$  & $-2E_{\mathrm{passive}}$ & $4W-6$ \\
Up quark    & $-1$  & $35$  & $-A$                     & $2W+A$ \\
Down quark  & $23$  & $-5$  & $2E-1$                   & $E-W$  \\
Electroweak & $1$   & $55$  & $A$                      & $3W+4$ \\
\bottomrule
\end{tabular}
\end{table}

\begin{table}[t]
\centering
\caption{Integer rungs for charged fermions.  Generation torsion:
$\tau_g\in\{0,11,17\}$ for generations $(1,2,3)$.}
\label{tab:rungs}
\begin{tabular}{lccc}
\toprule
& Gen~1 & Gen~2 & Gen~3 \\
\midrule
Charged leptons ($r_{\mathrm{base}}=2$) & $e:2$ & $\mu:13$ & $\tau:19$ \\
Up quarks ($r_{\mathrm{base}}=4$)       & $u:4$ & $c:15$   & $t:21$   \\
Down quarks ($r_{\mathrm{base}}=4$)     & $d:4$ & $s:15$   & $b:21$   \\
\bottomrule
\end{tabular}
\end{table}

\subsection{Charge integerization and the band label $Z$}
\label{sec:Z_map}

Electric charges are integerized via $\tildeQ:=6Q\in\mathbb{Z}$: \HYP{}
\begin{equation}
  \tildeQ_e = -6,\quad \tildeQ_u = 4,\quad \tildeQ_d = -2.
\end{equation}
The band label is: \HYP{}
\begin{equation}
  \Zidx(Q,\mathrm{sector}) =
  \begin{cases}
    \tildeQ^2 + \tildeQ^4, & \text{leptons},\\
    4 + \tildeQ^2 + \tildeQ^4, & \text{quarks}.
  \end{cases}
\end{equation}
This yields three equal-$Z$ families: \PROVED{}
\begin{equation}
  Z_\ell = 1332,\qquad Z_u = 276,\qquad Z_d = 24.
\end{equation}

\subsection{Equal-$Z$ corollary}

Within an equal-$Z$ family, the gap function cancels in mass ratios: \PROVED{}
\begin{equation}
  \frac{\mRS(i;\muStar)}{\mRS(j;\muStar)} = \phig^{r_i - r_j}
  \quad\text{(same sector, same $Z$)}.
\end{equation}

\subsection{Gap function values}

The three family gap values are: \PROVED{} (given the $Z$-map)
\begin{align}
  \mathrm{gap}(24)   &= \log_\phig(1 + 24/\phig) \approx 5.74, \\
  \mathrm{gap}(276)  &= \log_\phig(1 + 276/\phig) \approx 10.69, \\
  \mathrm{gap}(1332) &= \log_\phig(1 + 1332/\phig) \approx 13.95.
\end{align}


%=============================================================================
\section{Charged Lepton Mass Chain}
\label{sec:leptons}
%=============================================================================

The mass law~\eqref{eq:mass_law} organizes the spectrum at $\muStar$.  For
charged leptons, an additional pipeline yields \emph{absolute} predictions for
$m_e$, $m_\mu$, $m_\tau$ as a sequence of derived exponents.

\subsection{Electron break}

The electron break exponent is: \HYP{}
\begin{equation}
  \delta_e
  = 2W + \frac{W + E}{4E_{\mathrm{passive}}} + \alpha^2 + E\alpha^3,
  \label{eq:delta_e}
\end{equation}
where $W=17$, $E=12$, $E_{\mathrm{passive}}=11$, and $\alpha\approx 1/137.036$.
The first two terms are purely topological; the last two are small radiative
corrections organized by $\alpha$.

\subsection{Generation steps}

The electron-to-muon step: \HYP{}
\begin{equation}
  S_{e\to\mu} = E_{\mathrm{passive}} + \frac{1}{4\pi} - \alpha^2
  \approx 11.080.
  \label{eq:step_emu}
\end{equation}
The leading term $E_{\mathrm{passive}}=11$ is the passive edge count.

The muon-to-tau step: \HYP{}
\begin{equation}
  S_{\mu\to\tau} = F - \frac{2W+3}{2}\,\alpha
  \approx 5.866.
  \label{eq:step_mutau}
\end{equation}
The leading term $F=6$ is the cube face count.

\subsection{Predicted masses}

The electron mass prediction: \HYP{}
\begin{equation}
  m_e^{\mathrm{pred}} = m_{\mathrm{skel}}(e;\muStar)\cdot
  \phig^{\mathrm{gap}(1332)-\delta_e},
\end{equation}
where $m_{\mathrm{skel}}(e;\muStar)=A_{\mathrm{Lepton}}\cdot\phig^{r_e-8}$.

The muon and tau follow by accumulating the generation steps: \HYP{}
\begin{align}
  m_\mu^{\mathrm{pred}} &= m_e^{\mathrm{pred}}\cdot\phig^{S_{e\to\mu}}, \\
  m_\tau^{\mathrm{pred}} &= m_\mu^{\mathrm{pred}}\cdot\phig^{S_{\mu\to\tau}}.
\end{align}

\subsection{Validation against PDG}

Under the declared calibration seam and transport policy: \VAL{}
\begin{center}
\begin{tabular}{lrrr}
\toprule
Particle & Predicted (MeV) & PDG (MeV) & Rel.\ error \\
\midrule
$e$   & $0.51100$  & $0.51100$ & $\sim -4\times 10^{-7}$ \\
$\mu$ & $105.658$  & $105.658$ & $\sim -1\times 10^{-6}$ \\
$\tau$& $1776.5$   & $1776.9$  & $\sim -9\times 10^{-5}$ \\
\bottomrule
\end{tabular}
\end{center}

The electron and muon are reproduced to sub-ppm; the tau to $\sim 10^{-4}$.
These numbers are generated by the repository script
\texttt{tools/lepton\_chain\_table.py}.


%=============================================================================
\section{Quark Sector Predictions}
\label{sec:quarks}
%=============================================================================

\subsection{Mass organization at the anchor}

The six quarks share the same mass law~\eqref{eq:mass_law} with their respective
sector yardsticks and rungs (Table~\ref{tab:rungs}).  The equal-$Z$ families are:
\begin{itemize}[nosep]
  \item Up-type ($Z_u=276$): $u,c,t$ at rungs $4,15,21$,
  \item Down-type ($Z_d=24$): $d,s,b$ at rungs $4,15,21$.
\end{itemize}

Note that up-type and down-type quarks share the same rung \emph{values}
$(4,15,21)$ but have different sector yardsticks and different $Z$ values.
This is a structural prediction: the generation torsion $\{0,11,17\}$ is
universal across all sectors.

\subsection{Anchor-to-PDG transport}

Quark masses quoted by the PDG use diverse conventions ($\overline{\mathrm{MS}}$
running masses at various scales for light quarks; pole-like masses for top).
To compare, we define the transport display: \CERT{}
\begin{equation}
  \mPred(i;\mu_{\mathrm{target}})
  = \mRS(i;\muStar)\cdot\phig^{\fRG_i(\muStar,\mu_{\mathrm{target}})},
\end{equation}
where $\fRG_i$ is the SM renormalization-group transport exponent (4-loop QCD,
2-loop QED).

\subsection{Equal-$Z$ clustering test}

The most powerful test is \emph{not} the absolute mass values (which depend on
the calibration seam) but the clustering of transported data by equal-$Z$ families
at $\muStar$.  Transport PDG masses back to $\muStar$: \VAL{}
\begin{equation}
  f_i^{\mathrm{exp}}(\muStar)
  := \log_\phig\!\left(\frac{\mdata(i;\muStar)}{m_{\mathrm{skel}}(i;\muStar)}\right).
\end{equation}
The band-map hypothesis predicts that
$f_i^{\mathrm{exp}}(\muStar)\approx\mathrm{gap}(Z_i)$ within each family.
Under the declared transport policy, the nine charged fermions cluster by their
three $Z$-values to within tolerance $\sim 5\times 10^{-6}$ in residue space---a
$\sim 15.6\sigma$-equivalent result under simple null models. \VAL{}


%=============================================================================
\section{CKM Mixing from Cubic Ledger Topology}
\label{sec:CKM}
%=============================================================================

\subsection{The cubic ledger as a mixing graph}

The same 3-cube that organizes mass (vertices $V\!=\!8$, edges $E\!=\!12$,
faces $F\!=\!6$) also constrains flavor mixing.  The vertex--edge slot count
$S:=2E=24$ provides a natural normalization for transition amplitudes.

\subsection{CKM predictions}

\paragraph{$|V_{cb}|$: edge-dual normalization.} \HYP{}
The 2--3 quark mixing magnitude is identified with one admissible transition
out of $S=24$ vertex--edge slots:
\begin{equation}
  |V_{cb}|_{\mathrm{pred}} = \frac{1}{S} = \frac{1}{24} \approx 0.04167.
\end{equation}
PDG: $|V_{cb}|_{\mathrm{ref}}\approx 0.04182\pm 0.00085$. \VAL{}

\paragraph{$|V_{us}|$: golden projection with radiative correction.} \HYP{}
The Cabibbo mixing is a $\phig$-power suppressed by a cube-derived
$\alpha$-correction:
\begin{equation}
  |V_{us}|_{\mathrm{pred}} = \phig^{-3} - \frac{3}{2}\alpha \approx 0.22512.
\end{equation}
The coefficient $3/2=F/4$ (face count divided by 4).
PDG: $|V_{us}|_{\mathrm{ref}}\approx 0.22500\pm 0.00067$. \VAL{}

\paragraph{$|V_{ub}|$: fine-structure suppression.} \HYP{}
\begin{equation}
  |V_{ub}|_{\mathrm{pred}} = \frac{\alpha}{2} \approx 0.00365.
\end{equation}
PDG: $|V_{ub}|_{\mathrm{ref}}\approx 0.00369\pm 0.00011$. \VAL{}

\subsection{CKM CP violation: Jarlskog invariant}

The Jarlskog invariant is predicted from the product of the three mixing magnitudes
(no new parameters): \HYP{}
\begin{equation}
  J_{\mathrm{CKM}}^{\mathrm{pred}}
  = |V_{us}|\cdot|V_{cb}|\cdot|V_{ub}|
  = \left(\phig^{-3}-\tfrac{3}{2}\alpha\right)\cdot\frac{1}{24}\cdot\frac{\alpha}{2}
  \approx 3.4\times 10^{-5}.
\end{equation}
PDG: $J_{\mathrm{CKM}}^{\mathrm{ref}}\sim 3.1\times 10^{-5}$. \VAL{}


%=============================================================================
\section{PMNS Mixing from $\phig$-Harmonics}
\label{sec:PMNS}
%=============================================================================

\subsection{PMNS predictions}

The three PMNS mixing angles are proposed as closed-form expressions using
$\phig$ and $\alpha$ with cube-derived integer coefficients.

\paragraph{Reactor angle (octave-forced).} \HYP{}
\begin{equation}
  \sin^2\theta_{13}^{\mathrm{pred}} = \phig^{-8} \approx 0.02129.
\end{equation}
The exponent $8$ is the octave period---the same eight-tick count that defines
the mass coordinate origin.  NuFIT: $\sin^2\theta_{13}\approx 0.02220$. \VAL{}

\paragraph{Solar angle.} \HYP{}
\begin{equation}
  \sin^2\theta_{12}^{\mathrm{pred}} = \phig^{-2} - 10\alpha \approx 0.30899.
\end{equation}
The coefficient $10 = E-2$ (edges minus two constrained directions).
NuFIT: $\sin^2\theta_{12}\approx 0.303$. \VAL{}

\paragraph{Atmospheric angle (upper octant).} \HYP{}
\begin{equation}
  \sin^2\theta_{23}^{\mathrm{pred}} = \frac{1}{2} + 6\alpha \approx 0.54378.
\end{equation}
The coefficient $6=F$ (cube face count).  This predicts the \emph{upper octant}---a
sharp falsifier.  NuFIT: $\sin^2\theta_{23}\approx 0.572$ (upper octant preferred). \VAL{}


%=============================================================================
\section{Ablations and Falsifiers}
\label{sec:ablations}
%=============================================================================

\subsection{Mass framework ablations}

\begin{enumerate}
  \item \textbf{Drop the quark $+4$ offset}: replace quark $Z$ by
        $\tildeQ^2+\tildeQ^4$ (no $+4$).  Result: up/down $Z$-values no longer
        separate; equal-$Z$ clustering fails. \VAL{}

  \item \textbf{Drop the quartic term}: use $Z=\tildeQ^2$ only.
        Result: the three families cannot achieve the required gap hierarchy. \VAL{}

  \item \textbf{Change charge integerization}: use $\tildeQ=kQ$ with $k\neq 6$.
        Result: SM charges do not map to a stable, consistent integer family. \VAL{}

  \item \textbf{Drop band structure}: set $\mathrm{gap}(Z)\equiv 0$.
        Result: skeleton alone cannot reproduce the spectrum without per-species
        tuning. \VAL{}
\end{enumerate}

\subsection{Mixing falsifiers}

\begin{enumerate}
  \item $|V_{cb}|$ departing significantly from $1/24$ as CKM uncertainties tighten.
  \item Decisive lower-octant $\theta_{23}$ in PMNS (contradicts $1/2+6\alpha$).
  \item $\sin^2\theta_{13}$ measured outside the $\phig^{-8}$ prediction band.
\end{enumerate}


%=============================================================================
\section{Summary of Numerical Predictions}
\label{sec:summary_table}
%=============================================================================

\begin{table}[h]
\centering
\caption{Summary of RS predictions vs.\ experimental values.}
\begin{tabular}{llrl}
\toprule
Observable & RS prediction & Exp.\ value & Source \\
\midrule
$m_e$ & $0.51100\,\mathrm{MeV}$ & $0.51100\,\mathrm{MeV}$ & PDG \\
$m_\mu$ & $105.658\,\mathrm{MeV}$ & $105.658\,\mathrm{MeV}$ & PDG \\
$m_\tau$ & $1776.5\,\mathrm{MeV}$ & $1776.9\,\mathrm{MeV}$ & PDG \\
\midrule
$|V_{cb}|$ & $0.04167$ & $0.04182\pm 0.00085$ & PDG \\
$|V_{us}|$ & $0.22512$ & $0.22500\pm 0.00067$ & PDG \\
$|V_{ub}|$ & $0.00365$ & $0.00369\pm 0.00011$ & PDG \\
$J_{\mathrm{CKM}}$ & $3.4\times 10^{-5}$ & $3.1\times 10^{-5}$ & PDG \\
\midrule
$\sin^2\theta_{13}$ & $0.02129$ & $0.02220\pm 0.00068$ & NuFIT \\
$\sin^2\theta_{12}$ & $0.30899$ & $0.303\pm 0.012$ & NuFIT \\
$\sin^2\theta_{23}$ & $0.54378$ & $0.572\pm 0.018$ & NuFIT \\
\bottomrule
\end{tabular}
\end{table}


%=============================================================================
\section{Conclusions}
\label{sec:conclusions}
%=============================================================================

This paper has presented the phenomenological predictions of the \RS{} mass framework
for all nine charged fermion masses, CKM mixing, and PMNS mixing angles.

The inputs are:
\begin{itemize}[nosep]
  \item Five counting-layer integers: $V\!=\!8$, $E\!=\!12$, $F\!=\!6$, $W\!=\!17$, $A\!=\!1$,
  \item The golden ratio $\phig=(1+\sqrt{5})/2$ (forced by the cost functional),
  \item The fine-structure constant $\alpha\approx 1/137.036$ (derived from the same
        counting layer),
  \item Integer rungs with universal generation torsion $\{0,11,17\}$.
\end{itemize}

No per-species parameters are fitted.  The charged lepton masses are reproduced to
sub-ppm.  CKM magnitudes are within PDG uncertainties.  PMNS angles are consistent
with NuFIT at current precision.

Paper~III extends the framework to the neutrino sector, where fractional
$\phig$-ladder rungs yield predictions for absolute masses, mass splittings, and
the mass ordering.

\begin{thebibliography}{99}
\bibitem{PDG2024} R.~L.~Workman \textit{et al.} [Particle Data Group],
  Prog.\ Theor.\ Exp.\ Phys.\ \textbf{2022}, 083C01 (2022) and 2024 update.
\bibitem{NuFIT} I.~Esteban \textit{et al.}, NuFIT~5.x (2024);
  \url{http://www.nu-fit.org}.
\bibitem{Washburn2025} J.~Washburn,
  ``The Algebra of Reality: A Recognition Science Derivation of Physical Law,''
  \textit{Axioms} \textbf{15}(2), 90 (2025).
\bibitem{PaperI} J.~Washburn,
  ``The Origin of Mass in Recognition Science: Cost Geometry, Recognition Boundaries,
  and the $\varphi$-Ladder'' (Paper~I of this series).
\end{thebibliography}

\end{document}
