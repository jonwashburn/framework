\documentclass[11pt, reqno]{amsart}

%% Packages
\usepackage{amsmath, amssymb, amsthm, amsfonts}
\usepackage{mathtools}
\usepackage{geometry}
\geometry{margin=1.0in}
\usepackage[colorlinks=true, linkcolor=blue, citecolor=blue, urlcolor=blue]{hyperref}
\usepackage{microtype}

%% Theorems (lightweight; this is mostly a prose note)
\newtheorem{theorem}{Theorem}[section]
\newtheorem{lemma}[theorem]{Lemma}
\newtheorem{proposition}[theorem]{Proposition}
\newtheorem{corollary}[theorem]{Corollary}
\theoremstyle{definition}
\newtheorem{definition}[theorem]{Definition}
\newtheorem{remark}[theorem]{Remark}
\newtheorem{conjecture}[theorem]{Conjecture}

%% Macros
\newcommand{\R}{\mathbb{R}}
\newcommand{\Sbb}{\mathbb{S}}
\newcommand{\eps}{\varepsilon}
\newcommand{\dv}{\mathrm{div}}
\newcommand{\curl}{\mathrm{curl}}

\title[A Recognition Science view of Navier--Stokes]{What Recognition Science suggests about Navier--Stokes regularity: tail tightness as non-parasitic flux closure (prose note)}
\author{Jonathan Washburn}

\begin{document}

\begin{abstract}
Assuming Recognition Science (RS) is an accurate architecture of reality, this note explains---in prose, with a minimal amount of classical PDE notation---what RS would \emph{predict} about the remaining unknown elements in a running-max blow-up approach to Navier--Stokes regularity.
In the current proof program, many local reductions are available, but every route that tries to close the argument without global decay collapses at a single missing estimate: a \emph{global tightness} / \emph{no-multi-bubble} mechanism in blow-up variables, equivalently a vanishing of certain ``tail flux'' boundary terms at spatial infinity.
RS supplies a strong physical intuition for why such tail export cannot persist: Ledger/double-entry plus the closed-chain flux law (T3) forbids stable patterns whose persistence depends on net export, i.e.\ ``parasitic'' configurations.
We translate this RS content into a concrete PDE dictionary and isolate a small set of equivalent mathematical conjectures (RTD/UEWE/tail-flux vanishing) that would implement the RS prediction in classical terms.
\end{abstract}

\maketitle

\tableofcontents

\section{Context: where the classical proof program is stuck}

The 3D incompressible Navier--Stokes regularity problem can be phrased as: show that smooth initial data generate smooth solutions for all time.
In a running-max blow-up program, one assumes a finite-time singularity and extracts (after rescaling) a bounded-vorticity ancient solution (an ``ancient element'').
The program then aims to rule out such an ancient element by a chain of rigidity arguments.

\medskip
\noindent
\textbf{Current bottleneck (as of 2025-12-23).}
In the present draft implementation (see \texttt{navier-dec-12-rewrite.tex} and the companion technical note \texttt{papers/RM2U\_reduction\_note.tex}),
multiple attempts to close the final steps without assuming global decay have been executed and \emph{certified} to pivot to the same missing ingredient.
That ingredient is a global tail/tightness estimate for the running-max ancient element in blow-up variables.
One convenient formulation is a \emph{uniform exterior weighted enstrophy} (UEWE) bound; another is a \emph{relative tail depletion} (RTD) bound; another is a precise \emph{tail-flux vanishing} statement in an $\ell=2$ coefficient-energy identity.

\medskip
\noindent
\textbf{What is meant by ``tail/tightness''.}
Very informally: the running-max normalization forces strong control near the blow-up center but does \emph{not} forbid the existence of a second, comparably strong ``bubble'' far away in the rescaled variables.
Any such multi-bubble configuration defeats the desired passage from ``tail bounded'' to ``tail small'' and prevents absorption of forcing terms by coercive barriers.
This is exactly the point where classical estimates repeatedly fail.

\section{Recognition Science principles relevant to PDE closure (assumed true)}

This section extracts only the RS content that matters for the Navier--Stokes bottleneck.
We refer to \texttt{Recognition-Science-Full-Theory.txt} for the full architecture specification; here we use only its high-level commitments:

\begin{itemize}
\item \textbf{Meta-Principle (MP).} ``Nothing cannot recognize itself.'' (RS \texttt{@KERNEL})
\item \textbf{Ledger / double-entry.} Conservation plus discrete events forces a double-entry ledger structure: debit equals credit at every node (see RS \texttt{@KERNEL}; see also \texttt{@LEDGER} and \texttt{@MP\_TO\_LEDGER\_FORMALIZATION}).
\item \textbf{Closed-chain flux law (T3).} The net flux around any closed chain is zero (RS: ``T3 (flux=0) --- closed-chain conservation''); in continuum correspondence this is the continuity equation $\partial_t\rho+\nabla\cdot J=0$ (RS \texttt{MAP;T3}).
\item \textbf{Parasitism is unsustainable.} RS defines a \emph{parasitic pattern} as one that persists by exporting harm/imbalance, and asserts a theorem that parasitic patterns cannot persist indefinitely (RS \texttt{@EVIL\_PATHOLOGY}, \texttt{THEOREM; parasitism\_unsustainable}).
\item \textbf{Finite capacity / saturation.} RS asserts finite capacity at the fundamental ``light field'' layer; above saturation, a cost grows and forces a phase change / re-embodiment (RS \texttt{@PHASE\_SATURATION}).
\end{itemize}

\medskip
\noindent
\textbf{Interpretive stance for this note.}
We treat these RS items as physically true constraints on what stable patterns can exist, and we ask:
\emph{What classical PDE statement would implement these constraints in Navier--Stokes blow-up variables?}

\section{RS-to-PDE dictionary for the running-max ancient element}

The goal is not to ``prove Navier--Stokes from RS'' in one leap, but to use RS to focus the proof search on the uniquely missing mechanism.
The dictionary below is intentionally pragmatic: it maps RS primitives to the concrete PDE objects that appear in the current manuscript.

\subsection{Ledger to PDE: energy/enstrophy accounting}

In Navier--Stokes, there are many local balance laws: energy inequalities, enstrophy identities, and weighted local identities.
In a running-max extraction, one packages these identities into ``budgets'' on parabolic cylinders $Q_r(z_0)$.
This is a classical manifestation of a ledger: each cylinder has inflows/outflows, and each identity has a debit/credit form (dissipation paid by injection, etc.).

\subsection{Closed-chain flux zero to PDE: boundary flux at infinity must vanish}

In the $\ell=2$ RM2U sector, one derives a radial coefficient equation and an energy identity whose right-hand side contains a forcing/work term plus explicit boundary flux terms.
Classically, the obstruction is that the \emph{outer} boundary terms at radius $R$ cannot be controlled as $R\to\infty$ without a tightness input.

\medskip
\noindent
RS suggests a sharpened viewpoint:
\begin{quote}
\emph{If the only way a hypothetical singular ancient element can persist is by exporting ``work''/imbalance to infinity (a boundary flux that does not vanish), then that configuration is parasitic. Ledger + T3 forbids stable parasitic patterns. Therefore, in the correct variables, the outer boundary flux must vanish.}
\end{quote}

Mathematically, this is exactly the missing statement the proof needs: \emph{tail-flux vanishing}.

\subsection{Parasitism to PDE: multi-bubble / non-tight blow-up sequences}

The cleanest mathematical model of ``persistence by export'' in the running-max blow-up variables is the \emph{two-bubble} phenomenon:
one bubble at the origin (forced by normalization) and another bubble traveling to infinity in the rescaled variables while maintaining nontrivial amplitude.
This permits non-vanishing boundary flux at $R\to\infty$ and defeats attempts to close coercive inequalities.

RS frames such a configuration as an unstable, non-physical pattern: it persists only by exporting imbalance.
The corresponding PDE conjecture is a \emph{no-multi-bubble tightness theorem}.

\subsection{Finite capacity to PDE: no infinite payment over infinite history}

Running-max solutions satisfy a ``finite budget'' inequality: they cannot pay an unbounded amount of certain nonnegative costs over their entire backward history.
Several candidate closures reduce to proving a uniform small-scale bound of the form
\[
\sup_{z_0}\iint_{Q_r(z_0)} \rho^{3/2}\sigma_+ \to 0
\qquad (r\downarrow 0),
\]
where $\sigma_+$ is the positive part of a stretching density and $\rho=|\omega|$.
When this bound fails, one can interpret it as ``infinite payment'' (persistent positive injection) being hidden in the tail/harmonic modes.

RS suggests that such infinite payment is incompatible with finite capacity: the substrate cannot store unlimited imbalance.
Again, the classical implementation is a tightness statement that blocks leakage to infinity.

\section{Concrete RS-guided conjectures (classical correspondence targets)}

From the current manuscript, multiple equivalent ``global tail gate'' formulations are already isolated.
RS suggests focusing directly on one of these, since they are all morally ``no parasitic export''.

\begin{conjecture}[Relative tail depletion (RTD) in blow-up variables]
Along a running-max blow-up sequence, the rescaled vorticities are uniformly small in the far field:
there exists a decreasing envelope $h(R)\to 0$ such that
\[
\sup_k\ \sup_{s\le 0}\ \sup_{|y|\ge R} |\omega^{(k)}(y,s)| \le h(R).
\]
\end{conjecture}

\begin{conjecture}[Uniform exterior weighted enstrophy (UEWE)]
For the running-max ancient element $\omega^\infty$ one has a uniform-in-time exterior bound
\[
\sup_{t\le 0}\int_{|x|\ge 1}\left(\frac{|\omega^\infty(x,t)|^2}{|x|^2}+|\nabla\omega^\infty(x,t)|^2\right)\,dx < \infty.
\]
\end{conjecture}

\begin{conjecture}[Tail-flux vanishing (non-parasitism)]
In the $\ell=2$ coefficient energy identity for the RM2U sector, the outer boundary flux term at radius $R$ vanishes as $R\to\infty$ (uniformly for $t\le 0$ in the running-max ancient element).
Equivalently, the ``export to infinity'' term is zero: no persistent net flux can be maintained by the tail.
\end{conjecture}

\begin{remark}
In the proof engineering documents for this project, these conjectures are treated as different interfaces to the same missing global mechanism.
The Lean-facing interface is expressed as a \texttt{TailFluxVanish} hypothesis; the TeX-facing interface is expressed as UEWE/RTD-style tightness.
\end{remark}

\section{What RS changes about the search strategy}

Without RS, one might reasonably hope that some hidden algebraic cancellation closes the forcing pairing in the $\ell=2$ energy identity, or that the tail terms can be absorbed using only boundedness.
Those hopes have been tested and appear false in the current program: the outer transport/boundary terms are exactly where classical estimates pivot.

\medskip
\noindent
RS changes the search strategy by making a strong prediction:
\begin{quote}
\emph{There is no clever local cancellation that avoids the global gate. The correct theorem is global and structural: persistent export to infinity is physically forbidden.}
\end{quote}

Thus, the right mathematical work is to find a classical compactness/tightness mechanism for the running-max blow-up sequence.
In practice this means: identify a quantity that (i) detects multi-bubble tail persistence, (ii) has a sign/monotonicity forced by Navier--Stokes + the running-max normalization, and (iii) yields a contradiction with the finite-budget constraints.

\section{Conclusion}

Assuming RS is an accurate physical architecture, the remaining unknown element in the running-max Navier--Stokes program is not a mysterious local estimate: it is a single global prohibition on parasitic export.
The classical mathematical correspondence is a tightness statement (RTD/UEWE/tail-flux vanishing) for the running-max ancient element.
This note is intended to make that RS-to-PDE bridge explicit so that future work can target the uniquely missing theorem.

\end{document}


