\documentclass[11pt,a4paper]{article}
\usepackage[margin=1in]{geometry}
\usepackage{hyperref}
\usepackage{enumitem}
\usepackage{xcolor}

\hypersetup{
    colorlinks=true,
    linkcolor=blue,
    urlcolor=blue,
    citecolor=blue
}

\title{\textbf{Recognition Science: A Curated Syllabus}\\ \large The Logic of the Dependency Graph}
\author{Jonathan Washburn}
\date{\today}

\begin{document}

\maketitle

\section*{Introduction}
This syllabus mirrors the logical Dependency Graph (DAG) of Recognition Science. The papers are ordered not chronologically, but structurally: starting from the root geometry, ascending the ``spine'' of derivations, and then branching into specific domains (Particles, Computation, Gravity, Math, Consciousness, Ethics).

\section{I. The Root (Foundation)}

\subsection{1. Recognition Geometry}
\textbf{File:} \texttt{papers/tex/recognition-geometry.tex}
\textbf{Priority:} Tier 1 (Publish First)
\textbf{Description:} Defines the observational space, recognizers, and the recognition quotient.

\subsection{2. The Recognition Composition Law Primer}
\textbf{File:} \texttt{papers/tex/Recognition\_Composition\_Law\_Primer.tex}
\textbf{Priority:} Tier 1 (Publish First)
\textbf{Description:} Introduces the fundamental rule of comparison (RCL) governing how costs combine.

\subsection{3. Uniqueness of the Canonical Reciprocal Cost}
\textbf{File:} \texttt{Cost-9-1.pdf} (or latest revision)
\textbf{Priority:} Tier 1 (Publish First)
\textbf{Description:} Proves that the RCL forces the unique cost function $J(x) = \frac{1}{2}(x + x^{-1}) - 1$.

\subsection{4. Coherent Comparison as Information Cost: A Cost-First Ledger Framework}
\textbf{File:} \texttt{papers/pdf/2601.12194v1.pdf}
\textbf{Priority:} Tier 1 (Publish First)
\textbf{Description:} Derives the discrete ledger dynamics from the cost function.

\subsection{5. D'Alembert Inevitability: Polynomial Consistency Forces the Canonical Law}
\textbf{File:} \texttt{papers/tex/DAlembert\_Inevitability.tex}
\textbf{Priority:} Tier 1 (Publish First)
\textbf{Description:} Proves the inevitability of the composition law from polynomial consistency constraints.

\subsection{6. Model-Independent Exclusivity on the Quotient State Space}
\textbf{File:} \texttt{papers/tex/Model-Independent-Exclusivity-Quotient.tex}
\textbf{Priority:} Tier 1 (Publish First)
\textbf{Description:} Proves that any zero-parameter framework is observationally equivalent to RS.

\subsection{51. Gödel's Theorem Does Not Obstruct Physical Closure}
\textbf{File:} \texttt{papers/tex/godel\_dissolution.tex}
\textbf{Description:} Resolves the Gödelian objection by defining truth as cost-stabilization.

\subsection{53. The Recognition Stability Audit (RSA)}
\textbf{File:} \texttt{papers/tex/Recognition\_Stability\_Audit.tex}
\textbf{Description:} Formalizes the ``impossibility audit'' and certificates for existence claims.

\section{II. The Structure (Ontology)}

\subsection{7. The Cost of Existence}
\textbf{File:} \texttt{The\_Cost\_of\_Existence.tex}
\textbf{Priority:} Tier 2
\textbf{Description:} Derives existence from the infinite cost of the void ($J(0) \to \infty$).

\subsection{52. The Law of Existence}
\textbf{File:} \texttt{papers/tex/Law-of-Existence-arXiv.tex}
\textbf{Priority:} Tier 1
\textbf{Description:} The core ontological paper linking CPM, Darwin, and physical constants.

\subsection{8. Logic From Physical Cost}
\textbf{File:} \texttt{papers/tex/Logic\_From\_Physical\_Cost.tex}
\textbf{Priority:} Tier 2
\textbf{Description:} Derives logical consistency ($A=A$) as the zero-cost ground state ($J(1)=0$).

\subsection{9. The Recognition Operator}
\textbf{File:} \texttt{papers/root\_papers/The\_Recognition\_Operator.tex}
\textbf{Priority:} Tier 2
\textbf{Description:} Defines $\hat{R}$ (Recognition Operator) as the fundamental dynamic, replacing the Hamiltonian.

\subsection{10. The Golden Ratio as a Universal Coherence Eigenvalue}
\textbf{File:} \texttt{papers/tex/Penrose\_golden\_ratio\_and\_ledger\_structure.tex}
\textbf{Priority:} Tier 2
\textbf{Description:} Explains $\phi$ as the unique fixed point of the cost recursion.

\subsection{11. Dimensional Rigidity: D=3}
\textbf{File:} \texttt{papers/tex/Dimensional\_Rigidity\_D3.tex}
\textbf{Priority:} Tier 2
\textbf{Description:} Proves that $D=3$ is the only dimension permitting stable knotting and ledger closure.

\section{III. The Waist (The Big Unlock)}

\subsection{12. The Derivation of Physical Constants from the Meta-Principle}
\textbf{File:} \texttt{papers/tex/Formalized-Derivations-T1-T8.tex}
\textbf{Priority:} Tier 2
\textbf{Description:} The ``Grand Central Station'' paper deriving $\alpha, G, \hbar, c$ from the foundation.

\subsection{50. Reality-Native Measurements with a Single-Anchor SI Bridge}
\textbf{File:} \texttt{papers/RSNative-Measurement-Framework.tex}
\textbf{Priority:} Tier 2
\textbf{Description:} Protocol for mapping RS dimensionless quantities to SI units via a single anchor.

\subsection{61. Quantum Coherence as Gated Recognition}
\textbf{File:} \texttt{papers/tex/Quantum-Coherence-Theory.tex}
\textbf{Priority:} Tier 3
\textbf{Description:} Derives coherence time from the 8-tick cycle structure.

\subsection{58. The Octave System and the Particle Mass Spectrum}
\textbf{File:} \texttt{papers/tex/OCTAVE\_MASSES\_PAPER.tex}
\textbf{Description:} Details the ``Octave'' 8-tick mechanism forcing mass rungs.

\subsection{60. The Projection Operator \(\hat{\pi}\)}
\textbf{File:} \texttt{papers/tex/projection\_operator.tex}
\textbf{Priority:} Tier 3
\textbf{Description:} Active enforcement of information conservation; collapse/decision mechanism.

\section{IV. Track A: Particle Physics}

\subsection{13. A First-Principles Derivation of Particle Mass (Leptons)}
\textbf{File:} \texttt{papers/tex/Full\_First\_Principles\_Mass\_Derivation.tex}
\textbf{Priority:} Tier 3
\textbf{Description:} Integrated derivation of the charged lepton spectrum.

\subsection{14. CKM and PMNS Mixing from Cubic Ledger Topology}
\textbf{File:} \texttt{papers/tex/masses\_paper2\_mixing.tex}
\textbf{Priority:} Tier 3
\textbf{Description:} Extends the ledger to derive mixing angles from geometry.

\subsection{15. Neutrino Sector No-Go}
\textbf{File:} \texttt{papers/tex/Neutrino-Sector.tex}
\textbf{Description:} Negative result establishing constraints on the neutrino sector.

\subsection{16. Neutrino Masses and the Deep $\varphi$-Ladder}
\textbf{File:} \texttt{papers/tex/masses\_paper3\_neutrinos.tex}
\textbf{Priority:} Tier 3
\textbf{Description:} Derives absolute neutrino masses on fractional rungs.

\subsection{17. Recognition Science: Foundations Summary}
\textbf{File:} \texttt{papers/tex/RS-Foundations.tex}
\textbf{Description:} A summary synthesis of the physics track.

\section{V. Track B: Computation (LNAL)}

\subsection{18. Reality as Executable Code: LNAL Theory}
\textbf{File:} \texttt{papers/tex/Reality\_as\_Executable\_Code\_LNAL.tex} (check path)
\textbf{Description:} Defines the Light-Native Assembly Language (LNAL) and its instruction set.

\subsection{19. A Universal Register Mapping for LNAL}
\textbf{File:} \texttt{papers/tex/LNAL-Register-Mapping.tex}
\textbf{Description:} Technical spec for mapping physical systems into LNAL registers.

\section{VI. Track D: Gravity \& Cosmology}

\subsection{23. Recognition Science Baryogenesis}
\textbf{File:} \texttt{papers/tex/Baryogenesis-HubbleTensionSet.tex}
\textbf{Priority:} Tier 3
\textbf{Description:} Parameter-free origin of matter-antimatter asymmetry.

\subsection{24. Zero-Parameter Quantum Gravity}
\textbf{File:} \texttt{papers/tex/Quantum-Gravity-New-HubbleTensionSet.tex}
\textbf{Description:} Full QG derivation from discrete recognition calculus.

\subsection{59. Octave Gravity}
\textbf{File:} \texttt{papers/tex/octave-gravity.tex}
\textbf{Description:} Derives Geometric Gravity from the 8-step update cycle.

\subsection{25. ILG Scaffold}
\textbf{File:} \texttt{papers/tex/ILG-GPT5.tex}
\textbf{Description:} Audit-ready scaffold for Information-Limited Gravity.

\subsection{28. Gravity as Pressure}
\textbf{File:} \texttt{papers/tex/Pressure-Gravity.tex}
\textbf{Description:} Recasts ILG as an effective pressure field.

\subsection{29. The Coercive Projection Law of Gravity}
\textbf{File:} \texttt{papers/tex/CPM-Gravity.tex}
\textbf{Priority:} Tier 3
\textbf{Description:} Elevates ILG to a universal coercive projection principle.

\subsection{30. Zero-Parameter Galaxy Rotation Curves}
\textbf{File:} \texttt{papers/ILG\_Galaxy\_Rotation\_Curves.tex}
\textbf{Priority:} Tier 3
\textbf{Description:} Formal zero-parameter test against SPARC data.

\subsection{31. Convergence of Empirical Optimization}
\textbf{File:} \texttt{papers/ILG\_Validation\_Synthesis.tex}
\textbf{Description:} Compares blind optimization to RS-derived values.

\subsection{32. Information-Limited Gravity: Source-Side Tests (Dark Energy)}
\textbf{File:} \texttt{papers/tex/Dark-Energy-HubbleTensionSet.tex}
\textbf{Priority:} Tier 3
\textbf{Description:} Tests ILG kernel against cosmological observables.

\subsection{33. Late-time Recognition-Weighted Growth and Hubble Tension}
\textbf{File:} \texttt{papers/tex/Hubble-Tension-Resolution.tex}
\textbf{Priority:} Tier 3
\textbf{Description:} Applies RW kernel to resolve the Hubble Tension.

\section{VII. Track E: Mathematics}

\subsection{34. A Weighted Diagonal Operator... (Riemann Hypothesis)}
\textbf{File:} \texttt{papers/tex/Recognition-Riemann-Final.tex}
\textbf{Description:} RS approach to RH via spectral stability of the cost Hamiltonian.

\subsection{36. Goldbach via a Mod-8 Kernel}
\textbf{File:} \texttt{papers/tex/goldbach\_rs-arXiv.tex}
\textbf{Description:} Connects additive prime theory to the 8-tick ledger.

\section{VIII. Track F: Life \& Consciousness}

\subsection{54. Entropy Is an Interface}
\textbf{File:} \texttt{papers/tex/entropy-is-a-interface-arXiv.tex}
\textbf{Description:} Reframes entropy as code length; resolves reversibility paradox.

\subsection{55. The Statistical Mechanics of Recognition}
\textbf{File:} \texttt{papers/tex/Recognition\_Thermodynamics.tex}
\textbf{Description:} Thermodynamics of the cost function; Recognition Temperature.

\subsection{56. Darwin as Minimum Description Length}
\textbf{File:} \texttt{papers/tex/evolution-arXiv.tex}
\textbf{Description:} Unifies biological evolution with J-cost minimization (MDL).

\subsection{57. The Recognition Instrument for Abiogenesis}
\textbf{File:} \texttt{papers/tex/Recognition-Abiogenesis-arXiv.tex}
\textbf{Description:} Mechanism for origin of life via phi-timing gates.

\subsection{26. Protein Folding from First Principles}
\textbf{File:} \texttt{papers/tex/protein-dec-6.tex}
\textbf{Description:} Bio-Clocking theorem and hydration gearbox mechanism.

\subsection{27. A CPM Companion for Protein Folding}
\textbf{File:} \texttt{papers/tex/CPM-Folding-Companion-arXiv.tex}
\textbf{Description:} Instantiates CPM for the protein folding domain.

\subsection{43. Light as Consciousness}
\textbf{File:} \texttt{Light\_Consciousness\_Combined.tex}\\
\textbf{Date:} February 2026

\textbf{Description:}
Shows that the unique information-cost functional $J(x)=\frac{1}{2}(x+x^{-1})-1$ governs quantum measurement ($C=2A \Rightarrow$ Born weights), photonic operations (additive FOLD costs), and operational conscious selection. Proves a classification theorem: under bridge obligations, ConsciousProcess $\leftrightarrow$ PhotonChannel with uniqueness up to units, and only EM satisfies feasibility. Four classification lemmas (No-medium-knobs, Null-only, Maxwellization, BIOPHASE feasibility). Lean-verified.

\textbf{Dependencies:} Paper 3 (J-cost uniqueness), Paper 4 (Ledger Dynamics), Paper 19 (LNAL Registers).

\textbf{Why here?} It establishes the formal identity between light and operational consciousness---the bridge from physics to the meaning/consciousness track.

\subsection{63. Reciprocal Convex Costs for Ratio Matching}
\textbf{File:} \texttt{submitted-entropy-version-entropy-4136332.pdf}
\textbf{Description:} Formal characterization of the cost function (Entropy journal).

\subsection{64. CPM Method Closure}
\textbf{File:} \texttt{CPM\_Method\_Closure.tex}
\textbf{Description:} Domain-agnostic certificate for Coercive Projection.

\subsection{39. Optimization-Based Reference (Symbol Grounding)}
\textbf{File:} \texttt{papers/tex/Optimization\_Based\_Reference\_Symbol\_Grounding.tex}
\textbf{Description:} Resolves symbol grounding via internal argmin.

\subsection{40. Meaning is Forced}
\textbf{File:} \texttt{planning/papers/Meaning\_Is\_Forced.tex}
\textbf{Description:} Certificate bridge from closure to semantics.

\subsection{41. Universal Light Language}
\textbf{File:} \texttt{papers/tex/New-ULL-Periodic-Table-Meaning.tex}
\textbf{Description:} The ULL system paper; zero-parameter semantic pipeline.

\subsection{57/65. The Geometry of Transmutation}
\textbf{File:} \texttt{Geometry\_of\_Transmutation.tex}
\textbf{Description:} Phase-locking mechanism for non-local information transfer.

\subsection{62. Universal Light Qualia (ULQ)}
\textbf{File:} \texttt{papers/tex/light-field-saturation.tex} (Note: check content match)
\textbf{Description:} Geometry of feeling; qualia as strain tensor.

\subsection{46. Geometrodynamics of Consciousness}
\textbf{File:} \texttt{papers/tex/geometry\_of\_consciousness.tex}
\textbf{Description:} Mesoscale dynamics; 8-tick cadence as frame rate.

\subsection{47. The Topology of Self-Reference}
\textbf{File:} \texttt{papers/tex/Topology\_of\_Self\_Reference.tex}
\textbf{Description:} Topological characterization of the ``I am''.

\subsection{42. Phantom Light}
\textbf{File:} \texttt{papers/PhantomLight\_Paper.tex}
\textbf{Description:} Future neutrality constraints as present-time structure.

\section{IX. Track G: Ethics}

\subsection{44. Morality as a Conservation Law}
\textbf{File:} \texttt{papers/tex/Morality-As-Conservation-Law.tex}
\textbf{Description:} Derives moral law from physical ledger invariants.

\subsection{45. Virtues as Generators}
\textbf{File:} \texttt{papers/tex/Virtues-As-Generators.tex}
\textbf{Description:} Operationalizes ethics as 14 admissible operators.

\subsection{48. The Geometry of Evil}
\textbf{File:} \texttt{papers/tex/The\_Geometry\_of\_Evil.tex}
\textbf{Description:} Defines evil as geometric pathology (phantom loops).

\end{document}
