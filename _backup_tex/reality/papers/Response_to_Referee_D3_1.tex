\documentclass[11pt]{article}
\usepackage[margin=1in]{geometry}
\usepackage{amsmath,amssymb,amsthm,mathtools}
\usepackage[colorlinks=true,linkcolor=blue,citecolor=blue,urlcolor=blue]{hyperref}

\title{Response to Referee: Dimensional Rigidity / $D=3$}
\author{Washburn \& Zlatanovi\'c}
\date{}

\begin{document}
\maketitle

\paragraph{Referee comment (summary).}
The referee correctly notes that, as originally phrased, each of our three conditions (T),(K),(S) can read as a disguised restatement of $D=3$ rather than three \emph{convergently independent} constraints (in the set-theoretic sense that each condition allows multiple dimensions and only the intersection forces $D=3$). In particular, (S) is monotone in $D$ and therefore selects the smallest admissible $D$ once a lower bound is assumed.

\paragraph{Our response (high level).}
We agree with this point and have revised the manuscript to (i) make the logic explicit and non-circular, and (ii) separate two different roles played by our constraints:
\begin{itemize}
  \item \textbf{Weak (set-valued) constraints} that are each individually non-singleton and whose intersection is $\{3\}$ (a genuinely ``convergent'' pattern).
  \item \textbf{Sharp (characterizing) constraints} that, once adopted, already imply $D=3$; these are now presented as strengthened cross-checks / sharpenings rather than as the sole basis for an ``independence'' claim.
\end{itemize}

\paragraph{Where the manuscript changed.}
We updated (a) the Abstract language, (b) the Introduction (added an ``allowed-dimension set'' clarification and the weak A/B/C triad), (c) the statement/interpretation of (S) (explicitly a tie-breaker on an admissible set), and (d) downstream wording in the ``Main Result'' synthesis, summary table, and Conclusion to consistently reflect this logical structure.

\paragraph{1.\ ``Independent constraints'' and allowed-dimension sets.}
We added a short paragraph defining the allowed-dimension set
\[
\mathcal{A}_X := \{D\in\mathbb{N} : \text{constraint }(X)\text{ holds in dimension }D\}
\]
and clarified that our use of ``independent'' was intended to mean ``arising from distinct physical/mathematical sectors'' (topology vs.\ dynamics vs.\ computation), not that each $\mathcal{A}_X$ is non-singleton. To avoid ambiguity, we now explicitly include a \emph{set-theoretic} convergent triad (see item 3 below).

\paragraph{2.\ The role of (S) (synchronization).}
We agree that the minimization statement in (S) is an Occam/complexity tie-breaker once an admissible set of dimensions is specified. We revised the text around Theorem~(S) to state this plainly:
\begin{itemize}
  \item (S) does \emph{not} derive the lower bound (capacity) by itself; it selects the minimal synchronization overhead \emph{among admissible} $D$.
  \item We now treat (S) as a computational cost principle that is meaningful after other constraints have already ruled out low-dimensional cases (or after a capacity axiom fixes the admissible set).
\end{itemize}

\paragraph{3.\ New ``convergent independence'' triad (weak A/B/C).}
To address the referee's core logical concern, we now include (in the Introduction) a weaker triad of constraints whose allowed-dimension sets are non-singleton but whose intersection is $\{3\}$. Informally:
\begin{itemize}
  \item \textbf{(A) Same-dimension linking} (topological): existence of a nontrivial $\mathbb{Z}$-valued linking invariant between two same-dimensional extended objects implies $D$ is odd (so $\mathcal{A}_A=\{1,3,5,\dots\}$).
  \item \textbf{(B) Green-kernel orbital stability} (dynamical): stability of near-circular bound orbits under the Laplacian Green-kernel family forces $D<4$ (so $\mathcal{A}_B\supseteq\{1,2,3\}$ or, under mild exclusions, $\{2,3\}$).
  \item \textbf{(C) Minimal geometric capacity} (geometric/operational): a minimal capacity assumption excludes the $D=1$ case (so $\mathcal{A}_C\subseteq\{2,3,4,\dots\}$).
\end{itemize}
With these definitions, $\mathcal{A}_A\cap\mathcal{A}_B\cap\mathcal{A}_C=\{3\}$, matching the referee's requested ``nontrivial sets whose intersection is a singleton'' pattern.

\paragraph{4.\ Status of the sharper (T/K/S) statements.}
We retain the sharper statements (loop--loop linking; non-precession; $N=45$ synchronization minimality) but now label them as strengthened specializations and cross-checks, rather than the sole basis for an ``independent constraints'' claim.

\paragraph{5.\ Formalization note.}
We also tightened the Lean formalization of the $2^D$--45 arithmetic forcing by removing an artificial boundedness hypothesis from the Gap-45 derivation; the lemma is now fully general in $D$ (this does not change the paper's mathematical content, but improves the mechanized certificate).

\bigskip
\noindent We thank the referee for pushing us to sharpen the logical structure and presentation.

\end{document}

