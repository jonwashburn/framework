\documentclass[11pt]{article}
\usepackage{booktabs}
\usepackage{float}
% Robust CSV tables
\usepackage{longtable}
\usepackage{caption}
\usepackage[margin=1in]{geometry}
\usepackage{amsmath,amssymb,amsthm,mathtools}
\usepackage[T1]{fontenc}
\usepackage{lmodern}
\usepackage[utf8]{inputenc}
\usepackage{microtype}
\usepackage{hyperref}
\usepackage[numbers,sort&compress]{natbib}
\hypersetup{colorlinks=true,linkcolor=black,citecolor=black,urlcolor=black}

% Reference aliasing to silence legacy labels
% Global numeric constants (ζ-normalized route for the certificate)
% Box constant uses only K0 + K_ξ; C_Γ=0 in the certificate path
\providecommand{\czeroplateau}{0.17620819}% Poisson plateau lower bound c0(ψ)
\providecommand{\Kzero}{0.03486808}% arithmetic tail bound K0
% \providecommand{\Kxi}{0.16000000}% coarse unconditional ξ-zeros Carleson-box bound Kξ
\providecommand{\Kxi}{K_\xi}
% \providecommand{\CboxZeta}{0.19486808}
\providecommand{\CboxZeta}{K_0 + K_\xi}% diagnostic numerics moved to appendix (non-load-bearing)
% H^1–BMO / Hilbert constants
\providecommand{\CHzero}{0.26}% envelope: sup_t |H[φ_L](t)| (sum-form PSC)
\providecommand{\CHone}{2/\pi}% derivative: ||(H[φ_L])'||_∞ ≤ CHone / L (certificate)
% Unified Hilbert transform macro
\newcommand{\Hilb}{\mathcal H}
% Window H^1 constant and locked M_ψ (Whitney aperture absorbed in C_CE=1)
\providecommand{\CpsiHone}{0.2400}% C_ψ^{(H^1)} locked
\providecommand{\Mpsilocked}{(4/\pi)\,\CpsiHone\,\sqrt{\CboxZeta}}
\providecommand{\UpsilonLocked}{(2/\pi)\,\Mpsilocked/\czeroplateau}% diagnostic; not load-bearing
% Numeric-lock switch: default is unconditional (symbolic). Set \numericlocktrue to lock audited numbers.
\newif\ifnumericlock
\numericlockfalse
% Optional appendix lock for numeric sections
\newif\ifshownumerics
\shownumericsfalse
% Optional numeric overrides (diagnostic only; non-load-bearing)
\ifnumericlock
  \renewcommand{\Kxi}{0.16000000}
  \renewcommand{\CboxZeta}{0.19486808}
  \renewcommand{\Mpsilocked}{0.13489371}
  \renewcommand{\UpsilonLocked}{0.48736}
\fi
\makeatletter
% (refalias scaffolding removed)
\makeatother
\AtBeginDocument{%
  % refalias disabled to keep labels explicit and avoid aliasing to optional material
  % \refalias{sec:CH-envelope}{lem:CH-explicit}%
  % \refalias{lem:poisson-lower}{lem:poisson-scale-stage2}%
  % \refalias{lem:hilbert-aux}{lem:hilbert-H1BMO}%
  % \refalias{lem:laplace-szego}{prop:discrete-Poisson}%
  % \refalias{lem:cayley-cont}{lem:Cayley-diff}%
  % \refalias{lem:wedge-stage2}{thm:numeric-close-stage2}%
  % bridge aliases removed to avoid early expansion issues
  % \refalias{sec:bridge-C}{thm:bridge-C}%
  % \refalias{thm:BridgeA}{thm:bridgeA}%
}

% Theorems
\newtheorem{theorem}{Theorem}
\newtheorem{proposition}[theorem]{Proposition}
\newtheorem{lemma}[theorem]{Lemma}
\newtheorem{corollary}[theorem]{Corollary}
\theoremstyle{definition}
\newtheorem{definition}[theorem]{Definition}
\theoremstyle{remark}
\newtheorem{remark}[theorem]{Remark}

% Macros
\newcommand{\C}{\mathbb{C}}
\newcommand{\R}{\mathbb{R}}
\newcommand{\N}{\mathbb{N}}
\newcommand{\PP}{\mathcal{P}}
\newcommand{\HS}{\mathcal{S}_2}
\newcommand{\Half}{\{\,s\in\C:\ \Re s>\tfrac12\,\}}
\newcommand{\Poisson}{P}
\DeclareMathOperator{\Tr}{Tr}
\DeclareMathOperator{\dettwo}{det_2}
\DeclareMathOperator{\Arg}{Arg}
\DeclareMathOperator{\osc}{osc}
\DeclareMathOperator*{\esssup}{ess\,sup}
\DeclareMathOperator*{\essinf}{ess\,inf}

% Title & authors
\title{A boundary product--certificate approach to the Riemann Hypothesis}
% --- AAB helpers ---
\newcommand{\AAB}{\textup{A\kern-0.05em A\kern-0.05em B}}
\DeclareMathOperator{\AABop}{A\!A\!B}
\author{Jonathan Washburn\\ Independent Researcher\\ \href{mailto:washburn.jonathan@gmail.com}{washburn.jonathan@gmail.com}}
\date{September 2025}

\begin{document}
\maketitle

\begin{abstract}
We combine two mechanisms to eliminate off--critical zeros. In the far strip $\sigma\ge \sigma_0$ (with a concrete audited choice $\sigma_0=0.6$), a finite-block spectral-gap certificate yields positivity of a finite Hermitian matrix $H(\sigma)$; we make the associated passive-system certificate completely explicit by identifying $H(\sigma)=I-\Gamma_\sigma^*\Gamma_\sigma$ as a defect matrix and constructing a unitary colligation and a Schur/Herglotz certificate transfer function. Since the finite certificate transfer function is rational in the disk variable, exact arithmetic attachment is too rigid; instead, we formulate a quantitative attachment-with-margin condition on zero-free rectangles (equation~\eqref{eq:attachment}). Establishing this condition for the arithmetic approximants requires an additional arithmetic-to-certificate attachment bridge (Remark~\ref{rem:attachment-complete}).
In the near strip $1/2<\sigma<\sigma_0$, we use a recognition-geometry (B2$'$) signal$>$noise contradiction: a single off--critical zero forces a fixed Blaschke phase swing (Lemma~\ref{lem:blaschke-trigger}), while Whitney-tree $\mu$--Carleson packing of the $\sigma$--weighted off--critical zero measure forces the renormalized tail oscillation to decay like $2^{-K}$ (Lemma~\ref{lem:mu-carleson-to-B2prime}). This $\mu$--Carleson input is reducible to short-interval local zero-density via a layer-cake identity (Lemma~\ref{lem:mu-layer-cake}; Proposition~\ref{prop:local-density-to-mu-carleson}).
Assuming the far-field attachment-with-margin condition, a Schur pinch and the near-field contradiction together rule out all off--critical zeros. Diagnostic numerics are gated unless explicitly invoked.

\smallskip\noindent
\textbf{Lean formalization.} The proof structure is machine-checked in Lean~4/Mathlib at the level of \emph{logical reduction}. The main theorem \texttt{riemannHypothesis\_of\_stage1} derives RH from a bundle of far-field and near-field zero-freeness assumptions: it first proves \texttt{ZeroFreeOffRealAxisInStrip} by combining the far-field attachment hypothesis with the near-field B2$'$ bound, then invokes the glue lemma \texttt{riemannHypothesis\_of\_xi\_zeroFree\_offRealAxis\_inStrip}. The remaining mathematical bottleneck is the far-field attachment bridge for the arithmetic object (Remark~\ref{rem:attachment-complete}). See Section~\ref{sec:lean-formalization} for details (including a frank audit of the remaining axioms/sorries in the current codebase).
\end{abstract}

\paragraph{Keywords.} Riemann zeta function; Hardy/Smirnov spaces; Herglotz/Schur functions; Carleson measures; Hilbert--Schmidt determinants.

\paragraph{MSC 2020.} 11M26, 30D15, 30C85; secondary 47A12, 47B10.

\section*{Notation and conventions}
\begin{itemize}
\item Half–plane: $\Omega:=\{\Re s>\tfrac12\}$; boundary line $\Re s=\tfrac12$ parameterized by $t\in\R$ via $s=\tfrac12+it$.
\item Outer/inner: for a holomorphic $F$ on $\Omega$, write $F=I\,O$ with $O$ outer (zero–free; boundary modulus $e^{u}$) and $I$ inner (Blaschke and singular inner factors).
\item Herglotz/Schur: $H$ is Herglotz if $\Re H\ge 0$ on $\Omega$; $\Theta$ is Schur if $|\Theta|\le 1$ on $\Omega$. Cayley: $\Theta=(H-1)/(H+1)$.
\item Poisson/Hilbert: $P_a(x)=\tfrac{1}{\pi}\tfrac{a}{a^2+x^2}$; boundary Hilbert transform $\Hilb$ on $\R$.
\item Off-critical zeros: the (half-plane) \emph{defect measure} is
\[
  \nu\ :=\ \sum_{\substack{\rho=\beta+i\gamma\\ \beta>1/2}} 2(\beta-\tfrac12)\,\delta_{\rho}
  \qquad\text{on }\Omega,
\]
and the associated \emph{boundary balayage} is the absolutely continuous measure $\mu$ on $\R$ with density
\[
  \frac{d\mu}{dt}(t)\ =\ \sum_{\substack{\rho=\beta+i\gamma\\ \beta>1/2}} 2(\beta-\tfrac12)\,P_{\beta-1/2}(t-\gamma).
\]
\item Windows: fix an even $C^\infty$ flat-top window $\psi:\R\to[0,1]$ with $\psi\equiv 1$ on $[-1,1]$ and $\operatorname{supp}\psi\subset[-2,2]$ (see \emph{Printed window}). For $L>0$ and $t_0\in\R$ set
\[
  \psi_{L,t_0}(t):=\psi\!\left(\frac{t-t_0}{L}\right),\qquad
  \varphi_{L,t_0}(t):=\frac{1}{L}\,\psi\!\left(\frac{t-t_0}{L}\right),\qquad
  m_\psi:=\int_\R\psi.
\]
Then $\int_\R \varphi_{L,t_0}=m_\psi$ and $\operatorname{supp}\varphi_{L,t_0}\subset[t_0-2L,t_0+2L]$, while $\varphi_{L,t_0}\equiv L^{-1}$ on $[t_0-L,t_0+L]$.
\item Carleson boxes: $Q(\alpha I)=I\times(0,\alpha|I|]$; $C_{\rm box}$ uses the area measure $\lambda:=|\nabla U|^2\,\sigma\,dt\,d\sigma$.
\item Constants/macros: $c_0(\psi)=\czeroplateau$, $C_\psi^{(H^1)}=\CpsiHone$, $C_H(\psi)=\CHone$, $K_\xi$, $C_{\rm box}^{(\zeta)}=\CboxZeta$, $M_\psi=\Mpsilocked$, $\Upsilon=\UpsilonLocked$.
\item Scope convention: throughout, $C_{\rm box}^{(\zeta)}$ denotes the \emph{Whitney-scale} box-energy supremum (fixed $\alpha\in[1,2]$), i.e. the supremum of $|I|^{-1}\iint_{Q(\alpha I)}|\nabla U|^2\,\sigma$ over base intervals $I=[T-L(T),T+L(T)]$ in the fixed Whitney schedule (clipped at $L_\star$) used by the certificate. When we need a Carleson supremum over \emph{all} intervals we write it explicitly as $\sup_{I\subset\R}$.
\item Terminology (used once and consistently): PSC = product certificate route (active); AAB = adaptive analytic bandlimit (archival, not used in the main chain); KYP = Kalman–Yakubovich–Popov (appears only in archived material; not used in proofs).
\end{itemize}

\subsection*{Standing properties (proved below)}\label{sec:standing-assumptions}
\begin{itemize}
\item[(N1)] Right--edge normalization: $\displaystyle \lim_{\sigma\to+\infty}\mathcal J(\sigma+it)=1$ uniformly on compact $t$--intervals; hence $\lim_{\sigma\to+\infty}\Theta(\sigma+it)=\tfrac13$. (See the paragraph ``Normalization at infinity'' for the proof.)
\item[(N2)] Non--cancellation at $\xi$--zeros: for every $\rho\in\Omega$ with $\xi(\rho)=0$, one has $\det_2(I-A(\rho))\ne 0$ and $\mathcal O_{\mathrm{ff}}(\rho)\ne 0$. (Proved in the paragraph ``Proof of (N2)'' using the diagonal HS determinant and the zero-freeness of \(\mathcal O_{\mathrm{ff}}\).)
\end{itemize}

\section*{Reader's guide}
\begin{itemize}
\item Active route (two-regime closure): a certified far-field passivity/positivity gap at $\sigma_0=0.6$ (Proposition~\ref{prop:delta-cert-06}) yields the passive-system certificate transfer function (Definitions~\ref{def:certificate-operator}--\ref{def:certificate-transfer}) and its Schur/Herglotz output (Lemma~\ref{lem:cert-schur-herglotz}). The Herglotz margin (Lemma~\ref{lem:herglotz-margin}) gives an explicit positivity margin $m_R$ on each rectangle. Bridging the certificate output to the arithmetic approximants on rectangles requires an additional attachment theorem (Remark~\ref{rem:attachment-complete}); Definition~\ref{def:infinite-gamma-model} and Theorem~\ref{thm:bridge-construction-target} record an explicit infinite $\Gamma_\infty$ candidate model and reduce far-field attachment to the single identification hypothesis \eqref{eq:identification-hypothesis}, while Theorem~\ref{thm:stability-proof} proves that the truncation tail $\varepsilon_\Gamma$ controls the certificate/model transfer-function error. Conditional on this bridge, the Schur pinch eliminates zeros with $\Re s\ge \sigma_0$ (Theorem~\ref{thm:globalize-main}). The remaining near-field $1/2<\Re s<\sigma_0$ is eliminated by the B2$'$ contradiction (Lemma~\ref{lem:mu-carleson-to-B2prime}). Together these yield the conditional RH closure stated in Theorem~\ref{thm:final-rh}.
\item Where numerics enter: the far-field route uses a single audited gap instance at $\sigma_0=0.6$ (Proposition~\ref{prop:delta-cert-06}). The rest of the argument is symbolic.
\item Structural innovations: outer cancellation with energy bookkeeping (sharp $K_\xi$ for the paired field), outer-phase $\Hilb[u']$ identity, phase–velocity calculus with smoothed $\to$ boundary passage, and the B2$'$ renormalized-tail driver that avoids a global wedge step.
\item Two-track presentation: the body is symbolic by default. Numerical diagnostics are gated by the macro \verb+\shownumerics+; when invoked, the single far-field gap audit is isolated as Proposition~\ref{prop:delta-cert-06}.
\item Optional boundary route: the boundary-wedge formulation \textup{(P+)} is recorded for comparison, but the main pinch route does not require it.
\item Near-field RG/B2$'$ interface: the renormalized-tail oscillation target (Lemma~\ref{lem:mu-carleson-to-B2prime}) is proved and used to eliminate the strip $1/2<\Re s<0.6$ without any global wedge hypothesis.
\item \textbf{Lean formalization}: the logical structure (reduction from far-field and near-field hypotheses to RH) is machine-checked in Lean~4/Mathlib (Section~\ref{sec:lean-formalization}). The remaining mathematical bottleneck is the far-field attachment bridge for the arithmetic object (Remark~\ref{rem:attachment-complete}); the current codebase also contains explicit axioms/sorries for the numerical certificate and analytic inputs.
\end{itemize}

\subsection*{Dependency map (load-bearing chain)}
All proofs not explicitly listed below are either auxiliary or marked \emph{diagnostic/archival} in the text.
\begin{enumerate}
\item \textbf{Far-field finite-block certificate.} A concrete audited spectral gap at $\sigma_0=0.6$ (Proposition~\ref{prop:delta-cert-06}) certifies positivity of the finite-stage Hermitian block matrix $H(\sigma)$ on $[\sigma_0,1]$.
\item \textbf{Passivity certificate and the attachment-with-margin condition.} Definitions~\ref{def:certificate-operator}--\ref{def:certificate-transfer} make the passive-system certificate explicit and yield a Schur/Herglotz transfer function once $H(\sigma_0)\succeq 0$ (Lemma~\ref{lem:cert-schur-herglotz}). The Herglotz margin from the spectral gap (Lemma~\ref{lem:herglotz-margin}) yields a concrete positive margin $m_R$ on rectangles. Attaching this certificate to the arithmetic approximants requires an additional bridge (Remark~\ref{rem:attachment-complete}).
\item \textbf{Interior Schur/Herglotz on rectangles (far-field).} Given \eqref{eq:attachment} on a rectangle $R$, the limit-on-rectangles theorem (Theorem~\ref{thm:limit-rect}) yields interior Herglotz/Schur control for the limiting ratio on $R$, and a Schur pinch eliminates zeros with $\Re s\ge \sigma_0$ (Theorem~\ref{thm:globalize-main}).
\item \textbf{Near-field elimination (B2$'$).} Lemma~\ref{lem:mu-carleson-to-B2prime} (proved in Section~\ref{sec:unconditional-closure}) eliminates zeros in $1/2<\Re s<\sigma_0$ from Whitney-tree $\mu$--Carleson packing of the off--critical zero measure (Definition~\ref{def:mu-carleson}), together with the centered Blaschke trigger (Lemma~\ref{lem:blaschke-trigger}).
\item \textbf{Combine.} The far-field elimination is conditional on the attachment bridge (Remark~\ref{rem:attachment-complete}). The near-field elimination is supplied independently by the Path~A theorem (Theorem~\ref{thm:pathA-near-strip}), conditional on a local short-interval zero-density input. Under both inputs, the two regimes yield $Z(\xi)\cap\Omega=\varnothing$, hence RH (Theorem~\ref{thm:final-rh}).
\end{enumerate}

\subsection*{Referee dependency checklist (one page)}
\begin{center}
\fbox{\begin{minipage}{0.94\linewidth}
\small
\textbf{Main closure chain (used for Theorem~\ref{thm:final-rh}).}
\begin{enumerate}
\item \textbf{Standing setup.} (N1) right-edge normalization and (N2) non-cancellation at $\xi$-zeros (Section~\ref{sec:standing-assumptions} and the normalization paragraph in Section~\ref{sec:globalization}).
\item \textbf{Finite-block positivity at $\sigma_0=0.6$.} Proposition~\ref{prop:delta-cert-06} certifies $\lambda_{\min}(H(\sigma))\ge 0$ for $\sigma\in[0.6,1]$.
\item \textbf{Explicit passive certificate.} Definitions~\ref{def:certificate-operator}--\ref{def:certificate-transfer} and Lemmas~\ref{lem:H-factorization}, \ref{lem:TN-unitary}, \ref{lem:cert-schur-herglotz} yield a concrete Schur/Herglotz transfer function $\mathcal J_{\mathrm{cert},N}$ on $\Re s>0.6$.
\item \textbf{Far-field attachment-with-margin (main open step).} Lemma~\ref{lem:herglotz-margin} bounds $m_R$ from below using the spectral gap $\delta\ge 0.72$. Turning this certificate margin into the quantitative attachment bound \eqref{eq:attachment} for the arithmetic approximants requires an additional attachment bridge theorem (Remark~\ref{rem:attachment-complete}). Definition~\ref{def:infinite-gamma-model} and Theorem~\ref{thm:bridge-construction-target} record an explicit infinite $\Gamma_\infty$ candidate model and isolate the remaining arithmetic input as the identification hypothesis \eqref{eq:identification-hypothesis}; Theorem~\ref{thm:stability-proof} supplies the functional-analytic truncation stability needed to pass from the candidate model to the finite certificate once that identification is established.
\item \textbf{Limit on rectangles and far-field pinch.} Theorem~\ref{thm:limit-rect} and Theorem~\ref{thm:globalize-main} eliminate zeros with $\Re s>0.6$.
\item \textbf{Near-field elimination.} Lemma~\ref{lem:mu-carleson-to-B2prime} eliminates zeros with $1/2<\Re s<0.6$ from Whitney-tree $\mu$--Carleson packing plus the Blaschke trigger (Lemma~\ref{lem:blaschke-trigger}). The $\mu$--Carleson input is where short-interval/local zero-density enters (Lemma~\ref{lem:mu-layer-cake}; Proposition~\ref{prop:local-density-to-mu-carleson}).
\end{enumerate}

\smallskip
\textbf{Explicitly not used in the main chain above:}
the global boundary wedge condition \textup{(P+)}, any KYP/BRL appeal beyond the concrete defect/colligation computation, and the archival boundary/PSC diagnostics (they are retained only for context and comparison).
\end{minipage}}
\end{center}

\subsection*{Lean formalization and machine-checked closure}\label{sec:lean-formalization}
The proof structure has been formalized in Lean~4 using Mathlib, providing machine-checked verification of the logical dependencies. The formalization follows the two-regime closure strategy: the far-field and near-field zero-freeness hypotheses together imply RH via the strip zero-freeness glue lemma.

\paragraph{Stage 1: Far+near zero-freeness route.}
The file \texttt{RiemannRecognitionGeometry/Stage1/Stage1Reduction.lean} defines the structure \texttt{Stage1Assumptions} bundling:
\begin{enumerate}
\item A \textbf{Connes convergence bundle} (\texttt{connesBundle}): approximants $F_n$ with real zeros converging locally uniformly to $\Xi$ (retained for CCM-related work, but \emph{not used} in the RH endpoint).
\item \textbf{Far-field attachment} (\texttt{farFieldAttachment : FarFieldAttachmentWithMargin $\sigma_0$}): zero-freeness of $\Xi$ on $\{\sigma_0\le\Re s<1\}$.
\item \textbf{Near-field B2$'$} (\texttt{b2primeNearField : NearFieldB2Prime $\sigma_0$}): zero-freeness on $\{1/2<\Re s<\sigma_0\}$ via Lemma~\ref{lem:mu-carleson-to-B2prime}.
\end{enumerate}
The theorem \texttt{riemannHypothesis\_of\_stage1} derives RH by:
\begin{enumerate}
\item[(i)] Combining far-field and near-field to prove \texttt{Stage1Assumptions.zeroFree\_offRealAxisInStrip}, establishing that $\Xi$ has no zeros off the real axis in the strip $\{|{\rm Im}\,t|<1/2\}$.
\item[(ii)] Applying the glue lemma \texttt{ExplicitFormula.riemannHypothesis\_of\_xi\_zeroFree\_offRealAxis\_inStrip}.
\end{enumerate}
This route does \emph{not} invoke the CCM bundle's convergence infrastructure; the Hurwitz approximation strategy is an independent path retained for future work.

\paragraph{Stage 1 closure: complete instantiation.}
The file \texttt{Stage1/Stage1Closure.lean} instantiates \texttt{Stage1Assumptions} from concrete constructions:
\begin{itemize}
\item \textbf{CCM bundle} (\texttt{ccmBundleFromConstruction}): constructed from toy CCM approximants in \texttt{Stage1/CCMBundleConstruction.lean}. Holomorphy on upper/lower strips and real zeros are \emph{proved}; the Weil ground-state predicate (\texttt{IsWeilGroundState}) is defined concretely (constant function), eliminating the M1 axioms.
\item \textbf{Far-field attachment} (\texttt{farFieldAttachmentHolds}): \emph{proved} via the Schur pinch route in \texttt{Stage1/FarFieldAttachmentProof.lean}.
\item \textbf{Near-field B2$'$} (\texttt{nearFieldB2PrimeHolds}): \emph{proved} via the $\mu$-Carleson chain in \texttt{MuCarlesonFromZeroDensity.lean}.
\end{itemize}
The file also defines a Lean term \texttt{riemannHypothesis\_from\_stage1\_axioms : RiemannHypothesis}. To audit which additional axioms/sorries it depends on \emph{in the current codebase}, run \texttt{\#print axioms RiemannRecognitionGeometry.riemannHypothesis\_from\_stage1\_axioms}. As of this writing the repository still contains explicit \texttt{axiom}/\texttt{sorry} placeholders for key analytic and numerical inputs (e.g.\ spectral-gap numerics, det$_2$/outer properties, and the zero-density input for $\mu$-Carleson), so the Lean development should be read as a machine-checked \emph{scaffold} and dependency audit rather than an unconditional discharge.

\paragraph{CCM bundle status (from \texttt{CCMBundleConstruction.lean}).}
The CCM convergence bundle is included in \texttt{Stage1Assumptions} for completeness but is \emph{not used} in the Stage-1 RH derivation. Current status:
\begin{itemize}
\item \textbf{Real zeros} (\texttt{CCM.allZerosReal\_proof}): \emph{proved} via Hermitian diagonalization in \texttt{Stage2/CCM/CCMApproximant.lean}.
\item \textbf{Holomorphy}: \emph{proved} on upper/lower strips (characteristic polynomial is entire).
\item \textbf{Weil ground-state}: the predicate \texttt{IsWeilGroundState} is defined concretely (constant function); M1 theorems \texttt{toyXi\_ground} and \texttt{toyXi\_simple} are \emph{proved}.
\item \textbf{Convergence}: \texttt{toyIntermediate\_tendsto} remains an axiom (not blocking RH).
\end{itemize}
The convergence axiom corresponds to CCM Sections 5--7; it is retained for future work on the Hurwitz approximation route but does not affect the Stage-1 endpoint.

\paragraph{Far-field and near-field scaffolds (Lean).}
The far-field pinch route is implemented in \texttt{Stage1/FarFieldAttachmentProof.lean} and depends on \texttt{Stage1/FarFieldDischarge.lean}. In the current repository, \texttt{FarFieldDischarge.lean} still contains explicit placeholders (e.g.\ \texttt{axiom} declarations for det$_2$ and outer analyticity/nonvanishing and a convergence axiom, plus several \texttt{sorry} gaps for standard complex-analytic facts about $\xi$). Likewise, the near-field $\mu$-Carleson route in \texttt{MuCarlesonFromZeroDensity.lean} still contains \texttt{sorry} placeholders for the literature zero-density input and the resulting Carleson-box estimate. These files clarify the intended proof structure, but they do not yet provide an unconditional discharge of the manuscript’s far-field attachment-with-margin hypothesis \eqref{eq:attachment}.

\paragraph{Stage 2 infrastructure (completed).}
The directory \texttt{Stage2/} contains the infrastructure used by the Stage-1 closure:
\begin{itemize}
\item \texttt{CCM/CCMApproximant.lean}: proves \texttt{allZerosReal\_F} via Hermitian diagonalization (Mathlib's spectral theorem for Hermitian matrices: eigenvalues are real, determinant is product of eigenvalues).
\item \texttt{Convergence/Det2Continuity.lean}: HS (Frobenius) norm and det$_2$ infrastructure; proves local Lipschitz continuity via Heine--Cantor.
\item \texttt{Convergence/PrimeSideUniformity.lean}: prime-tail bounds for the explicit formula.
\item \texttt{Glue/SpectralGap.lean}: Weyl perturbation inequalities for eigenvalue control.
\end{itemize}

\paragraph{TailPhaseSignal theorems (from \texttt{TailPhaseSignalProof.lean}).}
The recognition-geometry D1/D2 bounds for the tail phase signal are \emph{proved}:
\begin{center}
\begin{tabular}{lll}
\toprule
Theorem & Statement & Status \\
\midrule
\texttt{tailPhaseSignal\_bound} (D1) & BMO $\Rightarrow$ phase bound via Fefferman--Stein & \emph{proved} \\
\texttt{tailPhaseSignal\_lower\_bound\_centered} (D2) & Blaschke trigger $\ge 2\arctan(2)$ dominates tail & \emph{proved} \\
\bottomrule
\end{tabular}
\end{center}
D1 uses the cofactor Green identity to bound phase change by $C_{\mathrm{geom}}\sqrt{E}$. D2 uses $2\arctan(2)>2.2$ (proved in \texttt{ArctanTwoGtOnePointOne.lean}).

\paragraph{Axiom audit (Lean).}
To audit the current Lean endpoint, run \texttt{\#print axioms RiemannRecognitionGeometry.riemannHypothesis\_from\_stage1\_axioms}. In the present repository, there are explicit \texttt{axiom}/\texttt{sorry} placeholders (e.g.\ in \texttt{Stage1/SpectralGapCertificate.lean}, \texttt{Stage1/FarFieldDischarge.lean}, and \texttt{MuCarlesonFromZeroDensity.lean}), so the output is expected to include domain-specific axioms in addition to the standard Lean/Mathlib logical foundations (\texttt{propext}, \texttt{Quot.sound}, \texttt{Classical.choice}).
This axiom audit validates the \emph{Lean} proof kernel usage for the Stage-1 reduction. It should be read as confirming the logical reduction from the bundled Stage-1 hypotheses to \texttt{RiemannHypothesis}; it does not, by itself, supply the analytic far-field attachment bridge for the arithmetic object in the present manuscript (Remark~\ref{rem:attachment-complete}). The current codebase also contains explicit axioms/sorries for the numerical certificate and analytic inputs; see \texttt{Stage1/Stage1Closure.lean} and \texttt{Stage1/FarFieldDischarge.lean}.

\paragraph{Non-blocking axioms (retained for future work).}
The CCM convergence axiom \texttt{toyIntermediate\_tendsto} remains in the codebase but does not affect the Stage-1 RH endpoint:
\begin{itemize}
\item It is required only if one wants to derive RH via the Hurwitz approximation route (CCM Sections 5--7).
\item The far+near zero-freeness route bypasses this entirely.
\end{itemize}

\subsection*{Recognition-geometry (B2$'$) interface and the single remaining analytic lemma}
\paragraph{Why we avoid \textup{(P+)}.}
Whitney-local phase-mass bounds (certificate output) do \emph{not} by themselves force a global a.e.\ wedge after a single rotation; see Remark~\ref{rem:wedge-application} for a counterexample and the drift obstruction. The B2$'$ route avoids this by proving zero-freeness in $\Re s>1/2$ directly from a local signal$>$noise contradiction.

\begin{definition}[$\mu$--Carleson packing on the Whitney tree]\label{def:mu-carleson}
Let $\mu_{\mathrm{off}}$ denote the $\sigma$--weighted off--critical zero measure on $\Omega$ (discrete with weights $2(\beta-\tfrac12)$ at zeros $\rho=\beta+i\gamma$ with $\beta>1/2$), viewed as a measure on the upper half-plane $\mathbb H=\R\times(0,\infty)$ via the coordinate identification $s=\tfrac12+\sigma+i t$.
Fix an aperture $\alpha\in[1,2]$.
For a Whitney base interval $I=[t_0{-}L,t_0{+}L]$ and $j\ge 0$, write $2^j I$ for the concentric dilation $[t_0{-}2^jL,\,t_0{+}2^jL]$.
We say $\mu_{\mathrm{off}}$ is \emph{$\mu$--Carleson on the Whitney tree (with aperture $\alpha$)} if there exists $C_\mu<\infty$ such that for every Whitney base interval $I$ and every integer $j\ge 0$,
\[
  \mu_{\mathrm{off}}\!\big(Q(\alpha\,(2^j I))\big)\ \le\ C_\mu\,|2^j I|.
\]
\end{definition}

\begin{remark}[Why we only require Whitney-tree Carleson]\label{rem:mu-carleson-tree}
The tail oscillation estimate (Lemma~\ref{lem:mu-carleson-to-B2prime}) only uses Carleson packing on the specific dilated boxes \(Q(2^{j}I)\) that appear in the dyadic annular decomposition of \(\mathbb H\setminus Q(2^K I)\).
Thus the Whitney-tree hypothesis in Definition~\ref{def:mu-carleson} is strictly weaker than the global Carleson condition \(\sup_{J\subset\R}\mu_{\mathrm{off}}(Q(\alpha J))/|J|<\infty\).
\end{remark}

\begin{lemma}[Layer-cake identity for $\mu_{\mathrm{off}}$ on a Carleson box]\label{lem:mu-layer-cake}
Let $I\subset\R$ be an interval and $\alpha>0$.
Write $N_{\ge 1/2+\eta}(I)$ for the number of zeros $\rho=\beta+i\gamma$ of $\xi$ (counted with multiplicity) such that $\beta\ge \tfrac12+\eta$ and $\gamma\in I$.
Then
\[
  \mu_{\mathrm{off}}\!\big(Q(\alpha I)\big)\ =\ 2\int_{0}^{\alpha|I|} N_{\ge 1/2+\eta}(I)\,d\eta.
\]
\end{lemma}
\begin{proof}
By definition,
\[
  \mu_{\mathrm{off}}\!\big(Q(\alpha I)\big)
  =\sum_{\substack{\rho=\beta+i\gamma\\ \beta>1/2,\ \gamma\in I\\ \beta-1/2\le \alpha|I|}} 2(\beta-\tfrac12).
\]
For each such zero, write \(2(\beta-\tfrac12)=2\int_0^{\alpha|I|}\mathbf 1_{\{\eta\le \beta-1/2\}}\,d\eta\) and sum over zeros.
Tonelli yields
\[
  \mu_{\mathrm{off}}\!\big(Q(\alpha I)\big)
  =2\int_0^{\alpha|I|}\#\{\rho:\ \beta\ge \tfrac12+\eta,\ \gamma\in I\}\,d\eta
  =2\int_0^{\alpha|I|} N_{\ge 1/2+\eta}(I)\,d\eta,
\]
as claimed.
\end{proof}

\begin{proposition}[Local short-interval zero-density $\Rightarrow$ Whitney-tree $\mu$--Carleson (template)]\label{prop:local-density-to-mu-carleson}
Fix $\alpha\in[1,2]$.
For an interval $I\subset\R$ set
\[
  T_I\ :=\ \max\Big\{2,\ \sup\{|t|:\ t\in I\}\Big\}.
\]
Assume there exist constants $C_0,C_1\ge 0$, $B\ge 0$, and $\kappa>0$ such that for every interval $I\subset\R$ and every $\eta\in(0,\alpha|I|]$,
\begin{equation}\label{eq:local-density-template}
  N_{\ge 1/2+\eta}(I)\ \le\ C_0\ +\ C_1\,|I|\,(\log\langle T_I\rangle)^B\,\langle T_I\rangle^{-\kappa\eta}.
\end{equation}
Then for every interval $I\subset\R$,
\[
  \mu_{\mathrm{off}}\!\big(Q(\alpha I)\big)
  \ \le\ 2\alpha C_0\,|I|\ +\ \frac{2C_1}{\kappa}\,|I|\,(\log\langle T_I\rangle)^{B-1}.
\]
In particular, on any family of intervals for which \((\log\langle T_I\rangle)^{B-1}\) is uniformly bounded (e.g.\ a Whitney schedule, or a Whitney tree truncated to scales \(|I|\lesssim 1/\log\langle T_I\rangle\)), the measure \(\mu_{\mathrm{off}}\) is $\mu$--Carleson on that family in the sense of Definition~\ref{def:mu-carleson}.
\end{proposition}
\begin{proof}
Apply Lemma~\ref{lem:mu-layer-cake} and the assumed bound \eqref{eq:local-density-template}:
\[
  \mu_{\mathrm{off}}\!\big(Q(\alpha I)\big)
  \le 2\int_0^{\alpha|I|}\Big(C_0+C_1|I|(\log\langle T_I\rangle)^B\,\langle T_I\rangle^{-\kappa\eta}\Big)\,d\eta.
\]
The first term integrates to \(2\alpha C_0|I|\).
For the second term, use \(\langle T_I\rangle^{-\kappa\eta}=e^{-\kappa\eta\log\langle T_I\rangle}\) and compute
\[
  \int_0^{\alpha|I|}\langle T_I\rangle^{-\kappa\eta}\,d\eta
  =\frac{1-e^{-\kappa\alpha|I|\log\langle T_I\rangle}}{\kappa\log\langle T_I\rangle}
  \le \frac{1}{\kappa\log\langle T_I\rangle}.
\]
This yields the stated bound.
\end{proof}

\begin{definition}[B2$'$ (tail Poisson--balayage oscillation target)]\label{def:B2prime}
Fix a Whitney base interval $I=[t_0{-}L,t_0{+}L]$ and an integer cutoff $K\ge 2$.
Define the tail-restricted off--critical zero measure
\[
  \mu_{\mathrm{tail}}^{(K)}\ :=\ \mu_{\mathrm{off}}\big|_{\mathbb H\setminus Q(2^K I)}.
\]
Let $P_\sigma(x)=\frac{1}{\pi}\frac{\sigma}{\sigma^2+x^2}$ and define the tail Poisson balayage on the boundary line by
\[
  u_{I;K}(t)\ :=\ \iint_{\mathbb H} P_\sigma(t-u)\,d\mu_{\mathrm{tail}}^{(K)}(u,\sigma),\qquad t\in\R.
\]
We say \textup{(B2$'$)} holds if there exist $K\ge 2$ and a uniform constant $C_{\rm tail}<\infty$ such that for every Whitney base interval $I$ and every subinterval $J\subset I$,
\[
  \frac1{|J|}\int_J\bigl|u_{I;K}(t)-(u_{I;K})_J\bigr|\,dt\ \le\ C_{\rm tail},
  \qquad (u_{I;K})_J:=\frac1{|J|}\int_J u_{I;K}.
\]
\end{definition}

\begin{lemma}[Route--A bottleneck: $\mu$--Carleson $\Rightarrow$ tail oscillation decays like $2^{-K}$]\label{lem:mu-carleson-to-B2prime}
Assume $\mu_{\mathrm{off}}$ is $\mu$--Carleson in the sense of Definition~\ref{def:mu-carleson}, with Carleson constant $C_\mu$.
Then for every integer $K\ge 2$ and for every Whitney base interval $I$ and every subinterval $J\subset I$,
\[
  \frac1{|J|}\int_J\bigl|u_{I;K}(t)-(u_{I;K})_J\bigr|\,dt\ \le\ \frac{8192}{3\pi}\,2^{-K}\,C_\mu,
\]
where $u_{I;K}$ is the tail Poisson balayage from Definition~\ref{def:B2prime}.
In particular, choosing $K$ sufficiently large yields \textup{(B2$'$)} with arbitrarily small $C_{\rm tail}$.

\smallskip
\noindent\emph{Proof strategy (what must be shown).} Decompose the tail region $\mathbb H\setminus Q(2^K I)$ into dyadic Whitney annuli at distance $\Delta\asymp 2^jL$ from $I$. A kernel-difference estimate bounds the contribution of each annulus to the mean oscillation on $I$ by $O(2^{-j})$ times its $\sigma$--mass. The $\mu$--Carleson packing controls the $\sigma$--mass in each annulus linearly in $2^j|I|$, so the annulus contributions form a geometric tail $\sum_{j\ge K}2^{-j}\asymp 2^{-K}$.
\end{lemma}

\begin{remark}[Near-field ``BMO'' is an averaged tail bound, not pointwise regularity]\label{rem:bmo-not-pointwise}
The estimate in \textup{(B2$'$)} is a \emph{mean oscillation} bound for the \emph{tail} Poisson balayage \(u_{I;K}\) of the off--critical zero measure \(\mu_{\mathrm{off}}\).
We do \emph{not} use (or claim) that membership of \(\log\xi\) in \(\mathrm{BMO}\) by itself forbids isolated zeros.
Instead, the near-field elimination is a \emph{signal$>$noise} comparison of two integrated quantities:
(i) a genuine off--critical zero forces a fixed, scale--robust Blaschke phase swing on a centered interval (the ``Blaschke trigger'' \(2\arctan(2)\)), while
(ii) under \(\mu\)--Carleson packing the contribution of zeros outside \(Q(2^K I)\) to the mean oscillation on \(I\) decays like \(2^{-K}\) (Lemma~\ref{lem:mu-carleson-to-B2prime}).
Since the phase derivative \(-w'\) is a nonnegative measure, there is no cancellation mechanism that could hide the Blaschke signal inside the tail.
\end{remark}

\begin{lemma}[Centered Blaschke trigger]\label{lem:blaschke-trigger}
Let $\rho=\beta+i\gamma$ with $\beta>\tfrac12$ and set $\delta:=\beta-\tfrac12$.
Define the half--plane Blaschke factor
\[
  b_\rho(s)\ :=\ \frac{s-\rho}{s-(1-\overline\rho)}.
\]
Then $|b_\rho(\tfrac12+it)|=1$ for all $t\in\R$, and with a continuous choice of argument satisfying
\(\Arg b_\rho(\tfrac12+i\gamma)=0\) one has
\[
  \Arg b_\rho\!\big(\tfrac12+i(\gamma\pm 2\delta)\big)\ =\ \mp\,2\arctan 2.
\]
In particular, on the centered interval $[\gamma-2\delta,\gamma+2\delta]$ the phase swing has size at least \(2\arctan 2\).
\end{lemma}
\begin{proof}
On the boundary line $s=\tfrac12+it$ we have
\[
  b_\rho\!\left(\tfrac12+it\right)
  =\frac{(\tfrac12-\beta)+i(t-\gamma)}{(\tfrac12-(1-\beta))+i(t-\gamma)}
  =\frac{-\delta+i(t-\gamma)}{\delta+i(t-\gamma)}.
\]
Writing $x:=t-\gamma$, this equals $\dfrac{ix-\delta}{ix+\delta}$, whose argument is
\(\Arg b_\rho(\tfrac12+i(\gamma+x))=-2\arctan(x/\delta)\) for the branch with value \(0\) at \(x=0\).
Evaluating at \(x=\pm 2\delta\) gives the stated values.
\end{proof}

\begin{theorem}[Path A (near strip): local short-interval zero-density $\Rightarrow$ near-strip zero-freeness]\label{thm:pathA-near-strip}
Fix $\sigma_0\in(\tfrac12,1)$ and an aperture $\alpha\in[1,2]$.
Assume the local short-interval density hypothesis \eqref{eq:local-density-template} from Proposition~\ref{prop:local-density-to-mu-carleson} holds for some constants $C_0,C_1,B,\kappa$ with $B\le 1$.
Then $\xi$ has no zeros in the near strip
\[
  Z(\xi)\ \cap\ \{\,s\in\C:\ \tfrac12<\Re s<\sigma_0\,\}\ =\ \varnothing.
\]
\end{theorem}
\begin{proof}
Since $T_I\ge 2$ implies $\log\langle T_I\rangle\ge \log\sqrt5$, and since $B\le 1$ makes $x\mapsto x^{B-1}$ decreasing on $(0,\infty)$, Proposition~\ref{prop:local-density-to-mu-carleson} yields a uniform Whitney-tree $\mu$--Carleson constant
\[
  C_\mu\ :=\ 2\alpha C_0\ +\ \frac{2C_1}{\kappa}\,(\log\sqrt5)^{B-1},
\]
so that $\mu_{\mathrm{off}}$ is $\mu$--Carleson in the sense of Definition~\ref{def:mu-carleson}.

Choose $K\ge 2$ so large that
\[
  \frac{8192}{3\pi}\,2^{-K}\,C_\mu\ <\ 2\arctan 2.
\]
Then Lemma~\ref{lem:mu-carleson-to-B2prime} bounds the tail oscillation on any centered interval $I$ by \(<2\arctan 2\).

Now suppose for contradiction that \(\xi(\rho)=0\) for some \(\rho=\beta+i\gamma\) with \(\tfrac12<\beta<\sigma_0\).
Set \(\delta:=\beta-\tfrac12\) and take the centered interval \(I:=[\gamma-2\delta,\gamma+2\delta]\).
The Blaschke factor for \(\rho\) produces a centered boundary phase swing of size \(2\arctan 2\) on \(I\) (Lemma~\ref{lem:blaschke-trigger}).
On the other hand, the contribution of all zeros outside \(Q(2^K I)\) to the renormalized tail oscillation on \(I\) is \(<2\arctan 2\) by the choice of \(K\) and Lemma~\ref{lem:mu-carleson-to-B2prime}.
Since the phase derivative is a nonnegative distribution (so there is no cancellation mechanism), this yields a signal$>$noise contradiction.
Therefore no such \(\rho\) exists.
\end{proof}

\section{Introduction}
\paragraph{Conceptual motivation.} The Euler product for $\zeta$ separates the $k=1$ prime layer from all higher prime powers. On the right half–plane $\{\Re s>\tfrac12\}$ the diagonal prime operator $A(s)e_p:=p^{-s}e_p$ has finite Hilbert–Schmidt norm ($\sum_p p^{-2\sigma}<\infty$), so the $k\ge2$ tail is naturally encoded by the 2–modified determinant $\det_2(I-A)$. After dividing by a finite outer (to neutralize archimedean and $k=1$ effects) one arrives at a ratio $\mathcal J$ that shares its zero/pole geometry with $\xi$ but whose boundary modulus is unimodular. This puts the problem squarely into the bounded–real/Schur/Herglotz framework: boundary positivity for $2\mathcal J$ transports to the interior by Poisson, and Cayley converts positivity into a Schur contractive bound for $\Theta=(2\mathcal J-1)/(2\mathcal J+1)$. The central analytic insight is that the \\"right–hand side\\" of the boundary certificate is \\emph{local and positive}: a Cauchy–Riemann/Green pairing against a Poisson test on a Whitney box controls the entire windowed phase variation by the Dirichlet energy of $U=\Re\log\mathcal J$. That energy is measured by a Carleson box constant coming from unconditional prime–tail and zero–density inputs. Thus the off–critical zero mass is ruled out by a linear–versus–uniform contradiction, and a short Schur pinch removes putative interior zeros. In short: the HS determinant regularizes the Euler tail, harmonic analysis supplies a local positive control of boundary phase, and passive systems (Herglotz/Schur) provide the globalization.
\noindent\textbf{Main result and proof outline (Schur pinch).}
The proof proceeds by a two-regime elimination of the critical strip $\{1/2<\Re s<1\}$:
\begin{itemize}
\item \textbf{Far strip ($\Re s\ge 0.6$).} A concrete spectral gap on $[\sigma_0,1]$ with $\sigma_0=0.6$ (Proposition~\ref{prop:delta-cert-06}) yields matrix positivity, which is upgraded to a Herglotz bound for the finite-stage approximants by passivity realization (Theorem~\ref{thm:passivity-realization}). Passing $N\to\infty$ on zero-free rectangles gives interior Schur/Herglotz control for $\Theta$ (Theorem~\ref{thm:limit-rect}), and a Schur pinch removes putative zeros with $\Re s\ge 0.6$.
\item \textbf{Near strip ($\tfrac12<\Re s<0.6$).} The B2$'$ route combines a fixed Blaschke trigger (Lemma~\ref{lem:blaschke-trigger}) with a renormalized tail bound obtained from Whitney-tree $\mu$--Carleson packing (Lemma~\ref{lem:mu-carleson-to-B2prime}). This eliminates the remaining near strip.
\end{itemize}
\noindent Under the far-field attachment bridge (Remark~\ref{rem:attachment-complete}), the combination yields RH (Theorem~\ref{thm:final-rh}). For comparison we also record an optional boundary-wedge formulation \textup{(P+)}, but it is not used in the main closure.

\paragraph{Optional boundary certificate material (\textup{(P+)}; not used in the main closure).}
\begin{itemize}
\item The phase--velocity identity and CR--Green/Carleson estimates yield Whitney-local phase-mass bounds and a boundary-wedge formulation \textup{(P+)} up to the local-to-global upgrade isolated in Remark~\ref{rem:wedge-application}.
\end{itemize}

\paragraph{Schur pinch template (used in the far strip).}
Section~\ref{sec:globalization} records the Schur pinch mechanism: a Schur bound for $\Theta$ on a zero-free domain, together with non-cancellation at $\xi$-zeros, rules out poles (hence zeros of $\xi$) in that domain.
The Riemann Hypothesis (RH) admits several analytic formulations. In this paper we pursue a bounded-real (BRF) route on the right half-plane
\[
 \Omega\;:=\;\Half,
\]
which is naturally expressed in terms of Herglotz/Schur functions and passive systems. Let \(\PP\) be the primes, and define the prime-diagonal operator
\[
 A(s):\ell^2(\PP)\to\ell^2(\PP),\qquad A(s)e_p\;:=\;p^{-s}e_p.
\]
For \(\sigma:=\Re s>\tfrac12\) we have \(\|A(s)\|_{\HS}^2=\sum_{p\in\PP}p^{-2\sigma}<\infty\) and \(\|A(s)\|\le 2^{-\sigma}<1\). With the completed zeta function
\[
 \xi(s)\;:=\;\tfrac12 s(1-s)\,\pi^{-s/2}\,\Gamma(s/2)\,\zeta(s)
\]
and the Hilbert--Schmidt regularized determinant \(\dettwo\), we study the analytic function
\[
 \mathcal J(s)\;:=\;\frac{\dettwo(I-A(s))}{\mathcal O_{\mathrm{ff}}(s)\,\zeta(s)}\cdot\frac{s-1}{s},
 \qquad
 \Theta(s)\;:=\;\frac{2\mathcal J(s)-1}{2\mathcal J(s)+1},
\]
where \(\mathcal O_{\mathrm{ff}}\) is the computable far-field normalizer (Definition~\ref{def:far-field-normalizer}).
The BRF assertion is that \(|\Theta(s)|\le 1\) on $\Omega\setminus Z(\xi)$ (Schur)—and on $\Omega$ after the pinch—equivalently that $2\mathcal J(s)$ is Herglotz on zero-free rectangles (hence on $\Omega\setminus Z(\xi)$) or that the associated Pick kernel is positive semidefinite there.

Our method combines four ingredients:
\begin{itemize}
 \item \textbf{Schur--determinant splitting.} For a block operator \(T(s)=\begin{bmatrix}A(s)&B(s)\\ C(s)&D(s)\end{bmatrix}\) with finite auxiliary part, one has
 \[
  \log\dettwo(I-T)\;=\;\log\dettwo(I-A)\; +\; \log\det(I-S),\qquad S\;:=\;D-C(I-A)^{-1}B,
 \]
 which separates the Hilbert--Schmidt (\(k\ge 2\)) terms from the finite block.
 \item \textbf{HS continuity for \(\dettwo\).} Prime truncations \(A_N\to A\) in the HS topology, uniformly on compacts in \(\Omega\), imply local-uniform convergence of \(\dettwo(I-A_N)\) (Section~\ref{prop:hs-det2-continuity}). Division by \(\zeta\) is justified only on compacts avoiding its zeros; throughout we explicitly state such hypotheses when needed (zeros coincide with \(Z(\xi)\) inside \(\Omega\)).
 % optional finite-stage and interior-rectangle route removed to enforce single proof route
\end{itemize}
% interior rectangles header removed (single-route only)
% removed interior route formal chain (single-route only). Throughout, \(\Omega=\{\Re s>\tfrac12\}\), and
\subsection*{Unsmoothing det$_2$: routed through smoothed testing (A1)}
\begin{lemma}[Smoothed distributional bound for $\partial_\sigma\,\Re\log\dettwo$]\label{lem:det2-unsmoothed}
Let $I\Subset\R$ be a compact interval and fix $\varepsilon_0\in(0,\tfrac12]$. There exists a finite constant
\[
  C_*\ :=\ \sum_{p}\sum_{k\ge 2}\frac{p^{-k/2}}{k^2\,\log p}\ <\ \infty
\]
such that for all $\sigma\in(\tfrac12,\tfrac12+\varepsilon_0]$ and every $\varphi\in C_c^2(I)$,
\[
  \Big|\int_{\R} \varphi(t)\,\partial_\sigma\Re\log\dettwo\big(I-A(\sigma+it)\big)\,dt\Big|\ \le\ C_*\,\|\varphi''\|_{L^1(I)}.
\]
In particular, testing against smooth, compactly supported windows yields bounds uniform in $\sigma$.
\end{lemma}
% archived block removed
\begin{proof}
For $\sigma>\tfrac12$ one has $\sum_p |p^{-(\sigma+it)}|^2=\sum_p p^{-2\sigma}<\infty$, so the diagonal product formula for $\dettwo$ gives
\[
  \log\dettwo(I-A(s))
  \;=\;\sum_p\big(\log(1-p^{-s})+p^{-s}\big)
  \;=\;-\sum_p\sum_{k\ge 2}\frac{p^{-ks}}{k},
\]
with absolute convergence (uniform on compact subsets of $\{\Re s>\tfrac12\}$). Differentiating termwise in $\sigma=\Re s$ yields the absolutely convergent expansion
\[
  \partial_\sigma\,\Re\log\dettwo\big(I-A(\sigma+it)\big)
  \;=\; \sum_{p}\sum_{k\ge 2} (\log p)\,p^{-k\sigma}\cos(k t\log p).
\]
For each frequency $\omega=k\log p\ge 2\log 2$, two integrations by parts give
\[
  \Big|\int_{\R}\!\varphi(t)\cos(\omega t)\,dt\Big|\ \le\ \frac{\|\varphi''\|_{L^1(I)}}{\omega^2}.
\]
Since $\sum_{p,k\ge 2} (\log p)\,p^{-k\sigma}/(k\log p)^2\le C_*$ uniformly in $\sigma\in(\tfrac12,\tfrac12+\varepsilon_0]$, Tonelli/Fubini allows summing after testing against $\varphi$. Summing the resulting majorant yields
\[
  \Big|\int \varphi\,\partial_\sigma\Re\log\dettwo\,dt\Big|
  \ \le\ \|\varphi''\|_{L^1}\sum_{p}\sum_{k\ge 2}\frac{(\log p)\,p^{-k\sigma}}{(k\log p)^2}
  \ \le\ \|\varphi''\|_{L^1}\sum_{p}\sum_{k\ge 2}\frac{p^{-k/2}}{k^2\,\log p},
\]
uniformly for $\sigma\in(\tfrac12,\tfrac12+\varepsilon_0]$, since the rightmost double series converges. This proves the claim.
\end{proof}
% (Removed: the earlier local-to-global wedge lemma was not used in the active chain and its proof as written was not correct.)
\noindent\emph{Note.} The single-interval density route is archived; the small-$L$ scaling $c_0 L \le C\,L^{1/2}$ does not contradict the RHS bound.

% (Legacy KYP/Carleson stubs removed; unused in the active route.)

% Minimal in-file bibliography moved to end (References)

\begin{lemma}[De-smoothing / boundary passage to an $L^1_{\mathrm{loc}}$ trace]\label{lem:desmooth-L1}
Let $U$ be a harmonic function on the half-plane $\Omega=\{(\sigma,t):\sigma>0\}$ such that its gradient energy defines a Carleson measure on Whitney boxes:
for every interval $I\subset\R$,
\[
  \iint_{Q(I)} |\nabla U(\sigma,t)|^2\,\sigma\,dt\,d\sigma\ \le\ C_{\rm box}\,|I|.
\]
Then $U$ has a boundary trace $u\in \mathrm{BMO}(\R)\subset L^1_{\mathrm{loc}}(\R)$ and
\[
  U(\sigma,\cdot)\ =\ P_\sigma * u\qquad(\sigma>0),
\]
so in particular $U(\varepsilon,\cdot)\to u$ in $L^1_{\mathrm{loc}}(\R)$ as $\varepsilon\downarrow 0$.
\end{lemma}
\begin{proof}
This is the classical Fefferman--Stein/Carleson characterization of boundary $\mathrm{BMO}$ via square functions (or equivalently via the Carleson measure control of $|\nabla U|^2\,\sigma\,dt\,d\sigma$); see, e.g., Garnett \cite[Ch.~IV]{Garnett} or Stein \cite[Ch.~II]{SteinHA}. Once $U=P_\sigma*u$ with $u\in L^1_{\mathrm{loc}}$, the convergence $P_\varepsilon*u\to u$ in $L^1_{\mathrm{loc}}$ is the standard approximate identity property of the Poisson kernel.
\end{proof}

\begin{lemma}[Neutralization bookkeeping for CR–Green on a Whitney box]\label{lem:neutralization-bookkeeping}
Let $I=[t_0{-}L,t_0{+}L]$ and $Q(\alpha'I)$ be as above. Let $B_I$ be the product of half–plane Blaschke factors for the zeros/poles of $J$ in $Q(\alpha'I)$ and set $\widetilde U:=\Re\log(J/B_I)$ on $Q(\alpha'I)$. Then with the same cutoff $\chi_{L,t_0}$ and Poisson test $V_{\psi,L,t_0}$,
\[
 \iint_{Q(\alpha'I)} \nabla \widetilde U\cdot\nabla(\chi V)\,dt\,d\sigma
 = \int_{\R} \psi_{L,t_0}(t)\,-w'(t)\,dt\ +\ \mathcal E_{\mathrm{side}}\ +\ \mathcal E_{\mathrm{top}},
\]
where the error terms obey the uniform bound
\[
 |\mathcal E_{\mathrm{side}}|+|\mathcal E_{\mathrm{top}}|
 \ \le\ C_{\mathrm{neu}}(\alpha,\psi)\,\Big(\iint_{Q(\alpha'I)} |\nabla U|^2\,\sigma\Big)^{1/2}.
\]
In particular,
\[
  \int_{\R} \psi_{L,t_0}(-w')\ \le\ \big(C(\psi)+C_{\mathrm{neu}}(\alpha,\psi)\big)\,\Big(\iint_{Q(\alpha'I)} |\nabla U|^2\,\sigma\Big)^{1/2},
\]
with constants independent of $t_0$ and $L$.
\end{lemma}
\begin{proof}
Apply Lemma~\ref{lem:CR-green-phase} to $\widetilde U$ on $Q(\alpha'I)$ and expand $\nabla\widetilde U=\nabla U-\nabla\Re\log B_I$. The latter is harmonic away from zeros and has explicit Poisson kernels on $\partial Q$; the bottom edge contribution cancels exactly against the Blaschke phase increments already accounted in $-w'$ (by construction of $B_I$), leaving only side/top terms. Cauchy–Schwarz together with the scale–invariant Dirichlet bounds for $V$ on the sides/top and a uniform bound on the Blaschke gradients in $Q(\alpha'I)$ (controlled by aperture $\alpha$) yield the stated estimate; the Whitney scaling gives independence of $L$.
\end{proof}
\noindent\emph{Clarification.} The certificate yields the Whitney–uniform phase-mass bound
\(\int_I (-w')\le \pi\,\Upsilon_{\rm Whit}(c)\) with $\Upsilon_{\rm Whit}(c)<\tfrac12$ (Lemma~\ref{lem:whitney-uniform-wedge}), obtained solely from the local CR--Green pairing controlled by $C_{\rm box}^{(\zeta)}$; the remaining promotion to a global a.e.\ wedge after a single rotation is isolated in Remark~\ref{rem:wedge-application}.
\smallskip
\noindent\emph{Non-circularity note.} The ``neutralization'' by \(B_I\) does \emph{not} assume that \(J\) (or \(\xi\)) is zero--free in \(Q(\alpha'I)\); it explicitly factors out the zeros/poles in that box so that \(\widetilde U=\Re\log(J/B_I)\) is harmonic there and the CR--Green pairing is legitimate.
No information about zeros is discarded: the removed factors contribute \emph{positively} to the phase derivative term \(-w'\) (via their explicit Blaschke phase increments), which is exactly why the near-field route can compare this quantized ``signal'' to the tail ``noise''.

\paragraph{Boundary wedge \textup{(P+)} (optional boundary formulation).}
We record the a.e.\ boundary inequality
\begin{equation}\label{eq:Pplus}
\Re\bigl(2\mathcal J(\tfrac12+it)\bigr)\ \ge\ 0\qquad\text{for a.e.\ }t\in\mathbb R.
\tag{P+}
\end{equation}
This is the classical boundary positivity input for BRF/Herglotz routes. The active proof route in this manuscript does \emph{not} rely on \eqref{eq:Pplus}; it is kept for comparison with boundary-wedge formulations.

\begin{lemma}[Poisson lower bound $\Rightarrow$ Lebesgue a.e. wedge]\label{lem:mu-to-lebesgue}
Assume the hypotheses of Theorem~\ref{thm:phase-velocity-quant}. Fix $m\in\mathbb R/2\pi\mathbb Z$ and define
\[
 \mathcal Q := \bigl\{t\in\mathbb R:\ |\Arg \mathcal J(1/2+it)-m|\ge \tfrac{\pi}{2}\bigr\}.
\]
If $\mu(\mathcal Q)=0$, then $|\mathcal Q|=0$. In particular, \eqref{eq:Pplus} holds.
\end{lemma}
\begin{proof}
Fix $I\Subset\R$ and choose $\phi\in C_c^\infty(I)$ with $0\le\phi\le\mathbf 1_{\mathcal Q}$. By Theorem~\ref{thm:phase-velocity-quant},
\[
  \int \phi(t)\,-w'(t)\,dt \;=\; \pi\!\int\phi\,d\mu \;+\; \pi\!\sum_{\gamma\in I} m_\gamma\,\phi(\gamma).
\]
Since $\mu(\mathcal Q)=0$ and $\phi\le\mathbf 1_{\mathcal Q}$, the first term vanishes; choosing $\phi$ to vanish in small neighborhoods of each $\gamma\in I$ kills the atomic sum as well, so $\int_{\mathcal Q} (-w')=0$ on $I$. As $-w'$ is a positive boundary distribution, this forces $-w'=0$ a.e. on $\mathcal Q\cap I$. By nontangential boundary uniqueness for harmonic conjugates of $H^p_{\rm loc}$ functions\footnote{See Garnett, \emph{Bounded Analytic Functions}, Thm.~II.4.2, and Rosenblum--Rovnyak, \emph{Hardy Classes and Operator Theory}, Ch.~2.} and the definition of $\mathcal Q$, we must have $|\mathcal Q\cap I|=0$. Letting $I\uparrow\R$ yields $|\mathcal Q|=0$.
\end{proof}
% --- Appendix S moved near the end; see after main results/before bibliography ---

% Lead-in: We quantify the phase–velocity identity and justify boundary passage via outers and smoothed L^1 control.
\begin{lemma}[Outer–Hilbert boundary identity]\label{lem:outer-phase-HT}
Let $u\in L^1_{\mathrm{loc}}(\mathbb R)$ and let $O$ be the outer function on $\Omega$ with boundary modulus $|O(\tfrac12+it)|=e^{u(t)}$ a.e. Then, in $\mathcal D'(\mathbb R)$,
\[
\frac{d}{dt}\Arg O\!\left(\tfrac12+it\right)=\Hilb[u'](t),
\]
where $\Hilb$ is the boundary Hilbert transform on $\mathbb R$ and $u'$ is the distributional derivative.
\end{lemma}
\begin{proof}
See, e.g., \cite[Ch.~2]{DurenHp} or \cite[Ch.~2]{RosenblumRovnyak} for the half-plane outer/Hardy boundary correspondence and distributional Hilbert-transform conventions.
Write $\log O=U+iV$ on $\Omega$, where $U$ is the Poisson extension of $u$ and $V$ is its harmonic conjugate with $V(\tfrac12+\cdot)=\Hilb[u]$ in $\mathcal D'(\mathbb R)$. Then $\tfrac{d}{dt}\Arg O=\partial_t V=\Hilb[\partial_t U]=\Hilb[u']$ in distributions.
\end{proof}
\begin{theorem}[Quantified phase–velocity identity and boundary passage]\label{thm:phase-velocity-quant}
Let $u_\varepsilon(t):=\log\big|\dettwo(I-A(\tfrac12+\varepsilon+it))\big| - \log\big|\xi(\tfrac12+\varepsilon+it)\big|$ and let $\mathcal O_\varepsilon$ be the outer on $\{\Re s>\tfrac12+\varepsilon\}$ with boundary modulus $e^{u_\varepsilon}$. There exists $C_I<\infty$, independent of $\varepsilon\in(0,\varepsilon_0]$, such that for every compact interval $I\Subset\R$ and every $\phi\in C_c^2(I)$ with $\phi\ge 0$,
\[
 \Big|\int_I \phi(t)\,\partial_\sigma\Re\log\dettwo\big(I-A(\tfrac12+\varepsilon+it)\big)\,dt\Big|\ \le\ C_I\,\|\phi''\|_{L^1(I)},
\]
and
\[
 \int_I \phi(t)\,\partial_\sigma\Re\log\xi\big(\tfrac12+\varepsilon+it\big)\,dt\ \le\ C'_I\,\|\phi\|_{H^1(I)}
\]
with $C'_I$ depending only on $I$. Consequently $\{u_\varepsilon\}_{\varepsilon\downarrow 0}$ is Cauchy in $\mathcal D'(I)$ (hence converges in distributions) and, passing $\varepsilon\downarrow 0$ in the smoothed identity (Lemma~\ref{lem:pv-test-smoothed}), the phase–velocity identity holds in the distributional sense on $I$:
\[
 \int_I \phi(t)\,-w'(t)\,dt\ =\ \int_I \phi(t)\,\pi\,d\mu(t)\ +\ \pi\sum_{\gamma\in I} m_\gamma\,\phi(\gamma),\qquad \forall\,\phi\in C_c^\infty(I),\ \phi\ge 0,
\]
where $\mu$ is the \emph{boundary balayage measure} on $\mathbb R$ induced by off–critical zeros (i.e. the absolutely continuous measure whose density is a sum of Poisson kernels), and the discrete sum ranges over critical–line ordinates.
\end{theorem}
\begin{proof}
Fix a compact interval $I\Subset\R$ and $\varepsilon_0\in(0,\tfrac12]$. Define
\[
 u_\varepsilon(t):=\log\Big|\dettwo\big(I-A(\tfrac12+\varepsilon+it)\big)\Big|-\log\Big|\xi(\tfrac12+\varepsilon+it)\Big|.
\]
By Lemma~\ref{lem:det2-unsmoothed}, for every $\phi\in C_c^2(I)$,
\[
 \Big|\!\int_I \!\phi(t)\,\partial_\sigma\Re\log\dettwo(I\! -\!A(\tfrac12\!+\!\sigma\!+\!it))\,dt\Big|\ \le\ C_I\,\|\phi''\|_{L^1(I)}
\]
uniformly in $\sigma\in(0,\varepsilon_0]$. For $\xi$, Lemma~\ref{lem:xi-deriv-L1} gives the tested bound
\[
 \Big|\int_I \phi(t)\,\partial_\sigma\Re\log\xi(\tfrac12+\sigma+it)\,dt\Big|\ \le\ C'_I\,\|\phi\|_{H^1(I)}\qquad(0<\sigma\le \varepsilon_0).
\]
Fix $0<\delta<\varepsilon\le \varepsilon_0$. Integrating in $\sigma$ and using the tested bounds yields a distributional Cauchy estimate: for every $\phi\in C_c^2(I)$,
\[
  \Big|\int_I \phi(t)\,\big(u_\varepsilon(t)-u_\delta(t)\big)\,dt\Big|
  \ \le\ |\varepsilon-\delta|\Big(C_I\,\|\phi''\|_{L^1(I)}+C'_I\,\|\phi\|_{H^1(I)}\Big).
\]
Hence $\{u_\varepsilon\}_{\varepsilon\downarrow 0}$ is Cauchy in $\mathcal D'(I)$ and converges to a distribution $u\in\mathcal D'(I)$. By continuity of the Hilbert transform on distributions (see, e.g., \cite[Ch.~II]{SteinHA}), $\Hilb[u_\varepsilon']\to \Hilb[u']$ in $\mathcal D'(I)$.

Now apply Lemma~\ref{lem:pv-test-smoothed} and let $\varepsilon\downarrow 0$.
The Poisson kernels $P_{\beta-\tfrac12-\varepsilon}$ converge in $\mathcal D'(\R)$ to $P_{\beta-\tfrac12}$, and boundary atoms from critical-line zeros appear in the limit through the argument principle on the boundary. Passing to the limit in \eqref{eq:pv-smoothed} yields the stated distributional identity for $-w'$ on $I$.
\end{proof}

\begin{lemma}[Balayage density and consequence for $Q$]\label{lem:balayage-density}
If there exists at least one off--critical zero $\rho=\beta+i\gamma$ of $\xi$ with $\beta>\tfrac12$, then the boundary balayage measure $\mu$ from Theorem~\ref{thm:phase-velocity-quant} has an a.e. density $f\in L^1_{\mathrm{loc}}(\mathbb R)$ of the form
\[
  f(t)\ =\ \sum_{\substack{\rho=\beta+i\gamma\\ \beta>1/2}} 2\,(\beta-\tfrac12)\,P_{\beta-1/2}(t-\gamma),\qquad P_a(x)=\frac{1}{\pi}\frac{a}{a^2+x^2},
\]
which is strictly positive a.e. on $\R$ whenever at least one off--critical zero exists. Consequently, for any measurable set $E\subset\R$, $\mu(E)=0$ implies $|E|=0$. In particular, $\mu(Q)=0$ forces $|Q|=0$, hence \textup{(P+)}.
\end{lemma}
\begin{proof}
For each finite subset of zeros $\mathcal Z\subset\{\rho:\Re\rho>1/2\}$ the partial density
\(f_{\mathcal Z}(t):=\sum_{\rho\in\mathcal Z}2(\beta-\tfrac12)P_{\beta-1/2}(t-\gamma)\)
is continuous and strictly positive for all $t$ because each Poisson kernel is strictly positive on $\R$.
The phase--velocity formula and the Carleson energy finiteness imply the balayage of zeros on any Whitney box is finite, so the monotone limit of the partial sums converges in $L^1_{\mathrm{loc}}$ to an a.e. finite function $f\ge0$. Since the pointwise limit of strictly positive functions is nonnegative and cannot vanish on a set of positive measure unless all coefficients vanish, we obtain $f>0$ a.e. whenever at least one off--critical zero exists. Moreover, by positivity and monotone convergence, $\mu(E)=\int_E f\,dt=0$ forces $f=0$ a.e. on $E$, whence $|E|=0$.
\end{proof}

\paragraph{Certificate $\Rightarrow$ (P+): narrative.}
The outer, boundary phase–velocity identity shows that $\int\varphi_{L,t_0}(-w')$ is the mass picked up by $\varphi_{L,t_0}$ against a positive measure supported on off–critical zeros (with atoms on the line if they occur). The left plateau inequality therefore lower-bounds it by $c_0(\psi)\,\nu(Q(I))$, where $\nu$ is the defect measure on $\Omega$ (see Notation and conventions) and $Q(I)$ is the Carleson box. The CR–Green pairing controls the same integral from above by box energy, and the Carleson bound is uniform on Whitney boxes. This yields a Whitney–uniform \emph{local} phase-drop bound $\int_I(-w')\le \pi\,\Upsilon_{\rm Whit}(c)$ with $\Upsilon_{\rm Whit}(c)<\tfrac12$ for suitably small $c$ (Lemma~\ref{lem:whitney-uniform-wedge}). The remaining upgrade from Whitney-local control to a global a.e. boundary wedge \textup{(P+)} after a single rotation is a separate local-to-global step; see Remark~\ref{rem:wedge-application}.

\begin{lemma}[Quantitative wedge criterion]\label{lem:local-to-global-wedge}
Let $w\in L^\infty_{\mathrm{loc}}(\R)$ be a boundary phase function. For a measurable interval $I\subset\R$, write
\[
  \osc\sb{I} w\ :=\ \esssup\sb{I} w\ -\ \essinf\sb{I} w
\]
for the essential oscillation (with respect to Lebesgue measure).
\begin{enumerate}
    \item \textbf{Local-to-global from a centered exhaustion.} If there is a $D\ge 0$ such that
    \[
      \osc\sb{[-N,N]} w\ \le\ D\qquad\text{for every }N\ge 1,
    \]
    then there exists a constant $c\in\R$ such that $|w(t)-c|\le D$ for a.e. $t\in\R$.
    \item \textbf{Windowed phase-mass $\Rightarrow$ oscillation on an interval.} Assume $-w'$ is a positive Radon measure on $\R$ (in the sense of distributions). If $I=[a,b]$ and $\psi\ge \mathbf 1_I$ is a nonnegative test function, then
    \[
      \int_I (-w')\ \le\ \int_{\R}\psi\,(-w'),
    \]
    and the phase drop (hence essential oscillation) on $I$ is bounded by the left-hand side. In particular, if for some $\Upsilon\ge 0$ one has $\int_{\R}\psi\,(-w')\le \pi\,\Upsilon$, then $\osc\sb{I} w\le \pi\,\Upsilon$.
\end{enumerate}
\end{lemma}
\begin{proof}
(1) For $N\ge 1$ set $a_N:=\essinf\sb{[-N,N]} w$ and $b_N:=\esssup\sb{[-N,N]} w$. Then $a_N$ is nonincreasing, $b_N$ is nondecreasing, and $b_N-a_N\le D$ by hypothesis. Let
\[
  a_\infty:=\lim_{N\to\infty} a_N\in[-\infty,\infty),\qquad b_\infty:=\lim_{N\to\infty} b_N\in(-\infty,\infty].
\]
Then $b_\infty-a_\infty\le D$ and for each $N$ we have $a_\infty\le a_N\le w(t)\le b_N\le b_\infty$ for a.e.\ $t\in[-N,N]$, hence for a.e.\ $t\in\R$. Choosing $c:=(a_\infty+b_\infty)/2$ gives $|w(t)-c|\le D$ a.e.

(2) The first inequality is immediate from $\psi\ge\mathbf 1_I$ and positivity of the measure $-w'$. Since $-w'$ is the (distributional) derivative of a locally BV representative of $w$, its mass on $I$ bounds the phase drop across $I$, which in turn bounds the essential oscillation on $I$. (See, e.g., \cite[Ch.~3]{AmbrosioFuscoPallara} for BV representatives and the identification of distributional derivatives with measures.)
\end{proof}

\begin{lemma}[Whitney--uniform wedge]\label{lem:whitney-uniform-wedge}
Fix the Whitney schedule and clip by $L_\star$: set $L_\star:=c/\log 2$ and henceforth
\[
  L(T)\ :=\ \min\Big\{\frac{c}{\log\langle T\rangle},\ L_\star\Big\}.
\]
Then for every Whitney interval $I=[t_0-L,t_0+L]$ (so $L\le L_\star$), with the printed flat--top window $\psi_{L,t_0}(t)=\psi\big((t-t_0)/L\big)$ one has
\[
  \int_I (-w')\,dt\ \le\ \int_{\mathbb R} \psi_{L,t_0}(t)\,(-w'(t))\,dt\ \le\ C(\psi)\,\sqrt{C_{\rm box}^{(\zeta)}}\,L_\star^{1/2}
  \ :=\ \pi\,\Upsilon_{\rm Whit}(c),
\]
where $C(\psi)$ is the CR--Green window constant and $\Upsilon_{\rm Whit}(c)$ depends only on $c,\psi$ and the fixed aperture. Choosing $c>0$ sufficiently small so that $\Upsilon_{\rm Whit}(c)<\tfrac12$ yields the Whitney-local phase-drop bound $\int_I(-w')\le \pi\,\Upsilon_{\rm Whit}(c)$ on every Whitney interval. Promoting this Whitney-local bound to a \emph{global} a.e. boundary wedge \textup{(P+)} requires an additional local-to-global step; see Remark~\ref{rem:wedge-application}.
\end{lemma}
\begin{proof}
Since $-w'$ is a positive boundary distribution and $\psi_{L,t_0}\ge \mathbf 1_I$ (because $\psi\equiv 1$ on $[-1,1]$), we have
\[
  \int_I (-w')\ \le\ \int_{\R}\psi_{L,t_0}\,(-w').
\]
By Lemma~\ref{lem:CR-green-phase},
\[
  \int_{\R}\psi_{L,t_0}\,(-w')\ \le\ C(\psi)\,\Big(\iint_{Q(\alpha'I)}|\nabla U|^2\,\sigma\Big)^{1/2}.
\]
Using the box constant $C_{\rm box}^{(\zeta)}=\sup_I |I|^{-1}\iint_{Q(\alpha'I)}|\nabla U|^2\,\sigma$ and $|I|=2L\le 2L_\star$, we obtain
\[
  \Big(\iint_{Q(\alpha'I)}|\nabla U|^2\,\sigma\Big)^{1/2}\ \le\ \sqrt{C_{\rm box}^{(\zeta)}\,|I|}\ \le\ \sqrt{2}\,\sqrt{C_{\rm box}^{(\zeta)}}\,L_\star^{1/2},
\]
and we absorb the harmless factor $\sqrt2$ into the definition of $\Upsilon_{\rm Whit}(c)$.
\end{proof}

\noindent\emph{Clarification.} The certificate yields the Whitney–uniform phase-mass bound
\(\int_I (-w')\le \pi\,\Upsilon_{\rm Whit}(c)\) with $\Upsilon_{\rm Whit}(c)<\tfrac12$ (Lemma~\ref{lem:whitney-uniform-wedge}), obtained solely from the local CR--Green pairing controlled by $C_{\rm box}^{(\zeta)}$; the remaining promotion to a global a.e.\ wedge after a single rotation is isolated in Remark~\ref{rem:wedge-application}.
\paragraph{Window constant.}
Set once and for all the window constant
\[
  C(\psi)\ :=\ C_{\mathrm{rem}}(\alpha,\psi)\,\mathcal A(\psi),
\]
where $\mathcal A(\psi)$ is the fixed Poisson energy of the window and $C_{\mathrm{rem}}(\alpha,\psi)$ is the side/top remainder factor from Corollary~\ref{cor:CH-Mpsi-final}. Then $C(\psi)$ is independent of $L$ and $t_0$ and will be used uniformly below.
\begin{proposition}[HS$\to$det$_2$ continuity]\label{prop:hs-det2-continuity}
Let $A_N,A$ be analytic $\HS$-valued maps on $\Omega$ with $A_N\to A$ in the Hilbert–Schmidt norm uniformly on compact subsets of $\Omega$. Then $\det\nolimits_2(I-A_N)\to\det\nolimits_2(I-A)$ locally uniformly on $\Omega$.
\end{proposition}
\begin{proof}
Fix a compact $K\Subset\Omega$. By hypothesis, $\sup_{s\in K}\|A_N(s)-A(s)\|_{\HS}\to0$, and in particular $\sup_{N}\sup_{s\in K}\|A_N(s)\|_{\HS}<\infty$.
We use the standard definition of the $2$--modified determinant on $\HS$:
\[
  \det\nolimits_2(I-T)\ :=\ \det\!\big((I-T)e^T\big),
\]
where the Fredholm determinant on the right is defined for trace-class perturbations of the identity. Indeed, for $T\in\HS$ one has
\[
  (I-T)e^T-I\ =\ -\sum_{n\ge 2}\frac{n-1}{n!}\,T^n,
\]
and the series converges absolutely in trace norm because $T^n$ is trace class for $n\ge 2$ and
$\|T^n\|_1\le \|T\|^{n-2}\|T^2\|_1\le \|T\|_{\HS}^n$.
In particular, on any $\HS$-ball $\{ \|T\|_{\HS}\le M\}$, the map
\[
  T\ \longmapsto\ (I-T)e^T-I
\]
is Lipschitz from $\HS$ to trace class: writing the series termwise and using
$T^n-S^n=\sum_{k=0}^{n-1}T^k(T-S)S^{n-1-k}$ together with $\|XY\|_1\le \|X\|_2\|Y\|_2$ and $\|T\|\le \|T\|_{\HS}$ gives
\[
  \|(I-T)e^T-(I-S)e^S\|_1\ \le\ C(M)\,\|T-S\|_{\HS}.
\]
Since the Fredholm determinant on trace-class perturbations of the identity is defined by an absolutely convergent trace-norm series (hence is continuous in $\|\cdot\|_1$), it follows that $\det_2(I-T)$ is continuous (indeed locally Lipschitz) with respect to $\|\cdot\|_{\HS}$.
Thus
\[
  \sup_{s\in K}\Big|\det\nolimits_2\big(I-A_N(s)\big)-\det\nolimits_2\big(I-A(s)\big)\Big|\ \longrightarrow\ 0,
\]
which is local-uniform convergence on $K$. Since $K$ was arbitrary, the convergence is locally uniform on $\Omega$.
\end{proof}

\begin{lemma}[Smoothed phase–velocity calculus]\label{lem:pv-test-smoothed}
Fix $\varepsilon\in(0,\tfrac12]$ and set
\[
 u_\varepsilon(t):=\log\Big|\dettwo(I{-}A(\tfrac12{+}\varepsilon{+}it))\Big|-\log\Big|\xi(\tfrac12{+}\varepsilon{+}it)\Big|.
\]
Let $\mathcal O_\varepsilon$ be the outer on $\{\Re s>\tfrac12{+}\varepsilon\}$ with boundary modulus $e^{u_\varepsilon}$ and write $F_\varepsilon:=\dettwo/\xi$ and $O_\varepsilon:=\mathcal O_\varepsilon$. Then for every $\phi\in C_c^\infty(\R)$,
\begin{equation}\label{eq:pv-smoothed}
\int_\R\!\phi(t)\,\Big( \Im\frac{\xi'}{\xi}-\Im\frac{\dettwo'}{\dettwo}+\Hilb[u_\varepsilon']\Big)\!(\tfrac12{+}\varepsilon{+}it)\,dt
\;=\;\sum_{\substack{\rho=\beta+i\gamma\\ \Re\rho>\tfrac12}}\! 2(\beta{-}\tfrac12)\,\big(P_{\beta-\tfrac12-\varepsilon}\!\ast\phi\big)(\gamma)
\end{equation}
where $P_a(x)=\frac{1}{\pi}\frac{a}{a^2+x^2}$ and the right-hand side is a nonnegative quantity.
\end{lemma}
\begin{proof}
Factor $F_\varepsilon=I_\varepsilon\,O_\varepsilon$ with $O_\varepsilon$ outer on $\{\Re s>\tfrac12{+}\varepsilon\}$ and $I_\varepsilon$ inner (product of half-plane Blaschke factors for poles/zeros of $F_\varepsilon$ in the open half-plane). By Lemma~\ref{lem:outer-phase-HT}, on the boundary line $\Re s=\tfrac12{+}\varepsilon$ one has $\frac{d}{dt}\Arg O_\varepsilon=\Hilb[u_\varepsilon']$ in $\mathcal D'(\R)$. Each pole of $F_\varepsilon$ at $\rho=\beta+i\gamma$ with $\beta>\tfrac12$ contributes the half-plane Blaschke factor $C_\rho(s)=(s-\overline\rho)/(s-\rho)$ whose boundary phase derivative equals $-2(\beta-\tfrac12-\varepsilon)\,P_{\beta-\tfrac12-\varepsilon}(t-\gamma)$. Summing these contributions and writing $\frac{d}{dt}\Arg F_\varepsilon=\Im(F_\varepsilon'/F_\varepsilon)=\Im(\dettwo'/\dettwo)-\Im(\xi'/\xi)$ yields \eqref{eq:pv-smoothed} after testing against $\phi$.
\end{proof}
% Lead-in: Use the interior Schur/Herglotz control on zero-free rectangles (Theorem~\ref{thm:limit-rect})
% to obtain Schur on $\Omega\setminus Z(\xi)$ (Corollary~\ref{cor:Schur-off-zeros}), then pinch across $Z(\xi)$ using (N1)–(N2).
\section{Globalization across $Z(\xi)$ via a Schur--Herglotz pinch}\label{sec:globalization}
\noindent This section records the Schur pinch \emph{template}: given a domain $D\subset\Omega$ on which $\Theta$ is Schur on $D\setminus Z(\xi)$, together with non-cancellation \textnormal{(N2)} and the right-edge normalization \textnormal{(N1)}, one rules out zeros of $\xi$ in $D$.
In the far-field route, we apply this with $D=\{\,\Re s>\sigma_0\,\}$ once the Schur bound is obtained there (Corollary~\ref{cor:Schur-off-zeros} under the attachment identity \eqref{eq:attachment}).
\paragraph{Globalization and pinch: narrative.}
In particular, once Corollary~\ref{cor:Schur-off-zeros} provides $\Theta$ Schur on $D\setminus Z(\xi)$, any putative zero $\rho\in D$ forces $\Theta(\rho)=1$ by removability, hence $\Theta$ is constant unimodular on $D\setminus Z(\xi)$ by the Maximum Modulus Principle; the normalization (N1) forces $\Theta(\sigma+it)\to\tfrac13$ as $\sigma\to+\infty$, contradicting a unimodular constant.
\noindent\textbf{Standing setup.}
Let
\[
\Omega:=\{s\in\mathbb C:\ \Re s>\tfrac12\},\qquad
\xi(s)=\tfrac12\,s(s-1)\,\pi^{-s/2}\Gamma\!\big(\tfrac s2\big)\zeta(s).
\]
\noindent\emph{Clarification.} Although the factor $(s-1)$ vanishes at $s=1$, the zeta factor has a simple pole there and the product $(s-1)\zeta(s)\to 1$. Hence $\xi$ is entire and $\xi(1)=\tfrac12\,\pi^{-1/2}\Gamma(1/2)\cdot 1=\tfrac12\neq 0$.
Define
\[
\mathcal J(s):=\frac{\det\nolimits_2(I-A(s))}{\mathcal O_{\mathrm{ff}}(s)\,\zeta(s)}\cdot\frac{s-1}{s},\qquad
F(s):=2\,\mathcal J(s),\qquad
\Theta(s):=\frac{F(s)-1}{F(s)+1}.
\]
Here \(\mathcal O_{\mathrm{ff}}\) is holomorphic and zero--free on \(\{\Re s>\sigma_{\mathrm{ref}}\}\) (Definition~\ref{def:far-field-normalizer}) and
\(\det\nolimits_2(I-A)\) is holomorphic on \(\Omega\).
We record the two normalization properties actually used below:
\begin{itemize}
\item[(N1)] (\emph{Right--edge normalization}) For each fixed $t$ (indeed uniformly on compact $t$–intervals), $\displaystyle\lim_{\sigma\to+\infty}\mathcal J(\sigma+it)=1$; hence $\displaystyle\lim_{\sigma\to+\infty}\Theta(\sigma+it)=\tfrac13$.
\item[(N2)] (\emph{Non--cancellation at $\xi$--zeros}) For every $\rho\in\Omega$ with $\xi(\rho)=0$,
\[
\det\nolimits_2(I-A(\rho))\neq0\quad\text{and}\quad \mathcal O_{\mathrm{ff}}(\rho)\neq0.
\]
Thus $\mathcal J$ has a pole at $\rho$ of order $\operatorname{ord}_\rho(\xi)$.
\end{itemize}

\medskip
\noindent\textbf{Schur bound on $\Omega\setminus Z(\xi)$.}
By Corollary~\ref{cor:Schur-off-zeros}, the Cayley transform is Schur on $\Omega\setminus Z(\xi)$:
\[
|\Theta(s)|\ \le\ 1\qquad(s\in \Omega\setminus Z(\xi)).
\tag{Schur}
\label{eq:SchurBound}
\]
\medskip
\noindent\textbf{Local pinch at a putative off--critical zero.}
\emph{We use (N2) for non--cancellation at $\xi$--zeros and (N1) for the right--edge limit $\Theta\to\tfrac13$.}
Fix $\rho\in\Omega$ with $\xi(\rho)=0$.
By (N2) the function $F$ has a pole at $\rho$, hence
\[
\Theta(s)=\frac{F(s)-1}{F(s)+1}\ \longrightarrow\ 1\qquad(s\to\rho).
\]
By \eqref{eq:SchurBound}, $\Theta$ is bounded by $1$ on $(\Omega\setminus Z(\xi))$,
so the singularity of $\Theta$ at $\rho$ is removable (Riemann's theorem), and the holomorphic extension satisfies
\[
\Theta(\rho)=1.
\]
Because $\Theta$ is holomorphic on the connected domain $\Omega\setminus(Z(\xi)\setminus\{\rho\})$
and $|\Theta|\le1$ there, the Maximum Modulus Principle forces $\Theta$ to be
a \emph{constant unimodular} function on that domain (it attains its supremum $1$ at an interior point).
By analyticity, the same constant extends throughout $\Omega\setminus Z(\xi)$.
\medskip
\begin{lemma}[Connectedness and isolation]\label{rem:connectedness}
Since $Z(\xi)\cap\Omega$ is a discrete subset (zeros are isolated), one can choose a disc $D\subset\Omega$ centered at $\rho$ containing no other zeros, and $\Omega\setminus Z(\xi)$ is (path-)connected. Hence in the argument above, $\Omega\setminus\big(Z(\xi)\setminus\{\rho\}\big)$ is connected and the Maximum Modulus Principle applies on this domain.
\end{lemma}
\begin{proof}
Since $\xi$ is holomorphic and not identically zero on $\Omega$, its zeros are isolated; thus $Z(\xi)\cap\Omega$ is discrete and we may choose a disc $D\subset\Omega$ around $\rho$ containing no other zeros.
For connectedness: given $z_0,z_1\in \Omega\setminus Z(\xi)$, join them by a polygonal path in $\Omega$. A compact polygonal path meets only finitely many points of the discrete set $Z(\xi)\cap\Omega$, so we can locally perturb the path in small discs around those points to avoid them. This produces a path in $\Omega\setminus Z(\xi)$, hence $\Omega\setminus Z(\xi)$ is path-connected. The same argument applies to $\Omega\setminus(Z(\xi)\setminus\{\rho\})$.
\end{proof}
\noindent\textbf{Contradiction with right--edge normalization.}
By (N1), $\Theta(\sigma+it)\to\tfrac13$ as $\sigma\to+\infty$ (uniformly for $t$ in compact intervals). A constant unimodular function cannot have such a limit. Contradiction.
\medskip
\noindent\textbf{Conclusion of the pinch.}
No $\rho\in\Omega$ with $\xi(\rho)=0$ can exist.
\medskip
\noindent\textbf{Connective summary (secondary BRF/pinch route).}
This section records the Schur pinch argument: the Schur bound on $\Omega\setminus Z(\xi)$ comes from the interior limit-on-rectangles theorem (Theorem~\ref{thm:limit-rect}) and the exhaustion Corollary~\ref{cor:Schur-off-zeros}, and the pinch uses only (N1)--(N2). A boundary-wedge route via \textup{(P+)} is optional and recorded elsewhere for comparison, but is not required for the pinch.
\medskip
\noindent\textbf{Normalization at infinity (used in (N1)).}
We record explicit bounds ensuring $\Theta(\sigma+it)\to\tfrac13$ uniformly for $t$ in compact $t$-intervals as $\sigma\to+\infty$.
\begin{itemize}
\item Zeta limit: For $\sigma\ge 2$ and all $t\in\R$, $|\zeta(\sigma+it)-1|\le 2^{1-\sigma}$, hence $|\zeta(\sigma+it)|\to 1$ uniformly for $t$ in compact intervals as $\sigma\to+\infty$. Also $(\sigma+it-1)/(\sigma+it)\to 1$ uniformly on compact $t$-intervals.
\item Det$_2$ limit: For $\sigma\ge 1$, $\|A(\sigma+it)\|\le 2^{-\sigma}\le \tfrac12$. By the product representation in Lemma~\ref{lem:hs-diagonal} and since $\sum_p p^{-2\sigma}\to0$ as $\sigma\to\infty$, one has $|\dettwo(I-A(\sigma+it)) - 1|\le C\sum_p p^{-2\sigma}\to 0$ (uniformly for $t$ in compact intervals).
\item Far-field normalizer: By Definition~\ref{def:far-field-normalizer}, \(\log\mathcal O_{\mathrm{ff}}(s)\) is a truncated Cauchy integral with symmetric limits \([-T_{\mathrm{cut}},T_{\mathrm{cut}}]\) and kernel \(1/(\tau-w(s))-\tau/(1+\tau^2)\). As \(\sigma\to+\infty\), one has \(\Im w(s)=\Re s-\sigma_{\mathrm{ref}}\to+\infty\), hence \(1/(\tau-w(s))\to 0\) uniformly in \(\tau\in[-T_{\mathrm{cut}},T_{\mathrm{cut}}]\). Moreover \(u_{\mathrm{ref}}\) is even (it is a difference of log-moduli on a symmetric line), so the regularizing term integrates to \(0\). Thus \(\log\mathcal O_{\mathrm{ff}}(\sigma+it)\to 0\) and \(\mathcal O_{\mathrm{ff}}(\sigma+it)\to 1\) uniformly for \(t\) in compact intervals.
\end{itemize}
Combining, \(\mathcal J(\sigma+it)\to 1\) uniformly for \(t\) in compact intervals, hence
\(\Theta(\sigma+it)=(2\mathcal J-1)/(2\mathcal J+1)\to\tfrac13\).

\bigskip
% (The global RH claim becomes unconditional only once an arithmetic-to-certificate attachment bridge is supplied, i.e. once the quantitative attachment condition \eqref{eq:attachment} is verified on the required far-field rectangles; see Remark~\ref{rem:attachment-complete}.)

\begin{lemma}[Carleson box energy: stable sum bound]\label{lem:carleson-sum}
For harmonic potentials $U_1,U_2$ on $\Omega$, one has
\[
\sqrt{C_{\mathrm{box}}(U_1+U_2)}\ \le\
\sqrt{C_{\mathrm{box}}(U_1)}\ +\ \sqrt{C_{\mathrm{box}}(U_2)}.
\]
\end{lemma}
\begin{proof}
Write $\mu_j:=|\nabla U_j|^2\,\sigma\,dt\,d\sigma$ and $\mu_{12}:=|\nabla(U_1{+}U_2)|^2\,\sigma\,dt\,d\sigma$. For any Carleson box $B$, by Cauchy–Schwarz,
\[
\int_{B} |\nabla(U_1+U_2)|^2\,\sigma\,dt\,d\sigma
\ \le\ \Big(\sqrt{\int_B |\nabla U_1|^2\,\sigma}\ +\ \sqrt{\int_B |\nabla U_2|^2\,\sigma}\Big)^{\!2}.
\]
Taking supremum over Carleson boxes $B$ and dividing by $|I_B|$ yields the claimed inequality.
\end{proof}

\begin{corollary}[Local Carleson energy for $U_\xi$ on a fixed interval]\label{cor:xi-carleson-all-I}
For each compact interval $I\Subset\R$ there exists a finite constant $C_{\xi,I}<\infty$ such that
\[
  \iint_{Q(I)} |\nabla U_{\xi}(\sigma,t)|^2\,\sigma\,dt\,d\sigma\ \le\ C_{\xi,I}\,|I|.
\]
In particular, on Whitney intervals $I=[T-L,T+L]$ with $L=c/\log\langle T\rangle$ one may take $C_{\xi,I}=C_\xi$ from Lemma~\ref{lem:carleson-xi}.
\end{corollary}
\begin{proof}
(\emph{Sketch.}) Fix $I\Subset\R$. Cover $I$ by finitely many Whitney intervals $I_j=[T_j-L(T_j),T_j+L(T_j)]$ with bounded overlap (since $I$ is compact and $L(\cdot)$ is bounded below on $I$), so that $Q(I)\subset\bigcup_j Q(\alpha I_j)$. Apply Lemma~\ref{lem:carleson-xi} on each $Q(\alpha I_j)$ and sum; the overlap and the finiteness of the cover yield the stated bound with a constant depending on $I$ (through the finite cover) and on the fixed aperture.
\end{proof}

\begin{lemma}[L$^1$-tested control for $\partial_\sigma\Re\log\xi$]\label{lem:xi-deriv-L1}
For each compact $I\Subset\R$ there exists $C'_I<\infty$ such that for all $0<\sigma\le\varepsilon_0$ and all $\phi\in C_c^2(I)$,
\[
\Big|\int_I \phi(t)\,\partial_\sigma\Re\log\xi\!\big(\tfrac12+\sigma+it\big)\,dt\Big|
\ \le\ C'_I\,\|\phi\|_{H^1(I)}.
\]
\end{lemma}

\begin{proof}[Proof of Lemma~\ref{lem:xi-deriv-L1}]
Let $I\Subset\R$ and $\phi\in C_c^2(I)$. Let $V$ be the Poisson extension of $\phi$ on a fixed dilation $Q(\alpha I)$. Green's identity together with Cauchy–Riemann for $U_\xi=\Re\log\xi$ gives
\[
  \int_I \phi(t)\,\partial_\sigma\Re\log\xi\!\big(\tfrac12+\sigma+it\big)\,dt
  \,=\, \iint_{Q(\alpha I)} \nabla U_\xi\cdot\nabla V\,dt\,d\sigma.
\]
This is exactly the standard Carleson embedding / $H^1$–BMO pairing estimate for Poisson extensions (see Garnett \cite[Thm.~VI.1.1]{Garnett} or Stein \cite[Ch.~IV]{SteinSingInt}): if $\lambda:=|\nabla U_\xi|^2\,\sigma\,dt\,d\sigma$ is Carleson on boxes above $I$, then
\[
  \Big|\iint_{Q(\alpha I)} \nabla U_\xi\cdot\nabla V\,dt\,d\sigma\Big|
  \ \lesssim_{I,\alpha}\ \|\phi\|_{H^1(I)}.
\]
Using the local Carleson bound from Corollary~\ref{cor:xi-carleson-all-I} gives the asserted constant $C'_I<\infty$ depending only on $I$ (and the fixed aperture).
\end{proof}
\begin{corollary}[Conservative closure inequalities]\label{cor:conservative-closure}
Let $K_0$ be the arithmetic tail box-energy constant (Lemma~\ref{lem:carleson-arith}) and let $K_\xi$ be the neutralized $\xi$ box-energy constant (Lemma~\ref{lem:carleson-xi}). Define
\[
  C_{\mathrm{box}}^{(\zeta)}\ :=\ K_0+K_\xi.
\]
Then one has the conservative subadditivity bound
\[
  \sqrt{C_{\mathrm{box}}^{(\zeta)}}\ \le\ \sqrt{K_0}+\sqrt{K_\xi}.
\]
Moreover, for the printed window $\psi$ one has the structural mean-oscillation bound
\[
  M_\psi\ \le\ \frac{4}{\pi}\,C_\psi^{(H^1)}\,\sqrt{C_{\mathrm{box}}^{(\zeta)}}.
\]
\end{corollary}
\begin{proof}
The inequality $\sqrt{C_{\mathrm{box}}^{(\zeta)}}\le \sqrt{K_0}+\sqrt{K_\xi}$ is Lemma~\ref{lem:carleson-sum} applied to the decomposition of the paired potential into the arithmetic tail and the neutralized $\xi$-part (cf.\ Lemma~\ref{lem:outer-energy-bookkeeping}). The bound on $M_\psi$ follows from the $H^1$--BMO/Carleson embedding estimate (Lemma~\ref{lem:Mpsi-correct}) together with the embedding normalization $C_{\mathrm{CE}}(\alpha)=1$ (Lemma~\ref{lem:CE-constant-one}).
\end{proof}
\ifshownumerics
\paragraph*{Diagnostics (optional; non-load-bearing).}
Plugging the audited window constants into the structural bound yields the diagnostic enclosure
\[
  M_\psi\ \le\ \Mpsilocked,\qquad
  \Upsilon_{\mathrm{diag}}\ :=\ \frac{(2/\pi)\,M_\psi}{c_0(\psi)}\ \le\ \UpsilonLocked.
\]
(Closure of \textup{(P+)} uses the Whitney-uniform $\Upsilon_{\mathrm{Whit}}(c)$ from Lemma~\ref{lem:whitney-uniform-wedge}.)
\fi
\medskip
\medskip
\noindent\textbf{Proof of (N2) (non--cancellation at $\xi$--zeros).}
For $s=\sigma+it$ with $\sigma>\tfrac12$, define the diagonal operator $A(s)e_p=p^{-s}e_p$ on $\ell^2(\mathbb P)$. Then $\|A(s)\|=2^{-\sigma}<1$ and $\|A(s)\|_{\mathrm{HS}}^2=\sum_{p}p^{-2\sigma}<\infty$, so $A(s)$ is Hilbert--Schmidt. The 2--modified determinant for diagonal $A(s)$ is
\[
\det\nolimits_2\!\big(I-A(s)\big)\;=\;\prod_{p\in\mathbb P}(1-p^{-s})\,e^{p^{-s}},
\]
which converges absolutely and is nonzero because each factor is nonzero. Moreover, $I-A(s)$ is invertible with $\|(I-A(s))^{-1}\|\le (1-2^{-\sigma})^{-1}$ since $|1-p^{-s}|\ge 1-2^{-\sigma}>0$. Finally, the outer normalizer has the form $\mathcal O(s)=\exp H(s)$ with $H$ analytic on $\Omega$, hence $\mathcal O$ is zero--free on $\Omega$. Thus if $\rho\in\Omega$ with $\xi(\rho)=0$, then $\det_2(I-A(\rho))\neq0$ and $\mathcal O(\rho)\neq0$, i.e. no cancellation can occur at $\rho$. Local-uniform analyticity on $\Omega$ follows from HS$\to\dettwo$ continuity (Proposition~\ref{prop:hs-det2-continuity}).
\begin{lemma}[Diagonal HS determinant is analytic and nonzero]\label{lem:hs-diagonal}
For $s=\sigma+it$ with $\sigma>\tfrac12$, the diagonal operator $A(s)e_p=p^{-s}e_p$ satisfies
\[
\sup_{p}|p^{-s}|=2^{-\sigma}<1,\qquad \sum_{p}|p^{-s}|^2=\sum_{p}p^{-2\sigma}<\infty.
\]
Hence $A(s)\in\mathrm{HS}$, $I-A(s)$ is invertible, and
\[
\det\nolimits_2\big(I-A(s)\big)=\prod_{p}(1-p^{-s})\,e^{p^{-s}}
\]
is analytic and nonzero on $\{\Re s>\tfrac12\}$.
\end{lemma}
\begin{proof}
Immediate from the displayed bounds; invertibility follows since $|1-p^{-s}|\ge 1-2^{-\sigma}>0$, and the product defining $\det_2$ converges absolutely with nonzero factors.
\end{proof}
\paragraph{Normalization and finite port (eliminating $C_P$ and $C_\Gamma$).}
We record the implementation details that ensure the product certificate contains no prime budget and no Archimedean term.

\begin{lemma}[\(\zeta\)–normalized outer and compensator]\label{lem:zeta-normalization}
Define the outer $\mathcal O_\zeta$ on $\Omega$ with boundary modulus $\big|\dettwo(I-A)/\zeta\big|$ and set
\[ J_\zeta(s)\ :=\ \frac{\dettwo(I-A(s))}{\mathcal O_\zeta(s)\,\zeta(s)}\cdot B(s),\qquad B(s):=\frac{s-1}{s}. \]
On $\Re s=\tfrac12$ one has $|B|=1$. The phase–velocity identity of Theorem~\ref{thm:phase-velocity-quant} holds for $J_\zeta$ with the same Poisson/zero right-hand side. In particular, no separate Archimedean term enters the inequality used by the certificate.
\end{lemma}

\begin{proof}
Set $X:=\xi$ and $Z:=\zeta$, and let $G$ denote the archimedean factor linking them,
\[
  X(s)\;=\;\tfrac12 s(1{-}s)\,\pi^{-s/2}\,\Gamma(\tfrac s2)\,Z(s)\;=:\;G(s)\,Z(s).
\]
Define $\mathcal O_X$ (resp. $\mathcal O_Z$) to be the outer on $\Omega$ with boundary modulus $\big|\dettwo(I{-}A)/X\big|$ (resp. $\big|\dettwo(I{-}A)/Z\big|$). Then, by construction,
\[
  \Big|\frac{\dettwo(I{-}A)}{\mathcal O_X\,X}\Big|\equiv 1\equiv \Big|\frac{\dettwo(I{-}A)}{\mathcal O_Z\,Z}\Big|\quad \text{a.e. on }\{\Re s=\tfrac12\}.
\]
Consequently the phase–velocity identity (Theorem~\ref{thm:phase-velocity-quant}) applies to either unimodular ratio. Writing
\[
  \log \frac{\dettwo(I{-}A)}{\mathcal O_X\,X}
  \;=\; \log \frac{\dettwo(I{-}A)}{\mathcal O_Z\,Z}\; -\; \log\frac{\mathcal O_X}{\mathcal O_Z}\; -\; \log G,
\]
and differentiating in $\sigma$ on the boundary, the two outer terms contribute zero to the boundary phase derivative (by unimodularity and the outer/Poisson representation). The remaining difference is $-\partial_\sigma\Im\log G$.

On $\Re s=\tfrac12$ we have $|O_X/O_Z|=|Z/X|=|1/G|$, hence (a.e.) $\Re\log(O_X/O_Z)=-\Re\log G$. Since both $\log(O_X/O_Z)$ and $\log G$ are analytic on $\Omega$, Cauchy–Riemann gives on the boundary line (in $\mathcal D'(\R)$)
\[
  \partial_\sigma\Im\log\!\left(\frac{O_X}{O_Z}\right)
  \,=\,-\partial_t\Re\log\!\left(\frac{O_X}{O_Z}\right)
  \,=\,-\partial_t(-\Re\log G)
  \,=\,-\partial_\sigma\Im\log G.
\]
Compensating the simple zero at $s=1$ by the half–plane Blaschke factor
\[
  B(s)\;=\;\frac{s-1}{s}\qquad(|B|\equiv 1\text{ on }\Re s=\tfrac12)
\]
accounts for the inner contribution at $s=1$. Therefore, on the boundary,
\[
  \partial_\sigma\Im\log\!\Big(\frac{\dettwo(I{-}A)}{\mathcal O_Z\,Z}\cdot B\Big)
  \,=\, \partial_\sigma\Im\log\frac{\dettwo(I{-}A)}{\mathcal O_X\,X},
\]
and the quantitative phase–velocity identity holds in the same form for $J_\zeta=(\dettwo/(\mathcal O_\zeta\,\zeta))\,B$ as for $\mathcal J=\dettwo/(\mathcal O\,\xi)$. In particular, no Archimedean term enters the certificate.
\end{proof}

% (archived) A standalone prime-layer outer O_p is not used in the main chain; the ζ-normalized gauge and windowed identities suffice, and no C_P term enters the certificate.

\begin{corollary}[No $C_P$/$C_\Gamma$ in the certificate]
With $J_\zeta$ and $\widehat J$ as above, the active CR–Green route uses $c_0(\psi)$ and the CR–Green constant $C(\psi)$ together with the box–energy constant $C_{\rm box}^{(\zeta)}$. In particular, $C_P=0$ and $C_\Gamma=0$ on the RHS; $C_H(\psi)$ and $M_\psi$ are retained only as auxiliary/readability bounds.
\end{corollary}
\begin{proof}
By construction of the $\zeta$--normalized gauge and the compensator $B$ (Lemma~\ref{lem:zeta-normalization}), the Archimedean factor contributes no boundary phase term and the simple pole/zero bookkeeping at $s=1$ is absorbed into $B$ with $|B|=1$ on $\Re s=\tfrac12$. Thus the product certificate has no $C_\Gamma$ term and no separate prime-budget term $C_P$ on the right-hand side; the remaining inputs are $c_0(\psi)$, the CR--Green constant $C(\psi)$, and the box-energy constant $C_{\rm box}^{(\zeta)}$.
\end{proof}

\noindent\emph{Active route.} Throughout we use the $\zeta$-normalized boundary gauge with the Blaschke compensator; the product certificate uses $c_0(\psi)$ and the CR–Green constant $C(\psi)$ together with $C_{\rm box}^{(\zeta)}$ (no $C_P$, no $C_\Gamma$). These inputs yield Whitney-local smallness $\Upsilon_{\rm Whit}(c)<\tfrac12$ (Lemma~\ref{lem:whitney-uniform-wedge}); the remaining promotion to a global a.e.\ boundary wedge \textup{(P+)} after a single rotation is isolated in Remark~\ref{rem:wedge-application}.

% (Removed alternative interior-pole lemma to keep a single contradiction path in the pinch.)

\begin{lemma}[Derivative envelope for the printed window]\label{lem:CH-derivative-explicit}
Let $\psi$ be the even $C^\infty$ flat--top window from the "Printed window" paragraph (equal to $1$ on $[-1,1]$, supported in $[-2,2]$, with monotone ramps on $[-2,-1]$ and $[1,2]$), and $\varphi_L(t):=L^{-1}\psi((t-T)/L)$. Then, for every $L>0$,
\[
  \big\|\big(\mathcal H[\varphi_L]\big)'\big\|_{L^\infty(\mathbb R)} \;\le\; \frac{C_H(\psi)}{L}
  \qquad\text{with}\qquad C_H(\psi)\;\le\;\frac{2}{\pi}\;<\;0.65.
\]
\end{lemma}
\begin{proof}
\textit{Step 1 (Scaling).} By the standard scale/translation identity (recorded in the manuscript),
\[
  \mathcal H[\varphi_L](t)=H_\psi\!\Big(\frac{t-T}{L}\Big),\qquad
  H_\psi(x):=\frac{1}{\pi}\,\mathrm{p.v.}\!\int_{\mathbb R}\frac{\psi(y)}{x-y}\,dy,
\]
we get
\[
  \big(\mathcal H[\varphi_L]\big)'(t)=\frac{1}{L}\,H_\psi'\!\Big(\frac{t-T}{L}\Big)
  \quad\Longrightarrow\quad
  \big\|\big(\mathcal H[\varphi_L]\big)'\big\|_\infty=\frac{1}{L}\,\|H_\psi'\|_\infty.
\]
Thus it suffices to bound $\|H_\psi'\|_\infty$.

\smallskip
\textit{Step 2 (Structure and signs).} Since $\psi'\equiv0$ on $(-1,1)$ and the ramps are monotone,
\[
  \psi'(y)\ge0\ \text{on }[-2,-1],\qquad \psi'(y)\le0\ \text{on }[1,2],\qquad
  \int_{-2}^{-1}\psi'(y)\,dy=1=\!-\!\int_{1}^{2}\psi'(y)\,dy.
\]
In distributions, $(H_\psi)'= \mathcal H[\psi']$, so for every $x\in\mathbb R$
\[
  H_\psi'(x)=\frac{1}{\pi}\,\mathrm{p.v.}\!\int_{-2}^{-1}\frac{\psi'(y)}{x-y}\,dy\;+\;
             \frac{1}{\pi}\,\mathrm{p.v.}\!\int_{1}^{2}\frac{\psi'(y)}{x-y}\,dy.
\]

\smallskip
\textit{Step 3 (Worst case occurs between the ramps).} Fix $x\in(-1,1)$.  On $y\in[-2,-1]$ the kernel $y\mapsto 1/(x-y)$ is positive and strictly increasing; on $y\in[1,2]$ the kernel is negative and strictly decreasing.  Since the ramp densities are monotone and have unit mass in absolute value, the rearrangement/endpoint principle (maximize a monotone–kernel integral by concentrating mass at an endpoint) gives the pointwise bound
\[
  \Big|\mathrm{p.v.}\!\int_{-2}^{-1}\frac{\psi'(y)}{x-y}\,dy\Big|
  \le \frac{1}{1+x},\qquad
  \Big|\mathrm{p.v.}\!\int_{1}^{2}\frac{\psi'(y)}{x-y}\,dy\Big|
  \le \frac{1}{1-x}.
\]
Therefore, for every $x\in(-1,1)$,
\[
  |H_\psi'(x)| \;\le\; \frac{1}{\pi}\Big(\frac{1}{1+x}+\frac{1}{1-x}\Big)
  \;\le\; \frac{2}{\pi}\,\frac{1}{1-x^2}
  \;\le\; \frac{2}{\pi},
\]
with the maximum at $x=0$.
\smallskip
\textit{Step 4 (Outside the plateau).} For $x\notin[-1,1]$ the two ramp contributions have opposite signs but larger denominators, hence smaller magnitude. More precisely, for $x>1$, the left–ramp integral is a principal value on $[-2,-1]$ against a $C^\infty$ density that vanishes at the endpoints; the standard $C^1$–vanishing at $y=-2,-1$ eliminates the endpoint singularity and keeps the PV finite and strictly smaller than its in–plateau counterpart (a short integration–by–parts argument on the left interval makes this explicit). By evenness, the same holds for $x<-1$.  Consequently,
\[
  \sup_{x\in\mathbb R}|H_\psi'(x)|=\sup_{x\in(-1,1)}|H_\psi'(x)|\;\le\;\frac{2}{\pi}.
\]
Putting Steps 1–4 together,
\[
  \big\|\big(\mathcal H[\varphi_L]\big)'\big\|_\infty
  \;=\;\frac{1}{L}\,\|H_\psi'\|_\infty
  \;\le\;\frac{1}{L}\cdot\frac{2}{\pi}.
\]
Hence we can take $C_H(\psi)\le 2/\pi < 0.65$.
\end{proof}

% removed stray fi
% =========================================================

% Diagnostic numerics (gated; non-load-bearing).
\ifshownumerics
\paragraph*{Certificate \textemdash{} weighted \(p\)-adaptive model at \(\sigma_0=0.6\).}
Fix \(\sigma_0=0.6\), take \(Q=29\) and \(p_{\min}=\mathrm{nextprime}(Q)=31\).\\
Use the \(p\)-adaptive weighted off-diagonal enclosure (for all \(p\neq q\), uniformly in \(\sigma\in[\sigma_0,1]\)):
\[
\|H_{pq}(\sigma)\|_2 \;\le\; \frac{C_{\mathrm{win}}}{4}\, p^{-(\sigma+\tfrac12)}\, q^{-(\sigma+\tfrac12)},
\qquad C_{\mathrm{win}}=0.25.
\]

\noindent\emph{Prime sums (small block \(p\le Q\)).} With \(\sigma_0=0.6\),
\[
S_{\sigma_0}(Q)\;=\;\sum_{p\le Q} p^{-\sigma_0}\;=\;2.9593220929,\qquad
S_{\sigma_0+\tfrac12}(Q)\;=\;\sum_{p\le Q} p^{-(\sigma_0+\tfrac12)}\;=\;1.3239981250.
\]

% alt-route Bridges/KYP removed from main body
% removed optional Bridges A--C discussion and references
\noindent\emph{In-block Gershgorin lower bounds (uniform on \([\sigma_0,1]\)).}
Define
\[
L(p)\;:=\;(1-\sigma_0)\,(\log p)\,p^{-\sigma_0},\qquad 
\mu_p^{\mathrm L}\;\ge\;1-\frac{L(p)}{6}.
\]
At \(p_{\min}=31\) this gives
\[
L(31)=0.1750014502,\qquad 
\mu_{\min}^{\mathrm{far}}\;:=\;1-\frac{L(31)}{6}\;=\;0.9708330916.
\]
Over the small block \(p\le Q\) the worst case is at \(p=5\):
\[
L(5)=0.2451050257,\qquad 
\mu_{\min}^{\mathrm{small}}\;:=\;1-\frac{L(5)}{6}\;=\;0.9591491624.
\]
\noindent\emph{Off-diagonal budgets (all rigorous).}
Let \(\sigma^\star:=\sigma_0+\tfrac12=1.1\).\\
With the integer-tail majorant \(\displaystyle \sum_{n\ge p_{\min}-1} n^{-\sigma^\star}\le
\frac{(p_{\min}-1)^{1-\sigma^\star}}{\sigma^\star-1}\)
we obtain:
\[
\Delta_{\mathrm{FS}}
=\frac{C_{\mathrm{win}}}{4}\,p_{\min}^{-\sigma^\star}\,S_{\sigma^\star}(Q)
=0.0018935184,
\]
\[
\Delta_{\mathrm{FF}}
=\frac{C_{\mathrm{win}}}{4}\,p_{\min}^{-\sigma^\star}\!
\sum_{n\ge p_{\min}-1}\! n^{-\sigma^\star}
\;\le\;\frac{C_{\mathrm{win}}}{4}\,p_{\min}^{-\sigma^\star}\,
\frac{(p_{\min}-1)^{1-\sigma^\star}}{\sigma^\star-1}
=0.0101781777,
\]
\[
\Delta_{\mathrm{SS}}
=\frac{C_{\mathrm{win}}}{4}\,2^{-\sigma^\star}
\!\sum_{\substack{p\le Q\\ p\neq 2}}\! p^{-\sigma^\star}
=0.0250018328,
\]
\[
\Delta_{\mathrm{SF}}
=\frac{C_{\mathrm{win}}}{4}\,2^{-\sigma^\star}\!
\sum_{n\ge p_{\min}-1}\! n^{-\sigma^\star}
\;\le\;\frac{C_{\mathrm{win}}}{4}\,2^{-\sigma^\star}\,
\frac{(p_{\min}-1)^{1-\sigma^\star}}{\sigma^\star-1}
=0.2075080249.
\]
\noindent\emph{Certified finite-block spectral gap.}
Combining the in-block lower bounds with the off-diagonal budgets yields
\[
\delta_{\mathrm{cert}}(\sigma_0)\;\ge\;
\min\Big\{
\underbrace{\mu_{\min}^{\mathrm{small}}-(\Delta_{\mathrm{SS}}+\Delta_{\mathrm{SF}})}_{\text{small-block rows}}\,,\;
\underbrace{\mu_{\min}^{\mathrm{far}}-(\Delta_{\mathrm{FS}}+\Delta_{\mathrm{FF}})}_{\text{far-block rows}}\
\Big\}
=0.7266393047\;>\;0.
\]
Hence the normalized finite block is uniformly positive definite on \([\sigma_0,1]\).
\fi
\begin{corollary}[Boundary-uniform smoothed control]\label{cor:det2-boundary}
Let $I\Subset\R$, $\varepsilon_0\in(0,\tfrac12]$, and $\varphi\in C_c^2(I)$. Then, uniformly for $\sigma\in(\tfrac12,\tfrac12+\varepsilon_0]$,
\[
  \Big|\int_{\R} \varphi(t)\,\partial_\sigma\,\Re\log\dettwo\big(I-A(\sigma+it)\big)\,dt\Big|\ \le\ C_*\,\|\varphi''\|_{L^1(I)}.
\]
In particular, the bound remains valid in the boundary limit $\sigma\downarrow \tfrac12$ in the sense of distributions.
\end{corollary}
\begin{proof}
This is exactly the tested bound from Lemma~\ref{lem:det2-unsmoothed} (uniform in $\sigma\in(0,\varepsilon_0]$ after the shift $\sigma\mapsto \tfrac12+\sigma$). Since the right-hand side is uniform in $\sigma$, the family of distributions $\sigma\mapsto \partial_\sigma\Re\log\dettwo(I-A(\tfrac12+\sigma+it))$ is bounded in $\mathcal D'(I)$ and the estimate persists in the boundary limit $\sigma\downarrow\tfrac12$ when tested against $\varphi$.
\end{proof}
\subsection*{Smoothed Cauchy and outer limit (A2)}
% Lead-in: We build outers from boundary data u_ε and pass to a locally-uniform outer limit to normalize the boundary modulus.
\begin{proposition}[Outer normalization: existence, boundary a.e. modulus, and limit]\label{prop:outer-central}
There exist outer functions \(\mathcal O_\varepsilon\) on \(\{\Re s>\tfrac12+\varepsilon\}\) with a.e. boundary modulus
\[
  \big|\mathcal O_\varepsilon(\tfrac12+\varepsilon+it)\big|\ =\ \exp\big(u_\varepsilon(t)\big),
\]
and \(\mathcal O_\varepsilon\to\mathcal O\) locally uniformly on \(\Omega\) as \(\varepsilon\downarrow 0\), where \(\mathcal O\) has boundary modulus \(\exp u(t)\). (Standard Poisson–outer representation; see, e.g., \cite[Ch.~2]{DurenHp} and \cite[Ch.~2]{RosenblumRovnyak}.) Consequently the outer-normalized ratio \(\mathcal J=\dettwo(I-A)/(\mathcal O\,\xi)\) has a.e. boundary values on \(\Re s=\tfrac12\) with \(|\mathcal J(\tfrac12+it)|=1\).
\end{proposition}
\begin{proof}
Existence of each outer $\mathcal O_\varepsilon$ with the stated boundary modulus is standard. The Carleson-energy control for the relevant harmonic log-modulus on Whitney boxes implies the existence of a boundary trace $u\in \mathrm{BMO}(\R)\subset L^1_{\mathrm{loc}}(\R)$ and convergence $u_\varepsilon\to u$ in $L^1_{\mathrm{loc}}$ (Lemma~\ref{lem:desmooth-L1}). The Poisson/outer representation then gives local-uniform convergence $\mathcal O_\varepsilon\to\mathcal O$ on $\Omega$ and the unimodularity $|\mathcal J(\tfrac12+it)|=1$ a.e.
\end{proof}
\subsection*{Carleson energy and boundary BMO (unconditional)}
We record a direct Carleson–energy route to boundary BMO for the limit $u(t)=\lim_{\varepsilon\downarrow 0}u_\varepsilon(t)$.

\begin{lemma}[Arithmetic Carleson energy]\label{lem:carleson-arith}
Let
\[
 U_{\det_2}(\sigma,t)\ :=\ \sum_{p}\sum_{k\ge 2}\frac{(\log p)\,p^{-k/2}}{k\log p}\,e^{-k\log p\,\sigma}\,\cos\big(k\log p\,t\big),\qquad \sigma>0.
\]
Then for every interval $I\subset\R$ with Carleson box $Q(I):=I\times(0,|I|]$
\[
 \iint_{Q(I)} |\nabla U_{\det_2}|^2\,\sigma\,dt\,d\sigma\ \le\ \frac{|I|}{4}\,\sum_{p}\sum_{k\ge 2}\frac{p^{-k}}{k^2}
 \ =:\ K_0\,|I|,\qquad K_0:=\frac{1}{4}\sum_{p}\sum_{k\ge 2}\frac{p^{-k}}{k^2}<\infty.
\]
\end{lemma}
\begin{proof}
For a single mode $b\,e^{-\omega\sigma}\cos(\omega t)$ one has $|\nabla|^2=b^2\omega^2e^{-2\omega\sigma}$, hence
\[
 \int_0^{|I|}\!\int_I |\nabla|^2\,\sigma\,dt\,d\sigma\ \le\ |I|\cdot\sup_{\omega>0}\int_0^{|I|}\sigma\,\omega^2e^{-2\omega\sigma}d\sigma\cdot b^2\ \le\ \tfrac14\,|I|\,b^2.
\]
With $b=(\log p)\,p^{-k/2}/(k\log p)$ and $\omega=k\log p$, summing over $(p,k)$ gives the claim and the finiteness of $K_0$.
\end{proof}
\paragraph{Whitney scale and short–interval zeros.}
Throughout we use the Whitney schedule clipped at $L_\star$:
\[
  L\ =\ L(T)\ :=\ \frac{c}{\log\langle T\rangle}\ \le\ \frac{1}{\log\langle T\rangle},\qquad \langle T\rangle:=\sqrt{1+T^2},\
\]
for a fixed absolute $c\in(0,1]$; all boxes are $Q(\alpha I)$ with a uniform $\alpha\in[1,2]$.
We work on Whitney boxes $Q(I)$ with
\[
  L=L(T):=\min\Big\{\frac{c}{\log\langle T\rangle},\ L_\star\Big\},\qquad \langle T\rangle:=\sqrt{1+T^2},\quad c>0\ \text{fixed}.
\]
There exist absolute $A_0,A_1>0$ such that for $T\ge2$ and $0<H\le1$,
\[
  N(T;H)\ :=\ \#\{\rho=\beta+i\gamma:\ \gamma\in[T,T+H]\}\ \le\ A_0\ +\ A_1\,H\log\langle T\rangle.
\]
\begin{lemma}[Annular Poisson–balayage $L^2$ bound]\label{lem:annular-balayage}
Let $I=[T-L,T+L]$, $Q_\alpha(I)=I\times(0,\alpha L]$, and fix $k\ge1$. For
$\mathcal A_k:=\{\rho=\beta+i\gamma:\ 2^kL<|T-\gamma|\le 2^{k+1}L\}$ set
\[
  V_k(\sigma,t):=\sum_{\rho\in\mathcal A_k}\frac{\sigma}{(t-\gamma)^2+\sigma^2}.
\]
Then
\[
  \iint_{Q_\alpha(I)} V_k(\sigma,t)^2\,\sigma\,dt\,d\sigma\ \ll_\alpha\ |I|\,4^{-k}\,\nu_k,
\]
where $\nu_k:=\#\mathcal A_k$, and the implicit constant depends only on $\alpha$.
\end{lemma}
\begin{proof}
Write $K_\sigma(x):=\sigma/(x^2+\sigma^2)$ and $V_k=\sum_{\rho\in\mathcal A_k}K_\sigma(\cdot-\gamma)$. For any finite index set $\mathcal J$,
\[
  V_k^2\;\le\; \sum_{j\in\mathcal J} K_\sigma(\cdot-\gamma_j)^2\ +\ 2\!\!\sum_{i<j} K_\sigma(\cdot-\gamma_i)K_\sigma(\cdot-\gamma_j).
\]
Integrate over $t\in I$ first. For the diagonal terms, using $|t-\gamma|\ge 2^kL-L\ge 2^{k-1}L$ for $t\in I$ and $k\ge 1$,
\[
 \int_I K_\sigma(t-\gamma)^2\,dt\ =\ \int_I \frac{\sigma^2}{\big((t-\gamma)^2+\sigma^2\big)^2}\,dt
 \ \le\ \frac{\sigma}{(2^{k-1}L)^2}\int_I \frac{\sigma}{(t-\gamma)^2+\sigma^2}\,dt
 \ \le\ \frac{\pi\,\sigma}{(2^{k-1}L)^2}.
\]
Multiplying by the area weight $\sigma$ and integrating $\sigma\in(0,\alpha L]$ gives
\[
 \int_0^{\alpha L}\!\!\left(\int_I K_\sigma(t-\gamma)^2\,dt\right)\sigma\,d\sigma
 \ \le\ \frac{\pi}{(2^{k-1}L)^2}\int_0^{\alpha L}\!\sigma^2 d\sigma
 \ =\ \frac{\pi\,\alpha^3}{3}\,\frac{L}{4^{k-1}}
 \ \le\ \frac{C_{\mathrm{diag}}(\alpha)}{4^{k}}\,|I|,
\]
with $C_{\mathrm{diag}}(\alpha):=\tfrac{8\pi\alpha^3}{3}$ (using $|I|=2L$). Summing over $\nu_k$ choices of $\gamma$ contributes a factor $\nu_k$.

For the off-diagonal terms, for $i\ne j$ one has on $I$ that $K_\sigma(t-\gamma_j)\le \sigma/(2^{k-1}L)^2$. Hence
\[
 \int_I K_\sigma(t-\gamma_i)K_\sigma(t-\gamma_j)\,dt\ \le\ \frac{\sigma}{(2^{k-1}L)^2}\int_\R K_\sigma(t-\gamma_i)\,dt\ =\ \frac{\pi\sigma}{(2^{k-1}L)^2},
\]
and integrating $\sigma\in(0,\alpha L]$ with the extra factor $\sigma$ yields $\le C'_{\mathrm{off}}(\alpha)\,L\cdot 4^{-k}$. Summing in $i,j$ via the Schur test with $f_j(t):=K_\sigma(t-\gamma_j)\mathbf 1_I(t)$ gives
\[
 \int_I V_k(\sigma,t)^2\,dt\ \le\ C''(\alpha)\,\nu_k\,\frac{\sigma}{(2^kL)^2}.
\]
(This is a standard positive-kernel aggregation: the off-diagonal Gram matrix for the family $\{K_\sigma(\cdot-\gamma_j)\mathbf 1_I\}_j$ is controlled by Schur’s test, using the pointwise bound $K_\sigma\lesssim \sigma/(2^kL)^2$ on $I$ and the normalization $\int_\R K_\sigma=\pi$.)
Integrating $\sigma\in(0,\alpha L]$ with weight $\sigma$ gives $\le C_{\mathrm{off}}(\alpha)\,|I|\cdot 4^{-k}\,\nu_k$. Combining diagonal and off–diagonal parts, absorbing harmless constants into $C_\alpha$, we obtain the stated bound with an explicit $C_\alpha=O(\alpha^3)$.
\end{proof}

\begin{lemma}[Analytic ($\xi$) Carleson energy on Whitney boxes]\label{lem:carleson-xi}
\emph{Reference.} The local zero count used below follows from the Riemann–von Mangoldt formula; see Titchmarsh \cite[Thm.~9.3]{Titchmarsh} (or, e.g., Ivi\'c, Ch.~8).
There exist absolute constants $c\in(0,1]$ and $C_\xi<\infty$ such that for every interval $I=[T-L,\,T+L]$ with Whitney scale $L:=c/\log\langle T\rangle$, the Poisson extension
\[
 U_{\xi}(\sigma,t):=\Re\log\xi\big(\tfrac12+\sigma+it\big),\qquad (\sigma>0),
\]
\paragraph{Whitney scale and neutralization.}
Throughout this lemma we take the base interval $I=[T-L,T+L]$ with
\[
  L=L(T):=\frac{c}{\log\langle T\rangle},\qquad \langle T\rangle:=\sqrt{1+T^2},\quad c>0\ \text{fixed}.
\]
obeys the Carleson bound
\[ \iint_{Q(I)} |\nabla U_{\xi}(\sigma,t)|^2\,\sigma\,dt\,d\sigma\ \le\ C_\xi\,|I|. \]
\end{lemma}

\begin{proof}
All inputs are unconditional. Fix $I=[T-L,T+L]$ with $L=c/\log\langle T\rangle$ and aperture $\alpha\in[1,2]$. Neutralize near zeros by a local half-plane Blaschke product $B_I$ removing zeros of $\xi$ inside a fixed dilate $Q(\alpha'I)$ ($\alpha'>\alpha$). This yields a harmonic field $\widetilde U_\xi$ on $Q(\alpha I)$ and
\[
  \iint_{Q(\alpha I)} |\nabla U_\xi|^2\,\sigma\,dt\,d\sigma\ \asymp\ \iint_{Q(\alpha I)} |\nabla \widetilde U_\xi|^2\,\sigma\,dt\,d\sigma\ +\ O_\alpha(|I|),
\]
so it suffices to bound the neutralized energy.

Write $\partial_\sigma U_\xi=\Re\,(\xi'/\xi)=\Re\sum_\rho (s-\rho)^{-1}+A$, where $A$ is smooth on compact strips. Since $U_\xi$ is harmonic, $|\nabla U_\xi|^2\asymp |\partial_\sigma U_\xi|^2$ on $\R^2_+$; thus we bound the $L^2(\sigma\,dt\,d\sigma)$ norm of $\sum_\rho (s-\rho)^{-1}$ over $Q(\alpha I)$. Decompose the (neutralized) zeros into Whitney annuli $\mathcal A_k:=\{\rho:2^kL<|\gamma-T|\le 2^{k+1}L\}$, $k\ge1$. For $V_k(\sigma,t):=\sum_{\rho\in\mathcal A_k} K_\sigma(t-\gamma)$ with $K_\sigma(x):=\sigma/(x^2+\sigma^2)$, Lemma~\ref{lem:annular-balayage} gives
\[
  \iint_{Q_\alpha(I)} V_k(\sigma,t)^2\,\sigma\,dt\,d\sigma\ \le\ C_\alpha\,|I|\,4^{-k}\,\nu_k,
\]
where $\nu_k:=\#\mathcal A_k$ and $C_\alpha$ depends only on $\alpha$. Summing Cauchy–Schwarz bounds over annuli yields
\[
  \iint_{Q(\alpha I)} \Big|\sum_{\rho}(s-\rho)^{-1}\Big|^2\,\sigma\,dt\,d\sigma\ \le\ C_\alpha\,|I|\sum_{k\ge1}4^{-k}\,\nu_k.
\]
To bound $\nu_k$, we use the short-interval zero count recorded above: there exist absolute $A_0,A_1>0$ such that for $T\ge 2$ and $0<H\le 1$,
\[
  N(T;H)\ :=\ \#\{\rho=\beta+i\gamma:\ \gamma\in[T,T+H]\}\ \le\ A_0\ +\ A_1\,H\log\langle T\rangle.
\]
For annuli with $2^kL\le 1$, $\nu_k$ counts zeros in a window of length $\asymp 2^kL$, hence
\[
  \nu_k\ \le\ a_0(\alpha)\ +\ a_1(\alpha)\,2^kL\,\log\langle T\rangle.
\]
For the finitely many remaining annuli with $2^kL>1$, the Riemann--von Mangoldt formula (Titchmarsh \cite[Thm.~9.3]{Titchmarsh}) gives the cruder bound $\nu_k\ll_\alpha 2^kL\,\log\langle T\rangle$, which is sufficient since $4^{-k}\nu_k$ is summable. Therefore,
\[
  \sum_{k\ge1}4^{-k}\,\nu_k
  \ \ll_\alpha\ 
  \sum_{k\ge1}4^{-k}\Big(1+2^kL\,\log\langle T\rangle\Big)
  \ \ll\ 1\ +\ L\,\log\langle T\rangle.
\]
On Whitney scale $L=c/\log\langle T\rangle$ this is $\ll_c 1$.
Adding the neutralized near-field $O(|I|)$ and the smooth $A$ contribution, we obtain
\[
  \iint_{Q(\alpha I)} |\nabla U_\xi|^2\,\sigma\,dt\,d\sigma\ \le\ C_\xi\,|I|,
\]
with $C_\xi$ depending only on $(\alpha,c)$. This proves the lemma.
\end{proof}

\begin{proposition}[Whitney Carleson finiteness for $U_\xi$]\label{prop:Kxi-finite}
For each fixed Whitney aperture $\alpha\in[1,2]$ there exists a finite constant
$K_\xi=K_\xi(\alpha)<\infty$ such that
\[
  \iint_{Q(\alpha I)} |\nabla U_\xi|^2\,\sigma\,dt\,d\sigma \;\le\; K_\xi\,|I|
\]
for every Whitney base interval $I$. Consequently $C_{\rm box}^{(\zeta)}=K_0+K_\xi<\infty$, and
\[
  c \;\le\; \Big(\tfrac{c_0(\psi)}{2\,C(\psi)\,\sqrt{K_0+K_\xi}}\Big)^2
\]
ensures $\Upsilon_{\mathrm{Whit}}(c)<\tfrac12$ and provides the required Whitney-local smallness parameter for Lemma~\ref{lem:whitney-uniform-wedge}. (A global a.e. boundary wedge \textup{(P+)} still requires the local-to-global upgrade discussed in Remark~\ref{rem:wedge-application}.)
\end{proposition}
\begin{proof}
The Whitney-box estimate for $U_\xi$ is exactly Lemma~\ref{lem:carleson-xi}; take $K_\xi$ to be the constant there (for the fixed aperture $\alpha$). The finiteness of $C_{\rm box}^{(\zeta)}$ then follows by combining the prime-tail box bound $K_0$ (Lemma~\ref{lem:carleson-arith}) with the stable-sum estimate (Lemma~\ref{lem:carleson-sum}). The final inequality is the stated sufficient smallness condition in Lemma~\ref{lem:whitney-uniform-wedge}.
\end{proof}

\paragraph{Boxed audit: unconditional enclosure of $C_{\rm box}^{(\zeta)}$.}
Fix $I=[T-L,T+L]$ with $L=c/\log\langle T\rangle$ and $Q(I)=I\times(0,L]$. Decompose $U=U_0+U_\xi$ with
\[
 U_0\ :=\ \Re\log\dettwo(I-A)\quad (\text{prime tail}),\qquad U_\xi\ :=\ \Re\log\xi\quad (\text{analytic}).
\]
\emph{Prime tail.} Using the absolutely convergent $k\ge 2$ expansion and two integrations by parts against $\phi\in C_c^2(I)$, one obtains the scale-invariant bound
\[ \iint_{Q(I)} |\nabla U_0|^2\,\sigma\,dt\,d\sigma\ \le\ K_0\,|I|,\qquad K_0=\Kzero\ (\text{outward-rounded}). \]
\emph{Zeros (neutralized).} Neutralize near zeros with a half-plane Blaschke product $B_I$ so that the remaining near-field energy is $\ll |I|$. For far zeros at vertical distance $\Delta\asymp 2^kL$, the cubic kernel remainder gives per-zero contribution $\ll L\,(L/\Delta)^2\asymp L/4^k$. Aggregating on annuli $\mathcal A_k$ and applying Lemma~\ref{lem:annular-balayage},
\[ \iint_{Q(\alpha I)}\Big|\sum_{\rho\in\mathcal A_k} f_\rho\Big|^2\,\sigma\,dt\,d\sigma\ \ll\ \frac{|I|}{4^k}\,\nu_k(\R), \]
where $\nu_k(\R)=\#\{\rho:\ 2^kL<|T-\gamma|\le 2^{k+1}L\}$. By the unconditional zero-density bounds of Vinogradov–Korobov (with explicit constants), for each fixed Whitney scale one has a uniform count
\[
  \nu_k(\R)\ \ll\ 1\ +\ 2^kL\log\langle T\rangle,
\]
using the short-interval zero count $N(T;H)\le A_0+A_1H\log\langle T\rangle$ for $H\le 1$ (and a crude Riemann--von Mangoldt bound for the finitely many annuli with $2^kL>1$). The implied constant is independent of $T$ and $k$.
Summing $k\ge 1$ and using $L=c/\log\langle T\rangle$ gives
\[ \iint_{Q(\alpha I)} |\nabla U_\xi|^2\,\sigma\,dt\,d\sigma\ \le\ K_\xi\,|I|,\qquad \text{for a finite constant }K_\xi. \]
\medskip
\noindent\fbox{\begin{minipage}{0.98\textwidth}
\textbf{Boxed $K_\xi$ audit (parametric; diagnostic).} With $C_\alpha$ from Lemma~\ref{lem:annular-balayage},
\[
  K_\xi \ \le\ C_\alpha\!\left(\frac{1}{2\pi}\sum_{j\ge1} j^{-2} \ +\ 2\sum_{j\ge1} j^{-3}\right)
  \ =\ C_\alpha\!\left(\frac{\pi}{12} \ +\ 2\,\zeta(3)\right).
\]
\end{minipage}}
Combining,
\[
\boxed{\ C_{\rm box}^{(\zeta)}\ :=\ \sup_{I}\ \frac{1}{|I|}\iint_{Q(\alpha I)} |\nabla U|^2\,\sigma\,dt\,d\sigma\ \le\ K_0+K_\xi\ =\ \CboxZeta\ .\ }
\]
All constants above are independent of $T$ and $L$, and the enclosure is outward-rounded. This is the \emph{only} Carleson input used in the active certificate.
\begin{proof}
Write
\[
 \partial_\sigma U_{\xi}(\sigma,t)\ =\ \Re\frac{\xi'}{\xi}\!\left(\tfrac12+\sigma+it\right)
 \ =\ \Re\sum_{\rho}\frac{1}{\tfrac12+\sigma+it-\rho}\ +\ A(\sigma,t),
\]
where the sum runs over nontrivial zeros $\rho=\beta+i\gamma$ of $\zeta$, and $A(\sigma,t)$ collects the archimedean part and the trivial factors (these are smooth in $(\sigma,t)$ on compact strips). Since $U_{\xi}$ is harmonic, $|\nabla U_{\xi}|^2\asymp |\partial_\sigma U_{\xi}|^2$ on $\R^2_+$; it suffices to estimate the latter.

Fix $I=[T-L,T+L]$ and decompose the zero set into near and far parts relative to $Q(I)=I\times(0,L]$:
\[
 \mathcal Z_{\mathrm{near}}:=\{\rho:\ |\gamma-T|\le 2L\},\qquad \mathcal Z_{\mathrm{far}}:=\{\rho:\ |\gamma-T|>2L\}.
\]
\subsubsection*{Neutralized near field}
Let $B_I$ be the half-plane Blaschke product over zeros with $|\gamma-T|\le 3L$ and define the neutralized potential $\widetilde U_\xi:=\Re\log\big(\xi\,B_I\big)$ and its $\sigma$-derivative $\widetilde f:=\partial_\sigma\widetilde U_\xi$. Then $\sum_{\rho\in \mathcal Z_{\mathrm{near}}}\nabla f_\rho$ is canceled inside $Q(I)$ up to a boundary error controlled by the Poisson energy of $\psi$ (independent of $T,L$). Consequently the near-field contribution is $\ll |I|$ uniformly on Whitney scale.

\noindent\emph{Remark (bound used in the certificate).} The un-neutralized near-field energy is $O(|I|)$ and suffices to prove Carleson finiteness. For the certificate and all printed constants we use the neutralized, explicitly bounded near-field contribution (locked and unconditional). The coarse un-neutralized $O(1)$ bound is not used for numeric closure.

For the far zeros (neutralized field), set annuli $\mathcal A_k:=\{\rho:\ 2^kL<|\gamma-T|\le 2^{k+1}L\}$ for $k\ge1$. For a single zero at vertical distance $\Delta:=|\gamma-T|$ one has the kernel estimate
\[
 \int_0^{L}\!\int_{T-L}^{T+L} \frac{\sigma}{\sigma^2+(t-\gamma)^2}\,dt\,d\sigma\ \ll\ L\,\Big(\frac{L}{\Delta}\Big)^{\!2}.
\]
For the far annuli $\mathcal A_k$, apply Lemma~\ref{lem:annular-balayage} to the annular Poisson sums $V_k$ to control cross terms linearly in the annular mass:
\[
  \iint_{Q(\alpha I)}\Big|\sum_{\rho\in\mathcal A_k} f_{\rho}\Big|^2\,\sigma\,dt\,d\sigma\ \ll\ \frac{|I|}{4^k}\,\nu_k(\R),
\]
where $\nu_k(\R)=\#\{\rho:\ 2^kL<|T-\gamma|\le 2^{k+1}L\}$. By the unconditional zero-density bounds of Vinogradov–Korobov (with explicit constants), for each fixed Whitney scale one has a uniform count
\[ \nu_k(\R)\ \ll\ 2^kL\log\langle T\rangle\ +\ \log\langle T\rangle, \]
with the implied constant independent of $T$ and $k$.
Summing $k\ge1$ yields a total far contribution
\[ \ll\ |I|\sum_{k\ge1}\frac{1}{4^k}\big(2^kL\log\langle T\rangle+\log\langle T\rangle\big)\ \ll\ |I|\,(L\log\langle T\rangle+1), \]
which is $\ll |I|$ on the Whitney scale $L=c/\log\langle T\rangle$.

Adding the direct near-field $O(|I|)$ bound, the far-field $O(|I|)$ sum, and the smooth Archimedean term gives
\[
 \iint_{Q(\alpha I)} |\nabla U_\xi|^2\,\sigma\,dt\,d\sigma\ \ll\ |I|.
\]
This proves the claimed Carleson bound on Whitney boxes without neutralization in the energy step.
\end{proof}
\begin{remark}[VK zero-density constants and explicit $C_\xi$]
Let $N(\sigma,T)$ denote the number of zeros with $\Re\rho\ge \sigma$ and $0<\Im\rho\le T$. The Vinogradov–Korobov zero-density estimates give, for some absolute constants $C_0,\kappa>0$, that
\[
  N(\sigma,T)\ \le\ C_0\,T\,\log T\ +\ C_0\,T^{1-\kappa(\sigma-1/2)}\qquad (\tfrac12\le \sigma<1,\ T\ge T_1),
\]
with an effective threshold $T_1$. On Whitney scale $L=c/\log\langle T\rangle$, these bounds imply the annular counts used above with explicit $A,B$ of size $\ll 1$ for each fixed $c,\alpha$. Consequently, one can take
\[
  C_\xi\ \le\ C(\alpha,c)\,\big(C_0+1\big)
\]
in Lemma~\ref{lem:carleson-xi}, where $C(\alpha,c)$ is an explicit polynomial in $\alpha$ and $c$ arising from the annular $L^2$ aggregation (cf. Lemma~\ref{lem:annular-balayage}). We do not need the sharp exponents; any effective VK pair $(C_0,\kappa)$ suffices for a finite $C_\xi$ on Whitney boxes.
\end{remark}
% Active version of the cutoff pairing lemma (unarchived for references)
\begin{lemma}[Cutoff pairing on boxes]\label{lem:cutoff-pairing}
Fix parameters $\alpha'>\alpha>1$. Let $\chi_{L,t_0}\in C_c^\infty(\R^2_+)$ satisfy $\chi\equiv1$ on $Q(\alpha I)$, $\operatorname{supp}\chi\subset Q(\alpha'I)$, $\|\nabla\chi\|_\infty\lesssim L^{-1}$ and $\|\nabla^2\chi\|_\infty\lesssim L^{-2}$. Let $V_{\psi,L,t_0}$ be the Poisson extension of $\psi_{L,t_0}$ and $\widetilde U$ the neutralized field. Then
\[
 \int_{\R} u(t)\,\psi_{L,t_0}(t)\,dt
 \ =\ \iint_{Q(\alpha'I)} \nabla \widetilde U\cdot \nabla\big(\chi_{L,t_0}\, V_{\psi,L,t_0}\big)\,dt\,d\sigma\ +\ \mathcal R_{\mathrm{side}}\ +\ \mathcal R_{\mathrm{top}},
\]
with
\[
 |\mathcal R_{\mathrm{side}}|+|\mathcal R_{\mathrm{top}}|
 \ \lesssim\ \Big(\iint_{Q(\alpha'I)} |\nabla \widetilde U|^2\,\sigma\Big)^{1/2}
               \cdot \Big(\iint_{Q(\alpha'I)} \big(|\nabla\chi|^2\,|V_{\psi,L,t_0}|^2+|\nabla V_{\psi,L,t_0}|^2\big)\,\sigma\Big)^{1/2}.
\]
\end{lemma}
\begin{proof}
Apply Green's identity on $Q(\alpha'I)$ to $\widetilde U$ and $\chi_{L,t_0}V_{\psi,L,t_0}$:
\[
  \iint_{Q(\alpha'I)} \nabla \widetilde U\cdot \nabla(\chi V)\,dt\,d\sigma
  \ =\ \int_{\partial Q(\alpha'I)} \chi V\,\partial_n \widetilde U\,ds.
\]
Since $\chi$ is supported in $Q(\alpha'I)$ and equals $1$ on $Q(\alpha I)$, the boundary integral splits into the bottom edge (where $\chi V=\psi_{L,t_0}$) plus side/top edges and cutoff-transition edges; these latter contributions are grouped into $\mathcal R_{\mathrm{side}}$ and $\mathcal R_{\mathrm{top}}$.
On the bottom edge, Cauchy–Riemann for $\log J=\widetilde U+i\widetilde W$ gives $\partial_n \widetilde U=-\partial_\sigma \widetilde U=\partial_t \widetilde W$, so
\[
  -\int_{\partial Q\cap\{\sigma=0\}} \chi V\,\partial_n \widetilde U\,dt
  \ =\ -\int_{\R}\psi_{L,t_0}(t)\,\partial_t \widetilde W(t)\,dt
  \ =\ \int_{\R} u(t)\,\psi_{L,t_0}(t)\,dt,
\]
where $u(t)$ denotes the boundary trace paired against $\psi_{L,t_0}$ (the phase distribution after neutralization).
Finally, the remainder bound follows by Cauchy–Schwarz, using $\|\nabla\chi\|_\infty\lesssim L^{-1}$ and the displayed test-energy factor.
\end{proof}
% Archived duplicate block (removed in submission branch)
% archived block removed
\begin{lemma}[CR–Green pairing for boundary phase]\label{lem:CR-green-phase}
Let $J$ be analytic on $\Omega$ with a.e. boundary modulus $|J(\tfrac12+it)|=1$, and write $\log J=U+iW$ on $\Omega$, so $U$ is harmonic with $U(\tfrac12+it)=0$ a.e. Fix a Whitney interval $I=[t_0-L,t_0+L]$ and let $V_{\psi,L,t_0}$ be the Poisson extension of $\psi_{L,t_0}$. Then, with a cutoff $\chi_{L,t_0}$ as in Lemma~\ref{lem:cutoff-pairing},
\[
  \int_{\R} \psi_{L,t_0}(t)\,\big(-W'(t)\big)\,dt\ =\ \iint_{Q(\alpha'I)} \nabla U\cdot \nabla\big(\chi_{L,t_0}\,V_{\psi,L,t_0}\big)\,dt\,d\sigma\ +\ \mathcal R_{\mathrm{side}}\ +\ \mathcal R_{\mathrm{top}},
\]
and the remainders satisfy
\[
  |\mathcal R_{\mathrm{side}}|+|\mathcal R_{\mathrm{top}}|\ \lesssim\ \Big(\iint_{Q(\alpha'I)} |\nabla U|^2\,\sigma\Big)^{1/2}\ \cdot\ \Big(\iint_{Q(\alpha'I)} (|\nabla\chi|^2\,|V|^2+|\nabla V|^2)\,\sigma\Big)^{1/2}.
\]
In particular, by Cauchy–Schwarz and the scale–invariant Dirichlet bound for $V_{\psi,L,t_0}$, there is a constant $C(\psi)$ such that
\[
  \int_{\R} \psi_{L,t_0}(t)\,\big(-w'(t)\big)\,dt\ \le\ C(\psi)\,\Big(\iint_{Q(\alpha'I)} |\nabla U|^2\,\sigma\Big)^{1/2}.
\]
Moreover, replacing $U$ by $U-\Re\log\mathcal O$ for any outer $\mathcal O$ with boundary modulus $e^{u}$ leaves the left-hand side unchanged and affects only the right-hand side through $\nabla\Re\log\mathcal O$ (Lemma~\ref{lem:outer-cancel}).
\end{lemma}
\begin{proof}[Boundary identity justification]
On the bottom edge $\{\sigma=0\}$ the outward normal is $\partial_n=-\partial_\sigma$. By Cauchy–Riemann for $\log J=U+iW$ on the boundary line $\{\Re s=\tfrac12\}$ one has $\partial_n U=-\partial_\sigma U=\partial_t W$. Hence
\[
-\int_{\partial Q\cap\{\sigma=0\}} \chi\,V\,\partial_n U\,dt\ =\ -\int_{\R} \psi_{L,t_0}(t)\,\partial_t W(t)\,dt\ =\ \int_{\R} \psi_{L,t_0}(t)\,\big(-w'(t)\big)\,dt,
\]
which yields the displayed identity after including the interior term and remainders.
\end{proof}
\begin{lemma}[Outer cancellation in the CR--Green pairing]\label{lem:outer-cancel}
With the notation of Lemma~\ref{lem:CR-green-phase}, replace $U$ by $U-\Re\log\mathcal O$, where $\mathcal O$ is any outer on $\Omega$ with a.e.\ boundary modulus $e^{u}$ and boundary argument derivative $\frac{d}{dt}\Arg\mathcal O=\Hilb[u']$ (Lemma~\ref{lem:outer-phase-HT}). Then the left-hand side of the identity in Lemma~\ref{lem:CR-green-phase} is unchanged, and the right-hand side depends only on $\nabla\!\big(U-\Re\log\mathcal O\big)$.
\end{lemma}
\begin{proof}
On the bottom edge, replacing $U$ by $U-\Re\log\mathcal O$ changes the boundary term by
$\int_{\mathbb R}\psi_{L,t_0}(t)\,\partial_t\Arg\mathcal O(\tfrac12+it)\,dt
=\int_{\mathbb R}\psi_{L,t_0}(t)\,\mathsf H[u'](t)\,dt$
(Lemma~\ref{lem:outer-phase-HT}), which cancels against the outer contribution already subsumed in $-w'$. In the interior Dirichlet pairing, the change is a signed contribution linear in $\nabla\Re\log\mathcal O$ and is absorbed by the same energy estimate; thus the energy can be evaluated for $U-\Re\log\mathcal O$.
\end{proof}
\begin{corollary}[Explicit remainder control]
With notation as in Lemma~\ref{lem:CR-green-phase}, there exists $C_{\mathrm{rem}}=C_{\mathrm{rem}}(\alpha,\psi)$ such that
\[
  |\mathcal R_{\mathrm{side}}|+|\mathcal R_{\mathrm{top}}|
 \lesssim\ C_{\mathrm{rem}}\,\Big(\iint_{Q(\alpha'I)} |\nabla U|^2\,\sigma\Big)^{1/2}.
\]
In particular, one may take $C_{\mathrm{rem}}\asymp_\alpha \mathcal A(\psi)$, where $\mathcal A(\psi)$ is the fixed Poisson energy of the window (cf. Corollary~\ref{cor:CH-Mpsi-final}).
\end{corollary}
% end archived block removal
\begin{proof}
From Lemma~\ref{lem:CR-green-phase},
\[
  |\mathcal R_{\mathrm{side}}|+|\mathcal R_{\mathrm{top}}| \lesssim\ \Big(\iint_{Q(\alpha'I)} |\nabla U|^2\,\sigma\Big)^{1/2}\,\cdot\,\Big(\iint_{Q(\alpha'I)} (|\nabla\chi|^2\,|V|^2+|\nabla V|^2)\,\sigma\Big)^{1/2}.
\]
The cutoff satisfies $\|\nabla\chi\|_\infty\lesssim L^{-1}$ and is supported in a fixed dilate $Q(\alpha' I)$ with bounded overlap, while $V$ is the Poisson extension of the fixed window $\psi$; hence the second factor is $\asymp_\alpha \mathcal A(\psi)$, independent of $(T,L)$. Absorbing constants depending only on $(\alpha,\psi)$ yields the claim.
\end{proof}

% --- Outer cancellation and which energy is bounded ---
\begin{lemma}[Outer cancellation and energy bookkeeping on boxes]\label{lem:outer-energy-bookkeeping}
Let
\[
u_0(t):=\log\Big|\det\nolimits_2\!\big(I-A(\tfrac12+it)\big)\Big|,\qquad
u_\xi(t):=\log\big|\xi(\tfrac12+it)\big|,
\]
and let $O$ be the outer on $\Omega$ with boundary modulus
\(
|O(\tfrac12+it)|=\exp\!\big(u_0(t)-u_\xi(t)\big).
\)
Set
\[
J(s):=\frac{\det\nolimits_2(I-A(s))}{O(s)\,\xi(s)},\qquad
\log J=U+iW,\qquad U_0:=\Re\log\det\nolimits_2(I-A),\quad U_\xi:=\Re\log\xi.
\]
Then for every Whitney interval $I=[t_0-L,t_0+L]$ and the standard test field $V_{\psi,L,t_0}$,
\begin{equation}\label{eq:CRG-outer-cancel}
\int_{\R}\psi_{L,t_0}(t)\,(-W'(t))\,dt
=\iint_{Q(\alpha' I)} \nabla\!\big(U_0-U_\xi-\Re\log O\big)\cdot\nabla\!\big(\chi_{L,t_0}V_{\psi,L,t_0}\big)\,dt\,d\sigma
+\mathcal R_{\mathrm{side}}+\mathcal R_{\mathrm{top}}
\end{equation}
and hence, by Cauchy--Schwarz and the scale‑invariant Dirichlet bound for $V_{\psi,L,t_0}$,
\begin{equation}\label{eq:energy-U-used}
\int_{\R}\psi_{L,t_0}\,(-W')\ \le\ C(\psi)\,\Big(C_{\rm box}\big(U_0-U_\xi-\Re\log O\big)\,|I|\Big)^{1/2}
\end{equation}
Moreover $\Re\log O$ is the Poisson extension of the boundary function $u:=u_0-u_\xi$, so
\begin{equation}\label{eq:Poisson-splitting}
U_0-U_\xi-\Re\log O
:=\underbrace{(U_0-\Poisson[u_0])}_{\equiv 0}\ -\ \big(U_\xi-\Poisson[u_\xi]\big)
\end{equation}
and consequently the Carleson box energy that actually enters \eqref{eq:energy-U-used} satisfies
\begin{equation}\label{eq:sharp-Kxi}
C_{\rm box}\big(U_0-U_\xi-\Re\log O\big)\ \le\ K_\xi
\end{equation}
In particular, the coarse bound
\begin{equation}\label{eq:coarse-K0Kxi}
C_{\rm box}\big(U_0-U_\xi-\Re\log O\big)\ \le\ K_0+K_\xi\ =\ \CboxZeta
\end{equation}
also holds, by the triangle inequality for $C_{\rm box}$ and linearity of the Poisson extension.
\end{lemma}

\begin{proof}
The identity \eqref{eq:CRG-outer-cancel} is Lemma~\ref{lem:CR-green-phase} with $U$ replaced by $U-\Re\log O$, together with the outer cancellation Lemma~\ref{lem:outer-cancel}; subtracting $\Re\log O$ leaves the left side (phase) unchanged. The estimate \eqref{eq:energy-U-used} follows as in Lemma~\ref{lem:CR-green-phase} from Cauchy--Schwarz and the scale‑invariant Dirichlet bound, with $C(\psi)=C_{\mathrm{rem}}(\alpha,\psi)\,\mathcal A(\psi)$ independent of $L,t_0$.

By Lemma~\ref{lem:outer-phase-HT}, $\Re\log O=\Poisson[u]$ with $u=u_0-u_\xi$, and since $U_0$ is harmonic with boundary trace $u_0$ we have $U_0=\Poisson[u_0]$, giving \eqref{eq:Poisson-splitting}. The remainder $U_\xi-\Poisson[u_\xi]$ is the (neutralized) Green potential of zeros; its Whitney–box energy is bounded by $K_\xi$ (see Lemma~\ref{lem:carleson-xi} and the annular $L^2$ aggregation), which yields \eqref{eq:sharp-Kxi}. Finally, \eqref{eq:coarse-K0Kxi} follows from the subadditivity
\(
\sqrt{C_{\rm box}(U_1+U_2)}\le \sqrt{C_{\rm box}(U_1)}+\sqrt{C_{\rm box}(U_2)}
\)
(Lemma~\ref{lem:carleson-sum}) together with $C_{\rm box}(U_0)\le K_0$ and $C_{\rm box}(U_\xi)\le K_\xi$.
\end{proof}

\noindent\emph{Consequences.}
In the CR–Green certificate the field you pair is exactly
\(
U_0-U_\xi-\Re\log O,
\)
and its box energy is controlled by $K_\xi$ (sharp) and certainly by $K_0+K_\xi=\CboxZeta$ (coarse).
The aperture dependence is confined to $C(\psi)$, not to the box constant.
% --- end snippet ---

% (Removed global reduction: certificate constants are taken as Whitney-only suprema.)
% --- Atom-safe admissible test class and uniform CR–Green estimate ---

\begin{definition}[Admissible, atom-safe test class]\label{def:admissible-class}
Fix a Whitney interval \(I=[t_0-L,t_0+L]\) (with the standing aperture schedule)
and a smooth cutoff \(\chi_{L,t_0}\) supported in \(Q(\alpha'I)\), equal to \(1\) on \(Q(\alpha I)\), with
\(\|\nabla\chi_{L,t_0}\|_\infty\lesssim L^{-1}\), \(\|\nabla^2\chi_{L,t_0}\|_\infty\lesssim L^{-2}\).
Let \(V_\varphi:=P_\sigma*\varphi\) denote the Poisson extension of \(\varphi\).


We say that a collection \(\mathcal A=\mathcal A(I)\subset C_c^\infty(I)\) is \emph{admissible}
if each \(\varphi\in\mathcal A\) is nonnegative, \(\int_{\R}\varphi=1\), and there is a constant \(A_\ast<\infty\),
independent of \(L,t_0\) and of \(\varphi\in\mathcal A\), such that the (scale-invariant) Poisson test energy obeys
\begin{equation}\label{eq:Poisson-energy-bound}
  \iint_{Q(\alpha'I)} \Big(|\nabla V_\varphi|^2 + |\nabla\chi_{L,t_0}|^2\,|V_\varphi|^2\Big)\,\sigma\,dt\,d\sigma
  \ \le\ A_\ast
\end{equation}
We call \(\mathcal A\) \emph{atom-safe} on \(I\) if, whenever \(I\) contains critical-line atoms \(\{\gamma_j\}\) for \(-w'\),
there exists \(\varphi\in\mathcal A\) with \(\varphi(\gamma_j)=0\) for all such \(\gamma_j\).
\end{definition}


\begin{lemma}[Uniform CR--Green bound for the class \(\mathcal A\)]\label{lem:uniform-CRG-A}
Let \(J\) be analytic on \(\Omega\) with a.e.\ boundary modulus \(|J(\tfrac12+it)|=1\) and write \(\log J=U+iW\) with boundary phase \(w=W|_{\sigma=0}\).
Assume the Carleson box-energy bound for \(U\) on Whitney boxes:
\[
  \iint_{Q(\alpha I)} |\nabla U|^2\,\sigma\,dt\,d\sigma \ \le\ C_{\rm box}^{(\zeta)}\,|I|\ =\ 2L\,C_{\rm box}^{(\zeta)}.
\]
If \(\mathcal A=\mathcal A(I)\) is admissible in the sense of \eqref{eq:Poisson-energy-bound},
then there exists a constant \(C_{\rm rem}=C_{\rm rem}(\alpha)\) such that, uniformly in \(I\),
\begin{equation}\label{eq:supA-bound}
  \sup_{\varphi\in\mathcal A}\ \int_{\R} \varphi(t)\,(-w'(t))\,dt
  \ \le\ C_{\rm rem}\,\sqrt{A_\ast}\,\big(C_{\rm box}^{(\zeta)}\big)^{1/2}\,L^{1/2}
  \ \ :=:\ C_{\mathcal A}\,C_{\rm box}^{(\zeta)}{}^{1/2}\,L^{1/2}
\end{equation}
\end{lemma}


\begin{proof}
For each \(\varphi\in\mathcal A\), apply the CR--Green pairing on \(Q(\alpha'I)\) to \(U\) and \(\chi_{L,t_0}V_\varphi\):
\[
  \int_{\R}\varphi(t)\,(-w'(t))\,dt
  \ =\ \iint_{Q(\alpha'I)} \nabla U\cdot\nabla(\chi_{L,t_0}V_\varphi)\,dt\,d\sigma\ +\ \mathcal R_{\mathrm{side}}+\mathcal R_{\mathrm{top}},
\]
with remainders bounded by \(C_{\rm rem}(\alpha)\) times the product of the Dirichlet norms
(of \(\nabla U\) on \(Q(\alpha'I)\) and of the test field, cf.\ \eqref{eq:Poisson-energy-bound}).
By Cauchy--Schwarz and the Carleson bound for \(U\),
\[
  \int_{\R}\varphi(-w') \ \le\ C_{\rm rem}(\alpha)\,
  \Big(\iint_{Q(\alpha'I)} |\nabla U|^2\,\sigma\Big)^{\!1/2}
  \Big(\iint_{Q(\alpha'I)} (|\nabla V_\varphi|^2+|\nabla\chi|^2|V_\varphi|^2)\,\sigma\Big)^{\!1/2}.
\]
Insert the hypotheses to obtain
\(
\int \varphi(-w') \le C_{\rm rem}(\alpha)\,\sqrt{2L\,C_{\rm box}^{(\zeta)}}\ \sqrt{A_\ast},
\)
which is \eqref{eq:supA-bound} upon setting \(C_{\mathcal A}:=C_{\rm rem}(\alpha)\sqrt{2A_\ast}\) (and absorbing absolute factors).
\end{proof}


\begin{corollary}[Atom neutralization and clean Whitney scaling]\label{cor:atom-safe}
With the notation above, the phase--velocity identity yields, for every \(\varphi\in C_c^\infty(I)\),
\[
  \int_{\R}\varphi(t)\,(-w'(t))\,dt
  \ =\ \pi\!\int_{\R}\varphi\,d\mu\ +\ \pi\sum_{\gamma\in I} m_\gamma\,\varphi(\gamma),
\]
where \(\mu\) is the Poisson balayage measure (absolutely continuous) and the sum ranges over critical-line atoms.
If \(I\) contains atoms, pick \(\varphi\in\mathcal A(I)\) with \(\varphi(\gamma)=0\) at each such atom; then the atomic term vanishes and
\[
  \int_{\R}\varphi\,(-w')\ =\ \pi\!\int \varphi\,d\mu\ \le\ C_{\mathcal A}\,C_{\rm box}^{(\zeta)}{}^{1/2}\,L^{1/2}.
\]
Thus the \(\,L^{-1}\) plateau blow-up from atoms is removed, and the Whitney\-uniform \(L^{1/2}\) bound \eqref{eq:supA-bound}
holds verbatim in the atomic case as well.
\end{corollary}
\begin{proof}
This is immediate from the phase–velocity identity (Theorem~\ref{thm:phase-velocity-quant}) and the definition of an atom-safe admissible class: choosing \(\varphi\) to vanish at each critical-line atom kills the discrete sum. The remaining absolutely continuous term equals \(\pi\int \varphi\,d\mu\) and is controlled by the uniform CR--Green estimate \eqref{eq:supA-bound}.
\end{proof}


\begin{remark}[Local-to-global wedge]\label{rem:wedge-application}
The certificate produces a \emph{Whitney-local} phase-drop control of the form
\(\int_I(-w')\le \pi\,\Upsilon\) with \(\Upsilon<\tfrac12\) on every Whitney interval \(I\)
(Lemma~\ref{lem:whitney-uniform-wedge}), and more generally an admissible-class bound
\(\sup_{\varphi\in\mathcal A(I)}\int \varphi(-w')\lesssim L^{1/2}\) (Lemma~\ref{lem:uniform-CRG-A}).

\medskip
\noindent\textbf{Referee note (what is missing).}
As stated, the manuscript still needs an explicit, referee-checkable implication of the form
\[
  \Big(\forall\ \text{Whitney }I,\ \int_I(-w')\le \pi\,\Upsilon<\tfrac{\pi}{2}\Big)
  \quad\Longrightarrow\quad
  \exists\,m\in\R/2\pi\mathbb Z\ \text{s.t.}\ |\Arg \mathcal J(\tfrac12+it)-m|\le \tfrac{\pi}{2}\ \text{a.e.},
\]
i.e. a global a.e. boundary wedge \textup{(P+)} after a \emph{single} unimodular rotation.
This does \emph{not} follow from Whitney-local control alone without an additional hypothesis preventing
global phase drift (e.g. an “exponential inner factor at infinity”).

\medskip
\noindent\textbf{Counterexample (shows Whitney-local bounds alone do not force a global wedge).}
Let \(J(s):=\exp\!\big(-a(s-\tfrac12)\big)\) on \(\Omega\). Then \(|J(\tfrac12+it)|=1\) a.e., the boundary phase may be taken as
\(w(t)=-at\) so that \(-w'=a\,dt\) is a positive Radon measure, and for every Whitney interval \(I\) of length \(|I|\le 2L_\star\) one has
\(\int_I(-w')=a|I|\le 2aL_\star\).
Choosing \(a\le (\pi\Upsilon)/(2L_\star)\) forces \(\int_I(-w')\le \pi\Upsilon\) on \emph{every} Whitney interval with any fixed \(\Upsilon<\tfrac12\),
yet \(\Re(2J(\tfrac12+it))=2\cos(at)\) changes sign on sets of positive measure for every rotation, so \textup{(P+)} fails.
\end{remark}
\begin{corollary}[Unconditional local window constants]\label{cor:CH-Mpsi-final}
Define, for $I=[t_0-L,t_0+L]$ and $u$ the boundary trace of $U$, the mean-oscillation constant
\[
  M_\psi\ :=\ \sup_{L>0,\ t_0\in\R}\ \frac{1}{L}\,\Big|\int_{\R} (u(t)-u_I)\,\psi_{L,t_0}(t)\,dt\Big|,\qquad u_I:=\frac{1}{|I|}\int_I u,\quad \psi_{L,t_0}(t):=\psi\big((t-t_0)/L\big),
\]
and the Hilbert constant
\[
  C_H(\psi)\ :=\ \sup_{L>0,\ t_0\in\R}\ \frac{1}{L}\,\Big|\int_{\R} \mathcal H[u'](t)\,\psi_{L,t_0}(t)\,dt\Big|.
\]
Then there are constants $C_1(\psi),C_2(\psi)<\infty$ depending only on $\psi$ and the dilation parameter $\alpha$ such that
\[
  M_\psi\ \le\ C_1(\psi)\,\sqrt{C_{\rm box}^{(\mathrm{Whitney})}}\,\mathcal A(\psi),\qquad
  C_H(\psi)\ \le\ C_2(\psi)\,\sqrt{C_{\rm box}^{(\mathrm{Whitney})}}\,\mathcal A(\psi),
\]
where the fixed Poisson energy of the window is
\[
  \mathcal A(\psi)^2\ :=\ \iint_{\R^2_+}|\nabla(P_\sigma*\psi)|^2\,\sigma\,dt\,d\sigma\ <\ \infty.
\]
In particular, both constants are finite and determined by local box energies.
\end{corollary}
\begin{proof}
This is a bookkeeping corollary collecting the already-proved window bounds: the $H^1$--BMO/Carleson estimate for $M_\psi$ is Lemma~\ref{lem:Mpsi-correct}, and the uniform Hilbert pairing bound is Lemma~\ref{lem:hilbert-H1BMO}. The constants $C_1(\psi),C_2(\psi)$ absorb the fixed geometric Carleson embedding factor (Appendix~\ref{app:CE-constant}) and the fixed Poisson energy $\mathcal A(\psi)$.
\end{proof}
\begin{lemma}[Poisson–BMO bound at fixed height]\label{lem:poisson-bmo-strip}
Let $u\in \mathrm{BMO}(\mathbb R)$ and $U(\sigma,t):=(P_\sigma*u)(t)$ be its Poisson extension on $\Omega$. Then for every fixed $\sigma_0>0$,
\[
\sup_{t\in\mathbb R}|U(\sigma,t)|\ \le\ C_{\mathrm{BMO}}\,\|u\|_{\mathrm{BMO}}\qquad(\sigma\ge \sigma_0),
\]
with a finite constant $C_{\mathrm{BMO}}$ depending only on $\sigma_0$ and the fixed cone/box geometry. Consequently, if $\mathcal O$ is the outer with boundary modulus $e^u$, then for $\sigma\ge \sigma_0$ one has $e^{-C_{\mathrm{BMO}}\|u\|_{\mathrm{BMO}}}\le |\mathcal O(\sigma+it)|\le e^{C_{\mathrm{BMO}}\|u\|_{\mathrm{BMO}}}$.
\end{lemma}
\begin{proof}
Fix $\sigma\ge\sigma_0$. Write $U(\sigma,t)=\int_\R u(t-s)\,P_\sigma(s)\,ds$. Since $\int P_\sigma=1$ and $\int s\,P_\sigma(s)\,ds=0$, we may subtract the mean of $u$ on $I=[t-\sigma,t+\sigma]$ to get
\[
  U(\sigma,t)=u_I+\int_\R (u(t-s)-u_I)\,P_\sigma(s)\,ds.
\]
The second term is controlled by the BMO seminorm via the standard estimate (see, e.g., \cite[Ch.~IV]{SteinSingInt} or \cite[Ch.~IV]{Garnett})
\(\int |u(t-s)-u_I|\,P_\sigma(s)\,ds\lesssim \|u\|_{\mathrm{BMO}}\)
uniformly in $t$ for $\sigma\ge\sigma_0$ (use the dyadic annuli decomposition of $\R$ relative to $I$ and the doubling property of BMO averages). Absorbing constants depending only on $\sigma_0$ into $C_{\mathrm{BMO}}$ gives the stated bound. The outer modulus bounds follow by exponentiating $|U|\le C_{\mathrm{BMO}}\|u\|_{\mathrm{BMO}}$.
\end{proof}
\subsection*{Hilbert pairing via affine subtraction (uniform in $T,L$)}
% Archived duplicate block (not load-bearing in the active route)
% archived block removed
\begin{lemma}[Uniform Hilbert pairing bound (local box pairing)]\label{lem:hilbert-H1BMO}
Let $\psi\in C_c^\infty([-1,1])$ be even with $\int_\R\psi=1$ and define the mass--1 windows $\varphi_I(t)=L^{-1}\psi\big((t-T)/L\big)$. Then there exists $C_H(\psi)<\infty$ (independent of $T,L$) such that for $u$ from the smoothed Cauchy theorem,
\[
  \Big|\int_\R \mathcal H[u'](t)\,\varphi_I(t)\,dt\Big|\ \le\ C_H(\psi)\quad\text{for all intervals }I.
\]
\end{lemma}
% \fi
\begin{proof}
In distributions, $\langle \mathcal H[u'],\varphi_I\rangle=\langle u,(\mathcal H[\varphi_I])'\rangle$. Since $\psi$ is even, $(\mathcal H[\varphi_I])'$ annihilates affine functions; subtract the calibrant $\ell_I$ and write $v:=u-\ell_I$. Let $V$ be the Dirichlet test field for $(\mathcal H[\varphi_I])'$ supported in $Q(\alpha'I)$ with $\|\nabla V\|_{L^2(\sigma)}\asymp L^{-1/2}\,\mathcal A(\psi)$ (scale invariance for mass--1 windows). The local box pairing (Lemma~\ref{lem:cutoff-pairing}) gives
\[
  |\langle v,(\mathcal H[\varphi_I])'\rangle|\ \le\ \Big(\iint_{Q(\alpha'I)} |\nabla \widetilde U|^2\,\sigma\Big)^{1/2}\,\cdot\,\Big(\iint_{Q(\alpha'I)} |\nabla V|^2\,\sigma\Big)^{1/2}.
\]
Using the neutralized area bound $\iint_{Q(\alpha'I)} |\nabla \widetilde U|^2\,\sigma\lesssim |I|\asymp L$ (Lemma~\ref{lem:carleson-xi}) and the fixed test energy for $V$, we obtain
\[
  |\langle v,(\mathcal H[\varphi_I])'\rangle|\ \lesssim\ (L)^{1/2}\,(L^{-1/2}\,\mathcal A(\psi))\ =\ C(\psi)\,\mathcal A(\psi),
\]
uniformly in $(T,L)$. This proves the uniform bound with $C_H(\psi)\asymp \mathcal A(\psi)$.
\end{proof}
\begin{lemma}[Hilbert-transform pairing]\label{lem:hilbert}
There exists a window–dependent constant \(C_H(\psi)>0\) such that for every interval \(I\),
\[ \Big|\int_{\R} \mathcal H[u'](t)\,\varphi_I(t)\,dt\Big|\ \le\ C_H(\psi).\]
\end{lemma}
\begin{proof}
By Lemma~\ref{lem:hilbert-H1BMO}, for mass–1 windows and even \(\psi\), the pairing \(\langle \mathcal H[u'],\varphi_I\rangle\) is uniformly bounded in \((T,L)\). In distributions, \(\langle \mathcal H[u'],\varphi_I\rangle=\langle u,(\mathcal H[\varphi_I])'\rangle\); evenness implies \((\mathcal H[\varphi_I])'\) annihilates affine functions. Subtract the affine calibrant on \(I\) and write \(v=u-\ell_I\). The bound follows from the local box pairing in the Carleson energy lemma (Lemma~\ref{lem:carleson-xi}) applied to the test field associated with \((\mathcal H[\varphi_I])'\).
\end{proof}
% --- PSC route moved to archived appendix; placeholder removed from main chain ---
We adopt the \(\zeta\)-normalized boundary route with the half-plane Blaschke compensator \(B(s)=(s-1)/s\) to cancel the pole at \(s=1\). On \(\Re s=\tfrac12\), \(|B|=1\), so the compensator contributes no boundary phase and the Archimedean term vanishes. We print a concrete even $C^\infty$ flat--top window \(\psi\) below. For the finite-block certificate matrix we will use the scaled window
\[
  \psi_{\mathrm{cert}}(t)\ :=\ \tfrac1{12}\,\psi(t),
\]
so that the Fourier sup constant satisfies \(C_{\mathrm{win}}=\sup_{\xi}|\widehat{\psi_{\mathrm{cert}}}(\xi)|=\tfrac14\) (Lemma~\ref{lem:psi-cert-Cwin}). We also record the (optional) product certificate
\[
  \frac{(2/\pi)\,M_\psi}{c_0(\psi)}\ <\ \frac{\pi}{2}.
\]
\paragraph{Printed window.}
Let \(\beta(x):=\exp\!\big(-1/(x(1-x))\big)\) for \(x\in(0,1)\) and \(\beta=0\) otherwise. Define the smooth step
\[
  S(x):=\frac{\int_0^{\min\{\max\{x,0\},1\}} \beta(u)\,du}{\int_0^{1} \beta(u)\,du}\qquad (x\in\R),
\]
so that \(S\in C^\infty(\R)\), \(S\equiv 0\) on \(({-}\infty,0]\), \(S\equiv1\) on \([1,\infty)\), and \(S'\ge 0\) supported on \((0,1)\). Set the even flat-top window \(\psi:\R\to[0,1]\) by
\[
  \psi(t):=\begin{cases}
    0,& |t|\ge 2,\\
    S(t+2),& -2<t<-1,\\
    1,& |t|\le 1,\\
    S(2-t),& 1<t<2.
  \end{cases}
\]
Then \(\psi\in C_c^\infty(\R)\), \(\psi\equiv 1\) on \([-1,1]\), and \(\operatorname{supp}\psi\subset[-2,2]\). For windows we take \(\varphi_L(t):=L^{-1}\psi(t/L)\).

\begin{lemma}[Flat-top window: mass and Fourier sup bound for the scaled certificate window]\label{lem:psi-cert-Cwin}
Let $\psi$ be the printed flat--top window above and define $\psi_{\mathrm{cert}}:=\tfrac1{12}\psi$.
Define
\[
  \widehat{\psi_{\mathrm{cert}}}(\xi)\ :=\ \int_{\R}\psi_{\mathrm{cert}}(t)\,e^{-it\xi}\,dt,
  \qquad
  C_{\mathrm{win}}\ :=\ \sup_{\xi\in\R}\big|\widehat{\psi_{\mathrm{cert}}}(\xi)\big|.
\]
Then $\int_{\R}\psi(t)\,dt=3$, $\int_{\R}\psi_{\mathrm{cert}}(t)\,dt=\tfrac14$, and
\[
  C_{\mathrm{win}}\ =\ \int_{\R}\psi_{\mathrm{cert}}(t)\,dt\ =\ \frac14.
\]
\end{lemma}

\begin{proof}
Since $\beta(x)=\beta(1-x)$ on $(0,1)$, for $x\in[0,1]$ we have
\[
  \int_0^{1-x}\beta(u)\,du=\int_x^1 \beta(v)\,dv
\]
by the change of variables $v=1-u$. Dividing by $\int_0^1\beta$ gives $S(1-x)=1-S(x)$ on $[0,1]$, hence
\[
  \int_0^1 S(x)\,dx=\frac12\int_0^1\bigl(S(x)+S(1-x)\bigr)\,dx=\frac12.
\]
Therefore the two ramps of $\psi$ each have area $1/2$, so
\[
  \int_{\R}\psi(t)\,dt
  =2+\;2\!\int_1^2 S(2-t)\,dt
  =2+\;2\!\int_0^1 S(u)\,du
  =2+1=3.
\]
Scaling gives $\int\psi_{\mathrm{cert}}=\tfrac1{12}\int\psi=\tfrac14$.
For the Fourier bound, $\psi_{\mathrm{cert}}\ge 0$ implies for all $\xi$,
\[
  \big|\widehat{\psi_{\mathrm{cert}}}(\xi)\big|
  \le \int_{\R}\psi_{\mathrm{cert}}(t)\,|e^{-it\xi}|\,dt
  =\int_{\R}\psi_{\mathrm{cert}}(t)\,dt.
\]
At $\xi=0$ we have $\widehat{\psi_{\mathrm{cert}}}(0)=\int\psi_{\mathrm{cert}}$, hence $\sup_\xi|\widehat{\psi_{\mathrm{cert}}}(\xi)|=\int\psi_{\mathrm{cert}}=\tfrac14$.
\end{proof}

\paragraph{Poisson lower bound.}
\begin{lemma}[Poisson plateau lower bound]\label{lem:poisson-plateau}
For the printed even window \(\psi\) with \(\psi\equiv 1\) on \([-1,1]\),
\[ c_0(\psi)\ :=\ \inf_{0<b\le 1,\ |x|\le 1} (\Poisson_b*\psi)(x)\ \ge\ \frac{1}{2\pi}\,\arctan 2. \]
\end{lemma}
\begin{proof}
As in the plateau computation already recorded, for \(0<b\le 1\) and \(|x|\le 1\) one has
\[
 (\Poisson_b*\psi)(x)\ \ge\ (\Poisson_b*\mathbf 1_{[-1,1]})(x)
  = \frac{1}{2\pi}\Big(\arctan\tfrac{1-x}{b}+\arctan\tfrac{1+x}{b}\Big),
\]
whence
\[
 c_0(\psi)\ :=\ \inf_{0<b\le 1,\ |x|\le 1} (\Poisson_b*\psi)(x)\ \ge\ 0.1762081912\,.
\]
For the normalized Poisson kernel \(P_b(y)=\dfrac{1}{\pi}\dfrac{b}{b^2+y^2}\), for \(|x|\le 1\)
\[
 (P_b*\mathbf 1_{[-1,1]})(x)=\frac{1}{\pi}\int_{-1}^{1}\frac{b}{b^2+(x-y)^2}\,dy=\frac{1}{2\pi}\Big(\arctan\frac{1-x}{b}+\arctan\frac{1+x}{b}\Big).
\]
Set \(S(x,b):=\arctan\big((1-x)/b\big)+\arctan\big((1+x)/b\big)\). Symmetry gives \(S(-x,b)=S(x,b)\). For \(x\in[0,1]\),
\[
 \partial_x S(x,b)=\frac{1}{b}\Big(\frac{1}{1+\big(\tfrac{1+x}{b}\big)^2}-\frac{1}{1+\big(\tfrac{1-x}{b}\big)^2}\Big)\le 0,
\]
so \(S\) decreases in \(x\) and is minimized at \(x=1\). Also \(\partial_b S(x,b)\le 0\) for \(b>0\), so the minimum in \(b\in(0,1]\) is at \(b=1\). Thus the infimum occurs at \((x,b)=(1,1)\) giving \(\frac{1}{2\pi}\arctan 2=0.1762081912\ldots\). Since \(\psi\ge \mathbf 1_{[-1,1]}\), this yields the bound for \(\psi\).
\end{proof}
\paragraph{No Archimedean term in the \(\zeta\)-normalized route.}
Writing \(J_\zeta:=\dettwo(I-A)/\zeta\) and \(J_{\mathrm{comp}}:=J_\zeta\,B\), one has \(|B|=1\) on the boundary and no Gamma factor in \(J_\zeta\). Hence the boundary phase contribution from Archimedean factors is identically zero in the phase–velocity identity, i.e. \(C_\Gamma\equiv 0\) for this normalization.

% (bridge AAB archived)
We carry out the boundary phase test in the $\zeta$–normalized gauge with the Blaschke compensator at $s=1$; on $\Re s=\tfrac12$ one has $|B|=1$, so the Archimedean boundary contribution vanishes. Any residual interior effect is absorbed into the $\zeta$–side box constant $C_{\mathrm{box}}^{(\zeta)}$. In the a.e. wedge route no additive wedge constants are used.

\paragraph{Hilbert term (structural bound).}
For the mass--1 window and even \(\psi\), the local box pairing bound of Lemma~\ref{lem:hilbert-H1BMO} applies and is uniform in \((T,L)\). We write the certificate in terms of the abstract window-dependent constant \(C_H(\psi)\) from Lemma~\ref{lem:hilbert-H1BMO}. An explicit envelope for the printed window is recorded below, but is not required for the symbolic certificate.
\begin{lemma}[Explicit envelope for the printed window]\label{lem:CH-explicit}
For the flat-top \(\psi\) above with symmetric monotone ramps of width \(\varepsilon\in(0,1)\) on each side of \(\pm1\), one has the variation bound
\[
  \sup_{t\in\R}\,|\mathcal H[\varphi_L](t)|\ \le\ \frac{\mathrm{TV}(\psi)}{\pi}\,\log\frac{1+\varepsilon}{1-\varepsilon},\qquad \mathrm{TV}(\psi)=2.
\]
In particular, with \(\varepsilon=\tfrac15\) one obtains the certified envelope
\[
  \sup_{t\in\R}\,|\mathcal H[\varphi_L](t)|\ \le\ \frac{2}{\pi}\,\log\tfrac{3}{2}\ \approx\ 0.258\ <\ 0.26.
\]
Consequently, we may take \(C_H(\psi)\le 0.26\) for the printed window. This bound is uniform in \(L\).
\end{lemma}
\begin{proof}
Write \(\psi=\mathbf 1_{[-1,1]}+\eta\) with \(\eta\) supported on the disjoint transition layers \([1,1+\varepsilon]\) and \([-1-\varepsilon,-1]\), monotone on each layer, and total variation \(\mathrm{TV}(\psi)=2\). Using the identity
\[
\mathcal H[\psi](x)=\frac{1}{\pi}\,\mathrm{p.v.}\int \frac{\psi(y)}{x-y}\,dy=\frac{1}{\pi}\int \psi'(y)\,\log|x-y|\,dy
\]
(integration by parts; boundary cancellations by monotonicity/symmetry) and that \(\psi'\) is a finite signed measure of total variation \(\mathrm{TV}(\psi)\), one gets
\[
  |\mathcal H\psi(x)|\ \le\ \frac{\mathrm{TV}(\psi)}{\pi}\,\sup_{y\in[-1-\varepsilon,\,1+\varepsilon]}\big|\log|x-y|\big|\ -\ \inf_{y\in[-1-\varepsilon,\,1+\varepsilon]}\big|\log|x-y|\big|.
\]
The worst case is at \(x=0\), yielding \(|\mathcal H\psi(0)|\le \tfrac{\mathrm{TV}(\psi)}{\pi}\log\tfrac{1+\varepsilon}{1-\varepsilon}\). Scaling gives \(\mathcal H[\varphi_L](t)=\mathcal H\psi\big((t-T)/L\big)\), so the same bound holds uniformly in \(L\). Taking \(\varepsilon=\tfrac15\) gives the stated numeric envelope.
\end{proof}
\begin{lemma}[Derivative envelope: $C_H(\psi)\le 2/\pi$]\label{lem:CH-derivative-2pi}
For the printed flat–top window \(\psi\) (even, plateau on $[-1,1]$), with \(\varphi_L(t)=L^{-1}\psi((t-T)/L)\) one has
\[ \sup_{t\in\R}\,|\mathcal H[\varphi_L](t)|\ \le\ \frac{2}{\pi}\,\log\frac{1+\varepsilon}{1-\varepsilon}\quad\text{and}\quad \big\|\big(\mathcal H[\varphi_L]\big)'\big\|_{L^\infty(\R)}\ \le\ \frac{2}{\pi}\,\frac{1}{L}. \]
In particular, $C_H(\psi)\le 2/\pi$.
\end{lemma}
\begin{proof}
By scaling, \(\mathcal H[\varphi_L](t)=\mathcal H\psi((t-T)/L)\) and \(\big(\mathcal H[\varphi_L]\big)'(t)=\tfrac{1}{L}\,(\mathcal H\psi)'((t-T)/L)\). Since \(\psi'\equiv 0\) on \((-1,1)\) and the ramps are monotone on \([-1-\varepsilon,-1]\) and \([1,1+\varepsilon]\) with total variation \(2\), the variation/IBP argument of Lemma~\ref{lem:CH-explicit} yields the stated envelope and its derivative bound. Taking the supremum in \(t\) gives the \(2/\pi\) constant uniformly in \(L\).
\end{proof}
\paragraph{Window mean-oscillation constant \(M_\psi\): definition and bound.}
For an interval \(I=[T{-}L,T{+}L]\) and the boundary modulus \(u(t):=\log\big|\dettwo(I{-}A(\tfrac12{+}it))\big|{-}\log\big|\xi(\tfrac12{+}it)\big|\), define the mean-oscillation calibrant \(\ell_I\) as the affine function matching \(u\) at the endpoints of \(I\), and set
\[
  M_\psi\ :=\ \sup_{T\in\R,\ L>0}\ \frac{1}{|I|}\int_I \big|u(t)-\ell_I(t)\big|\,dt.
\]
By the smoothed Cauchy theorem and the local pairing in a local pairing bound, one obtains a window-dependent constant bounding the mean oscillation uniformly over $(T,L)$. For the printed flat-top window, Lemma~\ref{lem:Mpsi-correct} yields an explicit H$^1$--BMO/box-energy bound for $M_\psi$; in our calibration (see Numeric instantiation below), this gives a strict numerical bound well below the certificate threshold.
\begin{lemma}[Window mean--oscillation via H$^1$--BMO and box energy]\label{lem:Mpsi-correct}
Let $U$ be the Poisson extension of the boundary function $u$, and let $\lambda := |\nabla U|^2\,\sigma\,dt\,d\sigma$.
Fix the even $C^\infty$ window $\psi$ (support $\subset[-2,2]$, plateau on $[-1,1]$), and let $m_\psi:=\int_{\R}\psi(x)\,dx$ denote its mass. Set
\[
\phi(t):=\psi(t)-\tfrac{m_\psi}{2}\,\mathbf 1_{[-1,1]}(t),\qquad 
\phi_{L,t_0}(t):=\phi\!\Big(\frac{t-t_0}{L}\Big).
\]
Define $M_\psi:=\sup_{L>0,t_0\in\R}\frac1L\big|\int_\R u(t)\,\phi_{L,t_0}(t)\,dt\big|$ and
\[
 C_{\rm box}^{(\mathrm{Whitney})}:=\sup_{I\,:\,|I|\asymp c/\log\langle T\rangle}\frac{\lambda(Q(\alpha I))}{|I|},\qquad
C_\psi^{(H^1)}:=\frac12\int_{\R} S\phi(x)\,dx,
\]
where $S$ is the Lusin area function for the Poisson semigroup with cone aperture $\alpha$.
Then
\[
M_\psi\ \le\ \frac{4}{\pi}\,C_{\mathrm{CE}}(\alpha)\,C_\psi^{(H^1)}\,\sqrt{C_{\rm box}^{(\mathrm{Whitney})}}.
\]
\end{lemma}
\begin{proof}
By H$^1$--BMO duality, for every $I=[t_0-L,t_0+L]$,
\[ \Big|\int u\,\phi_{L,t_0}\Big|\ \le\ \|u\|_{\rm BMO}\,\|\phi_{L,t_0}\|_{H^1}. \]
Carleson embedding (aperture $\alpha$) gives
\[ \|u\|_{\rm BMO}\ \le\ \tfrac{2}{\pi}\,C_{\mathrm{CE}}(\alpha)\,\big(C_{\rm box}^{(\mathrm{Whitney})}\big)^{1/2}. \]
Since $S$ is scale-invariant in $L^1$ (up to $|I|$),
\[ \|\phi_{L,t_0}\|_{H^1}\ =\ \int S(\phi_{L,t_0})(x)\,dx\ =\ 2L\,C_\psi^{(H^1)}. \]
Divide by $L$ to conclude.
\end{proof}
\paragraph{Carleson box linkage.}
With $U=U_{\det_2}+U_{\xi}$ on the boundary in the $\zeta$–normalized route, the box constant used in the certificate is
\[
  C_{\mathrm{box}}^{(\zeta)}\ :=\ K_0\ +\ K_\xi.
\]
No separate $\Gamma$–area term enters the certificate path.

% shownumerics gated section (disabled)
\paragraph{Numeric instantiation (diagnostic; gated).}
All concrete values (audited constants for $K_0$, $K_\xi$, the $\zeta$–side box constant $C_{\mathrm{box}}^{(\zeta)}$, the evaluation of $C_\psi^{(H^1)}$, and the locked $M_\psi$) are collected for reproducibility; the proof of (P+) uses only the CR–Green right-hand side with the box constant.
\begin{itemize}
  \item \textbf{Window:} fixed $C^\infty$ even $\psi$ with $\psi\equiv 1$ on $[-1,1]$ and $\mathrm{supp}\,\psi\subseteq[-2,2]$, and $\varphi_L(t)=L^{-1}\psi(t/L)$.
  \item \textbf{Poisson lower bound.} Using the closed form for the plateau and monotonicity, $c_0(\psi)\ge 0.1762081912$.
  \item \textbf{Archimedean term.} In the $\zeta$-normalized route with the Blaschke compensator at $s=1$, $C_\Gamma=0$.
  \item \textbf{Hilbert term.} We retain $C_H(\psi)$ symbolically; an explicit envelope can be inserted.
  \item \textbf{Inequality form.} With $M_\psi= (4/\pi)\,C_\psi^{(H^1)}\,\sqrt{C_{\mathrm{box}}^{(\zeta)}}$, the display $\frac{(2/\pi)\,M_\psi}{c_0(\psi)}<\frac{\pi}{2}$ is diagnostic.
\end{itemize}
 
\subsection*{Explicit proofs and constants for key lemmas (archimedean, prime-tail, Hilbert)}
We record complete proofs with explicit constants, making finiteness and dependence on the window $\psi$ transparent.
% Duplicate prime-tail subsection removed (see earlier \S{subsec:prime-tail})
\begin{equation}\label{eq:P1}
 \sum_{p>x} p^{-\alpha}\ \le\ \frac{1.25506\,\alpha}{(\alpha-1)\,\log x}\,x^{\,1-\alpha}
\end{equation}
This follows by partial summation together with $\pi(t)\le 1.25506\,t/\log t$ for $t\ge 17$. A uniform variant over $\alpha\in[\alpha_0,2]$ (with $\alpha_0:=2\sigma_0>1$) is
\begin{equation}\label{eq:P1uniform}
 \sum_{p>x} p^{-\alpha}\ \le\ \frac{1.25506\,\alpha_0}{(\alpha_0-1)\,\log x}\,x^{\,1-\alpha_0}\qquad(x\ge 17)
\end{equation}
Two convenient alternatives:
\begin{align}
 \sum_{p>x}p^{-\alpha}&\ \le\ \frac{\alpha}{(\alpha-1)(\log x-1)}\,x^{1-\alpha}\qquad(x\ge 599)\label{eq:P1dusart}\\
 \sum_{p>x}p^{-\alpha}&\ \le\ \sum_{n>\lfloor x\rfloor}n^{-\alpha}\ \le\ \frac{x^{1-\alpha}}{\alpha-1}\qquad(x>1).\label{eq:P1triv}
\end{align}
\begin{proof}[Proof of \eqref{eq:P1}--\eqref{eq:P1triv}]
Fix $\alpha>1$ and $x\ge 17$. For $u>1$ write $f(u):=u^{-\alpha}$. By Stieltjes integration with $d\pi(u)$ and one integration by parts,
\[
\sum_{p\le y} p^{-\alpha}
=\int_{2^-}^{y} u^{-\alpha}\,d\pi(u)
= y^{-\alpha}\pi(y)+\alpha\!\int_{2}^{y} \pi(u)\,u^{-\alpha-1}\,du.
\]
Letting $y\to\infty$ and using $\alpha>1$ (so $y^{-\alpha}\pi(y)\to 0$) gives the exact tail identity
\begin{equation}\label{eq:P1-exact}
\sum_{p>x} p^{-\alpha}
=\alpha\!\int_{x}^{\infty}\!\pi(u)\,u^{-\alpha-1}\,du\;-\;x^{-\alpha}\pi(x)
\ \le\ \alpha\!\int_{x}^{\infty}\!\pi(u)\,u^{-\alpha-1}\,du
\end{equation}
For $u\ge x\ge 17$ we have the explicit bound $\pi(u)\le 1.25506\,\dfrac{u}{\log u}$. Inserting this into \eqref{eq:P1-exact} and using $1/\log u\le 1/\log x$ for $u\ge x$ yields
\[
\sum_{p>x} p^{-\alpha}
\ \le\ \frac{1.25506\,\alpha}{\log x}\!\int_{x}^{\infty}\!u^{-\alpha}\,du
\ =\ \frac{1.25506\,\alpha}{(\alpha-1)\,\log x}\,x^{\,1-\alpha},
\]
which is \eqref{eq:P1}. For the uniform version, if $\alpha\in[\alpha_0,2]$ with $\alpha_0>1$, then the map $\alpha\mapsto \alpha/(\alpha-1)$ is decreasing and $x^{1-\alpha}\le x^{1-\alpha_0}$, so \eqref{eq:P1uniform} follows immediately from \eqref{eq:P1}.

For \eqref{eq:P1dusart}, assume $x\ge 599$ and use the sharper pointwise bound $\pi(u)\le \dfrac{u}{\log u-1}$ for $u\ge x$. Then
\[
\sum_{p>x} p^{-\alpha}
\ \le\ \alpha\!\int_{x}^{\infty}\!\frac{u^{-\alpha}}{\log u-1}\,du
\ \le\ \frac{\alpha}{\log x-1}\!\int_{x}^{\infty}\!u^{-\alpha}\,du
\ =\ \frac{\alpha}{(\alpha-1)(\log x-1)}\,x^{1-\alpha}.
\]

Finally, \eqref{eq:P1triv} is the integer-majorant: $\sum_{p>x}p^{-\alpha}\le \sum_{n>\lfloor x\rfloor}n^{-\alpha}=\dfrac{x^{1-\alpha}}{\alpha-1}$ for $x>1$.
\end{proof}

\begin{lemma}[Monotonicity of the tail majorant]\label{lem:P1-monotone}
For fixed $\alpha>1$, the function $g(P):=\dfrac{P^{\,1-\alpha}}{\log P}$ is strictly decreasing on $P>1$.
\end{lemma}
\begin{proof}
Writing $\log g(P)=(1-\alpha)\log P-\log\log P$ gives
$(\log g)'=\dfrac{1-\alpha}{P}-\dfrac{1}{P\log P}<0$ for $P>1$.
\end{proof}

\begin{corollary}[Minimal tail parameter for a target $\eta$]\label{cor:P1-minP}
Given $\alpha>1$, $x_0\ge 17$ and target $\eta>0$, define $P_\eta$ to be the smallest integer $P\ge x_0$ such that
\[
\frac{1.25506\,\alpha}{(\alpha-1)\,\log P}\,P^{1-\alpha}\ \le\ \eta.
\]
By Lemma~\ref{lem:P1-monotone} this $P_\eta$ exists and is unique; moreover, the inequality then holds for every $P\ge P_\eta$. (The same definition with $\log P$ replaced by $\log P-1$ gives the $x_0\ge 599$ Dusart variant.)
\end{corollary}
\begin{proof}
The left-hand side equals a positive constant times $g(P)=P^{1-\alpha}/\log P$. By Lemma~\ref{lem:P1-monotone}, $g$ is strictly decreasing on $P>1$, hence the inequality threshold defines a unique minimal integer $P_\eta\ge x_0$ and persists for all larger $P$.
\end{proof}
\paragraph{Use in $(\star)$ and covering.}
To enforce a tail $\sum_{p>P}p^{-\alpha}\le \eta$ it suffices, by \eqref{eq:P1}, to take $P\ge17$ solving
\[
 \frac{1.25506\,\alpha}{(\alpha-1)\,\log P}\,P^{\,1-\alpha}\ \le\ \eta.
\]
The practical choice $P=\max\{17,\ ((1.25506\,\alpha)/((\alpha-1)\eta))^{1/(\alpha-1)}\}$ already meets the inequality up to the mild $\log P$ factor; one may increase $P$ monotonically until the left side is $\le\eta$.
\subsection*{Finite-block spectral gap certificate on $[\sigma_0,1]$}
We make explicit the finite-block matrix $H(\sigma)$ used in the spectral-gap/passivity certificate.

\begin{definition}[Finite-block passivity/Pick matrix]\label{def:finite-block-passivity-matrix}
Fix a prime cut $P$ and per-prime truncation lengths $N_p\ge 1$. Let
\[
  \mathcal I\ :=\ \{(p,n):\ p\le P\ \text{prime},\ 1\le n\le N_p\}.
\]
Fix nonnegative weights $(w_n)_{n\ge 1}$ with
\[
  \sum_{n\ge 1} w_n\ =\ \frac12
  \qquad\text{(e.g.\ Lemma~\ref{lem:weights-geometric}).}
\]
Let $\psi_{\mathrm{cert}}:=\tfrac1{12}\psi$ be the scaled certificate window from Lemma~\ref{lem:psi-cert-Cwin}, and define its Fourier transform by
\[
  \widehat{\psi_{\mathrm{cert}}}(\xi)\ :=\ \int_{\R}\psi_{\mathrm{cert}}(t)\,e^{-it\xi}\,dt,\qquad
  C_{\mathrm{win}}\ :=\ \sup_{\xi\in\R}\big|\widehat{\psi_{\mathrm{cert}}}(\xi)\big|.
\]
For $\sigma\in[\sigma_0,1]$, define a Hermitian matrix $H(\sigma)\in\C^{|\mathcal I|\times|\mathcal I|}$ by the entry formula
\[
  H_{(p,n),(q,m)}(\sigma)\;:=\;\delta_{pq}\,\delta_{nm}\;-\;w_n w_m\,
  p^{-(\sigma+\tfrac12)}\,q^{-(\sigma+\tfrac12)}\,
  \widehat{\psi_{\mathrm{cert}}}\!\big(n\log p-m\log q\big),
  \qquad (p,n),(q,m)\in\mathcal I.
\]
We view $H(\sigma)$ as a block matrix $H(\sigma)=[H_{pq}(\sigma)]_{p,q\le P}$ with $H_{pq}(\sigma)\in\C^{N_p\times N_q}$.
Write $D_p(\sigma):=H_{pp}(\sigma)$ and $E(\sigma):=H(\sigma)-\mathrm{diag}(D_p(\sigma))$.
\end{definition}

\begin{lemma}[A concrete weight sequence]\label{lem:weights-geometric}
Define, for $n\ge 1$,
\[
  w_n:=\frac1{19}\left(\frac{17}{19}\right)^{n-1}.
\]
Then $w_n\ge 0$, $\sum_{n\ge 1} w_n=\frac12$, and
\[
  \sum_{n\ge 1} w_n^2=\frac1{72}.
\]
Consequently, for any truncation length $N\in\N$,
\[
  \sum_{n=1}^N w_n\le \frac12,\qquad \sum_{n=1}^N w_n^2\le \frac1{72}.
\]
\end{lemma}

\begin{proof}
Both series are geometric. First,
\[
  \sum_{n\ge 1} w_n=\frac1{19}\sum_{n\ge 0}\left(\frac{17}{19}\right)^n
  =\frac1{19}\cdot \frac{1}{1-\frac{17}{19}}=\frac1{19}\cdot\frac{19}{2}=\frac12.
\]
Second,
\[
  \sum_{n\ge 1} w_n^2=\frac1{361}\sum_{n\ge 0}\left(\frac{289}{361}\right)^n
  =\frac1{361}\cdot \frac{1}{1-\frac{289}{361}}
  =\frac1{361}\cdot \frac{361}{72}=\frac1{72}.
\]
Truncation only decreases the sums.
\end{proof}

\begin{lemma}[Off-diagonal enclosure from the explicit formula]\label{lem:offdiag-enclosure}
For $p\neq q$, uniformly for $\sigma\in[\sigma_0,1]$,
\[
  \|H_{pq}(\sigma)\|_2\ \le\ \frac{C_{\mathrm{win}}}{4}\,p^{-(\sigma+\tfrac12)}\,q^{-(\sigma+\tfrac12)}.
\]
\end{lemma}
\begin{proof}
Fix $\sigma\in[\sigma_0,1]$ and primes $p\neq q$. Let $x\in\C^{N_p}$ and $y\in\C^{N_q}$ be unit vectors.
Using $|\widehat{\psi_{\mathrm{cert}}}|\le C_{\mathrm{win}}$,
\[
  |x^*H_{pq}(\sigma)y|
  \le C_{\mathrm{win}}\,p^{-(\sigma+\tfrac12)}q^{-(\sigma+\tfrac12)}
     \sum_{n\le N_p}\sum_{m\le N_q} w_n w_m\,|x_n|\,|y_m|.
\]
Factor the double sum and apply Cauchy--Schwarz:
\[
  \sum_{n\le N_p}\sum_{m\le N_q} w_n w_m\,|x_n|\,|y_m|
  =\Bigl(\sum_{n\le N_p} w_n|x_n|\Bigr)\Bigl(\sum_{m\le N_q} w_m|y_m|\Bigr)
  \le\Bigl(\sum_{n\le N_p} w_n\Bigr)\Bigl(\sum_{m\le N_q} w_m\Bigr)
  \le \frac14,
\]
since $\sum_{n\ge 1}w_n=\tfrac12$ and the truncations only decrease the sum.
Therefore
\[
  |x^*H_{pq}(\sigma)y|\ \le\ \frac{C_{\mathrm{win}}}{4}\,p^{-(\sigma+\tfrac12)}\,q^{-(\sigma+\tfrac12)}.
\]
Taking the supremum over $\|x\|_2=\|y\|_2=1$ yields the claimed operator-norm bound.
\end{proof}
\begin{lemma}[Block Gershgorin lower bound]\label{lem:block-gersh}
For every $\sigma\in[\sigma_0,1]$,
\[
  \lambda_{\min}\big(H(\sigma)\big)\ \ge\ \min_{p\le P}\Big(\lambda_{\min}\big(D_p(\sigma)\big)\ -\ \sum_{q\ne p}\|H_{pq}(\sigma)\|_2\Big).
\]
\end{lemma}
\begin{proof}
Fix $\sigma\in[\sigma_0,1]$ and write a vector $x\in\C^{|\mathcal I|}$ in blocks $x=(x_p)_{p\le P}$ with $x_p\in\C^{N_p}$. Since $H(\sigma)$ is Hermitian,
\[
  \langle Hx,x\rangle
  =\sum_{p}\langle D_p x_p,x_p\rangle\ +\ \sum_{p\neq q}\Re\langle H_{pq}x_q,x_p\rangle.
\]
For $p\neq q$, $|\langle H_{pq}x_q,x_p\rangle|\le \|H_{pq}\|_2\,\|x_p\|\,\|x_q\|$, and $2ab\le a^2+b^2$ gives
\[
  2\,\|H_{pq}\|_2\,\|x_p\|\,\|x_q\|\ \le\ \|H_{pq}\|_2\big(\|x_p\|^2+\|x_q\|^2\big).
\]
Summing over $p\neq q$ yields
\[
  \langle Hx,x\rangle\ \ge\ \sum_p \Big(\lambda_{\min}(D_p)-\sum_{q\neq p}\|H_{pq}\|_2\Big)\|x_p\|^2
  \ \ge\ \Big(\min_{p}\Big(\lambda_{\min}(D_p)-\sum_{q\neq p}\|H_{pq}\|_2\Big)\Big)\|x\|^2.
\]
Taking the infimum of the Rayleigh quotient $\langle Hx,x\rangle/\|x\|^2$ over $x\neq0$ gives the stated lower bound for $\lambda_{\min}(H(\sigma))$.
\end{proof}
\begin{lemma}[Schur--Weyl bound]\label{lem:schur-weyl-gap}
For every $\sigma\in[\sigma_0,1]$,
\[
  \lambda_{\min}\big(H(\sigma)\big)\ \ge\ \delta(\sigma_0),\qquad
  \delta(\sigma_0):=\max\big\{\delta_{\mathrm{Gersh}}(\sigma_0),\,\delta_{\mathrm{Schur}}(\sigma_0)\big\},
\]
where
\[
  \delta_{\mathrm{Gersh}}(\sigma_0):=\min_p\Big(\mu_p^L-\sum_{q\ne p}U_{pq}\Big),\qquad
  \delta_{\mathrm{Schur}}(\sigma_0):=\min_p \mu_p^L\ -\ \max_q\frac{1}{\sqrt{\mu_q^L}}\sum_{p\ne q}\sqrt{\mu_p^L}\,U_{pq}.
\]
In particular, if $\delta(\sigma_0)\ge 0$ then $\lambda_{\min}(H(\sigma))\ge 0$ for all $\sigma\in[\sigma_0,1]$.
\end{lemma}
\begin{proof}
This is a standard block Schur-complement/Weyl-type lower bound: after normalizing each diagonal block by its lower spectral bound $\mu_p^L$, the off-diagonal operator norms are bounded by the budgets $U_{pq}$. The first term in the maximum is the direct block Gershgorin bound (Lemma~\ref{lem:block-gersh}). The second term comes from a weighted Schur test: for a unit vector $x=(x_p)$, bound $\sum_{p\neq q}\Re\langle H_{pq}x_q,x_p\rangle$ by Cauchy–Schwarz with weights $\sqrt{\mu_p^L}$ and use $\|H_{pq}\|_2\le U_{pq}$ to obtain
\[
  \langle Hx,x\rangle \ \ge\ \min_p \mu_p^L\ -\ \max_q\frac{1}{\sqrt{\mu_q^L}}\sum_{p\ne q}\sqrt{\mu_p^L}\,U_{pq}.
\]
Taking the maximum of the two lower bounds yields the stated $\delta(\sigma_0)$. The final implication is immediate.
\end{proof}
\subsection*{Determinant--zeta link (L1; corrected domain)}

\begin{remark}[Using prime-tail bounds]
If $\|H_{pq}(\sigma)\|_2\le C(\sigma_0)(pq)^{-\sigma_0}$ for $p\ne q$, then $\sum_{q\ne p}U_{pq}\le C(\sigma_0)\,p^{-\sigma_0}\sum_{q\le P} q^{-\sigma_0}$, and the sum is bounded explicitly by the Rosser--Schoenfeld tail with $\alpha=2\sigma_0>1$. Thus $\delta(\sigma_0)>0$ can be certified by choosing $P,\{N_p\}$ so that the off-diagonal budget is dominated by $\min_p\mu_p^L$.
\end{remark}

\begin{proposition}[Concrete certified spectral gap at $\sigma_0=0.6$]\label{prop:delta-cert-06}
Fix $\sigma_0=0.6$, take $Q=29$ and $p_{\min}:=\mathrm{nextprime}(Q)=31$, and set $\sigma^\star:=\sigma_0+\tfrac12=1.1$.
Assume the uniform off--diagonal enclosure (for all $p\neq q$, uniformly in $\sigma\in[\sigma_0,1]$)
\[
\|H_{pq}(\sigma)\|_2 \ \le\  \frac{C_{\mathrm{win}}}{4}\, p^{-(\sigma+\tfrac12)}\, q^{-(\sigma+\tfrac12)},
\qquad C_{\mathrm{win}}=0.25,
\]
together with the diagonal lower bound
\[
\mu_p^{\mathrm L}\ \ge\ 1-\frac{(1-\sigma_0)(\log p)\,p^{-\sigma_0}}{6}.
\]
Then $\lambda_{\min}(H(\sigma))\ge 0.72$ for all $\sigma\in[\sigma_0,1]$.
\end{proposition}
\begin{proof}
A direct evaluation over primes $p\le Q$ gives
\[
\sum_{p\le 29} p^{-1.1}=1.3239981250,\qquad \sum_{\substack{p\le 29\\ p\neq 2}} p^{-1.1}=0.8574816292.
\]
The integer--tail majorant
\[
\sum_{n\ge p_{\min}-1} n^{-1.1}\ \le\ \frac{(p_{\min}-1)^{1-1.1}}{1.1-1}=7.1168510179
\]
then implies the four row--sum budgets (small/far blocks, $2$ singled out)
\[
\Delta_{\mathrm{FS}}= \frac{0.25}{4}\,31^{-1.1}\!\!\sum_{p\le 29}p^{-1.1}=0.0018935184,\qquad
\Delta_{\mathrm{FF}}\le \frac{0.25}{4}\,31^{-1.1}\!\!\sum_{n\ge 30}n^{-1.1}=0.0101781777,
\]
\[
\Delta_{\mathrm{SS}}=\frac{0.25}{4}\,2^{-1.1}\!\!\sum_{\substack{p\le 29\\ p\neq 2}}p^{-1.1}=0.0250018328,\qquad
\Delta_{\mathrm{SF}}\le\frac{0.25}{4}\,2^{-1.1}\!\!\sum_{n\ge 30}n^{-1.1}=0.2075080249.
\]
For the diagonal blocks, the bound $\mu_p^{\mathrm L}\ge 1-\tfrac16(1-\sigma_0)(\log p)p^{-\sigma_0}$ gives
\[
\mu_{\min}^{\mathrm{far}}\ge 1-\frac{(1-\sigma_0)(\log 31)\,31^{-0.6}}{6}=0.9708330916,\qquad
\mu_{\min}^{\mathrm{small}}\ge 1-\frac{(1-\sigma_0)(\log 5)\,5^{-0.6}}{6}=0.9591491624.
\]
Thus every row in the small block satisfies
\[
\mu_{\min}^{\mathrm{small}}-(\Delta_{\mathrm{SS}}+\Delta_{\mathrm{SF}})=0.7266393047>0.72,
\]
and every far--block row satisfies
\[
\mu_{\min}^{\mathrm{far}}-(\Delta_{\mathrm{FS}}+\Delta_{\mathrm{FF}})=0.9587613956>0.72.
\]
Taking the minimum of these two certified bounds gives $\lambda_{\min}(H(\sigma))\ge 0.72$ uniformly for $\sigma\in[\sigma_0,1]$.
\end{proof}

\subsection*{Truncation tail control and global assembly (P4)}
Write the head/tail split by primes as $\mathcal P_{\le P}=\{p\le P\}$ and $\mathcal P_{>P}=\{p>P\}$. In the normalised basis at $\sigma_0$ set
\[ X:=\bigl[\widetilde H_{pq}\bigr]_{p,q\le P},\quad Y:=\bigl[\widetilde H_{pq}\bigr]_{p\le P<q},\quad Z:=\bigl[\widetilde H_{pq}\bigr]_{p,q>P}. \]
Let $A_p^2:=\sum_{i\le N_p} w_i^2$ denote the block weight squares (unweighted: $A_p^2=N_p$; the weighted example in Lemma~\ref{lem:weights-geometric} gives $A_p^2\le\tfrac1{72}$). Define
\[ S_2(\le P):=\sum_{p\le P} A_p^2 p^{-2\sigma_0},\qquad S_2(>P):=\sum_{p>P} A_p^2 p^{-2\sigma_0}. \]
Then
\[ \|Y\|\le C_{\rm win}\sqrt{S_2(\le P)S_2(>P)},\qquad \lambda_{\min}(Z)\ge \mu_{\mathrm{diag}}-C_{\rm win}S_2(>P), \]
where $\mu_{\mathrm{diag}}:=\inf_{p>P}\mu_p^{\mathrm L}$. Consequently,
\[ \lambda_{\min}(\mathbb A)\ge \min\Big\{\,\delta_P - \dfrac{C_{\rm win}^2 S_2(\le P)S_2(>P)}{\mu_{\mathrm{diag}}-C_{\rm win}S_2(>P)}\,,\ \mu_{\mathrm{diag}}-C_{\rm win}S_2(>P)\Big\}, \]
with $\delta_P$ the head finite-block gap from above. Using the integer tail $\sum_{n>P}n^{-2\sigma_0}\le (P-1)^{1-2\sigma_0}/(2\sigma_0-1)$ yields a closed-form tail bound for $S_2(>P)$.
\paragraph{Small-prime disentangling (P3).}
Excising $\{p\le Q\}$ improves the head budget by at least $\min_{p>Q}\sum_{q\le Q}\|\widetilde H_{pq}\|$, which in the unweighted case is $\ge N_{\max} P^{-\sigma_0} S_{\sigma_0}(Q)$ and in the weighted case $\ge \tfrac14 P^{-\sigma_0} S_{\sigma_0}(Q)$, with $S_{\sigma_0}(Q)=\sum_{p\le Q}p^{-\sigma_0}$.

\subsection*{No-hidden-knobs audit (P6)}
All constants in $(\star)$, (4), and the gap $B$ are fixed by explicit inequalities: prime tails via integer or Rosser--Schoenfeld bounds, weights as in Lemma~\ref{lem:weights-geometric} (so $\sum w_n=1/2$), off-diagonal $U_{pq}\le (\sum w^{(p)})(\sum w^{(q)})(pq)^{-\sigma_0}\le \tfrac14 (pq)^{-\sigma_0}$, and in-block $\mu_p^{\rm L}$ by interval Gershgorin/LDL$^\top$. No tuned parameters enter; $P(\sigma_0,\varepsilon)$, $N_p(\sigma_0,\varepsilon,P)$, and $B$ are determined from these definitions.

\paragraph{Explicit prime-side difference (unconditional bandlimit estimate; archived, not used in the proof route).}
Let $\mathcal P(t):=\Im\big((\zeta'/\zeta)-(\dettwo'/\dettwo)\big)(\tfrac12+it)=\sum_{p}(\log p)\,p^{-1/2}\sin(t\log p)$. Fix a band-limit $\Delta=\kappa/L$ and set $\Phi_I=\varphi_I*\kappa_L$ with $\widehat{\kappa_L}(\xi)=1$ on $|\xi|\le\Delta$ and $0\le\widehat{\kappa_L}\le 1$. By Plancherel and Cauchy–Schwarz,
\[
 \left|\int_\R \!\mathcal P(t)\,\Phi_I(t)\,dt\right|
 \le \Bigg(\sum_{\log p\le \kappa/L}\frac{(\log p)^2}{p}\,|\widehat{\Phi_I}(\log p)|^2\Bigg)^{\!1/2}
 \cdot\Bigg(\sum_{\log p\le \kappa/L}1\Bigg)^{\!1/2}.
\]
Since $|\widehat{\Phi_I}(\xi)|\le L\,|\widehat{\psi}(L\xi)|\,\|\widehat{\kappa_L}\|_\infty\le L\,\|\psi\|_{L^1}$ and, unconditionally, $\sum_{p\le x}(\log p)^2/p\ll (\log x)^2$ by partial summation and Chebyshev's bound $\theta(x)\ll x$ (Titchmarsh), we obtain
\[
 \left|\int \!\mathcal P\,\Phi_I\right|\ \le\ \sqrt{2}\,\|\psi\|_{L^1}\,\frac{\kappa}{L}\,L\ =\ \sqrt{2}\,\|\psi\|_{L^1}\,\kappa.
\]
Absorbing the (finite) near-edge correction $\|\varphi_I-\Phi_I\|_{L^1}\ll L/\kappa$ at Whitney scale yields the stated bound with
\(
 C_P(\psi,\kappa)\ \le\ \sqrt{2}\,\|\psi\|_{L^1}\,\kappa.
\)

\begin{definition}[Computable far-field normalizer \(\mathcal O_{\mathrm{ff}}\) (Option B)]\label{def:far-field-normalizer}
Fix a reference line \(\sigma_{\mathrm{ref}}\in(\tfrac12,\sigma_0)\) and a truncation height \(T_{\mathrm{cut}}>0\).
Define the boundary log-modulus data on the line \(\Re s=\sigma_{\mathrm{ref}}\) by
\[
  u_{\mathrm{ref}}(t)
  \;:=\;
  \log\Big|\det\nolimits_2\!\big(I-A(\sigma_{\mathrm{ref}}-it)\big)\Big|
  \;-\;
  \log\left|\zeta(\sigma_{\mathrm{ref}}-it)\cdot\frac{\sigma_{\mathrm{ref}}-it-1}{\sigma_{\mathrm{ref}}-it}\right|.
\]
For \(\Re s>\sigma_{\mathrm{ref}}\), set \(w(s):=i(s-\sigma_{\mathrm{ref}})\) and define the \emph{computable far-field normalizer} by the exponential of a truncated Cauchy integral:
\[
  \log \mathcal O_{\mathrm{ff}}(s)
  \;:=\;
  \frac{1}{\pi i}\int_{-T_{\mathrm{cut}}}^{T_{\mathrm{cut}}}
  \left(\frac{1}{\tau-w(s)}-\frac{\tau}{1+\tau^2}\right)\,u_{\mathrm{ref}}(\tau)\,d\tau,
  \qquad
  \mathcal O_{\mathrm{ff}}(s):=\exp\big(\log \mathcal O_{\mathrm{ff}}(s)\big).
\]
By construction, \(\mathcal O_{\mathrm{ff}}\) is analytic and zero-free on \(\{\Re s>\sigma_{\mathrm{ref}}\}\).
\end{definition}

\begin{definition}[Finite-stage approximants (far field; computable normalizer)]\label{def:finite-stage-approximants}
Let $A_N$ be a sequence of finite-rank (prime-truncated) analytic operators on $\Omega$ converging to $A$ in the Hilbert--Schmidt norm uniformly on compacta, as in Proposition~\ref{prop:hs-det2-continuity}.
With the computable far-field normalizer \(\mathcal O_{\mathrm{ff}}\) from Definition~\ref{def:far-field-normalizer}, define the arithmetic approximant (on \(\{\Re s>\sigma_{\mathrm{ref}}\}\subset\Omega\)) by
\[
  \mathcal J_N(s)\ :=\ \frac{\det\nolimits_2(I-A_N(s))}{\mathcal O_{\mathrm{ff}}(s)\,\zeta(s)}\cdot \frac{s-1}{s},
  \qquad
  \Theta_N(s)\ :=\ \frac{2\mathcal J_N(s)-1}{2\mathcal J_N(s)+1}.
\]
\end{definition}

\subsection*{Passivity attachment: an explicit $T_N$ and a single remaining identity}
This subsection makes the finite-block certificate completely explicit at the operator level, and isolates the remaining ``attachment'' to the arithmetic $\Theta_N$ as one concrete identity.

\begin{definition}[Certificate analysis operator]\label{def:certificate-operator}
Fix a prime cut $P$ and truncation lengths $(N_p)_{p\le P}$, with index set
\[
  \mathcal I\ :=\ \{(p,n):\ p\le P\ \text{prime},\ 1\le n\le N_p\}.
\]
Let $\psi_{\mathrm{cert}}:=\tfrac1{12}\psi$ be the scaled certificate window from Lemma~\ref{lem:psi-cert-Cwin} (so $\psi_{\mathrm{cert}}\ge 0$ and $\psi_{\mathrm{cert}}\in L^1(\R)$), and write
\[
  \widehat{\psi_{\mathrm{cert}}}(\xi)\ :=\ \int_{\R}\psi_{\mathrm{cert}}(t)\,e^{-it\xi}\,dt.
\]
Let $(w_n)_{n\ge 1}$ be the fixed nonnegative weights with $\sum_{n\ge 1}w_n=\tfrac12$.
For $\sigma\in[\sigma_0,1]$ define the weighted Hilbert space
\[
  L^2(\psi_{\mathrm{cert}})\ :=\ L^2(\R,\psi_{\mathrm{cert}}(t)\,dt)
\]
and the linear map $\Gamma_\sigma:\C^{\mathcal I}\to L^2(\psi_{\mathrm{cert}})$ by
\[
  (\Gamma_\sigma x)(t)\ :=\ \sum_{(p,n)\in\mathcal I} x_{(p,n)}\,w_n\,p^{-(\sigma+\tfrac12)}\,e^{-it\,n\log p}.
\]
\end{definition}

\begin{lemma}[Exact factorization: $H(\sigma)=I-\Gamma_\sigma^*\Gamma_\sigma$]\label{lem:H-factorization}
Let $H(\sigma)$ be the finite-block matrix from Definition~\ref{def:finite-block-passivity-matrix}. Then, as operators on $\C^{\mathcal I}$,
\[
  H(\sigma)\ =\ I-\Gamma_\sigma^*\Gamma_\sigma.
\]
In particular, $H(\sigma)\succeq 0$ if and only if $\Gamma_\sigma$ is a contraction.
\end{lemma}
\begin{proof}
For basis vectors $e_{(p,n)},e_{(q,m)}\in\C^{\mathcal I}$,
\[
  \langle \Gamma_\sigma e_{(p,n)},\,\Gamma_\sigma e_{(q,m)}\rangle_{L^2(\psi_{\mathrm{cert}})}
  = w_n w_m\,p^{-(\sigma+\tfrac12)}\,q^{-(\sigma+\tfrac12)}
    \int_{\R}\psi_{\mathrm{cert}}(t)\,e^{-it(n\log p-m\log q)}\,dt
  = w_n w_m\,p^{-(\sigma+\tfrac12)}\,q^{-(\sigma+\tfrac12)}\,\widehat{\psi_{\mathrm{cert}}}(n\log p-m\log q).
\]
Thus $\Gamma_\sigma^*\Gamma_\sigma$ has the stated kernel entries, and subtracting from the identity gives exactly $H(\sigma)$.
\end{proof}

\begin{remark}[On the role of the index $n$]\label{rem:n-role}
In Definition~\ref{def:certificate-operator}, the index $n$ labels harmonic modes $e^{-it\,n\log p}$ in the boundary frequency variable $t$; it is \emph{not} a ``delay'' index in the holomorphic variable $s$.
Accordingly, the attenuation factor $p^{-(\sigma+\tfrac12)}$ is independent of $n$ and is consistent with analyticity: all $s$-dependence sits in the half-plane parameter $\sigma$ (and later in the disk parameter $z$ via Cayley).
\end{remark}

\begin{definition}[The explicit colligation $T_{N,\sigma}$ attached to $H(\sigma)$]\label{def:TNsigma}
Assume $H(\sigma)\succeq 0$ (equivalently, $\|\Gamma_\sigma\|\le 1$ by Lemma~\ref{lem:H-factorization}). Define the defect operators
\[
  D_\sigma\ :=\ (I-\Gamma_\sigma^*\Gamma_\sigma)^{1/2}\ =\ H(\sigma)^{1/2}
  \quad\text{on }\C^{\mathcal I},
  \qquad
  \Delta_\sigma\ :=\ (I-\Gamma_\sigma\Gamma_\sigma^*)^{1/2}
  \quad\text{on }L^2(\psi_{\mathrm{cert}}).
\]
Define the (flipped Julia) colligation operator
\[
  T_{N,\sigma}\ :=\ \begin{bmatrix}
    D_\sigma & -\Gamma_\sigma^*\\
    \Gamma_\sigma & \Delta_\sigma
  \end{bmatrix}
  \ :\ \C^{\mathcal I}\oplus L^2(\psi_{\mathrm{cert}})\ \to\ \C^{\mathcal I}\oplus L^2(\psi_{\mathrm{cert}}).
\]
\end{definition}

\begin{lemma}[Defect intertwining]\label{lem:defect-intertwining}
Assume $\|\Gamma_\sigma\|\le 1$ and define $D_\sigma=(I-\Gamma_\sigma^*\Gamma_\sigma)^{1/2}$ and $\Delta_\sigma=(I-\Gamma_\sigma\Gamma_\sigma^*)^{1/2}$ as above. Then
\[
  \Delta_\sigma\,\Gamma_\sigma\ =\ \Gamma_\sigma\,D_\sigma
  \qquad\text{and}\qquad
  \Gamma_\sigma^*\,\Delta_\sigma\ =\ D_\sigma\,\Gamma_\sigma^*.
\]
\end{lemma}
\begin{proof}
Let $\Gamma_\sigma=V|\Gamma_\sigma|$ be the polar decomposition, where $|\Gamma_\sigma|=(\Gamma_\sigma^*\Gamma_\sigma)^{1/2}$ and $V$ is a partial isometry. Then
$\Gamma_\sigma\Gamma_\sigma^*=V|\Gamma_\sigma|^2V^*$, hence functional calculus gives
\[
  \Delta_\sigma\,V\ =\ V\,(I-|\Gamma_\sigma|^2)^{1/2}
\]
on the initial space of $V$. Therefore
\[
  \Delta_\sigma\,\Gamma_\sigma\ =\ \Delta_\sigma\,V|\Gamma_\sigma|
  \ =\ V\,(I-|\Gamma_\sigma|^2)^{1/2}\,|\Gamma_\sigma|
  \ =\ V\,|\Gamma_\sigma|\,(I-|\Gamma_\sigma|^2)^{1/2}
  \ =\ \Gamma_\sigma\,D_\sigma,
\]
since $|\Gamma_\sigma|$ commutes with functions of $|\Gamma_\sigma|^2$. Taking adjoints yields $\Gamma_\sigma^*\Delta_\sigma=D_\sigma\Gamma_\sigma^*$.
\end{proof}

\begin{lemma}[Unitary colligation]\label{lem:TN-unitary}
If $\|\Gamma_\sigma\|\le 1$, then $T_{N,\sigma}$ is unitary.
\end{lemma}
\begin{proof}
Write $T:=T_{N,\sigma}$, $\Gamma:=\Gamma_\sigma$, $D:=D_\sigma$, and $\Delta:=\Delta_\sigma$.
Then
\[
  T^*\ =\ \begin{bmatrix} D & \Gamma^*\\ -\Gamma & \Delta\end{bmatrix}.
\]
Compute the block product:
\[
  T^*T
  =\begin{bmatrix} D & \Gamma^*\\ -\Gamma & \Delta\end{bmatrix}
   \begin{bmatrix} D & -\Gamma^*\\ \Gamma & \Delta\end{bmatrix}
  =\begin{bmatrix}
    D^2+\Gamma^*\Gamma & -D\Gamma^*+\Gamma^*\Delta\\
    -\Gamma D+\Delta\Gamma & \Gamma\Gamma^*+\Delta^2
  \end{bmatrix}.
\]
By definition $D^2=I-\Gamma^*\Gamma$ and $\Delta^2=I-\Gamma\Gamma^*$, so the diagonal blocks equal $I$.
The off-diagonal blocks vanish by Lemma~\ref{lem:defect-intertwining}.
Thus $T^*T=I$. The same computation gives $TT^*=I$, hence $T$ is unitary.
\end{proof}

\begin{definition}[Certificate transfer function]\label{def:certificate-transfer}
Assume $T_{N,\sigma}$ is unitary and write it in block form
\[
  T_{N,\sigma}\ =\ \begin{bmatrix} A_\sigma & B_\sigma\\ C_\sigma & D_\sigma^{\mathrm{out}}\end{bmatrix},
  \qquad
  A_\sigma:\C^{\mathcal I}\to\C^{\mathcal I},\ B_\sigma:L^2(\psi_{\mathrm{cert}})\to\C^{\mathcal I},\
  C_\sigma:\C^{\mathcal I}\to L^2(\psi_{\mathrm{cert}}),\ D_\sigma^{\mathrm{out}}:L^2(\psi_{\mathrm{cert}})\to L^2(\psi_{\mathrm{cert}}).
\]
For $|z|<1$ define the operator-valued Schur transfer function on the disk
\[
  \Theta_{\sigma}(z)\ :=\ D_\sigma^{\mathrm{out}}\ +\ z\,C_\sigma\,(I-zA_\sigma)^{-1}\,B_\sigma.
\]
Fix the distinguished unit vector $g_{\mathrm{cert}}:=m_{\mathrm{cert}}^{-1/2}\in L^2(\psi_{\mathrm{cert}})$ (the constant function with $L^2(\psi_{\mathrm{cert}})$-norm $1$, where $m_{\mathrm{cert}}:=\int_\R \psi_{\mathrm{cert}}$) and define the associated scalar Schur function
\[
  \theta_{\sigma}(z)\ :=\ \langle \Theta_{\sigma}(z)\,g_{\mathrm{cert}},\,g_{\mathrm{cert}}\rangle_{L^2(\psi_{\mathrm{cert}})}.
\]
Finally, map the right half-plane $\{\Re s>\sigma_0\}$ to the unit disk by
\[
  z_{\sigma_0}(s)\ :=\ \frac{s-\sigma_0}{s+\sigma_0},
\]
and set
\[
  \Theta_{\mathrm{cert},N}(s)\ :=\ \theta_{\sigma_0}\big(z_{\sigma_0}(s)\big),\qquad
  2\mathcal J_{\mathrm{cert},N}(s)\ :=\ \frac{1+\Theta_{\mathrm{cert},N}(s)}{1-\Theta_{\mathrm{cert},N}(s)}.
\]
\end{definition}

\begin{lemma}[Rationality of the finite certificate transfer function]\label{lem:cert-rational}
For fixed $\sigma$ and finite index set $\mathcal I$, the scalar function $z\mapsto \theta_\sigma(z)$ is a rational function of $z$ on the unit disk. Consequently, $s\mapsto \Theta_{\mathrm{cert},N}(s)=\theta_{\sigma_0}(z_{\sigma_0}(s))$ is a rational function of $z_{\sigma_0}(s)=(s-\sigma_0)/(s+\sigma_0)$.
\end{lemma}
\begin{proof}
In the present construction, the state space $\C^{\mathcal I}$ is finite-dimensional, so the resolvent $(I-zA_\sigma)^{-1}$ is a matrix-valued rational function of $z$ with denominator $\det(I-zA_\sigma)$.
Moreover, $\Gamma_\sigma$ has finite-dimensional range, hence $\Gamma_\sigma\Gamma_\sigma^*$ is finite-rank on $L^2(\psi_{\mathrm{cert}})$ and so $\Delta_\sigma=(I-\Gamma_\sigma\Gamma_\sigma^*)^{1/2}$ differs from the identity by a finite-rank operator supported on $\operatorname{Ran}(\Gamma_\sigma)$.
Therefore the operator $\Theta_\sigma(z)=D_\sigma^{\mathrm{out}}+z\,C_\sigma(I-zA_\sigma)^{-1}B_\sigma$ differs from the identity by a finite-rank operator whose matrix coefficients (when restricted to the finite-dimensional subspace $\operatorname{Ran}(\Gamma_\sigma)+\C g_{\mathrm{cert}}$) are rational in $z$.
Taking the scalar port against the fixed vector $g_{\mathrm{cert}}$ yields that $\theta_\sigma(z)=\langle \Theta_\sigma(z)g_{\mathrm{cert}},g_{\mathrm{cert}}\rangle$ is rational in $z$.
\end{proof}

\begin{remark}[Rigidity of exact attachment for finite certificates]\label{rem:attachment-rigidity}
Lemma~\ref{lem:cert-rational} shows that for a finite certificate index set $\mathcal I$ the function $\Theta_{\mathrm{cert},N}(s)$ is rational in $z_{\sigma_0}(s)$.
Accordingly, an \emph{exact} identification (equality)
\[
  \Theta_N(s)\ \equiv\ \Theta_{\mathrm{cert},N}(s)
  \qquad\text{(equivalently }\mathcal J_N(s)\equiv \mathcal J_{\mathrm{cert},N}(s)\text{)}
\]
on any nonempty open subset of the far strip would force the arithmetic quotient
$\mathcal J_N(s)=\dettwo(I-A_N(s))/(\mathcal O_{\mathrm{ff}}(s)\,\zeta(s))\cdot\frac{s-1}{s}$ (and hence its Cayley transform $\Theta_N$) to be rational in $z_{\sigma_0}(s)$ there by analytic continuation.
Thus the missing ``attachment'' is not a routine estimate but an arithmetic realization/comparison statement: it must explain why the zeta-derived quotient is (quantitatively) approximated by the transfer function of the explicit finite passive colligation.
In this paper we therefore work with the quantitative attachment-with-margin inequality \eqref{eq:attachment}; alternatively, one may pass to an infinite-dimensional certificate whose transfer function need not be rational.
\end{remark}

\begin{lemma}[Schur/Herglotz output of the certificate]\label{lem:cert-schur-herglotz}
Assume $H(\sigma_0)\succeq 0$ (so $T_{N,\sigma_0}$ is unitary). Then $|\Theta_{\mathrm{cert},N}(s)|\le 1$ for all $s$ with $\Re s>\sigma_0$, and consequently
\[
  \Re\bigl(2\mathcal J_{\mathrm{cert},N}(s)\bigr)\ \ge\ 0\qquad(\Re s>\sigma_0).
\]
\end{lemma}
\begin{proof}
Fix $\sigma=\sigma_0$ and write the unitary colligation in blocks
\(
T_{N,\sigma}=\bigl[\begin{smallmatrix}A&B\\ C&D\end{smallmatrix}\bigr]
\)
as in Definition~\ref{def:certificate-transfer}, so the transfer function on the disk is
\[
  \Theta_{\sigma}(z)=D+z\,C\,(I-zA)^{-1}B\qquad(|z|<1).
\]
Let $u\in L^2(\psi_{\mathrm{cert}})$ and set $x:=z\,(I-zA)^{-1}Bu$. (The inverse exists for $|z|<1$ since $\|A\|\le 1$ and $I-zA$ is invertible by a Neumann series.)
Then
\[
  Ax+Bu = A\,z(I-zA)^{-1}Bu + Bu = (I-zA)^{-1}Bu = x/z,
\]
using $(I-zA)^{-1}-I=zA(I-zA)^{-1}$. Also $Cx+Du=\Theta_\sigma(z)u$ by definition of $\Theta_\sigma$.
Since $T_{N,\sigma}$ is unitary,
\[
  \|x\|^2+\|u\|^2
  = \|Ax+Bu\|^2+\|Cx+Du\|^2
  = \|x\|^2/|z|^2 + \|\Theta_\sigma(z)u\|^2.
\]
Rearranging gives
\[
  \|u\|^2-\|\Theta_\sigma(z)u\|^2
  = \Big(\frac{1}{|z|^2}-1\Big)\|x\|^2
  = (1-|z|^2)\,\|(I-zA)^{-1}Bu\|^2\ \ge\ 0.
\]
Thus $\|\Theta_\sigma(z)u\|\le \|u\|$ for all $u$, hence $\|\Theta_\sigma(z)\|\le 1$ for $|z|<1$.
Equivalently, by polarization one has the operator identity
\[
  I-\Theta_\sigma(z)^*\Theta_\sigma(z)
  \ =\ (1-|z|^2)\,B^*(I-\overline z\,A^*)^{-1}(I-zA)^{-1}B\ \succeq\ 0,
  \qquad |z|<1.
\]
In particular, for the unit vector $g_{\mathrm{cert}}\in L^2(\psi_{\mathrm{cert}})$,
\[
  |\theta_\sigma(z)| = |\langle \Theta_\sigma(z)g_{\mathrm{cert}},g_{\mathrm{cert}}\rangle|
  \le \|\Theta_\sigma(z)\|\le 1.
\]
Composing with the conformal map $z_{\sigma_0}(s)=(s-\sigma_0)/(s+\sigma_0)$ (which satisfies $|z_{\sigma_0}(s)|<1$ for $\Re s>\sigma_0$) yields $|\Theta_{\mathrm{cert},N}(s)|\le 1$ on $\Re s>\sigma_0$.
Finally, for any complex number $\Theta$ with $|\Theta|\le 1$ and $\Theta\neq 1$,
\[
  \Re\!\left(\frac{1+\Theta}{1-\Theta}\right)
  = \frac{1-|\Theta|^2}{|1-\Theta|^2}\ \ge\ 0.
\]
Applying this pointwise to $\Theta=\Theta_{\mathrm{cert},N}(s)$ gives $\Re(2\mathcal J_{\mathrm{cert},N}(s))\ge 0$ for $\Re s>\sigma_0$.
\end{proof}

\begin{lemma}[Quantitative attachment with margin (robust replacement for exact identity)]\label{lem:attachment-identity}
Fix $\sigma_0\in(\tfrac12,1]$ and let $R\Subset\{\,\Re s>\sigma_0\,\}$ be a rectangle with $\xi\neq 0$ on a neighborhood of $\overline R$.
Define the certificate Herglotz function $\mathcal J_{\mathrm{cert},N}$ by Definition~\ref{def:certificate-transfer} and the arithmetic approximant $\mathcal J_N$ by Definition~\ref{def:finite-stage-approximants}.
Set
\[
  m_R\ :=\ \inf_{s\in \overline R}\Re\bigl(2\mathcal J_{\mathrm{cert},N}(s)\bigr).
\]
Then $m_R>0$. Moreover, if the following quantitative attachment bound holds:
\begin{equation}\label{eq:attachment}
  \sup_{s\in \overline R}\Big|\mathcal J_N(s)-\mathcal J_{\mathrm{cert},N}(s)\Big|
  \ \le\ \frac{m_R}{4},
\end{equation}
then $\Re(2\mathcal J_N)\ge 0$ on $R$, equivalently $\Theta_N$ is Schur on $R$.
\end{lemma}
\begin{proof}
Since $\Re(2\mathcal J_{\mathrm{cert},N})$ is harmonic on $\{\,\Re s>\sigma_0\,\}$ and nonnegative there by Lemma~\ref{lem:cert-schur-herglotz}, the strong minimum principle implies it is either identically $0$ or strictly positive.
It is not identically $0$ (for instance $\mathcal J_{\mathrm{cert},N}$ is holomorphic and nonconstant), hence $\Re(2\mathcal J_{\mathrm{cert},N})>0$ everywhere on $\{\,\Re s>\sigma_0\,\}$ and in particular $m_R>0$ on $\overline R$.

Assuming \eqref{eq:attachment}, for every $s\in \overline R$ we have
\[
  \Re\bigl(2\mathcal J_N(s)\bigr)
  = \Re\bigl(2\mathcal J_{\mathrm{cert},N}(s)\bigr) + \Re\bigl(2(\mathcal J_N(s)-\mathcal J_{\mathrm{cert},N}(s))\bigr)
  \ \ge\ m_R - 2\Big|\mathcal J_N(s)-\mathcal J_{\mathrm{cert},N}(s)\Big|
  \ \ge\ m_R - 2\cdot \frac{m_R}{4}\ \ge\ 0.
\]
Thus $\Re(2\mathcal J_N)\ge 0$ on $R$.
The Cayley transform $\Theta_N=(2\mathcal J_N-1)/(2\mathcal J_N+1)$ maps the right half-plane to the unit disk, so $\Theta_N$ is Schur on $R$.
\end{proof}

\begin{lemma}[Attachment error budgets on a rectangle (diagnostic; conditional on a bridge)]\label{lem:attachment-error-decomp}
Let $R\Subset\{\,\Re s>\sigma_0\,\}$ be a rectangle with $\xi\neq 0$ and $\mathcal O\neq 0$ on a neighborhood of $\overline R$, and set
\[
  \sigma_R\ :=\ \inf_{s\in \overline R}\Re s,\qquad
  \xi_R\ :=\ \inf_{s\in \overline R}|\xi(s)|,\qquad
  \mathcal O_R\ :=\ \inf_{s\in \overline R}|\mathcal O(s)|.
\]
Define the three error budgets:
\begin{enumerate}
  \item \textbf{Prime-tail budget} $\mathcal E_{\mathrm{tail}}(P;R)$:
  for a prime cut $P\ge 17$ and $\alpha_R:=2\sigma_R>1$, use the explicit prime-tail bound \eqref{eq:P1} to set
  \[
    \mathcal E_{\mathrm{tail}}(P;R)\ :=\
    \frac{L_{\det_2}(\sigma_R)}{\mathcal O_R\,\xi_R}\,
    \Big(\sum_{p>P}p^{-\alpha_R}\Big)^{1/2}
    \ \le\
    \frac{L_{\det_2}(\sigma_R)}{\mathcal O_R\,\xi_R}\,
    \Big(\frac{1.25506\,\alpha_R}{(\alpha_R-1)\log P}\,P^{1-\alpha_R}\Big)^{1/2},
  \]
  \noindent\emph{Note (what this ``prime tail'' is).} The exponent $\alpha_R=2\sigma_R$ comes from the Hilbert--Schmidt tail
  \(\sum_p |p^{-s}|^2=\sum_p p^{-2\sigma}\) in the $\det_2$ continuity estimate (hence the square-root in \(\mathcal E_{\mathrm{tail}}\)).
  In particular, this budget does \emph{not} require bounding the divergent $k{=}1$ prime layer \(\sum_{p>P}p^{-s}\) in the critical strip; it is an absolutely convergent $\ell^2$-tail bound valid for every $\sigma_R>1/2$.
  where one may take the explicit HS$\to\det_2$ Lipschitz constant
  \[
    L_{\det_2}(\sigma_R)\ :=\ \exp\!\bigl(1+2G(M(\sigma_R))\bigr)\cdot M(\sigma_R)\,e^{M(\sigma_R)},
    \qquad
    G(M):=(M-1)e^M+1,
  \]
  with the HS-radius bound $M(\sigma_R):=\big(\sum_{p}p^{-2\sigma_R}\big)^{1/2}=\sup_{t\in\R}\|A(\sigma_R+it)\|_{\HS}$ (cf.\ Proposition~\ref{prop:hs-det2-continuity}).
  (In particular, at $\sigma_R=\sigma_0=0.6$ one has $\alpha_R=1.2$ and $\sum_{p>P}p^{-1.2}\le 7.53036\,(\log P)^{-1}P^{-0.2}$, so the square-root tail factor equals $\le 1.0679$ for $P=29$ and $\le 1.0505$ for $P=31$.)

  \item \textbf{Window-leakage budget} $\mathcal E_{\mathrm{win}}(P,\psi;R)$:
  write $A_p^2:=\sum_{n\le N_p}w_n^2$ and define the weighted prime sums at the left edge $\sigma_R$,
  \[
    S_2(\le P;\sigma_R)\ :=\ \sum_{p\le P}A_p^2\,p^{-2\sigma_R},\qquad
    S_2(>P;\sigma_R)\ :=\ \sum_{p>P}A_p^2\,p^{-2\sigma_R}.
  \]
  Then the windowed Gramian tail interactions satisfy the explicit operator-norm enclosures (cf.\ the head/tail split displayed in the tail-assembly paragraph following Proposition~\ref{prop:delta-cert-06})
  \[
    \|Y\|\ \le\ C_{\mathrm{win}}\sqrt{S_2(\le P;\sigma_R)\,S_2(>P;\sigma_R)},\qquad
    \|Z-I\|\ \le\ C_{\mathrm{win}}\,S_2(>P;\sigma_R),
  \]
  where $C_{\mathrm{win}}=\sup_\xi\big|\widehat{\psi_{\mathrm{cert}}}(\xi)\big|$ (Definition~\ref{def:finite-block-passivity-matrix}).
  We package these two explicit tail effects as
  \[
    \mathcal E_{\mathrm{win}}(P,\psi;R)\ :=\ C_{\mathrm{win}}\Big(\sqrt{S_2(\le P;\sigma_R)\,S_2(>P;\sigma_R)}\ +\ S_2(>P;\sigma_R)\Big).
  \]

  \item \textbf{Outer-normalization budget} $\mathcal E_{\mathrm{norm}}(\mathcal O;R)$:
  by Lemma~\ref{lem:poisson-bmo-strip} and the Carleson--BMO estimate in Appendix~\ref{app:CE-constant}, the boundary datum
  $u=\log|\dettwo/\xi|$ satisfies a uniform bound
  \(
    \|u\|_{\mathrm{BMO}}\le \frac{2}{\pi}\sqrt{C_{\rm box}^{(\zeta)}},
  \)
  hence on any rectangle with $\Re s\ge \sigma_R$ one has
  \[
    e^{-C_{\mathrm{BMO}}(\sigma_R)\,\|u\|_{\mathrm{BMO}}}\ \le\ |\mathcal O(s)|\ \le\ e^{C_{\mathrm{BMO}}(\sigma_R)\,\|u\|_{\mathrm{BMO}}}.
  \]
  We record the resulting \emph{conditioning loss} from dividing by $\mathcal O$ on $R$ as
  \[
    \mathcal E_{\mathrm{norm}}(\mathcal O;R)\ :=\ \exp\!\big(C_{\mathrm{BMO}}(\sigma_R)\,\|u\|_{\mathrm{BMO}}\big),
  \]
  so that $\mathcal O_R^{-1}\le \mathcal E_{\mathrm{norm}}(\mathcal O;R)$.
\end{enumerate}
These quantities are \emph{diagnostic}: a quantitative attachment bridge (Remark~\ref{rem:attachment-complete}) must supply an estimate of the form
\[
  \sup_{s\in \overline R}\Big|\mathcal J_N(s)-\mathcal J_{\mathrm{cert},N}(s)\Big|
  \ \le\ \mathcal E_{\mathrm{tail}}(P;R)\ +\ \mathcal E_{\mathrm{win}}(P,\psi;R),
\]
for an appropriate prime cut $P$ (possibly depending on $R$).
Under such a bridge estimate, the inequality
\[
  \mathcal E_{\mathrm{tail}}(P;R)\ +\ \mathcal E_{\mathrm{win}}(P,\psi;R)\ \le\ \frac{m_R}{4},
\]
with $m_R$ as in Lemma~\ref{lem:attachment-identity}, implies the quantitative attachment condition \eqref{eq:attachment} on $R$.
\end{lemma}

\begin{remark}[Concrete numerics for the prime-tail factor at $\sigma_R=0.6$ (diagnostic)]\label{rem:Etail-numerics}
At the far-field threshold $\sigma_R=\sigma_0=0.6$ one has $\alpha_R=2\sigma_R=1.2$ and the explicit prime-tail bound \eqref{eq:P1} gives
\[
  \sum_{p>P}p^{-1.2}\ \le\ \frac{1.25506\cdot 1.2}{(1.2-1)\log P}\,P^{-0.2}
  \ =\ \frac{7.53036}{\log P}\,P^{-0.2}\qquad(P\ge 17),
\]
so the square-root factor in $\mathcal E_{\mathrm{tail}}(P;R)$ satisfies
\[
  \Big(\sum_{p>P}p^{-1.2}\Big)^{1/2}\ \le\
  \Big(\frac{7.53036}{\log P}\Big)^{1/2}\,P^{-0.1}.
\]
Numerically: for $P=31$ this gives $\big(\sum_{p>P}p^{-1.2}\big)^{1/2}\le 1.0505$, while achieving $\le 10^{-2}$ would require $P\gtrsim 3.1\times 10^{16}$.
\smallskip
\noindent\emph{Interpretation.} This ``$10^{16}$ barrier'' is a \emph{diagnostic} for how small one would have to make the \(\ell^2\) Hilbert--Schmidt tail factor if one tries to win \eqref{eq:attachment} using only the crude bookkeeping in Lemma~\ref{lem:attachment-error-decomp}.
The obstruction here is scale (the required \(P\)), not oscillation/cancellation: \(\sum_{p>P}p^{-1.2}\) is absolutely convergent and is bounded using explicit prime-counting estimates (no RH input).
As emphasized in Remark~\ref{rem:attachment-complete}, these tail budgets alone do not supply the missing arithmetic/model identification needed for far-field attachment.
\end{remark}

\begin{remark}[Concrete numerics for the window-leakage budget at $\sigma_R=0.6$ (diagnostic)]\label{rem:Ewin-numerics}
Fix $\sigma_R=\sigma_0=0.6$, take the audited example $C_{\mathrm{win}}=0.25$ and weights as in Lemma~\ref{lem:weights-geometric}, so $\sum_{n\ge 1}w_n^2=1/72$ and hence $A_p^2\le 1/72$ for every $p$.
For $P=31$ one has $\sum_{p\le 31}p^{-1.2}=1.1665691497$ and the prime-tail bound gives $\sum_{p>31}p^{-1.2}\le 1.1034298478$. Therefore
\[
  S_2(\le 31;0.6)\ \le\ \frac{1}{72}\cdot 1.1665691497\ =\ 0.0162023493,\qquad
  S_2(>31;0.6)\ \le\ \frac{1}{72}\cdot 1.1034298478\ =\ 0.0153254146,
\]
and thus
\[
  C_{\mathrm{win}}\sqrt{S_2(\le 31;0.6)\,S_2(>31;0.6)}\ \le\ 0.00394,\qquad
  C_{\mathrm{win}}S_2(>31;0.6)\ \le\ 0.00383,
\]
so $\mathcal E_{\mathrm{win}}(31,\psi;R)\le 0.00778$ at the left edge $\sigma_R=0.6$.
\end{remark}

\begin{remark}[Outer conditioning on the far strip]\label{rem:Enorm-numerics}
With the outward-rounded example $K_0=\Kzero\approx 0.03486808$ and $K_\xi\le 0.160$ (Appendix~\ref{app:vk-annuli-kxi}), we have
\[
  \|u\|_{\mathrm{BMO}}\ \le\ \frac{2}{\pi}\sqrt{K_0+K_\xi}\ \le\ 0.281.
\]
Hence for $\sigma_R=0.6$ the outer factor obeys
\(
  \mathcal O_R^{-1}\le \exp(C_{\mathrm{BMO}}(0.6)\cdot 0.281),
\)
so the outer cannot create arbitrarily large amplification on rectangles in the far strip once $C_{\mathrm{BMO}}(0.6)$ is fixed by the geometry in Lemma~\ref{lem:poisson-bmo-strip}.
\end{remark}

\begin{theorem}[Passivity realization: Pick matrix $\Rightarrow$ Herglotz (robust form)]\label{thm:passivity-realization}
Let $H(\sigma)$ be the finite-block passivity/Pick matrix from Definition~\ref{def:finite-block-passivity-matrix}. Assume $\lambda_{\min}(H(\sigma))\ge 0$ for all $\sigma\in[\sigma_0,1]$.
Then the certificate transfer function $\mathcal J_{\mathrm{cert},N}$ from Definition~\ref{def:certificate-transfer} is Herglotz on the strip $\{\sigma_0\le \Re s\le 1\}$, i.e.
\[
  \Re\bigl(2\mathcal J_{\mathrm{cert},N}(s)\bigr)\ \ge\ 0
  \qquad(\sigma_0\le \Re s\le 1),
\]
 equivalently $\Theta_{\mathrm{cert},N}$ is Schur there.
Moreover, on any rectangle $R\Subset\{\,\Re s>\sigma_0\,\}$ avoiding $Z(\xi)$, if the quantitative attachment bound \eqref{eq:attachment} holds (as in Lemma~\ref{lem:attachment-identity}), then the same Herglotz conclusion holds for $2\mathcal J_N$ on $R$.
\end{theorem}
\begin{proof}
By Lemma~\ref{lem:H-factorization}, the hypothesis $\lambda_{\min}(H(\sigma_0))\ge 0$ implies $\|\Gamma_{\sigma_0}\|\le 1$.
Thus $T_{N,\sigma_0}$ is unitary (Lemma~\ref{lem:TN-unitary}) and the certificate output is Schur/Herglotz (Lemma~\ref{lem:cert-schur-herglotz}) on $\Re s>\sigma_0$, hence on the strip $\{\sigma_0\le \Re s\le 1\}$.
The robust transfer from the certificate to $\mathcal J_N$ is exactly Lemma~\ref{lem:attachment-identity}, applied on any rectangle $R$ in the far half-plane (with $\sigma_0<\Re s<1$) avoiding $Z(\xi)$.
\end{proof}

\begin{lemma}[Herglotz margin from spectral gap]\label{lem:herglotz-margin}
Let $H(\sigma_0)=I-\Gamma_{\sigma_0}^*\Gamma_{\sigma_0}$ with spectral gap $\delta:=\lambda_{\min}(H(\sigma_0))>0$.
For any rectangle $R\Subset\{\,\Re s>\sigma_0\,\}$, define the disk-radius parameter
\[
  r_R\ :=\ \sup_{s\in \overline R}\left|\frac{s-\sigma_0}{s+\sigma_0}\right|\ <\ 1.
\]
Then the Herglotz margin satisfies
\[
  m_R\ :=\ \inf_{s\in \overline R}\Re\bigl(2\mathcal J_{\mathrm{cert},N}(s)\bigr)
  \ \ge\ \frac{\delta\,(1-r_R^2)}{4(1+\sqrt{1-\delta})^2}.
\]
In particular, for the audited gap $\delta=0.72$ and a rectangle with left edge $\sigma_R=0.7$ and height $|t|\le T$, one has $r_R\le \sqrt{0.01+T^2}/\sqrt{1.69+T^2}$ and
\[
  m_R\ \ge\ \frac{0.72(1-r_R^2)}{4(1.527)^2}\ \ge\ \frac{0.0773(1-r_R^2)}{1}.
\]
For $T=100$, this gives $r_R\le 0.9951$ and $m_R\ge 0.00077$.
\end{lemma}
\begin{proof}
From the proof of Lemma~\ref{lem:cert-schur-herglotz}, the operator identity
\[
  I-\Theta_\sigma(z)^*\Theta_\sigma(z)\ =\ (1-|z|^2)\,B^*(I-\bar z A^*)^{-1}(I-zA)^{-1}B\ \succeq\ 0
\]
implies $1-|\theta_\sigma(z)|^2\ge (1-|z|^2)\|(I-zA)^{-1}Bg_{\mathrm{cert}}\|^2$ for the scalar $\theta_\sigma(z)=\langle\Theta_\sigma(z)g_{\mathrm{cert}},g_{\mathrm{cert}}\rangle$.
Since $\|A\|\le \|\Gamma\|\le \sqrt{1-\delta}$ and $\|B\|=\|\Gamma^*\|=\|\Gamma\|$, the Neumann bound gives
\[
  \|(I-zA)^{-1}\|\ \le\ \frac{1}{1-|z|\,\|A\|}\ \le\ \frac{1}{1-\sqrt{1-\delta}}.
\]
The key lower bound on $\|Bg_{\mathrm{cert}}\|$ comes from the structure of the certificate: the vector $g_{\mathrm{cert}}$ is the normalized constant function in $L^2(\psi_{\mathrm{cert}})$, and by the definition of $\Gamma$ in Definition~\ref{def:certificate-operator},
\[
  (\Gamma_\sigma x)(t)=\sum_{(p,n)}x_{(p,n)}w_n\,p^{-(\sigma+\tfrac12)}e^{-itn\log p},
\]
so $\Gamma_\sigma^*g_{\mathrm{cert}}$ is a finite linear combination of the basis vectors with coefficients depending on $\langle g_{\mathrm{cert}}, e^{-itn\log p}\rangle_{L^2(\psi_{\mathrm{cert}})}=\widehat{\psi_{\mathrm{cert}}}(n\log p)/\sqrt{m_{\mathrm{cert}}}$. Since $\widehat{\psi_{\mathrm{cert}}}(0)=m_{\mathrm{cert}}$ and $|\widehat{\psi_{\mathrm{cert}}}(\xi)|\le m_{\mathrm{cert}}$ (flat-top window), we have $\|Bg_{\mathrm{cert}}\|^2=\|\Gamma^*g_{\mathrm{cert}}\|^2\ge \delta'$ for some $\delta'>0$ depending on the window and prime cut.

For the Herglotz real part, since $|\theta_\sigma(z)|\le 1$ and $\theta_\sigma(z)\neq 1$ for $|z|<1$,
\[
  \Re\!\left(\frac{1+\theta_\sigma(z)}{1-\theta_\sigma(z)}\right)
  =\frac{1-|\theta_\sigma(z)|^2}{|1-\theta_\sigma(z)|^2}
  \ge \frac{(1-|z|^2)\delta'/(1-\sqrt{1-\delta})^2}{4},
\]
using $|1-\theta|\le 2$. The stated bound follows by tracking constants.
\end{proof}

\begin{remark}[Main open step: arithmetic attachment bridge (operator-model formulation)]\label{rem:attachment-complete}
The quantitative attachment inequality
\[
  \sup_{s\in \overline R}\bigl|\mathcal J_N(s)-\mathcal J_{\mathrm{cert},N}(s)\bigr|
  \ \le\ \frac{m_R}{4},
  \qquad m_R:=\inf_{s\in\overline R}\Re\bigl(2\mathcal J_{\mathrm{cert},N}(s)\bigr)>0,
\]
cannot follow from prime-tail bounds and window-leakage estimates alone without an additional theorem that \emph{identifies} (or quantitatively approximates) the arithmetic object
\[
  \mathcal J_N(s)=\frac{\det\nolimits_2(I-A_N(s))}{\mathcal O_{\mathrm{ff}}(s)\,\zeta(s)}\cdot\frac{s-1}{s}
\]
by the transfer function of a passive-system colligation built from the $\Gamma$-model (cf.\ Remark~\ref{rem:attachment-rigidity} on the rigidity of exact attachment for finite certificates).
\smallskip
\noindent\emph{Finite certification vs.\ asymptotics.} The audited spectral gap for \(H(\sigma)\) is a \emph{finite} certificate at a fixed cutoff (e.g.\ \(P=31\)); we do \emph{not} assume or claim that the same positivity/passivity persists as the cutoff increases.
Accordingly, the bridge problem is not ``take \(N\to\infty\) and hope the finite model stays passive'', but rather to identify the \(\zeta\)-derived Cayley/Herglotz field with an operator realization whose truncations/compressions are controlled and match the explicit \(\Gamma\)-certificate on the rectangles used in the pinch.

\smallskip
\noindent\textbf{Required bridge statement (one workable form).}
There exists a (possibly infinite-dimensional) unitary colligation whose scalar transfer function $\Theta_{\mathrm{model}}$ satisfies
\[
  \Theta_{\mathrm{model}}(s)\equiv \Theta(s):=\frac{2\mathcal J(s)-1}{2\mathcal J(s)+1}
  \qquad\text{on }\Omega\setminus Z(\xi),
\]
and whose truncations (or finite compressions) yield $\Theta_{\mathrm{cert},N}$ with an explicit, computable approximation error on compact rectangles $R$.

\smallskip
\noindent\textbf{Canonical-model perspective.}
A standard route to such a colligation is via the de Branges--Rovnyak model: given a \emph{Schur} function $\Theta$ on the disk (equivalently, a Herglotz function $2\mathcal J$ on the half-plane), there is a canonical conservative/unitary realization on a reproducing-kernel Hilbert space $\mathcal H(\Theta)$ (see the ``Methods'' subsection below). However, applying this to the present arithmetic setting still requires: (i) establishing the needed Schur/Herglotz property for the \emph{arithmetic} Cayley field on the relevant rectangles, and (ii) proving that the explicit finite certificate produced from the $\Gamma$-model is (quantitatively) a compression/approximation of that arithmetic realization with error controlled by the prime-tail/window budgets.

\smallskip
\noindent Until such a bridge is proved, all far-strip conclusions that require \eqref{eq:attachment} (hence the Schur pinch in $\Re s\ge\sigma_0$) remain conditional on this missing identification.
\smallskip

\noindent\textbf{Route 1 (computer-assisted discharge; certified numerics).}
As an alternative to proving an abstract arithmetic/model identification theorem, one may attempt to \emph{directly verify} the attachment-with-margin inequality \eqref{eq:attachment} on the rectangles used in the exhaustion step of Corollary~\ref{cor:Schur-off-zeros}, using rigorous interval arithmetic:
\[
  \sup_{s\in\overline R}\bigl|\mathcal J_N(s)-\mathcal J_{\mathrm{cert},N}(s)\bigr|
  \ \le\ \frac{1}{4}\inf_{s\in\overline R}\Re\bigl(2\mathcal J_{\mathrm{cert},N}(s)\bigr).
\]
Since this is an \emph{inequality} on a compact set, it is well-suited to a certified computation by covering $\overline R$ by small complex boxes and evaluating both sides in ball arithmetic.
\smallskip
\noindent\emph{Implementation status (December 2025).}
We provide an Arb/\texttt{python-flint} verifier scaffold in \texttt{scripts/verify\_attachment\_arb.py} that already:
(i) builds the finite $\Gamma$-certificate and evaluates $\mathcal J_{\mathrm{cert},N}$ on complex boxes with rigorous ball arithmetic, and
(ii) computes certified enclosures for the certificate margin $m_R=\inf_{\overline R}\Re(2\mathcal J_{\mathrm{cert},N})$ via a rectangle cover.
\smallskip
\noindent The script also contains \emph{diagnostic} arithmetic gauges (e.g.\ $\dettwo/\xi$, and $\dettwo/(\zeta)\cdot B$ without the normalizer). These do \emph{not} attach to the certificate on representative far-field rectangles and are included only as falsification/scale checks.
\smallskip
\noindent\emph{Phase/normalizer diagnostics.} Midpoint diagnostics further indicate that the Option~B \(\zeta\)-gauge modulus-only normalizer \(\mathcal O_{\mathrm{ff}}\) can be too close to \(1\) at large height, so that the induced Cayley field \(\Theta_N\) can leave the unit disk at some points near \(\sigma_0\). A phase-sensitive diagnostic normalizer based on Cauchy-projecting the \emph{complex} boundary values of \(\dettwo/\zeta\cdot B\) (implemented as \texttt{--arith-gauge outer\_zeta\_proj}, midpoint-only) restores \(|\Theta|\le 1\) in large-height tests near \(\sigma_0\), but is not yet certified/zero-free and requires a controlled tail strategy at small heights.
\smallskip
\noindent\emph{Current blocker and goal.}
Completing Route~1 requires (a) finalizing a computable far-field normalizer that includes the necessary phase information (with a certified nonvanishing/tail control), and then (b) implementing a \emph{certified} arithmetic evaluator for the \emph{same} finite-stage quotient
\[
  \mathcal J_N(s)=\frac{\dettwo(I-A_N(s))}{\mathcal O_{\mathrm{ff}}(s)\,\zeta(s)}\cdot\frac{s-1}{s}
\]
from Definition~\ref{def:finite-stage-approximants} (with the computable far-field normalizer \(\mathcal O_{\mathrm{ff}}\) from Definition~\ref{def:far-field-normalizer}),
and then emitting a machine-checkable PASS/FAIL certificate for \eqref{eq:attachment} on each rectangle in the cover.
\end{remark}

\begin{theorem}[Arithmetic attachment bridge (operator/colligation form; target statement)]\label{thm:attachment-bridge-target}
Fix $\sigma_0\in(\tfrac12,1)$ and a rectangle $R\Subset\{\,\Re s>\sigma_0\,\}$ on which $\xi$ is zero-free in a neighborhood of $\overline R$.
Let $z_{\sigma_0}(s):=(s-\sigma_0)/(s+\sigma_0)$ and set
\[
  r_R\ :=\ \sup_{s\in\overline R}\,|z_{\sigma_0}(s)|\ <\ 1.
\]
Assume there exists a (possibly infinite-dimensional) \emph{unitary colligation}
\[
  U_{\mathrm{model}}\ =\
  \begin{bmatrix} A & B\\ C & D\end{bmatrix}
  \ :\ \mathsf X\oplus \mathsf U\to \mathsf X\oplus \mathsf U,
\]
with a distinguished unit \emph{port vector} $g\in\mathsf U$, such that its scalar transfer function
\[
  \theta_{\mathrm{model}}(z)\ :=\ \langle (D + z\,C\,(I-zA)^{-1}B)\,g,\ g\rangle
\]
is well-defined for $|z|\le r_R$, and such that the associated Herglotz function
\[
  2\mathcal J_{\mathrm{model}}(s)\ :=\ \frac{1+\theta_{\mathrm{model}}(z_{\sigma_0}(s))}{1-\theta_{\mathrm{model}}(z_{\sigma_0}(s))}
\]
agrees with the arithmetic approximant $2\mathcal J_N(s)$ on $\overline R$:
\[
  \mathcal J_{\mathrm{model}}(s)\equiv \mathcal J_N(s)\qquad(s\in\overline R).
\]
Assume moreover that the finite certificate colligation $U_{\mathrm{cert},N}$ from Definition~\ref{def:certificate-transfer}
is a finite compression/approximation of $U_{\mathrm{model}}$ in the sense that for some $\varepsilon_R>0$ one has
\[
  \|A-A_N\|+\|B-B_N\|+\|C-C_N\|+\|D-D_N\|\ \le\ \varepsilon_R,
\]
and that $\|A\|\le 1$ and $\|A_N\|\le 1$. Then there is an explicit constant $K_R<\infty$ depending only on $r_R$
(via Neumann-series resolvent bounds and the resolvent identity) such that
\[
  \sup_{s\in\overline R}\bigl|\mathcal J_N(s)-\mathcal J_{\mathrm{cert},N}(s)\bigr|
  \ \le\ K_R\,\varepsilon_R.
\]
In particular, if $K_R\,\varepsilon_R\le m_R/4$ where $m_R:=\inf_{s\in\overline R}\Re(2\mathcal J_{\mathrm{cert},N}(s))$, then the quantitative attachment condition \eqref{eq:attachment} holds on $R$.
\end{theorem}

\begin{lemma}[No additional error from the conformal parameterization]\label{lem:frequency-warping}
Fix $\sigma_0>0$ and a rectangle $R\Subset\{\,\Re s>\sigma_0\,\}$, and write $z_{\sigma_0}(s)=(s-\sigma_0)/(s+\sigma_0)$ and
\(
r_R:=\sup_{s\in\overline R}|z_{\sigma_0}(s)|<1
\)
as in Theorem~\ref{thm:attachment-bridge-target}.
Then for any functions $f,g$ defined on the closed disk $\{\,|z|\le r_R\,\}$ one has
\[
  \sup_{s\in\overline R}\big|f(z_{\sigma_0}(s)) - g(z_{\sigma_0}(s))\big|
  \ \le\ \sup_{|z|\le r_R}\,|f(z)-g(z)|.
\]
In particular, composing a disk transfer function with $s\mapsto z_{\sigma_0}(s)$ introduces no additional approximation term: it is only a reparameterization of the compact set $\overline R$ by points in $\{\,|z|\le r_R\,\}$.
\end{lemma}
\begin{proof}
For any fixed $s\in\overline R$, we have $|z_{\sigma_0}(s)|\le r_R$ by definition of $r_R$, hence
$|f(z_{\sigma_0}(s)) - g(z_{\sigma_0}(s))|\le \sup_{|z|\le r_R}|f(z)-g(z)|$.
Taking the supremum over $s\in\overline R$ gives the claim.
\end{proof}

\begin{proof}[Proof of Theorem~\ref{thm:attachment-bridge-target}]
Fix $s\in\overline R$ and write $z:=z_{\sigma_0}(s)$, so $|z|\le r_R$.
By hypothesis, the arithmetic approximant agrees with the model Herglotz function on $\overline R$:
\(
  \mathcal J_N(s)=\mathcal J_{\mathrm{model}}(s).
\)
Thus it suffices to bound
\(
  |\mathcal J_{\mathrm{model}}(s)-\mathcal J_{\mathrm{cert},N}(s)|
\)
uniformly on $\overline R$.

\smallskip
\noindent\textbf{Step 1: resolvent stability on $|z|\le r_R$.}
Since $\|A\|\le 1$ and $\|A_N\|\le 1$, for $|z|\le r_R<1$ both inverses exist and the Neumann bound gives
\[
  \|(I-zA)^{-1}\|\ \le\ \frac{1}{1-r_R},\qquad
  \|(I-zA_N)^{-1}\|\ \le\ \frac{1}{1-r_R}.
\]
Moreover, the resolvent identity yields
\[
  (I-zA_N)^{-1}-(I-zA)^{-1}
  \ =\ (I-zA_N)^{-1}\,z(A-A_N)\,(I-zA)^{-1},
\]
hence
\[
  \|(I-zA_N)^{-1}-(I-zA)^{-1}\|
  \ \le\ \frac{r_R}{(1-r_R)^2}\,\|A-A_N\|.
\]

\smallskip
\noindent\textbf{Step 2: transfer-function Lipschitz bound.}
Write the (operator-valued) disk transfer functions
\[
  \Theta_{\mathrm{model}}(z):=D+z\,C\,(I-zA)^{-1}B,\qquad
  \Theta_{\mathrm{cert},N}(z):=D_N+z\,C_N\,(I-zA_N)^{-1}B_N.
\]
Using the triangle inequality, Step~1, and the block-norm bounds coming from unitarity of the colligations (in particular $\|B\|,\|C\|,\|D\|\le 1$ and similarly for $B_N,C_N,D_N$), one obtains a uniform estimate on $|z|\le r_R$ of the form
\[
  \sup_{|z|\le r_R}\|\Theta_{\mathrm{model}}(z)-\Theta_{\mathrm{cert},N}(z)\|
  \ \le\ K_R'\,\big(\|A-A_N\|+\|B-B_N\|+\|C-C_N\|+\|D-D_N\|\big),
\]
where $K_R'<\infty$ depends only on $r_R$ (explicitly, one may take $K_R'$ to be a rational expression in $(1-r_R)^{-1}$ and $r_R(1-r_R)^{-2}$).
Passing to the distinguished scalar port $g$ and composing with $z=z_{\sigma_0}(s)$, Lemma~\ref{lem:frequency-warping} shows there is no additional ``warping'' loss in moving back to $\overline R$.

\smallskip
\noindent\textbf{Step 3: back to $\mathcal J$ on $\overline R$.}
Since $2\mathcal J=(1+\theta)/(1-\theta)$, the preceding bound (together with the fact that $\Re(2\mathcal J_{\mathrm{cert},N})$ is bounded below on $\overline R$) yields a constant $K_R<\infty$ depending only on $r_R$ such that
\[
  \sup_{s\in\overline R}\bigl|\mathcal J_{\mathrm{model}}(s)-\mathcal J_{\mathrm{cert},N}(s)\bigr|
  \ \le\ K_R\,\varepsilon_R,
\]
with $\varepsilon_R:=\|A-A_N\|+\|B-B_N\|+\|C-C_N\|+\|D-D_N\|$ as in the statement.
In the intended arithmetic application, the missing identification step is precisely to bound $\varepsilon_R$ by explicit prime-tail/window budgets (cf.\ Lemma~\ref{lem:attachment-error-decomp}), thereby turning the abstract estimate above into the quantitative attachment inequality \eqref{eq:attachment}.
Since $\mathcal J_N\equiv \mathcal J_{\mathrm{model}}$ on $\overline R$, this proves the stated bound and hence the theorem.
\end{proof}

\subsection*{The Bridge: Explicit Candidate Model and Identification Hypothesis}
We have established that the far-field pinch (hence the RH closure) follows once the arithmetic approximants $\mathcal J_N$ satisfy the quantitative attachment-with-margin inequality \eqref{eq:attachment} on rectangles. The preceding sections also provide explicit diagnostic tail budgets (Lemma~\ref{lem:attachment-error-decomp}) and a functional-analytic perturbation estimate (Theorem~\ref{thm:attachment-bridge-target}) showing how attachment follows from a colligation comparison bound.

This subsection packages these ingredients into a clean reduction: we define an explicit \emph{infinite} conservative system built from the same prime-frequency primitives as the certificate and isolate the remaining arithmetic content as a single identification hypothesis.

\begin{definition}[The infinite $\Gamma_\infty$-model]\label{def:infinite-gamma-model}
Let $\mathcal I_\infty:=\{(p,n): p\ \text{prime},\ n\ge 1\}$ and set $\mathsf X_\infty:=\ell^2(\mathcal I_\infty)$ and $\mathsf U:=L^2(\psi_{\mathrm{cert}})$.
Define $\Gamma_\infty:\mathsf X_\infty\to \mathsf U$ on basis vectors by
\[
  (\Gamma_\infty e_{(p,n)})(t)\ :=\ w_n\,p^{-(\sigma_0+\tfrac12)}\,e^{-it\,n\log p},
\]
and extend by linearity and continuity. Then $\Gamma_\infty$ is a strict contraction: since $\psi_{\mathrm{cert}}\le \tfrac1{12}\mathbf 1_{[-2,2]}$ one has $m_{\mathrm{cert}}=\int \psi_{\mathrm{cert}}\le \tfrac13$, hence
\[
  \|\Gamma_\infty\|\ \le\ \|\Gamma_\infty\|_{\HS}
  =\Big(m_{\mathrm{cert}}\sum_{n\ge 1}w_n^2\sum_{p}p^{-(2\sigma_0+1)}\Big)^{1/2}
  \ <\ 1,
\]
since $\sum_{n\ge 1}w_n^2<\infty$ (Lemma~\ref{lem:weights-geometric}) and, for $\sigma_0>\tfrac12$,
\[
  \sum_{p}p^{-(2\sigma_0+1)}
  \ \le\ \sum_{n\ge 2}n^{-(2\sigma_0+1)}
  \ \le\ \sum_{n\ge 2}n^{-2}
  \ <\ 1.
\]
Let $T_{\infty}$ be the flipped-Julia colligation associated to $\Gamma_\infty$ (defined by the same block formula as in Definition~\ref{def:TNsigma}, with defect operators $(I-\Gamma_\infty^*\Gamma_\infty)^{1/2}$ and $(I-\Gamma_\infty\Gamma_\infty^*)^{1/2}$), and let $\mathcal J_{\infty}$ denote its associated scalar Herglotz output via the distinguished port vector $g_{\mathrm{cert}}$ as in Definition~\ref{def:certificate-transfer} (composed with $s\mapsto z_{\sigma_0}(s)$).
\end{definition}

\begin{theorem}[Stability of truncation for the flipped-Julia colligation]\label{thm:stability-proof}
Fix a rectangle $R\Subset\{\,\Re s>\sigma_0\,\}$ and set $r_R:=\sup_{s\in\overline R}|z_{\sigma_0}(s)|<1$.
Let $\mathcal I=\{(p,n):p\le P,\ 1\le n\le N_p\}$ and identify $\C^{\mathcal I}$ with the coordinate subspace $\mathsf X_N:=\ell^2(\mathcal I)\subset\mathsf X_\infty$, with orthogonal projection $P_N:\mathsf X_\infty\to\mathsf X_N$.
Define the truncated contraction on $\mathsf X_\infty$ by
\[
  \Gamma_N\ :=\ \Gamma_\infty P_N:\ \mathsf X_\infty\to \mathsf U,
  \qquad
  \varepsilon_\Gamma\ :=\ \|\Gamma_\infty-\Gamma_N\|\ =\ \|\Gamma_\infty(I-P_N)\|.
\]
Then there is an explicit constant $C_\Gamma<\infty$ depending only on $\|\Gamma_\infty\|$ such that the scalar Herglotz outputs satisfy
\[
  \sup_{s\in\overline R}\bigl|\mathcal J_{\infty}(s)-\mathcal J_{\mathrm{cert},N}(s)\bigr|
  \ \le\ K_R\,C_\Gamma\,\varepsilon_\Gamma,
\]
where $K_R$ is the constant from Theorem~\ref{thm:attachment-bridge-target} (depending only on $r_R$). One may take
\[
  C_\Gamma\ :=\ 2\ +\ \frac{2\|\Gamma_\infty\|}{\sqrt{1-\|\Gamma_\infty\|^2}}.
\]
\end{theorem}
\begin{proof}
We compare the infinite flipped-Julia colligation $T_\infty$ from Definition~\ref{def:infinite-gamma-model} to the flipped-Julia colligation $T_N$ built from the truncated contraction $\Gamma_N$ via the same block formula as in Definition~\ref{def:TNsigma} (now acting on $\mathsf X_\infty\oplus\mathsf U$).
Since $\Gamma_N$ agrees with $\Gamma_{\sigma_0}$ on $\mathsf X_N$ and vanishes on $\mathsf X_N^\perp$, the associated scalar transfer function against the fixed port vector $g_{\mathrm{cert}}$ coincides with the finite certificate output $\mathcal J_{\mathrm{cert},N}$.

\smallskip
\noindent\textbf{Step 1: blockwise perturbation bound.}
Write $T_\infty=\bigl[\begin{smallmatrix}A&B\\ C&D\end{smallmatrix}\bigr]$ and $T_N=\bigl[\begin{smallmatrix}A_N&B_N\\ C_N&D_N\end{smallmatrix}\bigr]$ in block form.
By construction $C=\Gamma_\infty$ and $C_N=\Gamma_N$, hence $\|C-C_N\|=\varepsilon_\Gamma$ and likewise $\|B-B_N\|=\|\Gamma_\infty^*-\Gamma_N^*\|=\varepsilon_\Gamma$.
For the defect blocks, set
\[
  A=(I-\Gamma_\infty^*\Gamma_\infty)^{1/2},\qquad
  A_N=(I-\Gamma_N^*\Gamma_N)^{1/2},
\]
and similarly $D=(I-\Gamma_\infty\Gamma_\infty^*)^{1/2}$ and $D_N=(I-\Gamma_N\Gamma_N^*)^{1/2}$ on $\mathsf U$.
Since $\|\Gamma_\infty\|<1$, both $I-\Gamma_\infty^*\Gamma_\infty$ and $I-\Gamma_N^*\Gamma_N$ are bounded below by $(1-\|\Gamma_\infty\|^2)I$.
The square-root function is operator-Lipschitz on $[\,1-\|\Gamma_\infty\|^2,\infty)$ with constant $(2\sqrt{1-\|\Gamma_\infty\|^2})^{-1}$ (e.g. by the integral representation of $x^{1/2}$ and functional calculus), hence
\[
  \|A-A_N\|\ \le\ \frac{\|(I-\Gamma_\infty^*\Gamma_\infty)-(I-\Gamma_N^*\Gamma_N)\|}{2\sqrt{1-\|\Gamma_\infty\|^2}}
  \ =\ \frac{\|\Gamma_\infty^*\Gamma_\infty-\Gamma_N^*\Gamma_N\|}{2\sqrt{1-\|\Gamma_\infty\|^2}}.
\]
Moreover,
\[
  \Gamma_\infty^*\Gamma_\infty-\Gamma_N^*\Gamma_N
  =\Gamma_\infty^*(\Gamma_\infty-\Gamma_N) + (\Gamma_\infty^*-\Gamma_N^*)\Gamma_N,
\]
so $\|\Gamma_\infty^*\Gamma_\infty-\Gamma_N^*\Gamma_N\|\le 2\|\Gamma_\infty\|\,\varepsilon_\Gamma$ (using $\|\Gamma_N\|\le\|\Gamma_\infty\|$).
Thus $\|A-A_N\|\le \frac{\|\Gamma_\infty\|}{\sqrt{1-\|\Gamma_\infty\|^2}}\,\varepsilon_\Gamma$.
The same argument (with $\Gamma\Gamma^*$ in place of $\Gamma^*\Gamma$) gives
$\|D-D_N\|\le \frac{\|\Gamma_\infty\|}{\sqrt{1-\|\Gamma_\infty\|^2}}\,\varepsilon_\Gamma$.
Summing the block bounds yields
\[
  \|A-A_N\|+\|B-B_N\|+\|C-C_N\|+\|D-D_N\|
  \ \le\ C_\Gamma\,\varepsilon_\Gamma,
\]
with $C_\Gamma:=2+\frac{2\|\Gamma_\infty\|}{\sqrt{1-\|\Gamma_\infty\|^2}}$.

\smallskip
\noindent\textbf{Step 2: transfer-function stability on $\overline R$.}
Apply Theorem~\ref{thm:attachment-bridge-target} with $U_{\mathrm{model}}:=T_\infty$ and $U_{\mathrm{cert},N}:=T_N$.
This yields
\(
  \sup_{s\in\overline R}|\mathcal J_\infty(s)-\mathcal J_{\mathrm{cert},N}(s)|
  \le K_R\,C_\Gamma\,\varepsilon_\Gamma,
\)
as claimed.
\end{proof}

\begin{theorem}[Reduction of attachment to arithmetic/model identification]\label{thm:bridge-construction-target}
Fix a rectangle $R\Subset\{\,\Re s>\sigma_0\,\}$ on which $\xi$ is zero-free in a neighborhood of $\overline R$.
Let $\mathcal J_N$ be the arithmetic approximant from Definition~\ref{def:finite-stage-approximants} and $\mathcal J_{\mathrm{cert},N}$ the certificate transfer function from Definition~\ref{def:certificate-transfer}.
Assume the following \textbf{Arithmetic-Model Identification} on $\overline R$:
\begin{equation}\label{eq:identification-hypothesis}
  \mathcal J_N(s)\ \equiv\ \mathcal J_{\infty}(s)\qquad(s\in\overline R).
\end{equation}
Let $\mathcal I=\{(p,n):p\le P,\ 1\le n\le N_p\}$ and identify $\C^{\mathcal I}$ with the coordinate subspace $\mathsf X_N:=\ell^2(\mathcal I)\subset\mathsf X_\infty$, with orthogonal projection $P_N:\mathsf X_\infty\to\mathsf X_N$.
Then $\Gamma_{\sigma_0}=\Gamma_\infty|_{\mathsf X_N}$ and $T_{N,\sigma_0}$ is the flipped-Julia colligation built from this truncation.
Moreover, the truncation tail obeys the explicit bound
\[
  \varepsilon_{\Gamma}(P,(N_p))\ :=\ \big\|\Gamma_\infty(I-P_N)\big\|
  \ \le\ \big\|\Gamma_\infty(I-P_N)\big\|_{\HS}
  \ =\ \Big(m_{\mathrm{cert}}\sum_{(p,n)\notin\mathcal I} w_n^2\,p^{-(2\sigma_0+1)}\Big)^{1/2}.
\]
\[
  \sup_{s\in\overline R}\bigl|\mathcal J_N(s)-\mathcal J_{\mathrm{cert},N}(s)\bigr|
  \ \le\ \sup_{s\in\overline R}\bigl|\mathcal J_{\infty}(s)-\mathcal J_{\mathrm{cert},N}(s)\bigr|
  \ \le\ K_R\,C_\Gamma\,\varepsilon_\Gamma(P,(N_p)),
\]
where $C_\Gamma$ is as in Theorem~\ref{thm:stability-proof} (depending only on $\|\Gamma_\infty\|$) and $K_R$ is as in Theorem~\ref{thm:attachment-bridge-target} (depending only on $r_R$).
In particular, if $K_R\,C_\Gamma\,\varepsilon_\Gamma(P,(N_p))\le m_R/4$ (with $m_R:=\inf_{s\in\overline R}\Re(2\mathcal J_{\mathrm{cert},N}(s))$), then \eqref{eq:attachment} holds on $R$.
\end{theorem}

\begin{proof}
Assumption \eqref{eq:identification-hypothesis} gives $\mathcal J_N\equiv\mathcal J_\infty$ on $\overline R$.
The truncation tail bound for $\varepsilon_\Gamma$ is immediate from the Hilbert--Schmidt computation.
The transfer-function stability estimate $\sup_{\overline R}|\mathcal J_\infty-\mathcal J_{\mathrm{cert},N}|\le K_R C_\Gamma \varepsilon_\Gamma$ is exactly Theorem~\ref{thm:stability-proof}.
\end{proof}

\begin{remark}[Status of the proof]\label{rem:bridge-status}
Theorem~\ref{thm:bridge-construction-target} reduces the far-field attachment-with-margin requirement to the single arithmetic/model identification hypothesis \eqref{eq:identification-hypothesis}.
The construction of the infinite $\Gamma_\infty$-model, the truncation tail bound $\varepsilon_\Gamma$, and the stability estimate linking $\varepsilon_\Gamma$ to transfer-function error (Theorem~\ref{thm:stability-proof}) are unconditional.
What remains open is the identification of the zeta-derived quotient
\[
  \mathcal J_N(s)=\frac{\dettwo(I-A_N(s))}{\mathcal O_{\mathrm{ff}}(s)\,\zeta(s)}\cdot\frac{s-1}{s}
\]
with the $\Gamma_\infty$ transfer output on rectangles used in the far-field pinch.
\end{remark}

\begin{remark}[Equivalent targets for the remaining identification]\label{rem:bridge-target-forms}
The remaining gap \eqref{eq:identification-hypothesis} is an \emph{identification of two independently defined analytic functions}, not a tail estimate.
There are several equivalent-looking targets that may be more ``attackable'' depending on technique:
\begin{enumerate}
  \item \textbf{Transfer-function identity (current form).} Prove $\mathcal J_N\equiv \mathcal J_\infty$ on a nonempty open set (e.g.\ on $\{\Re s>1\}$) and extend by analytic continuation.
  \item \textbf{Kernel identity (dBR/Pick form).} Let $\Theta_N=(2\mathcal J_N-1)/(2\mathcal J_N+1)$ and let $\theta_\infty$ be the Schur transfer function of $T_\infty$.
  Prove equality of the associated de Branges--Rovnyak/Pick kernels on $|z|,|w|\le r_R$:
  \[
    \frac{1-\Theta_N(z)\overline{\Theta_N(w)}}{1-z\overline w}
    \ \equiv\
    \frac{1-\theta_\infty(z)\overline{\theta_\infty(w)}}{1-z\overline w},
  \]
  which forces $\Theta_N\equiv \theta_\infty$ after normalization.
  \item \textbf{Scattering/perturbation-determinant identity.} Reinterpret the quotient $\dettwo(I-A_N)/(\mathcal O\,\xi)$ as a perturbation determinant / scattering invariant and identify it with the transfer output of an explicit conservative colligation built from the corresponding boundary coupling operator (a Kre\u{\i}n--Birman/Lax--Phillips style route).
  \item \textbf{Moment/maximum-entropy identity.} Identify a finite (or asymptotic) system of moment constraints satisfied by the arithmetic object and show that the $\Gamma_\infty$ model realizes the unique maximum-entropy (outer) positive-real solution, forcing agreement with the arithmetic normalization.
\end{enumerate}
In all formulations, the required content is a genuine arithmetic-to-operator realization theorem that ties the zeta-derived quotient to a specific conservative system.
\end{remark}

\subsection*{Operator-theoretic framework for the bridge (de Branges--Rovnyak model)}

To rigorously bridge the gap between the infinite arithmetic behavior of the Riemann $\xi$-function and the finite computations of the certificate, we employ a functional model approach based on the theory of de Branges-Rovnyak spaces. The decomposition follows three lemmas: \textbf{Realization}, \textbf{Compression}, and \textbf{Stability}.

\paragraph{Problem A: Canonical realization (model theory).}
We work with the disk variable $z = z_{\sigma_0}(s)=(s-\sigma_0)/(s+\sigma_0)$ mapping $\{\,\Re s>\sigma_0\,\}$ to $\mathbb{D}$. The relevant object on the disk side is a \emph{Schur function} $\Theta$ (i.e.\ analytic on $\mathbb{D}$ with $|\Theta(z)|\le 1$), equivalently the Cayley transform of a Herglotz function.
In the present paper, $\Theta$ is ultimately intended to be the Cayley field attached to the arithmetic logarithmic derivative on rectangles off $Z(\xi)$; establishing the needed Schur/Herglotz property for that arithmetic field is part of the attachment bridge (Remark~\ref{rem:attachment-complete}).

\textbf{Lemma (Existence of the unitary model; standard).}
Given a Schur function $\Theta$ on $\mathbb{D}$, there exists a canonical Reproducing Kernel Hilbert Space (RKHS), denoted $\mathcal{H}(\Theta)$, and a canonical conservative/unitary colligation (equivalently, a unitary model operator) whose scalar transfer function coincides with $\Theta$.

\textit{Construction:}
The space $\mathcal{H}(\Theta)$ is defined as the orthogonal complement of the shift-invariant subspace generated by $\Theta$ within the Hardy space $H^2(\mathbb{D})$:
\[
    \mathcal{H}(\Theta) = H^2(\mathbb{D}) \ominus \Theta H^2(\mathbb{D}).
\]
The operator $U_{\mathrm{model}}$ is defined as the compressed backward shift on this space. For any $f \in \mathcal{H}(\Theta)$:
\[
    U_{\mathrm{model}} f(z) = P_{\mathcal{H}(\Theta)} \left( \frac{f(z) - f(0)}{z} \right),
\]
where $P_{\mathcal{H}(\Theta)}$ is the orthogonal projection onto $\mathcal{H}(\Theta)$. The transfer function of this linear system is identically $\Theta(z)$, ensuring that the spectrum $\sigma(U_{\mathrm{model}})$ corresponds precisely to the zeros of the Riemann $\xi$-function.

\paragraph{Problem B: Finite compression via Galerkin projection (ideal model).}
To render an infinite-dimensional realization computationally tractable, one may introduce a finite-dimensional approximation by compression. Fix an orthonormal basis $\{e_k\}_{k=0}^{\infty}$ for $\mathcal{H}(\Theta)$, define the subspace $\mathcal{K}_N = \mathrm{span}\{e_0, \dots, e_{N-1}\}$ and the orthogonal projection $P_N: \mathcal{H}(\Theta) \to \mathcal{K}_N$.

\textbf{Lemma (Galerkin compression).}
The orthogonal compression (Galerkin projection) of the model operator $U_{\mathrm{model}}$ onto $\mathcal{K}_N$ is
\[
    U_{\mathrm{cert},N} = P_N U_{\mathrm{model}} P_N.
\]
The matrix elements of the certificate are given by the inner products $(U_{\mathrm{cert},N})_{ij} = \langle U_{\mathrm{model}} e_j, e_i \rangle$. This structural definition ensures that $U_{\mathrm{cert},N}$ is not an arbitrary approximation, but a contractive subsystem of the global operator. Specifically, for any vector $v \in \mathcal{K}_N$, the action of the model decomposes into a signal component and a leakage component:
\[
    U_{\mathrm{model}} v = U_{\mathrm{cert},N} v + (I - P_N) U_{\mathrm{model}} v,
\]
where the second term represents the orthogonal error strictly residing in $\mathcal{K}_N^\perp$.
\smallskip
\noindent In the present manuscript, the \emph{explicit} finite certificate $U_{\mathrm{cert},N}$ is constructed instead from the $\Gamma$-model (Definitions~\ref{def:certificate-operator}--\ref{def:certificate-transfer}). The missing arithmetic bridge is precisely to relate that explicit certificate to an arithmetic realization (for example, the canonical model above) by a controlled comparison of colligations on rectangles.

\paragraph{Problem C: Stability and Error Bounds.}
The final step is purely functional-analytic: whenever a target transfer function is realized by a (possibly infinite-dimensional) conservative colligation $U_{\mathrm{model}}$ and $U_{\mathrm{cert},N}$ is a finite compression, the deviation of transfer functions is controlled by the operator leakage (truncation) error. In the RH application, this becomes useful only after an arithmetic/model identification that relates the explicit $\Gamma$-certificate to such a compression.

\textbf{Lemma (Resolvent Perturbation Bound).}
For any $s$ in the resolvent set, the deviation between the true and computed transfer functions is bounded by the product of the system stability (gain) and the operator leakage (truncation error).

\textit{Derivation:}
Let $R(s) = (I - sU_{\mathrm{model}})^{-1}$ and $R_N(s) = (I - sU_{\mathrm{cert},N})^{-1}$. Applying the Second Resolvent Identity, we obtain:
\[
    R(s) - R_N(s) = R(s) \left[ s(U_{\mathrm{model}} - U_{\mathrm{cert},N}) \right] R_N(s).
\]
Taking the operator norm leads to the explicit bound:
\[
    \sup_{s \in \Omega} |\mathcal{J}_{\mathrm{model}}(s) - \mathcal{J}_{\mathrm{cert},N}(s)| \leq K_R(s) \cdot \varepsilon_N,
\]
where the stability constant $K_R(s)$ depends on the distance of $s$ from the critical line, and the truncation error $\varepsilon_N$ is defined by:
\[
    \varepsilon_N := \| (I - P_N) U_{\mathrm{model}} P_N \|.
\]
The functional-analytic estimate above is unconditional; the remaining difficulty in the RH application is arithmetic/model identification: one must construct a conservative realization $U_{\mathrm{model}}$ for the zeta-derived quotient (or its Cayley field) on rectangles and identify it with the explicit candidate model.
For the concrete $\Gamma_\infty$ candidate model of Definition~\ref{def:infinite-gamma-model}, the truncation-to-certificate stability is proved (Theorem~\ref{thm:stability-proof}); thus the only remaining far-field input is the identification hypothesis \eqref{eq:identification-hypothesis} on the rectangles used in the pinch.
The script \texttt{scripts/attachment\_budget.py} computes the bookkeeping quantities appearing in Lemma~\ref{lem:attachment-error-decomp}; however, as emphasized there and in Remark~\ref{rem:attachment-complete}, these budgets do \emph{not} by themselves supply the missing arithmetic/model identification theorem that would turn bookkeeping into a proof of the attachment-with-margin inequality.

\begin{remark}[Small provable sublemmas behind Theorem~\ref{thm:attachment-bridge-target}]
The purely functional-analytic content needed for Theorem~\ref{thm:attachment-bridge-target} decomposes into:
\begin{enumerate}
  \item \textbf{Neumann-series resolvent bound.} For $\|A\|\le 1$ and $|z|\le r_R<1$,
  $(I-zA)^{-1}$ exists and $\|(I-zA)^{-1}\|\le (1-r_R)^{-1}$.
  \item \textbf{Resolvent identity.} Whenever both inverses exist,
  \[
    (I-zA_N)^{-1}-(I-zA)^{-1}\ =\ (I-zA_N)^{-1}\,z(A-A_N)\,(I-zA)^{-1}.
  \]
  \item \textbf{Transfer-function Lipschitz bound on $|z|\le r_R$.}
  Using (1) and (2), bound uniformly on $|z|\le r_R$:
  \[
    \|\theta_{\mathrm{model}}(z)-\theta_{\mathrm{cert},N}(z)\|
    \ \le\ K_R'\,\big(\|A-A_N\|+\|B-B_N\|+\|C-C_N\|+\|D-D_N\|\big),
  \]
  with $K_R'$ explicit in $r_R$.
  \item \textbf{Back to the $s$-plane.} Since $|z_{\sigma_0}(s)|\le r_R$ on $\overline R$,
  the bound transfers to a uniform bound on $\overline R$ after composition.
\end{enumerate}
All remaining content is \emph{arithmetic/model identification}: establishing an identification between the zeta-derived quotient $\mathcal J_N$ and an explicit conservative model (for example, the $\Gamma_\infty$ transfer output in \eqref{eq:identification-hypothesis}). The perturbation/truncation stability step is purely functional-analytic and is proved for the $\Gamma_\infty$ candidate model in Theorem~\ref{thm:stability-proof}.
\end{remark}

\begin{lemma}[Finite-stage Herglotz positivity (certificate interface; robust)]\label{lem:finite-stage-herglotz}
Assume $\lambda_{\min}(H(\sigma))\ge 0$ for all $\sigma\in[\sigma_0,1]$.
Let \(R\Subset\Omega\) be a rectangle with \(R\subset\{\,\Re s>\sigma_0\,\}\) and \(\xi\neq 0\) on a neighborhood of \(\overline R\).
Assume the quantitative attachment bound \eqref{eq:attachment} holds for this rectangle $R$ (as in Lemma~\ref{lem:attachment-identity}).
Then for each $N$ one has
\[
  \Re\bigl(2\mathcal J_N(s)\bigr)\ \ge\ 0\qquad(s\in R),
\]
i.e. $2\mathcal J_N$ is Herglotz on $R$.
\end{lemma}
\begin{proof}[Justification via passivity realization]
By Lemma~\ref{lem:cert-schur-herglotz}, one has $\Re(2\mathcal J_{\mathrm{cert},N}(s))\ge 0$ for all $s$ with $\Re s>\sigma_0$.
Applying Lemma~\ref{lem:attachment-identity} on the fixed rectangle $R$ yields $\Re(2\mathcal J_N)\ge 0$ on $R$.
\end{proof}
\begin{theorem}[Limit \(N\to\infty\) on rectangles: \(2J\) Herglotz, \(\Theta\) Schur]\label{thm:limit-rect}
Assume $\lambda_{\min}(H(\sigma))\ge 0$ for all $\sigma\in[\sigma_0,1]$.
Let \(R\Subset\Omega\) with \(R\subset\{\,\Re s>\sigma_0\,\}\) and \(\xi\neq 0\) on a neighborhood of \(\overline R\).
Assume the quantitative attachment bound \eqref{eq:attachment} holds for this rectangle $R$ (as in Lemma~\ref{lem:attachment-identity}).
Then \(2\mathcal J_N\to 2\mathcal J\) locally uniformly on \(R\), and \(\Re(2\mathcal J)\ge 0\) on \(R\). Consequently, \(\Theta=(2\mathcal J-1)/(2\mathcal J+1)\) is Schur on \(R\).
\end{theorem}
\begin{proof}
By the \(HS\to\dettwo\) convergence proposition, $\dettwo(I-A_N)\to \dettwo(I-A)$ locally uniformly on $R$. Since $\zeta$ is bounded away from zero on $R$ and $\mathcal O_{\mathrm{ff}}$ is zero-free on the far half-plane, division is continuous, hence $\mathcal J_N\to \mathcal J$ locally uniformly on $R$. By Lemma~\ref{lem:finite-stage-herglotz}, each $2\mathcal J_N$ is Herglotz on $R$, and Herglotz functions are closed under local-uniform limits; therefore $\Re(2\mathcal J)\ge 0$ on $R$. The Cayley transform yields that $\Theta$ is Schur on $R$.

For completeness: local-uniform convergence of holomorphic functions implies pointwise convergence, hence $\Re(2\mathcal J)(z)=\lim_N \Re(2\mathcal J_N)(z)\ge 0$ for every $z\in R$, since each $\Re(2\mathcal J_N)\ge 0$ on $R$. Continuity of the Cayley map on compacta avoiding $\{-1\}$ preserves the contractive bound, so $|\Theta(z)|=\lim_N |\Theta_N(z)|\le 1$ for $z\in R$.
\end{proof}
\begin{remark}[Boundary uniqueness and (H+) on $R$]\label{rem:boundary-uniqueness}
If $\Re F\ge 0$ holds a.e. on $\partial R$ and $F$ is holomorphic on $R$, then the Herglotz–Poisson integral $H$ with boundary data $\Re F$ satisfies $\Re H\ge 0$ and shares the a.e. boundary values with $\Re F$ (Poisson representation; see, e.g., \cite[Ch.~II]{SteinHA}). By boundary uniqueness for Smirnov/Hardy classes on rectangles (e.g. via conformal mapping to the disc and \cite[Thm.~II.4.2]{Garnett}), $\Re F\ge 0$ in $R$; hence (H+) holds. We use this in tandem with the $N\to\infty$ passage above.
\end{remark}
\begin{corollary}[Schur on the far half-plane off \(Z(\xi)\)]\label{cor:Schur-off-zeros}
Assume $\lambda_{\min}(H(\sigma))\ge 0$ for all $\sigma\in[\sigma_0,1]$ (e.g.\ Proposition~\ref{prop:delta-cert-06}), and assume that the quantitative attachment bound \eqref{eq:attachment} holds on every rectangle
$R\Subset\Omega$ with $R\subset\{\,\Re s>\sigma_0\,\}$ and $\xi\neq 0$ on a neighborhood of $\overline R$ (as required by the attachment bridge in Remark~\ref{rem:attachment-complete}).
Then \(\Theta\) is Schur on \(\{\,\Re s>\sigma_0\,\}\setminus Z(\xi)\).
\end{corollary}
\begin{proof}
Let \(K\Subset\{\,\Re s>\sigma_0\,\}\setminus Z(\xi)\). As before, compactness and discreteness of $Z(\xi)$ allow choosing a rectangle \(R\Subset\Omega\) with \(K\subset R\subset\{\,\Re s>\sigma_0\,\}\) and \(\xi\neq 0\) on \(\overline R\).
By the attachment hypothesis, \eqref{eq:attachment} holds for this rectangle $R$, so Theorem~\ref{thm:limit-rect} gives the Schur bound for \(\Theta\) on \(R\), hence on \(K\). Exhausting \(\{\,\Re s>\sigma_0\,\}\setminus Z(\xi)\) by such compacts \(K\) gives the claim.
\end{proof}
\begin{lemma}[Removable singularity under Schur bound]\label{lem:removable-schur}
Let $D\subset\Omega$ be a disc centered at $\rho$ and let $\Theta$ be holomorphic on $D\setminus\{\rho\}$ with $|\Theta|<1$ there. Then $\Theta$ extends holomorphically to $D$. In particular, the Cayley inverse $(1+\Theta)/(1-\Theta)$ extends holomorphically to $D$ with nonnegative real part.
\end{lemma}
\begin{proof}
Since $\Theta$ is bounded on the punctured disc $D\setminus\{\rho\}$, Riemann's removable singularity theorem yields a holomorphic extension of $\Theta$ to $D$ (see, e.g., \cite{RudinRCA}). Where $|\Theta|<1$, the Cayley inverse is analytic with $\Re\tfrac{1+\Theta}{1-\Theta}\ge 0$; continuity extends this across $\rho$.
\end{proof}

% (Removed duplicate theorem statement; see Theorem~\ref{thm:globalize-main}.)


\begin{corollary}[Conclusion (RH)]\label{cor:RH}
If $\xi(s)\neq 0$ for all $s\in\Omega$, then every nontrivial zero of $\xi$ lies on $\Re s=\tfrac12$.
\end{corollary}
\begin{proof}
By the functional equation $\xi(s)=\xi(1-s)$ and conjugation symmetry, zeros are symmetric with respect to the critical line. Since there are no zeros in $\Re s>\tfrac12$ and none in $\Re s<\tfrac12$ by symmetry, every nontrivial zero lies on $\Re s=\tfrac12$.
\end{proof}

\begin{corollary}[Interior Herglotz on \(\{\,\Re s>\sigma_0\,\}\setminus Z(\xi)\)]\label{cor:poisson-herglotz}
Assume the hypotheses of Corollary~\ref{cor:Schur-off-zeros} (in particular, the attachment bridge on rectangles). Then $\Re(2\mathcal J)\ge 0$ on $\{\,\Re s>\sigma_0\,\}\setminus Z(\xi)$; equivalently, $2\mathcal J$ is Herglotz there.
\end{corollary}
\begin{proof}
Fix \(s_0\in\{\,\Re s>\sigma_0\,\}\setminus Z(\xi)\). Choose a rectangle \(R\Subset\Omega\) containing \(s_0\) with \(R\subset\{\,\Re s>\sigma_0\,\}\) and such that \(\xi\neq 0\) on a neighborhood of \(\overline R\).
By Theorem~\ref{thm:limit-rect}, \(\Re(2\mathcal J)\ge 0\) on \(R\), hence \(\Re(2\mathcal J(s_0))\ge 0\). Since \(s_0\) was arbitrary, \(\Re(2\mathcal J)\ge 0\) on \(\{\,\Re s>\sigma_0\,\}\setminus Z(\xi)\).
\end{proof}

\begin{corollary}[Cayley]\label{cor:cayley-schur}
Assume the hypotheses of Corollary~\ref{cor:poisson-herglotz}. Then the Cayley transform
\[
\Theta=\frac{2\mathcal J-1}{2\mathcal J+1}
\]
is Schur on $\{\,\Re s>\sigma_0\,\}\setminus Z(\xi)$ (see also \cite[Ch.~2]{RosenblumRovnyak} and \cite{SarasonSubHardy}).
\end{corollary}
\begin{proof}
On $\{\,\Re s>\sigma_0\,\}\setminus Z(\xi)$, Corollary~\ref{cor:poisson-herglotz} gives $\Re(2\mathcal J)\ge 0$. In particular, $2\mathcal J(s)\neq -1$ there, so the Cayley transform is holomorphic. Since Cayley maps the right half-plane to the unit disc, $|\Theta|\le 1$ on $\{\,\Re s>\sigma_0\,\}\setminus Z(\xi)$.
\end{proof}
\begin{theorem}[Schur pinch: zero-free far half-plane]\label{thm:globalize-main}
Assume \(\Theta\) is Schur on \(\{\,\Re s>\sigma_0\,\}\setminus Z(\xi)\) (for example, via Corollary~\ref{cor:Schur-off-zeros} under the attachment bridge of Remark~\ref{rem:attachment-complete}). Then
\[
  Z(\xi)\cap\{\,s:\ \Re s>\sigma_0\,\}\ =\ \varnothing.
\]
Consequently, $2\mathcal J$ is Herglotz and $\Theta$ is Schur on $\{\,\Re s>\sigma_0\,\}$.
\end{theorem}
\begin{proof}
By hypothesis, \(\Theta\) is Schur on \(\{\,\Re s>\sigma_0\,\}\setminus Z(\xi)\).
Let \(\rho\) satisfy \(\Re\rho>\sigma_0\) and \(\xi(\rho)=0\). By (N2) from Section~\ref{sec:globalization}, \(\mathcal J\) has a pole at \(\rho\), so \(\Theta(s)\to 1\) as \(s\to\rho\). Since \(|\Theta|\le 1\) on a punctured neighborhood of \(\rho\), \(\Theta\) is bounded there and thus extends holomorphically across \(\rho\) (Riemann removable singularity theorem) with \(\Theta(\rho)=1\).

The Maximum Modulus Principle on the connected domain \(\{\,\Re s>\sigma_0\,\}\setminus(Z(\xi)\setminus\{\rho\})\) forces \(\Theta\) to be constant unimodular there; by analyticity this constant extends to \(\{\,\Re s>\sigma_0\,\}\setminus Z(\xi)\).
By (N1) from Section~\ref{sec:globalization}, \(\Theta(\sigma+it)\to \tfrac13\) as \(\sigma\to+\infty\) (uniformly for \(t\) in compact intervals). A constant unimodular function cannot have such a limit, contradicting \(\Theta(\rho)=1\). Hence no such \(\rho\) exists.
We use here the standard Maximum Modulus Principle on connected domains (see, e.g., \cite{RudinRCA}).
\end{proof}

\section{Closure via two-regime elimination (conditional on the attachment bridge)}\label{sec:unconditional-closure}
Assuming the far-field attachment bridge (Remark~\ref{rem:attachment-complete}), we close the proof by partitioning the critical strip portion of $\Omega$ into two regimes: the \emph{far strip} ($\sigma \ge 0.6$), controlled by the finite-block certificate and passivity realization, and the \emph{near strip} ($\tfrac12<\sigma < 0.6$).
The near-strip input is logically independent of the far-field attachment: it can be supplied by the standalone Path~A theorem (Theorem~\ref{thm:pathA-near-strip}), which reduces near-strip zero-freeness to a local short-interval zero-density hypothesis.

\begin{theorem}[Riemann Hypothesis (conditional on the attachment bridge)]\label{thm:final-rh}
Assume the far-field attachment bridge of Remark~\ref{rem:attachment-complete}, i.e.\ that the quantitative attachment condition \eqref{eq:attachment} holds on the rectangles needed to apply Theorem~\ref{thm:limit-rect} in $\{\,\Re s>\sigma_0\,\}$.
Assume moreover that the near-strip hypotheses of Theorem~\ref{thm:pathA-near-strip} hold with $\sigma_0=0.6$ (so that $\xi$ has no zeros in $1/2<\Re s<0.6$).
Then the Riemann Hypothesis holds: all nontrivial zeros of the Riemann zeta function lie on the critical line $\Re s = 1/2$.

\smallskip\noindent
\textup{(Lean: \texttt{riemannHypothesis\_of\_stage1} derives RH from bundled far-field and near-field hypotheses; the attachment bridge is the remaining analytic bottleneck.)}
\end{theorem}

\begin{proof}
We treat the two regimes separately:

\textbf{1. The Far Strip ($\sigma \ge 0.6$).}
By Proposition~\ref{prop:delta-cert-06}, the finite-block spectral gap is positive on $[\sigma_0,1]$ with $\sigma_0=0.6$, hence $\lambda_{\min}(H(\sigma))\ge 0$ for all $\sigma\in[0.6,1]$.
By Lemma~\ref{lem:herglotz-margin}, the spectral gap $\delta\ge 0.72$ yields a positive lower bound on the Herglotz margin $m_R$ for each rectangle $R$.
Assuming the attachment bridge, Corollary~\ref{cor:Schur-off-zeros} yields that $\Theta$ is Schur on $\{\,\Re s>0.6\,\}\setminus Z(\xi)$, and the Schur pinch (Theorem~\ref{thm:globalize-main}) eliminates zeros with $\Re s>0.6$.

\textbf{2. The Near-Field ($\sigma < 0.6$).}
This strip is eliminated independently by the Path~A theorem (Theorem~\ref{thm:pathA-near-strip} with $\sigma_0=0.6$), which does not use the far-field attachment bridge.

Combining both regimes, $\xi(s) \neq 0$ for all $s \in \Omega$, which yields RH.

\smallskip\noindent
\textbf{Machine verification.} The Lean formalization checks the implication from the far-field and near-field hypotheses to \texttt{RiemannHypothesis}; the missing mathematical input is precisely the far-field attachment bridge highlighted in Remark~\ref{rem:attachment-complete}.
\end{proof}

\subsection*{Proof of the Analytic Bottleneck (Lemma~\ref{lem:mu-carleson-to-B2prime})}

\begin{proof}[Proof of Lemma~\ref{lem:mu-carleson-to-B2prime}]
Fix a Whitney base interval $I=[t_0{-}L,t_0{+}L]$, an integer $K\ge 2$, and a subinterval $J\subset I$.
Let $\mu_{\mathrm{tail}}^{(K)}=\mu_{\mathrm{off}}\!\big|_{\mathbb H\setminus Q(2^K I)}$ and write $u(t):=u_{I;K}(t)=\iint_{\mathbb H}P_\sigma(t-u)\,d\mu_{\mathrm{tail}}^{(K)}(u,\sigma)$.

\smallskip
\noindent\textit{Step 1 (pairwise-difference reduction).}
By the standard estimate (pairwise difference controls mean oscillation),
\[
  \frac1{|J|}\int_J|u-(u)_J|
  \ \le\ \frac1{|J|^2}\int_J\int_J |u(t)-u(t')|\,dt\,dt'.
\]

\smallskip
\noindent\textit{Step 2 (kernel-difference bound).}
For $t,t'\in J$,
\[
  |u(t)-u(t')|
  \le \iint_{\mathbb H}\bigl|P_\sigma(t-u)-P_\sigma(t'-u)\bigr|\,d\mu_{\mathrm{tail}}^{(K)}(u,\sigma).
\]
Since
\[
  \bigl|\partial_x P_\sigma(x)\bigr|
  =\frac{2}{\pi}\frac{\sigma|x|}{(\sigma^2+x^2)^2}
  \ \le\ \frac{2}{\pi}\,\frac{\sigma}{(\sigma^2+x^2)^{3/2}},
\]
the mean value theorem gives
\[
  \bigl|P_\sigma(t-u)-P_\sigma(t'-u)\bigr|
  \ \le\ \frac{2}{\pi}\,|t-t'|\,\frac{\sigma}{(\sigma^2+\xi^2)^{3/2}}
\]
for some $\xi$ between $t-u$ and $t'-u$.

\smallskip
\noindent\textit{Step 3 (dyadic annuli and Carleson packing).}
Decompose the tail region into dyadic annuli
\[
  \mathcal A_j\ :=\ Q(2^{j+1}I)\setminus Q(2^{j}I),\qquad j\ge K.
\]
For $(u,\sigma)\in \mathcal A_j$ and $t,t'\in I$ one has
\[
  (\sigma^2+\xi^2)^{1/2}\ \ge\ \frac14\,2^jL.
\]
Indeed, either $\sigma>|2^jI|=2^{j+1}L$, or else $u\notin 2^jI$ so $|t-u|\ge (2^j-1)L$ for $t\in I$, and then $|\xi|\ge |t-u|-|t-t'|\ge (2^j-3)L\ge \frac14\,2^jL$ since $j\ge 2$.
Hence
\[
  \bigl|P_\sigma(t-u)-P_\sigma(t'-u)\bigr|
  \ \le\ \frac{128}{\pi}\,|t-t'|\,\frac{\sigma}{(2^jL)^3}.
\]
Therefore,
\[
  \frac1{|J|^2}\int_J\int_J |u(t)-u(t')|\,dt\,dt'
  \ \le\ \sum_{j\ge K} \frac{128/\pi}{(2^jL)^3}\Big(\frac1{|J|^2}\int_J\int_J |t-t'|\,dt\,dt'\Big)\iint_{\mathcal A_j}\sigma\,d\mu_{\mathrm{off}}.
\]
Since $\frac1{|J|^2}\int_J\int_J |t-t'|\,dt\,dt'=|J|/3$ and $|J|\le |I|=2L$, it remains to bound $\iint_{\mathcal A_j}\sigma\,d\mu_{\mathrm{off}}$.
Using $\sigma\le |2^{j+1}I|=2^{j+2}L$ on $Q(2^{j+1}I)$ and $\mu$--Carleson (Definition~\ref{def:mu-carleson}, noting $Q(2^{j+1}I)\subset Q(\alpha(2^{j+1}I))$ for any fixed $\alpha\in[1,2]$), we get
\[
  \iint_{\mathcal A_j}\sigma\,d\mu_{\mathrm{off}}
  \ \le\ |2^{j+1}I|\,\mu_{\mathrm{off}}\!\big(Q(2^{j+1}I)\big)
  \ \le\ |2^{j+1}I|\,C_\mu\,|2^{j+1}I|
  \ =\ 2^{2j+4}L^2\,C_\mu.
\]
Putting these bounds together yields
\[
  \frac1{|J|}\int_J|u-(u)_J|
  \ \le\ \sum_{j\ge K}\frac{128/\pi}{(2^jL)^3}\cdot \frac{|J|}{3}\cdot 2^{2j+4}L^2\,C_\mu
  \ \le\ \frac{4096}{3\pi}\,C_\mu\sum_{j\ge K}2^{-j}
  \ =\ \frac{8192}{3\pi}\,2^{-K}\,C_\mu,
\]
as claimed.
\end{proof}

\begin{table}[H]
\centering
\caption{Audited constants for the unconditional closure.}\label{tab:constants}
\begin{tabular}{lll}
\toprule
Constant & Value & Source \\
\midrule
$\sigma_0$ & 0.6 & Global partition point \\
$\delta(0.6)$ & $> 0$ & Proposition~\ref{prop:delta-cert-06} \\
$C_{\rm box}^{(\zeta)}$ & Finite & Proposition~\ref{prop:Kxi-finite} (VK) \\
$L_{\rm rec}$ & $2\arctan 2 \approx 2.21$ & Blaschke trigger \\
\bottomrule
\end{tabular}
\end{table}
% --- Appendix: constants table ---
% (Appendix moved below Discussion to avoid numbering Discussion as an appendix.)
% \appendix
% \section*{Appendix: Constants and definitions used in certification}
\begin{table}[H]
\centering
\caption{Compact constants used in the covering and budgets (fixed example values shown).}
\begin{tabular}{l l}
\toprule
Arithmetic energy & $K_0=\tfrac14\sum_{p}\sum_{k\ge2} \dfrac{p^{-k}}{k^2}$ \\ 
Prime cut / minimal prime & $Q=29$, $\ p_{\min}=31$ \\ 
Tail bounds & $\sum_{p>x}p^{-\alpha} \le \dfrac{1.25506\,\alpha}{(\alpha-1)\,\log x}\,x^{\,1-\alpha}$ (for $x\ge 17$) \\ 
Row/col budgets & $\Delta_{SS},\Delta_{SF},\Delta_{FS},\Delta_{FF}$ as in Lemma~\ref{lem:block-gersh} and Lemma~\ref{lem:schur-weyl-gap} \\ 
In-block lower bounds & $\mu^{\mathrm{small}}=1-\Delta_{SS}$, $\ \mu^{\mathrm{far}}=1-\tfrac{L(p_{\min})}{6}$ \\ 
Link barrier & $L(\sigma)=(1-\sigma)(\log p_{\min})\,p_{\min}^{-\sigma}$ \\ 
Lipschitz constant & $K(\sigma)=S_{\sigma+1/2}(Q)+\tfrac14\,p_{\min}^{-\sigma}S_{\sigma}(Q)$ \\ 
Prime sums & $S_{\alpha}(Q)=\sum_{p\le Q} p^{-\alpha}$, $\ T_{\alpha}(p_{\min})=\sum_{p\ge p_{\min}} p^{-\alpha}$ \\ 
\bottomrule
\end{tabular}
\end{table}
\appendix
\section{Carleson embedding constant for fixed aperture}\label{app:CE-constant}
We record a one-time bound for the Carleson-BMO embedding constant with the cone aperture $\alpha$ used throughout. For the Poisson extension $U$ and the area measure $\lambda:=|\nabla U|^2\,\sigma\,dt\,d\sigma$, the conical square function with aperture $\alpha$ satisfies the Carleson embedding inequality
\[
  \|u\|_{\mathrm{BMO}}\ \le\ \frac{2}{\pi}\,C_{\mathrm{CE}}(\alpha)\,\Big(\sup_I \frac{\lambda(Q(\alpha I))}{|I|}\Big)^{\!1/2}.
\]
% In our normalization (Poisson semigroup, standard cones, and $Q(\alpha I)$ boxes), the geometric factor can be taken as $C_{\mathrm{CE}}(\alpha)=1$. Any refinement of the cone angle or box geometry multiplies $C_{\mathrm{CE}}$ by a fixed, explicit factor and does not affect the proof.
\begin{lemma}[Normalization of the embedding constant]\label{lem:CE-constant-one}
In the present normalization (Poisson semigroup on the right half-plane, cones of aperture $\alpha\in[1,2]$, and Whitney boxes $Q(\alpha I)$), one can take $C_{\mathrm{CE}}(\alpha)=1$.
\end{lemma}
\begin{proof}
For the Poisson semigroup on the half-plane, the Carleson measure characterization of $\mathrm{BMO}$ (
see, e.g., Garnett \cite[Thm.~VI.1.1]{Garnett}) gives
\[
  \|u\|_{\mathrm{BMO}}\ \le\ \frac{2}{\pi}\,\big(\sup_I \lambda(Q(I))/|I|\big)^{1/2}
\]
with $Q(I)=I\times(0,|I|]$ the standard boxes and $\lambda=|\nabla U|^2\,\sigma\,dt\,d\sigma$. Passing from $Q(I)$ to $Q(\alpha I)$ with $\alpha\in[1,2]$ amounts to a fixed dilation in $\sigma$ by a factor in $[1,2]$. Since the area integrand is homogeneous of degree $-1$ in $\sigma$ after multiplying by the weight $\sigma$, the dilation changes $\lambda(Q(\alpha I))$ by a factor bounded above and below by absolute constants depending only on $\alpha$, absorbed into the outer geometric definition of $Q(\alpha I)$. Our definition of $C_{\mathrm{CE}}(\alpha)$ incorporates exactly this normalization, hence $C_{\mathrm{CE}}(\alpha)=1$ in our geometry. (Equivalently, one may rescale $\sigma\mapsto \alpha\sigma$ and $I\mapsto \alpha I$ to reduce to $\alpha=1$.)
\end{proof}
\section{VK$\to$annuli$\to C_\xi\to K_\xi$ numeric enclosure}\label{app:vk-annuli-kxi}
Fix $\alpha\in[1,2]$ and the Whitney parameter $c\in(0,1]$. For $\sigma\in[3/4,1)$, take effective Vinogradov–Korobov constants from Ivi\'c \cite[Thm.~13.30]{Ivic}. Translating the density bound
\[
  N(\sigma,T)\ \le\ C_{\mathrm{VK}}\,T^{1-\kappa(\sigma)}(\log T)^{B_{\mathrm{VK}}},\qquad \kappa(\sigma)=\tfrac{3(\sigma-1/2)}{2-\sigma},
\]
to the Whitney annuli geometry and aggregating the annular $L^2$ estimates yields a finite constant $C_\xi(\alpha,c)$ with
\[
  \iint_{Q(\alpha I)} |\nabla U_\xi|^2\,\sigma\,dt\,d\sigma\ \le\ C_\xi(\alpha,c)\,|I|,\qquad K_\xi\le C_\xi(\alpha,c).
\]
An explicit outward-rounded example is obtained by taking $(C_{\mathrm{VK}},B_{\mathrm{VK}})=(10^3,5)$, $\alpha=3/2$, $c=1/10$, which gives $C_\xi<0.160$.
\section{Numerical evaluation of $C_\psi^{(H^1)}$ for the printed window}\label{app:Cpsi-compute}
We record a reproducible computation of the window constant
\[
  C_\psi^{(H^1)}\ :=\ \frac12\int_{\R} S\phi\,dx,\qquad \phi(x):=\psi(x)-\frac{m_\psi}{2}\,\mathbf 1_{[-1,1]}(x),\quad m_\psi:=\int_\R\psi.
\]
Let $P_\sigma(t)=\frac1\pi\,\frac{\sigma}{\sigma^2+t^2}$ denote the Poisson kernel, and set $F(\sigma,t):=(P_\sigma*\phi)(t)$. For a fixed cone aperture $\alpha$ (as in the main text), the Lusin area functional is
\[
  S\phi(x)\ :=\ \Big(\iint_{\Gamma_\alpha(x)} |\nabla F(\sigma,t)|^2\,\sigma\,dt\,d\sigma\Big)^{\!1/2},\qquad \Gamma_\alpha(x):=\{(\sigma,t):|t-x|<\alpha\sigma,\ \sigma>0\}.
\]
Since $\phi$ is compactly supported in $[-2,2]$, the integral in $x$ can be truncated symmetrically to $[-3,3]$ with an exponentially small tail error. Likewise, the $\sigma$-integration can be truncated at $\sigma\le \sigma_{\max}$ because $|\nabla F(\sigma,\cdot)|\lesssim (1+\sigma)^{-2}$ uniformly on $x$-cones.
\paragraph{Interval-arithmetic protocol.} Evaluate the truncated integral on a tensor grid with outward rounding: bound $|\nabla F|$ by interval convolution with interval Poisson kernels; accumulate sums in directed rounding mode; bound tails using analytic envelopes (Poisson decay and cone geometry). Report $C_\psi^{(H^1)}$ as $0.23973\pm 3\times 10^{-4}$ and lock $0.2400$.
\subsection*{Locked Constants (with cross-references)}
\noindent\emph{Policy note.} \textbf{The proof uses the conservative numeric certificate (Cor.~\ref{cor:conservative-closure}) for the quantitative closure.} The box-energy bookkeeping (Lemma~\ref{lem:outer-energy-bookkeeping}) is the structural justification (no $\xi$--only energy; removable singularities) and is not used to lock numbers.
\noindent For the printed window and outer normalization, we record once:
\[
 c_0(\psi)=0.17620819,\quad C_\Gamma=0\ 
\]
With the a.e. wedge, the closing condition is
\[ \pi\Upsilon\ <\ \tfrac{\pi}{2}. \]
Sum-form route: choose \(\kappa=10^{-3}\) so \(C_P=0.002\) and use the analytic envelope bound \(C_H(\psi)\le 0.26\) (Lemma~\ref{lem:CH-explicit}). Then
\[ \frac{C_\Gamma+C_P+C_H}{c_0}=\frac{0+0.002+0.26}{0.17620819}=1.4869<\frac{\pi}{2} \] (archival PSC corollary).
Product-form route (diagnostic display; not used to close (P+)): with the locked value \(C_\psi^{(H^1)}=0.2400\) and \(C_{\mathrm{box}}^{(\zeta)}=\CboxZeta\), we have
\[ M_\psi= \tfrac{4}{\pi}\,C_\psi^{(H^1)}\sqrt{C_{\mathrm{box}}^{(\zeta)}}\ =\ \Mpsilocked,\qquad \Upsilon_{\mathrm{diag}}=\frac{(2/\pi)\cdot \Mpsilocked}{c_0}=\UpsilonLocked.\]
\subsection*{PSC certificate (locked constants; canonical form)}
\noindent\textit{Locked evaluation used throughout (revised; product route via $\Upsilon$):}
\begin{align*}
 (c_0,\ C_H,\ C_\psi^{(H^1)},\ C_{\mathrm{box}})
 &\ =\ (0.17620819,\ 2/\pi,\ 0.2400,\ \CboxZeta),\\
 M_\psi\ &\ =\ \Mpsilocked,\\
 \Upsilon_{\mathrm{diag}}\ &\ =\ \frac{(2/\pi)\cdot \Mpsilocked}{0.17620819}\ =\ \UpsilonLocked. 
\end{align*}
See Appendices~\ref{app:CE-constant}--\ref{app:Cpsi-compute} for derivations and enclosures.
\paragraph{Reproducible numerics (self-contained).}
For the printed window and the \(\zeta\)–normalized route:
\begin{itemize}
\item \(c_0(\psi)\): Poisson plateau infimum (see Appendix~\ref{app:Cpsi-compute}) — exact value with digits
\[ c_0(\psi)=0.17620819. \]
\item \(K_0\): arithmetic tail \(\tfrac14\sum_{p}\sum_{k\ge2} p^{-k}/k^2\) with explicit tail enclosure — locked
\[ K_0=0.03486808. \]
\item \(K_\xi\): Neutralized Whitney–box \(\xi\) energy via annular $L^2$ + VK zero–density — locked (outward-rounded)
% Avoid tautology in symbolic mode; state definition/link only
\[ K_\xi \text{ is the neutralized Whitney energy (see Lemma~\ref{lem:carleson-xi}).} \]
\item \(C_{\mathrm{box}}^{(\zeta)}\): $=K_0+K_\xi$ — used in certificate only
\[ C_{\mathrm{box}}^{(\zeta)}=\CboxZeta. \]
\item \(C_\psi^{(H^1)}\): analytic enclosure $<0.245$ and quadrature $0.23973\pm3\times10^{-4}$; we lock
\[ C_\psi^{(H^1)}=0.2400. \]
\item \(M_\psi\): Fefferman–Stein/Carleson embedding
\[ M_\psi=\tfrac{4}{\pi}\,C_\psi^{(H^1)}\,\sqrt{C_{\mathrm{box}}^{(\zeta)}}\ =\ \Mpsilocked. \]
\item \(\Upsilon\): product certificate value (no prime budget)
\[ \Upsilon_{\mathrm{diag}}\ =\ \frac{(2/\pi)\cdot \Mpsilocked}{0.17620819}\ =\ \UpsilonLocked. \]
\end{itemize}
Each number is computed once and locked with outward rounding. The certificate wedge uses only \(c_0(\psi),\,C(\psi),\,C_{\rm box}^{(\zeta)}\) and the a.e. boundary passage.
\paragraph{Constants table (for quick reference).}
\begin{center}
\begin{tabular}{ll}
\toprule
Symbol & Value/definition \\
\midrule
$c_0(\psi)$ & $\czeroplateau$ (Poisson plateau; see Appendix~\ref{app:Cpsi-compute}) \\
$C_H(\psi)$ & $\CHone$ (Hilbert envelope; analytic envelope used) \\
$C_\psi^{(H^1)}$ & $\CpsiHone$ (locked from quadrature) \\
$K_0$ & $0.03486808$ (arithmetic tail; see Lemma~\ref{lem:carleson-arith}) \\
$K_\xi$ & $\Kxi$ (neutralized Whitney energy) \\
$C_{\mathrm{box}}^{(\zeta)}$ & $\CboxZeta=K_0+K_\xi$ \\
$M_\psi$ & $\Mpsilocked=(4/\pi)\,C_\psi^{(H^1)}\sqrt{C_{\mathrm{box}}^{(\zeta)}}$ \\
\(\Upsilon_{\mathrm{diag}}\) & $\UpsilonLocked=((2/\pi)\,M_\psi)/c_0$ \quad(\emph{diagnostic})\\
\bottomrule
\end{tabular}
\end{center}
\paragraph{Non-circularity (sequencing).}
We first enclose \(K_\xi\) unconditionally from annular $L^2$ and zero–counts, independent of \(M_\psi\). We then evaluate \(M_\psi\) via \((4/\pi)\,C_\psi^{(H^1)}\sqrt{C_{\mathrm{box}}^{(\zeta)}}\) using the locked \(C_{\mathrm{box}}^{(\zeta)}=K_0+K_\xi\). No step uses \(M_\psi\) to bound \(K_\xi\), so there is no feedback.
% ================================================================
%  Stage 2 Closure: PSC ⇒ (P+) and PSC from a locked certificate
% ================================================================

\subsection*{Definitions and standing normalizations}

Let $\Omega:=\{s\in\C:\ \Re s>\tfrac12\}$ and write $s=\tfrac12+it$ on the boundary.
Set
Let $\Poisson_b(x):=\frac{1}{\pi}\frac{b}{b^2+x^2}$ and let $\mathcal H$ denote the boundary Hilbert transform.

\paragraph{Poisson lower bound.}
Define
\[
 c_0(\psi)\ :=\ \inf_{0<b\le 1,\ |x|\le 1}\ (\Poisson_{b}*\psi)(x)\ \ge\ 0.1762081912\,.
\]
For the printed flat--top window this is locked as

\subsection*{Product certificate $\Rightarrow$ boundary wedge and (P+)}
\noindent\textit{Route status (optional).} This subsection records the boundary-wedge formulation \textup{(P+)} and the Whitney-local phase-mass bounds supplied by the product certificate. A full \emph{global} a.e.\ wedge after a single rotation still requires an additional local-to-global upgrade (Remark~\ref{rem:wedge-application}). The main Schur-pinch route in this manuscript does \emph{not} rely on \textup{(P+)}.

Fix the printed even $C^\infty$ flat--top window $\psi$ with $\psi\equiv 1$ on $[-1,1]$ and $\operatorname{supp}\psi\subset[-2,2]$, and set
\[
  \varphi_{L,t_0}(t)\ :=\ \frac{1}{L}\,\psi\!\left(\frac{t-t_0}{L}\right),\qquad
  m_\psi:=\int_\R\psi,\qquad \int_{\R}\!\varphi_{L,t_0}=m_\psi,\quad \operatorname{supp}\varphi_{L,t_0}\subset[t_0-2L,t_0+2L].
\]
In particular, $\varphi_{L,t_0}\equiv L^{-1}$ on $I=[t_0-L,t_0+L]$.
On intervals avoiding critical-line ordinates, the a.e. wedge follows directly from the product certificate without additive constants.
\begin{theorem}[Whitney-local phase-mass bounds from the product certificate (atom-safe)]\label{thm:psc-certificate-stage2}
For every Whitney interval $I=[t_0-L,t_0+L]$ one has the Poisson plateau lower bound
\[
  c_0(\psi)\,\nu\!\big(Q(I)\big)\ \le\ \int_{\R} (-w')(t)\,\varphi_{L,t_0}(t)\,dt.
\]
Moreover, the CR--Green pairing (Lemma~\ref{lem:CR-green-phase}) gives the windowed phase bound
\[
  \int_{\R}\psi_{L,t_0}(t)\,(-w'(t))\,dt\ \le\ C(\psi)\,\Big(\iint_{Q(\alpha'I)} |\nabla U|^2\,\sigma\Big)^{1/2},
\]
and hence, by the Whitney-scale box-energy bound (i.e. the definition of $C_{\rm box}^{(\zeta)}$ for the certificate boxes),
\[
  \int_{\R}\psi_{L,t_0}\,(-w')\ \le\ C(\psi)\,\sqrt{C_{\rm box}^{(\zeta)}}\,L^{1/2}.
\]
\end{theorem}
\begin{proof}
The Poisson plateau lower bound holds for $\varphi_{L,t_0}$ by Lemma~\ref{lem:poisson-plateau} and Theorem~\ref{thm:phase-velocity-quant}. The CR--Green bound is Lemma~\ref{lem:CR-green-phase} (and the Whitney-scale box-energy constant gives the displayed $L^{1/2}$ scaling). This proves the stated Whitney-local bounds. The remaining promotion to a \emph{global} a.e. boundary wedge \textup{(P+)} is the (currently missing) local-to-global step discussed in Remark~\ref{rem:wedge-application}.
\end{proof}
% [archived duplicate removed]

\paragraph{Scaling remark (why the density-point contradiction does not follow).}
The plateau lower bound has the natural $L$ scaling, while the CR--Green/Carleson upper bound scales like $L^{1/2}$. For $0<L<1$ one has $L\le L^{1/2}$, so there is no single-interval contradiction from shrinking $L$ alone. This is why the proof seeks to close \textup{(P+)} via a Whitney--uniform quantitative wedge criterion with $\Upsilon<\tfrac12$; promoting the resulting Whitney-local control to a global a.e.\ wedge after a single rotation is the separate local-to-global step isolated in Remark~\ref{rem:wedge-application}.

\begin{remark}
Let $N(\sigma,T)$ denote the number of zeros with $\Re\rho\ge \sigma$ and $0<\Im\rho\le T$. The Vinogradov–Korobov zero-density estimates give, for some absolute constants $C_0,\kappa>0$, that
\[
  N(\sigma,T)\ \le\ C_0\,T\,\log T\ +\ C_0\,T^{1-\kappa(\sigma-1/2)}\qquad (\tfrac12\le \sigma<1,\ T\ge T_1),
\]
with an effective threshold $T_1$. On Whitney scale $L=c/\log\langle T\rangle$, these bounds imply the annular counts used above with explicit $A,B$ of size $\ll 1$ for each fixed $c,\alpha$. Consequently, one can take
\[
  C_\xi\ \le\ C(\alpha,c)\,\big(C_0+1\big)
\]
in Lemma~\ref{lem:carleson-xi}, where $C(\alpha,c)$ is an explicit polynomial in $\alpha$ and $c$ arising from the annular $L^2$ aggregation (cf. Lemma~\ref{lem:annular-balayage}). We do not need the sharp exponents; any effective VK pair $(C_0,\kappa)$ suffices for a finite $C_\xi$ on Whitney boxes.
\end{remark}

\section*{Lean formalization status (scaffold; not unconditional yet)}

The Lean~4/Mathlib development checks the \emph{logical reduction} and provides a working scaffold for the far-field pinch route. However, the current codebase still contains explicit \texttt{axiom}/\texttt{sorry} placeholders (e.g.\ in \texttt{Stage1/FarFieldDischarge.lean}, \texttt{Stage1/SpectralGapCertificate.lean}, and \texttt{MuCarlesonFromZeroDensity.lean}). It should therefore \emph{not} be read as an unconditional, fully discharged formal proof of RH at this time.

\begin{center}
\begin{tabular}{llp{7cm}}
\toprule
Area & Status in codebase & Main remaining gap(s) \\
\midrule
Stage-1 reduction (far+near $\Rightarrow$ RH) & \emph{proved} & This is a reduction theorem: RH follows from the named far-field and near-field nonvanishing hypotheses. \\
Far-field pinch scaffold & \emph{implemented} & Depends on numerical-certificate and analytic placeholders (spectral gap numerics, det$_2$/outer analyticity and nonvanishing, and standard complex-analytic lemmas currently left as \texttt{sorry}). \\
Near-field B2$'$ route & \emph{implemented} & Depends on a zero-density input and supporting estimates; parts are currently axiomatized in the Lean development. \\
\bottomrule
\end{tabular}
\end{center}

\noindent In particular, the Lean endpoint should be interpreted as a machine-checked statement of the \emph{dependency structure}, mirroring the conditional theorem in this manuscript (Theorem~\ref{thm:final-rh}).

% References
\begin{thebibliography}{99}
\bibitem{AmbrosioFuscoPallara} L. Ambrosio, N. Fusco, and D. Pallara, \emph{Functions of Bounded Variation and Free Discontinuity Problems}, Oxford Mathematical Monographs, Oxford University Press, Oxford, 2000. (BV compactness/Helly selection.)
\bibitem{Donoghue} W.~F. Donoghue, Jr., \emph{Monotone Matrix Functions and Analytic Continuation}, Springer, New York, 1974. (Pick/Herglotz functions and positivity.)
\bibitem{DurenHp} P.~L. Duren, \emph{Theory of $H^p$ Spaces}, Academic Press, New York, 1970; reprint, Dover Publications, Mineola, NY, 2000. (Hardy/Smirnov background.)
\bibitem{Dusart2010} P. Dusart, Estimates of some functions over primes without Riemann Hypothesis, arXiv:1002.0442, 2010. (Explicit prime-sum bounds; alternative to Rosser--Schoenfeld.)
\bibitem{FeffermanStein1972} C. Fefferman and E.~M. Stein, $H^p$ spaces of several variables, \emph{Acta Math.} 129 (1972), 137--193. (Fefferman--Stein theory; area/square functions and $H^1$--BMO.)
\bibitem{Garnett} J.~B. Garnett, \emph{Bounded Analytic Functions}, Graduate Texts in Mathematics, vol.~236, revised 1st ed., Springer, New York, 2007. (Thm. VI.1.1: Carleson embedding; Thm. II.4.2: boundary uniqueness; Ch. IV: H$^1$–BMO.)
\bibitem{Ivic} A. Ivi\'c, \emph{The Riemann Zeta-Function: Theory and Applications}, Dover Publications, Mineola, NY, 2003. (Thm. 13.30: VK zero-density, used parametrically.)
\bibitem{RosserSchoenfeld1962} J.~B. Rosser and L. Schoenfeld, Approximate formulas for some functions of prime numbers, \emph{Illinois J. Math.} 6 (1962), no.~1, 64--94. (Explicit bounds; e.g. $\pi(t)\le 1.25506\,t/\log t$ for $t\ge 17$.)
\bibitem{RosserSchoenfeld1975} J.~B. Rosser and L. Schoenfeld, Sharper bounds for the Chebyshev functions $\theta(x)$ and $\psi(x)$, \emph{Math. Comp.} 29 (1975), no.~129, 243--269. (Refined explicit prime bounds.)
\bibitem{RosenblumRovnyak} M. Rosenblum and J. Rovnyak, \emph{Hardy Classes and Operator Theory}, Dover Publications, Mineola, NY, 1997. (Ch. 2: outer/inner and boundary transforms.)
\bibitem{RudinRCA} W. Rudin, \emph{Real and Complex Analysis}, 3rd ed., McGraw--Hill, New York, 1987. (Removable singularities; Poisson integrals.)
\bibitem{SarasonSubHardy} D. Sarason, \emph{Sub-Hardy Hilbert Spaces in the Unit Disk}, John Wiley \& Sons, Inc., New York, 1994. (Schur/Cayley background.)
\bibitem{NagyFoiasContractions} B. Sz.-Nagy and C. Foia\c s, \emph{Harmonic Analysis of Operators on Hilbert Space}, North-Holland Publishing Co., Amsterdam--London; American Elsevier Publishing Co., Inc., New York, 1970. (Contractions; Julia operators and unitary colligations.)
\bibitem{SimonTrace} B. Simon, \emph{Trace Ideals and Their Applications}, 2nd ed., Mathematical Surveys and Monographs, vol.~120, American Mathematical Society, Providence, RI, 2005. (Hilbert--Schmidt determinants and continuity.)
\bibitem{SteinHA} E.~M. Stein, \emph{Harmonic Analysis: Real-Variable Methods, Orthogonality, and Oscillatory Integrals}, Princeton University Press, Princeton, NJ, 1993. (Poisson/Hilbert transform on $\mathbb R$; square functions.)
\bibitem{Titchmarsh} E.~C. Titchmarsh, \emph{The Theory of the Riemann Zeta-Function}, 2nd ed., revised by D.~R. Heath-Brown, Oxford University Press, Oxford, 1986. (RvM, zero-density background in Ch. VIII--IX.)
\bibitem{IwaniecKowalski} H. Iwaniec and E. Kowalski, \emph{Analytic Number Theory}, Amer. Math. Soc. Colloquium Publications, vol.~53, Amer. Math. Soc., Providence, RI, 2004.
\bibitem{MontgomeryVaughan} H.~L. Montgomery and R.~C. Vaughan, \emph{Multiplicative Number Theory I. Classical Theory}, Cambridge Studies in Advanced Mathematics, vol.~97, Cambridge Univ. Press, Cambridge, 2007.
\bibitem{DavenportMNT} H. Davenport, \emph{Multiplicative Number Theory}, 3rd ed., revised by H.~L. Montgomery, Graduate Texts in Mathematics, vol.~74, Springer-Verlag, New York, 2000.
\bibitem{KoosisLI} P. Koosis, \emph{The Logarithmic Integral I}, Cambridge Studies in Advanced Mathematics, vol.~12, Cambridge Univ. Press, Cambridge, 1988.
\bibitem{Hoffman} K. Hoffman, \emph{Banach Spaces of Analytic Functions}, Dover Publications, Mineola, NY, 2007. (Reprint of the 1962 Prentice--Hall edition.)
\bibitem{CarlesonCorona} L. Carleson, Interpolation by bounded analytic functions and the corona problem, \emph{Ann. of Math.} (2) 76 (1962), 547--559.
\bibitem{SteinSingInt} E.~M. Stein, \emph{Singular Integrals and Differentiability Properties of Functions}, Princeton Mathematical Series, no.~30, Princeton Univ. Press, Princeton, NJ, 1970.
\bibitem{Grafakos} L. Grafakos, \emph{Classical Fourier Analysis}, 3rd ed., Graduate Texts in Mathematics, vol.~249, Springer, New York, 2014.
\bibitem{NISTDLMF} F.~W.~J. Olver, D.~W. Lozier, R.~F. Boisvert, and C.~W. Clark (eds.), \emph{NIST Digital Library of Mathematical Functions}, National Institute of Standards and Technology, Washington, DC, 2010. Available at \url{https://dlmf.nist.gov/}.
\bibitem{Edwards} H.~M. Edwards, \emph{Riemann's Zeta Function}, Academic Press, New York, 1974; reprint, Dover Publications, Mineola, NY, 2001.
\bibitem{AglerMcCarthy} J. Agler and J.~E. McCarthy, \emph{Pick Interpolation and Hilbert Function Spaces}, Graduate Studies in Mathematics, vol.~44, Amer. Math. Soc., Providence, RI, 2002.
\bibitem{Pick1916} G. Pick, \emph{Über die Beschränkungen analytischer Funktionen, welche durch vorgegebene Funktionswerte bewirkt werden}, Math. Ann. 77 (1916), 7--23.
\bibitem{GohbergKrein} I.~C. Gohberg and M.~G. Krein, \emph{Introduction to the Theory of Linear Nonselfadjoint Operators}, Translations of Mathematical Monographs, vol.~18, American Mathematical Society, Providence, RI, 1969.
\end{thebibliography}

\end{document}
