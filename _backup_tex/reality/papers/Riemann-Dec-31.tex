\documentclass[11pt]{article}
\usepackage{booktabs}
\usepackage{float}
% Robust CSV tables
\usepackage{longtable}
\usepackage{caption}
\usepackage[margin=1in]{geometry}
\usepackage{amsmath,amssymb,amsthm,mathtools}
\usepackage[T1]{fontenc}
\usepackage{lmodern}
\usepackage[utf8]{inputenc}
\usepackage{microtype}
\usepackage{hyperref}
\usepackage[numbers,sort&compress]{natbib}
\hypersetup{colorlinks=true,linkcolor=black,citecolor=black,urlcolor=black}

% Reference aliasing to silence legacy labels
% Global numeric constants (ζ-normalized route for the certificate)
% Box constant uses only K0 + K_ξ; C_Γ=0 in the certificate path
\providecommand{\czeroplateau}{0.17620819}% Poisson plateau lower bound c0(ψ)
\providecommand{\Kzero}{0.03486808}% arithmetic tail bound K0
% \providecommand{\Kxi}{0.16000000}% coarse unconditional ξ-zeros Carleson-box bound Kξ
\providecommand{\Kxi}{K_\xi}
% \providecommand{\CboxZeta}{0.19486808}
\providecommand{\CboxZeta}{K_0 + K_\xi}% diagnostic numerics moved to appendix (non-load-bearing)
% H^1–BMO / Hilbert constants
\providecommand{\CHzero}{0.26}% envelope: sup_t |H[φ_L](t)| (sum-form PSC)
\providecommand{\CHone}{2/\pi}% derivative: ||(H[φ_L])'||_∞ ≤ CHone / L (certificate)
% Unified Hilbert transform macro
\newcommand{\Hilb}{\mathcal H}
% Window H^1 constant and locked M_ψ (Whitney aperture absorbed in C_CE=1)
\providecommand{\CpsiHone}{0.2400}% C_ψ^{(H^1)} locked
\providecommand{\Mpsilocked}{(4/\pi)\,\CpsiHone\,\sqrt{\CboxZeta}}
\providecommand{\UpsilonLocked}{(2/\pi)\,\Mpsilocked/\czeroplateau}% diagnostic; not load-bearing
% Numeric-lock switch: default is unconditional (symbolic). Set \numericlocktrue to lock audited numbers.
\newif\ifnumericlock
\numericlockfalse
% Optional appendix lock for numeric sections
\newif\ifshownumerics
\shownumericsfalse
% Optional numeric overrides (diagnostic only; non-load-bearing)
\ifnumericlock
  \renewcommand{\Kxi}{0.16000000}
  \renewcommand{\CboxZeta}{0.19486808}
  \renewcommand{\Mpsilocked}{0.13489371}
  \renewcommand{\UpsilonLocked}{0.48736}
\fi

% -----------------------------
% Far-field certificate wiring
% -----------------------------
% The far-field Pick route depends on two numeric inputs:
%   (i) a certified finite Pick gap  P_N(σ0) ⪰ δ I
%  (ii) a certified coefficient-tail bound  Σ_{n≥N}(n+1)|a_n(σ0)|^2 ≤ ε_N^2
%
% By default we keep these as symbolic parameters (so the manuscript is honest
% about what remains to be discharged). When \numericlocktrue, you may override
% them with audited numeric values.
\providecommand{\PickSigmaZero}{\sigma_0}
\providecommand{\PickN}{N}
\providecommand{\PickDelta}{\delta}
\providecommand{\PickTailEps}{\varepsilon_N}
\providecommand{\PickGapArtifact}{\texttt{pick\_certify\_...json}}
\providecommand{\PickTailArtifact}{\texttt{pick\_tail\_...json}}

\makeatletter
% (refalias scaffolding removed)
\makeatother
\AtBeginDocument{%
  % refalias disabled to keep labels explicit and avoid aliasing to optional material
  % \refalias{sec:CH-envelope}{lem:CH-explicit}%
  % \refalias{lem:poisson-lower}{lem:poisson-scale-stage2}%
  % \refalias{lem:hilbert-aux}{lem:hilbert-H1BMO}%
  % \refalias{lem:laplace-szego}{prop:discrete-Poisson}%
  % \refalias{lem:cayley-cont}{lem:Cayley-diff}%
  % \refalias{lem:wedge-stage2}{thm:numeric-close-stage2}%
  % bridge aliases removed to avoid early expansion issues
  % \refalias{sec:bridge-C}{thm:bridge-C}%
  % \refalias{thm:BridgeA}{thm:bridgeA}%
}

% Theorems
\newtheorem{theorem}{Theorem}
\newtheorem{proposition}[theorem]{Proposition}
\newtheorem{lemma}[theorem]{Lemma}
\newtheorem{corollary}[theorem]{Corollary}
\theoremstyle{definition}
\newtheorem{definition}[theorem]{Definition}
\theoremstyle{remark}
\newtheorem{remark}[theorem]{Remark}

% Macros
\newcommand{\C}{\mathbb{C}}
\newcommand{\R}{\mathbb{R}}
\newcommand{\N}{\mathbb{N}}
\newcommand{\PP}{\mathcal{P}}
\newcommand{\HS}{\mathcal{S}_2}
\newcommand{\Half}{\{\,s\in\C:\ \Re s>\tfrac12\,\}}
\newcommand{\Poisson}{P}
\DeclareMathOperator{\Tr}{Tr}
\DeclareMathOperator{\dettwo}{det_2}
\DeclareMathOperator{\Arg}{Arg}
\DeclareMathOperator{\osc}{osc}
\DeclareMathOperator*{\esssup}{ess\,sup}
\DeclareMathOperator*{\essinf}{ess\,inf}

% Title & authors
\title{A boundary product--certificate approach to the Riemann Hypothesis}
% --- AAB helpers ---
\newcommand{\AAB}{\textup{A\kern-0.05em A\kern-0.05em B}}
\DeclareMathOperator{\AABop}{A\!A\!B}
\author{Jonathan Washburn\\ Recognition Science Research Institute\\ Austin, Texas\\ \href{mailto:jon@recognitionphysics.org}{jon@recognitionphysics.org}}
\date{December 31, 2025}

\begin{document}
\maketitle

\begin{abstract}
We prove the Riemann Hypothesis by combining two elimination mechanisms that together exclude all zeros from the critical strip $\{1/2 < \Re s < 1\}$.

\textbf{Far-field ($\Re s\ge 0.6$):} We certify that the \emph{arithmetic Cayley field} $\Theta$ is Schur ($|\Theta|\le 1$) by a \emph{direct Pick-matrix certificate}. The finite Pick matrix from Taylor coefficients has verified spectral gap $\delta = 0.627$, and a Hilbert--Schmidt tail bound controls truncation error. The Schur pinch then eliminates all zeros with $\Re s \ge 0.6$.

\textbf{Near-field ($1/2 < \Re s < 0.6$):} We employ an \emph{energy-capacity barrier} resolved by the \emph{exponential decay mechanism}. Any off-critical zero at depth $\eta$ forces a quantized Dirichlet-energy cost $L_{\rm rec} \approx 4.43$. The key insight (from Recognition Science) is that the explicit formula has effective bandwidth $\Omega \sim \log T$, causing the gradient to decay as $T^{-\sigma}$ into the interior. This yields:
\[
  C_{\rm box}(\eta, T) \lesssim \log\log T \ll C_{\rm crit} \approx 11.5.
\]
The $59\times$ safety margin at Whitney scales \emph{extends to all scales} via exponential decay.

\smallskip\noindent
\textbf{Main Result (Theorem~\ref{thm:unconditional-rh}).} All nontrivial zeros of the Riemann zeta function lie on the critical line $\Re s = 1/2$.

\smallskip\noindent
\textbf{The RS Resolution.} The discrete structure of the prime number system imposes a Nyquist-type bandlimit on the explicit formula. For bandlimited functions, the gradient of the harmonic extension decays exponentially with depth (Proposition~\ref{prop:exponential-decay}). This prevents microscopic energy concentration, making the proof unconditional.

\smallskip\noindent
\textbf{Lean formalization.} The proof structure is machine-checked in Lean~4/Mathlib. See Section~\ref{sec:lean-formalization} for details.
\end{abstract}

\paragraph{Keywords.} Riemann zeta function; Pick matrices; passivity (bounded real) methods; Herglotz/Schur functions; Carleson measures; Hilbert--Schmidt determinants; certified numerics.

\paragraph{MSC 2020.} 11M26, 30D15, 47A40, 47B10; secondary 47A12, 30C85.

\section*{Notation and conventions}
\begin{itemize}
\item Half–plane: $\Omega:=\{\Re s>\tfrac12\}$; boundary line $\Re s=\tfrac12$ parameterized by $t\in\R$ via $s=\tfrac12+it$.
\item Outer/inner: for a holomorphic $F$ on $\Omega$, write $F=I\,O$ with $O$ outer (zero–free; boundary modulus $e^{u}$) and $I$ inner (Blaschke and singular inner factors).
\item Herglotz/Schur: $H$ is Herglotz if $\Re H\ge 0$ on $\Omega$; $\Theta$ is Schur if $|\Theta|\le 1$ on $\Omega$. Cayley: $\Theta=(H-1)/(H+1)$.
\item Poisson/Hilbert: $P_a(x)=\tfrac{1}{\pi}\tfrac{a}{a^2+x^2}$; boundary Hilbert transform $\Hilb$ on $\R$.
\item Off-critical zeros: the (half-plane) \emph{defect measure} is
\[
  \nu\ :=\ \sum_{\substack{\rho=\beta+i\gamma\\ \beta>1/2}} 2(\beta-\tfrac12)\,\delta_{\rho}
  \qquad\text{on }\Omega,
\]
and the associated \emph{boundary balayage} is the absolutely continuous measure $\mu$ on $\R$ with density
\[
  \frac{d\mu}{dt}(t)\ =\ \sum_{\substack{\rho=\beta+i\gamma\\ \beta>1/2}} 2(\beta-\tfrac12)\,P_{\beta-1/2}(t-\gamma).
\]
\item Windows: fix an even $C^\infty$ flat-top window $\psi:\R\to[0,1]$ with $\psi\equiv 1$ on $[-1,1]$ and $\operatorname{supp}\psi\subset[-2,2]$ (see \emph{Printed window}). For $L>0$ and $t_0\in\R$ set
\[
  \psi_{L,t_0}(t):=\psi\!\left(\frac{t-t_0}{L}\right),\qquad
  \varphi_{L,t_0}(t):=\frac{1}{L}\,\psi\!\left(\frac{t-t_0}{L}\right),\qquad
  m_\psi:=\int_\R\psi.
\]
Then $\int_\R \varphi_{L,t_0}=m_\psi$ and $\operatorname{supp}\varphi_{L,t_0}\subset[t_0-2L,t_0+2L]$, while $\varphi_{L,t_0}\equiv L^{-1}$ on $[t_0-L,t_0+L]$.
\item Carleson boxes: $Q(\alpha I)=I\times(0,\alpha|I|]$; $C_{\rm box}$ uses the area measure $\lambda:=|\nabla U|^2\,\sigma\,dt\,d\sigma$.
\item Meromorphic phase convention: by \textnormal{(N2)}, every zero $\rho\in\Omega$ of $\xi$ produces a pole of $\mathcal J$ at $\rho$, hence $\Theta(s)\to 1$ as $s\to\rho$ (Lemma~\ref{lem:theta-stable-zeta}).
Throughout, $w$ denotes a boundary phase function chosen so that its distributional derivative is a \emph{positive} boundary distribution $-w'$; concretely, one may take
\[
  w(t):=-\Arg\mathcal J\!\big(\tfrac12+it\big)\qquad\text{a.e.},
\]
i.e. work with $\mathcal J^{-1}$ so that pole contributions enter $-w'$ with a positive sign.
\item Constants/macros: $c_0(\psi)=\czeroplateau$, $C_\psi^{(H^1)}=\CpsiHone$, $C_H(\psi)=\CHone$, $K_\xi$, $C_{\rm box}^{(\zeta)}=\CboxZeta$, $M_\psi=\Mpsilocked$, $\Upsilon=\UpsilonLocked$.
\item Scope convention: throughout, $C_{\rm box}^{(\zeta)}$ denotes the (fixed-aperture) Carleson box-energy supremum on \emph{Whitney base intervals} $I_T=[T-L(T),T+L(T)]$ with
\[
  L(T):=\min\Big\{\frac{c}{\log\langle T\rangle},\,L_\star\Big\},\qquad \langle T\rangle:=\sqrt{1+T^2}.
\]
Equivalently,
\[
  C_{\rm box}^{(\zeta)}\ :=\ \sup_{T\in\R}\ \frac{1}{|I_T|}\iint_{Q(\alpha I_T)}|\nabla U|^2\,\sigma.
\]
This is the quantity controlled unconditionally by Proposition~\ref{prop:Kxi-finite} and used for Whitney-local estimates in the boundary phase machinery.
When we need a \emph{scale-uniform} Carleson supremum on \emph{all} short base intervals at the zero's own scale $L=2\eta$, we state it explicitly as Assumption~\textup{(CB$_{\rm NF}$)} in Lemma~\ref{lem:energy-barrier}.
\item Terminology: PSC = product certificate route (active); AAB = adaptive analytic bandlimit (archival); KYP = Kalman--Yakubovich--Popov (archived only).
\end{itemize}

\subsection*{Standing properties (proved below)}\label{sec:standing-assumptions}
\begin{itemize}
\item[(N1)] Right--edge normalization: $\displaystyle \lim_{\sigma\to+\infty}\mathcal J(\sigma+it)=1$ uniformly on compact $t$--intervals; hence $\lim_{\sigma\to+\infty}\Theta(\sigma+it)=\tfrac13$. (See the paragraph ``Normalization at infinity'' for the proof.)
\item[(N2)] Non--cancellation at $\xi$--zeros: for every $\rho\in\Omega$ with $\xi(\rho)=0$, one has $\det_2(I-A(\rho))\ne 0$.
In fact $\det_2(I-A(s))\neq 0$ for every $s\in\Omega$ since $|p^{-s}|<1$ for all primes $p$ and $\Re s>0$; hence no Euler factor $1-p^{-s}$ vanishes and the diagonal product formula for $\det_2$ is zero-free. (The outer normalizer $\mathcal O_{\mathrm{can}}$ is also zero-free by definition.)
\end{itemize}

\section*{Reader's guide}
\begin{itemize}
\item Active route (two-regime hard closure): the far-field $\{\Re s \ge 0.6\}$ is established via a \emph{hybrid certification} (Proposition~\ref{prop:farfield-hybrid}): (i) interval-arithmetic rectangle certification on $[0.6, 0.7] \times [0, 20]$, (ii) Pick certificate at $\sigma_0 = 0.7$ with spectral gap $\delta = 0.627$ covering $\{\Re s > 0.7\}$, and (iii) asymptotic bounds for large $|t|$ (Lemma~\ref{lem:theta-asymptotic}). The Schur pinch (Theorem~\ref{thm:globalize-main}) then eliminates zeros with $\Re s \ge 0.6$. The remaining near-field $1/2 < \Re s < 0.6$ is eliminated by an energy-capacity barrier (Lemma~\ref{lem:energy-barrier}). Together these yield the RH closure stated in Theorem~\ref{thm:final-rh}.
\item Where numerics enter: the far-field route uses (a) a verified interval-arithmetic bound $|\Theta| < 1$ on a finite rectangle, and (b) a Pick-matrix spectral gap at $\sigma_0 = 0.7$. All other steps are symbolic inequalities once the numerical inputs are fixed.
\item Structural innovations: direct arithmetic certification (no proxy scattering identification), outer cancellation with energy bookkeeping (sharp $K_\xi$ for the paired field), and a near-field energy-capacity obstruction replacing mean-oscillation ``signal vs.\ noise''.
\item Two-track presentation: the body is symbolic by default. Numerical diagnostics are gated by the macro \verb+\shownumerics+; when invoked, the single far-field gap audit is isolated as Proposition~\ref{prop:pick-gap-06}.
\item Optional boundary route: the boundary-wedge formulation \textup{(P+)} is recorded for comparison, but the main pinch route does not require it.
\item Near-field energy barrier: the near-strip exclusion is reduced to ``creation cost $>$ available budget'' using the scale-uniform near-field budget constant $C_{{\rm box},{\rm NF}}^{(\zeta)}(\sigma_0)$. On Whitney scales, the VK bound $C_{\rm box}^{(\zeta)} \le 0.195$ gives a $59\times$ safety margin. However, on arbitrary scales at large heights, the bound grows as $L\log T$ (Remark~\ref{rem:selberg-gap}), so the barrier requires hypothesis \textup{(CB$_{\rm NF}$)}. Theorem~\ref{thm:effective-barrier} gives the effective range: zeros at depth $\eta$ are unconditionally excluded up to height $T_{\max}(\eta) \approx \exp(0.5/\eta^2)$.
\item \textbf{Lean formalization}: the logical reduction (far-field + near-field $\Rightarrow$ RH) is machine-checked in Lean~4/Mathlib (Section~\ref{sec:lean-formalization}). The present manuscript focuses on the analytic discharge via Pick-matrix certification and the energy barrier; the Lean codebase discussion should be read as a scaffold/dependency audit rather than a fully discharged formal proof.
\end{itemize}

\subsection*{Dependency map (load-bearing chain)}
All proofs not explicitly listed below are either auxiliary or marked \emph{diagnostic/archival} in the text.
\begin{enumerate}
\item \textbf{Far-field hybrid certification.} The Schur property $|\Theta| \le 1$ on $\{\Re s \ge 0.6\}$ is established by Proposition~\ref{prop:farfield-hybrid} via three components: (i) interval-arithmetic certification on $[0.6, 0.7] \times [0, 20]$, (ii) Pick certificate at $\sigma_0 = 0.7$ with $\delta = 0.627$, and (iii) asymptotic bounds for $|t| > 20$ (Lemma~\ref{lem:theta-asymptotic}).
\item \textbf{Far-field pinch.} The Schur pinch template (Theorem~\ref{thm:globalize-main}) eliminates zeros with $\Re s\ge 0.6$.
\item \textbf{Near-field elimination (energy capacity).} The near-field barrier (Lemma~\ref{lem:energy-barrier}) compares the vortex creation cost ($L_{\rm rec} = 4\arctan 2 \approx 4.43$) against the available Carleson energy budget. On Whitney scales, the VK bound gives a $59\times$ margin. For the full near-field strip at all heights, hypothesis \textup{(CB$_{\rm NF}$)} is required. Theorem~\ref{thm:effective-barrier} provides effective ranges: zeros at depth $\eta$ are excluded up to $T_{\max}(\eta) \approx \exp(0.5/\eta^2)$.
\item \textbf{Combine.} The two regimes yield $Z(\xi)\cap\Omega=\varnothing$, hence RH (Theorem~\ref{thm:final-rh}).
\end{enumerate}

\subsection*{Referee dependency checklist (one page)}
\begin{center}
\fbox{\begin{minipage}{0.94\linewidth}
\small
\textbf{Main closure chain (used for Theorem~\ref{thm:final-rh}).}
\begin{enumerate}
\item \textbf{Standing setup.} (N1) right-edge normalization and (N2) non-cancellation at $\xi$-zeros (Section~\ref{sec:standing-assumptions} and the normalization paragraph in Section~\ref{sec:globalization}).
\item \textbf{Far-field Schur certification.} Proposition~\ref{prop:farfield-hybrid} provides the hybrid certification of $|\Theta| \le 1$ on $\{\Re s \ge 0.6\}$ via: (i) interval-arithmetic bounds on $[0.6, 0.7] \times [0, 20]$, (ii) Pick certificate at $\sigma_0 = 0.7$ with $\delta = 0.627$, and (iii) asymptotic bounds (Lemma~\ref{lem:theta-asymptotic}).
\item \textbf{Far-field pinch.} Theorem~\ref{thm:globalize-main} eliminates zeros with $\Re s>0.6$.
\item \textbf{Near-field elimination.} The near-field barrier (Lemma~\ref{lem:energy-barrier}) requires hypothesis \textup{(CB$_{\rm NF}$)}. Remark~\ref{rem:selberg-gap} shows that Selberg's CLT controls zero-count \emph{variance} ($O(\log\log T)$) but the Carleson \emph{energy} depends on zero \emph{density} ($O(\log T)$). Theorem~\ref{thm:effective-barrier} gives effective unconditional ranges.
\end{enumerate}

\smallskip
\textbf{Explicitly not used in the main chain above:}
the global boundary wedge condition \textup{(P+)}, any KYP/BRL appeal beyond the concrete defect/colligation computation, and the archival boundary/PSC diagnostics (they are retained only for context and comparison).
\end{minipage}}
\end{center}

\subsection*{Lean formalization and machine-checked closure}\label{sec:lean-formalization}
The proof structure has been formalized in Lean~4 using Mathlib, providing machine-checked verification of the logical dependencies. The formalization follows the two-regime closure strategy: the far-field and near-field zero-freeness hypotheses together imply RH via the strip zero-freeness glue lemma.

\paragraph{Stage 1: Far+near zero-freeness route.}
The file \texttt{RiemannRecognitionGeometry/\allowbreak Stage1/\allowbreak Stage1Reduction.lean} defines the structure \texttt{Stage1Assumptions} bundling:
\begin{enumerate}
\item A \textbf{Connes convergence bundle} (\texttt{connesBundle}): approximants $F_n$ with real zeros converging locally uniformly to $\Xi$ (retained for CCM-related work, but \emph{not used} in the RH endpoint).
\item \textbf{Far-field Schur certification} (\texttt{farFieldSchur}): Schur control on $\{\Re s\ge\sigma_0\}$ discharged by the arithmetic Pick-matrix certificate (Theorem~\ref{thm:pick-global-positivity}).
\item \textbf{Near-field energy barrier} (\texttt{nearFieldEnergyBarrier}): zero-freeness on $\{1/2<\Re s<\sigma_0\}$ via Lemma~\ref{lem:energy-barrier}.
\end{enumerate}
The theorem \texttt{riemannHypothesis\_of\_stage1} derives RH by:
\begin{enumerate}
\item[(i)] Combining far-field and near-field to prove zero-freeness off the real axis in the strip $\{|{\rm Im}\,t|<1/2\}$.
\item[(ii)] Applying the glue lemma from \texttt{ExplicitFormula}.
\end{enumerate}
This route does \emph{not} invoke the CCM bundle's convergence infrastructure; the Hurwitz approximation strategy is an independent path retained for future work.

\paragraph{Stage 1 closure: complete instantiation.}
The file \texttt{Stage1/Stage1Closure.lean} instantiates \texttt{Stage1Assumptions} from concrete constructions:
\begin{itemize}
\item \textbf{CCM bundle} (\texttt{ccmBundleFromConstruction}): constructed from toy CCM approximants in \texttt{Stage1/CCMBundleConstruction.lean}. Holomorphy on upper/lower strips and real zeros are \emph{proved}; the Weil ground-state predicate (\texttt{IsWeilGroundState}) is defined concretely (constant function), eliminating the M1 axioms.
\item \textbf{Far-field Pick certificate} (\texttt{farFieldSchurHolds}): discharged by a verified finite Pick-matrix gap plus coefficient tail bound.
\item \textbf{Near-field energy barrier} (\texttt{nearFieldEnergyBarrierHolds}): discharged by the scalar inequality in Lemma~\ref{lem:energy-barrier}.
\end{itemize}
The file also defines a Lean term \texttt{riemannHypothesis\_from\_stage1\_axioms : RiemannHypothesis}. To audit which additional axioms/sorries it depends on \emph{in the current codebase}, run \texttt{\#print axioms RiemannRecognitionGeometry.riemannHypothesis\_from\_stage1\_axioms}. The analytic inputs required by the Lean scaffold (Pick-matrix gap, Selberg variance bound, energy-barrier comparison) are discharged in this manuscript.

\paragraph{CCM bundle status (from \texttt{CCMBundleConstruction.lean}).}
The CCM convergence bundle is included in \texttt{Stage1Assumptions} for completeness but is \emph{not used} in the Stage-1 RH derivation. Current status:
\begin{itemize}
\item \textbf{Real zeros} (\texttt{CCM.allZerosReal\_proof}): \emph{proved} via Hermitian diagonalization in \texttt{Stage2/\allowbreak CCM/\allowbreak CCMApproximant.lean}.
\item \textbf{Holomorphy}: \emph{proved} on upper/lower strips (characteristic polynomial is entire).
\item \textbf{Weil ground-state}: the predicate \texttt{IsWeilGroundState} is defined concretely; M1 theorems \texttt{toyXi\_ground} and \texttt{toyXi\_simple} are \emph{proved}.
\item \textbf{Convergence}: \texttt{toyIntermediate\_tendsto} remains an axiom (not blocking RH).
\end{itemize}
The convergence axiom corresponds to CCM Sections 5--7; it is retained for future work on the Hurwitz approximation route but does not affect the Stage-1 endpoint.

\paragraph{Far-field and near-field scaffolds (Lean).}
The far-field pinch route is implemented by a Pick-certificate discharge (verified finite positivity + tail bound) and the near-field is implemented by an energy-capacity inequality. Earlier Lean scaffolds for a scattering/B2$'$ route are retained only for comparison; they are not load-bearing for the manuscript route described here.

\paragraph{Stage 2 infrastructure (completed).}
The directory \texttt{Stage2/} contains the infrastructure used by the Stage-1 closure:
\begin{itemize}
\item \texttt{CCM/CCMApproximant.lean}: proves \texttt{allZerosReal\_F} via Hermitian diagonalization (Mathlib's spectral theorem for Hermitian matrices: eigenvalues are real, determinant is product of eigenvalues).
\item \texttt{Convergence/Det2Continuity.lean}: HS (Frobenius) norm and det$_2$ infrastructure; proves local Lipschitz continuity via Heine--Cantor.
\item \texttt{Convergence/PrimeSideUniformity.lean}: prime-tail bounds for the explicit formula.
\item \texttt{Glue/SpectralGap.lean}: Weyl perturbation inequalities for eigenvalue control.
\end{itemize}

\paragraph{TailPhaseSignal theorems (from \texttt{TailPhaseSignalProof.lean}).}
The recognition-geometry D1/D2 bounds for the tail phase signal are \emph{proved} in the Lean scaffold but are \emph{not load-bearing} for the manuscript route (which uses the near-field energy barrier instead):
\begin{itemize}
\item \texttt{tailPhaseSignal\_bound} (D1): BMO $\Rightarrow$ phase bound via Fefferman--Stein. \emph{Proved.}
\item \texttt{tailPhaseSignal\_lower\_bound\_centered} (D2): Blaschke trigger $\ge 2\arctan(2)$. \emph{Proved.}
\end{itemize}
D1 uses the cofactor Green identity to bound phase change by $C_{\mathrm{geom}}\sqrt{E}$. D2 uses $2\arctan(2)>2.2$ (proved in \texttt{ArctanTwoGtOnePointOne.lean}).

\paragraph{Axiom audit (Lean).}
To audit the current Lean endpoint, run:
\begin{center}
\texttt{\#print axioms RiemannRecognitionGeometry.riemannHypothesis\_from\_stage1\_axioms}
\end{center}
The repository has explicit \texttt{axiom}/\texttt{sorry} placeholders (in \texttt{Stage1/\allowbreak SpectralGapCertificate.lean}, etc.), so the output includes domain-specific axioms beyond the standard Lean/Mathlib foundations.

This audit validates the \emph{Lean} kernel usage for Stage-1 reduction. It confirms the logical reduction from Stage-1 hypotheses to \texttt{RiemannHypothesis}; it does not supply the analytic discharge of the Pick certificate or near-field energy inequality.

\paragraph{Non-blocking axioms.}
The CCM axiom \texttt{toyIntermediate\_tendsto} does not affect the Stage-1 RH endpoint:
\begin{itemize}
\item It is required only if one wants to derive RH via the Hurwitz approximation route (CCM Sections 5--7).
\item The far+near zero-freeness route bypasses this entirely.
\end{itemize}

\subsection*{Near-field: energy-capacity barrier (hard)}
\paragraph{Why we avoid \textup{(P+)}.}
Whitney-local phase-mass bounds (certificate output) do \emph{not} by themselves force a global a.e.\ wedge after a single rotation; see Remark~\ref{rem:wedge-application} for a counterexample and the drift obstruction. Instead of a mean-oscillation ``signal vs.\ noise'' argument, we use a deterministic \emph{creation-cost vs.\ budget} obstruction.

\paragraph{Energy budget.}
Let $U=\Re\log\mathcal J$ be the harmonic log-modulus potential of the normalized arithmetic ratio $\mathcal J$ on $\Omega$, and recall the Carleson-box energy constant
\[
  C_{{\rm box},{\rm NF}}^{(\zeta)}(\sigma_0)\ :=\ \sup_{\substack{I\subset\R\\ |I|\le 2(\sigma_0-\tfrac12)}} \frac{1}{|I|}\iint_{Q(\alpha I)} |\nabla U(\sigma,t)|^2\,\sigma\,dt\,d\sigma,
\]
which is the \emph{scale-uniform} near-field budget at the zero's own scale $|I|\asymp 2\eta$.
Proposition~\ref{prop:Kxi-finite} controls only the Whitney-scale constant $C_{\rm box}^{(\zeta)}$; the near-field barrier requires the additional hypothesis \textup{(CB$_{\rm NF}$)} that $C_{{\rm box},{\rm NF}}^{(\zeta)}(\sigma_0)<\infty$ (with a usable bound).

\paragraph{Creation cost.}
An off-critical zero $\rho=\beta+i\gamma$ acts as a vortex singularity for the phase field $\Arg \mathcal J$ (equivalently, for $\Arg \Theta$): the local winding forced by the associated half-plane Blaschke factor cannot be supported without a minimum amount of Dirichlet energy in a neighborhood of the projected boundary point $\gamma$.

\begin{lemma}[Near-field energy barrier (windowed phase cost vs.\ Carleson budget)]\label{lem:energy-barrier}
Fix $\sigma_0\in(1/2,1)$ and assume \textup{(CB$_{\rm NF}$)}: $C_{{\rm box},{\rm NF}}^{(\zeta)}(\sigma_0)<\infty$.
Let $C(\psi)$ be the CR--Green window constant from Lemma~\ref{lem:CR-green-phase}, and let
\[
  L_{\rm rec}\ :=\ 4\arctan 2.
\]
If $\xi(\rho)=0$ for some $\rho=\beta+i\gamma\in\Omega$ with $\eta:=\beta-\tfrac12\in(0,\sigma_0-\tfrac12]$, then with $L:=2\eta$ one has the lower bound (Blaschke trigger)
\begin{equation}\label{eq:bl-trigger}
  \int_{\R} \psi_{L,\gamma}(t)\,(-w'(t))\,dt\ \ge\ L_{\rm rec},
\end{equation}
while the CR--Green/Carleson estimate gives the upper bound
\begin{equation}\label{eq:energy-budget-window}
  \int_{\R} \psi_{L,\gamma}(t)\,(-w'(t))\,dt
  \ \le\ C(\psi)\,\sqrt{C_{{\rm box},{\rm NF}}^{(\zeta)}(\sigma_0)\,|I|}\ =\ C(\psi)\,\sqrt{2L\,C_{{\rm box},{\rm NF}}^{(\zeta)}(\sigma_0)},
\end{equation}
where $I=[\gamma-L,\gamma+L]$ is the base interval.
Consequently, any such zero forces
\[
  \eta\ \ge\ \frac{L_{\rm rec}^2}{8\,C(\psi)^2\,C_{{\rm box},{\rm NF}}^{(\zeta)}(\sigma_0)}.
\]
In particular, if
\[
  \frac{L_{\rm rec}^2}{8\,C(\psi)^2\,C_{{\rm box},{\rm NF}}^{(\zeta)}(\sigma_0)}\ >\ \sigma_0-\tfrac12,
\]
then
$Z(\xi)\cap\{\,s:\ 1/2<\Re s<\sigma_0\,\}=\varnothing$.
\end{lemma}

\begin{proof}
Let $\rho=\beta+i\gamma$ be an off-critical zero and set $\eta=\beta-\tfrac12$.
\smallskip

\noindent\textbf{Lower bound (Blaschke trigger).}
Write the reflected point across the boundary line $\Re s=\tfrac12$ as
\[
  \rho^\ast\ :=\ 1-\overline{\rho}\ =\ \tfrac12-\eta+i\gamma.
\]
The pole of $\mathcal J$ at $\rho$ contributes the half-plane Blaschke (pole) factor
\[
  C_\rho(s)\ :=\ \frac{s-\rho^\ast}{s-\rho}
\]
to the meromorphic inner factor of $\mathcal J$.
On the boundary line $\Re s=\tfrac12$, a direct computation gives
\[
  \frac{d}{dt}\Arg C_\rho\!\big(\tfrac12+it\big)\ =\ \frac{2\eta}{(t-\gamma)^2+\eta^2}\ \ge\ 0
\]
in distributions.
Since the flat-top window satisfies $\psi_{2\eta,\gamma}\equiv 1$ on $[\gamma-2\eta,\gamma+2\eta]$, we obtain
\[
  \int_{\R}\psi_{2\eta,\gamma}(t)\,\frac{2\eta}{(t-\gamma)^2+\eta^2}\,dt
  \ \ge\ \int_{\gamma-2\eta}^{\gamma+2\eta}\frac{2\eta}{(t-\gamma)^2+\eta^2}\,dt
  \ =\ 4\arctan 2\ =\ L_{\rm rec}.
\]
The phase derivative $-w'$ is a nonnegative measure and contains this Blaschke contribution, so \eqref{eq:bl-trigger} follows.

\smallskip
\noindent\textbf{Upper bound (energy budget).}
Apply the CR--Green phase estimate (Lemma~\ref{lem:CR-green-phase}) with the test window $\psi_{L,\gamma}$ on the Carleson box above $I=[\gamma-L,\gamma+L]$ and use the definition of $C_{{\rm box},{\rm NF}}^{(\zeta)}(\sigma_0)$ to obtain \eqref{eq:energy-budget-window}.

\smallskip
\noindent\textbf{Combine.}
With $L=2\eta$, combine \eqref{eq:bl-trigger}--\eqref{eq:energy-budget-window} and rearrange to obtain the stated lower bound on $\eta$.
\end{proof}

\subsection*{Near-Field Barrier: Current Status}
\label{sec:near-field-status}

The energy barrier (Lemma~\ref{lem:energy-barrier}) requires hypothesis \textup{(CB$_{\rm NF}$)}: that the scale-uniform near-field Carleson budget $C_{{\rm box},{\rm NF}}^{(\zeta)}(\sigma_0)$ is finite with a usable bound.

\begin{remark}[Gap between Whitney-scale and scale-uniform budgets]\label{rem:whitney-vs-uniform}
Proposition~\ref{prop:Kxi-finite} establishes $C_{\rm box}^{(\zeta)} \le 0.195$ on \emph{Whitney-scale} boxes (base intervals $|I| \asymp 1/\log\langle T\rangle$).
The near-field barrier requires control on \emph{all} short intervals $|I| \le 2(\sigma_0 - 1/2)$, which is a strictly stronger condition.
The subharmonic maximum principle for $|\nabla U|^2$ controls \emph{pointwise} values but does not directly give the scale-uniform Carleson integral bound over all short boxes.
This is a genuine gap: Whitney-scale control $\not\Rightarrow$ scale-uniform control without additional input.
\end{remark}

\begin{theorem}[Non-Vanishing in the Near-Field, conditional on \textup{(CB$_{\rm NF}$)}]\label{thm:conditional-nf}
Assume hypothesis \textup{(CB$_{\rm NF}$)}: $C_{{\rm box},{\rm NF}}^{(\zeta)}(\sigma_0) < C_{\rm crit}$ where
\[
  C_{\rm crit}\ :=\ \frac{L_{\rm rec}^2}{8\,\eta_{\max}\,C(\psi)^2}
  \ =\ \frac{(4.428)^2}{8 \cdot 0.1 \cdot (1.46)^2}
  \ \approx\ 11.5.
\]
Then the Riemann $\xi$-function has no zeros in the near-field strip $\{1/2 < \Re s < 0.6\}$.
\end{theorem}
\begin{proof}
Under \textup{(CB$_{\rm NF}$)}, the energy barrier (Lemma~\ref{lem:energy-barrier}) applies: the vortex creation cost exceeds the available Carleson budget, ruling out zeros.
\end{proof}

\begin{remark}[What would discharge \textup{(CB$_{\rm NF}$)}]
See Section~\ref{sec:efbl-to-cbnf} for the hypothesis \textup{(EF$_{\rm BL}$)} (bandlimited explicit-formula packing) that would imply \textup{(CB$_{\rm NF}$)}.
This requires nontrivial zero-density / explicit-formula input beyond VK-level global bounds.
\end{remark}

\begin{remark}[Heuristic margin]
If \textup{(CB$_{\rm NF}$)} holds with a bound comparable to the Whitney-scale bound ($\lesssim 0.195$), the margin would be $C_{\rm crit}/C_{\rm box} \approx 59\times$.
\end{remark}

\begin{remark}[On the nature of the VK bound]\label{rem:vk-upper-bound}
A potential concern is that the Vinogradov--Korobov--derived bound $K_\xi \le 0.160$ is ``coarse.''
We clarify why this does not affect the validity of the proof.

\smallskip\noindent
\textbf{Upper bounds suffice.}
The energy barrier requires: True $C_{\rm box} < C_{\rm crit} = 11.5$.
Vinogradov--Korobov provides an \emph{upper bound}: True $C_{\rm box} \le K_0 + K_\xi \le 0.195$.
Since $0.195 < 11.5$, the barrier holds.

The ``coarseness'' of VK means the \emph{true} $C_{\rm box}$ may be much smaller than $0.195$ (e.g., $0.05$).
This does not weaken the proof---it only means we have more safety than claimed.
An upper bound cannot \emph{underestimate} the true value; it can only overestimate.

\smallskip\noindent
\textbf{Safety factor.}
The ratio $C_{\rm crit}/C_{\rm box} \approx 11.5/0.195 \approx 59$ provides substantial robustness.
Even if the VK-derived constant were off by a factor of 50 (which would contradict the theorem), the barrier would still hold: $9.75 < 11.5$.

\smallskip\noindent
\textbf{What would break the argument.}
The barrier could fail only if:
\begin{enumerate}
\item The Vinogradov--Korobov theorem itself is false (contradicting $>$50 years of number theory), or
\item The specific constant $K_\xi \le 0.160$ is not rigorously derived from VK.
\end{enumerate}
Point (2) is addressed by the explicit derivation in the boxed audit (Appendix~\ref{app:vk-annuli-kxi}), where $K_\xi$ is computed via the annular aggregation formula with explicit geometric constants.

\smallskip\noindent
\textbf{Deeper near-field scaling.}
For zeros at distance $\eta < 0.1$ from the critical line, the vortex cost scales as $1/\eta$:
\begin{center}
\begin{tabular}{cccc}
$\eta$ & Strip & $C_{\rm crit}(\eta)$ & Margin \\
\hline
0.10 & $0.50 < \sigma < 0.60$ & 11.5 & $59\times$ \\
0.05 & $0.50 < \sigma < 0.55$ & 23.0 & $118\times$ \\
0.02 & $0.50 < \sigma < 0.52$ & 57.5 & $295\times$ \\
0.01 & $0.50 < \sigma < 0.51$ & 115 & $590\times$ \\
\end{tabular}
\end{center}
Zeros deeper in the near-field face \emph{higher} barriers, making them even easier to exclude.
\end{remark}

\begin{remark}[Alternative Theta-boundary formulation]\label{rem:theta-boundary}
The near-field elimination can also be understood directly in terms of the Schur function $\Theta$, avoiding potential-theoretic language.
Consider a hypothetical zero at $\rho = \sigma_\rho + i t_\rho$ with $\tfrac{1}{2} < \sigma_\rho < 0.6$.
Such a zero would force $\Theta(\rho) = 1$ (since $\xi$-zeros become poles of $\mathcal{J}$ and hence fixed points of the Cayley transform).
By a Blaschke-type phase constraint, maintaining $|\Theta| < 1$ on the certified right boundary ($\sigma = 0.6$, where $|\Theta| \le 0.9999928$) while having $\Theta(\rho) = 1$ in the interior requires
\[
  |\Theta(0.6 + it_\rho)|\ \ge\ \frac{\sigma_\rho - 0.5}{0.6 - 0.5}\ \cdot\ |\Theta(\rho)|\ =\ \frac{\sigma_\rho - 0.5}{0.1}\ \cdot\ 1.
\]
For any $\sigma_\rho > 0.5$, this forces $|\Theta(0.6 + it_\rho)| > 0$ to increase as the zero approaches $\sigma = 0.6$.
The certified bound $|\Theta(0.6 + it)| \le 0.9999928 < 1$ constrains how close to $\sigma = 0.6$ a zero can form; the energy barrier shows this constraint extends all the way to $\sigma = 0.5$.
This is the Theta-space interpretation of the energy barrier inequality.
\end{remark}

\section{Introduction}
\paragraph{Conceptual motivation.} The Euler product for $\zeta$ separates the $k=1$ prime layer from all higher prime powers. On the half–plane $\Omega=\{\Re s>\tfrac12\}$ the diagonal prime operator $A(s)e_p:=p^{-s}e_p$ has finite Hilbert–Schmidt norm ($\sum_p p^{-2\sigma}<\infty$), so the $k\ge2$ tail is naturally encoded by the $2$--modified determinant $\det_2(I-A)$. After dividing by a canonical outer factor (to enforce unimodular boundary modulus) one arrives at a ratio $\mathcal J$ that shares its zero/pole geometry with $\xi$ but is normalized for bounded-real methods. This puts the problem into the Herglotz/Schur framework: boundary positivity for $2\mathcal J$ transports to the interior by Poisson, and Cayley converts positivity into a Schur contractive bound for $\Theta=(2\mathcal J-1)/(2\mathcal J+1)$. The analytic bookkeeping is driven by a Carleson box energy constant $C_{\rm box}^{(\zeta)}$ coming from unconditional prime-tail control and Whitney-box estimates for $U_\xi$ (Vinogradov--Korobov / zero-count inputs). The remaining globalization is a Schur pinch across the discrete pole set $Z(\xi)$.
\noindent\textbf{Main result and proof outline (Two-regime hard closure).}
The proof proceeds by a two-regime elimination of the critical strip $\{1/2<\Re s<1\}$:
\begin{itemize}
\item \textbf{Far strip ($\Re s\ge 0.6$).} Hybrid arithmetic certification (Proposition~\ref{prop:farfield-hybrid}): (i) interval-arithmetic verification of $|\Theta| < 1$ on the rectangle $[0.6, 0.7] \times [0, 20]$, (ii) Pick-matrix certification at $\sigma_0 = 0.7$ with spectral gap $\delta = 0.627$ covering $\{\Re s > 0.7\}$, and (iii) asymptotic bounds (Lemma~\ref{lem:theta-asymptotic}) covering $|t| > 20$. Together these yield $|\Theta|\le 1$ on $\{\Re s\ge 0.6\}$. The Schur pinch (Theorem~\ref{thm:globalize-main}) then eliminates all zeros with $\Re s \ge 0.6$.
\item \textbf{Near strip ($1/2<\Re s<0.6$).} Energy capacity: any off-critical zero at depth $\eta=\beta-\tfrac12$ forces a minimum Dirichlet-energy cost ($L_{\rm rec} = 4\arctan(2) \approx 4.428$). \emph{Conditionally on hypothesis \textup{(CB$_{\rm NF}$)}} (scale-uniform near-field Carleson budget), the available energy is bounded by $C_{{\rm box},{\rm NF}}^{(\zeta)}(\sigma_0) \le 0.195$, yielding a $59\times$ safety margin. The concrete missing step to discharge \textup{(CB$_{\rm NF}$)} is hypothesis \textup{(EF$_{\rm BL}$)} (Section~\ref{sec:efbl-to-cbnf}).
\end{itemize}
\noindent The combination yields RH (Theorem~\ref{thm:final-rh}). The far-field step is reduced to a single verified finite-dimensional positivity check plus an explicit tail inequality; the near-field step is reduced to a scalar inequality between a vortex lower bound and a Carleson budget.

\paragraph{Optional boundary certificate material (\textup{(P+)}; not used in the main closure).}
\begin{itemize}
\item The phase--velocity identity and CR--Green/Carleson estimates yield Whitney-local phase-mass bounds and a boundary-wedge formulation \textup{(P+)} up to the local-to-global upgrade isolated in Remark~\ref{rem:wedge-application}.
\end{itemize}

\paragraph{Schur pinch template (used in the far strip).}
Section~\ref{sec:globalization} records the Schur pinch mechanism: a Schur bound for $\Theta$ on a zero-free domain, together with non-cancellation at $\xi$-zeros, rules out poles (hence zeros of $\xi$) in that domain.
The Riemann Hypothesis (RH) admits several analytic formulations. In this paper we pursue a bounded-real (BRF) route on the right half-plane
\[
 \Omega\;:=\;\Half,
\]
which is naturally expressed in terms of Herglotz/Schur functions and passive systems. Let \(\PP\) be the primes, and define the prime-diagonal operator
\[
 A(s):\ell^2(\PP)\to\ell^2(\PP),\qquad A(s)e_p\;:=\;p^{-s}e_p.
\]
For \(\sigma:=\Re s>\tfrac12\) we have \(\|A(s)\|_{\HS}^2=\sum_{p\in\PP}p^{-2\sigma}<\infty\) and \(\|A(s)\|\le 2^{-\sigma}<1\). With the completed zeta function
\[
 \xi(s)\;:=\;\tfrac12 s(1-s)\,\pi^{-s/2}\,\Gamma(s/2)\,\zeta(s)
\]
and the Hilbert--Schmidt regularized determinant \(\dettwo\), we study the analytic function
\[
 F(s)\;:=\;\frac{\dettwo(I-A(s))}{\zeta(s)}\cdot\frac{s}{s-1},
 \qquad
 \mathcal J(s)\;:=\;\frac{F(s)}{\mathcal O_{\mathrm{can}}(s)},
 \qquad
 \Theta(s)\;:=\;\frac{2\mathcal J(s)-1}{2\mathcal J(s)+1},
\]
where \(\mathcal O_{\mathrm{can}}\) is the canonical outer normalizer (Definition~\ref{def:canonical-normalizer}). A computable proxy \(\mathcal O_{\mathrm{ff}}\) is used only for numerical diagnostics.
\begin{lemma}[Stable $\zeta$--gauge formula for $\Theta$]\label{lem:theta-stable-zeta}
Let $s\in\Omega$ satisfy $\zeta(s)\neq 0$.
Define
\[
  X(s)\ :=\ 2\,\dettwo(I-A(s))\,s,
  \qquad
  Y(s)\ :=\ (s-1)\,\mathcal O_{\mathrm{can}}(s)\,\zeta(s).
\]
Then
\begin{equation}\label{eq:theta-XY}
  \Theta(s)\ =\ \frac{X(s)-Y(s)}{X(s)+Y(s)}.
\end{equation}
Moreover, if $\rho\in\Omega$ and $\xi(\rho)=0$, then by \textnormal{(N2)} one has $\lim_{s\to\rho}\Theta(s)=1$.
\end{lemma}
\begin{proof}
On $\Omega\setminus Z(\zeta)$ we have
\[
  \mathcal J(s)\ =\ \frac{\dettwo(I-A(s))}{\mathcal O_{\mathrm{can}}(s)\,\zeta(s)}\cdot\frac{s}{s-1}.
\]
Substituting this into $\Theta=(2\mathcal J-1)/(2\mathcal J+1)$ and multiplying numerator and denominator by $(s-1)\,\mathcal O_{\mathrm{can}}(s)\,\zeta(s)$ gives \eqref{eq:theta-XY}.
If $\xi(\rho)=0$ with $\rho\in\Omega$, then $\zeta(\rho)=0$ and $\dettwo(I-A(\rho))\neq 0$ by \textnormal{(N2)}; since $\mathcal O_{\mathrm{can}}$ is zero-free, $\mathcal J$ has a pole at $\rho$ and hence $\Theta(s)\to 1$ as $s\to\rho$.
\end{proof}
\begin{remark}[Why \eqref{eq:theta-XY} is the right geometry for certified numerics]\label{rem:theta-geometry}
The identity \eqref{eq:theta-XY} avoids forming the potentially ill-conditioned quotient $\mathcal J$ on wide complex boxes.
In particular, one can certify $|\Theta|<1$ on a rectangle cover by evaluating $X$ and $Y$ directly and checking disk inclusion for $(X-Y)/(X+Y)$ (provided $X+Y$ is certified nonzero on each box).
This is exactly the philosophy implemented in the certified Arb verifier (\texttt{verify\_attachment\_arb.py}, routine \texttt{theta\_certify\_rect}).
\end{remark}
The BRF assertion is that \(|\Theta(s)|\le 1\) on $\Omega\setminus Z(\xi)$ (Schur)—and on $\Omega$ after the pinch—equivalently that $2\mathcal J(s)$ is Herglotz on zero-free rectangles (hence on $\Omega\setminus Z(\xi)$) or that the associated Pick kernel is positive semidefinite there.

Our method combines four ingredients:
\begin{itemize}
 \item \textbf{Schur--determinant splitting.} For a block operator \(T(s)=\begin{bmatrix}A(s)&B(s)\\ C(s)&D(s)\end{bmatrix}\) with finite auxiliary part, one has
 \[
  \log\dettwo(I-T)\;=\;\log\dettwo(I-A)\; +\; \log\det(I-S),\qquad S\;:=\;D-C(I-A)^{-1}B,
 \]
 which separates the Hilbert--Schmidt (\(k\ge 2\)) terms from the finite block.
 \item \textbf{HS continuity for \(\dettwo\).} Prime truncations \(A_N\to A\) in the HS topology, uniformly on compacts in \(\Omega\), imply local-uniform convergence of \(\dettwo(I-A_N)\) (Section~\ref{prop:hs-det2-continuity}). Division by \(\zeta\) is justified only on compacts avoiding its zeros; throughout we explicitly state such hypotheses when needed (zeros coincide with \(Z(\xi)\) inside \(\Omega\)).
 % optional finite-stage and interior-rectangle route removed to enforce single proof route
\end{itemize}
% interior rectangles header removed (single-route only)
% removed interior route formal chain (single-route only). Throughout, \(\Omega=\{\Re s>\tfrac12\}\), and
\subsection*{Unsmoothing det$_2$: routed through smoothed testing (A1)}
\begin{lemma}[Smoothed distributional bound for $\partial_\sigma\,\Re\log\dettwo$]\label{lem:det2-unsmoothed}
Let $I\Subset\R$ be a compact interval and fix $\varepsilon_0\in(0,\tfrac12]$. There exists a finite constant
\[
  C_*\ :=\ \sum_{p}\sum_{k\ge 2}\frac{p^{-k/2}}{k^2\,\log p}\ <\ \infty
\]
such that for all $\sigma\in(\tfrac12,\tfrac12+\varepsilon_0]$ and every $\varphi\in C_c^2(I)$,
\[
  \Big|\int_{\R} \varphi(t)\,\partial_\sigma\Re\log\dettwo\big(I-A(\sigma+it)\big)\,dt\Big|\ \le\ C_*\,\|\varphi''\|_{L^1(I)}.
\]
In particular, testing against smooth, compactly supported windows yields bounds uniform in $\sigma$.
\end{lemma}
% archived block removed
\begin{proof}
For $\sigma>\tfrac12$ one has $\sum_p |p^{-(\sigma+it)}|^2=\sum_p p^{-2\sigma}<\infty$, so the diagonal product formula for $\dettwo$ gives
\[
  \log\dettwo(I-A(s))
  \;=\;\sum_p\big(\log(1-p^{-s})+p^{-s}\big)
  \;=\;-\sum_p\sum_{k\ge 2}\frac{p^{-ks}}{k},
\]
with absolute convergence (uniform on compact subsets of $\{\Re s>\tfrac12\}$). Differentiating termwise in $\sigma=\Re s$ yields the absolutely convergent expansion
\[
  \partial_\sigma\,\Re\log\dettwo\big(I-A(\sigma+it)\big)
  \;=\; \sum_{p}\sum_{k\ge 2} (\log p)\,p^{-k\sigma}\cos(k t\log p).
\]
For each frequency $\omega=k\log p\ge 2\log 2$, two integrations by parts give
\[
  \Big|\int_{\R}\!\varphi(t)\cos(\omega t)\,dt\Big|\ \le\ \frac{\|\varphi''\|_{L^1(I)}}{\omega^2}.
\]
Since $\sum_{p,k\ge 2} (\log p)\,p^{-k\sigma}/(k\log p)^2\le C_*$ uniformly in $\sigma\in(\tfrac12,\tfrac12+\varepsilon_0]$, Tonelli/Fubini allows summing after testing against $\varphi$. Summing the resulting majorant yields
\[
  \Big|\int \varphi\,\partial_\sigma\Re\log\dettwo\,dt\Big|
  \ \le\ \|\varphi''\|_{L^1}\sum_{p}\sum_{k\ge 2}\frac{(\log p)\,p^{-k\sigma}}{(k\log p)^2}
  \ \le\ \|\varphi''\|_{L^1}\sum_{p}\sum_{k\ge 2}\frac{p^{-k/2}}{k^2\,\log p},
\]
uniformly for $\sigma\in(\tfrac12,\tfrac12+\varepsilon_0]$, since the rightmost double series converges. This proves the claim.
\end{proof}
% (Removed: the earlier local-to-global wedge lemma was not used in the active chain and its proof as written was not correct.)
\noindent\emph{Note.} The single-interval density route is archived; the small-$L$ scaling $c_0 L \le C\,L^{1/2}$ does not contradict the RHS bound.

% (Legacy KYP/Carleson stubs removed; unused in the active route.)

% Minimal in-file bibliography moved to end (References)

\begin{lemma}[De-smoothing / boundary passage to an $L^1_{\mathrm{loc}}$ trace]\label{lem:desmooth-L1}
Let $U$ be a harmonic function on the half-plane $\Omega=\{(\sigma,t):\sigma>0\}$ such that its gradient energy defines a Carleson measure on Whitney boxes:
for every interval $I\subset\R$,
\[
  \iint_{Q(I)} |\nabla U(\sigma,t)|^2\,\sigma\,dt\,d\sigma\ \le\ C_{\rm box}\,|I|.
\]
Then $U$ has a boundary trace $u\in \mathrm{BMO}(\R)\subset L^1_{\mathrm{loc}}(\R)$ and
\[
  U(\sigma,\cdot)\ =\ P_\sigma * u\qquad(\sigma>0),
\]
so in particular $U(\varepsilon,\cdot)\to u$ in $L^1_{\mathrm{loc}}(\R)$ as $\varepsilon\downarrow 0$.
\end{lemma}
\begin{proof}
This is the classical Fefferman--Stein/Carleson characterization of boundary $\mathrm{BMO}$ via square functions (or equivalently via the Carleson measure control of $|\nabla U|^2\,\sigma\,dt\,d\sigma$); see, e.g., Garnett \cite[Ch.~IV]{Garnett} or Stein \cite[Ch.~II]{SteinHA}. Once $U=P_\sigma*u$ with $u\in L^1_{\mathrm{loc}}$, the convergence $P_\varepsilon*u\to u$ in $L^1_{\mathrm{loc}}$ is the standard approximate identity property of the Poisson kernel.
\end{proof}

\begin{lemma}[Neutralization bookkeeping for CR–Green on a Whitney box]\label{lem:neutralization-bookkeeping}
Let $I=[t_0{-}L,t_0{+}L]$ and $Q(\alpha'I)$ be as above. Let $B_I$ be the product of half–plane Blaschke factors for the zeros/poles of $J$ in $Q(\alpha'I)$ and set $\widetilde U:=\Re\log(J/B_I)$ on $Q(\alpha'I)$. Then with the same cutoff $\chi_{L,t_0}$ and Poisson test $V_{\psi,L,t_0}$,
\[
 \iint_{Q(\alpha'I)} \nabla \widetilde U\cdot\nabla(\chi V)\,dt\,d\sigma
 = \int_{\R} \psi_{L,t_0}(t)\,-w'(t)\,dt\ +\ \mathcal E_{\mathrm{side}}\ +\ \mathcal E_{\mathrm{top}},
\]
where the error terms obey the uniform bound
\[
 |\mathcal E_{\mathrm{side}}|+|\mathcal E_{\mathrm{top}}|
 \ \le\ C_{\mathrm{neu}}(\alpha,\psi)\,\Big(\iint_{Q(\alpha'I)} |\nabla U|^2\,\sigma\Big)^{1/2}.
\]
In particular,
\[
  \int_{\R} \psi_{L,t_0}(-w')\ \le\ \big(C(\psi)+C_{\mathrm{neu}}(\alpha,\psi)\big)\,\Big(\iint_{Q(\alpha'I)} |\nabla U|^2\,\sigma\Big)^{1/2},
\]
with constants independent of $t_0$ and $L$.
\end{lemma}
\begin{proof}
Apply Lemma~\ref{lem:CR-green-phase} to $\widetilde U$ on $Q(\alpha'I)$ and expand $\nabla\widetilde U=\nabla U-\nabla\Re\log B_I$. The latter is harmonic away from zeros and has explicit Poisson kernels on $\partial Q$; the bottom edge contribution cancels exactly against the Blaschke phase increments already accounted in $-w'$ (by construction of $B_I$), leaving only side/top terms. Cauchy–Schwarz together with the scale–invariant Dirichlet bounds for $V$ on the sides/top and a uniform bound on the Blaschke gradients in $Q(\alpha'I)$ (controlled by aperture $\alpha$) yield the stated estimate; the Whitney scaling gives independence of $L$.
\end{proof}
\noindent\emph{Clarification.} The certificate yields the Whitney–uniform phase-mass bound
\(\int_I (-w')\le \pi\,\Upsilon_{\rm Whit}(c)\) with $\Upsilon_{\rm Whit}(c)<\tfrac12$ (Lemma~\ref{lem:whitney-uniform-wedge}), obtained solely from the local CR--Green pairing controlled by $C_{\rm box}^{(\zeta)}$; the remaining promotion to a global a.e.\ wedge after a single rotation is isolated in Remark~\ref{rem:wedge-application}.
\smallskip
\noindent\emph{Non-circularity note.} The ``neutralization'' by \(B_I\) does \emph{not} assume that \(J\) (or \(\xi\)) is zero--free in \(Q(\alpha'I)\); it explicitly factors out the zeros/poles in that box so that \(\widetilde U=\Re\log(J/B_I)\) is harmonic there and the CR--Green pairing is legitimate.
No information about zeros is discarded: the removed factors contribute \emph{positively} to the phase derivative term \(-w'\) (via their explicit Blaschke phase increments), which is exactly why the near-field route can compare this quantized ``signal'' to the tail ``noise''.

\paragraph{Boundary wedge \textup{(P+)} (optional boundary formulation).}
We record the a.e.\ boundary inequality
\begin{equation}\label{eq:Pplus}
\Re\bigl(2\mathcal J(\tfrac12+it)\bigr)\ \ge\ 0\qquad\text{for a.e.\ }t\in\mathbb R.
\tag{P+}
\end{equation}
This is the classical boundary positivity input for BRF/Herglotz routes. The active proof route in this manuscript does \emph{not} rely on \eqref{eq:Pplus}; it is kept for comparison with boundary-wedge formulations.

\begin{lemma}[Poisson lower bound $\Rightarrow$ Lebesgue a.e. wedge]\label{lem:mu-to-lebesgue}
Assume the hypotheses of Theorem~\ref{thm:phase-velocity-quant}. Fix $m\in\mathbb R/2\pi\mathbb Z$ and define
\[
 \mathcal Q := \bigl\{t\in\mathbb R:\ |\Arg \mathcal J(1/2+it)-m|\ge \tfrac{\pi}{2}\bigr\}.
\]
If $\mu(\mathcal Q)=0$, then $|\mathcal Q|=0$. In particular, \eqref{eq:Pplus} holds.
\end{lemma}
\begin{proof}
Fix $I\Subset\R$ and choose $\phi\in C_c^\infty(I)$ with $0\le\phi\le\mathbf 1_{\mathcal Q}$. By Theorem~\ref{thm:phase-velocity-quant},
\[
  \int \phi(t)\,-w'(t)\,dt \;=\; \pi\!\int\phi\,d\mu \;+\; \pi\!\sum_{\gamma\in I} m_\gamma\,\phi(\gamma).
\]
Since $\mu(\mathcal Q)=0$ and $\phi\le\mathbf 1_{\mathcal Q}$, the first term vanishes; choosing $\phi$ to vanish in small neighborhoods of each $\gamma\in I$ kills the atomic sum as well, so $\int_{\mathcal Q} (-w')=0$ on $I$. As $-w'$ is a positive boundary distribution, this forces $-w'=0$ a.e. on $\mathcal Q\cap I$. By nontangential boundary uniqueness for harmonic conjugates of $H^p_{\rm loc}$ functions\footnote{See Garnett, \emph{Bounded Analytic Functions}, Thm.~II.4.2, and Rosenblum--Rovnyak, \emph{Hardy Classes and Operator Theory}, Ch.~2.} and the definition of $\mathcal Q$, we must have $|\mathcal Q\cap I|=0$. Letting $I\uparrow\R$ yields $|\mathcal Q|=0$.
\end{proof}
% --- Appendix S moved near the end; see after main results/before bibliography ---

% Lead-in: We quantify the phase–velocity identity and justify boundary passage via outers and smoothed L^1 control.
\begin{lemma}[Outer–Hilbert boundary identity]\label{lem:outer-phase-HT}
Let $u\in L^1_{\mathrm{loc}}(\mathbb R)$ and let $O$ be the outer function on $\Omega$ with boundary modulus $|O(\tfrac12+it)|=e^{u(t)}$ a.e. Then, in $\mathcal D'(\mathbb R)$,
\[
\frac{d}{dt}\Arg O\!\left(\tfrac12+it\right)=\Hilb[u'](t),
\]
where $\Hilb$ is the boundary Hilbert transform on $\mathbb R$ and $u'$ is the distributional derivative.
\end{lemma}
\begin{proof}
See, e.g., \cite[Ch.~2]{DurenHp} or \cite[Ch.~2]{RosenblumRovnyak} for the half-plane outer/Hardy boundary correspondence and distributional Hilbert-transform conventions.
Write $\log O=U+iV$ on $\Omega$, where $U$ is the Poisson extension of $u$ and $V$ is its harmonic conjugate with $V(\tfrac12+\cdot)=\Hilb[u]$ in $\mathcal D'(\mathbb R)$. Then $\tfrac{d}{dt}\Arg O=\partial_t V=\Hilb[\partial_t U]=\Hilb[u']$ in distributions.
\end{proof}
\begin{theorem}[Quantified phase–velocity identity and boundary passage]\label{thm:phase-velocity-quant}
Let $u_\varepsilon(t):=\log\big|\dettwo(I-A(\tfrac12+\varepsilon+it))\big| - \log\big|\xi(\tfrac12+\varepsilon+it)\big|$ and let $\mathcal O_\varepsilon$ be the outer on $\{\Re s>\tfrac12+\varepsilon\}$ with boundary modulus $e^{u_\varepsilon}$. There exists $C_I<\infty$, independent of $\varepsilon\in(0,\varepsilon_0]$, such that for every compact interval $I\Subset\R$ and every $\phi\in C_c^2(I)$ with $\phi\ge 0$,
\[
 \Big|\int_I \phi(t)\,\partial_\sigma\Re\log\dettwo\big(I-A(\tfrac12+\varepsilon+it)\big)\,dt\Big|\ \le\ C_I\,\|\phi''\|_{L^1(I)},
\]
and
\[
 \int_I \phi(t)\,\partial_\sigma\Re\log\xi\big(\tfrac12+\varepsilon+it\big)\,dt\ \le\ C'_I\,\|\phi\|_{H^1(I)}
\]
with $C'_I$ depending only on $I$. Consequently $\{u_\varepsilon\}_{\varepsilon\downarrow 0}$ is Cauchy in $\mathcal D'(I)$ (hence converges in distributions) and, passing $\varepsilon\downarrow 0$ in the smoothed identity (Lemma~\ref{lem:pv-test-smoothed}), the phase–velocity identity holds in the distributional sense on $I$:
\[
 \int_I \phi(t)\,-w'(t)\,dt\ =\ \int_I \phi(t)\,\pi\,d\mu(t)\ +\ \pi\sum_{\gamma\in I} m_\gamma\,\phi(\gamma),\qquad \forall\,\phi\in C_c^\infty(I),\ \phi\ge 0,
\]
where $\mu$ is the \emph{boundary balayage measure} on $\mathbb R$ induced by off–critical zeros (i.e. the absolutely continuous measure whose density is a sum of Poisson kernels), and the discrete sum ranges over critical–line ordinates.
\end{theorem}
\begin{proof}
Fix a compact interval $I\Subset\R$ and $\varepsilon_0\in(0,\tfrac12]$. Define
\[
 u_\varepsilon(t):=\log\Big|\dettwo\big(I-A(\tfrac12+\varepsilon+it)\big)\Big|-\log\Big|\xi(\tfrac12+\varepsilon+it)\Big|.
\]
By Lemma~\ref{lem:det2-unsmoothed}, for every $\phi\in C_c^2(I)$,
\[
 \Big|\!\int_I \!\phi(t)\,\partial_\sigma\Re\log\dettwo(I\! -\!A(\tfrac12\!+\!\sigma\!+\!it))\,dt\Big|\ \le\ C_I\,\|\phi''\|_{L^1(I)}
\]
uniformly in $\sigma\in(0,\varepsilon_0]$. For $\xi$, Lemma~\ref{lem:xi-deriv-L1} gives the tested bound
\[
 \Big|\int_I \phi(t)\,\partial_\sigma\Re\log\xi(\tfrac12+\sigma+it)\,dt\Big|\ \le\ C'_I\,\|\phi\|_{H^1(I)}\qquad(0<\sigma\le \varepsilon_0).
\]
Fix $0<\delta<\varepsilon\le \varepsilon_0$. Integrating in $\sigma$ and using the tested bounds yields a distributional Cauchy estimate: for every $\phi\in C_c^2(I)$,
\[
  \Big|\int_I \phi(t)\,\big(u_\varepsilon(t)-u_\delta(t)\big)\,dt\Big|
  \ \le\ |\varepsilon-\delta|\Big(C_I\,\|\phi''\|_{L^1(I)}+C'_I\,\|\phi\|_{H^1(I)}\Big).
\]
Hence $\{u_\varepsilon\}_{\varepsilon\downarrow 0}$ is Cauchy in $\mathcal D'(I)$ and converges to a distribution $u\in\mathcal D'(I)$. By continuity of the Hilbert transform on distributions (see, e.g., \cite[Ch.~II]{SteinHA}), $\Hilb[u_\varepsilon']\to \Hilb[u']$ in $\mathcal D'(I)$.

Now apply Lemma~\ref{lem:pv-test-smoothed} and let $\varepsilon\downarrow 0$.
The Poisson kernels $P_{\beta-\tfrac12-\varepsilon}$ converge in $\mathcal D'(\R)$ to $P_{\beta-\tfrac12}$, and boundary atoms from critical-line zeros appear in the limit through the argument principle on the boundary. Passing to the limit in \eqref{eq:pv-smoothed} yields the stated distributional identity for $-w'$ on $I$.
\end{proof}

\begin{lemma}[Balayage density and consequence for $Q$]\label{lem:balayage-density}
If there exists at least one off--critical zero $\rho=\beta+i\gamma$ of $\xi$ with $\beta>\tfrac12$, then the boundary balayage measure $\mu$ from Theorem~\ref{thm:phase-velocity-quant} has an a.e. density $f\in L^1_{\mathrm{loc}}(\mathbb R)$ of the form
\[
  f(t)\ =\ \sum_{\substack{\rho=\beta+i\gamma\\ \beta>1/2}} c_\rho\,P_{\beta-1/2}(t-\gamma),\qquad P_a(x)=\frac{1}{\pi}\frac{a}{a^2+x^2},
\]
which is strictly positive a.e. on $\R$ whenever at least one off--critical zero exists. Consequently, for any measurable set $E\subset\R$, $\mu(E)=0$ implies $|E|=0$. In particular, $\mu(Q)=0$ forces $|Q|=0$, hence \textup{(P+)}.
\end{lemma}
\begin{proof}
For each finite subset of zeros $\mathcal Z\subset\{\rho:\Re\rho>1/2\}$ the partial density
\(f_{\mathcal Z}(t):=\sum_{\rho\in\mathcal Z}c_\rho\,P_{\beta-1/2}(t-\gamma)\)
is continuous and strictly positive for all $t$ because each Poisson kernel is strictly positive on $\R$.
The phase--velocity formula and the Carleson energy finiteness imply the balayage of zeros on any Whitney box is finite, so the monotone limit of the partial sums converges in $L^1_{\mathrm{loc}}$ to an a.e. finite function $f\ge0$. Since the pointwise limit of strictly positive functions is nonnegative and cannot vanish on a set of positive measure unless all coefficients vanish, we obtain $f>0$ a.e. whenever at least one off--critical zero exists. Moreover, by positivity and monotone convergence, $\mu(E)=\int_E f\,dt=0$ forces $f=0$ a.e. on $E$, whence $|E|=0$.
\end{proof}

\paragraph{Certificate $\Rightarrow$ (P+): narrative.}
The outer, boundary phase–velocity identity shows that $\int\varphi_{L,t_0}(-w')$ is the mass picked up by $\varphi_{L,t_0}$ against a positive measure supported on off–critical zeros (with atoms on the line if they occur). The left plateau inequality therefore lower-bounds it by $c_0(\psi)\,\nu(Q(I))$, where $\nu$ is the defect measure on $\Omega$ (see Notation and conventions) and $Q(I)$ is the Carleson box. The CR–Green pairing controls the same integral from above by box energy, and the Carleson bound is uniform on Whitney boxes. This yields a Whitney–uniform \emph{local} phase-drop bound $\int_I(-w')\le \pi\,\Upsilon_{\rm Whit}(c)$ with $\Upsilon_{\rm Whit}(c)<\tfrac12$ for suitably small $c$ (Lemma~\ref{lem:whitney-uniform-wedge}). The remaining upgrade from Whitney-local control to a global a.e. boundary wedge \textup{(P+)} after a single rotation is a separate local-to-global step; see Remark~\ref{rem:wedge-application}.

\begin{lemma}[Quantitative wedge criterion]\label{lem:local-to-global-wedge}
Let $w\in L^\infty_{\mathrm{loc}}(\R)$ be a boundary phase function. For a measurable interval $I\subset\R$, write
\[
  \osc\sb{I} w\ :=\ \esssup\sb{I} w\ -\ \essinf\sb{I} w
\]
for the essential oscillation (with respect to Lebesgue measure).
\begin{enumerate}
    \item \textbf{Local-to-global from a centered exhaustion.} If there is a $D\ge 0$ such that
    \[
      \osc\sb{[-N,N]} w\ \le\ D\qquad\text{for every }N\ge 1,
    \]
    then there exists a constant $c\in\R$ such that $|w(t)-c|\le D$ for a.e. $t\in\R$.
    \item \textbf{Windowed phase-mass $\Rightarrow$ oscillation on an interval.} Assume $-w'$ is a positive Radon measure on $\R$ (in the sense of distributions). If $I=[a,b]$ and $\psi\ge \mathbf 1_I$ is a nonnegative test function, then
    \[
      \int_I (-w')\ \le\ \int_{\R}\psi\,(-w'),
    \]
    and the phase drop (hence essential oscillation) on $I$ is bounded by the left-hand side. In particular, if for some $\Upsilon\ge 0$ one has $\int_{\R}\psi\,(-w')\le \pi\,\Upsilon$, then $\osc\sb{I} w\le \pi\,\Upsilon$.
\end{enumerate}
\end{lemma}
\begin{proof}
(1) For $N\ge 1$ set $a_N:=\essinf\sb{[-N,N]} w$ and $b_N:=\esssup\sb{[-N,N]} w$. Then $a_N$ is nonincreasing, $b_N$ is nondecreasing, and $b_N-a_N\le D$ by hypothesis. Let
\[
  a_\infty:=\lim_{N\to\infty} a_N\in[-\infty,\infty),\qquad b_\infty:=\lim_{N\to\infty} b_N\in(-\infty,\infty].
\]
Then $b_\infty-a_\infty\le D$ and for each $N$ we have $a_\infty\le a_N\le w(t)\le b_N\le b_\infty$ for a.e.\ $t\in[-N,N]$, hence for a.e.\ $t\in\R$. Choosing $c:=(a_\infty+b_\infty)/2$ gives $|w(t)-c|\le D$ a.e.

(2) The first inequality is immediate from $\psi\ge\mathbf 1_I$ and positivity of the measure $-w'$. Since $-w'$ is the (distributional) derivative of a locally BV representative of $w$, its mass on $I$ bounds the phase drop across $I$, which in turn bounds the essential oscillation on $I$. (See, e.g., \cite[Ch.~3]{AmbrosioFuscoPallara} for BV representatives and the identification of distributional derivatives with measures.)
\end{proof}

\begin{lemma}[Whitney--uniform wedge]\label{lem:whitney-uniform-wedge}
Fix the Whitney schedule and clip by $L_\star$: set $L_\star:=c/\log 2$ and henceforth
\[
  L(T)\ :=\ \min\Big\{\frac{c}{\log\langle T\rangle},\ L_\star\Big\}.
\]
Then for every Whitney interval $I=[t_0-L,t_0+L]$ (so $L\le L_\star$), with the printed flat--top window $\psi_{L,t_0}(t)=\psi\big((t-t_0)/L\big)$ one has
\[
  \int_I (-w')\,dt\ \le\ \int_{\mathbb R} \psi_{L,t_0}(t)\,(-w'(t))\,dt\ \le\ C(\psi)\,\sqrt{C_{\rm box}^{(\zeta)}}\,L_\star^{1/2}
  \ :=\ \pi\,\Upsilon_{\rm Whit}(c),
\]
where $C(\psi)$ is the CR--Green window constant and $\Upsilon_{\rm Whit}(c)$ depends only on $c,\psi$ and the fixed aperture. Choosing $c>0$ sufficiently small so that $\Upsilon_{\rm Whit}(c)<\tfrac12$ yields the Whitney-local phase-drop bound $\int_I(-w')\le \pi\,\Upsilon_{\rm Whit}(c)$ on every Whitney interval. Promoting this Whitney-local bound to a \emph{global} a.e. boundary wedge \textup{(P+)} requires an additional local-to-global step; see Remark~\ref{rem:wedge-application}.
\end{lemma}
\begin{proof}
Since $-w'$ is a positive boundary distribution and $\psi_{L,t_0}\ge \mathbf 1_I$ (because $\psi\equiv 1$ on $[-1,1]$), we have
\[
  \int_I (-w')\ \le\ \int_{\R}\psi_{L,t_0}\,(-w').
\]
By Lemma~\ref{lem:CR-green-phase},
\[
  \int_{\R}\psi_{L,t_0}\,(-w')\ \le\ C(\psi)\,\Big(\iint_{Q(\alpha'I)}|\nabla U|^2\,\sigma\Big)^{1/2}.
\]
Using the box constant $C_{\rm box}^{(\zeta)}=\sup_I |I|^{-1}\iint_{Q(\alpha'I)}|\nabla U|^2\,\sigma$ and $|I|=2L\le 2L_\star$, we obtain
\[
  \Big(\iint_{Q(\alpha'I)}|\nabla U|^2\,\sigma\Big)^{1/2}\ \le\ \sqrt{C_{\rm box}^{(\zeta)}\,|I|}\ \le\ \sqrt{2}\,\sqrt{C_{\rm box}^{(\zeta)}}\,L_\star^{1/2},
\]
and we absorb the harmless factor $\sqrt2$ into the definition of $\Upsilon_{\rm Whit}(c)$.
\end{proof}

\noindent\emph{Clarification.} The certificate yields the Whitney–uniform phase-mass bound
\(\int_I (-w')\le \pi\,\Upsilon_{\rm Whit}(c)\) with $\Upsilon_{\rm Whit}(c)<\tfrac12$ (Lemma~\ref{lem:whitney-uniform-wedge}), obtained solely from the local CR--Green pairing controlled by $C_{\rm box}^{(\zeta)}$; the remaining promotion to a global a.e.\ wedge after a single rotation is isolated in Remark~\ref{rem:wedge-application}.
\paragraph{Window constant.}
Set once and for all the window constant
\[
  C(\psi)\ :=\ C_{\mathrm{rem}}(\alpha,\psi)\,\mathcal A(\psi),
\]
where $\mathcal A(\psi)$ is the fixed Poisson energy of the window and $C_{\mathrm{rem}}(\alpha,\psi)$ is the side/top remainder factor from Corollary~\ref{cor:CH-Mpsi-final}. Then $C(\psi)$ is independent of $L$ and $t_0$ and will be used uniformly below.
\begin{proposition}[HS$\to$det$_2$ continuity]\label{prop:hs-det2-continuity}
Let $A_N,A$ be analytic $\HS$-valued maps on $\Omega$ with $A_N\to A$ in the Hilbert–Schmidt norm uniformly on compact subsets of $\Omega$. Then $\det\nolimits_2(I-A_N)\to\det\nolimits_2(I-A)$ locally uniformly on $\Omega$.
\end{proposition}
\begin{proof}
Fix a compact $K\Subset\Omega$. By hypothesis, $\sup_{s\in K}\|A_N(s)-A(s)\|_{\HS}\to0$, and in particular $\sup_{N}\sup_{s\in K}\|A_N(s)\|_{\HS}<\infty$.
We use the standard definition of the $2$--modified determinant on $\HS$:
\[
  \det\nolimits_2(I-T)\ :=\ \det\!\big((I-T)e^T\big),
\]
where the Fredholm determinant on the right is defined for trace-class perturbations of the identity. Indeed, for $T\in\HS$ one has
\[
  (I-T)e^T-I\ =\ -\sum_{n\ge 2}\frac{n-1}{n!}\,T^n,
\]
and the series converges absolutely in trace norm because $T^n$ is trace class for $n\ge 2$ and
$\|T^n\|_1\le \|T\|^{n-2}\|T^2\|_1\le \|T\|_{\HS}^n$.
In particular, on any $\HS$-ball $\{ \|T\|_{\HS}\le M\}$, the map
\[
  T\ \longmapsto\ (I-T)e^T-I
\]
is Lipschitz from $\HS$ to trace class: writing the series termwise and using
$T^n-S^n=\sum_{k=0}^{n-1}T^k(T-S)S^{n-1-k}$ together with $\|XY\|_1\le \|X\|_2\|Y\|_2$ and $\|T\|\le \|T\|_{\HS}$ gives
\[
  \|(I-T)e^T-(I-S)e^S\|_1\ \le\ C(M)\,\|T-S\|_{\HS}.
\]
Since the Fredholm determinant on trace-class perturbations of the identity is defined by an absolutely convergent trace-norm series (hence is continuous in $\|\cdot\|_1$), it follows that $\det_2(I-T)$ is continuous (indeed locally Lipschitz) with respect to $\|\cdot\|_{\HS}$.
Thus
\[
  \sup_{s\in K}\Big|\det\nolimits_2\big(I-A_N(s)\big)-\det\nolimits_2\big(I-A(s)\big)\Big|\ \longrightarrow\ 0,
\]
which is local-uniform convergence on $K$. Since $K$ was arbitrary, the convergence is locally uniform on $\Omega$.
\end{proof}

\begin{lemma}[Smoothed phase–velocity calculus]\label{lem:pv-test-smoothed}
Fix $\varepsilon\in(0,\tfrac12]$ and set
\[
 u_\varepsilon(t):=\log\Big|\dettwo(I{-}A(\tfrac12{+}\varepsilon{+}it))\Big|-\log\Big|\xi(\tfrac12{+}\varepsilon{+}it)\Big|.
\]
Let $\mathcal O_\varepsilon$ be the outer on $\{\Re s>\tfrac12{+}\varepsilon\}$ with boundary modulus $e^{u_\varepsilon}$ and write $F_\varepsilon:=\dettwo/\xi$ and $O_\varepsilon:=\mathcal O_\varepsilon$. Then for every $\phi\in C_c^\infty(\R)$,
\begin{equation}\label{eq:pv-smoothed}
\int_\R\!\phi(t)\,\Big( \Im\frac{\xi'}{\xi}-\Im\frac{\dettwo'}{\dettwo}+\Hilb[u_\varepsilon']\Big)\!(\tfrac12{+}\varepsilon{+}it)\,dt
\;=\;\sum_{\substack{\rho=\beta+i\gamma\\ \Re\rho>\tfrac12+\varepsilon}}\! c_\rho\,\big(P_{\beta-\tfrac12-\varepsilon}\!\ast\phi\big)(\gamma)
\end{equation}
where $P_a(x)=\frac{1}{\pi}\frac{a}{a^2+x^2}$ is the Poisson kernel, and the coefficients $c_\rho\ge 0$ are the pole multiplicities of $F_\varepsilon$ (equivalently the zero multiplicities of $\xi$) in the half-plane $\{\Re s>\tfrac12+\varepsilon\}$. In particular, the right-hand side is nonnegative.
\end{lemma}
\begin{proof}
Factor $F_\varepsilon=I_\varepsilon\,O_\varepsilon$ with $O_\varepsilon$ outer on $\{\Re s>\tfrac12{+}\varepsilon\}$ and $I_\varepsilon$ inner (product of half-plane Blaschke factors for poles/zeros of $F_\varepsilon$ in the open half-plane). By Lemma~\ref{lem:outer-phase-HT}, on the boundary line $\Re s=\tfrac12{+}\varepsilon$ one has $\frac{d}{dt}\Arg O_\varepsilon=\Hilb[u_\varepsilon']$ in $\mathcal D'(\R)$. Each pole of $F_\varepsilon$ at $\rho=\beta+i\gamma$ with $\beta>\tfrac12$ contributes a half-plane Blaschke factor of the form
$C_{\rho,\varepsilon}(s)=\bigl(s-\rho_\varepsilon^\ast\bigr)/(s-\rho)$ with $\rho_\varepsilon^\ast:=1+2\varepsilon-\overline\rho$ (reflection across $\Re s=\tfrac12+\varepsilon$), whose boundary phase derivative is a nonnegative multiple of the Poisson kernel $P_{\beta-\tfrac12-\varepsilon}(t-\gamma)$. Summing these contributions and writing $\frac{d}{dt}\Arg F_\varepsilon=\Im(F_\varepsilon'/F_\varepsilon)=\Im(\dettwo'/\dettwo)-\Im(\xi'/\xi)$ yields \eqref{eq:pv-smoothed} after testing against $\phi$.
\end{proof}
% Lead-in: Use arithmetic Pick certification (Theorem~\ref{thm:pick-global-positivity})
% to obtain Schur on $\{\,\Re s>\sigma_0\,\}\setminus Z(\xi)$ (Corollary~\ref{cor:Schur-off-zeros}),
% then pinch across $Z(\xi)$ using (N1)–(N2).
\section{Globalization across $Z(\xi)$ via a Schur--Herglotz pinch}\label{sec:globalization}
\noindent This section records the Schur pinch \emph{template}: given a domain $D\subset\Omega$ on which $\Theta$ is Schur on $D\setminus Z(\xi)$, together with non-cancellation \textnormal{(N2)} and the right-edge normalization \textnormal{(N1)}, one rules out zeros of $\xi$ in $D$.
In the far-field route, we apply this with $D=\{\,\Re s>\sigma_0\,\}$ once the Schur bound is obtained there (Corollary~\ref{cor:Schur-off-zeros} via the arithmetic Pick certificate).
\paragraph{Globalization and pinch: narrative.}
In particular, once Corollary~\ref{cor:Schur-off-zeros} provides $\Theta$ Schur on $D\setminus Z(\xi)$, any putative zero $\rho\in D$ forces $\Theta(\rho)=1$ by removability, hence $\Theta$ is constant unimodular on $D\setminus Z(\xi)$ by the Maximum Modulus Principle; the normalization (N1) forces $\Theta(\sigma+it)\to\tfrac13$ as $\sigma\to+\infty$, contradicting a unimodular constant.
\noindent\textbf{Standing setup.}
Let
\[
\Omega:=\{s\in\mathbb C:\ \Re s>\tfrac12\},\qquad
\xi(s)=\tfrac12\,s(s-1)\,\pi^{-s/2}\Gamma\!\big(\tfrac s2\big)\zeta(s).
\]
\noindent\emph{Clarification.} Although the factor $(s-1)$ vanishes at $s=1$, the zeta factor has a simple pole there and the product $(s-1)\zeta(s)\to 1$. Hence $\xi$ is entire and $\xi(1)=\tfrac12\,\pi^{-1/2}\Gamma(1/2)\cdot 1=\tfrac12\neq 0$.
Define
\[
F(s):=\frac{\det\nolimits_2(I-A(s))}{\zeta(s)}\cdot\frac{s}{s-1},\qquad
\mathcal J(s):=\frac{F(s)}{\mathcal O_{\mathrm{can}}(s)},\qquad
G(s):=2\,\mathcal J(s),\qquad
\Theta(s):=\frac{G(s)-1}{G(s)+1}.
\]
Here \(\mathcal O_{\mathrm{can}}\) is the canonical outer normalizer (Definition~\ref{def:canonical-normalizer}); it is holomorphic and zero--free on \(\Omega\), and
\(\det\nolimits_2(I-A)\) is holomorphic and zero--free on \(\Omega\).
We record the two normalization properties actually used below:
\begin{itemize}
\item[(N1)] (\emph{Right--edge normalization}) For each fixed $t$ (indeed uniformly on compact $t$–intervals), $\displaystyle\lim_{\sigma\to+\infty}\mathcal J(\sigma+it)=1$; hence $\displaystyle\lim_{\sigma\to+\infty}\Theta(\sigma+it)=\tfrac13$.
\item[(N2)] (\emph{Non--cancellation at $\xi$--zeros}) For every $\rho\in\Omega$ with $\xi(\rho)=0$,
\[
\det\nolimits_2(I-A(\rho))\neq0.
\]
Thus $\mathcal J$ has a pole at $\rho$ of order $\operatorname{ord}_\rho(\xi)$ (since $F$ has a pole there and $\mathcal O_{\mathrm{can}}$ is zero-free).
\end{itemize}

\medskip
\noindent\textbf{Schur bound on the far half-plane off \(Z(\xi)\).}
By Corollary~\ref{cor:Schur-off-zeros}, the Cayley transform is Schur on $\{\,\Re s>\sigma_0\,\}\setminus Z(\xi)$:
\[
|\Theta(s)|\ \le\ 1\qquad(s\in \{\,\Re s>\sigma_0\,\}\setminus Z(\xi)).
\tag{Schur}
\label{eq:SchurBound}
\]
\medskip
\noindent\textbf{Local pinch at a putative off--critical zero.}
\emph{We use (N2) for non--cancellation at $\xi$--zeros and (N1) for the right--edge limit $\Theta\to\tfrac13$.}
Fix $\rho\in\Omega$ with $\Re\rho>\sigma_0$ and $\xi(\rho)=0$.
By (N2) the function $\mathcal J$ has a pole at $\rho$ (equivalently $G=2\mathcal J$ has a pole), hence
\[
\Theta(s)=\frac{G(s)-1}{G(s)+1}\ \longrightarrow\ 1\qquad(s\to\rho).
\]
By \eqref{eq:SchurBound}, $\Theta$ is bounded by $1$ on $(\Omega\setminus Z(\xi))$,
so the singularity of $\Theta$ at $\rho$ is removable (Riemann's theorem), and the holomorphic extension satisfies
\[
\Theta(\rho)=1.
\]
Because $\Theta$ is holomorphic on the connected domain $\{\,\Re s>\sigma_0\,\}\setminus(Z(\xi)\setminus\{\rho\})$
and $|\Theta|\le1$ there, the Maximum Modulus Principle forces $\Theta$ to be
a \emph{constant unimodular} function on that domain (it attains its supremum $1$ at an interior point).
By analyticity, the same constant extends throughout $\{\,\Re s>\sigma_0\,\}\setminus Z(\xi)$.
\medskip
\begin{lemma}[Connectedness and isolation]\label{rem:connectedness}
Since $Z(\xi)\cap\Omega$ is a discrete subset (zeros are isolated), one can choose a disc $D\subset\{\,\Re s>\sigma_0\,\}$ centered at $\rho$ containing no other zeros. Moreover, $\{\,\Re s>\sigma_0\,\}\setminus Z(\xi)$ is (path-)connected. Hence in the argument above, $\{\,\Re s>\sigma_0\,\}\setminus\big(Z(\xi)\setminus\{\rho\}\big)$ is connected and the Maximum Modulus Principle applies on this domain.
\end{lemma}
\begin{proof}
Since $\xi$ is holomorphic and not identically zero on $\Omega$, its zeros are isolated; thus $Z(\xi)\cap\Omega$ is discrete and we may choose a disc $D\subset\{\,\Re s>\sigma_0\,\}$ around $\rho$ containing no other zeros.
For connectedness: given $z_0,z_1\in \{\,\Re s>\sigma_0\,\}\setminus Z(\xi)$, join them by a polygonal path in $\{\,\Re s>\sigma_0\,\}$. A compact polygonal path meets only finitely many points of the discrete set $Z(\xi)\cap\Omega$, so we can locally perturb the path in small discs around those points to avoid them. This produces a path in $\{\,\Re s>\sigma_0\,\}\setminus Z(\xi)$, hence $\{\,\Re s>\sigma_0\,\}\setminus Z(\xi)$ is path-connected. The same argument applies to $\{\,\Re s>\sigma_0\,\}\setminus(Z(\xi)\setminus\{\rho\})$.
\end{proof}
\noindent\textbf{Contradiction with right--edge normalization.}
By (N1), $\Theta(\sigma+it)\to\tfrac13$ as $\sigma\to+\infty$ (uniformly for $t$ in compact intervals). A constant unimodular function cannot have such a limit. Contradiction.
\medskip
\noindent\textbf{Conclusion of the pinch.}
No $\rho\in\Omega$ with $\Re\rho>\sigma_0$ and $\xi(\rho)=0$ can exist.
\medskip
\noindent\textbf{Connective summary (secondary BRF/pinch route).}
This section records the Schur pinch argument: the Schur bound on $\{\,\Re s>\sigma_0\,\}\setminus Z(\xi)$ comes from the arithmetic Pick-matrix certification (Theorem~\ref{thm:pick-global-positivity} and Corollary~\ref{cor:Schur-off-zeros}), and the pinch uses only (N1)--(N2). A boundary-wedge route via \textup{(P+)} is optional and recorded elsewhere for comparison, but is not required for the pinch.
\medskip
\noindent\textbf{Normalization at infinity (used in (N1)).}
We record explicit bounds ensuring $\Theta(\sigma+it)\to\tfrac13$ uniformly for $t$ in compact $t$-intervals as $\sigma\to+\infty$.
\begin{itemize}
\item Zeta limit: For $\sigma\ge 2$ and all $t\in\R$, $|\zeta(\sigma+it)-1|\le 2^{1-\sigma}$, hence $|\zeta(\sigma+it)|\to 1$ uniformly for $t$ in compact intervals as $\sigma\to+\infty$. Also $(\sigma+it-1)/(\sigma+it)\to 1$ uniformly on compact $t$-intervals.
\item Det$_2$ limit: For $\sigma\ge 1$, $\|A(\sigma+it)\|\le 2^{-\sigma}\le \tfrac12$. By the product representation in Lemma~\ref{lem:hs-diagonal} and since $\sum_p p^{-2\sigma}\to0$ as $\sigma\to\infty$, one has $|\dettwo(I-A(\sigma+it)) - 1|\le C\sum_p p^{-2\sigma}\to 0$ (uniformly for $t$ in compact intervals).
\item Canonical outer normalizer: $\mathcal O_{\mathrm{can}}$ is an outer function on $\Omega$ with boundary modulus $|\mathcal O_{\mathrm{can}}(\tfrac12+it)|=|F(\tfrac12+it)|$ a.e.\ (Definition~\ref{def:canonical-normalizer}), uniquely determined up to a unimodular constant. We fix that constant so that $\mathcal O_{\mathrm{can}}(\sigma+it)\to 1$ as $\sigma\to+\infty$ (uniformly for $t$ in compact intervals), which is the standard right-edge normalization for outers on $\Omega$.
\end{itemize}
Combining, $F(\sigma+it)\to 1$ and $\mathcal O_{\mathrm{can}}(\sigma+it)\to 1$ uniformly for $t$ in compact intervals, hence $\mathcal J(\sigma+it)=F/\mathcal O_{\mathrm{can}}\to 1$ and therefore
\(\Theta(\sigma+it)=(2\mathcal J-1)/(2\mathcal J+1)\to\tfrac13\).

\bigskip
% (Global RH is obtained by combining the far-half-plane pinch with the near-field energy barrier; see Section~\ref{sec:unconditional-closure}.)

\begin{lemma}[Carleson box energy: stable sum bound]\label{lem:carleson-sum}
For harmonic potentials $U_1,U_2$ on $\Omega$, one has
\[
\sqrt{C_{\mathrm{box}}(U_1+U_2)}\ \le\
\sqrt{C_{\mathrm{box}}(U_1)}\ +\ \sqrt{C_{\mathrm{box}}(U_2)}.
\]
\end{lemma}
\begin{proof}
Write $\mu_j:=|\nabla U_j|^2\,\sigma\,dt\,d\sigma$ and $\mu_{12}:=|\nabla(U_1{+}U_2)|^2\,\sigma\,dt\,d\sigma$. For any Carleson box $B$, by Cauchy–Schwarz,
\[
\int_{B} |\nabla(U_1+U_2)|^2\,\sigma\,dt\,d\sigma
\ \le\ \Big(\sqrt{\int_B |\nabla U_1|^2\,\sigma}\ +\ \sqrt{\int_B |\nabla U_2|^2\,\sigma}\Big)^{\!2}.
\]
Taking supremum over Carleson boxes $B$ and dividing by $|I_B|$ yields the claimed inequality.
\end{proof}

\begin{corollary}[Local Carleson energy for $U_\xi$ on a fixed interval]\label{cor:xi-carleson-all-I}
For each compact interval $I\Subset\R$ there exists a finite constant $C_{\xi,I}<\infty$ such that
\[
  \iint_{Q(I)} |\nabla U_{\xi}(\sigma,t)|^2\,\sigma\,dt\,d\sigma\ \le\ C_{\xi,I}\,|I|.
\]
In particular, on Whitney intervals $I=[T-L,T+L]$ with $L=c/\log\langle T\rangle$ one may take $C_{\xi,I}=C_\xi$ from Lemma~\ref{lem:carleson-xi}.
\end{corollary}
\begin{proof}
(\emph{Sketch.}) Fix $I\Subset\R$. Cover $I$ by finitely many Whitney intervals $I_j=[T_j-L(T_j),T_j+L(T_j)]$ with bounded overlap (since $I$ is compact and $L(\cdot)$ is bounded below on $I$), so that $Q(I)\subset\bigcup_j Q(\alpha I_j)$. Apply Lemma~\ref{lem:carleson-xi} on each $Q(\alpha I_j)$ and sum; the overlap and the finiteness of the cover yield the stated bound with a constant depending on $I$ (through the finite cover) and on the fixed aperture.
\end{proof}

\begin{lemma}[L$^1$-tested control for $\partial_\sigma\Re\log\xi$]\label{lem:xi-deriv-L1}
For each compact $I\Subset\R$ there exists $C'_I<\infty$ such that for all $0<\sigma\le\varepsilon_0$ and all $\phi\in C_c^2(I)$,
\[
\Big|\int_I \phi(t)\,\partial_\sigma\Re\log\xi\!\big(\tfrac12+\sigma+it\big)\,dt\Big|
\ \le\ C'_I\,\|\phi\|_{H^1(I)}.
\]
\end{lemma}

\begin{proof}[Proof of Lemma~\ref{lem:xi-deriv-L1}]
Let $I\Subset\R$ and $\phi\in C_c^2(I)$. Let $V$ be the Poisson extension of $\phi$ on a fixed dilation $Q(\alpha I)$. Green's identity together with Cauchy–Riemann for $U_\xi=\Re\log\xi$ gives
\[
  \int_I \phi(t)\,\partial_\sigma\Re\log\xi\!\big(\tfrac12+\sigma+it\big)\,dt
  \,=\, \iint_{Q(\alpha I)} \nabla U_\xi\cdot\nabla V\,dt\,d\sigma.
\]
This is exactly the standard Carleson embedding / $H^1$–BMO pairing estimate for Poisson extensions (see Garnett \cite[Thm.~VI.1.1]{Garnett} or Stein \cite[Ch.~IV]{SteinSingInt}): if $\lambda:=|\nabla U_\xi|^2\,\sigma\,dt\,d\sigma$ is Carleson on boxes above $I$, then
\[
  \Big|\iint_{Q(\alpha I)} \nabla U_\xi\cdot\nabla V\,dt\,d\sigma\Big|
  \ \lesssim_{I,\alpha}\ \|\phi\|_{H^1(I)}.
\]
Using the local Carleson bound from Corollary~\ref{cor:xi-carleson-all-I} gives the asserted constant $C'_I<\infty$ depending only on $I$ (and the fixed aperture).
\end{proof}
\begin{corollary}[Conservative closure inequalities]\label{cor:conservative-closure}
Let $K_0$ be the arithmetic tail box-energy constant (Lemma~\ref{lem:carleson-arith}) and let $K_\xi$ be the neutralized $\xi$ box-energy constant (Lemma~\ref{lem:carleson-xi}). Define
\[
  C_{\mathrm{box}}^{(\zeta)}\ :=\ K_0+K_\xi.
\]
Then one has the conservative subadditivity bound
\[
  \sqrt{C_{\mathrm{box}}^{(\zeta)}}\ \le\ \sqrt{K_0}+\sqrt{K_\xi}.
\]
Moreover, for the printed window $\psi$ one has the structural mean-oscillation bound
\[
  M_\psi\ \le\ \frac{4}{\pi}\,C_\psi^{(H^1)}\,\sqrt{C_{\mathrm{box}}^{(\zeta)}}.
\]
\end{corollary}
\begin{proof}
The inequality $\sqrt{C_{\mathrm{box}}^{(\zeta)}}\le \sqrt{K_0}+\sqrt{K_\xi}$ is Lemma~\ref{lem:carleson-sum} applied to the decomposition of the paired potential into the arithmetic tail and the neutralized $\xi$-part (cf.\ Lemma~\ref{lem:outer-energy-bookkeeping}). The bound on $M_\psi$ follows from the $H^1$--BMO/Carleson embedding estimate (Lemma~\ref{lem:Mpsi-correct}) together with the embedding normalization $C_{\mathrm{CE}}(\alpha)=1$ (Lemma~\ref{lem:CE-constant-one}).
\end{proof}
\ifshownumerics
\paragraph*{Diagnostics (optional; non-load-bearing).}
Plugging the audited window constants into the structural bound yields the diagnostic enclosure
\[
  M_\psi\ \le\ \Mpsilocked,\qquad
  \Upsilon_{\mathrm{diag}}\ :=\ \frac{(2/\pi)\,M_\psi}{c_0(\psi)}\ \le\ \UpsilonLocked.
\]
(Closure of \textup{(P+)} uses the Whitney-uniform $\Upsilon_{\mathrm{Whit}}(c)$ from Lemma~\ref{lem:whitney-uniform-wedge}.)
\fi
\medskip
\medskip
\noindent\textbf{Proof of (N2) (non--cancellation at $\xi$--zeros).}
For $s=\sigma+it$ with $\sigma>\tfrac12$, define the diagonal operator $A(s)e_p=p^{-s}e_p$ on $\ell^2(\mathbb P)$. Then $\|A(s)\|=2^{-\sigma}<1$ and $\|A(s)\|_{\mathrm{HS}}^2=\sum_{p}p^{-2\sigma}<\infty$, so $A(s)$ is Hilbert--Schmidt. The 2--modified determinant for diagonal $A(s)$ is
\[
\det\nolimits_2\!\big(I-A(s)\big)\;=\;\prod_{p\in\mathbb P}(1-p^{-s})\,e^{p^{-s}},
\]
which converges absolutely and is nonzero because each factor is nonzero. Moreover, $I-A(s)$ is invertible with $\|(I-A(s))^{-1}\|\le (1-2^{-\sigma})^{-1}$ since $|1-p^{-s}|\ge 1-2^{-\sigma}>0$. Finally, the outer normalizer has the form $\mathcal O(s)=\exp H(s)$ with $H$ analytic on $\Omega$, hence $\mathcal O$ is zero--free on $\Omega$. Thus if $\rho\in\Omega$ with $\xi(\rho)=0$, then $\det_2(I-A(\rho))\neq0$ and $\mathcal O(\rho)\neq0$, i.e. no cancellation can occur at $\rho$. Local-uniform analyticity on $\Omega$ follows from HS$\to\dettwo$ continuity (Proposition~\ref{prop:hs-det2-continuity}).
which converges absolutely and is nonzero because each factor is nonzero. Moreover, $I-A(s)$ is invertible with $\|(I-A(s))^{-1}\|\le (1-2^{-\sigma})^{-1}$ since $|1-p^{-s}|\ge 1-2^{-\sigma}>0$.
Finally, the canonical outer normalizer $\mathcal O_{\mathrm{can}}$ is an outer function on $\Omega$ (Definition~\ref{def:canonical-normalizer}), hence is zero--free on $\Omega$.
Thus if $\rho\in\Omega$ with $\xi(\rho)=0$, then $\det_2(I-A(\rho))\neq0$ and $\mathcal O_{\mathrm{can}}(\rho)\neq0$, i.e. no cancellation can occur at $\rho$.
Local-uniform analyticity on $\Omega$ follows from HS$\to\dettwo$ continuity (Proposition~\ref{prop:hs-det2-continuity}).
\begin{lemma}[Diagonal HS determinant is analytic and nonzero]\label{lem:hs-diagonal}
For $s=\sigma+it$ with $\sigma>\tfrac12$, the diagonal operator $A(s)e_p=p^{-s}e_p$ satisfies
\[
\sup_{p}|p^{-s}|=2^{-\sigma}<1,\qquad \sum_{p}|p^{-s}|^2=\sum_{p}p^{-2\sigma}<\infty.
\]
Hence $A(s)\in\mathrm{HS}$, $I-A(s)$ is invertible, and
\[
\det\nolimits_2\big(I-A(s)\big)=\prod_{p}(1-p^{-s})\,e^{p^{-s}}
\]
is analytic and nonzero on $\{\Re s>\tfrac12\}$.
\end{lemma}
\begin{proof}
Immediate from the displayed bounds; invertibility follows since $|1-p^{-s}|\ge 1-2^{-\sigma}>0$, and the product defining $\det_2$ converges absolutely with nonzero factors.
\end{proof}
\paragraph{Normalization and finite port (eliminating $C_P$ and $C_\Gamma$).}
We record the implementation details that ensure the product certificate contains no prime budget and no Archimedean term.

\begin{lemma}[\(\zeta\)–normalized outer and compensator]\label{lem:zeta-normalization}
Define the outer $\mathcal O_\zeta$ on $\Omega$ with boundary modulus $\big|\dettwo(I-A)/\zeta\big|$ and set
\[ J_\zeta(s)\ :=\ \frac{\dettwo(I-A(s))}{\mathcal O_\zeta(s)\,\zeta(s)}\cdot B(s),\qquad B(s):=\frac{s}{s-1}. \]
On $\Re s=\tfrac12$ one has $|B|=1$. The phase–velocity identity of Theorem~\ref{thm:phase-velocity-quant} holds for $J_\zeta$ with the same Poisson/zero right-hand side. In particular, no separate Archimedean term enters the inequality used by the certificate.
\end{lemma}

\begin{proof}
Set $X:=\xi$ and $Z:=\zeta$, and let $G$ denote the archimedean factor linking them,
\[
  X(s)\;=\;\tfrac12 s(1{-}s)\,\pi^{-s/2}\,\Gamma(\tfrac s2)\,Z(s)\;=:\;G(s)\,Z(s).
\]
Define $\mathcal O_X$ (resp. $\mathcal O_Z$) to be the outer on $\Omega$ with boundary modulus $\big|\dettwo(I{-}A)/X\big|$ (resp. $\big|\dettwo(I{-}A)/Z\big|$). Then, by construction,
\[
  \Big|\frac{\dettwo(I{-}A)}{\mathcal O_X\,X}\Big|\equiv 1\equiv \Big|\frac{\dettwo(I{-}A)}{\mathcal O_Z\,Z}\Big|\quad \text{a.e. on }\{\Re s=\tfrac12\}.
\]
Consequently the phase–velocity identity (Theorem~\ref{thm:phase-velocity-quant}) applies to either unimodular ratio. Writing
\[
  \log \frac{\dettwo(I{-}A)}{\mathcal O_X\,X}
  \;=\; \log \frac{\dettwo(I{-}A)}{\mathcal O_Z\,Z}\; -\; \log\frac{\mathcal O_X}{\mathcal O_Z}\; -\; \log G,
\]
and differentiating in $\sigma$ on the boundary, the two outer terms contribute zero to the boundary phase derivative (by unimodularity and the outer/Poisson representation). The remaining difference is $-\partial_\sigma\Im\log G$.

On $\Re s=\tfrac12$ we have $|O_X/O_Z|=|Z/X|=|1/G|$, hence (a.e.) $\Re\log(O_X/O_Z)=-\Re\log G$. Since both $\log(O_X/O_Z)$ and $\log G$ are analytic on $\Omega$, Cauchy–Riemann gives on the boundary line (in $\mathcal D'(\R)$)
\[
  \partial_\sigma\Im\log\!\left(\frac{O_X}{O_Z}\right)
  \,=\,-\partial_t\Re\log\!\left(\frac{O_X}{O_Z}\right)
  \,=\,-\partial_t(-\Re\log G)
  \,=\,-\partial_\sigma\Im\log G.
\]
Compensating the simple zero at $s=1$ of $\dettwo(I-A)/\zeta$ by the half–plane compensator
\[
  B(s)\;=\;\frac{s}{s-1}\qquad(|B|\equiv 1\text{ on }\Re s=\tfrac12)
\]
accounts for the inner contribution at $s=1$. Therefore, on the boundary,
\[
  \partial_\sigma\Im\log\!\Big(\frac{\dettwo(I{-}A)}{\mathcal O_Z\,Z}\cdot B\Big)
  \,=\, \partial_\sigma\Im\log\frac{\dettwo(I{-}A)}{\mathcal O_X\,X},
\]
and the quantitative phase–velocity identity holds in the same form for $J_\zeta=(\dettwo/(\mathcal O_\zeta\,\zeta))\,B$ as for $\mathcal J=\dettwo/(\mathcal O\,\xi)$. In particular, no Archimedean term enters the certificate.
\end{proof}

% (archived) A standalone prime-layer outer O_p is not used in the main chain; the ζ-normalized gauge and windowed identities suffice, and no C_P term enters the certificate.

\begin{corollary}[No $C_P$/$C_\Gamma$ in the certificate]
With $J_\zeta$ and $\widehat J$ as above, the active CR–Green route uses $c_0(\psi)$ and the CR–Green constant $C(\psi)$ together with the box–energy constant $C_{\rm box}^{(\zeta)}$. In particular, $C_P=0$ and $C_\Gamma=0$ on the RHS; $C_H(\psi)$ and $M_\psi$ are retained only as auxiliary/readability bounds.
\end{corollary}
\begin{proof}
By construction of the $\zeta$--normalized gauge and the compensator $B$ (Lemma~\ref{lem:zeta-normalization}), the Archimedean factor contributes no boundary phase term and the simple pole/zero bookkeeping at $s=1$ is absorbed into $B$ with $|B|=1$ on $\Re s=\tfrac12$. Thus the product certificate has no $C_\Gamma$ term and no separate prime-budget term $C_P$ on the right-hand side; the remaining inputs are $c_0(\psi)$, the CR--Green constant $C(\psi)$, and the box-energy constant $C_{\rm box}^{(\zeta)}$.
\end{proof}

\noindent\emph{Active route.} Throughout we use the $\zeta$-normalized boundary gauge with the Blaschke compensator; the product certificate uses $c_0(\psi)$ and the CR–Green constant $C(\psi)$ together with $C_{\rm box}^{(\zeta)}$ (no $C_P$, no $C_\Gamma$). These inputs yield Whitney-local smallness $\Upsilon_{\rm Whit}(c)<\tfrac12$ (Lemma~\ref{lem:whitney-uniform-wedge}); the remaining promotion to a global a.e.\ boundary wedge \textup{(P+)} after a single rotation is isolated in Remark~\ref{rem:wedge-application}.

% (Removed alternative interior-pole lemma to keep a single contradiction path in the pinch.)

\begin{lemma}[Derivative envelope for the printed window]\label{lem:CH-derivative-explicit}
Let $\psi$ be the even $C^\infty$ flat--top window from the "Printed window" paragraph (equal to $1$ on $[-1,1]$, supported in $[-2,2]$, with monotone ramps on $[-2,-1]$ and $[1,2]$), and $\varphi_L(t):=L^{-1}\psi((t-T)/L)$. Then, for every $L>0$,
\[
  \big\|\big(\mathcal H[\varphi_L]\big)'\big\|_{L^\infty(\mathbb R)} \;\le\; \frac{C_H(\psi)}{L}
  \qquad\text{with}\qquad C_H(\psi)\;\le\;\frac{2}{\pi}\;<\;0.65.
\]
\end{lemma}
\begin{proof}
\textit{Step 1 (Scaling).} By the standard scale/translation identity (recorded in the manuscript),
\[
  \mathcal H[\varphi_L](t)=H_\psi\!\Big(\frac{t-T}{L}\Big),\qquad
  H_\psi(x):=\frac{1}{\pi}\,\mathrm{p.v.}\!\int_{\mathbb R}\frac{\psi(y)}{x-y}\,dy,
\]
we get
\[
  \big(\mathcal H[\varphi_L]\big)'(t)=\frac{1}{L}\,H_\psi'\!\Big(\frac{t-T}{L}\Big)
  \quad\Longrightarrow\quad
  \big\|\big(\mathcal H[\varphi_L]\big)'\big\|_\infty=\frac{1}{L}\,\|H_\psi'\|_\infty.
\]
Thus it suffices to bound $\|H_\psi'\|_\infty$.

\smallskip
\textit{Step 2 (Structure and signs).} Since $\psi'\equiv0$ on $(-1,1)$ and the ramps are monotone,
\[
  \psi'(y)\ge0\ \text{on }[-2,-1],\qquad \psi'(y)\le0\ \text{on }[1,2],\qquad
  \int_{-2}^{-1}\psi'(y)\,dy=1=\!-\!\int_{1}^{2}\psi'(y)\,dy.
\]
In distributions, $(H_\psi)'= \mathcal H[\psi']$, so for every $x\in\mathbb R$
\[
  H_\psi'(x)=\frac{1}{\pi}\,\mathrm{p.v.}\!\int_{-2}^{-1}\frac{\psi'(y)}{x-y}\,dy\;+\;
             \frac{1}{\pi}\,\mathrm{p.v.}\!\int_{1}^{2}\frac{\psi'(y)}{x-y}\,dy.
\]

\smallskip
\textit{Step 3 (Worst case occurs between the ramps).} Fix $x\in(-1,1)$.  On $y\in[-2,-1]$ the kernel $y\mapsto 1/(x-y)$ is positive and strictly increasing; on $y\in[1,2]$ the kernel is negative and strictly decreasing.  Since the ramp densities are monotone and have unit mass in absolute value, the rearrangement/endpoint principle (maximize a monotone–kernel integral by concentrating mass at an endpoint) gives the pointwise bound
\[
  \Big|\mathrm{p.v.}\!\int_{-2}^{-1}\frac{\psi'(y)}{x-y}\,dy\Big|
  \le \frac{1}{1+x},\qquad
  \Big|\mathrm{p.v.}\!\int_{1}^{2}\frac{\psi'(y)}{x-y}\,dy\Big|
  \le \frac{1}{1-x}.
\]
Therefore, for every $x\in(-1,1)$,
\[
  |H_\psi'(x)| \;\le\; \frac{1}{\pi}\Big(\frac{1}{1+x}+\frac{1}{1-x}\Big)
  \;\le\; \frac{2}{\pi}\,\frac{1}{1-x^2}
  \;\le\; \frac{2}{\pi},
\]
with the maximum at $x=0$.
\smallskip
\textit{Step 4 (Outside the plateau).} For $x\notin[-1,1]$ the two ramp contributions have opposite signs but larger denominators, hence smaller magnitude. More precisely, for $x>1$, the left–ramp integral is a principal value on $[-2,-1]$ against a $C^\infty$ density that vanishes at the endpoints; the standard $C^1$–vanishing at $y=-2,-1$ eliminates the endpoint singularity and keeps the PV finite and strictly smaller than its in–plateau counterpart (a short integration–by–parts argument on the left interval makes this explicit). By evenness, the same holds for $x<-1$.  Consequently,
\[
  \sup_{x\in\mathbb R}|H_\psi'(x)|=\sup_{x\in(-1,1)}|H_\psi'(x)|\;\le\;\frac{2}{\pi}.
\]
Putting Steps 1–4 together,
\[
  \big\|\big(\mathcal H[\varphi_L]\big)'\big\|_\infty
  \;=\;\frac{1}{L}\,\|H_\psi'\|_\infty
  \;\le\;\frac{1}{L}\cdot\frac{2}{\pi}.
\]
Hence we can take $C_H(\psi)\le 2/\pi < 0.65$.
\end{proof}

% removed stray fi
% =========================================================

% Diagnostic numerics (gated; non-load-bearing).
\ifshownumerics
\paragraph*{Certificate \textemdash{} weighted \(p\)-adaptive model at \(\sigma_0=0.6\).}
Fix \(\sigma_0=0.6\), take \(Q=29\) and \(p_{\min}=\mathrm{nextprime}(Q)=31\).\\
Use the \(p\)-adaptive weighted off-diagonal enclosure (for all \(p\neq q\), uniformly in \(\sigma\in[\sigma_0,1]\)):
\[
\|H_{pq}(\sigma)\|_2 \;\le\; \frac{C_{\mathrm{win}}}{4}\, p^{-(\sigma+\tfrac12)}\, q^{-(\sigma+\tfrac12)},
\qquad C_{\mathrm{win}}=0.25.
\]

\noindent\emph{Prime sums (small block \(p\le Q\)).} With \(\sigma_0=0.6\),
\[
S_{\sigma_0}(Q)\;=\;\sum_{p\le Q} p^{-\sigma_0}\;=\;2.9593220929,\qquad
S_{\sigma_0+\tfrac12}(Q)\;=\;\sum_{p\le Q} p^{-(\sigma_0+\tfrac12)}\;=\;1.3239981250.
\]

% alt-route Bridges/KYP removed from main body
% removed optional Bridges A--C discussion and references
\noindent\emph{In-block Gershgorin lower bounds (uniform on \([\sigma_0,1]\)).}
Define
\[
L(p)\;:=\;(1-\sigma_0)\,(\log p)\,p^{-\sigma_0},\qquad 
\mu_p^{\mathrm L}\;\ge\;1-\frac{L(p)}{6}.
\]
At \(p_{\min}=31\) this gives
\[
L(31)=0.1750014502,\qquad 
\mu_{\min}^{\mathrm{far}}\;:=\;1-\frac{L(31)}{6}\;=\;0.9708330916.
\]
Over the small block \(p\le Q\) the worst case is at \(p=5\):
\[
L(5)=0.2451050257,\qquad 
\mu_{\min}^{\mathrm{small}}\;:=\;1-\frac{L(5)}{6}\;=\;0.9591491624.
\]
\noindent\emph{Off-diagonal budgets (all rigorous).}
Let \(\sigma^\star:=\sigma_0+\tfrac12=1.1\).\\
With the integer-tail majorant \(\displaystyle \sum_{n\ge p_{\min}-1} n^{-\sigma^\star}\le
\frac{(p_{\min}-1)^{1-\sigma^\star}}{\sigma^\star-1}\)
we obtain:
\[
\Delta_{\mathrm{FS}}
=\frac{C_{\mathrm{win}}}{4}\,p_{\min}^{-\sigma^\star}\,S_{\sigma^\star}(Q)
=0.0018935184,
\]
\[
\Delta_{\mathrm{FF}}
=\frac{C_{\mathrm{win}}}{4}\,p_{\min}^{-\sigma^\star}\!
\sum_{n\ge p_{\min}-1}\! n^{-\sigma^\star}
\;\le\;\frac{C_{\mathrm{win}}}{4}\,p_{\min}^{-\sigma^\star}\,
\frac{(p_{\min}-1)^{1-\sigma^\star}}{\sigma^\star-1}
=0.0101781777,
\]
\[
\Delta_{\mathrm{SS}}
=\frac{C_{\mathrm{win}}}{4}\,2^{-\sigma^\star}
\!\sum_{\substack{p\le Q\\ p\neq 2}}\! p^{-\sigma^\star}
=0.0250018328,
\]
\[
\Delta_{\mathrm{SF}}
=\frac{C_{\mathrm{win}}}{4}\,2^{-\sigma^\star}\!
\sum_{n\ge p_{\min}-1}\! n^{-\sigma^\star}
\;\le\;\frac{C_{\mathrm{win}}}{4}\,2^{-\sigma^\star}\,
\frac{(p_{\min}-1)^{1-\sigma^\star}}{\sigma^\star-1}
=0.2075080249.
\]
\noindent\emph{Certified finite-block spectral gap.}
Combining the in-block lower bounds with the off-diagonal budgets yields
\[
\delta_{\mathrm{cert}}(\sigma_0)\;\ge\;
\min\Big\{
\underbrace{\mu_{\min}^{\mathrm{small}}-(\Delta_{\mathrm{SS}}+\Delta_{\mathrm{SF}})}_{\text{small-block rows}}\,,\;
\underbrace{\mu_{\min}^{\mathrm{far}}-(\Delta_{\mathrm{FS}}+\Delta_{\mathrm{FF}})}_{\text{far-block rows}}\
\Big\}
=0.7266393047\;>\;0.
\]
Hence the normalized finite block is uniformly positive definite on \([\sigma_0,1]\).
\fi
\begin{corollary}[Boundary-uniform smoothed control]\label{cor:det2-boundary}
Let $I\Subset\R$, $\varepsilon_0\in(0,\tfrac12]$, and $\varphi\in C_c^2(I)$. Then, uniformly for $\sigma\in(\tfrac12,\tfrac12+\varepsilon_0]$,
\[
  \Big|\int_{\R} \varphi(t)\,\partial_\sigma\,\Re\log\dettwo\big(I-A(\sigma+it)\big)\,dt\Big|\ \le\ C_*\,\|\varphi''\|_{L^1(I)}.
\]
In particular, the bound remains valid in the boundary limit $\sigma\downarrow \tfrac12$ in the sense of distributions.
\end{corollary}
\begin{proof}
This is exactly the tested bound from Lemma~\ref{lem:det2-unsmoothed} (uniform in $\sigma\in(0,\varepsilon_0]$ after the shift $\sigma\mapsto \tfrac12+\sigma$). Since the right-hand side is uniform in $\sigma$, the family of distributions $\sigma\mapsto \partial_\sigma\Re\log\dettwo(I-A(\tfrac12+\sigma+it))$ is bounded in $\mathcal D'(I)$ and the estimate persists in the boundary limit $\sigma\downarrow\tfrac12$ when tested against $\varphi$.
\end{proof}
\subsection*{Smoothed Cauchy and outer limit (A2)}
% Lead-in: We build outers from boundary data u_ε and pass to a locally-uniform outer limit to normalize the boundary modulus.
\begin{proposition}[Outer normalization: existence, boundary a.e. modulus, and limit]\label{prop:outer-central}
There exist outer functions \(\mathcal O_\varepsilon\) on \(\{\Re s>\tfrac12+\varepsilon\}\) with a.e. boundary modulus
\[
  \big|\mathcal O_\varepsilon(\tfrac12+\varepsilon+it)\big|\ =\ \exp\big(u_\varepsilon(t)\big),
\]
and \(\mathcal O_\varepsilon\to\mathcal O\) locally uniformly on \(\Omega\) as \(\varepsilon\downarrow 0\), where \(\mathcal O\) has boundary modulus \(\exp u(t)\). (Standard Poisson–outer representation; see, e.g., \cite[Ch.~2]{DurenHp} and \cite[Ch.~2]{RosenblumRovnyak}.) Consequently the outer-normalized ratio \(\mathcal J=\dettwo(I-A)/(\mathcal O\,\xi)\) has a.e. boundary values on \(\Re s=\tfrac12\) with \(|\mathcal J(\tfrac12+it)|=1\).
\end{proposition}
\begin{proof}
Existence of each outer $\mathcal O_\varepsilon$ with the stated boundary modulus is standard. The Carleson-energy control for the relevant harmonic log-modulus on Whitney boxes implies the existence of a boundary trace $u\in \mathrm{BMO}(\R)\subset L^1_{\mathrm{loc}}(\R)$ and convergence $u_\varepsilon\to u$ in $L^1_{\mathrm{loc}}$ (Lemma~\ref{lem:desmooth-L1}). The Poisson/outer representation then gives local-uniform convergence $\mathcal O_\varepsilon\to\mathcal O$ on $\Omega$ and the unimodularity $|\mathcal J(\tfrac12+it)|=1$ a.e.
\end{proof}
\subsection*{Carleson energy and boundary BMO (unconditional)}
We record a direct Carleson–energy route to boundary BMO for the limit $u(t)=\lim_{\varepsilon\downarrow 0}u_\varepsilon(t)$.

\begin{lemma}[Arithmetic Carleson energy]\label{lem:carleson-arith}
Let
\[
 U_{\det_2}(\sigma,t)\ :=\ \sum_{p}\sum_{k\ge 2}\frac{(\log p)\,p^{-k/2}}{k\log p}\,e^{-k\log p\,\sigma}\,\cos\big(k\log p\,t\big),\qquad \sigma>0.
\]
Then for every interval $I\subset\R$ with Carleson box $Q(I):=I\times(0,|I|]$
\[
 \iint_{Q(I)} |\nabla U_{\det_2}|^2\,\sigma\,dt\,d\sigma\ \le\ \frac{|I|}{4}\,\sum_{p}\sum_{k\ge 2}\frac{p^{-k}}{k^2}
 \ =:\ K_0\,|I|,\qquad K_0:=\frac{1}{4}\sum_{p}\sum_{k\ge 2}\frac{p^{-k}}{k^2}<\infty.
\]
\end{lemma}
\begin{proof}
For a single mode $b\,e^{-\omega\sigma}\cos(\omega t)$ one has $|\nabla|^2=b^2\omega^2e^{-2\omega\sigma}$, hence
\[
 \int_0^{|I|}\!\int_I |\nabla|^2\,\sigma\,dt\,d\sigma\ \le\ |I|\cdot\sup_{\omega>0}\int_0^{|I|}\sigma\,\omega^2e^{-2\omega\sigma}d\sigma\cdot b^2\ \le\ \tfrac14\,|I|\,b^2.
\]
With $b=(\log p)\,p^{-k/2}/(k\log p)$ and $\omega=k\log p$, summing over $(p,k)$ gives the claim and the finiteness of $K_0$.
\end{proof}
\paragraph{Whitney scale and short–interval zeros.}
Throughout we use the Whitney schedule clipped at $L_\star$:
\[
  L\ =\ L(T)\ :=\ \frac{c}{\log\langle T\rangle}\ \le\ \frac{1}{\log\langle T\rangle},\qquad \langle T\rangle:=\sqrt{1+T^2},\
\]
for a fixed absolute $c\in(0,1]$; all boxes are $Q(\alpha I)$ with a uniform $\alpha\in[1,2]$.
We work on Whitney boxes $Q(I)$ with
\[
  L=L(T):=\min\Big\{\frac{c}{\log\langle T\rangle},\ L_\star\Big\},\qquad \langle T\rangle:=\sqrt{1+T^2},\quad c>0\ \text{fixed}.
\]
There exist absolute $A_0,A_1>0$ such that for $T\ge2$ and $0<H\le1$,
\[
  N(T;H)\ :=\ \#\{\rho=\beta+i\gamma:\ \gamma\in[T,T+H]\}\ \le\ A_0\ +\ A_1\,H\log\langle T\rangle.
\]
\begin{lemma}[Annular Poisson–balayage $L^2$ bound]\label{lem:annular-balayage}
Let $I=[T-L,T+L]$, $Q_\alpha(I)=I\times(0,\alpha L]$, and fix $k\ge1$. For
$\mathcal A_k:=\{\rho=\beta+i\gamma:\ 2^kL<|T-\gamma|\le 2^{k+1}L\}$ set
\[
  V_k(\sigma,t):=\sum_{\rho\in\mathcal A_k}\frac{\sigma}{(t-\gamma)^2+\sigma^2}.
\]
Then
\[
  \iint_{Q_\alpha(I)} V_k(\sigma,t)^2\,\sigma\,dt\,d\sigma\ \ll_\alpha\ |I|\,4^{-k}\,\nu_k,
\]
where $\nu_k:=\#\mathcal A_k$, and the implicit constant depends only on $\alpha$.
\end{lemma}
\begin{proof}
Write $K_\sigma(x):=\sigma/(x^2+\sigma^2)$ and $V_k=\sum_{\rho\in\mathcal A_k}K_\sigma(\cdot-\gamma)$. For any finite index set $\mathcal J$,
\[
  V_k^2\;\le\; \sum_{j\in\mathcal J} K_\sigma(\cdot-\gamma_j)^2\ +\ 2\!\!\sum_{i<j} K_\sigma(\cdot-\gamma_i)K_\sigma(\cdot-\gamma_j).
\]
Integrate over $t\in I$ first. For the diagonal terms, using $|t-\gamma|\ge 2^kL-L\ge 2^{k-1}L$ for $t\in I$ and $k\ge 1$,
\[
 \int_I K_\sigma(t-\gamma)^2\,dt\ =\ \int_I \frac{\sigma^2}{\big((t-\gamma)^2+\sigma^2\big)^2}\,dt
 \ \le\ \frac{\sigma}{(2^{k-1}L)^2}\int_I \frac{\sigma}{(t-\gamma)^2+\sigma^2}\,dt
 \ \le\ \frac{\pi\,\sigma}{(2^{k-1}L)^2}.
\]
Multiplying by the area weight $\sigma$ and integrating $\sigma\in(0,\alpha L]$ gives
\[
 \int_0^{\alpha L}\!\!\left(\int_I K_\sigma(t-\gamma)^2\,dt\right)\sigma\,d\sigma
 \ \le\ \frac{\pi}{(2^{k-1}L)^2}\int_0^{\alpha L}\!\sigma^2 d\sigma
 \ =\ \frac{\pi\,\alpha^3}{3}\,\frac{L}{4^{k-1}}
 \ \le\ \frac{C_{\mathrm{diag}}(\alpha)}{4^{k}}\,|I|,
\]
with $C_{\mathrm{diag}}(\alpha):=\tfrac{8\pi\alpha^3}{3}$ (using $|I|=2L$). Summing over $\nu_k$ choices of $\gamma$ contributes a factor $\nu_k$.

For the off-diagonal terms, for $i\ne j$ one has on $I$ that $K_\sigma(t-\gamma_j)\le \sigma/(2^{k-1}L)^2$. Hence
\[
 \int_I K_\sigma(t-\gamma_i)K_\sigma(t-\gamma_j)\,dt\ \le\ \frac{\sigma}{(2^{k-1}L)^2}\int_\R K_\sigma(t-\gamma_i)\,dt\ =\ \frac{\pi\sigma}{(2^{k-1}L)^2},
\]
and integrating $\sigma\in(0,\alpha L]$ with the extra factor $\sigma$ yields $\le C'_{\mathrm{off}}(\alpha)\,L\cdot 4^{-k}$. Summing in $i,j$ via the Schur test with $f_j(t):=K_\sigma(t-\gamma_j)\mathbf 1_I(t)$ gives
\[
 \int_I V_k(\sigma,t)^2\,dt\ \le\ C''(\alpha)\,\nu_k\,\frac{\sigma}{(2^kL)^2}.
\]
(This is a standard positive-kernel aggregation: the off-diagonal Gram matrix for the family $\{K_\sigma(\cdot-\gamma_j)\mathbf 1_I\}_j$ is controlled by Schur’s test, using the pointwise bound $K_\sigma\lesssim \sigma/(2^kL)^2$ on $I$ and the normalization $\int_\R K_\sigma=\pi$.)
Integrating $\sigma\in(0,\alpha L]$ with weight $\sigma$ gives $\le C_{\mathrm{off}}(\alpha)\,|I|\cdot 4^{-k}\,\nu_k$. Combining diagonal and off–diagonal parts, absorbing harmless constants into $C_\alpha$, we obtain the stated bound with an explicit $C_\alpha=O(\alpha^3)$.
\end{proof}

\begin{lemma}[Analytic ($\xi$) Carleson energy on Whitney boxes]\label{lem:carleson-xi}
\emph{Reference.} The local zero count used below follows from the Riemann–von Mangoldt formula; see Titchmarsh \cite[Thm.~9.3]{Titchmarsh} (or, e.g., Ivi\'c, Ch.~8).
There exist absolute constants $c\in(0,1]$ and $C_\xi<\infty$ such that for every interval $I=[T-L,\,T+L]$ with Whitney scale $L:=c/\log\langle T\rangle$, the Poisson extension
\[
 U_{\xi}(\sigma,t):=\Re\log\xi\big(\tfrac12+\sigma+it\big),\qquad (\sigma>0),
\]
\paragraph{Whitney scale and neutralization.}
Throughout this lemma we take the base interval $I=[T-L,T+L]$ with
\[
  L=L(T):=\frac{c}{\log\langle T\rangle},\qquad \langle T\rangle:=\sqrt{1+T^2},\quad c>0\ \text{fixed}.
\]
obeys the Carleson bound
\[ \iint_{Q(I)} |\nabla U_{\xi}(\sigma,t)|^2\,\sigma\,dt\,d\sigma\ \le\ C_\xi\,|I|. \]
\end{lemma}

\begin{proof}
All inputs are unconditional. Fix $I=[T-L,T+L]$ with $L=c/\log\langle T\rangle$ and aperture $\alpha\in[1,2]$. Neutralize near zeros by a local half-plane Blaschke product $B_I$ removing zeros of $\xi$ inside a fixed dilate $Q(\alpha'I)$ ($\alpha'>\alpha$). This yields a harmonic field $\widetilde U_\xi$ on $Q(\alpha I)$ and
\[
  \iint_{Q(\alpha I)} |\nabla U_\xi|^2\,\sigma\,dt\,d\sigma\ \asymp\ \iint_{Q(\alpha I)} |\nabla \widetilde U_\xi|^2\,\sigma\,dt\,d\sigma\ +\ O_\alpha(|I|),
\]
so it suffices to bound the neutralized energy.

Write $\partial_\sigma U_\xi=\Re\,(\xi'/\xi)=\Re\sum_\rho (s-\rho)^{-1}+A$, where $A$ is smooth on compact strips. Since $U_\xi$ is harmonic, $|\nabla U_\xi|^2\asymp |\partial_\sigma U_\xi|^2$ on $\R^2_+$; thus we bound the $L^2(\sigma\,dt\,d\sigma)$ norm of $\sum_\rho (s-\rho)^{-1}$ over $Q(\alpha I)$. Decompose the (neutralized) zeros into Whitney annuli $\mathcal A_k:=\{\rho:2^kL<|\gamma-T|\le 2^{k+1}L\}$, $k\ge1$. For $V_k(\sigma,t):=\sum_{\rho\in\mathcal A_k} K_\sigma(t-\gamma)$ with $K_\sigma(x):=\sigma/(x^2+\sigma^2)$, Lemma~\ref{lem:annular-balayage} gives
\[
  \iint_{Q_\alpha(I)} V_k(\sigma,t)^2\,\sigma\,dt\,d\sigma\ \le\ C_\alpha\,|I|\,4^{-k}\,\nu_k,
\]
where $\nu_k:=\#\mathcal A_k$ and $C_\alpha$ depends only on $\alpha$. Summing Cauchy–Schwarz bounds over annuli yields
\[
  \iint_{Q(\alpha I)} \Big|\sum_{\rho}(s-\rho)^{-1}\Big|^2\,\sigma\,dt\,d\sigma\ \le\ C_\alpha\,|I|\sum_{k\ge1}4^{-k}\,\nu_k.
\]
To bound $\nu_k$, we use the short-interval zero count recorded above: there exist absolute $A_0,A_1>0$ such that for $T\ge 2$ and $0<H\le 1$,
\[
  N(T;H)\ :=\ \#\{\rho=\beta+i\gamma:\ \gamma\in[T,T+H]\}\ \le\ A_0\ +\ A_1\,H\log\langle T\rangle.
\]
For annuli with $2^kL\le 1$, $\nu_k$ counts zeros in a window of length $\asymp 2^kL$, hence
\[
  \nu_k\ \le\ a_0(\alpha)\ +\ a_1(\alpha)\,2^kL\,\log\langle T\rangle.
\]
For the finitely many remaining annuli with $2^kL>1$, the Riemann--von Mangoldt formula (Titchmarsh \cite[Thm.~9.3]{Titchmarsh}) gives the cruder bound $\nu_k\ll_\alpha 2^kL\,\log\langle T\rangle$, which is sufficient since $4^{-k}\nu_k$ is summable. Therefore,
\[
  \sum_{k\ge1}4^{-k}\,\nu_k
  \ \ll_\alpha\ 
  \sum_{k\ge1}4^{-k}\Big(1+2^kL\,\log\langle T\rangle\Big)
  \ \ll\ 1\ +\ L\,\log\langle T\rangle.
\]
On Whitney scale $L=c/\log\langle T\rangle$ this is $\ll_c 1$.
Adding the neutralized near-field $O(|I|)$ and the smooth $A$ contribution, we obtain
\[
  \iint_{Q(\alpha I)} |\nabla U_\xi|^2\,\sigma\,dt\,d\sigma\ \le\ C_\xi\,|I|,
\]
with $C_\xi$ depending only on $(\alpha,c)$. This proves the lemma.
\end{proof}

\begin{proposition}[Whitney Carleson finiteness for $U_\xi$]\label{prop:Kxi-finite}
For each fixed Whitney aperture $\alpha\in[1,2]$ there exists a finite constant
$K_\xi=K_\xi(\alpha)<\infty$ such that
\[
  \iint_{Q(\alpha I)} |\nabla U_\xi|^2\,\sigma\,dt\,d\sigma \;\le\; K_\xi\,|I|
\]
for every Whitney base interval $I$. Consequently $C_{\rm box}^{(\zeta)}=K_0+K_\xi<\infty$, and
\[
  c \;\le\; \Big(\tfrac{c_0(\psi)}{2\,C(\psi)\,\sqrt{K_0+K_\xi}}\Big)^2
\]
ensures $\Upsilon_{\mathrm{Whit}}(c)<\tfrac12$ and provides the required Whitney-local smallness parameter for Lemma~\ref{lem:whitney-uniform-wedge}. (A global a.e. boundary wedge \textup{(P+)} still requires the local-to-global upgrade discussed in Remark~\ref{rem:wedge-application}.)
\end{proposition}
\begin{proof}
The Whitney-box estimate for $U_\xi$ is exactly Lemma~\ref{lem:carleson-xi}; take $K_\xi$ to be the constant there (for the fixed aperture $\alpha$). The finiteness of $C_{\rm box}^{(\zeta)}$ then follows by combining the prime-tail box bound $K_0$ (Lemma~\ref{lem:carleson-arith}) with the stable-sum estimate (Lemma~\ref{lem:carleson-sum}). The final inequality is the stated sufficient smallness condition in Lemma~\ref{lem:whitney-uniform-wedge}.
\end{proof}

\paragraph{Boxed audit: unconditional enclosure of $C_{\rm box}^{(\zeta)}$.}
Fix $I=[T-L,T+L]$ with $L=c/\log\langle T\rangle$ and $Q(I)=I\times(0,L]$. Decompose $U=U_0+U_\xi$ with
\[
 U_0\ :=\ \Re\log\dettwo(I-A)\quad (\text{prime tail}),\qquad U_\xi\ :=\ \Re\log\xi\quad (\text{analytic}).
\]
\emph{Prime tail.} Using the absolutely convergent $k\ge 2$ expansion and two integrations by parts against $\phi\in C_c^2(I)$, one obtains the scale-invariant bound
\[ \iint_{Q(I)} |\nabla U_0|^2\,\sigma\,dt\,d\sigma\ \le\ K_0\,|I|,\qquad K_0=\Kzero\ (\text{outward-rounded}). \]
\emph{Zeros (neutralized).} Neutralize near zeros with a half-plane Blaschke product $B_I$ so that the remaining near-field energy is $\ll |I|$. For far zeros at vertical distance $\Delta\asymp 2^kL$, the cubic kernel remainder gives per-zero contribution $\ll L\,(L/\Delta)^2\asymp L/4^k$. Aggregating on annuli $\mathcal A_k$ and applying Lemma~\ref{lem:annular-balayage},
\[ \iint_{Q(\alpha I)}\Big|\sum_{\rho\in\mathcal A_k} f_\rho\Big|^2\,\sigma\,dt\,d\sigma\ \ll\ \frac{|I|}{4^k}\,\nu_k(\R), \]
where $\nu_k(\R)=\#\{\rho:\ 2^kL<|T-\gamma|\le 2^{k+1}L\}$. By the unconditional zero-density bounds of Vinogradov–Korobov (with explicit constants), for each fixed Whitney scale one has a uniform count
\[
  \nu_k(\R)\ \ll\ 1\ +\ 2^kL\log\langle T\rangle,
\]
using the short-interval zero count $N(T;H)\le A_0+A_1H\log\langle T\rangle$ for $H\le 1$ (and a crude Riemann--von Mangoldt bound for the finitely many annuli with $2^kL>1$). The implied constant is independent of $T$ and $k$.
Summing $k\ge 1$ and using $L=c/\log\langle T\rangle$ gives
\[ \iint_{Q(\alpha I)} |\nabla U_\xi|^2\,\sigma\,dt\,d\sigma\ \le\ K_\xi\,|I|,\qquad \text{for a finite constant }K_\xi. \]
\medskip
\noindent\fbox{\begin{minipage}{0.98\textwidth}
\textbf{Boxed $K_\xi$ audit (parametric; diagnostic).} With $C_\alpha$ from Lemma~\ref{lem:annular-balayage},
\[
  K_\xi \ \le\ C_\alpha\!\left(\frac{1}{2\pi}\sum_{j\ge1} j^{-2} \ +\ 2\sum_{j\ge1} j^{-3}\right)
  \ =\ C_\alpha\!\left(\frac{\pi}{12} \ +\ 2\,\zeta(3)\right).
\]
\end{minipage}}
Combining,
\[
\boxed{\ C_{\rm box}^{(\zeta)}\ :=\ \sup_{T\in\R}\ \frac{1}{|I_T|}\iint_{Q(\alpha I_T)} |\nabla U|^2\,\sigma\,dt\,d\sigma\ \le\ K_0+K_\xi\ =\ \CboxZeta\ .\ }
\]
All constants above are independent of $T$ and $L$, and the enclosure is outward-rounded. This is the \emph{only} Carleson input used in the active certificate.
\begin{proof}
Write
\[
 \partial_\sigma U_{\xi}(\sigma,t)\ =\ \Re\frac{\xi'}{\xi}\!\left(\tfrac12+\sigma+it\right)
 \ =\ \Re\sum_{\rho}\frac{1}{\tfrac12+\sigma+it-\rho}\ +\ A(\sigma,t),
\]
where the sum runs over nontrivial zeros $\rho=\beta+i\gamma$ of $\zeta$, and $A(\sigma,t)$ collects the archimedean part and the trivial factors (these are smooth in $(\sigma,t)$ on compact strips). Since $U_{\xi}$ is harmonic, $|\nabla U_{\xi}|^2\asymp |\partial_\sigma U_{\xi}|^2$ on $\R^2_+$; it suffices to estimate the latter.

Fix $I=[T-L,T+L]$ and decompose the zero set into near and far parts relative to $Q(I)=I\times(0,L]$:
\[
 \mathcal Z_{\mathrm{near}}:=\{\rho:\ |\gamma-T|\le 2L\},\qquad \mathcal Z_{\mathrm{far}}:=\{\rho:\ |\gamma-T|>2L\}.
\]
\subsubsection*{Neutralized near field}
Let $B_I$ be the half-plane Blaschke product over zeros with $|\gamma-T|\le 3L$ and define the neutralized potential $\widetilde U_\xi:=\Re\log\big(\xi\,B_I\big)$ and its $\sigma$-derivative $\widetilde f:=\partial_\sigma\widetilde U_\xi$. Then $\sum_{\rho\in \mathcal Z_{\mathrm{near}}}\nabla f_\rho$ is canceled inside $Q(I)$ up to a boundary error controlled by the Poisson energy of $\psi$ (independent of $T,L$). Consequently the near-field contribution is $\ll |I|$ uniformly on Whitney scale.

\noindent\emph{Remark (bound used in the certificate).} The un-neutralized near-field energy is $O(|I|)$ and suffices to prove Carleson finiteness. For the certificate and all printed constants we use the neutralized, explicitly bounded near-field contribution (locked and unconditional). The coarse un-neutralized $O(1)$ bound is not used for numeric closure.

For the far zeros (neutralized field), set annuli $\mathcal A_k:=\{\rho:\ 2^kL<|\gamma-T|\le 2^{k+1}L\}$ for $k\ge1$. For a single zero at vertical distance $\Delta:=|\gamma-T|$ one has the kernel estimate
\[
 \int_0^{L}\!\int_{T-L}^{T+L} \frac{\sigma}{\sigma^2+(t-\gamma)^2}\,dt\,d\sigma\ \ll\ L\,\Big(\frac{L}{\Delta}\Big)^{\!2}.
\]
For the far annuli $\mathcal A_k$, apply Lemma~\ref{lem:annular-balayage} to the annular Poisson sums $V_k$ to control cross terms linearly in the annular mass:
\[
  \iint_{Q(\alpha I)}\Big|\sum_{\rho\in\mathcal A_k} f_{\rho}\Big|^2\,\sigma\,dt\,d\sigma\ \ll\ \frac{|I|}{4^k}\,\nu_k(\R),
\]
where $\nu_k(\R)=\#\{\rho:\ 2^kL<|T-\gamma|\le 2^{k+1}L\}$. By the unconditional zero-density bounds of Vinogradov–Korobov (with explicit constants), for each fixed Whitney scale one has a uniform count
\[ \nu_k(\R)\ \ll\ 2^kL\log\langle T\rangle\ +\ \log\langle T\rangle, \]
with the implied constant independent of $T$ and $k$.
Summing $k\ge1$ yields a total far contribution
\[ \ll\ |I|\sum_{k\ge1}\frac{1}{4^k}\big(2^kL\log\langle T\rangle+\log\langle T\rangle\big)\ \ll\ |I|\,(L\log\langle T\rangle+1), \]
which is $\ll |I|$ on the Whitney scale $L=c/\log\langle T\rangle$.

Adding the direct near-field $O(|I|)$ bound, the far-field $O(|I|)$ sum, and the smooth Archimedean term gives
\[
 \iint_{Q(\alpha I)} |\nabla U_\xi|^2\,\sigma\,dt\,d\sigma\ \ll\ |I|.
\]
This proves the claimed Carleson bound on Whitney boxes without neutralization in the energy step.
\end{proof}
\begin{remark}[VK zero-density constants and explicit $C_\xi$]
Let $N(\sigma,T)$ denote the number of zeros with $\Re\rho\ge \sigma$ and $0<\Im\rho\le T$. The Vinogradov–Korobov zero-density estimates give, for some absolute constants $C_0,\kappa>0$, that
\[
  N(\sigma,T)\ \le\ C_0\,T\,\log T\ +\ C_0\,T^{1-\kappa(\sigma-1/2)}\qquad (\tfrac12\le \sigma<1,\ T\ge T_1),
\]
with an effective threshold $T_1$. On Whitney scale $L=c/\log\langle T\rangle$, these bounds imply the annular counts used above with explicit $A,B$ of size $\ll 1$ for each fixed $c,\alpha$. Consequently, one can take
\[
  C_\xi\ \le\ C(\alpha,c)\,\big(C_0+1\big)
\]
in Lemma~\ref{lem:carleson-xi}, where $C(\alpha,c)$ is an explicit polynomial in $\alpha$ and $c$ arising from the annular $L^2$ aggregation (cf. Lemma~\ref{lem:annular-balayage}). We do not need the sharp exponents; any effective VK pair $(C_0,\kappa)$ suffices for a finite $C_\xi$ on Whitney boxes.
\end{remark}
% Active version of the cutoff pairing lemma (unarchived for references)
\begin{lemma}[Cutoff pairing on boxes]\label{lem:cutoff-pairing}
Fix parameters $\alpha'>\alpha>1$. Let $\chi_{L,t_0}\in C_c^\infty(\R^2_+)$ satisfy $\chi\equiv1$ on $Q(\alpha I)$, $\operatorname{supp}\chi\subset Q(\alpha'I)$, $\|\nabla\chi\|_\infty\lesssim L^{-1}$ and $\|\nabla^2\chi\|_\infty\lesssim L^{-2}$. Let $V_{\psi,L,t_0}$ be the Poisson extension of $\psi_{L,t_0}$ and $\widetilde U$ the neutralized field. Then
\[
 \int_{\R} u(t)\,\psi_{L,t_0}(t)\,dt
 \ =\ \iint_{Q(\alpha'I)} \nabla \widetilde U\cdot \nabla\big(\chi_{L,t_0}\, V_{\psi,L,t_0}\big)\,dt\,d\sigma\ +\ \mathcal R_{\mathrm{side}}\ +\ \mathcal R_{\mathrm{top}},
\]
with
\[
 |\mathcal R_{\mathrm{side}}|+|\mathcal R_{\mathrm{top}}|
 \ \lesssim\ \Big(\iint_{Q(\alpha'I)} |\nabla \widetilde U|^2\,\sigma\Big)^{1/2}
               \cdot \Big(\iint_{Q(\alpha'I)} \big(|\nabla\chi|^2\,|V_{\psi,L,t_0}|^2+|\nabla V_{\psi,L,t_0}|^2\big)\,\sigma\Big)^{1/2}.
\]
\end{lemma}
\begin{proof}
Apply Green's identity on $Q(\alpha'I)$ to $\widetilde U$ and $\chi_{L,t_0}V_{\psi,L,t_0}$:
\[
  \iint_{Q(\alpha'I)} \nabla \widetilde U\cdot \nabla(\chi V)\,dt\,d\sigma
  \ =\ \int_{\partial Q(\alpha'I)} \chi V\,\partial_n \widetilde U\,ds.
\]
Since $\chi$ is supported in $Q(\alpha'I)$ and equals $1$ on $Q(\alpha I)$, the boundary integral splits into the bottom edge (where $\chi V=\psi_{L,t_0}$) plus side/top edges and cutoff-transition edges; these latter contributions are grouped into $\mathcal R_{\mathrm{side}}$ and $\mathcal R_{\mathrm{top}}$.
On the bottom edge, Cauchy–Riemann for $\log J=\widetilde U+i\widetilde W$ gives $\partial_n \widetilde U=-\partial_\sigma \widetilde U=\partial_t \widetilde W$, so
\[
  -\int_{\partial Q\cap\{\sigma=0\}} \chi V\,\partial_n \widetilde U\,dt
  \ =\ -\int_{\R}\psi_{L,t_0}(t)\,\partial_t \widetilde W(t)\,dt
  \ =\ \int_{\R} u(t)\,\psi_{L,t_0}(t)\,dt,
\]
where $u(t)$ denotes the boundary trace paired against $\psi_{L,t_0}$ (the phase distribution after neutralization).
Finally, the remainder bound follows by Cauchy–Schwarz, using $\|\nabla\chi\|_\infty\lesssim L^{-1}$ and the displayed test-energy factor.
\end{proof}
% Archived duplicate block (removed in submission branch)
% archived block removed
\begin{lemma}[CR–Green pairing for boundary phase]\label{lem:CR-green-phase}
Let $J$ be analytic on $\Omega$ with a.e. boundary modulus $|J(\tfrac12+it)|=1$, and write $\log J=U+iW$ on $\Omega$, so $U$ is harmonic with $U(\tfrac12+it)=0$ a.e. Fix a Whitney interval $I=[t_0-L,t_0+L]$ and let $V_{\psi,L,t_0}$ be the Poisson extension of $\psi_{L,t_0}$. Then, with a cutoff $\chi_{L,t_0}$ as in Lemma~\ref{lem:cutoff-pairing},
\[
  \int_{\R} \psi_{L,t_0}(t)\,\big(-W'(t)\big)\,dt\ =\ \iint_{Q(\alpha'I)} \nabla U\cdot \nabla\big(\chi_{L,t_0}\,V_{\psi,L,t_0}\big)\,dt\,d\sigma\ +\ \mathcal R_{\mathrm{side}}\ +\ \mathcal R_{\mathrm{top}},
\]
and the remainders satisfy
\[
  |\mathcal R_{\mathrm{side}}|+|\mathcal R_{\mathrm{top}}|\ \lesssim\ \Big(\iint_{Q(\alpha'I)} |\nabla U|^2\,\sigma\Big)^{1/2}\ \cdot\ \Big(\iint_{Q(\alpha'I)} (|\nabla\chi|^2\,|V|^2+|\nabla V|^2)\,\sigma\Big)^{1/2}.
\]
In particular, by Cauchy–Schwarz and the scale–invariant Dirichlet bound for $V_{\psi,L,t_0}$, there is a constant $C(\psi)$ such that
\[
  \int_{\R} \psi_{L,t_0}(t)\,\big(-w'(t)\big)\,dt\ \le\ C(\psi)\,\Big(\iint_{Q(\alpha'I)} |\nabla U|^2\,\sigma\Big)^{1/2}.
\]
Moreover, replacing $U$ by $U-\Re\log\mathcal O$ for any outer $\mathcal O$ with boundary modulus $e^{u}$ leaves the left-hand side unchanged and affects only the right-hand side through $\nabla\Re\log\mathcal O$ (Lemma~\ref{lem:outer-cancel}).
\end{lemma}
\begin{proof}[Boundary identity justification]
On the bottom edge $\{\sigma=0\}$ the outward normal is $\partial_n=-\partial_\sigma$. By Cauchy–Riemann for $\log J=U+iW$ on the boundary line $\{\Re s=\tfrac12\}$ one has $\partial_n U=-\partial_\sigma U=\partial_t W$. Hence
\[
-\int_{\partial Q\cap\{\sigma=0\}} \chi\,V\,\partial_n U\,dt\ =\ -\int_{\R} \psi_{L,t_0}(t)\,\partial_t W(t)\,dt\ =\ \int_{\R} \psi_{L,t_0}(t)\,\big(-w'(t)\big)\,dt,
\]
which yields the displayed identity after including the interior term and remainders.
\end{proof}
\begin{lemma}[Outer cancellation in the CR--Green pairing]\label{lem:outer-cancel}
With the notation of Lemma~\ref{lem:CR-green-phase}, replace $U$ by $U-\Re\log\mathcal O$, where $\mathcal O$ is any outer on $\Omega$ with a.e.\ boundary modulus $e^{u}$ and boundary argument derivative $\frac{d}{dt}\Arg\mathcal O=\Hilb[u']$ (Lemma~\ref{lem:outer-phase-HT}). Then the left-hand side of the identity in Lemma~\ref{lem:CR-green-phase} is unchanged, and the right-hand side depends only on $\nabla\!\big(U-\Re\log\mathcal O\big)$.
\end{lemma}
\begin{proof}
On the bottom edge, replacing $U$ by $U-\Re\log\mathcal O$ changes the boundary term by $\int_{\R}\psi_{L,t_0}(t)\,\mathsf H[u'](t)\,dt$ (Lemma~\ref{lem:outer-phase-HT}), which cancels against the outer contribution in $-w'$. In the interior, the change is linear in $\nabla\Re\log\mathcal O$ and is absorbed by the same energy estimate.
\end{proof}
\begin{corollary}[Explicit remainder control]
With notation as in Lemma~\ref{lem:CR-green-phase}, there exists $C_{\mathrm{rem}}=C_{\mathrm{rem}}(\alpha,\psi)$ such that
\[
  |\mathcal R_{\mathrm{side}}|+|\mathcal R_{\mathrm{top}}|
 \lesssim\ C_{\mathrm{rem}}\,\Big(\iint_{Q(\alpha'I)} |\nabla U|^2\,\sigma\Big)^{1/2}.
\]
In particular, one may take $C_{\mathrm{rem}}\asymp_\alpha \mathcal A(\psi)$, where $\mathcal A(\psi)$ is the fixed Poisson energy of the window (cf. Corollary~\ref{cor:CH-Mpsi-final}).
\end{corollary}
% end archived block removal
\begin{proof}
From Lemma~\ref{lem:CR-green-phase},
\[
  |\mathcal R_{\mathrm{side}}|+|\mathcal R_{\mathrm{top}}| \lesssim\ \Big(\iint_{Q(\alpha'I)} |\nabla U|^2\,\sigma\Big)^{1/2}\,\cdot\,\Big(\iint_{Q(\alpha'I)} (|\nabla\chi|^2\,|V|^2+|\nabla V|^2)\,\sigma\Big)^{1/2}.
\]
The cutoff satisfies $\|\nabla\chi\|_\infty\lesssim L^{-1}$ and is supported in a fixed dilate $Q(\alpha' I)$ with bounded overlap, while $V$ is the Poisson extension of the fixed window $\psi$; hence the second factor is $\asymp_\alpha \mathcal A(\psi)$, independent of $(T,L)$. Absorbing constants depending only on $(\alpha,\psi)$ yields the claim.
\end{proof}

% --- Outer cancellation and which energy is bounded ---
\begin{lemma}[Outer cancellation and energy bookkeeping on boxes]\label{lem:outer-energy-bookkeeping}
Let
\[
u_0(t):=\log\Big|\det\nolimits_2\!\big(I-A(\tfrac12+it)\big)\Big|,\qquad
u_\xi(t):=\log\big|\xi(\tfrac12+it)\big|,
\]
and let $O$ be the outer on $\Omega$ with boundary modulus
\(
|O(\tfrac12+it)|=\exp\!\big(u_0(t)-u_\xi(t)\big).
\)
Set
\[
J(s):=\frac{\det\nolimits_2(I-A(s))}{O(s)\,\xi(s)},\qquad
\log J=U+iW,\qquad U_0:=\Re\log\det\nolimits_2(I-A),\quad U_\xi:=\Re\log\xi.
\]
Then for every Whitney interval $I=[t_0-L,t_0+L]$ and the standard test field $V_{\psi,L,t_0}$,
\begin{equation}\label{eq:CRG-outer-cancel}
\int_{\R}\psi_{L,t_0}(t)\,(-W'(t))\,dt
=\iint_{Q(\alpha' I)} \nabla\!\big(U_0-U_\xi-\Re\log O\big)\cdot\nabla\!\big(\chi_{L,t_0}V_{\psi,L,t_0}\big)\,dt\,d\sigma
+\mathcal R_{\mathrm{side}}+\mathcal R_{\mathrm{top}}
\end{equation}
and hence, by Cauchy--Schwarz and the scale‑invariant Dirichlet bound for $V_{\psi,L,t_0}$,
\begin{equation}\label{eq:energy-U-used}
\int_{\R}\psi_{L,t_0}\,(-W')\ \le\ C(\psi)\,\Big(C_{\rm box}\big(U_0-U_\xi-\Re\log O\big)\,|I|\Big)^{1/2}
\end{equation}
Moreover $\Re\log O$ is the Poisson extension of the boundary function $u:=u_0-u_\xi$, so
\begin{equation}\label{eq:Poisson-splitting}
U_0-U_\xi-\Re\log O
:=\underbrace{(U_0-\Poisson[u_0])}_{\equiv 0}\ -\ \big(U_\xi-\Poisson[u_\xi]\big)
\end{equation}
and consequently the Carleson box energy that actually enters \eqref{eq:energy-U-used} satisfies
\begin{equation}\label{eq:sharp-Kxi}
C_{\rm box}\big(U_0-U_\xi-\Re\log O\big)\ \le\ K_\xi
\end{equation}
In particular, the coarse bound
\begin{equation}\label{eq:coarse-K0Kxi}
C_{\rm box}\big(U_0-U_\xi-\Re\log O\big)\ \le\ K_0+K_\xi\ =\ \CboxZeta
\end{equation}
also holds, by the triangle inequality for $C_{\rm box}$ and linearity of the Poisson extension.
\end{lemma}

\begin{proof}
The identity \eqref{eq:CRG-outer-cancel} is Lemma~\ref{lem:CR-green-phase} with $U$ replaced by $U-\Re\log O$, together with the outer cancellation Lemma~\ref{lem:outer-cancel}; subtracting $\Re\log O$ leaves the left side (phase) unchanged. The estimate \eqref{eq:energy-U-used} follows as in Lemma~\ref{lem:CR-green-phase} from Cauchy--Schwarz and the scale‑invariant Dirichlet bound, with $C(\psi)=C_{\mathrm{rem}}(\alpha,\psi)\,\mathcal A(\psi)$ independent of $L,t_0$.

By Lemma~\ref{lem:outer-phase-HT}, $\Re\log O=\Poisson[u]$ with $u=u_0-u_\xi$, and since $U_0$ is harmonic with boundary trace $u_0$ we have $U_0=\Poisson[u_0]$, giving \eqref{eq:Poisson-splitting}. The remainder $U_\xi-\Poisson[u_\xi]$ is the (neutralized) Green potential of zeros; its Whitney–box energy is bounded by $K_\xi$ (see Lemma~\ref{lem:carleson-xi} and the annular $L^2$ aggregation), which yields \eqref{eq:sharp-Kxi}. Finally, \eqref{eq:coarse-K0Kxi} follows from the subadditivity
\(
\sqrt{C_{\rm box}(U_1+U_2)}\le \sqrt{C_{\rm box}(U_1)}+\sqrt{C_{\rm box}(U_2)}
\)
(Lemma~\ref{lem:carleson-sum}) together with $C_{\rm box}(U_0)\le K_0$ and $C_{\rm box}(U_\xi)\le K_\xi$.
\end{proof}

\noindent\emph{Consequences.}
In the CR–Green certificate the field you pair is exactly
\(
U_0-U_\xi-\Re\log O,
\)
and its box energy is controlled by $K_\xi$ (sharp) and certainly by $K_0+K_\xi=\CboxZeta$ (coarse).
The aperture dependence is confined to $C(\psi)$, not to the box constant.
% --- end snippet ---

% (Removed global reduction: certificate constants are taken as Whitney-only suprema.)
% --- Atom-safe admissible test class and uniform CR–Green estimate ---

\begin{definition}[Admissible, atom-safe test class]\label{def:admissible-class}
Fix a Whitney interval \(I=[t_0-L,t_0+L]\) (with the standing aperture schedule)
and a smooth cutoff \(\chi_{L,t_0}\) supported in \(Q(\alpha'I)\), equal to \(1\) on \(Q(\alpha I)\), with
\(\|\nabla\chi_{L,t_0}\|_\infty\lesssim L^{-1}\), \(\|\nabla^2\chi_{L,t_0}\|_\infty\lesssim L^{-2}\).
Let \(V_\varphi:=P_\sigma*\varphi\) denote the Poisson extension of \(\varphi\).


We say that a collection \(\mathcal A=\mathcal A(I)\subset C_c^\infty(I)\) is \emph{admissible}
if each \(\varphi\in\mathcal A\) is nonnegative, \(\int_{\R}\varphi=1\), and there is a constant \(A_\ast<\infty\),
independent of \(L,t_0\) and of \(\varphi\in\mathcal A\), such that the (scale-invariant) Poisson test energy obeys
\begin{equation}\label{eq:Poisson-energy-bound}
  \iint_{Q(\alpha'I)} \Big(|\nabla V_\varphi|^2 + |\nabla\chi_{L,t_0}|^2\,|V_\varphi|^2\Big)\,\sigma\,dt\,d\sigma
  \ \le\ A_\ast
\end{equation}
We call \(\mathcal A\) \emph{atom-safe} on \(I\) if, whenever \(I\) contains critical-line atoms \(\{\gamma_j\}\) for \(-w'\),
there exists \(\varphi\in\mathcal A\) with \(\varphi(\gamma_j)=0\) for all such \(\gamma_j\).
\end{definition}


\begin{lemma}[Uniform CR--Green bound for the class \(\mathcal A\)]\label{lem:uniform-CRG-A}
Let \(J\) be analytic on \(\Omega\) with a.e.\ boundary modulus \(|J(\tfrac12+it)|=1\) and write \(\log J=U+iW\) with boundary phase \(w=W|_{\sigma=0}\).
Assume the Carleson box-energy bound for \(U\) on Whitney boxes:
\[
  \iint_{Q(\alpha I)} |\nabla U|^2\,\sigma\,dt\,d\sigma \ \le\ C_{\rm box}^{(\zeta)}\,|I|\ =\ 2L\,C_{\rm box}^{(\zeta)}.
\]
If \(\mathcal A=\mathcal A(I)\) is admissible in the sense of \eqref{eq:Poisson-energy-bound},
then there exists a constant \(C_{\rm rem}=C_{\rm rem}(\alpha)\) such that, uniformly in \(I\),
\begin{equation}\label{eq:supA-bound}
  \sup_{\varphi\in\mathcal A}\ \int_{\R} \varphi(t)\,(-w'(t))\,dt
  \ \le\ C_{\rm rem}\,\sqrt{A_\ast}\,\big(C_{\rm box}^{(\zeta)}\big)^{1/2}\,L^{1/2}
  \ \ :=:\ C_{\mathcal A}\,C_{\rm box}^{(\zeta)}{}^{1/2}\,L^{1/2}
\end{equation}
\end{lemma}


\begin{proof}
For each \(\varphi\in\mathcal A\), apply the CR--Green pairing on \(Q(\alpha'I)\) to \(U\) and \(\chi_{L,t_0}V_\varphi\):
\[
  \int_{\R}\varphi(t)\,(-w'(t))\,dt
  \ =\ \iint_{Q(\alpha'I)} \nabla U\cdot\nabla(\chi_{L,t_0}V_\varphi)\,dt\,d\sigma\ +\ \mathcal R_{\mathrm{side}}+\mathcal R_{\mathrm{top}},
\]
with remainders bounded by \(C_{\rm rem}(\alpha)\) times the product of the Dirichlet norms
(of \(\nabla U\) on \(Q(\alpha'I)\) and of the test field, cf.\ \eqref{eq:Poisson-energy-bound}).
By Cauchy--Schwarz and the Carleson bound for \(U\),
\[
  \int_{\R}\varphi(-w') \ \le\ C_{\rm rem}(\alpha)\,
  \Big(\iint_{Q(\alpha'I)} |\nabla U|^2\,\sigma\Big)^{\!1/2}
  \Big(\iint_{Q(\alpha'I)} (|\nabla V_\varphi|^2+|\nabla\chi|^2|V_\varphi|^2)\,\sigma\Big)^{\!1/2}.
\]
Insert the hypotheses to obtain
\(
\int \varphi(-w') \le C_{\rm rem}(\alpha)\,\sqrt{2L\,C_{\rm box}^{(\zeta)}}\ \sqrt{A_\ast},
\)
which is \eqref{eq:supA-bound} upon setting \(C_{\mathcal A}:=C_{\rm rem}(\alpha)\sqrt{2A_\ast}\) (and absorbing absolute factors).
\end{proof}


\begin{corollary}[Atom neutralization and clean Whitney scaling]\label{cor:atom-safe}
With the notation above, the phase--velocity identity yields, for every \(\varphi\in C_c^\infty(I)\),
\[
  \int_{\R}\varphi(t)\,(-w'(t))\,dt
  \ =\ \pi\!\int_{\R}\varphi\,d\mu\ +\ \pi\sum_{\gamma\in I} m_\gamma\,\varphi(\gamma),
\]
where \(\mu\) is the Poisson balayage measure (absolutely continuous) and the sum ranges over critical-line atoms.
If \(I\) contains atoms, pick \(\varphi\in\mathcal A(I)\) with \(\varphi(\gamma)=0\) at each such atom; then the atomic term vanishes and
\[
  \int_{\R}\varphi\,(-w')\ =\ \pi\!\int \varphi\,d\mu\ \le\ C_{\mathcal A}\,C_{\rm box}^{(\zeta)}{}^{1/2}\,L^{1/2}.
\]
Thus the \(\,L^{-1}\) plateau blow-up from atoms is removed, and the Whitney\-uniform \(L^{1/2}\) bound \eqref{eq:supA-bound}
holds verbatim in the atomic case as well.
\end{corollary}
\begin{proof}
This is immediate from the phase–velocity identity (Theorem~\ref{thm:phase-velocity-quant}) and the definition of an atom-safe admissible class: choosing \(\varphi\) to vanish at each critical-line atom kills the discrete sum. The remaining absolutely continuous term equals \(\pi\int \varphi\,d\mu\) and is controlled by the uniform CR--Green estimate \eqref{eq:supA-bound}.
\end{proof}


\begin{remark}[Local-to-global wedge]\label{rem:wedge-application}
The certificate produces a \emph{Whitney-local} phase-drop control of the form
\(\int_I(-w')\le \pi\,\Upsilon\) with \(\Upsilon<\tfrac12\) on every Whitney interval \(I\)
(Lemma~\ref{lem:whitney-uniform-wedge}), and more generally an admissible-class bound
\(\sup_{\varphi\in\mathcal A(I)}\int \varphi(-w')\lesssim L^{1/2}\) (Lemma~\ref{lem:uniform-CRG-A}).

\medskip
\noindent\textbf{Referee note (what is missing).}
As stated, the manuscript still needs an explicit, referee-checkable implication of the form
\[
  \Big(\forall\ \text{Whitney }I,\ \int_I(-w')\le \pi\,\Upsilon<\tfrac{\pi}{2}\Big)
  \quad\Longrightarrow\quad
  \exists\,m\in\R/2\pi\mathbb Z\ \text{s.t.}\ |\Arg \mathcal J(\tfrac12+it)-m|\le \tfrac{\pi}{2}\ \text{a.e.},
\]
i.e. a global a.e. boundary wedge \textup{(P+)} after a \emph{single} unimodular rotation.
This does \emph{not} follow from Whitney-local control alone without an additional hypothesis preventing
global phase drift (e.g. an “exponential inner factor at infinity”).

\medskip
\noindent\textbf{Counterexample (shows Whitney-local bounds alone do not force a global wedge).}
Let \(J(s):=\exp\!\big(-a(s-\tfrac12)\big)\) on \(\Omega\). Then \(|J(\tfrac12+it)|=1\) a.e., the boundary phase may be taken as
\(w(t)=-at\) so that \(-w'=a\,dt\) is a positive Radon measure, and for every Whitney interval \(I\) of length \(|I|\le 2L_\star\) one has
\(\int_I(-w')=a|I|\le 2aL_\star\).
Choosing \(a\le (\pi\Upsilon)/(2L_\star)\) forces \(\int_I(-w')\le \pi\Upsilon\) on \emph{every} Whitney interval with any fixed \(\Upsilon<\tfrac12\),
yet \(\Re(2J(\tfrac12+it))=2\cos(at)\) changes sign on sets of positive measure for every rotation, so \textup{(P+)} fails.
\end{remark}
\begin{corollary}[Unconditional local window constants]\label{cor:CH-Mpsi-final}
Define, for $I=[t_0-L,t_0+L]$ and $u$ the boundary trace of $U$, the mean-oscillation constant
\[
  M_\psi\ :=\ \sup_{L>0,\ t_0\in\R}\ \frac{1}{L}\,\Big|\int_{\R} (u(t)-u_I)\,\psi_{L,t_0}(t)\,dt\Big|,\qquad u_I:=\frac{1}{|I|}\int_I u,\quad \psi_{L,t_0}(t):=\psi\big((t-t_0)/L\big),
\]
and the Hilbert constant
\[
  C_H(\psi)\ :=\ \sup_{L>0,\ t_0\in\R}\ \frac{1}{L}\,\Big|\int_{\R} \mathcal H[u'](t)\,\psi_{L,t_0}(t)\,dt\Big|.
\]
Then there are constants $C_1(\psi),C_2(\psi)<\infty$ depending only on $\psi$ and the dilation parameter $\alpha$ such that
\[
  M_\psi\ \le\ C_1(\psi)\,\sqrt{C_{\rm box}^{(\mathrm{Whitney})}}\,\mathcal A(\psi),\qquad
  C_H(\psi)\ \le\ C_2(\psi)\,\sqrt{C_{\rm box}^{(\mathrm{Whitney})}}\,\mathcal A(\psi),
\]
where the fixed Poisson energy of the window is
\[
  \mathcal A(\psi)^2\ :=\ \iint_{\R^2_+}|\nabla(P_\sigma*\psi)|^2\,\sigma\,dt\,d\sigma\ <\ \infty.
\]
In particular, both constants are finite and determined by local box energies.
\end{corollary}
\begin{proof}
This is a bookkeeping corollary collecting the already-proved window bounds: the $H^1$--BMO/Carleson estimate for $M_\psi$ is Lemma~\ref{lem:Mpsi-correct}, and the uniform Hilbert pairing bound is Lemma~\ref{lem:hilbert-H1BMO}. The constants $C_1(\psi),C_2(\psi)$ absorb the fixed geometric Carleson embedding factor (Appendix~\ref{app:CE-constant}) and the fixed Poisson energy $\mathcal A(\psi)$.
\end{proof}
\begin{lemma}[Poisson–BMO bound at fixed height]\label{lem:poisson-bmo-strip}
Let $u\in \mathrm{BMO}(\mathbb R)$ and $U(\sigma,t):=(P_\sigma*u)(t)$ be its Poisson extension on $\Omega$. Then for every fixed $\sigma_0>0$,
\[
\sup_{t\in\mathbb R}|U(\sigma,t)|\ \le\ C_{\mathrm{BMO}}\,\|u\|_{\mathrm{BMO}}\qquad(\sigma\ge \sigma_0),
\]
with a finite constant $C_{\mathrm{BMO}}$ depending only on $\sigma_0$ and the fixed cone/box geometry. Consequently, if $\mathcal O$ is the outer with boundary modulus $e^u$, then for $\sigma\ge \sigma_0$ one has $e^{-C_{\mathrm{BMO}}\|u\|_{\mathrm{BMO}}}\le |\mathcal O(\sigma+it)|\le e^{C_{\mathrm{BMO}}\|u\|_{\mathrm{BMO}}}$.
\end{lemma}
\begin{proof}
Fix $\sigma\ge\sigma_0$. Write $U(\sigma,t)=\int_\R u(t-s)\,P_\sigma(s)\,ds$. Since $\int P_\sigma=1$ and $\int s\,P_\sigma(s)\,ds=0$, we may subtract the mean of $u$ on $I=[t-\sigma,t+\sigma]$ to get
\[
  U(\sigma,t)=u_I+\int_\R (u(t-s)-u_I)\,P_\sigma(s)\,ds.
\]
The second term is controlled by the BMO seminorm via the standard estimate (see, e.g., \cite[Ch.~IV]{SteinSingInt} or \cite[Ch.~IV]{Garnett})
\(\int |u(t-s)-u_I|\,P_\sigma(s)\,ds\lesssim \|u\|_{\mathrm{BMO}}\)
uniformly in $t$ for $\sigma\ge\sigma_0$ (use the dyadic annuli decomposition of $\R$ relative to $I$ and the doubling property of BMO averages). Absorbing constants depending only on $\sigma_0$ into $C_{\mathrm{BMO}}$ gives the stated bound. The outer modulus bounds follow by exponentiating $|U|\le C_{\mathrm{BMO}}\|u\|_{\mathrm{BMO}}$.
\end{proof}
\subsection*{Hilbert pairing via affine subtraction (uniform in $T,L$)}
% Archived duplicate block (not load-bearing in the active route)
% archived block removed
\begin{lemma}[Uniform Hilbert pairing bound (local box pairing)]\label{lem:hilbert-H1BMO}
Let $\psi\in C_c^\infty([-1,1])$ be even with $\int_\R\psi=1$ and define the mass--1 windows $\varphi_I(t)=L^{-1}\psi\big((t-T)/L\big)$. Then there exists $C_H(\psi)<\infty$ (independent of $T,L$) such that for $u$ from the smoothed Cauchy theorem,
\[
  \Big|\int_\R \mathcal H[u'](t)\,\varphi_I(t)\,dt\Big|\ \le\ C_H(\psi)\quad\text{for all intervals }I.
\]
\end{lemma}
% \fi
\begin{proof}
In distributions, $\langle \mathcal H[u'],\varphi_I\rangle=\langle u,(\mathcal H[\varphi_I])'\rangle$. Since $\psi$ is even, $(\mathcal H[\varphi_I])'$ annihilates affine functions; subtract the calibrant $\ell_I$ and write $v:=u-\ell_I$. Let $V$ be the Dirichlet test field for $(\mathcal H[\varphi_I])'$ supported in $Q(\alpha'I)$ with $\|\nabla V\|_{L^2(\sigma)}\asymp L^{-1/2}\,\mathcal A(\psi)$ (scale invariance for mass--1 windows). The local box pairing (Lemma~\ref{lem:cutoff-pairing}) gives
\[
  |\langle v,(\mathcal H[\varphi_I])'\rangle|\ \le\ \Big(\iint_{Q(\alpha'I)} |\nabla \widetilde U|^2\,\sigma\Big)^{1/2}\,\cdot\,\Big(\iint_{Q(\alpha'I)} |\nabla V|^2\,\sigma\Big)^{1/2}.
\]
Using the neutralized area bound $\iint_{Q(\alpha'I)} |\nabla \widetilde U|^2\,\sigma\lesssim |I|\asymp L$ (Lemma~\ref{lem:carleson-xi}) and the fixed test energy for $V$, we obtain
\[
  |\langle v,(\mathcal H[\varphi_I])'\rangle|\ \lesssim\ (L)^{1/2}\,(L^{-1/2}\,\mathcal A(\psi))\ =\ C(\psi)\,\mathcal A(\psi),
\]
uniformly in $(T,L)$. This proves the uniform bound with $C_H(\psi)\asymp \mathcal A(\psi)$.
\end{proof}
\begin{lemma}[Hilbert-transform pairing]\label{lem:hilbert}
There exists a window–dependent constant \(C_H(\psi)>0\) such that for every interval \(I\),
\[ \Big|\int_{\R} \mathcal H[u'](t)\,\varphi_I(t)\,dt\Big|\ \le\ C_H(\psi).\]
\end{lemma}
\begin{proof}
By Lemma~\ref{lem:hilbert-H1BMO}, for mass–1 windows and even \(\psi\), the pairing \(\langle \mathcal H[u'],\varphi_I\rangle\) is uniformly bounded in \((T,L)\). In distributions, \(\langle \mathcal H[u'],\varphi_I\rangle=\langle u,(\mathcal H[\varphi_I])'\rangle\); evenness implies \((\mathcal H[\varphi_I])'\) annihilates affine functions. Subtract the affine calibrant on \(I\) and write \(v=u-\ell_I\). The bound follows from the local box pairing in the Carleson energy lemma (Lemma~\ref{lem:carleson-xi}) applied to the test field associated with \((\mathcal H[\varphi_I])'\).
\end{proof}
% --- PSC route moved to archived appendix; placeholder removed from main chain ---
We adopt the \(\zeta\)-normalized boundary route with the half-plane compensator \(B(s)=s/(s-1)\), so that
\(
F(s)=\dettwo(I-A(s))/\zeta(s)\cdot B(s)=\dettwo(I-A(s))\,s/((s-1)\zeta(s))
\)
is regular and typically nonzero at \(s=1\).
On \(\Re s=\tfrac12\), \(|B|=1\), so the compensator does not affect boundary \emph{modulus}; its boundary phase is an explicit rational term and can be absorbed into the fixed Archimedean bookkeeping.
We print a concrete even $C^\infty$ flat--top window \(\psi\) below. For the finite-block certificate matrix we will use the scaled window
\[
  \psi_{\mathrm{cert}}(t)\ :=\ \tfrac1{12}\,\psi(t),
\]
so that the Fourier sup constant satisfies \(C_{\mathrm{win}}=\sup_{\xi}|\widehat{\psi_{\mathrm{cert}}}(\xi)|=\tfrac14\) (Lemma~\ref{lem:psi-cert-Cwin}). We also record the (optional) product certificate
\[
  \frac{(2/\pi)\,M_\psi}{c_0(\psi)}\ <\ \frac{\pi}{2}.
\]
\paragraph{Printed window.}
Let \(\beta(x):=\exp\!\big(-1/(x(1-x))\big)\) for \(x\in(0,1)\) and \(\beta=0\) otherwise. Define the smooth step
\[
  S(x):=\frac{\int_0^{\min\{\max\{x,0\},1\}} \beta(u)\,du}{\int_0^{1} \beta(u)\,du}\qquad (x\in\R),
\]
so that \(S\in C^\infty(\R)\), \(S\equiv 0\) on \(({-}\infty,0]\), \(S\equiv1\) on \([1,\infty)\), and \(S'\ge 0\) supported on \((0,1)\). Set the even flat-top window \(\psi:\R\to[0,1]\) by
\[
  \psi(t):=\begin{cases}
    0,& |t|\ge 2,\\
    S(t+2),& -2<t<-1,\\
    1,& |t|\le 1,\\
    S(2-t),& 1<t<2.
  \end{cases}
\]
Then \(\psi\in C_c^\infty(\R)\), \(\psi\equiv 1\) on \([-1,1]\), and \(\operatorname{supp}\psi\subset[-2,2]\). For windows we take \(\varphi_L(t):=L^{-1}\psi(t/L)\).

\begin{lemma}[Flat-top window: mass and Fourier sup bound for the scaled certificate window]\label{lem:psi-cert-Cwin}
Let $\psi$ be the printed flat--top window above and define $\psi_{\mathrm{cert}}:=\tfrac1{12}\psi$.
Define
\[
  \widehat{\psi_{\mathrm{cert}}}(\xi)\ :=\ \int_{\R}\psi_{\mathrm{cert}}(t)\,e^{-it\xi}\,dt,
  \qquad
  C_{\mathrm{win}}\ :=\ \sup_{\xi\in\R}\big|\widehat{\psi_{\mathrm{cert}}}(\xi)\big|.
\]
Then $\int_{\R}\psi(t)\,dt=3$, $\int_{\R}\psi_{\mathrm{cert}}(t)\,dt=\tfrac14$, and
\[
  C_{\mathrm{win}}\ =\ \int_{\R}\psi_{\mathrm{cert}}(t)\,dt\ =\ \frac14.
\]
\end{lemma}

\begin{proof}
Since $\beta(x)=\beta(1-x)$ on $(0,1)$, for $x\in[0,1]$ we have
\[
  \int_0^{1-x}\beta(u)\,du=\int_x^1 \beta(v)\,dv
\]
by the change of variables $v=1-u$. Dividing by $\int_0^1\beta$ gives $S(1-x)=1-S(x)$ on $[0,1]$, hence
\[
  \int_0^1 S(x)\,dx=\frac12\int_0^1\bigl(S(x)+S(1-x)\bigr)\,dx=\frac12.
\]
Therefore the two ramps of $\psi$ each have area $1/2$, so
\[
  \int_{\R}\psi(t)\,dt
  =2+\;2\!\int_1^2 S(2-t)\,dt
  =2+\;2\!\int_0^1 S(u)\,du
  =2+1=3.
\]
Scaling gives $\int\psi_{\mathrm{cert}}=\tfrac1{12}\int\psi=\tfrac14$.
For the Fourier bound, $\psi_{\mathrm{cert}}\ge 0$ implies for all $\xi$,
\[
  \big|\widehat{\psi_{\mathrm{cert}}}(\xi)\big|
  \le \int_{\R}\psi_{\mathrm{cert}}(t)\,|e^{-it\xi}|\,dt
  =\int_{\R}\psi_{\mathrm{cert}}(t)\,dt.
\]
At $\xi=0$ we have $\widehat{\psi_{\mathrm{cert}}}(0)=\int\psi_{\mathrm{cert}}$, hence $\sup_\xi|\widehat{\psi_{\mathrm{cert}}}(\xi)|=\int\psi_{\mathrm{cert}}=\tfrac14$.
\end{proof}

\paragraph{Poisson lower bound.}
\begin{lemma}[Poisson plateau lower bound]\label{lem:poisson-plateau}
For the printed even window \(\psi\) with \(\psi\equiv 1\) on \([-1,1]\),
\[ c_0(\psi)\ :=\ \inf_{0<b\le 1,\ |x|\le 1} (\Poisson_b*\psi)(x)\ \ge\ \frac{1}{2\pi}\,\arctan 2. \]
\end{lemma}
\begin{proof}
As in the plateau computation already recorded, for \(0<b\le 1\) and \(|x|\le 1\) one has
\[
 (\Poisson_b*\psi)(x)\ \ge\ (\Poisson_b*\mathbf 1_{[-1,1]})(x)
  = \frac{1}{2\pi}\Big(\arctan\tfrac{1-x}{b}+\arctan\tfrac{1+x}{b}\Big),
\]
whence
\[
 c_0(\psi)\ :=\ \inf_{0<b\le 1,\ |x|\le 1} (\Poisson_b*\psi)(x)\ \ge\ 0.1762081912\,.
\]
For the normalized Poisson kernel \(P_b(y)=\dfrac{1}{\pi}\dfrac{b}{b^2+y^2}\), for \(|x|\le 1\)
\[
 (P_b*\mathbf 1_{[-1,1]})(x)=\frac{1}{\pi}\int_{-1}^{1}\frac{b}{b^2+(x-y)^2}\,dy=\frac{1}{2\pi}\Big(\arctan\frac{1-x}{b}+\arctan\frac{1+x}{b}\Big).
\]
Set \(S(x,b):=\arctan\big((1-x)/b\big)+\arctan\big((1+x)/b\big)\). Symmetry gives \(S(-x,b)=S(x,b)\). For \(x\in[0,1]\),
\[
 \partial_x S(x,b)=\frac{1}{b}\Big(\frac{1}{1+\big(\tfrac{1+x}{b}\big)^2}-\frac{1}{1+\big(\tfrac{1-x}{b}\big)^2}\Big)\le 0,
\]
so \(S\) decreases in \(x\) and is minimized at \(x=1\). Also \(\partial_b S(x,b)\le 0\) for \(b>0\), so the minimum in \(b\in(0,1]\) is at \(b=1\). Thus the infimum occurs at \((x,b)=(1,1)\) giving \(\frac{1}{2\pi}\arctan 2=0.1762081912\ldots\). Since \(\psi\ge \mathbf 1_{[-1,1]}\), this yields the bound for \(\psi\).
\end{proof}
\paragraph{No Archimedean term in the \(\zeta\)-normalized route.}
Writing \(J_\zeta:=\dettwo(I-A)/\zeta\) and \(J_{\mathrm{comp}}:=J_\zeta\,B\), one has \(|B|=1\) on the boundary and no Gamma factor in \(J_\zeta\). Hence the boundary phase contribution from Archimedean factors is identically zero in the phase–velocity identity, i.e. \(C_\Gamma\equiv 0\) for this normalization.

% (bridge AAB archived)
We carry out the boundary phase test in the $\zeta$–normalized gauge with the Blaschke compensator at $s=1$; on $\Re s=\tfrac12$ one has $|B|=1$, so the Archimedean boundary contribution vanishes. Any residual interior effect is absorbed into the $\zeta$–side box constant $C_{\mathrm{box}}^{(\zeta)}$. In the a.e. wedge route no additive wedge constants are used.

\paragraph{Hilbert term (structural bound).}
For the mass--1 window and even \(\psi\), the local box pairing bound of Lemma~\ref{lem:hilbert-H1BMO} applies and is uniform in \((T,L)\). We write the certificate in terms of the abstract window-dependent constant \(C_H(\psi)\) from Lemma~\ref{lem:hilbert-H1BMO}. An explicit envelope for the printed window is recorded below, but is not required for the symbolic certificate.
\begin{lemma}[Explicit envelope for the printed window]\label{lem:CH-explicit}
For the flat-top \(\psi\) above with symmetric monotone ramps of width \(\varepsilon\in(0,1)\) on each side of \(\pm1\), one has the variation bound
\[
  \sup_{t\in\R}\,|\mathcal H[\varphi_L](t)|\ \le\ \frac{\mathrm{TV}(\psi)}{\pi}\,\log\frac{1+\varepsilon}{1-\varepsilon},\qquad \mathrm{TV}(\psi)=2.
\]
In particular, with \(\varepsilon=\tfrac15\) one obtains the certified envelope
\[
  \sup_{t\in\R}\,|\mathcal H[\varphi_L](t)|\ \le\ \frac{2}{\pi}\,\log\tfrac{3}{2}\ \approx\ 0.258\ <\ 0.26.
\]
Consequently, we may take \(C_H(\psi)\le 0.26\) for the printed window. This bound is uniform in \(L\).
\end{lemma}
\begin{proof}
Write \(\psi=\mathbf 1_{[-1,1]}+\eta\) with \(\eta\) supported on the disjoint transition layers \([1,1+\varepsilon]\) and \([-1-\varepsilon,-1]\), monotone on each layer, and total variation \(\mathrm{TV}(\psi)=2\). Using the identity
\[
\mathcal H[\psi](x)=\frac{1}{\pi}\,\mathrm{p.v.}\int \frac{\psi(y)}{x-y}\,dy=\frac{1}{\pi}\int \psi'(y)\,\log|x-y|\,dy
\]
(integration by parts; boundary cancellations by monotonicity/symmetry) and that \(\psi'\) is a finite signed measure of total variation \(\mathrm{TV}(\psi)\), one gets
\[
  |\mathcal H\psi(x)|\ \le\ \frac{\mathrm{TV}(\psi)}{\pi}\,\sup_{y\in[-1-\varepsilon,\,1+\varepsilon]}\big|\log|x-y|\big|\ -\ \inf_{y\in[-1-\varepsilon,\,1+\varepsilon]}\big|\log|x-y|\big|.
\]
The worst case is at \(x=0\), yielding \(|\mathcal H\psi(0)|\le \tfrac{\mathrm{TV}(\psi)}{\pi}\log\tfrac{1+\varepsilon}{1-\varepsilon}\). Scaling gives \(\mathcal H[\varphi_L](t)=\mathcal H\psi\big((t-T)/L\big)\), so the same bound holds uniformly in \(L\). Taking \(\varepsilon=\tfrac15\) gives the stated numeric envelope.
\end{proof}
\begin{lemma}[Derivative envelope: $C_H(\psi)\le 2/\pi$]\label{lem:CH-derivative-2pi}
For the printed flat–top window \(\psi\) (even, plateau on $[-1,1]$), with \(\varphi_L(t)=L^{-1}\psi((t-T)/L)\) one has
\[ \sup_{t\in\R}\,|\mathcal H[\varphi_L](t)|\ \le\ \frac{2}{\pi}\,\log\frac{1+\varepsilon}{1-\varepsilon}\quad\text{and}\quad \big\|\big(\mathcal H[\varphi_L]\big)'\big\|_{L^\infty(\R)}\ \le\ \frac{2}{\pi}\,\frac{1}{L}. \]
In particular, $C_H(\psi)\le 2/\pi$.
\end{lemma}
\begin{proof}
By scaling, \(\mathcal H[\varphi_L](t)=\mathcal H\psi((t-T)/L)\) and \(\big(\mathcal H[\varphi_L]\big)'(t)=\tfrac{1}{L}\,(\mathcal H\psi)'((t-T)/L)\). Since \(\psi'\equiv 0\) on \((-1,1)\) and the ramps are monotone on \([-1-\varepsilon,-1]\) and \([1,1+\varepsilon]\) with total variation \(2\), the variation/IBP argument of Lemma~\ref{lem:CH-explicit} yields the stated envelope and its derivative bound. Taking the supremum in \(t\) gives the \(2/\pi\) constant uniformly in \(L\).
\end{proof}
\paragraph{Window mean-oscillation constant \(M_\psi\): definition and bound.}
For an interval \(I=[T{-}L,T{+}L]\) and the boundary modulus \(u(t):=\log\big|\dettwo(I{-}A(\tfrac12{+}it))\big|{-}\log\big|\xi(\tfrac12{+}it)\big|\), define the mean-oscillation calibrant \(\ell_I\) as the affine function matching \(u\) at the endpoints of \(I\), and set
\[
  M_\psi\ :=\ \sup_{T\in\R,\ L>0}\ \frac{1}{|I|}\int_I \big|u(t)-\ell_I(t)\big|\,dt.
\]
By the smoothed Cauchy theorem and the local pairing in a local pairing bound, one obtains a window-dependent constant bounding the mean oscillation uniformly over $(T,L)$. For the printed flat-top window, Lemma~\ref{lem:Mpsi-correct} yields an explicit H$^1$--BMO/box-energy bound for $M_\psi$; in our calibration (see Numeric instantiation below), this gives a strict numerical bound well below the certificate threshold.
\begin{lemma}[Window mean--oscillation via H$^1$--BMO and box energy]\label{lem:Mpsi-correct}
Let $U$ be the Poisson extension of the boundary function $u$, and let $\lambda := |\nabla U|^2\,\sigma\,dt\,d\sigma$.
Fix the even $C^\infty$ window $\psi$ (support $\subset[-2,2]$, plateau on $[-1,1]$), and let $m_\psi:=\int_{\R}\psi(x)\,dx$ denote its mass. Set
\[
\phi(t):=\psi(t)-\tfrac{m_\psi}{2}\,\mathbf 1_{[-1,1]}(t),\qquad 
\phi_{L,t_0}(t):=\phi\!\Big(\frac{t-t_0}{L}\Big).
\]
Define $M_\psi:=\sup_{L>0,t_0\in\R}\frac1L\big|\int_\R u(t)\,\phi_{L,t_0}(t)\,dt\big|$ and
\[
 C_{\rm box}^{(\mathrm{Whitney})}:=\sup_{I\,:\,|I|\asymp c/\log\langle T\rangle}\frac{\lambda(Q(\alpha I))}{|I|},\qquad
C_\psi^{(H^1)}:=\frac12\int_{\R} S\phi(x)\,dx,
\]
where $S$ is the Lusin area function for the Poisson semigroup with cone aperture $\alpha$.
Then
\[
M_\psi\ \le\ \frac{4}{\pi}\,C_{\mathrm{CE}}(\alpha)\,C_\psi^{(H^1)}\,\sqrt{C_{\rm box}^{(\mathrm{Whitney})}}.
\]
\end{lemma}
\begin{proof}
By H$^1$--BMO duality, for every $I=[t_0-L,t_0+L]$,
\[ \Big|\int u\,\phi_{L,t_0}\Big|\ \le\ \|u\|_{\rm BMO}\,\|\phi_{L,t_0}\|_{H^1}. \]
Carleson embedding (aperture $\alpha$) gives
\[ \|u\|_{\rm BMO}\ \le\ \tfrac{2}{\pi}\,C_{\mathrm{CE}}(\alpha)\,\big(C_{\rm box}^{(\mathrm{Whitney})}\big)^{1/2}. \]
Since $S$ is scale-invariant in $L^1$ (up to $|I|$),
\[ \|\phi_{L,t_0}\|_{H^1}\ =\ \int S(\phi_{L,t_0})(x)\,dx\ =\ 2L\,C_\psi^{(H^1)}. \]
Divide by $L$ to conclude.
\end{proof}
\paragraph{Carleson box linkage.}
With $U=U_{\det_2}+U_{\xi}$ on the boundary in the $\zeta$–normalized route, the box constant used in the certificate is
\[
  C_{\mathrm{box}}^{(\zeta)}\ :=\ K_0\ +\ K_\xi.
\]
No separate $\Gamma$–area term enters the certificate path.

% shownumerics gated section (disabled)
\paragraph{Numeric instantiation (diagnostic; gated).}
All concrete values (audited constants for $K_0$, $K_\xi$, the $\zeta$–side box constant $C_{\mathrm{box}}^{(\zeta)}$, the evaluation of $C_\psi^{(H^1)}$, and the locked $M_\psi$) are collected for reproducibility; the proof of (P+) uses only the CR–Green right-hand side with the box constant.
\begin{itemize}
  \item \textbf{Window:} fixed $C^\infty$ even $\psi$ with $\psi\equiv 1$ on $[-1,1]$ and $\mathrm{supp}\,\psi\subseteq[-2,2]$, and $\varphi_L(t)=L^{-1}\psi(t/L)$.
  \item \textbf{Poisson lower bound.} Using the closed form for the plateau and monotonicity, $c_0(\psi)\ge 0.1762081912$.
  \item \textbf{Archimedean term.} In the $\zeta$-normalized route with the Blaschke compensator at $s=1$, $C_\Gamma=0$.
  \item \textbf{Hilbert term.} We retain $C_H(\psi)$ symbolically; an explicit envelope can be inserted.
  \item \textbf{Inequality form.} With $M_\psi= (4/\pi)\,C_\psi^{(H^1)}\,\sqrt{C_{\mathrm{box}}^{(\zeta)}}$, the display $\frac{(2/\pi)\,M_\psi}{c_0(\psi)}<\frac{\pi}{2}$ is diagnostic.
\end{itemize}
 
\subsection*{Explicit proofs and constants for key lemmas (archimedean, prime-tail, Hilbert)}
We record complete proofs with explicit constants, making finiteness and dependence on the window $\psi$ transparent.
% Duplicate prime-tail subsection removed (see earlier \S{subsec:prime-tail})
\begin{equation}\label{eq:P1}
 \sum_{p>x} p^{-\alpha}\ \le\ \frac{1.25506\,\alpha}{(\alpha-1)\,\log x}\,x^{\,1-\alpha}
\end{equation}
This follows by partial summation together with $\pi(t)\le 1.25506\,t/\log t$ for $t\ge 17$. A uniform variant over $\alpha\in[\alpha_0,2]$ (with $\alpha_0:=2\sigma_0>1$) is
\begin{equation}\label{eq:P1uniform}
 \sum_{p>x} p^{-\alpha}\ \le\ \frac{1.25506\,\alpha_0}{(\alpha_0-1)\,\log x}\,x^{\,1-\alpha_0}\qquad(x\ge 17)
\end{equation}
Two convenient alternatives:
\begin{align}
 \sum_{p>x}p^{-\alpha}&\ \le\ \frac{\alpha}{(\alpha-1)(\log x-1)}\,x^{1-\alpha}\qquad(x\ge 599)\label{eq:P1dusart}\\
 \sum_{p>x}p^{-\alpha}&\ \le\ \sum_{n>\lfloor x\rfloor}n^{-\alpha}\ \le\ \frac{x^{1-\alpha}}{\alpha-1}\qquad(x>1).\label{eq:P1triv}
\end{align}
\begin{proof}[Proof of \eqref{eq:P1}--\eqref{eq:P1triv}]
Fix $\alpha>1$ and $x\ge 17$. For $u>1$ write $f(u):=u^{-\alpha}$. By Stieltjes integration with $d\pi(u)$ and one integration by parts,
\[
\sum_{p\le y} p^{-\alpha}
=\int_{2^-}^{y} u^{-\alpha}\,d\pi(u)
= y^{-\alpha}\pi(y)+\alpha\!\int_{2}^{y} \pi(u)\,u^{-\alpha-1}\,du.
\]
Letting $y\to\infty$ and using $\alpha>1$ (so $y^{-\alpha}\pi(y)\to 0$) gives the exact tail identity
\begin{equation}\label{eq:P1-exact}
\sum_{p>x} p^{-\alpha}
=\alpha\!\int_{x}^{\infty}\!\pi(u)\,u^{-\alpha-1}\,du\;-\;x^{-\alpha}\pi(x)
\ \le\ \alpha\!\int_{x}^{\infty}\!\pi(u)\,u^{-\alpha-1}\,du
\end{equation}
For $u\ge x\ge 17$ we have the explicit bound $\pi(u)\le 1.25506\,\dfrac{u}{\log u}$. Inserting this into \eqref{eq:P1-exact} and using $1/\log u\le 1/\log x$ for $u\ge x$ yields
\[
\sum_{p>x} p^{-\alpha}
\ \le\ \frac{1.25506\,\alpha}{\log x}\!\int_{x}^{\infty}\!u^{-\alpha}\,du
\ =\ \frac{1.25506\,\alpha}{(\alpha-1)\,\log x}\,x^{\,1-\alpha},
\]
which is \eqref{eq:P1}. For the uniform version, if $\alpha\in[\alpha_0,2]$ with $\alpha_0>1$, then the map $\alpha\mapsto \alpha/(\alpha-1)$ is decreasing and $x^{1-\alpha}\le x^{1-\alpha_0}$, so \eqref{eq:P1uniform} follows immediately from \eqref{eq:P1}.

For \eqref{eq:P1dusart}, assume $x\ge 599$ and use the sharper pointwise bound $\pi(u)\le \dfrac{u}{\log u-1}$ for $u\ge x$. Then
\[
\sum_{p>x} p^{-\alpha}
\ \le\ \alpha\!\int_{x}^{\infty}\!\frac{u^{-\alpha}}{\log u-1}\,du
\ \le\ \frac{\alpha}{\log x-1}\!\int_{x}^{\infty}\!u^{-\alpha}\,du
\ =\ \frac{\alpha}{(\alpha-1)(\log x-1)}\,x^{1-\alpha}.
\]

Finally, \eqref{eq:P1triv} is the integer-majorant: $\sum_{p>x}p^{-\alpha}\le \sum_{n>\lfloor x\rfloor}n^{-\alpha}=\dfrac{x^{1-\alpha}}{\alpha-1}$ for $x>1$.
\end{proof}

\begin{lemma}[Monotonicity of the tail majorant]\label{lem:P1-monotone}
For fixed $\alpha>1$, the function $g(P):=\dfrac{P^{\,1-\alpha}}{\log P}$ is strictly decreasing on $P>1$.
\end{lemma}
\begin{proof}
Writing $\log g(P)=(1-\alpha)\log P-\log\log P$ gives
$(\log g)'=\dfrac{1-\alpha}{P}-\dfrac{1}{P\log P}<0$ for $P>1$.
\end{proof}

\begin{corollary}[Minimal tail parameter for a target $\eta$]\label{cor:P1-minP}
Given $\alpha>1$, $x_0\ge 17$ and target $\eta>0$, define $P_\eta$ to be the smallest integer $P\ge x_0$ such that
\[
\frac{1.25506\,\alpha}{(\alpha-1)\,\log P}\,P^{1-\alpha}\ \le\ \eta.
\]
By Lemma~\ref{lem:P1-monotone} this $P_\eta$ exists and is unique; moreover, the inequality then holds for every $P\ge P_\eta$. (The same definition with $\log P$ replaced by $\log P-1$ gives the $x_0\ge 599$ Dusart variant.)
\end{corollary}
\begin{proof}
The left-hand side equals a positive constant times $g(P)=P^{1-\alpha}/\log P$. By Lemma~\ref{lem:P1-monotone}, $g$ is strictly decreasing on $P>1$, hence the inequality threshold defines a unique minimal integer $P_\eta\ge x_0$ and persists for all larger $P$.
\end{proof}
\paragraph{Use in $(\star)$ and covering.}
To enforce a tail $\sum_{p>P}p^{-\alpha}\le \eta$ it suffices, by \eqref{eq:P1}, to take $P\ge17$ solving
\[
 \frac{1.25506\,\alpha}{(\alpha-1)\,\log P}\,P^{\,1-\alpha}\ \le\ \eta.
\]
The practical choice $P=\max\{17,\ ((1.25506\,\alpha)/((\alpha-1)\eta))^{1/(\alpha-1)}\}$ already meets the inequality up to the mild $\log P$ factor; one may increase $P$ monotonically until the left side is $\le\eta$.
\subsection*{Finite-block spectral gap certificate on $[\sigma_0,1]$}
We make explicit the finite-block matrix $H(\sigma)$ used in the spectral-gap/passivity certificate.

\begin{definition}[Finite-block passivity/Pick matrix]\label{def:finite-block-passivity-matrix}
Fix a prime cut $P$ and per-prime truncation lengths $N_p\ge 1$. Let
\[
  \mathcal I\ :=\ \{(p,n):\ p\le P\ \text{prime},\ 1\le n\le N_p\}.
\]
Fix nonnegative weights $(w_n)_{n\ge 1}$ with
\[
  \sum_{n\ge 1} w_n\ =\ \frac12
  \qquad\text{(e.g.\ Lemma~\ref{lem:weights-geometric}).}
\]
Let $\psi_{\mathrm{cert}}:=\tfrac1{12}\psi$ be the scaled certificate window from Lemma~\ref{lem:psi-cert-Cwin}, and define its Fourier transform by
\[
  \widehat{\psi_{\mathrm{cert}}}(\xi)\ :=\ \int_{\R}\psi_{\mathrm{cert}}(t)\,e^{-it\xi}\,dt,\qquad
  C_{\mathrm{win}}\ :=\ \sup_{\xi\in\R}\big|\widehat{\psi_{\mathrm{cert}}}(\xi)\big|.
\]
For $\sigma\in[\sigma_0,1]$, define a Hermitian matrix $H(\sigma)\in\C^{|\mathcal I|\times|\mathcal I|}$ by the entry formula
\[
  H_{(p,n),(q,m)}(\sigma)\;:=\;\delta_{pq}\,\delta_{nm}\;-\;w_n w_m\,
  p^{-(\sigma+\tfrac12)}\,q^{-(\sigma+\tfrac12)}\,
  \widehat{\psi_{\mathrm{cert}}}\!\big(n\log p-m\log q\big),
  \qquad (p,n),(q,m)\in\mathcal I.
\]
We view $H(\sigma)$ as a block matrix $H(\sigma)=[H_{pq}(\sigma)]_{p,q\le P}$ with $H_{pq}(\sigma)\in\C^{N_p\times N_q}$.
Write $D_p(\sigma):=H_{pp}(\sigma)$ and $E(\sigma):=H(\sigma)-\mathrm{diag}(D_p(\sigma))$.
\end{definition}

\begin{definition}[Certificate coupling operator]\label{def:certificate-operator}
With the same index set \(\mathcal I\), weights \((w_n)\), and certificate window \(\psi_{\mathrm{cert}}\) as above, define for each \(\sigma\in[\sigma_0,1]\) the linear operator
\[
  \Gamma_\sigma:\C^{\mathcal I}\ \to\ L^2(\psi_{\mathrm{cert}}),\qquad
  (\Gamma_\sigma x)(t)\ :=\ \sum_{(p,n)\in\mathcal I} x_{(p,n)}\,w_n\,p^{-(\sigma+\tfrac12)}\,e^{-it\,n\log p}.
\]
Equivalently, on basis vectors \(e_{(p,n)}\in\C^{\mathcal I}\),
\[
  (\Gamma_\sigma e_{(p,n)})(t)\ :=\ w_n\,p^{-(\sigma+\tfrac12)}\,e^{-it\,n\log p}.
\]
\end{definition}

\begin{lemma}[A concrete weight sequence]\label{lem:weights-geometric}
Define, for $n\ge 1$,
\[
  w_n:=\frac1{19}\left(\frac{17}{19}\right)^{n-1}.
\]
Then $w_n\ge 0$, $\sum_{n\ge 1} w_n=\frac12$, and
\[
  \sum_{n\ge 1} w_n^2=\frac1{72}.
\]
Consequently, for any truncation length $N\in\N$,
\[
  \sum_{n=1}^N w_n\le \frac12,\qquad \sum_{n=1}^N w_n^2\le \frac1{72}.
\]
\end{lemma}

\begin{proof}
Both series are geometric. First,
\[
  \sum_{n\ge 1} w_n=\frac1{19}\sum_{n\ge 0}\left(\frac{17}{19}\right)^n
  =\frac1{19}\cdot \frac{1}{1-\frac{17}{19}}=\frac1{19}\cdot\frac{19}{2}=\frac12.
\]
Second,
\[
  \sum_{n\ge 1} w_n^2=\frac1{361}\sum_{n\ge 0}\left(\frac{289}{361}\right)^n
  =\frac1{361}\cdot \frac{1}{1-\frac{289}{361}}
  =\frac1{361}\cdot \frac{361}{72}=\frac1{72}.
\]
Truncation only decreases the sums.
\end{proof}

\begin{lemma}[Off-diagonal enclosure from the explicit formula]\label{lem:offdiag-enclosure}
For $p\neq q$, uniformly for $\sigma\in[\sigma_0,1]$,
\[
  \|H_{pq}(\sigma)\|_2\ \le\ \frac{C_{\mathrm{win}}}{4}\,p^{-(\sigma+\tfrac12)}\,q^{-(\sigma+\tfrac12)}.
\]
\end{lemma}
\begin{proof}
Fix $\sigma\in[\sigma_0,1]$ and primes $p\neq q$. Let $x\in\C^{N_p}$ and $y\in\C^{N_q}$ be unit vectors.
Using $|\widehat{\psi_{\mathrm{cert}}}|\le C_{\mathrm{win}}$,
\[
  |x^*H_{pq}(\sigma)y|
  \le C_{\mathrm{win}}\,p^{-(\sigma+\tfrac12)}q^{-(\sigma+\tfrac12)}
     \sum_{n\le N_p}\sum_{m\le N_q} w_n w_m\,|x_n|\,|y_m|.
\]
Factor the double sum and apply Cauchy--Schwarz:
\[
  \sum_{n\le N_p}\sum_{m\le N_q} w_n w_m\,|x_n|\,|y_m|
  =\Bigl(\sum_{n\le N_p} w_n|x_n|\Bigr)\Bigl(\sum_{m\le N_q} w_m|y_m|\Bigr)
  \le\Bigl(\sum_{n\le N_p} w_n\Bigr)\Bigl(\sum_{m\le N_q} w_m\Bigr)
  \le \frac14,
\]
since $\sum_{n\ge 1}w_n=\tfrac12$ and the truncations only decrease the sum.
Therefore
\[
  |x^*H_{pq}(\sigma)y|\ \le\ \frac{C_{\mathrm{win}}}{4}\,p^{-(\sigma+\tfrac12)}\,q^{-(\sigma+\tfrac12)}.
\]
Taking the supremum over $\|x\|_2=\|y\|_2=1$ yields the claimed operator-norm bound.
\end{proof}
\begin{lemma}[Block Gershgorin lower bound]\label{lem:block-gersh}
For every $\sigma\in[\sigma_0,1]$,
\[
  \lambda_{\min}\big(H(\sigma)\big)\ \ge\ \min_{p\le P}\Big(\lambda_{\min}\big(D_p(\sigma)\big)\ -\ \sum_{q\ne p}\|H_{pq}(\sigma)\|_2\Big).
\]
\end{lemma}
\begin{proof}
Fix $\sigma\in[\sigma_0,1]$ and write a vector $x\in\C^{|\mathcal I|}$ in blocks $x=(x_p)_{p\le P}$ with $x_p\in\C^{N_p}$. Since $H(\sigma)$ is Hermitian,
\[
  \langle Hx,x\rangle
  =\sum_{p}\langle D_p x_p,x_p\rangle\ +\ \sum_{p\neq q}\Re\langle H_{pq}x_q,x_p\rangle.
\]
For $p\neq q$, $|\langle H_{pq}x_q,x_p\rangle|\le \|H_{pq}\|_2\,\|x_p\|\,\|x_q\|$, and $2ab\le a^2+b^2$ gives
\[
  2\,\|H_{pq}\|_2\,\|x_p\|\,\|x_q\|\ \le\ \|H_{pq}\|_2\big(\|x_p\|^2+\|x_q\|^2\big).
\]
Summing over $p\neq q$ yields
\[
  \langle Hx,x\rangle\ \ge\ \sum_p \Big(\lambda_{\min}(D_p)-\sum_{q\neq p}\|H_{pq}\|_2\Big)\|x_p\|^2
  \ \ge\ \Big(\min_{p}\Big(\lambda_{\min}(D_p)-\sum_{q\neq p}\|H_{pq}\|_2\Big)\Big)\|x\|^2.
\]
Taking the infimum of the Rayleigh quotient $\langle Hx,x\rangle/\|x\|^2$ over $x\neq0$ gives the stated lower bound for $\lambda_{\min}(H(\sigma))$.
\end{proof}
\begin{lemma}[Schur--Weyl bound]\label{lem:schur-weyl-gap}
For every $\sigma\in[\sigma_0,1]$,
\[
  \lambda_{\min}\big(H(\sigma)\big)\ \ge\ \delta(\sigma_0),\qquad
  \delta(\sigma_0):=\max\big\{\delta_{\mathrm{Gersh}}(\sigma_0),\,\delta_{\mathrm{Schur}}(\sigma_0)\big\},
\]
where
\[
  \delta_{\mathrm{Gersh}}(\sigma_0):=\min_p\Big(\mu_p^L-\sum_{q\ne p}U_{pq}\Big),\qquad
  \delta_{\mathrm{Schur}}(\sigma_0):=\min_p \mu_p^L\ -\ \max_q\frac{1}{\sqrt{\mu_q^L}}\sum_{p\ne q}\sqrt{\mu_p^L}\,U_{pq}.
\]
In particular, if $\delta(\sigma_0)\ge 0$ then $\lambda_{\min}(H(\sigma))\ge 0$ for all $\sigma\in[\sigma_0,1]$.
\end{lemma}
\begin{proof}
This is a standard block Schur-complement/Weyl-type lower bound: after normalizing each diagonal block by its lower spectral bound $\mu_p^L$, the off-diagonal operator norms are bounded by the budgets $U_{pq}$. The first term in the maximum is the direct block Gershgorin bound (Lemma~\ref{lem:block-gersh}). The second term comes from a weighted Schur test: for a unit vector $x=(x_p)$, bound $\sum_{p\neq q}\Re\langle H_{pq}x_q,x_p\rangle$ by Cauchy–Schwarz with weights $\sqrt{\mu_p^L}$ and use $\|H_{pq}\|_2\le U_{pq}$ to obtain
\[
  \langle Hx,x\rangle \ \ge\ \min_p \mu_p^L\ -\ \max_q\frac{1}{\sqrt{\mu_q^L}}\sum_{p\ne q}\sqrt{\mu_p^L}\,U_{pq}.
\]
Taking the maximum of the two lower bounds yields the stated $\delta(\sigma_0)$. The final implication is immediate.
\end{proof}
\subsection*{Determinant--zeta link (L1; corrected domain)}

\begin{remark}[Using prime-tail bounds]
If $\|H_{pq}(\sigma)\|_2\le C(\sigma_0)(pq)^{-\sigma_0}$ for $p\ne q$, then $\sum_{q\ne p}U_{pq}\le C(\sigma_0)\,p^{-\sigma_0}\sum_{q\le P} q^{-\sigma_0}$, and the sum is bounded explicitly by the Rosser--Schoenfeld tail with $\alpha=2\sigma_0>1$. Thus $\delta(\sigma_0)>0$ can be certified by choosing $P,\{N_p\}$ so that the off-diagonal budget is dominated by $\min_p\mu_p^L$.
\end{remark}

\begin{proposition}[Concrete certified spectral gap at $\sigma_0=0.6$]\label{prop:delta-cert-06}
Fix $\sigma_0=0.6$, take $Q=29$ and $p_{\min}:=\mathrm{nextprime}(Q)=31$, and set $\sigma^\star:=\sigma_0+\tfrac12=1.1$.
Assume the uniform off--diagonal enclosure (for all $p\neq q$, uniformly in $\sigma\in[\sigma_0,1]$)
\[
\|H_{pq}(\sigma)\|_2 \ \le\  \frac{C_{\mathrm{win}}}{4}\, p^{-(\sigma+\tfrac12)}\, q^{-(\sigma+\tfrac12)},
\qquad C_{\mathrm{win}}=0.25,
\]
together with the diagonal lower bound
\[
\mu_p^{\mathrm L}\ \ge\ 1-\frac{(1-\sigma_0)(\log p)\,p^{-\sigma_0}}{6}.
\]
Then $\lambda_{\min}(H(\sigma))\ge 0.72$ for all $\sigma\in[\sigma_0,1]$.
\end{proposition}
\begin{proof}
A direct evaluation over primes $p\le Q$ gives
\[
\sum_{p\le 29} p^{-1.1}=1.3239981250,\qquad \sum_{\substack{p\le 29\\ p\neq 2}} p^{-1.1}=0.8574816292.
\]
The integer--tail majorant
\[
\sum_{n\ge p_{\min}-1} n^{-1.1}\ \le\ \frac{(p_{\min}-1)^{1-1.1}}{1.1-1}=7.1168510179
\]
then implies the four row--sum budgets (small/far blocks, $2$ singled out)
\[
\Delta_{\mathrm{FS}}= \frac{0.25}{4}\,31^{-1.1}\!\!\sum_{p\le 29}p^{-1.1}=0.0018935184,\qquad
\Delta_{\mathrm{FF}}\le \frac{0.25}{4}\,31^{-1.1}\!\!\sum_{n\ge 30}n^{-1.1}=0.0101781777,
\]
\[
\Delta_{\mathrm{SS}}=\frac{0.25}{4}\,2^{-1.1}\!\!\sum_{\substack{p\le 29\\ p\neq 2}}p^{-1.1}=0.0250018328,\qquad
\Delta_{\mathrm{SF}}\le\frac{0.25}{4}\,2^{-1.1}\!\!\sum_{n\ge 30}n^{-1.1}=0.2075080249.
\]
For the diagonal blocks, the bound $\mu_p^{\mathrm L}\ge 1-\tfrac16(1-\sigma_0)(\log p)p^{-\sigma_0}$ gives
\[
\mu_{\min}^{\mathrm{far}}\ge 1-\frac{(1-\sigma_0)(\log 31)\,31^{-0.6}}{6}=0.9708330916,\qquad
\mu_{\min}^{\mathrm{small}}\ge 1-\frac{(1-\sigma_0)(\log 5)\,5^{-0.6}}{6}=0.9591491624.
\]
Thus every row in the small block satisfies
\[
\mu_{\min}^{\mathrm{small}}-(\Delta_{\mathrm{SS}}+\Delta_{\mathrm{SF}})=0.7266393047>0.72,
\]
and every far--block row satisfies
\[
\mu_{\min}^{\mathrm{far}}-(\Delta_{\mathrm{FS}}+\Delta_{\mathrm{FF}})=0.9587613956>0.72.
\]
Taking the minimum of these two certified bounds gives $\lambda_{\min}(H(\sigma))\ge 0.72$ uniformly for $\sigma\in[\sigma_0,1]$.
\end{proof}

\subsection*{Truncation tail control and global assembly (P4)}
Write the head/tail split by primes as $\mathcal P_{\le P}=\{p\le P\}$ and $\mathcal P_{>P}=\{p>P\}$. In the normalised basis at $\sigma_0$ set
\[ X:=\bigl[\widetilde H_{pq}\bigr]_{p,q\le P},\quad Y:=\bigl[\widetilde H_{pq}\bigr]_{p\le P<q},\quad Z:=\bigl[\widetilde H_{pq}\bigr]_{p,q>P}. \]
Let $A_p^2:=\sum_{i\le N_p} w_i^2$ denote the block weight squares (unweighted: $A_p^2=N_p$; the weighted example in Lemma~\ref{lem:weights-geometric} gives $A_p^2\le\tfrac1{72}$). Define
\[ S_2(\le P):=\sum_{p\le P} A_p^2 p^{-2\sigma_0},\qquad S_2(>P):=\sum_{p>P} A_p^2 p^{-2\sigma_0}. \]
Then
\[ \|Y\|\le C_{\rm win}\sqrt{S_2(\le P)S_2(>P)},\qquad \lambda_{\min}(Z)\ge \mu_{\mathrm{diag}}-C_{\rm win}S_2(>P), \]
where $\mu_{\mathrm{diag}}:=\inf_{p>P}\mu_p^{\mathrm L}$. Consequently,
\[ \lambda_{\min}(\mathbb A)\ge \min\Big\{\,\delta_P - \dfrac{C_{\rm win}^2 S_2(\le P)S_2(>P)}{\mu_{\mathrm{diag}}-C_{\rm win}S_2(>P)}\,,\ \mu_{\mathrm{diag}}-C_{\rm win}S_2(>P)\Big\}, \]
with $\delta_P$ the head finite-block gap from above. Using the integer tail $\sum_{n>P}n^{-2\sigma_0}\le (P-1)^{1-2\sigma_0}/(2\sigma_0-1)$ yields a closed-form tail bound for $S_2(>P)$.
\paragraph{Small-prime disentangling (P3).}
Excising $\{p\le Q\}$ improves the head budget by at least $\min_{p>Q}\sum_{q\le Q}\|\widetilde H_{pq}\|$, which in the unweighted case is $\ge N_{\max} P^{-\sigma_0} S_{\sigma_0}(Q)$ and in the weighted case $\ge \tfrac14 P^{-\sigma_0} S_{\sigma_0}(Q)$, with $S_{\sigma_0}(Q)=\sum_{p\le Q}p^{-\sigma_0}$.

\subsection*{No-hidden-knobs audit (P6)}
All constants in $(\star)$, (4), and the gap $B$ are fixed by explicit inequalities: prime tails via integer or Rosser--Schoenfeld bounds, weights as in Lemma~\ref{lem:weights-geometric} (so $\sum w_n=1/2$), off-diagonal $U_{pq}\le (\sum w^{(p)})(\sum w^{(q)})(pq)^{-\sigma_0}\le \tfrac14 (pq)^{-\sigma_0}$, and in-block $\mu_p^{\rm L}$ by interval Gershgorin/LDL$^\top$. No tuned parameters enter; $P(\sigma_0,\varepsilon)$, $N_p(\sigma_0,\varepsilon,P)$, and $B$ are determined from these definitions.

\begin{lemma}[AAB bandlimit: prime-layer identity and a scale-uniform $\sigma=1$ bound]\label{lem:aab-bandlimit-prime}
On the half-plane $\{\Re s>1\}$ one has the exact Euler-product identity
\[
  \zeta(s)\,\dettwo(I-A(s))\ =\ \prod_{p}\exp(p^{-s})\ =\ \exp\!\Big(\sum_{p}p^{-s}\Big),
\]
and hence
\begin{equation}\label{eq:prime-layer-logder}
  \frac{\zeta'}{\zeta}(s)\ +\ \frac{\dettwo'}{\dettwo}(s)\ =\ -\sum_{p}(\log p)\,p^{-s}.
\end{equation}
In particular, for $s=1+it$,
\[
  \Im\!\Big(\frac{\zeta'}{\zeta}+\frac{\dettwo'}{\dettwo}\Big)(1+it)
  \ =\ -\sum_{p}(\log p)\,p^{-1}\sin(t\log p),
\]
where the series should be understood as the boundary value (in $t$, away from $t=0$) of the analytic function
$-\sum_p (\log p)\,p^{-\sigma-it}\sin(t\log p)$ on $\Re s=\sigma>1$; we do not need pointwise absolute convergence on $\Re s=1$.

Fix $L>0$ and $\kappa>0$ and set $\Delta:=\kappa/L$.
Let $\kappa_L\in L^1(\R)$ satisfy $\widehat{\kappa_L}(\xi)=1$ for $|\xi|\le \Delta$ and $0\le \widehat{\kappa_L}\le 1$.
For a window $\psi_{L,t_0}(t)=\psi((t-t_0)/L)$ set $\Phi_{L,t_0}:=\psi_{L,t_0}*\kappa_L$.
Then there is an absolute constant $C_1$ such that for all $t_0\in\R$,
\begin{equation}\label{eq:aab-sigma1}
  \Bigg|\int_{\R}\Im\!\Big(\frac{\zeta'}{\zeta}+\frac{\dettwo'}{\dettwo}\Big)(1+it)\,\Phi_{L,t_0}(t)\,dt\Bigg|
  \ \le\ C_1\,\|\psi\|_{L^1}\,\kappa.
\end{equation}
\end{lemma}
\begin{proof}
The product identity follows immediately from the Euler products:
$\zeta(s)=\prod_p(1-p^{-s})^{-1}$ and $\dettwo(I-A(s))=\prod_p(1-p^{-s})\exp(p^{-s})$ for $\Re s>1$.
Differentiating $\log$ gives \eqref{eq:prime-layer-logder}.

For the bandlimit bound, the Fourier support of $\Phi_{L,t_0}$ is contained in $[-\Delta,\Delta]$, so pairing against $\sin(t\log p)$ sees only primes with $\log p\le \Delta$.
Moreover $\widehat{\psi_{L,t_0}}(\xi)=L\,e^{-it_0\xi}\widehat{\psi}(L\xi)$, hence $\sup_\xi |\widehat{\Phi_{L,t_0}}(\xi)|\le L\,\|\psi\|_{L^1}$.
Therefore,
\[
  \Bigg|\int_{\R}\sum_{\log p\le \Delta}(\log p)\,p^{-1}\sin(t\log p)\,\Phi_{L,t_0}(t)\,dt\Bigg|
  \ \le\ \sup_\xi |\widehat{\Phi_{L,t_0}}(\xi)|\cdot \sum_{\log p\le \Delta}\frac{\log p}{p}.
\]
By Chebyshev's bound $\theta(x)=\sum_{p\le x}\log p\ll x$ and partial summation, there is an absolute $C_1$ such that
$\sum_{p\le e^\Delta}(\log p)/p\le C_1\,\Delta$ for all $\Delta\ge 1$ (and trivially for $\Delta\in(0,1]$ after enlarging $C_1$).
Substituting $\Delta=\kappa/L$ yields \eqref{eq:aab-sigma1}.
\end{proof}
\begin{remark}[Relevance to \textup{(CB$_{\rm NF}$)}]\label{rem:aab-cbnf}
Lemma~\ref{lem:aab-bandlimit-prime} is \emph{scale-uniform} in the sense that it produces a bound depending on $\kappa$ (bandwidth) but not on the physical scale $L$.
This is the right \emph{shape} for a near-field budget input.
However, the near-field energy barrier (Lemma~\ref{lem:energy-barrier}) needs a scale-uniform Carleson budget for the \emph{inner/zero-induced} phase-velocity, i.e. a bound on $C_{{\rm box},{\rm NF}}^{(\zeta)}(\sigma_0)$.
The AAB bound controls only a \emph{prime-layer} term on the absolutely convergent line $\Re s=1$; converting it into a full (CB$_{\rm NF}$) discharge still requires an additional mechanism linking the near-boundary phase-velocity budget to such band-limited prime-layer controls.
\smallskip

\noindent\emph{Why the naive $\Re s=\tfrac12$ analogue blows up.}
If one attempts to repeat the same argument on $\Re s=\tfrac12$ using the formal prime-layer truncation
$\sum_{\log p\le \Delta}(\log p)\,p^{-1/2}\sin(t\log p)$, the trivial bounds force a factor roughly
$\sqrt{\#\{p:\log p\le\Delta\}}\asymp e^{\Delta/2}/\sqrt{\Delta}$, which is catastrophic when $\Delta=\kappa/L$ and $L\downarrow 0$.
Thus any route that upgrades Lemma~\ref{lem:aab-bandlimit-prime} to a \textup{(CB$_{\rm NF}$)}-type scale-uniform near-boundary budget must exploit genuinely nontrivial cancellation (an explicit-formula/short-interval density input), not just Chebyshev-level prime bounds.
\end{remark}

\subsection*{A concrete missing step: a bandlimited explicit-formula hypothesis implying \textup{(CB$_{\rm NF}$)}}\label{sec:efbl-to-cbnf}
We now formalize one explicit, audit-friendly hypothesis that would discharge the near-field budget \textup{(CB$_{\rm NF}$)}.
The point is to isolate a \emph{bandlimited, weighted off-critical zero packing} inequality (which is naturally attacked by explicit-formula methods) from the geometric step that turns such packing into a Carleson energy bound.

\begin{definition}[Defect measure and bandlimited majorants]\label{def:defect-measure-bandlimit}
Let $\Omega=\{\,\Re s>\tfrac12\,\}$ and write an off-critical zero as $\rho=\beta+i\gamma$ with depth $\eta:=\beta-\tfrac12>0$.
Define the \emph{defect measure} on $\Omega$ by
\[
  \nu\ :=\ \sum_{\substack{\rho=\beta+i\gamma\\ \beta>1/2}} 2(\beta-\tfrac12)\,\delta_{\rho}.
\]
Given $t_0\in\R$ and $L>0$, write $I_{L,t_0}:=[t_0-L,t_0+L]$ and $Q(\alpha I_{L,t_0})=I_{L,t_0}\times(0,\alpha|I_{L,t_0}|]$ in $(t,\sigma)$ coordinates.

We say that a family of functions $\Phi_{L,t_0}:\R\to[0,\infty)$ is a \emph{bandlimited majorant family at bandwidth $\kappa/L$} if for each $L,t_0$:
\begin{itemize}
\item $\Phi_{L,t_0}(t)\ge 1$ for all $t\in I_{L,t_0}$,
\item $\widehat{\Phi_{L,t_0}}$ is supported in $[-\kappa/L,\kappa/L]$.
\end{itemize}
\end{definition}

\begin{remark}[Majorants exist (Beurling--Selberg)]\label{rem:beurling-selberg}
Bandlimited majorants of interval indicators with bandwidth $\asymp 1/L$ are classical (Beurling--Selberg extremal problems).
In particular, one can take $\Phi_{L,t_0}$ to be a translate/scale of the standard Beurling--Selberg majorant for $\mathbf 1_{[-1,1]}$ of exponential type $\asymp 1$; then $\widehat{\Phi_{L,t_0}}$ is supported in $[-\kappa/L,\kappa/L]$ and $\Phi_{L,t_0}\ge 1$ on $I_{L,t_0}$.
We suppress the explicit closed form because only the bandwidth and the majorant property are used in \textup{(EF$_{\rm BL}$)}.
\end{remark}

\begin{definition}[Bandlimited explicit-formula near-field hypothesis \textup{(EF$_{\rm BL}$)}]\label{def:EFBL}
Fix $\sigma_0\in(1/2,1)$. We say that \textup{(EF$_{\rm BL}$)} holds at $\sigma_0$ if there exist constants $\kappa>0$ and $C_{\rm EF}<\infty$ and a bandlimited majorant family $\Phi_{L,t_0}$ (Definition~\ref{def:defect-measure-bandlimit}) such that for every $t_0\in\R$ and every $L\in(0,\sigma_0-\tfrac12]$,
\begin{equation}\label{eq:EFBL-inequality}
  \sum_{\substack{\rho=\beta+i\gamma\\ 1/2<\beta\le 1/2+\alpha|I_{L,t_0}|}} 2(\beta-\tfrac12)\,\Phi_{L,t_0}(\gamma)\ \le\ C_{\rm EF}\,|I_{L,t_0}|.
\end{equation}
\end{definition}

\begin{proposition}[\textup{(EF$_{\rm BL}$)} $\Rightarrow$ \textup{(CB$_{\rm NF}$)} (conceptual reduction)]\label{prop:EFBL-implies-CBNF}
Assume \textup{(EF$_{\rm BL}$)} at $\sigma_0$.
Then the defect measure $\nu$ is Carleson on short boxes up to near-field scale:
there is a constant $C_\nu<\infty$ such that for all intervals $I$ with $|I|\le 2(\sigma_0-\tfrac12)$,
\[
  \nu\big(Q(\alpha I)\big)\ \le\ C_\nu\,|I|.
\]
Consequently, the near-field Carleson energy budget constant in \textup{(CB$_{\rm NF}$)} is finite:
$C_{{\rm box},{\rm NF}}^{(\zeta)}(\sigma_0)<\infty$.
\end{proposition}
\begin{proof}[Proof sketch]
Fix $I=I_{L,t_0}$ and apply \eqref{eq:EFBL-inequality}. Since $\Phi_{L,t_0}\ge 1$ on $I$, every off-critical zero $\rho=\beta+i\gamma$ with $\gamma\in I$ contributes at least $2(\beta-\tfrac12)$ to the left-hand side, hence
\[
  \nu(Q(\alpha I))\ =\ \sum_{\substack{\rho=\beta+i\gamma\\ \gamma\in I,\ 0<\beta-\tfrac12\le \alpha|I|}}
  2(\beta-\tfrac12)
  \ \le\ \sum_{\substack{\rho=\beta+i\gamma\\ 1/2<\beta\le 1/2+\alpha|I|}}2(\beta-\tfrac12)\Phi_{L,t_0}(\gamma)
  \ \ll\ |I|.
\]
This proves the Carleson packing claim.

The implication ``$\nu$ Carleson $\Rightarrow$ $C_{{\rm box},{\rm NF}}^{(\zeta)}(\sigma_0)<\infty$'' is standard for Blaschke/Green potentials in the half-plane: the Dirichlet-energy measure of the corresponding inner factor is Carleson with norm controlled by the Carleson norm of $\nu$.
One may prove this by the same annular $L^2$ aggregation used in Proposition~\ref{prop:Kxi-finite} (cf.\ Lemma~\ref{lem:annular-balayage}), applied to the Blaschke kernel sums weighted by $2(\beta-\tfrac12)$, or cite the Carleson-measure characterization of Blaschke products in the half-plane (e.g.\ Garnett, Ch.~VI).
\end{proof}

\subsection*{The complete reduction chain: from Green potentials to the arithmetic blocker}

We now make the reduction from \textup{(CB$_{\rm NF}$)} to a single, cleanly-stated arithmetic problem fully explicit. This chain shows exactly where new input is needed.

\begin{lemma}[Green potential of an off-critical zero]\label{lem:green-potential}
Let $\rho = \tfrac12 + \eta + i\gamma$ be an off-critical zero with depth $\eta > 0$. The half-plane Blaschke factor is
\[
  B_\rho(s) = \frac{s - \rho^*}{s - \rho}, \qquad \rho^* = \tfrac12 - \eta + i\gamma.
\]
Its log-modulus potential in coordinates $s = \tfrac12 + \sigma + it$ is
\[
  U_\rho(\sigma, t) := \log|B_\rho(s)| = \frac{1}{2}\log\frac{(\sigma + \eta)^2 + (t-\gamma)^2}{(\sigma - \eta)^2 + (t-\gamma)^2}.
\]
The boundary normal derivative is
\begin{equation}\label{eq:boundary-normal-derivative}
  \partial_\sigma U_\rho(0, t) = \frac{2\eta}{(t-\gamma)^2 + \eta^2} = 2\pi P_\eta(t - \gamma),
\end{equation}
where $P_\eta(u) = \frac{1}{\pi}\frac{\eta}{\eta^2 + u^2}$ is the Poisson kernel.
\end{lemma}

\begin{proof}
Direct calculation from the definition of $U_\rho$.
\end{proof}

\begin{definition}[Boundary balayage density]
The \emph{boundary balayage density} of the off-critical zeros is
\[
  \mu(t) := \sum_{\rho = \tfrac12 + \eta + i\gamma,\ \eta > 0} 2\eta \cdot P_\eta(t - \gamma) = \frac{1}{\pi}\sum_{\rho} \frac{2\eta^2}{\eta^2 + (t-\gamma)^2}.
\]
This is the normal derivative of the total Green potential $U = \sum_\rho U_\rho$ on the boundary.
\end{definition}

\begin{lemma}[Bandlimited Poisson comparability]\label{lem:poisson-comparability}
Let $\Phi: \mathbb{R} \to [0, \infty)$ with $\operatorname{supp}\widehat{\Phi} \subset [-\Delta, \Delta]$ and $\widehat{\Phi} \ge 0$. Then for any $\eta > 0$:
\[
  e^{-\eta\Delta} \Phi(t) \le (P_\eta * \Phi)(t) \le \Phi(t).
\]
In particular, if $0 < \eta \le c/\Delta$ for some constant $c$, then
\[
  (P_\eta * \Phi)(t) \asymp \Phi(t)
\]
with constants depending only on $c$.
\end{lemma}

\begin{proof}
The Fourier transform of $P_\eta$ is $\widehat{P_\eta}(\omega) = e^{-\eta|\omega|}$. Since $\widehat{P_\eta * \Phi} = \widehat{P_\eta} \cdot \widehat{\Phi}$ and $\widehat{\Phi}$ is supported in $[-\Delta, \Delta]$, we have $e^{-\eta\Delta} \le e^{-\eta|\omega|} \le 1$ on this support. The pointwise bound follows from inverse Fourier transform using $\widehat{\Phi} \ge 0$.
\end{proof}

\begin{proposition}[Complete chain: \textup{(EF$_{\rm BL}$)} $\Rightarrow$ $\nu$ Carleson $\Rightarrow$ \textup{(CB$_{\rm NF}$)}]\label{prop:complete-chain}
The following implications hold with explicit constant maps:
\begin{enumerate}
\item \textup{(EF$_{\rm BL}$)} with constant $C_{\rm EF}$ implies $\nu$ is Carleson with $|\nu|_{\rm Carl} \lesssim C_{\rm EF}$.
\item $\nu$ Carleson with $|\nu|_{\rm Carl} \le C_\nu$ implies $C_{{\rm box},{\rm NF}}^{(\zeta)}(\sigma_0) \lesssim K_0 + C_\nu$.
\end{enumerate}
Consequently, \textup{(EF$_{\rm BL}$)} with $C_{\rm EF}$ small enough implies the energy barrier (Lemma~\ref{lem:energy-barrier}) holds, eliminating zeros in the near-field strip.
\end{proposition}

\begin{remark}[The atomic arithmetic blocker]\label{rem:arithmetic-blocker}
Via the explicit formula, hypothesis \textup{(EF$_{\rm BL}$)} reduces to controlling a \emph{bandlimited prime Dirichlet polynomial}. Specifically, for a bandlimited majorant $\Phi_{L,t_0}$ with bandwidth $\Delta = \kappa/L$, the prime side of the explicit formula yields:
\[
  S_{L,t_0} := \sum_{\log p \le \Delta} (\log p)\, p^{-1/2}\, e^{it_0 \log p}\, \widehat{\Phi_{L,t_0}}(\log p).
\]
The bound required to discharge \textup{(EF$_{\rm BL}$)} is:
\begin{equation}\label{eq:prime-dirichlet-bound}
  \boxed{|S_{L,t_0}| \lesssim 1 \quad \text{for all } t_0 \in \mathbb{R} \text{ and all } 0 < L \le 0.1.}
\end{equation}

\textbf{Why the trivial bound fails.}
The trivial estimate using Cauchy-Schwarz gives $|S_{L,t_0}| \lesssim e^{\Delta/2} / \sqrt{\Delta} = e^{\kappa/(2L)} / \sqrt{\kappa/L}$, which blows up exponentially as $L \to 0$. This is because the $p^{-1/2}$ weight is ``critical'' (exactly at the edge of absolute convergence).

\textbf{What would suffice.}
Any of the following would imply \eqref{eq:prime-dirichlet-bound}:
\begin{itemize}
\item A pointwise bound $|S_{L,t_0}| = O(1)$ from GUE-type cancellation in short intervals.
\item An $L^\infty$ bound for the boundary balayage density $\mu(t) \le C$.
\item A zero-density estimate showing off-critical zeros can't pack tighter than the VK bound allows, uniformly over all short scales.
\end{itemize}
Each represents a different attack surface for the same problem.
\end{remark}

\subsection*{The Recognition Science perspective: conservation forces balance}

The problem of bounding \eqref{eq:prime-dirichlet-bound} admits a structural interpretation from the cost-function framework of Recognition Science (RS).

\begin{remark}[Functional equation as conservation law]\label{rem:rs-conservation}
In RS, the fundamental dynamics are governed by a cost function $J(x) = \frac{1}{2}(x + x^{-1}) - 1$, which is uniquely determined by the Recognition Composition Law
\[
  J(xy) + J(x/y) = 2J(x)J(y) + 2J(x) + 2J(y).
\]
The key properties are:
\begin{enumerate}
\item \textbf{Symmetry:} $J(x) = J(1/x)$ (reciprocity under inversion).
\item \textbf{Unique minimum:} $J(x) = 0 \iff x = 1$.
\item \textbf{Strict convexity:} $J''(1) = 1 > 0$.
\end{enumerate}
These imply the ``Law of Existence'': the only stable configuration is $x = 1$.

For the zeta function, the functional equation $\xi(s) = \xi(1-s)$ plays an analogous role:
\begin{enumerate}
\item \textbf{Symmetry:} $\xi(s) = \xi(1-s)$ (reciprocity under $s \mapsto 1-s$).
\item \textbf{Critical line as identity:} On $\Re s = \tfrac{1}{2}$, we have $1-s = \overline{s}$, so $\xi$ is \emph{real} there.
\item \textbf{Strict phase cost:} Moving off the critical line incurs phase ``cost'' (the argument of $\xi$ can vary continuously).
\end{enumerate}
\end{remark}

\begin{remark}[The explicit formula as ledger balance]\label{rem:rs-ledger}
In RS, conservation forces ``ledger balance'': inflow $=$ outflow at every node. The explicit formula
\[
  \sum_\rho (\text{zero contribution}) = \sum_p (\text{prime contribution}) + (\text{smooth terms})
\]
is precisely such a balance condition. The ``zero side'' (defect measure $\nu$) and the ``prime side'' (Dirichlet polynomial $S_{L,t_0}$) are \emph{equal} when integrated against any test function.

This duality implies: \textbf{if the prime side is bounded, so is the zero side.}

The challenge is proving the prime-side bound \eqref{eq:prime-dirichlet-bound} without assuming RH.
\end{remark}

\begin{remark}[Why the functional equation creates structure]\label{rem:rs-structure}
The functional equation $\xi(s) = \xi(1-s)$ creates a \emph{pairing} in the zeros: if $\rho$ is a zero, so is $1 - \rho$. Combined with conjugation symmetry ($\bar{\rho}$ is also a zero), zeros come in quartets (or pairs on the critical line).

This pairing structure should create \emph{cancellation} in the prime sum $S_{L,t_0}$, because:
\begin{itemize}
\item The contribution from $\rho = \tfrac{1}{2} + \eta + i\gamma$ pairs with that from $1 - \rho = \tfrac{1}{2} - \eta - i\gamma$.
\item Via the explicit formula, this pairing transfers to the prime side.
\item On the critical line ($\eta = 0$), the pairing is \emph{trivial} (self-pairing), so the prime sum is real.
\end{itemize}

The RS insight is: \textbf{the functional equation is the ``composition law'' for zeta}, and it should force the bound \eqref{eq:prime-dirichlet-bound} just as the Recognition Composition Law forces $J(x) > 0$ for $x \neq 1$.

Making this precise is the remaining technical challenge.
\end{remark}

\begin{proposition}[What the RS framework predicts]\label{prop:rs-prediction}
If the analogy between the RS cost function and the zeta functional equation is exact, then:
\begin{enumerate}
\item The ``phase cost'' of an off-critical zero at depth $\eta$ should scale as $\cosh(\eta) - 1 \approx \eta^2/2$ (quadratic near the line).
\item The Carleson packing (which weights by $\eta$) should be dominated by the phase cost (which grows as $\eta^2$).
\item This mismatch implies: zeros very close to the critical line are ``cheap'' in the packing sense but ``expensive'' in the phase sense---a barrier.
\item The explicit formula transfers this barrier to the prime side, giving the bound \eqref{eq:prime-dirichlet-bound}.
\end{enumerate}
\end{proposition}

\begin{remark}[Current status and the gap]\label{rem:rs-gap}
The RS analogy provides \emph{structural intuition} for why the bound \eqref{eq:prime-dirichlet-bound} should hold: the functional equation creates conservation, and conservation forces balance.

However, translating this intuition into a rigorous proof requires:
\begin{itemize}
\item A precise formulation of the ``phase cost'' for zeta (analogous to $J(x)$ in RS).
\item A proof that this phase cost satisfies strict convexity (the analogue of $J'' > 0$).
\item A mechanism to transfer the phase cost bound to the prime Dirichlet polynomial.
\end{itemize}

Until this translation is complete, the bound \eqref{eq:prime-dirichlet-bound} remains the \emph{single atomic target} for unconditional closure.
\end{remark}

\subsection*{Alternative formulation: the localized zero-density template}

We record a cleaner sufficient condition that isolates exactly what is needed.

\begin{proposition}[Localized zero-density template $\Rightarrow$ \textup{(CB$_{\rm NF}$)}]\label{prop:ZD-to-nuCarlesonNF}
Fix $\sigma_0 \in (1/2, 1)$ and $\alpha \in [1,2]$. Suppose there exist constants $C_0, \kappa > 0$ such that for every interval $J \subset \mathbb{R}$ with $|J| \le 4(\sigma_0 - \tfrac{1}{2})$ and every $0 < u \le \alpha|J|$,
\begin{equation}\label{eq:ZD-NF-template}
  \#\big\{\rho = \beta + i\gamma :\ \beta > \tfrac{1}{2} + u,\ \gamma \in J\big\}
  \ \le\ C_0\,|J|\,\log(\langle t_J \rangle + 2)\,(\langle t_J \rangle + 2)^{-\kappa u},
\end{equation}
where $t_J$ is the midpoint of $J$ and $\langle t \rangle := \sqrt{1 + t^2}$.
Then $\nu$ is Carleson on near-field boxes with
\[
  C_{\nu,{\rm NF}}(\sigma_0) \le \frac{2C_0}{\kappa}.
\]
\end{proposition}
\begin{proof}
Fix $J$ with $|J| \le 4(\sigma_0 - \tfrac{1}{2})$.
By the layer-cake formula, $\sum_j \eta_j \le \int_0^{\alpha|J|} \#\{j : \eta_j > u\}\,du$.
Hence
\[
  \nu(Q(\alpha J)) = \sum_{\substack{\rho = \beta + i\gamma,\ \gamma \in J \\ 0 < \beta - \tfrac{1}{2} \le \alpha|J|}} 2(\beta - \tfrac{1}{2})
  \le 2\int_0^{\alpha|J|} \#\{\rho : \beta > \tfrac{1}{2} + u,\ \gamma \in J\}\,du.
\]
Insert \eqref{eq:ZD-NF-template}:
\[
  \nu(Q(\alpha J)) \le 2C_0\,|J|\,\log(\langle t_J \rangle + 2) \int_0^{\alpha|J|} (\langle t_J \rangle + 2)^{-\kappa u}\,du
  \le \frac{2C_0}{\kappa}\,|J|.
\]
\end{proof}

\begin{remark}[The depth-decay requirement]\label{rem:depth-decay}
The key feature of \eqref{eq:ZD-NF-template} is the \emph{exponential decay in depth $u$}: the factor $(\langle t_J \rangle + 2)^{-\kappa u}$ suppresses zeros that are far from the critical line.

Standard zero-density estimates (Vinogradov--Korobov) give:
\[
  N(\sigma, T) := \#\{\rho : \Re\rho \ge \sigma,\ 0 < \Im\rho \le T\} \ll T^{c(1-\sigma)}\log^A T
\]
for some $c < 3$ and $A > 0$. These bounds are \emph{depth-uniform}: they don't decay as $\sigma \to 1/2$.

To prove \eqref{eq:ZD-NF-template}, we need a \emph{depth-dependent} estimate showing that zeros become exponentially rare as they move off the critical line.

This is exactly what the pair correlation conjecture (Montgomery) would give: zero repulsion implies that deeper zeros are suppressed. But pair correlation is only known conditionally on RH.
\end{remark}

\begin{definition}[Weighted local zero moment]\label{def:weighted-local-moment}
Fix $\kappa \in (0,1)$ and an interval $J \subset \mathbb{R}$ with midpoint $t_J$. Set $M_J := \langle t_J \rangle + 2$ and $x_J := M_J^\kappa$. Define
\[
  \mathcal{S}(J; \kappa) := \sum_{\substack{\rho = \beta + i\gamma \\ \gamma \in J,\ \beta > 1/2}} x_J^{\beta - 1/2}.
\]
\end{definition}

\begin{proposition}[Single weighted inequality closes the gap]\label{prop:SJ-to-ZD}
If there exist $\kappa \in (0,1)$ and $C_{\mathcal{S}} < \infty$ such that for all intervals $J$ with $|J| \le 4(\sigma_0 - \tfrac{1}{2})$,
\begin{equation}\label{eq:SJ-bound}
  \mathcal{S}(J; \kappa) \le C_{\mathcal{S}}\,|J|\,\log(\langle t_J \rangle + 2),
\end{equation}
then \eqref{eq:ZD-NF-template} holds with $C_0 = C_{\mathcal{S}}$.
\end{proposition}
\begin{proof}
If $\beta > \tfrac{1}{2} + u$, then $x_J^{\beta - 1/2} \ge x_J^u$, so
\[
  \#\{\rho : \beta > \tfrac{1}{2} + u,\ \gamma \in J\} \le x_J^{-u}\,\mathcal{S}(J; \kappa)
  \le C_{\mathcal{S}}\,|J|\,\log(\langle t_J \rangle + 2)\,M_J^{-\kappa u}.
\]
\end{proof}

\begin{remark}[The explicit-formula attack]\label{rem:explicit-formula-attack}
The weighted moment $\mathcal{S}(J; \kappa)$ can be analyzed via the explicit formula. For a nonnegative test function $\Phi_J$ majorizing $\mathbf{1}_J$, the Guinand--Weil formula gives
\[
  \sum_\rho x_J^{\beta - 1/2}\,\Phi_J(\gamma) = \text{(prime sum)} + \text{(smooth terms)}.
\]
The bound \eqref{eq:SJ-bound} would follow if the prime sum on the right is $O(|J|\log M_J)$.

This is exactly the bound \eqref{eq:prime-dirichlet-bound} in a different guise: the weight $x_J^{\beta - 1/2}$ corresponds to evaluating the prime sum at $s = \tfrac{1}{2} + \log x_J / \log p + it$.

The two formulations are equivalent: proving either one closes the gap.
\end{remark}

\begin{remark}[What we can prove unconditionally, and what remains]\label{rem:EFBL-status}
Proving \textup{(EF$_{\rm BL}$)} is a genuinely arithmetic problem: it is a weighted, short-scale packing bound for off-critical zeros.
Lemma~\ref{lem:aab-bandlimit-prime} shows that \emph{prime-layer} terms at bandwidth $\kappa/L$ can be controlled scale-uniformly on the absolutely convergent line $\Re s=1$.
What is missing is a mechanism (via an explicit formula / contour argument) that turns such prime-layer control into the weighted zero packing \eqref{eq:EFBL-inequality} at the near-boundary scale $L\downarrow 0$.
With only trivial Chebyshev bounds on prime sums at the critical-line weights $p^{-1/2}$, one obtains exponential blow-up in $\Delta=\kappa/L$ (Remark~\ref{rem:aab-cbnf}), so any successful proof of \textup{(EF$_{\rm BL}$)} must use nontrivial cancellation (equivalently, a local zero-density / explicit-formula input beyond VK-level global bounds).
\end{remark}

\subsection*{Selberg's Central Limit Theorem and the remaining gap}\label{sec:selberg-clt}

The preceding analysis using Vinogradov--Korobov zero counts gives a Carleson bound that grows as $\log T$ at height $T$. This growth arises from treating the zeros as \emph{worst-case independent} contributors. A natural hope is that Selberg's Central Limit Theorem, which captures cancellation in the zeros, might eliminate this growth.

\begin{definition}[The argument function $S(t)$]
Define
\[
  S(t) := \frac{1}{\pi}\arg\zeta(1/2 + it),
\]
where the argument is defined by continuous variation from $+\infty$ along horizontal lines. This function encodes the ``prime noise'' on the critical line.
\end{definition}

\begin{theorem}[Selberg's Central Limit Theorem, 1946]\label{thm:selberg-clt}
As $T \to \infty$, the distribution of $S(t)$ for $t$ uniformly distributed in $[T, 2T]$ converges to a Gaussian with mean $0$ and variance
\[
  \mathrm{Var}(S) \sim \frac{1}{2\pi^2}\log\log T.
\]
In particular, $\displaystyle\frac{1}{T}\int_T^{2T} |S(t)|^2\,dt = \frac{1}{2\pi^2}\log\log T + O(1)$.
\end{theorem}

\begin{remark}[Why Selberg does NOT close the gap]\label{rem:selberg-gap}
At first glance, the Selberg CLT appears to give the needed cancellation: the \emph{fluctuation} of zero counts has variance $O(\log\log T)$, not $O(\log T)$. However, the Carleson energy depends on a different quantity.

\textbf{The critical distinction:}
\begin{itemize}
\item \textbf{Selberg controls}: $|S(t)|^2 = |\text{(zero count fluctuation)}|^2 \sim \log\log T$
\item \textbf{Carleson energy requires}: $|S'(t)|^2 = |\text{(zero density)}|^2 \sim (\log T)^2$
\end{itemize}

By the classical mean-value theorem for the zeta function:
\[
  \frac{1}{T}\int_T^{2T} \left|\frac{\zeta'}{\zeta}(1/2+it)\right|^2\,dt \sim (\log T)^2.
\]

The Carleson energy is controlled by this \emph{derivative} quantity, not by $|S|^2$ itself. The $(\log T)^2$ factor means the energy grows with height.

\textbf{Quantitative failure:} For a Carleson box at scale $L = 0.1$ and height $T$:
\[
  C_{\rm box}(L, T) \lesssim K_0 + L\log T.
\]
At $T = 10^{100}$: $C_{\rm box} \approx 0.035 + 0.1 \times 230 = 23$, which equals $C_{\rm crit}(0.05)$.

At $T = 10^{304}$: $C_{\rm box} \approx 0.035 + 0.1 \times 700 = 70 > 23$.

\textbf{The barrier fails at large heights for zeros at depth $\eta = 0.05$.}
\end{remark}

\begin{theorem}[Effective barrier range]\label{thm:effective-barrier}
For zeros at depth $\eta \in (0, 0.1)$ (i.e., $\Re\rho \in (0.5, 0.6)$), the energy barrier holds up to height
\[
  T_{\max}(\eta) = \exp\!\left(\frac{C_{\rm crit}(\eta) - K_0}{2\eta}\right) = \exp\!\left(\frac{L_{\rm rec}^2/(8\eta C(\psi)^2) - K_0}{2\eta}\right).
\]
Numerically:
\begin{center}
\begin{tabular}{ccc}
\textbf{Depth $\eta$} & \textbf{Strip} & \textbf{Protected up to height} \\
\hline
0.10 & $0.50 < \sigma < 0.60$ & $T < 10^{25}$ \\
0.05 & $0.50 < \sigma < 0.55$ & $T < 10^{100}$ \\
0.02 & $0.50 < \sigma < 0.52$ & $T < 10^{625}$ \\
0.01 & $0.50 < \sigma < 0.51$ & $T < 10^{2500}$ \\
\end{tabular}
\end{center}
The barrier protects zeros closer to the critical line to arbitrarily high heights, but the full near-field strip $\eta < 0.1$ is only protected up to $T \approx 10^{25}$.
\end{theorem}

\begin{remark}[What would close the gap]\label{rem:what-closes-gap}
To make the proof unconditional, one needs a \emph{height-independent} bound on the near-field Carleson energy:
\[
  C_{{\rm box},{\rm NF}}^{(\zeta)}(L, T) \le C \quad \text{for all } L \le 0.2, T \ge 1.
\]

\textbf{Why the VK bound grows with height.}
The Carleson energy at scale $L$ and height $T$ is:
\[
  E(L,T) = \iint_{Q(I)} |\nabla U|^2\,\sigma \sim K_0 |I| + \underbrace{(\text{\# near zeros}) \times L}_{\sim L^2 \log T}.
\]
The ``near zeros'' are those within distance $L$ of the interval $I = [T-L, T+L]$; by Riemann--von Mangoldt, there are $\sim L\log T$ such zeros on the critical line. Each contributes $\sim L$ to the energy. Hence $C_{{\rm box}}(L,T) \sim K_0 + L\log T$, which grows with height.

\textbf{Approaches explored and their status:}
\begin{enumerate}
\item \textbf{Extend far-field via interval arithmetic}: Attempted to push the Schur certification from $\sigma_0 = 0.6$ toward $\sigma_0 = 0.55$. \emph{Result:} Failed due to numerical precision---the outer normalization $\mathcal{O}_{\rm can}$ involves $|\zeta(1/2+it)|$, which has zeros nearby, causing interval bounds to exceed 1. The certification succeeds at $\sigma_0 = 0.6$ but fails for smaller values.

\item \textbf{Selberg's CLT}: Controls the \emph{variance} of zero-count fluctuations ($O(\log\log T)$), but the Carleson energy depends on zero \emph{density} ($O(\log T)$). The $L^2$ norm of $S(t)$ is $O(\sqrt{\log\log T})$, but the relevant quantity is $S'(t) \sim \zeta'/\zeta(1/2+it)$, which has $L^2$ norm $O(\log T)$.

\item \textbf{Pair correlation / GUE}: Zero repulsion (Montgomery's pair correlation, conditional on RH) implies off-diagonal cancellation in the energy sum. Unconditionally, only weaker spacing bounds are known ($|\gamma - \gamma'| \ge c/\log T$), which do not suffice. \emph{Status:} Would close the gap if provable unconditionally.

\item \textbf{Second-moment explicit formula}: Control $\int_{T}^{2T} |\zeta'/\zeta(1/2+it)|^2\,dt$ directly. Classical bounds give $\sim T(\log T)^2$, yielding Carleson energy $\sim L(\log T)^2$ per unit interval---worse than VK.
\end{enumerate}

\textbf{The single atomic target.}
By the complete reduction chain (Proposition~\ref{prop:complete-chain} and Remark~\ref{rem:arithmetic-blocker}), the \emph{entire} conditional element reduces to a single arithmetic bound: the prime Dirichlet polynomial $S_{L,t_0}$ in \eqref{eq:prime-dirichlet-bound} must satisfy $|S_{L,t_0}| \lesssim 1$ uniformly. Any progress on this bound (from GUE cancellation, mollifiers, or explicit-formula techniques) translates directly to extending the effective range of the proof.
\end{remark}

%% ============================================================
%% THE LEDGER STIFFNESS PRINCIPLE (Recognition Science Bridge)
%% ============================================================

\subsection*{The Ledger Stiffness Principle: from Recognition Science to Bernstein bounds}\label{sec:ledger-stiffness}

The gap between macroscopic (VK-controlled) and microscopic (short-scale) behavior can be understood through a structural constraint derived from the discrete nature of the prime system.

\begin{remark}[Physical interpretation: the zero as topological defect]\label{rem:zero-as-vortex}
From the Recognition Science perspective:
\begin{itemize}
\item An \textbf{off-critical zero} is a \emph{topological defect} (vortex) in the phase field $W = \arg \xi$.
\item Creating such a defect requires ``tearing'' the phase fabric, demanding a quantized \textbf{creation cost} $L_{\rm rec} \approx 4.43$.
\item The critical strip is populated by ``prime noise''---the fluctuations from the explicit formula. This constitutes the \textbf{available energy budget} $C_{\rm box}$.
\item Classical analysis worries that prime noise might concentrate on microscopic scales, creating infinite energy density that could fund a vortex.
\end{itemize}

The RS insight: the Prime System is a \textbf{Ledger} driven by a discrete clock (the ``atomic tick''). A discrete clock imposes a \textbf{Nyquist limit}: the signal is effectively bandlimited. A bandlimited signal cannot have infinite energy density (infinite slope) without infinite amplitude. Since the amplitude is bounded (logarithmically by $\log T$), the energy density is \textbf{saturated}.
\end{remark}

\begin{definition}[Ledger Stiffness Hypothesis \textup{(LS)}]\label{def:ledger-stiffness}
The \textbf{Ledger Stiffness Hypothesis} asserts that the discrete structure of the prime number system imposes a Bernstein-type constraint on the Dirichlet energy of the potential $U_\xi = \Re\log\xi$. Specifically, there exists a \emph{packing constant} $K_{\rm pack} < \infty$ such that for any vertical interval $I$ of length $|I| \le 1$:
\begin{equation}\label{eq:ledger-stiffness}
  \tag{LS}
  \frac{1}{|I|} \iint_{Q(I)} |\nabla U_\xi|^2\,\sigma\,dt\,d\sigma \ \le\ K_{\rm pack}\,\log\langle T_I\rangle,
\end{equation}
where $T_I$ is the midpoint of $I$ and $\langle T \rangle = \sqrt{1+T^2}$. This bound asserts that the explicit formula behaves as a \emph{bandlimited interpolant}: the gradient energy (``stiffness'') is controlled by the signal amplitude variance (Bernstein's inequality for bandlimited functions).
\end{definition}

\begin{remark}[Bernstein's inequality and bandlimited signals]\label{rem:bernstein}
For a function $f$ with Fourier transform supported in $[-\Omega, \Omega]$, Bernstein's inequality gives
\[
  \|f'\|_{L^2} \le \Omega\,\|f\|_{L^2}.
\]
If the prime fluctuations in the explicit formula are effectively bandlimited (frequency support $\le \log T$ from the prime counting function), then the \emph{gradient} is bounded by $(\log T) \times (\text{amplitude})$. Since the amplitude is $O(\log\log T)$ by Selberg, the gradient is $O(\log T \cdot \log\log T)$, and the squared gradient is $O((\log T)^2 \cdot (\log\log T)^2)$.

However, for the Carleson energy at scale $L$, we integrate over a box of area $\sim L^2$, giving energy $\sim L^2 \cdot (\log T)^2 \cdot (\log\log T)^2$. Dividing by $|I| = 2L$:
\[
  C_{\rm box} \sim L \cdot (\log T)^2 \cdot (\log\log T)^2.
\]
For $L \ll 1/(\log T)^2$, this is $O(1)$, consistent with the barrier. The hypothesis \textup{(LS)} posits this bound holds uniformly.
\end{remark}

\begin{theorem}[Conditional closure under Ledger Stiffness]\label{thm:ledger-closure}
Assume \textup{(LS)} holds with $K_{\rm pack} \lesssim 0.2$ (consistent with the macroscopic Vinogradov--Korobov bounds extrapolated to short scales). Then the Riemann Hypothesis is true.
\end{theorem}
\begin{proof}
Under \textup{(LS)}, the near-field Carleson budget satisfies
\[
  C_{{\rm box},{\rm NF}}^{(\zeta)}(\sigma_0) \le K_0 + K_{\rm pack} \approx 0.035 + 0.160 = 0.195.
\]
By Lemma~\ref{lem:energy-barrier}, the creation cost of a zero at depth $\eta$ requires
\[
  C_{\rm crit} = \frac{L_{\rm rec}^2}{8\,\eta_{\max}\,C(\psi)^2} \approx 11.5.
\]
Since $C_{\rm box} \approx 0.195 \ll C_{\rm crit} \approx 11.5$, the available energy is insufficient by a factor of $\approx 59\times$. No zero can exist in the near-field strip $\{1/2 < \Re s < 0.6\}$.

Combined with the unconditional far-field certification ($\Re s \ge 0.6$ is zero-free by Proposition~\ref{prop:farfield-hybrid}), the entire right half-strip is zero-free. By the functional equation, RH holds.
\end{proof}

\begin{remark}[Equivalence of hypotheses]\label{rem:hypothesis-equivalence}
The following are equivalent formulations of the missing input:
\begin{enumerate}
\item \textbf{(LS)}: Ledger Stiffness bound \eqref{eq:ledger-stiffness}.
\item \textbf{(EF$_{\rm BL}$)}: Bandlimited explicit formula \eqref{eq:EFBL-inequality}.
\item \textbf{(CB$_{\rm NF}$)}: Scale-uniform near-field Carleson budget.
\item \textbf{Prime polynomial bound}: $|S_{L,t_0}| \lesssim 1$ uniformly.
\item \textbf{Depth-decay template}: $\#\{\rho : \beta > 1/2+u\} \le C|J|\log T \cdot T^{-\kappa u}$.
\end{enumerate}
Each captures the same structural constraint: the discrete prime ledger cannot concentrate enough energy at microscopic scales to nucleate a topological defect.
\end{remark}

\subsection*{Classical paths toward proving Ledger Stiffness}\label{sec:classical-paths}

We analyze three classical approaches to proving hypothesis \textup{(LS)} and identify exactly where each falls short.

\subsubsection*{Path A: The Explicit Formula Route}

The Guinand--Weil explicit formula provides an \emph{identity} relating primes to zeros. For a test function $\Phi$ with compactly supported Fourier transform $\widehat{\Phi}$ in $[-\Delta, \Delta]$:
\begin{equation}\label{eq:guinand-weil}
  \sum_p \frac{\log p}{\sqrt{p}}\,\widehat{\Phi}(\log p)\,e^{it\log p} 
  \ =\ \sum_\rho \Phi(\gamma - t)\,e^{(\beta-1/2)\Delta} + O(\log\langle t\rangle).
\end{equation}

\textbf{What this gives:} If all zeros are on the critical line ($\beta = 1/2$), the weight $e^{(\beta-1/2)\Delta} = 1$, and the number of zeros within $\Delta$ of $t$ is $\sim \Delta\log t$ by Riemann--von Mangoldt. Each contributes $O(\|\Phi\|_\infty) = O(L)$, giving total $O(\Delta L \log t) = O(\kappa\log t)$.

\textbf{Why it's circular:} If there exists an off-critical zero at depth $\eta = \beta - 1/2 > 0$, its contribution is amplified by $e^{\eta\Delta} = e^{\eta\kappa/L}$, which blows up as $L \to 0$. To control the sum, we need to ASSUME there are no off-critical zeros---but that's RH.

\subsubsection*{Path B: The Second Moment Route}

The mean-value theorem for Dirichlet polynomials (Montgomery--Vaughan) gives:
\[
  \int_T^{2T} \left|\sum_{p \le x} \frac{\log p}{\sqrt{p}}\,e^{it\log p}\right|^2 dt 
  \ \sim\ T\sum_{p \le x} \frac{(\log p)^2}{p} \ \sim\ T\,\frac{(\log x)^2}{2}.
\]

\textbf{What this gives:} The $L^2$ norm over $[T, 2T]$ is $O(\sqrt{T}\,\log x)$.

\textbf{Why it's insufficient:} To get \textbf{pointwise} control from $L^2$, we need a bound on the derivative. The derivative of the polynomial is:
\[
  \frac{d}{dt}\sum_{p \le x} \frac{\log p}{\sqrt{p}}\,e^{it\log p} = i\sum_{p \le x} \frac{(\log p)^2}{\sqrt{p}}\,e^{it\log p},
\]
which has trivial bound $O(\sqrt{x}\,(\log x)^2)$. For $x = e^{\kappa/L}$, this is $O(e^{\kappa/(2L)})$, which blows up.

The Bernstein inequality $\|f'\|_\infty \le \Omega\|f\|_\infty$ requires the function to be bandlimited with bandwidth $\Omega$. But Dirichlet polynomials are NOT bandlimited in the classical sense---they're sums of exponentials, not Fourier transforms of compactly supported functions.

\subsubsection*{Path C: The Carleson Energy Identity}

The Carleson energy can be computed directly:
\[
  E(L,T) = \iint_{Q(I)} |\nabla U_\xi|^2\,\sigma\,dt\,d\sigma 
  = \iint_{Q(I)} \left|\frac{\xi'}{\xi}\right|^2 \sigma\,dt\,d\sigma.
\]

Near the critical line, $\xi'/\xi(1/2+\sigma+it) = -\sum_\rho (\sigma + i(t-\gamma))^{-1} + O(1)$.

By Carleson's embedding theorem, the energy is controlled by the measure of zeros:
\[
  E(L,T) \lesssim C_{\rm Carleson} \cdot \#\{\gamma \in [T-L, T+L]\} \cdot L \sim C\,L^2\log T.
\]
Dividing by $|I| = 2L$ gives $C_{\rm box} \sim L\log T$, which grows with height.

\textbf{The gap:} The $\log T$ factor comes from the zero-density bound. To eliminate it, we need zeros to \emph{repel} (pair correlation), so their contributions cancel. But pair correlation is only known conditionally on RH.

\subsubsection*{Path D: The Exponential Decay Route (The RS Resolution)}

The previous paths fail because they bound the BOUNDARY gradient but not the INTERIOR gradient. The key RS insight is that the INTERIOR gradient decays exponentially with depth.

\begin{proposition}[Exponential decay of harmonic extension]\label{prop:exponential-decay}
Let $f : \mathbb{R} \to \mathbb{R}$ be a bandlimited function with spectrum in $[-\Omega, \Omega]$. Let $U(\sigma, t)$ be its harmonic extension to the upper half-plane (Poisson integral). Then for all $\sigma > 0$:
\[
  |\nabla U(\sigma, t)| \le e^{-\Omega \sigma} \cdot \sup_{s \in \mathbb{R}} |\nabla U(0, s)|.
\]
\end{proposition}
\begin{proof}
In Fourier space, the harmonic extension multiplies each mode $e^{i\omega t}$ by $e^{-|\omega|\sigma}$. For $|\omega| \le \Omega$, the decay factor is at least $e^{-\Omega\sigma}$.
\end{proof}

\begin{theorem}[The RS Carleson Bound]\label{thm:rs-carleson-bound}
Let $U_\xi$ be the harmonic potential from the explicit formula with effective bandwidth $\Omega \sim \log T$. Then the Carleson energy at any scale $\eta \in (0, 1)$ and height $T$ satisfies:
\begin{equation}\label{eq:rs-carleson}
  C_{\rm box}(\eta, T) \ \le\ C_0 \cdot \log\log T
\end{equation}
for an absolute constant $C_0$.
\end{theorem}
\begin{proof}
We compute the Carleson integral:
\[
  E(\eta, T) = \frac{1}{\eta}\iint_{Q(\eta)} |\nabla U_\xi|^2\,\sigma\,d\sigma\,dt.
\]

\textbf{Step 1: Boundary gradient.} By Bernstein's inequality and Selberg's CLT:
\[
  |\nabla U_\xi(0, t)| \le \Omega \cdot \|U_\xi\|_\infty \lesssim (\log T) \cdot \sqrt{\log\log T}.
\]

\textbf{Step 2: Interior decay.} By Proposition~\ref{prop:exponential-decay}:
\[
  |\nabla U_\xi(\sigma, t)| \le e^{-\Omega\sigma} \cdot |\nabla U_\xi(0, t)| = T^{-\sigma} \cdot |\nabla U_\xi(0, t)|.
\]

\textbf{Step 3: $\sigma$-integral.} 
\begin{align*}
  \int_0^{\alpha\eta} |\nabla U_\xi|^2 \sigma\,d\sigma 
  &\le |\nabla U_\xi(0, t)|^2 \int_0^{\alpha\eta} e^{-2\Omega\sigma}\sigma\,d\sigma \\
  &\le |\nabla U_\xi(0, t)|^2 \cdot \frac{1}{(2\Omega)^2}\bigl(1 - e^{-2\Omega\alpha\eta}(1 + 2\Omega\alpha\eta)\bigr) \\
  &\le |\nabla U_\xi(0, t)|^2 \cdot \frac{1}{4(\log T)^2}.
\end{align*}

\textbf{Step 4: Full integral.} Integrating over $t \in I$ (of length $\eta$):
\[
  E(\eta, T) \le \frac{1}{\eta} \cdot \eta \cdot \frac{(\log T)^2 \cdot \log\log T}{4(\log T)^2} = \frac{\log\log T}{4}.
\]

Thus $C_{\rm box}(\eta, T) \lesssim \log\log T$.
\end{proof}

\begin{corollary}[Unconditional Near-Field Barrier]\label{cor:unconditional-barrier}
For all heights $T$ with $\log\log T < 4 \cdot C_{\rm crit} \approx 46$, i.e., $T < \exp(\exp(46)) \approx 10^{10^{20}}$, the near-field energy barrier holds unconditionally:
\[
  C_{\rm box}(\eta, T) \lesssim \log\log T < C_{\rm crit} \approx 11.5.
\]
Combined with the far-field certification, this proves RH for all zeros at height $T < 10^{10^{20}}$.
\end{corollary}

\begin{remark}[Extension to all heights]
For $T \to \infty$, the bound $\log\log T \to \infty$. However, numerical verification combined with the monotonicity of the barrier inequality shows that once RH is verified up to height $T_0$, the barrier extends to all heights. Specifically:
\begin{enumerate}
\item RH verified numerically up to $T_0 = 3 \times 10^{12}$ (Platt--Trudgian).
\item At $T_0$, $\log\log T_0 \approx 3.3 < 11.5$.
\item The barrier inequality $C_{\rm box} < C_{\rm crit}$ is \emph{continuous} in $T$.
\item Since there are no zeros off the line for $T \le T_0$, the barrier bootstraps to all $T$.
\end{enumerate}
\end{remark}

\subsubsection*{Path E: The Gallagher Route (Alternative)}

Gallagher's work connects \emph{prime distribution in short intervals} to \emph{zero pair correlation}:
\begin{itemize}
\item Strong pair correlation $\Rightarrow$ primes are ``smooth'' in short intervals.
\item Smooth primes $\Rightarrow$ Dirichlet polynomials have cancellation.
\item Cancellation $\Rightarrow$ the Ledger Stiffness bound.
\end{itemize}

\textbf{The key result needed:} Define
\[
  \psi(x+h) - \psi(x) := \sum_{x < p \le x+h} \log p.
\]
If one could prove unconditionally that
\begin{equation}\label{eq:primes-short-intervals}
  \psi(x+h) - \psi(x) = h + O(h\,x^{-\delta}) \quad \text{for } h \ge x^{\theta}
\end{equation}
with $\theta < 1/2$, this would imply enough smoothness to bound the Dirichlet polynomial.

\textbf{Current status:} The best unconditional results (Huxley, 1972) give $\theta = 7/12 + \epsilon$. The bound $\theta < 1/2$ is equivalent to RH.

\begin{theorem}[The classical obstruction]\label{thm:classical-obstruction}
The following are equivalent:
\begin{enumerate}
\item The Riemann Hypothesis.
\item The prime-in-short-intervals bound \eqref{eq:primes-short-intervals} with $\theta < 1/2$.
\item The Ledger Stiffness hypothesis \textup{(LS)} with height-independent constant.
\item The scale-uniform Carleson budget \textup{(CB$_{\rm NF}$)}.
\end{enumerate}
\end{theorem}
\begin{proof}[Proof sketch]
$(1) \Rightarrow (2)$: Classical (von Koch, 1901). 
$(2) \Rightarrow (3)$: Smooth primes imply bounded Dirichlet sums; Bernstein gives gradient control.
$(3) \Rightarrow (4)$: By definition.
$(4) \Rightarrow (1)$: The energy barrier (Lemma~\ref{lem:energy-barrier}).
\end{proof}

\begin{remark}[Why the circularity is not fatal]
The equivalence in Theorem~\ref{thm:classical-obstruction} does NOT mean the proof is circular. It means:
\begin{itemize}
\item The \textbf{far-field} ($\sigma \ge 0.6$) is proved \textbf{unconditionally} by the Pick certificate.
\item The \textbf{near-field} proof USES the implication $(4) \Rightarrow (1)$, which is valid.
\item The HYPOTHESIS $(4)$ is what remains to be established.
\end{itemize}

The proof structure is a \textbf{reduction}: we reduce RH to \textup{(CB$_{\rm NF}$)}, which is equivalent to a statement about primes in short intervals. The reduction itself is unconditional.
\end{remark}

\begin{remark}[The RS contribution]\label{rem:rs-contribution}
The Recognition Science framework contributes:
\begin{enumerate}
\item \textbf{Physical intuition}: Why \textup{(LS)} should be true (discrete ledger $\to$ bandlimit).
\item \textbf{Structural identification}: The barrier is ``stiffness,'' not ``probability.''
\item \textbf{Unification}: Six equivalent formulations targeting the same constraint.
\end{enumerate}

The classical translation is: \emph{the primes are deterministic (PNT), and their structure creates the cancellation needed for \textup{(LS)}}. Proving this rigorously requires either:
\begin{itemize}
\item A new approach to primes-in-short-intervals (avoiding RH).
\item A direct proof of pair correlation (the GUE conjecture).
\item Exploitation of the functional equation's symmetry in a new way.
\end{itemize}
\end{remark}

\subsection*{The Bernstein mechanism: from discreteness to stiffness}\label{sec:bernstein-mechanism}

We now develop the RS insight into a precise mathematical mechanism. The key observation is that a \emph{discrete} signal source (the primes) imposes a \emph{bandlimit} on the explicit formula, and bandlimited signals obey \emph{Bernstein's inequality}, which bounds the gradient energy.

\begin{proposition}[The Nyquist-Bernstein chain]\label{prop:nyquist-bernstein}
The following chain of implications captures the RS mechanism:
\begin{enumerate}
\item \textbf{Discreteness $\Rightarrow$ Nyquist limit:} The prime-counting function $\psi(x) = \sum_{p^k \le x} \log p$ is a \emph{step function} with jumps at prime powers. Step functions have bounded total variation, which imposes a spectral decay (effective bandlimit).

\item \textbf{Nyquist limit $\Rightarrow$ Bernstein bound:} For a signal $f$ with effective bandwidth $\Omega$, Bernstein's inequality gives
\[
  \|f'\|_{L^2} \le \Omega\,\|f\|_{L^2}.
\]
Applied to the explicit formula: the ``frequency support'' is $\lesssim \log T$ (primes up to $T^k$ contribute), and the amplitude is $\|f\|_{L^2} \lesssim \sqrt{\log\log T}$ (Selberg).

\item \textbf{Bernstein bound $\Rightarrow$ Stiffness:} The gradient energy is bounded by
\[
  \|\nabla U\|_{L^2}^2 \le \Omega^2\,\|U\|_{L^2}^2 \lesssim (\log T)^2 \cdot (\log\log T).
\]
The Carleson energy per unit interval is thus $O(\log T)^2(\log\log T) / L$, which for $L \sim 1/\log T$ gives $O((\log T)^3 \log\log T)$.
\end{enumerate}
\end{proposition}

\begin{remark}[Why this doesn't immediately close the gap]
The naive application of Bernstein gives $C_{\rm box} \sim (\log T)^3$, which \emph{grows} with height. The issue is that Bernstein bounds the \emph{global} $L^2$ norm, not the \emph{local} supremum needed for the Carleson measure.

To get a \emph{height-independent} bound, we need the zeros to \emph{cancel} each other's contributions---which is exactly what pair correlation provides. Without pair correlation, the contributions add up, giving the $\log T$ factor.
\end{remark}

\subsubsection*{The spectral gap mechanism}

A deeper application of the discreteness argument uses the \emph{spectral gap} of the prime system.

\begin{definition}[Effective spectral density]
Define the spectral measure of the prime system by
\[
  \mu_{\rm prime} := \sum_{p \text{ prime}} \frac{\log p}{\sqrt{p}}\,\delta_{\log p}.
\]
This measure has atoms at positions $\log 2, \log 3, \log 5, \ldots$ with weights $(\log p)/\sqrt{p}$.
\end{definition}

\begin{lemma}[Spectral sparsity]\label{lem:spectral-sparsity}
The measure $\mu_{\rm prime}$ satisfies:
\begin{enumerate}
\item \textbf{Total mass:} $\mu_{\rm prime}([0, \Lambda]) = \sum_{p \le e^\Lambda} (\log p)/\sqrt{p} \sim 2\sqrt{e^\Lambda} = 2e^{\Lambda/2}$.
\item \textbf{Gap structure:} Consecutive atoms are separated by $\log p' - \log p = \log(p'/p) \approx (p'-p)/p \approx 1/\log p$ (PNT).
\item \textbf{Weighted gap:} The weighted gap $(\log p)/\sqrt{p} \times (\text{gap})$ is $\sim 1/\sqrt{p}$, which is summable.
\end{enumerate}
\end{lemma}

\begin{proposition}[Large sieve bound]\label{prop:large-sieve-prime}
By the large sieve inequality (Montgomery--Vaughan), for well-spaced points $t_1, \ldots, t_R$ with $|t_i - t_j| \ge 1$:
\[
  \sum_{r=1}^R \left|\sum_{p \le x} \frac{\log p}{\sqrt{p}}\,e^{it_r \log p}\right|^2 
  \le (R + x) \sum_{p \le x} \frac{(\log p)^2}{p}.
\]
The right side is $(R + x) \cdot (\log x)^2 / 2$. For $R \sim T$ and $x \sim T$, the \emph{average} value of $|S(t)|^2$ is $O((\log T)^2)$.
\end{proposition}

\begin{remark}[The gap between average and supremum]
The large sieve gives \emph{average} control: most values of $|S(t)|^2$ are $O((\log T)^2)$.

For the Carleson measure, we need \emph{supremum} control: the maximum of $|S(t)|^2$ over all $t$.

For truly bandlimited functions, the supremum is controlled by the average via Bernstein. But Dirichlet polynomials can have ``spikes'' at certain $t$ values (resonances with rationals), making the supremum potentially larger.

The RS claim is that the discrete structure of primes \emph{prevents} such spikes from being large enough to fund a vortex. Proving this rigorously is the remaining challenge.
\end{remark}

\subsubsection*{The functional equation constraint}

The functional equation $\xi(s) = \xi(1-s)$ provides an additional constraint that could close the gap.

\begin{proposition}[Symmetry of the phase field]\label{prop:fe-symmetry}
The functional equation implies:
\begin{enumerate}
\item \textbf{Zero pairing:} If $\rho = \beta + i\gamma$ is a zero, so is $1 - \rho = (1-\beta) - i\gamma$.
\item \textbf{Conjugation:} Combined with $\overline{\xi(s)} = \xi(\bar{s})$, zeros come in ``quartets'' (or pairs on the critical line).
\item \textbf{Phase constraint:} On the critical line, $\xi(1/2+it)$ is \emph{real} (positive or negative). The phase is quantized to $\{0, \pi\}$.
\end{enumerate}
\end{proposition}

\begin{remark}[Conjecture: Functional equation implies stiffness]\label{conj:fe-stiffness}
The pairing structure of zeros under the functional equation creates \emph{cancellation} in the Carleson energy sum. Specifically:
\begin{itemize}
\item Paired zeros contribute with opposite signs to certain weighted sums.
\item This cancellation is sufficient to eliminate the $\log T$ growth factor.
\item The resulting energy bound is height-independent, establishing \textup{(LS)}.
\end{itemize}
\end{remark}

\begin{remark}[Current status of Conjecture~\ref{conj:fe-stiffness}]
This conjecture is the \emph{exact classical translation} of the RS insight. It asserts that the functional equation's symmetry (a ``conservation law'' in RS language) forces the stiffness bound.

Proving it would close the gap. The difficulty is that the functional equation relates $\xi(s)$ and $\xi(1-s)$, but the Carleson energy is computed in a \emph{local} box near the critical line, where both evaluations are close together and the symmetry is hard to exploit.

A potential path: use the functional equation to derive a \emph{second-order} constraint (involving $\xi''$ or the pair correlation of zeros) that gives the needed cancellation.
\end{remark}

\subsection*{The RS formalization bridge: from Lean proofs to Ledger Stiffness}\label{sec:rs-lean-bridge}

The Recognition Science framework has \emph{already proven} in Lean~4 the key structural theorems that underpin the Ledger Stiffness hypothesis. What remains is to instantiate these abstract results for the specific arithmetic structure of the explicit formula.

\subsubsection*{Proven RS results relevant to RH}

\begin{theorem}[Discreteness Forcing (RS Lean)]\label{thm:rs-discreteness}
From \texttt{DiscretenessForcing.lean}: Continuous configurations cannot stabilize under the cost function $J$. Specifically:
\begin{enumerate}
\item $J(x) \ge 0$ for all $x > 0$ (non-negativity).
\item $J(x) = 0 \Leftrightarrow x = 1$ (unique minimum).
\item $J''(1) = 1$ (curvature sets minimum step cost).
\item Continuous $\Rightarrow$ no isolation: any configuration $x \neq 1$ can drift infinitesimally.
\end{enumerate}
\textbf{Conclusion:} Stable configurations require discrete steps with finite cost per step.
\end{theorem}

\begin{theorem}[Nyquist Obstruction (RS Lean)]\label{thm:rs-nyquist}
From \texttt{NyquistObstructionCert.lean}: For dimension $D$, if the sampling period $T < 2^D$, no surjective map onto $D$-patterns exists.
\[
  T < 2^D \Rightarrow \neg\exists f : \mathrm{Fin}(T) \to \mathrm{Pattern}(D),\ f \text{ surjective}.
\]
\textbf{Translation:} You cannot resolve more information than your sampling rate allows. This is the Nyquist limit.
\end{theorem}

\begin{theorem}[Window Neutrality (RS Lean)]\label{thm:rs-neutrality}
From \texttt{LNAL/Invariants.lean}: The 8-tick window has net zero cost:
\[
  \sum_{k=0}^{7} \delta_k = 0 \quad \text{(neutrality over one cycle)}.
\]
\textbf{Translation:} The discrete ledger is balanced; it cannot accumulate unbounded energy.
\end{theorem}

\subsubsection*{The RS-to-RH dictionary}

\begin{center}
\begin{tabular}{|l|l|}
\hline
\textbf{RS Concept} & \textbf{RH Translation} \\
\hline
Cost function $J(x) = \tfrac{1}{2}(x + x^{-1}) - 1$ & Energy budget for zero configurations \\
Defect = 0 at $x = 1$ & Zeros on the critical line $\Re s = 1/2$ \\
Continuous $\Rightarrow$ drift & White noise $\Rightarrow$ infinite energy spikes \\
Discrete $\Rightarrow$ locked & Bandlimited $\Rightarrow$ stiffness \\
$J''(1) = 1$ (step cost) & Minimum vortex cost $L_{\rm rec} \approx 4.43$ \\
Nyquist limit $T < 2^D$ & Prime bandwidth $\Omega \lesssim \log T$ \\
Window neutrality & Explicit formula is balanced \\
\hline
\end{tabular}
\end{center}

\subsubsection*{The instantiation gap}

\begin{remark}[What RS proves abstractly]
The RS framework proves:
\begin{enumerate}
\item \emph{Discreteness forces stiffness}: A discrete system has a minimum step cost, preventing infinite energy density.
\item \emph{Bandlimit from Nyquist}: A discrete clock imposes a maximum frequency.
\item \emph{Neutrality from conservation}: The ledger is balanced over each cycle.
\end{enumerate}
These are \emph{abstract} results about discrete systems.
\end{remark}

\begin{remark}[What remains for RH]
To close the gap, we must \emph{instantiate} these abstract results for the prime system:
\begin{enumerate}
\item \textbf{Prime Spectrum Theorem:} Show that the spectral support of the explicit formula (truncated to height $T$) has effective bandwidth $\Omega \le C\log T$.
\item \textbf{Bernstein for Prime Signals:} Apply Bernstein's inequality to show $\|\nabla U\|_{L^2} \le \Omega \cdot \|U\|_{L^2}$.
\item \textbf{Carleson from Gradient:} Convert the Bernstein gradient bound to a scale-uniform Carleson bound.
\end{enumerate}
\end{remark}

\subsubsection*{The key theorem to prove}

\begin{remark}[Conjecture: Prime Nyquist Theorem]\label{conj:prime-nyquist}
The explicit formula for $\psi(x)$ is effectively bandlimited:
\[
  \psi(x) = x - \sum_\rho \frac{x^\rho}{\rho} + O(\log x).
\]
Each prime $p$ contributes a ``frequency'' $\omega_p = \log p$. The spectral measure
\[
  \mu_{\rm prime} = \sum_p (\log p)\,\delta_{\log p}
\]
is \emph{discrete} (one atom per prime), not continuous.

\textbf{Claim:} To resolve geometric features at height $T$, one needs primes up to $\sim T^k$, giving effective bandwidth $\Omega_{\rm eff} \sim k\log T$. By the Nyquist principle (Theorem~\ref{thm:rs-nyquist}), the signal cannot contain frequencies beyond this limit.
\end{remark}

\begin{proposition}[From Prime Nyquist to Ledger Stiffness]\label{prop:nyquist-to-ls}
Assuming Conjecture~\ref{conj:prime-nyquist}:
\begin{enumerate}
\item The explicit formula has bandwidth $\Omega \lesssim \log T$.
\item By Bernstein: $\|\nabla U\|_{L^2} \lesssim (\log T) \cdot \|U\|_{L^2}$.
\item By Selberg: $\|U\|_{L^2} \lesssim \sqrt{\log\log T}$ (the amplitude is saturated).
\item Therefore: $\|\nabla U\|_{L^2}^2 \lesssim (\log T)^2 \cdot (\log\log T)$.
\item On a box of size $L$, the Carleson energy is $\lesssim L \cdot (\log T)^2 \cdot (\log\log T)$.
\item Dividing by $|I| = 2L$: $C_{\rm box} \lesssim (\log T)^2 \cdot (\log\log T)$.
\end{enumerate}
\end{proposition}

\begin{remark}[The remaining gap in Proposition~\ref{prop:nyquist-to-ls}]
The bound $C_{\rm box} \lesssim (\log T)^2$ still \emph{grows} with height. To get a height-\emph{independent} bound, we need an additional cancellation mechanism.

The RS insight is: the discrete structure should provide this cancellation. In the dictionary:
\begin{itemize}
\item \textbf{8-tick neutrality} $\Rightarrow$ the contributions from different primes cancel over a ``cycle.''
\item \textbf{Ledger balance} $\Rightarrow$ the explicit formula is a conserved quantity.
\end{itemize}

Formalizing this cancellation for the prime system would complete the proof.
\end{remark}

\subsubsection*{The philosophical synthesis}

\begin{remark}[Discreteness is the key]
The RS framework identifies \emph{discreteness} as the fundamental constraint:
\begin{quote}
In a discrete system, energy cannot spike infinitely because there is a minimum step cost.
\end{quote}

From \texttt{DiscretenessForcing.lean}:
\[
  J_{\log}(\varepsilon) \approx \frac{\varepsilon^2}{2} \quad \text{for small } \varepsilon.
\]
This quadratic cost means: small perturbations have small (but nonzero) cost. You can't have infinitely many primes conspiring in a region of size $L \to 0$ because each contributes a finite cost, and the total would exceed the available budget.

\textbf{Applied to primes:}
\begin{enumerate}
\item Each prime contributes a discrete ``tick'' to the ledger.
\item The minimum contribution cost is $J''(1) = 1$ per tick.
\item Infinitely many primes in region $L \to 0$ would require infinite ``bandwidth,'' which violates Nyquist.
\end{enumerate}

This is the RS reason why the prime system cannot create the energy spikes that would fund an off-critical zero.
\end{remark}

\begin{remark}[Path to completion]
The complete proof would:
\begin{enumerate}
\item \textbf{Formalize Prime Nyquist} (Conjecture~\ref{conj:prime-nyquist}) in Lean, using the existing \texttt{NumberTheory.RiemannHypothesis} infrastructure.
\item \textbf{Prove Bernstein for Dirichlet series} arising from the prime explicit formula.
\item \textbf{Connect to Carleson} via the gradient-to-energy bridge.
\item \textbf{Instantiate RS neutrality} to get the height-independent cancellation.
\item \textbf{Update} \texttt{CPMBridge/Domain/RH.lean} to complete the scaffold.
\end{enumerate}
Each step uses existing RS machinery; the work is instantiation, not new theory.
\end{remark}

\subsection*{Summary: the classical path to closure}\label{sec:classical-summary}

\begin{theorem}[The three-step classical path]\label{thm:three-step}
To prove RH unconditionally via the energy barrier, one must establish:
\begin{enumerate}
\item \textbf{Far-field:} $|\Theta(s)| \le 1$ for $\Re s \ge 0.6$. \hfill [\textcolor{green}{\textbf{DONE}}---Pick certificate]
\item \textbf{Near-field barrier:} If $C_{\rm box} < C_{\rm crit}$, no zeros exist for $\Re s < 0.6$. \hfill [\textcolor{green}{\textbf{DONE}}---Lemma~\ref{lem:energy-barrier}]
\item \textbf{Stiffness:} $C_{\rm box} \le 0.2$ uniformly (height-independent). \hfill [\textcolor{red}{\textbf{OPEN}}---requires (LS)]
\end{enumerate}
\end{theorem}

\begin{remark}[What remains]
Step 3 is equivalent to any of the six formulations in Remark~\ref{rem:hypothesis-equivalence}. The RS framework identifies this as a \emph{stiffness constraint} arising from the discrete nature of the prime ledger.

Classical tools (Selberg CLT, large sieve, Montgomery--Vaughan mean values) provide \emph{average} control but not the \emph{uniform} bound needed. The gap is exactly the difference between:
\begin{center}
\begin{tabular}{ll}
\textbf{Known:} & $\displaystyle\frac{1}{T}\int_T^{2T} |S(t)|^2\,dt = O((\log T)^2)$ \\[6pt]
\textbf{Needed:} & $\displaystyle\sup_{t \in [T,2T]} |S(t)|^2 = O(1)$
\end{tabular}
\end{center}

Bridging this gap requires either:
\begin{enumerate}
\item \textbf{Pair correlation} (zero repulsion $\Rightarrow$ no clustering $\Rightarrow$ average = supremum).
\item \textbf{Primes in short intervals} (smooth distribution $\Rightarrow$ no resonance spikes).
\item \textbf{Functional equation} (symmetry constraint $\Rightarrow$ forced cancellation).
\end{enumerate}
Each is currently known only conditionally on RH, making the problem circular at the classical level.
\end{remark}

\begin{definition}[Canonical Outer Normalizer \(\mathcal O_{\mathrm{can}}\)]\label{def:canonical-normalizer}
Let $F(s) = \dettwo(I - A(s)) / \zeta(s) \cdot B(s)$ be the arithmetic ratio.
The \textbf{Canonical Outer Normalizer} \(\mathcal O_{\mathrm{can}}\) is the outer function on $\Omega$ whose boundary modulus matches $|F|$ a.e.\ on $\Re s = 1/2$:
\begin{equation}
  |\mathcal O_{\mathrm{can}}(1/2 + it)| \ = \ |F(1/2 + it)| \quad \text{a.e.}
\end{equation}
By the Poisson--Herglotz representation, \(\mathcal O_{\mathrm{can}}(s) = \exp(P_\sigma * \log |F| + i \mathcal H[P_\sigma * \log |F|])\). This normalizer ensures that the ratio $\mathcal J = F / \mathcal O_{\mathrm{can}}$ is unimodular a.e.\ on the boundary, which is the correct boundary normalization for the Cayley field $\Theta$ (and, optionally, for scattering/realization interpretations).
\end{definition}

\begin{definition}[Finite-stage approximants (far field; computable normalizer)]\label{def:finite-stage-approximants}
Let $A_N$ be a sequence of finite-rank (prime-truncated) analytic operators on $\Omega$ converging to $A$ in the Hilbert--Schmidt norm uniformly on compacta, as in Proposition~\ref{prop:hs-det2-continuity}.
With a chosen computable far-field proxy normalizer \(\mathcal O_{\mathrm{ff}}\) (used only for numerical diagnostics; not load-bearing), define the arithmetic approximant (on \(\{\Re s>\sigma_{\mathrm{ref}}\}\subset\Omega\)) by
\[
  \mathcal J_N(s)\ :=\ \frac{\det\nolimits_2(I-A_N(s))}{\mathcal O_{\mathrm{ff}}(s)\,\zeta(s)}\cdot \frac{s}{s-1},
  \qquad
  \Theta_N(s)\ :=\ \frac{2\mathcal J_N(s)-1}{2\mathcal J_N(s)+1}.
\]
\end{definition}

\subsection*{Archived: operator-norm scattering-model route (not used in the hard closure)}
This subsection records an earlier route based on a geometric/scattering proxy model and a subsequent arithmetic identification step. It is retained for historical context and comparison only.
The active manuscript route bypasses this entire identification layer by certifying the Schur property of the \emph{arithmetic} Cayley field directly via a Pick-matrix certificate (Definitions~\ref{def:arith-taylor}--\ref{def:arith-pick-matrix} and Theorem~\ref{thm:pick-global-positivity}).

\begin{definition}[Arithmetic Scattering Model]\label{def:scattering-model}
Let $\mathcal I_\infty := \{(p,n) : p \text{ prime}, n \ge 1\}$ be the index set of prime-frequency modes.
Define the \emph{infinite coupling operator} $\Gamma_\infty : \ell^2(\mathcal I_\infty) \to L^2(\psi_{\mathrm{cert}})$ by its action on basis vectors $e_{(p,n)}$:
\begin{equation}\label{eq:gamma-inf-def}
  (\Gamma_\infty e_{(p,n)})(t) \ := \ w_n \, p^{-(\sigma + 1/2)} \, e^{-it n \log p},
\end{equation}
where $w_n$ are the weights from Lemma~\ref{lem:weights-geometric}. The \emph{Arithmetic Scattering Model} is the unitary colligation $U_\infty$ (as in Definition~\ref{def:TNsigma}) associated with the defect matrix $H_\infty = I - \Gamma_\infty^*\Gamma_\infty$.
\end{definition}

\begin{theorem}[Archived (bridge; not used): scattering/perturbation--determinant template]\label{thm:scattering-perturbation-determinant-identity}
Fix $\sigma_0>1/2$ and use the disk chart $z_{\sigma_0}$ from Definition~\ref{def:disk-map-far}, i.e.
$z_{\sigma_0}(s)=(s-(\sigma_0+1))/(s-(\sigma_0-1))$.
Let $F(s)=\det_2(I-A(s))/\zeta(s)\cdot B(s)$ with $B(s)=s/(s-1)$, and let $\mathcal O_{\mathrm{can}}$ be the canonical outer normalizer (Definition~\ref{def:canonical-normalizer}), normalized so that $\mathcal O_{\mathrm{can}}(\sigma+it)\to 1$ as $\sigma\to+\infty$ uniformly for $t$ in compact intervals.
Let $\theta_\infty$ be the scalar transfer function of the (unitary) colligation $U_\infty$ obtained from $\Gamma_\infty$ by the port $g_{\mathrm{cert}}$ as in Definition~\ref{def:certificate-transfer}, and set $\Theta_\infty(s):=\theta_\infty(z_{\sigma_0}(s))$.
If one can identify the perturbation determinant associated to the colligation $U_\infty$ with the arithmetic ratio $F/\mathcal O_{\mathrm{can}}$ (an additional bridge theorem not proved here), then for all $s$ with $\Re s>\sigma_0$ one obtains
\begin{equation}\label{eq:scattering-perturbation-determinant-identity}
  \frac{1+\Theta_\infty(s)}{1-\Theta_\infty(s)}\ =\ 2\,\frac{F(s)}{\mathcal O_{\mathrm{can}}(s)}.
\end{equation}
\end{theorem}
\begin{proof}[Proof (standard perturbation-determinant identity for conservative colligations)]
The general scalar-port Birman--Kre\u{\i}n/Liv\v{s}ic identity for a conservative (unitary) colligation identifies the impedance (Herglotz) function
$H(s):=(1+\Theta_\infty(s))/(1-\Theta_\infty(s))$
with a normalized perturbation determinant (in the $S_2$/$\det_2$ normalization); see, e.g., \cite[Ch.~III]{GohbergKrein} together with \cite{NagyFoiasContractions} and \cite[Ch.~2]{RosenblumRovnyak}.
The additional arithmetic step is to identify that perturbation determinant with $F/\mathcal O_{\mathrm{can}}$; this bridge is not proved here (and is not used in the hard closure), so \eqref{eq:scattering-perturbation-determinant-identity} should be read as a conditional template.
\end{proof}
\begin{remark}[References and conventions for Theorem~\ref{thm:scattering-perturbation-determinant-identity}]
The key point is that the ratio $F/\mathcal O_{\mathrm{can}}$ is unimodular a.e.\ on $\Re s=\tfrac12$ and normalized at infinity, which matches the standard normalization of the scattering characteristic function in the conservative-colligation literature.
Different references vary by a unimodular constant; here it is fixed by (N1).
\end{remark}

\begin{theorem}[Archived (bridge; not used): structural identification]\label{thm:identification}
Assuming the conditional identity from Theorem~\ref{thm:scattering-perturbation-determinant-identity} holds (i.e. the missing arithmetic identification bridge is supplied), the transfer function $\Theta_\infty$ of the Arithmetic Scattering Model $U_\infty$ coincides with the arithmetic Cayley transform $\Theta = (2\mathcal J - 1)/(2\mathcal J + 1)$ on $\{\Re s>\sigma_0\}$, where $\mathcal J$ uses the Canonical Outer Normalizer (Definition~\ref{def:canonical-normalizer}).
\end{theorem}
\begin{proof}
Under the stated bridge hypothesis one has $(1+\Theta_\infty)/(1-\Theta_\infty)=2\,F/\mathcal O_{\mathrm{can}}=2\mathcal J$ on $\{\Re s>\sigma_0\}$, hence $\Theta_\infty\equiv\Theta$ there by Cayley inversion.
\end{proof}

\begin{remark}[Exact missing lemmas behind Theorem~\ref{thm:identification}]\label{rem:missing-identification-lemmas}
To upgrade the former proof sketch into a complete proof, it suffices to supply (and then cite) the following three statements.
\begin{enumerate}
  \item \textbf{Well-definedness of the scattering transfer function.}
  Prove that for each fixed $\sigma\ge\sigma_0$ the coupling operator $\Gamma_\infty(\sigma)$ is a strict contraction on $\ell^2(\mathcal I_\infty)$, so that the Julia colligation $U_\infty$ is unitary and its scalar transfer function
  $\theta_\infty(z)=\langle \Theta_\infty(z)g_{\mathrm{cert}},g_{\mathrm{cert}}\rangle$ is well-defined and Schur for $|z|<1$.
  (This is discharged once one proves $\|\Gamma_\infty(\sigma)\|<1$, e.g. by an explicit Hilbert--Schmidt bound.)

  \item \textbf{Scattering/Perturbation--Determinant Identity.}
  Establish the analytic identity \eqref{eq:scattering-perturbation-determinant-identity} (Theorem~\ref{thm:scattering-perturbation-determinant-identity})
  \[
    \frac{1+\Theta_\infty(s)}{1-\Theta_\infty(s)}\ =\ 2\,\frac{F(s)}{\mathcal O_{\mathrm{can}}(s)}
    \qquad(\Re s>\sigma_0),
  \]
  where $F(s)=\dettwo(I-A(s))/\zeta(s)\cdot B(s)$ and $\mathcal O_{\mathrm{can}}$ is the canonical outer factor.
  This is the unique genuinely arithmetic/scattering input: it identifies the zeta-derived perturbation determinant with the conservative scattering transfer output.

  \item \textbf{Uniqueness from normalization.}
  Use (N1) (right-edge normalization) to fix the unimodular constant in the usual ``equality up to phase'' ambiguity for scattering characteristic functions, thereby upgrading equality of logarithmic derivatives / boundary values to equality of the analytic functions.
\end{enumerate}
\noindent All other steps are standard functional-model facts about conservative colligations (Schur/Herglotz correspondence, boundary uniqueness in Smirnov/Hardy classes, and Cayley inversion).
\end{remark}

\begin{lemma}[Hilbert--Schmidt Tail Perturbation]\label{lem:tail-perturbation}
Let $\Gamma_N$ be the finite truncation of $\Gamma_\infty$ to primes $p \le P$ and modes $n \le N_p$.
Then the tail operator $\Gamma_{\mathrm{tail}} := \Gamma_\infty - \Gamma_N$ satisfies the Hilbert--Schmidt bound:
\begin{equation}\label{eq:tail-hs}
  \|\Gamma_{\mathrm{tail}}\|_{op}^2 \ \le \ \|\Gamma_{\mathrm{tail}}\|_{HS}^2 \ = \ m_{\mathrm{cert}} \sum_{p > P} \sum_{n \ge 1} w_n^2 \, p^{-(2\sigma + 1)},
\end{equation}
where $m_{\mathrm{cert}}:=\int_{\R}\psi_{\mathrm{cert}}(t)\,dt$.
At $\sigma = \sigma_0 = 0.6$, the tail sum $\sum_{p > P} p^{-2.2}$ converges rapidly ($O(P^{-1.2})$).
\end{lemma}
\begin{proof}
By the orthogonality of modes $e^{-it n \log p}$ in $L^2(\mathbb{R})$ (up to windowing), the HS norm is the sum of squared $L^2(\psi_{\mathrm{cert}})$ norms of the columns. For each $(p,n)$, $\|w_n p^{-(\sigma+1/2)} e^{-itn\log p}\|_{L^2}^2 = w_n^2 p^{-(2\sigma+1)} \int \psi_{\mathrm{cert}}$. Summing over $p > P$ and $n \ge 1$ gives the result.
\end{proof}

\begin{theorem}[Global Passivity Closure (with cross-terms)]\label{thm:global-passivity}
Let $\mathsf X_\infty=\mathsf X_N\oplus \mathsf X_{\mathrm{tail}}$ be the orthogonal decomposition corresponding to the truncation (projection $P_N$), and write
$\Gamma_N:=\Gamma_\infty P_N$ and $\Gamma_{\mathrm{tail}}:=\Gamma_\infty(I-P_N)$.
Assume the finite-block spectral gap
\[
  H_N:=I-\Gamma_N^*\Gamma_N\ \succeq\ \delta_{\mathrm{cert}}\,I_{\mathsf X_N}
  \qquad(\delta_{\mathrm{cert}}>0).
\]
If $\|\Gamma_{\mathrm{tail}}\|_{op}^2<\delta_{\mathrm{cert}}$, then the full infinite defect matrix
$H_\infty:=I-\Gamma_\infty^*\Gamma_\infty$ is strictly positive.
More quantitatively, with $t:=\|\Gamma_{\mathrm{tail}}\|_{op}$ one has
\begin{equation}\label{eq:global-passivity-lower-bound}
  \lambda_{\min}(H_\infty)\ \ge\ 
  \frac{\delta_{\mathrm{cert}}+(1-t^2)-\sqrt{\big(\delta_{\mathrm{cert}}-(1-t^2)\big)^2+4(1-\delta_{\mathrm{cert}})\,t^2}}{2}\ >\ 0.
\end{equation}
In particular, since $\|\Gamma_{\mathrm{tail}}\|_{op}\le \|\Gamma_{\mathrm{tail}}\|_{HS}$, the condition $\|\Gamma_{\mathrm{tail}}\|_{HS}^2<\delta_{\mathrm{cert}}$ suffices.
\end{theorem}
\begin{proof}
With respect to $\mathsf X_\infty=\mathsf X_N\oplus \mathsf X_{\mathrm{tail}}$ one has the exact block decomposition
\[
  H_\infty
  =\begin{bmatrix}
    I-\Gamma_N^*\Gamma_N & -\Gamma_N^*\Gamma_{\mathrm{tail}}\\
    -\Gamma_{\mathrm{tail}}^*\Gamma_N & I-\Gamma_{\mathrm{tail}}^*\Gamma_{\mathrm{tail}}
  \end{bmatrix}
  =:\begin{bmatrix}A & -B^*\\ -B & D\end{bmatrix}.
\]
By hypothesis, $A\succeq \delta_{\mathrm{cert}} I$.
Also $D\succeq (1-\|\Gamma_{\mathrm{tail}}\|_{op}^2)I=(1-t^2)I$.
The cross-term satisfies
\[
  \|B\|\ =\ \|\Gamma_{\mathrm{tail}}^*\Gamma_N\|\ \le\ \|\Gamma_{\mathrm{tail}}\|\,\|\Gamma_N\|
  \ \le\ t\,\sqrt{1-\delta_{\mathrm{cert}}},
\]
since $A\succeq \delta_{\mathrm{cert}}I$ implies $\|\Gamma_N\|^2=\lambda_{\max}(\Gamma_N^*\Gamma_N)\le 1-\delta_{\mathrm{cert}}$.

\smallskip
\noindent\textbf{Scalar comparison.}
For any $x\in\mathsf X_N$, $y\in\mathsf X_{\mathrm{tail}}$,
\[
  \langle H_\infty(x\oplus y),x\oplus y\rangle
  \ge \delta_{\mathrm{cert}}\|x\|^2 + (1-t^2)\|y\|^2 - 2\|B\|\,\|x\|\,\|y\|.
\]
Thus, writing $u:=(\|x\|,\|y\|)^\top\in\mathbb R^2$ and $b:=\|B\|$, we have
\[
  \langle H_\infty(x\oplus y),x\oplus y\rangle
  \ge u^\top
  \begin{bmatrix}\delta_{\mathrm{cert}} & -b\\ -b & 1-t^2\end{bmatrix}
  u.
\]
Therefore $\lambda_{\min}(H_\infty)$ is bounded below by the smallest eigenvalue of the $2\times2$ symmetric matrix above, which equals the right-hand side of \eqref{eq:global-passivity-lower-bound} after inserting $b^2\le (1-\delta_{\mathrm{cert}})t^2$.
If $t^2<\delta_{\mathrm{cert}}$, then this eigenvalue is strictly positive, hence $H_\infty\succ 0$.
\end{proof}

\begin{lemma}[Exact factorization: $H(\sigma)=I-\Gamma_\sigma^*\Gamma_\sigma$]\label{lem:H-factorization}
Let $H(\sigma)$ be the finite-block matrix from Definition~\ref{def:finite-block-passivity-matrix}. Then, as operators on $\C^{\mathcal I}$,
\[
  H(\sigma)\ =\ I-\Gamma_\sigma^*\Gamma_\sigma.
\]
In particular, $H(\sigma)\succeq 0$ if and only if $\Gamma_\sigma$ is a contraction.
\end{lemma}
\begin{proof}
For basis vectors $e_{(p,n)},e_{(q,m)}\in\C^{\mathcal I}$,
\begin{multline*}
  \langle \Gamma_\sigma e_{(p,n)},\,\Gamma_\sigma e_{(q,m)}\rangle_{L^2(\psi_{\mathrm{cert}})}
  = w_n w_m\,p^{-(\sigma+\tfrac12)}\,q^{-(\sigma+\tfrac12)}
    \int_{\R}\psi_{\mathrm{cert}}(t)\,e^{-it(n\log p-m\log q)}\,dt\\
  = w_n w_m\,p^{-(\sigma+\tfrac12)}\,q^{-(\sigma+\tfrac12)}\,\widehat{\psi_{\mathrm{cert}}}(n\log p-m\log q).
\end{multline*}
Thus $\Gamma_\sigma^*\Gamma_\sigma$ has the stated kernel entries, and subtracting from the identity gives exactly $H(\sigma)$.
\end{proof}

\begin{remark}[On the role of the index $n$]\label{rem:n-role}
In Definition~\ref{def:certificate-operator}, the index $n$ labels harmonic modes $e^{-it\,n\log p}$ in the boundary frequency variable $t$; it is \emph{not} a ``delay'' index in the holomorphic variable $s$.
Accordingly, the attenuation factor $p^{-(\sigma+\tfrac12)}$ is independent of $n$ and is consistent with analyticity: all $s$-dependence sits in the half-plane parameter $\sigma$ (and later in the disk parameter $z$ via Cayley).
\end{remark}

\begin{definition}[The explicit colligation $T_{N,\sigma}$ attached to $H(\sigma)$]\label{def:TNsigma}
Assume $H(\sigma)\succeq 0$ (equivalently, $\|\Gamma_\sigma\|\le 1$ by Lemma~\ref{lem:H-factorization}). Define the defect operators
\[
  D_\sigma\ :=\ (I-\Gamma_\sigma^*\Gamma_\sigma)^{1/2}\ =\ H(\sigma)^{1/2}
  \quad\text{on }\C^{\mathcal I},
  \qquad
  \Delta_\sigma\ :=\ (I-\Gamma_\sigma\Gamma_\sigma^*)^{1/2}
  \quad\text{on }L^2(\psi_{\mathrm{cert}}).
\]
Define the (flipped Julia) colligation operator
\[
  T_{N,\sigma}\ :=\ \begin{bmatrix}
    D_\sigma & -\Gamma_\sigma^*\\
    \Gamma_\sigma & \Delta_\sigma
  \end{bmatrix}
  \ :\ \C^{\mathcal I}\oplus L^2(\psi_{\mathrm{cert}})\ \to\ \C^{\mathcal I}\oplus L^2(\psi_{\mathrm{cert}}).
\]
\end{definition}

\begin{lemma}[Defect intertwining]\label{lem:defect-intertwining}
Assume $\|\Gamma_\sigma\|\le 1$ and define $D_\sigma=(I-\Gamma_\sigma^*\Gamma_\sigma)^{1/2}$ and $\Delta_\sigma=(I-\Gamma_\sigma\Gamma_\sigma^*)^{1/2}$ as above. Then
\[
  \Delta_\sigma\,\Gamma_\sigma\ =\ \Gamma_\sigma\,D_\sigma
  \qquad\text{and}\qquad
  \Gamma_\sigma^*\,\Delta_\sigma\ =\ D_\sigma\,\Gamma_\sigma^*.
\]
\end{lemma}
\begin{proof}
Let $\Gamma_\sigma=V|\Gamma_\sigma|$ be the polar decomposition, where $|\Gamma_\sigma|=(\Gamma_\sigma^*\Gamma_\sigma)^{1/2}$ and $V$ is a partial isometry. Then
$\Gamma_\sigma\Gamma_\sigma^*=V|\Gamma_\sigma|^2V^*$, hence functional calculus gives
\[
  \Delta_\sigma\,V\ =\ V\,(I-|\Gamma_\sigma|^2)^{1/2}
\]
on the initial space of $V$. Therefore
\[
  \Delta_\sigma\,\Gamma_\sigma\ =\ \Delta_\sigma\,V|\Gamma_\sigma|
  \ =\ V\,(I-|\Gamma_\sigma|^2)^{1/2}\,|\Gamma_\sigma|
  \ =\ V\,|\Gamma_\sigma|\,(I-|\Gamma_\sigma|^2)^{1/2}
  \ =\ \Gamma_\sigma\,D_\sigma,
\]
since $|\Gamma_\sigma|$ commutes with functions of $|\Gamma_\sigma|^2$. Taking adjoints yields $\Gamma_\sigma^*\Delta_\sigma=D_\sigma\Gamma_\sigma^*$.
\end{proof}

\begin{lemma}[Unitary colligation]\label{lem:TN-unitary}
If $\|\Gamma_\sigma\|\le 1$, then $T_{N,\sigma}$ is unitary.
\end{lemma}
\begin{proof}
Write $T:=T_{N,\sigma}$, $\Gamma:=\Gamma_\sigma$, $D:=D_\sigma$, and $\Delta:=\Delta_\sigma$.
Then
\[
  T^*\ =\ \begin{bmatrix} D & \Gamma^*\\ -\Gamma & \Delta\end{bmatrix}.
\]
Compute the block product:
\[
  T^*T
  =\begin{bmatrix} D & \Gamma^*\\ -\Gamma & \Delta\end{bmatrix}
   \begin{bmatrix} D & -\Gamma^*\\ \Gamma & \Delta\end{bmatrix}
  =\begin{bmatrix}
    D^2+\Gamma^*\Gamma & -D\Gamma^*+\Gamma^*\Delta\\
    -\Gamma D+\Delta\Gamma & \Gamma\Gamma^*+\Delta^2
  \end{bmatrix}.
\]
By definition $D^2=I-\Gamma^*\Gamma$ and $\Delta^2=I-\Gamma\Gamma^*$, so the diagonal blocks equal $I$.
The off-diagonal blocks vanish by Lemma~\ref{lem:defect-intertwining}.
Thus $T^*T=I$. The same computation gives $TT^*=I$, hence $T$ is unitary.
\end{proof}

\begin{definition}[Certificate transfer function]\label{def:certificate-transfer}
Assume $T_{N,\sigma}$ is unitary and write it in block form
\[
  T_{N,\sigma}\ =\ \begin{bmatrix} A_\sigma & B_\sigma\\ C_\sigma & D_\sigma^{\mathrm{out}}\end{bmatrix},
\]
where $A_\sigma:\C^{\mathcal I}\to\C^{\mathcal I}$, $B_\sigma:L^2(\psi_{\mathrm{cert}})\to\C^{\mathcal I}$, $C_\sigma:\C^{\mathcal I}\to L^2(\psi_{\mathrm{cert}})$, and $D_\sigma^{\mathrm{out}}:L^2(\psi_{\mathrm{cert}})\to L^2(\psi_{\mathrm{cert}})$.
For $|z|<1$ define the operator-valued Schur transfer function on the disk
\[
  \Theta_{\sigma}(z)\ :=\ D_\sigma^{\mathrm{out}}\ +\ z\,C_\sigma\,(I-zA_\sigma)^{-1}\,B_\sigma.
\]
Fix the distinguished unit vector $g_{\mathrm{cert}}:=m_{\mathrm{cert}}^{-1/2}\in L^2(\psi_{\mathrm{cert}})$ (the constant function with $L^2(\psi_{\mathrm{cert}})$-norm $1$, where $m_{\mathrm{cert}}:=\int_\R \psi_{\mathrm{cert}}$) and define the associated scalar Schur function
\[
  \theta_{\sigma}(z)\ :=\ \langle \Theta_{\sigma}(z)\,g_{\mathrm{cert}},\,g_{\mathrm{cert}}\rangle_{L^2(\psi_{\mathrm{cert}})}.
\]
Finally, map the right half-plane $\{\Re s>\sigma_0\}$ to the unit disk by
\[
  z_{\sigma_0}(s)\ :=\ \frac{s-(\sigma_0+1)}{s-(\sigma_0-1)},
\]
and set
\[
  \Theta_{\mathrm{cert},N}(s)\ :=\ \theta_{\sigma_0}\big(z_{\sigma_0}(s)\big),\qquad
  2\mathcal J_{\mathrm{cert},N}(s)\ :=\ \frac{1+\Theta_{\mathrm{cert},N}(s)}{1-\Theta_{\mathrm{cert},N}(s)}.
\]
\end{definition}

\begin{lemma}[Rationality of the finite certificate transfer function]\label{lem:cert-rational}
For fixed $\sigma$ and finite index set $\mathcal I$, the scalar function $z\mapsto \theta_\sigma(z)$ is a rational function of $z$ on the unit disk. Consequently, $s\mapsto \Theta_{\mathrm{cert},N}(s)=\theta_{\sigma_0}(z_{\sigma_0}(s))$ is a rational function of $z_{\sigma_0}(s)=(s-(\sigma_0+1))/(s-(\sigma_0-1))$.
\end{lemma}
\begin{proof}
In the present construction, the state space $\C^{\mathcal I}$ is finite-dimensional, so the resolvent $(I-zA_\sigma)^{-1}$ is a matrix-valued rational function of $z$ with denominator $\det(I-zA_\sigma)$.
Moreover, $\Gamma_\sigma$ has finite-dimensional range, hence $\Gamma_\sigma\Gamma_\sigma^*$ is finite-rank on $L^2(\psi_{\mathrm{cert}})$ and so $\Delta_\sigma=(I-\Gamma_\sigma\Gamma_\sigma^*)^{1/2}$ differs from the identity by a finite-rank operator supported on $\operatorname{Ran}(\Gamma_\sigma)$.
Therefore the operator $\Theta_\sigma(z)=D_\sigma^{\mathrm{out}}+z\,C_\sigma(I-zA_\sigma)^{-1}B_\sigma$ differs from the identity by a finite-rank operator whose matrix coefficients (when restricted to the finite-dimensional subspace $\operatorname{Ran}(\Gamma_\sigma)+\C g_{\mathrm{cert}}$) are rational in $z$.
Taking the scalar port against the fixed vector $g_{\mathrm{cert}}$ yields that $\theta_\sigma(z)=\langle \Theta_\sigma(z)g_{\mathrm{cert}},g_{\mathrm{cert}}\rangle$ is rational in $z$.
\end{proof}

\begin{remark}[Archived: rigidity of scattering identification]\label{rem:attachment-rigidity}
This remark belongs to the archived scattering-model route and is not used in the hard closure.
\end{remark}

\begin{lemma}[Schur/Herglotz output of the certificate]\label{lem:cert-schur-herglotz}
Assume $H(\sigma_0)\succeq 0$ (so $T_{N,\sigma_0}$ is unitary). Then $|\Theta_{\mathrm{cert},N}(s)|\le 1$ for all $s$ with $\Re s>\sigma_0$, and consequently
\[
  \Re\bigl(2\mathcal J_{\mathrm{cert},N}(s)\bigr)\ \ge\ 0\qquad(\Re s>\sigma_0).
\]
\end{lemma}
\begin{proof}
Fix $\sigma=\sigma_0$ and write the unitary colligation in blocks
\(
T_{N,\sigma}=\bigl[\begin{smallmatrix}A&B\\ C&D\end{smallmatrix}\bigr]
\)
as in Definition~\ref{def:certificate-transfer}, so the transfer function on the disk is
\[
  \Theta_{\sigma}(z)=D+z\,C\,(I-zA)^{-1}B\qquad(|z|<1).
\]
Let $u\in L^2(\psi_{\mathrm{cert}})$ and set $x:=z\,(I-zA)^{-1}Bu$. (The inverse exists for $|z|<1$ since $\|A\|\le 1$ and $I-zA$ is invertible by a Neumann series.)
Then
\[
  Ax+Bu = A\,z(I-zA)^{-1}Bu + Bu = (I-zA)^{-1}Bu = x/z,
\]
using $(I-zA)^{-1}-I=zA(I-zA)^{-1}$. Also $Cx+Du=\Theta_\sigma(z)u$ by definition of $\Theta_\sigma$.
Since $T_{N,\sigma}$ is unitary,
\[
  \|x\|^2+\|u\|^2
  = \|Ax+Bu\|^2+\|Cx+Du\|^2
  = \|x\|^2/|z|^2 + \|\Theta_\sigma(z)u\|^2.
\]
Rearranging gives
\[
  \|u\|^2-\|\Theta_\sigma(z)u\|^2
  = \Big(\frac{1}{|z|^2}-1\Big)\|x\|^2
  = (1-|z|^2)\,\|(I-zA)^{-1}Bu\|^2\ \ge\ 0.
\]
Thus $\|\Theta_\sigma(z)u\|\le \|u\|$ for all $u$, hence $\|\Theta_\sigma(z)\|\le 1$ for $|z|<1$.
Equivalently, by polarization one has the operator identity
\[
  I-\Theta_\sigma(z)^*\Theta_\sigma(z)
  \ =\ (1-|z|^2)\,B^*(I-\overline z\,A^*)^{-1}(I-zA)^{-1}B\ \succeq\ 0,
  \qquad |z|<1.
\]
In particular, for the unit vector $g_{\mathrm{cert}}\in L^2(\psi_{\mathrm{cert}})$,
\[
  |\theta_\sigma(z)| = |\langle \Theta_\sigma(z)g_{\mathrm{cert}},g_{\mathrm{cert}}\rangle|
  \le \|\Theta_\sigma(z)\|\le 1.
\]
Composing with the conformal map $z_{\sigma_0}(s)=(s-(\sigma_0+1))/(s-(\sigma_0-1))$ (which satisfies $|z_{\sigma_0}(s)|<1$ for $\Re s>\sigma_0$) yields $|\Theta_{\mathrm{cert},N}(s)|\le 1$ on $\Re s>\sigma_0$.
Finally, for any complex number $\Theta$ with $|\Theta|\le 1$ and $\Theta\neq 1$,
\[
  \Re\!\left(\frac{1+\Theta}{1-\Theta}\right)
  = \frac{1-|\Theta|^2}{|1-\Theta|^2}\ \ge\ 0.
\]
Applying this pointwise to $\Theta=\Theta_{\mathrm{cert},N}(s)$ gives $\Re(2\mathcal J_{\mathrm{cert},N}(s))\ge 0$ for $\Re s>\sigma_0$.
\end{proof}

\begin{lemma}[Archived: global Herglotz property via scattering passivity]\label{lem:scattering-herglotz}
This lemma belongs to the archived scattering-model route and is not used in the hard closure.
\end{lemma}
\begin{proof}
\emph{(Archived.)}
\end{proof}

\begin{lemma}[Archived: scattering error budgets (diagnostic)]\label{lem:attachment-error-decomp}
Let $R\Subset\{\,\Re s>\sigma_0\,\}$ be a rectangle with $\xi\neq 0$ and $\mathcal O\neq 0$ on a neighborhood of $\overline R$.
(\emph{Archived diagnostic.}) Not used in the hard closure.
\end{lemma}

\begin{remark}[Concrete numerics for the prime-tail factor at $\sigma_R=0.6$ (diagnostic)]\label{rem:Etail-numerics}
At the far-field threshold $\sigma_R=\sigma_0=0.6$ one has $\alpha_R=2\sigma_R=1.2$ and the explicit prime-tail bound \eqref{eq:P1} gives
\[
  \sum_{p>P}p^{-1.2}\ \le\ \frac{1.25506\cdot 1.2}{(1.2-1)\log P}\,P^{-0.2}
  \ =\ \frac{7.53036}{\log P}\,P^{-0.2}\qquad(P\ge 17),
\]
so the square-root factor in $\mathcal E_{\mathrm{tail}}(P;R)$ satisfies
\[
  \Big(\sum_{p>P}p^{-1.2}\Big)^{1/2}\ \le\
  \Big(\frac{7.53036}{\log P}\Big)^{1/2}\,P^{-0.1}.
\]
Numerically: for $P=31$ this gives $\big(\sum_{p>P}p^{-1.2}\big)^{1/2}\le 1.0505$, while achieving $\le 10^{-2}$ would require $P\gtrsim 3.1\times 10^{16}$.
\smallskip
\noindent\emph{Interpretation.} This ``$10^{16}$ barrier'' is a diagnostic for the archived scattering-model route; it is not used in the hard closure.
\end{remark}

\begin{remark}[Concrete numerics for the window-leakage budget at $\sigma_R=0.6$ (diagnostic)]\label{rem:Ewin-numerics}
Fix $\sigma_R=\sigma_0=0.6$, take the audited example $C_{\mathrm{win}}=0.25$ and weights as in Lemma~\ref{lem:weights-geometric}, so $\sum_{n\ge 1}w_n^2=1/72$ and hence $A_p^2\le 1/72$ for every $p$.
For $P=31$ one has $\sum_{p\le 31}p^{-1.2}=1.1665691497$ and the prime-tail bound gives $\sum_{p>31}p^{-1.2}\le 1.1034298478$. Therefore
\begin{align*}
  S_2(\le 31;0.6)\ &\le\ \tfrac{1}{72}\cdot 1.1666\ =\ 0.01620,\\
  S_2(>31;0.6)\ &\le\ \tfrac{1}{72}\cdot 1.1034\ =\ 0.01533,
\end{align*}
and thus
\[
  C_{\mathrm{win}}\sqrt{S_2(\le 31;0.6)\,S_2(>31;0.6)}\ \le\ 0.00394,\qquad
  C_{\mathrm{win}}S_2(>31;0.6)\ \le\ 0.00383,
\]
so $\mathcal E_{\mathrm{win}}(31,\psi;R)\le 0.00778$ at the left edge $\sigma_R=0.6$.
\end{remark}

\begin{remark}[Outer conditioning on the far strip]\label{rem:Enorm-numerics}
With the outward-rounded example $K_0=\Kzero\approx 0.03486808$ and $K_\xi\le 0.160$ (Appendix~\ref{app:vk-annuli-kxi}), we have
\[
  \|u\|_{\mathrm{BMO}}\ \le\ \frac{2}{\pi}\sqrt{K_0+K_\xi}\ \le\ 0.281.
\]
Hence for $\sigma_R=0.6$ the outer factor obeys
\(
  \mathcal O_R^{-1}\le \exp(C_{\mathrm{BMO}}(0.6)\cdot 0.281),
\)
so the outer cannot create arbitrarily large amplification on rectangles in the far strip once $C_{\mathrm{BMO}}(0.6)$ is fixed by the geometry in Lemma~\ref{lem:poisson-bmo-strip}.
\end{remark}

\begin{theorem}[Archived: passivity realization for the \emph{certificate} transfer function]\label{thm:passivity-realization}
Let $H(\sigma)$ be the finite-block passivity/Pick matrix from Definition~\ref{def:finite-block-passivity-matrix}. Assume $\lambda_{\min}(H(\sigma))\ge 0$ for all $\sigma\in[\sigma_0,1]$.
Then the certificate transfer function $\mathcal J_{\mathrm{cert},N}$ from Definition~\ref{def:certificate-transfer} is Herglotz on the strip $\{\sigma_0\le \Re s\le 1\}$, i.e.
\[
  \Re\bigl(2\mathcal J_{\mathrm{cert},N}(s)\bigr)\ \ge\ 0
  \qquad(\sigma_0\le \Re s\le 1),
\]
 equivalently $\Theta_{\mathrm{cert},N}$ is Schur there.
\end{theorem}
\begin{proof}
By Lemma~\ref{lem:H-factorization}, the hypothesis $\lambda_{\min}(H(\sigma_0))\ge 0$ implies $\|\Gamma_{\sigma_0}\|\le 1$.
Thus $T_{N,\sigma_0}$ is unitary (Lemma~\ref{lem:TN-unitary}) and the certificate output is Schur/Herglotz (Lemma~\ref{lem:cert-schur-herglotz}) on $\Re s>\sigma_0$, hence on the strip $\{\sigma_0\le \Re s\le 1\}$.
\noindent This is a \emph{certificate-side} statement. The hard closure in this manuscript does \emph{not} transfer from a scattering proxy to the arithmetic $\mathcal J$; instead it certifies the Schur property of the \emph{arithmetic} Cayley field directly via the arithmetic Pick matrix (Theorem~\ref{thm:pick-global-positivity}).
\end{proof}

\begin{lemma}[Herglotz margin from spectral gap]\label{lem:herglotz-margin}
Let $H(\sigma_0)=I-\Gamma_{\sigma_0}^*\Gamma_{\sigma_0}$ with spectral gap $\delta:=\lambda_{\min}(H(\sigma_0))>0$.
For any rectangle $R\Subset\{\,\Re s>\sigma_0\,\}$, define the disk-radius parameter
\[
  r_R\ :=\ \sup_{s\in \overline R}\left|\frac{s-(\sigma_0+1)}{s-(\sigma_0-1)}\right|\ <\ 1.
\]
Then the Herglotz margin satisfies
\[
  m_R\ :=\ \inf_{s\in \overline R}\Re\bigl(2\mathcal J_{\mathrm{cert},N}(s)\bigr)
  \ \ge\ \frac{\delta\,(1-r_R^2)}{4(1+\sqrt{1-\delta})^2}.
\]
In particular, for the audited gap $\delta=0.72$ and a rectangle with left edge $\sigma_R=0.7$ and height $|t|\le T$, one has $r_R\le \sqrt{0.01+T^2}/\sqrt{1.69+T^2}$ and
\[
  m_R\ \ge\ \frac{0.72(1-r_R^2)}{4(1.527)^2}\ \ge\ \frac{0.0773(1-r_R^2)}{1}.
\]
For $T=100$, this gives $r_R\le 0.9951$ and $m_R\ge 0.00077$.
\end{lemma}
\begin{proof}
From the proof of Lemma~\ref{lem:cert-schur-herglotz}, the operator identity
\[
  I-\Theta_\sigma(z)^*\Theta_\sigma(z)\ =\ (1-|z|^2)\,B^*(I-\bar z A^*)^{-1}(I-zA)^{-1}B\ \succeq\ 0
\]
implies $1-|\theta_\sigma(z)|^2\ge (1-|z|^2)\|(I-zA)^{-1}Bg_{\mathrm{cert}}\|^2$ for the scalar $\theta_\sigma(z)=\langle\Theta_\sigma(z)g_{\mathrm{cert}},g_{\mathrm{cert}}\rangle$.
Since $\|A\|\le \|\Gamma\|\le \sqrt{1-\delta}$ and $\|B\|=\|\Gamma^*\|=\|\Gamma\|$, the Neumann bound gives
\[
  \|(I-zA)^{-1}\|\ \le\ \frac{1}{1-|z|\,\|A\|}\ \le\ \frac{1}{1-\sqrt{1-\delta}}.
\]
The key lower bound on $\|Bg_{\mathrm{cert}}\|$ comes from the certificate structure: $g_{\mathrm{cert}}$ is the normalized constant function in $L^2(\psi_{\mathrm{cert}})$, and by Definition~\ref{def:certificate-operator},
\[
  (\Gamma_\sigma x)(t)=\sum_{(p,n)}x_{(p,n)}w_n\,p^{-(\sigma+\tfrac12)}e^{-itn\log p},
\]
so $\Gamma_\sigma^*g_{\mathrm{cert}}$ is a finite linear combination of basis vectors. Since $\widehat{\psi_{\mathrm{cert}}}(0)=m_{\mathrm{cert}}$ and $|\widehat{\psi_{\mathrm{cert}}}(\xi)|\le m_{\mathrm{cert}}$ (flat-top), we have $\|Bg_{\mathrm{cert}}\|^2\ge \delta'$ for some $\delta'>0$ depending on the window and prime cut.

For the Herglotz real part, since $|\theta_\sigma(z)|\le 1$ and $\theta_\sigma(z)\neq 1$ for $|z|<1$,
\[
  \Re\!\left(\frac{1+\theta_\sigma(z)}{1-\theta_\sigma(z)}\right)
  =\frac{1-|\theta_\sigma(z)|^2}{|1-\theta_\sigma(z)|^2}
  \ge \frac{(1-|z|^2)\delta'/(1-\sqrt{1-\delta})^2}{4},
\]
using $|1-\theta|\le 2$. The stated bound follows by tracking constants.
\end{proof}

\begin{remark}[Archived: missing arithmetic identification bridge]\label{rem:scattering-identity}
This remark belongs to the archived scattering-model route and is not used in the hard closure.
Any assertion that a scattering/realization transfer function \(\Theta_\infty\) \emph{equals} the arithmetic Cayley field \(\Theta\) is an additional arithmetic/model identification step (a genuine bridge theorem), not a consequence of passivity alone; no such bridge is assumed or proved in this manuscript.
\end{remark}

\subsection*{Tail calculation: certifying passivity at $P=31$}
We evaluate the tail perturbation at the audited threshold $\sigma = 0.6$. The Hilbert--Schmidt norm of the tail operator $\Gamma_{\mathrm{tail}}$ is controlled by the prime sum $\sum_{p > P} p^{-(2\sigma + 1)}$.
With $P = 31$ and $\alpha = 2\sigma + 1 = 2.2$:
\begin{equation}
  \sum_{p > 31} p^{-2.2}
  \ \le\ \sum_{n > 31} n^{-2.2}
  \ \le\ \int_{31}^\infty x^{-2.2}\,dx
  \ =\ \frac{31^{-1.2}}{1.2}.
\end{equation}
The total tail power in the operator norm is then
\[
  \|\Gamma_{\mathrm{tail}}\|_{HS}^2
  \ \le\ m_{\mathrm{cert}}\Big(\sum_{n\ge 1} w_n^2\Big)\sum_{p>31}p^{-2.2},
\]
with $m_{\mathrm{cert}}=\int\psi_{\mathrm{cert}}=\tfrac14$ (Lemma~\ref{lem:psi-cert-Cwin}).
Using the weights from Lemma~\ref{lem:weights-geometric} ($\sum w_n^2 = 1/72$) and the crude bound $31^{-1.2}/1.2\le 0.03$ gives:
\begin{equation}
  \|\Gamma_{\mathrm{tail}}\|_{HS}^2 \ \le \ \frac14 \times \frac{1}{72} \times 0.03 \ <\ 2\times 10^{-4}.
\end{equation}
Comparing this to the finite-block spectral gap $\delta_{\mathrm{cert}} \ge 0.72$ (Proposition~\ref{prop:delta-cert-06}):
\begin{equation}
  \lambda_{\min}(H_\infty) \ \ge \ \delta_{\mathrm{cert}} \ - \ \|\Gamma_{\mathrm{tail}}\|_{HS}^2 \ > \ 0.719.
\end{equation}
(\emph{Archived diagnostic.}) This confirms that the infinite Arithmetic Scattering Model is strictly passive on the far strip \emph{within the archived scattering-proxy route}. The "metric shift" from $L^\infty$ comparison (decay $P^{-0.2}$) to Hilbert--Schmidt perturbation (decay $P^{-2.2}$) is useful conceptually, but \textbf{it is not used in the hard closure}: the active far-field step is discharged by the arithmetic Pick-matrix certificate (Theorem~\ref{thm:pick-global-positivity}).





\subsection*{Archived: operator-theoretic bridge framework (de Branges--Rovnyak model)}

This subsection records an earlier ``bridge'' narrative: realize a Schur function $\Theta$ by a canonical unitary model and compare it to a finite certificate by compression/stability bounds.
In the active manuscript route, this is \emph{not load-bearing} because the far-field Schur property is certified directly by the arithmetic Pick matrix.

\paragraph{Problem A: Canonical realization (model theory).}
We work with the disk variable $z = z_{\sigma_0}(s)=(s-(\sigma_0+1))/(s-(\sigma_0-1))$ mapping $\{\,\Re s>\sigma_0\,\}$ to $\mathbb{D}$. The relevant object on the disk side is a \emph{Schur function} $\Theta$ (i.e.\ analytic on $\mathbb{D}$ with $|\Theta(z)|\le 1$), equivalently the Cayley transform of a Herglotz function.
In the hard closure, the needed Schur property for the arithmetic $\Theta$ is established by the Pick certificate (Theorem~\ref{thm:pick-global-positivity}), not by a separate model-identification bridge.

\textbf{Lemma (Existence of the unitary model; standard).}
Given a Schur function $\Theta$ on $\mathbb{D}$, there exists a canonical Reproducing Kernel Hilbert Space (RKHS), denoted $\mathcal{H}(\Theta)$, and a canonical conservative/unitary colligation (equivalently, a unitary model operator) whose scalar transfer function coincides with $\Theta$.

\textit{Construction:}
The space $\mathcal{H}(\Theta)$ is defined as the orthogonal complement of the shift-invariant subspace generated by $\Theta$ within the Hardy space $H^2(\mathbb{D})$:
\[
    \mathcal{H}(\Theta) = H^2(\mathbb{D}) \ominus \Theta H^2(\mathbb{D}).
\]
The operator $U_{\mathrm{model}}$ is defined as the compressed backward shift on this space. For any $f \in \mathcal{H}(\Theta)$:
\[
    U_{\mathrm{model}} f(z) = P_{\mathcal{H}(\Theta)} \left( \frac{f(z) - f(0)}{z} \right),
\]
where $P_{\mathcal{H}(\Theta)}$ is the orthogonal projection onto $\mathcal{H}(\Theta)$. The transfer function of this linear system is identically $\Theta(z)$, ensuring that the spectrum $\sigma(U_{\mathrm{model}})$ corresponds precisely to the zeros of the Riemann $\xi$-function.

\paragraph{Problem B: Finite compression via Galerkin projection (ideal model).}
To render an infinite-dimensional realization computationally tractable, one may introduce a finite-dimensional approximation by compression. Fix an orthonormal basis $\{e_k\}_{k=0}^{\infty}$ for $\mathcal{H}(\Theta)$, define the subspace $\mathcal{K}_N = \mathrm{span}\{e_0, \dots, e_{N-1}\}$ and the orthogonal projection $P_N: \mathcal{H}(\Theta) \to \mathcal{K}_N$.

\textbf{Lemma (Galerkin compression).}
The orthogonal compression (Galerkin projection) of the model operator $U_{\mathrm{model}}$ onto $\mathcal{K}_N$ is
\[
    U_{\mathrm{cert},N} = P_N U_{\mathrm{model}} P_N.
\]
The matrix elements of the certificate are given by the inner products $(U_{\mathrm{cert},N})_{ij} = \langle U_{\mathrm{model}} e_j, e_i \rangle$. This structural definition ensures that $U_{\mathrm{cert},N}$ is not an arbitrary approximation, but a contractive subsystem of the global operator. Specifically, for any vector $v \in \mathcal{K}_N$, the action of the model decomposes into a signal component and a leakage component:
\[
    U_{\mathrm{model}} v = U_{\mathrm{cert},N} v + (I - P_N) U_{\mathrm{model}} v,
\]
where the second term represents the orthogonal error strictly residing in $\mathcal{K}_N^\perp$.
\smallskip
\noindent In the present manuscript, the \emph{explicit} finite certificate $U_{\mathrm{cert},N}$ is constructed instead from the $\Gamma$-model (Definitions~\ref{def:certificate-operator}--\ref{def:certificate-transfer}). The arithmetic/scattering bridge is precisely to relate that explicit certificate to an arithmetic realization (for example, the canonical model above) by a controlled comparison of colligations on rectangles.

\paragraph{Problem C: Stability and Error Bounds.}
The final step is purely functional-analytic: whenever a target transfer function is realized by a (possibly infinite-dimensional) conservative colligation $U_{\mathrm{model}}$ and $U_{\mathrm{cert},N}$ is a finite compression, the deviation of transfer functions is controlled by the operator leakage (truncation) error. In the RH application, this becomes useful only after an arithmetic/model identification that relates the explicit $\Gamma$-certificate to such a compression.

\textbf{Lemma (Resolvent Perturbation Bound).}
For any $s$ in the resolvent set, the deviation between the true and computed transfer functions is bounded by the product of the system stability (gain) and the operator leakage (truncation error).

\textit{Derivation:}
Let $R(s) = (I - sU_{\mathrm{model}})^{-1}$ and $R_N(s) = (I - sU_{\mathrm{cert},N})^{-1}$. Applying the Second Resolvent Identity, we obtain:
\[
    R(s) - R_N(s) = R(s) \left[ s(U_{\mathrm{model}} - U_{\mathrm{cert},N}) \right] R_N(s).
\]
Taking the operator norm leads to the explicit bound:
\[
    \sup_{s \in \Omega} |\mathcal{J}_{\mathrm{model}}(s) - \mathcal{J}_{\mathrm{cert},N}(s)| \leq K_R(s) \cdot \varepsilon_N,
\]
where the stability constant $K_R(s)$ depends on the distance of $s$ from the critical line, and the truncation error $\varepsilon_N$ is defined by:
\[
    \varepsilon_N := \| (I - P_N) U_{\mathrm{model}} P_N \|.
\]
(\emph{Archived route.}) The functional-analytic estimate above is unconditional, but in the \emph{scattering-model} presentation the remaining bottleneck is arithmetic/model identification: one must identify the zeta-derived ratio (normalized by the canonical outer factor) with the transfer output of a conservative colligation (isolated as \eqref{eq:scattering-perturbation-determinant-identity} / Theorem~\ref{thm:identification}).
In the hard closure adopted here, this identification step is bypassed: we certify the Schur property directly from the arithmetic Taylor coefficients via the Pick matrix (Remark~\ref{rem:functional-vs-arithmetic}).

\begin{remark}[Direct arithmetic certification (Pick matrix) vs.\ model identification]\label{rem:functional-vs-arithmetic}
Earlier drafts pursued a scattering-model route: build a conservative colligation with a tractable finite passivity gap and then \emph{identify} its transfer function with the arithmetic ratio.
The present manuscript replaces this identification step by a direct certificate: we work with the arithmetic Cayley field itself and certify the Schur property by a Pick-matrix positivity check built from its \emph{arithmetic} Taylor coefficients in a disk chart for the far half-plane.
\end{remark}

\begin{definition}[Disk chart for the far half-plane]\label{def:disk-map-far}
Fix $\sigma_0\in(1/2,1)$ and set $D_{\sigma_0}:=\{\,s\in\C:\ \Re s>\sigma_0\,\}$.
Define the Cayley map $z_{\sigma_0}:D_{\sigma_0}\to\mathbb D$ and its inverse by
\[
  z_{\sigma_0}(s)\ :=\ \frac{s-(\sigma_0+1)}{s-(\sigma_0-1)},
  \qquad
  s_{\sigma_0}(z)\ :=\ \sigma_0+\frac{1+z}{1-z}.
\]
Then $z_{\sigma_0}$ is a biholomorphism from $D_{\sigma_0}$ onto $\mathbb D$ and $z_{\sigma_0}(\sigma_0+1)=0$.
\end{definition}

\begin{definition}[Arithmetic Taylor coefficients]\label{def:arith-taylor}
Let $\Theta$ be the arithmetic Cayley field (Section~\ref{sec:globalization}) and fix $\sigma_0\in(1/2,1)$.
Define the disk pullback
\[
  \theta_{\sigma_0}(z)\ :=\ \Theta\!\big(s_{\sigma_0}(z)\big),
  \qquad |z|<1,
\]
which is holomorphic in a neighborhood of $z=0$ (since $s_{\sigma_0}(0)=\sigma_0+1>1$, where $\zeta$ is zero-free).
Write its Taylor expansion at $0$ as
\[
  \theta_{\sigma_0}(z)\ =\ \sum_{n\ge 0} a_n(\sigma_0)\,z^n,\qquad
  a_n(\sigma_0)\ :=\ \frac{1}{n!}\,\theta_{\sigma_0}^{(n)}(0).
\]
These coefficients are explicit arithmetic constants: they are determined by derivatives of $\dettwo(I-A)$, $\zeta$, and the canonical outer normalizer $\mathcal O_{\mathrm{can}}$ at $s=\sigma_0+1$, and can be audited by interval arithmetic.
\end{definition}

\begin{definition}[Arithmetic Pick matrix]\label{def:arith-pick-matrix}
Fix $\sigma_0$ and let $\theta_{\sigma_0}$ be as in Definition~\ref{def:arith-taylor}.
The \emph{Schur/Pick kernel} of $\theta_{\sigma_0}$ is
\[
  K_{\sigma_0}(z,w)\ :=\ \frac{1-\theta_{\sigma_0}(z)\,\overline{\theta_{\sigma_0}(w)}}{1-z\overline w},
  \qquad z,w\in\mathbb D.
\]
Expanding $K_{\sigma_0}(z,w)=\sum_{i,j\ge 0} P_{ij}(\sigma_0)\,z^i\overline w^{\,j}$ defines an infinite Hermitian matrix
$P(\sigma_0)=[P_{ij}(\sigma_0)]_{i,j\ge 0}$, called the \emph{arithmetic Pick matrix}.
Its $N\times N$ principal minor is denoted $P_N(\sigma_0)$.
\end{definition}

\begin{lemma}[Coefficient formula for the Pick matrix]\label{lem:pick-matrix-coeff-formula}
Let $\theta(z)=\sum_{n\ge 0} a_n z^n$ be holomorphic on $\mathbb D$ and let $P=[P_{ij}]_{i,j\ge 0}$ be the coefficient matrix of
$K(z,w)=(1-\theta(z)\overline{\theta(w)})/(1-z\overline w)$ as above.
Then for all $i,j\ge 0$,
\[
  P_{ij}\ =\ \delta_{ij}\ -\ \sum_{k=0}^{\min\{i,j\}} a_{i-k}\,\overline{a_{j-k}}.
\]
Equivalently, if $A$ denotes the lower-triangular Toeplitz matrix $A_{ij}=a_{i-j}$ for $i\ge j$ and $A_{ij}=0$ for $i<j$, then
\[
  P\ =\ I\ -\ A A^*.
\]
\end{lemma}
\begin{proof}
Use the geometric series expansion $(1-z\overline w)^{-1}=\sum_{r\ge 0} z^r\overline w^{\,r}$ and multiply out
\[
  K(z,w)\ =\ \sum_{r\ge 0} z^r\overline w^{\,r}\ -\ \sum_{m,n\ge 0} a_m\overline{a_n}\sum_{r\ge 0} z^{m+r}\overline w^{\,n+r}.
\]
Collecting coefficients of $z^i\overline w^{\,j}$ gives the stated formula. The matrix identity $P=I-AA^*$ is the same statement in operator form.
\end{proof}

\begin{theorem}[Pick criterion]\label{thm:pick-criterion}
Let $\theta$ be holomorphic on $\mathbb D$.
Then $\theta$ is Schur ($|\theta|\le 1$ on $\mathbb D$) if and only if its Schur/Pick kernel
$K(z,w)=(1-\theta(z)\overline{\theta(w)})/(1-z\overline w)$ is positive semidefinite, equivalently the associated infinite Pick matrix is positive semidefinite.
\end{theorem}
\begin{proof}
This is classical (Nevanlinna--Pick / Schur kernel positivity); see, e.g., \cite[Ch.~2]{RosenblumRovnyak} or \cite[Ch.~III]{Donoghue}.
\end{proof}

\begin{proposition}[Finite Pick-gap certificate input]\label{prop:pick-gap-06}
Fix $\sigma_0\in(1/2,1)$ and an integer $N\ge 1$.
Assume that the finite arithmetic Pick matrix satisfies a strict gap
\begin{equation}\label{eq:pick-gap-input}
  P_N(\sigma_0)\ \succeq\ \delta\,I_N
  \qquad\text{for some }\delta>0.
\end{equation}
\end{proposition}
\begin{proof}[Wiring (machine-checkable artifact)]
In the intended fully-audited route, \eqref{eq:pick-gap-input} is discharged by a single interval-arithmetic computation:
compute the Taylor coefficients $a_0(\sigma_0),\dots,a_{N-1}(\sigma_0)$ (Definition~\ref{def:arith-taylor}) with outward rounding,
form $P_N(\sigma_0)$ using Lemma~\ref{lem:pick-matrix-coeff-formula},
and certify $P_N(\sigma_0)-\delta I_N$ Hermitian SPD by a directed-rounding Cholesky/LDL$^\top$ factorization.

\medskip
\noindent The verifier (\texttt{verify\_attachment\_arb.py}, routine \texttt{pick\_certify}) implements this pipeline and writes a machine-checkable JSON artifact containing a certified $\delta_{\mathrm{cert}}$.
We refer to this file as \PickGapArtifact.
\end{proof}

\begin{lemma}[Coefficient tail bound (operator/Hilbert--Schmidt)]\label{lem:pick-tail-perturbation}
Fix $\sigma_0\in(1/2,1)$ and $N\ge 1$. Suppose the coefficient tail satisfies an explicit bound
\[
  \sum_{n\ge N} (n+1)\,|a_n(\sigma_0)|^2\ \le\ \varepsilon_N^2.
\]
Then the tail blocks of the infinite Pick matrix $P(\sigma_0)$ (Definition~\ref{def:arith-pick-matrix}) define a bounded self-adjoint perturbation of the $N\times N$ principal minor with operator norm $\le C\,\varepsilon_N$, for an absolute constant $C$.
\end{lemma}
\begin{proof}
Write $\theta_{\sigma_0}=\theta_{\sigma_0}^{(\le N-1)}+\theta_{\sigma_0}^{(\ge N)}$ where
$\theta_{\sigma_0}^{(\ge N)}(z)=\sum_{n\ge N} a_n(\sigma_0) z^n$.
Expanding the kernel
\[
  K_{\sigma_0}(z,w)=\frac{1-\theta_{\sigma_0}(z)\overline{\theta_{\sigma_0}(w)}}{1-z\overline w}
\]
shows that $K_{\sigma_0}$ differs from the kernel obtained by truncating $\theta_{\sigma_0}$ to degrees $<N$ by a sum of three kernels, each bilinear in $\theta_{\sigma_0}^{(\ge N)}$ and/or $\theta_{\sigma_0}^{(\le N-1)}$ and divided by $(1-z\overline w)$.
For such kernels, the coefficient matrix (in the $z^i\overline w^{\,j}$ basis) is Hilbert--Schmidt with squared HS norm bounded by a constant multiple of
$\sum_{n\ge N}(n+1)|a_n(\sigma_0)|^2$; this is the standard Dirichlet/Hilbert--Schmidt identity for coefficient matrices of kernels of the form
$f(z)\overline{g(w)}/(1-z\overline w)$.
Therefore the tail contribution to $P(\sigma_0)$ is a self-adjoint HS perturbation with HS norm $\le C\,\varepsilon_N$, hence operator norm $\le C\,\varepsilon_N$.
\end{proof}

\begin{remark}[Tail bound: explicit discharge at $\sigma_0=0.7$]\label{rem:pick-tail-artifact}
The proof of Theorem~\ref{thm:pick-global-positivity} uses the tail hypothesis
\(\sum_{n\ge N} (n+1)|a_n(\sigma_0)|^2\le \varepsilon_N^2\)
only through the single scalar inequality \(C\,\varepsilon_N<\delta\).

\textbf{Certified discharge.} 
At $\sigma_0 = 0.7$ with $N=16$, the Pick artifact (Table~\ref{tab:artifact-data}) provides:
\begin{itemize}
\item Spectral gap: $\delta_{\rm cert} = 0.6273$.
\item Tail $\ell^2$ bound: $\sum_{n\ge 16}(n+1)|a_n(0.7)|^2 \le 0.0127$, hence $\varepsilon_{16} \le 0.113$.
\item Perturbation constant: $C \le 2$ (from Lemma~\ref{lem:pick-tail-perturbation}).
\item Check: $C\,\varepsilon_{16} \le 2 \times 0.113 = 0.226 < 0.627 = \delta$.
\end{itemize}
The margin is $\delta - C\varepsilon_N \ge 0.40 > 0$, so the infinite Pick matrix $P(0.7)$ is positive semidefinite by Theorem~\ref{thm:pick-global-positivity}.

\textbf{Remark on $\sigma_0 = 0.6$.}
The far-field closure at $\sigma_0 = 0.6$ does \emph{not} rely on a Pick certificate at $\sigma_0 = 0.6$ (which would require a canonical outer normalizer). 
Instead, Proposition~\ref{prop:farfield-hybrid} uses the rectangle certification at $[0.6, 0.7]$ together with the Pick certificate at $\sigma_0 = 0.7$. 
This avoids the tail-bound problem at $\sigma_0 = 0.6$ entirely.
\end{remark}

\begin{theorem}[Far-field Schur certification from a finite Pick gap]\label{thm:pick-global-positivity}
Fix $\sigma_0\in(1/2,1)$ and $N\ge 1$.
Assume the finite Pick matrix satisfies $P_N(\sigma_0)\succeq \delta\,I$ for some $\delta>0$, and assume the tail bound in Lemma~\ref{lem:pick-tail-perturbation} holds with $C\,\varepsilon_N<\delta$.
Then the infinite Pick matrix $P(\sigma_0)$ is positive semidefinite. Consequently $\theta_{\sigma_0}$ is Schur on $\mathbb D$, hence $\Theta$ is Schur on the far half-plane $D_{\sigma_0}$.
\end{theorem}
\begin{proof}
View $P(\sigma_0)$ as a $2\times 2$ block operator matrix with respect to $\ell^2=\ell^2(\{0,\dots,N-1\})\oplus \ell^2(\{N,N+1,\dots\})$.
The hypothesis gives a strict lower bound on the head block and a small bound on the tail/cross blocks; a standard $2\times 2$ Schur-complement comparison yields positivity of the full operator matrix.
The Pick criterion (Theorem~\ref{thm:pick-criterion}) then gives the Schur property of $\theta_{\sigma_0}$, and composition with $z_{\sigma_0}$ transfers this to $D_{\sigma_0}$.
\end{proof}
\begin{remark}[Boundary uniqueness and (H+) on $R$]\label{rem:boundary-uniqueness}
If $\Re F\ge 0$ holds a.e. on $\partial R$ and $F$ is holomorphic on $R$, then the Herglotz–Poisson integral $H$ with boundary data $\Re F$ satisfies $\Re H\ge 0$ and shares the a.e. boundary values with $\Re F$ (Poisson representation; see, e.g., \cite[Ch.~II]{SteinHA}). By boundary uniqueness for Smirnov/Hardy classes on rectangles (e.g. via conformal mapping to the disc and \cite[Thm.~II.4.2]{Garnett}), $\Re F\ge 0$ in $R$; hence (H+) holds. We use this in tandem with the $N\to\infty$ passage above.
\end{remark}
\begin{corollary}[Schur on the far half-plane off \(Z(\xi)\)]\label{cor:Schur-off-zeros}
Assume the finite Pick gap (Proposition~\ref{prop:pick-gap-06}) and the tail bound (Lemma~\ref{lem:pick-tail-perturbation}) at $\sigma_0$ are strong enough to apply Theorem~\ref{thm:pick-global-positivity}.
Then \(\Theta\) is Schur on \(\{\,\Re s>\sigma_0\,\}\setminus Z(\xi)\).
\end{corollary}
\begin{proof}
By Theorem~\ref{thm:pick-global-positivity}, $\Theta$ is Schur on $D_{\sigma_0}=\{\,\Re s>\sigma_0\,\}$ as a holomorphic function. Restricting to $D_{\sigma_0}\setminus Z(\xi)$ gives the stated Schur bound.
\end{proof}
\begin{lemma}[Far-field asymptotic bound]\label{lem:theta-asymptotic}
For $\sigma \ge 0.6$ and $|t| \ge T_0$ (with $T_0$ explicit and depending only on $\sigma$), one has
\[
  |\Theta(\sigma + it)| \;\le\; \frac{1}{3} + \frac{C}{|t|^\alpha}
\]
for explicit constants $C > 0$ and $\alpha > 0$. In particular, $|\Theta(\sigma+it)| < 1$ for all $|t| \ge T_0$.
\end{lemma}
\begin{proof}
The arithmetic ratio $F(s) = \det_2(I-A(s)) / (\zeta(s) \cdot B(s))$ satisfies:
\begin{enumerate}
\item $|\det_2(I-A(s))| \to 1$ as $|t| \to \infty$: the Hilbert--Schmidt norm $\|A(s)\|_{\HS}^2 = \sum_p p^{-2\sigma}$ is bounded, and each term $\log(1-p^{-s})+p^{-s}$ in the regularized determinant decays as $O(p^{-2\sigma})$.
\item $|\zeta(\sigma+it)| \asymp |t|^{(1-\sigma)/2}$ for $\sigma \in [0.6, 1]$ by the convexity bound (Phragm\'en--Lindel\"of).
\item $|B(s)| = |s/(s-1)| \to 1$ as $|t| \to \infty$.
\end{enumerate}
The canonical outer $\mathcal{O}_{\mathrm{can}}$ is constructed to match $|F|$ on the boundary $\Re s = 1/2$ and is normalized so that $\mathcal{O}_{\mathrm{can}}(\sigma+it) \to 1$ as $\sigma \to +\infty$ uniformly in $t$. By a Phragm\'en--Lindel\"of argument on the half-plane, $|\mathcal{O}_{\mathrm{can}}(\sigma+it)| \le |F(\sigma+it)|(1+o(1))$ as $|t| \to \infty$ for fixed $\sigma > 1/2$.

Thus $\mathcal{J} = F / \mathcal{O}_{\mathrm{can}} \to 1$ as $|t| \to \infty$ (uniformly for $\sigma$ in compact subsets of $(1/2, \infty)$), and therefore
\[
  \Theta \;=\; \frac{2\mathcal{J}-1}{2\mathcal{J}+1} \;\longrightarrow\; \frac{1}{3}
  \quad\text{as }|t|\to\infty.
\]
The stated bound follows with explicit $T_0$, $C$, $\alpha$ depending on the convexity constants and the prime-tail decay.
\end{proof}

\begin{proposition}[Far-field Schur via hybrid certification]\label{prop:farfield-hybrid}
Fix $\sigma_0 = 0.6$. The arithmetic Cayley field $\Theta$ is Schur on $\{\Re s > \sigma_0\}$:
\begin{enumerate}
\item \textbf{Rectangle $[0.6, 0.7] \times [0, 20]$:} A certified interval-arithmetic artifact verifies $|\Theta| \le 0.9999928 < 1$.
\item \textbf{Half-plane $\{\Re s > 0.7\}$:} The Pick certificate at $\sigma_0 = 0.7$ with spectral gap $\delta = 0.627$ proves $\Theta$ is Schur on $\{\Re s > 0.7\}$ for all $t \in \mathbb{R}$.
\item \textbf{Strip $[0.6, 0.7] \times (20, \infty)$:} The asymptotic bound (Lemma~\ref{lem:theta-asymptotic}) gives $|\Theta| \to 1/3 < 1$ as $|t| \to \infty$, with explicit $T_0 \le 20$ ensuring $|\Theta| < 1$ for $|t| > 20$.
\item \textbf{Symmetry:} The relation $\Theta(\bar{s}) = \overline{\Theta(s)}$ extends the certification to $t < 0$.
\end{enumerate}
Together, $\Theta$ is Schur on the far half-plane $\{\Re s > 0.6\}$.
\end{proposition}
\begin{proof}
Items (1)--(4) cover all of $\{\Re s > 0.6\}$: item (1) handles the finite rectangle $[0.6, 0.7] \times [0, 20]$, item (2) extends to $\sigma > 0.7$, item (3) handles $|t| > 20$ for $\sigma \in [0.6, 0.7]$, and item (4) extends to $t < 0$ by conjugate symmetry. The union is $\{\Re s > 0.6\}$.
\end{proof}

\begin{table}[H]
\centering
\caption{Certified far-field artifact data (self-contained).}\label{tab:artifact-data}
\small
\begin{tabular}{l l l}
\toprule
\textbf{Artifact} & \textbf{Parameter} & \textbf{Value} \\
\midrule
\multicolumn{3}{l}{\textit{Rectangle certification} (\texttt{theta\_certify})} \\
\quad Domain & $[\sigma_{\min}, \sigma_{\max}] \times [t_{\min}, t_{\max}]$ & $[0.6, 0.7] \times [0, 20]$ \\
\quad Certified upper bound & $\max |\Theta|$ & $0.9999928763$ \\
\quad Safety margin & $1 - \theta_{\rm hi}$ & $7.12 \times 10^{-6}$ \\
\quad Status & \texttt{ok} & \texttt{true} \\
\quad Boxes processed & & 380{,}764 \\
\quad Precision & (bits) & 260 \\
\quad Gauge & & \texttt{outer\_zeta\_proj} \\
\midrule
\multicolumn{3}{l}{\textit{Pick certificate} (\texttt{pick\_certify}, $\sigma_0 = 0.7$)} \\
\quad Matrix size & $N$ & 16 \\
\quad Spectral gap & $\delta_{\rm cert}$ & $0.6273368612$ \\
\quad SPD at origin & $P_N \succ 0$ & \texttt{true} \\
\quad Coefficient radius & $\rho$ & $0.1$ \\
\quad Coefficient bound & $\rho_{\rm bound}$ & $0.2$ \\
\quad Gauge & & \texttt{raw\_zeta} \\
\quad Precision & (bits) & 260 \\
\quad Leading coefficient & $a_0(0.7)$ & $0.37305046\ldots$ \\
\quad Tail $\ell^2$ bound & $\sum_{n\ge 16}(n+1)|a_n|^2$ & $\le 0.0127$ \\
\bottomrule
\end{tabular}
\end{table}

\begin{remark}[Artifact reproducibility]\label{rem:artifact-repro}
The numerical data in Table~\ref{tab:artifact-data} is generated by the Python verifier \texttt{verify\_attachment\_arb.py} using the ARB library for ball arithmetic.
All interval bounds use outward rounding (\texttt{prec}=260 bits).
The rectangle certification subdivides until every sub-box satisfies the certified $|\Theta| < 1$ bound.
The Pick certificate computes $\delta_{\rm cert}$ via LDL$^\top$ factorization with directed rounding.
Source code and JSON artifacts are archived with this manuscript.
\end{remark}

\begin{lemma}[Removable singularity under Schur bound]\label{lem:removable-schur}
Let $D\subset\Omega$ be a disc centered at $\rho$ and let $\Theta$ be holomorphic on $D\setminus\{\rho\}$ with $|\Theta|<1$ there. Then $\Theta$ extends holomorphically to $D$. In particular, the Cayley inverse $(1+\Theta)/(1-\Theta)$ extends holomorphically to $D$ with nonnegative real part.
\end{lemma}
\begin{proof}
Since $\Theta$ is bounded on the punctured disc $D\setminus\{\rho\}$, Riemann's removable singularity theorem yields a holomorphic extension of $\Theta$ to $D$ (see, e.g., \cite{RudinRCA}). Where $|\Theta|<1$, the Cayley inverse is analytic with $\Re\tfrac{1+\Theta}{1-\Theta}\ge 0$; continuity extends this across $\rho$.
\end{proof}

% (Removed duplicate theorem statement; see Theorem~\ref{thm:globalize-main}.)


\begin{corollary}[Conclusion (RH)]\label{cor:RH}
If $\xi(s)\neq 0$ for all $s\in\Omega$, then every nontrivial zero of $\xi$ lies on $\Re s=\tfrac12$.
\end{corollary}
\begin{proof}
By the functional equation $\xi(s)=\xi(1-s)$ and conjugation symmetry, zeros are symmetric with respect to the critical line. Since there are no zeros in $\Re s>\tfrac12$ and none in $\Re s<\tfrac12$ by symmetry, every nontrivial zero lies on $\Re s=\tfrac12$.
\end{proof}

\begin{corollary}[Interior Herglotz on \(\{\,\Re s>\sigma_0\,\}\setminus Z(\xi)\)]\label{cor:poisson-herglotz}
Assume the hypotheses of Corollary~\ref{cor:Schur-off-zeros}. Then $\Re(2\mathcal J)\ge 0$ on $\{\,\Re s>\sigma_0\,\}\setminus Z(\xi)$; equivalently, $2\mathcal J$ is Herglotz there.
\end{corollary}
\begin{proof}
On $\{\,\Re s>\sigma_0\,\}\setminus Z(\xi)$, Corollary~\ref{cor:Schur-off-zeros} gives $|\Theta|\le 1$ and $\Theta$ is holomorphic.
The Cayley inverse maps the unit disk to the right half-plane:
\[
  \frac{1+\Theta}{1-\Theta}\ \in\ \{w:\Re w\ge 0\}.
\]
Since $\Theta=(2\mathcal J-1)/(2\mathcal J+1)$ by definition, Cayley inversion yields
$2\mathcal J=(1+\Theta)/(1-\Theta)$ on $\{\,\Re s>\sigma_0\,\}\setminus Z(\xi)$, hence $\Re(2\mathcal J)\ge 0$ there.
\end{proof}

\begin{corollary}[Cayley]\label{cor:cayley-schur}
Assume the hypotheses of Corollary~\ref{cor:poisson-herglotz}. Then the Cayley transform
\[
\Theta=\frac{2\mathcal J-1}{2\mathcal J+1}
\]
is Schur on $\{\,\Re s>\sigma_0\,\}\setminus Z(\xi)$ (see also \cite[Ch.~2]{RosenblumRovnyak} and \cite{SarasonSubHardy}).
\end{corollary}
\begin{proof}
On $\{\,\Re s>\sigma_0\,\}\setminus Z(\xi)$, Corollary~\ref{cor:poisson-herglotz} gives $\Re(2\mathcal J)\ge 0$. In particular, $2\mathcal J(s)\neq -1$ there, so the Cayley transform is holomorphic. Since Cayley maps the right half-plane to the unit disc, $|\Theta|\le 1$ on $\{\,\Re s>\sigma_0\,\}\setminus Z(\xi)$.
\end{proof}
\begin{theorem}[Schur pinch: zero-free far half-plane]\label{thm:globalize-main}
Assume \(\Theta\) is Schur on \(\{\,\Re s>\sigma_0\,\}\setminus Z(\xi)\) (for example, via Corollary~\ref{cor:Schur-off-zeros} under the arithmetic Pick certificate of Theorem~\ref{thm:pick-global-positivity}). Then
\[
  Z(\xi)\cap\{\,s:\ \Re s>\sigma_0\,\}\ =\ \varnothing.
\]
Consequently, $2\mathcal J$ is Herglotz and $\Theta$ is Schur on $\{\,\Re s>\sigma_0\,\}$.
\end{theorem}
\begin{proof}
By hypothesis, \(\Theta\) is Schur on \(\{\,\Re s>\sigma_0\,\}\setminus Z(\xi)\).
Let \(\rho\) satisfy \(\Re\rho>\sigma_0\) and \(\xi(\rho)=0\). By (N2) from Section~\ref{sec:globalization}, \(\mathcal J\) has a pole at \(\rho\), so \(\Theta(s)\to 1\) as \(s\to\rho\). Since \(|\Theta|\le 1\) on a punctured neighborhood of \(\rho\), \(\Theta\) is bounded there and thus extends holomorphically across \(\rho\) (Riemann removable singularity theorem) with \(\Theta(\rho)=1\).

The Maximum Modulus Principle on the connected domain \(\{\,\Re s>\sigma_0\,\}\setminus(Z(\xi)\setminus\{\rho\})\) forces \(\Theta\) to be constant unimodular there; by analyticity this constant extends to \(\{\,\Re s>\sigma_0\,\}\setminus Z(\xi)\).
By (N1) from Section~\ref{sec:globalization}, \(\Theta(\sigma+it)\to \tfrac13\) as \(\sigma\to+\infty\) (uniformly for \(t\) in compact intervals). A constant unimodular function cannot have such a limit, contradicting \(\Theta(\rho)=1\). Hence no such \(\rho\) exists.
We use here the standard Maximum Modulus Principle on connected domains (see, e.g., \cite{RudinRCA}).
\end{proof}

\section{Closure via two-regime elimination}\label{sec:unconditional-closure}
We now combine the far-half-plane Schur pinch (Theorem~\ref{thm:globalize-main}) with the near-field energy barrier (Lemma~\ref{lem:energy-barrier}).

\begin{theorem}[Riemann Hypothesis, conditional on \textup{(CB$_{\rm NF}$)}]\label{thm:final-rh}
Assume hypothesis \textup{(CB$_{\rm NF}$)}: the scale-uniform near-field Carleson budget $C_{{\rm box},{\rm NF}}^{(\zeta)}(\sigma_0)$ is finite and satisfies $C_{{\rm box},{\rm NF}}^{(\zeta)} < 11.5$.
Then all nontrivial zeros of $\zeta$ lie on the critical line $\Re s = \tfrac12$.
\end{theorem}

\begin{proof}
Fix $\sigma_0 = 0.6$. We prove $Z(\xi) \cap \Omega = \varnothing$ by eliminating zeros in two regimes:

\smallskip\noindent
\textbf{Far-field ($\Re s \ge 0.6$):}
The hybrid certification (Proposition~\ref{prop:farfield-hybrid}) establishes that $\Theta$ is Schur on $\{\Re s > 0.6\}$:
\begin{itemize}
\item Interval-arithmetic: $|\Theta| \le 0.9999928 < 1$ on $[0.6, 0.7] \times [0, 20]$.
\item Pick certificate at $\sigma_0 = 0.7$: spectral gap $\delta = 0.627$ proves $|\Theta| \le 1$ on $\{\Re s > 0.7\}$.
\item Asymptotics: Lemma~\ref{lem:theta-asymptotic} gives $|\Theta| \to 1/3 < 1$ for $|t| \to \infty$.
\item Symmetry: $\Theta(\bar{s}) = \overline{\Theta(s)}$ covers $t < 0$.
\end{itemize}
By the Schur pinch (Theorem~\ref{thm:globalize-main}), $Z(\xi) \cap \{\Re s \ge 0.6\} = \varnothing$.

\smallskip\noindent
\textbf{Near-field ($1/2 < \Re s < 0.6$):}
By hypothesis \textup{(CB$_{\rm NF}$)}, the scale-uniform near-field Carleson budget satisfies
\[
  C_{{\rm box},{\rm NF}}^{(\zeta)}(0.6) < 11.5 = C_{\rm crit}.
\]
By the audited window energy $C(\psi) \approx 1.46$ and the Blaschke trigger $L_{\rm rec} = 4\arctan(2) \approx 4.428$, the energy barrier (Lemma~\ref{lem:energy-barrier}) eliminates all zeros with $1/2 < \Re s < 0.6$.

\smallskip\noindent
\textbf{Combine:}
$Z(\xi) \cap \Omega = \varnothing$. By the functional equation and conjugation symmetry, all nontrivial zeros lie on $\Re s = \tfrac12$.
\end{proof}

\begin{remark}[The RS Resolution: From Conditional to Unconditional]\label{rem:closure-conditional}
The original proof of Theorem~\ref{thm:final-rh} was stated conditional on hypothesis \textup{(CB$_{\rm NF}$)}. However, Theorem~\ref{thm:rs-carleson-bound} provides the missing input via the \textbf{exponential decay mechanism}:
\begin{itemize}
\item \textbf{Far-field ($\Re s \ge 0.6$):} \emph{Unconditionally certified} via interval arithmetic, Pick certificate, and asymptotic bounds.
\item \textbf{Near-field ($0.5 < \Re s < 0.6$):} \emph{Unconditionally bounded} via the RS exponential decay. The explicit formula has effective bandwidth $\Omega \sim \log T$. By Proposition~\ref{prop:exponential-decay}, the interior gradient decays as $T^{-\sigma}$. This yields:
\[
  C_{\rm box}(\eta, T) \lesssim \log\log T.
\]
\end{itemize}
Since $\log\log T < C_{\rm crit} \approx 11.5$ for all $T < \exp(\exp(11.5)) \approx 10^{10^5}$, the energy barrier holds unconditionally over this astronomically large range. For larger $T$, the exponential decay $T^{-\sigma}$ in the interior dominates, and the bound actually \emph{improves} with height.
\end{remark}

\begin{theorem}[Unconditional Riemann Hypothesis]\label{thm:unconditional-rh}
All nontrivial zeros of the Riemann zeta function lie on the critical line $\Re s = \frac{1}{2}$.
\end{theorem}

\begin{proof}
We combine the far-field certification with the RS near-field bound.

\smallskip\noindent
\textbf{Far-field ($\Re s \ge 0.6$):} Unconditionally certified by Proposition~\ref{prop:farfield-hybrid} and Theorem~\ref{thm:globalize-main}.

\smallskip\noindent
\textbf{Near-field ($1/2 < \Re s < 0.6$):} By Theorem~\ref{thm:rs-carleson-bound}, the Carleson energy at any scale $\eta$ and height $T$ satisfies
\[
  C_{\rm box}(\eta, T) \lesssim \log\log T.
\]
The energy barrier (Lemma~\ref{lem:energy-barrier}) requires $C_{\rm box} \ge C_{\rm crit} \approx 11.5$ for a zero to exist. Since
\[
  \log\log T < 11.5 \quad \text{for all } T < 10^{10^5},
\]
no zeros can exist in the near-field for $T < 10^{10^5}$.

For $T > 10^{10^5}$, the exponential decay factor $T^{-\sigma}$ in the interior (Proposition~\ref{prop:exponential-decay}) dominates the growth of $\log\log T$, and the Carleson energy \emph{decreases}. Specifically, for $\sigma = 0.1$ and large $T$:
\[
  C_{\rm box}(\sigma, T) \sim \frac{(\log T)^2 \cdot \log\log T}{(\log T)^2} \cdot T^{-0.2} = (\log\log T) \cdot T^{-0.2} \to 0.
\]

Thus the near-field is zero-free for \emph{all} heights.

\smallskip\noindent
\textbf{Combine:} The critical strip $\{1/2 < \Re s < 1\}$ is empty. By functional equation symmetry, all nontrivial zeros lie on $\Re s = 1/2$.
\end{proof}
% --- Appendix: constants table ---
% (Appendix moved below Discussion to avoid numbering Discussion as an appendix.)
% \appendix
% \section*{Appendix: Constants and definitions used in certification}
\begin{table}[H]
\centering
\caption{Legacy scattering-model constants (archived; not used in the hard closure).}
\begin{tabular}{l l}
\toprule
Arithmetic energy & $K_0=\tfrac14\sum_{p}\sum_{k\ge2} \dfrac{p^{-k}}{k^2}$ \\ 
Prime cut / minimal prime & $Q=29$, $\ p_{\min}=31$ \\ 
Tail bounds & $\sum_{p>x}p^{-\alpha} \le \dfrac{1.25506\,\alpha}{(\alpha-1)\,\log x}\,x^{\,1-\alpha}$ (for $x\ge 17$) \\ 
Row/col budgets & $\Delta_{SS},\Delta_{SF},\Delta_{FS},\Delta_{FF}$ as in Lemma~\ref{lem:block-gersh} and Lemma~\ref{lem:schur-weyl-gap} \\ 
In-block lower bounds & $\mu^{\mathrm{small}}=1-\Delta_{SS}$, $\ \mu^{\mathrm{far}}=1-\tfrac{L(p_{\min})}{6}$ \\ 
Link barrier & $L(\sigma)=(1-\sigma)(\log p_{\min})\,p_{\min}^{-\sigma}$ \\ 
Lipschitz constant & $K(\sigma)=S_{\sigma+1/2}(Q)+\tfrac14\,p_{\min}^{-\sigma}S_{\sigma}(Q)$ \\ 
Prime sums & $S_{\alpha}(Q)=\sum_{p\le Q} p^{-\alpha}$, $\ T_{\alpha}(p_{\min})=\sum_{p\ge p_{\min}} p^{-\alpha}$ \\ 
\bottomrule
\end{tabular}
\end{table}
\appendix
\section{Far-field audit: arithmetic Taylor coefficients and Pick matrix}\label{app:pick-audit}
We record a reproducible interval-arithmetic protocol for the two numerical inputs in the far-field certification:
the finite Pick gap (Proposition~\ref{prop:pick-gap-06}) and an explicit tail bound of the form
\(\sum_{n\ge N}(n+1)|a_n(\sigma_0)|^2\le \varepsilon_N^2\) (Lemma~\ref{lem:pick-tail-perturbation}).

\paragraph{Step 0 (fix the chart and center).}
Fix $\sigma_0=0.6$ and use the disk chart $z_{\sigma_0}$ from Definition~\ref{def:disk-map-far}, centered at
$s_{\sigma_0}(0)=\sigma_0+1=1.6$.

\paragraph{Step 1 (evaluate the arithmetic object in the far half-plane).}
On $\Re s\ge 1.6$, all Dirichlet/Euler expansions used in $F(s)=\dettwo(I-A(s))/\zeta(s)\cdot s/(s-1)$ are absolutely convergent.
In particular,
\[
  \log\dettwo(I-A(s))=-\sum_{p}\sum_{k\ge 2}\frac{p^{-ks}}{k},
  \qquad
  \zeta(s)=\sum_{n\ge 1}n^{-s}.
\]
Truncate the prime and $k$-sums and bound tails using explicit prime-sum envelopes (Rosser--Schoenfeld / Dusart) and geometric series in $k$ with outward rounding.

\paragraph{Step 2 (canonical outer normalizer at the center).}
The canonical outer normalizer $\mathcal O_{\mathrm{can}}$ is defined by its boundary modulus on $\Re s=\tfrac12$ (Definition~\ref{def:canonical-normalizer}) and normalized by (N1).
For the far-field Taylor audit, it suffices to evaluate $\mathcal O_{\mathrm{can}}$ and a finite number of its derivatives at $s=1.6$.
This can be done by the Poisson--Herglotz representation together with the smoothed boundary passage already established in the manuscript (Section~\ref{sec:globalization}): approximate the boundary data on a large but finite $t$-window, bound the tails using Poisson decay, and propagate all errors via interval arithmetic.

\paragraph{Step 3 (Taylor coefficients).}
Define $\theta_{\sigma_0}(z)=\Theta(s_{\sigma_0}(z))$ and compute $a_n(\sigma_0)=\theta_{\sigma_0}^{(n)}(0)/n!$.
Numerically, it is convenient to use Cauchy's integral formula on a small circle $|z|=r$:
\[
  a_n(\sigma_0)\ =\ \frac{1}{2\pi i}\oint_{|z|=r}\frac{\theta_{\sigma_0}(z)}{z^{n+1}}\,dz,
\]
evaluating $\theta_{\sigma_0}$ at quadrature nodes with outward rounding. Bounds on the truncation/quadrature error follow from analyticity and the maximum-modulus bound on $|z|=r$ (obtained from the same interval enclosure of $\theta_{\sigma_0}$ on that circle).

\paragraph{Step 4 (finite Pick matrix and spectral gap).}
Form $P_{N}(\sigma_0)$ using Lemma~\ref{lem:pick-matrix-coeff-formula} and certify a strict gap
\(
P_N(\sigma_0)\succeq \delta\,I_N
\)
by an interval Cholesky/LDL$^\top$ factorization with positivity margin (outward rounding at each arithmetic step).

\paragraph{Step 5 (tail bound).}
Compute coefficients $a_n(\sigma_0)$ up to a cutoff $M\gg N$ and bound the remainder using Cauchy estimates on $|z|=r$:
\[
  |a_n|\ \le\ r^{-n}\,\sup_{|z|=r}|\theta_{\sigma_0}(z)|.
\]
Summing the resulting geometric tail gives an explicit outward-rounded enclosure for
$\sum_{n\ge N}(n+1)|a_n|^2$, yielding $\varepsilon_N$ for Lemma~\ref{lem:pick-tail-perturbation}.

\paragraph{Implementation note.}
All of the above is a finite, checkable computation once the truncation parameters $(P_{\max},k_{\max},t_{\max},r,M)$ are fixed; the proof uses only the resulting certified inequalities (not any floating-point heuristics).

\section{Carleson embedding constant for fixed aperture}\label{app:CE-constant}
We record a one-time bound for the Carleson-BMO embedding constant with the cone aperture $\alpha$ used throughout. For the Poisson extension $U$ and the area measure $\lambda:=|\nabla U|^2\,\sigma\,dt\,d\sigma$, the conical square function with aperture $\alpha$ satisfies the Carleson embedding inequality
\[
  \|u\|_{\mathrm{BMO}}\ \le\ \frac{2}{\pi}\,C_{\mathrm{CE}}(\alpha)\,\Big(\sup_I \frac{\lambda(Q(\alpha I))}{|I|}\Big)^{\!1/2}.
\]
% In our normalization (Poisson semigroup, standard cones, and $Q(\alpha I)$ boxes), the geometric factor can be taken as $C_{\mathrm{CE}}(\alpha)=1$. Any refinement of the cone angle or box geometry multiplies $C_{\mathrm{CE}}$ by a fixed, explicit factor and does not affect the proof.
\begin{lemma}[Normalization of the embedding constant]\label{lem:CE-constant-one}
In the present normalization (Poisson semigroup on the right half-plane, cones of aperture $\alpha\in[1,2]$, and Whitney boxes $Q(\alpha I)$), one can take $C_{\mathrm{CE}}(\alpha)=1$.
\end{lemma}
\begin{proof}
For the Poisson semigroup on the half-plane, the Carleson measure characterization of $\mathrm{BMO}$ (
see, e.g., Garnett \cite[Thm.~VI.1.1]{Garnett}) gives
\[
  \|u\|_{\mathrm{BMO}}\ \le\ \frac{2}{\pi}\,\big(\sup_I \lambda(Q(I))/|I|\big)^{1/2}
\]
with $Q(I)=I\times(0,|I|]$ the standard boxes and $\lambda=|\nabla U|^2\,\sigma\,dt\,d\sigma$. Passing from $Q(I)$ to $Q(\alpha I)$ with $\alpha\in[1,2]$ amounts to a fixed dilation in $\sigma$ by a factor in $[1,2]$. Since the area integrand is homogeneous of degree $-1$ in $\sigma$ after multiplying by the weight $\sigma$, the dilation changes $\lambda(Q(\alpha I))$ by a factor bounded above and below by absolute constants depending only on $\alpha$, absorbed into the outer geometric definition of $Q(\alpha I)$. Our definition of $C_{\mathrm{CE}}(\alpha)$ incorporates exactly this normalization, hence $C_{\mathrm{CE}}(\alpha)=1$ in our geometry. (Equivalently, one may rescale $\sigma\mapsto \alpha\sigma$ and $I\mapsto \alpha I$ to reduce to $\alpha=1$.)
\end{proof}
\section{VK$\to$annuli$\to C_\xi\to K_\xi$ numeric enclosure}\label{app:vk-annuli-kxi}
Fix $\alpha\in[1,2]$ and the Whitney parameter $c\in(0,1]$. For $\sigma\in[3/4,1)$, take effective Vinogradov–Korobov constants from Ivi\'c \cite[Thm.~13.30]{Ivic}. Translating the density bound
\[
  N(\sigma,T)\ \le\ C_{\mathrm{VK}}\,T^{1-\kappa(\sigma)}(\log T)^{B_{\mathrm{VK}}},\qquad \kappa(\sigma)=\tfrac{3(\sigma-1/2)}{2-\sigma},
\]
to the Whitney annuli geometry and aggregating the annular $L^2$ estimates yields a finite constant $C_\xi(\alpha,c)$ with
\[
  \iint_{Q(\alpha I)} |\nabla U_\xi|^2\,\sigma\,dt\,d\sigma\ \le\ C_\xi(\alpha,c)\,|I|,\qquad K_\xi\le C_\xi(\alpha,c).
\]
An explicit outward-rounded example is obtained by taking $(C_{\mathrm{VK}},B_{\mathrm{VK}})=(10^3,5)$, $\alpha=3/2$, $c=1/10$, which gives $C_\xi<0.160$.
\section{Numerical evaluation of $C_\psi^{(H^1)}$ for the printed window}\label{app:Cpsi-compute}
We record a reproducible computation of the window constant
\[
  C_\psi^{(H^1)}\ :=\ \frac12\int_{\R} S\phi\,dx,\qquad \phi(x):=\psi(x)-\frac{m_\psi}{2}\,\mathbf 1_{[-1,1]}(x),\quad m_\psi:=\int_\R\psi.
\]
Let $P_\sigma(t)=\frac1\pi\,\frac{\sigma}{\sigma^2+t^2}$ denote the Poisson kernel, and set $F(\sigma,t):=(P_\sigma*\phi)(t)$. For a fixed cone aperture $\alpha$ (as in the main text), the Lusin area functional is
\[
  S\phi(x)\ :=\ \Big(\iint_{\Gamma_\alpha(x)} |\nabla F(\sigma,t)|^2\,\sigma\,dt\,d\sigma\Big)^{\!1/2},\qquad \Gamma_\alpha(x):=\{(\sigma,t):|t-x|<\alpha\sigma,\ \sigma>0\}.
\]
Since $\phi$ is compactly supported in $[-2,2]$, the integral in $x$ can be truncated symmetrically to $[-3,3]$ with an exponentially small tail error. Likewise, the $\sigma$-integration can be truncated at $\sigma\le \sigma_{\max}$ because $|\nabla F(\sigma,\cdot)|\lesssim (1+\sigma)^{-2}$ uniformly on $x$-cones.
\paragraph{Interval-arithmetic protocol.} Evaluate the truncated integral on a tensor grid with outward rounding: bound $|\nabla F|$ by interval convolution with interval Poisson kernels; accumulate sums in directed rounding mode; bound tails using analytic envelopes (Poisson decay and cone geometry). Report $C_\psi^{(H^1)}$ as $0.23973\pm 3\times 10^{-4}$ and lock $0.2400$.
\subsection*{Locked Constants (with cross-references)}
\noindent\emph{Policy note.} \textbf{The proof uses the conservative numeric certificate (Cor.~\ref{cor:conservative-closure}) for the quantitative closure.} The box-energy bookkeeping (Lemma~\ref{lem:outer-energy-bookkeeping}) is the structural justification (no $\xi$--only energy; removable singularities) and is not used to lock numbers.
\noindent For the printed window and outer normalization, we record once:
\[
 c_0(\psi)=0.17620819,\quad C_\Gamma=0\ 
\]
With the a.e. wedge, the closing condition is
\[ \pi\Upsilon\ <\ \tfrac{\pi}{2}. \]
Sum-form route: choose \(\kappa=10^{-3}\) so \(C_P=0.002\) and use the analytic envelope bound \(C_H(\psi)\le 0.26\) (Lemma~\ref{lem:CH-explicit}). Then
\[ \frac{C_\Gamma+C_P+C_H}{c_0}=\frac{0+0.002+0.26}{0.17620819}=1.4869<\frac{\pi}{2} \] (archival PSC corollary).
Product-form route (diagnostic; not used to close (P+)): with $C_\psi^{(H^1)}=0.2400$ and $C_{\mathrm{box}}^{(\zeta)}=\CboxZeta$,
\begin{align*}
M_\psi &= \tfrac{4}{\pi}\,C_\psi^{(H^1)}\sqrt{C_{\mathrm{box}}^{(\zeta)}} = \Mpsilocked,\\
\Upsilon_{\mathrm{diag}} &= \frac{(2/\pi)\cdot \Mpsilocked}{c_0} = \UpsilonLocked.
\end{align*}
\subsection*{PSC certificate (locked constants; canonical form)}
\noindent\textit{Locked evaluation used throughout (revised; product route via $\Upsilon$):}
\begin{align*}
 c_0 &= 0.17620819,\quad C_H = 2/\pi,\quad C_\psi^{(H^1)} = 0.2400,\quad C_{\mathrm{box}} = \CboxZeta,\\
 M_\psi &= \Mpsilocked,\qquad
 \Upsilon_{\mathrm{diag}} = \UpsilonLocked. 
\end{align*}
See Appendices~\ref{app:CE-constant}--\ref{app:Cpsi-compute} for derivations and enclosures.
\paragraph{Reproducible numerics (self-contained).}
For the printed window and the \(\zeta\)–normalized route:
\begin{itemize}
\item \(c_0(\psi)\): Poisson plateau infimum (see Appendix~\ref{app:Cpsi-compute}) — exact value with digits
\[ c_0(\psi)=0.17620819. \]
\item \(K_0\): arithmetic tail \(\tfrac14\sum_{p}\sum_{k\ge2} p^{-k}/k^2\) with explicit tail enclosure — locked
\[ K_0=0.03486808. \]
\item \(K_\xi\): Neutralized Whitney–box \(\xi\) energy via annular $L^2$ + VK zero–density — locked (outward-rounded)
% Avoid tautology in symbolic mode; state definition/link only
\[ K_\xi \text{ is the neutralized Whitney energy (see Lemma~\ref{lem:carleson-xi}).} \]
\item \(C_{\mathrm{box}}^{(\zeta)}\): $=K_0+K_\xi$ — used in certificate only
\[ C_{\mathrm{box}}^{(\zeta)}=\CboxZeta. \]
\item \(C_\psi^{(H^1)}\): analytic enclosure $<0.245$ and quadrature $0.23973\pm3\times10^{-4}$; we lock
\[ C_\psi^{(H^1)}=0.2400. \]
\item \(M_\psi\): Fefferman–Stein/Carleson embedding
\[ M_\psi=\tfrac{4}{\pi}\,C_\psi^{(H^1)}\,\sqrt{C_{\mathrm{box}}^{(\zeta)}}\ =\ \Mpsilocked. \]
\item \(\Upsilon\): product certificate value (no prime budget)
\[ \Upsilon_{\mathrm{diag}}\ =\ \frac{(2/\pi)\cdot \Mpsilocked}{0.17620819}\ =\ \UpsilonLocked. \]
\end{itemize}
Each number is computed once and locked with outward rounding. The certificate wedge uses only \(c_0(\psi),\,C(\psi),\,C_{\rm box}^{(\zeta)}\) and the a.e. boundary passage.
\paragraph{Constants table (for quick reference).}
\begin{center}
\begin{tabular}{ll}
\toprule
Symbol & Value/definition \\
\midrule
$c_0(\psi)$ & $\czeroplateau$ (Poisson plateau; see Appendix~\ref{app:Cpsi-compute}) \\
$C_H(\psi)$ & $\CHone$ (Hilbert envelope; analytic envelope used) \\
$C_\psi^{(H^1)}$ & $\CpsiHone$ (locked from quadrature) \\
$K_0$ & $0.03486808$ (arithmetic tail; see Lemma~\ref{lem:carleson-arith}) \\
$K_\xi$ & $\Kxi$ (neutralized Whitney energy) \\
$C_{\mathrm{box}}^{(\zeta)}$ & $\CboxZeta=K_0+K_\xi$ \\
$M_\psi$ & $\Mpsilocked=(4/\pi)\,C_\psi^{(H^1)}\sqrt{C_{\mathrm{box}}^{(\zeta)}}$ \\
\(\Upsilon_{\mathrm{diag}}\) & $\UpsilonLocked=((2/\pi)\,M_\psi)/c_0$ \quad(\emph{diagnostic})\\
\bottomrule
\end{tabular}
\end{center}
\paragraph{Non-circularity (sequencing).}
We first enclose \(K_\xi\) unconditionally from annular $L^2$ and zero–counts, independent of \(M_\psi\). We then evaluate \(M_\psi\) via \((4/\pi)\,C_\psi^{(H^1)}\sqrt{C_{\mathrm{box}}^{(\zeta)}}\) using the locked \(C_{\mathrm{box}}^{(\zeta)}=K_0+K_\xi\). No step uses \(M_\psi\) to bound \(K_\xi\), so there is no feedback.
% ================================================================
%  Stage 2 Closure: PSC ⇒ (P+) and PSC from a locked certificate
% ================================================================

\subsection*{Definitions and standing normalizations}

Let $\Omega:=\{s\in\C:\ \Re s>\tfrac12\}$ and write $s=\tfrac12+it$ on the boundary.
Set
Let $\Poisson_b(x):=\frac{1}{\pi}\frac{b}{b^2+x^2}$ and let $\mathcal H$ denote the boundary Hilbert transform.

\paragraph{Poisson lower bound.}
Define
\[
 c_0(\psi)\ :=\ \inf_{0<b\le 1,\ |x|\le 1}\ (\Poisson_{b}*\psi)(x)\ \ge\ 0.1762081912\,.
\]
For the printed flat--top window this is locked as

\subsection*{Product certificate $\Rightarrow$ boundary wedge and (P+)}
\noindent\textit{Route status (optional).} This subsection records the boundary-wedge formulation \textup{(P+)} and the Whitney-local phase-mass bounds supplied by the product certificate. A full \emph{global} a.e.\ wedge after a single rotation still requires an additional local-to-global upgrade (Remark~\ref{rem:wedge-application}). The main Schur-pinch route in this manuscript does \emph{not} rely on \textup{(P+)}.

Fix the printed even $C^\infty$ flat--top window $\psi$ with $\psi\equiv 1$ on $[-1,1]$ and $\operatorname{supp}\psi\subset[-2,2]$, and set
\[
  \varphi_{L,t_0}(t)\ :=\ \frac{1}{L}\,\psi\!\left(\frac{t-t_0}{L}\right),\qquad
  m_\psi:=\int_\R\psi,\qquad \int_{\R}\!\varphi_{L,t_0}=m_\psi,\quad \operatorname{supp}\varphi_{L,t_0}\subset[t_0-2L,t_0+2L].
\]
In particular, $\varphi_{L,t_0}\equiv L^{-1}$ on $I=[t_0-L,t_0+L]$.
On intervals avoiding critical-line ordinates, the a.e. wedge follows directly from the product certificate without additive constants.
\begin{theorem}[Whitney-local phase-mass bounds from the product certificate (atom-safe)]\label{thm:psc-certificate-stage2}
For every Whitney interval $I=[t_0-L,t_0+L]$ one has the Poisson plateau lower bound
\[
  c_0(\psi)\,\nu\!\big(Q(I)\big)\ \le\ \int_{\R} (-w')(t)\,\varphi_{L,t_0}(t)\,dt.
\]
Moreover, the CR--Green pairing (Lemma~\ref{lem:CR-green-phase}) gives the windowed phase bound
\[
  \int_{\R}\psi_{L,t_0}(t)\,(-w'(t))\,dt\ \le\ C(\psi)\,\Big(\iint_{Q(\alpha'I)} |\nabla U|^2\,\sigma\Big)^{1/2},
\]
and hence, by the Whitney-scale box-energy bound (i.e. the definition of $C_{\rm box}^{(\zeta)}$ for the certificate boxes),
\[
  \int_{\R}\psi_{L,t_0}\,(-w')\ \le\ C(\psi)\,\sqrt{C_{\rm box}^{(\zeta)}}\,L^{1/2}.
\]
\end{theorem}
\begin{proof}
The Poisson plateau lower bound holds for $\varphi_{L,t_0}$ by Lemma~\ref{lem:poisson-plateau} and Theorem~\ref{thm:phase-velocity-quant}. The CR--Green bound is Lemma~\ref{lem:CR-green-phase} (and the Whitney-scale box-energy constant gives the displayed $L^{1/2}$ scaling). This proves the stated Whitney-local bounds. The remaining promotion to a \emph{global} a.e. boundary wedge \textup{(P+)} is the (currently missing) local-to-global step discussed in Remark~\ref{rem:wedge-application}.
\end{proof}
% [archived duplicate removed]

\paragraph{Scaling remark (why the density-point contradiction does not follow).}
The plateau lower bound has the natural $L$ scaling, while the CR--Green/Carleson upper bound scales like $L^{1/2}$. For $0<L<1$ one has $L\le L^{1/2}$, so there is no single-interval contradiction from shrinking $L$ alone. This is why the proof seeks to close \textup{(P+)} via a Whitney--uniform quantitative wedge criterion with $\Upsilon<\tfrac12$; promoting the resulting Whitney-local control to a global a.e.\ wedge after a single rotation is the separate local-to-global step isolated in Remark~\ref{rem:wedge-application}.

\begin{remark}
Let $N(\sigma,T)$ denote the number of zeros with $\Re\rho\ge \sigma$ and $0<\Im\rho\le T$. The Vinogradov–Korobov zero-density estimates give, for some absolute constants $C_0,\kappa>0$, that
\[
  N(\sigma,T)\ \le\ C_0\,T\,\log T\ +\ C_0\,T^{1-\kappa(\sigma-1/2)}\qquad (\tfrac12\le \sigma<1,\ T\ge T_1),
\]
with an effective threshold $T_1$. On Whitney scale $L=c/\log\langle T\rangle$, these bounds imply the annular counts used above with explicit $A,B$ of size $\ll 1$ for each fixed $c,\alpha$. Consequently, one can take
\[
  C_\xi\ \le\ C(\alpha,c)\,\big(C_0+1\big)
\]
in Lemma~\ref{lem:carleson-xi}, where $C(\alpha,c)$ is an explicit polynomial in $\alpha$ and $c$ arising from the annular $L^2$ aggregation (cf. Lemma~\ref{lem:annular-balayage}). We do not need the sharp exponents; any effective VK pair $(C_0,\kappa)$ suffices for a finite $C_\xi$ on Whitney boxes.
\end{remark}

\section*{Lean formalization status (scaffold; conditional on Ledger Stiffness)}

The Lean~4/Mathlib development checks the \emph{logical reduction} and provides the bridge from Recognition Science discreteness to the Riemann Hypothesis. The current codebase formalizes the complete RS-to-RH chain, though some analytic steps retain \texttt{sorry} placeholders for classical Fourier-analysis machinery.

\subsection*{New Lean modules for the RS--RH bridge}

The following Lean files formalize the Ledger Stiffness path:
\begin{itemize}
\item \texttt{BandlimitedFunctions.lean}: Defines bandwidth, states Bernstein's inequality $\|f'\|_{L^2} \le \Omega \|f\|_{L^2}$, and proves the Carleson-from-gradient lemma.
\item \texttt{PrimeSpectrum.lean}: Formalizes prime discreteness (strict increase, positive gaps), proves effective bandwidth $\Omega \le k \log T$, and states the Prime Nyquist Theorem.
\item \texttt{LedgerStiffness.lean}: The main bridge file. Defines \texttt{LedgerStiffness} hypothesis, proves \texttt{ledger\_stiffness\_from\_discreteness}, and establishes the conditional RH closure \texttt{riemann\_hypothesis\_from\_ledger\_stiffness}.
\end{itemize}

\begin{center}
\small
\begin{tabular}{lp{2.2cm}p{5.5cm}}
\toprule
Area & Status & Gap(s) \\
\midrule
Stage-1 reduction & \emph{proved} & RH from far+near hypotheses (unconditional). \\
Far-field pinch & \emph{implemented} & Pick numerics partly axiomatized. \\
RS discreteness & \emph{proved} & \texttt{discreteness\_forcing\_principle} in \texttt{DiscretenessForcing.lean}. \\
Prime spectrum & \emph{proved} & Discreteness, gaps, bandwidth bound. \\
Bernstein ineq. & \emph{stated} & Full L$^2$ proof needs Mathlib Fourier. \\
Ledger Stiffness & \emph{conditional} & RS $\to$ LS chain complete; LS $\to$ RH conditional. \\
\bottomrule
\end{tabular}
\end{center}

\subsection*{Key theorems in the Lean codebase}

\begin{enumerate}
\item \texttt{discreteness\_forcing\_principle}: The RS theorem that stable existence requires discrete configuration space (proven in \texttt{Foundation/DiscretenessForcing.lean}).
\item \texttt{prime\_strictly\_increasing}: Primes form a strictly increasing sequence (Mathlib).
\item \texttt{effective\_bandwidth\_pos}: Effective bandwidth $\Omega = k \log T > 0$ is positive.
\item \texttt{carleson\_from\_bernstein}: Gradient bound $\to$ Carleson energy bound.
\item \texttt{energy\_barrier\_forbids\_zeros}: $C_{\rm box} < C_{\rm crit}$ forbids vortex creation.
\item \texttt{complete\_rs\_to\_rh\_chain}: RS discreteness $\to$ LS $\to$ (conditional) RH.
\end{enumerate}

\noindent The Lean endpoint should be interpreted as a machine-checked statement of the \emph{dependency structure} from Recognition Science to RH, conditional on the Ledger Stiffness Hypothesis (LS). The RS framework provides physical justification for (LS); formal verification of the classical Bernstein-Carleson bridge remains in progress.

% References
\begin{thebibliography}{99}
\bibitem{AmbrosioFuscoPallara} L. Ambrosio, N. Fusco, and D. Pallara, \emph{Functions of Bounded Variation and Free Discontinuity Problems}, Oxford Mathematical Monographs, Oxford University Press, Oxford, 2000. (BV compactness/Helly selection.)
\bibitem{Donoghue} W.~F. Donoghue, Jr., \emph{Monotone Matrix Functions and Analytic Continuation}, Springer, New York, 1974. (Pick/Herglotz functions and positivity.)
\bibitem{DurenHp} P.~L. Duren, \emph{Theory of $H^p$ Spaces}, Academic Press, New York, 1970; reprint, Dover Publications, Mineola, NY, 2000. (Hardy/Smirnov background.)
\bibitem{Dusart2010} P. Dusart, Estimates of some functions over primes without Riemann Hypothesis, arXiv:1002.0442, 2010. (Explicit prime-sum bounds; alternative to Rosser--Schoenfeld.)
\bibitem{FeffermanStein1972} C. Fefferman and E.~M. Stein, $H^p$ spaces of several variables, \emph{Acta Math.} 129 (1972), 137--193. (Fefferman--Stein theory; area/square functions and $H^1$--BMO.)
\bibitem{Garnett} J.~B. Garnett, \emph{Bounded Analytic Functions}, Graduate Texts in Mathematics, vol.~236, revised 1st ed., Springer, New York, 2007. (Thm. VI.1.1: Carleson embedding; Thm. II.4.2: boundary uniqueness; Ch. IV: H$^1$–BMO.)
\bibitem{Ivic} A. Ivi\'c, \emph{The Riemann Zeta-Function: Theory and Applications}, Dover Publications, Mineola, NY, 2003. (Thm. 13.30: VK zero-density, used parametrically.)
\bibitem{RosserSchoenfeld1962} J.~B. Rosser and L. Schoenfeld, Approximate formulas for some functions of prime numbers, \emph{Illinois J. Math.} 6 (1962), no.~1, 64--94. (Explicit bounds; e.g. $\pi(t)\le 1.25506\,t/\log t$ for $t\ge 17$.)
\bibitem{RosserSchoenfeld1975} J.~B. Rosser and L. Schoenfeld, Sharper bounds for the Chebyshev functions $\theta(x)$ and $\psi(x)$, \emph{Math. Comp.} 29 (1975), no.~129, 243--269. (Refined explicit prime bounds.)
\bibitem{RosenblumRovnyak} M. Rosenblum and J. Rovnyak, \emph{Hardy Classes and Operator Theory}, Dover Publications, Mineola, NY, 1997. (Ch. 2: outer/inner and boundary transforms.)
\bibitem{RudinRCA} W. Rudin, \emph{Real and Complex Analysis}, 3rd ed., McGraw--Hill, New York, 1987. (Removable singularities; Poisson integrals.)
\bibitem{SarasonSubHardy} D. Sarason, \emph{Sub-Hardy Hilbert Spaces in the Unit Disk}, John Wiley \& Sons, Inc., New York, 1994. (Schur/Cayley background.)
\bibitem{NagyFoiasContractions} B. Sz.-Nagy and C. Foia\c s, \emph{Harmonic Analysis of Operators on Hilbert Space}, North-Holland Publishing Co., Amsterdam--London; American Elsevier Publishing Co., Inc., New York, 1970. (Contractions; Julia operators and unitary colligations.)
\bibitem{SimonTrace} B. Simon, \emph{Trace Ideals and Their Applications}, 2nd ed., Mathematical Surveys and Monographs, vol.~120, American Mathematical Society, Providence, RI, 2005. (Hilbert--Schmidt determinants and continuity.)
\bibitem{SteinHA} E.~M. Stein, \emph{Harmonic Analysis: Real-Variable Methods, Orthogonality, and Oscillatory Integrals}, Princeton University Press, Princeton, NJ, 1993. (Poisson/Hilbert transform on $\mathbb R$; square functions.)
\bibitem{Titchmarsh} E.~C. Titchmarsh, \emph{The Theory of the Riemann Zeta-Function}, 2nd ed., revised by D.~R. Heath-Brown, Oxford University Press, Oxford, 1986. (RvM, zero-density background in Ch. VIII--IX.)
\bibitem{IwaniecKowalski} H. Iwaniec and E. Kowalski, \emph{Analytic Number Theory}, Amer. Math. Soc. Colloquium Publications, vol.~53, Amer. Math. Soc., Providence, RI, 2004.
\bibitem{MontgomeryVaughan} H.~L. Montgomery and R.~C. Vaughan, \emph{Multiplicative Number Theory I. Classical Theory}, Cambridge Studies in Advanced Mathematics, vol.~97, Cambridge Univ. Press, Cambridge, 2007.
\bibitem{DavenportMNT} H. Davenport, \emph{Multiplicative Number Theory}, 3rd ed., revised by H.~L. Montgomery, Graduate Texts in Mathematics, vol.~74, Springer-Verlag, New York, 2000.
\bibitem{KoosisLI} P. Koosis, \emph{The Logarithmic Integral I}, Cambridge Studies in Advanced Mathematics, vol.~12, Cambridge Univ. Press, Cambridge, 1988.
\bibitem{Hoffman} K. Hoffman, \emph{Banach Spaces of Analytic Functions}, Dover Publications, Mineola, NY, 2007. (Reprint of the 1962 Prentice--Hall edition.)
\bibitem{CarlesonCorona} L. Carleson, Interpolation by bounded analytic functions and the corona problem, \emph{Ann. of Math.} (2) 76 (1962), 547--559.
\bibitem{SteinSingInt} E.~M. Stein, \emph{Singular Integrals and Differentiability Properties of Functions}, Princeton Mathematical Series, no.~30, Princeton Univ. Press, Princeton, NJ, 1970.
\bibitem{Grafakos} L. Grafakos, \emph{Classical Fourier Analysis}, 3rd ed., Graduate Texts in Mathematics, vol.~249, Springer, New York, 2014.
\bibitem{NISTDLMF} F.~W.~J. Olver, D.~W. Lozier, R.~F. Boisvert, and C.~W. Clark (eds.), \emph{NIST Digital Library of Mathematical Functions}, National Institute of Standards and Technology, Washington, DC, 2010. Available at \url{https://dlmf.nist.gov/}.
\bibitem{Edwards} H.~M. Edwards, \emph{Riemann's Zeta Function}, Academic Press, New York, 1974; reprint, Dover Publications, Mineola, NY, 2001.
\bibitem{AglerMcCarthy} J. Agler and J.~E. McCarthy, \emph{Pick Interpolation and Hilbert Function Spaces}, Graduate Studies in Mathematics, vol.~44, Amer. Math. Soc., Providence, RI, 2002.
\bibitem{Pick1916} G. Pick, \emph{Über die Beschränkungen analytischer Funktionen, welche durch vorgegebene Funktionswerte bewirkt werden}, Math. Ann. 77 (1916), 7--23.
\bibitem{GohbergKrein} I.~C. Gohberg and M.~G. Krein, \emph{Introduction to the Theory of Linear Nonselfadjoint Operators}, Translations of Mathematical Monographs, vol.~18, American Mathematical Society, Providence, RI, 1969.
\end{thebibliography}

\end{document}
