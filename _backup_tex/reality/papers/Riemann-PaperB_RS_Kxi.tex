\documentclass[11pt]{article}
\usepackage[margin=1in]{geometry}
\usepackage{amsmath,amssymb,amsthm,mathtools}
\usepackage{microtype}
\usepackage{hyperref}
\usepackage[numbers,sort&compress]{natbib}
\hypersetup{colorlinks=true,linkcolor=black,citecolor=black,urlcolor=black}

% Theorem environments
\newtheorem{theorem}{Theorem}
\newtheorem{proposition}[theorem]{Proposition}
\newtheorem{lemma}[theorem]{Lemma}
\newtheorem{corollary}[theorem]{Corollary}
\newtheorem{hypothesis}[theorem]{Hypothesis}
\newtheorem{axiomenv}[theorem]{Axiom}
\theoremstyle{remark}
\newtheorem{remark}[theorem]{Remark}

% Macros
\newcommand{\C}{\mathbb C}
\newcommand{\R}{\mathbb R}
\newcommand{\N}{\mathbb N}
\newcommand{\Half}{\{\,s\in\C:\ \Re s>\tfrac12\,\}}
\DeclareMathOperator{\Rea}{Re}
\DeclareMathOperator{\Imz}{Im}
\DeclareMathOperator{\Arg}{Arg}
\DeclareMathOperator{\dettwo}{det_2}

\title{Recognition Science closes the far-field attachment gate\\
\large (Paper B in a two-paper Recognition Science proof of the Riemann Hypothesis)}
\author{Jonathan Washburn\\ \href{mailto:washburn.jonathan@gmail.com}{washburn.jonathan@gmail.com}}
\date{December 2025}

\begin{document}
\maketitle

\begin{abstract}
Paper~A \cite{WashburnChristmas2025} gives a two-regime route to the Riemann Hypothesis (RH).
The near strip $1/2<\Re s<\sigma_0$ (with $\sigma_0=0.6$) is eliminated unconditionally by a
recognition-geometry/B2$'$ signal$>$noise contradiction. The far strip $\Re s\ge\sigma_0$ is eliminated
provided one supplies a single missing analytic inequality: a quantitative \emph{attachment-with-margin}
estimate comparing the arithmetic approximant $\mathcal J_N$ with a finite passive certificate transfer
function $\mathcal J_{\mathrm{cert},N}$ on each far-field zero-free rectangle.

Assuming Recognition Science (RS), we translate this remaining inequality into a physical correlation:
it is an impedance-matching (gap-vs-perturbation) statement asserting that the prime event-stream cannot
inject enough mismatch to break a passive, gapped ledger certificate. We then repackage the attachment gate
as one crisp classical conjecture/lemma and decompose it into separately provable targets (det$_2$ continuity,
prime-tail/window budgets, and a single ``arithmetic deformation'' identification statement). This produces a
precise next objective for either a classical proof effort or an RS$\Rightarrow$RH bridge theorem.
\end{abstract}

\paragraph{Keywords.} Riemann zeta function; Herglotz/Schur functions; passive systems; Hilbert--Schmidt determinants; prime tails; Recognition Science.

\section{Reader's guide}

\begin{itemize}
  \item \textbf{Paper A (Christmas route).} \cite{WashburnChristmas2025} proves RH from a two-regime closure.
  The only missing far-field step is the attachment-with-margin inequality \eqref{eq:attachment} below.

  \item \textbf{This paper (Paper B).} We focus on the single missing far-field inequality and give:
  (i) the one-line statement to attack next (Hypothesis~\ref{hyp:attachment}),
  (ii) its RS interpretation as a physical correlation, and
  (iii) a classical decomposition into sublemmas that can be proved independently.

  \item \textbf{Relation to the older active BRF route.} The active boundary route \cite{WashburnActive2025}
  isolates a different bottleneck (a scale-free Whitney $\xi$--energy constant $K_\xi$). The Christmas route
  is written to avoid the global wedge obstruction and instead isolates \eqref{eq:attachment} as the far-field gate.
\end{itemize}

\section{The far-field attachment gate (the remaining gap)}\label{sec:gap}

We follow the notational conventions of \cite{WashburnChristmas2025}. Paper~A constructs:
\begin{itemize}
  \item a finite passive certificate transfer function $\mathcal J_{\mathrm{cert},N}$ on $\{\Re s>\sigma_0\}$,
  produced from a finite-block positivity/spectral gap certificate via a unitary colligation realization; and
  \item an arithmetic approximant $\mathcal J_N$ derived from primes (via $\dettwo$ and an outer normalization),
  which shares the non-cancellation property at $\xi$-zeros.
\end{itemize}

The far-field chain is then purely classical \emph{once} one knows that $\mathcal J_N$ is uniformly close to
$\mathcal J_{\mathrm{cert},N}$ with margin on each zero-free rectangle.

\begin{hypothesis}[Attachment-with-margin on far-field rectangles]\label{hyp:attachment}
Fix $\sigma_0\in(\tfrac12,1]$ and let $R\Subset\{\,\Re s>\sigma_0\,\}$ be a rectangle with $\xi\neq 0$ on a
neighborhood of $\overline R$. Define
\[
  m_R\ :=\ \inf_{s\in \overline R}\Rea\bigl(2\mathcal J_{\mathrm{cert},N}(s)\bigr)\ >\ 0.
\]
Assume the quantitative attachment bound
\begin{equation}\label{eq:attachment}
  \sup_{s\in \overline R}\Big|\mathcal J_N(s)-\mathcal J_{\mathrm{cert},N}(s)\Big|
  \ \le\ \frac{m_R}{4}.
\end{equation}
\end{hypothesis}

\begin{remark}[Why this is the right ``one-line'' target]
Hypothesis~\ref{hyp:attachment} is exactly \cite[Eq.~(attachment) and Lemma~(attachment-identity)]{WashburnChristmas2025}.
Once it is available on the rectangles used in the far-strip pinch, \cite{WashburnChristmas2025} closes RH.
\end{remark}

\section{RS translation: what physical correlation is \eqref{eq:attachment}?}\label{sec:rs-translation}

Assume RS is an accurate description of reality. Then the far-field objects have the following physical reading:
\begin{itemize}
  \item $\mathcal J_{\mathrm{cert},N}$ is a \emph{passive, lossless} finite system (a gapped ledger certificate).
  Passivity is the analytic proxy for conservation/no-exported-harm.
  \item $\mathcal J_N$ is the response of the \emph{arithmetic write-head}: the prime event-stream coupled into the same channel,
  observed through the same normalization.
\end{itemize}

\noindent In this dictionary, \eqref{eq:attachment} is the statement
\[
  \text{(mismatch amplitude)}\ <\ \text{(stability margin)}.
\]
It is an \emph{impedance match} condition: the ledger certificate has a robustness margin $m_R$, and the primes can only inject
a bounded perturbation. If the perturbation is below the margin, the system cannot cross the passivity boundary
$\Rea(2\mathcal J)=0$ on the far strip.

\section{What the answer is under RS (and how to decompose it classically)}\label{sec:decompose}

Under RS, the expected answer is that Hypothesis~\ref{hyp:attachment} holds: ledger locality + conservation + finite resolution prevent
long-range phase drift from accumulating into a far-field passivity failure.

Classically, Hypothesis~\ref{hyp:attachment} decomposes into the following independently checkable problems:

\medskip
\noindent\textbf{(A) Certificate margin.}
Lower bound $m_R$ on the far-field rectangles of interest. This is certificate-side (finite-block positivity and passive realization).

\medskip
\noindent\textbf{(B) Determinant continuity (pure analysis).}
Prove a uniform continuity/Lipschitz estimate for $\dettwo(I-A)$ with respect to Hilbert--Schmidt perturbations, as in
\cite[Prop.~(hs-det2-continuity)]{WashburnActive2025}.

\medskip
\noindent\textbf{(C) Prime-tail and window-leakage budgets (explicit inequalities).}
Bound the truncation and leakage contributions using explicit prime-sum bounds (e.g. Rosser--Schoenfeld/Dusart style tails),
as recorded in \cite[Lemma~(attachment-error-decomp)]{WashburnChristmas2025}.

\medskip
\noindent\textbf{(D) The true boss: arithmetic deformation / identification.}
Supply a theorem explaining \emph{why} the arithmetic approximant tracks the certificate transfer function uniformly on the far strip
with an error small compared to $m_R$. This is the only place where genuinely new mathematics (or RS input) is expected to live.

\begin{remark}[A crisp classical conjecture to attack next]
The core task can be stated without RS as:
\emph{prove Hypothesis~\ref{hyp:attachment} on the far-field rectangles needed in \cite{WashburnChristmas2025}.}
Equivalently, prove that the arithmetic approximant $\mathcal J_N$ is uniformly close to a passive certificate transfer function
with a uniform positive-real margin on those rectangles.
\end{remark}

\section{Main conclusion}

\begin{theorem}[RH under RS, via the Christmas route]\label{thm:rh-under-rs}
Assume RS implies Hypothesis~\ref{hyp:attachment} on the far-field rectangles required by \cite{WashburnChristmas2025}.
Then the Riemann Hypothesis holds.
\end{theorem}

\begin{proof}
By \cite{WashburnChristmas2025}, the near strip $1/2<\Re s<\sigma_0$ is eliminated unconditionally (B2$'$ signal$>$noise).
Assuming Hypothesis~\ref{hyp:attachment}, the far-strip Schur/Herglotz machinery applies on each rectangle and the Schur pinch
eliminates zeros with $\Re s\ge\sigma_0$. Together these rule out all off-critical zeros, hence RH.
\end{proof}

\section*{Status and Lean formalization notes (non-load-bearing)}

\begin{itemize}
  \item The purely algebraic implication ``attachment-with-margin $\Rightarrow$ Schur/Herglotz on rectangles'' is mirrored in Lean
  as `IndisputableMonolith.NumberTheory.RiemannHypothesis.AttachmentWithMargin` (together with existing Cayley/Schur plumbing).
  \item The remaining hard work is proving \eqref{eq:attachment} as a uniform analytic inequality on the far-field rectangles.
\end{itemize}

\begin{thebibliography}{9}

\bibitem{WashburnChristmas2025}
J.~Washburn,
\emph{Riemann--Christmas (passivity + near-field recognition-geometry route)},
preprint (September 2025).
Source: \texttt{docs/primes/Riemann-proofs/Riemann-Christmas.tex}.

\bibitem{WashburnActive2025}
J.~Washburn,
\emph{A boundary product--certificate reduction of the Riemann Hypothesis},
preprint (September 2025).
Source: \texttt{docs/primes/Riemann-proofs/Riemann-active.tex}.

\end{thebibliography}

\end{document}


