\documentclass[11pt]{article}
% Robust CSV tables
\usepackage[margin=1in]{geometry}
\usepackage{amsmath,amssymb,amsthm,mathtools}
\usepackage[T1]{fontenc}
\usepackage{lmodern}
\usepackage[utf8]{inputenc}
\usepackage{microtype}
% Minimal preamble: this manuscript records unconditional Whitney-box energy bounds
% for $U_\xi=\Re\log\xi$ and an averaged bound for the weighted microscopic
% $S(T)$-variation obstruction.

% Theorems
\newtheorem{theorem}{Theorem}
\newtheorem{proposition}[theorem]{Proposition}
\newtheorem{lemma}[theorem]{Lemma}
\newtheorem{corollary}[theorem]{Corollary}
\theoremstyle{definition}
\newtheorem{definition}[theorem]{Definition}
\theoremstyle{remark}
\newtheorem{remark}[theorem]{Remark}

% Macros
\newcommand{\C}{\mathbb{C}}
\newcommand{\R}{\mathbb{R}}
\newcommand{\N}{\mathbb{N}}
\DeclareMathOperator{\Arg}{Arg}

% Title & authors
\title{Whitney box energy for $\log\xi$ and weighted microscopic variation of $S(T)$}
\author{Jonathan Washburn\\ Recognition Science, Recognition Physics Institute\\ Austin, Texas, USA\\ jon@recognitionphysics.org}
\date{September 2025}

\begin{document}
\maketitle

\begin{abstract}
We study the harmonic field \(U_\xi(\sigma,t)=\Re\log\xi(\tfrac12+\sigma+it)\) on the right half-plane and its Whitney-box Dirichlet energy at the microscopic scale \(L\asymp 1/\log\langle T\rangle\).
Unconditionally, we prove a Whitney-box growth bound of size \(O(|I|\log\langle T\rangle)\) (Lemma~\ref{lem:carleson-xi}).
We also show that, at Whitney scale \(L=c/\log\langle T\rangle\), the box energy admits a sharpened upper bound in terms of a weighted short-interval variation of the classical argument term \(S(T)=\frac1\pi\Arg\zeta(\tfrac12+iT)\) (Lemma~\ref{lem:xi-energy-Svariation}), isolating a precise obstruction to scale-free energy control.
\end{abstract}

\paragraph{Keywords.} Riemann zeta function; Hardy/Smirnov spaces; Herglotz/Schur functions; Carleson measures; Hilbert--Schmidt determinants.

\paragraph{MSC 2020.} 11M26, 30D15, 30C85; secondary 47A12, 47B10.

\section*{Notation and conventions}
\begin{itemize}
\item Half–plane and coordinates: $\Omega:=\{\Re s>\tfrac12\}$; write $s=\tfrac12+\sigma+it$ with $\sigma>0$, $t\in\R$.
\item Completed zeta: $\xi(s):=\tfrac12 s(1-s)\,\pi^{-s/2}\Gamma(s/2)\,\zeta(s)$, and
\[
U_\xi(\sigma,t):=\Re\log\xi\!\left(\tfrac12+\sigma+it\right)\qquad(\sigma>0).
\]
\item Riemann--von Mangoldt: $N(T)=\frac{T}{2\pi}\log\frac{T}{2\pi}-\frac{T}{2\pi}+\frac78+S(T)+O(1)$ with
$S(T):=\frac{1}{\pi}\Arg\zeta(\tfrac12+iT)$.
\item Poisson kernel: $P_a(x)=\tfrac{1}{\pi}\tfrac{a}{a^2+x^2}$.
\item Whitney scale and boxes: fix $c\in(0,1]$ and $\langle T\rangle:=\sqrt{1+T^2}$, set
\[
L:=\frac{c}{\log\langle T\rangle},\qquad I:=[T-L,T+L],\qquad Q(\alpha I):=I\times(0,\alpha L],
\]
with a fixed aperture $\alpha\in[1,2]$.
\end{itemize}

\section*{Reader's guide}
\begin{itemize}
\item \textbf{Main unconditional results.} An annular \(L^2\) Poisson--balayage bound (Lemma~\ref{lem:annular-balayage}); unconditional Whitney-box energy growth for \(U_\xi\) (Lemma~\ref{lem:carleson-xi} and Proposition~\ref{prop:xi-energy-growth}); the reduction of Whitney-scale \(\xi\)-energy to weighted microscopic variation of \(S(T)\) (Lemma~\ref{lem:xi-energy-Svariation}); and an averaged (dyadic) mean-square bound for that weighted term (Theorem~\ref{thm:weighted-S-mean-square}).
\item \textbf{What is pinned down as the obstruction.} Scale-free Whitney energy at \(L=c/\log\langle T\rangle\) would follow from a uniform bound on the weighted short-interval variation term in Lemma~\ref{lem:xi-energy-Svariation}; informally, it forbids arbitrarily large zero clusters in windows of length \(\asymp 1/\log T\).
\item \textbf{Scope.} We do \emph{not} claim an unconditional proof of the Riemann Hypothesis here; this manuscript records the unconditional analytic core and a precise formulation of what remains open.
\end{itemize}

\subsection*{Dependency map (unconditional core)}
All proofs not explicitly listed below are auxiliary.
\begin{enumerate}
\item \textbf{Annular bookkeeping.} Lemma~\ref{lem:annular-balayage} provides the basic annular \(L^2\) Poisson–balayage estimate used to control far-field contributions from zeros.
\item \textbf{Whitney box energy for $U_\xi$.} Lemma~\ref{lem:carleson-xi} gives an unconditional \(O(|I|\log\langle T\rangle)\) growth bound on Whitney boxes. Lemma~\ref{lem:xi-energy-Svariation} refines the bookkeeping to express the only non–scale-free contribution as a weighted microscopic variation of \(S(T)\).
\item \textbf{Average control of the obstruction.} Theorem~\ref{thm:weighted-S-mean-square} gives a dyadic mean-square bound for the weighted microscopic variation term, via Selberg's mean-square estimate for \(S\) (Lemma~\ref{lem:selberg-mean-square}).
\item \textbf{Open problem (isolated).} Upgrade the Whitney-scale \(O(|I|\log\langle T\rangle)\) growth bound to a scale-free \(O(|I|)\) bound uniform in \(T\) at scale \(L=c/\log\langle T\rangle\). By Lemma~\ref{lem:xi-energy-Svariation}, this would follow from uniform control of the weighted short-interval variation of \(S(T)\) at Whitney scale.
\end{enumerate}

\section{Introduction}
\paragraph{Conceptual motivation.} The Euler product for $\zeta$ separates the $k=1$ prime layer from all higher prime powers. On the right half–plane $\{\Re s>\tfrac12\}$ the diagonal prime operator $A(s)e_p:=p^{-s}e_p$ has finite Hilbert–Schmidt norm ($\sum_p p^{-2\sigma}<\infty$), so the $k\ge2$ tail is naturally encoded by the 2–modified determinant $\det_2(I-A)$. For the present manuscript, the central analytic object is the $\xi$--field
\[
U_\xi(\sigma,t)=\Re\log\xi\!\left(\tfrac12+\sigma+it\right),
\]
and its Whitney-box Dirichlet energy at the microscopic scale \(L\asymp 1/\log\langle T\rangle\).

\paragraph{Scope and contributions.} We prove an unconditional Whitney-box growth bound of size \(O(|I|\log\langle T\rangle)\) for \(U_\xi\) on Whitney boxes (Lemma~\ref{lem:carleson-xi}). We also refine the bookkeeping to show that the only non–scale-free contribution can be expressed as a weighted microscopic short-interval variation of the classical argument term \(S(T)=\frac1\pi\Arg\zeta(\tfrac12+iT)\) (Lemma~\ref{lem:xi-energy-Svariation}). These results isolate a precise obstruction to scale-free energy control at length \(1/\log T\). No unconditional proof of the Riemann Hypothesis is claimed here.
\section{Whitney-box energy bounds}
We record unconditional Carleson-energy bounds for the arithmetic tail and for $U_\xi$ on Whitney boxes, and we isolate the precise obstruction to a scale-free bound.

\begin{lemma}[Arithmetic Carleson energy]\label{lem:carleson-arith}
Let
\[
 U_{\det_2}(\sigma,t)\ :=\ \sum_{p}\sum_{k\ge 2}\frac{(\log p)\,p^{-k/2}}{k\log p}\,e^{-k\log p\,\sigma}\,\cos\big(k\log p\,t\big),\qquad \sigma>0.
\]
Then for every interval $I\subset\R$ with Carleson box $Q(I):=I\times(0,|I|]$
\[
 \iint_{Q(I)} |\nabla U_{\det_2}|^2\,\sigma\,dt\,d\sigma\ \le\ \frac{|I|}{4}\,\sum_{p}\sum_{k\ge 2}\frac{p^{-k}}{k^2}
 \ =:\ K_0\,|I|,\qquad K_0:=\frac{1}{4}\sum_{p}\sum_{k\ge 2}\frac{p^{-k}}{k^2}<\infty.
\]
\end{lemma}
\begin{proof}
For a single mode $b\,e^{-\omega\sigma}\cos(\omega t)$ one has $|\nabla|^2=b^2\omega^2e^{-2\omega\sigma}$, hence
\[
 \int_0^{|I|}\!\int_I |\nabla|^2\,\sigma\,dt\,d\sigma\ \le\ |I|\cdot\sup_{\omega>0}\int_0^{|I|}\sigma\,\omega^2e^{-2\omega\sigma}d\sigma\cdot b^2\ \le\ \tfrac14\,|I|\,b^2.
\]
With $b=(\log p)\,p^{-k/2}/(k\log p)$ and $\omega=k\log p$, summing over $(p,k)$ gives the claim and the finiteness of $K_0$.
\end{proof}
\paragraph{Whitney scale and short–interval zero counts (what is actually known unconditionally).}
Fix a Whitney parameter $c\in(0,1]$ and set $L:=c/\log\langle T\rangle$ with $\langle T\rangle:=\sqrt{1+T^2}$.
The only short-interval input we use is the local Riemann--von Mangoldt bound (see, e.g., Ivi\'c): for $T\ge 2$ and $H\in(0,1]$,
\[
  N(T+H)-N(T-H)\ \ll\ H\log\langle T\rangle\ +\ \log\langle T\rangle.
\]
In particular, on Whitney scale $H\asymp L(T)=c/\log\langle T\rangle$ this gives at best $O(\log\langle T\rangle)$ zeros in a window of length $\asymp 1/\log\langle T\rangle$; we do \emph{not} assume or claim a uniform $O(1)$ bound at that microscopic scale.
\begin{lemma}[Annular Poisson–balayage $L^2$ bound]\label{lem:annular-balayage}
Let $I=[T-L,T+L]$, $Q_\alpha(I)=I\times(0,\alpha L]$, and fix $k\ge1$. For
$\mathcal A_k:=\{\rho=\beta+i\gamma:\ 2^kL<|T-\gamma|\le 2^{k+1}L\}$ set
\[
  V_k(\sigma,t):=\sum_{\rho\in\mathcal A_k}\frac{\sigma}{(t-\gamma)^2+\sigma^2}.
\]
Then
\[
  \iint_{Q_\alpha(I)} V_k(\sigma,t)^2\,\sigma\,dt\,d\sigma\ \ll_\alpha\ |I|\,4^{-k}\,\nu_k,
\]
where $\nu_k:=\#\mathcal A_k$, and the implicit constant depends only on $\alpha$.
\end{lemma}
\begin{proof}
Write $K_\sigma(x):=\sigma/(x^2+\sigma^2)$ and $V_k=\sum_{\rho\in\mathcal A_k}K_\sigma(\cdot-\gamma)$. For any finite index set $\mathcal J$,
\[
  V_k^2\;\le\; \sum_{j\in\mathcal J} K_\sigma(\cdot-\gamma_j)^2\ +\ 2\!\!\sum_{i<j} K_\sigma(\cdot-\gamma_i)K_\sigma(\cdot-\gamma_j).
\]
Integrate over $t\in I$ first. For the diagonal terms, using $|t-\gamma|\ge 2^kL-L\ge 2^{k-1}L$ for $t\in I$ and $k\ge 1$,
\[
 \int_I K_\sigma(t-\gamma)^2\,dt\ =\ \sigma^2\!\int_I \frac{dt}{\big((t-\gamma)^2+\sigma^2\big)^2}\ \le\ \frac{L}{(2^{k-1}L)^2}\,\sigma\ \le\ \frac{\sigma}{4^{k-1}L}.
\]
Multiplying by the area weight $\sigma$ and integrating $\sigma\in(0,\alpha L]$ gives
\[
 \int_0^{\alpha L}\!\!\left(\int_I K_\sigma(t-\gamma)^2\,dt\right)\sigma\,d\sigma\ \le\ \frac{1}{4^{k-1}L}\int_0^{\alpha L}\!\sigma^2 d\sigma\ =\ \frac{\alpha^3 L^2}{3\cdot 4^{k-1}}\ \le\ \frac{C_{\mathrm{diag}}(\alpha)}{4^{k}}\,|I|,
\]
with $C_{\mathrm{diag}}(\alpha):=\tfrac{4\alpha^3}{3}\cdot\tfrac{L}{|I|}\asymp_\alpha 1$. Summing over $\nu_k$ choices of $\gamma$ contributes a factor $\nu_k$.

For the off-diagonal terms, for $i\ne j$ one has on $I$ that $K_\sigma(t-\gamma_j)\le \sigma/(2^{k-1}L)^2$. Hence
\[
 \int_I K_\sigma(t-\gamma_i)K_\sigma(t-\gamma_j)\,dt\ \le\ \frac{\sigma}{(2^{k-1}L)^2}\int_\R K_\sigma(t-\gamma_i)\,dt\ =\ \frac{\pi\sigma}{(2^{k-1}L)^2},
\]
and integrating $\sigma\in(0,\alpha L]$ with the extra factor $\sigma$ yields $\le C'_{\mathrm{off}}(\alpha)\,L\cdot 4^{-k}$. Summing in $i,j$ via the Schur test with $f_j(t):=K_\sigma(t-\gamma_j)\mathbf 1_I(t)$ gives
\[
 \int_I V_k(\sigma,t)^2\,dt\ \le\ C''(\alpha)\,\nu_k\,\frac{\sigma}{(2^kL)^2}.
\]
Integrating $\sigma\in(0,\alpha L]$ with weight $\sigma$ gives $\le C_{\mathrm{off}}(\alpha)\,|I|\cdot 4^{-k}\,\nu_k$. Combining diagonal and off–diagonal parts, absorbing harmless constants into $C_\alpha$, we obtain the stated bound with an explicit $C_\alpha=O(\alpha^3)$.
\end{proof}

\begin{lemma}[Analytic ($\xi$) box energy on Whitney boxes (unconditional, but not scale-free)]\label{lem:carleson-xi}
\emph{Reference.} The local zero count used below follows from the Riemann--von Mangoldt formula; see, e.g., \cite{Titchmarsh,Ivic}.
Fix a Whitney parameter $c\in(0,1]$ and let $I=[T-L,\,T+L]$ with Whitney scale $L:=c/\log\langle T\rangle$.
Then for the Poisson extension
\[
 U_{\xi}(\sigma,t):=\Re\log\xi\big(\tfrac12+\sigma+it\big),\qquad (\sigma>0),
\]
and any fixed aperture $\alpha\in[1,2]$, one has the unconditional bound
\[
  \iint_{Q(\alpha I)} |\nabla U_{\xi}(\sigma,t)|^2\,\sigma\,dt\,d\sigma\ \ll_{\alpha,c}\ |I|\,\log\langle T\rangle.
\]
\end{lemma}

\begin{proof}
All inputs are unconditional. Fix $I=[T-L,T+L]$ with $L=c/\log\langle T\rangle$ and aperture $\alpha\in[1,2]$. Neutralize near zeros by a local half-plane Blaschke product $B_I$ removing zeros of $\xi$ inside a fixed dilate $Q(\alpha'I)$ ($\alpha'>\alpha$). This yields a harmonic field $\widetilde U_\xi$ on $Q(\alpha I)$ and
\[
  \iint_{Q(\alpha I)} |\nabla U_\xi|^2\,\sigma\,dt\,d\sigma\ \asymp\ \iint_{Q(\alpha I)} |\nabla \widetilde U_\xi|^2\,\sigma\,dt\,d\sigma\ +\ O_\alpha(|I|),
\]
so it suffices to bound the neutralized energy.

Write $\partial_\sigma U_\xi=\Re\,(\xi'/\xi)=\Re\sum_\rho (s-\rho)^{-1}+A$, where $A$ is smooth on compact strips. Since $U_\xi$ is harmonic, $|\nabla U_\xi|^2\asymp |\partial_\sigma U_\xi|^2$ on $\R^2_+$; thus we bound the $L^2(\sigma\,dt\,d\sigma)$ norm of $\sum_\rho (s-\rho)^{-1}$ over $Q(\alpha I)$. Decompose the (neutralized) zeros into Whitney annuli $\mathcal A_k:=\{\rho:2^kL<|\gamma-T|\le 2^{k+1}L\}$, $k\ge1$. For $V_k(\sigma,t):=\sum_{\rho\in\mathcal A_k} K_\sigma(t-\gamma)$ with $K_\sigma(x):=\sigma/(x^2+\sigma^2)$, Lemma~\ref{lem:annular-balayage} gives
\[
  \iint_{Q_\alpha(I)} V_k(\sigma,t)^2\,\sigma\,dt\,d\sigma\ \le\ C_\alpha\,|I|\,4^{-k}\,\nu_k,
\]
where $\nu_k:=\#\mathcal A_k$ and $C_\alpha$ depends only on $\alpha$. Summing Cauchy–Schwarz bounds over annuli yields
\[
  \iint_{Q(\alpha I)} \Big|\sum_{\rho}(s-\rho)^{-1}\Big|^2\,\sigma\,dt\,d\sigma\ \le\ C_\alpha\,|I|\sum_{k\ge1}4^{-k}\,\nu_k.
\]
To bound $\nu_k$, it suffices to use the local zero count on short vertical intervals from the Riemann--von Mangoldt formula (Titchmarsh ): for $H>0$,
\[
  N(T+H)-N(T-H)\ \ll\ H\log\langle T\rangle\ +\ \log\langle T\rangle.
\]
Since $\nu_k$ counts (a subset of) zeros with ordinates in a window of length $\asymp 2^kL$, this yields, for some absolute $a_1(\alpha),a_2(\alpha)$,
\[
  \nu_k\ \le\ a_1(\alpha)\,2^k L\,\log\langle T\rangle\ +\ a_2(\alpha)\,\log\langle T\rangle.
\]
Therefore,
\[
  \sum_{k\ge1}4^{-k}\,\nu_k\ \le\ a_1(\alpha)\,L\,\log\langle T\rangle\sum_{k\ge1}2^{-k}\ +\ a_2(\alpha)\,\log\langle T\rangle\sum_{k\ge1}4^{-k}\ \ll\ L\,\log\langle T\rangle\ +\ \log\langle T\rangle.
\]
On Whitney scale $L=c/\log\langle T\rangle$ this is $\ll \log\langle T\rangle$. Adding the neutralized near-field $O(|I|)$ and the smooth $A$ contribution, we obtain
\[
  \iint_{Q(\alpha I)} |\nabla U_\xi|^2\,\sigma\,dt\,d\sigma\ \le\ C_\xi\,|I|,
\]
with $C_\xi$ depending on $\alpha$ and growing at most like $O(\log\langle T\rangle)$ on Whitney scale. This proves the lemma.
\end{proof}

\begin{proposition}[Whitney box energy growth for $U_\xi$ (unconditional)]\label{prop:xi-energy-growth}
Fix $\alpha\in[1,2]$ and a Whitney parameter $c\in(0,1]$. For each Whitney base interval
$I=[T-L,T+L]$ with $L=c/\log\langle T\rangle$ one has
\[
  \iint_{Q(\alpha I)} |\nabla U_\xi|^2\,\sigma\,dt\,d\sigma \;\ll_{\alpha,c}\; |I|\,\log\langle T\rangle.
\]
\end{proposition}
\begin{proof}
This is exactly Lemma~\ref{lem:carleson-xi}.
\end{proof}

\begin{definition}[Scale-free Whitney box-energy bound for $U_\xi$]\label{def:Kxi-uniform}
Fix $\alpha\in[1,2]$ and a Whitney parameter $c\in(0,1]$.
We say that \(U_\xi\) satisfies a \emph{scale-free Whitney box-energy bound} (at \((\alpha,c)\)) if there exists a finite constant $K_\xi=K_\xi(\alpha,c)<\infty$ such that for every Whitney base interval
$I=[T-L,T+L]$ with $L=c/\log\langle T\rangle$,
\[
  \iint_{Q(\alpha I)} |\nabla U_\xi|^2\,\sigma\,dt\,d\sigma \;\le\; K_\xi\,|I|.
\]
\end{definition}
\begin{remark}[Number-theoretic interpretation]
The scale-free Whitney box-energy bound in Definition~\ref{def:Kxi-uniform} is not merely a technical Carleson constant: it is a scale-free prohibition on extreme microscopic zero clustering at scale $H\asymp 1/\log\langle T\rangle$.
One clean way to pin this to a classical object is via the $S(T)$ term in the Riemann--von Mangoldt formula: Lemma~\ref{lem:xi-energy-Svariation} shows that a uniform weighted short-interval variation bound for $S$ at Whitney scale implies Definition~\ref{def:Kxi-uniform}.
\end{remark}
\begin{lemma}[Whitney $\xi$-energy reduces to weighted short-interval variation of $S$]\label{lem:xi-energy-Svariation}
Fix $\alpha\in[1,2]$ and $c\in(0,1]$.
Let $T\ge 3$, set $L:=c/\log\langle T\rangle$, and let $I:=[T-L,T+L]$.
Define
\[
U_\xi(\sigma,t):=\Re\log\xi\!\left(\tfrac12+\sigma+it\right)\qquad(\sigma>0),
\]
and write the Riemann--von Mangoldt decomposition
\[
 N(T)=\frac{T}{2\pi}\log\frac{T}{2\pi}-\frac{T}{2\pi}+\frac78+S(T)+O(1/T),
\qquad S(T):=\frac{1}{\pi}\Arg \zeta\!\left(\tfrac12+iT\right).
\]
Then there exists a constant $C_{\alpha,c}<\infty$ such that
\[
\iint_{Q(\alpha I)} |\nabla U_\xi(\sigma,t)|^2\,\sigma\,dt\,d\sigma
\ \le\
C_{\alpha,c}\,|I|\Bigg(1+\sum_{k\ge 1}4^{-k}\,\big|S(T+2^kL)-S(T-2^kL)\big|\Bigg).
\]
In particular, a uniform bound
\[
\sup_{T\ge 3}\ \sum_{k\ge 1}4^{-k}\,\big|S(T+2^kL)-S(T-2^kL)\big|\ <\ \infty
\qquad\Big(L=\frac{c}{\log\langle T\rangle}\Big)
\]
implies the scale-free Whitney box-energy bound of Definition~\ref{def:Kxi-uniform} for some finite $K_\xi=K_\xi(\alpha,c)$ independent of $T$.
\end{lemma}

\begin{proof}
This is a sharpened version of the annular bookkeeping in Lemma~\ref{lem:carleson-xi}.
As there, neutralize zeros of $\xi$ in a fixed dilate $Q(\alpha'I)$ by a local half-plane Blaschke product and reduce to bounding the neutralized energy on $Q(\alpha I)$; this changes the energy by at most $O_\alpha(|I|)$.

Decompose the (neutralized) zeros by dyadic annuli in ordinate,
\[
  \mathcal A_k:=\{\rho=\beta+i\gamma:\ 2^kL<|\gamma-T|\le 2^{k+1}L\},\qquad \nu_k:=\#\mathcal A_k,
\]
and let $V_k(\sigma,t):=\sum_{\rho\in\mathcal A_k}K_\sigma(t-\gamma)$ with $K_\sigma(x):=\sigma/(x^2+\sigma^2)$.
Arguing exactly as in the proof of Lemma~\ref{lem:carleson-xi} and using Lemma~\ref{lem:annular-balayage},
\[
 \iint_{Q(\alpha I)} |\nabla U_\xi|^2\,\sigma\,dt\,d\sigma
  \ \ll_{\alpha}\ |I|\Big(1+\sum_{k\ge 1}4^{-k}\,\nu_k\Big).
\]

To control $\nu_k$, note that trivially
\[
  \nu_k\ \le\ N(T+2^{k+1}L)-N(T-2^{k+1}L).
\]
Using the displayed form of $N(T)$ and Taylor expansion of the main term on $[T-2^{k+1}L,\,T+2^{k+1}L]$, one has
\[
  N(T+H)-N(T-H)
  \ =\ \frac{H}{\pi}\log\frac{T}{2\pi}\ +\ O(1+H^2/T)\ +\ \big(S(T+H)-S(T-H)\big)
\]
for $T\ge 3$ and $0<H\le T/2$.
Applying this with $H=2^{k+1}L$ (and absorbing the finitely many $k$ with $2^{k+1}L>T/2$ into the harmless $O(1)$ term, since their $4^{-k}$ weights are summable) yields
\[
  \nu_k\ \ll_c\ 2^kL\log\langle T\rangle\ +\ 1\ +\ \big|S(T+2^{k+1}L)-S(T-2^{k+1}L)\big|.
\]
Insert this into the weighted sum:
\[
  \sum_{k\ge 1}4^{-k}\,\nu_k
  \ \ll_c\ (L\log\langle T\rangle)\sum_{k\ge 1}2^{-k}\ +\ \sum_{k\ge 1}4^{-k}
  \ +\ \sum_{k\ge 1}4^{-k}\,\big|S(T+2^{k+1}L)-S(T-2^{k+1}L)\big|.
\]
Since $L\log\langle T\rangle=c$ and the geometric series converge, the first two terms are $O_c(1)$.
Reindexing the last term ($j=k+1$) and absorbing the factor $4$ into the constant gives
\[
  \sum_{k\ge 1}4^{-k}\,\nu_k\ \ll_c\ 1+\sum_{j\ge 1}4^{-j}\,\big|S(T+2^{j}L)-S(T-2^{j}L)\big|.
\]
Substituting back into the energy bound yields the stated estimate (with constants depending only on $(\alpha,c)$).
\end{proof}

\section{Averaged control of the weighted microscopic $S$-variation}
The scale-free bound of Definition~\ref{def:Kxi-uniform} remains open. Nevertheless, one can control the weighted short-interval variation term from Lemma~\ref{lem:xi-energy-Svariation} on average using Selberg's mean-square bound for $S(t)$.

\begin{lemma}[Selberg mean-square bound for $S$]\label{lem:selberg-mean-square}
For every $X\ge 3$ one has
\[
  \int_0^{X} S(t)^2\,dt\ \ll\ X\log\log X.
\]
\end{lemma}
\begin{proof}
This is classical (Selberg); see, e.g., \cite{Ivic}.
\end{proof}

\begin{theorem}[Dyadic mean-square bound for the weighted $S$-variation]\label{thm:weighted-S-mean-square}
Fix $c\in(0,1]$ and let $T\ge 3$. Set $L:=c/\log\langle T\rangle$ and define, for $t\in[T,2T]$,
\[
  V_T(t)\ :=\ \sum_{k\ge 1}4^{-k}\,\big|S(t+2^kL)-S(t-2^kL)\big|.
\]
Then
\[
  \frac{1}{T}\int_T^{2T} V_T(t)^2\,dt\ \ll\ \log\log T.
\]
In particular,
\[
  \frac{1}{T}\int_T^{2T} V_T(t)\,dt\ \ll\ \sqrt{\log\log T},
\]
and for any $A>0$,
\[
  \big|\{t\in[T,2T]:\ V_T(t)>A\sqrt{\log\log T}\}\big|\ \ll\ A^{-2}\,T.
\]
\end{theorem}
\begin{proof}
Let $\Delta_k(t):=S(t+2^kL)-S(t-2^kL)$. By Cauchy--Schwarz with weights $w_k:=4^{-k}$,
\[
  V_T(t)^2\ \le\ \Big(\sum_{k\ge 1}w_k\Big)\,\sum_{k\ge 1}w_k\,|\Delta_k(t)|^2.
\]
Using $|a-b|^2\le 2(a^2+b^2)$, we have
\[
  |\Delta_k(t)|^2\ \le\ 2\Big(S(t+2^kL)^2+S(t-2^kL)^2\Big).
\]
Integrating $t\in[T,2T]$ and changing variables yields
\[
  \int_T^{2T}|\Delta_k(t)|^2\,dt
  \ \le\ 2\int_{T-2^kL}^{2T+2^kL}\!S(u)^2\,du\ +\ 2\int_{T-2^kL}^{2T+2^kL}\!S(u)^2\,du
  \ \le\ 4\int_0^{4T}\!S(u)^2\,du.
\]
Therefore
\[
  \int_T^{2T}V_T(t)^2\,dt\ \ll\ \Big(\sum_{k\ge 1}4^{-k}\Big)\,\sum_{k\ge 1}4^{-k}\int_0^{4T}\!S(u)^2\,du
  \ \ll\ \int_0^{4T}\!S(u)^2\,du
  \ \ll\ T\log\log T
\]
by Lemma~\ref{lem:selberg-mean-square}. The $L^1$ bound follows from Cauchy--Schwarz, and the large-deviation bound from Chebyshev.
\end{proof}

\begin{remark}
Theorem~\ref{thm:weighted-S-mean-square} gives a density-one control at the scale $\sqrt{\log\log T}$, but it does not imply the scale-free bound of Definition~\ref{def:Kxi-uniform}; it is compatible with rare microscopic windows where $V_T(t)$ is large.
\end{remark}

\begin{remark}[Bottleneck / open problem]\label{rem:Kxi-bottleneck}
An open problem is to upgrade the Whitney-scale estimate of Proposition~\ref{prop:xi-energy-growth}
to a \emph{scale-free} bound uniform in \(T\), i.e. to prove that \(U_\xi\) satisfies Definition~\ref{def:Kxi-uniform}.
By Lemma~\ref{lem:xi-energy-Svariation}, this would follow from a uniform (in \(T\)) bound on the weighted short-interval variation term
\[
  \sum_{k\ge 1}4^{-k}\,\big|S(T+2^kL)-S(T-2^kL)\big|
  \qquad\Big(L=\frac{c}{\log\langle T\rangle}\Big),
\]
or by any equivalent analytic input strong enough to replace Proposition~\ref{prop:xi-energy-growth} by a scale-free \(O(|I|)\) estimate at Whitney scale.
\end{remark}

\begin{thebibliography}{99}
\bibitem{Titchmarsh}
E.~C.~Titchmarsh, revised by D.~R.~Heath-Brown,
\emph{The Theory of the Riemann Zeta-Function}, 2nd ed.,
Oxford University Press, 1986.

\bibitem{Ivic}
A.~Ivi\'c,
\emph{The Riemann Zeta-Function: Theory and Applications},
Dover Publications, 2003.
\end{thebibliography}

\end{document}
