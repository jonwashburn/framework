\documentclass[11pt,reqno]{amsart}
%%%%%%%%%%%%%%%%%%%%%%%%%%%%%%%%%%%%%%%%%%%%%%%%%%%%%%%%%%%%%%%%%%%%%%%%%%%%%%%%%%%%%%%%%%%%%%%%%%%%%%%%%%%%%%%%%%%%%%%%%%%%%%%%%%%%%%%%%%%%%%%%%%%%%%%%%%%%%%%%%%%%%%%%%%%%%%%%%%%%%%%%%%%%%%%%%%%%%%%%%%%%%%%%%%%%%%%%%%%%%%%%%%%%%%%%%%%%%%%%%%%%%%%%%%%%
\usepackage[pagebackref,colorlinks=true]{hyperref}
\usepackage{verbatim}
\usepackage{eucal,url,amssymb,stmaryrd,enumerate,amscd,}
%\usepackage{showkeys}
%\usepackage{refcheck}
\usepackage{amsfonts,amsmath,amsthm,amssymb,amscd,enumerate,eucal,url,stmaryrd}
\usepackage{mathtools}
\usepackage[margin=1in]{geometry}

\usepackage{mathrsfs}
%\setcounter{MaxMatrixCols}{10}

\numberwithin{equation}{section}


\newcommand{\R}{\mathbb{R}}
\newcommand{\Q}{\mathbb{Q}}
\newcommand{\Rp}{\R_{>0}}
\newcommand{\e}{\mathrm{e}}

\newtheorem{theorem}{Theorem}[section]
\newtheorem{lemma}{Lemma}[section]
\newtheorem{proposition}{Proposition}[section]
\newtheorem{corollary}{Corollary}[section]
\newtheorem{definition}{Definition}[section]
\newtheorem{remark}{Remark}[section]
\newtheorem{example}{Example}[section]
\newtheorem{notation}{Notation}
\newtheorem{convention}{Convention}
\usepackage{amsthm}

\newtheoremstyle{axiomstyle}
  {}{}                 % razmaci
  {\itshape}           % telo
  {}                   % indent
  {\bfseries}          % naslov
  {.}                  % tačka iza broja
  {0.5em}              % razmak
  {\thmname{#1}~\thmnumber{#2}\thmnote{ (#3)}} % OVO PRIKAZUJE (RG0...)
  
\theoremstyle{axiomstyle}
\newtheorem{axiom}{Axiom}

%\linespread{1.06} \sloppy \allowdisplaybreaks
%\def\R{\hbox{\ddpp R}}
%\def\medR{\hbox{\rcmed R}}
%\def\C{{\hbox{\ddpp C}}}
%\def\peqC{{\hbox{\rcpeq C}}}
%\def\toro{\hbox{\ddpp T}}
%\def\Z{\hbox{\ddpp Z}}
%\def\Q{\hbox{\ddpp Q}}
%\def\ene{\hbox{\ddpp N}}
%\def\Q{\hbox{\ddpp Q}}
%\def\L{\hbox{\ddpp L}}
%\def\P{\hbox{\ddpp P}}
%\def\ppp{\hspace*{-6pt}{\sc .}\hspace*{6pt}}
%\def\fra{{\frak a}}
%\def\frg{{\frak g}}
%\def\frh{{\frak h}}
%\def\frk{{\frak K}}
%\def\frt{{\frak t}}
%\def\gc{\frg_\peqC}
%\def\Re{{\frak R}{\frak e}\,}
%\def\Im{{\frak I}{\frak m}\,}
%\def\nilm{\Gamma\backslash G}
%\def\Mod{{\mathcal M}}
%\def\db{{\bar{\partial}}}
%\def\zzz{{\!\!\!}}
%\def\sqi{{\sqrt{-1\,}}}


\begin{document}

\begin{abstract} Cont...
\vskip1.mm\noindent
\textbf{Keywords}:  

\vskip1.mm
\noindent
\textbf{Mathematics Subject Classifications (2010)}: 
\end{abstract}

\title[Uniqueness of the Canonical Reciprocal Cost]{Uniqueness of the Canonical Reciprocal Cost}
\date{\today}


%\author{Jonathan Washburn}
%\address[Jonathan Washburn]{Recognition Physics Institute Austin, Texas, USA}
%\email{jon@recognitionphysics.org}
%\author{Milan Zlatanovi\'c}
%\address[Milan Zlatanovi\'c]{Department of Mathematics, Faculty of Science and Mathematics, University of Ni\v s, Vi\v segradska 33, 18000 Ni\v s, Serbia}
%\email{zlatmilan@yahoo.com}
%\author{Elshad Allahyarov}
%\address[Elshad Allahyarov]{Recognition Physics Institute, Austin, TX, USA \\Institut für Theoretische Physik II: Weiche Materie, Heinrich-Heine-Universität Düsseldorf, Germany \\ 
%Theoretical Department, Joint Institute for High Temperatures, RAS, Moscow, Russia \\  Department of Physics, Case Western Reserve University, Cleveland, OH, USA }
%\email{elshad.allakhyarov@case.edu}

 
 
 

 
\maketitle 

\setcounter{tocdepth}{3}

%\tableofcontents

%%%\author{xxxx...}
%%%\address[xxxx...]{Univ...}
%%%\email{xxxx...}
\newcommand{\config}{\mathcal{C}}
\newcommand{\configR}{\mathcal{C}_R}



\section{Motivation and Introduction}


In many areas of mathematics and physics, one quantifies ``how different'' two states are by selecting a scalar functional: a cost, energy, divergence, or action. Such choices are often guided by symmetry, convenience, or tradition (e.g., quadratic penalties, log-likelihoods, or variational principles). But once a framework is built atop a chosen cost, it is easy to overlook that the cost itself may be a \emph{dial}: different choices can preserve superficial qualitative behavior while materially changing quantitative outputs.

If the goal is explanatory or predictive, this dial matters. In particular, claims of being ``parameter-free'' can be undermined even in the absence of explicit tunable constants: if one can vary the functional form of the cost while keeping the rest of the story fixed, then the cost selection plays the role of an implicit parameter family. For this reason, a credible parameter-free derivation program needs \emph{uniqueness} results: conditions under which the cost is forced, rather than chosen.



Recognition Science aiming to derive a coherent mathematical scaffold for dynamics from a small set of structural constraints, including a ledger-style consistency discipline and a composition law for ``recognition'' amplitudes. In that setting, a cost on multiplicative ratios appears as a primitive interface between composition and measurement. The present paper does \emph{not} require the broader Recognition Science framework; we mention it only as motivation for why one is led to consider reciprocal costs and  functional equations.

Our aim here is narrower and purely mathematical: to prove that a natural set of explicit hypotheses forces a unique closed-form cost on \(\Rp\).

\subsection{The canonical reciprocal cost}

\begin{definition}[Canonical reciprocal cost]
We define the function \(J:\Rp\to\R\) by
\[
J(x)\;:=\;\frac{x+x^{-1}}{2}-1.
\]
\end{definition}

The function \(J\) satisfies the following properties:
\begin{itemize}
    \item[(i)] \emph{Reciprocity symmetry:} \(J(x)=J(x^{-1})\);
    \item[(ii)] \emph{Normalization:} \(J(1)=0\);
    \item[(iii)] \emph{Nonnegativity:} For all $x\in \Rp$
    \[
    J(x)=\frac{(x-1)^2}{2x}\ge 0.
    \]
\end{itemize}

If we substitute $t=\ln x$, then
\[
J(\e^t)=\cosh(t)-1.
\]

\subsection{Main result}
We now state the main theorem proved in this paper.

\begin{theorem}[Uniqueness of the canonical reciprocal cost]\label{thm:main}
Let \(F:\Rp\to\R\). If $F$ satisfies:
\begin{enumerate}
  \item[(i)] {Normalization:} \(F(1)=0\).
  \item[(ii)] {Composition law on \(\Rp\):} for all \(x,y>0\),
  \begin{equation}\label{eq1}
  F(xy)+F\Big(\dfrac xy\Big)=2\,F(x)\,F(y)+2\,F(x)+2\,F(y).
  \end{equation}
  \item[(iii)] {Quadratic calibration at the identity:}
  \[
  \lim_{t\to 0}\frac{2\,F(\e^t)}{t^2}=1.
  \]
\end{enumerate}
Then for all \(x>0\),
\[
F(x)=\frac{x+x^{-1}}{2}-1 \;=\; J(x).
\]
\end{theorem}

 



\subsection{Paper organization}
 CONT...


\section{Definitions and basic properties}\label{sec:prelim}


We work on the following domain
\[
\Rp:=\{x\in\R:\ x>0\}.
\]


\begin{definition}\label{def:recip-norm}
A function \(F:\Rp\to\R\) is called a \emph{reciprocal cost} if
\[
F(x)=F(x^{-1})\qquad\text{for all }x>0.
\]
It is \emph{normalized} if \(F(1)=0\).
\end{definition}
From \(F(1)=0\) it immediately follows that \(G(0)=0\) and \(H(0)=1\).
\medskip

\noindent Let us consider \(F:\Rp\to\R\), and define 
\[
G(t):=F(\e^t),\qquad H(t):=G(t)+1=F(\e^t)+1,\qquad t\in\R.
\]

\begin{lemma}\label{lem:recip-even}
If \(F\) is reciprocal, then \(G\) and \(H\) are even.
%\[
%G(-t)=G(t),\qquad H(-t)=H(t)\qquad(t\in\R).
%\]
\end{lemma}
\begin{proof}
Since \(\e^{-t}=(\e^t)^{-1}\) and \(F(x)=F(x^{-1})\), we have
\[
G(-t)=F(\e^{-t})=F\bigl((\e^t)^{-1}\bigr)=F(\e^t)=G(t).
\]
Further, for $H(-t)$, we have \[H(-t)=G(-t)+1=G(t)+1=H(t)\].
\end{proof}


Let us define the   function
\[
J(x):=\frac{x+x^{-1}}{2}-1,\qquad x>0.
\]
Clearly, for all \(x>0\),
\[
J(x)=\frac{(x-1)^2}{2x}\ge 0,
\]
with equality if and only if \(x=1\).
 

 

 
For all \(t\in\R\),
\(
J(\e^t)=\cosh(t)-1.
\)
So, we have \(G(t)=\cosh(t)-1\) and \(H(t)=\cosh(t)\).

\subsection{The d'Alembert functional equation} 
The key structural identity considered in this paper is the d’Alembert functional equation, which is also known in the literature as the cosine equation or the Poisson equation.

\begin{definition} \label{def:dalembert}
A function \(H:\R\to\R\) is said to satisfy the d'Alembert functional equation if,
for all \(t,u\in\R\),
\begin{equation}\label{dal}
H(t+u)+H(t-u)=2\,H(t)\,H(u).
\end{equation}
\end{definition}
D'Alembert's functional equation
has a long history going back to d'Alembert \cite{dAlembert1769}, Poisson \cite{Poisson1804}, and Picard \cite{Picard1922}. The equation plays an important role in determining the sum of two vectors in
various Euclidean and non-Euclidean geometries.\\

If $t=u= 0$ in (\ref{dal}), then $2H(0) = 2H(0)^2$ so that, we have
$$H(0) =0\quad\mbox{or}\quad H(0) = 1.$$
If $H(0) =0$, then for any real $x$
$$0 = 2H(x)H(0) =H(x + 0) +H(x - 0) = 2H(x). $$ Therefore, $H$ is the identically zero function. We will assume from now $H(0)=1$.




\begin{theorem}
If \(H:\R\to\R\) is a continuous   function and satisfies equation (\ref{dal}), then the only solutions are
\[H(x)\equiv 0,\quad H(x)\equiv 1, \quad H(x)=\cos(kx), \quad
H(x)=\cosh(kx),\] where \(k\) is a real constant. The classical Cauchy
method determines \(H\) on a dense subset of \(\R\) and extends it to the
whole real line by continuity.
    \end{theorem}

\begin{lemma} \label{lem:dalembert-even}
If \(H\) satisfies Definition~\ref{def:dalembert}, then \(H\) is even.
\end{lemma}
\begin{proof}
Fix \(u\in\R\) and apply the d'Alembert equation (\ref{dal}) with \(t=0\):
\[
H(u)+H(-u)=2\,H(0)\,H(u)=2H(u),
\]
so \(H(-u)=H(u)\).
\end{proof}

\begin{lemma}\label{lem:dalembert-product}
If \(H\) satisfies Definition~\ref{def:dalembert}, then for all \(t,u\in\R\),
\[
H(t+u)\,H(t-u)=H(t)^2+H(u)^2-1.
\]
\end{lemma}
\begin{proof}
Apply (\ref{dal}) with \(a=t+u\) and \(b=t-u\):
\[
H((t+u)+(t-u))+H((t+u)-(t-u))=2H(t+u)H(t-u),
\]
so \[H(2t)+H(2u)=2H(t+u)H(t-u).\] Using that
\[
H(2t)=2H(t)^2-1
\]
(obtained from (\ref{dal}) with \((t,t)\) and \(H(0)=1\)), and similarly for \(u\), yields the claim.
\end{proof}

\begin{lemma}\label{lem:dalembert-diff-square}
If \(H\) satisfies Definition~\ref{def:dalembert}, then for all \(t,u\in\R\),
\[
\bigl(H(t+u)-H(t-u)\bigr)^2=4\,(H(t)^2-1)\,(H(u)^2-1).
\]
\end{lemma}
\begin{proof}
Let \(A:=H(t+u)\) and \(B:=H(t-u)\). Then \(A+B=2H(t)H(u)\) by Definition \ref{def:dalembert}, and \(AB=H(t)^2+H(u)^2-1\)
by Lemma~\ref{lem:dalembert-product}. Hence
\begin{align*}
        (A-B)^2&=(A+B)^2-4AB\\
&=4H(t)^2H(u)^2-4(H(t)^2+H(u)^2-1)\\
&=4(H(t)^2-1)(H(u)^2-1).
\end{align*}
\end{proof}

\begin{lemma}\label{lem:dalembert-continuity}
If \(H\) satisfies Definition~\ref{def:dalembert} and 
\(\lim_{t\to 0}2(H(t)-1)/t^2\) exists. Then \(H\) is continuous on \(\R\).
\end{lemma}
\begin{proof}
The limit assumption implies that \(\lim_{t\to 0} H(t)=1\).
Since \(H(0)=1\), it follows that \(H\) is continuous at \(0\).


Fix \(t\in\R\). For \(u\to 0\), the equation (\ref{dal}) gives
\[
\lim_{u\to 0}\bigl(H(t+u)+H(t-u)\bigr)
=2H(t)\lim_{u\to 0}H(u)
=2H(t).
\]
By Lemma~\ref{lem:dalembert-diff-square}, we have
\[
\lim_{u\to 0}\bigl(H(t+u)-H(t-u)\bigr)^2
=4\bigl(H(t)^2-1\bigr)\lim_{u\to 0}\bigl(H(u)^2-1\bigr)=0,
\]
hence
\[
\lim_{u\to 0}\bigl(H(t+u)-H(t-u)\bigr)=0.
\]
Moreover,
\[
H(t+u)
=\frac{H(t+u)+H(t-u)}{2}
+\frac{H(t+u)-H(t-u)}{2}.
\]
Taking limits as \(u\to 0\) and using
\[
\lim_{u\to 0}\bigl(H(t+u)+H(t-u)\bigr)=2H(t),
\qquad
\lim_{u\to 0}\bigl(H(t+u)-H(t-u)\bigr)=0,
\]
we obtain
\[
\lim_{u\to 0} H(t+u)=H(t).
\]
Similarly, \(\lim_{u\to 0} H(t-u)=H(t)\).
Therefore, \(H\) is continuous at every \(t\in\R\).
\end{proof}



If \(H\) satisfies the d'Alembert equation (\ref{dal}) and for $G=H-1$,
one proves that \(G\) satisfies the following identity
\[
G(t+u)+G(t-u)=2\,G(t)\,G(u)+2\,G(t)+2\,G(u).
\]
For example, 
\(
H(t)=J(\e^t)+1=\cosh(t)
\)
satisfies the d'Alembert  (\ref{dal}).


\subsection{Composition law on \texorpdfstring{$\Rp$}{R\_>0}}\label{sec:rp-law}
The main theorem of this paper is stated directly on \(\Rp\), so that log-coordinates appear only as
a proof technique.

\begin{definition}\label{def:rp-law}
A function \(F:\Rp\to\R\) satisfies the \emph{d'Alembert composition law on \(\Rp\)} if for all
\(x,y>0\),
\[
F(xy)+F\Big(\dfrac xy\Big)=2\,F(x)\,F(y)+2\,F(x)+2\,F(y).
\]
\end{definition}

\begin{lemma}\label{lem:equiv-rp-dalembert}
Let \(F:\Rp\to\R\), and  \(H:\R\to\R\) such that \(H(t)=F(\e^t)+1\).
Then \(F\) satisfies Definition~\ref{def:rp-law} if and only if \(H\) satisfies the d'Alembert
equation (\ref{dal}).
\end{lemma}
\begin{proof}
Assume \(F\) satisfies Definition~\ref{def:rp-law}. Let \(t,u\in\R\) and set \(x=\e^t\), \(y=\e^u\),
so \(xy=\e^{t+u}\) and \(x/y=\e^{t-u}\). Then
\begin{align*}
H(t+u)+H(t-u)
&=\bigl(F(\e^{t+u})+1\bigr)+\bigl(F(\e^{t-u})+1\bigr)\\
&=\bigl(F(xy)+F\Big(\dfrac xy\Big)\bigr)+2\\
&=\bigl(2F(x)F(y)+2F(x)+2F(y)\bigr)+2\\
&=2\bigl(F(x)+1\bigr)\bigl(F(y)+1\bigr)
=2H(t)H(u),
\end{align*}
so \(H\) satisfies (\ref{dal}).

Conversely, if \(H\) satisfies (\ref{dal}), by reverse calculation with \(x=\e^t\), \(y=\e^u\), we obtain
that $F$ satisfies Definition~\ref{def:rp-law}.
\end{proof}

\begin{definition}\label{def:calibration}
Let \(F:\Rp\to\R\). 
Define the \emph{log-curvature} of \(F\), denoted \(\kappa(F)\), as
\[
\kappa(F) \;:=\; \lim_{t\to 0}\frac{2\,F(\e^t)}{t^2}
\]
provided this limit exists.

The limit above exists if and only if 
\[
\lim_{x\to 1}\frac{2\,F(x)}{(\log x)^2}
\]
exists, and when one exists the two limits coincide.  
\end{definition}




\section{Main results}\label{sec:results}

In this section, we prove the main theorem in the paper. 

\begin{lemma}\label{lem:J-meets}
Let \(J(x)=\tfrac12(x+x^{-1})-1\) on \(\Rp\).
Then:
\begin{enumerate}
  \item[(i)] \(J\) is reciprocal and normalized: \(J(x)=J(x^{-1})\) for all \(x>0\) and \(J(1)=0\).
  \item[(ii)] \(J\) satisfies the d'Alembert composition law on \(\Rp\) (Definition~\ref{def:rp-law}).
  \item[(iii)] \(J\) has unit log-curvature: \(\kappa(J)=1\).
\end{enumerate}
\end{lemma}
\begin{proof}
(i) Reciprocity follows directly from the definition of $J(x)$. Also \(J(1)=\tfrac12(1+1)-1=0\).

\noindent(ii) Let \(H(t)=J(\e^t)+1, \; t\in\R.\)
Let us first compute \(H\)
\begin{align*}
H(t)
&= \left(\frac12\bigl(\e^t+\e^{-t}\bigr)-1\right)+1
 = \frac12\bigl(\e^t+\e^{-t}\bigr)
 = \cosh(t).
\end{align*}
Hence \(H(t)=\cosh(t)\) for all \(t\in\R\).

The function \(\cosh\) satisfies the d'Alembert
equation (\ref{dal}), i.e.,
\[
\cosh(t+u)+\cosh(t-u)=2\,\cosh(t)\cosh(u)
\qquad\text{for all }t,u\in\R.
\]
Since \(H(t)=\cosh(t)\), it follows that \(H\) satisfies the d'Alembert equation  
\begin{equation}\label{eq:H-dalembert}
H(t+u)+H(t-u)=2\,H(t)\,H(u)\qquad\text{for all }t,u\in\R.
\end{equation}
Let \(x,y>0\) and
\[
t=\log x,\qquad u=\log y.
\]
Then \(\e^t=x\), \(\e^u=y\), and consequently
\[
\e^{t+u}=xy,\qquad \e^{t-u}=\frac{x}{y}.
\]
Using the definition of \(H\), we can rewrite \eqref{eq:H-dalembert} as
\begin{align*}
\bigl(J(\e^{t+u})+1\bigr)+\bigl(J(\e^{t-u})+1\bigr)
&=2\bigl(J(\e^t)+1\bigr)\bigl(J(\e^u)+1\bigr).
\end{align*}
Substituting \(\e^{t+u}=xy\), \(\e^{t-u}=x/y\), \(\e^t=x\), and \(\e^u=y\), we obtain
\[
\bigl(J(xy)+1\bigr)+\bigl(J\Big(\dfrac xy\Big)+1\bigr)
=2\bigl(J(x)+1\bigr)\bigl(J(y)+1\bigr),
\]
i.e.
\[
J(xy)+J\Big(\dfrac xy\Big)+2
=2J(x)J(y)+2J(x)+2J(y)+2.
\]
Finally, it follows the d'Alembert composition law
on \(\Rp\) given by Definition \ref{def:rp-law}:
\[
J(xy)+J\Bigl(\frac{x}{y}\Bigr)
=2\,J(x)\,J(y)+2\,J(x)+2\,J(y),
\qquad x,y>0.
\]

\noindent (iii) Using \(J(\e^t)=\cosh(t)-1\) and the Taylor expansion
\(\cosh(t)=1+t^2/2+o(t^2)\) as \(t\to 0\), we have
\[
\lim_{t\to 0}\frac{2J(\e^t)}{t^2}
=\lim_{t\to 0}\frac{2(\cosh(t)-1)}{t^2}=1,
\]
so \(\kappa(J)=1\).
\end{proof}
 
\begin{theorem}\label{thm:cosh-unique}
Let \(H:\R\to\R\) satisfy the d'Alembert equation (\ref{dal}).
Assume the following limit exists:
\[
\kappa_H:=\lim_{t\to 0}\frac{2\,(H(t)-1)}{t^2}\in\R.
\]
Then:
\begin{enumerate}
  \item If \(\kappa_H>0\), then \(H(t)=\cosh(\sqrt{\kappa_H}\,t)\) for all \(t\in\R\).
  \item If \(\kappa_H<0\), then \(H(t)=\cos(\sqrt{-\kappa_H}\,t)\) for all \(t\in\R\).
  \item If \(\kappa_H=0\), then \(H(t)=1\) for all \(t\in\R\).
\end{enumerate}
In particular, if \(\kappa_H=1\), then \(H(t)=\cosh(t)\) for all \(t\in\R\).
\end{theorem}

\begin{proof}
{\color{red}\% MZ Please check the proof, since it was missing, and I wrote it.}



By Lemma~\ref{lem:dalembert-continuity}, the existence of
\[
\lim_{t\to 0}\frac{2\,(H(t)-1)}{t^2}\in\R
\]
implies that \(H\) is continuous on \(\R\).
Hence we will apply the classical classification of continuous real-valued
solutions of the d'Alembert equation \eqref{dal} ({\color{red}\%MZ need literature for this})
\[
H(t)\equiv 1,\quad
H(t)=\cos(kt)\qquad H(t)=\cosh(kt)\;k\in\R.
\]

If \(H(t)\equiv 1\), then \(H(t)-1\equiv 0\), so \(\kappa_H=0\), and the conclusion in (3) holds.

\medskip
If \(H(t)=\cosh(kt)\) for some \(k\in\R\).
Using the Taylor expansion \(\cosh(t)=1+t^2/2+o(t^2)\) as \(t\to 0\), we have 
\[
\cosh(kt)-1=\frac{(kt)^2}{2}+o(t^2),\qquad t\to 0.
\]
Therefore,
\[
\kappa_H
=\lim_{t\to 0}\frac{2(\cosh(kt)-1)}{t^2}
=\lim_{t\to 0}\frac{2\left(\frac{k^2t^2}{2}+o(t^2)\right)}{t^2}
=k^2\ge 0.
\]
If \(\kappa_H>0\), then \(k\neq 0\) and \(|k|=\sqrt{\kappa_H}\), hence
\[
H(t)=\cosh(kt)=\cosh(\sqrt{\kappa_H}\,t),\qquad t\in\R,
\]
since \(\cosh\) is even. If \(k=0\), then \(H\equiv 1\) and \(\kappa_H=0\),
which is already covered by the firts case.

\medskip
Similarly to the second case, if \(H(t)=\cos(kt)\) for some \(k\in\R\),  the claim follows.
\end{proof}

\begin{corollary}\label{cor:cost-unique}
Let \(F:\Rp\to\R\) be normalized (\(F(1)=0\)). Assume \(F\) satisfies the
\(\Rp\) composition law (Definition~\ref{def:rp-law}) and has unit log-curvature \(\kappa(F)=1\). Then
\[
F(x)=\frac{x+x^{-1}}{2}-1
\qquad\text{for all }x>0.
\]
\end{corollary}
\begin{proof}
Let \(H(t)=F(\e^t)+1\). By Lemma~\ref{lem:equiv-rp-dalembert}, \(H\) satisfies d'Alembert
equation (\ref{dal}), and
moreover,
\[
\kappa_H=\lim_{t\to 0}\frac{2(H(t)-1)}{t^2}=\lim_{t\to 0}\frac{2F(\e^t)}{t^2}=\kappa(F)=1.
\]
From Theorem~\ref{thm:cosh-unique} then \(H(t)=\cosh(t)\), hence
\(
F(\e^t)=\cosh(t)-1=J(\e^t)
\). For \(x>0\) and \(x=\e^{\log x}\), we have \(F(x)=J(x)\).
\end{proof}



\section{Conclusion} 

%%

\begin{thebibliography}{99}

 

\bibitem{Papp}
F.J. Papp, {\it The D'Alembert functional equation,}
Amer. Math. Monthly \textbf{92} (1985), 273--275.

\bibitem{dAlembert1769}
J.~d'Alembert,
{\it M\'emoire sur les principes de m\'ecanique,}
 Hist. Acad. Sci. Paris (1769), 278--286.

\bibitem{Poisson1804} S.~Poisson,
{\it Du parall\'elogramme des forces,} Correspondance sur l'\'Ecole Polytechnique \textbf{1} (1804), 356--360.

 

\bibitem{Picard1922}
C.-E.~Picard,
{\it  Deux le\c{c}ons sur certaines \'equations fonctionnelles et la g\'eom\'etrie non-euclidienne,} Bull. Soc. Math. France \textbf{46} (1922), 404--416, 425--432.

\end{thebibliography}

 

\end{document}
 
 
\begin{lemma}\label{lem:cosh-dalembert}
The function \(\cosh:\R\to\R\) satisfies the d'Alembert equation
\[
\cosh(t+u)+\cosh(t-u)=2\,\cosh(t)\cosh(u),
\qquad \text{for all } t,u\in\R.
\]
\end{lemma}
\begin{proof}
Recall that \(\cosh(s)=\frac{e^{s}+e^{-s}}{2}\). Then
\begin{align*}
\cosh(t+u)+\cosh(t-u)
&=\frac{e^{t+u}+e^{-(t+u)}}{2}+\frac{e^{t-u}+e^{-(t-u)}}{2}\\
&=\frac{e^{t}(e^{u}+e^{-u})+e^{-t}(e^{u}+e^{-u})}{2}\\
&=\frac{(e^{t}+e^{-t})(e^{u}+e^{-u})}{2}\\
&=2\cdot \frac{e^{t}+e^{-t}}{2}\cdot \frac{e^{u}+e^{-u}}{2}\\
&=2\,\cosh(t)\cosh(u).
\end{align*}
\end{proof}