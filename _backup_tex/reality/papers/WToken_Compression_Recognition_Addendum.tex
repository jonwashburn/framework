\documentclass[12pt]{article}

% Packages (kept minimal for portability)
\usepackage[margin=1in]{geometry}
\usepackage{amsmath,amssymb,amsthm}
\usepackage{mathtools}
\usepackage{hyperref}
\usepackage{microtype}
\usepackage{booktabs}

\hypersetup{
  colorlinks=true,
  linkcolor=blue!70!black,
  citecolor=blue!70!black,
  urlcolor=blue!70!black
}

% Theorem environments
\theoremstyle{plain}
\newtheorem{theorem}{Theorem}[section]
\newtheorem{lemma}[theorem]{Lemma}
\newtheorem{corollary}[theorem]{Corollary}
\theoremstyle{definition}
\newtheorem{definition}[theorem]{Definition}
\theoremstyle{remark}
\newtheorem{remark}[theorem]{Remark}

% Notation
\newcommand{\R}{\mathbb{R}}
\newcommand{\Rp}{\R_{>0}}
\newcommand{\phiG}{\varphi}
\newcommand{\J}{J}
\newcommand{\WT}{\mathsf{WToken}}
\newcommand{\Concept}{\mathsf{Concept}}

\title{\textbf{WTokens as Compression}\\[0.25em]
\large Selection = Recognition (WToken-Specific Addendum)}

\author{Jonathan Washburn}
\date{\today}

\begin{document}
\maketitle

\begin{abstract}
This short addendum isolates the new WToken-specific consequences of the Algebra of Aboutness.
The general reference framework defines a \emph{symbol} as a configuration that both \emph{means} an object
(minimizes reference cost) and \emph{compresses} it (has strictly lower intrinsic cost).
We formalize two new claims in the Recognition Science WToken setting:
(1) \emph{WTokens as compression:} if \(\J(\WT) < \J(\Concept)\), the WToken can serve as a valid symbolic
carrier for the concept; in particular, Level-0 WTokens satisfy \(\J=0\) and therefore compress every
positive-cost concept.
(2) \emph{Selection = recognition:} choosing the optimal WToken is exactly a cost-minimizing projection onto
a discrete \(\phiG\)-lattice, i.e., an explicit \(\arg\min\) of reference mismatch.
We cite the corresponding certified results from the Lean development.
\end{abstract}

\section{Scope}
The Algebra of Aboutness provides a general, configuration-independent theory of reference.
This addendum does \emph{not} re-derive those general foundations.
Instead, it isolates the new material introduced by instantiating that framework with WTokens:
\begin{itemize}
  \item the induced WToken cost functional \(\J_W\),
  \item the compression guarantee for Level-0 WTokens, and
  \item the cost-minimizing projection (geodesic selection) used for recognition.
\end{itemize}

\section{The universal cost functional}
\begin{definition}[Recognition cost]
Let \(x\in\Rp\). Define
\begin{equation}
  \J(x) := \frac{1}{2}\bigl(x + x^{-1}\bigr) - 1.
\end{equation}
\end{definition}

This \(\J\) is nonnegative on \(\Rp\), symmetric under inversion, and has a unique zero-point at \(x=1\).

\section{WTokens as a \(\phiG\)-lattice cost space}
WToken identity includes several discrete coordinates (mode family, \(\phiG\)-level, and a \(\tau\)-offset).
For cost and reference selection, the only intrinsic coordinate is the \(\phiG\)-level.

\begin{definition}[\(\phiG\)-level ratio map]
Fix the golden ratio \(\phiG = \tfrac{1+\sqrt{5}}{2}\).
A WToken has discrete level \(k\in\{0,1,2,3\}\) and is assigned a characteristic ratio \(\phiG^k\).
Its intrinsic cost is
\begin{equation}
  \J_W(w) := \J\bigl(\phiG^k\bigr).
\end{equation}
\end{definition}

In the codebase, this is realized by \texttt{wtokenRatio} and \texttt{wtokenCost} in
\texttt{IndisputableMonolith/Foundation/WTokenReference.lean}.

\section{WTokens as compression: the Level-0 universality theorem}
In the Aboutness framework, the \emph{compression half} of symbolhood is the strict inequality
\(\J_S(s) < \J_O(o)\). The WToken specialization makes this strikingly explicit.

\begin{theorem}[Level-0 WTokens have zero intrinsic cost]
Let \(w\) be any Level-0 WToken. Then \(\J_W(w)=0\).
\end{theorem}
\begin{proof}
This is exactly the certified theorem \texttt{level0\_zero\_cost}.
Since Level-0 corresponds to exponent \(k=0\), the associated ratio is \(\phiG^0=1\), and \(\J(1)=0\).
\end{proof}

\begin{theorem}[Universal compression by Level-0]
Let \(c\) be any concept with positive intrinsic cost \(\J(c)>0\).
Let \(w_0\) be a fixed Level-0 WToken. Then
\begin{equation}
  \J_W(w_0) < \J(c).
\end{equation}
\end{theorem}
\begin{proof}
This is exactly the certified theorem \texttt{level0\_wtoken\_is\_universal\_symbol}.
By the previous theorem, \(\J_W(w_0)=0\), so the inequality reduces to \(0<\J(c)\).
\end{proof}

\begin{remark}[Interpretation]
Level-0 WTokens act as \emph{universal compressors} for any positive-cost concept.
This provides the WToken-specific semantic content of the ``mathematical backbone'':
zero intrinsic cost means the symbol can be attached to any \(\J>0\) object without violating the
compression budget.
\end{remark}

\section{Selection = recognition: cost-minimizing projection onto the ladder}
Compression alone is not the entire story; the Aboutness framework also requires that a symbol
\emph{means} an object by minimizing reference mismatch cost.
In the WToken setting, the central operational step is a projection onto the \(\phiG\)-ladder.

\subsection{Reference mismatch cost}
For a positive input ratio \(r\in\Rp\) and a candidate \(\phiG\)-level \(k\in\{0,1,2,3\}\), define
\begin{equation}
  \mathrm{Ref}(r,k) := \J\bigl(r / \phiG^k\bigr).
\end{equation}

A WToken selection rule is then an explicit \(\arg\min\) of \(\mathrm{Ref}(r,k)\) over the finite ladder.

\subsection{Geodesic selection theorem (certified)}
\begin{theorem}[Geodesic projection minimizes reference mismatch]
Let \(r\in\Rp\).
Let \(k^*\in\{0,1,2,3\}\) be the selected \(\phiG\)-level returned by \texttt{projectOntoPhiLattice}.
Then for every ladder level \(k\in\{0,1,2,3\}\),
\begin{equation}
  \J\bigl(r/\phiG^{k^*}\bigr) \le \J\bigl(r/\phiG^{k}\bigr).
\end{equation}
\end{theorem}
\begin{proof}
This is exactly the certified theorem \texttt{projection\_minimizes\_reference}.
In Lean, \texttt{projectOntoPhiLattice} is defined as the minimizer of \(\mathrm{Ref}(r,k)\) over a
finite set of \(\phiG\)-levels.
\end{proof}

\begin{remark}[Log-rounding as an efficient implementation]
Although the proof is formulated as a finite \(\arg\min\), the minimizer is well-approximated by
\(k^* \approx \mathrm{round}(\log_{\phiG}(r))\).
This explains the phenomenological ``snap'' from a continuous input field to discrete semantic atoms.
\end{remark}

\section{Recognition operator as cost minimization}
Recognition Science treats recognition as the act of selecting cost-minimizing configurations.
The WToken specialization makes this explicit: recognition of a ratio-valued input is exactly the
minimizing projection onto the \(\phiG\)-ladder.

\begin{remark}[What this addendum establishes]
\begin{itemize}
  \item \textbf{Compression:} Level-0 WTokens satisfy \(\J_W=0\) and therefore compress any \(\J>0\)
  concept.
  \item \textbf{Selection:} The WToken \(\phiG\)-level is selected by minimizing reference mismatch cost.
\end{itemize}
The remaining (domain-specific) work is to characterize when the selected WToken also satisfies the
full \emph{meaning} condition for a chosen concept class.
\end{remark}

\section*{Formal verification anchors}
\begin{itemize}
  \item \texttt{IndisputableMonolith/Foundation/WTokenReference.lean}
    \begin{itemize}
      \item \texttt{wtokenCost}, \texttt{wtokenRatio}
      \item \texttt{level0\_zero\_cost}
      \item \texttt{level0\_wtoken\_is\_universal\_symbol}
      \item \texttt{projectOntoPhiLattice}
      \item \texttt{projection\_minimizes\_reference}
    \end{itemize}
\end{itemize}

\begin{thebibliography}{9}
\bibitem{aboutness}
J. Washburn,
\emph{The Algebra of Aboutness: Reference as Cost-Minimizing Compression},
preprint (see \texttt{papers/tex/Algebra\_of\_Aboutness.tex}).
\end{thebibliography}

\end{document}
