\documentclass[12pt]{article}
\usepackage[margin=1in]{geometry}
\usepackage{amsmath,amssymb,amsthm}
\usepackage{graphicx}
\usepackage{enumitem}
\usepackage{array}
\usepackage{hyperref}

% Simple page style
\pagestyle{plain}

\newtheorem{theorem}{Theorem}
\newtheorem{lemma}[theorem]{Lemma}
\newtheorem{definition}{Definition}
\newtheorem{corollary}[theorem]{Corollary}

\begin{document}

\begin{center}
\textbf{\LARGE PATENT APPLICATION}\\[0.5cm]
\textbf{\Large Method and System for Coherence-Controlled Fusion Barrier Scaling\\and Effective Temperature Regulation}\\[1cm]

\begin{tabular}{rl}
\textbf{Application Type:} & Utility Patent \\
\textbf{Filing Date:} & January 25, 2026 \\
\textbf{Inventor:} & Jonathan Washburn \\
\textbf{Technology Field:} & Fusion Energy / Plasma Control / Quantum Engineering \\
\textbf{International Class:} & G21B 1/00; H05H 1/00; G05B 13/00 \\
\end{tabular}
\end{center}

\vspace{1cm}
\hrule
\vspace{0.5cm}

\section*{ABSTRACT}

A method and system for controlling nuclear fusion reaction rates by modulating the coherence of an energy delivery system. The invention introduces a ``Barrier Scale'' metric $S = 1/(1 + C_\varphi + C_\sigma)$ computed from a temporal coherence metric ($C_\varphi$) and a symmetry ledger synchronization metric ($C_\sigma$). The barrier scale is applied within an explicit, auditable tunneling proxy by computing an RS-adjusted exponent $\eta_{\text{RS}} = S \cdot \eta_{\text{classical}}$ and an RS tunneling proxy $P_{\text{RS}} = \exp(-\eta_{\text{RS}})$. In this proxy, the system also computes an effective-temperature identity $T_{\text{eff}} = T/S^2$ (a model-layer equivalence used for control decisions and certificates). The method provides a deterministic, verifiable control loop for regulating fusion gain via pulse timing precision and symmetry optimization rather than solely by increasing thermal energy, while explicitly separating certified computations from facility calibration seams.

\vspace{0.5cm}
\hrule
\vspace{0.5cm}

\section{BACKGROUND OF THE INVENTION}

\subsection{Technical Field}

This invention relates generally to nuclear fusion reactor control, and more particularly to methods for enhancing fusion reaction rates by manipulating the quantum-mechanical tunneling probability through coherence optimization of the driver system.

\subsection{Description of Related Art}

Achieving controlled nuclear fusion requires overcoming the Coulomb barrier, the electrostatic repulsion between positively charged nuclei. In conventional fusion approaches (magnetic confinement, inertial confinement), this is achieved primarily by heating the fuel to extreme temperatures ($T > 10$ keV) so that the high-energy tail of the Maxwell-Boltzmann distribution has sufficient energy to tunnel through the barrier.

The tunneling probability is governed by the Gamow factor $\exp(-\eta)$, where $\eta \propto 1/\sqrt{T}$. Conventional control strategies focus on maximizing $T$ (heating) and $n\tau$ (confinement). However, heating is energy-intensive and leads to instabilities (e.g., disruptions, radiative losses). There is a need for a control method that can enhance the reaction rate without solely relying on brute-force temperature increases.

\section{SUMMARY OF THE INVENTION}

The present invention provides a method for ``Coherence-Controlled Fusion,'' where the effective tunneling barrier is reduced by increasing the coherence of the energy delivery system.

The core innovation is the computation and application of a \textbf{Barrier Scale} factor $S$, defined as:
\[
S = \frac{1}{1 + C_\varphi + C_\sigma}
\]
where $C_\varphi \in [0,1]$ is the temporal $\varphi$-coherence of the driver pulses (e.g., jitter minimization, golden-ratio timing) and $C_\sigma \in [0,1]$ is the ledger synchronization of the implosion symmetry.

The invention exploits the discovery that the effective Gamow exponent in a coherent system scales as $\eta_{\text{RS}} = S \cdot \eta_{\text{classical}}$. Since $S \le 1$, this reduces the exponential suppression of the reaction rate. This is mathematically equivalent to operating at an effective temperature $T_{\text{eff}} = T/S^2$.

By measuring $C_\varphi$ and $C_\sigma$ in real-time (or predicting them from shot parameters), the control system can:
\begin{enumerate}
    \item Compute the enhanced tunneling probability $P_{\text{RS}}$.
    \item Determine the required physical temperature $T_{\text{needed}} = S^2 \cdot T_{\text{classical}}$ to achieve a target reaction rate.
    \item Actuate driver parameters (timing precision, beam balance) to \textbf{minimize $S$} (equivalently, maximize $1/S^2$) and thus minimize the energy input required for a target proxy reaction rate.
\end{enumerate}

\section{BRIEF DESCRIPTION OF THE DRAWINGS}

\begin{itemize}
    \item \textbf{FIG. 1} is a block diagram of the coherence-controlled fusion system.
    \item \textbf{FIG. 2} illustrates the Barrier Scale $S$ as a function of coherence parameters $C_\varphi$ and $C_\sigma$.
    \item \textbf{FIG. 3} shows the effective temperature gain $T_{\text{eff}}/T$ versus coherence.
    \item \textbf{FIG. 4} is a flowchart of the control method.
\end{itemize}

\section{DETAILED DESCRIPTION OF EMBODIMENTS}

\subsection{Definitions}

\begin{itemize}
    \item \textbf{Barrier Scale ($S$):} A dimensionless factor $S \in (0, 1]$ quantifying the reduction in the effective Coulomb barrier.
    \item \textbf{$\varphi$-Coherence ($C_\varphi$):} A normalized metric $\in [0,1]$ quantifying the temporal precision and phase alignment of the driver pulses relative to a golden-ratio schedule.
    \item \textbf{Ledger Synchronization ($C_\sigma$):} A normalized metric $\in [0,1]$ quantifying the spatial symmetry of the implosion, derived from the Symmetry Ledger $L$.
    \item \textbf{Gamow Exponent ($\eta$):} The exponent governing tunneling probability, $\eta \approx 31.3 Z_1 Z_2 \sqrt{\mu/T}$.
    \item \textbf{Effective Temperature ($T_{\text{eff}}$):} The temperature at which a classical system would exhibit the same tunneling probability as the coherent system at physical temperature $T$.
\end{itemize}

\subsection{System Architecture}

The system comprises:
\begin{enumerate}
    \item \textbf{Diagnostic Module:} Inputs raw data (pulse timing logs, X-ray imaging) and computes the raw error metrics.
    \item \textbf{Coherence Estimator:} Maps raw errors to normalized metrics $C_\varphi$ and $C_\sigma$ using facility-specific calibration curves.
    \item \textbf{Barrier Scale Engine:} Computes $S = 1/(1 + C_\varphi + C_\sigma)$ and the resulting effective physics parameters.
    \item \textbf{Reactor Controller:} Adjusts machine parameters (laser timing, beam power) to reduce $S$ (equivalently, increase $1/S^2$) and regulate the fusion burn.
\end{enumerate}

\subsection{Method of Operation}

The control loop executes the following steps:

\subsubsection{1. Measurement}
The system measures the temporal jitter $\delta t$ of the driver pulses and the mode asymmetry amplitudes $a_{\ell m}$ of the plasma.

\subsubsection{2. Coherence Computation}
The raw measurements are mapped to normalized coherence metrics.
For temporal coherence:
\[
C_\varphi = \mathrm{clamp}_{[0,1]}\!\Bigl(R \cdot \mathrm{timingScore} \cdot \mathrm{skewScore}\Bigr)
\]
where $R \in [0,1]$ is a phase-alignment (mean-resultant-length) metric and the timing and skew scores are (in one Lean-aligned embodiment):
\[
\mathrm{timingScore} = \frac{1}{1 + (j_{\mathrm{rms}}/j_{\mathrm{scale}})^2},\qquad
\mathrm{skewScore} = \frac{1}{1 + (s_{\mathrm{rms}}/s_{\mathrm{scale}})^2}.
\]
Here $j_{\mathrm{rms}}$ is RMS timing jitter from measured vs expected pulse times, $s_{\mathrm{rms}}$ is RMS inter-channel timing skew, and $j_{\mathrm{scale}}, s_{\mathrm{scale}}$ are facility calibration scales. Other monotone calibration mappings (e.g., exponential) are permitted as an explicit seam so long as the emitted artifact records the mapping and parameters.
For symmetry synchronization:
\[
C_\sigma = \frac{1}{1 + L/\Lambda}
\]
where $L$ is the Symmetry Ledger value (weighted sum of J-costs of mode ratios) and $\Lambda$ is a normalization constant.

\subsubsection{3. Barrier Scale Calculation}
The core computation is performed:
\[
S = \frac{1}{1 + C_\varphi + C_\sigma}
\]
This calculation is rigorous and verified. Note that if $C_\varphi = C_\sigma = 0$ (incoherent), then $S=1$ (classical limit). If $C_\varphi = C_\sigma = 1$ (perfect coherence), $S = 1/3$.

\subsubsection{4. Effective Parameter Derivation}
The system derives the effective physics parameters for the current state:
\begin{itemize}
    \item \textbf{Effective Gamow Exponent:} $\eta_{\text{RS}} = S \cdot \eta_{\text{classical}}(T)$
    \item \textbf{Tunneling Probability:} $P_{\text{tunnel}} = \exp(-\eta_{\text{RS}})$
    \item \textbf{Effective Temperature:} $T_{\text{eff}} = T / S^2$
\end{itemize}
The relation $T_{\text{eff}} = T/S^2$ is an exact identity in this model. For $S=1/3$, the effective temperature is $9\times$ the physical temperature.

\subsubsection{5. Control Action}
The controller compares $P_{\text{tunnel}}$ to the target reaction rate. If the rate is too low, the controller can:
\begin{itemize}
    \item Increase physical heating (conventional).
    \item \textbf{Increase coherence} (the inventive step): Reduce jitter, improve beam balance, or refine pulse shaping to increase $C_\varphi$ and $C_\sigma$, thereby decreasing $S$ and increasing $P_{\text{tunnel}}$ without adding thermal energy.
\end{itemize}

\subsection{Seams and Calibration}

The formula $S = 1/(1 + C_\varphi + C_\sigma)$ is the certified core of the invention. However, the mapping from physical diagnostics to the abstract metrics $C_\varphi$ and $C_\sigma$ constitutes an \textbf{empirical seam}.
\begin{itemize}
    \item The system requires a \textbf{Calibration Envelope} defining the functions $f_{\text{time}}(\delta t) \to C_\varphi$ and $f_{\text{sym}}(L) \to C_\sigma$.
    \item These calibration functions are facility-specific and must be determined via reference shots.
    \item The patent claims the \textit{method of using this computed S for control}, regardless of the specific calibration constants used to normalize the inputs.
\end{itemize}

\section{CLAIMS}

\begin{enumerate}
    \item \textbf{A method for controlling a nuclear fusion reactor, comprising:}
    \begin{enumerate}
        \item measuring a temporal coherence metric $C_\varphi$ characterizing the timing precision of energy delivery pulses;
        \item measuring a symmetry synchronization metric $C_\sigma$ characterizing the spatial symmetry of the fusion fuel implosion;
        \item computing a barrier scale factor $S$ according to the relationship $S = 1 / (1 + C_\varphi + C_\sigma)$;
        \item determining an effective fusion reaction temperature $T_{\text{eff}}$ based on a physical temperature $T$ and the barrier scale factor $S$; and
        \item adjusting reactor control parameters to \textbf{minimize} the barrier scale factor $S$ (or maximize an effective-temperature gain $1/S^2$) or maintain $T_{\text{eff}}$ above a target ignition threshold.
    \end{enumerate}

    \item The method of claim 1, wherein the effective fusion reaction temperature is computed as $T_{\text{eff}} = T / S^2$.

    \item The method of claim 1, wherein adjusting reactor control parameters comprises reducing timing jitter of driver pulses to increase $C_\varphi$.

    \item The method of claim 1, wherein the symmetry synchronization metric $C_\sigma$ is derived from a convex symmetry ledger functional $L$ of normalized mode amplitudes.

    \item \textbf{A fusion reactor control system comprising:}
    \begin{enumerate}
        \item a coherence estimator configured to compute normalized temporal and spatial coherence metrics from reactor diagnostics;
        \item a barrier scale engine configured to calculate a scalar $S$ inversely proportional to the sum of unity and the coherence metrics;
        \item a tunneling probability calculator configured to compute a reaction rate proxy $\exp(-S \cdot \eta)$ where $\eta$ is a classical Gamow exponent; and
        \item a feedback controller configured to modulate energy delivery parameters to reduce the calculated scalar $S$ (or increase an effective-temperature gain $1/S^2$).
    \end{enumerate}

    \item The system of claim 5, wherein the barrier scale engine implements the formula $S = 1/(1 + C_\varphi + C_\sigma)$.

    \item \textbf{A non-transitory computer-readable medium storing instructions that, when executed by a processor, cause a control system to:}
    \begin{enumerate}
        \item receive diagnostic data regarding driver pulse timing and target implosion symmetry;
        \item calculate a barrier scale factor $S$ quantifying the reduction in effective Coulomb barrier due to coherence;
        \item compute an effective temperature gain factor $G = 1/S^2$; and
        \item generate control signals to maintain $G$ above a predetermined efficiency threshold.
    \end{enumerate}
\end{enumerate}

\section*{APPENDIX: Implementation Evidence}

The core logic of this invention is implemented in the accompanying software artifacts:
\begin{itemize}
    \item \textbf{Python Implementation:} \texttt{fusion/simulator/coherence/barrier\_scale.py} implements the \texttt{compute\_rs\_barrier\_scale} function and the \texttt{RSCoherenceParams} data structure.
    \item \textbf{Formal Verification:} The mathematical derivations are verified in Lean 4 in \texttt{IndisputableMonolith/Fusion/ReactionNetworkRates.lean}, specifically the theorems \texttt{rsBarrierScale\_le\_one} and \texttt{rsGamowExponent\_le\_gamowExponent}.
\end{itemize}

\end{document}
