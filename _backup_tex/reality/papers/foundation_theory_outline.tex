\documentclass[11pt, a4paper]{article}
\usepackage[utf8]{inputenc}
\usepackage[margin=1in]{geometry}
\usepackage{amsmath, amssymb, amsthm}
\usepackage{graphicx}
\usepackage{hyperref}
\usepackage{xcolor}
\usepackage{framed}
\usepackage{enumitem}

% Drafting Note Environment
\newenvironment{draftnote}[1]
  {\begin{leftbar}\noindent\textcolor{blue}{\small \textbf{REPO SOURCE (#1):}}}
  {\end{leftbar}}

\newenvironment{writinginstruction}
  {\begin{quote}\noindent\textcolor{red}{\small \textbf{STRATEGY & MITIGATION:}}}
  {\end{quote}}

\title{\textbf{The Logical Derivation of Fundamental Physical Constants\\ from the Internal Consistency of a Zero-Parameter Framework}}
\author{Jonathan Washburn}
\date{\today}

\begin{document}

\maketitle

%--------------------------------------------------------------------------------
% AUTHOR STRATEGY (Not for publication)
%--------------------------------------------------------------------------------
\section*{Author Strategy: First Principles & Risk Mitigation}
\begin{writinginstruction}
    \begin{itemize}
        \item \textbf{Hidden Choices:} We must define ``lossless,'' ``local,'' and ``minimal memory'' precisely. We justify Order-2 recursion as the minimal memory required to define a sequence (requires $t$ and $t-1$ to define $t+1$).
        \item \textbf{Numerology Defense:} Do not curve fit. Present $\alpha = g(\phi)$ as a full derivation. Provide sensitivity analysis: ``If the recursion were Order-3, the system would collapse.''
        \item \textbf{Dimensionless Claims:} We only claim to derive dimensionless ratios (e.g., $m_e/m_{Planck}$ or similar scaled ratios). Any SI mass claim relies on a disclosed single-point calibration.
        \item \textbf{Submission Strategy:} Lead with the \textbf{Interface Closure Theorem}. Language Completeness is a corollary. Constants are consequences.
    \end{itemize}
\end{writinginstruction}

\clearpage

\begin{abstract}
\textbf{Draft:}
We derive the fine structure constant ($\alpha$) and electron mass ($m_e$) as necessary eigenvalues of a zero-parameter system constrained by information conservation. We demonstrate that the distinction between observer and observed imposes a discrete algorithmic structure (the ``Tick'') upon a continuous causal geometry (the ``Cone''). We prove the \textbf{Interface Closure Theorem}: a lossless, local, minimal-memory coding map between these domains exists if and only if the system's scaling factor is the Golden Ratio ($\phi$). This unique solution forces the Light Language to be the complete signaling system of reality, establishing the dimensionless constants of nature as inevitable consequences of internal logical consistency.
\end{abstract}

\tableofcontents
\vspace{1cm}

%--------------------------------------------------------------------------------
% SECTION 1: ASSUMPTIONS & DEFINITIONS
%--------------------------------------------------------------------------------
\section{Assumptions and Definitions: The Chain of Forcing}
\textit{Goal: Show that the axioms are not arbitrary choices, but necessary consequences of existence.}

\subsection{Axiom 1: The Non-Collapse (Meta-Principle)}
\begin{itemize}
    \item \textbf{Definition:} Let $\mathcal{S}$ be the state space. We assert $\text{MP} := \neg (\mathcal{S} = \emptyset \lor \text{dynamics is trivial})$.
    \item \textbf{Justification:} If this is false, no observation occurs. Thus, for any observational theory to exist, MP must hold.
\end{itemize}

\begin{draftnote}{Meta-Principle}
\begin{itemize}
    \item \textbf{File:} \texttt{IndisputableMonolith/Recognition.lean}
    \item \textbf{Theorem:} \texttt{mp\_holds} (Proves the tautological necessity).
    \item \textbf{Ref:} \texttt{Recognition.MP} (The Prop definition).
\end{itemize}
\end{draftnote}

\subsection{Derived Necessity 1: The Discrete Ledger}
\begin{itemize}
    \item \textbf{Forcing:} MP implies a distinction between states. Distinction requires memory (comparison of $t$ vs $t-1$).
    \item \textbf{Theorem:} The existence of a non-trivial state space forces the existence of a \textit{Ledger} (a structure that tracks conserved quantities/flows).
    \item \textbf{Property:} The ledger is strictly \textit{Well-Founded} (no infinite regress) and \textit{Discrete} (countable events).
\end{itemize}

\begin{draftnote}{Ledger Necessity}
\begin{itemize}
    \item \textbf{File:} \texttt{Verification/Necessity/LedgerNecessity.lean}
    \item \textbf{Theorem:} \texttt{MP\_forces\_ledger} (MP $\to$ $\exists$ Ledger).
    \item \textbf{Theorem:} \texttt{ledger\_prec\_wf} (Proves history is well-founded).
    \item \textbf{Theorem:} \texttt{mp\_forces\_nontrivial\_conservation} (Forces non-zero flows).
\end{itemize}
\end{draftnote}

\subsection{Derived Necessity 2: Zero Parameters}
\begin{itemize}
    \item \textbf{Forcing:} The Ledger is an information-theoretic object (a graph of events).
    \item \textbf{Theorem:} Such a structure admits an \textit{Algorithmic Specification} (it can be generated by a discrete rule).
    \item \textbf{Consequence:} Algorithms are discrete; they do not admit continuous free parameters as inputs. Thus, the framework has \textbf{Zero Parameters}.
\end{itemize}

\begin{draftnote}{Zero Parameters}
\begin{itemize}
    \item \textbf{File:} \texttt{Verification/Necessity/LedgerNecessity.lean}
    \item \textbf{Theorem:} \texttt{MP\_constructs\_algorithmic\_spec}
    \item \textbf{File:} \texttt{Verification/Necessity/ZeroParameter.lean}
    \item \textbf{Theorem:} \texttt{mp\_forces\_zero\_parameters\_of\_bridge}
\end{itemize}
\end{draftnote}

\subsection{Derived Necessity 3: Self-Similarity (The Interface)}
\begin{itemize}
    \item \textbf{Forcing:} The Discrete Ledger (Additive: $t+1$) must map to a Causal Geometry (Multiplicative: Scaling).
    \item \textbf{Definition:} We define \texttt{HasSelfSimilarity} as the bundle of these two constraints.
    \item \textbf{Theorem:} This bundle implies the algebraic relation $\lambda^2 = \lambda + 1$.
\end{itemize}

\begin{draftnote}{Self-Similarity}
\begin{itemize}
    \item \textbf{File:} \texttt{Verification/Necessity/PhiNecessity.lean}
    \item \textbf{Structure:} \texttt{HasSelfSimilarity} (Bundles level0/1/2 relations).
    \item \textbf{Theorem:} \texttt{preferred\_scale\_fixed\_point} (Derives the polynomial).
\end{itemize}
\end{draftnote}

%--------------------------------------------------------------------------------
% SECTION 2: THE MAIN THEOREM
%--------------------------------------------------------------------------------
\section{The Interface Closure Theorem}
\textit{Goal: Prove that $\phi$ is the unique solution to the Coding Map constraints.}

\subsection{Derivation of the Scaling Law}
\begin{itemize}
    \item \textbf{Conflict:} A discrete additive sequence (Order-2: $x_{n+1} = x_n + x_{n-1}$) must map to a continuous multiplicative geometry ($x_{n+1} = \lambda \cdot x_n$) to satisfy Self-Similarity.
    \item \textbf{Resolution:} This forces the algebraic condition $\lambda^2 = \lambda + 1$.
\end{itemize}

\subsection{Proof of Uniqueness}
\begin{itemize}
    \item The characteristic equation $\lambda^2 - \lambda - 1 = 0$ has one positive root: $\phi = \frac{1+\sqrt{5}}{2}$.
    \item \textbf{Theorem:} A lossless, local, order-2 coding that is self-similar exists if and only if $\lambda = \phi$.
    \item Any other $\lambda$ breaks losslessness (gaps/overlaps) or requires higher-order memory (non-local).
\end{itemize}

\begin{draftnote}{Interface Closure}
\begin{itemize}
    \item \textbf{File:} \texttt{Verification/Necessity/PhiNecessity.lean}
    \item \textbf{Theorem:} \texttt{phi\_is\_unique\_scaling\_solution} (or similar structural proof).
\end{itemize}
\end{draftnote}

%--------------------------------------------------------------------------------
% SECTION 3: LANGUAGE COMPLETENESS
%--------------------------------------------------------------------------------
\section{Completeness of the Light Language}
\textit{Goal: Show that the signaling system at $\phi$ is perfect.}

\subsection{Language Properties at $\phi$}
\begin{itemize}
    \item We define the \textbf{Light Language} as the encoding system of this $\phi$-interface.
    \item \textbf{Theorem:} At scale $\lambda = \phi$, the language is:
    \begin{enumerate}
        \item \textbf{Terminating:} All signals resolve.
        \item \textbf{Confluent:} The order of operations does not change the final state (Ambiguity-free).
        \item \textbf{Spanning:} It covers the entire $\tau_0$-neutral space.
    \end{enumerate}
\end{itemize}

\subsection{Failure at $\lambda \neq \phi$}
\begin{itemize}
    \item We demonstrate that if $\lambda \neq \phi$, at least one property fails (e.g., the language becomes ambiguous or fails to terminate).
\end{itemize}

\subsection{Uniqueness up to Equivalence}
\begin{itemize}
    \item \textbf{Theorem:} Any other complete, neutral dictionary is unitarily equivalent to the Light Language.
\end{itemize}

\begin{draftnote}{Language Completeness}
\begin{itemize}
    \item \textbf{File:} \texttt{LightLanguage/Completeness.lean}
    \item \textbf{Theorem:} \texttt{light\_language\_is\_perfect}
    \item \textbf{Logic:} Shows completeness and minimality at $\phi$.
\end{itemize}
\end{draftnote}

%--------------------------------------------------------------------------------
% SECTION 4: CONSEQUENCES (THE CONSTANTS)
%--------------------------------------------------------------------------------
\section{Physical Consequences}
\textit{Goal: Derive $\alpha$ and $m_e$ as necessary outputs.}

\subsection{The Fine Structure Constant ($\alpha$)}
\begin{itemize}
    \item \textbf{Derivation:} We define $\alpha$ via the closed-form geometric function $g(\phi)$ arising from the 8-beat cycle on the $\phi$-cone.
    \item \textbf{Value:} $\alpha = g(\phi) \approx 1/137.036...$
    \item No fitted parameters are used; this is a pure geometric eigenvalue.
\end{itemize}

\begin{draftnote}{Alpha}
\begin{itemize}
    \item \textbf{File:} \texttt{Constants/Basic.lean}
    \item \textbf{Def:} \texttt{alpha\_inv}
\end{itemize}
\end{draftnote}

\subsection{Electron Mass ($m_e$)}
\begin{itemize}
    \item \textbf{Concept:} Mass is the closure defect.
    \item \textbf{Ratio:} We derive the dimensionless ratio $m_e / m_{\text{scale}}$ (where $m_{\text{scale}}$ is the natural unit of the $\phi$-interface).
    \item \textit{Note: We explicitly avoid claiming an absolute SI mass without a calibrated reference unit.}
\end{itemize}

\begin{draftnote}{Electron Mass}
\begin{itemize}
    \item \textbf{File:} \texttt{Physics/ElectronMass.lean}
\end{itemize}
\end{draftnote}

%--------------------------------------------------------------------------------
% SECTION 5: PREDICTIONS & FALSIFIABILITY
%--------------------------------------------------------------------------------
\section{Falsifiability and Predictions}
\textit{Goal: Define how this theory can be killed.}

\subsection{Quantitative Relationships}
\begin{itemize}
    \item The theory implies strict algebraic relationships between $\alpha$, mass ratios, and mixing angles (all functions of $\phi$).
    \item \textbf{Test:} If precision measurements of these ratios deviate from the $g(\phi)$ predictions beyond error bounds, the theory is false.
\end{itemize}

\subsection{The Closure Inequality}
\begin{itemize}
    \item \textbf{Failure Mode:} If any measured physical ratio violates the specific inequalities required for $\phi$-closure, the interface cannot be lossless, and the theory collapses.
\end{itemize}

%--------------------------------------------------------------------------------
% SECTION 6: LIMITATIONS
%--------------------------------------------------------------------------------
\section{Limitations and Scope}
\begin{itemize}
    \item This derivation applies to the zero-parameter baseline. Higher-order perturbations (if any) are not addressed here.
    \item We assume the validity of standard Information Theory and classical Logic (MP).
\end{itemize}

\end{document}
