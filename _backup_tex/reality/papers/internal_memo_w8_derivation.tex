\documentclass[11pt]{article}

\usepackage[margin=1in]{geometry}
\usepackage{amsmath, amssymb, mathtools}
\usepackage{booktabs}
\usepackage{enumitem}
\usepackage{xcolor}
\usepackage[hidelinks]{hyperref}
\usepackage{listings}

\definecolor{codebg}{RGB}{248,248,248}
\definecolor{codeframe}{RGB}{220,220,220}
\lstset{
  basicstyle=\ttfamily\small,
  backgroundcolor=\color{codebg},
  frame=single,
  rulecolor=\color{codeframe},
  columns=fullflexible,
  keepspaces=true,
  showstringspaces=false
}

\newcommand{\phiRS}{\varphi}
\newcommand{\wEight}{w_8}
\newcommand{\Fin}{\mathrm{Fin}}

\title{Internal Memo: Derivation and Rationale for the RS Gap Weight $\wEight$}
\author{Recognition Science Engineering (Lean formalization)}
\date{\today}

\begin{document}
\maketitle

\section*{Executive summary (one page)}
\begin{itemize}[leftmargin=1.5em]
  \item \textbf{Where $\wEight$ is used.} In the Lean $\alpha$ pipeline we use a single gap term
  \[
    f_{\mathrm{gap}} = \wEight \ln(\phiRS),
    \qquad
    \alpha^{-1} = 4\pi\cdot 11 - \bigl(f_{\mathrm{gap}} + \delta_\kappa\bigr).
  \]
  \item \textbf{Canonical (parameter-free) definition.} The current Lean source of truth is
  \[
    \boxed{
    \wEight \;:=\; \frac{348 + 210\sqrt{2} - (204 + 130\sqrt{2})\,\phiRS}{7}
    }
    \;\approx\; 2.49056927545\ldots
  \]
  (Lean: \texttt{IndisputableMonolith/Constants/GapWeight.lean}).
  \item \textbf{How it is derived (high level).}
  \begin{enumerate}[leftmargin=1.5em]
    \item Take the canonical 8-sample \emph{$\phi$-pattern} $p(t)=\phiRS^t$ on $t\in\Fin 8$.
    \item Decompose $p$ in the \emph{unitary DFT-8 basis} $c_k=\sum_t \overline{\mathrm{DFT}_{t,k}}\,p(t)$.
    \item Form a weighted neutral-mode energy sum $\sum_{k\neq 0} |c_k|^2\,g_k(\phiRS)$, with
    $g_k(\phiRS)=\sin^2(\pi k/8)\,\phiRS^{-k}$.
    \item Normalize by total DFT energy (Parseval) to remove dependence on overall scaling,
    and scale to RS ``per 8-tick cube-cell'' units. The normalization is the entire difference
    between the intuitive ``raw energy'' calculations and the pipeline $\wEight$.
  \end{enumerate}
  \item \textbf{Why the team scientist got $\approx 0.970692\ldots$.}
  That value corresponds to a \emph{different quantity}:
  \[
    \wEight^{\mathrm{ratio}} \;:=\; \sum_{k=1}^{7}\frac{|c_k|^2}{|c_0|^2}
    \;\approx\;0.970692434668\ldots
  \]
  It is a diagnostic ``neutral-vs-DC energy ratio'' for the raw $\phi$-pattern and does \emph{not}
  include (i) geometric weights or (ii) the projection/normalization needed to make a dimensionless
  RS gap coefficient suitable to multiply $\ln(\phiRS)$.
\end{itemize}

\section{Context: what problem $\wEight$ solves}
RS uses an 8-tick discrete clock (T8) as the fundamental ledger-compatible period. Independently,
RS uses $\phiRS$ as the unique scale factor connecting discrete ledger recursion to continuous
self-similar scaling. When you sample a $\phi$-ladder walk on an 8-tick window, the resulting
pattern is \emph{not} purely DC; it has neutral oscillatory content. The gap term measures the
information/strain cost of representing that neutral content when the ledger insists on 8-tick
periodicity.

In other words:
\begin{quote}
  \emph{The gap is the penalty for forcing a continuous self-similar $\phi$-ladder onto an 8-tick
  discrete clock.}
\end{quote}

\section{Step 1: the canonical $\phi$-pattern on eight ticks}
Define the canonical 8-tick sample of the $\phi$-ladder by
\[
  p(t) := \phiRS^t, \qquad t\in\{0,1,\dots,7\}.
\]
Lean: \texttt{Constants/GapWeight/Formula.lean} defines \texttt{phiPattern}.

\subsection*{Why this is the right starting point (the ``why'')}
\begin{itemize}[leftmargin=1.5em]
  \item \textbf{Eight ticks are not a choice.} In RS, ledger-compatible evolution forces a
  $2^D$-period; with $D=3$ this is 8 ticks. So the minimal clock window is length 8.
  \item \textbf{$\phiRS$ is not a choice.} RS derives $\phiRS$ as the unique fixed point of the
  reciprocal-symmetric cost recursion (the same reason Fibonacci asymptotically approaches $\phiRS$).
  \item \textbf{Sampling $t\mapsto \phiRS^t$ is the minimal $\phi$-ladder witness.} It is the
  simplest ``pure scaling'' sequence. Any other $\phi$-ladder walk differs by a phase/offset or
  by additional structure (which must be justified separately).
\end{itemize}

\section{Step 2: DFT-8 decomposition}
Let $\omega := e^{-2\pi i/8}$. The unitary DFT-8 matrix entry is
\[
  \mathrm{DFT}_{t,k} := \frac{\omega^{tk}}{\sqrt{8}}.
\]
Define DFT coefficients by the standard unitary convention
\[
  c_k := \sum_{t=0}^{7} \overline{\mathrm{DFT}_{t,k}}\;p(t)
  = \frac{1}{\sqrt{8}}\sum_{t=0}^{7} \phiRS^t \omega^{-tk}.
\]
Lean: \texttt{LightLanguage/Basis/DFT8.lean} defines \texttt{dft8\_entry}; the coefficient is
\texttt{phiDFTCoeff} in \texttt{Constants/GapWeight/Formula.lean}.

\subsection*{Why DFT-8 is forced (the ``why'')}
\begin{itemize}[leftmargin=1.5em]
  \item \textbf{DFT-8 is the unique unitary diagonalizer of the 8-tick shift.}
  The 8-tick clock has a canonical ``shift by one tick'' symmetry. The DFT basis is the
  (essentially unique) unitary basis that diagonalizes this cyclic shift. If we want a basis
  that respects 8-tick time-translation, DFT-8 is canonical.
  \item \textbf{DFT separates ``mean'' from ``shape''.}
  The $k=0$ coefficient is the DC component (the mean / overall scale). Modes $k\neq 0$ are the
  neutral, mean-free components. A gap penalty should only charge the \emph{shape} mismatch, not
  the overall scale offset, hence we focus on $k\neq 0$.
\end{itemize}

\section{Step 3: neutral-mode energies and geometric weights}
Define the mode energies
\[
  A_k := |c_k|^2,\qquad k=0,1,\dots,7.
\]
Define the geometric weights for $k\neq 0$ by
\[
  g_k(\phiRS) := \sin^2\!\Bigl(\frac{\pi k}{8}\Bigr)\,\phiRS^{-k}, \qquad g_0:=0.
\]
The \emph{raw weighted neutral sum} is then
\[
  C_{\mathrm{raw}}(\phiRS) := \sum_{k=1}^{7} A_k\,g_k(\phiRS).
\]

\subsection*{Why these weights (the ``why'')}
\begin{itemize}[leftmargin=1.5em]
  \item \textbf{$\sin^2(\pi k/8)$ is the \emph{forced} spectral weight of the 8-tick derivative/Laplacian.}
  On the 8-tick clock, the canonical symmetry is the one-tick cyclic shift $S$.
  The DFT-8 modes are eigenvectors of $S$ with eigenvalues $\omega^k$.
  The unique (up to scale) local, shift-invariant quadratic ``variation'' energy on the cycle is
  the discrete derivative / Laplacian energy, e.g.\ $\sum_t |(S-I)x(t)|^2$.
  On mode $k$ this contributes the eigenvalue weight
  \[
    |\,\omega^k-1\,|^2 \;=\; 2-2\cos(2\pi k/8) \;=\; 4\sin^2(\pi k/8).
  \]
  Therefore the appearance of $\sin^2(\pi k/8)$ is not ad hoc: it is exactly the spectral
  footprint of the canonical 8-tick derivative/Laplacian.
  \item \textbf{$\phiRS^{-k}$ is the \emph{forced} transport attenuation from $\phi$-ladder rung coupling.}
  RS already uses $\phi$ as the unique multiplicative scale step. If coupling/transport decreases
  exponentially with rung separation (the minimal parameter-free choice), then each additional
  separation step multiplies by $\phiRS^{-1}$, hence mode index $k$ contributes $\phiRS^{-k}$.
  \item \textbf{Net effect:} low-$k$ modes are weighted strongly (they are the dominant, physically
  coherent deviations); high-$k$ modes are weighted weakly (they are rapidly oscillatory and do not
  carry long-range coherent ``gap'' structure).
\end{itemize}

\section{Step 4: the projection/normalization that makes $\wEight$ a true RS coefficient}
The raw sum $C_{\mathrm{raw}}$ is \emph{not} an RS coefficient yet: it scales quadratically with
the overall scale of the pattern. To produce a dimensionless, scale-invariant coefficient we:
\begin{enumerate}[leftmargin=1.5em]
  \item \textbf{Normalize by total energy.} Let
  \[
    E_{\mathrm{tot}} := \sum_{k=0}^{7} A_k.
  \]
  By Parseval (unitary DFT),
  \[
    E_{\mathrm{tot}} = \sum_{t=0}^{7} |p(t)|^2 = \sum_{t=0}^{7} \phiRS^{2t}.
  \]
  This removes dependence on scaling $p\mapsto \lambda p$.
  \item \textbf{Scale to the RS 8-tick cube-cell.}
  RS is ledger-local on a $Q_3$ cell (8 vertices) and uses an octave clock (8 ticks).
  Once we have a scale-invariant fraction, we must choose the \emph{measure} over which the gap
  cost is accumulated. The canonical choice is the fundamental interface cell:
  \[
    \text{cell index set} = (\text{ticks})\times(\text{vertices})
    = \Fin 8 \times \Fin 8,
    \qquad |\Fin 8 \times \Fin 8| = 8\cdot 8 = 64.
  \]
  Multiplying by $64$ converts the invariant fraction into a per-cell integrated weight.
  This is now recorded explicitly in Lean as a definition-level ``projection operator''
  (see \texttt{IndisputableMonolith/Constants/GapWeight/Projection.lean}).
\end{enumerate}

We therefore define the \emph{normalized projection weight}
\[
  \boxed{
  \wEight \;:=\; 64\;\frac{C_{\mathrm{raw}}(\phiRS)}{E_{\mathrm{tot}}(\phiRS)}.
  }
\]
Numerically, for $p(t)=\phiRS^t$, this evaluates to
\[
  \wEight \approx 2.49056927545\ldots
\]
and in Lean is exposed in the closed form
\[
  \wEight = \frac{348 + 210\sqrt{2} - (204 + 130\sqrt{2})\,\phiRS}{7}.
\]

\subsection*{Why this normalization is necessary (the ``why'')}
\begin{itemize}[leftmargin=1.5em]
  \item \textbf{A coefficient multiplying $\ln(\phiRS)$ must be scale-invariant.}
  Since $\ln(\phiRS)$ is a fixed bit-cost per scale step, its prefactor must not depend on how we
  choose units for the $\phi$-pattern amplitude.
  \item \textbf{The raw DFT sum is not invariant under $p\mapsto \lambda p$.}
  $A_k$ scales as $\lambda^2$, so any unnormalized sum of $A_k$ is an arbitrary choice of amplitude
  scale, i.e.\ an illicit parameter.
  \item \textbf{Normalizing by total energy makes it parameter-free.}
  $C_{\mathrm{raw}}/E_{\mathrm{tot}}$ is invariant under $p\mapsto \lambda p$.
  \item \textbf{The remaining $64$ is a unit convention tied to the RS fundamental cell.}
  It converts a dimensionless fraction into the per-8-tick-per-$Q_3$-cell coefficient that belongs
  in the $\alpha^{-1}$ ledger accounting.
\end{itemize}

\section{Why the scientist got $\approx 0.970692\ldots$ (and why it is not $\wEight$)}
The common ``neutral energy ratio'' is
\[
  \wEight^{\mathrm{ratio}} := \sum_{k=1}^{7}\frac{|c_k|^2}{|c_0|^2}.
\]
For the $\phi$-pattern, $\wEight^{\mathrm{ratio}}\approx 0.970692434668\ldots$
(this appears in the paper \texttt{papers/Dic\_23\_refereed.tex}).

\textbf{Key point:} $\wEight^{\mathrm{ratio}}$ answers:
\begin{quote}
  ``How much non-DC energy exists relative to the DC energy in one 8-sample $\phi$ window?''
\end{quote}
Whereas $\wEight$ answers:
\begin{quote}
  ``What is the \emph{normalized, geometrically weighted} projection cost of the neutral spectrum,
  expressed in the RS 8-tick cube-cell units, suitable to multiply $\ln(\phiRS)$?''
\end{quote}
These are different by design. The ratio is a useful diagnostic sanity-check; it is not the
pipeline coefficient.

\section{Implementation pointers (Lean source of truth)}
\begin{itemize}[leftmargin=1.5em]
  \item \textbf{Canonical value used in $\alpha$:} \texttt{IndisputableMonolith/Constants/GapWeight.lean}
  defines \texttt{w8\_from\_eight\_tick} in closed form and \texttt{f\_gap := w8 * log(phi)}.
  \item \textbf{DFT definitions:} \texttt{IndisputableMonolith/LightLanguage/Basis/DFT8.lean}.
  \item \textbf{$\phi$-pattern DFT scaffolding:} \texttt{IndisputableMonolith/Constants/GapWeight/Formula.lean}
  defines \texttt{phiPattern}, \texttt{phiDFTCoeff}, \texttt{phiDFTAmplitude}, and \texttt{geometricWeight}.
  \item \textbf{Rigorous bounds:} \texttt{IndisputableMonolith/Numerics/Interval/W8Bounds.lean}.
  \item \textbf{$\alpha^{-1}$ assembly:} \texttt{IndisputableMonolith/Constants/Alpha.lean} and
  interval bounds \texttt{IndisputableMonolith/Numerics/Interval/AlphaBounds.lean}.
\end{itemize}

\section{Open item: reconciling with older ``certified'' $2.488254\ldots$}
Some older artifacts (e.g.\ \texttt{data/certificates/w8.json}) record $2.488254397846$ as ``$w_8$''.
The current Lean definition is \emph{parameter-free} and evaluates to $2.49056927545\ldots$.

If the $2.488254\ldots$ number is intended to be canonical, then the framework must specify an
additional, explicit projection/transport step (often described informally as ``gap-45 resonance'')
that changes the effective weight. That step must be stated as a mathematical operator and then
proved to yield the number; otherwise the $2.488254\ldots$ value is just an external calibration.

\section*{Appendix: quick numerical sanity-check}
For $\phiRS=(1+\sqrt5)/2$:
\[
  \wEight \approx 2.49056927545,\qquad
  f_{\mathrm{gap}} = \wEight\ln(\phiRS) \approx 1.19849138648.
\]

\end{document}


