\documentclass[11pt]{article}
\usepackage[margin=1in]{geometry}
\usepackage{hyperref}
\usepackage{enumitem}
\title{Recognition Science: 100-Paper Publishing Order}
\date{\today}
\begin{document}
\maketitle
\section*{Method}
This document proposes a 100-paper publishing order that follows Recognition Science derivation dependencies. Each entry includes detected RS topics, inferred prerequisites, and a short rationale. Topic tagging is keyword-based and intended for triage; it can be refined as needed.
\section*{Dependency Legend}
\begin{itemize}[leftmargin=*]
\item RG: Recognition Geometry / quotient observables
\item RCL: Recognition Composition Law / d\'Alembert structure
\item Jcost: Canonical reciprocal cost uniqueness
\item Existence: Law of existence / defect = 0
\item Discreteness: Discrete forcing / finite resolution / 8-tick
\item Ledger: Double-entry ledger / conservation
\item Phi: Golden ratio forcing
\item Dimension: D=3 forcing
\item Rhat: Recognition operator / dynamics
\item Constants: c, hbar, G, alpha derivations
\end{itemize}
\section*{Ordered List (100 Papers)}
\subsection*{1.  Recognition Geometry \textbackslash{}\textbackslash{}  A Complete Mathematical Framework \textbackslash{}\textbackslash{}  Formalized in Lean 4}
\textbf{File:} \texttt{papers/tex/recognition-geometry.tex}\\
\textbf{Detected topics:} RG, Discreteness, Dimension\\
\textbf{Prereqs:} Jcost, Ledger, Phi\\
\textbf{Reasoning:} Depends on Jcost, Ledger, Phi; advances topics: RG, Discreteness, Dimension.\\
\textbf{Summary:} Recognition Geometry is a new geometric framework that inverts the traditional relationship between space and measurement. In classical geometry, space is primitive and measurements are operations performed on a pre-existing spatial substrate. Recognition Geometry reverses this: recognition maps are primitive, and spac...
\subsection*{2.  The Recognition Composition Law\textbackslash{}\textbackslash{}[0.5em]  The Single Primitive of Recognition Science}
\textbf{File:} \texttt{papers/tex/Recognition\_Composition\_Law\_Primer.tex}\\
\textbf{Detected topics:} RCL, Jcost, Existence\\
\textbf{Prereqs:} RCL, Jcost\\
\textbf{Reasoning:} Depends on RCL, Jcost; advances topics: RCL, Jcost, Existence.\\
\textbf{Summary:} We present the Recognition Composition Law, the foundational axiom of Recognition Science from which all physical structure emerges. This functional equation constrains how recognition costs compose under multiplication and division, and---combined with minimal normalization conditions---uniquely determines the cost fu...
\subsection*{3. Uniqueness of the Canonical Reciprocal Cost}
\textbf{File:} \texttt{papers/UNIQUENESS OF THE CANONICAL RECIPROCAL COST.tex}\\
\textbf{Detected topics:} Jcost\\
\textbf{Prereqs:} RCL\\
\textbf{Reasoning:} Depends on RCL; advances topics: Jcost.\\
\textbf{Summary:} Cont... Keywords: Mathematics Subject Classifications (2010):
\subsection*{4. D'Alembert Inevitability:\textbackslash{}; Polynomial Consistency Forces the Canonical Composition Law on }
\textbf{File:} \texttt{papers/tex/DAlembert\_Inevitability.tex}\\
\textbf{Detected topics:} RCL, Ledger, Exclusivity\\
\textbf{Prereqs:} Jcost, Discreteness, Phi, Dimension\\
\textbf{Reasoning:} Depends on Jcost, Discreteness, Phi, Dimension; advances topics: RCL, Ledger, Exclusivity.\\
\textbf{Summary:} Let F: be a real-valued functional on multiplicative ratios. A common modeling move in functional-equation based theories is to postulate a specific composition identity relating F(xy)+F(x/y) to the pair (F(x),F(y)). Even in ``zero-parameter'' settings, the choice of composition law is itself an implicit parameter choi...
\subsection*{5. Model-Independent Exclusivity on the Quotient State Space\textbackslash{}\textbackslash{}  Recognition Science as an Inevitability Theorem for Zero-Parameter Frameworks}
\textbf{File:} \texttt{papers/tex/Model-Independent-Exclusivity-Quotient.tex}\\
\textbf{Detected topics:} RCL, Jcost, Existence, Phi, Exclusivity\\
\textbf{Prereqs:} RCL, Jcost, Ledger, Phi, Dimension\\
\textbf{Reasoning:} Depends on RCL, Jcost, Ledger, Phi, Dimension; advances topics: RCL, Jcost, Existence, Phi, Exclusivity.\\
\textbf{Summary:} We prove a model-independent exclusivity theorem for Recognition Science (RS) on the quotient state space: states are identified when they are observationally indistinguishable (i.e.\textbackslash{} yield the same observable output). Working with an abstract ``physics framework'' consisting of a state space, an evolution operator, an...
\subsection*{6. The Cost of Existence: A First-Principles Derivation of Physical Law from the Recognition Composition Law}
\textbf{File:} \texttt{The\_Cost\_of\_Existence.tex}\\
\textbf{Detected topics:} RCL, Jcost, Existence, Discreteness, Ledger, Dimension, Constants\\
\textbf{Prereqs:} RCL, Jcost, Discreteness, Ledger, Phi, Dimension\\
\textbf{Reasoning:} Depends on RCL, Jcost, Discreteness, Ledger, Phi, Dimension; advances topics: RCL, Jcost, Existence, Discreteness, Ledger, Dimension, Constants.\\
\textbf{Summary:} Standard physical theories typically postulate the existence of a manifold, a set of logical axioms, and initial conditions as irreducible priors. This paper proposes a ``Cost-First'' foundation where these elements are derived rather than assumed. Starting from a single primitive constraint--the Recognition Composition...
\subsection*{7. The\_Law\_of\_Inevitable\_Unity.tex}
\textbf{File:} \texttt{papers/The\_Law\_of\_Inevitable\_Unity.tex}\\
\textbf{Detected topics:} RCL, Jcost, Dimension, Language\\
\textbf{Prereqs:} RCL, Jcost, Existence, Ledger, Phi\\
\textbf{Reasoning:} Depends on RCL, Jcost, Existence, Ledger, Phi; advances topics: RCL, Jcost, Dimension, Language.\\
\textbf{Summary:} \textbackslash{} (RS) is a parameter-free framework whose single primitive is a coherence law for costs on ratios. Its primitive---the Recognition Composition Law ()---is equivalent (under log substitution) to the d'Alembert functional equation and, with standard cost side-conditions (symmetry, positivity, coercivity, and normalizati...
\subsection*{8. Logic Emerges from Physical Cost\textbackslash{}\textbackslash{}[0.3em]  Logical Consistency as a Low-Energy State\textbackslash{}\textbackslash{}[0.2em] Proof as Geodesic, Existence as Stability}
\textbf{File:} \texttt{papers/tex/Logic\_From\_Physical\_Cost.tex}\\
\textbf{Detected topics:} RCL, Jcost, Existence, Language\\
\textbf{Prereqs:} RCL, Jcost, Existence\\
\textbf{Reasoning:} Depends on RCL, Jcost, Existence; advances topics: RCL, Jcost, Existence, Language.\\
\textbf{Summary:} We develop a radical inversion of the traditional relationship between logic and physics: rather than assuming logic as a pre-given foundation upon which physics is built, we derive a canonical logical semantics as emergent from a physical cost functional. The canonical reciprocal cost J(x) = 1\{2(x + x\textasciicircum{}\{-1) - 1 is uniq...
\subsection*{9. Recognition Science: A Zero-Parameter Framework \textbackslash{}\textbackslash{}[0.5em] Deriving Fundamental Constants from Logical Necessity}
\textbf{File:} \texttt{papers/tex/RS-Foundations.tex}\\
\textbf{Detected topics:} Existence, Discreteness, Ledger, Phi, Constants, Mass, Gravity, Exclusivity\\
\textbf{Prereqs:} Jcost, Discreteness, Ledger, Phi, Dimension, Rhat, Constants\\
\textbf{Reasoning:} Depends on Jcost, Discreteness, Ledger, Phi, Dimension, Rhat, Constants; advances topics: Existence, Discreteness, Ledger, Phi, Constants, Mass, Gravity, Exclusivity.\\
\textbf{Summary:} We present Recognition Science (RS), a theoretical framework that derives all fundamental physical constants---the speed of light c, Planck's constant , Newton's gravitational constant G, and the fine-structure constant ---from a single logical principle with zero adjustable parameters. The framework simultaneously res...
\subsection*{10. EightAxiomsForced.tex}
\textbf{File:} \texttt{papers/tex/EightAxiomsForced.tex}\\
\textbf{Detected topics:} Jcost, Existence, Ledger, Phi, Constants, Gravity, Exclusivity\\
\textbf{Prereqs:} RCL, Jcost, Discreteness, Ledger, Phi, Dimension, Constants\\
\textbf{Reasoning:} Depends on RCL, Jcost, Discreteness, Ledger, Phi, Dimension, Constants; advances topics: Jcost, Existence, Ledger, Phi, Constants, Gravity, Exclusivity.\\
\textbf{Summary:} We show that a single logical tautology--the Meta\textbackslash{},Principle (MP), "nothing cannot recognize itself"--forces eight core theorems (T1-T8) that pin down the recognition ledger, the unique convex symmetric cost J(x)=12(x+x\textasciicircum{}\{-1)-1 (with fixed local scale), the golden\textbackslash{},ratio fixed point via \textasciicircum{}2=+1, an eight\textbackslash{},tick minimal updat...
\subsection*{11. The Golden Ratio as a Universal Coherence Eigenvalue:\textbackslash{}\textbackslash{} Bridging Penrose Aperiodic Order and Information-Theoretic Comparison}
\textbf{File:} \texttt{papers/tex/Penrose\_golden\_ratio\_and\_ledger\_structure.tex}\\
\textbf{Detected topics:} Jcost, Ledger, Phi\\
\textbf{Prereqs:} RCL, Jcost, Discreteness, Ledger\\
\textbf{Reasoning:} Depends on RCL, Jcost, Discreteness, Ledger; advances topics: Jcost, Ledger, Phi.\\
\textbf{Summary:} The golden ratio \textbackslash{}(=1+5\{2\textbackslash{}) occupies a distinguished position in mathematics, appearing across diverse domains from number theory and dynamical systems to geometric tilings and quasicrystal physics. This paper establishes a formal bridge between two independently motivated occurrences of \textbackslash{}(\textbackslash{}): as the forced inflation e...
\subsection*{12. Geometric Necessity of Recognition Angle \textbackslash{}\textbackslash{}  Forcing  = 1/4 from Minimal Axioms \textbackslash{}\textbackslash{}[0.3cm]  With Complete Lean 4 Formalization}
\textbf{File:} \texttt{papers/tex/Geometric-Necessity-Recognition-Angle.tex}\\
\textbf{Detected topics:} RCL, Jcost, Dimension\\
\textbf{Prereqs:} RCL, Ledger, Phi\\
\textbf{Reasoning:} Depends on RCL, Ledger, Phi; advances topics: RCL, Jcost, Dimension.\\
\textbf{Summary:} We prove that the recognition angle = (1/4) 75.52° is forced in the highest sense: it is the unique value consistent with minimal axioms, with no free parameters. Starting from first principles of binary recognition, we establish two rigidity theorems: enumerate Coupling Rigidity (Angle T5): The d'Alembert functional e...
\subsection*{13. Dimensional Rigidity: D=3 from Linking of Loops, Kepler Stability, and Minimal Dyadic Synchronization}
\textbf{File:} \texttt{papers/tex/Dimensional\_Rigidity\_D3.tex}\\
\textbf{Detected topics:} RCL, Discreteness, Phi, Dimension\\
\textbf{Prereqs:} Jcost, Ledger, Phi\\
\textbf{Reasoning:} Depends on Jcost, Ledger, Phi; advances topics: RCL, Discreteness, Phi, Dimension.\\
\textbf{Summary:} We give three mathematically precise constraints that each single out the spatial dimension D=3. First, we show that an integer-valued linking invariant for disjoint oriented loops (embedded copies of S\textasciicircum{}1) exists only in D=3; this follows from Alexander duality, since for an embedded circle K \textasciicircum{}D the group H\_1(\textasciicircum{}D K) is...
\subsection*{14. The\_Recognition\_Operator.tex}
\textbf{File:} \texttt{papers/root\_papers/The\_Recognition\_Operator.tex}\\
\textbf{Detected topics:} Rhat\\
\textbf{Prereqs:} Jcost, Ledger, Dimension\\
\textbf{Reasoning:} Depends on Jcost, Ledger, Dimension; advances topics: Rhat.\\
\textbf{Summary:} \{ The Recognition Operator
\subsection*{15. The Derivation of Physical Constants from the Meta-Principle:\textbackslash{}\textbackslash{}[0.5em]  A Complete Chain of Custody from Logic to Cosmology}
\textbf{File:} \texttt{papers/tex/Formalized-Derivations-T1-T8.tex}\\
\textbf{Detected topics:} Existence, Discreteness, Ledger, Dimension, Constants, Mass, Gravity\\
\textbf{Prereqs:} Jcost, Discreteness, Ledger, Phi, Dimension, Rhat, Constants\\
\textbf{Reasoning:} Depends on Jcost, Discreteness, Ledger, Phi, Dimension, Rhat, Constants; advances topics: Existence, Discreteness, Ledger, Dimension, Constants, Mass, Gravity.\\
\textbf{Summary:} We present a rigorous derivation of the fundamental constants of physics starting from a single logical axiom: the Meta-Principle (MP) stating that ``Nothing cannot recognize itself.'' We demonstrate that this axiom forces a discrete recognition process, which in turn imposes a unique topological structure on the vacuu...
\subsection*{16. Coercive Projection Method:\textbackslash{}\textbackslash{} Rigorous Derivation of Constants from First Principles\textbackslash{}\textbackslash{}[0.5em]  Supporting Technical Document}
\textbf{File:} \texttt{papers/tex/CPM\_Constants\_Derivation.tex}\\
\textbf{Detected topics:} Phi, Dimension, Constants, Gravity\\
\textbf{Prereqs:} Jcost, Ledger, Phi, Dimension, Constants\\
\textbf{Reasoning:} Depends on Jcost, Ledger, Phi, Dimension, Constants; advances topics: Phi, Dimension, Constants, Gravity.\\
\textbf{Summary:} This document provides rigorous mathematical derivations of all constants appearing in the Coercive Projection Method (CPM) and its gravitational instantiation (CPM-Gravity / ILG). Every constant is derived from explicit axioms or standard mathematical results---no assumptions are made without proof. This document dire...
\subsection*{17.  A First-Principles Derivation of Particle Mass\textbackslash{}\textbackslash{}[0.3em]  Geometric Origin of the Charged Lepton Spectrum}
\textbf{File:} \texttt{papers/tex/Full\_First\_Principles\_Mass\_Derivation.tex}\\
\textbf{Detected topics:} RG, Jcost, Existence, Discreteness, Ledger, Dimension, Constants, Mass\\
\textbf{Prereqs:} RCL, Jcost, Discreteness, Ledger, Phi, Dimension, Rhat, Constants\\
\textbf{Reasoning:} Depends on RCL, Jcost, Discreteness, Ledger, Phi, Dimension, Rhat, Constants; advances topics: RG, Jcost, Existence, Discreteness, Ledger, Dimension, Constants, Mass.\\
\textbf{Summary:} The Standard Model of particle physics is remarkably successful but structurally incomplete: it requires the masses of fermions to be inserted as free parameters (Yukawa couplings). This paper presents a first-principles derivation of these masses within the framework of Recognition Science. By treating particle existe...
\subsection*{18.  Neutrino Sector No--Go under Dirac Z\_=0 at a Single Anchor:\textbackslash{} Acceptance Failure and Paths to Resolution}
\textbf{File:} \texttt{papers/tex/Neutrino-Sector.tex}\\
\textbf{Detected topics:} Existence, Discreteness, Mass\\
\textbf{Prereqs:} Jcost, Rhat, Constants\\
\textbf{Reasoning:} Depends on Jcost, Rhat, Constants; advances topics: Existence, Discreteness, Mass.\\
\textbf{Summary:} We report a no--go result for closing the light neutrino sector within the mass framework under the current axioms: Dirac neutrinos with vanishing word--charge at the universal anchor (Z\_=0), a single common transport , and the formal rung triplet (r\_1,r\_2,r\_3)=(0,11,19). Using the same acceptance test that organizes c...
\subsection*{19. Interacting, BRST-Consistent Quantum Gravity in the Recognition Calculus:\textbackslash{}\textbackslash{} Proven Zero-Parameter Framework, Construction, and Audit Interfaces}
\textbf{File:} \texttt{papers/root\_papers/quantum\_gravity\_B\_v4.tex}\\
\textbf{Detected topics:} Existence, Discreteness, Phi, Constants, Mass, Gravity, Exclusivity\\
\textbf{Prereqs:} Jcost, Ledger, Phi, Dimension, Rhat, Constants\\
\textbf{Reasoning:} Depends on Jcost, Ledger, Phi, Dimension, Rhat, Constants; advances topics: Existence, Discreteness, Phi, Constants, Mass, Gravity, Exclusivity.\\
\textbf{Summary:} We present the Recognition Calculus (RC) quantum gravity construction as a proved, zero-parameter framework: all fundamental constants \textbackslash{}\{c,,G,\textasciicircum{}\{-1,,E\_\{coh,\_0,\_\{rec\textbackslash{} and the mass-to-light ratio M/L are derived from the Meta-Principle (MP) with no free parameters. The RC exclusivity and completeness theorems referenced h...
\subsection*{20. Information-Limited Gravity II: Test Program, Linear Signatures, and Stage-IV Tests}
\textbf{File:} \texttt{papers/root\_papers/Dark\_Energy\_Paper2\_v4.tex}\\
\textbf{Detected topics:} Ledger, Constants, Gravity\\
\textbf{Prereqs:} Jcost, Discreteness, Ledger, Phi, Dimension, Constants\\
\textbf{Reasoning:} Depends on Jcost, Discreteness, Ledger, Phi, Dimension, Constants; advances topics: Ledger, Constants, Gravity.\\
\textbf{Summary:} We present Paper\textasciitilde{}II of Information-Limited Gravity (). Building on the fixed-kernel framework and mathematical foundations established in Paper\textasciitilde{}I PaperI, we assemble an observationally vulnerable test program and a compact set of working expressions for linear, k-resolved signatures. \textbackslash{} is a source-side modification of...
\subsection*{21. Zero-Parameter Galaxy Rotation Curves from Information-Limited Gravity:\textbackslash{}\textbackslash{} A Lean-Verified Test Against 99 SPARC Galaxies}
\textbf{File:} \texttt{papers/ILG\_Galaxy\_Rotation\_Curves.tex}\\
\textbf{Detected topics:} Phi, Constants, Mass, Gravity\\
\textbf{Prereqs:} Jcost, Ledger, Phi, Dimension, Rhat, Constants\\
\textbf{Reasoning:} Depends on Jcost, Ledger, Phi, Dimension, Rhat, Constants; advances topics: Phi, Constants, Mass, Gravity.\\
\textbf{Summary:} We present the first formally-verified, zero-parameter numerical test of a modified gravity theory against empirical galaxy rotation curves. Information-Limited Gravity (ILG), derived from Recognition Science axioms, predicts rotation velocities using parameters determined entirely by the golden ratio = (1+5)/2: the dy...
\subsection*{22. Convergence of Empirical Optimization and First-Principles Derivation in Galactic Dynamics: A Unified Validation of Recognition Science}
\textbf{File:} \texttt{papers/ILG\_Validation\_Synthesis.tex}\\
\textbf{Detected topics:} Phi, Gravity\\
\textbf{Prereqs:} Ledger, Dimension, Constants\\
\textbf{Reasoning:} Depends on Ledger, Dimension, Constants; advances topics: Phi, Gravity.\\
\textbf{Summary:} We present a unified analysis of two independent tests of the Information-Limited Gravity (ILG) framework against the SPARC galaxy rotation curve database. The first test, a phenomenological optimization, treated the model's seven global parameters as free variables, using differential evolution to minimize residuals a...
\subsection*{23. Recognition-Riemann-Final.tex}
\textbf{File:} \texttt{papers/tex/Recognition-Riemann-Final.tex}\\
\textbf{Detected topics:} Phi, Riemann\\
\textbf{Prereqs:} RG, Jcost, Ledger\\
\textbf{Reasoning:} Depends on RG, Jcost, Ledger; advances topics: Phi, Riemann.\\
\textbf{Summary:} We realise (s)\textasciicircum{}\{-1 as a -regularised Fredholm determinant \_2 of A(s)=e\textasciicircum{}\{-sH, where the arithmetic Hamiltonian H\_\{p=( p)\_\{p acts on the weighted space \_\{=\textasciicircum{}\{2(P,p\textasciicircum{}\{-2(1+)) with =-10.618. On this space A(s) is Hilbert--Schmidt precisely for the half--strip 12< s<1, and within that domain \textbackslash{}[ \_\{2(I-A(s))E(s)=(s)\textasciicircum{}\{-1, \textbackslash{}] whe...
\subsection*{24.  Protein Folding from First Principles\textbackslash{}\textbackslash{}[0.8em]  Recognition Science Without Machine Learning\textbackslash{}\textbackslash{}[0.5em]  How the Bio-Clocking Theorem Resolves Levinthal's Paradox}
\textbf{File:} \texttt{papers/tex/protein-dec-6.tex}\\
\textbf{Detected topics:} Jcost, Existence, Discreteness, Phi, Constants, Biology\\
\textbf{Prereqs:} RCL, Jcost, Ledger, Phi, Dimension, Rhat\\
\textbf{Reasoning:} Depends on RCL, Jcost, Ledger, Phi, Dimension, Rhat; advances topics: Jcost, Existence, Discreteness, Phi, Constants, Biology.\\
\textbf{Summary:} The protein folding problem has been approached primarily through data-driven methods, with recent breakthroughs from AlphaFold and ESMFold achieving remarkable accuracy by learning from millions of known structures. We present a fundamentally different approach: deriving protein folding behavior from atomic chemistry...
\subsection*{25. Topological Origins of Nuclear Binding Energy Corrections\textbackslash{}\textbackslash{}[0.3cm]  Deriving Shell Structure from an 8-Tick Ledger Topology}
\textbf{File:} \texttt{papers/tex/Topological\_Origins\_Nuclear\_Binding\_Energy.tex}\\
\textbf{Detected topics:} Discreteness, Ledger, Mass, Fusion\\
\textbf{Prereqs:} Jcost, Discreteness, Rhat, Constants\\
\textbf{Reasoning:} Depends on Jcost, Discreteness, Rhat, Constants; advances topics: Discreteness, Ledger, Mass, Fusion.\\
\textbf{Summary:} We present a novel derivation of nuclear shell corrections to the semi-empirical mass formula (SEMF) from first principles, based on a discrete topological structure we call the ``8-tick ledger.'' The magic numbers \textbackslash{}\{2, 8, 20, 28, 50, 82, 126\textbackslash{} emerge naturally as stability maxima in this framework, without requiring th...
\subsection*{26. Nuclear Magic Numbers from Ledger Topology:\textbackslash{}\textbackslash{} A Recognition Science Derivation}
\textbf{File:} \texttt{fusion/papers/Nuclear\_Magic\_Numbers\_RS\_Derivation.tex}\\
\textbf{Detected topics:} Discreteness, Ledger, Dimension, Fusion\\
\textbf{Prereqs:} Jcost, Discreteness, Ledger, Phi, Rhat, Constants\\
\textbf{Reasoning:} Depends on Jcost, Discreteness, Ledger, Phi, Rhat, Constants; advances topics: Discreteness, Ledger, Dimension, Fusion.\\
\textbf{Summary:} We derive the nuclear magic numbers \textbackslash{}\{2, 8, 20, 28, 50, 82, 126\textbackslash{} from Recognition Science (RS) first principles, demonstrating that these stability markers emerge from the same 8-tick ledger topology that forces noble gas closures in chemistry. Unlike standard nuclear physics, which fits these numbers using Woods-Saxon...
\subsection*{27. Recognition geometry}
\textbf{File:} \texttt{papers/tex/RG-9-12.tex}\\
\textbf{Detected topics:} RG\\
\textbf{Prereqs:} None (base-level)\\
\textbf{Reasoning:} Depends on None (base-level); advances topics: RG.\\
\textbf{Summary:} Recognition geometry is.. Keywords: Mathematics Subject Classifications (2010)
\subsection*{28. Recognition geometry}
\textbf{File:} \texttt{tmp/riemann-rs-geometry/riemann-rs-geometry-main/RG-9-12.tex}\\
\textbf{Detected topics:} RG\\
\textbf{Prereqs:} None (base-level)\\
\textbf{Reasoning:} Depends on None (base-level); advances topics: RG.\\
\textbf{Summary:} Recognition geometry is.. Keywords: Mathematics Subject Classifications (2010)
\subsection*{29. Response to Recognition Geometry Comments}
\textbf{File:} \texttt{papers/tex/RG-Response-Dec11.tex}\\
\textbf{Detected topics:} RG\\
\textbf{Prereqs:} None (base-level)\\
\textbf{Reasoning:} Depends on None (base-level); advances topics: RG.\\
\textbf{Summary:} Add a remark noting these extensions for future work.
\subsection*{30. T5 Cost Uniqueness and the Certificate Circle\textbackslash{}\textbackslash{}  What Completing ``T5'' Certifies in the reality Repository}
\textbf{File:} \texttt{papers/tex/T5\_Cost\_Uniqueness\_Certificate\_Circle.tex}\\
\textbf{Detected topics:} RCL, Jcost\\
\textbf{Prereqs:} RCL\\
\textbf{Reasoning:} Depends on RCL; advances topics: RCL, Jcost.\\
\textbf{Summary:} This note explains the mathematical and engineering content of completing ``T5'' in the reality repository's Lean formalization workflow. In this codebase, T5 is packaged as a certificate (IndisputableMonolith/Verification/T5UniqueCert.lean) asserting a uniqueness theorem for the Recognition Science cost function \textbackslash{}(J\textbackslash{})...
\subsection*{31. The Recognition Composition Law for Zeta Zeros:\textbackslash{}\textbackslash{} A New Mathematical Framework}
\textbf{File:} \texttt{papers/tex/RECOGNITION\_COMPOSITION\_LAW.tex}\\
\textbf{Detected topics:} RCL, Existence, Riemann\\
\textbf{Prereqs:} RG, Jcost\\
\textbf{Reasoning:} Depends on RG, Jcost; advances topics: RCL, Existence, Riemann.\\
\textbf{Summary:} We introduce a new mathematical structure---the Recognition Composition Law---that connects the d'Alembert functional equation governing the RS cost function to constraints on the zero distribution of the Riemann zeta function. We define the zero defect functional and prove several rigorous theorems about its relations...
\subsection*{32. The Prime Stiffness Theorem and the Riemann Hypothesis\textbackslash{}\textbackslash{}[0.5em]  A Conditional Framework with Identified Technical Gaps}
\textbf{File:} \texttt{papers/tex/RH\_Prime\_Stiffness\_Proof-alt.tex}\\
\textbf{Detected topics:} Discreteness, Riemann\\
\textbf{Prereqs:} RG, Jcost\\
\textbf{Reasoning:} Depends on RG, Jcost; advances topics: Discreteness, Riemann.\\
\textbf{Summary:} We present a framework for proving the Riemann Hypothesis from the discrete nature of prime numbers. The key insight is the Prime Stiffness Theorem: because primes are distinct integers with gaps 1, finite prime sums are bandwidth-limited, which implies gradient bounds via Bernstein's inequality. Status: This framework...
\subsection*{33. Prolate spheroidal operator and Zeta}
\textbf{File:} \texttt{../AI/repos/riemann/Notes/Papers/Connes/Draft2.tex}\\
\textbf{Detected topics:} Discreteness, Riemann\\
\textbf{Prereqs:} RG, Jcost\\
\textbf{Reasoning:} Depends on RG, Jcost; advances topics: Discreteness, Riemann.\\
\textbf{Summary:} In this paper we describe a remarkable new property of the self-adjoint extension of the prolate spheroidal operator introduced in college98,CMbook. The restriction of this operator to the interval J whose characteristic function commutes with it is well known, has discrete positive spectrum and is well understood Slep...
\subsection*{34. Prolate spheroidal operator and Zeta}
\textbf{File:} \texttt{../Riemann/riemann-main/Notes/Papers/Connes/Draft2.tex}\\
\textbf{Detected topics:} Discreteness, Riemann\\
\textbf{Prereqs:} RG, Jcost\\
\textbf{Reasoning:} Depends on RG, Jcost; advances topics: Discreteness, Riemann.\\
\textbf{Summary:} In this paper we describe a remarkable new property of the self-adjoint extension of the prolate spheroidal operator introduced in college98,CMbook. The restriction of this operator to the interval J whose characteristic function commutes with it is well known, has discrete positive spectrum and is well understood Slep...
\subsection*{35. Prolate spheroidal operator and Zeta}
\textbf{File:} \texttt{../archive/goldbach/riemann\_lean\_repo/Notes/Papers/Connes/Draft2.tex}\\
\textbf{Detected topics:} Discreteness, Riemann\\
\textbf{Prereqs:} RG, Jcost\\
\textbf{Reasoning:} Depends on RG, Jcost; advances topics: Discreteness, Riemann.\\
\textbf{Summary:} In this paper we describe a remarkable new property of the self-adjoint extension of the prolate spheroidal operator introduced in college98,CMbook. The restriction of this operator to the interval J whose characteristic function commutes with it is well known, has discrete positive spectrum and is well understood Slep...
\subsection*{36. The Energetic Necessity of the Riemann Hypothesis:\textbackslash{} the Prime Distribution from the Law of Existence}
\textbf{File:} \texttt{papers/tex/RS\_Axiomatic\_Proof\_RH.tex}\\
\textbf{Detected topics:} Existence, Riemann\\
\textbf{Prereqs:} RG, Jcost\\
\textbf{Reasoning:} Depends on RG, Jcost; advances topics: Existence, Riemann.\\
\textbf{Summary:} The Riemann Hypothesis (RH) remains unproven in standard Zermelo-Fraenkel set theory (ZFC) because ZFC treats all logically consistent objects as equally existent, regardless of their complexity or ``cost.'' We present a proof of RH within the axiomatic framework of Recognition Science (RS). By adopting the Law of Exis...
\subsection*{37. The Medium-Arc L\textasciicircum{}4 Saving Conjecture\textbackslash{}\textbackslash{}[0.5em]  A Sufficient Condition for Goldbach via the Circle Method}
\textbf{File:} \texttt{../archive/goldbach/MED\_L4\_THEOREM.tex}\\
\textbf{Detected topics:} RCL, Riemann\\
\textbf{Prereqs:} RG, Jcost\\
\textbf{Reasoning:} Depends on RG, Jcost; advances topics: RCL, Riemann.\\
\textbf{Summary:} We formulate a precise conjecture about exponential sums over primes on medium arcs in the circle method. This conjecture, if true, would imply that every sufficiently large even integer is a sum of two primes. We state the conjecture, explain its context, and outline a potential proof strategy based on dispersion meth...
\subsection*{38. The Symmetry Resonance Theorem:\textbackslash{}\textbackslash{} A Novel Characterization of the Critical Line}
\textbf{File:} \texttt{papers/tex/SYMMETRY\_RESONANCE\_THEOREM.tex}\\
\textbf{Detected topics:} Existence, Riemann\\
\textbf{Prereqs:} RG, Jcost\\
\textbf{Reasoning:} Depends on RG, Jcost; advances topics: Existence, Riemann.\\
\textbf{Summary:} We introduce the concept of symmetry resonance for zeta zeros and prove that the critical line is uniquely characterized as the locus where two fundamental symmetries---the functional equation and complex conjugation---become resonant (i.e., identical in their action). We prove that this resonance imposes constraints o...
\subsection*{39.  A Proof of the Riemann Hypothesis:\textbackslash{}\textbackslash{} Via Transfer Operator Spectral Analysis}
\textbf{File:} \texttt{papers/tex/Riemann-July-7.tex}\\
\textbf{Detected topics:} Riemann\\
\textbf{Prereqs:} RG, Jcost\\
\textbf{Reasoning:} Depends on RG, Jcost; advances topics: Riemann.\\
\textbf{Summary:} We present a proof of the Riemann Hypothesis through construction of a transfer operator whose Fredholm determinant equals (s)\textasciicircum{}\{-1 and whose spectral gap off the critical line forces all zeros to s = 1/2. All technical components have been rigorously established.
\subsection*{40.  Hypothesis Proof \textbackslash{}\textbackslash{}  Complete Lean 4 Source Code \textbackslash{}\textbackslash{}  Repository: recognition-riemann}
\textbf{File:} \texttt{../Riemann/Riemann-final/Riemann-Not-Used/riemann-verification/recognition-riemann/riemann/no-zeros/riemann\_hypothesis\_proof\_source.tex}\\
\textbf{Detected topics:} Riemann\\
\textbf{Prereqs:} RG, Jcost\\
\textbf{Reasoning:} Depends on RG, Jcost; advances topics: Riemann.\\
\textbf{Summary:} This document contains the complete Lean 4 source code for a formalization of a proof of the Riemann Hypothesis. The repository structure follows a pinch-certificate approach using Schur bounds, Poisson representation, and removable singularity theory. Current Status: The proof uses a private axiom in the export layer...
\subsection*{41.  From Local Height Diagonalization to Birch--Swinnerton--Dyer:\textbackslash{}\textbackslash{} A Prime-wise Program for =0,  Finiteness, and p-Parts}
\textbf{File:} \texttt{../arXiv Papers/New Papers/Papers/BSD-Full-Closure.tex}\\
\textbf{Detected topics:} Riemann\\
\textbf{Prereqs:} RG, Jcost\\
\textbf{Reasoning:} Depends on RG, Jcost; advances topics: Riemann.\\
\textbf{Summary:} We present a classical, prime-wise route toward the Birch--Swinnerton--Dyer conjecture for elliptic curves over Q that converts local p-adic height information into global consequences. The core mechanism is a reduction-order separation criterion that upper-triangularizes the cyclotomic p-adic height Gram matrix modulo...
\subsection*{42. A certified zero-free region for the Riemann zeta function\textbackslash{} the half-plane  s  0.6}
\textbf{File:} \texttt{../Holder/paper1\_farfield\_v4\_static.tex}\\
\textbf{Detected topics:} Riemann\\
\textbf{Prereqs:} RG, Jcost\\
\textbf{Reasoning:} Depends on RG, Jcost; advances topics: Riemann.\\
\textbf{Summary:} We prove unconditionally that the Riemann zeta function (s) has no zeros in the fixed half-plane \textbackslash{}\{\textbackslash{}, s 0.6\textbackslash{},\textbackslash{}. The argument is function-theoretic. On =\textbackslash{}\{\textbackslash{}, s>12\textbackslash{},\textbackslash{} we form an arithmetic ratio J(s) whose poles encode zeros of , and pass to its Cayley transform (s)=(2 J(s)-1)/(2 J(s)+1). A Schur bound || 1 on a domain f...
\subsection*{43. A certified zero-free region for the Riemann zeta function\textbackslash{} the half-plane  s  0.6}
\textbf{File:} \texttt{../Jan-26/paper1\_farfield.tex}\\
\textbf{Detected topics:} Riemann\\
\textbf{Prereqs:} RG, Jcost\\
\textbf{Reasoning:} Depends on RG, Jcost; advances topics: Riemann.\\
\textbf{Summary:} We prove unconditionally that the Riemann zeta function (s) has no zeros in the fixed half-plane \textbackslash{}\{ s 0.6\textbackslash{}. The argument is function-theoretic: we encode the arithmetic ratio into a Cayley transform and show is Schur on \textbackslash{}\{ s>0.6\textbackslash{}, after which a Schur--Herglotz pinch eliminates zeros. The Schur property is certified by...
\subsection*{44. A certified zero-free region for the Riemann zeta function\textbackslash{} the half-plane  s  0.6}
\textbf{File:} \texttt{../Riemann/paper1\_farfield.tex}\\
\textbf{Detected topics:} Riemann\\
\textbf{Prereqs:} RG, Jcost\\
\textbf{Reasoning:} Depends on RG, Jcost; advances topics: Riemann.\\
\textbf{Summary:} We prove unconditionally that the Riemann zeta function (s) has no zeros in the fixed half-plane \textbackslash{}\{\textbackslash{}, s 0.6\textbackslash{},\textbackslash{}. The argument is function-theoretic. On =\textbackslash{}\{\textbackslash{}, s>12\textbackslash{},\textbackslash{} we form an arithmetic ratio J(s) whose poles encode zeros of , and pass to its Cayley transform (s)=(2 J(s)-1)/(2 J(s)+1). A Schur bound || 1 on a domain f...
\subsection*{45. A certified zero-free region for the Riemann zeta function\textbackslash{} the half-plane  s  0.6}
\textbf{File:} \texttt{../Riemann/paper1\_farfield\_v2.tex}\\
\textbf{Detected topics:} Riemann\\
\textbf{Prereqs:} RG, Jcost\\
\textbf{Reasoning:} Depends on RG, Jcost; advances topics: Riemann.\\
\textbf{Summary:} We prove unconditionally that the Riemann zeta function (s) has no zeros in the fixed half-plane \textbackslash{}\{\textbackslash{}, s 0.6\textbackslash{},\textbackslash{}. The argument is function-theoretic. On =\textbackslash{}\{\textbackslash{}, s>12\textbackslash{},\textbackslash{} we form an arithmetic ratio J(s) whose poles encode zeros of , and pass to its Cayley transform (s)=(2 J(s)-1)/(2 J(s)+1). A Schur bound || 1 on a domain f...
\subsection*{46. A certified zero-free region for the Riemann zeta function\textbackslash{} the half-plane  s  0.6}
\textbf{File:} \texttt{../Riemann/paper1\_farfield\_v4.tex}\\
\textbf{Detected topics:} Riemann\\
\textbf{Prereqs:} RG, Jcost\\
\textbf{Reasoning:} Depends on RG, Jcost; advances topics: Riemann.\\
\textbf{Summary:} We prove unconditionally that the Riemann zeta function (s) has no zeros in the fixed half-plane \textbackslash{}\{\textbackslash{}, s 0.6\textbackslash{},\textbackslash{}. The argument is function-theoretic. On =\textbackslash{}\{\textbackslash{}, s>12\textbackslash{},\textbackslash{} we form an arithmetic ratio J(s) whose poles encode zeros of , and pass to its Cayley transform (s)=(2 J(s)-1)/(2 J(s)+1). A Schur bound || 1 on a domain f...
\subsection*{47. A certified zero-free region for the Riemann zeta function\textbackslash{} the half-plane  s  0.6}
\textbf{File:} \texttt{../Riemann/paper1\_farfield\_v5.tex}\\
\textbf{Detected topics:} Riemann\\
\textbf{Prereqs:} RG, Jcost\\
\textbf{Reasoning:} Depends on RG, Jcost; advances topics: Riemann.\\
\textbf{Summary:} We prove unconditionally that the Riemann zeta function (s) has no zeros in the fixed half-plane \textbackslash{}\{\textbackslash{}, s 0.6\textbackslash{},\textbackslash{}. The argument is function-theoretic. On =\textbackslash{}\{\textbackslash{}, s>12\textbackslash{},\textbackslash{} we form an arithmetic ratio J(s) whose poles encode zeros of , and pass to its Cayley transform (s)=(2 J(s)-1)/(2 J(s)+1). A Schur bound || 1 on a domain f...
\subsection*{48. A certified zero-free region for the Riemann zeta function\textbackslash{} the half-plane  s  0.6}
\textbf{File:} \texttt{../Riemann/paper1\_handoff\_2026-01-22/papers/paper1\_farfield.tex}\\
\textbf{Detected topics:} Riemann\\
\textbf{Prereqs:} RG, Jcost\\
\textbf{Reasoning:} Depends on RG, Jcost; advances topics: Riemann.\\
\textbf{Summary:} We prove unconditionally that the Riemann zeta function (s) has no zeros in the fixed half-plane \textbackslash{}\{\textbackslash{}, s 0.6\textbackslash{},\textbackslash{}. The argument is function-theoretic. On =\textbackslash{}\{\textbackslash{}, s>12\textbackslash{},\textbackslash{} we form an arithmetic ratio J(s) whose poles encode zeros of , and pass to its Cayley transform (s)=(2 J(s)-1)/(2 J(s)+1). A Schur bound || 1 on a domain f...
\subsection*{49. A certified zero-free region for the Riemann zeta function\textbackslash{} the half-plane  s  0.6}
\textbf{File:} \texttt{../Riemann/paper1\_handoff\_v2\_2026-01-22/papers/paper1\_farfield.tex}\\
\textbf{Detected topics:} Riemann\\
\textbf{Prereqs:} RG, Jcost\\
\textbf{Reasoning:} Depends on RG, Jcost; advances topics: Riemann.\\
\textbf{Summary:} We prove unconditionally that the Riemann zeta function (s) has no zeros in the fixed half-plane \textbackslash{}\{\textbackslash{}, s 0.6\textbackslash{},\textbackslash{}. The argument is function-theoretic. On =\textbackslash{}\{\textbackslash{}, s>12\textbackslash{},\textbackslash{} we form an arithmetic ratio J(s) whose poles encode zeros of , and pass to its Cayley transform (s)=(2 J(s)-1)/(2 J(s)+1). A Schur bound || 1 on a domain f...
\subsection*{50. A certified zero-free region for the Riemann zeta function\textbackslash{} the half-plane  s  0.6}
\textbf{File:} \texttt{../Riemann/riemann\_handoff\_2026-01-29/papers/paper1\_farfield.tex}\\
\textbf{Detected topics:} Riemann\\
\textbf{Prereqs:} RG, Jcost\\
\textbf{Reasoning:} Depends on RG, Jcost; advances topics: Riemann.\\
\textbf{Summary:} We prove unconditionally that the Riemann zeta function (s) has no zeros in the fixed half-plane \textbackslash{}\{\textbackslash{}, s 0.6\textbackslash{},\textbackslash{}. The argument is function-theoretic. On =\textbackslash{}\{\textbackslash{}, s>12\textbackslash{},\textbackslash{} we form an arithmetic ratio J(s) whose poles encode zeros of , and pass to its Cayley transform (s)=(2 J(s)-1)/(2 J(s)+1). A Schur bound || 1 on a domain f...
\subsection*{51. A certified zero-free region for the Riemann zeta function\textbackslash{} the half-plane  s  0.6}
\textbf{File:} \texttt{../Riemann/riemann\_proof\_backup\_2026-01-08 2/papers/paper1\_farfield.tex}\\
\textbf{Detected topics:} Riemann\\
\textbf{Prereqs:} RG, Jcost\\
\textbf{Reasoning:} Depends on RG, Jcost; advances topics: Riemann.\\
\textbf{Summary:} We prove unconditionally that the Riemann zeta function (s) has no zeros in the fixed half-plane \textbackslash{}\{ s 0.6\textbackslash{}. The argument is function-theoretic: we encode the arithmetic ratio into a Cayley transform and show is Schur on \textbackslash{}\{ s>0.6\textbackslash{}, after which a Schur--Herglotz pinch eliminates zeros. The Schur property is certified by...
\subsection*{52. A certified zero-free region for the Riemann zeta function\textbackslash{} the half-plane  s  0.6}
\textbf{File:} \texttt{../Riemann/riemann\_proof\_backup\_2026-01-08/papers/paper1\_farfield.tex}\\
\textbf{Detected topics:} Riemann\\
\textbf{Prereqs:} RG, Jcost\\
\textbf{Reasoning:} Depends on RG, Jcost; advances topics: Riemann.\\
\textbf{Summary:} We prove unconditionally that the Riemann zeta function (s) has no zeros in the fixed half-plane \textbackslash{}\{ s 0.6\textbackslash{}. The argument is function-theoretic: we encode the arithmetic ratio into a Cayley transform and show is Schur on \textbackslash{}\{ s>0.6\textbackslash{}, after which a Schur--Herglotz pinch eliminates zeros. The Schur property is certified by...
\subsection*{53. A certified zero-free region for the Riemann zeta function\textbackslash{} the half-plane  s  0.6}
\textbf{File:} \texttt{../Riemann/run\_out/referee\_package\_2026-02-03/paper/paper1\_farfield.tex}\\
\textbf{Detected topics:} Riemann\\
\textbf{Prereqs:} RG, Jcost\\
\textbf{Reasoning:} Depends on RG, Jcost; advances topics: Riemann.\\
\textbf{Summary:} We prove unconditionally that the Riemann zeta function (s) has no zeros in the fixed half-plane \textbackslash{}\{\textbackslash{}, s 0.6\textbackslash{},\textbackslash{}. The argument is function-theoretic. On =\textbackslash{}\{\textbackslash{}, s>12\textbackslash{},\textbackslash{} we form an arithmetic ratio J(s) whose poles encode zeros of , and pass to its Cayley transform (s)=(2 J(s)-1)/(2 J(s)+1). A Schur bound || 1 on a domain f...
\subsection*{54. A certified zero-free region for the Riemann zeta function\textbackslash{} the half-plane  s  0.6}
\textbf{File:} \texttt{../arXiv Papers/Millennium Prize Problems/paper1\_farfield.tex}\\
\textbf{Detected topics:} Riemann\\
\textbf{Prereqs:} RG, Jcost\\
\textbf{Reasoning:} Depends on RG, Jcost; advances topics: Riemann.\\
\textbf{Summary:} We prove unconditionally that the Riemann zeta function (s) has no zeros in the fixed half-plane \textbackslash{}\{\textbackslash{}, s 0.6\textbackslash{},\textbackslash{}. The argument is function-theoretic. On =\textbackslash{}\{\textbackslash{}, s>12\textbackslash{},\textbackslash{} we form an arithmetic ratio J(s) whose poles encode zeros of , and pass to its Cayley transform (s)=(2 J(s)-1)/(2 J(s)+1). A Schur bound || 1 on a domain f...
\subsection*{55. A certified zero-free region for the Riemann zeta function\textbackslash{} the half-plane  s  0.6}
\textbf{File:} \texttt{../theory-report-from-amir-to-jon/paper1\_farfield.tex}\\
\textbf{Detected topics:} Riemann\\
\textbf{Prereqs:} RG, Jcost\\
\textbf{Reasoning:} Depends on RG, Jcost; advances topics: Riemann.\\
\textbf{Summary:} We prove unconditionally that the Riemann zeta function (s) has no zeros in the fixed half-plane \textbackslash{}\{\textbackslash{}, s 0.6\textbackslash{},\textbackslash{}. The argument is function-theoretic. On =\textbackslash{}\{\textbackslash{}, s>12\textbackslash{},\textbackslash{} we form an arithmetic ratio J(s) whose poles encode zeros of , and pass to its Cayley transform (s)=(2 J(s)-1)/(2 J(s)+1). A Schur bound || 1 on a domain f...
\subsection*{56. A certified zero-free region for the Riemann zeta function\textbackslash{} \textbackslash{}(0.6  s  0.999\textbackslash{}), \textbackslash{}(| s| 20\textbackslash{})}
\textbf{File:} \texttt{../Riemann/paper1\_farfield\_v3.tex}\\
\textbf{Detected topics:} Riemann\\
\textbf{Prereqs:} RG, Jcost\\
\textbf{Reasoning:} Depends on RG, Jcost; advances topics: Riemann.\\
\textbf{Summary:} We certify (via rigorous complex ball arithmetic) that the Riemann zeta function (s) has no zeros on the compact far-field region \textbackslash{}[ R\textbackslash{} :=\textbackslash{} [0.6,0.999][-20,20]. \textbackslash{}] The argument is function-theoretic. On =\textbackslash{}\{\textbackslash{}, s>12\textbackslash{},\textbackslash{} we form an arithmetic ratio J(s) whose poles encode zeros of , and pass to its Cayley transform (s)=(2...
\subsection*{57. A certified zero-free region for the Riemann zeta function\textbackslash{} \textbackslash{}(0.6  s  0.999\textbackslash{}), \textbackslash{}(| s| 20\textbackslash{})}
\textbf{File:} \texttt{../Riemann/referee\_step3\_response\_bundle\_v5/paper1\_farfield\_v5.tex}\\
\textbf{Detected topics:} Riemann\\
\textbf{Prereqs:} RG, Jcost\\
\textbf{Reasoning:} Depends on RG, Jcost; advances topics: Riemann.\\
\textbf{Summary:} We certify (via rigorous complex ball arithmetic) that the Riemann zeta function (s) has no zeros on the compact far-field region \textbackslash{}[ R\textbackslash{} :=\textbackslash{} [0.6,0.999][-20,20]. \textbackslash{}] The argument is function-theoretic. On =\textbackslash{}\{\textbackslash{}, s>12\textbackslash{},\textbackslash{} we form an arithmetic ratio J(s) whose poles encode zeros of , and pass to its Cayley transform (s)=(2...
\subsection*{58. A Complete Operator--Theoretic Proof of the Riemann Hypothesis}
\textbf{File:} \texttt{../arXiv Papers/New Papers/Papers/Riemann-Proof-Lean.tex}\\
\textbf{Detected topics:} Riemann\\
\textbf{Prereqs:} RG, Jcost\\
\textbf{Reasoning:} Depends on RG, Jcost; advances topics: Riemann.\\
\textbf{Summary:} The purpose of this manuscript is to present a self--contained, fully rigorous operator--theoretic proof of the Riemann Hypothesis (RH)
\subsection*{59. A cutoff principle and conditional closure\textbackslash{} the Riemann Hypothesis}
\textbf{File:} \texttt{../Jan-26/paper3\_cutoff\_conditional.tex}\\
\textbf{Detected topics:} Riemann\\
\textbf{Prereqs:} RG, Jcost\\
\textbf{Reasoning:} Depends on RG, Jcost; advances topics: Riemann.\\
\textbf{Summary:} This paper states a concrete cutoff principle (a bandlimit/Nyquist hypothesis for prime-frequency observables) and shows that, within the near-field barrier framework, it implies a uniform arithmetic blocker and a height-uniform Carleson bound. Combined with the unconditional far-field zero-freeness of Paper\textasciitilde{}I, this yi...
\subsection*{60. A cutoff principle and conditional closure\textbackslash{} the Riemann Hypothesis}
\textbf{File:} \texttt{../Riemann/riemann\_proof\_backup\_2026-01-08 2/papers/paper3\_cutoff\_conditional.tex}\\
\textbf{Detected topics:} Riemann\\
\textbf{Prereqs:} RG, Jcost\\
\textbf{Reasoning:} Depends on RG, Jcost; advances topics: Riemann.\\
\textbf{Summary:} This paper states a concrete cutoff principle (a bandlimit/Nyquist hypothesis for prime-frequency observables) and shows that, within the near-field barrier framework, it implies a uniform arithmetic blocker and a height-uniform Carleson bound. Combined with the unconditional far-field zero-freeness of Paper\textasciitilde{}I, this yi...
\subsection*{61. A cutoff principle and conditional closure\textbackslash{} the Riemann Hypothesis}
\textbf{File:} \texttt{../Riemann/riemann\_proof\_backup\_2026-01-08/papers/paper3\_cutoff\_conditional.tex}\\
\textbf{Detected topics:} Riemann\\
\textbf{Prereqs:} RG, Jcost\\
\textbf{Reasoning:} Depends on RG, Jcost; advances topics: Riemann.\\
\textbf{Summary:} This paper states a concrete cutoff principle (a bandlimit/Nyquist hypothesis for prime-frequency observables) and shows that, within the near-field barrier framework, it implies a uniform arithmetic blocker and a height-uniform Carleson bound. Combined with the unconditional far-field zero-freeness of Paper\textasciitilde{}I, this yi...
\subsection*{62. A divergent Vasyunin correction}
\textbf{File:} \texttt{../AI/repos/riemann/Notes/Papers/Romanov/VasyuninDiverges.tex}\\
\textbf{Detected topics:} Riemann\\
\textbf{Prereqs:} RG, Jcost\\
\textbf{Reasoning:} Depends on RG, Jcost; advances topics: Riemann.\\
\textbf{Summary:} V. I. Vasyunin has introduced special sequences of step functions related to the strong Nyman-Beurling criterion that converge pointwise to 1 in [1,). We show here that the first and simplest such sequence considered by Vasyunin diverges in L\_1((1,),x\textasciicircum{}\{-2dx), which of course precludes the L\_2((1,),x\textasciicircum{}\{-2dx)-convergence...
\subsection*{63. A divergent Vasyunin correction}
\textbf{File:} \texttt{../Riemann/riemann-main/Notes/Papers/Romanov/VasyuninDiverges.tex}\\
\textbf{Detected topics:} Riemann\\
\textbf{Prereqs:} RG, Jcost\\
\textbf{Reasoning:} Depends on RG, Jcost; advances topics: Riemann.\\
\textbf{Summary:} V. I. Vasyunin has introduced special sequences of step functions related to the strong Nyman-Beurling criterion that converge pointwise to 1 in [1,). We show here that the first and simplest such sequence considered by Vasyunin diverges in L\_1((1,),x\textasciicircum{}\{-2dx), which of course precludes the L\_2((1,),x\textasciicircum{}\{-2dx)-convergence...
\subsection*{64. A divergent Vasyunin correction}
\textbf{File:} \texttt{../archive/goldbach/riemann\_lean\_repo/Notes/Papers/Romanov/VasyuninDiverges.tex}\\
\textbf{Detected topics:} Riemann\\
\textbf{Prereqs:} RG, Jcost\\
\textbf{Reasoning:} Depends on RG, Jcost; advances topics: Riemann.\\
\textbf{Summary:} V. I. Vasyunin has introduced special sequences of step functions related to the strong Nyman-Beurling criterion that converge pointwise to 1 in [1,). We show here that the first and simplest such sequence considered by Vasyunin diverges in L\_1((1,),x\textasciicircum{}\{-2dx), which of course precludes the L\_2((1,),x\textasciicircum{}\{-2dx)-convergence...
\subsection*{65. A general strong Nyman-Beurling criterion for the Riemann Hypothesis}
\textbf{File:} \texttt{../AI/repos/riemann/Notes/Papers/Romanov/GeneralStrongNyman.tex}\\
\textbf{Detected topics:} Riemann\\
\textbf{Prereqs:} RG, Jcost\\
\textbf{Reasoning:} Depends on RG, Jcost; advances topics: Riemann.\\
\textbf{Summary:} For each f:[0,) formally consider its co-Poisson or M\textbackslash{}"\{untz transform g(x)=\_\{n 1f(nx)-1\{x\_0\textasciicircum{} f(t)dt. For certain f's with both f, g L\_2(0,) it is true that the Riemann hypothesis holds if and only if f is in the L\_2 closure of the vector space generated by the dilations g(kx), k. Such is the case for example when f=\_\{...
\subsection*{66. A general strong Nyman-Beurling criterion for the Riemann Hypothesis}
\textbf{File:} \texttt{../Riemann/riemann-main/Notes/Papers/Romanov/GeneralStrongNyman.tex}\\
\textbf{Detected topics:} Riemann\\
\textbf{Prereqs:} RG, Jcost\\
\textbf{Reasoning:} Depends on RG, Jcost; advances topics: Riemann.\\
\textbf{Summary:} For each f:[0,) formally consider its co-Poisson or M\textbackslash{}"\{untz transform g(x)=\_\{n 1f(nx)-1\{x\_0\textasciicircum{} f(t)dt. For certain f's with both f, g L\_2(0,) it is true that the Riemann hypothesis holds if and only if f is in the L\_2 closure of the vector space generated by the dilations g(kx), k. Such is the case for example when f=\_\{...
\subsection*{67. A general strong Nyman-Beurling criterion for the Riemann Hypothesis}
\textbf{File:} \texttt{../archive/goldbach/riemann\_lean\_repo/Notes/Papers/Romanov/GeneralStrongNyman.tex}\\
\textbf{Detected topics:} Riemann\\
\textbf{Prereqs:} RG, Jcost\\
\textbf{Reasoning:} Depends on RG, Jcost; advances topics: Riemann.\\
\textbf{Summary:} For each f:[0,) formally consider its co-Poisson or M\textbackslash{}"\{untz transform g(x)=\_\{n 1f(nx)-1\{x\_0\textasciicircum{} f(t)dt. For certain f's with both f, g L\_2(0,) it is true that the Riemann hypothesis holds if and only if f is in the L\_2 closure of the vector space generated by the dilations g(kx), k. Such is the case for example when f=\_\{...
\subsection*{68. Energy barriers and Carleson budgets\textbackslash{} off-critical zeros of the Riemann zeta function}
\textbf{File:} \texttt{../Jan-26/paper2\_energy\_barrier.tex}\\
\textbf{Detected topics:} Riemann\\
\textbf{Prereqs:} RG, Jcost\\
\textbf{Reasoning:} Depends on RG, Jcost; advances topics: Riemann.\\
\textbf{Summary:} This paper develops effective near-field barriers against off-critical zeros of (s) by comparing a quantized windowed phase cost to a scale-tracked Carleson energy budget. It yields an explicit protection height T\_\{ safe() such that any zero =+i with 12<<0.6 and =-12 must satisfy ||>T\_\{ safe().
\subsection*{69. Energy barriers and Carleson budgets\textbackslash{} off-critical zeros of the Riemann zeta function}
\textbf{File:} \texttt{../Riemann/paper2\_energy\_barrier.tex}\\
\textbf{Detected topics:} Riemann\\
\textbf{Prereqs:} RG, Jcost\\
\textbf{Reasoning:} Depends on RG, Jcost; advances topics: Riemann.\\
\textbf{Summary:} This paper develops an effective near-field barrier against off-critical zeros of the Riemann zeta function in the strip 12< s<0.6. Writing a hypothetical zero as =12++i with (0,0.1), we compare a universal windowed phase-cost lower bound (a half-plane Blaschke trigger) to a scale-tracked Carleson-box energy budget for...
\subsection*{70. Energy barriers and Carleson budgets\textbackslash{} off-critical zeros of the Riemann zeta function}
\textbf{File:} \texttt{../Riemann/riemann\_handoff\_2026-01-29/papers/paper2\_energy\_barrier.tex}\\
\textbf{Detected topics:} Riemann\\
\textbf{Prereqs:} RG, Jcost\\
\textbf{Reasoning:} Depends on RG, Jcost; advances topics: Riemann.\\
\textbf{Summary:} This paper develops an effective near-field barrier against off-critical zeros of the Riemann zeta function in the strip 12< s<0.6. Writing a hypothetical zero as =12++i with (0,0.1), we compare a universal windowed phase-cost lower bound (a half-plane Blaschke trigger) to a scale-tracked Carleson-box energy budget for...
\subsection*{71. Energy barriers and Carleson budgets\textbackslash{} off-critical zeros of the Riemann zeta function}
\textbf{File:} \texttt{../Riemann/riemann\_proof\_backup\_2026-01-08 2/papers/paper2\_energy\_barrier.tex}\\
\textbf{Detected topics:} Riemann\\
\textbf{Prereqs:} RG, Jcost\\
\textbf{Reasoning:} Depends on RG, Jcost; advances topics: Riemann.\\
\textbf{Summary:} This paper develops effective near-field barriers against off-critical zeros of (s) by comparing a quantized windowed phase cost to a scale-tracked Carleson energy budget. It yields an explicit protection height T\_\{ safe() such that any zero =+i with 12<<0.6 and =-12 must satisfy ||>T\_\{ safe().
\subsection*{72. Energy barriers and Carleson budgets\textbackslash{} off-critical zeros of the Riemann zeta function}
\textbf{File:} \texttt{../Riemann/riemann\_proof\_backup\_2026-01-08/papers/paper2\_energy\_barrier.tex}\\
\textbf{Detected topics:} Riemann\\
\textbf{Prereqs:} RG, Jcost\\
\textbf{Reasoning:} Depends on RG, Jcost; advances topics: Riemann.\\
\textbf{Summary:} This paper develops effective near-field barriers against off-critical zeros of (s) by comparing a quantized windowed phase cost to a scale-tracked Carleson energy budget. It yields an explicit protection height T\_\{ safe() such that any zero =+i with 12<<0.6 and =-12 must satisfy ||>T\_\{ safe().
\subsection*{73. Energy barriers and Carleson budgets\textbackslash{} off-critical zeros of the Riemann zeta function}
\textbf{File:} \texttt{../arXiv Papers/Millennium Prize Problems/paper2\_energy\_barrier.tex}\\
\textbf{Detected topics:} Riemann\\
\textbf{Prereqs:} RG, Jcost\\
\textbf{Reasoning:} Depends on RG, Jcost; advances topics: Riemann.\\
\textbf{Summary:} This paper develops an effective near-field barrier against off-critical zeros of the Riemann zeta function in the strip 12< s<0.6. Writing a hypothetical zero as =12++i with (0,0.1), we compare a universal windowed phase-cost lower bound (a half-plane Blaschke trigger) to a scale-tracked Carleson-box energy budget for...
\subsection*{74. Is the Riemann zeta function in a short interval \textbackslash{}\textbackslash{} a 1-RSB spin glass ?}
\textbf{File:} \texttt{../AI/repos/riemann/Notes/Papers/Arguin/arXiv-1706.08462v4/RZF-1RSB\_oct23\_2018.tex}\\
\textbf{Detected topics:} Riemann\\
\textbf{Prereqs:} RG, Jcost\\
\textbf{Reasoning:} Depends on RG, Jcost; advances topics: Riemann.\\
\textbf{Summary:} Fyodorov, Hiary \textbackslash{}\& Keating established an intriguing connection between the maxima of log-correlated processes and the ones of the Riemann zeta function on a short interval of the critical line. In particular, they suggest that the analogue of the free energy of the Riemann zeta function is identical to the one of the...
\subsection*{75. Main conjectures for non-CM elliptic curves at good ordinary primes}
\textbf{File:} \texttt{../Riemann/riemann-geometry-rs/notes/papers/bsd/arxiv-src/2412.20078/ref.tex}\\
\textbf{Detected topics:} Riemann\\
\textbf{Prereqs:} RG, Jcost\\
\textbf{Reasoning:} Depends on RG, Jcost; advances topics: Riemann.\\
\textbf{Summary:} Let E/ be an elliptic curve and p > 2 be a prime of good ordinary reduction for E. Assume that the residue representation associated with (E, p) is irreducible. In this paper, we prove more cases on several Iwasawa main conjectures for E. As applications, we prove more general cases of p-converse theorem and p-part BSD...
\subsection*{76. Meta-Principle-arXiv.tex}
\textbf{File:} \texttt{papers/tex/Meta-Principle-arXiv.tex}\\
\textbf{Detected topics:} Existence\\
\textbf{Prereqs:} Jcost\\
\textbf{Reasoning:} Depends on Jcost; advances topics: Existence.\\
\textbf{Summary:} The twentieth century accelerated this trend with the development of General Relativity and the Standard Model of particle physics. Yet, this success has revealed a profound challenge smolin2006trouble.
\subsection*{77. Meta-Principle.tex}
\textbf{File:} \texttt{../arXiv Papers/Meta-Principle.tex}\\
\textbf{Detected topics:} Existence\\
\textbf{Prereqs:} Jcost\\
\textbf{Reasoning:} Depends on Jcost; advances topics: Existence.\\
\textbf{Summary:} The twentieth century accelerated this trend with the development of General Relativity and the Standard Model of particle physics. Yet, this success has revealed a profound challenge smolin2006trouble.
\subsection*{78. Meta-Principle.tex}
\textbf{File:} \texttt{../rs-website/Meta-Principle.tex}\\
\textbf{Detected topics:} Existence\\
\textbf{Prereqs:} Jcost\\
\textbf{Reasoning:} Depends on Jcost; advances topics: Existence.\\
\textbf{Summary:} The twentieth century accelerated this trend with the development of General Relativity and the Standard Model of particle physics. Yet, this success has revealed a profound challenge smolin2006trouble.
\subsection*{79. Note to understand the Meta-Principle, Eight Theorems, and Parameter-Free}
\textbf{File:} \texttt{papers/tex/Note\_anil.tex}\\
\textbf{Detected topics:} Existence\\
\textbf{Prereqs:} Jcost\\
\textbf{Reasoning:} Depends on Jcost; advances topics: Existence.\\
\textbf{Summary:} This is my note and attempt to understand the Recognition Science foundational theory. Most of the things I write below is with the help of AI and asking questions (hopefully right ones).
\subsection*{80. On congruences between Kato's Euler systems}
\textbf{File:} \texttt{../Riemann/riemann-geometry-rs/tmp/arxiv/1909.01764/kim-lee-ponsinet-19.tex}\\
\textbf{Detected topics:} Riemann\\
\textbf{Prereqs:} RG, Jcost\\
\textbf{Reasoning:} Depends on RG, Jcost; advances topics: Riemann.\\
\textbf{Summary:} We study the behavior under congruences of the Iwasawa invariants of the Iwasawa modules which appear in Kato's main conjecture without p-adic L-functions. It generalizes the work of Greenberg--Vatsal, Emerton--Pollack--Weston, B.D. Kim, Greenberg--Iovita--Pollack, and one of us simultaneously. As a consequence, we est...
\subsection*{81. On zeroes and poles of Helson zeta functions}
\textbf{File:} \texttt{../AI/repos/riemann/Notes/Papers/Romanov/Helson\_zeta\_function.tex}\\
\textbf{Detected topics:} Riemann\\
\textbf{Prereqs:} RG, Jcost\\
\textbf{Reasoning:} Depends on RG, Jcost; advances topics: Riemann.\\
\textbf{Summary:} We show that the analytic continuations of Helson zeta functions \_ (s)= \_1\textasciicircum{}\{(n)n\textasciicircum{}\{-s can have essentially arbitrary poles and zeroes in the strip 21/40 < s < 1 (unconditionally), and in the whole critical strip 1/2 < s <1 under Riemann Hypothesis.
\subsection*{82. On zeroes and poles of Helson zeta functions}
\textbf{File:} \texttt{../Riemann/riemann-main/Notes/Papers/Romanov/Helson\_zeta\_function.tex}\\
\textbf{Detected topics:} Riemann\\
\textbf{Prereqs:} RG, Jcost\\
\textbf{Reasoning:} Depends on RG, Jcost; advances topics: Riemann.\\
\textbf{Summary:} We show that the analytic continuations of Helson zeta functions \_ (s)= \_1\textasciicircum{}\{(n)n\textasciicircum{}\{-s can have essentially arbitrary poles and zeroes in the strip 21/40 < s < 1 (unconditionally), and in the whole critical strip 1/2 < s <1 under Riemann Hypothesis.
\subsection*{83. On zeroes and poles of Helson zeta functions}
\textbf{File:} \texttt{../archive/goldbach/riemann\_lean\_repo/Notes/Papers/Romanov/Helson\_zeta\_function.tex}\\
\textbf{Detected topics:} Riemann\\
\textbf{Prereqs:} RG, Jcost\\
\textbf{Reasoning:} Depends on RG, Jcost; advances topics: Riemann.\\
\textbf{Summary:} We show that the analytic continuations of Helson zeta functions \_ (s)= \_1\textasciicircum{}\{(n)n\textasciicircum{}\{-s can have essentially arbitrary poles and zeroes in the strip 21/40 < s < 1 (unconditionally), and in the whole critical strip 1/2 < s <1 under Riemann Hypothesis.
\subsection*{84. Prime-Graph Expansion and Guerra Interpolation \textbackslash{}\textbackslash{} for an Arithmetic Spin Glass}
\textbf{File:} \texttt{../AI/repos/riemann/Notes/Papers/CW/Prime-Graph.tex}\\
\textbf{Detected topics:} Riemann\\
\textbf{Prereqs:} RG, Jcost\\
\textbf{Reasoning:} Depends on RG, Jcost; advances topics: Riemann.\\
\textbf{Summary:} We propose and partially analyze an ``arithmetic spin glass'' whose Hamiltonian is a truncated log--correlated field built from prime divisibility on a block (N,2N]. The core input is the strong local expansion of the prime divisibility graph recently proved by Helfgott and Radziwi\textbackslash{} HR-expansion,HR-explanation, which w...
\subsection*{85. Recognition Science, Prime Numbers, and the Riemann Hypothesis:\textbackslash{}\textbackslash{} A Standalone Roadmap of What We Know, What We Built, and What Still Blocks Us}
\textbf{File:} \texttt{papers/tex/RecognitionScience\_Primes\_RH\_Blockers.tex}\\
\textbf{Detected topics:} Riemann\\
\textbf{Prereqs:} RG, Jcost\\
\textbf{Reasoning:} Depends on RG, Jcost; advances topics: Riemann.\\
\textbf{Summary:} This note is a standalone ``state-of-the-art'' writeup for a specific research codebase (riemann-geometry-rs) and a specific guiding narrative (``Recognition Science''). We assume, as a working hypothesis, that Recognition Science (RS) is the correct architecture of reality, and we explain what that hypothesis suggests...
\subsection*{86. Riemann Hypothesis Proof Track\textbackslash{}\textbackslash{}  Axiom-Free Formalization in Lean 4}
\textbf{File:} \texttt{../Riemann/Riemann-final/Riemann-Not-Used/gg/RH\_PROOF\_TRACK.tex}\\
\textbf{Detected topics:} Riemann\\
\textbf{Prereqs:} RG, Jcost\\
\textbf{Reasoning:} Depends on RG, Jcost; advances topics: Riemann.\\
\textbf{Summary:} This document presents the complete proof track for the Riemann Hypothesis formalization in Lean 4. The proof is axiom-free, building entirely on mathlib foundations. We document the structure of 71 Lean files organized into modules, showing how they combine to prove RiemannHypothesis from mathlib's number theory libra...
\subsection*{87. Riemann Hypothesis Proof Track\textbackslash{}\textbackslash{}  Axiom-Free Formalization in Lean 4}
\textbf{File:} \texttt{../Riemann/Riemann-final/Riemann-Not-Used/gg2/RH\_PROOF\_TRACK.tex}\\
\textbf{Detected topics:} Riemann\\
\textbf{Prereqs:} RG, Jcost\\
\textbf{Reasoning:} Depends on RG, Jcost; advances topics: Riemann.\\
\textbf{Summary:} This document presents the complete proof track for the Riemann Hypothesis formalization in Lean 4. The proof is axiom-free, building entirely on mathlib foundations. We document the structure of 71 Lean files organized into modules, showing how they combine to prove RiemannHypothesis from mathlib's number theory libra...
\subsection*{88. Riemann Hypothesis Proof Track\textbackslash{}\textbackslash{}  Axiom-Free Formalization in Lean 4}
\textbf{File:} \texttt{../Riemann/Riemann-final/Riemann-Not-Used/nbd/zeros/RH\_PROOF\_TRACK.tex}\\
\textbf{Detected topics:} Riemann\\
\textbf{Prereqs:} RG, Jcost\\
\textbf{Reasoning:} Depends on RG, Jcost; advances topics: Riemann.\\
\textbf{Summary:} This document presents the complete proof track for the Riemann Hypothesis formalization in Lean 4. The proof is axiom-free, building entirely on mathlib foundations. We document the structure of 71 Lean files organized into modules, showing how they combine to prove RiemannHypothesis from mathlib's number theory libra...
\subsection*{89. Riemann Hypothesis Proof Track\textbackslash{}\textbackslash{}  Axiom-Free Formalization in Lean 4}
\textbf{File:} \texttt{../Riemann/Riemann-final/Riemann-Not-Used/riemann-67/archive/root\_files/RH\_PROOF\_TRACK.tex}\\
\textbf{Detected topics:} Riemann\\
\textbf{Prereqs:} RG, Jcost\\
\textbf{Reasoning:} Depends on RG, Jcost; advances topics: Riemann.\\
\textbf{Summary:} This document presents the complete proof track for the Riemann Hypothesis formalization in Lean 4. The proof is axiom-free, building entirely on mathlib foundations. We document the structure of 71 Lean files organized into modules, showing how they combine to prove RiemannHypothesis from mathlib's number theory libra...
\subsection*{90. Riemann Hypothesis Proof Track\textbackslash{}\textbackslash{}  Axiom-Free Formalization in Lean 4}
\textbf{File:} \texttt{../Riemann/Riemann-final/Riemann-Not-Used/rocket-ship/vendor/zeros/RH\_PROOF\_TRACK.tex}\\
\textbf{Detected topics:} Riemann\\
\textbf{Prereqs:} RG, Jcost\\
\textbf{Reasoning:} Depends on RG, Jcost; advances topics: Riemann.\\
\textbf{Summary:} This document presents the complete proof track for the Riemann Hypothesis formalization in Lean 4. The proof is axiom-free, building entirely on mathlib foundations. We document the structure of 71 Lean files organized into modules, showing how they combine to prove RiemannHypothesis from mathlib's number theory libra...
\subsection*{91. Riemann Hypothesis Proof Track\textbackslash{}\textbackslash{}  Axiom-Free Formalization in Lean 4}
\textbf{File:} \texttt{../Riemann/Riemann-final/Riemann-Not-Used/zeros-done/zeros/RH\_PROOF\_TRACK.tex}\\
\textbf{Detected topics:} Riemann\\
\textbf{Prereqs:} RG, Jcost\\
\textbf{Reasoning:} Depends on RG, Jcost; advances topics: Riemann.\\
\textbf{Summary:} This document presents the complete proof track for the Riemann Hypothesis formalization in Lean 4. The proof is axiom-free, building entirely on mathlib foundations. We document the structure of 71 Lean files organized into modules, showing how they combine to prove RiemannHypothesis from mathlib's number theory libra...
\subsection*{92. Riemann Hypothesis Proof Track\textbackslash{}\textbackslash{}  Axiom-Free Formalization in Lean 4}
\textbf{File:} \texttt{../Riemann/Riemann-final/Riemann-Not-Used/zeros-ref/RH\_PROOF\_TRACK.tex}\\
\textbf{Detected topics:} Riemann\\
\textbf{Prereqs:} RG, Jcost\\
\textbf{Reasoning:} Depends on RG, Jcost; advances topics: Riemann.\\
\textbf{Summary:} This document presents the complete proof track for the Riemann Hypothesis formalization in Lean 4. The proof is axiom-free, building entirely on mathlib foundations. We document the structure of 71 Lean files organized into modules, showing how they combine to prove RiemannHypothesis from mathlib's number theory libra...
\subsection*{93. Riemann Hypothesis Proof Track\textbackslash{}\textbackslash{}  Axiom-Free Formalization in Lean 4}
\textbf{File:} \texttt{../Riemann/Riemann-final/Riemann-Not-Used/zeros/RH\_PROOF\_TRACK.tex}\\
\textbf{Detected topics:} Riemann\\
\textbf{Prereqs:} RG, Jcost\\
\textbf{Reasoning:} Depends on RG, Jcost; advances topics: Riemann.\\
\textbf{Summary:} This document presents the complete proof track for the Riemann Hypothesis formalization in Lean 4. The proof is axiom-free, building entirely on mathlib foundations. We document the structure of 71 Lean files organized into modules, showing how they combine to prove RiemannHypothesis from mathlib's number theory libra...
\subsection*{94. Riemann Hypothesis Proof Track\textbackslash{}\textbackslash{}  Axiom-Free Formalization in Lean 4}
\textbf{File:} \texttt{../Riemann/Riemann-final/Riemann-Not-Used/zeta/archive/RH\_PROOF\_TRACK.tex}\\
\textbf{Detected topics:} Riemann\\
\textbf{Prereqs:} RG, Jcost\\
\textbf{Reasoning:} Depends on RG, Jcost; advances topics: Riemann.\\
\textbf{Summary:} This document presents the complete proof track for the Riemann Hypothesis formalization in Lean 4. The proof is axiom-free, building entirely on mathlib foundations. We document the structure of 71 Lean files organized into modules, showing how they combine to prove RiemannHypothesis from mathlib's number theory libra...
\subsection*{95. Riemann Hypothesis Proof Track\textbackslash{}\textbackslash{}  Axiom-Free Formalization in Lean 4}
\textbf{File:} \texttt{../Riemann/Riemann-final/Riemann-Not-Used/zeta/vendor/zeros/RH\_PROOF\_TRACK.tex}\\
\textbf{Detected topics:} Riemann\\
\textbf{Prereqs:} RG, Jcost\\
\textbf{Reasoning:} Depends on RG, Jcost; advances topics: Riemann.\\
\textbf{Summary:} This document presents the complete proof track for the Riemann Hypothesis formalization in Lean 4. The proof is axiom-free, building entirely on mathlib foundations. We document the structure of 71 Lean files organized into modules, showing how they combine to prove RiemannHypothesis from mathlib's number theory libra...
\subsection*{96. Selberg's central limit theorem for  |( 12+it)|}
\textbf{File:} \texttt{../AI/repos/riemann/Notes/Papers/Radziwill\&/CLT.tex}\\
\textbf{Detected topics:} Riemann\\
\textbf{Prereqs:} RG, Jcost\\
\textbf{Reasoning:} Depends on RG, Jcost; advances topics: Riemann.\\
\textbf{Summary:} There is considerable latitude in choosing parameters such as W, but to fix ideas we select equation 1.0 W= ( T)\textasciicircum{}4, \textbackslash{} \textbackslash{} X = T\textasciicircum{}\{1/( T)\textasciicircum{}2, \textbackslash{}\textbackslash{} and Y = T\textasciicircum{}\{1/( T)\textasciicircum{}2. equation Here X and Y are two parameters that will appear short

\section*{Proposed Additional Papers to Strengthen the Stack}
\begin{enumerate}
\item \textbf{Bridge Theorem (Consistency + Interaction \\Rightarrow d'Alembert)} -- removes the explicit bridge hypothesis by proving the log-lift must satisfy d'Alembert under minimal regularity.
\item \textbf{Eight-Tick Forcing} -- a standalone derivation of the 8-tick cycle from the discrete ledger / recognition constraints.
\item \textbf{Ledger Dynamics Specification} -- formal definition of admissible states, constraints, and the optimization problem defining R\^{}\^{ }.
\item \textbf{Calibration \& Unit Fixing} -- how $G''(0)=1$ and unit choices map the dimensionless RS quantities to SI units.
\item \textbf{Measurement Protocol \& Data Certificates} -- formal pipeline for turning experiments into checkable certificates.
\item \textbf{Uncertainty / Tolerance Propagation} -- rigorous handling of measurement error and prediction intervals.
\item \textbf{Axiom Independence \& Necessity} -- small countermodels showing each axiom/gate is required.
\item \textbf{Continuum-Limit Emergence} -- explicit derivation of Hamiltonian / Schrödinger / GR approximations.
\item \textbf{Empirical Falsification Suite} -- unified list of RS-specific tests with thresholds and failure criteria.
\item \textbf{Reproducibility \& Formalization Audit} -- clear map from papers <-> Lean modules <-> build artifacts.
\end{enumerate}

\end{document}