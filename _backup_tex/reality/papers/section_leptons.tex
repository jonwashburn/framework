\section{Lepton Mass Chain (T9/T10)}
\noindent\fbox{\parbox{0.97\linewidth}{%
\textbf{Section summary.}
The anchor law of Sec.~5 organizes the charged spectrum at \(\muStar\).
This section presents an additional, lepton-specific pipeline that yields absolute predictions for
\(m_e\), \(m_\mu\), and \(m_\tau\) as a sequence of derived ladder exponents.
The pipeline has two parts: (i) an electron ``break'' exponent (a large shift) fixed from the same counting layer and coupling constant \(\alpha\),
and (ii) generation-step exponents from electron\(\to\)muon and muon\(\to\)tau.
All numerical comparisons are labeled as validation against PDG.}}

\subsection{Electron baseline at the anchor}
For leptons the family band label is \(Z_\ell=1332\) (Sec.~5). \PROVED
Write the lepton skeleton mass at the anchor as
\begin{equation}
  m_{\mathrm{skel}}(e;\muStar) := A_{\mathrm{Lepton}}\,\phig^{\,r_e-8}. \PROVED
  \label{eq:electron_skeleton}
\end{equation}
Then the anchor display law specializes to
\begin{equation}
  \mRS(e;\muStar) = m_{\mathrm{skel}}(e;\muStar)\,\phig^{\mathrm{gap}(1332)}. \PROVED
  \label{eq:electron_anchor_display}
\end{equation}
This anchor display is an organizational coordinate statement; by itself it is not yet the low-energy electron mass. \HYP

\subsection{The electron break (refined shift)}
To obtain an absolute electron mass prediction, we introduce a lepton-specific exponent shift \(\delta_e\) (the ``break''). \HYP
It is fixed by the same integer layer \((W,E_{\mathrm{total}},E_{\mathrm{passive}})\) together with the fine-structure constant \(\alpha\): \HYP
\begin{equation}
  \delta_e
  \;:=\;
  2W \;+\; \frac{W + E_{\mathrm{total}}}{4E_{\mathrm{passive}}}
  \;+\; \alpha^2 \;+\; E_{\mathrm{total}}\alpha^3.
  \HYP
  \label{eq:delta_e}
\end{equation}
The interpretation is that the first two terms capture a purely topological ledger contribution, while the latter two terms are small radiative corrections
organized by \(\alpha\). \HYP

With \(\delta_e\) fixed, the electron mass prediction is
\begin{equation}
  m_e^{\mathrm{pred}}
  \;:=\;
  m_{\mathrm{skel}}(e;\muStar)\,\phig^{\mathrm{gap}(1332)-\delta_e}.
  \HYP
  \label{eq:me_pred}
\end{equation}

\subsection{Generation steps: \texorpdfstring{electron\(\to\)muon\(\to\)tau}{electron to muon to tau}}
The muon and tau are obtained by adding two step exponents to the electron residue. \HYP
Define the electron\(\to\)muon step as
\begin{equation}
  S_{e\to\mu}
  \;:=\;
  E_{\mathrm{passive}} \;+\; \frac{1}{4\pi} \;-\; \alpha^2.
  \HYP
  \label{eq:step_emu}
\end{equation}
The leading term \(E_{\mathrm{passive}}=11\) is an integer rung jump; the remaining terms provide small geometry/coupling corrections. \HYP

Define the muon\(\to\)tau step as
\begin{equation}
  S_{\mu\to\tau}
  \;:=\;
  6 \;-\; \frac{2W+3}{2}\,\alpha.
  \HYP
  \label{eq:step_mutau}
\end{equation}
The leading term \(6\) is again an integer jump (the cube face count), with a small \(\alpha\)-dependent correction. \HYP

Using these steps, the muon and tau predictions are
\begin{align}
  m_\mu^{\mathrm{pred}}
  &:= m_{\mathrm{skel}}(e;\muStar)\,\phig^{\mathrm{gap}(1332)-\delta_e + S_{e\to\mu}},
  \HYP
  \label{eq:mmu_pred}\\
  m_\tau^{\mathrm{pred}}
  &:= m_{\mathrm{skel}}(e;\muStar)\,\phig^{\mathrm{gap}(1332)-\delta_e + S_{e\to\mu} + S_{\mu\to\tau}}.
  \HYP
  \label{eq:mtau_pred}
\end{align}

\subsection{Validation table (PDG comparison)}
We report the numerical predictions in MeV under the declared unit convention (Sec.~4) and compare to PDG values \cite{PDG2024}. \VAL
The table below is generated automatically from the repository scripts (no manual editing). \PROVED

% Auto-generated by tools/lepton_chain_table.py
\begin{table}[h]
  \centering
  \caption{Lepton chain prediction (T9--T10) from first-principles constants. Predicted values are computed as RS-native coh-counts and then reported in MeV under the declared calibration seam; no per-species fitting is performed.}
  \label{tab:lepton_chain_pred_vs_pdg}
  \begin{tabular}{lrrrr}
    \toprule
    Species & Pred. (MeV) & PDG (MeV) & Abs. err & Rel. err \\
    \midrule
    e & 0.510999 & 0.510999 & -1.9546e-07 & -3.82506e-07 \\
    mu & 105.658 & 105.658 & -0.000112323 & -1.06307e-06 \\
    tau & 1776.71 & 1776.86 & -0.154158 & -8.67587e-05 \\
    \bottomrule
  \end{tabular}
\end{table}


\paragraph{Classical correspondence.}
The lepton mass chain has no direct classical analog in the Standard Model, where the
electron, muon, and tau masses are independent Yukawa inputs.
The closest conceptual relatives are:
(i)~topological linking arguments (Jordan curve theorem, Alexander polynomials) that assign
integer invariants to knotted configurations, analogous to how the generation steps
$S_{e\to\mu}$ and $S_{\mu\to\tau}$ are fixed by integer counts $(E_{\mathrm{passive}}, F)$; and
(ii)~radiative correction hierarchies in QED, where $\alpha$-dependent terms appear as
perturbative shifts to leading-order results.
The key difference is that the lepton chain fixes the $\alpha$-corrections from the same
integer layer rather than fitting them to data. \HYP

\section{Transport and PDG Comparison}
\noindent\fbox{\parbox{0.97\linewidth}{%
\textbf{Section summary.}
Any comparison to external mass conventions is scheme- and scale-dependent.
We therefore separate two distinct roles:
the structural band coordinate \(\mathrm{gap}(Z)\) (large, family-defining) and the Standard-Model RG transport exponent \(f^{RG}\)
(small, bookkeeping-only).
We explicitly do \emph{not} identify \(f^{RG}\) with \(\mathrm{gap}(Z)\).}}

\subsection{What a ``PDG mass'' means (why transport is unavoidable)}
The phrase ``the mass of a particle'' is not a single number in quantum field theory.
Depending on the particle and convention, quoted values may refer to:
\begin{itemize}
  \item \textbf{Pole masses} (commonly used for charged leptons), or
  \item \textbf{running masses} (commonly used for quarks in \(\overline{\mathrm{MS}}\)) evaluated at a stated scale.
\end{itemize}
Therefore, any numerical objection or comparison must state the target \((\text{scheme},\mu)\). \PROVED

\subsection{Two different exponents (do not conflate)}
The structural band coordinate is
\[
  f^{\mathrm{Rec}}(Z) := \mathrm{gap}(Z). \PROVED
\]
It is a closed-form, family-defining exponent shift (order \(\sim 6\)--\(14\) for the charged families). \PROVED

By contrast, the RG transport exponent \(f^{RG}\) is a scheme/scale bookkeeping quantity defined from the Standard Model running mass \(m_i(\mu)\) by
\begin{equation}
  \fRG_i(\mu_1,\mu_2)
  \;:=\;
  \log_{\phig}\!\left(\frac{m_i(\mu_2)}{m_i(\mu_1)}\right)
  \;=\;
  \frac{1}{\ln\phig}\ln\!\left(\frac{m_i(\mu_2)}{m_i(\mu_1)}\right).
  \CERT
  \label{eq:frg_def}
\end{equation}
In typical SM running between \(\muStar\) and low-energy reference points, \(f^{RG}\) is small (order \(10^{-2}\) to \(10^{-1}\) for leptons). \CERT
It is therefore neither conceptually nor numerically plausible to identify \(\fRG\) with \(\mathrm{gap}(Z)\). \PROVED

\subsection{Transport display (bookkeeping only)}
Given a declared target scheme/scale \(\muT\), the transport display is
\begin{equation}
  \mPred(i;\muT)
  \;:=\;
  \mRS(i;\muStar)\,\phig^{\fRG_i(\muStar,\muT)}.
  \CERT
  \label{eq:transport_display}
\end{equation}
\noindent
\textbf{Crucial distinction:} Eq.~\eqref{eq:transport_display} is bookkeeping that aligns an anchor-defined quantity with an external convention.
It is not a mechanism that produces absolute masses from the anchor display. \PROVED
For the charged leptons in this paper, the absolute predictions are provided by the separate lepton chain of Sec.~6. \PROVED

\subsection{Pinned transport policy used for reproducible tables (CERT)}
To make the bookkeeping convention auditable, the repository pins a specific transport policy (loop orders, thresholds, integrator, and targets)
and provides a reproducible certificate of the resulting transport exponents. \CERT
We include the certificate table here to emphasize the order-of-magnitude separation between transport and band structure:

% Auto-generated by tools/rg_transport_table.py
% Policy details: QCD=4L, QED=2L, $\alpha$-run=0L, RK4=10000/ln, thresholds=(1.27,4.18,162.5)\,GeV
\begin{table}[h]
  \centering
  \caption{Pinned SM RG transport exponent certificate (CERT). Policy=\texttt{RS\_CANONICAL\_2025\_Q4}, anchor $\mu_\star=182.201\,\mathrm{GeV}$. (QCD=4L, QED=2L, $\alpha$-run=0L, RK4=10000/ln, thresholds=(1.27,4.18,162.5)\,GeV). These values are used only for scheme/scale bookkeeping and must not be conflated with $\mathrm{gap}(Z)$.}
  \begin{tabular}{lrr}
    \toprule
    Species & $\mu_{\mathrm{end}}$ [GeV] & $f^{RG}(\mu_\star,\mu_{\mathrm{end}})$ \\
    \midrule
    e & 0.000510999 & 0.0494258 \\
    mu & 0.105658 & 0.0287906 \\
    tau & 1.77686 & 0.0178757 \\
    u & 2 & 0.482193 \\
    d & 2 & 0.476388 \\
    s & 2 & 0.476388 \\
    c & 1.27 & 0.547013 \\
    b & 4.18 & 0.380746 \\
    t & 162.5 & 0.00979749 \\
    \bottomrule
  \end{tabular}
\end{table}


\subsection{The diagnostic band test (how to test \texorpdfstring{$\mathrm{gap}(Z)$}{gap(Z)} against transported data)}
If one wants to test whether the charge-derived band map clusters the charged families at the anchor, the correct diagnostic is to
transport the external mass data back to the anchor under the declared RG policy:
\begin{align}
  \mdata(i;\muStar)
  &:= \mdata(i;\muT)\,\phig^{-\fRG_i(\muStar,\muT)}, \VAL \label{eq:data_to_anchor}\\
  f_i^{\mathrm{exp}}(\muStar)
  &:= \log_{\phig}\!\left(\frac{\mdata(i;\muStar)}{m_{\mathrm{skel}}(i;\muStar)}\right). \VAL \label{eq:fexp_def}
\end{align}
Then the band-map validation statement is that \(f_i^{\mathrm{exp}}(\muStar)\) clusters by equal \(Z\) and is consistent with \(\mathrm{gap}(Z_i)\)
under the declared transport policy. \VAL

\subsection{Reviewer checklist for any numerical comparison}
Any numerical objection or alternative comparison must specify:
\begin{itemize}
  \item the target scheme (pole vs.\ \(\overline{\mathrm{MS}}\) vs.\ other),
  \item the target scale \(\muT\),
  \item the RG policy choices (loop orders, thresholds, coupling treatment, integrator), and
  \item the exact statement being tested (anchor organization vs.\ absolute-mass pipeline).
\end{itemize}

\paragraph{Classical correspondence.}
The transport exponent $f^{RG}$ is mathematically identical to the standard logarithmic
running of masses under Standard-Model renormalization-group evolution.
The distinction emphasized in this section---that $f^{RG}$ is bookkeeping while
$\mathrm{gap}(Z)$ is structural---corresponds to the distinction in effective field
theory between scheme-dependent running and scheme-independent physical observables.
The transport display Eq.~\eqref{eq:transport_display} is the analog of an RG-improved
prediction: a fixed-point quantity (the anchor mass) is transported to a comparison scale
using the SM beta functions.
The key difference is that the anchor mass itself is derived from discrete structure,
not fitted to data at any scale. \CERT

\section{Ablations and Falsifiers}
\noindent\fbox{\parbox{0.97\linewidth}{%
\textbf{Section summary.}
The goal of this paper is not merely to fit numbers; it is to propose a small set of structural ingredients that can be \emph{refuted}.
We therefore list ablations (remove one ingredient and observe failure) and falsifiers (observations that would rule out the framework).
The tests below are phrased so that a skeptical reader can reproduce them with alternative scheme/scale choices, provided the choices are stated explicitly.}}

\subsection{Ablations (drop one ingredient and see what breaks)}
\paragraph{Ablation A: remove the quark offset in the \(Z\)-map.}
Replace the quark branch of Eq.~\eqref{eq:Zmap} by \(Z=\tildeQ^2+\tildeQ^4\) (i.e.\ drop the ``\(+4\)''). \HYP
Then the charged-family labeling no longer separates cleanly between up-type and down-type quarks at the anchor: the equal-\(Z\) family clustering that motivates
the band coordinate fails. \VAL

\paragraph{Ablation B: remove the quartic term in the \(Z\)-map.}
Replace Eq.~\eqref{eq:Zmap} by a purely quadratic rule (drop \(\tildeQ^4\)). \HYP
Then the three charged-family \(Z\) values cannot be realized in the required hierarchy, and the band function \(\mathrm{gap}(Z)\) no longer produces the
observed separation between the three charged families. \VAL

\paragraph{Ablation C: change charge integerization.}
Replace \(\tildeQ=6Q\) in Eq.~\eqref{eq:charge_integerization} by \(\tildeQ=kQ\) with \(k\neq 6\) (e.g.\ \(k=3\) or \(k=5\)). \HYP
Then the Standard Model charge set does not map to a stable, sector-consistent integer family labeling, and the equal-\(Z\) family structure breaks. \VAL

\paragraph{Ablation D: remove band structure (skeleton-only).}
Drop the band factor entirely by setting \(\mathrm{gap}(Z)\equiv 0\) in Eq.~\eqref{eq:masslaw_anchor}. \HYP
The remaining skeleton \(m_{\mathrm{skel}}(i;\muStar)\) cannot reproduce the observed charged spectrum without per-species tuning, which is forbidden by the model contract. \VAL

\subsection{Falsifiers (observations that would rule out the framework)}
\paragraph{Falsifier 1: failure of equal-\(Z\) clustering at the anchor.}
Using the diagnostic protocol of Sec.~7 (Eqs.~\eqref{eq:data_to_anchor}--\eqref{eq:fexp_def}) under a declared transport policy,
compute \(f_i^{\mathrm{exp}}(\muStar)\) for the charged fermions.
If the values do not cluster by the three family labels \(Z\in\{24,276,1332\}\), the charge-derived band hypothesis is refuted. \VAL

\paragraph{Falsifier 2: need for per-particle offsets.}
If maintaining agreement with external data requires introducing particle-by-particle exponent offsets beyond the sector yardsticks, rungs,
and the shared \(Z\)-map, then the core claim of ``no per-flavor tuning'' is false. \VAL

\paragraph{Falsifier 3: lepton chain failure beyond declared tolerance.}
The lepton absolute pipeline of Sec.~6 makes a concrete numerical prediction for \(m_e,m_\mu,m_\tau\) under a declared unit convention. \VAL
If future refined measurements (or corrected convention choices) move the PDG targets outside the declared tolerance band of the prediction pipeline,
then the lepton chain is refuted as a universal mechanism. \VAL

\paragraph{Falsifier 4: scheme/scale dependence masquerading as structure.}
If the qualitative conclusions of the framework (family clustering at the anchor; order-of-magnitude separation between \(\mathrm{gap}(Z)\) and \(f^{RG}\);
and the lepton-chain hierarchy) disappear under reasonable alternative scheme/scale declarations, then the framework is not describing an invariant structural signal. \VAL

\paragraph{Classical correspondence.}
The ablation and falsification methodology corresponds to standard hypothesis testing in physics:
ablations are analogous to removing terms from a Lagrangian to check which are essential,
while falsifiers are analogous to the critical tests that distinguish competing theories.
The key methodological point is that the framework is designed to be \emph{refutable}:
unlike a fit with enough free parameters to match any data, the discrete structure here
makes sharp predictions that can fail. \VAL

