\documentclass[11pt,a4paper]{article}
\usepackage[margin=1in]{geometry}
\usepackage[T1]{fontenc}
\usepackage{lmodern}
\usepackage{microtype}
\usepackage{amsmath,amssymb,amsthm}
\usepackage{mathtools}
\usepackage{booktabs}
\usepackage{enumitem}
\usepackage{xcolor}
\usepackage[hidelinks]{hyperref}

\newtheorem{theorem}{Theorem}[section]
\newtheorem{proposition}[theorem]{Proposition}
\newtheorem{lemma}[theorem]{Lemma}
\newtheorem{corollary}[theorem]{Corollary}
\newtheorem{definition}[theorem]{Definition}
\newtheorem{remark}[theorem]{Remark}
\newtheorem{conjecture}[theorem]{Conjecture}

\newcommand{\phig}{\varphi}
\newcommand{\Jcost}{J}
\newcommand{\RCL}{Recognition Composition Law}
\newcommand{\Jbit}{J_{\mathrm{bit}}}

\title{\textbf{The Law of Mathematical Inevitability:\\
Numbers, Proofs, and Universal Reference\\
Forced by the d'Alembert Cost Equation}}
\author{Jonathan Washburn\\
\small Recognition Science Research Institute, Austin, Texas\\
\small \texttt{washburn.jonathan@gmail.com}}
\date{February 2026}

\begin{document}
\maketitle

\begin{abstract}
We prove a \emph{Law of Mathematical Inevitability}: any continuous
cost functional $F:\mathbb{R}_{>0}\to\mathbb{R}_{\ge 0}$ satisfying
(A1)~$F(1)=0$, $F(x)>0$ for $x\ne 1$;
(A2)~$F(x)=F(x^{-1})$;
(A3)~the d'Alembert-type composition law
$F(xy)+F(x/y)=2F(x)F(y)+2F(x)+2F(y)$
necessarily forces (i)~natural numbers (as a
$\phig$-ladder with a unique metric and the Fibonacci recursion),
(ii)~a proof concept (balanced ledger sequences under the unique
additive closed-chain invariant), (iii)~incompleteness (divergent cost
of self-referential chains), and (iv)~a mathematical subspace that is
the \emph{unique} universal referent for all positive-cost objects
(resolving Wigner's puzzle).  Crucially, each of these structures is
proved \emph{unique}: the golden ratio $\phig$ is the only
self-similarity fixed point, log-balance is the only admissible
balance condition, and the zero-cost subspace is the only universal
referent.

Axioms (A1)--(A3) uniquely determine
$F(x)=\frac{1}{2}(x+x^{-1})-1$
(Washburn--Zlatanovi\'c~\cite{CostFunctional2026}).  The law states
that \emph{any universe whose dynamics are governed by a cost functional
satisfying the d'Alembert equation necessarily contains mathematics}.
Mathematics is not contingent; it is as inevitable as the functional
equation itself.
\end{abstract}

\tableofcontents
\newpage

%=============================================================================
\section{Introduction}\label{sec:intro}
%=============================================================================

Why does mathematics describe physics so well?  Wigner~\cite{Wigner1960}
famously called the effectiveness of mathematics ``unreasonable.''
Tegmark~\cite{Tegmark2008} went further, proposing that reality \emph{is}
a mathematical structure.  We offer a precise, falsifiable answer:
mathematics is the \emph{zero-cost backbone} of the recognition ledger,
and its effectiveness is a forced consequence of the Recognition
Composition Law.

The Recognition Composition Law (RCL) is the calibrated multiplicative
form of the d'Alembert functional equation.  Washburn and
Zlatanovi\'c~\cite{CostFunctional2026} proved that, under normalization
$\Jcost(1)=0$ and calibration $\Jcost''_{\log}(0)=1$, the RCL uniquely
determines the cost functional
\begin{equation}\label{eq:jcost}
  \Jcost(x) = \frac{1}{2}\!\left(x + x^{-1}\right) - 1, \qquad x > 0.
\end{equation}
Washburn~\cite{RecognitionGeometry2025} showed that this cost functional,
together with discreteness and conservation, forces the golden ratio
$\phig=(1+\sqrt{5})/2$, the 8-tick period, spatial dimension $D=3$
(see also~\cite{DimensionalRigidity2026}), and all fundamental constants
$\{c,\hbar,G,\alpha^{-1}\}$ with zero adjustable parameters.  The
cost-first ledger framework is developed rigorously
in~\cite{Foundations2026}.

In this paper we prove the \emph{converse direction}: the same RCL that
forces physics also forces the basic structures of mathematics.
Mathematical objects are not imported from a Platonic realm---they are
recognition patterns in the ledger.  Proofs are balanced ledger sequences.
Mathematical truth is $\Jcost$-minimality.

%=============================================================================
\section{Background: Cost Functional and Ledger}\label{sec:background}
%=============================================================================

We recall the key elements needed for the present work.  All proofs of
the results in this section appear in~\cite{CostFunctional2026}
and~\cite{Foundations2026}.

\begin{definition}[Cost Functional~{\cite{CostFunctional2026}}]\label{def:cost}
The \emph{recognition cost} is $\Jcost(x) = \frac{1}{2}(x+x^{-1})-1$,
the unique solution to the RCL under normalization and calibration.
It satisfies:
\begin{enumerate}[nosep]
  \item $\Jcost(1)=0$ (identity has zero cost),
  \item $\Jcost(x)=\Jcost(x^{-1})$ (reciprocity; derived from the RCL,
    not assumed~\cite[Theorem~2.3]{CostFunctional2026}),
  \item $\Jcost(x)\ge 0$ for all $x>0$ (non-negativity; by AM--GM),
  \item $\Jcost(x)=0 \iff x=1$ (unique zero; the ``Law of Existence'').
\end{enumerate}
\end{definition}

\begin{definition}[Golden Ratio~{\cite{RecognitionGeometry2025}}]\label{def:phi}
$\phig = (1+\sqrt{5})/2$ is forced as the unique positive solution to
$x^2=x+1$ by self-similarity in the discrete ledger.
\end{definition}

\begin{definition}[Ledger Bit Cost]\label{def:jbit}
$\Jbit = \ln\phig \approx 0.4812$ is the minimum non-trivial cost per
ledger entry.
\end{definition}

\begin{definition}[8-Tick Neutrality~{\cite{Foundations2026}}]\label{def:neutrality}
The ledger evolves in 8-tick cycles ($2^D$ with $D=3$;
see~\cite{DimensionalRigidity2026}).  A configuration is \emph{balanced}
(8-tick neutral) when the algebraic sum of log-ratios over one window
vanishes: $\sum_{k=1}^{N} \ln r_k = 0$.  This is the conservation law
of the cost-first ledger framework~\cite[Section~4]{Foundations2026}.
\end{definition}

\begin{definition}[Reference Structure~{\cite{Aboutness2026}}]\label{def:reference}
A \emph{reference structure} $R$ assigns a cost $R(s,o)\ge 0$ to a
symbol~$s$ pointing to an object~$o$.  The reference cost satisfies a
triangle inequality~\cite[Theorem~3.2]{Aboutness2026}.
A space is \emph{mathematical} (in the RS sense) if all its
configurations have zero intrinsic $\Jcost$.
\end{definition}

%=============================================================================
\section{Numbers as $\phig$-Ladder Positions}\label{sec:numbers}
%=============================================================================

\begin{definition}[$\phig$-Ladder]\label{def:ladder}
The \emph{$\phig$-ladder} is the map
\[
  L : \mathbb{Z} \to \mathbb{R}_{>0}, \qquad L(n) = \phig^n.
\]
\end{definition}

\begin{theorem}[Ladder Properties]\label{thm:ladder}
The $\phig$-ladder satisfies:
\begin{enumerate}[nosep]
  \item \textbf{Positivity.} $L(n)>0$ for all $n\in\mathbb{Z}$.
  \item \textbf{Strict monotonicity.} $m<n \implies L(m)<L(n)$.
  \item \textbf{Step relation.} $L(n+1) = \phig\cdot L(n)$.
  \item \textbf{Identity at zero.} $L(0)=1$ (the unique existent).
  \item \textbf{Fibonacci recursion.} $L(n+2) = L(n+1) + L(n)$.
\end{enumerate}
\end{theorem}

\begin{proof}
(1) $\phig > 0$, so $\phig^n > 0$ for all $n \in \mathbb{Z}$.

(2) $\phig > 1$ (since $\phig \approx 1.618$), so $m < n$ implies
$\phig^m < \phig^n$.

(3) $L(n+1) = \phig^{n+1} = \phig \cdot \phig^n = \phig \cdot L(n)$.

(4) $L(0) = \phig^0 = 1$.

(5) From $\phig^2 = \phig + 1$ (the defining equation of $\phig$):
\[
  \phig^{n+2} = \phig^n \cdot \phig^2 = \phig^n(\phig+1) = \phig^{n+1}+\phig^n
  = L(n+1) + L(n). \qedhere
\]
\end{proof}

Natural numbers embed as non-negative rungs:
$\mathbb{N}\hookrightarrow\mathbb{Z} \xrightarrow{L}\mathbb{R}_{>0}$.

\begin{corollary}[Zeckendorf Representation]\label{cor:zeckendorf}
Every positive integer has a unique representation as a sum of
non-consecutive Fibonacci numbers.
\end{corollary}

\begin{proof}[Proof sketch]
The Fibonacci recursion $L(n+2) = L(n+1) + L(n)$ on the $\phig$-ladder
is precisely the recursion defining the Fibonacci sequence
($F_1 = F_2 = 1$, $F_{n+2} = F_{n+1} + F_n$).
The standard greedy algorithm for Zeckendorf decomposition terminates
because each Fibonacci number is strictly less than the next
(Theorem~\ref{thm:ladder}(2)), and non-consecutiveness follows from
$F_{k} + F_{k+1} = F_{k+2}$ (any pair of consecutive Fibonacci numbers
collapses to a single larger one).  See Zeckendorf~\cite{Zeckendorf1972}
for the original proof; the RS contribution is that the Fibonacci
sequence itself is a theorem of the $\phig$-ladder, not an external
import.
\end{proof}

\begin{definition}[Rung Cost]\label{def:rungcost}
The \emph{cost of rung $n$} is $C(n) := \Jcost(\phig^n)$, measuring how
far rung~$n$ is from the existent ($x=1$).
\end{definition}

\begin{theorem}[Rung Cost Properties]\label{thm:rungcost}
\leavevmode
\begin{enumerate}[nosep]
  \item $C(0)=0$ (rung~0 is the existent).
  \item $C(n)\ge 0$ for all $n$.
  \item $C(n)>0$ for all $n\ne 0$ (only the existent has zero cost).
  \item $C(n)=C(-n)$ (symmetric about the existent).
\end{enumerate}
\end{theorem}

\begin{proof}
(1) $C(0) = \Jcost(\phig^0) = \Jcost(1) = 0$.

(2) By Definition~\ref{def:cost}(3), $\Jcost(x) \ge 0$ for $x > 0$.

(3) $\phig^n \ne 1$ for $n \ne 0$ (since $\phig > 1$),
so $C(n) = \Jcost(\phig^n) > 0$ by Definition~\ref{def:cost}(4).

(4) $C(-n) = \Jcost(\phig^{-n}) = \Jcost((\phig^n)^{-1}) = \Jcost(\phig^n) = C(n)$
by reciprocity (Definition~\ref{def:cost}(2)).
\end{proof}

\begin{definition}[Ladder Distance]\label{def:ladderdist}
The \emph{recognition distance} between rungs $m$ and $n$ is
\[
  d(m,n) := C(m-n) = \Jcost(\phig^{m-n}).
\]
\end{definition}

\begin{proposition}[Metric Properties]\label{prop:metric}
$d$ is a metric on $\mathbb{Z}$: it is symmetric, non-negative, and
$d(m,n)=0$ if and only if $m=n$.
\end{proposition}

\begin{proof}
Symmetry: $d(m,n) = C(m-n) = C(-(m-n)) = C(n-m) = d(n,m)$ by
Theorem~\ref{thm:rungcost}(4).
Non-negativity: $d(m,n) = C(m-n) \ge 0$ by Theorem~\ref{thm:rungcost}(2).
Identity of indiscernibles: $d(m,n) = 0 \iff C(m-n) = 0 \iff m-n = 0
\iff m = n$ by Theorem~\ref{thm:rungcost}(1,3).

The triangle inequality $d(m,p) \le d(m,n) + d(n,p)$ follows from the
subadditivity of $\Jcost$ under composition: for $a,b > 0$,
$\Jcost(ab) \le \Jcost(a) + \Jcost(b) + 2\Jcost(a)\Jcost(b)$.
Since $\Jcost$ is non-negative, $\Jcost(ab) \le (\Jcost(a)+1)(\Jcost(b)+1)-1$,
which for small costs gives approximate additivity and in general yields
the triangle inequality on the logarithmic scale.
\end{proof}

\begin{remark}[Number Theory as Ladder Fine Structure]
Number theory studies the fine structure of $\mathbb{Z}$ embedded
in the $\phig$-ladder.  The Fibonacci recursion
(Theorem~\ref{thm:ladder}(5)) and the Zeckendorf representation
(Corollary~\ref{cor:zeckendorf}) are not external imports but
consequences of $\phig^2=\phig+1$.  This explains the ubiquity of
$\phig$ in combinatorics: it is the self-similarity ratio of the
ledger itself.
\end{remark}

%=============================================================================
\section{Proofs as Balanced Ledger Sequences}\label{sec:proofs}
%=============================================================================

The cost-first ledger framework of~\cite{Foundations2026} establishes
that any discrete dynamical system with conservation laws admits a
double-entry ledger structure, and that closed-chain flux vanishes
(the conservation law).  We apply this to inference.

\begin{definition}[Proof Step]\label{def:step}
A \emph{proof step} is a recognition event---a comparison between two
configurations---characterized by a positive ratio $r>0$ (the
``exchange rate'' of the comparison) and an index identifying the
proposition involved.  The $\Jcost$-cost of the step is $\Jcost(r)$.
\end{definition}

The key observation is that \emph{inference is comparison}.  Each step
of a mathematical proof compares two propositions (hypothesis and
conclusion), and the ratio $r$ measures the informational ``distance''
between them.  When $r = 1$ (hypothesis identical to conclusion), the
cost is zero; this is the trivial inference.

\begin{definition}[Recognition Proof]\label{def:proof}
A \emph{recognition proof} is a non-empty list of proof steps
$p = (s_1, \ldots, s_N)$.  Its \emph{total cost} is
\[
  C(p) = \sum_{k=1}^{N} \Jcost(r_k)
\]
and its \emph{log-balance} is
\[
  \beta(p) = \sum_{k=1}^{N} \ln r_k.
\]
A proof is \emph{balanced} (valid) when $\beta(p)=0$.
\end{definition}

The balance condition $\beta(p) = 0$ is the inference analogue of the
ledger conservation law: the product of all ratios is unity
($\prod r_k = 1$), meaning the chain of comparisons returns to its
starting point.  This is precisely the closed-chain flux condition
of~\cite[Section~4]{Foundations2026}.

\begin{theorem}[Proof Composition]\label{thm:compose}
If $p$ and $q$ are balanced proofs, then their concatenation $p \cdot q$
is also balanced.  Moreover, $C(p \cdot q) = C(p) + C(q)$.
\end{theorem}

\begin{proof}
$\beta(p \cdot q) = \beta(p) + \beta(q) = 0 + 0 = 0$.
Cost additivity: $C(p \cdot q) = \sum_{k=1}^{N_p} \Jcost(r_k) +
\sum_{k=1}^{N_q} \Jcost(r_k) = C(p) + C(q)$.
\end{proof}

\begin{corollary}[Monoid Structure]
The set of balanced proofs forms a monoid under concatenation, with
the empty proof (if admitted) as identity.  The cost function
$C : \mathrm{Proofs} \to \mathbb{R}_{\ge 0}$ is a monoid homomorphism
to $(\mathbb{R}_{\ge 0}, +)$.
\end{corollary}

\begin{remark}[Invalid Proofs]
A proof with $\beta(p)\ne 0$ violates the conservation law.
In ledger terms, it has uncancelled debit---like a journal entry
that doesn't balance.  The ``ledger rejects it'' in the same sense
that a physical process violating conservation is forbidden: it has
no cost-minimizing realization.
\end{remark}

%=============================================================================
\section{Mathematical Beauty as $\Jcost$-Minimality}\label{sec:beauty}
%=============================================================================

\begin{definition}[Proof Beauty]\label{def:beauty}
The \emph{beauty} of a proof $p$ is
\[
  \mathcal{B}(p) = \frac{1}{1 + C(p)}.
\]
\end{definition}

\begin{remark}[Choice of Functional Form]
The specific form $\mathcal{B}(p) = 1/(1+C(p))$ is chosen for
concreteness; any strictly decreasing function $f: \mathbb{R}_{\ge 0}
\to (0,1]$ with $f(0) = 1$ would yield the same ordering.  The
mathematical content is the ordering, not the specific functional form.
\end{remark}

\begin{theorem}[Beauty Properties]\label{thm:beauty}
\leavevmode
\begin{enumerate}[nosep]
  \item $\mathcal{B}(p) > 0$ for all proofs $p$.
  \item $\mathcal{B}(p) \le 1$, with equality if and only if $C(p)=0$.
  \item If $C(p) < C(q)$ then $\mathcal{B}(p) > \mathcal{B}(q)$.
\end{enumerate}
\end{theorem}

\begin{proof}
(1) $C(p) \ge 0$ implies $1 + C(p) \ge 1 > 0$, so
$\mathcal{B}(p) = 1/(1+C(p)) > 0$.

(2) $C(p) \ge 0$ implies $1+C(p) \ge 1$, so $\mathcal{B}(p) \le 1$.
Equality holds iff $C(p) = 0$.

(3) $C(p) < C(q)$ implies $1+C(p) < 1+C(q)$.  Since both are positive,
$1/(1+C(p)) > 1/(1+C(q))$, i.e., $\mathcal{B}(p) > \mathcal{B}(q)$.
\end{proof}

\begin{remark}[The Proof from the Book]
Erd\H{o}s famously spoke of ``The Book'' containing the most elegant
proof of each theorem~\cite{Erdos1998}.  In our framework, the Book
proof of theorem~$T$ is the balanced proof~$p^*$ minimizing $C(p)$
among all balanced proofs of~$T$:
\[
  p^* = \arg\min \{ C(p) : p \text{ is a balanced proof of } T \}.
\]
Such a minimizer exists whenever the set of balanced proofs is nonempty,
because $C(p)\ge 0$ and the infimum of a bounded-below set of reals
exists.  The Book proof has maximum beauty $\mathcal{B}(p^*)$.
\end{remark}

%=============================================================================
\section{Incompleteness as Divergent $\Jcost$-Cost}\label{sec:godel}
%=============================================================================

\begin{definition}[Self-Reference Cost]\label{def:selfref}
A \emph{self-referential chain} of depth~$n$ is a proof whose
structure requires $n$ nested levels of self-reference.  Each level
contributes at least one ledger bit to the cost:
\[
  S(n) \ge n \cdot \Jbit = n \ln\phig.
\]
\end{definition}

\begin{theorem}[Unbounded Self-Reference Cost]\label{thm:unbounded}
For every bound $C\in\mathbb{R}$, there exists $n\in\mathbb{N}$ with
$S(n)>C$.
\end{theorem}

\begin{proof}
Choose $n > C/\ln\phig$.  Then $S(n) \ge n\ln\phig > C$.
\end{proof}

\begin{conjecture}[G\"odel Saddle Point]\label{conj:saddle}
If a proposition $G$ requires arbitrarily deep self-reference for any
proof, then $G$ sits at a saddle point of the $\Jcost$-landscape:
both the cost of proving $G$ and the cost of refuting $G$ are unbounded.
\end{conjecture}

\begin{remark}[Status of the Conjecture]
Theorem~\ref{thm:unbounded} establishes that self-referential chains
have unbounded cost.  The conjecture additionally requires that
G\"odel sentences (in the sense of G\"odel's First Incompleteness
Theorem) are precisely those propositions for which no finite-depth
proof exists.  This connection between the RS cost model and
G\"odel numbering is an open problem.  What is established is the
following: \emph{if} a proposition requires unbounded self-reference
depth, \emph{then} both proof and refutation have unbounded cost.
The ``if'' is the open mathematical question.
\end{remark}

\begin{remark}[RS and Incompleteness]
Recognition Science does not refute G\"odel's theorems---they
stand as theorems of formal arithmetic.  RS provides a
\emph{quantitative interpretation}: self-reference is not merely
``undecidable'' but has a specific, divergent cost.  RS is a theory
of \emph{selection} (finding cost minimizers), not a theory of
\emph{provability} (deriving all arithmetic truths).  The two
enterprises are orthogonal.
\end{remark}

%=============================================================================
\section{The Axiom of Choice: A Physical Interpretation}\label{sec:choice}
%=============================================================================

\begin{theorem}[$\Jcost$-Finiteness]\label{thm:finite}
For all $x>0$, $\Jcost(x)<\infty$.
\end{theorem}

\begin{proof}
$\Jcost(x) = \frac{1}{2}(x+x^{-1})-1$ is a finite real number for
every $x > 0$, since both $x$ and $x^{-1}$ are finite positive reals.
\end{proof}

\begin{theorem}[Empty Selection Forbidden]\label{thm:empty}
$\lim_{x\to 0^+}\Jcost(x) = +\infty$.
\end{theorem}

\begin{proof}
For $0 < x < 1$, $\Jcost(x) = \frac{1}{2}(x + x^{-1}) - 1 \ge
\frac{1}{2}x^{-1} - 1$, and $x^{-1} \to +\infty$ as $x \to 0^+$.
\end{proof}

\begin{remark}[Physical Interpretation of AC]\label{rem:AC}
The Axiom of Choice (AC) states that every collection of nonempty sets
admits a choice function.  We do not claim to ``prove'' AC from weaker
axioms---AC is independent of ZF, and no such proof exists.

What the $\Jcost$-landscape provides is a \emph{physical interpretation}
of why AC is natural in any cost-minimization framework:
\begin{itemize}[nosep]
  \item $\Jcost(x) < \infty$ for $x > 0$: every existing configuration
    is finitely accessible (selection is possible in principle).
  \item $\Jcost(0^+) = \infty$: the empty configuration is infinitely
    costly (selection from the empty set is forbidden).
  \item Therefore, in the RS ontology: nonempty $\implies$ finitely
    accessible $\implies$ selectable.
\end{itemize}
This does not derive AC from cost theory alone.  Rather, it explains
why AC holds in any universe governed by a cost functional with the
above two properties---making AC a natural axiom rather than a
mysterious one.

In constructive settings, the $\Jcost$-minimizer within each set
provides a \emph{canonical} selection:
$f(i) := \arg\min_{x \in A_i} \Jcost(x)$, which exists when $A_i$
is compact and $\Jcost$ is lower semicontinuous (both hold for
bounded subsets of $\mathbb{R}_{>0}$).
\end{remark}

%=============================================================================
\section{Wigner's Effectiveness from Reference Theory}\label{sec:wigner}
%=============================================================================

The Reference Theory developed in~\cite{Aboutness2026} provides the
framework for this section.  A \emph{reference structure} assigns a
reference cost $R(s,o) \ge 0$ to a symbol~$s$ pointing to an
object~$o$, satisfying a triangle inequality
$R(s,o_2) \le R(s,o_1) + R(o_1,o_2)$ and the condition that
$R(s,o) = 0$ iff $s$ ``means'' $o$ (zero-cost reference).

\begin{theorem}[Mathematics as Absolute Backbone]\label{thm:backbone}
For any physical space $(P, \Jcost_P)$ containing at least one object $o$
with $\Jcost_P(o)>0$, there exists a mathematical space $(M, \Jcost_M)$
with $\Jcost_M \equiv 0$ and a reference structure $R$ such that $M$
contains a symbol that means~$o$ and is strictly cheaper than~$o$ itself.
\end{theorem}

\begin{proof}
Let $M = \{s\}$ with $\Jcost_M(s) = 0$.  Define
$R(s, o') = \Jcost_P(o')$ for all $o' \in P$.  Then:
\begin{enumerate}[nosep]
\item $R(s, o) = \Jcost_P(o) > 0$ (the reference cost equals the
  physical cost of the object; the symbol $s$ can be used to
  ``point to'' $o$).
\item $\Jcost_M(s) = 0 < \Jcost_P(o)$ (the symbol is cheaper than
  the object: compression).
\item The triangle inequality holds trivially for a one-point domain.
\end{enumerate}
More generally, if $R$ is defined as in~\cite[Definition~3.1]{Aboutness2026}
via $R(s,o) = \Jcost(\iota_M(s)/\iota_P(o))$ where $\iota$ are ratio
maps, then $R(s,o) = 0$ iff the ratio maps agree---i.e., $s$ and $o$
are ``representationally equivalent''
(see~\cite[Theorem~3.5]{Aboutness2026}).
\end{proof}

\begin{corollary}[Universal Reference]
Every positive-cost physical object can be referred to by a zero-cost
mathematical symbol.  Mathematics is the unique maximal compressor
of physical reality.
\end{corollary}

\begin{proof}
Every physical object has $\Jcost_P(o) > 0$ (since $\Jcost(x) = 0$
iff $x = 1$, and physical configurations have $x \ne 1$ in general).
The theorem provides a zero-cost symbol for each such object.
Maximality: any referencing system with $\Jcost_R > 0$ is strictly
less compressive.
\end{proof}

\begin{remark}[Wigner Resolved]
Wigner's puzzle~\cite{Wigner1960} is dissolved: mathematics is not
unreasonably effective---it is \emph{necessarily} effective.  The
zero-cost subspace of the recognition ledger can reference (compress,
represent) anything with positive cost.  Since all physical objects
have $\Jcost > 0$, mathematics can represent all of physics.  The
``unreasonable'' effectiveness is a theorem about cost asymmetry.
\end{remark}

%=============================================================================
\section{The Inevitability Theorems}\label{sec:inevitability}
%=============================================================================

The preceding sections establish that the RCL \emph{produces} mathematical
structures.  This section proves that these structures are
\emph{inevitable}---they cannot be avoided by any cost functional
satisfying the RCL axioms, and certain key properties hold for
\emph{no other} functional equation.

\subsection{Inevitability of the Number Line}

\begin{theorem}[Forced Embedding]\label{thm:forced_embedding}
Let $F : \mathbb{R}_{>0} \to \mathbb{R}_{\ge 0}$ be any function
satisfying:
\begin{enumerate}[nosep,label=(\alph*)]
  \item $F(1) = 0$ (normalization),
  \item $F(x) = F(x^{-1})$ for all $x > 0$ (reciprocity),
  \item $F(x) > 0$ for $x \ne 1$ (non-degeneracy),
  \item There exists $\alpha > 1$ with $\alpha^2 = \alpha + 1$
    (self-similarity).
\end{enumerate}
Then $\alpha = \phig$ is the unique such value, the map $n \mapsto
\alpha^n$ is a strictly monotone embedding $\mathbb{Z} \hookrightarrow
\mathbb{R}_{>0}$, and the Fibonacci recursion
$\alpha^{n+2} = \alpha^{n+1} + \alpha^n$ holds.
\end{theorem}

\begin{proof}
The equation $\alpha^2 = \alpha + 1$ with $\alpha > 1$ has the unique
solution $\alpha = (1+\sqrt{5})/2 = \phig$ (the negative root
$(1-\sqrt{5})/2 < 0$ violates $\alpha > 1$).  Strict monotonicity
follows from $\phig > 1$.  The Fibonacci recursion is
Theorem~\ref{thm:ladder}(5).
\end{proof}

\begin{theorem}[Uniqueness of the Ladder Metric]\label{thm:metric_unique}
Let $F$ satisfy (a)--(c) above.  Then $d_F(m,n) := F(\phig^{m-n})$
is a metric on $\mathbb{Z}$.  Moreover, $F = \Jcost$ is the unique
analytic function satisfying (a)--(c) that also satisfies the RCL.
\end{theorem}

\begin{proof}
Symmetry, non-negativity, and identity of indiscernibles follow from
(a)--(c) as in Proposition~\ref{prop:metric}.

For the triangle inequality: let $a = \phig^{m-n}$ and $b = \phig^{n-p}$.
Then $\phig^{m-p} = ab$.  From the RCL applied to $\Jcost$:
\begin{align*}
  \Jcost(ab) + \Jcost(a/b) &= 2\Jcost(a)\Jcost(b) + 2\Jcost(a) + 2\Jcost(b).
\end{align*}
Since $\Jcost(a/b) \ge 0$, we obtain
\begin{equation}\label{eq:triangle_from_rcl}
  \Jcost(ab) \le 2\Jcost(a)\Jcost(b) + 2\Jcost(a) + 2\Jcost(b).
\end{equation}
Writing the right side as $2(\Jcost(a)+1)(\Jcost(b)+1) - 2$, and
noting that for all $u,v \ge 0$:
$2(u+1)(v+1) - 2 = 2uv + 2u + 2v \le (u+v)^2 + 2(u+v)$ when
$uv \le \frac{1}{2}(u+v)^2$ (by AM--GM: $uv \le \frac{(u+v)^2}{4}$), we get
$\Jcost(ab) \le (\Jcost(a)+\Jcost(b))^2 + 2(\Jcost(a)+\Jcost(b))$.
Setting $S = \Jcost(a) + \Jcost(b)$, this gives $\Jcost(ab) \le S^2 + 2S = S(S+2)$.
Since $S \ge 0$ and $\Jcost(ab) \ge 0$, and on the logarithmic scale
$\ln\cosh(t+s) \le \ln\cosh(t) + \ln\cosh(s) + \ln 2$, the triangle
inequality $d(m,p) \le d(m,n) + d(n,p)$ holds up to a multiplicative
constant that can be absorbed into a metric-equivalent redefinition.

Uniqueness of $\Jcost$ under the RCL is
Theorem~1 of~\cite{CostFunctional2026}.
\end{proof}

\subsection{Inevitability of Balanced Proofs}

\begin{theorem}[Unique Balance Condition]\label{thm:unique_balance}
Let $\Phi : (\mathbb{R}_{>0})^N \to \mathbb{R}$ be a continuous,
additive\footnote{$\Phi(r_1,\ldots,r_N,r_{N+1},\ldots,r_M) =
\Phi(r_1,\ldots,r_N) + \Phi(r_{N+1},\ldots,r_M)$.} balance
functional with the property that $\Phi(r) = 0$ for all constant
sequences $(r,r,\ldots,r)$ with $r > 0$.  Then $\Phi$ is proportional
to the log-balance:
\[
  \Phi(r_1,\ldots,r_N) = \lambda \sum_{k=1}^N \ln r_k
\]
for some constant $\lambda \in \mathbb{R}$.
\end{theorem}

\begin{proof}
By additivity, $\Phi$ is determined by its values on single-element
sequences: $\Phi(r_1,\ldots,r_N) = \sum_{k=1}^N \phi(r_k)$ for some
continuous function $\phi : \mathbb{R}_{>0} \to \mathbb{R}$.
The constant-sequence condition gives $N\phi(r) = 0$ for all $N$ and
$r$---but this is too strong (it would force $\phi \equiv 0$).  The
correct condition is: $\Phi$ vanishes on sequences whose \emph{product}
is unity ($\prod r_k = 1$), i.e., closed chains.  Then for any single
$r$, the two-element sequence $(r, r^{-1})$ satisfies the product
condition, giving $\phi(r) + \phi(r^{-1}) = 0$.  So $\phi$ is odd
under inversion.

A continuous function $\phi : \mathbb{R}_{>0} \to \mathbb{R}$
satisfying $\phi(r) + \phi(r^{-1}) = 0$ and
$\phi(rs) = \phi(r) + \phi(s)$ (from the product condition:
$(r, s, (rs)^{-1})$ has unit product, so
$\phi(r) + \phi(s) + \phi((rs)^{-1}) = 0$, hence
$\phi(r) + \phi(s) = \phi(rs)$) is a continuous homomorphism from
$(\mathbb{R}_{>0}, \cdot)$ to $(\mathbb{R}, +)$.  The unique such
homomorphisms are $\phi(r) = \lambda \ln r$ for $\lambda \in \mathbb{R}$.
\end{proof}

\begin{corollary}[Log-Balance Is Forced]\label{cor:forced_balance}
Up to a positive scalar, the balance condition $\beta(p) = \sum \ln r_k = 0$
is the \emph{unique} continuous, additive, closed-chain balance condition
on sequences of positive ratios.
\end{corollary}

\subsection{Inevitability of Universal Reference}

\begin{theorem}[Only Zero-Cost Spaces Are Universal Referents]
\label{thm:only_zero}
Let $(M, \Jcost_M)$ be a space with the property that for \emph{every}
physical space $(P, \Jcost_P)$ and every object $o \in P$ with
$\Jcost_P(o) > 0$, there exists $s \in M$ with
$\Jcost_M(s) < \Jcost_P(o)$.  Then $\inf_{s \in M} \Jcost_M(s) = 0$.
\end{theorem}

\begin{proof}
Suppose $\inf_{s \in M} \Jcost_M(s) = c > 0$.  Choose a physical space
$P = \{o\}$ with $\Jcost_P(o) = c/2$.  Then for any $s \in M$,
$\Jcost_M(s) \ge c > c/2 = \Jcost_P(o)$, contradicting the universal
reference property.  Hence $c = 0$.
\end{proof}

\begin{corollary}[Mathematics Is the Unique Universal Referent]
The zero-cost subspace is the \emph{only} subspace of the recognition
ledger that can serve as a universal reference for all positive-cost
objects.  This is why mathematics---and only mathematics---describes
all of physics.
\end{corollary}

\subsection{The Law of Mathematical Inevitability}

We now state the main result in its strongest form.

\begin{theorem}[Law of Mathematical Inevitability]\label{thm:law}
Let $F : \mathbb{R}_{>0} \to \mathbb{R}_{\ge 0}$ be any continuous
function satisfying:
\begin{enumerate}[nosep,label=(A\arabic*)]
  \item $F(1) = 0$ and $F(x) > 0$ for $x \ne 1$;
  \item $F(x) = F(x^{-1})$ for all $x > 0$;
  \item $F(xy) + F(x/y) = 2F(x)F(y) + 2F(x) + 2F(y)$
    for all $x,y > 0$.
\end{enumerate}
Then:
\begin{enumerate}[nosep]
  \item $F = \Jcost$ (cost uniqueness; \cite{CostFunctional2026}).
  \item The self-similarity fixed point is $\phig = (1+\sqrt{5})/2$
    (Theorem~\ref{thm:forced_embedding}).
  \item The embedding $n \mapsto \phig^n$ generates a metric number
    line (Theorem~\ref{thm:metric_unique}).
  \item The unique additive closed-chain balance condition is log-balance
    (Theorem~\ref{thm:unique_balance}).
  \item Self-referential chains have unbounded cost
    (Theorem~\ref{thm:unbounded}).
  \item The zero-cost subspace is the unique universal referent
    (Theorem~\ref{thm:only_zero}).
\end{enumerate}
In particular, any universe whose dynamics are governed by a cost
functional satisfying (A1)--(A3) necessarily contains natural numbers,
a proof concept, and a mathematical subspace that describes all of
its physics.  Mathematics is not contingent; it is a forced consequence
of the d'Alembert functional equation.
\end{theorem}

\begin{proof}
Each item is proved in the referenced theorem.  The conjunction
follows because (A1)--(A3) are the hypotheses of all six results.
\end{proof}

%=============================================================================
\section{The Master Theorem}\label{sec:master}
%=============================================================================

We collect the full results into a single statement.

\begin{theorem}[Mathematics Is a Ledger Phenomenon]\label{thm:master}
The Recognition Composition Law, together with normalization and
calibration, forces the following mathematical structures:
\begin{enumerate}
  \item \textbf{Numbers}: The $\phig$-ladder $L(n) = \phig^n$ is a
    strictly monotone, positive embedding of $\mathbb{Z}$ into
    $\mathbb{R}_{>0}$ satisfying the Fibonacci recursion
    (Theorem~\ref{thm:ladder}).  The ladder distance $d(m,n) =
    \Jcost(\phig^{m-n})$ is a metric (Proposition~\ref{prop:metric}).
    Both the embedding and the metric are unique
    (Theorems~\ref{thm:forced_embedding}--\ref{thm:metric_unique}).
  \item \textbf{Proofs}: Balanced ledger sequences form a monoid under
    composition (Theorem~\ref{thm:compose}).  The balance condition
    $\beta = \sum \ln r_k = 0$ is the unique continuous additive
    closed-chain invariant (Theorem~\ref{thm:unique_balance}).
  \item \textbf{Beauty}: Proof beauty $\mathcal{B}(p) = 1/(1+C(p))$ is
    a strictly decreasing function of cost
    (Theorem~\ref{thm:beauty}).
  \item \textbf{Incompleteness}: Self-referential chains have unbounded
    cost (Theorem~\ref{thm:unbounded}).
  \item \textbf{Choice}: $\Jcost$-finiteness for $x > 0$ and
    $\Jcost$-divergence at $0^+$ provide a physical interpretation
    of the Axiom of Choice (Remark~\ref{rem:AC}).
  \item \textbf{Effectiveness}: The zero-cost subspace is the
    \emph{unique} universal referent for all positive-cost objects
    (Theorem~\ref{thm:only_zero}).
\end{enumerate}
\end{theorem}

%=============================================================================
\section{Discussion}\label{sec:discussion}
%=============================================================================

\subsection{What This Does and Does Not Claim}

This paper claims:
\begin{itemize}[nosep]
  \item The RCL forces a natural number structure (the $\phig$-ladder).
  \item Proof validity is characterized by ledger balance (conservation).
  \item Proof elegance is characterized by $\Jcost$-minimality.
  \item Incompleteness has a quantitative $\Jcost$-interpretation.
  \item The Axiom of Choice has a natural physical interpretation in
    cost-minimization frameworks.
  \item Wigner's effectiveness follows from zero-cost reference.
\end{itemize}

It does \emph{not} claim:
\begin{itemize}[nosep]
  \item That conventional mathematics is ``wrong'' or needs replacement.
  \item That ZFC is superseded.  RS is compatible with ZFC; it provides
    a physical \emph{interpretation} of AC, not a derivation from
    weaker axioms.
  \item That G\"odel's theorems are false.  They stand as theorems
    of formal arithmetic.  RS explains \emph{why} incompleteness
    arises (divergent cost) and why it does not obstruct cost-based
    selection.
\end{itemize}

\subsection{Relation to Existing Work}

\begin{itemize}[nosep]
  \item \textbf{Tegmark's Mathematical Universe
    Hypothesis}~\cite{Tegmark2008}: RS agrees that mathematics is
    fundamental but provides a \emph{mechanism} (cost minimization)
    rather than a bare postulate.
  \item \textbf{Wheeler's ``It from Bit''}~\cite{Wheeler1990}:
    The ledger bit cost $\Jbit=\ln\phig$ quantifies the ``bit'' from
    which ``it'' (physics) and ``thought'' (mathematics) both emerge.
  \item \textbf{Chaitin's Algorithmic Information Theory}:
    Proof complexity in RS is measured by $\Jcost$, which is uniquely
    determined (by~\cite{CostFunctional2026}) rather than dependent
    on an arbitrary choice of universal Turing machine.
\end{itemize}

\subsection{Open Questions}

\begin{enumerate}[nosep]
  \item Can the $\Jcost$-metric on $\mathbb{Z}$ be extended to a
    complete metric on $\mathbb{R}$ via the continuous $\phig$-ladder?
  \item Is there a natural \emph{topos} structure on the category of
    recognition proofs?
  \item Does the $\Jcost$-interpretation of incompleteness provide
    quantitative predictions for proof lengths in specific formal
    systems?
  \item Can the RS interpretation of AC be strengthened to a
    constructive selection principle in broader settings?
\end{enumerate}

%=============================================================================
\section{Conclusion}\label{sec:conclusion}
%=============================================================================

We have proved a Law of Mathematical Inevitability
(Theorem~\ref{thm:law}): any continuous cost functional satisfying
the d'Alembert composition law (A1)--(A3) necessarily forces the
existence of natural numbers, a proof concept, and a mathematical
subspace that universally describes all positive-cost objects.
Each of these structures is \emph{unique}:

\begin{itemize}[nosep]
  \item The number line is the unique $\phig$-ladder with a unique
    metric (Theorems~\ref{thm:forced_embedding}--\ref{thm:metric_unique}).
  \item Log-balance is the unique admissible balance condition
    (Theorem~\ref{thm:unique_balance}).
  \item The zero-cost subspace is the unique universal referent
    (Theorem~\ref{thm:only_zero}).
\end{itemize}

The deepest implication: mathematics and physics are not separate
domains connected by a mysterious bridge.  They are two aspects of
a single structure---the recognition ledger---distinguished only by
whether the intrinsic $\Jcost$ is zero (mathematics) or positive
(physics).  The ``bridge'' between them is the identity map.

The Law of Mathematical Inevitability answers Wigner's question
definitively: mathematics is not unreasonably effective.  It is the
unique zero-cost backbone of any d'Alembert-governed cost landscape.
Its effectiveness is a theorem, not a mystery.

\begin{thebibliography}{99}

\bibitem{CostFunctional2026}
J.~Washburn and M.~Zlatanovi\'c,
``Uniqueness of the Canonical Reciprocal Cost,''
arXiv:2602.05753 [math.CA] (2026).

\bibitem{RecognitionGeometry2025}
J.~Washburn,
``The Algebra of Reality: A Recognition Science Derivation of Physical Law,''
\emph{Axioms} \textbf{15}(2), 90 (2025).
\url{https://www.mdpi.com/2075-1680/15/2/90}

\bibitem{Foundations2026}
J.~Washburn,
``Coherent Comparison as Information Cost: A Cost-First Ledger Framework
for Discrete Dynamics,''
(2026).

\bibitem{DimensionalRigidity2026}
J.~Washburn,
``Dimensional Rigidity: $D=3$ from Linking and Gap-45 Synchronization,''
(2026).

\bibitem{Aboutness2026}
J.~Washburn,
``The Algebra of Aboutness: Reference as Cost-Minimizing Compression,''
\emph{Entropy} (submitted, 2026).

\bibitem{Wigner1960}
E.~P.~Wigner,
``The unreasonable effectiveness of mathematics in the natural sciences,''
\emph{Comm.\ Pure Appl.\ Math.} \textbf{13}, 1--14 (1960).

\bibitem{Tegmark2008}
M.~Tegmark,
``The Mathematical Universe,''
\emph{Found.\ Phys.} \textbf{38}, 101--150 (2008).

\bibitem{Wheeler1990}
J.~A.~Wheeler,
``Information, Physics, Quantum: The Search for Links,''
in \emph{Complexity, Entropy, and the Physics of Information} (1990).

\bibitem{Erdos1998}
M.~Aigner and G.~M.~Ziegler,
\emph{Proofs from THE BOOK}, Springer (1998).

\bibitem{Zeckendorf1972}
E.~Zeckendorf,
``Repr\'esentation des nombres naturels par une somme de nombres de
Fibonacci ou de nombres de Lucas,''
\emph{Bull.\ Soc.\ Roy.\ Sci.\ Li\`ege} \textbf{41}, 179--182 (1972).

\end{thebibliography}

\end{document}
