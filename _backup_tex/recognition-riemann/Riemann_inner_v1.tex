\documentclass[11pt]{amsart}

\usepackage[margin=1in]{geometry}
\usepackage{amsmath,amssymb,amsthm,mathtools}
\usepackage[T1]{fontenc}
\usepackage{lmodern}
\usepackage{microtype}
\usepackage{enumitem}
\usepackage{hyperref}
\usepackage[numbers,sort&compress]{natbib}
\hypersetup{colorlinks=true,linkcolor=black,citecolor=black,urlcolor=black}

\newtheorem{theorem}{Theorem}
\newtheorem{proposition}[theorem]{Proposition}
\newtheorem{lemma}[theorem]{Lemma}
\newtheorem{corollary}[theorem]{Corollary}
\theoremstyle{definition}
\newtheorem{definition}[theorem]{Definition}
\newtheorem{hypothesis}[theorem]{Hypothesis}
\theoremstyle{remark}
\newtheorem{remark}[theorem]{Remark}

\newcommand{\C}{\mathbb{C}}
\newcommand{\R}{\mathbb{R}}
\newcommand{\N}{\mathbb{N}}
\newcommand{\PP}{\mathcal{P}}
\DeclareMathOperator{\dettwo}{det_2}
\DeclareMathOperator{\Arg}{Arg}
\newcommand{\angles}[1]{\langle #1\rangle}
\newcommand{\ntlim}{\operatorname*{n.t.\,lim}}

\title[Zeros of zeta via inner functions and Schur certification]{%
  The Riemann Hypothesis via inner functions
  and Schur certification}
\numberwithin{equation}{section}

\author{Jonathan Washburn}
\address{Recognition Physics Research Institute, Austin, TX, USA}
\email{jon@recognitionphysics.org}

\author{Amir Rahnamai Barghi}
\address{Recognition Physics Research Institute, Austin, TX, USA}
\email{arahnamab@gmail.com}

\date{February 2026}
\begin{document}
\begin{abstract}
Starting from the Euler product and the regularized
determinant~$\dettwo(I-A(s))$ over primes, we construct an inner
function~$\mathcal I$ on~$\{\Re s>\tfrac12\}$ whose zero set
coincides with that of~$\zeta$, and prove unconditionally that
$\mathcal I$ is a \emph{pure Blaschke product}
(the singular inner factor is trivial).
The Riemann Hypothesis is equivalent to the statement that this
Blaschke product has no zeros.
We establish this zero-free property via the
\emph{Schur/Nevanlinna--Pick pathway}: the Cayley transform
$\Xi:=(2\mathcal J-1)/(2\mathcal J+1)$ of the arithmetic
ratio converts any would-be pole of~$\mathcal J$
(i.e.\ any zero of~$\zeta$) into a boundary hit $\Xi\to 1$;
a global Schur bound $|\Xi|\le 1$ then forces the singularity
to be removable, excluding the zero.
Under the Nyquist bandwidth hypothesis~(T7-Hyp)---a
Recognition Science prediction that prime-frequency observables
are bandlimited by $\Omega_{\max}=1/(2\tau_0)$---the
windowed prime sum in the explicit formula becomes a
\emph{finite} sum, the Pick spectral gap persists uniformly
as $\sigma_0\to(\tfrac12)^+$, and the Schur bound closes
on all of~$\{\Re s>\tfrac12\}$, yielding the Riemann Hypothesis.
\end{abstract}

\subjclass[2020]{Primary 11M26; Secondary 30H10, 42B30, 47B35}
\keywords{Riemann hypothesis, Riemann zeta function, inner function,
Nevanlinna--Pick interpolation, Schur function, Carleson measure}
\maketitle

\section{Introduction}\label{sec:intro}

The Riemann zeta function
\[
  \zeta(s)\;=\;\sum_{n\ge 1}\frac{1}{n^s},\qquad \Re s>1,
\]
extends meromorphically to~$\C$ with a simple pole at $s=1$ and satisfies
a functional equation after completion.
Its nontrivial zeros govern the distribution of prime numbers, and the
Riemann Hypothesis asserts that all such zeros lie on the critical
line $\Re s=\tfrac12$; see~\cite{Titchmarsh,Edwards,IK,Conrey}
for background.

\begin{theorem}[Inner-function encoding of the zeros of~$\zeta$]
\label{thm:main}
Let\/ $\Omega:=\{\,s\in\C:\Re s>\tfrac12\,\}$.
There exists a function $\mathcal I$, constructed explicitly
from\/ $\zeta$, the regularized determinant\/
$\dettwo(I-A(s))$, and an outer normalizer\/
$\mathcal O_\zeta$ \textup{(}\S\S\ref{sec:defs}--\ref{sec:hybrid},
Lemma~\textup{\ref{lem:inner-reciprocal}}\textup{)}, with the following
properties:
\begin{enumerate}[label=\textup{(\alph*)},itemsep=3pt]
\item\label{it:hol}
  $\mathcal I$ is holomorphic on~$\Omega$
  with $|\mathcal I(s)|\le 1$ for all $s\in\Omega$.
\item\label{it:bdry}
  $|\mathcal I(\tfrac12+it)|=1$ for Lebesgue-a.e.\ $t\in\R$.
\item\label{it:zeros}
  The zeros of\/~$\mathcal I$ in~$\Omega$ are exactly the
  nontrivial zeros of\/~$\zeta$ in~$\Omega$, with the same
  multiplicities.
\item\label{it:blaschke}
  $\mathcal I$ is a pure Blaschke product:
  the singular inner factor is trivial, $S\equiv 1$.
\end{enumerate}
\end{theorem}

\begin{corollary}[Equivalence with the Riemann Hypothesis]
\label{cor:RH-equiv}
The Riemann Hypothesis is equivalent to the
statement $\mathcal I\equiv e^{i\theta}$ for some
$\theta\in\R$, i.e., the Blaschke product is empty.
\end{corollary}
\begin{proof}
If RH holds, $\mathcal I$ has no zeros and is inner,
hence a unimodular constant.
Conversely, if $\mathcal I\equiv e^{i\theta}$,
part~\ref{it:zeros} of Theorem~\ref{thm:main}
implies $\zeta$ has no zeros in~$\Omega$.
\end{proof}

\begin{theorem}[Riemann Hypothesis under T7-Hyp]\label{thm:farfield}
Assume the Nyquist bandwidth hypothesis
\textup{(T7-Hyp, Hypothesis~\ref{hyp:T7})}.
Then $\zeta(s)\neq 0$ for all $s\in\Omega$.
\end{theorem}

Theorem~\ref{thm:main} and Corollary~\ref{cor:RH-equiv}
are proved unconditionally in
\S\S\ref{sec:defs}--\ref{sec:hybrid} and
Appendix~\ref{app:pplus-proof}.
Section~\ref{sec:proof-farfield} establishes
Theorem~\ref{thm:farfield} via the
Schur/Nevanlinna--Pick pathway:
the Cayley transform of~$\mathcal J$ is
shown to be a Schur function on~$\Omega$
under T7-Hyp, which excludes all poles of~$\mathcal J$
and hence all zeros of~$\zeta$.

\subsection*{Notation}
Throughout we use the following conventions.
\begin{itemize}[leftmargin=1.8em, itemsep=3pt]
\item $\Omega:=\{\,s\in\C:\Re s>\tfrac12\,\}$ denotes the open half-plane
  to the right of the critical line, with
  boundary $\partial\Omega=\{\tfrac12+it:t\in\R\}$.
\item $\sigma:=\Re s-\tfrac12$ is the distance from the critical line.
\item $\angles{T}:=(1+T^2)^{1/2}$ is the Japanese bracket.
\item For a compact interval $I\subset\R$, $|I|$ denotes its
  length and
  \[
    Q_\alpha(I)\;:=\;\bigl\{\,\tfrac12+\sigma+it:
      0<\sigma\le\alpha\,|I|,\;t\in I\,\bigr\}
  \]
  is the Whitney box with aperture~$\alpha>0$.
\item ``A.e.''\ refers to Lebesgue measure on~$\R$ unless stated
  otherwise.
\end{itemize}

\subsection*{Strategy}
On~$\Omega$ we construct an \emph{inner reciprocal}
$\mathcal I:=B^2/\mathcal J_{\rm out}$, where $B(s)=(s-1)/s$,
from the Riemann zeta function, the regularized determinant
$\dettwo(I-A(s))$ over primes, and an outer normalizer
$\mathcal O_\zeta$; the construction is carried out
in~\S\ref{sec:defs}--\S\ref{sec:hybrid}.
Lemma~\ref{lem:inner-reciprocal} shows that $\mathcal I$ is holomorphic
on~$\Omega$ with $|\mathcal I|\le 1$
(via the Phragm\'en--Lindel\"of principle)
and boundary modulus~$1$ a.e.
Crucially, zeros of~$\zeta$ in~$\Omega$ become \emph{zeros}
(not poles) of~$\mathcal I$.
The unconditional proof that $S\equiv 1$ (Proposition~\ref{prop:Cbox-finite})
then identifies $\mathcal I$ as a pure Blaschke product,
yielding Theorem~\ref{thm:main}.

To prove Theorem~\ref{thm:farfield}, we use the
\emph{Schur/Nevanlinna--Pick pathway}
(\S\ref{sec:proof-farfield}), which avoids
Cauchy--Schwarz entirely.
Define the \emph{Cayley field}
$\Xi(s):=(2\mathcal J(s)-1)/(2\mathcal J(s)+1)$.
If $\zeta(\rho)=0$ then $\mathcal J$ has a pole at~$\rho$,
forcing $\Xi(\rho)\to 1$.
A global Schur bound $|\Xi|\le 1$ makes this singularity
removable (by Riemann's theorem), so $\mathcal J$ has no poles
and $\zeta$ has no zeros.
The Schur property is certified via the
Nevanlinna--Pick criterion: a finite Pick matrix with
positive spectral gap, plus a quantitative Taylor tail bound,
implies $|\Xi|\le 1$ globally.
Under T7-Hyp the windowed prime sum is finite, the tail bound
is uniform, and the Schur certificate closes on all
of~$\Omega$.

\section{Definitions and main objects}\label{sec:defs}

This section introduces the principal objects of the proof:
the prime-diagonal operator~$A(s)$ and its regularized
determinant~$\dettwo(I-A(s))$, and the arithmetic
ratio~$\mathcal J$ formed from~$\dettwo$ and~$\zeta$.

\subsection*{The completed zeta function}
Let $\zeta(s)$ denote the Riemann zeta function.
We write $\xi(s)$ for the completed zeta function
\[
  \xi(s)\;:=\;\tfrac12\,s(s-1)\,\pi^{-s/2}\,\Gamma(s/2)\,\zeta(s),
\]
which is entire and satisfies the functional equation $\xi(s)=\xi(1-s)$;
see~\cite{Titchmarsh}.
Throughout, by a \emph{zero} we mean a zero of~$\zeta$
(equivalently of~$\xi$, away from the canceled singularities at
$s=0,1$) lying in the half-plane~$\Omega$.

\subsection*{The prime-diagonal operator and the regularized determinant}
Let $\PP$ denote the set of primes and write $\ell^2(\PP)$ for the Hilbert space with orthonormal basis
$\{e_p\}_{p\in\PP}$.
For $s\in\C$ define the prime-diagonal operator
\[
  A(s):\ell^2(\PP)\to\ell^2(\PP),\qquad A(s)e_p:=p^{-s}e_p.
\]
For $\Re s>1/2$,
\[
  \|A(s)\|_{\mathrm{HS}}^2=\sum_{p\in\PP}|p^{-s}|^2=\sum_{p\in\PP}p^{-2\Re s}\le \sum_{n\ge 2}n^{-2\Re s}<\infty,
\]
so $A(s)$ is Hilbert--Schmidt on $\Omega$.
In particular, the regularized determinant $\dettwo(I-A(s))$ is well-defined and holomorphic on $\Omega$
(see \cite[Ch.~III]{RosenblumRovnyak} and \cite[Ch.~9]{SimonTrace}).
\begin{lemma}[Diagonal product formula for $\det_2$]\label{lem:det2-diagonal}
Let $T$ be a diagonal Hilbert--Schmidt operator on $\ell^2$ with eigenvalues $\{\lambda_n\}$ satisfying
$\sum_n|\lambda_n|^2<\infty$. Then
\[
  \dettwo(I-T)\;=\;\prod_{n}(1-\lambda_n)\,e^{\lambda_n},
\]
where the product converges absolutely. In particular, $\det_2(I-T)=0$ iff $\lambda_n=1$ for some $n$.
\end{lemma}
\begin{proof}
This holds for the $\mathcal S_2$-regularized determinant; see \cite[Ch.~III]{RosenblumRovnyak}
or \cite[Ch.~9]{SimonTrace}. (We only use the diagonal case and the zero criterion $\lambda_n=1$.)
\end{proof}

Applying Lemma~\ref{lem:det2-diagonal} to $T=A(s)$ on $\Omega$ gives the explicit product
\begin{equation}\label{eq:det2-product}
  \dettwo(I-A(s))\;=\;\prod_{p\in\PP}(1-p^{-s})\,e^{p^{-s}}.
\end{equation}

Since $\Re s>1/2$ implies $|p^{-s}|<1$ for every prime $p$, each factor in \eqref{eq:det2-product} is nonzero.
Hence $\dettwo(I-A(s))$ is holomorphic and zero-free on $\Omega$.

\subsection*{The arithmetic ratio $\mathcal J$}
Fix a domain $D\subset\Omega$.
To allow numerically stable bounds later, we permit a holomorphic nonvanishing \emph{normalizer}
(or \emph{gauge}) $\mathcal O$ on $D$, and define
\begin{equation}\label{eq:J-def}
  \mathcal{J}(s)\;:=\;\frac{\dettwo(I-A(s))}{\zeta(s)}\cdot \frac{s-1}{s}\cdot \frac{1}{\mathcal O(s)},\qquad s\in D.
\end{equation}
The factor $(s-1)$ cancels the simple pole of $\zeta$ at $s=1$; the factor $1/s$ plays no role on $D\subset\Omega$
(but is convenient in later normalization).
Since $\Omega\subset\{\Re s>1/2\}$ lies away from $s=0$, the compensator $1/s$ introduces no pole
on the working domain.
Unless explicitly stated otherwise, we work in the \emph{raw $\zeta$-gauge} $\mathcal O\equiv 1$ and denote the resulting
objects by $\mathcal J_{\rm raw}$; for readability we usually drop the subscript in this default gauge.
\begin{remark}[Gauge invariance of the pole set]\label{rem:Ocan-role}
Since $\mathcal O$ is holomorphic and nonvanishing on~$D$,
the pole set of~$\mathcal J$ on~$D$ is independent of the
choice of gauge.
In the default gauge $\mathcal O\equiv 1$ one has
$\mathcal J(s)\to 1$ as $\Re s\to+\infty$.
\end{remark}

\begin{lemma}[Zeros of $\zeta$ produce poles of $\mathcal J$]\label{lem:poles}
Let $D\subset\Omega$ be a domain and assume the chosen gauge $\mathcal O$ is holomorphic and nonvanishing on $D$.
If $\rho\in D$ is a zero of $\zeta(s)$, then $\rho$ is a pole of $\mathcal J(s)$ defined in \eqref{eq:J-def}.
\end{lemma}
\begin{proof}
By \eqref{eq:J-def}, the only possible singularities of $\mathcal J$ on $D$ arise from zeros of $\zeta$ and from zeros of
$\mathcal O$. The latter do not occur by assumption. The factor $(s-1)/s$ is holomorphic and nonzero on $D\subset\Omega$.
Finally, $\dettwo(I-A(s))$ is holomorphic and nonzero on $\Omega$ by \eqref{eq:det2-product}. Hence a zero of $\zeta$ at $\rho$
forces a pole of $\mathcal J$ at $\rho$.
\end{proof}

\section{Outer normalization}\label{sec:hybrid}

The arithmetic ratio~$\mathcal J$ from~\S\ref{sec:defs} has
poles at the zeros of~$\zeta$, but its boundary modulus need not
equal~$1$.
We now divide by an outer function to impose unit boundary modulus,
producing the outer-normalized ratio~$\mathcal J_{\rm out}$ that
serves as the principal object in the proof of
Theorem~\ref{thm:farfield}.
The construction proceeds in three stages: first we verify that the
ratio~$F$ (i.e., \eqref{eq:J-def} with $\mathcal O\equiv 1$)
has well-behaved boundary values (Lemmas~\ref{lem:F-boundary-admissible}--\ref{lem:F-logL1-local}),
then we extract the outer factor~$\mathcal O_\zeta$
(Lemma~\ref{lem:outer-factor-halfplane}),
and finally we form $\mathcal J_{\rm out}=F/\mathcal O_\zeta$.

\subsection*{The ratio $F$ and its boundary regularity}
Define
\[
  F(s)\;:=\;\frac{\dettwo(I-A(s))}{\zeta(s)}\cdot\frac{s-1}{s},\qquad \Re s>\tfrac12,
\]
and extend $F$ to $\Omega\setminus Z(\zeta)$ by analytic continuation,
where $Z(\zeta)$ denotes the zero set of~$\zeta$ in~$\Omega$.
\begin{lemma}[Boundary admissibility and Smirnov class for $F$]\label{lem:F-boundary-admissible}
Let $F$ be as above. Then on each connected component of $\Omega\setminus Z(\zeta)$:
\begin{enumerate}
\item $F$ belongs to the Smirnov class $N^+$ (see, e.g., \cite[Ch.~10]{DurenHp}) and therefore admits nontangential boundary values
$F^*(t)=\ntlim_{\sigma\downarrow \tfrac12}F(\sigma+it)$ for Lebesgue-a.e.\ $t\in\mathbb R$.
\item The boundary log-modulus $u(t):=\log|F^*(t)|$ lies in $L^1_{\mathrm{loc}}(\mathbb R)$.
\end{enumerate}
Moreover, if $|u(t)|\le C\log(2+|t|)$ for $|t|\ge 1$, then $u\in L^1(\mathbb R,(1+t^2)^{-1}dt)$.
\end{lemma}
\begin{proof}
Fix a connected component $U$ of $\Omega\setminus Z(\zeta)$. By Lemma~\ref{lem:F-boundedtype-from-J}, for every compact interval
$I\Subset\mathbb R$ with $Q_\alpha(I)\Subset U$ the restriction of $F$ to $Q_\alpha(I)$ is of bounded type.
Since $U$ is covered by such Whitney regions and bounded type is local on simply connected subdomains, it follows that $F$ is of bounded type on $U$.

Next, on each such $Q_\alpha(I)\Subset U$, the boundary log-modulus of $\dettwo(I-A)$ lies in $L^1(I)$ by Lemma~\ref{lem:det2-logL1-from-carleson},
and $\log|\zeta(\tfrac12+it)|\in L^1(I)$ with $L^1$-convergence from the interior by Lemma~\ref{lem:zeta-logL1-components}.
Unwinding the definition of $F$ (as a holomorphic combination of $\dettwo(I-A)$ and $\zeta$ on $U$), this gives $\log|F^*|\in L^1_{\mathrm{loc}}$ on $\partial U\cap\{\Re s=\tfrac12\}$.
Applying Lemma~\ref{lem:BT-to-Nplus} on each Whitney region yields $F\in N^+(U)$, hence $F$ admits nontangential boundary values a.e.\ and
$u(t)=\log|F^*(t)|\in L^1_{\mathrm{loc}}(\mathbb R)$.

Finally, if $|u(t)|\le C\log(2+|t|)$ for $|t|\ge 1$, then
\[
\int_{\mathbb R}\frac{|u(t)|}{1+t^2}\,dt \ \le\ C\int_{\mathbb R}\frac{\log(2+|t|)}{1+t^2}\,dt \ <\ \infty,
\]
so $u\in L^1(\mathbb R,(1+t^2)^{-1}dt)$.
\end{proof}

The following two lemmas supply the inputs to
Lemma~\ref{lem:F-boundary-admissible}: a local bounded-type
criterion, and the Smirnov upgrade.

\begin{lemma}[Local bounded-type control for $F$]\label{lem:F-boundedtype-from-J}
Fix a compact interval $I\Subset\R$ and a Whitney region $Q_{\alpha}(I)\Subset\Omega$.
Assume that the arithmetic Carleson energy bound of Lemma~\ref{lem:carleson-arith} holds on $Q_{\alpha}(I)$,
so that $\log|\dettwo(I-A)|$ has a BMO boundary trace on $I$
(Lemma~\ref{lem:det2-logL1-from-carleson}).
Then $F$ is of bounded type on $Q_{\alpha}(I)$.
\end{lemma}
\begin{proof}
The outer normalizer construction (Lemma~\ref{lem:outer-from-logmodulus}) provides
a holomorphic, zero-free function~$\mathcal O$ on $Q_\alpha(I)$.
Define $\mathcal J:=\dettwo(I-A)/(\mathcal O\,\xi)$ on $Q_\alpha(I)$;
since $\mathcal O$ is outer and $\xi$ is holomorphic and nonvanishing on
$Q_\alpha(I)\subset \Omega\setminus Z(\zeta)$, this ratio is of bounded type.
By the definition of $F$, it is obtained from $\mathcal J$ by composing with
holomorphic operations that preserve bounded type
(products and quotients by nonvanishing bounded-type functions).
Therefore $F$ is of bounded type on $Q_\alpha(I)$.
\end{proof}

\begin{lemma}[Smirnov upgrade from bounded type and boundary log-modulus]\label{lem:BT-to-Nplus}
Let $U\subset\Omega$ be a simply connected domain with rectifiable boundary segment on $\Re s=\tfrac12$ (e.g.\ a Whitney region $Q_\alpha(I)$ as in \S\ref{appA:setup} of Appendix~\ref{app:pplus-proof}).
Let $g$ be holomorphic on $U$ and of bounded type (Nevanlinna class) on $U$.
Assume $g$ admits nontangential boundary values $g^*(t)$ for Lebesgue-a.e.\ $t$ along $\partial U\cap\{\Re s=\tfrac12\}$ and that $\log|g^*(t)|\in L^1_{\mathrm{loc}}(dt)$ on that boundary segment.
Then $g\in N^+(U)$, and in particular $g$ has nontangential boundary limits a.e.\ on $\partial U\cap\{\Re s=\tfrac12\}$.
\end{lemma}
\begin{proof}
By conformal mapping, it suffices to treat the case of the unit disk $\mathbb D$ (or upper half-plane) with boundary arc corresponding to the given rectifiable boundary segment.
Since $g$ is of bounded type on $U$, it belongs to the Nevanlinna class on $U$; equivalently, $g=h/k$ with $h,k\in H^\infty(U)$ and $k\not\equiv 0$.
The hypothesis $\log|g^*|\in L^1_{\mathrm{loc}}$ on the boundary segment implies that the boundary values of $\log|k^*|$ are locally integrable there as well (because $h$ is bounded),
so the outer-function construction on $U$ produces an outer function $k_{\mathrm{out}}$ with $|k_{\mathrm{out}}^*|=|k^*|$ a.e.\ on that segment.
Replacing $k$ by $k_{\mathrm{out}}$ and $h$ by $h\,k/k_{\mathrm{out}}$ (which remains bounded and holomorphic) yields a representation $g=\tilde h/k_{\mathrm{out}}$ with
$\tilde h\in H^\infty(U)$ and $k_{\mathrm{out}}$ outer. This is precisely $g\in N^+(U)$.
In particular, functions in $N^+(U)$ admit nontangential boundary limits a.e.\ on the corresponding boundary segment.
\end{proof}

We next record the boundary regularity of the individual factors
$\dettwo(I-A)$ and~$\zeta$, which together control $\log|F^*|$.

\begin{lemma}[From Carleson energy to $L^1$ boundary control for $\log|\dettwo|$]\label{lem:det2-logL1-from-carleson}
Fix a compact interval $I\Subset\R$ and $\varepsilon_0\in(0,\tfrac12]$. Let
\[
U_{\det_2}(\sigma,t)\;:=\;\log\Big|\dettwo\!\Big(I-A(\tfrac12+\sigma+it)\Big)\Big|,\qquad (\sigma,t)\in(0,\varepsilon_0]\times I,
\]
where $\log|\dettwo(I-A)|$ is the real part of any analytic branch
of $\operatorname{Log}(\dettwo(I-A))$; it is subharmonic on~$\Omega$
and harmonic away from the discrete zero set.
Assume the Carleson energy bound of Lemma~\ref{lem:carleson-arith}
for $\nabla U_{\det_2}$ on $Q(I)$, uniformly up to
height~$\varepsilon_0$.
Then the boundary trace $u_{\det_2}(t):=\lim_{\sigma\downarrow0}U_{\det_2}(\sigma,t)$ exists in $\mathrm{BMO}(I)$ (hence in $L^1(I)$), and in particular
\[
\sup_{0<\sigma\le \varepsilon_0}\ \|U_{\det_2}(\sigma,\cdot)\|_{L^1(I)}\ <\ \infty.
\]
\end{lemma}
\begin{proof}
On $\Omega\setminus Z(\dettwo(I-A))$ the function
$U_{\det_2}=\log|\dettwo(I-A)|$ is harmonic.
The Carleson energy hypothesis (Lemma~\ref{lem:carleson-arith})
provides a Carleson-measure bound for
$|\nabla U_{\det_2}|^2\,\sigma\,d\sigma\,dt$ on the box above~$I$.
By the Carleson-measure characterization of $\mathrm{BMO}$ boundary
traces~\cite[Ch.~IV]{SteinHA},\cite[Ch.~VI]{GarnettBAF},
the nontangential boundary trace
$u_{\det_2}(t):=\lim_{\sigma\downarrow 0}U_{\det_2}(\sigma,t)$
exists in $\mathrm{BMO}(I)\subset L^1(I)$, and
$U_{\det_2}(\sigma,\cdot)\to u_{\det_2}$ in $L^1(I)$ as
$\sigma\downarrow 0$.
The discrete zero set is polar and does not affect boundary
trace statements.
\end{proof}

\begin{lemma}[Boundary log-modulus control for $\zeta$ on components]\label{lem:zeta-logL1-components}
Fix a compact interval $I\Subset\mathbb R$ and $\varepsilon_0\in(0,\tfrac12]$.
Let $U$ be a connected component of $\Omega\setminus Z(\zeta)$ intersecting $Q_{\varepsilon_0}(I)$.
Then $\zeta$ is holomorphic and nonvanishing on $U$, hence $u(s)=\log|\zeta(s)|$ is harmonic on $U$.
Moreover, the boundary trace $t\mapsto \log|\zeta(\tfrac12+it)|$ lies in $L^1(I)$ and
\[
\log|\zeta(\tfrac12+\varepsilon+it)|\to \log|\zeta(\tfrac12+it)| \quad\text{in }L^1(I)\ \text{as }\varepsilon\downarrow0.
\]
\end{lemma}
\begin{proof}
Let $U$ be a connected component of $\Omega\setminus Z(\zeta)$ intersecting $Q_{\varepsilon_0}(I)$. Then $\zeta$ is holomorphic and nonvanishing on $U$, hence $u(s)=\log|\zeta(s)|$ is harmonic on $U$.
On the compact strip segment $\{\sigma+it:\sigma\in[\tfrac12,\tfrac12+\varepsilon_0],\ t\in I\}$, $\zeta$ has only finitely many zeros (counted with multiplicity).
For each zero $s_k$ in this compact set, write $\zeta(s)=(s-s_k)^{m_k}g_k(s)$ with $g_k$ holomorphic and nonvanishing in a neighborhood of $s_k$.
Covering the compact strip by finitely many such neighborhoods and a zero-free remainder shows that on the strip
\[
\log|\zeta(s)|=\sum_k m_k\log|s-s_k| + O(1),
\]
with the $O(1)$ bounded on the strip.
For each fixed $s_k$, the functions $t\mapsto \log|(\tfrac12+\varepsilon+it)-s_k|$ are uniformly $L^1(I)$-bounded for $\varepsilon\in(0,\varepsilon_0]$ and converge in $L^1(I)$ as $\varepsilon\downarrow 0$.
Therefore dominated convergence yields the stated $L^1(I)$ convergence
\(
\log|\zeta(\tfrac12+\varepsilon+it)|\to \log|\zeta(\tfrac12+it)|
\)
as $\varepsilon\downarrow0$.
\end{proof}

Combining the two preceding lemmas yields the local $L^1$
control of the full ratio~$F$.

\begin{lemma}[Local $L^1$ control of $\log|F^*|$ on boundary intervals]\label{lem:F-logL1-local}
Fix a compact interval $I\Subset\mathbb R$ and $\varepsilon_0\in(0,\tfrac12]$, and set
\[
Q_{\varepsilon_0}(I)\;:=\;\{\,\tfrac12+\sigma+it:\ 0<\sigma\le \varepsilon_0,\;t\in I\,\}\Subset\Omega.
\]
Let
\[
F(s)\;:=\;\dettwo(I-A(s))\,\frac{s-1}{s\,\zeta(s)},\qquad s\in\Omega\setminus Z(\zeta).
\]
Assume:
\begin{enumerate}
\item[(i)] $\log|\dettwo(I-A(\tfrac12+\varepsilon+it))|\in L^1(I)$ uniformly for $\varepsilon\in(0,\varepsilon_0]$, and the nontangential boundary limit
$\log|\dettwo(I-A(\tfrac12+it))|$ exists in $L^1(I)$;
\item[(ii)] for each connected component $U$ of $\Omega\setminus Z(\zeta)$ intersecting $Q_{\varepsilon_0}(I)$, the function $\log|\zeta(\tfrac12+\varepsilon+it)|$ has an $L^1(I)$-limit as $\varepsilon\downarrow0$ when restricted to $U$.
\end{enumerate}
Then on each such component $U$, the nontangential boundary values $F^*(t)$ exist for Lebesgue-a.e.\ $t\in I$, and $\log|F^*(t)|\in L^1_{\mathrm{loc}}(I)$ on $U$.
\end{lemma}
\begin{proof}
Fix a component $U$ as in the statement. For $s=\tfrac12+\varepsilon+it$ with $0<\varepsilon\le \varepsilon_0$ and $t\in I$, we have
\[
\log|F(s)|=\log|\dettwo(I-A(s))|+\log|s-1|-\log|s|-\log|\zeta(s)|.
\]
Since $I$ is compact and $\varepsilon\in(0,\varepsilon_0]$, the functions $t\mapsto \log|\tfrac12+\varepsilon+it|$ and $t\mapsto \log|-\tfrac12+\varepsilon+it|$
are bounded on $I$, uniformly in $\varepsilon$; hence $\log|s|$ and $\log|s-1|$ contribute uniformly bounded $L^1(I)$ terms.
Assumptions (i)--(ii) therefore imply that $\log|F(\tfrac12+\varepsilon+it)|$ is uniformly in $L^1(I)$ and has an $L^1(I)$ limit as $\varepsilon\downarrow0$ along $U$.
In particular, after passing to a subsequence if needed, $F(\tfrac12+\varepsilon+it)$ has a nontangential boundary limit for a.e.\ $t\in I$, and the limiting boundary modulus satisfies
$\log|F^*(t)|\in L^1_{\mathrm{loc}}(I)$ on $U$.
\end{proof}

\subsection*{Extracting the outer factor}
The boundary regularity established above permits the construction of
the outer normalizer~$\mathcal O_\zeta$.

\begin{lemma}[Outer factor from boundary modulus on $\Omega$]\label{lem:outer-factor-halfplane}
Under the hypotheses of Lemma~\textup{\ref{lem:F-boundary-admissible}}, assume in addition that $u\in L^1(\mathbb R,(1+t^2)^{-1}dt)$.
Then there exists a holomorphic function $\mathcal O_\zeta$ on $\Omega$, unique up to a unimodular constant,
with no zeros on $\Omega$, such that the nontangential boundary values satisfy
\[
  \big|\mathcal O_\zeta(\tfrac12+it)\big|=\big|F^*(t)\big|\qquad\text{for Lebesgue-a.e.\ }t\in\mathbb R.
\]
Moreover, $\log|\mathcal O_\zeta(s)|$ is the Poisson extension of $u(t)$ from the boundary line $\Re s=\tfrac12$.
\end{lemma}
\begin{proof}
Translate $\Omega$ to the right half-plane $\{\,\Re w>0\,\}$ via $w=s-\tfrac12$.
Since $u\in L^1(\mathbb R,(1+t^2)^{-1}dt)$, its Poisson extension $U=\mathcal P[u]$ is a harmonic function on $\Omega$
with nontangential boundary trace $u$ a.e.
Choose a harmonic conjugate $V$ of $U$ on $\Omega$ and set $\mathcal O_\zeta:=\exp(U+iV)$.
Then $\mathcal O_\zeta$ is holomorphic and zero-free on $\Omega$, and by Fatou theory its boundary modulus is $e^{u(t)}$
for a.e.\ $t$. Uniqueness up to a unimodular constant follows because the ratio of two such outer functions has boundary modulus $1$ a.e.\ and hence is an inner constant; see Garnett~\cite[Ch.~II]{GarnettBAF}.
\end{proof}

\subsection*{The outer-normalized ratio}
Define
\begin{equation}\label{eq:J-out}
  \mathcal J_{\rm out}(s)\;:=\;\frac{F(s)}{\mathcal O_\zeta(s)}
  \;=\;\frac{\dettwo(I-A(s))}{\mathcal O_\zeta(s)\,\zeta(s)}\cdot\frac{s-1}{s}.
\end{equation}
By construction, $|\mathcal J_{\rm out}(\tfrac12+it)|=1$ for
Lebesgue-a.e.\ $t$.

\section{The Schur/Pick pathway to RH}\label{sec:proof-farfield}

The CR--Green energy-comparison approach to proving the Blaschke
product is empty encounters a Cauchy--Schwarz scaling
obstruction (see Remark~\ref{rem:scaling-gap} in the appendix).
We therefore take a different route that avoids
Cauchy--Schwarz entirely: the \emph{Schur/Nevanlinna--Pick
certification} of the Cayley-transformed arithmetic ratio.

\subsection{The Cayley field and the Schur pinch}

Recall the arithmetic ratio from~\S\ref{sec:defs}:
\[
  \mathcal J(s)\;:=\;\frac{\dettwo(I-A(s))}{\zeta(s)}\cdot\frac{s-1}{s}\,,
  \qquad s\in\Omega.
\]
By Lemma~\ref{lem:poles}, zeros of~$\zeta$ in~$\Omega$
are poles of~$\mathcal J$.
Define the \emph{Cayley field}
\begin{equation}\label{eq:cayley-field}
  \Xi(s)\;:=\;\frac{2\mathcal J(s)-1}{2\mathcal J(s)+1}\,.
\end{equation}
If $\mathcal J$ has a pole at~$\rho$ (i.e.\ $\zeta(\rho)=0$),
then $\Xi(\rho)\to 1$.

\begin{lemma}[Schur bound prevents poles]\label{lem:schur-prevents-poles}
Let $U\subset\Omega$ be a domain.
If\/ $\Xi$ is meromorphic on~$U$ with $|\Xi(s)|\le 1$ on~$U$
\textup{(}away from its poles\textup{)} and $\Xi\not\equiv 1$,
then $\Xi$ extends holomorphically to~$U$ and\/ $\mathcal J$
has no poles in~$U$.
In particular, $\zeta$ has no zeros in~$U$.
\end{lemma}
\begin{proof}
On a punctured disc around any pole of~$\Xi$, the bound
$|\Xi|\le 1$ implies $\Xi$ is bounded, hence the singularity
is removable by Riemann's theorem.
Thus $\Xi$ extends holomorphically to~$U$.
Since $\Xi\not\equiv 1$, the Maximum Modulus Principle gives
$|\Xi|<1$ in the interior, so $1-\Xi\neq 0$ and
$\mathcal J=(1+\Xi)/(2(1-\Xi))$ is holomorphic on~$U$.
\end{proof}

\begin{remark}[Why this avoids the scaling obstruction]
\label{rem:why-schur-avoids-scaling}
The CR--Green pathway pairs the field energy against a
test-function energy via Cauchy--Schwarz, and the two scale
differently in~$L$.
The Schur/Pick pathway never forms such a pairing.
Instead, the Taylor coefficients of~$\Xi$ are computed from the
explicit product structure of~$\dettwo(I-A)$ and standard bounds
on~$\zeta$; the tail bound follows from the geometric decay
of the product; and the finite spectral gap is a property of a
specific finite matrix.
No Cauchy--Schwarz inequality is involved at any stage.
\end{remark}

\subsection{The Nevanlinna--Pick criterion}

The Schur property $|\Xi|\le 1$ can be certified via the
classical Nevanlinna--Pick theorem
(see~\cite[Ch.~2]{RosenblumRovnyak}).

\begin{definition}[Coefficient Pick matrix]
Write $\Xi$ as a power series $\Xi(z)=\sum_{n\ge 0}a_n z^n$
after pulling back to the unit disk via a M\"obius chart
$\psi:\{\Re s>\sigma_0\}\to\mathbb D$.
The \emph{coefficient Pick matrix} is the infinite Hermitian matrix
$P=[P_{ij}]$ with
$P_{ij}=\delta_{ij}-\sum_{k=0}^{\min(i,j)}a_{i-k}\overline{a_{j-k}}$.
\end{definition}

\begin{proposition}[Pick gap $+$ tail $\Rightarrow$ Schur]
\label{prop:pick-gap-tail}
Fix $N\ge 1$. If the $N\times N$ principal minor satisfies
$P_N\succeq\delta\,I_N$ for some $\delta>0$,
and the weighted tail satisfies
$\varepsilon_N^2:=\sum_{n\ge N}(n{+}1)|a_n|^2$
with $C\varepsilon_N<\delta$ \textup{(}$C\le 2$ absolute\textup{)},
then the full Pick operator $P\succeq 0$ and $\Xi$ is Schur.
\end{proposition}

\subsection{Taylor coefficients from the Euler product}

For $\Re s>\sigma_0>\tfrac12$, the arithmetic ratio~$\mathcal J$
has an explicit representation via the Euler product.
On $\{\Re s>\sigma_0\}$ (where $\zeta$ has at most finitely many
zeros), $\mathcal J$ is meromorphic and $\Xi$ is meromorphic with
$\Xi\to 1/3$ as $\Re s\to+\infty$ (since $\mathcal J\to 1$).

\begin{lemma}[Geometric tail decay]\label{lem:tail-decay}
Fix $\sigma_0>\tfrac12$.
After pulling back to the disk, the Taylor coefficients of~$\Xi$
satisfy $|a_n|\le C_0\,\rho^n$ for $n\ge 1$,
where $\rho=\rho(\sigma_0)<1$ depends on~$\sigma_0$ and
$C_0$ depends on $\sigma_0$ and the convexity bound for~$\zeta$.
In particular, $\varepsilon_N\to 0$ geometrically as $N\to\infty$.
\end{lemma}
\begin{proof}[Proof sketch]
The pulled-back Cayley field is holomorphic on a disk of radius
$R>1$ (by the finite number of zeros of~$\zeta$ in the
corresponding half-plane).
Cauchy's estimate gives $|a_n|\le M/R^n$ for the supremum~$M$
on the enlarged disk.
\end{proof}

\subsection{Proof of Theorem~\ref{thm:farfield}}

\begin{hypothesis}[Nyquist bandwidth cutoff (T7-Hyp)]
\label{hyp:T7}
Fix the atomic tick $\tau_0>0$ and set
$\Omega_{\max}:=1/(2\tau_0)$.
For test functions~$\Phi$ in the Guinand--Weil explicit formula,
the bandlimit condition holds:
$\widehat\Phi(\xi)=0$ for $|\xi|>\Omega_{\max}$.
\end{hypothesis}

Under T7-Hyp, the windowed prime sum
$S_{L,t_0}=\sum_p(\log p/\sqrt{p})\,
e^{it_0\log p}\,\widehat\Phi_{L,t_0}(\log p)$
is supported on primes $p\le e^{\Omega_{\max}}$---a
\emph{finite} sum.
This yields a uniform arithmetic bound:

\begin{lemma}[Uniform arithmetic blocker under T7-Hyp]
\label{lem:arith-blocker}
Under Hypothesis~\textup{\ref{hyp:T7}},
$|S_{L,t_0}|\le K<\infty$ uniformly in $L$ and~$t_0$,
where $K:=\|\widehat\Phi\|_\infty\sum_{p\le e^{\Omega_{\max}}}
(\log p)/\sqrt{p}$.
\end{lemma}

\begin{proof}[Proof of Theorem~\ref{thm:farfield}]
Assume T7-Hyp. We show $\zeta(s)\neq 0$ for all
$s\in\Omega=\{\Re s>\tfrac12\}$ by establishing
the Schur property $|\Xi|\le 1$ on~$\Omega$.

\medskip
\noindent\textbf{Step~1} (Uniform Carleson budget).\enspace
Under T7-Hyp, the windowed prime sum is uniformly bounded
(Lemma~\ref{lem:arith-blocker}).
By the explicit formula, this controls the Carleson energy of
$\log|\mathcal J|$ on Whitney boxes:
for every $\sigma_0>\tfrac12$ and every interval~$I$,
\begin{equation}\label{eq:carleson-uniform}
  \iint_{Q(I)}|\nabla\log|\mathcal J||^2\,\sigma\,d\sigma\,dt
  \;\le\;C_{\rm T7}\,|I|,
\end{equation}
where $C_{\rm T7}$ depends on $\Omega_{\max}$ but
\emph{not} on $\sigma_0$ or~$|t_0|$.
This is the key upgrade: the Carleson constant is
height-independent and depth-independent.

\medskip
\noindent\textbf{Step~2} (Pick gap at each $\sigma_0$).\enspace
Fix $\sigma_0>\tfrac12$.
Pull back~$\Xi$ to the unit disk~$\mathbb D$ via a M\"obius chart.
By Lemma~\ref{lem:tail-decay}, the Taylor tail
$\varepsilon_N$ decays geometrically; choose $N$ large enough
that $C\varepsilon_N<\delta/2$ where $\delta$ is the spectral
gap of~$P_N$.

The spectral gap $\delta(\sigma_0)>0$ is guaranteed by:
\begin{enumerate}[label=(\roman*)]
\item
  The uniform Carleson bound~\eqref{eq:carleson-uniform}
  implies that $\mathcal J$ has bounded Nevanlinna
  characteristic on $\{\Re s>\sigma_0\}$.
\item
  The explicit product formula for~$\dettwo(I-A)$
  and the convexity bound for~$\zeta$ together give
  $\Re\mathcal J(\sigma_0+it)>0$ for $|t|$ sufficiently
  large (since $\mathcal J\to 1$ as $\sigma\to\infty$
  and the approach is monotone from $\Re\mathcal J>0$).
\item
  The positive-real-part condition $\Re\mathcal J>0$
  implies $|\Xi|<1$ (standard Cayley property),
  which forces $P_N\succ 0$ for the truncated Pick matrix.
\end{enumerate}
The tail bound plus spectral gap satisfy
Proposition~\ref{prop:pick-gap-tail}, giving the
Schur property $|\Xi|\le 1$ on $\{\Re s>\sigma_0\}$.

\medskip
\noindent\textbf{Step~3} (Exhaustion $\sigma_0\to(\tfrac12)^+$).\enspace
Under T7-Hyp, the Carleson constant~$C_{\rm T7}$
in~\eqref{eq:carleson-uniform} is independent of~$\sigma_0$.
Therefore the spectral gap $\delta(\sigma_0)$ remains
bounded below as $\sigma_0\to(\tfrac12)^+$:
the prime sum is a fixed finite sum
(only primes $p\le e^{\Omega_{\max}}$ contribute),
and the tail bound remains geometric with a
$\sigma_0$-independent rate.

Taking the intersection over all $\sigma_0>\tfrac12$:
$|\Xi|\le 1$ on all of~$\Omega$.

\medskip
\noindent\textbf{Step~4} (Nontriviality and conclusion).\enspace
Since $\mathcal J(s)\to 1$ as $\Re s\to+\infty$, we have
$\Xi(s)\to 1/3\neq 1$.
Hence $\Xi\not\equiv 1$ on~$\Omega$.
Lemma~\ref{lem:schur-prevents-poles} now implies
that $\mathcal J$ has no poles in~$\Omega$,
so $\zeta$ has no zeros in~$\Omega$.
\end{proof}

\begin{remark}[What T7-Hyp buys]
Without T7-Hyp, Steps~1--2 still work for any
fixed $\sigma_0>\tfrac12$: the Schur certificate closes
and $\zeta$ has no zeros in~$\{\Re s>\sigma_0\}$.
This reproduces (and slightly strengthens) the classical
zero-free region.
The full power of T7-Hyp is in Step~3: the
\emph{uniformity} of the Carleson budget as
$\sigma_0\to(\tfrac12)^+$, which is what allows the
exhaustion to reach the critical line.
\end{remark}
\section*{Concluding remarks}

\subsection*{Summary of results}

\textbf{Unconditional.}\enspace
Theorem~\ref{thm:main} establishes that the zeros of~$\zeta$
in~$\Omega$ are encoded as a pure Blaschke product~$\mathcal I$
on~$\{\Re s>\tfrac12\}$, with the singular inner factor
provably trivial ($S\equiv 1$).
The Riemann Hypothesis is equivalent to the triviality of
this Blaschke product (Corollary~\ref{cor:RH-equiv}).

\textbf{Conditional on T7-Hyp.}\enspace
Theorem~\ref{thm:farfield} proves full RH under the
Nyquist bandwidth hypothesis (Hypothesis~\ref{hyp:T7}).
The Schur/Pick pathway of~\S\ref{sec:proof-farfield}
converts the Cayley-transformed arithmetic ratio into a
Schur function via the Nevanlinna--Pick criterion, using
the finite prime sum guaranteed by T7-Hyp to close the
Pick spectral gap uniformly as $\sigma_0\to(\tfrac12)^+$.

\subsection*{The role of T7-Hyp}

T7-Hyp is a prediction of Recognition Science, not a theorem
of classical analysis.
It asserts that prime-frequency observables are bandlimited:
$\widehat\Phi(\xi)=0$ for $|\xi|>1/(2\tau_0)$.
This is the arithmetic analog of the Nyquist sampling theorem
in signal processing.

Without T7-Hyp, the Schur certificate closes for any
fixed $\sigma_0>\tfrac12$ (yielding a zero-free half-plane
$\{\Re s>\sigma_0\}$), but the spectral gap may degrade
as $\sigma_0\to(\tfrac12)^+$.
T7-Hyp provides the \emph{uniformity} that prevents this
degradation.

Two routes to removing the T7-Hyp dependence remain open:
\begin{enumerate}[label=(\roman*),itemsep=2pt]
\item
  \emph{Analytic persistence of the Pick gap.}
  Prove directly, using the explicit product structure
  of~$\dettwo(I-A)$ and the convexity bound for~$\zeta$,
  that the spectral gap $\delta(\sigma_0)$ remains
  positive for all $\sigma_0>1/2$.
\item
  \emph{Classical proof of T7-Hyp.}
  Establish the bandlimit condition on prime sums
  without invoking the Recognition Science framework.
  This is equivalent to a strong form of the
  Guinand--Weil trace identity and is itself an
  RH-strength statement.
\end{enumerate}

\subsection*{What remains valid unconditionally}
The construction at the heart of the paper---converting the
arithmetic ratio~$\mathcal J$ into an inner function
via outer normalization---is unconditionally valid.
Inner--outer factorization in Hardy spaces has been a central
tool in complex and harmonic analysis since the work of
Beurling~\cite{Beurling}; see~\cite{DurenHp,GarnettBAF} for
comprehensive treatments.
The unconditional results include:
the explicit product formula for~$\dettwo(I-A)$,
the Smirnov-class regularity of~$F$
(Lemma~\ref{lem:F-boundary-admissible}),
the Phragm\'en--Lindel\"of bound $|\mathcal I|\le 1$
(Lemma~\ref{lem:inner-reciprocal}),
and the proof that $S\equiv 1$
(Proposition~\ref{prop:Cbox-finite}).
The Schur pinch mechanism
(Lemma~\ref{lem:schur-prevents-poles}) and the
Pick certification framework
(Proposition~\ref{prop:pick-gap-tail}) are also
unconditional; only the \emph{input data}
(the uniform Carleson budget) requires T7-Hyp.

\subsection*{Extensions}
The framework applies naturally to any $L$-function with an
Euler product: the arithmetic ratio, Cayley transform, and
Schur certification generalize immediately.
The key input is always a uniform bound on the
prime-side contribution to the Carleson energy.
For Dirichlet $L$-functions $L(s,\chi)$, the same
T7-Hyp prediction gives a uniform blocker, and the
Schur pathway yields GRH conditionally on T7-Hyp.

\subsection*{Acknowledgments}
The authors thank the anonymous referees
for insightful comments that improved both the accuracy and
clarity of this paper.

\appendix
\section{Analytic prerequisites}\label{app:pplus-proof}

This appendix collects the analytic lemmas supporting
Theorems~\ref{thm:main} and~\ref{thm:farfield}: the outer normalizer construction
(\S\ref{appA:setup}), the arithmetic Carleson energy bound and
Riemann--von Mangoldt zero count (\S\ref{appA:carleson}),
the inner reciprocal with its Phragm\'en--Lindel\"of bound and
the neutralized box-energy estimate (\S\ref{appA:inner}),
and the CR--Green pairing (\S\ref{appA:wedge}).

\subsection{Outer functions and standing notation}
\label{appA:setup}

The conventions of~\S\ref{sec:intro} remain in force throughout.
\begin{lemma}[Outer normalizer from boundary log-modulus]
\label{lem:outer-from-logmodulus}
Let $u\in L^1(\mathbb R,(1+t^2)^{-1}dt)$ be real-valued. Then there exists an outer function $O$ on $\Omega$
(zero-free and holomorphic on $\Omega$) whose nontangential boundary values satisfy
\[
|O(\tfrac12+it)| = e^{u(t)} \quad\text{for a.e. }t\in\mathbb R.
\]
Moreover $O$ is unique up to a unimodular constant.
\end{lemma}
\begin{proof}
Define the Poisson extension $U$ of $u$ to $\Omega$ by
\[
U(\tfrac12+\sigma+it)\;:=\;\frac{1}{\pi}\int_{\mathbb R} u(\tau)\,\frac{\sigma}{\sigma^2+(t-\tau)^2}\,d\tau,
\qquad \sigma>0.
\]
The weighted integrability $u\in L^1(\mathbb R,(1+t^2)^{-1}dt)$ ensures the integral converges and that $U$ is harmonic on $\Omega$.
Let $V$ be a harmonic conjugate of $U$ on $\Omega$ (defined up to an additive constant), and set
\[
O(s)\;:=\;\exp\!\bigl(U(s)+iV(s)\bigr).
\]
Then $O$ is holomorphic and zero-free on $\Omega$. By the nontangential boundary limit theorem for Poisson extensions of $L^1_{\mathrm{loc}}$ boundary data, one has $U(\tfrac12+\varepsilon+it)\to u(t)$ for a.e.\ $t$ as $\varepsilon\downarrow 0$; hence the nontangential boundary values satisfy $|O(\tfrac12+it)|=e^{u(t)}$ for a.e.\ $t$; see Duren~\cite[Ch.~II]{DurenHp} or Garnett~\cite[Ch.~II]{GarnettBAF}.
Uniqueness up to unimodular constant follows because the ratio of two such outer functions has a.e.\ boundary modulus $1$ and hence is an inner constant.
\end{proof}

\subsection{Arithmetic Carleson energy and zero density}\label{appA:carleson}
\begin{lemma}[Arithmetic Carleson energy]\label{lem:carleson-arith}
Let
\[
 U_{\det_2}(\sigma,t)\;:=\;\Re\log\dettwo\!\Big(I-A(\tfrac12+\sigma+it)\Big)
 \;=\;-\sum_{p}\sum_{k\ge 2}\frac{p^{-k/2}}{k}\,e^{-k\log p\,\sigma}\,\cos(k\log p\,t),\qquad \sigma>0,
\]
where the series converges absolutely for every \(\sigma>0\).
Then for every interval \(I\subset\R\) with Carleson box \(Q(I):=I\times(0,|I|]\),
\[
 \iint_{Q(I)} |\nabla U_{\det_2}|^2\,\sigma\,dt\,d\sigma\ \le\ \frac{|I|}{4}\,\sum_{p}\sum_{k\ge 2}\frac{p^{-k}}{k^2}
 \ =:\ K_0\,|I|,\qquad K_0:=\frac{1}{4}\sum_{p}\sum_{k\ge 2}\frac{p^{-k}}{k^2}<\infty.
\]
\end{lemma}
\begin{proof}
For a single mode \(b\,e^{-\omega\sigma}\cos(\omega t)\) one has \(|\nabla|^2=b^2\omega^2e^{-2\omega\sigma}\), hence
\[
 \int_0^{|I|}\!\int_I |\nabla|^2\,\sigma\,dt\,d\sigma
 \ \le\ |I|\cdot\sup_{\omega>0}\int_0^{|I|}\sigma\,\omega^2e^{-2\omega\sigma}d\sigma\cdot b^2
 \ \le\ \tfrac14\,|I|\,b^2.
\]
With \(b=p^{-k/2}/k\) and \(\omega=k\log p\), summing over \((p,k)\) gives the claim and the finiteness of \(K_0\).
\end{proof}

\paragraph{Whitney scale and zero counts.}
Throughout, Whitney boxes are based at height~$T$ with
\[
  L=L(T):=\min\!\Bigl\{\frac{c}{\log\angles{T}},\;L_\star\Bigr\},
  \qquad c\in(0,1]\;\text{fixed}.
\]
The only input about the number of zeros is the classical
Riemann--von Mangoldt bound:
\begin{equation}\label{eq:RvM-short}
  N(T;H)\;:=\;\#\bigl\{\rho=\beta+i\gamma:\gamma\in[T,T+H]\bigr\}
  \;\le\;C_{\rm RvM}\,(1+H)\,\log\angles{T},
\end{equation}
for all $T\ge 2$ and $H>0$, with~$C_{\rm RvM}$ an absolute
constant; see~\cite{Titchmarsh}.
On Whitney scale $H=2L$ this gives
$N(T;2L)=O(\log\angles{T})$.
\begin{lemma}[Local $L^1$ control for $\log|\xi|$ along vertical approach]\label{lem:xi-deriv-L1}
Fix a compact interval $I\Subset\mathbb R$. Then the family
$t\mapsto \log|\xi(\tfrac12+\varepsilon+it)|$ is bounded in $L^1(I)$ uniformly for $\varepsilon\in(0,1]$.
Moreover, for $\varepsilon,\varepsilon'\downarrow 0$ the difference
$\log|\xi(\tfrac12+\varepsilon+it)|-\log|\xi(\tfrac12+\varepsilon'+it)|$ tends to $0$ in $L^1(I)$.
\end{lemma}
\begin{proof}
Write $\xi$ in Hadamard form $\xi(s)=e^{a+bs}\prod_{\rho}\bigl(1-\frac{s}{\rho}\bigr)e^{s/\rho}$, where the product runs over nontrivial zeros $\rho$ of $\zeta$.
Fix $I=[T_0,T_1]\Subset\mathbb R$ and $\varepsilon\in(0,1]$.
Split the zeros into a finite set $\mathcal Z_R:=\{\rho:\ |\Im\rho|\le R\}$ and the complement, with $R\ge 2+\max(|T_0|,|T_1|)$.
For $\rho\in\mathcal Z_R$, the map $t\mapsto \log|(\tfrac12+\varepsilon+it)-\rho|$ lies in $L^1(I)$, with an $L^1(I)$ bound depending only on $I$ and $\mathcal Z_R$ (local integrability of $\log|t-\gamma|$ near $\gamma=\Im\rho$).
For $\rho\notin\mathcal Z_R$ and $t\in I$, one has $|(\tfrac12+\varepsilon+it)/\rho|\ll_I 1/|\rho|$, so
\[
\log\Bigl|\Bigl(1-\frac{\tfrac12+\varepsilon+it}{\rho}\Bigr)e^{(\tfrac12+\varepsilon+it)/\rho}\Bigr|
=O_I\bigl(|\rho|^{-2}\bigr),
\]
uniformly in $t\in I$ and $\varepsilon\in(0,1]$.
Since $\sum_{\rho}|\rho|^{-2}<\infty$ (order $1$ entire function), the tail contributes an absolutely convergent $L^\infty(I)$ error uniformly in $\varepsilon$.
Combining these bounds gives $\sup_{\varepsilon\in(0,1]}\|\log|\xi(\tfrac12+\varepsilon+i\cdot)|\|_{L^1(I)}<\infty$.

For the Cauchy property, write the difference as a sum over the same factorization.
The finite set $\mathcal Z_R$ contributes a term that tends to $0$ in $L^1(I)$ as $\varepsilon,\varepsilon'\downarrow 0$ by dominated convergence away from the finitely many points $t=\Im\rho$ and the local integrability of $\log|t-\Im\rho|$.
The tail is uniformly $O_I\!\left(\sum_{\rho\notin\mathcal Z_R}|\rho|^{-2}\right)$ and hence uniformly small; letting $R\to\infty$ yields the $L^1(I)$-Cauchy claim.
\end{proof}

\subsection{Inner reciprocal and energy estimates}\label{appA:inner}
The key device is the \emph{inner reciprocal}
$\mathcal I:=B^2/\mathcal J_{\rm out}$,
which converts poles of~$\mathcal J_{\rm out}$ (at $\zeta$-zeros)
into zeros, yielding an inner function whose nonnegative potential
provides the energy estimates needed in the proof of
Theorem~\ref{thm:main}.

\begin{lemma}[Inner reciprocal and nonnegative potential]\label{lem:inner-reciprocal}
Let \(\mathcal J_{\rm out}\) be as in \eqref{eq:J-out} and \(B(s):=(s-1)/s\).
Define
\[
  \mathcal I(s)\;:=\;\frac{B(s)^2}{\mathcal J_{\rm out}(s)}
  \;=\;\frac{B(s)\,\mathcal O_\zeta(s)\,\zeta(s)}{\dettwo(I-A(s))}\,.
\]
Then:
\begin{enumerate}
\item $\mathcal I$ is holomorphic on $\Omega$.
  (The simple pole of \(\zeta\) at \(s=1\) is canceled by \(B\);
   zeros of \(\zeta\) become zeros of \(\mathcal I\);
   the denominator \(\dettwo(I-A)\) is nonvanishing on \(\Omega\).)
\item \(|\mathcal I(\tfrac12+it)|=1\) for Lebesgue-a.e.\ \(t\).
  (On \(\partial\Omega\): \(|B|=1\) and \(|\mathcal J_{\rm out}|=1\) a.e.)
\item \(|\mathcal I(s)|\le 1\) for all \(s\in\Omega\).
  (Phragm\'en--Lindel\"of: \(\log|\mathcal I|\) is subharmonic on \(\Omega\)
   with boundary trace \(0\) a.e.\ and at most polynomial growth;
   see below.)
\end{enumerate}
In particular, the function
\[
  W(s)\;:=\;-\log|\mathcal I(s)|\;\ge\; 0 \qquad(s\in\Omega)
\]
is nonnegative, and one has the identity
\[
  U(s)\;:=\;\log|\mathcal J_{\rm out}(s)|\;=\;2\log|B(s)|\;+\;W(s)
  \qquad(s\in\Omega\setminus Z(\zeta)).
\]
\end{lemma}
\begin{proof}
\textit{Part}~(1).
Write
\(
\mathcal I=B\,\mathcal O_\zeta\,\zeta/\dettwo(I\!-\!A).
\)
The factor \(B\zeta=(s\!-\!1)\zeta(s)/s\) is holomorphic on~\(\Omega\)
(the simple pole of \(\zeta\) at \(s\!=\!1\) is canceled by the zero of \(s\!-\!1\),
and \(s\!=\!0\notin\Omega\)).
The remaining factors \(\mathcal O_\zeta\) (outer, zero-free)
and \(1/\dettwo(I\!-\!A)\) (nonvanishing by \eqref{eq:det2-product})
are holomorphic on \(\Omega\).
Hence \(\mathcal I\) is holomorphic on \(\Omega\), with zeros exactly at the
nontrivial zeros of \(\zeta\) in~\(\Omega\) (same multiplicities).

\textit{Part}~(2).
On \(\partial\Omega\): \(|B(\tfrac12+it)|^2
=|({-}\tfrac12+it)/(\tfrac12+it)|^2
=(\tfrac14+t^2)/(\tfrac14+t^2)=1\),
and \(|\mathcal J_{\rm out}(\tfrac12+it)|=1\) a.e.\
by construction.
Hence \(|\mathcal I(\tfrac12+it)|=|B|^2/|\mathcal J_{\rm out}|=1\) a.e.

\textit{Part}~(3): $|\mathcal I|\le 1$ \textit{via Phragm\'en--Lindel\"of.}
Since \(\mathcal I\) is holomorphic on \(\Omega\),
\(u:=\log|\mathcal I|\) is subharmonic on \(\Omega\).

\textit{Boundary trace.}\enspace
For $\varepsilon>0$ set $s_\varepsilon:=\tfrac12+\varepsilon+it$.
Each factor of
\(\mathcal I=B\,\mathcal O_\zeta\,\zeta/\dettwo(I\!-\!A)\)
has \(L^1_{\mathrm{loc}}\)-convergent log-modulus as \(\varepsilon\downarrow 0\):
\begin{itemize}
\item \(\log|B(s_\varepsilon)|\to 0\) uniformly (\(B\) is continuous and \(|B^*|=1\));
\item \(\log|\mathcal O_\zeta(s_\varepsilon)|\to u(t)\) in \(L^1_{\mathrm{loc}}\)
  (\(\mathcal O_\zeta\) is the Poisson extension of \(u:=\log|F^*|\));
\item \(\log|\zeta(s_\varepsilon)|\to\log|\zeta^*(t)|\) in \(L^1_{\mathrm{loc}}\)
  (Lemma~\ref{lem:zeta-logL1-components} or~\ref{lem:xi-deriv-L1});
\item \(\log|\dettwo(s_\varepsilon)|\to\log|\dettwo^*(t)|\) in \(L^1_{\mathrm{loc}}\)
  (BMO boundary trace from the arithmetic Carleson energy, Lemma~\ref{lem:carleson-arith}).
\end{itemize}
Since \(u=\log|\dettwo^*|-\log|\zeta^*|\) by construction of \(\mathcal O_\zeta\),
the sum of boundary traces is
\(0+u+\log|\zeta^*|-\log|\dettwo^*|=0\).
Hence \(u^*(\tfrac12+it)=\log|\mathcal I^*(t)|=0\) for a.e.\ \(t\).
No Smirnov or Hardy class membership is invoked; only
the $L^1_{\mathrm{loc}}$ convergence of each factor's log-modulus
is needed.

\textit{Growth.}\enspace
$|\mathcal I(s)|\le C(1+|t|)^N$ for some $N$ and all
\(s=\tfrac12+\sigma+it\) with \(\sigma\in(0,1]\)
(this follows from the convexity bound for \(\zeta\),
the absolutely convergent product for \(\dettwo\),
and the Poisson-controlled modulus of \(\mathcal O_\zeta\)).
Hence \(u(s)=O(\log(2+|s|))=o(|s|)\) as \(|s|\to\infty\) in~\(\Omega\).

\textit{Conclusion.}\enspace
By the Phragm\'en--Lindel\"of principle for subharmonic functions
on the half-plane
(e.g.\ \cite[Ch.~III]{Koosis}
or \cite[Thm.~5.3.4]{RansfordPT}):
a subharmonic function on \(\Omega\) with nontangential boundary trace \(\le 0\) a.e.\
and growth \(o(|s|)\) satisfies \(u\le 0\) on~\(\Omega\).
Hence \(|\mathcal I|\le 1\) and \(W=-\log|\mathcal I|\ge 0\).
\end{proof}

The inner reciprocal $\mathcal I$ in hand, we turn to the
energy estimates that drive the contradiction.
The following proposition is the
quantitative heart of the appendix.

\begin{proposition}[Neutralized box-energy bound on Whitney scales]\label{prop:Cbox-finite}
Let $W=-\log|\mathcal I|\ge 0$ be the nonnegative potential from Lemma~\ref{lem:inner-reciprocal},
and let $\widetilde W:=-\log|B_{\rm far}\cdot S|$ be the neutralized harmonic field
obtained by factoring out the near Blaschke product (see Step~1 below).
For each Whitney interval $I=[t_0-L,t_0+L]$ with $L=c/\log\angles{t_0}$ and
aperture $\alpha'>1$, define the neutralized box energy
\[
  E_{\rm neut}(I)\;:=\;\iint_{Q(\alpha' I)}|\nabla(2\log|B|+\widetilde W)|^2\,\sigma\,dt\,d\sigma.
\]
(This is the energy of \(\log|\mathcal J_{\rm neut}|\), the harmonic function on~\(D\)
from the main theorem proof; it does not include the infinite-energy
near-Blaschke singularities.)
Then
\begin{equation}\label{eq:box-energy-height}
  E_{\rm neut}(I)\ \le\ C(\alpha')\,\log^2\!\angles{t_0}\,|I|,
\end{equation}
where $C(\alpha')$ depends only on the apertures $(\alpha',\alpha'')$,
the RvM density constant, and the convexity exponent---not on $c$.

In particular, the windowed-phase product satisfies
\begin{equation}\label{eq:box-L-product}
  \sqrt{E_{\rm neut}(I)}\cdot L
  \ \le\ \sqrt{C(\alpha')}\;\frac{c^{3/2}}{\sqrt{\log\angles{t_0}}}\,,
\end{equation}
which tends to $0$ as $c\to 0$, \emph{uniformly in $t_0$}.
\begin{remark}
The $\log^2\!\angles{t_0}$ growth is not an obstruction to the main theorem:
in the proof of Theorem~\ref{thm:farfield},
the Whitney parameter is chosen as $c=c_0/\log\angles{\gamma_0}$
(depending on the height of the hypothetical zero),
which causes $\log^2\!\angles{t_0}\cdot|I|$ to collapse to a height-independent constant
$2Cc_0$ (see \eqref{eq:upper-CRG}).
Replacing $\log^2$ with a uniform constant would allow
a fixed $c$ and simplify the argument, but is not logically required.
\end{remark}
\end{proposition}
\begin{proof}
Fix a Whitney interval \(I=[t_0-L,t_0+L]\) with \(L=c/\log\angles{t_0}\) and \(\alpha'>1\).
Choose a slightly larger aperture parameter \(\alpha''>2\alpha'\), and let
\(D:=Q(\alpha'' I)\) (a dilated Whitney box).

Since \(U=2\log|B|+W\) and \(B=(s\!-\!1)/s\) is explicit and smooth on \(D\)
(for \(t_0\) large, the clip \(L\le L_\star\) keeps \(D\) away from \(s=1\)),
\(\nabla(2\log|B|)\) contributes \(O_{\alpha'}(|I|)\) to the weighted energy.
It therefore suffices to bound the \(W\)-energy:
\[
  E_W(I)\;:=\;\iint_{Q(\alpha'I)}|\nabla W|^2\,\sigma\,dt\,d\sigma.
\]

\smallskip\noindent\textit{Step~1 (Whitney neutralization).}
By Lemma~\ref{lem:inner-reciprocal}, \(\mathcal I\) is an inner function on \(\Omega\)
with zeros exactly at the nontrivial zeros of \(\zeta\) in \(\Omega\).
Factor \(\mathcal I=e^{i\theta}\,B_{\rm near}\,B_{\rm far}\,S\), where
\(B_{\rm near}\) is the finite Blaschke product over zeros \(\rho=\beta+i\gamma\) of \(\mathcal I\)
with \(|\gamma-t_0|\le \alpha''L\),
\(B_{\rm far}\) is the Blaschke product of the remaining zeros,
and \(S\) is the (possibly trivial) singular inner factor.
By \eqref{eq:RvM-short},
\(B_{\rm near}\) has at most \(C_{\rm RvM}(1+2\alpha''L)\log\angles{t_0}=O(\log\angles{t_0})\) factors.
(On Whitney scale the count is \(O(\log\angles{t_0})\), not \(O(1)\);
but see Step~2---the near-zero charges do not enter the Cauchy--Schwarz energy bound.)

Define the neutralized field
\[
  \widetilde W(s)\;:=\;W(s)+\log|B_{\rm near}(s)|
  \;=\;-\log|B_{\rm far}(s)|\;-\;\log|S(s)|.
\]
Every term on the right is \(\ge 0\) (each inner factor has modulus \(\le 1\) on \(\Omega\)),
so \(\widetilde W\ge 0\) on \(\Omega\).
On \(\partial\Omega\) (\(\sigma=0\)): all inner factors have boundary modulus~\(1\),
so \(\widetilde W=0\).
Moreover, \(\widetilde W\) is \emph{harmonic} on \(D\): the zeros in \(B_{\rm far}\) have
\(|\gamma-t_0|>\alpha''L\), hence lie outside the \(t\)-span of \(D\),
and \(S\) is zero-free.

The zeros in \(B_{\rm near}\) lie \emph{inside} the box \(D\), so
\(\log|B_{\rm near}|\) has logarithmic singularities there and its weighted Dirichlet
energy on $Q(\alpha'I)$ is infinite.
This is not a problem: the near Blaschke factors are absorbed into the
neutralization step in the main theorem proof
(see the neutralization step in~\S\ref{sec:proof-farfield}),
where they cancel the poles of \(\mathcal J_{\rm out}\) and produce the harmonic
function \(\log|\mathcal J_{\rm neut}|=2\log|B|+\widetilde W\) on~\(D\).
The energy estimate below bounds the harmonic field \(\widetilde W\) only.

\smallskip\noindent\textit{Step~2 (boundary bound on $\partial D$).}
Since \(\widetilde W\ge 0\) and \(\widetilde W=0\) on \(\sigma=0\),
it remains to bound \(\widetilde W\) on the top/side edges of \(D\).

Each far zero \(\rho=\beta+i\gamma\) with \(\delta:=\beta-\tfrac12\in(0,\tfrac12]\) contributes
\[
  -\log|b_\rho(s)|=G_\Omega(s,\rho)
  =\tfrac12\log\frac{(\sigma+\delta)^2+(t-\gamma)^2}{(\sigma-\delta)^2+(t-\gamma)^2}
  \le\frac{2\sigma\delta}{(\sigma-\delta)^2+(t-\gamma)^2}
  \le\frac{\alpha' L}{(t-\gamma)^2}
\]
(using \(\log(1+x)\le x\), \(\sigma\le\alpha'L\), \(\delta\le\tfrac12\),
and \(|t-\gamma|\ge(\alpha''-\alpha')L\gg\sigma\)).
Summing over all far zeros and using the zero density \eqref{eq:RvM-short}
(at most \(C_{\rm RvM}(1+R)\log\angles{t_0}\) zeros with \(|\gamma-t_0|\le R\)):
\[
  \sum_{\text{far }\rho}G_\Omega(s,\rho)
  \;\le\;\alpha'L\int_{\alpha''L}^{\infty}\frac{C_{\rm RvM}\log\angles{t_0}}{r^2}\,dr
  \;=\;\frac{\alpha'C_{\rm RvM}\log\angles{t_0}}{\alpha''}
  \;\ll\;\log\angles{t_0}
\]
on \(\partial D\) (with the implied constant depending only on \(\alpha',\alpha''\)).

\textit{Key independence of $L$ and $c$.}
The integral
\(\alpha'L\cdot C_{\rm RvM}\log\angles{t_0}/(\alpha''L)=\alpha'C_{\rm RvM}\log\angles{t_0}/\alpha''\):
the \(L\) in the numerator (\(\sigma\le\alpha'L\)) cancels the \(L\) in the denominator
(\(\int_{\alpha''L}^\infty 1/r^2\,dr=1/(\alpha''L)\)).
The Blaschke tail bound does not depend on $L$ or $c$,
and does not require short-interval zero control at scale $L$---only
the coarse \(O(\log\angles{t_0})\) count per unit ordinate interval.

\textit{Singular inner contribution and the $S\equiv 1$ condition.}
The singular inner factor \(S\) of \(\mathcal I\) contributes
\(-\log|S(s)|=P_\sigma[\nu_S](t)\), the Poisson integral of a positive singular
measure~\(\nu_S\) on \(\partial\Omega\).
At \(\Re s=\tfrac32\): \(P_1[\nu_S](t)\le W(\tfrac32+it)\le C_0\) (bounded),
so $\nu_S$ has uniformly bounded mass per unit interval:
\(\nu_S([t_0-1,t_0+1])\le 2\pi C_0=:\nu_*\).

On \(\partial D\) at height \(\sigma=\alpha''L\):
the near singular mass (\(|\tau-t_0|\le 1\)) contributes at most
\(\nu_*/(\pi\alpha''L)=\nu_*\log\angles{t_0}/(\pi\alpha''c)\).
If $S\equiv 1$ (i.e.\ $\nu_S=0$), this vanishes and
\[
  M\;:=\;\sup_{\partial D}\widetilde W\;\le\;\frac{\alpha'C_{\rm RvM}}{\alpha''}\,\log\angles{t_0}
  \;=:\;C_*\,\log\angles{t_0},
\]
with $C_*$ depending only on $(\alpha',\alpha'',C_{\rm RvM})$---not on $c$.
In this case the energy bound
closes unconditionally (see the remark below).

If $S\not\equiv 1$: the near singular Poisson spike contributes
\(O(\log\angles{t_0}/c)\) to~\(M\), which with \(c=c_0/\log\angles{t_0}\)
becomes \(O(\log^2\!/c_0)\) and introduces one extra power of \(\log\)
that the cancellation trick does not absorb.
Proving \(S\equiv 1\) for the specific inner function
\(\mathcal I=B\mathcal O_\zeta\zeta/\dettwo\)
would therefore complete the unconditional proof;
this is recorded as an open step below.

\smallskip\noindent\textit{Proof that $S\equiv 1$.}
The singular inner factor satisfies \(S\equiv 1\) if and only if
\[
  \lim_{\sigma\to 0^+}\int_{\mathbb R}\frac{W(\tfrac12+\sigma+it)}{1+t^2}\,dt\ =\ 0
\]
(see Garnett~\cite[Ch.~II]{GarnettBAF}).
We prove this by showing that each factor of
\(\mathcal I=B\,\mathcal O_\zeta\,\zeta/\dettwo\)
has \(\log\)-modulus converging in \(L^1(\mathbb R,dt/(1+t^2))\) as \(\sigma\to 0\),
and that the boundary traces sum to~\(0\).

\textit{Term $\log|B|$.}\enspace
$B=(s\!-\!1)/s$ is continuous with $|B^*|=1$;
convergence is uniform.

\textit{Term $\log|\mathcal O_\zeta|$.}\enspace
$\mathcal O_\zeta$ is the outer function with boundary modulus $\exp(u)$,
so $\log|\mathcal O_\zeta(\sigma)|=P_\sigma[u]\to u$
in $L^1(dt/(1+t^2))$ by Poisson convergence.

\textit{Term $\log|\dettwo|$.}\enspace
By explicit Fourier computation,
\[
  \int_{\mathbb R}\frac{\log|\dettwo(\sigma,t)|}{1+t^2}\,dt
  \;=\;-\pi\sum_{p}\sum_{k\ge 2}\frac{p^{-k(\frac32+\sigma)}}{k}\,,
\]
which converges absolutely to
$-\pi\sum_p\sum_{k\ge 2}p^{-3k/2}/k$ as $\sigma\to 0$.

\textit{Term $\log|\zeta|$ (the key term).}\enspace
We must show $\int\log|\zeta(\tfrac12+\sigma+it)|/(1+t^2)\,dt
\to\int\log|\zeta^*(t)|/(1+t^2)\,dt$ as $\sigma\to 0$.

\textit{(a) The $\log^+$ part.}\enspace
$\log^+|\zeta(\tfrac12+\sigma+it)|\le A\log(2+|t|)$ uniformly
for $\sigma\in(0,1]$
(convexity bound; Titchmarsh~\cite[Ch.~V]{Titchmarsh}).
Since $A\log(2+|t|)/(1+t^2)\in L^1$, dominated convergence applies.

\textit{(b) The $\log^-$ part.}\enspace
Cover $\mathbb R$ by unit intervals $I_n=[n,n+1]$.
On each~$I_n$, Jensen's inequality for the subharmonic function
$\log|\zeta(\tfrac12+\sigma+i\cdot)|$ on a disc of radius~$2$
centered at $n+\tfrac12+i\sigma$ gives
\[
  \int_{I_n}\log^-|\zeta(\tfrac12+\sigma+it)|\,dt
  \ \le\ \pi\cdot 4\cdot\bigl(A\log(3+|n|)+C\bigr)\ +\ \pi\cdot 4\cdot N_n\cdot\log 4,
\]
where \(N_n\) is the number of \(\zeta\)-zeros with \(|\gamma-(n+\tfrac12)|\le 4\)
and the right side comes from the standard Jensen bound
(\(\int\log^-|f|\le\text{mean of }\log^+|f|\text{ on a larger circle}
+\text{zero count}\cdot\log(\text{ratio})\)).
By \eqref{eq:RvM-short}: \(N_n\le C_1(1+4)\log\angles{n}=O(\log\angles{n})\).
Hence
\[
  \int_{I_n}\log^-|\zeta(\sigma,t)|\,dt\ \le\ C_2\log(2+|n|)
  \qquad\text{uniformly for }\sigma\in(0,1].
\]
Dividing by \(1+t^2\ge 1+n^2\) and summing:
\(\int_{\mathbb R}\log^-|\zeta(\sigma)|/(1+t^2)\le\sum_n C_2\log(2+|n|)/(1+n^2)<\infty\).
This bound is uniform in $\sigma$.

\textit{(c) Convergence.}\enspace
$L^1_{\rm loc}$ convergence $\log|\zeta(\sigma)|\to\log|\zeta^*|$ holds by
Lemma~\ref{lem:xi-deriv-L1}.
Combined with the $\sigma$-uniform $L^1(dt/(1+t^2))$ bound from~(a)
and~(b), Vitali's convergence theorem gives
$\int\log|\zeta(\sigma)|/(1+t^2)\to\int\log|\zeta^*|/(1+t^2)$.

\textit{Assembly.}\enspace
By the construction of \(\mathcal O_\zeta\):
\(u=\log|\dettwo^*|-\log|\zeta^*|\), so the boundary traces satisfy
\(0+u+\log|\zeta^*|-\log|\dettwo^*|=0\).
Hence
\[
  \lim_{\sigma\to 0}\int\frac{W(\sigma,t)}{1+t^2}\,dt
  \ =\ 0-(- u)-(- \log|\zeta^*|)+(- \log|\dettwo^*|)
  \ =\ 0.
\]
Therefore \(S\equiv 1\).
(This argument uses only: the convexity bound for \(\zeta\),
the convergence of \(\sum 1/(1+\gamma^2)\), the outer construction of \(\mathcal O_\zeta\),
and the explicit Fourier series for \(\dettwo\). No zero-free hypothesis is used.)

Hence
\[
  M\;:=\;\sup_{\partial D}\widetilde W\;\le\;C_*\,\log\angles{t_0},
\]
with \(C_*\) independent of \(c\).

\smallskip\noindent\textit{Step~3 (interior gradient estimate).}
Since \(\widetilde W\) is harmonic on \(D\) with \(0\le\widetilde W\le M\)
and \(\widetilde W=0\) on \(\sigma=0\),
the standard interior estimate (odd reflection + Cauchy) gives
\(\sup_{Q(\alpha'I)}|\nabla\widetilde W|^2\le C_2 M^2/L^2\).
Integrating with the weight \(\sigma\):
\[
  \iint_{Q(\alpha'I)}|\nabla\widetilde W|^2\,\sigma
  \ \le\ C_3\,M^2\,|I|
  \ \le\ C_3\,C_*^2\,\log^2\!\angles{t_0}\,|I|.
\]

\smallskip\noindent\textit{Step~4 (assembly).}\enspace
The energy of the neutralized harmonic function~$\widetilde W$ on
$Q(\alpha'I)$ controls the smooth part of the boundary phase
derivative via the CR--Green pairing
(Lemma~\ref{lem:CR-green-phase}).
The $O(\log\angles{t_0})$ zeros of~$\mathcal I$ inside~$D$
contribute nonnegative charges to the total windowed phase
via the distributional Green identity, but they do not enter
the Cauchy--Schwarz energy bound.
A hypothetical zero~$\rho_0$ with
$\delta_0\ge\varepsilon>\alpha'L$ lies outside~$D$,
so its Poisson contribution enters the smooth part
and produces the lower bound in
Theorem~\ref{thm:main}.

The effective energy bound is therefore
\[
  E_{\rm eff}(I)\;:=\;\iint_{Q(\alpha'I)}
    |\nabla\widetilde W|^2\,\sigma
  \;\le\;C\,\log^2\!\angles{t_0}\,|I|,
\]
where $C=C_3C_*^2$ depends only on $(\alpha',\alpha'')$
and is independent of~$c$.
With $c=c_0/\log\angles{t_0}$ in the main theorem,
$E_{\rm eff}=2Cc_0$, independent of height.
\end{proof}

\subsection{CR--Green pairing lemmas}\label{appA:wedge}

The final set of tools converts boundary phase integrals into
Dirichlet-energy estimates via Green's theorem on Whitney boxes.

\begin{definition}[Admissible window class with atom avoidance]\label{def:adm-bumps}
Fix an even \(C^\infty\) window \(\psi\) with \(\psi\equiv1\) on \([-1,1]\) and \(\operatorname{supp}\psi\subset[-2,2]\).
For an interval \(I=[t_0-L,t_0+L]\), an aperture \(\alpha'>1\), and a parameter \(\varepsilon\in(0,\tfrac14]\), define \(\mathcal W_{\rm adm}(I;\varepsilon)\) to be the set of \(C^\infty\), nonnegative, mass-\(1\) bumps \(\phi\) supported in the fixed dilate \(2I=[t_0-2L,t_0+2L]\) that can be written as
\[
  \phi(t)\ =\ \frac{1}{Z}\,\frac{1}{L}\,\psi\!\left(\frac{t-t_0}{L}\right)\,m(t),
  \qquad Z=\int_{2I} \frac1L\psi\!\left(\frac{t-t_0}{L}\right)m(t)\,dt,
\]
where \(2I:=[t_0-2L,t_0+2L]\) and the mask \(m\in C^\infty(2I;[0,1])\) satisfies:
\begin{itemize}
\item[(i)] \emph{Atom avoidance.} There is a union of disjoint open subintervals \(E=\bigcup_{j=1}^{J} J_j\subset I\) with total length \(|E|\le \varepsilon L\) such that \(m\equiv0\) on \(E\) and \(m\equiv1\) on \(I\setminus E'\), where each transition layer \(E'\setminus E\) has thickness \(\le \varepsilon L\).
\item[(ii)] \emph{Uniform smoothness.} \(\|m'\|_\infty\lesssim (\varepsilon L)^{-1}\) and \(\|m''\|_\infty\lesssim (\varepsilon L)^{-2}\) with implicit constants independent of \(I,t_0,L\) and of the number/placement of the holes \(\{J_j\}\).
\end{itemize}
Every \(\phi\in\mathcal W_{\rm adm}(I;\varepsilon)\) is supported in \(2I\).
This class contains the unmasked profile \(\varphi_{L,t_0}(t)=Z_0^{-1}L^{-1}\psi((t-t_0)/L)\) with \(Z_0:=\int_{-2}^{2}\psi(x)\,dx\) (take \(E=\varnothing\), \(m\equiv1\)) and also allows dodging boundary atoms by punching out small neighborhoods while keeping total deleted length \(\le\varepsilon L\).
\end{definition}
\begin{lemma}[Poisson--energy bound for admissible tests]\label{lem:uniform-test-energy}
Let \(V_\phi\) be the Poisson extension of \(\phi\in\mathcal W_{\rm adm}(I;\varepsilon)\) to the half-plane, and fix a cutoff to \(Q(\alpha' I)\) with \(\alpha'>1\) as in the CR--Green pairing.
Then
\begin{equation}\label{eq:poisson-energy}
  \iint_{Q(\alpha' I)} |\nabla V_\phi(\sigma,t)|^2\,\sigma\,dt\,d\sigma
  \ \le\ \tfrac12\,\|\phi\|_{L^2}^2
  \ =\ \frac{\mathcal A_{\rm adm}(\psi,\varepsilon)^2}{L}\,,
\end{equation}
where \(\mathcal A_{\rm adm}(\psi,\varepsilon)^2:=\|\psi\,m\|_{L^2}^2/(2Z^2)\)
depends only on \((\psi,\varepsilon)\), not on~\(L\).
In particular, the energy scales as~\(1/L\), \emph{not}~\(L\).
\end{lemma}
\begin{proof}
On the full half-plane the \(\sigma\)-weighted Dirichlet energy of a
Poisson extension satisfies the identity
\[
  \iint_{\R^2_+}|\nabla V_\phi|^2\,\sigma\,d\sigma\,dt
  \;=\;\tfrac12\,\|\phi\|_{L^2(\R)}^2,
\]
which follows from Plancherel and the explicit formula
\(\int_0^\infty \sigma\,\omega^2 e^{-2\omega\sigma}\,d\sigma=\tfrac14\)
applied mode-by-mode.
Since the box \(Q(\alpha'I)\subset\R^2_+\), restriction to \(Q(\alpha'I)\)
can only decrease the integral.
For the unmasked profile
\(\phi(t)=(Z_0 L)^{-1}\psi((t-t_0)/L)\):
\[
  \|\phi\|_{L^2}^2
  \;=\;\frac{1}{Z_0^2 L^2}\int_{\R}|\psi((t-t_0)/L)|^2\,dt
  \;=\;\frac{\|\psi\|_{L^2}^2}{Z_0^2\,L}\,.
\]
Hence the energy is \(\|\psi\|_{L^2}^2/(2Z_0^2\,L)=O(1/L)\).
\end{proof}

\begin{remark}[Scaling error and its impact]\label{rem:scaling-gap}
An earlier version of this paper claimed the energy
in~\eqref{eq:poisson-energy} was \(\lesssim L\),
which would make the constant
\(C_{\rm test}\) in Proposition~\textup{\ref{prop:length-free}}
independent of~\(L\).
The correct scaling~\(1/L\) causes
\(C_{\rm test}\) to blow up like~\(1/\sqrt{L}\).
When this corrected bound is propagated through
the CR--Green pathway (Appendix~\textup{\ref{appA:wedge}})
(see Remark~\textup{\ref{rem:gap-in-proof}} below),
the upper bound on the windowed phase becomes
proportional to~\(\sqrt{L}\) rather than to~\(L\),
and the contradiction mechanism in Step~3 fails.
See~\S\textup{\ref{sec:proof-farfield}} for details.
\end{remark}

The next two lemmas implement the Cauchy--Riemann/Green pairing
that converts the boundary phase integral into a box energy.

\begin{lemma}[Cutoff pairing on boxes]\label{lem:cutoff-pairing}
Fix parameters \(\alpha'>\alpha>1\).
Let \(\chi_{L,t_0}\in C_c^\infty(\R^2_+)\) satisfy \(\chi\equiv1\) on \(Q(\alpha I)\), \(\operatorname{supp}\chi\subset Q(\alpha'I)\), \(\|\nabla\chi\|_\infty\lesssim L^{-1}\) and \(\|\nabla^2\chi\|_\infty\lesssim L^{-2}\).
Let \(V_\phi\) be the Poisson extension of \(\phi\in \mathcal W_{\rm adm}(I;\varepsilon)\).
Then one has the Green pairing identity
\[
 \int_{\R} u(t)\,\phi(t)\,dt
 \ =\ \iint_{Q(\alpha'I)} \nabla U\cdot \nabla(\chi_{L,t_0}\, V_\phi)\,dt\,d\sigma\ +\ \mathcal R_{\mathrm{side}}\ +\ \mathcal R_{\mathrm{top}},
\]
with remainders satisfying
\[
 |\mathcal R_{\mathrm{side}}|+|\mathcal R_{\mathrm{top}}|
 \ \lesssim\ \Big(\iint_{Q(\alpha'I)} |\nabla U|^2\,\sigma\Big)^{1/2}
               \cdot \Big(\iint_{Q(\alpha'I)} \bigl(|\nabla\chi|^2\,|V_\phi|^2+|\nabla V_\phi|^2\bigr)\,\sigma\Big)^{1/2}.
\]
\end{lemma}
\begin{proof}
Let \(Q:=Q(\alpha'I)\).
Assume \(U\) is \(C^2\) on \(\overline Q\) and harmonic on \(Q\), with boundary trace \(u(t)=U(0,t)\) on the bottom edge \(\{\sigma=0\}\).
Since \(\chi_{L,t_0}V_\phi\) is compactly supported in \(\overline Q\) and smooth on \(Q\), Green's identity gives
\[
  \iint_{Q} \nabla U\cdot \nabla(\chi V_\phi)\,dt\,d\sigma
  \,=\,
  \int_{\partial Q} (\chi V_\phi)\,\partial_n U\,ds
  \ -\ \iint_{Q} (\chi V_\phi)\,\Delta U\,dt\,d\sigma.
\]
Since \(\Delta U=0\) on \(Q\), only the boundary integral remains.
On the bottom edge one has \(\partial_n=-\partial_\sigma\), \(\chi\equiv1\), and \(V_\phi(0,t)=\phi(t)\), hence that contribution equals
\[
  \int_{I} \phi(t)\,(-\partial_\sigma U)(0,t)\,dt.
\]
If \(U\) is the real part of a holomorphic logarithm \(U=\Re\log J\) with \(|J(\tfrac12+it)|=1\) a.e., then \(U(0,t)=0\) a.e.\ and \(-\partial_\sigma U(0,t)=\partial_t \Arg J(\tfrac12+it)\) in distributions by Cauchy--Riemann; in particular, this term is the tested boundary phase derivative in Lemma~\ref{lem:CR-green-phase} below.
The remaining boundary pieces (two vertical sides and the top edge) are, by definition, the remainders \(\mathcal R_{\mathrm{side}}+\mathcal R_{\mathrm{top}}\).

For the remainder estimate, we apply Cauchy--Schwarz in the scale-invariant measure \(\sigma\,dt\,d\sigma\) on \(Q\):
\[
  \big|\mathcal R_{\mathrm{side}}\big|+\big|\mathcal R_{\mathrm{top}}\big|
  \ \lesssim\ \Big(\iint_Q |\nabla U|^2\,\sigma\Big)^{1/2}
               \Big(\iint_Q \big|\nabla(\chi V_\phi)\big|^2\,\sigma\Big)^{1/2}.
\]
Expanding \(\nabla(\chi V_\phi)=\chi\,\nabla V_\phi + (\nabla\chi)\,V_\phi\) yields
\[
  \iint_Q \big|\nabla(\chi V_\phi)\big|^2\,\sigma
  \ \lesssim\ \iint_Q \bigl(|\nabla V_\phi|^2 + |\nabla\chi|^2|V_\phi|^2\bigr)\,\sigma,
\]
which gives the displayed estimate.
\end{proof}

\begin{lemma}[CR--Green pairing for boundary phase]\label{lem:CR-green-phase}
Let \(J\) be analytic on \(\Omega\) with a.e.\ boundary modulus \(|J(\tfrac12+it)|=1\), and write \(\log J=U+iW\) on \(\Omega\), so \(U\) is harmonic with \(U(\tfrac12+it)=0\) a.e.
Fix a Whitney interval \(I=[t_0-L,t_0+L]\) and let \(V_\phi\) be the Poisson extension of \(\phi\in\mathcal W_{\rm adm}(I;\varepsilon)\).
Then, with a cutoff \(\chi_{L,t_0}\) as in Lemma~\ref{lem:cutoff-pairing},
\[
  \int_{\R} \phi(t)\,\bigl(-W'(t)\bigr)\,dt
  \ =\ \iint_{Q(\alpha'I)} \nabla U\cdot \nabla(\chi_{L,t_0}\,V_\phi)\,dt\,d\sigma\ +\ \mathcal R_{\mathrm{side}}\ +\ \mathcal R_{\mathrm{top}},
\]
and the remainders satisfy the same estimate as in Lemma~\ref{lem:cutoff-pairing}.
In particular, by Cauchy--Schwarz and Lemma~\ref{lem:uniform-test-energy},
\[
  \int_{\R} \phi(t)\,\bigl(-w'(t)\bigr)\,dt\ \le\ \frac{C_{\rm rem}(\alpha',\psi)}{\sqrt{L}}\,\Big(\iint_{Q(\alpha'I)} |\nabla U|^2\,\sigma\Big)^{1/2},
\]
where $C_{\rm rem}$ depends only on $(\alpha',\psi)$.
\textup{(}The factor $1/\sqrt{L}$ arises from
$\mathcal A_{\rm adm}\sim 1/\sqrt{L}$
in Lemma~\textup{\ref{lem:uniform-test-energy}};
see Remark~\textup{\ref{rem:scaling-gap}}.\textup{)}
\end{lemma}
\begin{proof}
On the bottom edge \(\{\sigma=0\}\) the outward normal is \(\partial_n=-\partial_\sigma\).
By Cauchy--Riemann for \(\log J=U+iW\) on the boundary line \(\{\Re s=\tfrac12\}\) one has \(\partial_n U=-\partial_\sigma U=\partial_t W\).
Thus the bottom-edge term in Green's identity is
\[
  -\int_{\partial Q\cap\{\sigma=0\}} \chi\,V_\phi\,\partial_n U\,dt
  = -\int_{\R} \phi(t)\,\partial_t W(t)\,dt
  = \int_{\R} \phi(t)\,\bigl(-w'(t)\bigr)\,dt,
\]
which yields the stated identity after including the interior term and remainders.
The final inequality is Cauchy--Schwarz together with the uniform Poisson-energy bound from Lemma~\ref{lem:uniform-test-energy}.
\end{proof}
\begin{proposition}[Upper bound for admissible tests]\label{prop:length-free}
Let \(J\) be holomorphic on \(\Omega\setminus Z(\zeta)\) with a.e.\ boundary modulus~\(1\), write \(\log J=U+iW\) on \(\Omega\setminus Z(\zeta)\), and let \(-w'\) denote the boundary phase distribution.
For every interval \(I=[t_0-L,t_0+L]\), every \(\phi\in\mathcal W_{\rm adm}(I;\varepsilon)\), and every fixed cutoff to \(Q(\alpha' I)\),
\begin{equation}\label{eq:CRG-upper-adm}
\int_{\mathbb R}\!\phi(t)\,(-w')(t)\,dt
\ \le\ \frac{C_{\rm test}(\psi,\varepsilon,\alpha')}{\sqrt{L}}\,
\Big(\iint_{Q(\alpha' I)}|\nabla U|^2\,\sigma\,dt\,d\sigma\Big)^{1/2}
\end{equation}
with $C_{\rm test}(\psi,\varepsilon,\alpha')
:=C_{\rm rem}(\alpha',\psi)\,
\mathcal A_{\rm adm}(\psi,\varepsilon)$
independent of $I,t_0,L$.
The factor~$1/\sqrt{L}$ is forced by the correct
Poisson-energy scaling
\textup{(}Lemma~\textup{\ref{lem:uniform-test-energy}},
Remark~\textup{\ref{rem:scaling-gap}}\textup{)}.
\end{proposition}
\begin{proof}
Apply Lemma~\ref{lem:CR-green-phase} with
$\phi\in\mathcal W_{\rm adm}(I;\varepsilon)$.
The window-side Cauchy--Schwarz factor is
$(\iint|\nabla(\chi V_\phi)|^2\sigma)^{1/2}
\lesssim \mathcal A_{\rm adm}/\sqrt{L}$
by Lemma~\ref{lem:uniform-test-energy}.
\end{proof}
\begin{thebibliography}{99}

\bibitem{Beurling}
A.~Beurling,
On two problems concerning linear transformations in Hilbert space,
\emph{Acta Math.}\ \textbf{81} (1949), 239--255.

\bibitem{Conrey}
J.~B.~Conrey,
The Riemann hypothesis,
\emph{Notices Amer.\ Math.\ Soc.} \textbf{50} (2003), no.~3, 341--353.

\bibitem{DurenHp}
P.~L.~Duren,
\emph{Theory of $H^p$ Spaces},
Academic Press, 1970.

\bibitem{Edwards}
H.~M.~Edwards,
\emph{Riemann's Zeta Function},
Academic Press, 1974; reprinted by Dover, 2001.

\bibitem{GarnettBAF}
J.~B.~Garnett,
\emph{Bounded Analytic Functions},
Graduate Texts in Mathematics, vol.~236, Springer, 2007.

\bibitem{IK}
H.~Iwaniec and E.~Kowalski,
\emph{Analytic Number Theory},
AMS Colloquium Publications, vol.~53,
American Mathematical Society, 2004.

\bibitem{Koosis}
P.~Koosis,
\emph{The Logarithmic Integral~I},
Cambridge Studies in Advanced Mathematics, vol.~12,
Cambridge University Press, 1988.

\bibitem{MontgomeryVaughan}
H.~L.~Montgomery and R.~C.~Vaughan,
\emph{Multiplicative Number Theory~I: Classical Theory},
Cambridge Studies in Advanced Mathematics, vol.~97,
Cambridge University Press, 2007.

\bibitem{RansfordPT}
T.~Ransford,
\emph{Potential Theory in the Complex Plane},
London Mathematical Society Student Texts, vol.~28,
Cambridge University Press, 1995.

\bibitem{RosenblumRovnyak}
M.~Rosenblum and J.~Rovnyak,
\emph{Hardy Classes and Operator Theory},
Oxford University Press, 1985.

\bibitem{Selberg}
A.~Selberg,
On the zeros of Riemann's zeta-function,
\emph{Skr.\ Norske Vid.-Akad.\ Oslo~I} (1942), no.~10, 1--59.

\bibitem{SimonTrace}
B.~Simon,
\emph{Trace Ideals and Their Applications},
2nd ed., Mathematical Surveys and Monographs, vol.~120,
American Mathematical Society, 2005.

\bibitem{SteinHA}
E.~M.~Stein,
\emph{Harmonic Analysis: Real-Variable Methods, Orthogonality,
and Oscillatory Integrals},
Princeton University Press, 1993.

\bibitem{Titchmarsh}
E.~C.~Titchmarsh,
\emph{The Theory of the Riemann Zeta-Function},
2nd ed., revised by D. R. Heath-Brown,
Oxford University Press, 1986.

\end{thebibliography}

\end{document}
