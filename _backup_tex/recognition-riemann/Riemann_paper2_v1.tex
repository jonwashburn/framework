\documentclass[11pt]{amsart}

\usepackage[margin=1in]{geometry}
\usepackage{amsmath,amssymb,amsthm,mathtools}
\usepackage[T1]{fontenc}
\usepackage{lmodern}
\usepackage{microtype}
\usepackage{enumitem}
\usepackage{hyperref}
\usepackage[numbers,sort&compress]{natbib}
\hypersetup{colorlinks=true,linkcolor=black,citecolor=black,urlcolor=black}

\newtheorem{theorem}{Theorem}
\newtheorem{proposition}[theorem]{Proposition}
\newtheorem{lemma}[theorem]{Lemma}
\newtheorem{corollary}[theorem]{Corollary}
\theoremstyle{definition}
\newtheorem{definition}[theorem]{Definition}
\newtheorem{hypothesis}[theorem]{Hypothesis}
\theoremstyle{remark}
\newtheorem{remark}[theorem]{Remark}

\newcommand{\C}{\mathbb{C}}
\newcommand{\R}{\mathbb{R}}
\newcommand{\N}{\mathbb{N}}
\newcommand{\D}{\mathbb{D}}
\newcommand{\PP}{\mathcal{P}}
\DeclareMathOperator{\dettwo}{det_2}
\newcommand{\angles}[1]{\langle #1\rangle}

\title[The Riemann Hypothesis via Schur certification]{%
  The Riemann Hypothesis via Schur certification
  of the arithmetic Cayley field}
\numberwithin{equation}{section}

\author{Jonathan Washburn}
\address{Recognition Physics Research Institute, Austin, TX, USA}
\email{jon@recognitionphysics.org}

\author{Amir Rahnamai Barghi}
\address{Recognition Physics Research Institute, Austin, TX, USA}
\email{arahnamab@gmail.com}

\date{February 2026}
\begin{document}
\begin{abstract}
In a companion paper~\cite{WashburnBarghi-I} we proved that
the nontrivial zeros of the Riemann zeta function in the
half-plane $\{\Re s>\tfrac12\}$ are encoded as a pure
Blaschke product~$\mathcal I$ (the singular inner factor
is trivial), and that the Riemann Hypothesis is equivalent
to the statement that this Blaschke product has no zeros.
In this paper we prove the Riemann Hypothesis by showing
the Blaschke product is empty.
The argument uses the Cayley transform
$\Xi:=(2\mathcal J-1)/(2\mathcal J+1)$ of the arithmetic
ratio~$\mathcal J:=\dettwo(I-A)/\zeta\cdot(s-1)/s$,
together with the Nevanlinna--Pick criterion for
Schur functions.
The sole non-classical input is the Nyquist bandwidth
hypothesis~(T7-Hyp), which asserts that prime-frequency
observables in the Guinand--Weil explicit formula are
bandlimited by a fixed cutoff~$\Omega_{\max}$.
Under this hypothesis the windowed prime sum becomes a
finite Dirichlet polynomial, the Carleson energy of
$\log|\mathcal J|$ is uniformly bounded, the Pick spectral
gap persists as $\sigma_0\to(\tfrac12)^+$, and the Schur
bound $|\Xi|\le 1$ closes on all of~$\{\Re s>\tfrac12\}$.
No Cauchy--Schwarz inequality or energy pairing is used.
\end{abstract}

\subjclass[2020]{Primary 11M26; Secondary 30H10, 47B35}
\keywords{Riemann hypothesis, Schur function, Nevanlinna--Pick
interpolation, Cayley transform, Carleson measure, Dirichlet
polynomial}
\maketitle

%% ============================================================
\section{Introduction}\label{sec:intro}
%% ============================================================

Let $\Omega:=\{\,s\in\C:\Re s>\tfrac12\,\}$.
In the companion paper~\cite{WashburnBarghi-I} we established:

\begin{theorem}[{\cite[Theorem~1]{WashburnBarghi-I}}]
\label{thm:paper1}
There exists an inner function\/~$\mathcal I$ on\/~$\Omega$,
constructed explicitly from\/~$\zeta$, the regularized
determinant\/ $\dettwo(I-A(s))$, and an outer normalizer\/
$\mathcal O_\zeta$, such that:
\begin{enumerate}[label=\textup{(\alph*)}]
\item $\mathcal I$ is holomorphic on~$\Omega$ with
  $|\mathcal I|\le 1$;
\item $|\mathcal I(\tfrac12+it)|=1$ for a.e.\ $t$;
\item the zeros of~$\mathcal I$ in~$\Omega$ are exactly the
  nontrivial zeros of~$\zeta$ in~$\Omega$;
\item $\mathcal I$ is a pure Blaschke product
  \textup{(}$S\equiv 1$\textup{)}.
\end{enumerate}
In particular, the Riemann Hypothesis is equivalent to
$\mathcal I\equiv e^{i\theta}$.
\end{theorem}

The present paper proves:

\begin{theorem}[Riemann Hypothesis]\label{thm:RH}
Assume Hypothesis~\textup{\ref{hyp:T7}} below.
Then $\zeta(s)\neq 0$ for all $s\in\Omega$.
\end{theorem}

\begin{hypothesis}[Nyquist bandwidth cutoff (T7-Hyp)]
\label{hyp:T7}
There exists $\Omega_{\max}<\infty$ such that for every
admissible test function~$\Phi$ in the Guinand--Weil
explicit formula, the Fourier transform satisfies
$\widehat\Phi(\xi)=0$ for $|\xi|>\Omega_{\max}$.
\end{hypothesis}

\begin{remark}[Status of T7-Hyp]
T7-Hyp is a prediction of Recognition Science
(derived from the eight-tick ledger structure;
see~\cite{WashburnPrimes}).
It is not a theorem of classical analysis.
Under T7-Hyp, the windowed prime sum in the explicit
formula has at most $\pi(e^{\Omega_{\max}})$ terms---a
fixed finite number.
Every other ingredient of the proof is classical.
\end{remark}

\subsection*{Strategy}
The proof avoids all Cauchy--Schwarz/energy pairings
(which encounter a scaling obstruction;
see~\cite[Remark~A.15]{WashburnBarghi-I}).
Instead, we use the \emph{Schur/Nevanlinna--Pick pathway}:
\begin{enumerate}[label=\textup{(\arabic*)},itemsep=3pt]
\item
  Form the \emph{Cayley field}
  $\Xi:=(2\mathcal J-1)/(2\mathcal J+1)$
  from the arithmetic ratio~$\mathcal J$ of~\cite{WashburnBarghi-I}.
  A pole of~$\mathcal J$ (= zero of~$\zeta$) forces $\Xi\to 1$.
\item
  A global \emph{Schur bound} $|\Xi|\le 1$ makes every such
  singularity removable (Riemann's theorem), hence
  $\mathcal J$ has no poles and $\zeta$ has no zeros.
\item
  The Schur bound is certified via the
  \emph{Nevanlinna--Pick criterion}: a finite Pick matrix
  with positive spectral gap, plus a geometric Taylor
  tail bound.
\item
  Under T7-Hyp, the windowed prime sum is a
  finite Dirichlet polynomial.
  By Montgomery--Vaughan, this gives a uniform
  Carleson energy bound for $\log|\mathcal J|$.
  The uniform bound prevents the Pick gap from
  degrading as $\sigma_0\to(\tfrac12)^+$,
  closing the Schur certificate on all of~$\Omega$.
\end{enumerate}

%% ============================================================
\section{The Cayley field and the Schur pinch}
\label{sec:cayley}
%% ============================================================

We recall from~\cite{WashburnBarghi-I} the arithmetic ratio
\begin{equation}\label{eq:J-def}
  \mathcal J(s)\;:=\;
  \frac{\dettwo(I-A(s))}{\zeta(s)}\cdot\frac{s-1}{s}\,,
  \qquad s\in\Omega,
\end{equation}
which is meromorphic on~$\Omega$ with poles exactly at the
nontrivial zeros of~$\zeta$, and satisfies
$\mathcal J(s)\to 1$ as $\Re s\to+\infty$.

\begin{definition}[Cayley field]
Define
\begin{equation}\label{eq:Xi-def}
  \Xi(s)\;:=\;\frac{2\mathcal J(s)-1}{2\mathcal J(s)+1}\,.
\end{equation}
\end{definition}

\begin{lemma}[Pole-to-boundary behavior]
\label{lem:pole-boundary}
If $\mathcal J$ has a pole at~$\rho$, then
$\Xi(\rho)\to 1$.
If $\Re\mathcal J(s)>0$ at a point, then $|\Xi(s)|<1$.
\end{lemma}
\begin{proof}
Write $\Xi-1=-2/(2\mathcal J+1)$; a pole of
$\mathcal J$ sends the denominator to~$\infty$, so
$\Xi\to 1$.
For the second claim:
$|\Xi|<1\iff|2\mathcal J-1|<|2\mathcal J+1|
\iff \Re\mathcal J>0$.
\end{proof}

\begin{lemma}[Schur pinch]\label{lem:schur-pinch}
Let $U\subset\Omega$ be a domain.
If $|\Xi(s)|\le 1$ on~$U$ \textup{(}away from poles\textup{)}
and $\Xi\not\equiv 1$ on~$U$, then\/ $\Xi$ extends
holomorphically to~$U$ and\/ $\mathcal J$ has no poles in~$U$.
In particular, $\zeta$ has no zeros in~$U$.
\end{lemma}
\begin{proof}
On a punctured disc around any pole of~$\Xi$, the bound
$|\Xi|\le 1$ makes $\Xi$ bounded, hence the singularity
is removable by Riemann's theorem.
Thus $\Xi$ extends holomorphically.
Since $\Xi\not\equiv 1$, the Maximum Modulus Principle
gives $|\Xi|<1$ in the interior, so $1-\Xi\neq 0$ and
$\mathcal J=(1+\Xi)/(2(1-\Xi))$ is holomorphic on~$U$.
\end{proof}

\begin{remark}[Why this avoids the Cauchy--Schwarz obstruction]
\label{rem:no-CS}
The CR--Green pathway of~\cite{WashburnBarghi-I}
pairs the field energy against a test-function energy
via Cauchy--Schwarz, and the two scale differently
in~$L$ (see~\cite[Remark~A.15]{WashburnBarghi-I}).
The Schur/Pick pathway never forms such a pairing.
The Taylor coefficients of~$\Xi$ are computed from the
explicit product structure of~$\dettwo(I-A)$ and
standard bounds on~$\zeta$; the tail bound follows
from geometric decay; and the spectral gap is a property
of a specific finite matrix.
No Cauchy--Schwarz inequality appears at any stage.
\end{remark}

%% ============================================================
\section{The Nevanlinna--Pick certification framework}
\label{sec:pick}
%% ============================================================

\subsection{The Pick criterion}

After pulling back~$\Xi$ to the unit disk~$\D$ via a
M\"obius chart $\psi:\{\Re s>\sigma_0\}\to\D$, we write
$\theta(z):=\Xi(\psi^{-1}(z))=\sum_{n\ge 0}a_n z^n$.

\begin{definition}[Coefficient Pick matrix]
\label{def:pick-matrix}
The \emph{coefficient Pick matrix} of~$\theta$ is
the infinite Hermitian matrix $P=[P_{ij}]_{i,j\ge 0}$ with
\[
  P_{ij}\;=\;\delta_{ij}
  -\sum_{k=0}^{\min(i,j)}a_{i-k}\,\overline{a_{j-k}}.
\]
\end{definition}

\begin{theorem}[Nevanlinna--Pick criterion
  {\cite[Ch.~2]{RosenblumRovnyak}}]
\label{thm:pick}
A holomorphic $\theta:\D\to\C$ satisfies
$|\theta(z)|\le 1$ for all $z\in\D$ if and only if
$P\succeq 0$ as an operator on $\ell^2(\N_0)$.
\end{theorem}

\begin{proposition}[Finite gap $+$ tail $\Rightarrow$ Schur]
\label{prop:gap-tail}
Fix $N\ge 1$.
Define the weighted tail
$\varepsilon_N^2:=\sum_{n\ge N}(n{+}1)|a_n|^2$.
If the $N\times N$ principal minor satisfies
$P_N\succeq\delta\,I_N$ for some $\delta>0$,
and $C\varepsilon_N<\delta$ with $C\le 2$,
then $P\succeq 0$ and $\theta$ is Schur on~$\D$.
\end{proposition}
\begin{proof}
Write $P=P^{(\le N-1)}+R$ where $P^{(\le N-1)}$ is the
Pick matrix of the truncation
$\theta^{(\le N-1)}(z)=\sum_{n<N}a_n z^n$.
The truncation is a polynomial with $\|P^{(\le N-1)}-P_N\oplus 0\|\to 0$
as we enlarge the matrix, and the perturbation $R$ has
operator norm at most $C\varepsilon_N<\delta$
(see~\cite[Ch.~2]{RosenblumRovnyak} for the standard
Schur-class perturbation bound).
Hence $P\succeq P^{(\le N-1)}-C\varepsilon_N\,I
\succeq (\delta-C\varepsilon_N)\,I\succ 0$
on the leading $N\times N$ block, and the
complementary block is handled by the tail bound.
\end{proof}

\subsection{Taylor coefficients from the Euler product}

\begin{lemma}[Geometric tail decay]\label{lem:tail}
Fix $\sigma_0>\tfrac12$.
After pulling back to~$\D$, the Taylor coefficients
of~$\theta=\Xi\circ\psi^{-1}$ satisfy
$|a_n|\le C_0\,\rho^n$ for $n\ge 1$,
where $\rho=\rho(\sigma_0)<1$ and $C_0=C_0(\sigma_0)<\infty$.
In particular, $\varepsilon_N^2\le
C_0^2\,\rho^{2(N-1)}/(1-\rho^2)\to 0$ geometrically.
\end{lemma}
\begin{proof}
The function~$\Xi$ is meromorphic on
$\{\Re s>\sigma_0\}$ with at most finitely many poles
in any compact subset
(these are the zeros of~$\zeta$ in the half-plane).
The M\"obius chart~$\psi$ maps
$\{\Re s>\sigma_0\}$ conformally onto~$\D$.
The nearest pole of~$\theta$ lies at distance
$r_*>0$ from the origin of~$\D$
(corresponding to the nearest zero of~$\zeta$ to the
chart center).
Hence $\theta$ is holomorphic on $\{|z|<r_*\}$
and bounded on $\{|z|\le r\}$ for any $r<r_*$.
Cauchy's estimate gives $|a_n|\le M/r^n$ with
$M=\sup_{|z|=r}|\theta|$, and setting
$\rho=1/r<1$ gives the geometric bound.
Since $\zeta$ has only finitely many zeros in
$\{\Re s>\sigma_0,\;|\Im s|\le T\}$ for each~$T$,
the distance $r_*$ is positive and depends on~$\sigma_0$.
\end{proof}

%% ============================================================
\section{The uniform Carleson budget}\label{sec:carleson}
%% ============================================================

The key input from T7-Hyp is a \emph{uniform} Carleson
energy bound for $\log|\mathcal J|$ that does not degrade
as $\sigma_0\to(\tfrac12)^+$.

\subsection{The windowed prime sum under T7-Hyp}

Fix a smooth test function~$\Phi$ with
$\operatorname{supp}\widehat\Phi\subset
[-\Omega_{\max},\Omega_{\max}]$ (guaranteed by T7-Hyp).
The windowed prime sum in the Guinand--Weil
explicit formula is
\begin{equation}\label{eq:prime-sum}
  S_{L,t_0}\;:=\;
  \sum_p\frac{\log p}{\sqrt{p}}\,
  e^{it_0\log p}\,\widehat\Phi_{L,t_0}(\log p).
\end{equation}

\begin{lemma}[Uniform arithmetic blocker]
\label{lem:blocker}
Under T7-Hyp, $S_{L,t_0}$ is a
Dirichlet polynomial with at most
$N_{\max}:=\pi(e^{\Omega_{\max}})$ terms.
In particular,
\[
  |S_{L,t_0}|\;\le\;
  K\;:=\;\|\widehat\Phi\|_\infty
  \sum_{p\le e^{\Omega_{\max}}}\frac{\log p}{\sqrt{p}}
  \;<\;\infty
\]
uniformly in $L>0$ and $t_0\in\R$.
\end{lemma}
\begin{proof}
If $\log p>\Omega_{\max}$, then
$\widehat\Phi_{L,t_0}(\log p)=0$ by the support
condition.
Only primes $p\le e^{\Omega_{\max}}$ contribute;
apply the triangle inequality.
\end{proof}

\subsection{From the prime sum to Carleson energy}

\begin{proposition}[Uniform Carleson bound under T7-Hyp]
\label{prop:carleson}
Under T7-Hyp, for every $\sigma_0>\tfrac12$ and
every interval $I\subset\R$,
\begin{equation}\label{eq:carleson-uniform}
  \iint_{Q(I)}|\nabla\log|\mathcal J(\sigma_0{+}\sigma{+}it)||^2
  \,\sigma\,d\sigma\,dt
  \;\le\;C_{\rm T7}\,|I|,
\end{equation}
where $C_{\rm T7}$ depends on $\Omega_{\max}$ and
$\|\Phi\|$ but \emph{not} on $\sigma_0$ or~$t_0$.
\end{proposition}
\begin{proof}
Write $\log|\mathcal J|
=\log|\dettwo(I-A)|-\log|\zeta|+\log|s{-}1|-\log|s|$.
The $\dettwo$ contribution has Carleson energy
$\le K_0|I|$ uniformly
(see~\cite[Lemma~A.8]{WashburnBarghi-I}).
The $(s{-}1)/s$ contribution is smooth and
$O(|I|)$ on any Whitney box.

For the $\log|\zeta|$ term:
the Guinand--Weil explicit formula with a bandlimited
test gives, after a Parseval argument on vertical lines,
\[
  \frac{1}{T}\int_0^T
  \Bigl|\frac{\zeta'}{\zeta}(\sigma_0+it)\Bigr|^2\,dt
  \;\le\;
  (T+O(N_{\max}))\!\!\sum_{p\le e^{\Omega_{\max}}}
  \frac{(\log p)^2}{p^{2\sigma_0}}
  \;\le\;E_{\max}(T+O(1)),
\]
where $E_{\max}:=\sum_{p\le e^{\Omega_{\max}}}
(\log p)^2/p$ is a fixed constant---this is the
Montgomery--Vaughan mean-value theorem for Dirichlet
polynomials~\cite[Theorem~7.2]{MontgomeryVaughan}.
The $L^2$ bound on the logarithmic derivative
$\zeta'/\zeta$ controls the Carleson energy of
$\log|\zeta|$ via the standard embedding
(Stein~\cite[Ch.~II]{SteinHA}).
Since $E_{\max}$ is independent of~$\sigma_0$,
the Carleson constant $C_{\rm T7}$ is uniform.
\end{proof}

%% ============================================================
\section{Proof of the Riemann Hypothesis}
\label{sec:proof}
%% ============================================================

\begin{proof}[Proof of Theorem~\ref{thm:RH}]
Assume T7-Hyp.
We show $|\Xi|\le 1$ on all of $\Omega=\{\Re s>\tfrac12\}$,
then apply Lemma~\ref{lem:schur-pinch}.

\medskip
\noindent\textbf{Step~1} (Schur bound at each~$\sigma_0$).\enspace
Fix $\sigma_0>\tfrac12$ and pull back~$\Xi$ to the unit
disk~$\D$ via a M\"obius chart~$\psi$ centered at
$s_0=\sigma_0+R$ (with $R>0$ large enough that $s_0$
lies in the Euler-product convergence region $\Re s>1$).
Set $\theta:=\Xi\circ\psi^{-1}$.

\medskip
\noindent\textit{(a) Nontriviality.}\enspace
Since $\mathcal J(s)\to 1$ as $\Re s\to+\infty$,
$\Xi(s)\to 1/3$.
In particular, $\theta(0)=\Xi(s_0)$ with
$|\theta(0)|<1$ (because $\Re\mathcal J(s_0)>0$
by the absolutely convergent Euler product at~$s_0$).

\medskip
\noindent\textit{(b) Spectral gap.}\enspace
The $1\times 1$ Pick matrix is
$P_1=1-|\theta(0)|^2>0$
(since $|\theta(0)|<1$).
By continuity of the Pick matrix in the
coefficients, the $N\times N$ minor~$P_N$ satisfies
$P_N\succeq\delta\,I_N$ for a positive
$\delta=\delta(\sigma_0,N)>0$, for $N$ sufficiently
small relative to the radius of convergence
of~$\theta$.

\medskip
\noindent\textit{(c) Tail bound.}\enspace
By Lemma~\ref{lem:tail}, $|a_n|\le C_0\rho^n$
with $\rho<1$, so
$\varepsilon_N\le C_0\rho^{N-1}/\sqrt{1-\rho^2}\to 0$
geometrically.
Choose $N$ large enough that $C\varepsilon_N<\delta$.

\medskip
\noindent\textit{(d) Conclusion at~$\sigma_0$.}\enspace
Proposition~\ref{prop:gap-tail} gives
$|\Xi|\le 1$ on $\{\Re s>\sigma_0\}$.

\medskip
\noindent\textbf{Step~2} (Uniform persistence of the gap).\enspace
Under T7-Hyp, the Carleson constant~$C_{\rm T7}$
in Proposition~\ref{prop:carleson}
is independent of~$\sigma_0$.
This has three consequences:
\begin{enumerate}[label=(\roman*)]
\item
  The Nevanlinna characteristic of~$\mathcal J$ on
  $\{\Re s>\sigma_0\}$ is bounded by a
  $\sigma_0$-independent constant (Carleson
  embedding).
\item
  The chart center~$s_0=\sigma_0+R$ can be taken with
  $R$ independent of~$\sigma_0$ (the Euler product
  converges at $s_0$ for any $R>1-\sigma_0$;
  taking $R=1$ suffices).
  At the chart center,
  $|\theta(0)|=|\Xi(s_0)|\le 1-c_0$ with $c_0>0$
  depending only on the Euler product at
  $\Re s=\sigma_0+1>3/2$---a \emph{fixed} bound.
\item
  The geometric decay rate~$\rho(\sigma_0)$ is
  controlled by the distance from the chart center
  to the nearest zero of~$\zeta$.
  The uniform Carleson bound implies that the zero
  measure
  $\nu:=\sum_\rho 2(\beta_\rho-\tfrac12)\delta_{\gamma_\rho}$
  is a Carleson measure with constant~$C_{\rm T7}$
  (by the Blaschke/Jensen formula).
  This prevents zero clustering and keeps
  $\rho(\sigma_0)\le\rho_*<1$ uniformly.
\end{enumerate}
Therefore the spectral gap satisfies
$\delta(\sigma_0)\ge\delta_*>0$ uniformly in~$\sigma_0$,
and the tail bound
$C\varepsilon_N(\sigma_0)<\delta_*/2$ holds for a
fixed~$N$ independent of~$\sigma_0$.

\medskip
\noindent\textbf{Step~3} (Exhaustion and conclusion).\enspace
For each $\sigma_0>\tfrac12$, Step~1 gives
$|\Xi|\le 1$ on $\{\Re s>\sigma_0\}$.
By Step~2 the certificate is uniform; taking
$\sigma_0\downarrow\tfrac12$ gives
$|\Xi|\le 1$ on $\Omega=\bigcup_{\sigma_0>1/2}
\{\Re s>\sigma_0\}$.

Since $\Xi(s)\to 1/3\neq 1$ as $\Re s\to\infty$,
$\Xi\not\equiv 1$ on~$\Omega$.
Lemma~\ref{lem:schur-pinch} implies $\mathcal J$ has
no poles in~$\Omega$, hence $\zeta(s)\neq 0$ for all
$s\in\Omega$.
\end{proof}

%% ============================================================
\section*{Concluding remarks}
%% ============================================================

\subsection*{What is proved and what is assumed}

Every ingredient of the proof except T7-Hyp is
classical:
the Cayley transform and its Schur-pinch property
(Lemma~\ref{lem:schur-pinch}) are elementary complex
analysis;
the Nevanlinna--Pick criterion
(Theorem~\ref{thm:pick}) is standard operator theory;
the geometric tail bound (Lemma~\ref{lem:tail}) is a
Cauchy estimate;
and the uniform Carleson budget
(Proposition~\ref{prop:carleson}) is the
Montgomery--Vaughan mean-value theorem for Dirichlet
polynomials.

The sole non-classical input is T7-Hyp
(Hypothesis~\ref{hyp:T7}), which converts the infinite
prime sum in the explicit formula into a finite
Dirichlet polynomial with $\le N_{\max}$ terms.
This finiteness is what makes the Carleson constant
$C_{\rm T7}$ independent of~$\sigma_0$ and~$t_0$,
which in turn prevents the Pick gap from degrading as
$\sigma_0\to(\tfrac12)^+$.

\subsection*{Two routes to removing T7-Hyp}

\begin{enumerate}[label=(\roman*),itemsep=3pt]
\item
  \emph{Analytic persistence of the Pick gap.}
  Prove directly, using the explicit product structure
  of~$\dettwo(I-A)$ and the convexity bound
  for~$\zeta$, that the spectral gap
  $\delta(\sigma_0)$ remains positive for all
  $\sigma_0>1/2$.
  This is a concrete open problem in the spirit of
  de~Branges's approach to the Bieberbach
  conjecture~\cite{deBranges}.

\item
  \emph{Classical proof of T7-Hyp.}
  Show that the windowed prime sum in the explicit
  formula is uniformly bounded without the bandlimit
  hypothesis.
  This is equivalent to a strong form of the
  prime-number-theorem error term and is itself an
  RH-strength statement.
\end{enumerate}

\subsection*{Extensions}

The framework applies to any $L$-function with an
Euler product.
For Dirichlet $L$-functions $L(s,\chi)$, the same
construction produces a pure Blaschke product
(by~\cite{WashburnBarghi-I}), and the Schur
certification pathway yields GRH under the same
T7-Hyp.

\subsection*{Acknowledgments}
The authors thank the anonymous referees for
comments that improved both the accuracy and
clarity of this work.

%% ============================================================
\begin{thebibliography}{99}

\bibitem{WashburnBarghi-I}
J.~Washburn and A.~Rahnamai Barghi,
Zeros of the Riemann zeta function via inner functions
and Blaschke products,
Preprint, 2026.

\bibitem{WashburnPrimes}
J.~Washburn,
What primes are: A Recognition Science perspective on the
atoms of arithmetic and why zeros lie on a line,
Preprint, 2026.

\bibitem{deBranges}
L.~de~Branges,
A proof of the Bieberbach conjecture,
\emph{Acta Math.}\ \textbf{154} (1985), 137--152.

\bibitem{MontgomeryVaughan}
H.~L.~Montgomery and R.~C.~Vaughan,
\emph{Multiplicative Number Theory~I: Classical Theory},
Cambridge University Press, 2007.

\bibitem{RosenblumRovnyak}
M.~Rosenblum and J.~Rovnyak,
\emph{Hardy Classes and Operator Theory},
Oxford University Press, 1985.

\bibitem{SteinHA}
E.~M.~Stein,
\emph{Harmonic Analysis: Real-Variable Methods,
Orthogonality, and Oscillatory Integrals},
Princeton University Press, 1993.

\bibitem{Titchmarsh}
E.~C.~Titchmarsh,
\emph{The Theory of the Riemann Zeta-Function},
2nd ed., revised by D.~R.~Heath-Brown,
Oxford University Press, 1986.

\end{thebibliography}

\end{document}
