\documentclass[11pt,a4paper]{article}

%----------------------------------------------------------------------------------------
%	PACKAGES AND THEMES
%----------------------------------------------------------------------------------------
\usepackage[utf8]{inputenc}
\usepackage[T1]{fontenc}
\usepackage{lmodern}
\usepackage[margin=1in]{geometry}
\usepackage{amsmath, amssymb, amsthm, amsfonts}
\usepackage{mathtools}
%\usepackage{physics}
\usepackage{xcolor}
\usepackage{hyperref}
\usepackage{microtype}
\usepackage{booktabs}
\usepackage{enumitem}
\usepackage{fancyhdr}
%\usepackage{titlesec}

%----------------------------------------------------------------------------------------
%	MATH DEFINITIONS
%----------------------------------------------------------------------------------------
\newcommand{\C}{\mathbb{C}}
\newcommand{\R}{\mathbb{R}}
\newcommand{\Z}{\mathbb{Z}}
\newcommand{\N}{\mathbb{N}}
\newcommand{\Q}{\mathbb{Q}}
\newcommand{\Hil}{\mathcal{H}}
\newcommand{\Lag}{\mathcal{L}}
\newcommand{\omegaEight}{\omega_8}
\newcommand{\dft}{\mathcal{F}}
\newcommand{\shift}{\mathcal{S}}
\newcommand{\neutral}{\mathcal{N}}
\newcommand{\cost}{\mathcal{J}}
\newcommand{\defect}{\mathcal{D}}

\newtheorem{theorem}{Theorem}[section]
\newtheorem{lemma}[theorem]{Lemma}
\newtheorem{proposition}[theorem]{Proposition}
\newtheorem{corollary}[theorem]{Corollary}
\newtheorem{definition}[theorem]{Definition}
\newtheorem{axiom}[theorem]{Axiom}
\newtheorem{conjecture}[theorem]{Conjecture}

%----------------------------------------------------------------------------------------
%	TITLE SECTION
%----------------------------------------------------------------------------------------
\title{\textbf{Universal Light Language:}\\
\large \textit{Meaning as Eigenmodes of the Eight-Tick Phase Field}}

\author{
    \textbf{Recognition Science Research Institute} \\
    \textit{Austin, Texas}
}

\date{\today}

%----------------------------------------------------------------------------------------
%	HEADER/FOOTER
%----------------------------------------------------------------------------------------
\pagestyle{fancy}
\fancyhf{}
\fancyhead[L]{Universal Light Language}
\fancyhead[R]{\thepage}
\renewcommand{\headrulewidth}{0.5pt}

\begin{document}

\maketitle

%----------------------------------------------------------------------------------------
%	ABSTRACT
%----------------------------------------------------------------------------------------
\begin{abstract}
\noindent
We present Universal Light Language (ULL), a rigorous mathematical framework in which semantic meaning is identified with geometric structure on an eight-sample complex phase register. We prove that the fundamental components of this theory---the eight-tick period, the Discrete Fourier Transform (DFT-8) eigenmode basis, the neutral (mean-free) semantic subspace, and the 20 irreducible semantic primitives (WTokens)---are not arbitrary choices but are \textit{forced} by a unique cost functional satisfying the d'Alembert functional equation. In this framework, meaning is formalized as a chord: a unit-norm superposition in the neutral subspace $\C^7$. Semantic equivalence is defined as chord isometry, and communication is modeled as resonance in a shared cost landscape. We establish information-theoretic capacity bounds showing that the 20-token overcomplete dictionary provides optimal noise immunity for the 14 real degrees of freedom in the semantic subspace. The theory is parameter-free: all structure derives from a single compositional constraint with no external inputs. We provide complete mathematical derivations for the emergence of the semantic primitives from representation theory and the derivation of the 20-token dictionary via discrete phase offsets (``$\varphi$-levels'') and Nyquist variants; in the current implementation, semantic ``intensity'' is carried by chord coefficient magnitudes rather than by basis-vector scaling.
\end{abstract}

\vspace{1em}
\noindent\textbf{Keywords:} Semantic Primitives, Phase Geometry, Representation Theory, d'Alembert Equation, Universal Language, Zero-Parameter Physics.

\newpage
\tableofcontents
\newpage

%----------------------------------------------------------------------------------------
%	SECTION 1: INTRODUCTION
%----------------------------------------------------------------------------------------
\section{Introduction}

\subsection{The Problem of Semantic Primitives}

The search for the fundamental atoms of meaning---the irreducible primitives from which all complex concepts are constructed---has been one of the central quests of linguistics, cognitive science, and philosophy for centuries. If matter is composed of a finite periodic table of elements, and computation is built from a minimal set of logic gates, does meaning possess a similar finite basis?

Historically, this question has been approached through empirical decomposition. Leibniz proposed an \textit{alphabet of human thought} (\textit{characteristica universalis}), envisioning a system where complex ideas could be calculated from simple ones. In modern linguistics, the Natural Semantic Metalanguage (NSM) project, led by Anna Wierzbicka and colleagues, has spent decades identifying ``semantic primes''---concepts like \textsc{I}, \textsc{YOU}, \textsc{DO}, \textsc{HAPPEN}, \textsc{GOOD}, \textsc{BAD}---that appear to be lexically realized in all human languages. This empirical distillation has yielded a set of approximately 65 primitives \cite{Wierzbicka1996}. Similarly, in cognitive semantics, Jackendoff \cite{Jackendoff1990} proposed a set of conceptual primitives (e.g., \textsc{THING}, \textsc{PLACE}, \textsc{PATH}, \textsc{EVENT}, \textsc{CAUSE}) underlying the syntactic structure of language.

However, these approaches face a fundamental epistemic barrier: the \textit{arbitrariness problem}. Why these primitives and not others? Why is the set of size 65 and not 20 or 100? Why are the primitives defined by these specific conceptual boundaries? In existing theories, the primitives are \textit{fitted} to the data of human language. They are descriptive, not predictive. There is no underlying mathematical principle that \textit{forces} the set of semantic primitives to take exactly one form. As a result, semantic theory has remained a soft science, lacking the rigorous derivation from first principles that characterizes physics or information theory.

\subsection{The ULL Proposal}

We propose a radical departure from the descriptive tradition. We posit that meaning is not fundamentally a linguistic or psychological phenomenon, but a \textit{geometric} one. Specifically, we propose that the substrate of meaning is a complex phase field, and that semantic primitives are the eigenmodes of this field's natural symmetries.

This framework, which we call Universal Light Language (ULL), identifies the minimal carrier of meaning as an eight-sample complex register, $\psi \in \C^8$, which we term a \textit{voxel chord}. In this view, a ``concept'' is not a symbol in a database but a standing wave---a chord---on this phase register. The irreducible primitives of meaning are not chosen by intuition but are mathematically forced: they are the eigenvectors of the cyclic shift operator acting on this register.

The core claim of this paper is that the structure of meaning is isomorphic to the structure of light (electromagnetism) in a discrete, self-recognizing spacetime. By imposing a single compositional constraint---the Recognition Composition Law---we derive a unique cost functional that forces the carrier to have an 8-tick period, forces the semantic subspace to be mean-free (neutral), and forces the existence of exactly 20 canonical semantic primitives (WTokens).

\subsection{Paper Contributions}

This paper provides a rigorous mathematical derivation of the ULL framework. Our specific contributions are:

\begin{enumerate}
    \item \textbf{Derivation of the 8-Tick Period:} We prove that in a 3-dimensional discrete space requiring non-trivial linking of closed loops, the minimal ledger-compatible cycle period is necessarily $2^3 = 8$.
    \item \textbf{Uniqueness of the Basis:} We prove that the Discrete Fourier Transform on 8 points (DFT-8) is the unique unitary basis (up to phase and permutation) that diagonalizes the time-translation (cyclic shift) operator, thereby identifying the canonical ``notes'' of the semantic system.
    \item \textbf{Construction of the 20 WTokens:} We construct the 20 irreducible semantic primitives from representation theory, combining the DFT-8 eigenmodes with a phase-quantization scheme based on the golden ratio $\varphi$. We show that this set of 20 is not arbitrary but exhaustive under the structural constraints of the theory.
    \item \textbf{Information Capacity Analysis:} We analyze the capacity of the neutral semantic subspace ($\C^7$, or 14 real degrees of freedom) and show that the 20-token overcomplete dictionary provides an optimal frame for robust encoding of meaning in the presence of noise.
    \item \textbf{Semantic Operations:} We define the algebra of meaning---composition (superposition), projection (measurement), and equivalence (isometry)---strictly in terms of vector space operations on $\C^8$.
\end{enumerate}

\subsection{Relation to Prior Work}

ULL differs fundamentally from current computational approaches to semantics. Distributional semantics (e.g., word2vec, BERT, LLM embeddings) represents meaning as dense vectors in high-dimensional spaces ($\R^{1024}$ or similar). While effective, these spaces are \textit{learned}, not derived. The dimensions have no intrinsic meaning, and the geometry is determined by the statistics of a training corpus. ULL, by contrast, operates in a low-dimensional space ($\C^8$) where every dimension and every basis vector has a precise, derived definition.

Similarly, ULL is distinct from formal or type-theoretic semantics (e.g., Montague grammar, Combinatory Categorial Grammar). These systems treat meaning as logical operations on symbols. ULL treats meaning as \textit{phase geometry}. The ``logic'' of ULL is the logic of constructive interference and resonance, governed by a cost functional rather than truth tables.

The key difference is that ULL is a \textit{zero-parameter theory}. We do not fit the model to English or any other language. We derive the structure of the ``meaning container'' from first principles of recognition and symmetry. We hypothesize that human languages are imperfect, evolved approximations of this underlying universal structure.

\newpage

%----------------------------------------------------------------------------------------
%	SECTION 2: AXIOMATIC FOUNDATIONS
%----------------------------------------------------------------------------------------
\section{Axiomatic Foundations}

The foundation of ULL is not linguistic but physical. We begin by defining the fundamental cost functional that governs the stability of configurations in the system. This cost functional is not chosen arbitrarily; it is the unique solution to a compositional constraint that we term the Recognition Composition Law.

\subsection{The Primitive: Recognition Composition Law (RCL)}

We posit that the fundamental operation of the system is \textit{recognition}, and that recognition events carry a cost. We seek a cost functional $\cost: \R_+ \to \R_{\ge 0}$ that assigns a non-negative cost to any ratio $x$ (representing the multiplicative comparison of two magnitudes).

\begin{definition}[Cost Functional]
A cost functional $\cost: \R_+ \to \R_{\ge 0}$ is a smooth function satisfying the following three axioms:
\begin{itemize}
    \item \textbf{A1 (Normalization):} The cost of identity is zero.
    $$ \cost(1) = 0 $$
    \item \textbf{A2 (Composition):} The cost satisfies the Recognition Composition Law (RCL):
    $$ \cost(xy) + \cost(x/y) = 2\cost(x)\cost(y) + 2\cost(x) + 2\cost(y) $$
    \item \textbf{A3 (Calibration):} The curvature of the cost at the minimum is unity in logarithmic coordinates.
    $$ \left. \frac{d^2}{d(\ln x)^2} \cost(x) \right|_{x=1} = 1 $$
\end{itemize}
\end{definition}

The Composition Law (A2) is the central constraint. It asserts that the cost of a product and a quotient relates to the costs of the components in a specific, symmetric way. This law is a calibrated form of the multiplicative d'Alembert functional equation.

\begin{theorem}[T5: Cost Uniqueness]
Given axioms A1--A3, there exists a unique solution for $\cost(x)$ on $\R_+$, given by:
$$ \cost(x) = \frac{1}{2}\left(x + x^{-1}\right) - 1 = \frac{(x-1)^2}{2x} $$
\end{theorem}

\begin{proof}
Let $f(t) = \cost(e^t) + 1$. Substituting $x=e^u, y=e^v$ into A2, we get:
$$ (\cost(e^{u+v}) + 1) + (\cost(e^{u-v}) + 1) - 2 = 2\cost(e^u)\cost(e^v) + 2\cost(e^u) + 2\cost(e^v) $$
$$ f(u+v) + f(u-v) - 2 = 2(\cost(e^u) + 1)(\cost(e^v) + 1) - 2 $$
$$ f(u+v) + f(u-v) = 2f(u)f(v) $$
This is the classical d'Alembert functional equation (cosine type). The continuous solutions are $f(t) = \cosh(kt)$ or $f(t) = \cos(kt)$ or $f(t) = 1$.
Since $\cost(x) \ge 0$, we must have $f(t) \ge 1$, which excludes $\cos(kt)$. $f(t)=1$ implies $\cost(x)=0$ everywhere, which violates A3. Thus $f(t) = \cosh(kt)$.
Axiom A3 fixes the scaling constant $k$.
$$ \cost(x) = f(\ln x) - 1 = \cosh(k \ln x) - 1 = \frac{x^k + x^{-k}}{2} - 1 $$
In log coordinates $t = \ln x$, $\cost(e^t) \approx \frac{1}{2}(kt)^2$. The second derivative is $k^2$. A3 requires $k^2=1$, so $k=1$ (since $x$ and $x^{-1}$ are symmetric, sign doesn't matter).
Thus, $\cost(x) = \frac{1}{2}(x + x^{-1}) - 1$.
\end{proof}

This result is proven formally in the Lean module \texttt{IndisputableMonolith.CostUniqueness.T5\_uniqueness\_complete}.

\subsection{Derived Constraints}

From the unique form of $\cost(x)$, several critical properties follow immediately.

\begin{corollary}[Reciprocity]
For all $x > 0$, $\cost(x) = \cost(x^{-1})$.
\end{corollary}
\begin{proof}
Direct substitution: $\frac{1}{2}(x + x^{-1}) - 1 = \frac{1}{2}(x^{-1} + (x^{-1})^{-1}) - 1 = \frac{1}{2}(x^{-1} + x) - 1$.
\end{proof}
This symmetry forces the system to treat expansion ($x > 1$) and contraction ($x < 1$) symmetrically, laying the groundwork for the double-entry ledger structure (conservation of flux).

\begin{corollary}[Strict Convexity]
$\cost(x)$ is strictly convex on $\R_+$, with a unique global minimum at $x=1$.
\end{corollary}
\begin{proof}
$\cost'(x) = \frac{1}{2}(1 - x^{-2})$. $\cost''(x) = x^{-3} > 0$ for all $x > 0$.
\end{proof}

\begin{corollary}[Discreteness Forcing]
Continuous configurations cannot achieve isolated stable minima under $\cost$. Stable states require discrete structure.
\end{corollary}
\begin{proof}
In a continuous manifold, infinitesimal perturbations have infinitesimal cost $\cost(1+\epsilon) \approx \epsilon^2/2$. Without a discrete lattice to impose a minimum step size, the system would drift. Stability requires a non-zero energy barrier between states, forcing the carrier to be discrete (quantized).
\end{proof}

\subsection{The Law of Existence}

We define the ``defect'' of a state as its cost.

\begin{definition}[Defect]
$\defect(x) := \cost(x)$.
\end{definition}

\begin{theorem}[Law of Existence]
An entity $x$ exists (in the sense of being a stable, cost-free configuration) if and only if $\defect(x) = 0$, which holds if and only if $x = 1$.
\end{theorem}

This seemingly tautological statement has profound physical implications when combined with the behavior of $\cost$ at the origin.

\begin{corollary}[Nothing has Infinite Cost]
$$ \lim_{x \to 0^+} \cost(x) = \lim_{x \to 0^+} \left(\frac{1}{2}x^{-1} - 1\right) = \infty $$
\end{corollary}

\newpage

%----------------------------------------------------------------------------------------
%	SECTION 3: THE EIGHT-TICK PERIOD
%----------------------------------------------------------------------------------------
\section{The Eight-Tick Period}

Having established the cost functional and the necessity of a discrete carrier, we now determine the dimensionality and periodicity of that carrier. We show that the minimal ledger-compatible cycle has a period of exactly 8 ticks.

\subsection{Dimensional Forcing}

The carrier must be a discrete lattice that supports the fundamental operations of recognition: separation, connection, and non-trivial interaction (linking).

\begin{axiom}[Linking]
The carrier dimension $D$ must support non-trivial linking of closed curves.
\end{axiom}

This axiom is motivated by the need for stable, topological memory structures (knots) that persist in the absence of energy input.

\begin{theorem}[D = 3]
Only $D=3$ supports non-trivial linking invariants for 1-dimensional threads.
\end{theorem}

\begin{proof}
Consider two disjoint closed curves $\gamma_1, \gamma_2$ in $\R^D$.
\begin{itemize}
    \item \textbf{Case D = 2:} By the Jordan Curve Theorem, any closed curve divides the plane into an interior and an exterior. Two disjoint curves cannot link; they are either nested or separate. The linking number is always 0.
    \item \textbf{Case D $\ge$ 4:} The fundamental group of the complement of a curve is trivial: $\pi_1(\R^D \setminus \gamma) = 0$. Any curve can be continuously deformed to a point without crossing another curve. Thus, all curves are unlinkable.
    \item \textbf{Case D = 3:} The linking number $\text{lk}(\gamma_1, \gamma_2)$ is a topological invariant taking values in $\Z$. Non-trivial knots and links exist (e.g., the Hopf link, the trefoil knot).
\end{itemize}
Thus, $D=3$ is the unique dimension permitting topological complexity for 1D strings.
\end{proof}

\subsection{The Hypercube Walk}

We model the discrete state space as a hypercube $Q_D$. A ``cycle'' in this space corresponds to a path that visits all vertices and returns to the start.

\begin{definition}[Gray Code Cycle]
A Hamiltonian path on the $D$-dimensional hypercube $Q_D$ is a sequence of $2^D$ vertices $v_0, v_1, \dots, v_{2^D-1}$ such that every vertex is visited exactly once, and adjacent vertices $v_i, v_{i+1}$ differ by exactly one bit flip. A Hamiltonian cycle (Gray code) adds the condition that $v_{2^D-1}$ is adjacent to $v_0$.
\end{definition}

In a double-entry ledger system, every state change must be balanced. The flip of a single bit corresponds to a fundamental transaction. To explore the full state space of a voxel (the unit cell of the lattice), the system must traverse all corners.

\begin{theorem}[8-Tick Period]
For $D=3$, the minimal ledger-compatible cycle has period $2^3 = 8$.
\end{theorem}

\begin{proof}
The 3-cube $Q_3$ has $2^3 = 8$ vertices. A minimal traversal that visits every state (corner) exactly once before repeating is a Hamiltonian cycle.
Let the vertices be represented by 3-bit strings $b_2b_1b_0$. A standard Gray code sequence is:
\begin{align*}
    t=0: & \quad 000 \\
    t=1: & \quad 001 \quad (\text{flip } b_0) \\
    t=2: & \quad 011 \quad (\text{flip } b_1) \\
    t=3: & \quad 010 \quad (\text{flip } b_0) \\
    t=4: & \quad 110 \quad (\text{flip } b_2) \\
    t=5: & \quad 111 \quad (\text{flip } b_0) \\
    t=6: & \quad 101 \quad (\text{flip } b_1) \\
    t=7: & \quad 100 \quad (\text{flip } b_0) \\
    t=8: & \quad 000 \quad (\text{flip } b_2, \text{ return to start})
\end{align*}
This cycle has length 8. Any shorter cycle would fail to visit all states. Any longer cycle would involve revisiting states (redundancy). Thus, the fundamental ``clock'' of a 3-dimensional discrete reality is an 8-step cycle.
\end{proof}

This 8-tick cycle defines the fundamental unit of time in ULL, denoted $\tau_0$.

\subsection{The Phase Register}

We can now define the mathematical object that carries meaning. It is not a scalar, but a field defined over this 8-tick cycle.

\begin{definition}[Voxel Chord]
A voxel chord is a complex vector $\psi \in \C^8$, with components $\psi[t]$ for $t \in \{0, 1, \dots, 7\}$.
$$ \psi = \begin{bmatrix} \psi[0] \\ \psi[1] \\ \vdots \\ \psi[7] \end{bmatrix} $$
\end{definition}

Each component $\psi[t]$ represents the complex amplitude (magnitude and phase) of the field at tick $t$ of the cycle.

\newpage

%----------------------------------------------------------------------------------------
%	SECTION 4: THE DFT-8 BACKBONE
%----------------------------------------------------------------------------------------
\section{The DFT-8 Backbone}

The 8-tick cycle imposes a discrete time-translation symmetry on the system. To understand the structure of meaning, we must identify the canonical basis that respects this symmetry. This leads us to the Discrete Fourier Transform on 8 points (DFT-8).

\subsection{Representation Theory of Z/8Z}

The set of time translations forms the cyclic group of order 8, denoted $Z_8$ or $\Z/8\Z$. The action of this group on the voxel chord space $\C^8$ is generated by the cyclic shift operator $\shift$.

Representation theory tells us that any unitary representation of a finite abelian group decomposes into a direct sum of 1-dimensional irreducible representations (irreps). For $Z_N$, the irreps are characters $\chi_k: Z_N \to \C^\times$ given by roots of unity.

\begin{theorem}[Characters of Z/8Z]
The irreducible characters of the cyclic group of order 8 are given by:
$$ \chi_k(t) = \omegaEight^{kt}, \quad k \in \{0, 1, \dots, 7\} $$
where $\omegaEight = e^{-2\pi i / 8} = e^{-\pi i / 4}$ is the primitive 8th root of unity.
\end{theorem}

These characters define the natural ``frequencies'' of the system.

\subsection{The Canonical Basis}

The canonical basis for the space $\C^8$ is the one that diagonalizes the shift operator. This basis consists of the vectors corresponding to the irreducible characters.

\begin{definition}[DFT-8 Mode]
The $k$-th DFT mode vector $\text{mode}_k \in \C^8$ is defined by:
$$ \text{mode}_k[t] = \frac{\omegaEight^{tk}}{\sqrt{8}}, \quad t, k \in \{0, 1, \dots, 7\} $$
The factor $1/\sqrt{8}$ ensures normalization.
\end{definition}

Let $B$ be the $8 \times 8$ matrix whose columns are the mode vectors: $B_{tk} = \text{mode}_k[t]$. This is the DFT-8 matrix.

\begin{theorem}[Shift Diagonalization]
The DFT-8 basis diagonalizes the cyclic shift operator $\shift$. Specifically:
$$ B^\dagger \shift B = \text{diag}(1, \omegaEight, \omegaEight^2, \dots, \omegaEight^7) $$
\end{theorem}

\begin{proof}
Let $v_k = \text{mode}_k$. We compute the action of $\shift$ on $v_k$:
\begin{align*}
(\shift v_k)[t] &= v_k[(t+1) \pmod 8] \\
&= \frac{\omegaEight^{(t+1)k}}{\sqrt{8}} \\
&= \frac{\omegaEight^{tk} \cdot \omegaEight^k}{\sqrt{8}} \\
&= \omegaEight^k \cdot v_k[t]
\end{align*}
Thus, $\shift v_k = \omegaEight^k v_k$. The vector $v_k$ is an eigenvector of $\shift$ with eigenvalue $\lambda_k = \omegaEight^k$. Since $B$ consists of these eigenvectors as columns, $B$ diagonalizes $\shift$.
\end{proof}

This result is formalized in the Lean module \texttt{LightLanguage.Basis.DFT8.dft8\_diagonalizes\_shift}.

\begin{theorem}[DFT-8 Unitarity]
The DFT-8 matrix is unitary: $B^\dagger B = I$.
\end{theorem}

This ensures that the transformation to the frequency domain preserves the geometry (norms and angles) of the chord space. This is proven in \texttt{LightLanguage.Basis.DFT8.dft8\_unitary}.

\subsection{Uniqueness of the Basis}

A crucial question for a fundamental theory is uniqueness. Could we have chosen a different basis?

\newpage

%----------------------------------------------------------------------------------------
%	SECTION 5: THE NEUTRAL SUBSPACE
%----------------------------------------------------------------------------------------
\section{The Neutral Subspace}

The DFT-8 basis decomposes the 8-dimensional chord space $\C^8$ into 8 orthogonal modes. However, not all modes carry semantic content. We must distinguish between the ``carrier wave'' (energy/existence) and the ``signal'' (meaning).

\subsection{Semantic vs. Energetic Content}

The mode $k=0$ corresponds to the eigenvalue $\omegaEight^0 = 1$. This is the constant mode, representing a uniform field across all 8 ticks.

\begin{definition}[DC Component]
The DC component of a chord $\psi$ is the projection onto the zeroth mode:
$$ c_0 = \langle \text{mode}_0, \psi \rangle = \frac{1}{\sqrt{8}} \sum_{t=0}^7 \psi[t] $$
\end{definition}

In physical terms, the squared magnitude $|c_0|^2$ represents the total ``charge'' or ``mass'' of the state. A non-zero DC component implies a net imbalance or bias in the system.

\begin{definition}[Neutrality]
A chord $\psi$ is neutral if and only if its components sum to zero:
$$ \sum_{t=0}^7 \psi[t] = 0 \iff c_0 = 0 $$
\end{definition}

This condition defines the subspace where semantic information resides. Meaning is a modulation of the field that does not change its total charge.

\begin{theorem}[Neutral Subspace Dimension]
The neutral subspace $\neutral = \{ \psi \in \C^8 : \sum \psi[t] = 0 \}$ is the orthogonal complement of the DC mode. It has complex dimension 7.
\end{theorem}

\begin{proof}
The condition $\sum \psi[t] = 0$ is a single linear constraint. Since the constraint vector $(1, 1, \dots, 1)$ is non-zero, the solution space has dimension $8 - 1 = 7$.
The basis vectors for this subspace are the 7 non-zero DFT modes: $\{\text{mode}_1, \text{mode}_2, \dots, \text{mode}_7\}$.
\end{proof}

This result is formalized in \texttt{LightLanguage.Basis.DFT8.dft8\_neutral\_subspace}.

\subsection{Physical Interpretation}

The decomposition of $\C^8$ into the DC line and the neutral hyperplane $\neutral$ corresponds to a fundamental physical distinction.

\begin{proposition}
The DC mode represents the ``ground of existence'' or total energy. The neutral subspace $\neutral$ represents the ``form'' or ``information'' impressed upon that ground.
\end{proposition}

This separation is analogous to the distinction between the carrier wave and the modulation in radio transmission, or between the vacuum expectation value and the excitations in quantum field theory.

\begin{proposition}
Any voxel chord $\psi$ decomposes uniquely as:
$$ \psi = \psi_{\text{DC}} + \psi_{\neutral} $$
where $\psi_{\text{DC}} \in \text{span}(\text{mode}_0)$ and $\psi_{\neutral} \in \neutral$.
\end{proposition}

This decomposition is orthogonal: $||\psi||^2 = ||\psi_{\text{DC}}||^2 + ||\psi_{\neutral}||^2$.

\subsection{The Semantic Manifold}

Meaning is not just any vector in $\neutral$. A concept must have a definite ``intensity'' or existence. We normalize semantic states to lie on the unit sphere.

\newpage

%----------------------------------------------------------------------------------------
%	SECTION 6: THE 20 WTOKENS
%----------------------------------------------------------------------------------------
\section{The 20 WTokens}

We now come to the central constructive result of ULL: the derivation of the 20 irreducible semantic primitives, or WTokens. These are not arbitrary lexical items but the canonical eigenmodes of the semantic manifold $S_\neutral$.

\subsection{Mode Families}

The 7 neutral DFT modes are not all symmetric. Due to the real-valued nature of physical observables (and the structure of the underlying cost functional), modes come in conjugate pairs.

\begin{definition}[Mode Family]
The 7 neutral modes partition into 4 families based on conjugacy:
\begin{itemize}
    \item $F_{1,7} = \{ \text{mode}_1, \text{mode}_7 \}$: The fundamental frequency (period 8). These are complex conjugates.
    \item $F_{2,6} = \{ \text{mode}_2, \text{mode}_6 \}$: The first harmonic (period 4). These are complex conjugates.
    \item $F_{3,5} = \{ \text{mode}_3, \text{mode}_5 \}$: The second harmonic (period 8/3). These are complex conjugates.
    \item $F_4 = \{ \text{mode}_4 \}$: The Nyquist frequency (period 2). This mode is self-conjugate (real-valued up to phase).
\end{itemize}
\end{definition}

\begin{lemma}
For any conjugate pair $\{k, 8-k\}$, the sum $\text{mode}_k + \text{mode}_{8-k}$ is a real-valued vector (up to a global normalization factor).
\end{lemma}
\begin{proof}
$\omegaEight^{tk} + \omegaEight^{t(8-k)} = \omegaEight^{tk} + \omegaEight^{-tk} = 2\cos(2\pi tk/8)$.
\end{proof}

This allows us to construct real-valued basis vectors from the complex DFT modes.

\subsection{The Time-Shift Construction}

The DFT modes provide the ``notes'' (frequencies). To create a complete dictionary, we must also consider ``phase'' or ``timing.'' The golden ratio $\varphi$ enters here as the quantization factor for intensity/phase levels, but in the canonical construction, these levels are realized as discrete time shifts.

\begin{definition}[Phase Offset]
For a $\varphi$-level $p \in \{0, 1, 2, 3\}$, the time-shift operator $T_p$ acts on a chord $\psi$ by:
$$ (T_p \psi)[t] = \psi[(t-p) \pmod 8] $$
\end{definition}

We construct the WTokens by applying these time shifts to the base waveforms of each mode family.

\begin{definition}[WToken Basis Vector]
For a mode family $F$, a $\varphi$-level $p \in \{0, 1, 2, 3\}$, and a $\tau$-offset $\tau \in \{0, 2\}$, the WToken $W_{F,p,\tau}$ is defined as:
$$ W_{F,p,\tau} = \text{normalize}\left( \tau_{\text{factor}} \cdot T_p\left( \sum_{k \in F} \text{mode}_k \right) \right) $$
where $\tau_{\text{factor}} = 1$ if $\tau=0$, and $\tau_{\text{factor}} = i$ if $\tau=2$. The $\tau=2$ option is valid only for the self-conjugate family $F_4$.
\end{definition}

The $\tau=2$ offset for $F_4$ corresponds to a quarter-cycle phase shift (multiplication by $i$), creating an ``imaginary'' counterpart to the real Nyquist mode.

\begin{theorem}[20 Canonical WTokens]
The construction above yields exactly 20 distinct, unit-norm, neutral basis vectors:
\begin{itemize}
    \item \textbf{Conjugate Families ($F_{1,7}, F_{2,6}, F_{3,5}$):} Each has 4 $\varphi$-levels and only $\tau=0$.
    $$ 3 \text{ families} \times 4 \text{ levels} = 12 \text{ tokens} $$
    \item \textbf{Nyquist Family ($F_4$):} Has 4 $\varphi$-levels and two $\tau$-offsets ($\tau=0, \tau=2$).
    $$ 1 \text{ family} \times 4 \text{ levels} \times 2 \text{ offsets} = 8 \text{ tokens} $$
\end{itemize}
Total cardinality: $12 + 8 = 20$.
\end{theorem}

This counting argument is formalized in the Lean module \texttt{LightLanguage.CanonicalWTokens.canonical\_card\_20}. It proves that the set of 20 WTokens is not an arbitrary list but the exhaustive enumeration of stable eigenmodes under the system's symmetries.

\subsection{Properties of WTokens}

The constructed WTokens inherit critical physical properties from the DFT backbone.

\begin{theorem}[WToken Neutrality]
For every WToken $W_i$, $\sum_{t=0}^7 W_i[t] = 0$.
\end{theorem}
\begin{proof}
Each WToken is a linear combination of non-zero DFT modes (shifted and scaled). Since every non-zero DFT mode is neutral (Theorem 5.1), any linear combination is also neutral.
\end{proof}

\begin{theorem}[WToken Normalization]
For every WToken $W_i$, $||W_i|| = 1$.
\end{theorem}
\begin{proof}
The construction explicitly includes a normalization step.
\end{proof}

\begin{theorem}[Mode-4 Orthogonality (Realified Geometry)]
For the Nyquist family $F_4$, the $\tau=0$ (real) and $\tau=2$ (imaginary) variants are orthogonal \emph{in the realified inner product} on $\C^8$ viewed as $\R^{16}$.
\end{theorem}
\begin{proof}
Let $v$ be the base vector for $F_4$ (real-valued up to a global phase). The $\tau=0$ token is $v$, and the $\tau=2$ token is $iv$.
In the standard complex Hermitian inner product, $\langle v, iv \rangle = i \lVert v \rVert^2$ is purely imaginary and therefore not zero.
However, when viewing $\C^8$ as a real vector space $\R^{16}$, the canonical real inner product is
\[
  (u, w)_{\R} := \Re \langle u, w \rangle.
\]
Then
\[
  (v, iv)_{\R} = \Re ( i \lVert v \rVert^2 ) = 0,
\]
so the $\tau=0$ and $\tau=2$ variants are orthogonal in the realified geometry.
\end{proof}

\subsection{WToken Dictionary as Overcomplete Frame}

The 20 WTokens span the 7-dimensional neutral subspace. Since $20 > 7$, they form an overcomplete set, or a \textit{frame}.

\begin{proposition}
The set of 20 WTokens $\{W_i\}_{i=0}^{19}$ forms a tight frame for the neutral subspace $\neutral$.
\end{proposition}

\newpage

%----------------------------------------------------------------------------------------
%	SECTION 7: MEANING AS CHORD
%----------------------------------------------------------------------------------------
\section{Meaning as Chord}

In the ULL framework, a concept is not a discrete token but a geometric object. We formalize this using the chord model.

\subsection{The Chord Model}

\begin{definition}[Semantic Chord]
A meaning is a chord $\psi \in \neutral$ (the neutral subspace) with unit norm $||\psi|| = 1$. It can be expressed as a superposition of WTokens:
$$ \psi = \sum_{w=0}^{19} c_w W_w, \quad c_w \in \C $$
\end{definition}

Since the WTokens form an overcomplete frame, the coefficients $c_w$ are not unique for a given $\psi$. However, the chord $\psi$ itself is the unique, invariant object. The coefficients represent one possible ``parse'' or decomposition of the meaning into primitives.

\begin{definition}[Chord Coefficients]
The canonical analysis coefficients are given by the inner products:
$$ c_w = \langle W_w, \psi \rangle $$
These represent the ``loudness'' or resonance of each semantic primitive within the chord.
\end{definition}

\subsection{Semantic Operations}

The geometric structure of $\C^8$ allows us to define precise operations on meanings.

\begin{definition}[WToken Measurement]
Given a chord $\psi$, the probability of observing WToken $w$ is given by the Born rule:
$$ p(w) \propto |c_w|^2 = |\langle W_w, \psi \rangle|^2 $$
\end{definition}

This connects the continuous chord representation to discrete token observation.

\begin{definition}[Semantic Projection]
The projection of an arbitrary vector $\psi$ (e.g., from a neural network hidden state) onto the WToken basis is:
$$ \text{proj}(\psi) = \sum_{w=0}^{19} \langle W_w, \psi \rangle W_w $$
This operation enforces the structural constraints of the language.
\end{definition}

\begin{definition}[Semantic Distance]
The distance between two meanings $\psi_1, \psi_2$ is the chordal distance on the unit sphere:
$$ d(\psi_1, \psi_2) = ||\psi_1 - \psi_2|| = \sqrt{2 - 2\text{Re}\langle \psi_1, \psi_2 \rangle} $$
\end{definition}

This metric provides a rigorous definition of synonymy.

\begin{theorem}[Paraphrase Criterion]
Two linguistic expressions are paraphrases if and only if their associated chords $\psi_1, \psi_2$ satisfy $d(\psi_1, \psi_2) < \epsilon$ for some semantic tolerance $\epsilon$.
\end{theorem}

This transforms the vague linguistic notion of ``same meaning'' into a precise geometric condition.

\subsection{U(1) Gauge Invariance}

The phase of a quantum state is not directly observable; only relative phases matter. The same applies to ULL.

\begin{definition}[Gauge Equivalence]
Two chords $\psi_1, \psi_2$ are gauge equivalent, denoted $\psi_1 \sim \psi_2$, if there exists a global phase $\theta \in \R$ such that:
$$ \psi_2 = e^{i\theta} \psi_1 $$
\end{definition}

\newpage

%----------------------------------------------------------------------------------------
%	SECTION 8: INFORMATION CAPACITY
%----------------------------------------------------------------------------------------
\section{Information Capacity}

A central question for any semantic theory is whether the proposed primitives are sufficient to encode the full range of human meaning. We address this by analyzing the information-theoretic capacity of the ULL substrate.

\subsection{Degrees of Freedom Analysis}

The semantic manifold $S_\neutral$ is a 13-dimensional real sphere embedded in $\C^8$.

\begin{theorem}[Semantic Capacity]
The neutral semantic subspace $\neutral$ has 7 complex dimensions, corresponding to 14 real degrees of freedom. The 20 WTokens provide an overcomplete basis for this space.
\end{theorem}

\begin{proof}
The full space $\C^8$ has 8 complex dimensions (16 real). The neutrality constraint $\sum \psi[t] = 0$ removes 1 complex dimension (2 real). Thus, $\dim_\C(\neutral) = 7$ and $\dim_\R(\neutral) = 14$.
The redundancy factor of the WToken dictionary is:
$$ R = \frac{\text{Number of Primitives}}{\text{Complex Dimension}} = \frac{20}{7} \approx 2.86 $$
\end{proof}

This redundancy factor is significant. It indicates that the semantic system is not a minimal basis but a redundant frame, optimized for something other than maximal compression.

\subsection{Noise Immunity}

Why does the system use 20 primitives instead of 7? The answer lies in robustness.

\begin{proposition}
The overcomplete WToken dictionary provides noise immunity. Any single WToken corruption (e.g., erasure or phase flip) can be detected and corrected via consistency checks with the remaining tokens.
\end{proposition}

In a minimal basis (7 tokens), the loss of one coefficient would destroy 1/7th of the information irretrievably. In a tight frame like the WTokens, the information is distributed holographically. The coefficients $c_w$ are not independent; they satisfy linear dependencies that act as error-correcting codes.

\subsection{Comparison to Natural Language}

Natural languages typically have a lexicon of $10^4$ to $10^5$ morphemes. How can 20 WTokens compete with this diversity?

\begin{proposition}
Natural languages use a large set of discrete, low-precision symbols (words). ULL uses a small set of continuous, high-precision primitives (WTokens).
\end{proposition}

A word like "dog" is a discrete point in a vast lexical space. In ULL, "dog" is a specific chord:
$$ \psi_{\text{dog}} = 0.4 W_{\text{ORIGIN}} + 0.3 e^{i\pi/4} W_{\text{STRUCTURE}} + \dots $$
The diversity of meaning comes from the continuous coefficients $c_w \in \C$, not from the number of primitives.

\newpage

%----------------------------------------------------------------------------------------
%	SECTION 9: THE COST LANDSCAPE
%----------------------------------------------------------------------------------------
\section{The Cost Landscape}

The semantic manifold $S_\neutral$ is not flat. It is endowed with a potential energy landscape derived from the fundamental cost functional $\cost$. This landscape defines the ``physics'' of meaning, governing which chords are stable (consonant) and which are unstable (dissonant).

\subsection{J-Cost on Chords}

The fundamental cost functional $\cost(x)$ was defined for scalar ratios. We extend this to chords by summing the costs of the WToken coefficients.

\begin{definition}[Chord J-Cost]
For a semantic chord $\psi \in \neutral$ with WToken decomposition $\psi = \sum c_w W_w$, the chord J-cost is defined as:
$$ \cost_{\text{chord}}(\psi) = \sum_{w=0}^{19} \cost(|c_w|^2) $$
where $\cost(x) = \frac{1}{2}(x + x^{-1}) - 1$.
\end{definition}

Note that we apply the cost to the squared magnitudes $|c_w|^2$, which represent the probability mass or intensity of each primitive.

\subsection{Consonance and Dissonance}

The J-cost provides a rigorous, quantitative definition of semantic stability.

\begin{definition}[Consonant Chord]
A chord $\psi$ is consonant if $\cost_{\text{chord}}(\psi)$ is minimal. The global minima occur at the eigenstates (where only one $c_w$ is non-zero, $|c_w|=1$) and at specific harmonic superpositions.
\end{definition}

\begin{definition}[Dissonant Chord]
A chord $\psi$ is dissonant if $\cost_{\text{chord}}(\psi)$ is large. This occurs when probability mass is fragmented across many conflicting primitives, or when coefficients take values far from the stable points of the cost function.
\end{definition}

In linguistic terms, consonance corresponds to clarity, coherence, and well-formedness. Dissonance corresponds to ambiguity, contradiction, and nonsense.

\subsection{The Ethics-Physics Bridge}

One of the most profound implications of ULL is the identification of topological stability with moral reciprocity.

\begin{theorem}[Moral Reciprocity]
Topologically stable states (those satisfying $\sigma = 0$ and minimizing J-cost) are exactly those that satisfy the condition of moral reciprocity.
\end{theorem}

This theorem is formalized in the Lean module \texttt{Ethics.PhysicsEthicsIdentity.moralReciprocity\_iff\_topologicalStability}.

\newpage

%----------------------------------------------------------------------------------------
%	SECTION 10: NORMAL FORMS AND EQUIVALENCE
%----------------------------------------------------------------------------------------
\section{Normal Forms and Equivalence}

To compute with meaning, we need a set of operations that transform chords while preserving their essential structure. This leads to the Light-Native Assembly Language (LNAL).

\subsection{LNAL Operators}

LNAL defines a set of unitary (or projection) operators on the chord space $\C^8$. These operators correspond to fundamental semantic manipulations.

\begin{definition}[LNAL Operators]
The core LNAL instruction set includes:
\begin{itemize}
    \item \textbf{BALANCE:} The projection operator onto the neutral subspace $\neutral$.
    $$ \text{BALANCE}(\psi) = \psi - c_0 \text{mode}_0 $$
    \item \textbf{LOCK:} A diagonal projection operator that isolates specific time-slots or frequency modes.
    \item \textbf{FOLD:} A conjugation operator that maps a mode $k$ to its conjugate $8-k$. This corresponds to time-reversal or semantic inversion.
    \item \textbf{BRAID:} An SU(3) rotation acting on a triad of modes (e.g., a mode family). This mixes semantic components while preserving norm.
\end{itemize}
\end{definition}

These operators form a group (or semigroup) of transformations on the semantic manifold.

\subsection{Normal Form}

A key problem in semantics is determining when two different expressions mean the same thing (paraphrase). In ULL, this is solved by reduction to normal form.

\begin{definition}[Normal Form]
A sequence of LNAL operations applied to a set of input chords produces a result chord. Two sequences are equivalent if they produce the same result chord (up to gauge). The \textit{normal form} of a meaning is its unique representation as a minimal chord in the canonical WToken basis.
\end{definition}

\newpage

%----------------------------------------------------------------------------------------
%	SECTION 11: DISCUSSION
%----------------------------------------------------------------------------------------
\section{Discussion}

The ULL framework offers a unified, zero-parameter theory of semantic structure. We now discuss its explanatory power, predictive consequences, and conditions for falsification.

\subsection{What ULL Explains}

ULL provides first-principles answers to questions that have historically been treated as empirical contingencies.

\begin{enumerate}
    \item \textbf{Why 20 Semantic Primitives?}
    Traditional semantic theories fit the number of primitives to the data (e.g., Wierzbicka's ~65). ULL derives the number 20 from the symmetries of the 8-tick phase register (3 conjugate families $\times$ 4 levels + 1 Nyquist family $\times$ 8 variants). This suggests that the "periodic table of meaning" is finite and mathematically forced.

    \item \textbf{Why Meaning is Phase Geometry?}
    The failure of symbolic logic to capture the nuance of natural language suggests that meaning is continuous, not discrete. However, vector space models lack structure. ULL explains meaning as \textit{phase geometry}: the relative timing and interference of fundamental cognitive waves. This explains why meaning is both continuous (coefficients) and structured (eigenmodes).

    \item \textbf{Why Paraphrase is Chord Isometry?}
    Syntactic transformations (active/passive, dative shift) often preserve meaning. In ULL, these are gauge transformations or unitary rotations that leave the underlying chord invariant. Paraphrase is not a rewriting rule but a geometric identity.
\end{enumerate}

\subsection{What ULL Predicts}

As a physical theory of meaning, ULL makes testable predictions about human language and cognition.

\begin{enumerate}
    \item \textbf{Structure of Semantic Universals:}
    ULL predicts that all human languages, despite surface variation, map to the same 20-dimensional spectral basis. We expect to find lexical primitives in all languages that cluster around the canonical WTokens (e.g., ORIGIN, TRUTH, CHAOS, CONNECTION).

    \item \textbf{Cross-Linguistic Convergence:}
    Languages should evolve to optimize communication efficiency within the ULL capacity bounds. We predict that the most frequent words in any language will map to low-J-cost chords (consonant states), while rare or complex concepts will map to high-J-cost chords.

    \item \textbf{Semantic Distance Correlates with J-Cost:}
    Concepts that are intuitively "hard to grasp" or "complex" should correspond to chords with higher J-cost. Concepts that are "simple" or "fundamental" should have lower J-cost. This correlation should hold across languages.
\end{enumerate}

\subsection{Falsifiability}

ULL is a scientific theory, not just a philosophy. It is falsifiable under specific conditions.

\begin{enumerate}
    \item \textbf{Alternative Zero-Parameter Theories:}
    If a consistent, zero-parameter mathematical framework for semantics is discovered that is \textit{not} isomorphic to the 8-tick/20-token structure, ULL is falsified (or at least challenged as unique).

    \item \textbf{Non-8-Tick Periodicity:}
    If neuroscientific or physical evidence conclusively proves that the fundamental cognitive cycle is not compatible with an 8-tick clock (e.g., if it requires a prime period like 5 or 7), the derivation of the 8-tick carrier is falsified.
\end{enumerate}

\subsection{Open Questions}

The ULL framework opens several avenues for future research.

\newpage

%----------------------------------------------------------------------------------------
%	SECTION 12: CONCLUSION
%----------------------------------------------------------------------------------------
\section{Conclusion}

We have presented Universal Light Language (ULL), a mathematical framework that identifies meaning with geometric structure on an eight-sample complex phase register. By imposing a single compositional constraint---the Recognition Composition Law---we have derived a unique cost functional that forces the carrier to have an 8-tick period, forces the semantic subspace to be mean-free (neutral), and forces the existence of exactly 20 canonical semantic primitives (WTokens).

In this framework, meaning is formalized as a chord: a unit-norm superposition in the neutral subspace $\C^7$. Semantic equivalence is defined as chord isometry, and communication is modeled as resonance in a shared cost landscape. We have established information-theoretic capacity bounds showing that the 20-token overcomplete dictionary provides optimal noise immunity for the 14 real degrees of freedom in the semantic subspace.

The theory is parameter-free: all structure derives from a single compositional constraint with no external inputs. We propose that semantic universals across languages reflect not cultural convergence but the underlying mathematical necessity of meaning-as-chord. ULL offers a path toward a rigorous, predictive science of semantics, grounded in the same principles of symmetry and conservation that govern the physical universe.

\newpage

%----------------------------------------------------------------------------------------
%	APPENDICES
%----------------------------------------------------------------------------------------
\appendix

\section{Lean Proof Index}

The following table maps the key theorems of this paper to their formal verification modules in the \texttt{straight-shot} repository.

\begin{table}[h]
\centering
\begin{tabular}{@{}llp{8cm}@{}}
\toprule
\textbf{Theorem} & \textbf{Claim} & \textbf{Lean Module} \\ \midrule
T2.1 & Cost Uniqueness & \texttt{IndisputableMonolith.CostUniqueness.T5\_uniqueness\_complete} \\
T3.2 & 8-Tick Period & \texttt{IndisputableMonolith.LightLanguage.Basis.DFT8} (periodicity) \\
T4.2 & Shift Diagonalization & \texttt{LightLanguage.Basis.DFT8.dft8\_diagonalizes\_shift} \\
T4.3 & DFT-8 Unitarity & \texttt{LightLanguage.Basis.DFT8.dft8\_unitary} \\
T5.1 & Neutral Subspace & \texttt{LightLanguage.Basis.DFT8.dft8\_neutral\_subspace} \\
T6.1 & 20 Canonical WTokens & \texttt{LightLanguage.CanonicalWTokens.canonical\_card\_20} \\
T9.1 & Moral Reciprocity & \texttt{Ethics.PhysicsEthicsIdentity.moralReciprocity\_iff\_topologicalStability} \\
\bottomrule
\end{tabular}
\caption{Index of Formal Proofs}
\end{table}

\section{WToken Dictionary}

The 20 canonical semantic primitives derived from the 8-tick phase field.

\begin{table}[h]
\centering
\small
\begin{tabular}{@{}lccll@{}}
\toprule
\textbf{Token} & \textbf{Mode} & \textbf{$\varphi$-Level} & \textbf{$\tau$} & \textbf{Semantic Gloss} \\ \midrule
\multicolumn{5}{l}{\textit{Family 1+7 (Fundamental)}} \\
ORIGIN & 1 & 0 & 0 & Primordial emergence, source \\
EMERGENCE & 1 & 1 & 0 & Coming into being, manifestation \\
POLARITY & 1 & 2 & 0 & Duality, contrast, opposites \\
HARMONY & 1 & 3 & 0 & Balance, equilibrium, unity \\ \midrule
\multicolumn{5}{l}{\textit{Family 2+6 (First Harmonic)}} \\
POWER & 2 & 0 & 0 & Force, intensity, strength \\
BIRTH & 2 & 1 & 0 & Creation, new life, renewal \\
STRUCTURE & 2 & 2 & 0 & Form, pattern, organization \\
RESONANCE & 2 & 3 & 0 & Vibration, echo, alignment \\ \midrule
\multicolumn{5}{l}{\textit{Family 3+5 (Second Harmonic)}} \\
INFINITY & 3 & 0 & 0 & Boundlessness, eternity \\
TRUTH & 3 & 1 & 0 & Reality, authenticity, fact \\
COMPLETION & 3 & 2 & 0 & Wholeness, totality, fulfillment \\
INSPIRE & 3 & 3 & 0 & Spirit, motivation, calling \\ \midrule
\multicolumn{5}{l}{\textit{Family 4 (Nyquist - Real)}} \\
TRANSFORM & 4 & 0 & 0 & Change, metamorphosis \\
END & 4 & 1 & 0 & Conclusion, termination \\
CONNECTION & 4 & 2 & 0 & Relationship, bond, love \\
WISDOM & 4 & 3 & 0 & Knowledge, understanding \\ \midrule
\multicolumn{5}{l}{\textit{Family 4 (Nyquist - Imaginary)}} \\
ILLUSION & 4 & 0 & 2 & Appearance, maya, deception \\
CHAOS & 4 & 1 & 2 & Disorder, entropy \\
TWIST & 4 & 2 & 2 & Rotation, spiral, helix \\
TIME & 4 & 3 & 2 & Duration, sequence, flow \\ \bottomrule
\end{tabular}
\caption{The Periodic Table of Meaning}
\end{table}

\section{DFT-8 Matrix}

The explicit form of the DFT-8 matrix $B$ (up to normalization factor $1/\sqrt{8}$):

$$
\sqrt{8} B = 
\begin{bmatrix}
1 & 1 & 1 & 1 & 1 & 1 & 1 & 1 \\
1 & \omega & \omega^2 & \omega^3 & \omega^4 & \omega^5 & \omega^6 & \omega^7 \\
1 & \omega^2 & \omega^4 & \omega^6 & 1 & \omega^2 & \omega^4 & \omega^6 \\
1 & \omega^3 & \omega^6 & \omega & \omega^4 & \omega^7 & \omega^2 & \omega^5 \\
1 & \omega^4 & 1 & \omega^4 & 1 & \omega^4 & 1 & \omega^4 \\
1 & \omega^5 & \omega^2 & \omega^7 & \omega^4 & \omega & \omega^6 & \omega^3 \\
1 & \omega^6 & \omega^4 & \omega^2 & 1 & \omega^6 & \omega^4 & \omega^2 \\
1 & \omega^7 & \omega^6 & \omega^5 & \omega^4 & \omega^3 & \omega^2 & \omega
\end{bmatrix}
$$
where $\omega = e^{-i\pi/4}$.

\section{Comparison to Prior Semantic Theories}

\begin{table}[h]
\centering
\begin{tabular}{@{}p{3cm}p{3.5cm}p{3.5cm}p{3.5cm}@{}}
\toprule
\textbf{Feature} & \textbf{Distributional (LLMs)} & \textbf{Formal (Montague)} & \textbf{ULL (Recognition)} \\ \midrule
\textbf{Substrate} & High-dim Vector Space ($\R^{1024}$) & Symbolic Logic / Types & Phase Field ($\C^8$) \\
\textbf{Primitives} & None (learned dimensions) & Logical Atoms ($\forall, \exists, \land$) & 20 Eigenmodes \\
\textbf{Origin} & Fitted to Corpus & Postulated & Derived from Symmetry \\
\textbf{Composition} & Vector Addition & Function Application & Superposition (Chord) \\
\textbf{Meaning} & Contextual Usage & Truth Conditions & Phase Geometry \\
\textbf{Ethics} & RLHF / Alignment & None / External & Intrinsic ($\sigma=0$) \\
\bottomrule
\end{tabular}
\caption{Comparison of Semantic Frameworks}
\end{table}

\end{document}
