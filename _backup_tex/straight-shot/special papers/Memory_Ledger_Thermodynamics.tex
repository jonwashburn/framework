\documentclass[11pt,twocolumn]{article}

% Packages
\usepackage[utf8]{inputenc}
\usepackage[T1]{fontenc}
\usepackage{amsmath,amssymb,amsthm}
\usepackage{mathtools}
\usepackage{graphicx}
\usepackage{hyperref}
\usepackage{booktabs}
\usepackage[margin=1in]{geometry}
\usepackage{xcolor}

% Theorem environments
\newtheorem{theorem}{Theorem}[section]
\newtheorem{lemma}[theorem]{Lemma}
\newtheorem{proposition}[theorem]{Proposition}
\newtheorem{corollary}[theorem]{Corollary}
\newtheorem{definition}[theorem]{Definition}
\newtheorem{axiom}{Axiom}
\newtheorem{prediction}{Prediction}
\newtheorem{falsifier}{Falsifier}
\theoremstyle{remark}
\newtheorem{remark}[theorem]{Remark}

% Custom commands
\newcommand{\Jcost}{J}
\newcommand{\Jmem}{J_{\text{mem}}}
\newcommand{\TR}{T_R}
\newcommand{\FR}{F_R}
\newcommand{\SR}{S_R}
\newcommand{\Rhat}{\hat{R}}
\newcommand{\phival}{\varphi}
\newcommand{\tauzero}{\tau_0}
\newcommand{\taubreath}{\tau_{\text{b}}}
\newcommand{\dd}{\mathrm{d}}

% Title
\title{The Thermodynamics of Memory:\\
A Cost-Functional Approach to Retention and Forgetting}

\author{
Jonathan Washburn\\
\textit{Recognition Science Research Institute}\\
\textit{Austin, Texas}\\
\texttt{washburn.jonathan@gmail.com}
}

\date{\today}

\begin{document}

\maketitle

\begin{abstract}
We present a thermodynamic theory of memory derived from Recognition Science, which posits a unique cost functional $\Jcost(x) = \frac{1}{2}(x + x^{-1}) - 1$ arising from the d'Alembert composition law. Memory is formalized as a cost-minimizing dynamical system where retention competes with free-energy decay. We derive forgetting dynamics that yield exponential decay for short times and power-law decay for long times, unifying the Ebbinghaus and Wixted-Ebbesen observations. The memory-specific cost $\Jmem$ depends on pattern complexity, emotional weight, and ledger balance through principled dimensional analysis. The theory predicts: (1) working memory capacity in the range $[\phival^2, \phival^4] \approx [2.6, 6.9]$ items; (2) emotional retention advantage factor of $\phival \approx 1.618$; (3) spacing advantage scaling as $\log(\Delta t)$; (4) sleep consolidation ratio $\gamma_{\text{Deep}}/\gamma_{\text{Light}} = \phival^2 \approx 2.618$; (5) PTSD threshold at ledger imbalance $|\beta| \geq 2\beta_0$. The framework provides falsifiable predictions with explicit refutation conditions.
\end{abstract}

\section{Introduction}

Memory has traditionally been studied as an information storage and retrieval problem \cite{atkinson1968}. However, the physics of memory---why some memories persist while others fade, why emotional content enhances retention, why sleep consolidates learning---remains poorly understood at a fundamental level.

We propose that memory is not fundamentally about storage but about \textit{cost minimization}. Every memory trace has an associated maintenance cost, and the dynamics of retention versus forgetting emerge from the minimization of a free energy functional. This perspective unifies diverse phenomena including:

\begin{itemize}
    \item The Ebbinghaus forgetting curve \cite{ebbinghaus1885}
    \item Working memory capacity limits \cite{miller1956,cowan2001}
    \item Emotional memory enhancement \cite{mcgaugh2004}
    \item Sleep-dependent consolidation \cite{stickgold2005}
    \item Spacing effect in learning \cite{cepeda2006}
    \item Power-law forgetting \cite{wixted1991}
    \item Traumatic memory persistence (PTSD) \cite{brewin2010}
\end{itemize}

Our approach is grounded in Recognition Science (RS), a framework that derives physical constants and structures from a single cost functional satisfying the d'Alembert composition law \cite{washburn2025rs}. We extend this framework to cognitive systems, treating memory traces as configurations with associated costs.

\subsection{Why Cost Minimization?}

The cost-minimization framework is motivated by three observations:

\begin{enumerate}
    \item \textbf{Metabolic constraint}: Neural activity requires energy. Maintaining a memory trace has an ongoing metabolic cost proportional to synaptic strength \cite{attwell2001}.
    
    \item \textbf{Interference cost}: Each stored memory occupies representational space, creating interference with other memories \cite{underwood1957}.
    
    \item \textbf{Equilibrium forgetting}: In the absence of rehearsal, memories decay---consistent with relaxation toward a minimum-energy state.
\end{enumerate}

The question is: what is the correct cost functional? We argue that the d'Alembert functional equation, which governs multiplicative composition, provides a unique answer.

\subsection{Relation to Computational Models}

Before proceeding, we situate this work relative to existing computational memory models:

\textbf{ACT-R} \cite{anderson2004}: Models memory retrieval via base-level activation that decays as a power law. Our theory derives this power-law from thermodynamic principles rather than assuming it.

\textbf{SAM/REM} \cite{raaijmakers1981}: Uses strength-based retrieval with separate storage and retrieval processes. We unify these via the cost functional.

\textbf{SIMPLE} \cite{brown2007}: Emphasizes temporal distinctiveness. Our ledger balance term captures a related interference mechanism.

The present theory differs by deriving memory dynamics from a single, principled cost functional rather than fitting phenomenological equations.

\section{The Cost Functional Framework}

\subsection{The d'Alembert Functional Equation}

Consider a cost functional $\Jcost: \mathbb{R}^+ \to \mathbb{R}$ that satisfies the multiplicative composition law:
\begin{equation}\label{eq:dalembert}
    \Jcost(xy) + \Jcost(x/y) = 2[\Jcost(x) + 1][\Jcost(y) + 1] - 2
\end{equation}

This equation arises when we require that the cost of a composite state factors multiplicatively---a natural requirement for systems where configurations combine via multiplication (e.g., probabilities, growth rates, or attention weights).

\begin{theorem}[Uniqueness of J-Cost]\label{thm:uniqueness}
The unique smooth solution to \eqref{eq:dalembert} with normalization $\Jcost(1) = 0$ and calibration $\Jcost''(1) = 1$ is:
\begin{equation}\label{eq:jcost}
    \Jcost(x) = \frac{1}{2}\left(x + \frac{1}{x}\right) - 1 = \frac{(x-1)^2}{2x}
\end{equation}
\end{theorem}

\begin{proof}
The functional equation \eqref{eq:dalembert} is solved by $\Jcost(x) + 1 = \cosh(\alpha \ln x)$ for any $\alpha$. To verify:
\begin{align}
    &[\cosh(\alpha \ln xy) - 1] + [\cosh(\alpha \ln(x/y)) - 1] \notag \\
    &= \cosh(\alpha(\ln x + \ln y)) + \cosh(\alpha(\ln x - \ln y)) - 2 \notag \\
    &= 2\cosh(\alpha \ln x)\cosh(\alpha \ln y) - 2
\end{align}
using the identity $\cosh(a+b) + \cosh(a-b) = 2\cosh a \cosh b$.

The RHS of \eqref{eq:dalembert} is $2[\cosh(\alpha \ln x)][\cosh(\alpha \ln y)] - 2$, confirming equality.

The constraint $\Jcost''(1) = 1$ requires $\alpha^2 = 1$, so $\alpha = 1$ (taking positive root). Thus $\Jcost(x) = \cosh(\ln x) - 1 = \frac{1}{2}(e^{\ln x} + e^{-\ln x}) - 1 = \frac{1}{2}(x + x^{-1}) - 1$.
\end{proof}

Key properties of $\Jcost$:
\begin{itemize}
    \item $\Jcost(x) \geq 0$ for all $x > 0$, with equality iff $x = 1$
    \item $\Jcost(x) = \Jcost(1/x)$ (symmetry under inversion)
    \item $\Jcost(x) \to \infty$ as $x \to 0^+$ or $x \to \infty$
    \item Near $x = 1$: $\Jcost(x) \approx \frac{1}{2}(x-1)^2$ (harmonic approximation)
\end{itemize}

\subsection{The Golden Ratio from Discrete Self-Similarity}

When the cost functional is discretized on a lattice with self-similar structure, the golden ratio emerges uniquely:

\begin{theorem}[Golden Ratio from Minimal Lattice]\label{thm:phi}
Consider a discrete lattice $\{x_n\}_{n \in \mathbb{Z}}$ with constant ratio $r = x_{n+1}/x_n$. The requirement that the lattice be \textit{self-similar under decimation} (removing every other point yields an isomorphic lattice) uniquely fixes:
\begin{equation}
    r = \phival = \frac{1 + \sqrt{5}}{2} \approx 1.618034
\end{equation}
\end{theorem}

\begin{proof}
Self-similarity under decimation means the lattice $\{x_{2n}\}$ has the same structure as $\{x_n\}$ up to scaling. The ratio for the decimated lattice is $x_{2(n+1)}/x_{2n} = x_{2n+2}/x_{2n} = r^2$.

For self-similarity, we require $r^2 = r \cdot c$ for some universal constant $c$. The minimal non-trivial choice is $c = r + 1$ (the lattice spacing equals the sum of the previous two), giving:
\begin{equation}
    r^2 = r + 1 \implies r = \frac{1 + \sqrt{5}}{2} = \phival
\end{equation}
This is equivalent to requiring $x_{n+2} = x_{n+1} + x_n$ (Fibonacci recursion).
\end{proof}

\begin{remark}
The choice $c = r + 1$ is ``minimal'' in that it involves only the two adjacent lattice points. Any other choice would introduce non-local dependencies.
\end{remark}

\subsection{Fundamental Time Scales}

\begin{definition}[Time Scales]\label{def:timescales}
The theory involves two fundamental time scales:
\begin{itemize}
    \item $\tauzero \approx 25$ ms: The minimal recognition time (``tick''), corresponding to a 40 Hz neural oscillation cycle.
    \item $\taubreath = 1024 \cdot \tauzero \approx 25.6$ s: The ``breath cycle,'' the characteristic period for consolidation.
\end{itemize}
The factor $1024 = 2^{10}$ arises from the 8-tick working memory cycle ($8\tauzero \approx 200$ ms) iterated through the $\phival$-ladder: $8 \times 128 = 1024$ where 128 corresponds to $\sim 7$ $\phival$-steps.
\end{definition}

\subsection{Recognition Thermodynamics}

At finite Recognition Temperature $\TR > 0$, configurations are distributed according to the Gibbs measure:
\begin{equation}\label{eq:gibbs}
    p(x) = \frac{1}{Z} \exp\left(-\frac{\Jcost(x)}{\TR}\right)
\end{equation}
where $Z = \int_0^\infty \exp(-\Jcost(x)/\TR) \, \dd x$ is the partition function.

\begin{definition}[Recognition Free Energy]\label{def:free_energy}
The Recognition Free Energy is:
\begin{equation}\label{eq:free_energy}
    \FR = \langle \Jcost \rangle - \TR \cdot \SR
\end{equation}
where $\SR = -\int p(x) \ln p(x) \, \dd x$ is the Recognition Entropy.
\end{definition}

\begin{axiom}[Second Law of Recognition]\label{ax:second_law}
Under RS dynamics, $\FR$ is non-increasing:
\begin{equation}\label{eq:second_law}
    \frac{\dd\FR}{\dd t} \leq 0
\end{equation}
\end{axiom}

This axiom is the foundation for forgetting: systems relax toward minimum free energy.

\section{Memory Ledger Theory}

\subsection{Memory Trace Structure}

\begin{definition}[Memory Trace]\label{def:trace}
A memory trace $\mathcal{T}$ consists of:
\begin{itemize}
    \item \textbf{Complexity} $C > 0$: Pattern information content (in bits)
    \item \textbf{Emotional weight} $\omega \in [0,1]$: Affective salience
    \item \textbf{Strength} $S \in (0,1]$: Current accessibility/activation
    \item \textbf{Encoding time} $t_0$: When the trace was formed
    \item \textbf{Ledger balance} $\beta \in \mathbb{Z}$: Recalls minus re-encodings
    \item \textbf{Consolidation state} $\chi \in \{0,1\}$: Working (0) vs. long-term (1)
\end{itemize}
\end{definition}

\subsection{Derivation of Memory J-Cost}

The cost of maintaining a memory trace must be dimensionless (when divided by $\TR$). We construct $\Jmem$ from first principles:

\begin{proposition}[Memory Cost Construction]\label{prop:jmem}
The memory cost is:
\begin{equation}\label{eq:jmem}
    \Jmem(\mathcal{T}, t) = \epsilon(\omega) \cdot \left[ C \cdot \Jcost(S) + \Jcost\left(\frac{t - t_0}{\taubreath} + 1\right) + \Jcost\left(1 + \frac{|\beta|}{\beta_0}\right) \right]
\end{equation}
where:
\begin{itemize}
    \item $\epsilon(\omega) = 1 - \omega(1 - \phival^{-1})$ is the emotional discount factor
    \item $\beta_0 = \phival^3 \approx 4.24$ is the interference scale (equal to WM capacity)
\end{itemize}
\end{proposition}

\begin{proof}[Construction]
Each term is motivated by a distinct physical consideration:
\begin{enumerate}
    \item \textbf{Complexity term} $C \cdot \Jcost(S)$: A pattern of $C$ bits at strength $S$ has cost proportional to $C$. When $S = 1$ (perfect recall), $\Jcost(1) = 0$. As $S \to 0$, the trace becomes incoherent and cost diverges.
    
    \item \textbf{Time term} $\Jcost((t-t_0)/\taubreath + 1)$: Time since encoding, measured in breath cycles, increases maintenance cost. The $+1$ ensures cost is zero at encoding ($t = t_0$).
    
    \item \textbf{Interference term} $\Jcost(1 + |\beta|/\beta_0)$: Imbalanced ledger (many recalls without re-encoding, or vice versa) creates interference. The scale $\beta_0 = \phival^3$ links to working memory capacity.
    
    \item \textbf{Emotional discount} $\epsilon(\omega)$: Emotional tagging reduces cost multiplicatively.
\end{enumerate}
The universal use of $\Jcost$ ensures: (a) non-negativity, (b) dimensional consistency, (c) connection to the fundamental composition law.
\end{proof}

\subsection{Emotional Discount Factor}

\begin{proposition}[Emotional Reduction]\label{prop:emotional}
The emotional discount factor $\epsilon(\omega) = 1 - \omega(1 - \phival^{-1})$ satisfies:
\begin{enumerate}
    \item $\epsilon(0) = 1$ (neutral memories: full cost)
    \item $\epsilon(1) = \phival^{-1} \approx 0.618$ (maximal emotion: 38\% cost reduction)
    \item $\dd\epsilon/\dd\omega = -(1 - \phival^{-1}) = -\phival^{-2} < 0$ (strictly decreasing)
\end{enumerate}
\end{proposition}

\begin{proof}
Direct calculation. Note that $1 - \phival^{-1} = \phival^{-2}$ follows from $\phival^{-1} = \phival - 1$.
\end{proof}

\begin{corollary}[Flashbulb Memory Effect]\label{cor:flashbulb}
For traces with identical $C$, $S$, and timing, higher $\omega$ implies lower $\Jmem$ and thus slower forgetting, explaining the persistence of emotional memories \cite{brown1977}.
\end{corollary}

\subsection{Forgetting Dynamics}

The Second Law (Axiom~\ref{ax:second_law}) implies memory strength decays to minimize free energy:

\begin{equation}\label{eq:dynamics}
    \frac{\dd S}{\dd t} = -\frac{1}{\TR} \cdot \frac{\partial \FR}{\partial S} = -\frac{1}{\TR} \cdot \frac{\partial \Jmem}{\partial S}
\end{equation}

Computing the derivative of the complexity term:
\begin{equation}\label{eq:derivative}
    \frac{\partial}{\partial S}\left[C \cdot \Jcost(S)\right] = \frac{C}{2}\left(1 - \frac{1}{S^2}\right)
\end{equation}

\begin{theorem}[Forgetting Dynamics]\label{thm:forgetting}
For the complexity-dominated regime ($C \cdot \Jcost(S) \gg$ time and interference terms), the strength evolution is:
\begin{equation}\label{eq:forgetting_ode}
    \frac{\dd S}{\dd t} = -\frac{\epsilon(\omega) C}{2\TR}\left(1 - \frac{1}{S^2}\right)
\end{equation}
\end{theorem}

\begin{proof}
Direct substitution of \eqref{eq:derivative} into \eqref{eq:dynamics}, retaining only the complexity term.
\end{proof}

\begin{corollary}[Short-Time Exponential Decay]\label{cor:exponential}
For $S \approx 1$ (recent memories), $1 - S^{-2} \approx 2(S-1)/S \approx 2(1-S)$, giving:
\begin{equation}
    \frac{\dd S}{\dd t} \approx -\frac{\epsilon C}{\TR}(1 - S)
\end{equation}
This has solution $1 - S(t) = (1 - S_0)e^{-t/\sigma}$ where $\sigma = \TR/(\epsilon C)$, i.e., approximately exponential decay of the deficit from unity.
\end{corollary}

\begin{corollary}[Long-Time Power-Law Decay]\label{cor:powerlaw}
For $S \ll 1$, $1 - S^{-2} \approx -S^{-2}$, giving:
\begin{equation}
    \frac{\dd S}{\dd t} \approx \frac{\epsilon C}{2\TR S^2}
\end{equation}
This has solution $S(t)^3 = S_0^3 + \frac{3\epsilon C}{2\TR}(t - t_0)$, or $S(t) \propto t^{-1/3}$ for large $t$---a power law.
\end{corollary}

\begin{remark}[Unifying Ebbinghaus and Wixted]
The theory naturally interpolates between exponential decay (short times, $S \approx 1$) and power-law decay (long times, $S \ll 1$), unifying the observations of Ebbinghaus \cite{ebbinghaus1885} and Wixted-Ebbesen \cite{wixted1991}.
\end{remark}

\begin{table}[h]
\centering
\caption{Predicted retention for exponential regime ($S \approx 1$)}
\label{tab:forgetting}
\begin{tabular}{lccc}
\toprule
$t/\sigma$ & Neutral ($\omega=0$) & Emotional ($\omega=1$) & Ratio \\
\midrule
0 & 1.000 & 1.000 & 1.00 \\
1 & 0.368 & 0.535 & 1.45 \\
2 & 0.135 & 0.286 & 2.12 \\
3 & 0.050 & 0.153 & 3.08 \\
$\phival$ & 0.199 & 0.368 & 1.85 \\
\bottomrule
\end{tabular}
\vspace{0.5em}

\textit{Retention $R(t) = e^{-t/\sigma}$. Emotional memories have $\sigma_{\text{em}} = \phival \cdot \sigma_{\text{neutral}}$, yielding asymptotic advantage $\to \phival$.}
\end{table}

\section{Working Memory Capacity}

\subsection{Phenomenological Axioms}

Working memory operates on the 8-tick cycle ($8\tauzero \approx 200$ ms), matching the theta oscillation period. We introduce two phenomenological axioms grounded in neural resource constraints:

\begin{axiom}[Attention Resource]\label{ax:attention}
Total attention capacity per 8-tick cycle is $A_{\text{total}} = \phival^4 \approx 6.85$ (in $\phival$-units).
\end{axiom}

\begin{remark}
This value is not arbitrary: $\phival^4$ is the fourth power in the $\phival$-ladder, corresponding to ``four levels'' of hierarchical processing within a single cycle. Neurally, this reflects the bandwidth of $\sim 7$ Hz theta oscillations.
\end{remark}

\begin{axiom}[Coherence Threshold]\label{ax:threshold}
Each item requires minimum attention $A_{\min} = \phival^{-1} \approx 0.618$ to remain above the coherence threshold.
\end{axiom}

\begin{remark}
The threshold $\phival^{-1}$ is the reciprocal of the golden ratio, representing the ``minimal viable'' allocation. Items below this threshold fail to cohere and are lost.
\end{remark}

\subsection{Capacity Derivation}

\begin{theorem}[Working Memory Capacity]\label{thm:miller}
With pairwise interference cost $A_{\text{int}} = \phival^{-3}$, the working memory capacity is:
\begin{equation}\label{eq:wm_capacity}
    N_{\text{WM}} \in [\phival^2, \phival^4] \approx [2.62, 6.85]
\end{equation}
with typical value $N_{\text{WM}} \approx \phival^3 \approx 4.24$.
\end{theorem}

\begin{proof}
The total attention budget constraint is:
\begin{equation}
    N \cdot A_{\min} + \binom{N}{2} A_{\text{int}} \leq A_{\text{total}}
\end{equation}

Substituting $A_{\min} = \phival^{-1}$, $A_{\text{int}} = \phival^{-3}$, $A_{\text{total}} = \phival^4$:
\begin{equation}
    N\phival^{-1} + \frac{N(N-1)}{2}\phival^{-3} \leq \phival^4
\end{equation}

Multiplying by $\phival^3$:
\begin{equation}
    N\phival^2 + \frac{N(N-1)}{2} \leq \phival^7
\end{equation}

For $N = \phival^3 \approx 4.24$:
\begin{align}
    \text{LHS} &= \phival^3 \cdot \phival^2 + \frac{\phival^3(\phival^3-1)}{2} \\
    &= \phival^5 + \frac{\phival^6 - \phival^3}{2} \\
    &\approx 11.09 + \frac{17.94 - 4.24}{2} \approx 17.9
\end{align}

And $\phival^7 \approx 29.0$, so the constraint is satisfied.

For $N = \phival^4 \approx 6.85$: LHS $\approx 50 > 29$, violating the constraint.

Thus $\phival^3 < N_{\text{WM}} < \phival^4$, and allowing for individual variation in parameters gives the range $[\phival^2, \phival^4]$.
\end{proof}

\begin{remark}
This result matches Cowan's ``magical number 4'' \cite{cowan2001}. Miller's ``$7 \pm 2$'' \cite{miller1956} includes chunking effects that effectively increase the unit size.
\end{remark}

\section{Sleep and Consolidation}

\subsection{Consolidation Rate by Sleep Stage}

Memory consolidation occurs during sleep when low-frequency oscillations enable hippocampal-cortical transfer.

\begin{definition}[Consolidation Rates]\label{def:consolidation}
The consolidation rate by sleep stage is:
\begin{align}
    \gamma_{\text{Wake}} &= 0 & &\text{(no consolidation)} \label{eq:gamma_wake}\\
    \gamma_{\text{Light}} &= \phival^{-2} \approx 0.382 & &\text{(N1/N2 sleep)} \label{eq:gamma_light}\\
    \gamma_{\text{REM}} &= \phival^{-1} \approx 0.618 & &\text{(REM sleep)} \label{eq:gamma_rem}\\
    \gamma_{\text{Deep}} &= 1 & &\text{(N3/SWS)} \label{eq:gamma_deep}
\end{align}
\end{definition}

The rates form a $\phival$-ladder: $\gamma_{\text{Deep}} : \gamma_{\text{REM}} : \gamma_{\text{Light}} = 1 : \phival^{-1} : \phival^{-2} = \phival^2 : \phival : 1$.

\begin{theorem}[Deep Sleep Optimality]\label{thm:deep}
Deep sleep (NREM stage 3/SWS) maximizes consolidation rate:
\begin{equation}
    \gamma_{\text{Deep}} > \gamma_{\text{REM}} > \gamma_{\text{Light}} > \gamma_{\text{Wake}}
\end{equation}
\end{theorem}

\begin{proof}
Since $\phival \approx 1.618 > 1$: $\phival^{-1} < 1$ and $\phival^{-2} < \phival^{-1}$. The ordering follows.
\end{proof}

\begin{corollary}[Deep/Light Ratio]\label{cor:sleep_ratio}
\begin{equation}
    \frac{\gamma_{\text{Deep}}}{\gamma_{\text{Light}}} = \phival^2 \approx 2.618
\end{equation}
\end{corollary}

This ratio is testable via polysomnography with targeted memory reactivation \cite{stickgold2005}.

\subsection{Consolidation Threshold}

\begin{proposition}[Minimum Strength for Consolidation]\label{prop:consol_thresh}
Memories require strength $S > \phival^{-1} \approx 0.618$ to be consolidated from working to long-term storage.
\end{proposition}

This explains why weak traces are forgotten during sleep rather than consolidated---they fall below the coherence threshold (Axiom~\ref{ax:threshold}).

\section{Learning Dynamics}

\subsection{Learning as Cost Reduction}

Learning modifies the Recognition Operator $\Rhat$ to reduce the cost of recognizing specific patterns, rather than ``storing'' information.

\begin{definition}[Learning Event]\label{def:learning}
A learning event is characterized by:
\begin{itemize}
    \item Attention level $a \in [0,1]$
    \item Repetition count $k \in \mathbb{N}$
    \item Spacing interval $\Delta t \geq 0$ since last exposure
\end{itemize}
\end{definition}

\begin{definition}[Learning Rate]\label{def:eta}
The learning rate is:
\begin{equation}\label{eq:learning}
    \eta(a, k, \Delta t) = a \cdot \phival^{-k} \cdot \left(1 + \frac{\ln(1 + \Delta t / 8\tauzero)}{\ln \phival}\right)
\end{equation}
\end{definition}

The $\phival^{-k}$ factor reflects diminishing returns from repetition. The logarithmic spacing bonus captures the empirical spacing effect.

\begin{theorem}[Spacing Effect]\label{thm:spacing}
Spaced practice with interval $\Delta t > 0$ produces greater learning than massed practice:
\begin{equation}
    \eta(a, k, \Delta t) > \eta(a, k, 0) \quad \forall \Delta t > 0
\end{equation}
\end{theorem}

\begin{proof}
The spacing bonus $\ln(1 + \Delta t/8\tauzero)/\ln\phival > 0$ for $\Delta t > 0$, since $\ln(1+x) > 0$ for $x > 0$.
\end{proof}

\begin{corollary}[Quantitative Spacing Advantage]\label{cor:spacing_quant}
\begin{itemize}
    \item $\Delta t = 8\tauzero$ (one WM cycle, $\sim 200$ ms): ratio $= 1 + \ln 2/\ln\phival \approx 2.44$
    \item $\Delta t = \taubreath$ ($\sim 25$ s): ratio $\approx 11$
    \item $\Delta t = 1$ hour: ratio $\approx 25$
    \item $\Delta t = 1$ day: ratio $\approx 35$
\end{itemize}
\end{corollary}

This is consistent with the dramatic advantage of distributed practice \cite{cepeda2006}.

\section{Trauma and PTSD}

\subsection{Traumatic Trace Characterization}

Traumatic memories exhibit distinctive features within the ledger model:

\begin{enumerate}
    \item \textbf{High emotional weight}: $\omega \geq \phival^{-1} \approx 0.618$
    \item \textbf{Ledger imbalance}: $|\beta| \geq 2\beta_0$ (involuntary recalls dominate)
    \item \textbf{Persistent strength}: $S \geq \phival^{-1}$ (resists decay)
\end{enumerate}

\begin{definition}[PTSD Threshold]\label{def:ptsd}
A trace enters a PTSD state when the ledger imbalance exceeds twice the interference scale:
\begin{equation}\label{eq:ptsd_threshold}
    |\beta| \geq 2\beta_0 = 2\phival^3 \approx 8.5
\end{equation}
equivalently, when interference cost exceeds complexity cost:
\begin{equation}
    \Jcost\left(1 + \frac{|\beta|}{\beta_0}\right) > C \cdot \Jcost(S)
\end{equation}
\end{definition}

\begin{proposition}[PTSD as Non-Equilibrium]\label{prop:ptsd_noneq}
In PTSD, the high interference cost prevents thermodynamic equilibration. The ledger imbalance creates a ``stuck'' high-free-energy state that cannot relax via normal forgetting.
\end{proposition}

Despite the emotional discount (which would normally enhance retention beneficially), the interference term dominates, causing distressing intrusions rather than adaptive memory.

\subsection{Therapeutic Mechanism}

The theory suggests that effective trauma therapy rebalances the ledger:
\begin{equation}
    \beta_{\text{new}} = \beta_{\text{old}} + n_{\text{controlled}}
\end{equation}
where controlled therapeutic exposures add credits to offset involuntary debit recalls.

\begin{proposition}[Exposure Therapy Rationale]\label{prop:exposure}
Reducing $|\beta|$ below $2\beta_0$ decreases the interference term, allowing normal forgetting dynamics to resume and the trace to reach equilibrium.
\end{proposition}

This is consistent with prolonged exposure therapy mechanisms \cite{foa2007}.

\section{Falsifiable Predictions}

\begin{prediction}[WM Capacity Range]\label{pred:wm}
Working memory capacity (unitary items, no chunking) lies in:
\begin{equation}
    N_{\text{WM}} \in [\phival^2, \phival^4] \approx [2.62, 6.85]
\end{equation}
with central tendency at $\phival^3 \approx 4.24$.
\end{prediction}

\begin{prediction}[Emotional Retention Advantage]\label{pred:emotion}
Emotional memories ($\omega = 1$) have time constant $\sigma_{\text{em}} = \phival \cdot \sigma_{\text{neutral}}$, yielding asymptotic retention advantage of factor $\phival \approx 1.618$.
\end{prediction}

\begin{prediction}[Logarithmic Spacing]\label{pred:spaced}
Learning advantage scales as $\log(\Delta t)$:
\begin{equation}
    \frac{\eta(\Delta t)}{\eta(0)} = 1 + \frac{\ln(1 + \Delta t/8\tauzero)}{\ln\phival}
\end{equation}
\end{prediction}

\begin{prediction}[Sleep Consolidation Ratio]\label{pred:sleep}
\begin{equation}
    \frac{\gamma_{\text{Deep}}}{\gamma_{\text{Light}}} = \phival^2 \approx 2.618
\end{equation}
\end{prediction}

\begin{prediction}[PTSD Threshold]\label{pred:ptsd}
PTSD symptom onset occurs when ledger imbalance $|\beta| \geq 2\phival^3 \approx 8.5$ intrusive recalls without controlled re-encoding.
\end{prediction}

\subsection{Falsification Conditions}

\begin{falsifier}[WM Out of Range]
If WM capacity (unitary, non-chunked items) is consistently $< 2$ or $> 8$ items, the theory is falsified.
\end{falsifier}

\begin{falsifier}[Emotional Decay Reversal]
If emotional memories consistently decay \textit{faster} than neutral memories (controlling for rehearsal), the theory is falsified.
\end{falsifier}

\begin{falsifier}[Sleep Stage Inversion]
If light sleep consolidates memories more effectively than deep sleep (controlling for duration), the theory is falsified.
\end{falsifier}

\begin{falsifier}[Non-Logarithmic Spacing]
If spacing advantage scales linearly rather than logarithmically with $\Delta t$, the logarithmic model is falsified.
\end{falsifier}

\begin{falsifier}[PTSD Threshold Mismatch]
If PTSD symptom onset consistently occurs at ledger imbalance $< 4$ or $> 15$, the threshold prediction is falsified.
\end{falsifier}

\section{Discussion}

\subsection{Relation to Existing Theories}

\textbf{ACT-R} \cite{anderson2004}: Our power-law decay (Corollary~\ref{cor:powerlaw}) matches ACT-R's base-level learning equation, but we derive it rather than assume it.

\textbf{Levels of Processing} \cite{craik1972}: Deeper processing corresponds to higher $\omega$, reducing $\Jmem$.

\textbf{Interference Theory} \cite{underwood1957}: The ledger balance term explicitly models retrieval-encoding imbalance.

\textbf{Consolidation Theory} \cite{mcgaugh2000}: The breath-cycle transfer maps to WM $\to$ LTM consolidation.

\textbf{Power-Law Forgetting} \cite{wixted1991}: Derived from our dynamics for $S \ll 1$.

\subsection{Novel Predictions}

\begin{enumerate}
    \item WM capacity clusters near $\phival^3 \approx 4.24$
    \item Emotional retention advantage is precisely $\phival \approx 1.618$
    \item Deep/light sleep consolidation ratio is $\phival^2 \approx 2.618$
    \item PTSD onset threshold at $|\beta| \approx 8-9$ intrusions
    \item Spacing advantage is logarithmic, not linear
    \item Short-time forgetting is exponential; long-time is power-law $\propto t^{-1/3}$
\end{enumerate}

\subsection{Limitations}

The theory does not address:
\begin{itemize}
    \item Neural implementation of the cost functional
    \item Individual differences in $\phival$-parameters
    \item Semantic vs. episodic memory distinctions
    \item False memory formation
    \item Reconsolidation effects
    \item Age-related memory decline
\end{itemize}

These are directions for future work.

\section{Conclusion}

We have presented a thermodynamic theory of memory in which retention and forgetting emerge from cost minimization. The framework:

\begin{enumerate}
    \item \textbf{Derives} $\Jcost$ uniquely from the d'Alembert equation
    \item \textbf{Derives} $\phival$ from discrete self-similarity
    \item \textbf{Constructs} $\Jmem$ via principled dimensional analysis
    \item \textbf{Predicts} quantitative values ($\phival^3$ capacity, $\phival^2$ sleep ratio)
    \item \textbf{Unifies} exponential and power-law forgetting
    \item \textbf{Explains} emotional memory, spacing effects, PTSD
    \item \textbf{Provides} falsifiable predictions with explicit thresholds
\end{enumerate}

The key insight is that memory is not storage---it is a cost-minimizing dynamical system. Remembering is maintaining low-cost configurations; forgetting is relaxation toward thermodynamic equilibrium.

\section*{Acknowledgments}

The author thanks the Recognition Science research community for discussions on thermodynamic extensions.

\section*{Data Availability}

Lean 4 formalization available at: \texttt{github.com/recognition-science/reality}

\bibliographystyle{apalike}
\begin{thebibliography}{99}

\bibitem[Anderson et al.(2004)]{anderson2004}
Anderson, J.~R., Bothell, D., Byrne, M.~D., Douglass, S., Lebiere, C., \& Qin, Y. (2004).
\newblock An integrated theory of the mind.
\newblock \textit{Psychological Review}, 111(4), 1036--1060.

\bibitem[Atkinson \& Shiffrin(1968)]{atkinson1968}
Atkinson, R.~C., \& Shiffrin, R.~M. (1968).
\newblock Human memory: A proposed system and its control processes.
\newblock In \textit{Psychology of Learning and Motivation} (Vol. 2, pp. 89--195). Academic Press.

\bibitem[Attwell \& Laughlin(2001)]{attwell2001}
Attwell, D., \& Laughlin, S.~B. (2001).
\newblock An energy budget for signaling in the grey matter of the brain.
\newblock \textit{Journal of Cerebral Blood Flow \& Metabolism}, 21(10), 1133--1145.

\bibitem[Brewin et al.(2010)]{brewin2010}
Brewin, C.~R., Gregory, J.~D., Lipton, M., \& Burgess, N. (2010).
\newblock Intrusive images in psychological disorders.
\newblock \textit{Psychological Review}, 117(1), 210--232.

\bibitem[Brown \& Kulik(1977)]{brown1977}
Brown, R., \& Kulik, J. (1977).
\newblock Flashbulb memories.
\newblock \textit{Cognition}, 5(1), 73--99.

\bibitem[Brown et al.(2007)]{brown2007}
Brown, G.~D., Neath, I., \& Chater, N. (2007).
\newblock A temporal ratio model of memory.
\newblock \textit{Psychological Review}, 114(3), 539--576.

\bibitem[Cepeda et al.(2006)]{cepeda2006}
Cepeda, N.~J., Pashler, H., Vul, E., Wixted, J.~T., \& Rohrer, D. (2006).
\newblock Distributed practice in verbal recall tasks.
\newblock \textit{Psychological Bulletin}, 132(3), 354--380.

\bibitem[Cowan(2001)]{cowan2001}
Cowan, N. (2001).
\newblock The magical number 4 in short-term memory.
\newblock \textit{Behavioral and Brain Sciences}, 24(1), 87--114.

\bibitem[Craik \& Lockhart(1972)]{craik1972}
Craik, F.~I., \& Lockhart, R.~S. (1972).
\newblock Levels of processing.
\newblock \textit{Journal of Verbal Learning and Verbal Behavior}, 11(6), 671--684.

\bibitem[Ebbinghaus(1885)]{ebbinghaus1885}
Ebbinghaus, H. (1885).
\newblock \textit{{\"U}ber das Ged{\"a}chtnis}.
\newblock Leipzig: Duncker \& Humblot.

\bibitem[Foa et al.(2007)]{foa2007}
Foa, E.~B., Hembree, E.~A., \& Rothbaum, B.~O. (2007).
\newblock \textit{Prolonged Exposure Therapy for PTSD}.
\newblock Oxford University Press.

\bibitem[McGaugh(2000)]{mcgaugh2000}
McGaugh, J.~L. (2000).
\newblock Memory---a century of consolidation.
\newblock \textit{Science}, 287(5451), 248--251.

\bibitem[McGaugh(2004)]{mcgaugh2004}
McGaugh, J.~L. (2004).
\newblock The amygdala modulates consolidation of emotionally arousing memories.
\newblock \textit{Annual Review of Neuroscience}, 27, 1--28.

\bibitem[Miller(1956)]{miller1956}
Miller, G.~A. (1956).
\newblock The magical number seven, plus or minus two.
\newblock \textit{Psychological Review}, 63(2), 81--97.

\bibitem[Raaijmakers \& Shiffrin(1981)]{raaijmakers1981}
Raaijmakers, J.~G., \& Shiffrin, R.~M. (1981).
\newblock Search of associative memory.
\newblock \textit{Psychological Review}, 88(2), 93--134.

\bibitem[Roediger \& Karpicke(2006)]{roediger2006}
Roediger, H.~L., \& Karpicke, J.~D. (2006).
\newblock Test-enhanced learning.
\newblock \textit{Psychological Science}, 17(3), 249--255.

\bibitem[Stickgold(2005)]{stickgold2005}
Stickgold, R. (2005).
\newblock Sleep-dependent memory consolidation.
\newblock \textit{Nature}, 437(7063), 1272--1278.

\bibitem[Underwood(1957)]{underwood1957}
Underwood, B.~J. (1957).
\newblock Interference and forgetting.
\newblock \textit{Psychological Review}, 64(1), 49--60.

\bibitem[Washburn(2025)]{washburn2025rs}
Washburn, J. (2025).
\newblock Recognition Science: A cost-functional approach to physics.
\newblock \textit{Recognition Science Technical Reports}.

\bibitem[Wixted \& Ebbesen(1991)]{wixted1991}
Wixted, J.~T., \& Ebbesen, E.~B. (1991).
\newblock On the form of forgetting.
\newblock \textit{Psychological Science}, 2(6), 409--415.

\end{thebibliography}

\appendix

\section{Lean 4 Formalization}

Key structures from the Mathlib-based formalization:

\begin{verbatim}
structure LedgerMemoryTrace where
  complexity : Real
  emotional_weight : Real  -- in [0,1]
  strength : Real          -- in (0,1]
  encoding_tick : Nat
  ledger_balance : Int
  consolidated : Bool

def phi : Real := (1 + Real.sqrt 5) / 2
def beta_0 : Real := phi ^ 3

noncomputable def emotional_discount 
    (omega : Real) : Real :=
  1 - omega * (1 - 1/phi)

noncomputable def memory_cost 
    (trace : LedgerMemoryTrace) 
    (t : Nat) : Real := 
  let eps := emotional_discount trace.emotional_weight
  let age := (t - trace.encoding_tick : Real) / breath_cycle
  eps * (trace.complexity * Jcost trace.strength
       + Jcost (age + 1)
       + Jcost (1 + (abs trace.ledger_balance) / beta_0))

theorem miller_law : 
  phi^2 <= working_memory_capacity 
    /\ working_memory_capacity <= phi^4 := by
  -- proof via attention budget constraint
  sorry

theorem emotional_reduces_cost (h : omega1 > omega2) :
  memory_cost trace1 t < memory_cost trace2 t := by
  -- follows from emotional_discount decreasing
  sorry

theorem spacing_advantage (h : delta_t > 0) :
  learning_rate a k delta_t > learning_rate a k 0 := by
  -- follows from ln(1+x) > 0 for x > 0
  sorry

theorem powerlaw_decay (h : S << 1) :
  -- exists alpha > 0, S t approx t^(-alpha)
  True := by trivial
\end{verbatim}

\section{Derivation Details}

\subsection{J-Cost from d'Alembert}

The functional equation $\Jcost(xy) + \Jcost(x/y) = 2[\Jcost(x)+1][\Jcost(y)+1] - 2$ with $\Jcost(1) = 0$ has general solution $\Jcost(x) = \cosh(\alpha \ln x) - 1$.

Setting $\Jcost''(1) = 1$: Since $\Jcost(x) = \cosh(\alpha \ln x) - 1$:
\begin{align}
    \Jcost'(x) &= \alpha \sinh(\alpha \ln x) / x \\
    \Jcost''(x) &= \alpha^2 \cosh(\alpha \ln x) / x^2 - \alpha \sinh(\alpha \ln x) / x^2
\end{align}
At $x = 1$: $\Jcost''(1) = \alpha^2 \cdot 1 - 0 = \alpha^2 = 1$, so $\alpha = 1$.

\subsection{Power-Law Exponent}

From $\dd S/\dd t \approx C\epsilon/(2\TR S^2)$ for $S \ll 1$:
\begin{equation}
    S^2 \dd S = \frac{C\epsilon}{2\TR} \dd t
\end{equation}
Integrating: $S^3/3 = C\epsilon t/(2\TR) + \text{const}$, so $S \propto t^{1/3}$.

For retention $R = S/S_0$, this gives $R \propto t^{-1/3}$---a power law with exponent $-1/3$.

\end{document}
