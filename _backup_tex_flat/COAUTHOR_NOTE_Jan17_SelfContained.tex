\documentclass[11pt]{article}

\usepackage[T1]{fontenc}
\usepackage[utf8]{inputenc}
\usepackage{lmodern}
\usepackage{microtype}
\usepackage[margin=1in]{geometry}
\usepackage{amsmath,amssymb}
\usepackage{booktabs}
\usepackage{xcolor}
\usepackage{hyperref}
\usepackage{fancyhdr}

% Colors
\definecolor{RSBlue}{RGB}{25,55,95}
\definecolor{RSGold}{RGB}{180,140,50}

% Hyperref setup
\hypersetup{
  colorlinks=true,
  linkcolor=RSBlue,
  urlcolor=RSBlue,
  pdftitle={Co-Author Note (Self-Contained): Derivations Previously Backed by Lean},
  pdfauthor={Internal (RS co-authors)}
}

% Header/footer
\pagestyle{fancy}
\fancyhf{}
\fancyhead[L]{\textcolor{RSBlue}{\textsc{Recognition Science --- Internal}}}
\fancyhead[R]{\textcolor{gray}{\small Co-Author Note (Self-Contained)}}
\fancyfoot[C]{\thepage}
\renewcommand{\headrulewidth}{0.4pt}
\setlength{\headheight}{14pt}

% Notation
\newcommand{\R}{\mathbb{R}}
\newcommand{\N}{\mathbb{N}}
\newcommand{\phiGR}{\varphi}
\newcommand{\Jcost}{J}
\newcommand{\alphamem}{\alpha}
\newcommand{\alphatotal}{\widetilde{\alpha}}

\begin{document}

\begin{center}
{\Large\bfseries\color{RSBlue} CO-AUTHOR NOTE (SELF-CONTAINED)}\\[0.4em]
{\large Derivations formerly referenced via Lean (now written out)}\\[0.9em]
\begin{tabular}{@{}rl@{}}
\textbf{Date:} & January 17, 2026 \\
\textbf{Paper:} & \emph{Toward a Discrete Informational Framework for Classical Gravity} (v7) \\
\textbf{Audience:} & Co-author without Lean repository access \\
\end{tabular}
\end{center}

\vspace{0.75em}
\hrule
\vspace{1.0em}

\section*{Executive Summary}

This memo is intentionally \textbf{self-contained}: every claim that was previously justified by pointing to Lean code is rewritten below as an explicit derivation or proof.

\begin{itemize}
\item \textbf{Dimension forcing (D=3):} Under explicit hypotheses (8-tick coverage and compatibility conditions), \(D=3\) is the \emph{unique} solution. The paper’s caution is correct: the \emph{physical necessity} of the 45-tick cycle remains a separate physical argument, even if a plausible motivation exists.
\item \textbf{Scale-free latency vs ILG:} The manuscript shows scale-free latency is \emph{sufficient} for the ILG form; it should explain more clearly why it is plausibly \emph{necessary} (or label the argument as sufficiency-only).
\item \textbf{Factor-of-two ambiguity:} The paper’s \(\alphamem \approx 0.191\) and some notes’ \(\alphatotal \approx 0.382\) differ by exactly a factor of two: \(\alphatotal = 2\alphamem = 1-\phiGR^{-1}\). The \(1/2\) is not a convention; it comes from the two-subloop decomposition in the two-scale self-similarity argument.
\item \textbf{``137'' optics:} The parameter-free electromagnetic derivation is most cleanly expressed as
\[
\alpha^{-1} \;=\; 4\pi\cdot 11 \;-\; w_8\ln(\phiGR) \;-\; \kappa,
\]
with \(w_8\) and \(\kappa\) defined by closed-form, parameter-free expressions. This avoids any impression that ``137'' is an input.
\end{itemize}

\section{Dimension Forcing: A Self-Contained Proof Sketch}

\subsection{Hypotheses (explicit)}
We separate the \emph{mathematical uniqueness} statement (conditional theorem) from the \emph{physical claim} (Nature satisfies the hypotheses).

\paragraph{H1 (8-tick coverage law).}
In \(D\) spatial dimensions, the minimal ``ledger coverage'' period is \(2^D\). We denote this period by
\[
T(D) := 2^D.
\]
\emph{Interpretation:} each tick resolves one of the \(2^D\) binary orthants/corners of a \(D\)-dimensional sign/cube structure; ledger neutrality requires a full traversal.

\paragraph{H2 (8-tick empirical/structural requirement).}
The ledger-neutrality period is \(8\). That is,
\[
T(D) = 8.
\]

\paragraph{H3 (Gap-45 synchronization condition; conjectural physical input).}
There exists a second cycle of period \(45\) that must synchronize with the 8-tick cycle; equivalently,
\[
\mathrm{lcm}(8,45)=360,
\]
and the combined structure is compatible with physical rotations/closure. (This is precisely the point the paper treats cautiously; see Remark A.4.)

\subsection{Lemma 1: \(\mathrm{lcm}(8,45)=360\)}
Compute \(\gcd(8,45)=1\) since \(8=2^3\) and \(45=3^2\cdot 5\) share no prime factors. Therefore
\[
\mathrm{lcm}(8,45) = \frac{8\cdot 45}{\gcd(8,45)} = 8\cdot 45 = 360.
\]

\subsection{Lemma 2: If \(2^D=8\) for \(D\in\N\), then \(D=3\)}
Since \(8=2^3\), the equation \(2^D=8\) is equivalent to \(2^D=2^3\). The function \(n\mapsto 2^n\) is strictly increasing on \(\N\), hence injective; therefore \(D=3\).

\paragraph{Elementary proof without ``injective exponentiation''.}
If \(D\le 2\), then \(2^D\le 4<8\). If \(D\ge 4\), then \(2^D\ge 16>8\). Hence the only possibility is \(D=3\).

\subsection{Theorem (conditional uniqueness): \(D=3\) is forced under H1--H2}
\textbf{Theorem.} Assume H1--H2. Then the unique dimension \(D\in\N\) satisfying \(T(D)=8\) is \(D=3\).

\textbf{Proof.}
Existence: \(D=3\) satisfies \(T(3)=2^3=8\).
Uniqueness: if \(T(D)=8\), then \(2^D=8\), hence \(D=3\) by Lemma 2. \(\square\)

\subsection{Why the paper’s caution about ``45'' is correct}
The theorem above is a \emph{conditional} mathematical statement: it is only as strong as H1--H2 (and any additional structure bundled into ``RS-compatible''). The manuscript’s Remark A.4 correctly notes that the \emph{physical necessity} of the 45-tick cycle is not yet established at the same level of inevitability; it remains a motivated conjecture (even if a candidate physical explanation is offered).

\section{Gap-45: Physical Motivation (Write-up)}

The motivating idea can be expressed without any code:

\subsection{Step 1: Closure principle (``fence-post'')}
An 8-tick traversal is not automatically a \emph{closed} loop in state space. To return to the initial phase/state after 8 transitions, one needs a closure step, giving \(8+1=9\) states/marks for a closed cycle.

\subsection{Step 2: Cumulative phase as a triangular sum}
Assume each tick \(k\) contributes a phase increment proportional to \(k\) (linear accumulation). Then the total phase over \(n\) steps is
\[
\sum_{k=1}^{n} k \;=\; \frac{n(n+1)}{2}.
\]
For the closed 8-tick cycle, \(n=9\), hence
\[
T(9) = \frac{9\cdot 10}{2} = 45.
\]

\subsection{Step 3: Synchronization}
Requiring compatibility of an 8-cycle and a 45-cycle forces a synchronization period \(\mathrm{lcm}(8,45)=360\) (Lemma 1).

\paragraph{Status.}
This is a \emph{plausible physical story} for why 45 appears. Whether it is \emph{necessary} remains the open point the paper correctly marks.

\section{Concern: Scale-Free Latency is Necessary vs.\ Merely Sufficient}

\subsection*{Concern (as requested)}
\textbf{Concern:} The paper could more explicitly discuss why scale-free latency is \emph{necessary} rather than merely \emph{sufficient} for the ILG form.

\subsection*{Expanded explanation}
The manuscript argues
\[
\text{scale-free closure latency} \;\Rightarrow\; \text{fractional memory} \;\Rightarrow\; \text{ILG kernel}.
\]
But a reader may ask why the closure process cannot contain a characteristic time scale \(\tau_\ast\).

\paragraph{Why this matters.}
If \(\tau_\ast\) exists, the memory kernel typically develops a crossover: power-law behavior for \(t\ll \tau_\ast\) and a cutoff or exponential suppression for \(t\gg \tau_\ast\). That would (i) introduce at least one new free parameter, and (ii) generally spoil the scale-free behavior that motivates ILG.

\paragraph{Recommended manuscript insertion.}
Add a short paragraph stating that (a) RS aims to avoid injecting an \emph{ad hoc} physical scale into closure latency, and (b) the ILG power-law form is the stable scale-invariant fixed point of closure dynamics; non-scale-free latencies generically imply crossover kernels.

\section{Factor-of-Two Ambiguity in \texorpdfstring{$\alpha$}{alpha}: Full Resolution}

\subsection{Definitions}
Let \(\phiGR\) be the golden ratio:
\[
\phiGR = \frac{1+\sqrt{5}}{2}, \qquad \phiGR^2 = \phiGR + 1, \qquad \phiGR^{-1} = \phiGR - 1.
\]
Define the \emph{fractional-memory exponent} (paper’s \(\alphamem\)) by
\[
\alphamem := \frac{1-\phiGR^{-1}}{2}.
\]
Define the \emph{acceleration-parameterized} exponent (some notes’ \(\alphatotal\)) by
\[
\alphatotal := 1-\phiGR^{-1}.
\]

\subsection{Algebraic identity}
Immediately,
\[
\alphatotal = 1-\phiGR^{-1} = 2\cdot \frac{1-\phiGR^{-1}}{2} = 2\alphamem.
\]
Numerically, \(\phiGR^{-1}\approx 0.618\), so \(\alphatotal\approx 0.382\) and \(\alphamem\approx 0.191\).

\subsection{Physical origin of the factor \(1/2\)}
The factor \(1/2\) is not a convention; it arises from the two-scale decomposition used in the self-similarity argument:
\begin{itemize}
\item a loop at scale \(s\) decomposes into two sub-loops at scales \(1\) and \(1/s\),
\item the ``incomplete fraction'' is \(1-1/s\),
\item self-similarity forces \(s=\phiGR\) (see next section),
\item and the exponent is shared equally by the two sub-loops, hence \(2\alphamem = 1-\phiGR^{-1}\).
\end{itemize}

\section{Self-Similarity Forces \texorpdfstring{$\alpha$}{alpha}: A Worked Derivation}

\subsection{Step 1: Two-scale decomposition forces \(s=\phiGR\)}
Let \(s>0\) be the scale ratio between a loop and its decomposition. The paper’s self-similarity model asserts a loop at scale \(s\) decomposes into two loops with relative scales \(1\) and \(1/s\). The scale-additivity condition is
\[
s = 1 + \frac{1}{s}.
\]
Multiplying by \(s\) yields
\[
s^2 = s + 1.
\]
The positive solution is \(s=\phiGR\).

\subsection{Step 2: Exponent constraint}
The ``incomplete fraction'' per loop is \(1 - 1/s\). With \(s=\phiGR\), this becomes
\[
1 - \frac{1}{\phiGR}.
\]
Because the decomposition has \emph{two} sub-loops that contribute symmetrically, the fractional-memory exponent per sub-loop is half of the total:
\[
2\alphamem = 1-\phiGR^{-1}
\quad\Longrightarrow\quad
\alphamem = \frac{1-\phiGR^{-1}}{2}.
\]
This is exactly the value used in the paper (\(\alphamem\approx 0.191\)).

\section{Electromagnetic \texorpdfstring{$\alpha^{-1}$}{alpha^{-1}} Derivation: Explicit Formula (No ``137'' Input)}

\subsection{Canonical parameter-free decomposition}
Write the inverse fine-structure constant as
\[
\alpha^{-1} \;=\; \alpha_{\mathrm{seed}} \;-\; \bigl(f_{\mathrm{gap}} + \delta_\kappa\bigr).
\]
Each term is specified by an explicit, parameter-free formula:
\begin{align*}
\alpha_{\mathrm{seed}} &:= 4\pi\cdot 11, \\
f_{\mathrm{gap}} &:= w_8\ln(\phiGR), \\
\delta_\kappa &:= -\frac{103}{102\,\pi^5}.
\end{align*}

\subsection{Interpretation of the terms}
\paragraph{\(\alpha_{\mathrm{seed}} = 4\pi\cdot 11\).}
This is ``total solid angle'' \((4\pi)\) times the \emph{passive-edge count} \((11)\) of the 3-cube \(Q_3\). The cube has \(12\) edges; the RS bookkeeping distinguishes one ``active'' edge (recognition channel) and \(11\) passive edges contributing to baseline coupling/closure.

\paragraph{\(f_{\mathrm{gap}} = w_8\ln(\phiGR)\).}
The gap term is a projection weight \(w_8\) onto the 8-tick basis, multiplied by the elementary bit-cost \(\ln(\phiGR)\). The weight is given in closed form:
\[
w_8 \;:=\; \frac{348 + 210\sqrt{2} - (204 + 130\sqrt{2})\,\phiGR}{7}.
\]
(Numerically \(w_8\approx 2.49056927545\).)

\paragraph{\(\delta_\kappa = -103/(102\pi^5)\).}
This is a curvature correction representing the mismatch between spherical and voxel/cubic boundary seam counting; the ratio \(103/102\) is a crystallographic/wallpaper-group count (\(6\) faces \(\times\) \(17\) groups \(=102\)) plus one corrective seam, and \(\pi^5\) arises from the geometry normalization used in the curvature term.

\subsection{Numerical sanity check (optional)}
Compute:
\[
\alpha_{\mathrm{seed}} = 44\pi \approx 138.230076757.
\]
Also \(\ln(\phiGR)\approx 0.481211825\), so with \(w_8\approx 2.490569275\),
\[
f_{\mathrm{gap}} \approx 2.490569275 \times 0.481211825 \approx 1.198\ (\text{approx}).
\]
Finally,
\[
\delta_\kappa = -\frac{103}{102\pi^5} \approx -0.00330,
\]
hence \(f_{\mathrm{gap}}+\delta_\kappa \approx 1.198 - 0.0033 \approx 1.195\). Therefore
\[
\alpha^{-1} \approx 138.2301 - 1.195 \approx 137.035,
\]
consistent with \(\alpha^{-1}\approx 137.036\) (CODATA) within the displayed rounding.

\section{Post-Hoc vs.\ Predictive: How to State the Logic Cleanly}

\subsection{The issue}
Fitting \((A,\alphamem,r_0)\) to SPARC and then observing agreement with \((\phiGR^{-2},(1-\phiGR^{-1})/2,\dots)\) is \emph{post-hoc} unless the prediction is logically prior to the fit.

\subsection{Recommended statement (explicit separation)}
\begin{quote}
\textbf{Theory (no galaxy data):} derive \(\phiGR\) from self-similarity and then derive \(\alphamem=(1-\phiGR^{-1})/2\) and \(C=\phiGR^{-2}\).\\
\textbf{Observation (galaxy data only):} fit \((A,\alphamem,r_0)\) freely on SPARC.\\
\textbf{Comparison:} check whether the fitted values are within uncertainty of the predicted values.
\end{quote}
This is the minimum logical separation required to prevent a ``post-hoc'' critique.

\section*{Action Items (Manuscript-Level)}
\begin{itemize}
\item Add a paragraph clarifying why scale-free latency is plausibly necessary (or label the argument as sufficiency-only).
\item Add a short boxed note in \S III.E: \(\alphatotal = 2\alphamem\) and explain the two-subloop origin.
\item Keep Remark A.4 cautious: the 45-tick physical necessity is not fully compelled (even if motivated).
\item For ``137 optics'': emphasize the explicit \(4\pi\cdot 11\) seed and corrections; avoid phrasing that treats ``137'' as an input.
\end{itemize}

\vspace{1.0em}
\hrule
\vspace{0.75em}

\noindent\textit{Internal note:} The repository contains formalizations of many of these steps; this memo is written so the co-author can review the logic without repository access.

\end{document}

