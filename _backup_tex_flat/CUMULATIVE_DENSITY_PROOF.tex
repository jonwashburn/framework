\documentclass[11pt]{article}
\usepackage{amsmath,amssymb,amsthm}
\usepackage[margin=1in]{geometry}

\newtheorem{theorem}{Theorem}
\newtheorem{lemma}[theorem]{Lemma}
\newtheorem{proposition}[theorem]{Proposition}
\newtheorem{corollary}[theorem]{Corollary}
\theoremstyle{remark}
\newtheorem{remark}[theorem]{Remark}

\newcommand{\R}{\mathbb{R}}
\newcommand{\C}{\mathbb{C}}

\title{The Cumulative Density Argument:\\
Global Energy Constraints on Off-Line Zeros}
\author{Recognition Physics Institute}
\date{January 1, 2026}

\begin{document}
\maketitle

\begin{abstract}
We develop a global energy argument showing that the total ``off-line cost'' of 
all zeros is bounded by the prime layer energy. Combined with individual lower 
bounds on off-line zero costs, this severely constrains the number and distribution 
of off-line zeros.
\end{abstract}

\section{Setup}

\subsection{The Energy Framework}

Define the \textbf{Dirichlet energy} of $\log|\xi|$ in the half-strip:
\[
E(T) = \iint_{\Omega_T} |\nabla \log|\xi||^2 \, \sigma \, d\sigma \, dt
\]
where $\Omega_T = \{1/2 < \sigma < 1, \; 0 < t < T\}$.

This energy has contributions from:
\begin{enumerate}
\item \textbf{Prime layer}: $E_{\rm prime}(T)$ from the Euler product
\item \textbf{Zeros}: $E_{\rm zeros}(T)$ from logarithmic singularities
\end{enumerate}

\subsection{The Prime Layer Energy}

\begin{lemma}[Prime Energy Bound]\label{lem:prime-energy}
The prime layer contributes:
\[
E_{\rm prime}(T) \leq C_{\rm prime} \cdot T
\]
where $C_{\rm prime}$ is a constant derived from Mertens' theorem.
\end{lemma}

\begin{proof}[Sketch]
The prime layer potential is:
\[
U_{\rm prime}(\sigma, t) = \sum_p \frac{\log p}{p^\sigma} \cos(t \log p)
\]
The gradient squared integrates to give energy proportional to $T$.
\end{proof}

\subsection{The Zero Energy}

\begin{lemma}[On-Line Zero Energy]\label{lem:online-energy}
An on-line zero at height $\gamma$ contributes finite energy:
\[
E_{\rm on}(\gamma) \leq C_{\rm on}
\]
This is because the singularity is on the boundary of $\Omega_T$, and the half-disk 
regularization gives a finite contribution.
\end{lemma}

\begin{lemma}[Off-Line Zero Energy]\label{lem:offline-energy}
An off-line zero at depth $\eta > 0$ and height $\gamma$ contributes:
\[
E_{\rm off}(\eta, \gamma) \geq L_{\rm rec} + |\log(2\eta)|
\]
where $L_{\rm rec} = 4\arctan 2 \approx 4.43$ is the Blaschke trigger.
\end{lemma}

\section{The Global Constraint}

\begin{theorem}[Energy Balance]\label{thm:balance}
The total energy satisfies:
\[
E_{\rm prime}(T) \geq \sum_{\text{on-line}, |\gamma| \leq T} E_{\rm on}(\gamma) 
+ \sum_{\text{off-line}, |\gamma| \leq T} E_{\rm off}(\eta_\rho, \gamma)
\]
\end{theorem}

\begin{corollary}[Off-Line Zero Bound]\label{cor:offbound}
Let $N_{\rm off}(T)$ be the number of off-line zeros up to height $T$. Then:
\[
N_{\rm off}(T) \cdot L_{\rm rec} \leq E_{\rm prime}(T) \leq C_{\rm prime} \cdot T
\]
which gives:
\[
N_{\rm off}(T) \leq \frac{C_{\rm prime}}{L_{\rm rec}} \cdot T \approx 0.044 \cdot T
\]
\end{corollary}

\section{The Density Improvement}

\begin{theorem}[Fraction Bound]\label{thm:fraction}
The fraction of off-line zeros among all zeros satisfies:
\[
\frac{N_{\rm off}(T)}{N(T)} \leq \frac{0.044 \cdot T}{(T/2\pi) \log T} = \frac{0.28}{\log T} \to 0
\]
as $T \to \infty$.
\end{theorem}

This proves that \textbf{almost all zeros are on the line}, but not that \textbf{all} 
zeros are on the line.

\section{Attempting to Close the Gap}

\subsection{The Deep Off-Line Constraint}

\begin{lemma}[Coulomb Enhancement]\label{lem:coulomb-enhance}
If an off-line zero has depth $\eta_\rho$, its energy contribution is at least:
\[
E_{\rm off}(\eta_\rho) \geq 4.43 + |\log(2\eta_\rho)|
\]
For $\eta_\rho < 0.5$, this is at least $4.43 + 0.69 = 5.12$.
\end{lemma}

This doesn't fundamentally change the bound; we still get $N_{\rm off} = O(T)$.

\subsection{The Depth-Weighted Bound}

\begin{theorem}[Weighted Constraint]\label{thm:weighted}
Define $\Delta(T) = \sum_{\text{off-line}, |\gamma| \leq T} |\log(2\eta_\rho)|$.
Then:
\[
\Delta(T) \leq E_{\rm prime}(T) - N_{\rm off}(T) \cdot L_{\rm rec} \leq C_{\rm prime} \cdot T
\]
\end{theorem}

This shows that the \textbf{total Coulomb cost} is bounded by $O(T)$.

\subsection{What Would Give RH}

\begin{remark}[The Missing Ingredient]
To prove $N_{\rm off} = 0$, we would need either:
\begin{enumerate}
\item $E_{\rm prime}(T) = o(T)$, which is false.
\item $E_{\rm off}(\eta) \to \infty$ uniformly, which only happens as $\eta \to 0$.
\item A structural constraint that makes even one off-line zero impossible.
\end{enumerate}
\end{remark}

\section{The Honest Conclusion}

The cumulative density argument proves:
\begin{center}
\fbox{\parbox{0.8\textwidth}{
\textbf{Theorem (Density of On-Line Zeros)}\\
The proportion of zeros on the critical line approaches 1:
\[
\lim_{T \to \infty} \frac{N_{\rm on}(T)}{N(T)} = 1
\]
}}
\end{center}

This is \textbf{not} equivalent to RH, which requires $N_{\rm off}(T) = 0$ for all $T$.

\textbf{Gap remaining}: The argument allows $N_{\rm off}(T) \sim cT$ off-line zeros, 
as long as $c < 0.044$.

\section{What Would Close the Gap}

\begin{enumerate}
\item \textbf{Zero-density exponent improvement}: If we could prove $N(\sigma, T) = O(T^{\epsilon})$ 
for any $\sigma > 1/2$ and any $\epsilon > 0$, this would bound $N_{\rm off} = o(T)$.

\item \textbf{Individual lower bound improvement}: If we could prove each off-line 
zero costs at least $c \cdot T^\epsilon$ for some $\epsilon > 0$, the energy bound 
would give $N_{\rm off} = O(T^{1-\epsilon})$, which combined with density theorems 
might give a contradiction.

\item \textbf{Structural constraint}: Show that the Euler product or functional 
equation directly forbids even one off-line zero.
\end{enumerate}

\end{document}


