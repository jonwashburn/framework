\documentclass[11pt]{article}

\usepackage[margin=1in]{geometry}
\usepackage[T1]{fontenc}
\usepackage[utf8]{inputenc}
\usepackage{lmodern}
\usepackage{microtype}
\usepackage{amsmath,amssymb,amsthm,mathtools}
\usepackage[colorlinks=true,linkcolor=blue,citecolor=blue,urlcolor=blue]{hyperref}
\usepackage[nameinlink]{cleveref}
\usepackage{booktabs}
\usepackage{enumitem}
\setlist{nosep}

\title{Audit Report: \texttt{Draft\_v1.tex} Theorem Inventory and Lean Mapping}
\author{Automated audit (Cursor agent)}
\date{\today}

\begin{document}
\maketitle

\section{Scope}
This report audits \texttt{Draft\_v1.tex} by:
\begin{itemize}
  \item identifying every \texttt{theorem}/\texttt{proposition}/\texttt{corollary} environment with \textbf{exact line ranges},
  \item mapping each statement to a corresponding Lean theorem in this repository (when present),
  \item flagging missing or only-partially-formalized statements for follow-up.
\end{itemize}

\paragraph{Important note on ``matches''.}
This audit treats a match as \emph{statement-level equivalence} (possibly up to notational translation),
not merely ``a theorem with a similar conclusion''.

\section{Inventory (12 formal statements)}
\begin{enumerate}
  \item \textbf{Theorem} (\texttt{\textbackslash label\{thm:main\}}) \emph{Dimensional Rigidity in Recognition Geometry} \\
  \textbf{Location:} \texttt{Draft\_v1.tex} lines 105--113 \\
  \textbf{Statement:} If the recognition quotient $\mathcal{C}_R$ satisfies constraints (T), (K), (S) then $\dim(\mathcal{C}_R)=3$. \\
  \textbf{Lean mapping (current):} \emph{partial}. A paper-companion module was added:
  \texttt{IndisputableMonolith.Papers.DraftV1.dimensional\_rigidity\_main}
  (in \texttt{IndisputableMonolith/Papers/DraftV1.lean}), which provides a forward-direction
  $D=3$ forcing result using the formalized (K) algebraic core. The full (T/K/S) structure
  (especially Alexander duality and linking invariants) is not yet formalized. \\
  \textbf{Status:} \textbf{PARTIAL}.

  \item \textbf{Theorem} \emph{Injectivity of Observable Map} \\
  \textbf{Location:} \texttt{Draft\_v1.tex} lines 177--179 \\
  \textbf{Statement:} The induced map $\overline{R}:\mathcal{C}_R\to\mathcal{E}$ with $\overline{R}([c]_R)=R(c)$ is injective. \\
  \textbf{Lean mapping:} \texttt{IndisputableMonolith.RecogGeom.Quotient.quotientEventMap\_injective} (in \texttt{IndisputableMonolith/RecogGeom/Quotient.lean}). \\
  \textbf{Status:} \textbf{MATCH}.

  \item \textbf{Theorem} \emph{Refinement} \\
  \textbf{Location:} \texttt{Draft\_v1.tex} lines 202--204 \\
  \textbf{Statement:} The quotient $\mathcal{C}_{R_1\otimes R_2}$ refines $\mathcal{C}_{R_1}$ and $\mathcal{C}_{R_2}$. \\
  \textbf{Lean mapping:} \texttt{IndisputableMonolith.RecogGeom.Composition.refinement\_theorem} (in \texttt{IndisputableMonolith/RecogGeom/Composition.lean}). \\
  \textbf{Status:} \textbf{MATCH}.

  \item \textbf{Theorem} (\texttt{\textbackslash label\{thm:alexander\}}) \emph{Alexander Duality for Recognition Quotients} \\
  \textbf{Location:} \texttt{Draft\_v1.tex} lines 244--251 \\
  \textbf{Statement:} For an embedded circle $K\subset\mathcal{C}_R$ in a homology $D$-manifold homology sphere, $H_1(\mathcal{C}_R\setminus K)\cong\mathbb{Z}\iff D=3$. \\
  \textbf{Lean mapping (current):} Not found as a proved theorem. A placeholder hypothesis interface exists in
  \texttt{IndisputableMonolith/Papers/DraftV1.lean} (\texttt{AlexanderDualityApplies}), but the actual homology/Alexander duality theorem is not yet in Lean. \\
  \textbf{Status:} \textbf{MISSING}.

  \item \textbf{Proposition} (\texttt{\textbackslash label\{prop:linking\}}) \emph{Linking Selection Principle} \\
  \textbf{Location:} \texttt{Draft\_v1.tex} lines 286--288 \\
  \textbf{Statement:} If $\mathcal{C}_R$ supports a nontrivial integer-valued loop-linking invariant, then $\dim(\mathcal{C}_R)=3$. \\
  \textbf{Lean mapping (current):} \emph{partial / different assumptions}. The repo has
  \texttt{IndisputableMonolith.Foundation.DimensionForcing.linking\_requires\_D3}
  (spinor+eight-tick notion of ``linking''), and a paper-facing hypothesis interface
  \texttt{IndisputableMonolith.Papers.DraftV1.LinkingSelectionPrincipleHypothesis}
  plus a wrapper \texttt{...DraftV1.linking\_selection\_principle}. \\
  \textbf{Status:} \textbf{PARTIAL}.

  \item \textbf{Proposition} \emph{RG Conditions for Duality} \\
  \textbf{Location:} \texttt{Draft\_v1.tex} lines 307--318 \\
  \textbf{Statement:} Minimal RG hypotheses on $(\mathcal{C},\tau_N)\to(\mathcal{C}_R,\tau_R)$ imply local contractibility of $\mathcal{C}_R$ and thus applicability of Alexander duality. \\
  \textbf{Lean mapping (current):} Placeholder interface exists:
  \texttt{IndisputableMonolith.Papers.DraftV1.rg\_conditions\_for\_duality}
  (in \texttt{IndisputableMonolith/Papers/DraftV1.lean}). \\
  \textbf{Status:} \textbf{MISSING} (not yet a proved topology theorem).

  \item \textbf{Proposition} \emph{RG Derivation of Central Potentials} \\
  \textbf{Location:} \texttt{Draft\_v1.tex} lines 443--452 \\
  \textbf{Statement:} Green-kernel potential form $V_D(r)\propto \ln r$ (D=2), $V_D(r)\propto -r^{2-D}$ (D$\ge3$). \\
  \textbf{Lean mapping (current):} Placeholder interface exists:
  \texttt{IndisputableMonolith.Papers.DraftV1.rg\_derivation\_of\_central\_potentials}
  (in \texttt{IndisputableMonolith/Papers/DraftV1.lean}). \\
  \textbf{Status:} \textbf{MISSING} (not yet a proved Green-kernel theorem).

  \item \textbf{Theorem} (\texttt{\textbackslash label\{thm:kepler\}}) \emph{Kepler Selection Principle} \\
  \textbf{Location:} \texttt{Draft\_v1.tex} lines 460--462 \\
  \textbf{Statement:} For $V_D(r)\propto -r^{2-D}$ on a smooth $D$-manifold, near-circular orbits are stable and non-precessing iff $D=3$. \\
  \textbf{Lean mapping:} \texttt{IndisputableMonolith.Papers.DraftV1.kepler\_selection\_principle}
  (in \texttt{IndisputableMonolith/Papers/DraftV1.lean}). \\
  \textbf{Note:} This formalizes the paper's \emph{algebraic core} after the substitution
  $\Delta\theta(D)=2\pi/\sqrt{4-D}$; a full Binet/Bertrand formalization is still future work. \\
  \textbf{Status:} \textbf{PARTIAL MATCH}.

  \item \textbf{Proposition} (\texttt{\textbackslash label\{prop:robustness\}}) \emph{Robustness of $D=3$ Signature} \\
  \textbf{Location:} \texttt{Draft\_v1.tex} lines 499--501 \\
  \textbf{Statement:} Binet linearization and apsidal-angle computation survive small RG-compatible perturbations. \\
  \textbf{Lean mapping (current):} Placeholder interface exists:
  \texttt{IndisputableMonolith.Papers.DraftV1.robustness\_of\_D3\_signature}
  (in \texttt{IndisputableMonolith/Papers/DraftV1.lean}). \\
  \textbf{Status:} \textbf{MISSING} (not yet a proved perturbation/continuity theorem).

  \item \textbf{Theorem} (\texttt{\textbackslash label\{thm:sync\}}) \emph{Synchronization Selection Principle} \\
  \textbf{Location:} \texttt{Draft\_v1.tex} lines 578--584 \\
  \textbf{Statement:} Under $D\ge3$, $S(D)=\mathrm{lcm}(2^D,45)$ is minimized uniquely at $D=3$; consequently $\mathcal{F}(D,45)=\alpha S(D)+\beta D$ (with $\alpha>0,\beta\ge0$) is minimized at $D=3$. \\
  \textbf{Lean mapping:} \texttt{IndisputableMonolith.Papers.DraftV1.synchronization\_selection\_principle}
  and \texttt{...DraftV1.sync\_resource\_functional\_minimized}
  (in \texttt{IndisputableMonolith/Papers/DraftV1.lean}). \\
  \textbf{Status:} \textbf{MATCH}.

  \item \textbf{Theorem} (\texttt{\textbackslash label\{thm:full\}}) \emph{Dimensional Rigidity in Recognition Geometry---Full Statement} \\
  \textbf{Location:} \texttt{Draft\_v1.tex} lines 616--626 \\
  \textbf{Statement:} (T/K/S)$\Rightarrow D=3$ and a partial converse ($D=3$ plus structural hypotheses $\Rightarrow$ T/K/S). \\
  \textbf{Lean mapping (current):} \emph{partial}. Forward-direction packaging exists as
  \texttt{IndisputableMonolith.Papers.DraftV1.dimensional\_rigidity\_full\_forward}. \\
  \textbf{Status:} \textbf{PARTIAL}.

  \item \textbf{Corollary} \emph{No Higher-Dimensional Alternative} \\
  \textbf{Location:} \texttt{Draft\_v1.tex} lines 666--668 \\
  \textbf{Statement:} No $D>3$ satisfies all three constraints simultaneously. \\
  \textbf{Lean mapping:} \texttt{IndisputableMonolith.Papers.DraftV1.no\_higher\_dimensional\_alternative}
  (in \texttt{IndisputableMonolith/Papers/DraftV1.lean}). \\
  \textbf{Status:} \textbf{PARTIAL} (depends on the currently formalized (K) core; (T) remains a placeholder).
\end{enumerate}

\section{Next steps (to complete the audit)}
\begin{itemize}
  \item Fully formalize the Alexander-duality/linking portion (requires homology + embeddings infrastructure)
  \item Formalize the Green-kernel potential derivation (Laplacian/Green's function) in Lean
  \item Upgrade the Kepler proof from the closed-form apsidal-angle core to a full Binet/Bertrand formalization
  \item Formalize the robustness proposition (perturbations / IFT / continuity), if needed for the paper
  \item Add a Lean formalization for the ``RG Conditions for Duality'' proposition (topology of quotients)
  \item Decide policy: fully formalize heavy topology/dynamics in Mathlib terms vs keep explicit hypothesis interfaces (conditional verification)
\end{itemize}

\end{document}

