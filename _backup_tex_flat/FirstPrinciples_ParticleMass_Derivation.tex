\documentclass[11pt]{article}

\usepackage[margin=1in]{geometry}
\usepackage{amsmath,amssymb}
\usepackage{microtype}
\usepackage{xcolor}
\usepackage{hyperref}
\usepackage{booktabs}

\hypersetup{
  colorlinks=true,
  linkcolor=blue,
  citecolor=blue,
  urlcolor=blue
}

% --- Notation ---
\newcommand{\phiG}{\varphi}
\newcommand{\lnphi}{\ln\phiG}
\newcommand{\Ecoh}{E_{\mathrm{coh}}}
\newcommand{\Fgap}{\mathcal F}

\title{\textbf{Internal Derivation Note: Particle Masses from First Principles}\\[0.3em]
\large What is derived, where it is derived (Lean), and what is still an empirical interface}
\author{Recognition (workspace: \texttt{reality})}
\date{\today}

\begin{document}
\maketitle

\section*{Purpose (what this note resolves)}
Anil's objection is methodological: if a ``recognition term'' is \emph{extracted} by rearranging the same measured masses being tested,
then nothing has been predicted.
This note provides the missing ``first principles'' provenance for the mass-law constants used in the repo and points to the new Lean
formalization that derives (or explicitly flags) each step.

\paragraph{Scope.}
This note focuses on the \textbf{derived structural mass law and lepton masses} (electron/muon/tau), because that is where the repo contains
the most explicit first-principles chain.
The \textbf{SM single-anchor RG identity} is treated separately as an \emph{empirical RG-side claim} (with a certified interface available).

\section{First-principles integer layer (cube geometry + crystallographic constant)}

\subsection*{Cube combinatorics (proved in Lean)}
The recognition ledger uses the cubic cell \(Q_3\) (8 vertices, 12 edges, 6 faces) as the minimal 3D unit cell.
These integers are derived in:
\begin{itemize}
  \item \texttt{IndisputableMonolith/Constants/AlphaDerivation.lean}
\end{itemize}
In particular:
\begin{itemize}
  \item \(E_{\text{total}}=12\) edges, and
  \item \(E_{\text{passive}}=11\) passive edges (12 total minus 1 active edge per atomic tick).
\end{itemize}

\subsection*{Wallpaper groups (documented constant)}
The integer \(W=17\) (wallpaper groups) is treated as a crystallographic classification constant inside
\texttt{IndisputableMonolith/Constants/AlphaDerivation.lean}. This is not a fit parameter; it is a standard mathematical constant.

\section{Derived constants used in the mass law}

\subsection*{Golden ratio and the gap/display function}
The project fixes \(\phiG=\frac{1+\sqrt5}{2}\), and uses the closed form
\[
  \Fgap(Z)=\frac{\ln(1+Z/\phiG)}{\ln\phiG}.
\]
Mathematical properties of \(\Fgap\) are verified in Lean:
\begin{itemize}
  \item \texttt{IndisputableMonolith/RSBridge/GapProperties.lean}
\end{itemize}

\subsection*{Coherence energy}
In the model layer, \(\Ecoh\) is defined as \(\phiG^{-5}\) (dimensionless) and then given physical units by a display convention
(\(\Ecoh\approx \phiG^{-5}\,\mathrm{eV}\)).
See:
\begin{itemize}
  \item \texttt{IndisputableMonolith/Masses/Anchor.lean} (\texttt{Anchor.E\_coh})
\end{itemize}

\subsection*{Sector yardsticks are now explicitly derivable}
The common criticism is that the sector yardsticks look like tuned knobs.
We addressed this directly: the frozen yardstick integers \((B_{\mathrm{pow}},r_0)\) are now proven equal to simple formulas in terms
of the first-principles integers \((E_{\text{total}},E_{\text{passive}},W)\).

\paragraph{Lean file (new).}
\begin{itemize}
  \item \texttt{IndisputableMonolith/Masses/AnchorDerivation.lean}
\end{itemize}

\paragraph{What it proves.}
It proves that the constants in \texttt{Masses/Anchor.lean} match derived expressions, e.g.:
\begin{itemize}
  \item \(B_{\mathrm{pow}}(\text{Lepton})=-2E_{\text{passive}}=-22\),
  \item \(B_{\mathrm{pow}}(\text{DownQuark})=2E_{\text{total}}-1=23\),
  \item \(r_0(\text{DownQuark})=E_{\text{total}}-W=12-17=-5\),
  \item \(r_0(\text{UpQuark})=2W+1=35\),
  \item \(r_0(\text{Lepton})=4W-(8-2)=62\) (uses the 8-tick offset and the baseline lepton rung \(r_e=2\)).
\end{itemize}
These are \emph{sector-global} integers; no per-species tuning is introduced.

\section{Lepton masses (derived chain in Lean)}

\subsection*{Electron (T9)}
The electron derivation uses:
\begin{itemize}
  \item the derived fine-structure constant pipeline (\texttt{Constants/AlphaDerivation.lean}),
  \item the derived topological shift \(\delta\) (\texttt{Physics/MassTopology.lean}),
  \item the lepton band value \(\Fgap(1332)\) (\texttt{RSBridge/GapProperties.lean}),
  \item and the derived sector yardstick constants (now derived as above).
\end{itemize}

The electron mass construction is formalized (with interval-bounds bookkeeping) in:
\begin{itemize}
  \item \texttt{IndisputableMonolith/Physics/ElectronMass.lean}
  \item \texttt{IndisputableMonolith/Physics/ElectronMass/Necessity.lean}
\end{itemize}

\subsection*{Muon and tau (T10)}
The muon and tau derivations are built by adding derived ``step'' exponents to the electron residue.
They are formalized in:
\begin{itemize}
  \item \texttt{IndisputableMonolith/Physics/LeptonGenerations.lean}
  \item \texttt{IndisputableMonolith/Physics/LeptonGenerations/Necessity.lean}
\end{itemize}

\subsection*{Numerical evaluation (reproducible, non-circular)}
We provide a small evaluator that mirrors the Lean definitions and produces a crisp
``predicted vs.\ PDG'' table without per-species fitting:
\begin{itemize}
  \item script: \texttt{tools/lepton\_chain\_table.py}
  \item outputs:
    \texttt{out/masses/lepton\_chain\_pred\_vs\_pdg.csv} and
    \texttt{out/masses/lepton\_chain\_pred\_vs\_pdg.tex}.
\end{itemize}
The LaTeX snippet is included below:

% Auto-generated by tools/lepton_chain_table.py
\begin{table}[h]
  \centering
  \caption{Lepton chain prediction (T9--T10) from first-principles constants. Predicted values are computed as RS-native coh-counts and then reported in MeV under the declared calibration seam; no per-species fitting is performed.}
  \label{tab:lepton_chain_pred_vs_pdg}
  \begin{tabular}{lrrrr}
    \toprule
    Species & Pred. (MeV) & PDG (MeV) & Abs. err & Rel. err \\
    \midrule
    e & 0.510999 & 0.510999 & -1.9546e-07 & -3.82506e-07 \\
    mu & 105.658 & 105.658 & -0.000112323 & -1.06307e-06 \\
    tau & 1776.71 & 1776.86 & -0.154158 & -8.67587e-05 \\
    \bottomrule
  \end{tabular}
\end{table}


\section{What remains empirical / external (and how we make it auditably non-circular)}

\subsection*{SM single-anchor RG identity is an empirical claim}
The claim ``the SM RG residue at \(\mu_\star\) equals \(\Fgap(Z)\)'' depends on multi-loop SM kernels and threshold policy.
Those kernels are \emph{not} implemented in Lean; in the repo they are represented as:
\begin{itemize}
  \item a hypothesis interface (\texttt{IndisputableMonolith/Physics/AnchorPolicy.lean}), or
  \item a certificate interface (externally computed residue intervals) (\texttt{IndisputableMonolith/Physics/AnchorPolicyCertified.lean}), or
  \item a purely definitional model (for algebraic consequences only) (\texttt{IndisputableMonolith/Physics/AnchorPolicyModel.lean}).
\end{itemize}
This is the honest separation between \emph{first-principles RS derivations} and \emph{empirical SM-RG phenomenology}.

\section*{Bottom line}
For the lepton sector, the repo already contains a first-principles derivation chain in Lean for:
\begin{itemize}
  \item the core integers (cube geometry and passive edges),
  \item the derived \(\alpha\) pipeline,
  \item the electron topological shift and lepton generation steps,
  \item and (new) the sector yardstick constants as derived expressions rather than fit knobs.
\end{itemize}
For the SM single-anchor identity, the repo treats the claim as RG-side phenomenology and provides a certified interface so the claim remains auditable and non-circular.

\end{document}


