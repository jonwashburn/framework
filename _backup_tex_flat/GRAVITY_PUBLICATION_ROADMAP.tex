\documentclass[11pt]{article}

\usepackage[margin=1in]{geometry}
\usepackage[T1]{fontenc}
\usepackage[utf8]{inputenc}
\usepackage{amsmath,amssymb}
\usepackage[hidelinks]{hyperref}

\setlength{\parindent}{0pt}
\setlength{\parskip}{0.55em}

\begin{document}

\begin{center}
{\LARGE \textbf{Gravity publication roadmap (draft)}}\\
\end{center}

\section*{Goal}
Produce a \textbf{coherent, end-to-end, publication-ready documentation set} for the full gravitational theory (RS/RC foundations $\rightarrow$ classical limit/GR $\rightarrow$ effective kernels in galaxies/cosmology $\rightarrow$ quantum ``display layer'' + falsifiers), with clear separation between:
\begin{itemize}
  \item \textbf{proved / machine-verified} statements (Lean certificates / rigorous math),
  \item \textbf{empirical fits} (parameter values inferred from data),
  \item \textbf{hypotheses / effective-theory closures} (explicitly labeled).
\end{itemize}

\section*{What has already been drafted (core papers deep-read)}
\begin{itemize}
  \item \textbf{Galaxy phenomenology (global-only causal response)}: \texttt{gravity-papers/0-0-0-gravity-submission-aaa-v07-shorted-v04.tex}
    \begin{itemize}
      \item Scope: rotation curves, global-only fitting protocol, cluster lensing prediction, Caldeira--Leggett realization.
    \end{itemize}

  \item \textbf{Cosmology (ILG Paper I: source-side kernel + linear regime theory)}: \texttt{gravity-papers/dark\_energy\_paper1\_mesedits.tex}
    \begin{itemize}
      \item Scope: fixed kernel \(w(k,a)=1+C(k\tau_0/a)^{-\alpha}\); well-posedness for \(\alpha\in(0,\tfrac12)\); \textbf{zero Buchert backreaction}; qualitative falsifiers (ISW suppression, scale-dependent growth, tracer-independent \(E_G\), \(X\)-collapse).
      \item Explicitly defers: quantitative \(\Lambda\)CDM amplitudes + nonlinear modeling (``Paper II'').
    \end{itemize}

  \item \textbf{Discrete foundations for a classical gravity limit (ledger $\rightarrow$ DEC $\rightarrow$ kernel)}: \texttt{gravity-papers/quantum\_gravity\_A\_mesedits.tex}
    \begin{itemize}
      \item Scope: MP + ledger axioms; DEC bridge to Poisson; fractional-memory mechanism for \(k^{-\alpha}\); SPARC-only fits (with parameters free); catalogs open gaps (units quotient, parameter fixing, public Lean release, operational definition of ``recognition event'').
    \end{itemize}

  \item \textbf{Quantum completion display layer (BRST + DEC + audit interfaces)}: \texttt{gravity-papers/quantum\_gravity\_B\_v4.tex}
    \begin{itemize}
      \item Scope: background-field / de Donder gauge; BRST framing; DEC nonperturbative outline; anomaly checks; BH thermodynamics check; \textbf{GW ``audit bands''} pipeline.
    \end{itemize}
\end{itemize}

\section*{Other relevant internal drafts (awareness; not assumed publishable as-is)}
\begin{itemize}
  \item \textbf{Coercive Projection Method (CPM) gravity law + certificates}: \texttt{gravity-papers/Gravity Set/CPM-Gravity.tex}
  \item \textbf{Gravity-as-pressure equivalence display}: \texttt{gravity-papers/Gravity Set/Pressure-Gravity.tex}
  \item \textbf{Disk-dynamics ILG series draft (older ``Paper I'')}: \texttt{gravity-papers/Gravity Set/Information-Limited-Gravity-Paper1-Sept26.tex}
  \item \textbf{Late-time kernel + Hubble tension (draft)}: \texttt{gravity-papers/Gravity Set/Hubble-Tension-Resolution.tex}
  \item \textbf{Universe origin / inflation / primordial signatures (draft)}: \texttt{gravity-papers/Gravity Set/Universe-Origin.tex}
  \item \textbf{Baryogenesis (draft)}: \texttt{gravity-papers/Gravity Set/Baryogenesis.tex}
  \item \textbf{Measurement/collapse bridge (not gravity per se, but overlaps RC cost/action)}: \texttt{gravity-papers/Gravity Set/gravity-coherence.tex}
\end{itemize}

\hrule

\section*{Proposed publication set (what still needs to be written)}

\subsection*{Paper 1 --- Recognition/Ledger Foundations for Gravity (formal core)}
\textbf{Working title:} \emph{Recognition/Ledger Foundations for Gravity: discrete axioms, cost uniqueness, cadence, and the admissible-units quotient}
\begin{itemize}
  \item \textbf{Purpose}: Provide the \emph{canonical} statement of RS/RC primitives that later gravity papers cite as axioms/theorems, with machine-verifiable status.
  \item \textbf{Must include}:
    \begin{itemize}
      \item Precise definitions: ledger, postings, neutrality/exactness, cost \(J\), cadence/Octave constraints, and \textbf{admissible-units quotient} (parameter-free policy).
      \item A \textbf{public, checkable artifact story} (Lean build, certificate ledger, versioning).
    \end{itemize}
  \item \textbf{Inputs to reuse}:
    \begin{itemize}
      \item \texttt{gravity-papers/quantum\_gravity\_A\_mesedits.tex} (model + theorem sketches)
      \item repo Lean sources (already present in \texttt{IndisputableMonolith/**})
    \end{itemize}
  \item \textbf{Missing today (by the paper's own status notes)}:
    \begin{itemize}
      \item Public Lean release / citation-quality certificate summary for T2--T7 (or downgraded claims until released).
      \item Completed units-quotient formalization (explicitly flagged as a blocker in Paper A).
    \end{itemize}
\end{itemize}

\subsection*{Paper 2 --- Machine-verified GR emergence (classical continuum target)}
\textbf{Working title:} \emph{Machine-Verified Emergence of Einstein Dynamics from Recognition Science}
\begin{itemize}
  \item \textbf{Purpose}: Bridge RS primitives to \textbf{covariant GR} (Einstein--Hilbert action, EFE, stress-energy definition); the canonical citation for ``RS $\rightarrow$ GR''.
  \item \textbf{Inputs to reuse}:
    \begin{itemize}
      \item \texttt{docs/GRAVITATIONAL\_EMERGENCE\_PAPER.tex} (already structured around proof-status honesty)
      \item Lean files in \texttt{IndisputableMonolith/Relativity/**} (variational sector, Palatini, EFE emergence)
    \end{itemize}
  \item \textbf{What still needs writing/work (to reach ``full machine verified'')}:
    \begin{itemize}
      \item Close remaining \texttt{sorry}s and remove remaining gravity-sector scaffolds as per \texttt{docs/GR\_EMERGENCE\_PLAN.md}.
      \item Decide the precise statement of the RS$\rightarrow$GR mapping (what is definition, what is theorem, what is hypothesis).
    \end{itemize}
\end{itemize}

\subsection*{Paper 3 --- ILG Cosmology II (the missing quantitative confrontation)}
\textbf{Working title:} \emph{Information-Limited Gravity II: quantitative \(\Lambda\)CDM predictions, nonlinear calibration, and likelihood-level tests}
\begin{itemize}
  \item \textbf{Purpose}: Provide the missing ``Paper II'' promised by \texttt{dark\_energy\_paper1\_mesedits.tex}: realistic amplitudes, parameter-free forecasts, and hard tests.
  \item \textbf{Must include}:
    \begin{itemize}
      \item Numerical integration in \(\Lambda\)CDM for \(D(k,a)\), \(f(k,z)\), \(R_L\), ISW predictions.
      \item Nonlinear modeling plan: N-body or emulators + baryonic feedback treatment.
      \item Likelihood pipeline that preserves the required \((k,z)\) structure (no scale-averaging that erases the signal).
    \end{itemize}
  \item \textbf{Inputs to reuse}:
    \begin{itemize}
      \item \texttt{gravity-papers/dark\_energy\_paper1\_mesedits.tex} (definitions + theory)
      \item any existing analysis code (if present in repo) + explicit artifact bundle.
    \end{itemize}
  \item \textbf{Status}: \textbf{not yet written} (only referenced).
\end{itemize}

\subsection*{Paper 4 --- Galaxy-scale kernel paper (choose/merge the competing ``galaxy kernels'')}
\textbf{Working title:} \emph{Information-limited response in galaxies: nonlocal kernel prediction, disk convolution, and global-only inference}
\begin{itemize}
  \item \textbf{Purpose}: Provide a single, canonical galaxy-scale prediction pipeline consistent with the cosmology kernel (if unification is intended), and/or explain why galaxy and cosmology kernels differ.
  \item \textbf{Why this paper is needed}:
    \begin{itemize}
      \item Multiple overlapping galaxy presentations exist (ILG kernel as \(w(k)\)/\(w(r)\) vs. causal-response \(w(r)\) with morphology factors). The publication set needs \textbf{one ``source of truth''} or a clear regime map.
    \end{itemize}
  \item \textbf{Must include}:
    \begin{itemize}
      \item Full nonlocal prediction for disks (Hankel/FFT convolution) as the primary forward model (not only the effective \(v^2 \approx w(r)v_b^2\) closure).
      \item Strict global-only policy and a reproducibility bundle (SPARC snapshot + masks + code).
      \item A ``regime map'' explaining which assumptions are phenomenological (e.g., exponential-memory kernel, morphology factor) vs.\ derived.
    \end{itemize}
  \item \textbf{Inputs to reuse}:
    \begin{itemize}
      \item \texttt{gravity-papers/0-0-0-gravity-submission-aaa-v07-shorted-v04.tex} (global-only causal response)
      \item \texttt{gravity-papers/quantum\_gravity\_A\_mesedits.tex} (kernel-from-latency story + SPARC evidence)
      \item \texttt{gravity-papers/Gravity Set/Pressure-Gravity.tex} (pressure display)
      \item \texttt{gravity-papers/Gravity Set/Information-Limited-Gravity-Paper1-Sept26.tex} (older ILG disk framing)
    \end{itemize}
\end{itemize}

\subsection*{Paper 5 --- Covariant completion of the source-side kernel (the core conceptual gap)}
\textbf{Working title:} \emph{Covariant completion of source-side information-limited gravity: action, Bianchi identity, causality, and GW sector}
\begin{itemize}
  \item \textbf{Purpose}: ILG Paper I calls this the main theoretical gap. A full ``gravity theory'' needs a covariant embedding (even if phenomenology is tested first).
  \item \textbf{Must include}:
    \begin{itemize}
      \item Covariant operator realization (e.g., nonlocal \(\mathcal{F}[\Box]\), auxiliary-field completion, or equivalent) reducing to the Poisson multiplier in the Newtonian/quasi-static regime.
      \item Demonstrate: Bianchi identity compatibility, causal (retarded) structure, ghost-freedom, and GW-sector predictions consistent with GW170817-style constraints.
    \end{itemize}
  \item \textbf{Inputs to reuse}:
    \begin{itemize}
      \item candidate operator sketches already in \texttt{dark\_energy\_paper1\_mesedits.tex} (covariant completion section)
      \item Caldeira--Leggett style constructions (already in the galaxy paper) if relevant.
    \end{itemize}
\end{itemize}

\subsection*{Paper 6 --- Quantum completion + audit interfaces (tighten the artifact story)}
\textbf{Working title:} \emph{Recognition Calculus quantum gravity: BRST + DEC realization and falsifiable audit bands}
\begin{itemize}
  \item \textbf{Purpose}: Keep \texttt{quantum\_gravity\_B\_v4.tex} as the quantum-facing paper, but ensure it cleanly references proved artifacts and does not overclaim beyond what is certified.
  \item \textbf{Still needed}:
    \begin{itemize}
      \item Consolidated public artifact bundle: certificate hashes, where Lean lives, and how a referee rebuilds it.
      \item Clear separation between ``standard EFT/BRST review'' vs.\ ``RC-specific additional claims''.
    \end{itemize}
\end{itemize}

\hrule

\section*{Optional papers (valuable, but not strictly required for ``gravity core'')}
\begin{itemize}
  \item \textbf{Hubble tension / late-time inference}: formalize whether ``late-time kernel'' genuinely changes \emph{inference} vs.\ \emph{physics} and how this interacts with Paper 3.
  \item \textbf{Coercive projection law + certificate schema} (CPM): a methods paper that becomes the audit standard across RS phenomenology papers.
  \item \textbf{Origin / inflation / primordial signatures} (Universe-Origin): if you want full gravitational \emph{cosmology}, this becomes part of the suite.
  \item \textbf{Baryogenesis}: if included, should reference the same constants/units policy and the same GR emergence baseline.
  \item \textbf{Collapse/coherence bridge}: adjacent to gravity via RC cost/action, but likely a separate measurement series.
\end{itemize}

\hrule

\section*{Cross-paper consistency checklist (must be harmonized before submission)}
\begin{itemize}
  \item \textbf{Kernel naming}: distinguish \(w(r)\) (galaxy display), \(w(k,a)\) (cosmology), and any transfer function \(H(i\omega)\) (memory model). Spell out the mapping/assumptions between them.
  \item \textbf{Constants and ``parameter-free'' claims}: every paper must separate:
    \begin{itemize}
      \item \textbf{derived constants} (with certificates),
      \item \textbf{calibration constants} (SI mappings / survey \(M/L\) choices),
      \item \textbf{fit parameters} (if any are used, label them honestly).
    \end{itemize}
  \item \textbf{Units policy}: adopt one ``quotient-first, units-last'' policy and apply it everywhere (avoid mixing micro tick \(\tau_0\) with macro anchors like \(\tau_\star\) without explicit bridges).
  \item \textbf{Nonlocal vs.\ local closures}: if using an effective local \(w(r)\) multiply rule, label it as a closure and provide a plan (or results) for the full convolution check.
  \item \textbf{Covariant completion boundary}: every effective kernel paper should state what is and is not claimed without a covariant embedding.
\end{itemize}

\hrule

\section*{Recommended execution order}
\begin{enumerate}
  \item Paper 2 (Lean GR emergence): close proof debt $\rightarrow$ establishes the GR baseline.
  \item Paper 1 (foundations + units quotient): finalize the RS/RC primitives and parameter policy.
  \item Paper 3 (ILG cosmology II): quantitative tests + pipeline.
  \item Paper 4 (galaxy-scale canonical kernel paper): unify/choose the galaxy story with full convolution.
  \item Paper 5 (covariant completion): once phenomenology survives, complete the conceptual embedding.
  \item Paper 6 (quantum + audits): finalize the QG display layer and public artifacts.
\end{enumerate}

\end{document}


