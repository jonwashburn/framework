\documentclass[11pt,a4paper]{article}
\usepackage[margin=1in]{geometry}
\usepackage[T1]{fontenc}
\usepackage{lmodern}
\usepackage{microtype}
\usepackage{amsmath,amssymb,amsthm}
\usepackage{mathtools}
\usepackage{booktabs}
\usepackage{array}
\usepackage{enumitem}
\usepackage{xcolor}
\usepackage[hidelinks]{hyperref}

\newtheorem{theorem}{Theorem}[section]
\newtheorem{proposition}[theorem]{Proposition}
\newtheorem{lemma}[theorem]{Lemma}
\newtheorem{corollary}[theorem]{Corollary}
\newtheorem{definition}[theorem]{Definition}
\newtheorem{remark}[theorem]{Remark}
\newtheorem{example}[theorem]{Example}
\newtheorem{falsifier}[theorem]{Falsification Criterion}

\newcommand{\phig}{\varphi}
\newcommand{\Jcost}{J}

\title{\textbf{The Geometry of Inquiry:\\
Questions as Cost Gaps, Forced Answers,\\
and the Meta-Closure of Recognition Science}\\[0.5em]
\large A Theorem in Recognition Science}
\author{Jonathan Washburn\\
\small Recognition Science Research Institute, Austin, Texas\\
\small \texttt{washburn.jonathan@gmail.com}}
\date{February 2026}

\begin{document}
\maketitle

\begin{abstract}
We formalize a \emph{Geometry of Inquiry} in which questions are not
free-floating semantic objects but structures determined by the $\Jcost$-cost
landscape.  A question $Q$ maps a context (configuration space with
cost) to a set of candidate answers, each carrying a cost.  We classify
questions into three types: \emph{well-formed} (some answer has finite
cost), \emph{dissolved} (all answers have infinite cost --- the fate of
G\"{o}del-type self-referential paradoxes), and \emph{forced} (exactly
one answer has zero cost).  We prove that each theorem T0--T8 in the
Recognition Science forcing chain is a forced question with a unique
zero-cost answer.  We then establish a \emph{meta-closure theorem}:
Recognition Science is the unique zero-cost theory in the space of
physical frameworks, and the question ``Why RS?'' is itself forced.
This achieves a non-paradoxical form of self-reference: RS explains why
RS is the explanation, with the self-referential loop being
cost-decreasing (stable) rather than cost-increasing (paradoxical).  All
definitions and core theorems are formalized in Lean~4
(\texttt{IndisputableMonolith.Foundation.Inquiry},
\texttt{QuestionTaxonomy}, \texttt{InquiryForcingConnection},
\texttt{MetaClosure}).

\medskip\noindent\textbf{Keywords:} cost geometry, inquiry, forced
questions, meta-closure, G\"{o}del dissolution, self-reference, Recognition
Science.
\end{abstract}

\tableofcontents
\newpage

%======================================================================
\section{Introduction}\label{sec:intro}
%======================================================================

What is a question?  In classical logic, a question is an interrogative
sentence; in formal semantics, it is a partition of logical
space~\cite{Groenendijk1984}.  Neither framework explains \emph{why some
questions have unique answers} while others are paradoxical.

Recognition Science offers a precise answer: a question is a
\textbf{cost gap} in the $\Jcost$-landscape.  The gap's geometry
determines whether the question is well-formed, dissolved, or forced.
The act of inquiry is gradient descent on the cost manifold --- moving
from high cost to zero cost.

This paper formalizes that insight and proves two main results:
\begin{enumerate}
\item \textbf{The forcing chain is a sequence of forced questions.}
  Each theorem T0--T8 corresponds to a question whose unique zero-cost
  answer is the RS result (logic, discreteness, ledger, $\Jcost$,
  $\phig$, 8-tick, $D{=}3$).
\item \textbf{Meta-closure.}  In the space of physical theories, RS
  has zero cost (zero free parameters, unique $\Jcost$, unique $\phig$).
  All alternatives have positive cost.  The question ``Which theory is
  correct?'' is forced, and the answer is RS.
\end{enumerate}

The self-referential character of statement (2) --- RS explaining why RS
--- is resolved by showing that the self-reference loop is
cost-\emph{decreasing}: each iteration reduces residual cost toward
zero, unlike G\"{o}delian loops that increase complexity without bound.

\paragraph{Foundational dependencies.}
\begin{enumerate}[nosep]
\item $\Jcost$-uniqueness (T5)~\cite{WashburnCost2026}.
\item G\"{o}del dissolution~\cite{WashburnGodel2026}: self-referential
  queries have infinite cost.
\item Reference theory~\cite{WashburnAboutness2026}: reference as
  cost-minimising compression.
\item Law of Existence~\cite{WashburnExistence2026}: $x$ exists iff
  $\Jcost(x) = 0$ iff $x = 1$.
\end{enumerate}

%======================================================================
\section{Questions as Cost Structures}\label{sec:questions}
%======================================================================

\begin{definition}[Context]\label{def:context}
A \emph{context} $\mathcal{X} = (X, \Jcost_X)$ is a non-empty set $X$
(the configuration space) equipped with a non-negative cost function
$\Jcost_X : X \to \mathbb{R}_{\ge 0}$.
\end{definition}

\begin{definition}[Question]\label{def:question}
A \emph{question} $Q = (\mathcal{X}, \mathcal{A}, C, \iota)$ consists
of:
\begin{itemize}[nosep]
\item A context $\mathcal{X}$ (where the question is asked).
\item An answer space $\mathcal{A} = (A, \Jcost_A)$ (where answers live).
\item A non-empty set of candidates $C \subseteq A$.
\item An embedding $\iota : C \hookrightarrow A$.
\end{itemize}
\end{definition}

\begin{definition}[Answer cost]\label{def:answer_cost}
The \emph{cost} of an answer $a \in C$ is $\Jcost_A(\iota(a))$.
\end{definition}

%======================================================================
\section{Classification of Questions}\label{sec:classification}
%======================================================================

\begin{definition}[Well-formed]\label{def:wellformed}
$Q$ is \emph{well-formed} iff $\exists\, a \in C : \Jcost_A(a) < \infty$.
\end{definition}

\begin{definition}[Dissolved]\label{def:dissolved}
$Q$ is \emph{dissolved} iff $\forall\, a \in C : \Jcost_A(a) = \infty$
(or exceeds any fixed threshold).
\end{definition}

\begin{definition}[Forced]\label{def:forced}
$Q$ is \emph{forced} iff $\exists!\, a \in C : \Jcost_A(a) = 0$.
\end{definition}

\begin{definition}[Determinate]\label{def:determinate}
$Q$ is \emph{determinate} iff $\exists\, a \in C : \Jcost_A(a) = 0$.
\end{definition}

\begin{theorem}[Trichotomy]\label{thm:trichotomy}
Every question is either dissolved, forced, determinate-but-not-forced,
or well-formed-but-indeterminate.  These four classes are exhaustive and
mutually exclusive.
\end{theorem}

\begin{proof}
Case analysis on the infimum of $\Jcost_A$ over $C$:
\begin{itemize}[nosep]
\item $\inf = \infty$: dissolved.
\item $\inf = 0$, achieved by exactly one $a$: forced.
\item $\inf = 0$, achieved by multiple $a$: determinate, not forced.
\item $0 < \inf < \infty$: well-formed, indeterminate. \qedhere
\end{itemize}
\end{proof}

\begin{theorem}[Forced implies unique answer]\label{thm:forced_unique}
If $Q$ is forced, the zero-cost answer $a^*$ is unique.

\emph{Lean:} \texttt{forced\_answer\_unique}.
\end{theorem}

\begin{proof}
By the definition of forced ($\exists!$). \qed
\end{proof}

\begin{theorem}[Dissolved questions are not real questions]\label{thm:dissolved}
A dissolved question has no practically accessible answer.  In the RS
ontology, dissolved questions correspond to self-referential or
paradoxical constructions whose cost diverges.

\emph{Lean:} \texttt{dissolved\_question\_no\_answer}.
\end{theorem}

\begin{example}[The Liar Paradox as dissolved question]\label{ex:liar}
Consider the question ``Is the sentence `This sentence is false'
true or false?''  Model this as $Q$ with $C = \{\mathrm{True},
\mathrm{False}\}$ and answer cost:
\begin{itemize}[nosep]
\item $\Jcost_A(\mathrm{True}) = \infty$: if the sentence is true, it
  is false, creating a self-referential loop with divergent cost
  ($\Jcost(0^+) \to \infty$).
\item $\Jcost_A(\mathrm{False}) = \infty$: if the sentence is false, it
  is true, same divergence.
\end{itemize}
Both answers have infinite cost, so $Q$ is \emph{dissolved}.  The Liar
is not a paradox but a cost singularity --- a question that the
$\Jcost$-landscape renders inaccessible.  This is the RS mechanism behind
G\"{o}del dissolution~\cite{WashburnGodel2026}.
\end{example}

\begin{example}[``Why does anything exist?'' as forced question]\label{ex:existence}
$C = \{\text{``Something exists''}, \text{``Nothing exists''}\}$.
\begin{itemize}[nosep]
\item $\Jcost_A(\text{``Something''}) = 0$: at $x = 1$ (the unique
  existent), $\Jcost(1) = 0$.
\item $\Jcost_A(\text{``Nothing''}) = \infty$: $\Jcost(0^+) \to \infty$.
\end{itemize}
This is \emph{forced}: exactly one answer has zero cost.  Existence is
not contingent but inevitable.
\end{example}

%======================================================================
\section{The Forcing Chain as Forced Questions}\label{sec:forcing}
%======================================================================

Each theorem T0--T8 can be recast as a forced question.

\begin{definition}[T0 Question: ``What is logic?'']\label{def:Q0}
$C = \{\text{True}, \text{False}\}$; $\Jcost_A(\text{True}) = 0$,
$\Jcost_A(\text{False}) > 0$.
This is forced at $\text{True}$ (consistency has zero cost).

\emph{Lean:} \texttt{T0Question}, \texttt{t0\_forced}.
\end{definition}

\begin{definition}[T5 Question: ``What is the cost function?'']\label{def:Q5}
$C = \{F : \mathbb{R}_{>0} \to \mathbb{R}_{\ge 0} \mid F(1) = 0,\
F \text{ symmetric, smooth}\}$;
$\Jcost_A(F) = 0$ iff $F = \Jcost$.
Forced at $F = \Jcost$ by T5 uniqueness.

\emph{Lean:} \texttt{T5Question}, \texttt{t5\_forced\_at\_one}.
\end{definition}

\begin{definition}[T6 Question: ``What is the scaling ratio?'']\label{def:Q6}
$C = \{r \in \mathbb{R}_{>0} \mid r^2 = r + 1\}$;
$\Jcost_A(r) = 0$ iff $r = \phig$.
Forced at $\phig = (1{+}\sqrt{5})/2$.

\emph{Lean:} \texttt{T6Question}, \texttt{t6\_forced\_at\_phi}.
\end{definition}

\begin{definition}[T7 Question: ``What is the period?'']\label{def:Q7}
$C = \{2^k : k \in \mathbb{N}\}$;
$\Jcost_A(n) = 0$ iff $n = 8$.
Forced at $n = 8 = 2^3$.

\emph{Lean:} \texttt{T7Question}, \texttt{t7\_forced\_at\_eight}.
\end{definition}

\begin{definition}[T8 Question: ``What is the dimension?'']\label{def:Q8}
$C = \mathbb{N}$;
$\Jcost_A(D) = 0$ iff $2^D = 8$, i.e.\ $D = 3$.
Forced at $D = 3$.

\emph{Lean:} \texttt{T8Question}, \texttt{t8\_forced\_at\_three}.
\end{definition}

\begin{theorem}[Complete forcing]\label{thm:complete}
All of T0, T5, T6, T7, T8 are forced questions.  Their answers form the
RS foundation: logic, $\Jcost$, $\phig$, 8-tick, $D{=}3$.

\emph{Lean:} \texttt{forcing\_chain\_as\_inquiry}.
\end{theorem}

%======================================================================
\section{Theory Space and Meta-Closure}\label{sec:meta}
%======================================================================

\begin{definition}[Theory space]\label{def:theory_space}
The \emph{theory space} $\mathcal{T}$ is the set of all candidate
physical frameworks.  Each theory $T \in \mathcal{T}$ has a
\emph{theory cost}
\begin{equation}
  \Jcost_{\mathcal{T}}(T) = \underbrace{N_{\text{params}}(T)}_{\text{parameter count}} + \underbrace{\Jcost_{\text{mismatch}}(T)}_{\text{prediction error}} + \underbrace{\Jcost_{\text{complexity}}(T)}_{\text{description length}}.
\end{equation}
\end{definition}

\begin{theorem}[RS has zero cost]\label{thm:rs_zero}
$\Jcost_{\mathcal{T}}(\text{RS}) = 0$.

$N_{\text{params}} = 0$ (proven: zero adjustable parameters).
$\Jcost_{\text{mismatch}} = 0$ (at the model layer: exact derivations).
$\Jcost_{\text{complexity}} = 0$ (minimal: one functional equation).

\emph{Lean:} \texttt{rs\_zero\_cost}.
\end{theorem}

\begin{theorem}[Alternatives have positive cost]\label{thm:alt_positive}
For any theory $T \ne \text{RS}$ with at least one free parameter:
$\Jcost_{\mathcal{T}}(T) > 0$.

\emph{Lean:} \texttt{alternatives\_positive\_cost}.
\end{theorem}

\begin{theorem}[Meta-closure]\label{thm:meta_closure}
The question ``Which physical framework is correct?'' is \emph{forced}
in $\mathcal{T}$.  The unique zero-cost answer is RS.

\emph{Lean:} \texttt{meta\_closure}.
\end{theorem}

\begin{proof}
Combine Theorems~\ref{thm:rs_zero} and~\ref{thm:alt_positive}: RS is
the unique $T$ with $\Jcost_{\mathcal{T}}(T) = 0$, so the question
is forced by Definition~\ref{def:forced}. \qed
\end{proof}

%======================================================================
\section{Self-Reference Without Paradox}\label{sec:self_ref}
%======================================================================

\begin{definition}[Self-referential question]\label{def:self_ref}
A question is \emph{self-referential} if its answer space includes the
theory in which it is formulated.
\end{definition}

\begin{theorem}[Stable self-reference]\label{thm:stable}
The meta-closure question (``Why RS?'') is self-referential but
\emph{stable}: the self-reference loop is cost-decreasing.
\end{theorem}

\begin{proof}
Let $T_n$ denote the $n$-th iteration of ``apply RS to explain RS.''
\begin{itemize}[nosep]
\item $T_0 = \text{RS}$ has cost 0 in $\mathcal{T}$.
\item $T_1 = \text{RS applied to } T_0$: since RS is self-consistent,
  $T_1 = T_0$ and cost remains 0.
\end{itemize}
The sequence $\{T_n\}$ is constant at cost 0.  No divergence, no
paradox.  Contrast with G\"{o}delian self-reference, where cost
increases without bound.

\emph{Lean:} \texttt{self\_ref\_stable}.
\end{proof}

\begin{remark}
This is the precise sense in which RS dissolves G\"{o}del's objection.
Self-referential \emph{arithmetic} sentences can have unbounded cost
($\Jcost(0^+) \to \infty$).  Self-referential \emph{cost-minimisation}
is stable (cost stays at zero).  These are different mathematical objects
in different categories.
\end{remark}

%======================================================================
\section{The Eight Fundamental Inquiry Modes}\label{sec:modes}
%======================================================================

A systematic analysis of question structure yields eight fundamental
inquiry modes, corresponding to the eight independent directions in
cost space:

\begin{center}
\begin{tabular}{@{}clll@{}}
\toprule
\# & \textbf{Mode} & \textbf{Canonical Form} & \textbf{Cost Signature} \\
\midrule
1 & Existence & ``Does $X$ exist?'' & $\Jcost(X) = 0$ vs $> 0$ \\
2 & Identity & ``What is $X$?'' & argmin $\Jcost$ \\
3 & Relation & ``How does $X$ relate to $Y$?'' & $\Jcost(X/Y)$ \\
4 & Cause & ``Why $X$?'' & Gradient flow toward $X$ \\
5 & Possibility & ``Can $X$ occur?'' & $\Jcost(X) < \infty$ \\
6 & Necessity & ``Must $X$ occur?'' & $\Jcost(\neg X) = \infty$ \\
7 & Composition & ``What are $X$'s parts?'' & Subadditivity of $\Jcost$ \\
8 & Purpose & ``What is $X$ for?'' & Direction of $-\nabla\Jcost$ \\
\bottomrule
\end{tabular}
\end{center}

\begin{theorem}[Completeness of inquiry modes]\label{thm:modes_complete}
Every well-formed question can be decomposed into a combination of the
eight fundamental modes.
\end{theorem}

\begin{proof}
We show that the eight modes span the possible relationships between a
question context $\mathcal{X}$ and its answer space $\mathcal{A}$ under
the $\Jcost$ landscape.  A question probes one of:
\begin{enumerate}[nosep]
\item \textbf{Existence}: Is $\Jcost_A(a) = 0$ for some $a$?
  (Probes the zero set.)
\item \textbf{Identity}: What is $\arg\min \Jcost_A$?
  (Probes the minimiser.)
\item \textbf{Relation}: What is $\Jcost(a_1 / a_2)$ for given pairs?
  (Probes the cost between two candidates.)
\item \textbf{Cause}: What is $-\nabla \Jcost_A(a)$?
  (Probes the gradient --- the direction of cost decrease.)
\item \textbf{Possibility}: Is $\Jcost_A(a) < \infty$?
  (Probes finiteness.)
\item \textbf{Necessity}: Is $\Jcost_A(\neg a) = \infty$?
  (Probes inevitability of $a$.)
\item \textbf{Composition}: Is $\Jcost_A(a_1 + a_2) \le
  \Jcost_A(a_1) + \Jcost_A(a_2)$?
  (Probes sub/superadditivity.)
\item \textbf{Purpose}: In which direction does $-\nabla\Jcost$ point
  from $a$?  (Probes teleology --- what $a$ is ``for'' in the cost landscape.)
\end{enumerate}
These exhaust the first- and second-order properties of $\Jcost_A$:
zero set (1,6), minimiser (2), gradient (4,8), pairwise cost (3),
finiteness (5), and composition structure (7).  Any well-formed
question about the landscape reduces to a combination of these
probes.

\emph{Lean:} \texttt{QuestionTaxonomy.modes\_complete}.
\end{proof}

\begin{remark}[Why eight?]
The count ``eight'' is not arbitrary.  The cost landscape $\Jcost$ on
$\mathbb{R}_{>0}$ is a one-dimensional function with: a zero (mode 1),
a minimum (mode 2), a first derivative (modes 4, 8), pairwise structure
(mode 3), global finiteness properties (modes 5, 6), and composition
rules (mode 7).  These are the \emph{complete} set of qualitatively
distinct probes of a smooth convex function --- parallelling the 8 DFT
modes of the eight-tick cycle.  The coincidence of ``8 inquiry modes''
with ``8-tick period'' is structural, not numerical.
\end{remark}

%======================================================================
\section{Implications}\label{sec:implications}
%======================================================================

\begin{enumerate}
\item \textbf{The regress problem is dissolved.}  ``Why does $X$ hold?''
  terminates when $\Jcost(X) = 0$.  The chain of ``why'' questions
  converges to the zero-cost ground state, not to an infinite regress.

\item \textbf{Inquiry IS physics.}  Questions are cost gaps; inquiry is
  gradient descent; answers are cost minima.  There is no distinction
  between ``the universe finding its ground state'' and ``an agent
  finding an answer.''

\item \textbf{RS is self-justifying.}  The meta-closure theorem shows
  RS is the unique zero-cost framework, and the question ``Why RS?'' is
  forced.  This is not circular --- it is a fixed point.
\end{enumerate}

%======================================================================
\section{Comparison with Existing Approaches}\label{sec:prior}
%======================================================================

\begin{center}
\small
\renewcommand{\arraystretch}{1.15}
\begin{tabular}{@{}>{\bfseries}l p{5cm} p{5.5cm}@{}}
\toprule
Feature & Standard & RS (this paper) \\
\midrule
Groenendijk--Stokhof~\cite{Groenendijk1984}
  & Partition semantics & Cost-gap semantics \\
Hintikka~\cite{Hintikka1976}
  & Game-theoretic questions & Forced questions ($\Jcost = 0$) \\
Floridi~\cite{Floridi2011}
  & Levels of abstraction & Cost hierarchy ($\Jcost$ values) \\
G\"{o}del & Incompleteness (syntactic) & Dissolved questions (cost $= \infty$) \\
Tarski & Undefinability of truth & Truth $=$ $\Jcost$-minimality (defined) \\
\bottomrule
\end{tabular}
\end{center}

\begin{remark}
The RS framework does not contradict G\"{o}del or Tarski.  G\"{o}del
sentences are \emph{dissolved} (infinite cost), not ``false'' or
``unprovable in the strong sense.''  Tarski's result applies to
\emph{truth predicates within formal languages}; the RS notion of truth
($\Jcost = 0$) is a \emph{physical} criterion, not a linguistic one.
The two are compatible because they operate in different categories.
\end{remark}

%======================================================================
\section{Discussion}\label{sec:discussion}
%======================================================================

\subsection*{Claims and non-claims}

We formalise questions as cost structures and prove that the RS forcing
chain T0--T8 consists of forced questions.  We do \emph{not} claim that
all philosophical questions can be formulated within this framework, or
that the meta-closure theorem constitutes a ``proof of RS'' in the
circular sense.  The meta-closure is a \emph{fixed point}: RS is
self-consistent, not self-proving.

\subsection*{Open problems}

\begin{enumerate}[label=\textup{(Q\arabic*)},nosep]
\item Can the eight inquiry modes be axiomatised independently
  (i.e.\ proved to be a basis for a ``logic of questions'')?
\item Is there a category-theoretic formulation (e.g.\ questions as
  morphisms in a topos)?
\item Can dissolved questions be ``regularised'' by modifying the
  cost landscape (analogous to renormalisation)?
\item Does the theory-space cost $\Jcost_{\mathcal{T}}$ have a
  measurable proxy (e.g.\ description length + prediction error)?
\end{enumerate}

%======================================================================
\section{Falsification Criteria}\label{sec:falsifiers}
%======================================================================

\begin{falsifier}[Non-forced physical constant]
If a fundamental constant (e.g.\ $\phig$, $\alpha^{-1}$) can be
shown to have multiple zero-cost explanations in $\mathcal{T}$ (the
question is determinate but not forced), meta-closure fails.
\end{falsifier}

\begin{falsifier}[Zero-cost alternative]
If a zero-parameter framework $T \ne \text{RS}$ is exhibited with
$\Jcost_{\mathcal{T}}(T) = 0$, the uniqueness claim is falsified.
\end{falsifier}

%======================================================================
\section{Lean Formalization}\label{sec:lean}
%======================================================================

\begin{center}
\begin{tabular}{@{}ll@{}}
\toprule
\textbf{Module} & \textbf{Content} \\
\midrule
\texttt{Foundation.Inquiry} & Context, Question, answerCost, classification \\
\texttt{Foundation.QuestionTaxonomy} & 8 modes, completeness \\
\texttt{Foundation.InquiryForcingConnection} & T0--T8 as forced questions \\
\texttt{Foundation.MetaClosure} & Theory space, RS zero cost, meta-closure \\
\bottomrule
\end{tabular}
\end{center}

\begin{thebibliography}{9}
\bibitem{WashburnCost2026}
J.~Washburn and M.~Zlatanovi\'{c},
``The Cost of Coherent Comparison,''
arXiv:2602.05753v1, 2026.

\bibitem{WashburnGodel2026}
J.~Washburn,
``G\"{o}del's Theorem Does Not Obstruct Physical Closure,''
Lean: \texttt{Foundation.GodelDissolution}, 2026.

\bibitem{WashburnAboutness2026}
J.~Washburn,
``Optimization-Based Reference,''
Lean: \texttt{Foundation.Reference}, 2026.

\bibitem{WashburnExistence2026}
J.~Washburn,
``The Cost of Existence,''
Recognition Science preprint, 2026.

\bibitem{Groenendijk1984}
J.~Groenendijk and M.~Stokhof,
``Studies on the Semantics of Questions and the Pragmatics of Answers,''
PhD thesis, University of Amsterdam, 1984.

\bibitem{Hintikka1976}
J.~Hintikka,
``The semantics of questions and the questions of semantics,''
\textit{Acta Philosophica Fennica}, 28(4), 1976.

\bibitem{Floridi2011}
L.~Floridi,
\textit{The Philosophy of Information},
Oxford University Press, 2011.
\end{thebibliography}

\end{document}
