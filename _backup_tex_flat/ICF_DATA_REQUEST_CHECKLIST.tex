\documentclass[11pt,letterpaper]{article}

\usepackage[margin=1in]{geometry}
\usepackage[T1]{fontenc}
\usepackage[utf8]{inputenc}
\usepackage{amsmath,amssymb}
\usepackage{booktabs}
\usepackage{xcolor}
\usepackage{hyperref}

\hypersetup{
  colorlinks=true,
  linkcolor=blue!70!black,
  urlcolor=blue!70!black
}

\title{\textbf{Project Helios: ICF Data Request Checklist}\\
\large (What to Provide to Validate RS Coherence-Controlled Fusion)}
\author{Jonathan Washburn (Project Helios)}
\date{2026-01-25}

\begin{document}
\maketitle

\section*{Purpose (plain language)}
We are building an \textbf{ICF-first control and audit system} that computes:
\begin{itemize}
  \item \textbf{\(C_\phi\)}: timing/phase alignment of a pulsed driver schedule (``\(\phi\)-coherence'').
  \item \textbf{\(C_\sigma\)}: symmetry/synchronization derived from diagnostic mode ratios via the RS symmetry ledger.
  \item \textbf{\(S = 1/(1 + C_\phi + C_\sigma)\)}: barrier scale factor.
  \item \textbf{Enhancement \(E\)}: predicted multiplicative improvement in tunneling/reactivity proxy implied by \(S\).
\end{itemize}

This document specifies the \textbf{minimum shot-level diagnostics} needed to run the control/analysis pipeline on real facility data and produce auditable certificate bundles.

\bigskip
\noindent\textbf{Important:} we are not asking for classified design details. If raw data cannot be exported, we include a fallback ``reduced export'' option below.

\section*{A. Minimum viable dataset (per shot)}
\subsection*{A1. Shot identity + metadata}
\begin{itemize}
  \item \textbf{shot\_id}: unique shot identifier.
  \item \textbf{timestamp} (ISO-8601): when the shot occurred.
  \item \textbf{facility}: e.g.\ NIF, OMEGA, OMEGA-EP.
  \item \textbf{target type / campaign} (free text ok).
  \item \textbf{notes}: any caveats (diagnostic outages, known timing offsets, etc.).
\end{itemize}

\subsection*{A2. Pulse timing (for \(C_\phi\))}
Provide two arrays of equal length \(N\) (units: seconds):
\begin{itemize}
  \item \textbf{expected\_pulse\_times[0..N-1]}: planned driver event times for the schedule.
  \item \textbf{measured\_pulse\_times[0..N-1]}: measured driver event times from timing diagnostics.
\end{itemize}

Optional (improves fidelity, still minimal):
\begin{itemize}
  \item \textbf{channel\_phases[0..M-1]} (radians): phase angles for relevant channels in the timing window.
  \item \textbf{channel\_time\_offsets[0..M-1]} (seconds): per-channel offsets / skews (if measured).
  \item \textbf{jitter\_scale} (seconds): facility-defined scale for interpreting RMS timing jitter.
  \item \textbf{skew\_scale} (seconds): facility-defined scale for interpreting RMS channel skew.
\end{itemize}

\subsection*{A3. Symmetry diagnostic mode ratios (for \(C_\sigma\))}
Provide \(K\) mode ratios (dimensionless), plus \emph{definitions}:
\begin{itemize}
  \item \textbf{mode\_ratios[0..K-1]}: e.g.\ \(P_2/P_0\), \(P_4/P_0\), etc. (ideally all near 1.0).
  \item \textbf{mode\_labels[0..K-1]}: strings describing each ratio (e.g.\ ``P2/P0'').
  \item \textbf{mode\_weights[0..K-1]}: nonnegative weights for the symmetry ledger aggregation.
\end{itemize}

\subsection*{A4. Optional outcome proxies (for validation)}
These are not required to compute \(C_\phi,C_\sigma,S,E\), but they are needed to test whether predicted enhancement correlates with performance:
\begin{itemize}
  \item \textbf{yield proxy}: neutron yield / burn proxy / alpha yield proxy (as allowed).
  \item \textbf{bang time}, \textbf{\(\rho R\)} proxy, or other standard campaign metrics (as allowed).
\end{itemize}

\section*{B. Preferred file format}
\subsection*{B1. One JSON per shot (recommended)}
Filename: \texttt{shot\_<shot\_id>.json}

\smallskip
\noindent Minimal JSON structure (example):
\begin{verbatim}
{
  "shot_id": "NIF-YYYYMMDD-####",
  "timestamp": "2026-01-25T00:00:00Z",
  "facility": "NIF",

  "expected_pulse_times": [1.0e-9, 1.618e-9, 2.618e-9],
  "measured_pulse_times": [1.00002e-9, 1.61798e-9, 2.61801e-9],

  "mode_labels": ["P2/P0", "P4/P0"],
  "mode_ratios": [1.03, 0.97],
  "mode_weights": [0.5, 0.5],

  "yield_proxy": null,
  "notes": "ok"
}
\end{verbatim}

\subsection*{B2. CSV (acceptable)}
If CSV is easier, we can provide a schema on request. The key is keeping:
\texttt{expected\_pulse\_times} and \texttt{measured\_pulse\_times} paired and ordered, and mode ratios labeled.

\section*{C. Reduced export option (if raw arrays cannot be shared)}
If you cannot export raw pulse times or mode ratios, provide instead:
\begin{itemize}
  \item \textbf{Computed \(C_\phi\)} (and the jitter RMS / scales used)
  \item \textbf{Computed ledger value} (symmetry ledger) and resulting \textbf{\(C_\sigma\)}
  \item \textbf{Calibration/provenance}: which diagnostics, which time windows, and any smoothing/filters applied
  \item \textbf{Hashes} of the raw internal records (so we can cross-check consistency in an on-site audit)
\end{itemize}

\section*{D. Where this plugs into Project Helios}
Once you have shot JSONs, we can ingest them into the control/analysis pipeline and emit:
\begin{itemize}
  \item \textbf{Per-shot certificates} with computed \(C_\phi,C_\sigma,S,E\) and theorem references.
  \item \textbf{Validation plots/tables}: observed yield proxy vs predicted \(E\).
\end{itemize}

For development without facility data, a synthetic demo exists in the repo:
\begin{itemize}
  \item \texttt{python scripts/icf\_control\_demo.py}
\end{itemize}

\section*{E. Access pointers (public portals)}
These sites document user access workflows (shot data is typically account-controlled):
\begin{itemize}
  \item NIF ``Shot RI Resources'': \url{https://lasers.llnl.gov/for-users/shot-ri-resources}
  \item OMEGA ``Omega Users'': \url{https://www.lle.rochester.edu/omega-users/}
\end{itemize}

\end{document}

