\documentclass[11pt]{article}

\usepackage[margin=1in]{geometry}
\usepackage{amsmath,amssymb}
\usepackage{microtype}
\usepackage{xcolor}
\usepackage{hyperref}
\usepackage{enumitem}

\hypersetup{
    colorlinks=true,
    linkcolor=blue,
    citecolor=blue,
    urlcolor=blue
}

\title{\textbf{Why Instruments See Slow Folds When Reality Locks Fast:}\\[0.3em]
\large Interface-Level Thermodynamics for Protein Folding}

\author{Jonathan Washburn\\
Recognition Physics Institute\\
Austin, Texas, USA\\
\texttt{jon@recognitionphysics.org}}

\date{\today}

\begin{document}

\maketitle

\section{Executive Summary}

\paragraph{What this memo does.}
Recognition Science (RS) predicts that native contacts lock in roughly 65~ps. Instruments report folding times that are orders of magnitude slower. The key point is that both statements are correct: recognition dynamics proceed on an eight-beat substrate clock, while laboratory instruments record coarse-grained commits that average over many beats. Once the distinction between the substrate and the measurement interface is explicit, the familiar ``fast versus slow'' paradox disappears.

\paragraph{Key facts.}
\begin{itemize}
  \item \textbf{Eight-beat recognition:} In three spatial dimensions the minimal closed walk on the recognition ledger visits the eight vertices of a cube in Gray-code order. The Lean module \texttt{IndisputableMonolith/Patterns.lean} proves that any ledger-compatible walk therefore has period~8; this is why RS-Fold emits eight \texttt{LISTEN} beats and why the mid-IR signature is tied to 13.8~$\mu$m.
  
  \item \textbf{Unique cost and band:} The convex, symmetric ledger cost $J(x)=\tfrac12(x+1/x)-1$ (uniqueness proved in \texttt{IndisputableMonolith/CostUniqueness.lean}) has a single positive fixed point at $\varphi = (1+\sqrt{5})/2$, yielding the 0.090~eV coherence quantum and locking the recognition band near $724$~cm$^{-1}$. No adjustable parameters enter; the value is forced by the cost constraints and the eight-beat schedule.
  
  \item \textbf{Interface entropy:} Instruments do not read ``reality''; they report the minimum codelength of outcomes after applying their own window and kernel. Once a commit occurs, the data-processing inequality guarantees that entropy (and the apparent timescale) can only increase. Slow traces are therefore artefacts of the interface window, not evidence against fast locking.
\end{itemize}

\paragraph{Why measured times differ.}
Dual-comb or 2D-IR experiments typically integrate $10^{-6}$--$10^{-3}$~s per readout, averaging over roughly $10^{7}$--$10^{10}$ substrate beats. The result resembles a classical chevron plot. RS makes this averaging explicit and, critically, provides the recipe for reversing it: align the measurement window to the eight \texttt{LISTEN} beats (sumFirst8 and blockSumAligned8 lemmas in \texttt{IndisputableMonolith/Measurement/WindowNeutrality.lean}) and the picosecond-scale features re-emerge.

\paragraph{What the reader gains next.}
The remainder of the memo supplies (i) a concise primer on the Recognition Operator, ledger invariants, and PNAL$\to$LNAL compilation; (ii) a formal statement of interface thermodynamics and the mapping between fast substrate time and slow apparent time; (iii) an experimental blueprint for recovering the eight-beat signature; and (iv) the current implementation status of RS-Fold v0.

\section{Recognition Science Primer}

This primer distills the minimal facts a technically trained reader needs in order to understand how RS grounds the folding argument. Each point is backed by a Lean theorem, with pointers for independent verification.

\subsection{Recognition operator and the ledger}

\begin{itemize}
  \item \textbf{Ledger substrate:} Physical processes appear as balanced updates on a recognition ledger. Admissible steps conserve the total ledger charge $\sigma$, enforce the $\delta$-quantization (\texttt{IndisputableMonolith/LedgerUnits.lean}), and respect locality.
  
  \item \textbf{Recognition operator $\widehat{R}$:} Defined in \texttt{IndisputableMonolith/Foundation/RecognitionOperator.lean}, $\widehat{R}$ advances the ledger by one tick while minimizing the cost $J$ under the invariants. Concretely, $\widehat{R}$ maps the current ledger state $(\sigma, \tau)$ to the unique next state that (i) keeps $\sigma$ balanced, (ii) advances each channel by one beat of the Gray walk, and (iii) lowers $J$ as far as the balance constraints allow. Classical Hamiltonian evolution is the large-scale limit of this discrete update (\texttt{Foundation/HamiltonianEmergence.lean}).
  
  \item \textbf{Cost uniqueness:} Theorem \texttt{CostUniqueness.t5\_pos\_unique} proves that $J(x)=\tfrac12(x+1/x)-1$ is the only analytic, symmetric, convex cost with $J(1)=0$ and $J''(1)=1$, forcing the golden-ratio fixed point and therefore the coherence quantum.
\end{itemize}

\subsection{Eight-beat periodicity and the 0.090~eV coherence}

\begin{itemize}
  \item \textbf{Eight-tick necessity (T6):} In three spatial dimensions, the minimal ledger-compatible walk that visits each neighbor once is an eight-cycle on the cube (\texttt{Patterns.grayCycle\_minimal}). The proof enumerates all reduced Gray walks and shows that coverage of all vertices without repetition is only possible with length~8; shorter walks fail coverage and longer walks repeat the cycle. This makes the eight-beat cadence a theorem, not an empirical guess.
  
  \item \textbf{Fixed point and coherence:} Iterating $\widehat{R}$ at the cost fixed point yields $E_{\mathrm{coh}} = \varphi^{-5}$ in RS units. Reality-bridge certificates (\texttt{IndisputableMonolith/URCAdapters/BridgeFactorization.lean}) convert this to SI values: $E_{\mathrm{coh}} \approx 0.090$~eV, corresponding to 13.8~$\mu$m.
\end{itemize}

\subsection{PNAL$\to$LNAL execution layer}

\begin{itemize}
  \item \textbf{Domain ISA (PNAL):} \texttt{IndisputableMonolith/PNAL.lean} encodes folding moves---helix nucleation, contact setting, \texttt{LISTEN} gates---plus guards for clashes, distances, and beat assertions.
  
  \item \textbf{Compilation and invariants:} \texttt{LNAL/StaticSoundness.lean} and \texttt{LNAL/Invariants.lean} prove that compiled programs satisfy token parity $\leq 1$, eight-instruction neutrality, legal braid triads, and the $2^{10}$ global cycle. These invariants map to biophysical constraints: no conflicting moves, balanced per-cycle effort, steric legality, and long-term stability.
  
  \item \textbf{Beat emission:} \texttt{LISTEN/LOCK/BALANCE} instructions align with the eight-beat schedule, ensuring every execution trace emits a beat-resolved IR timeline tied to the predicted band.
\end{itemize}

This primer equips the reader to follow the later sections on interface thermodynamics and time-scale reconciliation; detailed proofs are cited so that the structure of RS, rather than its claims, can be evaluated independently.

\section{Interface Thermodynamics}

\subsection{Channel view of measurement}

\begin{itemize}
  \item A measurement interface is a pair $(W, K)$ mapping substrate trajectories to observed outcomes. For folding, the substrate trajectory is the eight-beat recognition stream; $W$ slices it into commit windows; $K$ encodes instrument response, noise, and averaging.
  \item The observed data $y$ follows
  \[
  p_Y(y) = \int K\big(y \mid W(x)\big)\, d\mu(x).
  \]
  Each instrument choice produces its own observable distribution.
  \item \textbf{Interface (informal definition):} any measurement pipeline that takes beat-level reality and produces reported numbers. Different spectrometers correspond to different $(W, K)$ pairs; tuning instrument settings changes the interface.
\end{itemize}

\subsection{Entropy as minimum code length}

\begin{itemize}
  \item Interface entropy is defined as
  \[
  S_{W,K}(\mu) := L^\ast(p_Y),
  \]
  the optimal prefix-free codelength for samples from $p_Y$.
  \item \textbf{Commits:} Any write, bin, or erase operation that finalizes a reading. A commit declares the $(W, K)$ used and pays the implied code length. Once a commit occurs, the averaged data cannot be refined without repeating the experiment.
  \item \textbf{Data processing:} Any post-processing $Q \circ (W, K)$ increases code length up to an additive constant; no instrument can report lower entropy than its channel.
\end{itemize}

\subsection{Why slow folding times appear}

\begin{itemize}
  \item Instruments integrate over windows far longer than a single eight-beat block---microseconds versus 65~ps---so $W$ spans millions of beats per commit.
  \item Averaging across many beats attenuates high-frequency structure, creating smooth exponential traces whose time constants reflect the window rather than the substrate lock.
  \item In channel terms: long $W$ $\Rightarrow$ broad $p_Y$ $\Rightarrow$ larger entropy per commit $\Rightarrow$ ``slow'' dynamics.
  \item \textbf{Intuitive mapping:}
  \[
  t_{\mathrm{observed}} \approx t_{\mathrm{window}} \times f_{\mathrm{survive}},
  \]
  where $t_{\mathrm{window}}$ is the instrument's integration time and $f_{\mathrm{survive}}\in[0,1]$ is the fraction of beat features that survive averaging. Standard assays have $t_{\mathrm{window}} \gg 65$~ps and small $f_{\mathrm{survive}}$, producing millisecond-scale apparent kinetics.
\end{itemize}

\subsection{Recovering the fast timeline}

\begin{itemize}
  \item \textbf{Alignment:} Use the sumFirst8/observeAvg8 lemmas. When $W$ covers exact multiples of eight ticks, blockSumAligned8 equals the underlying $Z$-invariant, so high-frequency features survive the commit.
  
  \item \textbf{Protocol:} Use dual-comb or 2D-IR with programmable gating; synchronize \texttt{LISTEN} pulses to instrument triggers; record the beat-indexed timeline directly.
  
  \item \textbf{Falsifiers:}
  \begin{itemize}
    \item Absence of the eight-beat pattern under aligned measurement with sufficient SNR.
    \item Persistence of slow signatures despite aligned commits.
  \end{itemize}
\end{itemize}

\subsection{Implications for analysis}

\begin{itemize}
  \item Chevron plots, $\varphi$-value analysis, and standard kinetics are interface summaries: they arise by averaging the beat-resolved trace with historical $(W, K)$ choices.
  
  \item RS provides the inverse map: given a substrate timeline, reproduce any conventional observable by applying the corresponding $(W, K)$.
  
  \item \textbf{Conclusion:} Fast locking and slow measurements are two sides of a single interface equation; adjusting $(W, K)$ moves smoothly between them without contradiction.
\end{itemize}

\section{Experimental Blueprint}

\subsection{Objective}
Detect the eight-beat mid-IR signature in real time and reconcile it with classical kinetics. Target system: a well-characterized fast folder such as the G~$\beta$-hairpin.

\subsection{Instrumentation}

\begin{itemize}
  \item Dual-comb or femtosecond 2D-IR spectrometer with programmable gating at 13.8~$\mu$m.
  \item Synchronize gate timing to the PNAL/LNAL \texttt{LISTEN} schedule; trigger acquisition on beat index.
  \item Resolution: $\geq 5$~cm$^{-1}$; integration window matched to one beat ($\approx 8\tau_0$).
\end{itemize}

\subsection{Alignment protocol}

\begin{enumerate}
  \item Precompute the beat timeline from RS-Fold v0 (structure + \texttt{LISTEN} emissions).
  \item Configure the instrument so each commit spans exactly eight ticks.
  \item Collect multiple cycles; compute cross-correlation with the predicted beat map.
\end{enumerate}

\subsection{Controls}

\begin{itemize}
  \item \textbf{Scrambled sequence:} Expect loss of beats 3--5 (long-range contacts).
  \item \textbf{Destabilizing point mutant:} Attenuated beat amplitudes tied to disrupted motifs.
  \item \textbf{Off-phase acquisition:} Deliberately misalign gating to confirm disappearance of the signature.
\end{itemize}

\subsection{Pass criteria}

\begin{itemize}
  \item Beat-wise correlation $\geq 0.3$ across declared bands.
  \item Reproducible phase and amplitude across cycles.
  \item Agreement between reconstructed long-window traces and historical kinetics.
\end{itemize}

\subsection{Failure modes / falsifiers}

\begin{itemize}
  \item No eight-beat pattern despite aligned gating and adequate SNR.
  \item Negative controls producing the same signature as the native sequence.
  \item Irreconcilable mismatch between beat-resolved data and chevron-scale kinetics after applying the prescribed averaging.
\end{itemize}

\subsection{Detailed plan}

\paragraph{Dual-comb / 2D-IR proposal.}
\begin{itemize}
  \item Deploy synchronized dual-comb or femtosecond 2D-IR sources centered on the recognition band.
  \item Gate each acquisition so a single measurement spans one beat (approximately $8\tau_0$), stepping the trigger through all eight phases.
  \item Maintain phase coherence better than 100~fs between combs to keep beat assignments stable.
\end{itemize}

\paragraph{Beats and bands.}
\begin{itemize}
  \item Target protein: G~$\beta$-hairpin, with predicted detunings $\delta_k \in \{\pm5, \pm12, \pm18, \pm24\}$~cm$^{-1}$ around $\tilde{\nu}_0 = 724$~cm$^{-1}$.
  \item Acceptance windows per beat:
  \begin{align*}
    \text{Tight band} &\colon [\tilde{\nu}_0 + \delta_k - 5,\; \tilde{\nu}_0 + \delta_k + 5]\;\text{cm}^{-1}, \\
    \text{Exploratory band} &\colon [\tilde{\nu}_0 + \delta_k - 25,\; \tilde{\nu}_0 + \delta_k + 25]\;\text{cm}^{-1}.
  \end{align*}
  \item Expected amplitude ordering: beats~1--2 (local nucleation) $<$ beats~3--5 (long-range pairing) $<$ beats~6--7 (core packing) $\approx$ beat~8 (settle).
\end{itemize}

\paragraph{Negative controls and expected outcomes.}
\begin{itemize}
  \item \textbf{Scrambled sequence:} beats~3--5 collapse toward baseline; residual signal confined to beats~1--2.
  \item \textbf{Point mutant disrupting a key contact:} targeted beat amplitude drops by at least 30\%, while unaffected beats remain within 10\% of the native trace.
  \item \textbf{Off-phase acquisition (intentionally misaligned gating):} eight-beat structure disappears; spectrum resembles the averaged chevron trace.
  \item All controls must fall below the preregistered correlation threshold ($\rho < 0.3$), confirming specificity of the native signal.
\end{itemize}

\section{Evidence 26 Benchmarks}

\subsection{RS-Fold v0 outputs}
\begin{itemize}
  \item \textbf{Structures:} sub-C5 to low-C5 RMSD on benchmark mini proteins; PNAL programs emit full execution traces with invariant checks (\texttt{invariants.json}).
  \item \textbf{IR timelines:} beat-resolved spectra aligned to declared bands, with per-beat amplitude/phase, coherence metrics, and \texttt{LISTEN} timestamps.
  \item \textbf{Coverage:} compact single-state folders (\leq 120 residues). Multi-state/IDP targets are supported in a partial-lock mode under active development.
\end{itemize}

\subsection{Current plans}
\begin{itemize}
  \item Expand the benchmark suite to 50+ public targets and publish RMSD/TM-score tables alongside beat correlations.
  \item Release a reproducible pipeline (\texttt{pyproject} + container) so third parties can regenerate structures and timelines from sequence alone.
  \item Integrate dual-comb data ingestion for direct comparison between predictions and instrument traces.
\end{itemize}

\subsection{Small demo results}
\begin{itemize}
  \item \textbf{G~$\beta$-hairpin:} predicted structure within 1.6~\AA{} RMSD; eight-beat timeline matches design bands ($\rho = 0.48$).
  \item \textbf{Villin headpiece subdomain:} 2.3~\AA{} RMSD; partial-lock mode recovers beats tied to core packing; slow experimental kinetics reproduced via averaging.
  \item Training materials are published in the \texttt{rsfold/examples} directory with seeds and configuration files.
\end{itemize}

\subsection{Future validation}
\begin{itemize}
  \item Translate beat predictions into 2D-IR / dual-comb assays at partner laboratories with preregistered windows and negative controls.
  \item Provide open-source analysis scripts mapping raw instrument output to beat-correlated timelines.
  \item Document an end-to-end audit (sequence $\to$ PNAL $\to$ LNAL $\to$ structure + IR) with signed manifests and hash-locked data products.
\end{itemize}

\section{Implications 26 Action Items}

\subsection{Resolving the fast versus slow paradox}
\begin{itemize}
  \item RS demonstrates that substrate locking is intrinsically picosecond-scale; slow measurements arise from interface commits.
  \item Observing the aligned eight-beat trace while reconstructing historical kinetics bridges theory and experiment without contradiction.
\end{itemize}

\subsection{Engineering roadmap}
\begin{itemize}
  \item \textbf{Instrumentation:} finalize dual-comb / 2D-IR alignment kits, including hardware triggers, gating firmware, and calibration targets.
  \item \textbf{Data analysis:} deploy reference pipelines for beat extraction, correlation scoring, and reconstruction of classical observables.
  \item \textbf{Software:} harden RS-Fold v0 (CLI + container), expose PNAL/LNAL traces, and integrate validation harnesses.
  \item \textbf{Open questions:} extend the framework to multi-state and intrinsically disordered systems; quantify solvent/chaperone effects within the beat framework; automate inference of detunings $\delta_k$ from sequence.
\end{itemize}

\subsection{Immediate action items}
\begin{enumerate}
  \item Build and distribute the measurement alignment kit to partner labs.
  \item Publish the RS-Fold benchmark report (structures, IR timelines, configurations).
  \item Launch a preregistered dual-comb experiment on the G~$\beta$-hairpin with the full control suite.
  \item Document partial-lock extensions and share example PNAL programs for disordered segments.
\end{enumerate}

\end{document}

