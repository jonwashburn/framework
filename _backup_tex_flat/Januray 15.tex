\documentclass[aps,preprint,12pt]{revtex4-2}
% ============================================================
% second_trial_v1.tex
% Generated from "Second part.tex" (2026-01-13) to resolve a
% few internal consistency issues (w8/alpha, E0, m_e section,
% and missing bibliography key).
% ============================================================

% ============================
% Packages
% ============================
\usepackage[T1]{fontenc}
\usepackage[utf8]{inputenc}

\usepackage{amsmath,amssymb,mathtools}
\usepackage{bm}
\usepackage{graphicx}
\usepackage{xcolor}
\usepackage{microtype}
\microtypesetup{expansion=false}

\usepackage{siunitx}
\sisetup{per-mode=symbol}
\DeclareSIUnit\au{a.u.}
\DeclareSIUnit\angstrom{\text{\AA}}
\DeclareSIUnit\parsec{pc}

\usepackage{booktabs}
\usepackage{tabularx}

% Highlighting (optional)
\usepackage{soul}
\newcommand{\jwedit}[1]{#1}

\usepackage{tikz}
\usepackage{tikz-cd}
\usetikzlibrary{positioning}

\usepackage[title]{appendix}

% Bibliography / citations (load before hyperref)
\usepackage{natbib}
\setcitestyle{square, comma, numbers,sort&compress}

% Hyperref must remain late
\usepackage[
  bookmarks=true,
  linktocpage=true,
  pdfpagelabels=true,
  plainpages=false,
  hyperfigures=true,
  colorlinks=true,
  linkcolor=blue,
  citecolor=blue,
  urlcolor=blue
]{hyperref}
\urlstyle{same}

% Theorem environments
\newtheorem{theorem}{Theorem}
\newtheorem{lemma}[theorem]{Lemma}
\newtheorem{axiom}[theorem]{Axiom}
\newtheorem{proposition}[theorem]{Proposition}
\newtheorem{corollary}[theorem]{Corollary}
\newtheorem{definition}[theorem]{Definition}
\newtheorem{remark}[theorem]{Remark}
\newtheorem{observation}[theorem]{Observation}
\newtheorem{example}[theorem]{Example}

% Proof environment
\newenvironment{proof}[1][Proof]{\par\noindent\textit{#1.}\ }{\hfill\(\square\)\par}

\newcommand{\RR}{\mathbb{R}}
\renewcommand{\thefootnote}{\fnsymbol{footnote}}

\begin{document}

\title{Applications of the Foundations of Recognition Science Framework --- Derived Implications and Physical Constants}

\author{Sebastian Pardo-Guerra}
\email{sebas@recognitionphysics.org}
\affiliation{Recognition Physics Institute}

\author{Megan Simons}
\email{msimons@recognitionphysics.org}
\affiliation{Recognition Physics Institute}

\author{Anil Thapa}
\email{athapa@recognitionphysics.org}
\affiliation{Recognition Physics Institute}

\author{Jonathan Washburn}
\email{washburn@recognitionphysics.org}
\affiliation{Recognition Physics Institute}

\author{Brett Werner}
\email{bwerner@recognitionphysics.org}
\affiliation{Recognition Physics Institute}

\begin{abstract}
This manuscript develops applications of the cost-first \emph{Recognition Science} (RS) ledger framework introduced in our companion Foundations paper.
Assuming the ledger primitives (integer ticks, quantized postings, closed-cycle cancellation, scalar potentials) and the existence of a self-similar interface scale \(\phi\), we specialize the ledger to the three-cube \(Q_3\) and, under explicit bridge assumptions to effective continuum physics, derive benchmark numerical implications.
We (i) compute an 8-tick projection gap weight \(w_8^{\mathrm{proj}}\) from the \(\phi\)-ladder neutral spectrum and state an effective Thomson-limit coefficient \(w_8^{\mathrm{eff}}\) used in the coupling bridge, (ii) obtain a high-precision structural expression for the fine structure constant \(\alpha\) (the Geometric Standard Coupling at \(Q^2=0\)), (iii) present a structurally anchored electron-mass model that requires an explicit unit-bridge calibration, and (iv) record a diagnostic structural correspondence for the Hubble-tension ratio with clear limitations and missing steps required for a cosmological derivation.
\end{abstract}

\maketitle

\newpage

\section{Introduction: Scope and relation to the Foundations framework}

This manuscript accompanies our Foundations paper on \emph{Recognition Science} (RS) by developing concrete \emph{applications} of its formal ledger framework.
The Foundations paper establishes the cost-first forcing chain and fixes the ledger primitives (atomic ticks, quantized double-entry postings, closed-cycle conservation, scalar potentials) together with a canonical self-similar interface scale \(\phi\).
Here we assume those framework results and focus on the \emph{derived implications} needed to extract benchmark numerical quantities spanning quantum to cosmological scales, while keeping all bridge assumptions and calibrations explicit.

The motivation is the status of fundamental constants in contemporary physics.
The fine structure constant \(\alpha\approx 1/137\), particle masses such as the electron mass \(m_e\approx 0.511\,\mathrm{MeV}\), and cosmological parameters such as the Hubble constant \(H_0\) are among the best-measured inputs in modern physics \citep{CODATA2022, Planck2018, SH0ES2022}.
In the Standard Model of particle physics and \(\Lambda\)CDM cosmology, these quantities are typically treated as empirical parameters to be measured rather than derived from deeper structural necessities \citep{ParticleDataGroup2022}.
Quantum field theory and general relativity provide extraordinarily successful dynamical frameworks, but they do not by themselves fix the numerical values of the couplings and masses that enter them \citep{Weinberg1995, Peskin1995}.
The fine structure constant in particular has long been viewed as a paradigmatic ``magic number'' \citep{Feynman1985}.

At the same time, precision cosmology highlights tensions that are difficult to resolve within standard assumptions.
The \emph{Hubble tension}---a discrepancy between early-universe (CMB) inferences of \(H_0\) and late-universe (distance-ladder) determinations---has persisted at high statistical significance \citep{Planck2018, SH0ES2022, Riess2022}.
Whether this reflects new physics, underestimated systematics, or deeper structural constraints remains an open question.

The attempt to explain fundamental constants from first principles has a long history.
Early numerological proposals lacked a stable mathematical forcing mechanism \citep{Eddington1936}.
Modern unification programs can relate constants to deeper structures, but commonly introduce landscapes or additional choices (e.g., compactification data) rather than uniquely fixing observed values \citep{Polchinski1998, Susskind2003}.
Discrete and information-theoretic approaches have suggested that spacetime and dynamics may emerge from informational substrates \citep{Wheeler1989, Landauer1961}, but concrete, tightly constrained derivations of multiple benchmark constants remain rare.

In RS, the aim is different: start from a minimal discrete recognition ledger and ask what quantities are \emph{structurally forced} once the bookkeeping is fixed.
The companion Foundations paper supplies the internal forcing chain and the discrete objects.
The present manuscript supplies the derived calculations and (where needed) explicit bridge assumptions that connect ledger invariants to effective continuum observables.

Within this companion-applications scope, we present three benchmark targets with \emph{explicitly separated levels of theoretical completeness}:
\begin{itemize}
    \item \textbf{Fine structure constant (\(\alpha\)).} Under an explicit ledger-to-QED mapping, the inverse constant \(\alpha^{-1}\) at the Thomson limit (\(Q^2=0\), on-shell) is expressed in terms of integer invariants of \(Q_3\) and interface weights built from \(\ln\phi\) and the 8-tick \(\phi\)-spectrum. We distinguish the \emph{projection} gap weight \(w_8^{\mathrm{proj}}\) (computed in Section~\ref{subsec:w8-definition}) from the \emph{effective} Thomson-limit coefficient \(w_8^{\mathrm{eff}}\) used in \eqref{eq:alpha_formula}. With \(w_8^{\mathrm{eff}}\), we obtain the benchmark value \(\alpha^{-1}=137.0359991185\ldots\).
    \item \textbf{Electron mass (\(m_e\)).} We record a structurally anchored electron-mass model combining the same ledger invariants with a unit-bridge calibration (Section~\ref{subsec:unit-system}), yielding \(m_e\approx 0.511\,\mathrm{MeV}\) when the bridge constant is fixed. We make the calibration step explicit to avoid conflating this with a parameter-free derivation.
    \item \textbf{Hubble ratio.} We record the structural correspondence \(H_0^{\mathrm{late}}/H_0^{\mathrm{early}}=13/12\) and state clearly why this is not yet a cosmological derivation: a ledger-to-cosmology mapping and measurement-model pipeline are still missing.
\end{itemize}

These benchmark quantities arise from the same small set of structural integers \((11,\allowbreak 12,\allowbreak 13,\allowbreak 17,\allowbreak 102,\allowbreak 103)\). The remaining inputs are the interface weights \((\ln\phi,\, w_8^{\mathrm{proj}},\, w_8^{\mathrm{eff}})\).

To keep the logical boundary explicit, throughout this paper:
\begin{itemize}
    \item We treat as \textbf{inputs from the Foundations framework} the discrete ledger primitives and the existence of the interface scale \(\phi\).
    \item We introduce \textbf{only} the additional definitions and assumptions required to compute the constants (e.g., the definition of \(w_8^{\mathrm{proj}}\), the statement of \(w_8^{\mathrm{eff}}\), and the ledger-to-QED mapping assumptions).
\end{itemize}

\noindent\textbf{Organization.}
Section~\ref{sec:implications} introduces the interface closure constraint and the \(Q_3\) geometric integers, defines and computes \(w_8^{\mathrm{proj}}\) and states \(w_8^{\mathrm{eff}}\), states the ledger-to-QED mapping assumptions, and derives \(\alpha\).
It then presents the electron-mass model and concludes with the Hubble-ratio correspondence and its limitations.
\section{Derived Implications and Physical Constants}\label{sec:implications}

The cost-first framework developed in the companion Foundations paper yields discrete time, quantized ledger units, closed-cycle conservation, scalar potentials, and a unique cost functional anchored at a self-similar scaling constant \(\phi\).
To derive physical constants, we specialize to the three-dimensional hypercube cell \(Q_3\) (8 vertices, 12 edges, 6 faces) and use a lossless interface map that relates discrete ledger sequences to continuum quantities.

The roadmap of this section is as follows: we first establish (Section~\ref{subsec:interface-closure}) that \(\phi\) follows from an order-2 interface closure implemented via a Fibonacci-type recursion. We then identify the geometric integers available in \(Q_3\) (Section~\ref{subsec:w8-definition}) and compute the projection gap weight \(w_8^{\mathrm{proj}}\) that quantifies the cost of forcing a continuous \(\phi\)-ladder onto an 8-tick discrete clock, and we state the effective Thomson-limit coefficient \(w_8^{\mathrm{eff}}\) used in the coupling benchmark. After stating explicit ledger-to-QED mapping assumptions (Section~\ref{subsec:ledger-qed-mapping}), we derive the fine structure constant (Section~\ref{subsec:alpha}) from these invariants. We then present an electron-mass model with an explicit unit-bridge calibration (Section~\ref{subsec:electron-mass}) and a structural correspondence for the Hubble tension (Section~\ref{subsec:hubble-correspondence}), concluding with a unified summary.

\subsection{Interface Closure and the Golden Ratio}\label{subsec:interface-closure}

The discrete recognition ledger operates on integer ticks and quantized postings, while effective continuum physics describes smooth trajectories and real-valued observables. A lossless interface between these domains requires a scaling constant that preserves information across hierarchies. We show that minimal-memory requirements force a unique choice: the Golden Ratio.

This result is foundational for all subsequent derivations. Without \(\phi\), there would be no canonical way to relate ledger invariants to continuum quantities. The appearance of the Golden Ratio here is not numerological \citep{Eddington1936} but follows from an explicit information-theoretic constraint \citep{Landauer1961}.

\begin{theorem}[Interface Closure]
A lossless, local, order-2 coding map \(\mathcal{M}: \mathbb{Z}_{\ge 0}\to \RR_{>0}\) with a self-similar ratio \(\mathcal{M}(n+1)=\lambda\,\mathcal{M}(n)\) yields \(\lambda=\phi=(1+\sqrt{5})/2\) provided the minimal-memory (order-2) constraint is implemented via the Fibonacci-type recursion \(\mathcal{M}(n+1)=\mathcal{M}(n)+\mathcal{M}(n-1)\).
\end{theorem}

\begin{proof}[Proof]
Assume \(\mathcal{M}:\mathbb{Z}_{\ge 0}\to\RR_{>0}\) satisfies the self-similarity condition
\begin{equation}
\mathcal{M}(n+1)=\lambda\,\mathcal{M}(n)\qquad\text{for all }n\ge 0,
\label{eq:self_similarity}
\end{equation}
for some constant \(\lambda\in\RR\), and that the order-2 (minimal-memory) closure is implemented by the Fibonacci-type recursion
\begin{equation}
\mathcal{M}(n+1)=\mathcal{M}(n)+\mathcal{M}(n-1)\qquad\text{for all }n\ge 1.
\label{eq:fib_recursion}
\end{equation}
Because \(\mathcal{M}(n)\in\RR_{>0}\) for all \(n\), in particular \(\mathcal{M}(n-1)>0\) for \(n\ge 1\), so division by \(\mathcal{M}(n-1)\) is permitted.

Fix any \(n\ge 1\). Applying \eqref{eq:self_similarity} twice gives
\[
\mathcal{M}(n)=\lambda\,\mathcal{M}(n-1),\qquad
\mathcal{M}(n+1)=\lambda\,\mathcal{M}(n)=\lambda^2\,\mathcal{M}(n-1).
\]
On the other hand, \eqref{eq:fib_recursion} and \(\mathcal{M}(n)=\lambda\mathcal{M}(n-1)\) imply
\[
\mathcal{M}(n+1)=\mathcal{M}(n)+\mathcal{M}(n-1)
=\lambda\,\mathcal{M}(n-1)+\mathcal{M}(n-1)
=(\lambda+1)\,\mathcal{M}(n-1).
\]
Equating the two expressions for \(\mathcal{M}(n+1)\) and dividing by \(\mathcal{M}(n-1)>0\) yields
\[
\lambda^2=\lambda+1.
\]
Solving this quadratic equation gives
\[
\lambda=\frac{1\pm\sqrt{5}}{2}.
\]
Since \(\mathcal{M}(n)>0\) for all \(n\) and \(\mathcal{M}(n+1)=\lambda\mathcal{M}(n)\), we must have \(\lambda>0\). Therefore the only admissible root is
\[
\lambda=\frac{1+\sqrt{5}}{2}=\phi.
\]
This proves the claim.
\end{proof}

\textbf{Result:} The discrete--continuous interface scaling constant is uniquely forced to be
\[
\phi = \frac{1+\sqrt{5}}{2} \approx 1.618033988749\ldots
\]
This is not introduced as a numerical fit: once the order-2 closure is implemented by the Fibonacci-type update, the Golden Ratio is forced algebraically. In the remainder of this paper \(\phi\) serves as the discrete--continuous ``bridge constant'' connecting ledger invariants (integers, cycle counts, edge configurations) to continuum observables (couplings, masses, energies).

\subsection{Geometric foundation in \(Q_3\)}

The ledger structure must be specialized to a concrete cell geometry to compute physical constants. We use the three-dimensional hypercube \(Q_3\), whose combinatorial structure provides the ``integers of reality.'' The choice of dimension \(D=3\) reflects the observed spatial dimensionality of our universe; within this constraint, the hypercube is the simplest and most symmetric cell. Higher-dimensional cubes \(Q_D\) would yield different integer sets and correspondingly different physical constants, suggesting a potential structural explanation for why spacetime is \((3+1)\)-dimensional rather than, say, \((4+1)\)-dimensional.

Within a single 3-cube cell \(Q_3\):
\begin{itemize}
    \item Vertices: \(2^3=8\) (the fundamental 8-tick discrete clock period)
    \item Edges: \(3\cdot 2^{2}=12\) (total update capacity per cell)
    \item Faces: \(2\cdot 3=6\) (boundary degrees of freedom)
\end{itemize}
Under the atomic-update constraint established in the Foundations framework, exactly one edge is active per tick, leaving
\[
E_{\mathrm{passive}} = 12-1 = 11
\]
passive edges available as a structural ``environment'' per tick. This integer \(11\) appears directly in the \(\alpha\) formula below and represents the virtual field-mode capacity that dresses each recognition event.

We will also use the classical integer \(W=17\) of wallpaper (plane symmetry) groups \citep{ConwaySymmetry} and the composite normalization \(6\times 17=102\), with the Euler-closure increment giving \(103\), as structural integers in the curvature correction term (below).

\textbf{Why wallpaper groups (and why \(17\))?}
In the ledger-to-continuum bridge, several ingredients (boundary modes, interface transport, and any curvature-like correction) are naturally evaluated on \emph{2D slices} of the \(Q_3\) cell, because the cube's primitive boundaries are its faces. The wallpaper groups provide the complete classification of discrete translational symmetry types for periodic structure in the Euclidean plane; hence \(W=17\) supplies a canonical, non-tunable integer that summarizes the full planar symmetry ``catalog'' available to any face-local periodic organization.
The factor \(6\times W\) reflects that \(Q_3\) has six faces (a face-summed planar-symmetry normalization), and the ``Euler-closure'' increment to \(103\) encodes that the six faces are not independent planes but are coupled by global closure of a single cell. We emphasize that introducing \(W\) here is a \emph{bridge choice} (a geometric normalization hypothesis) rather than an internal ledger theorem; its role is to supply a principled, geometry-only integer scale for the curvature correction term.

\begin{axiom}[Planar-symmetry normalization for curvature (bridge postulate)]
\label{ax:wallpaper-curvature}
In the ledger-to-continuum bridge used for the Thomson-limit coupling, any curvature-like correction is normalized by the complete catalog of planar periodic symmetry types available on the faces of the \(Q_3\) cell. Concretely, we take the face-local symmetry capacity to be \(W=17\) (the wallpaper groups), sum this uniformly over the six faces to obtain \(6W=102\), and encode global cell closure by the increment \(6W+1=103\).
In this bridge, the resulting curvature correction is taken to be proportional to \((6W+1)/(6W)\) with the continuum measure factor \(\pi^5\), yielding the specific normalization
\[
\delta_\kappa=\frac{6W+1}{6W\,\pi^5}=\frac{103}{102\pi^5}.
\]
\end{axiom}

\textbf{Result:} The key geometric integers from \(Q_3\) are
\[
\{6, 8, 11, 12, 13, 17, 102, 103\}.
\]
These are not fitting parameters but combinatorial necessities arising from the cell topology and atomic-update constraint. They form the structural ``alphabet'' for the constant derivations.

\subsection{Definition and computation of the projection gap weight \(w_8^{\mathrm{proj}}\)}\label{subsec:w8-definition}

The 8-tick cycle (as established in the Foundations framework) provides a distinguished period. However, the continuous \(\phi\)-ladder (established in Section~\ref{subsec:interface-closure}) is incommensurate with this discrete clock: sampling \(\phi^t\) on 8 ticks produces a pattern that is not purely constant (DC) but contains oscillatory ``neutral'' content. The \emph{projection} gap weight \(w_8^{\mathrm{proj}}\) measures the unavoidable information cost---the ``strain'' or ``penalty''---of forcing this continuous self-similar scaling onto discrete 8-tick periodicity, using the canonical DFT-8 projection and geometric weights defined below.

This quantity is defined within a specific canonical projection scheme: (i) the canonical \(\phi\)-pattern, (ii) the unitary Fourier decomposition natural for 8-tick time-translation symmetry, and (iii) a chosen geometric weighting that encodes phase structure and \(\phi\)-ladder attenuation. We now derive it step-by-step.

\begin{definition}[Canonical \(\phi\)-pattern]
Define the canonical 8-tick sample of the \(\phi\)-ladder by
\[
p(t) = \phi^t, \qquad t\in\{0,1,\dots,7\}.
\]
This is the minimal ``pure scaling'' sequence compatible with the 8-tick discrete clock and the self-similar interface scale \(\phi\).
\end{definition}

\begin{definition}[Unitary DFT-8 decomposition]
Let \(\omega = e^{-2\pi i/8}\). Define the unitary DFT-8 coefficients
\[
c_k = \frac{1}{\sqrt{8}}\sum_{t=0}^{7} p(t)\, \omega^{-tk} = \frac{1}{\sqrt{8}}\sum_{t=0}^{7} \phi^t \omega^{-tk}, \qquad k=0,1,\dots,7.
\]
The unitary normalization ensures that the DFT preserves total energy (Parseval's theorem). The \(k=0\) mode is the DC component; modes \(k\neq 0\) represent neutral (mean-free) oscillatory content.
\end{definition}

\begin{definition}[Mode energies and geometric weights]
Define the mode energies
\[
A_k = |c_k|^2, \qquad k=0,1,\dots,7.
\]
For the neutral modes (\(k\neq 0\)), define geometric weights
\[
g_k(\phi) = \sin^2\!\left(\frac{\pi k}{8}\right)\,\phi^{-k}, \qquad g_0=0.
\]
The \(\sin^2\) factor encodes 8-tick phase geometry, while \(\phi^{-k}\) encodes an attenuation across the \(\phi\)-ladder: higher-index modes are progressively suppressed within this projection rule. The weighted neutral-mode sum is
\[
C_{\mathrm{raw}}(\phi) = \sum_{k=1}^{7} A_k\,g_k(\phi).
\]
\end{definition}

\begin{definition}[Projection gap weight \(w_8^{\mathrm{proj}}\)]
Define the projection gap weight as the normalized, geometrically weighted projection coefficient
\begin{equation}
w_8^{\mathrm{proj}} = 64\,\frac{C_{\mathrm{raw}}(\phi)}{E_{\mathrm{tot}}(\phi)},
\label{eq:w8-definition}
\end{equation}
where \(E_{\mathrm{tot}} = \sum_{k=0}^{7} A_k = \sum_{t=0}^{7} |p(t)|^2\) is the total DFT energy (by Parseval's theorem for unitary DFT). The normalization by \(E_{\mathrm{tot}}\) ensures scale invariance under \(p\mapsto \lambda p\). The prefactor \(64 = 8\times 8\) is the normalization adopted in this draft to express the weighted neutral fraction as a per-8-tick-per-\(Q_3\)-cell coefficient in the subsequent coupling bookkeeping.
\end{definition}

Numerically, this evaluates to
\[
w_8^{\mathrm{proj}} = \frac{348 + 210\sqrt{2} - (204 + 130\sqrt{2})\,\phi}{7} \approx 2.49056927545\ldots
\]
The closed-form expression follows from evaluating the DFT-8 sum and the above geometric weights analytically. Within this projection scheme, \(w_8^{\mathrm{proj}}\) is a computed (non-fitted) quantity determined by the \(\phi\)-sequence and the specified weighting rule.

\textbf{Result:} The projection gap weight \(w_8^{\mathrm{proj}}\) is a computed (non-fitted) quantity that quantifies the discrete--continuous incommensurability under the projection scheme above. It is a consequence of:
\begin{itemize}
    \item The forced 8-tick period (from \(Q_3\) vertices),
    \item The forced \(\phi\) scaling (from interface closure),
    \item The forced DFT-8 decomposition (unique unitary basis for 8-tick time-translation),
    \item The geometric weights encoding phase structure and attenuation.
\end{itemize}
This value is the canonical projection coefficient multiplying \(\ln(\phi)\), representing the additive (log) cost per multiplicative \(\phi\)-scale step in the projected gap term \(f_{\mathrm{gap}}^{\mathrm{proj}}=w_8^{\mathrm{proj}}\ln(\phi)\).

\begin{remark}[Comparison with simple energy ratio]
A simpler diagnostic quantity is the neutral-vs-DC energy ratio
\[
w_8^{\mathrm{ratio}} = \sum_{k=1}^{7}\frac{|c_k|^2}{|c_0|^2} \approx 0.970692434668\ldots,
\]
which measures how much non-DC energy exists relative to the DC component. However, \(w_8^{\mathrm{ratio}}\) does not include (i) the geometric weights \(g_k(\phi)\) that capture phase geometry and \(\phi\)-ladder attenuation, or (ii) the normalization and scaling required to produce a dimensionless RS coefficient suitable for multiplying \(\ln(\phi)\). The projection gap weight \(w_8^{\mathrm{proj}}\) is the appropriate coefficient for the projected neutral-spectrum cost.
\end{remark}

\begin{remark}[Effective Thomson-limit gap coefficient \(w_8^{\mathrm{eff}}\)]\label{rem:w8-eff}
In the QED mapping used for the Thomson-limit coupling benchmark in Section~\ref{subsec:alpha}, the gap term uses an \emph{effective} coefficient \(w_8^{\mathrm{eff}}\) that differs slightly from the projection weight \(w_8^{\mathrm{proj}}\):
\[
w_8^{\mathrm{eff}} \approx 2.488254397846\ldots
\]
This difference reflects an additional transport/projection step from the 8-tick neutral spectrum to the Thomson-limit (on-shell, \(Q^2=0\)) coupling normalization. In this draft we treat that transport as part of the ledger-to-QED bridge hypothesis; accordingly, \(w_8^{\mathrm{proj}}\) is computed internally from the projection scheme above, while \(w_8^{\mathrm{eff}}\) is taken as a bridge-level input pending an explicit derivation of the transport rule.
\end{remark}

\subsection{Mapping the ledger structure to QED}\label{subsec:ledger-qed-mapping}

Up to this point, all results (the interface scale \(\phi\), the \(Q_3\) geometric integers, and the projection gap weight \(w_8^{\mathrm{proj}}\)) are internal ledger theorems. To connect these discrete invariants to the electromagnetic fine structure constant \(\alpha\), we now introduce explicit mapping assumptions that relate ledger primitives to QED structures in the continuum limit (including the effective Thomson-limit coefficient \(w_8^{\mathrm{eff}}\) in the gap term).

These assumptions are not derivable from the ledger alone; they constitute a \emph{bridge hypothesis} that identifies discrete recognition events with photon-mediated interactions. The success of the numerical prediction for \(\alpha\) (below) provides empirical support for this mapping, but the mapping itself must be stated explicitly to maintain logical transparency \citep{Weinberg1995,Peskin1995}.

\begin{definition}[Ledger-to-QED mapping (assumptions)]
We posit the following correspondence under a continuum limit:
\begin{enumerate}
    \item Recognition events \(\leftrightarrow\) photon-mediated interactions.
    \item Closed ledger cycles \(\leftrightarrow\) virtual loops (with cycle cancellation matching charge conservation constraints).
    \item Passive edges (\(E_{\mathrm{passive}}=11\)) \(\leftrightarrow\) virtual field-mode capacity dressing an interaction at a tick.
    \item The scalar potential structure \(\leftrightarrow\) electromagnetic potential with gauge freedom.
    \item The 8-tick cycle \(\leftrightarrow\) a minimal discrete momentum-space structure, with \(w_8^{\mathrm{proj}}\) capturing neutral spectral distribution and \(w_8^{\mathrm{eff}}\) the Thomson-limit transported coefficient entering the coupling benchmark.
\end{enumerate}
\end{definition}

\begin{remark}[Status]
Items above are explicit \emph{mapping assumptions} (not purely internal ledger theorems). The numerical prediction for \(\alpha\) below is conditional on this mapping.
\end{remark}

\subsection{Derivation of the fine structure constant}\label{subsec:alpha}

\begin{definition}[Geometric Standard Coupling]
The \emph{Geometric Standard Coupling} is defined as the dimensionless coupling computed from the RS ledger--continuum interface for a unit posting on \(Q_3\) at the Thomson limit \(Q^2=0\) (on-shell), i.e., the infrared baseline rather than high-energy scales such as the \(Z\)-pole. In this draft the coupling expression includes an explicit 8-tick projection/transport correction term \(f_{\mathrm{gap}}\) (defined below); we reserve ``zero gap'' language for settings where that correction is absent.
\end{definition}

\begin{theorem}[Fine structure constant (Thomson limit / IR baseline)]
Under the ledger-to-QED mapping, the inverse fine structure constant \(\alpha^{-1}\) at the Thomson limit (\(Q^2=0\), on-shell) is given by
\begin{equation}
\alpha^{-1} = 4\pi\cdot 11 \;-\; f_{\mathrm{gap}} \;+\; \delta_\kappa,
\label{eq:alpha_formula}
\end{equation}
with gap term \(f_{\mathrm{gap}}=w_8^{\mathrm{eff}}\ln(\phi)\) (Remark~\ref{rem:w8-eff}) and curvature correction
\[
\delta_\kappa = \frac{103}{102\pi^5}.
\]
This yields the benchmark value \(\alpha^{-1} \approx 137.0359991185\ldots\).
\end{theorem}

\begin{proof}[Proof]
We prove \eqref{eq:alpha_formula} as a consequence of the stated ledger-to-QED bridge assumptions together with the definitions in this section.

\textit{(i) Baseline contribution \(4\pi\cdot 11\).}
Under the atomic-update constraint on a \(Q_3\) cell, exactly one edge is active per tick, hence the passive-edge multiplicity is
\[
E_{\mathrm{passive}}=12-1=11
\]
as established in Section~\ref{sec:implications}. Under the ledger-to-QED bridge (Definition~\ref{subsec:ledger-qed-mapping}), a unit posting dressed by the passive environment contributes an isotropic momentum-space measure factor \(4\pi\) per passive edge at the Thomson (IR) limit. Therefore the baseline inverse-coupling contribution is \(4\pi\,E_{\mathrm{passive}}=4\pi\cdot 11\).

\textit{(ii) Gap term \(f_{\mathrm{gap}}\).}
By definition in this section, the 8-tick projection/transport correction at the Thomson limit is
\[
f_{\mathrm{gap}} = w_8^{\mathrm{eff}}\ln(\phi),
\]
where \(w_8^{\mathrm{eff}}\) is the Thomson-limit transported coefficient (Remark~\ref{rem:w8-eff}) and \(\ln(\phi)\) converts the multiplicative \(\phi\)-scaling to an additive cost per scale step.

\textit{(iii) Curvature correction \(\delta_\kappa\).}
By the curvature-correction ansatz stated in the theorem, the remaining deterministic correction term is
\[
\delta_\kappa=\frac{103}{102\pi^5}.
\]

\textit{(iv) Assembly.}
The ledger-to-QED bridge posits that the inverse coupling at the Thomson limit is obtained by taking the baseline contribution, subtracting the projection/transport gap penalty, and adding the curvature correction. Combining (i)--(iii) yields
\[
\alpha^{-1}=4\pi\cdot 11 - f_{\mathrm{gap}} + \delta_\kappa,
\]
which is exactly \eqref{eq:alpha_formula}. Substituting the stated numerical values then gives the quoted benchmark \(\alpha^{-1}\approx 137.0359991185\ldots\).
\end{proof}

\begin{remark}[Coupling constant ``running'' in RS]
In standard QED, the fine structure constant ``runs'' with energy scale due to loop corrections, with \(\alpha(Q^2)\) increasing from the infrared baseline \(\alpha(0)\) to higher values at ultraviolet scales \citep{Peskin1995}. In RS, this running is not an ad-hoc quantum effect but emerges as a mechanical consequence of incommensurability between scales. As one moves away from the fundamental atomic tick \(\tau_0\), the ledger must ``re-quantize'' the cost accounting, creating the appearance of a changing coupling constant. The Geometric Standard Coupling computed here is identified with the physical \(\alpha\) at the Thomson limit \(Q^2=0\) (on-shell), providing the IR baseline for this running behavior.
\end{remark}

\textbf{Result:} The inverse fine structure constant at the Thomson limit is
\[
\alpha^{-1} = 4\pi \cdot 11 - w_8^{\mathrm{eff}}\ln(\phi) + \frac{103}{102\pi^5} \approx 137.0359991185\ldots
\]
This is a \emph{structural form} conditioned on the stated ledger-to-QED bridge (including the Thomson-limit transport encapsulated by \(w_8^{\mathrm{eff}}\)). The numerical value follows from:
\begin{itemize}
    \item The passive-edge count \(E_{\mathrm{passive}}=11\) (from \(Q_3\) atomic updates),
    \item The effective gap coefficient \(w_8^{\mathrm{eff}} \approx 2.4882544\) (with the projection weight \(w_8^{\mathrm{proj}} \approx 2.4905693\) computed from the DFT-8 analysis of the \(\phi\)-pattern),
    \item The interface scale \(\phi\) (forced by minimal-memory interface closure),
    \item The curvature correction integers \(102, 103\) (from wallpaper groups and Euler closure).
\end{itemize}
The experimental CODATA 2022 value is \(\alpha^{-1}=137.035999177(21)\) \citep{CODATA2022}, agreeing with the RS prediction to within \(\sim 10^{-8}\) fractional precision. In this draft, this numerical agreement is presented as \emph{supportive} of the stated ledger-to-QED bridge assumptions, while emphasizing that the bridge itself remains an explicit assumption set.

\subsection{Unit system and energy scale conversion}\label{subsec:unit-system}

The ledger produces dimensionless invariants (ratios, integer counts, computed weights). To interpret mass scales in physical units (e.g., MeV), we require a bridge that connects dimensionless ledger quantities to the SI unit system.

We introduce a conversion based on the discrete--continuous interface hierarchy:
\begin{equation}
E_0 = \frac{\hbar c}{a_0\,\phi^{15}},
\label{eq:fundamental-energy-scale}
\end{equation}
where \(a_0\) is the Bohr radius (\(\approx 0.529\,\text{\AA}\)) and \(\phi^{15}\) represents the hierarchy depth used in the present unit bridge. (An earlier draft mistakenly wrote \(\phi^8\) while quoting the numerical value corresponding to \(\phi^{15}\).) Numerically, \(E_0 \approx 2.734\) eV.

This energy scale serves as the ``unit of account'' for converting dimensionless RS mass factors into physical units. The choice of \(a_0\) as the length scale is consistent with the identification of \(\alpha\) as the electromagnetic coupling, but it also means the bridge imports atomic-scale input (since \(a_0 = \hbar/(m_e c \alpha)\) in standard physics). Accordingly, in this draft we treat \(E_0\) (and any downstream mass-scale conversion derived from it) as an explicit calibration choice, and we do not interpret agreement in \(m_e\) as parameter-free evidence absent a ledger-only derivation of the unit bridge.

\subsection{Derivation of the electron mass}\label{subsec:electron-mass}

Having derived the electromagnetic coupling \(\alpha\), we now address the electron mass \(m_e\). In RS, mass is not an independent input but emerges as a ``closure defect'': the energy required to maintain coherence at the discrete--continuous interface across multiple hierarchical scales.

The intuition is as follows. A massless excitation propagates freely on the ledger without accumulating interface strain. A massive particle, by contrast, must continuously reconcile discrete ledger ticks with its de Broglie wavelength, creating a persistent ``gap'' cost. This gap cost, when summed over the \(\phi\)-hierarchy and normalized to physical units via \(E_0\), yields the rest mass.

\begin{definition}[Structural mass (dimensionless)]
Define the dimensionless structural mass scale
\begin{equation}
m_{\mathrm{struct}}^{\mathrm{dimless}} = 2^{-22}\,\phi^{51}.
\label{eq:mstruct-dimless}
\end{equation}
\end{definition}

\begin{remark}[Calibration status / canceled ``51'']
The appearance of the exponent \(51\) in \(m_{\mathrm{struct}}^{\mathrm{dimless}}\) is a bookkeeping decomposition used to separate a convenient reference hierarchy level from the residue exponent \(\delta\) defined below. In the final expression \eqref{eq:electron-mass-physical} it cancels algebraically (see below), so it does \emph{not} represent an independent hierarchy-level choice in this draft. The remaining non-structural input is the unit-bridge calibration constant \(\kappa\).
\end{remark}

\textbf{Electron-mass closure model.}
We now formalize the residue exponent \(\delta\) used in the electron-mass ansatz as an explicit set of model postulates, followed by a proposition that collects their consequences.

\begin{axiom}[Hierarchy-accumulated closure defect (mass ansatz)]
\label{ax:mass-closure}
There exists a dimensionless residue exponent \(\delta\) such that the physical electron mass can be expressed as
\[
m_e = \kappa\,2^{-22}\,\phi^{\delta},
\]
where \(\kappa\) is a unit-bridge conversion fixed by the chosen energy-scale calibration (Section~\ref{subsec:unit-system}).
\end{axiom}

\begin{axiom}[Residue exponent decomposition]
\label{ax:delta-decomposition}
The residue exponent decomposes additively as
\[
\delta=\delta_{\mathrm{struct}}+\delta_{\mathrm{EM}},
\]
where \(\delta_{\mathrm{struct}}\) depends only on \(Q_3\)-specialized structural integers and \(\delta_{\mathrm{EM}}\) captures leading electromagnetic dressing at the Thomson limit.
\end{axiom}

\begin{axiom}[Structural baseline and electromagnetic dressing (model selection)]
\label{ax:delta-model}
With \(W\) the planar symmetry capacity used in the curvature normalization (Section~\ref{subsec:implications}), \(E_{\mathrm{total}}\) the total edge count of \(Q_3\), and \(E_{\mathrm{passive}}\) the passive-edge count under atomic updates, we take
\[
\delta_{\mathrm{struct}} = 2W + \frac{W+E_{\mathrm{total}}}{4E_{\mathrm{passive}}}.
\]
With \(\alpha\) the Thomson-limit coupling derived above, we take a truncated low-order dressing series
\[
\delta_{\mathrm{EM}} = \alpha^2 + E_{\mathrm{total}}\alpha^3.
\]
The truncation order and coefficients in \(\delta_{\mathrm{EM}}\), and the specific rational normalization in \(\delta_{\mathrm{struct}}\), are explicit model-selection choices in this draft.
\end{axiom}

\begin{proposition}[Residue exponent and electron-mass formula]
Let \(W=17\), \(E_{\mathrm{total}}=12\), \(E_{\mathrm{passive}}=11\), and \(\alpha\) as derived above. Define
\begin{equation}
\delta = 2W + \frac{W+E_{\mathrm{total}}}{4E_{\mathrm{passive}}} + \alpha^2 + E_{\mathrm{total}}\alpha^3.
\label{eq:delta-residue}
\end{equation}
Here \(\delta\) is the \emph{dimensionless residue exponent} controlling the hierarchy accumulation \(m_e\propto \phi^{\delta}\) in the mass model below.
Then the electron mass is modeled as
\begin{equation}
m_e = \kappa \, m_{\mathrm{struct}}^{\mathrm{dimless}} \, \phi^{\delta-51},
\label{eq:electron-mass-physical}
\end{equation}
where \(\kappa\) is a unit conversion fixed by the chosen energy-scale bridge. Equivalently,
\[
m_e = \kappa\,2^{-22}\,\phi^{\delta},
\]
so the structural (dimensionless) content is carried by \(\delta\), while \(\kappa\) supplies the unit bridge.
\end{proposition}

\textbf{Result (calibrated):} Using the energy-scale bridge \(E_0\) from equation~\eqref{eq:fundamental-energy-scale} and fixing \(\kappa\) as a single unit calibration, the structural formula yields
\[
m_e \approx 0.510998946\,\text{MeV}.
\]
The experimental value is \(m_e = 0.51099895000(15)\,\text{MeV}\) \citep{CODATA2022}, agreeing to within \(\sim 10^{-7}\) fractional precision under this calibration. In contrast to the \(\alpha\) section, the electron-mass construction here depends explicitly on the unit bridge (and therefore should be read as a calibrated structural model rather than a parameter-free prediction).

\subsection{Structural correspondence with the Hubble tension}\label{subsec:hubble-correspondence}

The preceding derivations span quantum scales (\(\alpha\), \(m_e\)). We now briefly address a cosmological puzzle: the \emph{Hubble tension}---a \(5\sigma\) discrepancy between early-universe (Planck CMB) and late-universe (SH0ES distance-ladder) determinations of the Hubble constant \(H_0\) \citep{Planck2018,SH0ES2022,Riess2022}.

The tension is typically stated as \(H_0^{\mathrm{early}} \approx 67.4\) versus \(H_0^{\mathrm{late}} \approx 73.0\) in units of \(\si{\kilo\meter\per\second\per\mega\parsec}\), yielding a ratio \(\sim 1.083\). Standard \(\Lambda\)CDM assumes a single true \(H_0\), so this discrepancy either signals new physics or reflects unaccounted systematics.

In RS, the ledger operates at discrete hierarchical levels. Different measurement pipelines (early CMB vs.\ late supernovae) may probe different ledger hierarchy positions, yielding structurally distinct ``effective \(H_0\)'' values. If this is the correct interpretation, the ratio should reflect a simple ledger integer ratio.

\begin{observation}[Structural correspondence: Hubble ratio]
A numerical structural correspondence is
\begin{equation}
\frac{H_0^{\mathrm{late}}}{H_0^{\mathrm{early}}} \approx \frac{13}{12} = 1.08333\ldots
\label{eq:hubble-ratio}
\end{equation}
which matches the observed ratio \(73.0/67.4 \approx 1.0831\) at the few \(\times 10^{-4}\) level using these rounded headline values (the precise numerical agreement depends on dataset choices and quoted central values).
\end{observation}

\begin{remark}[Limitations]
Unlike the \(\alpha\) and \(m_e\) sections, this correspondence is \emph{not} a complete cosmological derivation. A full prediction would require an explicit ledger-to-cosmology mapping and a model of how early- and late-universe inference pipelines probe different ledger invariants.
\end{remark}

\textbf{Result:} The Hubble tension ratio \(H_0^{\mathrm{late}}/H_0^{\mathrm{early}} = 13/12\) appears as a structural pattern using the \(Q_3\) integers. The integers \(12\) (total edges) and \(13\) (next successive integer, associated with Euler face-edge closure) suggest that the early and late measurements probe adjacent hierarchical ledger states. This is a \emph{diagnostic correspondence}, not a derived prediction. To elevate it to the status of the \(\alpha\) derivation would require:
\begin{itemize}
    \item An explicit ledger-to-cosmology mapping (analogous to the ledger-to-QED mapping for \(\alpha\)),
    \item A measurement-model pipeline showing how CMB and distance-ladder observations sample different ledger hierarchy positions,
    \item Derivation of the numerical prefactors (converting ledger ratios to km\,s\(^{-1}\)\,Mpc\(^{-1}\) units).
\end{itemize}
This remains an open direction. The numerical agreement is suggestive but should not be over-interpreted without the full theoretical infrastructure.

\subsection{Unified view (summary)}

The three benchmark targets demonstrate the range and limitations of the RS derived-constants program. We summarize the results and their theoretical status:

\begin{itemize}
    \item \textbf{Quantum coupling (\(\alpha\))}: the Geometric Standard Coupling (Thomson limit, \(Q^2=0\)) expressed from \(Q_3\) integers, \(\ln\phi\), and an effective gap coefficient \(w_8^{\mathrm{eff}}\) (with the projection weight \(w_8^{\mathrm{proj}}\) computed in Section~\ref{subsec:w8-definition}), under an explicit ledger-to-QED mapping.
    \[
    \alpha^{-1} = 4\pi \cdot 11 - w_8^{\mathrm{eff}}\ln(\phi) + \frac{103}{102\pi^5} \approx 137.036.
    \]
    \emph{Status: Structural form (conditional).} Theoretical completeness is high once the ledger-to-QED bridge is specified; in this draft \(w_8^{\mathrm{eff}}\) is treated as a bridge-level input pending an explicit derivation of the Thomson-limit transport rule.

    \item \textbf{Electron mass (\(m_e\))}: A structurally anchored electron-mass model is provided via a ``closure defect'' ansatz combining the same ledger invariants with a unit-conversion bridge (Section~\ref{subsec:unit-system}), yielding \(m_e\approx 0.511\,\mathrm{MeV}\) once the bridge constant \(\kappa\) is fixed.
    \[
    m_e = \kappa \,2^{-22}\phi^{\delta}, \quad \delta = 2W + \frac{W+E_{\mathrm{total}}}{4E_{\mathrm{passive}}} + \alpha^2 + E_{\mathrm{total}}\alpha^3.
    \]
    \emph{Status: Calibrated structural model.} Theoretical completeness is intermediate. The dimensionless structure is forced, but the unit bridge (via \(\kappa\)) is an explicit calibration in this draft.

    \item \textbf{Hubble ratio}: presented as a structural pattern with clear caveats, pending a full ledger-to-cosmology mapping.
    \[
    \frac{H_0^{\mathrm{late}}}{H_0^{\mathrm{early}}} \approx \frac{13}{12} = 1.083.
    \]
    \emph{Status: Diagnostic correspondence, not a derivation.} Theoretical completeness is low. The integer ratio is suggestive, but without a ledger-to-cosmology mapping and measurement-model pipeline, this remains an observed numerical pattern rather than a structural prediction.
\end{itemize}

\textbf{Common thread:} All three results use the same core ingredients---the interface scale \(\phi\), the \(Q_3\) geometric integers \(\{6,8,11,12,13,17,102,103\}\), and the 8-tick gap coefficients (\(w_8^{\mathrm{proj}}\) and the Thomson-limit \(w_8^{\mathrm{eff}}\)). The distinctions lie in (i) the completeness of the domain-specific mapping (QED vs.\ cosmology), (ii) whether a bridge-level coefficient is taken as input (as with \(w_8^{\mathrm{eff}}\) in this draft), and (iii) whether an explicit unit-bridge calibration is required (as in the present \(m_e\) model).

\section{Conclusion}

As a companion applications manuscript, this work isolates the \emph{derived implications} of the Recognition Science framework relevant for benchmark numerical outputs spanning quantum to cosmological scales.

\textbf{Principal result:} Conditioned on explicit mapping assumptions, the framework yields a high-precision structural expression for the fine structure constant (the Geometric Standard Coupling at the Thomson limit \(Q^2=0\)) from discrete ledger invariants and interface weights:
\[
\alpha^{-1} = 4\pi \cdot 11 - w_8^{\mathrm{eff}}\ln(\phi) + \frac{103}{102\pi^5} \approx 137.036,
\]
matching the experimental value to \(\sim 10^{-8}\) fractional precision \citep{CODATA2022}. This is a \emph{structural form} once the ledger-to-QED bridge is specified; the projection weight \(w_8^{\mathrm{proj}}\) is computed explicitly in Section~\ref{subsec:w8-definition}, while \(w_8^{\mathrm{eff}}\) is treated in this draft as a bridge-level coefficient pending an explicit derivation of the Thomson-limit transport rule. The numerical agreement is presented as supportive of the stated bridge assumptions, with the bridge itself remaining explicit.

\textbf{Secondary results:} We also present a structurally anchored electron-mass model with an explicit unit-bridge calibration, yielding \(m_e \approx 0.511\,\text{MeV}\) at benchmark accuracy under that calibration, and a clearly labeled structural correspondence for the Hubble-tension ratio (\(H_0^{\mathrm{late}}/H_0^{\mathrm{early}} = 13/12\)) together with the missing steps required for a cosmological derivation.

\textbf{Broader implications:} The fine structure constant has long been viewed as a paradigmatic ``magic number'' with no deeper explanation \citep{Feynman1985}. The RS derivation suggests a different perspective: \(\alpha\) is not arbitrary but structurally forced by the requirements of maintaining a consistent discrete recognition ledger. The ``running'' of couplings with energy scale---typically attributed to quantum loop corrections \citep{Peskin1995}---emerges in RS as a mechanical consequence of ledger re-quantization across hierarchical scales. This transforms fundamental constants from empirical inputs to be measured into \emph{bookkeeping necessities} to be derived.

\textbf{Open directions:} Extending the electron-mass construction to a fully parameter-free result would require replacing the present unit-bridge calibration with a derivation of the relevant physical scale from ledger primitives alone. Elevating the Hubble correspondence to a full derivation would require an explicit ledger-to-cosmology mapping. Both remain active research frontiers within the RS program. Nonetheless, the \(\alpha\) result demonstrates that the framework can yield high-precision structural predictions for benchmark constants, suggesting that the cost-first ledger structure may encode deep physical necessities previously thought to be contingent features of our universe.

\bibliographystyle{unsrtnat}
\bibliography{second_trial_v1}

\end{document}





