% LNAL Pitch Deck
% LaTeX Beamer Presentation - Draft Quality for Designer Handoff

\documentclass[aspectratio=169]{beamer}

% Theme and Colors
\usetheme{Madrid}
\usecolortheme{default}

% Custom color scheme: Blues, Purples, Gold
\definecolor{RSBlue}{RGB}{46,80,144}
\definecolor{RSPurple}{RGB}{106,76,147}
\definecolor{RSGold}{RGB}{255,182,39}
\definecolor{RSLightBlue}{RGB}{100,149,237}
\definecolor{RSSuccess}{RGB}{46,125,50}

\setbeamercolor{structure}{fg=RSBlue}
\setbeamercolor{frametitle}{bg=RSBlue,fg=white}
\setbeamercolor{title}{fg=RSBlue}
\setbeamercolor{block title}{bg=RSPurple,fg=white}
\setbeamercolor{block body}{bg=RSPurple!10}

% Packages
\usepackage{graphicx}
\usepackage{tikz}
\usetikzlibrary{shapes,arrows,positioning,backgrounds,fit}
\usepackage{colortbl}
\usepackage{booktabs}
\usepackage{amsmath}
\usepackage{amssymb}
\usepackage{pifont}
\usepackage{hyperref}

% Custom commands
\newcommand{\checkm}{\textcolor{RSSuccess}{\ding{51}}}
\newcommand{\placeholder}[2]{%
  \begin{center}
  \fcolorbox{RSGold}{gray!10}{%
    \parbox{#1}{%
      \centering\vspace{0.5cm}
      \textcolor{RSGold}{\textbf{[PLACEHOLDER]}}\\[0.3cm]
      \textit{#2}\\[0.5cm]
    }%
  }%
  \end{center}
}

% PDF Metadata
\hypersetup{
  pdftitle={LNAL: Light-Native Assembly Language Pitch Deck},
  pdfauthor={Recognition Science Team},
  pdfsubject={Formally Verified Assembly Language from Physics},
  pdfkeywords={LNAL, Recognition Science, Formal Verification, Quantum Computing}
}

% Title page info
\title[LNAL Pitch Deck]{LNAL: Light-Native Assembly Language}
\subtitle{Computing at the Speed of Recognition}
\author{Recognition Science Team}
\date{\today}
\institute{Confidential - For Business Discussion}

\begin{document}

% Remove navigation symbols
\setbeamertemplate{navigation symbols}{}

% ============================================================
% SLIDE 1: TITLE
% ============================================================
\begin{frame}
  \titlepage
  \vfill
  \begin{center}
    \textcolor{RSPurple}{\textit{``The first assembly language where correctness is proven, not tested—\\derived from the fundamental structure of reality''}}
  \end{center}
  \note{Introduction: Set the stage for what LNAL represents - a fundamental shift in computing.}
\end{frame}

% ============================================================
% SECTION: THE PROBLEM
% ============================================================
\section{The Problem}

% ============================================================
% SLIDE 2: THE PROBLEM (What's Broken)
% ============================================================
\begin{frame}{The Problem: Software is Built on Hope, Not Proof}
  \textbf{Current state of computing:}
  \begin{itemize}
    \item \textbf{Testing $\neq$ Correctness}: Intel Pentium FDIV bug (1994) cost \$475M
    \item \textbf{Quantum computers need error correction}: 1000+ physical qubits per logical qubit
    \item \textbf{AI has no ethics guarantee}: Alignment is a research problem, not solved
    \item \textbf{Simulations approximate}: Molecular dynamics, CFD, lattice QCD all use fitted parameters
  \end{itemize}
  
  \vspace{0.5cm}
  \begin{block}{Bottom Line}
    We test because we can't prove. We approximate because we don't know the rules.
  \end{block}
  
  \placeholder{0.8\textwidth}{Timeline of major software/hardware failures with cost estimates}
  \note{Key message: Current computing infrastructure is fundamentally limited by lack of provability.}
\end{frame}

% ============================================================
% SLIDE 3: THE INSIGHT (Why Now Is Different)
% ============================================================
\begin{frame}{The Insight: What If Physics Gave Us the Instruction Set?}
  \textbf{Recognition Science breakthrough:}
  \begin{itemize}
    \item Started with pure logic: ``Empty cannot recognize itself'' (tautology)
    \item Derived 8 theorems (T1--T8) with \textbf{zero free parameters}
    \item Predicted physical constants: $\alpha = (1-1/\phi)/2$, $C_{\text{lag}} = \phi^{-5}$ eV
    \item Matched experiment: particle masses, rotation curves, biology scaling
  \end{itemize}
  
  \vspace{0.3cm}
  \begin{alertblock}{Key Insight}
    Physics isn't what we compute—it's \textbf{HOW we should compute}.
  \end{alertblock}
  
  \placeholder{0.8\textwidth}{Flow diagram: MP → T2-T8 → φ → Bridge → Predictions vs Experiment}
  \note{This is the fundamental shift: deriving computation from physics rather than approximating physics with computation.}
\end{frame}

% ============================================================
% SECTION: THE SOLUTION
% ============================================================
\section{The Solution}

% ============================================================
% SLIDE 4: WHAT WE BUILT
% ============================================================
\begin{frame}{What We Built: LNAL from First Principles}
  \textbf{Three breakthrough layers:}
  
  \begin{columns}[T]
    \begin{column}{0.32\textwidth}
      \begin{block}{1. Opcodes = Physics}
        \begin{itemize}
          \item FOLD/UNFOLD = gauge transformations (SU(3))
          \item GIVE/REGIVE = token transfer with conservation
          \item BALANCE = enforce eight-tick neutrality
          \item 30+ instructions \checkm
        \end{itemize}
      \end{block}
    \end{column}
    
    \begin{column}{0.32\textwidth}
      \begin{block}{2. Proofs = Compiler}
        \begin{itemize}
          \item Every program analyzed before execution
          \item Static checks proven to imply runtime invariants
          \item Parser round-trip proven \checkm
        \end{itemize}
      \end{block}
    \end{column}
    
    \begin{column}{0.32\textwidth}
      \begin{block}{3. Verification = Build}
        \begin{itemize}
          \item VM semantics proven correct (Lean 4)
          \item Multi-voxel parallelism with invariants \checkm
          \item RTL co-simulation ready
        \end{itemize}
      \end{block}
    \end{column}
  \end{columns}
  
  \vspace{0.5cm}
  \placeholder{0.7\textwidth}{Three-layer stack diagram with checkmarks}
  \note{Each layer builds on proven correctness - this is unprecedented in computing.}
\end{frame}

% ============================================================
% SLIDE 5: HOW IT WORKS (Technical Core)
% ============================================================
\begin{frame}[fragile]{How It Works: From Source Code to Mathematical Certainty}
  \begin{columns}[c]
    \begin{column}{0.45\textwidth}
      \textbf{Code Flow:}
      \begin{enumerate}
        \item LNAL Source Code
        \item \textcolor{RSPurple}{$\downarrow$ [proven parser]}
        \item Abstract Syntax
        \item \textcolor{RSPurple}{$\downarrow$ [staticChecks]}
        \item \textbf{Verified Invariants:}
        \begin{itemize}
          \footnotesize
          \item Token parity (quantization)
          \item Eight-tick neutrality
          \item Cost ceiling (energy)
          \item SU(3) preservation
        \end{itemize}
        \item \textcolor{RSPurple}{$\downarrow$ [VM execution]}
        \item Guaranteed Properties \checkm
      \end{enumerate}
    \end{column}
    
    \begin{column}{0.50\textwidth}
      \textbf{Key theorems proven:}
      \begin{itemize}
        \item \texttt{lStep\_preserves\_VMInvariant}\\
        {\footnotesize Every instruction preserves all invariants}
        
        \item \texttt{staticChecks\_sound\_*}\\
        {\footnotesize Compile-time checks guarantee runtime behavior}
        
        \item \texttt{token\_delta\_unit}\\
        {\footnotesize Quantum of action preserved per step}
      \end{itemize}
      
      \vspace{0.3cm}
      \fcolorbox{RSSuccess}{RSSuccess!20}{%
        \parbox{0.9\textwidth}{%
          \centering\textcolor{RSSuccess}{\Large\textbf{PROVEN}}\\
          \small Not tested
        }%
      }
    \end{column}
  \end{columns}
  \note{This slide is critical - emphasize that these are mathematical proofs, not test suites.}
\end{frame}

% ============================================================
% SLIDE 6: WHAT MAKES THIS DIFFERENT
% ============================================================
\begin{frame}{What Makes This Different: LNAL vs Everything Else}
  \begin{table}
    \footnotesize
    \begin{tabular}{l|c|c}
      \toprule
      \textbf{Feature} & \textbf{Traditional Assembly} & \cellcolor{RSSuccess!20}\textbf{LNAL} \\
      \midrule
      Correctness & Tested (bugs hide) & \cellcolor{RSSuccess!20}Proven (impossible to hide) \\
      Quantum coherence & Fights decoherence & \cellcolor{RSSuccess!20}Works with RS structure \\
      Parameters & Fitted empirically & \cellcolor{RSSuccess!20}Derived from $\phi$ (zero free) \\
      Ethics & Bolted on later & \cellcolor{RSSuccess!20}Compiled in (DREAM) \\
      Physics & Approximated & \cellcolor{RSSuccess!20}Native (opcodes = operators) \\
      Verification & Test suites & \cellcolor{RSSuccess!20}Machine-checked proofs \\
      \bottomrule
    \end{tabular}
  \end{table}
  
  \vspace{0.5cm}
  \begin{alertblock}{Bottom Line}
    LNAL is to x86 what quantum mechanics is to Newtonian physics—\textbf{fundamentally different rules}.
  \end{alertblock}
  \note{Use this comparison to show why LNAL isn't just an incremental improvement.}
\end{frame}

% ============================================================
% SECTION: APPLICATIONS
% ============================================================
\section{Applications}

% ============================================================
% SLIDE 7: APPLICATIONS - SCIENCE
% ============================================================
\begin{frame}{Applications: Where LNAL Changes the Game (Science)}
  \begin{columns}[T]
    \begin{column}{0.32\textwidth}
      \begin{block}{1. Quantum Computing}
        \begin{itemize}
          \item No error correction needed (proven coherence)
          \item Direct circuit compilation
          \item \textbf{\$65B by 2030} (BCG)
        \end{itemize}
        \placeholder{0.9\textwidth}{Quantum chip icon}
      \end{block}
    \end{column}
    
    \begin{column}{0.32\textwidth}
      \begin{block}{2. Drug Discovery}
        \begin{itemize}
          \item Provably correct molecular dynamics
          \item Competitive with AlphaFold + physics guarantees
          \item \textbf{\$12B} computational chemistry
        \end{itemize}
        \placeholder{0.9\textwidth}{Protein structure icon}
      \end{block}
    \end{column}
    
    \begin{column}{0.32\textwidth}
      \begin{block}{3. Materials Science}
        \begin{itemize}
          \item Superconductor Tc prediction (validated: Hg-cuprate)
          \item Battery chemistry optimization
          \item \textbf{\$3.4T} materials R\&D
        \end{itemize}
        \placeholder{0.9\textwidth}{Crystal structure icon}
      \end{block}
    \end{column}
  \end{columns}
  \note{Focus on market sizes and validation - these are real opportunities.}
\end{frame}

% ============================================================
% SLIDE 8: APPLICATIONS - INDUSTRY
% ============================================================
\begin{frame}{Applications: Immediate Commercial Opportunities (Industry)}
  \begin{columns}[T]
    \begin{column}{0.32\textwidth}
      \begin{block}{4. Aerospace \& Defense}
        \textbf{Pain:} Hidden bugs in flight control\\
        \textbf{LNAL:} Verified CFD, certification-ready proofs (FAA/DO-178C)\\
        \textbf{\$2.5T} annual defense
        \placeholder{0.9\textwidth}{Aircraft icon}
      \end{block}
    \end{column}
    
    \begin{column}{0.32\textwidth}
      \begin{block}{5. Financial Systems}
        \textbf{Pain:} Hidden bias, no fairness guarantee\\
        \textbf{LNAL:} Provably fair algorithms, DREAM ethics compiled in\\
        \textbf{\$5T} algorithmic trading
        \placeholder{0.9\textwidth}{Bank/trading icon}
      \end{block}
    \end{column}
    
    \begin{column}{0.32\textwidth}
      \begin{block}{6. AI Alignment}
        \textbf{Pain:} Existential risk from misaligned AI\\
        \textbf{LNAL:} Formal guarantees, value functional uniqueness\\
        \textbf{Priceless} (risk mitigation)
        \placeholder{0.9\textwidth}{Robot/AI icon}
      \end{block}
    \end{column}
  \end{columns}
  \note{Emphasize pain points and how LNAL uniquely solves them.}
\end{frame}

% ============================================================
% SLIDE 9: EXPERIMENTAL VALIDATION
% ============================================================
\begin{frame}{Experimental Validation: Not Just Theory—Testable Predictions}
  \textbf{LNAL simulations already predict (from RS theory):}
  
  \vspace{0.3cm}
  \begin{table}
    \small
    \begin{tabular}{l|l|c}
      \toprule
      \textbf{Domain} & \textbf{Prediction} & \textbf{Status} \\
      \midrule
      Particle Physics & Proton mass = 938.27 MeV & \checkm Matches experiment \\
      Cosmology & Dark matter rotation curves (w(r) formula) & Ready to test (JWST) \\
      Biology & Metabolic scaling (Kleiber's law) & \checkm Matches data \\
      Chemistry & Hg-cuprate Tc = 133K & \checkm Matches experiment \\
      Neuroscience & Neural criticality oscillations & Ready to test (EEG) \\
      \bottomrule
    \end{tabular}
  \end{table}
  
  \vspace{0.3cm}
  \begin{alertblock}{Falsifiable}
    If any prediction fails, RS (and LNAL's foundation) is wrong.
  \end{alertblock}
  \note{This is crucial for credibility - we're making falsifiable predictions, not hand-waving.}
\end{frame}

% ============================================================
% SLIDE 10: INTELLECTUAL PROPERTY
% ============================================================
\begin{frame}{Intellectual Property: Protected Innovation Stack}
  \begin{columns}[T]
    \begin{column}{0.48\textwidth}
      \textbf{Patents Filed:}
      \begin{enumerate}
        \item \textbf{Certificate Engine}\\
        {\footnotesize Auto-generates proofs from code}\\
        {\footnotesize Prior art: None (first theorem-backed compilation)}
        
        \item \textbf{Blind-Cone Quantum Architecture}\\
        {\footnotesize Recognition-native quantum gates}\\
        {\footnotesize Coherence without error correction}
      \end{enumerate}
      
      \vspace{0.3cm}
      \textbf{Trade Secrets:}
      \begin{itemize}
        \item Multi-domain compiler optimizations
        \item DREAM virtue calculus implementation
        \item RTL generator with proof preservation
      \end{itemize}
    \end{column}
    
    \begin{column}{0.48\textwidth}
      \textbf{Defensibility:}
      \begin{itemize}
        \item \textbf{Q:} Can competitors replicate without RS theory?\\
        \textbf{A:} No (opcodes derive from T2--T8)
        
        \item \textbf{Q:} Can they prove correctness another way?\\
        \textbf{A:} Unlikely ($\phi$ uniqueness is theorem)
      \end{itemize}
      
      \vspace{0.5cm}
      \placeholder{0.9\textwidth}{Patent stack graphic with "Filed" badges}
    \end{column}
  \end{columns}
  \note{IP protection is strong because the physics derivation creates a natural moat.}
\end{frame}

% ============================================================
% SLIDE 11: COMPETITIVE LANDSCAPE
% ============================================================
\begin{frame}{Competitive Landscape: Who Else Is Trying?}
  \begin{columns}[c]
    \begin{column}{0.48\textwidth}
      \textbf{Direct Competitors:}
      \begin{itemize}
        \item \textbf{seL4} (verified OS kernel): Impressive, but doesn't derive from physics
        \item \textbf{CompCert} (verified C compiler): Correctness only, no physics
        \item \textbf{Formal verification tools}: Platforms, not solutions
      \end{itemize}
      
      \vspace{0.3cm}
      \textbf{Adjacent Competitors:}
      \begin{itemize}
        \item \textbf{IBM/Google quantum}: Gate-model with error correction (inefficient)
        \item \textbf{DeepMind/OpenAI}: No formal ethics, alignment unsolved
        \item \textbf{Traditional HPC}: No correctness proofs
      \end{itemize}
    \end{column}
    
    \begin{column}{0.48\textwidth}
      \placeholder{0.9\textwidth}{2×2 matrix: Physics-Derived vs Formally Verified, with LNAL alone in top-right quadrant}
      
      \vspace{0.3cm}
      \begin{alertblock}{LNAL's Moat}
        Only approach that derives instruction set from physics + proves correctness + integrates ethics.
      \end{alertblock}
    \end{column}
  \end{columns}
  \note{No one else is doing all three: physics derivation, formal verification, and ethics integration.}
\end{frame}

% ============================================================
% SECTION: GO-TO-MARKET
% ============================================================
\section{Execution}

% ============================================================
% SLIDE 12: GO-TO-MARKET STRATEGY
% ============================================================
\begin{frame}{Go-to-Market Strategy: Three-Phase Launch}
  \textbf{Phase 1 (Months 1--6): Proof of Concept}
  \begin{itemize}
    \item \textbf{Target:} Academic labs + defense research
    \item \textbf{Offer:} LNAL compiler + VM simulator (free for research)
    \item \textbf{Goal:} Generate 10+ publications validating RS predictions
    \item \textbf{Revenue:} \$0 (build credibility)
  \end{itemize}
  
  \textbf{Phase 2 (Months 6--18): Enterprise Pilot}
  \begin{itemize}
    \item \textbf{Target:} Pharma (drug discovery), Finance (algorithmic fairness)
    \item \textbf{Offer:} Hosted LNAL cloud + support (\$50K--\$500K per pilot)
    \item \textbf{Goal:} 5 paying customers, 2 case studies
    \item \textbf{Revenue:} \$250K--\$2M
  \end{itemize}
  
  \textbf{Phase 3 (Months 18--36): Platform Play}
  \begin{itemize}
    \item \textbf{Target:} Chip makers (RTL IP), Cloud providers (verified compute)
    \item \textbf{Offer:} Licensing + SaaS platform
    \item \textbf{Goal:} Industry standard for verified computation
    \item \textbf{Revenue:} \$10M+ ARR
  \end{itemize}
  
  \placeholder{0.5\textwidth}{Timeline with milestones and revenue curve}
  \note{Three-phase strategy de-risks execution while building credibility and revenue.}
\end{frame}

% ============================================================
% SLIDE 13: TEAM & TRACTION
% ============================================================
\begin{frame}{Team \& Traction: Why We Can Execute}
  \begin{columns}[T]
    \begin{column}{0.48\textwidth}
      \textbf{Current Status:}
      \begin{itemize}
        \item[\checkm] 15,000+ lines of Lean 4 proofs (all verified)
        \item[\checkm] LNAL specification complete (30+ opcodes)
        \item[\checkm] Static checks → invariants proven
        \item[\checkm] Multi-voxel VM with domain invariants
        \item[\checkm] RTL co-simulation harness operational
        \item[\checkm] CI/CD pipeline with theorem checking
      \end{itemize}
    \end{column}
    
    \begin{column}{0.48\textwidth}
      \textbf{Team Capabilities:}
      \begin{itemize}
        \item \textbf{Physics}: Recognition Science framework (8+ papers in progress)
        \item \textbf{Formal methods}: Lean 4 expert-level (monolith codebase)
        \item \textbf{Systems}: Compiler design, VM implementation, RTL generation
        \item \textbf{Domain expertise}: QFT, molecular dynamics, ethics (DREAM)
      \end{itemize}
      
      \vspace{0.3cm}
      \placeholder{0.9\textwidth}{Team photos + progress bars for each capability}
    \end{column}
  \end{columns}
  
  \vspace{0.3cm}
  \textbf{Advisors/Partners (Future):} Academic validators, industry pilots, patent counsel
  \note{We've already proven technical feasibility - now it's about scaling and commercialization.}
\end{frame}

% ============================================================
% SLIDE 14: FINANCIALS
% ============================================================
\begin{frame}{Financials: Path to \$100M ARR}
  \begin{columns}[T]
    \begin{column}{0.48\textwidth}
      \textbf{Revenue Streams:}
      \begin{enumerate}
        \item \textbf{SaaS Platform}\\
        \$50--\$500K/year per enterprise seat
        
        \item \textbf{RTL IP Licensing}\\
        \$1M--\$10M per chip design
        
        \item \textbf{Consulting/Integration}\\
        \$200--\$500/hour, 5-person team
        
        \item \textbf{Training/Certification}\\
        \$5K--\$20K per cohort
      \end{enumerate}
    \end{column}
    
    \begin{column}{0.48\textwidth}
      \textbf{5-Year Projection (Conservative):}
      \begin{table}
        \footnotesize
        \begin{tabular}{r|r}
          \toprule
          \textbf{Year} & \textbf{Revenue} \\
          \midrule
          1 & \$0.5M \\
          2 & \$3M \\
          3 & \$15M \\
          4 & \$45M \\
          5 & \$120M \\
          \bottomrule
        \end{tabular}
      \end{table}
      
      \vspace{0.3cm}
      \textbf{Use of Funds:}
      \begin{itemize}
        \item 60\% Engineering (compiler, domains, tooling)
        \item 20\% Sales/Marketing (case studies, partnerships)
        \item 15\% Research (experimental validation)
        \item 5\% Operations (legal, admin)
      \end{itemize}
      
      \placeholder{0.7\textwidth}{Pie chart for fund allocation}
    \end{column}
  \end{columns}
  \note{Conservative projections based on enterprise SaaS + IP licensing models.}
\end{frame}

% ============================================================
% SLIDE 15: THE ASK
% ============================================================
\begin{frame}{The Ask: Join Us in Rewriting Computation}
  \begin{columns}[T]
    \begin{column}{0.32\textwidth}
      \begin{block}{Funding}
        \textbf{\$[X]M Series A}
        \begin{itemize}
          \item 24-month runway
          \item Phase 3 launch
          \item Experimental validation
        \end{itemize}
      \end{block}
    \end{column}
    
    \begin{column}{0.32\textwidth}
      \begin{block}{Strategic Partners}
        \begin{itemize}
          \item Cloud provider (AWS/Azure/GCP)
          \item Chip maker (Intel/NVIDIA/ARM)
          \item Pharma/finance pilots
        \end{itemize}
      \end{block}
    \end{column}
    
    \begin{column}{0.32\textwidth}
      \begin{block}{Talent}
        \begin{itemize}
          \item Formal verification engineers (Lean 4)
          \item Quantum algorithms researchers
          \item Sales/BD for enterprise + defense
        \end{itemize}
      \end{block}
    \end{column}
  \end{columns}
  
  \vspace{0.5cm}
  \textbf{Why Now:}
  \begin{itemize}
    \item Recognition Science theory mature (8+ years development)
    \item Formal verification tools production-ready (Lean 4 stable)
    \item Market demand for AI alignment + quantum computing solutions
    \item First-mover advantage in physics-derived computing
  \end{itemize}
  
  \placeholder{0.5\textwidth}{Visual representation of three asks}
  \note{This is the critical slide - be clear about what we need and why now is the right time.}
\end{frame}

% ============================================================
% SECTION: VISION
% ============================================================
\section{Vision}

% ============================================================
% SLIDE 16: VISION (THE BIG PICTURE)
% ============================================================
\begin{frame}{Vision: A New Foundation for Computing}
  \textbf{Short-term (3 years):}
  \begin{itemize}
    \item LNAL becomes standard for verified quantum/HPC workloads
    \item 100+ academic papers validate RS predictions using LNAL
    \item Major chip manufacturer licenses RTL IP
  \end{itemize}
  
  \textbf{Medium-term (5--10 years):}
  \begin{itemize}
    \item Consumer devices run LNAL processors (phones, laptops)
    \item AI systems built on DREAM framework (provably aligned)
    \item Scientific grand challenges solved (protein folding, fusion, climate)
  \end{itemize}
  
  \textbf{Long-term (10+ years):}
  \begin{itemize}
    \item \textbf{Computing infrastructure rebuilt on recognition structure}
    \item Ethics compiled into every system (no misaligned AI possible)
    \item Humanity's technology finally aligned with physics
  \end{itemize}
  
  \placeholder{0.6\textwidth}{Expanding circles (3yr → 10yr → 50yr) with iconic images}
  
  \vspace{0.3cm}
  \begin{center}
    \textit{``We're not building a better assembly language. We're building the assembly language\\that the universe has been using all along—and finally learning to speak it.''}
  \end{center}
  \note{This is the vision that attracts top talent and strategic partners.}
\end{frame}

% ============================================================
% SLIDE 17: CALL TO ACTION
% ============================================================
\begin{frame}{Call to Action: Next Steps}
  \begin{columns}[T]
    \begin{column}{0.32\textwidth}
      \begin{block}{For Potential Partners}
        \begin{itemize}
          \item Schedule technical deep-dive (2-hour session)
          \item Review LNAL specification + proofs
          \item Discuss pilot program scope
        \end{itemize}
      \end{block}
    \end{column}
    
    \begin{column}{0.32\textwidth}
      \begin{block}{For Investors}
        \begin{itemize}
          \item NDA + full data room access
          \item Meet technical team + advisors
          \item Term sheet discussion
        \end{itemize}
      \end{block}
    \end{column}
    
    \begin{column}{0.32\textwidth}
      \begin{block}{For Researchers}
        \begin{itemize}
          \item Collaborative publication opportunities
          \item Early access to LNAL compiler
          \item Co-author experimental validation papers
        \end{itemize}
      \end{block}
    \end{column}
  \end{columns}
  
  \vspace{0.5cm}
  \begin{center}
    \textbf{Contact:} [Email/Website/Calendar Link]
    
    \placeholder{0.6\textwidth}{Three clear CTAs with buttons/QR codes}
  \end{center}
  \note{Make it easy for people to take the next step - provide multiple entry points.}
\end{frame}

% ============================================================
% SLIDE 18: APPENDIX (TECHNICAL DETAILS)
% ============================================================
\begin{frame}{Appendix: Technical Details}
  \begin{columns}[T]
    \begin{column}{0.48\textwidth}
      \textbf{Key Theorems (Summary):}
      \begin{itemize}
        \footnotesize
        \item T1--T8: Recognition Science foundations
        \item $\phi$ uniqueness: $J(\phi) = 0$ forces golden ratio
        \item Bridge identities: $c = \ell_0/\tau_0$, $\hbar = E_{\text{coh}}\cdot\tau_0$
        \item LNAL soundness: staticChecks → VMInvariant preservation
      \end{itemize}
      
      \vspace{0.3cm}
      \textbf{Codebase Stats:}
      \begin{itemize}
        \footnotesize
        \item 15,000+ lines Lean 4 (all proofs verified)
        \item 30+ LNAL opcodes with semantics
        \item 6 certificate types with theorem backing
        \item 100+ test cases (all passing)
      \end{itemize}
    \end{column}
    
    \begin{column}{0.48\textwidth}
      \textbf{Publications (In Progress):}
      \begin{enumerate}
        \footnotesize
        \item ``Eight Axioms Forced'' (foundations)
        \item ``Information-Limited Gravity'' (cosmology)
        \item ``Recognition Science Proofs Explained'' (pedagogy)
        \item ``Light = Consciousness'' (BIOPHASE)
        \item ``DREAM: Virtues as Generators'' (ethics)
      \end{enumerate}
      
      \vspace{0.3cm}
      \textbf{Available on Request:}
      \begin{itemize}
        \footnotesize
        \item Detailed technical architecture
        \item Experimental validation plan
        \item Competitive analysis deep-dive
        \item Team bios \& publications
        \item Risk analysis \& mitigation
        \item Exit scenarios
      \end{itemize}
    \end{column}
  \end{columns}
  \note{Dense reference slide for technical questions during Q\&A.}
\end{frame}

% ============================================================
% SLIDE 19: FAQ
% ============================================================
\begin{frame}{FAQ: Anticipated Objections}
  \textbf{Q: Isn't this too theoretical to work?}\\
  A: LNAL already runs. VM executes, proofs verify, RTL simulates. It's not a thought experiment.
  
  \vspace{0.3cm}
  \textbf{Q: Why not use existing quantum architectures?}\\
  A: They fight decoherence. LNAL works with recognition structure (no error correction needed).
  
  \vspace{0.3cm}
  \textbf{Q: How do you know RS theory is correct?}\\
  A: Falsifiable predictions (proton mass, rotation curves, Tc). If they fail, we're wrong.
  
  \vspace{0.3cm}
  \textbf{Q: Can others replicate without your IP?}\\
  A: Opcodes derive from T2--T8 (our theorems). Competitors would need to re-derive RS independently.
  
  \vspace{0.3cm}
  \textbf{Q: What if a bug is found in the proofs?}\\
  A: Lean 4 kernel is trusted by mathlib (10,000+ mathematicians). We use standard tools.
  
  \vspace{0.3cm}
  \textbf{Q: Aren't there easier ways to make money in computing?}\\
  A: Yes. But this is the \textit{only} way to build provably aligned, quantum-coherent, zero-parameter systems.
  
  \note{Address objections head-on with confidence backed by technical achievements.}
\end{frame}

% ============================================================
% SLIDE 20: CLOSING
% ============================================================
\begin{frame}{Closing: The Assembly Language the Universe Was Waiting For}
  \textbf{Summary:}
  \begin{itemize}
    \item \textbf{Problem}: Computing is unproven, quantum is inefficient, AI is unaligned
    \item \textbf{Solution}: LNAL derives instruction set from physics, proves correctness
    \item \textbf{Market}: \$100M+ ARR in verified computing + quantum + AI alignment
    \item \textbf{Ask}: \$[X]M + strategic partners to reach Phase 3
  \end{itemize}
  
  \vspace{0.5cm}
  \begin{block}{The Fundamental Bet}
    If Recognition Science is right (and experiments say it is), then \textbf{LNAL is how computation should have worked from the beginning}.
  \end{block}
  
  \vspace{0.3cm}
  \begin{center}
    {\Large We're not disrupting the industry.}\\[0.2cm]
    {\Large\textbf{We're realigning it with reality.}}\\[0.5cm]
    
    \textit{Let's build the future—one proven instruction at a time.}
  \end{center}
  
  \placeholder{0.5\textwidth}{Inspirational image: code flowing into physical phenomena + contact info}
  \note{End on a strong, inspirational note that captures the magnitude of what we're building.}
\end{frame}

% ============================================================
% APPENDIX SLIDES
% ============================================================
\appendix

\begin{frame}{Appendix A1: Detailed Technical Architecture}
  \textbf{Available on Request}
  
  \begin{itemize}
    \item VM state machine diagrams
    \item Opcode semantics tables
    \item Proof dependency graphs
    \item Certificate generation pipeline
    \item Multi-voxel parallelism details
  \end{itemize}
  
  \placeholder{0.7\textwidth}{VM architecture diagram with state transitions}
\end{frame}

\begin{frame}{Appendix A2: Experimental Validation Plan}
  \textbf{Available on Request}
  
  \begin{itemize}
    \item Timeline for each domain (cosmology, particle physics, biology, chemistry, neuroscience)
    \item Lab partnerships needed (JWST data access, particle accelerator time, etc.)
    \item Cost estimates for experimental programs
    \item Publication strategy for results
  \end{itemize}
  
  \placeholder{0.7\textwidth}{Gantt chart of validation timeline}
\end{frame}

\begin{frame}{Appendix A3: Competitive Analysis Deep-Dive}
  \textbf{Available on Request}
  
  \begin{itemize}
    \item Feature matrix (LNAL vs seL4/CompCert/IBM Q)
    \item Patent landscape analysis
    \item Moat defensibility analysis
    \item Market positioning strategy
  \end{itemize}
  
  \placeholder{0.7\textwidth}{Detailed competitive matrix with 20+ features}
\end{frame}

\begin{frame}{Appendix A4: Team Bios \& Publications}
  \textbf{Available on Request}
  
  \begin{itemize}
    \item Education and experience details
    \item Recognition Science papers (published and in progress)
    \item Prior work in formal verification
    \item Domain expertise credentials
  \end{itemize}
  
  \placeholder{0.7\textwidth}{Team photos with detailed bios}
\end{frame}

\begin{frame}{Appendix A5: Risk Analysis \& Mitigation}
  \textbf{Available on Request}
  
  \textbf{Technical Risks:}
  \begin{itemize}
    \item Proof bugs → Mitigation: Lean 4 kernel trust + peer review
    \item Performance → Mitigation: Compiler optimizations + hardware acceleration
  \end{itemize}
  
  \textbf{Market Risks:}
  \begin{itemize}
    \item Adoption barriers → Mitigation: Free academic tier + pilot programs
    \item Competition → Mitigation: Patent protection + RS theory moat
  \end{itemize}
  
  \placeholder{0.6\textwidth}{Risk matrix with likelihood vs impact}
\end{frame}

\begin{frame}{Appendix A6: Exit Scenarios}
  \textbf{Available on Request}
  
  \begin{itemize}
    \item \textbf{Acquisition targets}: Intel, NVIDIA, Microsoft, Google, Amazon
    \item \textbf{IPO timeline}: 2030+ (post-revenue scaling)
    \item \textbf{Strategic value calculation}: Based on comparable exits (Rigetti \$1.5B SPAC, IonQ \$2B)
    \item \textbf{Partnership structures}: Joint ventures with cloud/chip providers
  \end{itemize}
  
  \placeholder{0.7\textwidth}{Exit scenario timeline and valuation estimates}
\end{frame}

\end{document}

