\documentclass[11pt,twocolumn]{article}

% --- Packages ---
\usepackage[utf8]{inputenc}
\usepackage[T1]{fontenc}
\usepackage{amsmath,amssymb,amsthm}
\usepackage{mathtools}
\usepackage{graphicx}
\usepackage{booktabs}
\usepackage{hyperref}
\usepackage{geometry}
\usepackage{xcolor}
\usepackage{enumitem}

\geometry{margin=1in}

% --- Theorem environments ---
\newtheorem{definition}{Definition}[section]
\newtheorem{theorem}{Theorem}[section]
\newtheorem{lemma}[theorem]{Lemma}
\newtheorem{corollary}[theorem]{Corollary}
\newtheorem{proposition}[theorem]{Proposition}
\newtheorem{remark}{Remark}[section]
\newtheorem{hypothesis}{Hypothesis}[section]

% --- Custom commands ---
\newcommand{\R}{\mathbb{R}}
\newcommand{\Z}{\mathbb{Z}}
\newcommand{\N}{\mathbb{N}}
\newcommand{\phival}{\varphi}
\newcommand{\Pdrive}{P_{\text{drive}}}
\newcommand{\Pout}{P_{\text{out}}}
\newcommand{\Pthermal}{P_{\text{thermal}}}
\newcommand{\Pstored}{P_{\text{stored}}}

% --- Title ---
\title{%
  Generator-Mode Operation, Active Detuning Governors,\\
  and Responsible Disclosure for\\
  High-Gain Resonant Field Systems%
}

\author{%
  Recognition Science Research Institute\\
  \texttt{[correspondence address]}
}

\date{February 1, 2026}

\begin{document}

\maketitle

% ============================================================================
% ABSTRACT
% ============================================================================
\begin{abstract}
If a resonant rotating-field system exhibits high gain (energy or force amplification), the dominant risks become runaway behavior and unsafe disclosure. This paper defines rigorous energy accounting for ``generator mode'' operation, specifies a safety governor based on intentional detuning (phase slip injection), and establishes a governance model for staged disclosure.

We formalize the thermodynamic conditions for self-sustaining operation, analyze runaway hazard modes, and prove that active detuning collapses system efficiency to prevent catastrophic failure. The framework is strictly data-gated: no claims of over-unity or novel effects are made absent verified measurements following the protocols of NM-0 (Foundation) and NM-1 (Engineering and Metrology).

\textbf{Keywords:} safety governor, detuning, phase slip, resonance collapse, energy accounting, runaway prevention, responsible disclosure, export control
\end{abstract}

% ============================================================================
% 1. INTRODUCTION
% ============================================================================
\section{Introduction}
\label{sec:intro}

\subsection{Motivation}

This memo addresses two failure modes for high-gain resonant systems:

\begin{enumerate}
    \item \textbf{Technical failure (runaway):} Loss of load or controller fault causes uncontrolled acceleration, leading to mechanical failure or worse.
    \item \textbf{Social failure (premature disclosure):} Publishing results before adequate verification enables misuse or reputational damage.
\end{enumerate}

Both failures must be prevented by design, not by caution alone.

\subsection{Scientific Posture}

All content in this paper is hypothesis-gated and measurement-gated:
\begin{itemize}
    \item No claim of ``over-unity'' or novel effect is made.
    \item All hypotheses are formalized with falsifiers.
    \item Claims require auditable data products per NM-1 protocols.
\end{itemize}

\subsection{Relationship to Project Architecture}

\begin{itemize}
    \item \textbf{NM-0 (Foundation):} Geometry, scheduling, resonance-map predictions.
    \item \textbf{NM-1 (Engineering):} Virtual rotor, metrology, null tests.
    \item \textbf{NM-2 (This paper):} Generator mode, safety governor, governance.
\end{itemize}

% ============================================================================
% 2. ENERGY ACCOUNTING
% ============================================================================
\section{Energy Accounting}
\label{sec:accounting}

\subsection{Power Flow Definitions}

\begin{definition}[Power Flows]
\label{def:power}
At any time $t$, the system has four power channels:
\begin{align}
    \Pdrive(t) &: \text{Electrical input to drivers} \label{eq:pdrive}\\
    \Pstored(t) &: \text{Rate of change of stored energy} \label{eq:pstored}\\
    \Pthermal(t) &: \text{Heat flow to/from environment} \label{eq:pthermal}\\
    \Pout(t) &: \text{Harvested electrical output} \label{eq:pout}
\end{align}
\end{definition}

\noindent Energy conservation (first law) requires:
\begin{equation}
    \Pdrive(t) = \Pout(t) + \Pthermal(t) + \frac{d}{dt}E_{\text{stored}}(t).
    \label{eq:firstlaw}
\end{equation}

\subsection{Generator Mode Definition}

\begin{definition}[Generator Mode]
\label{def:generator}
The system operates in \emph{generator mode} if, over a measurement interval $[t_1, t_2]$:
\begin{equation}
    \int_{t_1}^{t_2} \Pout(t)\,dt > \int_{t_1}^{t_2} \Pdrive(t)\,dt + \delta E_{\text{stored}}
    \label{eq:generatormode}
\end{equation}
where $\delta E_{\text{stored}} = E_{\text{stored}}(t_1) - E_{\text{stored}}(t_2)$ accounts for any depletion of stored energy.
\end{definition}

\begin{remark}
This definition is \emph{accounting-based}, not effect-based. It can be evaluated from measurements without assuming any particular physical mechanism.
\end{remark}

\subsection{Self-Sustaining Threshold}

\begin{definition}[Self-Sustaining Operation]
\label{def:selfsustain}
The system is \emph{self-sustaining} at time $t$ if:
\begin{equation}
    \Pout(t) > \Pdrive(t).
    \label{eq:selfsustain}
\end{equation}
\end{definition}

\noindent Self-sustaining operation implies that the output power exceeds the drive power required to maintain resonance. This is a \textbf{data-gated hypothesis}---it is defined precisely so that it can be tested and falsified.

\subsection{Entropic Cooling Hypothesis}

\begin{hypothesis}[Entropic Cooling]
\label{hyp:cooling}
If a resonant system orders the vacuum (reduces vacuum entropy), it absorbs thermal energy from the environment:
\begin{equation}
    \Delta S_{\text{vacuum}} < 0 \implies \Delta Q_{\text{env}} < 0.
    \label{eq:entropiccooling}
\end{equation}
\end{hypothesis}

\noindent This is an empirically testable prediction: during high-resonance operation, the device temperature should \emph{decrease} relative to a matched-heating control.

\begin{definition}[Entropic Cooling Predicate]
\label{def:entropiccooling}
The entropic cooling condition is formalized as:
\begin{equation}
    \text{EntropicCooling}(\Delta S_{\text{vac}}, \Delta Q_{\text{env}}) \coloneqq \left( \Delta S_{\text{vac}} < 0 \right) \to \left( \Delta Q_{\text{env}} < 0 \right).
\end{equation}
\end{definition}

\subsection{Vacuum Power Scaling}

\begin{hypothesis}[Vacuum Power Scaling]
\label{hyp:vacpower}
The power output in generator mode scales with frequency $f$ and metric gradient $\sigma_{\nabla}$:
\begin{equation}
    P_{\text{vac}} \propto f \cdot \sigma_{\nabla}.
    \label{eq:vacpower}
\end{equation}
\end{hypothesis}

\noindent This provides a testable prediction: output power should increase linearly with drive frequency at fixed resonance conditions.

\subsection{Falsifiers and Disqualifiers}

The generator-mode hypothesis is \textbf{falsified} if:
\begin{enumerate}
    \item Apparent over-unity disappears under improved thermal/EMI controls.
    \item The effect correlates only with sensor pickup channels (measurement artifact).
    \item Matched-heating runs produce the same ``output'' signal.
\end{enumerate}

% ============================================================================
% 3. GENERATOR ARCHITECTURE
% ============================================================================
\section{Generator Architecture}
\label{sec:architecture}

\subsection{System Block Diagram}

The generator-mode system comprises:

\begin{enumerate}
    \item \textbf{Core:} Virtual rotor (NM-1) generating rotating field.
    \item \textbf{Pickup:} Secondary coil array (stator) around the core.
    \item \textbf{Rectification:} AC-to-DC conversion of induced EMF.
    \item \textbf{Buffer:} Energy storage (capacitor bank or battery).
    \item \textbf{Load interface:} Grid-tie or resistive load.
    \item \textbf{Controller:} Schedule generation, resonance lock, logging.
    \item \textbf{Governor:} Safety system with detune/shutdown authority.
\end{enumerate}

\subsection{Startup Sequence}

\begin{enumerate}
    \item \textbf{Initialize:} External power brings system to idle.
    \item \textbf{Resonance search:} Sweep through candidate frequencies (NM-0 resonance map).
    \item \textbf{Lock detection:} Identify resonance peak via coherence metric.
    \item \textbf{Load stabilization:} Connect load to provide braking.
    \item \textbf{Self-power transition:} (Data-gated) If $\Pout > \Pdrive$, reduce external power.
\end{enumerate}

\subsection{Load as Stabilizer}

\begin{theorem}[Load Stabilization]
\label{thm:loadstable}
When the output load draws power, it acts as a stabilizing brake on the resonant system. Load disconnect removes this braking instantaneously.
\end{theorem}

\begin{proof}[Argument]
By Lenz's law, current drawn from the pickup coils creates a magnetic field opposing the rotating field, providing a restoring (damping) force proportional to load current. When load $= 0$, this damping vanishes.
\end{proof}

\noindent This is the fundamental hazard: loss of load $\Rightarrow$ loss of braking $\Rightarrow$ runaway.

% ============================================================================
% 4. RUNAWAY HAZARD ANALYSIS
% ============================================================================
\section{Runaway Hazard Analysis}
\label{sec:hazard}

\subsection{Runaway Signature}

\begin{definition}[Runaway Signature]
\label{def:runaway}
A system exhibits the \emph{runaway signature} if, when load approaches zero, the rotational proxy (RPM or field frequency) increases without bound:
\begin{equation}
    \text{RunawaySignature}(\text{load}, \omega) \coloneqq \left( \text{load} = 0 \right) \to \left( \frac{d\omega}{dt} > 0 \right).
    \label{eq:runaway}
\end{equation}
\end{definition}

\noindent This signature distinguishes the hypothesized resonant generator from conventional generators, which slow down when unloaded.

\subsection{Runaway Scenarios}

\begin{table}[h]
\centering
\caption{Runaway Hazard Enumeration}
\label{tab:hazards}
\begin{tabular}{@{}lll@{}}
\toprule
Scenario & Cause & Observable \\
\midrule
Loss of load & Open circuit, load fault & $\omega \uparrow$, $I_{\text{load}} \to 0$ \\
Controller fault & Stuck-on pulses & $\omega \uparrow$, pulse timing frozen \\
Clock instability & Jitter blowup & $\omega$ erratic, coherence $\downarrow$ \\
Thermal runaway & Driver overheating & $T \uparrow$, current limit hit \\
Mechanical failure & Structural resonance & Vibration spike, strain alarm \\
\bottomrule
\end{tabular}
\end{table}

\subsection{Runaway Detection Observables}

The following observables indicate runaway onset:
\begin{enumerate}
    \item Rapid rise of $\omega$ (RPM or field frequency).
    \item Rising harmonic content / instability in resonance score.
    \item Overcurrent or overvoltage events.
    \item Strain/vibration spikes.
\end{enumerate}

\subsection{Safety Requirements}

\begin{enumerate}
    \item \textbf{Detection time:} Runaway must be detected within $t_{\det} < 100$ ms.
    \item \textbf{Mitigation time:} Resonance must be collapsed within $t_{\text{mit}} < 1$ s.
    \item \textbf{Fail-safe:} Controller crash must result in detune (not drive continuation).
\end{enumerate}

% ============================================================================
% 5. ACTIVE DETUNING GOVERNOR
% ============================================================================
\section{Active Detuning Governor}
\label{sec:governor}

\subsection{Governor Concept}

Rather than relying on mechanical friction braking (which may be insufficient for high-gain systems), we use \textbf{active detuning}: intentionally inject phase slip to collapse the resonance condition.

\begin{definition}[Phase Slip]
\label{def:phaseslip}
A \emph{phase slip} of magnitude $\delta\phi$ is an intentional timing error in the drive schedule, shifting pulses away from the optimal 8-tick alignment.
\end{definition}

\subsection{Governor Function}

\begin{definition}[Governor Function]
\label{def:governor}
The governor function outputs a phase slip command based on the rotational proxy $\omega$ and a limit $\omega_{\max}$:
\begin{equation}
    \text{Governor}(\omega, \omega_{\max}) \coloneqq 
    \begin{cases}
        \delta\phi_{\text{detune}} & \text{if } \omega > \omega_{\max} \\
        0 & \text{otherwise}
    \end{cases}
    \label{eq:governor}
\end{equation}
where $\delta\phi_{\text{detune}} > 0$ is a fixed detune magnitude (e.g., 0.1 radians or 10\% phase error).
\end{definition}

\subsection{Efficiency Collapse}

\begin{definition}[Efficiency Function]
\label{def:efficiency}
The system efficiency under phase error $\delta\phi$ is modeled as:
\begin{equation}
    \eta(\delta\phi) \coloneqq e^{-(\delta\phi)^2}.
    \label{eq:efficiency}
\end{equation}
\end{definition}

\noindent This Gaussian roll-off captures the intuition that small phase errors cause quadratic efficiency loss, with rapid collapse for larger errors.

\begin{theorem}[Detuning Stops Runaway]
\label{thm:detune}
If $\omega > \omega_{\max}$, then applying the governor function collapses efficiency below unity:
\begin{equation}
    \omega > \omega_{\max} \implies \eta\bigl(\text{Governor}(\omega, \omega_{\max})\bigr) < 1.
\end{equation}
\end{theorem}

\begin{proof}
When $\omega > \omega_{\max}$, the governor outputs $\delta\phi = \delta\phi_{\text{detune}} > 0$. Then:
\begin{equation}
    \eta(\delta\phi_{\text{detune}}) = e^{-(\delta\phi_{\text{detune}})^2} < e^0 = 1.
\end{equation}
For $\delta\phi_{\text{detune}} = 0.1$:
\begin{equation}
    \eta(0.1) = e^{-0.01} \approx 0.99.
\end{equation}
Larger detune values (e.g., $\delta\phi = 0.5$) yield $\eta \approx 0.78$.
\end{proof}

\begin{corollary}[Efficiency Floor]
\label{cor:floor}
To guarantee efficiency below any threshold $\eta_{\text{target}}$, choose:
\begin{equation}
    \delta\phi_{\text{detune}} \geq \sqrt{-\ln(\eta_{\text{target}})}.
\end{equation}
For $\eta_{\text{target}} = 0.5$: $\delta\phi_{\text{detune}} \geq 0.83$ radians.
\end{corollary}

\subsection{Governor State Machine}

The governor operates as a state machine:

\begin{enumerate}
    \item \textbf{NOMINAL:} $\omega < \omega_{\max}$; output $\delta\phi = 0$.
    \item \textbf{DETUNE:} $\omega_{\max} \leq \omega < \omega_{\text{crit}}$; output $\delta\phi = \delta\phi_{\text{detune}}$.
    \item \textbf{SHUTDOWN:} $\omega \geq \omega_{\text{crit}}$ or fault detected; output SHUTDOWN command.
\end{enumerate}

\noindent Transitions: NOMINAL $\to$ DETUNE $\to$ SHUTDOWN (escalation ladder). Recovery: DETUNE $\to$ NOMINAL (if $\omega$ drops below $\omega_{\max} - \epsilon$).

\subsection{Hardware Interlocks}

In addition to the software governor, the following hardware interlocks are required:

\begin{enumerate}
    \item \textbf{Independent watchdog:} Separate microcontroller monitors $\omega$ and triggers detune if main controller fails.
    \item \textbf{Physical E-stop:} Manual emergency stop accessible to operator.
    \item \textbf{Dump load / crowbar:} Resistive load that can be switched in to absorb excess energy.
    \item \textbf{Default-safe behavior:} On loss of control signal, drivers default to detune pattern.
\end{enumerate}

% ============================================================================
% 6. OPERATIONAL SAFETY PROTOCOLS
% ============================================================================
\section{Operational Safety Protocols}
\label{sec:protocols}

\subsection{Test Gating and Escalation}

Experiments proceed through power levels with mandatory gates:

\begin{table}[h]
\centering
\caption{Power Escalation Ladder}
\label{tab:escalation}
\begin{tabular}{@{}llll@{}}
\toprule
Level & Power & Gate Requirement & Replications \\
\midrule
L1 & $< 10$ W & Basic safety check & 3 \\
L2 & $10$--$100$ W & L1 complete, governor verified & 5 \\
L3 & $100$--$1000$ W & L2 complete, remote operation & 10 \\
L4 & $> 1$ kW & L3 complete, facility review & 20 \\
\bottomrule
\end{tabular}
\end{table}

\subsection{Facility and Personnel Safety}

\begin{enumerate}
    \item \textbf{Shielding:} Faraday enclosure; RF-absorbing materials if needed.
    \item \textbf{Standoff distances:} No personnel within 2 m during L3+ operation.
    \item \textbf{Remote operation:} All L3+ tests controlled from separate room.
    \item \textbf{Electrical safety:} GFCI protection; lockout/tagout procedures.
\end{enumerate}

\subsection{Incident Response}

\begin{enumerate}
    \item \textbf{Automatic logging:} All sensor data preserved with tamper-evident checksums.
    \item \textbf{Post-incident lockout:} System remains disabled until root-cause analysis complete.
    \item \textbf{Root-cause workflow:} Standardized incident report template; independent review.
\end{enumerate}

% ============================================================================
% 7. GOVERNANCE AND DISCLOSURE
% ============================================================================
\section{Governance and Responsible Disclosure}
\label{sec:governance}

\subsection{Disclosure Categories}

\begin{table}[h]
\centering
\caption{Disclosure Classification}
\label{tab:disclosure}
\begin{tabular}{@{}ll@{}}
\toprule
Category & Content \\
\midrule
Public & Geometry definitions, scheduling discipline, metrology protocols \\
Patent & Broad claims, high-level diagrams, operating principles \\
Trade Secret & Exact operating tables, PCB layouts, switching details \\
\bottomrule
\end{tabular}
\end{table}

\subsection{Replication Strategy}

Responsible replication proceeds as follows:
\begin{enumerate}
    \item \textbf{Internal replication:} $\geq 3$ independent builds within organization.
    \item \textbf{NDA replication:} External parties under NDA with pre-registered protocols.
    \item \textbf{Public release:} Only after reproducibility meets internal criteria.
\end{enumerate}

\subsection{Export Control Considerations}

\begin{enumerate}
    \item \textbf{Domestic-first filing:} All patents filed domestically before international.
    \item \textbf{Counsel review:} Legal review before any international dissemination.
    \item \textbf{ITAR/EAR awareness:} Monitor for potential dual-use classification.
\end{enumerate}

\subsection{Ethical Framework}

\begin{enumerate}
    \item \textbf{Safety by design:} Governor integration is mandatory, not optional.
    \item \textbf{Misuse risk:} Explicit non-goal for weaponization applications.
    \item \textbf{Transparency:} Null results published with same rigor as positive results.
\end{enumerate}

% ============================================================================
% 8. CONCLUSION
% ============================================================================
\section{Conclusion}
\label{sec:conclusion}

This paper has established the safety and governance framework for high-gain resonant field systems operating in generator mode. The key contributions are:

\begin{enumerate}
    \item \textbf{Energy accounting:} Rigorous definitions of power flows, generator-mode threshold, and self-sustaining operation.
    \item \textbf{Falsifiable hypotheses:} Entropic cooling and vacuum power scaling, with explicit falsifiers.
    \item \textbf{Hazard analysis:} Enumeration of runaway scenarios with detection observables and safety requirements.
    \item \textbf{Active detuning governor:} Phase-slip-based efficiency collapse with proof that detuning prevents runaway.
    \item \textbf{Operational protocols:} Power escalation ladder, facility safety, incident response.
    \item \textbf{Governance model:} Disclosure categories, replication strategy, export control awareness.
\end{enumerate}

The framework ensures that any claimed effects are rigorously verified before disclosure, and that safety measures are in place to prevent catastrophic failure during testing.

% ============================================================================
% APPENDICES
% ============================================================================
\appendix

\section{Energy Accounting Worksheet}
\label{app:worksheet}

\begin{verbatim}
Run ID: ______________
Date: ______________

INPUTS:
  P_drive (avg, W): ______________
  Duration (s): ______________
  E_drive = P_drive x duration: ______________

STORED ENERGY:
  E_stored_start (J): ______________
  E_stored_end (J): ______________
  Delta_E_stored: ______________

OUTPUTS:
  P_out (avg, W): ______________
  E_out = P_out x duration: ______________

THERMAL:
  Delta_T_device (C): ______________
  Delta_T_environment (C): ______________
  Thermal anomaly (Y/N): ______________

BALANCE:
  E_drive + Delta_E_stored = __________ (input)
  E_out + E_thermal = __________ (output)
  Discrepancy: ______________

GENERATOR MODE (Y/N): ______________
\end{verbatim}

\section{Governor State Machine}
\label{app:statemachine}

\textbf{States:}
\begin{itemize}
    \item NOMINAL: Normal operation, $\delta\phi = 0$
    \item DETUNE: Active braking, $\delta\phi = \delta\phi_{\text{detune}}$
    \item SHUTDOWN: All drives disabled
\end{itemize}

\textbf{Transitions:}
\begin{enumerate}
    \item NOMINAL $\to$ DETUNE: $\omega > \omega_{\max}$
    \item DETUNE $\to$ NOMINAL: $\omega < \omega_{\max} - \epsilon$ for $t > t_{\text{hold}}$
    \item DETUNE $\to$ SHUTDOWN: $\omega > \omega_{\text{crit}}$ OR fault detected
    \item ANY $\to$ SHUTDOWN: E-stop pressed OR watchdog timeout
\end{enumerate}

\textbf{Pseudocode:}
\begin{verbatim}
loop:
  w = read_rpm()
  fault = check_faults()
  
  if state == NOMINAL:
    if w > w_max:
      state = DETUNE
      set_phase_slip(dphi_detune)
      
  elif state == DETUNE:
    if w > w_crit or fault:
      state = SHUTDOWN
      disable_all_drives()
    elif w < w_max - eps for t_hold:
      state = NOMINAL
      set_phase_slip(0)
      
  elif state == SHUTDOWN:
    disable_all_drives()
    log_incident()
    wait_for_manual_reset()
\end{verbatim}

\section{Redaction Checklist}
\label{app:redaction}

Before external sharing, verify removal of:
\begin{enumerate}
    \item[$\square$] Exact operating frequency tables
    \item[$\square$] PCB layout files and Gerber data
    \item[$\square$] Driver schematic details
    \item[$\square$] Manufacturing tolerances
    \item[$\square$] Specific component part numbers
    \item[$\square$] Calibration constants
    \item[$\square$] Internal test results (unless publication-approved)
\end{enumerate}

% ============================================================================
% REFERENCES
% ============================================================================
\begin{thebibliography}{9}

\bibitem{powersafety}
IEEE Std 1584-2018,
\textit{IEEE Guide for Performing Arc-Flash Hazard Calculations}.
IEEE, 2018.

\bibitem{resonance}
J.~P.~Den Hartog,
\textit{Mechanical Vibrations}, 4th ed.
Dover, 1985.

\bibitem{gridtie}
IEEE Std 1547-2018,
\textit{IEEE Standard for Interconnection and Interoperability of Distributed Energy Resources}.
IEEE, 2018.

\bibitem{disclosure}
S.~D.~Scalet,
``Responsible Disclosure of Security Vulnerabilities,''
\textit{Communications of the ACM}, vol.~49, no.~4, pp.~83--88, 2006.

\bibitem{dualuse}
National Academies of Sciences,
\textit{Dual Use Research of Concern in the Life Sciences}.
National Academies Press, 2017.

\end{thebibliography}

\end{document}
