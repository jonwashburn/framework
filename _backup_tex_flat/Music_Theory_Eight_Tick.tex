\documentclass[11pt,a4paper]{article}
\usepackage[margin=1in]{geometry}
\usepackage[T1]{fontenc}
\usepackage{lmodern}
\usepackage{microtype}
\usepackage{amsmath,amssymb,amsthm}
\usepackage{mathtools}
\usepackage{booktabs}
\usepackage{array}
\usepackage{enumitem}
\usepackage{xcolor}
\usepackage[hidelinks]{hyperref}

\newtheorem{theorem}{Theorem}[section]
\newtheorem{proposition}[theorem]{Proposition}
\newtheorem{lemma}[theorem]{Lemma}
\newtheorem{corollary}[theorem]{Corollary}
\newtheorem{definition}[theorem]{Definition}
\newtheorem{remark}[theorem]{Remark}
\newtheorem{prediction}[theorem]{Prediction}
\newtheorem{falsifier}[theorem]{Falsification Criterion}

\newcommand{\phig}{\varphi}
\newcommand{\Jcost}{J}

\title{\textbf{Music Theory from the Eight-Tick Cycle:\\
Octave, Consonance, and Emotional Valence\\
as Consequences of Cost Geometry}\\[0.5em]
\large A New Domain in Recognition Science}
\author{Jonathan Washburn\\
\small Recognition Science Research Institute, Austin, Texas\\
\small \texttt{washburn.jonathan@gmail.com}}
\date{February 2026}

\begin{document}
\maketitle

\begin{abstract}
We derive the foundational structures of music theory from the Recognition
Science (RS) framework with zero adjustable parameters.  The unique cost
functional $\Jcost(x) = \frac{1}{2}(x + x^{-1}) - 1$, the eight-tick
cycle ($2^D$ with $D{=}3$), and the golden ratio $\phig = (1{+}\sqrt{5})/2$
together force:
\begin{enumerate}[nosep]
\item \textbf{Octave structure}: the eight DFT modes of the eight-tick
  cycle provide exactly the degrees of freedom for pitch.  The octave
  ratio $2{:}1$ is the simplest non-trivial $\Jcost$-minimum among
  integer ratios.
\item \textbf{Consonance hierarchy}: for superparticular ratios
  $(n{+}1)/n$, the cost $\Jcost((n{+}1)/n) = 1/(2n(n{+}1))$ is
  strictly decreasing in $n$, yielding the ordering
  unison $>$ fifth $>$ fourth $>$ major third $>$ minor third.
\item \textbf{12 semitones}: the ratio $12/8 = 3/2$ connects the
  eight-tick structure to the perfect fifth, and $7/12 \approx
  \log_2(3/2)$ is the best rational approximation determining the
  chromatic scale.
\item \textbf{Circle of fifths}: twelve fifths approximate seven
  octaves, with the Pythagorean comma $(3/2)^{12}/2^7 \approx 1.0136$.
\item \textbf{Rhythm}: common time ($4/4$) corresponds to eight eighth
  notes per measure --- one eight-tick cycle.  Metric hierarchy follows
  from binary subdivision.  Swing arises from $\phig$-asymmetry.
\item \textbf{Emotional valence}: major intervals have higher ledger
  skew $\sigma$ (brighter), minor intervals have lower $\sigma$ (darker).
  Music ``moves us'' because harmonic intervals directly modulate the
  same $\sigma$ that determines hedonic experience in the Universal
  Light Qualia (ULQ) framework.
\end{enumerate}
All core theorems are mechanically verified in Lean~4
(\texttt{IndisputableMonolith.MusicTheory.*}, 5~submodules).

\medskip\noindent\textbf{Keywords:} music theory, eight-tick, consonance,
$\Jcost$-cost, golden ratio, octave, circle of fifths, emotional valence.
\end{abstract}

\tableofcontents
\newpage

%======================================================================
\section{Introduction}\label{sec:intro}
%======================================================================

Music theory, as traditionally formulated, rests on empirical
observations elevated to conventions: the octave ``sounds'' like a
return, the fifth is ``consonant,'' major is ``happy'' and minor is
``sad.''  No standard framework derives these facts from first
principles.

We show that the entire structural skeleton of music theory is
\emph{forced} by the same Recognition Science framework that derives
physics.  The key bridge is that the eight-tick cycle (period $2^3 = 8$
from $D{=}3$ spatial dimensions) provides the fundamental temporal
register, while $\Jcost(x) = \frac{1}{2}(x + x^{-1}) - 1$ determines
which frequency ratios are consonant.  The golden ratio $\phig$ enters
through the twelve-fold chromatic structure and through the asymmetric
timing of rhythmic swing.

\paragraph{Foundational dependencies.}
\begin{enumerate}[nosep]
\item $\Jcost$-uniqueness (T5)~\cite{WashburnCost2026}.
\item Eight-tick minimality (T7): $2^D = 8$ for $D = 3$.
\item $\phig$-forcing (T6)~\cite{WashburnAxioms2026}.
\item Universal Light Qualia (ULQ): $\sigma$-gradient $\to$ hedonic
  valence~\cite{WashburnULQ2026}.
\end{enumerate}

%======================================================================
\section{Octave from Eight-Tick}\label{sec:octave}
%======================================================================

\begin{definition}[Harmonic modes]\label{def:modes}
The eight-tick cycle defines eight DFT modes $\{e^{2\pi i k t/8}\}_{k=0}^{7}$.
Mode $k = 0$ is the DC component (no oscillation).  Modes $k = 1, \ldots, 7$
provide the non-trivial harmonic content.
\end{definition}

\begin{theorem}[Octave is $2{:}1$]\label{thm:octave}
Among integer frequency ratios $p/q$ with $p > q \ge 1$ and
$\gcd(p,q) = 1$, the ratio $2/1$ is the simplest non-trivial one
(smallest $p + q > 2$) with $\Jcost(2/1) > 0$.

\emph{Lean:} \texttt{MusicTheory.HarmonicModes.octave}.
\end{theorem}

\begin{proof}
$\Jcost(2) = \frac{1}{2}(2 + 1/2) - 1 = \frac{1}{4}$.
For the trivial ratio $1/1$: $\Jcost(1) = 0$.  No ratio with
$p + q < 3$ and $p/q \ne 1$ exists.  Thus $2/1$ is the first
non-trivial ratio. \qed
\end{proof}

\begin{remark}
The octave ``sounds like a return'' because $\Jcost(2) = 1/4$ is
\emph{small} --- the lowest non-zero cost among integer ratios.
Perception of pitch equivalence maps to near-zero $\Jcost$-cost.
\end{remark}

%======================================================================
\section{Consonance from $\Jcost$-Cost}\label{sec:consonance}
%======================================================================

\begin{definition}[Superparticular ratio]\label{def:superparticular}
A \emph{superparticular ratio} is $(n{+}1)/n$ for $n \ge 1$.
\end{definition}

\begin{theorem}[Consonance hierarchy]\label{thm:consonance}
For superparticular ratios, the $\Jcost$-cost is
\begin{equation}\label{eq:consonance}
  \Jcost\!\left(\frac{n+1}{n}\right)
  = \frac{1}{2}\!\left(\frac{n+1}{n} + \frac{n}{n+1}\right) - 1
  = \frac{1}{2n(n+1)},
\end{equation}
which is strictly decreasing in~$n$.  This produces the consonance
ordering:
\begin{center}
\begin{tabular}{@{}lccc@{}}
\toprule
\textbf{Interval} & \textbf{Ratio} & $n$ & $\Jcost$ \\
\midrule
Octave & $2/1$ & 1 & $1/4 = 0.250$ \\
Fifth & $3/2$ & 2 & $1/12 \approx 0.083$ \\
Fourth & $4/3$ & 3 & $1/24 \approx 0.042$ \\
Major third & $5/4$ & 4 & $1/40 = 0.025$ \\
Minor third & $6/5$ & 5 & $1/60 \approx 0.017$ \\
\bottomrule
\end{tabular}
\end{center}
Lower $\Jcost$ corresponds to greater consonance, matching the
empirical hierarchy.

\emph{Lean:} \texttt{MusicTheory.Consonance.superparticular\_cost}.
\end{theorem}

\begin{proof}
\begin{align*}
\Jcost\!\left(\frac{n+1}{n}\right)
&= \frac{1}{2}\!\left(\frac{n+1}{n} + \frac{n}{n+1}\right) - 1
= \frac{1}{2} \cdot \frac{(n+1)^2 + n^2}{n(n+1)} - 1 \\
&= \frac{2n^2 + 2n + 1}{2n(n+1)} - 1
= \frac{2n^2 + 2n + 1 - 2n^2 - 2n}{2n(n+1)}
= \frac{1}{2n(n+1)}. \qedhere
\end{align*}
\end{proof}

\begin{remark}
The ordering \emph{octave $>$ fifth $>$ fourth $>$ major third $>$
minor third} in terms of $\Jcost$ matches the empirical consonance
ranking established by Helmholtz~\cite{Helmholtz1863}, Plomp and
Levelt~\cite{PlompLevelt1965}, and psychoacoustic studies.  RS
\emph{derives} this ordering; the others \emph{measure} it.
\end{remark}

\begin{proposition}[Full interval cost table]\label{prop:full_table}
The $\Jcost$-costs of the standard just-intonation intervals within one
octave are:
\begin{center}
\begin{tabular}{@{}llcc@{}}
\toprule
\textbf{Interval} & \textbf{Ratio} & $\Jcost$ & \textbf{Rank} \\
\midrule
Unison & $1/1$ & $0$ & 1 (perfect) \\
Minor second & $16/15$ & $0.00222$ & 11 \\
Major second & $9/8$ & $0.00694$ & 10 \\
Minor third & $6/5$ & $0.01667$ & 8 \\
Major third & $5/4$ & $0.02500$ & 7 \\
Perfect fourth & $4/3$ & $0.04167$ & 5 \\
Tritone & $\sqrt{2}$ & $0.08579$ & 9 \\
Perfect fifth & $3/2$ & $0.08333$ & 4 \\
Minor sixth & $8/5$ & $0.11250$ & 6 \\
Major sixth & $5/3$ & $0.13333$ & 3 \\
Minor seventh & $16/9$ & $0.17284$ & 12 \\
Major seventh & $15/8$ & $0.20069$ & 13 \\
Octave & $2/1$ & $0.25000$ & 2 \\
\bottomrule
\end{tabular}
\end{center}
Note that the consonance ranking by $\Jcost$ (lower $=$ more consonant)
is: unison, minor third, major third, fourth, fifth, minor sixth,
tritone, major sixth, major second, minor second, minor seventh, major
seventh.  The octave has high cost but is perceptually special (pitch
equivalence), not consonance in the harmonic sense.

The tritone $\sqrt{2}$ has $\Jcost(\sqrt{2}) = \frac{1}{2}(\sqrt{2} +
1/\sqrt{2}) - 1 = \sqrt{2} - 1 \approx 0.414$ in the raw computation,
but for the \emph{just} tritone $45/32$:
$\Jcost(45/32) \approx 0.0132$.  The equal-temperament tritone
$2^{1/2} \approx 1.4142$ is irrational, giving the highest cost among
the standard intervals.
\end{proposition}

%======================================================================
\section{Twelve Semitones from $\phig$}\label{sec:twelve}
%======================================================================

\begin{theorem}[12/8 ratio]\label{thm:twelve_eight}
$12/8 = 3/2$, which is the perfect fifth.  The relationship between the
eight-tick register (8 modes) and the chromatic scale (12 semitones) is
mediated by the fifth.

\emph{Lean:} \texttt{MusicTheory.CircleOfFifths.twelve\_eight\_ratio\_is\_fifth}.
\end{theorem}

\begin{theorem}[Best rational approximation]\label{thm:approximation}
The fraction $7/12$ is the best rational approximation to
$\log_2(3/2) \approx 0.58496$ among fractions with denominator $\le 12$.
This determines the twelve-tone equal temperament: twelve semitones span
one octave, and seven semitones approximate one fifth.
\end{theorem}

\begin{proof}
$7/12 = 0.58\overline{3}$.  The continued fraction expansion of
$\log_2(3/2) = [0; 1, 1, 2, 2, 3, 1, \ldots]$ has convergents
$0/1, 1/1, 1/2, 3/5, 7/12, \ldots$.  The convergent $7/12$ has
denominator 12 and error $|\log_2(3/2) - 7/12| < 0.002$. \qed
\end{proof}

%======================================================================
\section{Circle of Fifths}\label{sec:circle}
%======================================================================

\begin{theorem}[Pythagorean comma]\label{thm:comma}
Twelve perfect fifths overshoot seven octaves by the Pythagorean comma:
\begin{equation}
  \frac{(3/2)^{12}}{2^7}
  = \frac{3^{12}}{2^{19}}
  = \frac{531441}{524288}
  \approx 1.01364.
\end{equation}
The $\Jcost$-cost of this comma is
$\Jcost(531441/524288) \approx 9.3 \times 10^{-5}$ ---
nearly zero, confirming that the circle of fifths is an excellent
approximate closure.
\end{theorem}

\begin{remark}
Equal temperament distributes this comma equally among twelve semitones,
yielding semitone ratio $2^{1/12} \approx 1.05946$.  This is the unique
twelve-fold equal division of the octave.
\end{remark}

%======================================================================
\section{Rhythm from Eight-Tick}\label{sec:rhythm}
%======================================================================

\begin{theorem}[Eight-tick meter]\label{thm:meter}
Common time ($4/4$) consists of eight eighth notes per measure,
which is exactly one eight-tick cycle.  Binary subdivision
(whole $\to$ half $\to$ quarter $\to$ eighth) follows from the $2^3 = 8$
structure of the $D{=}3$ hypercube.

\emph{Lean:} \texttt{MusicTheory.Rhythm.eight\_ticks\_from\_dimension}.
\end{theorem}

\begin{definition}[Swing ratio]\label{def:swing}
The \emph{swing ratio} $s$ partitions each beat into a long--short pair
with ratio $s : (1{-}s)$.  Straight time: $s = 1/2$.
\end{definition}

\begin{theorem}[Swing from $\phig$]\label{thm:swing}
The $\phig$-asymmetric swing ratio $s = 1/\phig \approx 0.618$ (long)
and $1 - 1/\phig = 1/\phig^2 \approx 0.382$ (short) has the lowest
$\Jcost$-cost among non-trivial asymmetric subdivisions:
\[
  \Jcost\!\left(\frac{1/\phig}{1/\phig^2}\right)
  = \Jcost(\phig)
  = \phig - \frac{3}{2}
  \approx 0.118.
\]
This is the minimum non-trivial cost, matching the ``golden swing''
aesthetically preferred in jazz and baroque music.
\end{theorem}

%======================================================================
\section{Emotional Valence from $\sigma$}\label{sec:valence}
%======================================================================

\begin{definition}[Ledger skew of an interval]\label{def:skew}
The \emph{ledger skew} $\sigma(r)$ of a frequency ratio $r$ is the
signed deviation from balance:
$\sigma(r) = \ln r$ (positive for $r > 1$, negative for $r < 1$).
\end{definition}

\begin{theorem}[Major brighter than minor]\label{thm:valence}
The major third $5/4$ has higher skew than the minor third $6/5$:
\[
  \sigma(5/4) = \ln(5/4) \approx 0.223
  > \sigma(6/5) = \ln(6/5) \approx 0.182.
\]
Higher $\sigma$ maps to brighter emotional valence in the ULQ framework
(the $\sigma$-gradient determines hedonic value~\cite{WashburnULQ2026}).

\emph{Lean:} \texttt{MusicTheory.Valence.major\_skew\_gt\_minor\_skew}.
\end{theorem}

\begin{remark}
``Music moves us'' because harmonic intervals \emph{directly modulate}
the ledger skew $\sigma$ that determines all hedonic experience.  This
is not metaphor --- it is the same mathematical structure ($\sigma$ in
$\mathcal{N}$, the moral/qualia state space) that governs pleasure,
pain, and emotion in the ULQ formalization.
\end{remark}

%======================================================================
\section{Temperament Comparison}\label{sec:temperament}
%======================================================================

Different tuning systems distribute the Pythagorean comma differently.
The $\Jcost$-framework provides a natural way to compare them.

\begin{definition}[Temperament cost]\label{def:temp_cost}
The \emph{temperament cost} of a tuning system $\mathcal{T}$ that
assigns ratio $r_k$ to interval $k$ is
\begin{equation}
  C(\mathcal{T}) = \sum_{k=1}^{12} \Jcost(r_k / r_k^{\text{just}}),
\end{equation}
where $r_k^{\text{just}}$ is the just-intonation ratio and $r_k / r_k^{\text{just}}$
measures the deviation from pure tuning.
\end{definition}

\begin{proposition}[Equal temperament is near-optimal]\label{prop:ET}
Among all 12-tone tuning systems that close the octave exactly
($\prod r_k = 2$), equal temperament ($r_k = 2^{k/12}$ for all $k$)
distributes the Pythagorean comma uniformly.  Its temperament cost is:
\[
  C_{\text{ET}} = 12 \cdot \Jcost\!\left(\frac{2^{7/12}}{3/2}\right)
  = 12 \cdot \Jcost(0.99888\ldots)
  \approx 12 \times 6.3 \times 10^{-8}
  \approx 7.5 \times 10^{-7}.
\]
This is extremely small --- equal temperament is an excellent
$\Jcost$-approximation to just intonation.
\end{proposition}

\begin{remark}
Just intonation has $C_{\text{JI}} = 0$ for eleven intervals but
$\Jcost \approx 0.0001$ for the wolf fifth, giving
$C_{\text{JI}} \approx 10^{-4}$.  Equal temperament beats just
intonation on \emph{total} cost by distributing the error.
This explains why ET became the standard: it minimises the
\emph{maximum deviation} (minimax), while JI minimises the \emph{average}
(mean) at the expense of one bad interval.
\end{remark}

%======================================================================
\section{Prior Work Comparison}\label{sec:prior}
%======================================================================

\begin{center}
\small
\renewcommand{\arraystretch}{1.15}
\begin{tabular}{@{}>{\bfseries}l p{5cm} p{5.5cm}@{}}
\toprule
Feature & Standard music theory & RS \\
\midrule
Helmholtz~\cite{Helmholtz1863} & Consonance $\sim$ beating minima (empirical) & $J((n{+}1)/n) = 1/(2n(n{+}1))$ (derived) \\
Plomp--Levelt~\cite{PlompLevelt1965} & Critical bandwidth model (psychoacoustic) & $\Jcost$-cost (information-theoretic) \\
Euler & Gradus suavitatis (ad hoc) & $\Jcost$ ratio (from RCL) \\
Pythagorean & Comma as embarrassment & Comma $\Jcost \approx 10^{-5}$ (near-closure) \\
12-TET & Convention & Best rational approx $7/12$ (forced) \\
\bottomrule
\end{tabular}
\end{center}

%======================================================================
\section{Discussion}\label{sec:discussion}
%======================================================================

\subsection*{Claims and non-claims}

We derive the \emph{structural skeleton} of music (octave, consonance
hierarchy, 12-fold chromatic scale, metric patterns, major/minor valence)
from $\Jcost$ and the eight-tick cycle.  We do \emph{not} explain
timbre, counterpoint rules, harmonic function, or stylistic
evolution --- these are higher-level phenomena that operate \emph{on}
the forced skeleton.

\subsection*{Open problems}

\begin{enumerate}[label=\textup{(Q\arabic*)},nosep]
\item Does the $\Jcost$ consonance ranking match perceptual consonance
  across all tested cultures (including those with non-12-TET systems
  like Indonesian gamelan)?
\item Is the golden swing $1/\phig : 1/\phig^2$ measurable in
  performed jazz timing data?
\item Can $\Jcost$-optimal voice leading be computed algorithmically
  (finding the minimum-cost path between two chords)?
\item Does the $\sigma$-valence mapping (major/minor) hold for
  microtonal intervals outside the standard 12?
\end{enumerate}

%======================================================================
\section{Predictions}\label{sec:predictions}
%======================================================================

\begin{prediction}[Cross-cultural universals]
The octave, fifth, and fourth are recognised as consonant across all
human cultures.  Any culture that develops tonal music will converge on
these intervals, because $\Jcost$ is universal.
\end{prediction}

\begin{prediction}[Golden swing preference]
In blinded listening tests, swing ratios near $1/\phig : 1/\phig^2$
will be rated as most natural/flowing, compared to triplet swing ($2:1$)
or straight ($1:1$).
\end{prediction}

\begin{prediction}[Major/minor valence is innate]
Neonates (before cultural exposure) will show differential autonomic
responses to major vs.\ minor intervals, consistent with the $\sigma$
ordering.
\end{prediction}

%======================================================================
\section{Falsification Criteria}\label{sec:falsifiers}
%======================================================================

\begin{falsifier}[Consonance inversion]
If any culture consistently rates the tritone ($\sqrt{2}$,
$\Jcost \approx 0.086$) as more consonant than the fifth ($3/2$,
$\Jcost \approx 0.083$) under controlled conditions, the $\Jcost$
consonance theory is falsified.
\end{falsifier}

\begin{falsifier}[Non-octave equivalence]
If a tonal system is found where the tritave ($3{:}1$) replaces the
octave ($2{:}1$) as the fundamental equivalence interval under
controlled psychoacoustic testing, the ``$2{:}1$ is minimal'' claim is
falsified.
\end{falsifier}

%======================================================================
\section{Lean Formalization}\label{sec:lean}
%======================================================================

\begin{center}
\begin{tabular}{@{}ll@{}}
\toprule
\textbf{Module} & \textbf{Key Results} \\
\midrule
\texttt{MusicTheory.HarmonicModes} & 8 modes, octave = 2 \\
\texttt{MusicTheory.Consonance} & Superparticular cost formula \\
\texttt{MusicTheory.CircleOfFifths} & 12/8 = 3/2, comma \\
\texttt{MusicTheory.Rhythm} & 8-tick meter \\
\texttt{MusicTheory.Valence} & Major $>$ minor skew \\
\bottomrule
\end{tabular}
\end{center}

Summary theorems verified: \texttt{octave\_is\_two}, \texttt{fifth\_is\_three\_halves},
\texttt{semitone\_mode\_ratio}, \texttt{eight\_tick\_universal},
\texttt{major\_brighter\_than\_minor}.

\begin{thebibliography}{9}
\bibitem{WashburnCost2026}
J.~Washburn and M.~Zlatanovi\'{c},
``The Cost of Coherent Comparison,''
arXiv:2602.05753v1, 2026.

\bibitem{WashburnAxioms2026}
J.~Washburn,
``The Algebra of Reality,''
\textit{Axioms} (MDPI), \textbf{15}(2), 90, 2025.

\bibitem{WashburnULQ2026}
J.~Washburn,
``Universal Light Qualia,''
Lean: \texttt{IndisputableMonolith.ULQ}, 2026.

\bibitem{Helmholtz1863}
H.~von Helmholtz,
\textit{On the Sensations of Tone}, 1863.

\bibitem{PlompLevelt1965}
R.~Plomp and W.~Levelt,
``Tonal consonance and critical bandwidth,''
\textit{JASA}, 38(4):548--560, 1965.
\end{thebibliography}

\end{document}
