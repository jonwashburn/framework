\documentclass[11pt]{article}

\usepackage[margin=1in]{geometry}
\usepackage{amsmath,amssymb}
\usepackage[colorlinks=true,linkcolor=blue,citecolor=blue,urlcolor=blue]{hyperref}

\newcommand{\R}{\mathbb{R}}

\setlength{\parindent}{0pt}
\setlength{\parskip}{0.45em}

\begin{document}

\begin{center}
{\Large Closing Lemmas 2.6--2.9 (Dec 11 version): suggested completion}\\
{\small \today}
\end{center}

\textbf{To:} Milan Zlatanovi\'c\\
\textbf{Cc:} Prof.\ Elshad Allahyarov\\
\textbf{Project:} NS Overleaf (\href{https://www.overleaf.com/3989828692dqrcmngchgvs#eba565}{link})

\vspace{0.4em}

Dear Milan,

Thank you for clarifying. You are right: my earlier response was tied to the older (Dec~8) ``single Lemma~2.6'' iteration. I have now reviewed your Dec~11 restructuring into Lemmas~2.6--2.9 (singular point; normalization; rescaled domain; ancient limit) in \texttt{new-version-12-11.tex}. The structure is good and much easier to check.

\textbf{What is still needed to fully close the chain:}

\begin{enumerate}
  \item \textbf{Lemma 2.6 (existence of a CKN singular point):} the logic is correct (if no singular point exists at $T^*$, $\varepsilon$-regularity gives local boundedness near $t=T^*$, hence continuation past $T^*$, contradiction). This is the right ``anchor'' for the blow-up.

  \item \textbf{Lemma 2.7 (blow-up normalization):} as currently written it chooses $x_k$ from the \emph{global} vorticity supremum at times $t_k\uparrow T^*$. This leaves an implicit issue: $x_k$ could in principle drift (the noncompactness worry you flagged in the Dec~8 comments).

  The clean fix is to \emph{anchor the blow-up near a fixed singular point $x^*$ from Lemma~2.6}. Concretely, replace the choice of $x_k$ by either:
  \begin{itemize}
    \item \emph{(Local vorticity choice)} choose $x_k\in B_1(x^*)$ with $|\omega(x_k,t_k)|=\|\omega(\cdot,t_k)\|_{L^\infty(B_1(x^*))}$, and note this local supremum must diverge as $t_k\uparrow T^*$ if $(x^*,T^*)$ is singular; or
    \item \emph{(CKN-normalization choice, recommended)} choose scales $r_k\downarrow0$ such that the CKN functional at $(x^*,T^*)$ satisfies
    \[
      r_k^{-2}\iint_{Q_{r_k}(x^*,T^*)}\bigl(|u|^3+|p|^{3/2}\bigr)\,dx\,dt \ge \varepsilon_{\mathrm{CKN}},
    \]
    and define the blow-up by $\lambda_k:=r_k$ and center $x_k:=x^*$ (so no drift can occur).
  \end{itemize}
  The second option makes the nontriviality in Lemma~2.9 essentially automatic by semicontinuity.

  \item \textbf{Lemma 2.8 (domain exhaustion):} looks correct.

  \item \textbf{Lemma 2.9 (ancient limit):} this is the main place where the proof needs to be written out. There are two distinct sub-steps:
  \begin{enumerate}
    \item \emph{Compactness / passage to the limit.} State explicitly the uniform estimates on each fixed cylinder $Q_R=B_R\times(-R^2,0)$ inherited from the local energy inequality under scaling, e.g.
    \[
      \sup_{s\in(-R^2,0)}\int_{B_R}|u^{(k)}(s)|^2 + \int_{Q_R}|\nabla u^{(k)}|^2 \le C(R),
      \qquad
      \|p^{(k)}\|_{L^{3/2}(Q_R)}\le C(R),
    \]
    plus a standard bound on $\partial_s u^{(k)}$ in a negative Sobolev space (so Aubin--Lions applies). This yields a subsequence converging strongly in $L^p_{\mathrm{loc}}$ (for $p<3$) and is enough to pass the nonlinear term and the local energy inequality, giving a suitable weak limit $(u^\infty,p^\infty)$ on $\R^3\times(-\infty,0)$.

    \item \emph{Nontriviality of the limit (your item (iii)).} This is easiest if the blow-up is anchored by a \emph{scale-invariant lower bound} (CKN functional) rather than a pointwise vorticity normalization. With the CKN-normalization choice in Lemma~2.7, we get on $Q_1$:
    \[
      \iint_{Q_1}\bigl(|u^{(k)}|^3+|p^{(k)}|^{3/2}\bigr)\,dx\,dt \ge \varepsilon_{\mathrm{CKN}}.
    \]
    By strong/weak convergence and lower semicontinuity, the same lower bound holds for the limit, hence $u^\infty\not\equiv 0$, which implies your desired statement $\int_{Q_r}|u^\infty|^3\ge c>0$ for some $r,c$.
  \end{enumerate}
\end{enumerate}

\textbf{One stylistic point:} I recommend removing any claim that the ancient limit is locally $L^\infty$ in space; from local energy bounds one naturally inherits $L^\infty_tL^2_x\cap L^2_tH^1_x$ and the scale-invariant $L^3$/$L^{3/2}$ controls, which are what the later ``geometric depletion'' steps use.

\vspace{0.4em}
Thanks again for doing the hard work of restructuring this. If you agree, I suggest implementing the CKN-normalized blow-up (anchored at $x^*$) as the main route, and keeping the vorticity normalization as an optional remark (it is conceptually nice, but technically less robust for proving nontriviality of the ancient limit).

\vspace{0.5em}
Sincerely,\\
Jonathan Washburn

\end{document}


