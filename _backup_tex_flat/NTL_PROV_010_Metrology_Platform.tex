\documentclass[11pt]{article}

% Keep packages minimal for TeX Live "basic" installs.
\usepackage[utf8]{inputenc}
\usepackage[T1]{fontenc}
\usepackage{geometry}
\usepackage{hyperref}
\usepackage{amsmath,amssymb}
\usepackage{graphicx}
\usepackage{booktabs}
\usepackage{xcolor}
\usepackage{enumitem}
\usepackage{array}

\geometry{margin=1in}
\hypersetup{
  colorlinks=true,
  linkcolor=blue,
  urlcolor=blue
}

% ---------------------------------------------------------------------------
% Convenience macros (avoid Unicode Greek in text; use LaTeX math symbols)
% ---------------------------------------------------------------------------
\newcommand{\R}{\mathbb{R}}
\newcommand{\N}{\mathbb{N}}

\newcommand{\PatentTitle}{Synchronized Multi-Sensor Metrology Platforms for Rotating-Field Experiments with Force, Thermal, EMI, Vibration, and Environmental Instrumentation}
\newcommand{\Docket}{NTL-PROV-010}
\newcommand{\Inventors}{[Inventor Names]}
\newcommand{\Assignee}{[Assignee / Organization]}
\newcommand{\FilingDate}{February 1, 2026}

\begin{document}

\begin{center}
{\LARGE \textbf{\PatentTitle}}\\[0.75em]
{\large \textbf{Docket:} \Docket}\\[0.25em]
{\large \textbf{Inventors:} \Inventors}\\[0.25em]
{\large \textbf{Assignee:} \Assignee}\\[0.25em]
{\large \textbf{Date:} \FilingDate}\\[0.75em]
\end{center}

\vspace{-0.5em}
\hrule
\vspace{0.75em}

% ===========================================================================
% ABSTRACT (PATENT)
% ===========================================================================
\section*{Abstract}

Disclosed are apparatus, systems, methods, and non-transitory computer-readable media for constructing and operating a synchronized metrology platform for rotating-field experiments, including mechanical rotors and solid-state phased electromagnetic arrays. The platform integrates multiple sensor modalities (force/thrust proxies, temperature/thermal flux, electromagnetic interference (EMI) probes, magnetometers, vibration/acoustic sensors, and environmental sensors) on a common timebase and provides calibration, drift tracking, and artifact correlation analysis.

In various embodiments, a device under test (DUT) is mounted on a mechanically isolated measurement stage (e.g., load cell or torsion pendulum) inside an enclosure that may include a vacuum chamber. The platform measures per-run inputs and outputs including driver power, sensor responses, and environmental conditions. The platform outputs time-aligned data products and uncertainty estimates suitable for publication-grade reporting and for closed-loop control integration. The disclosed metrology platform improves reproducibility and reduces false positives by enabling synchronized artifact rejection and rigorous energy/force accounting.

% ===========================================================================
% TECHNICAL FIELD
% ===========================================================================
\section*{Technical Field}

The present disclosure relates to instrumentation and metrology for rotating-field systems, and more particularly to synchronized multi-sensor metrology platforms that measure force/weight proxies, thermal behavior, EMI/vibration confounders, and environmental variables with a unified timebase and calibration procedures.

% ===========================================================================
% BACKGROUND
% ===========================================================================
\section*{Background}

Claims of anomalous effects in rotating-field experiments are frequently confounded by ordinary phenomena including thermal buoyancy, vibration coupling, electromagnetic pickup, and measurement drift. Even in conventional systems, measurement credibility depends on careful calibration and synchronized logging across heterogeneous sensors.

Conventional experimental setups often use disparate instruments with independent clocks, leading to time misalignment between drive signals and sensor responses and enabling spurious correlations. Additionally, many setups lack a standardized null-test measurement stack and do not quantify uncertainty in a rigorous, repeatable manner.

Accordingly, there is a need for a standardized metrology platform that integrates multiple sensor modalities on a common timebase, provides calibration and drift tracking, and produces deterministic, auditable data products.

% ===========================================================================
% SUMMARY
% ===========================================================================
\section*{Summary}

This disclosure provides a synchronized metrology platform for rotating-field experiments.

In one aspect, the platform comprises: (i) a DUT mounting and isolation subsystem, (ii) a force measurement subsystem, (iii) a thermal measurement subsystem, (iv) an EMI and magnetic measurement subsystem, (v) a vibration/acoustic subsystem, (vi) an environmental subsystem, (vii) a synchronized data acquisition subsystem, and (viii) a calibration and uncertainty subsystem.

In another aspect, the platform outputs time-aligned sensor streams and derived metrics (e.g., correlations and coherence between confounder sensors and force sensors) and flags artifact conditions.

% ===========================================================================
% BRIEF DESCRIPTION OF DRAWINGS
% ===========================================================================
\section*{Brief Description of the Drawings}

Drawings may be provided later. For purposes of this specification:
\begin{itemize}[leftmargin=*]
  \item \textbf{FIG. 1} depicts a DUT on a force measurement stage with synchronized sensor suites.
  \item \textbf{FIG. 2} depicts a metrology platform block diagram including timebase distribution and logging.
  \item \textbf{FIG. 3} depicts force measurement embodiments (load cell stage and torsion pendulum).
  \item \textbf{FIG. 4} depicts EMI and vibration sensor placement and shielding.
  \item \textbf{FIG. 5} depicts calibration workflows and uncertainty accounting.
  \item \textbf{FIG. 6} depicts an optional vacuum chamber embodiment and feedthrough sensor integration.
\end{itemize}

% ===========================================================================
% DEFINITIONS
% ===========================================================================
\section*{Definitions and Notation}

Unless otherwise indicated:
\begin{itemize}[leftmargin=*]
  \item A \emph{device under test (DUT)} refers to a rotating-field generator and any associated structures (e.g., driver coils, rotors, enclosures).
  \item A \emph{force proxy} refers to a measured quantity representing force, weight, thrust, or torque (e.g., load cell output, torsion angle).
  \item A \emph{timebase} refers to a clock used to timestamp samples from all sensors.
  \item A \emph{confounder sensor} refers to a sensor measuring a variable that can induce artifacts (e.g., vibration, temperature gradient, EMI).
  \item A \emph{calibration} refers to mapping sensor outputs to physical units with quantified uncertainty.
  \item A \emph{drift} refers to slow variation of sensor output not attributable to DUT effects.
\end{itemize}

% ===========================================================================
% DETAILED DESCRIPTION
% ===========================================================================
\section*{Detailed Description}

\subsection*{1. Platform Architecture Overview}

In one embodiment, the metrology platform comprises:
\begin{itemize}[leftmargin=*]
  \item a mechanically isolated DUT mounting stage;
  \item a force measurement subsystem;
  \item thermal sensors and heat-flow sensors;
  \item EMI probes and magnetic sensors;
  \item vibration and acoustic sensors;
  \item environmental sensors (pressure, ambient temperature, airflow);
  \item a synchronized data acquisition and logging subsystem;
  \item calibration artifacts and procedures.
\end{itemize}

\subsection*{2. DUT Mounting and Mechanical Isolation}

In one embodiment, the DUT is mounted to a stage that provides:
\begin{itemize}[leftmargin=*]
  \item vibration isolation (e.g., elastomeric isolators, pneumatic isolation);
  \item symmetric mounting to reduce cross-axis coupling;
  \item controlled cable routing and strain relief to reduce cable force artifacts;
  \item tilt sensors to detect and compensate for stage tilt.
\end{itemize}

\subsection*{3. Force Measurement Subsystem}

\paragraph{3.1 Load cell stage embodiment.}
In one embodiment, the platform comprises one or more load cells configured to measure vertical force, optionally with multi-axis force measurement. The platform includes calibration masses and procedures to produce a calibrated force signal \(F(t)\) with uncertainty bounds.

\paragraph{3.2 Torsion pendulum embodiment.}
In one embodiment, the platform comprises a torsion pendulum with optical displacement sensing. The torsion constant is calibrated and the pendulum output is converted to torque or lateral force proxy.

\paragraph{3.3 Cross-axis and tilt characterization.}
In one embodiment, the platform characterizes cross-axis sensitivity by applying known forces in orthogonal directions and measuring coupling.

\subsection*{4. Thermal Measurement Subsystem}

In one embodiment, thermal measurement includes:
\begin{itemize}[leftmargin=*]
  \item contact sensors (RTDs, thermistors) at multiple locations on the DUT and stage;
  \item ambient temperature sensors;
  \item optional infrared imaging with emissivity calibration;
  \item optional heat-flux sensors or calorimetric measurement of energy balance.
\end{itemize}

Thermal drift is tracked and correlated with force proxy to detect buoyancy artifacts.

\subsection*{5. EMI and Magnetic Measurement Subsystem}

In one embodiment, EMI measurement includes:
\begin{itemize}[leftmargin=*]
  \item near-field E-field and H-field probes placed near sensor electronics and DUT;
  \item cable current probes (common-mode monitoring);
  \item spectrum snapshots synchronized to DUT drive states.
\end{itemize}

In one embodiment, magnetic measurement includes:
\begin{itemize}[leftmargin=*]
  \item 3-axis magnetometers placed at defined locations around the DUT;
  \item reference magnetometer channels used as controls.
\end{itemize}

\subsection*{6. Vibration, Acoustic, and Structural Sensors}

In one embodiment, vibration measurement includes:
\begin{itemize}[leftmargin=*]
  \item accelerometers on the DUT, on the stage, and on the environment/vacuum chamber;
  \item optional strain gauges on supports;
  \item optional acoustic microphones for airborne vibration monitoring.
\end{itemize}

The platform computes coherence between vibration channels and force channels to detect vibration-induced artifacts.

\subsection*{7. Environmental Measurement Subsystem}

In one embodiment, the platform measures:
\begin{itemize}[leftmargin=*]
  \item pressure (for vacuum experiments);
  \item ambient temperature and humidity;
  \item airflow or convective conditions near the stage.
\end{itemize}

\subsection*{8. Synchronized Data Acquisition and Timebase}

\paragraph{8.1 Common clock.}
In one embodiment, all sensors share a common clock or are timestamped against a monotonic timebase. Sensor sampling times are aligned to drive schedule timing.

\paragraph{8.2 Anti-aliasing and sampling strategy.}
In one embodiment, each sensor channel has appropriate anti-aliasing filtering and sampling rate chosen based on expected signal bandwidth.

\paragraph{8.3 Latency measurement and compensation.}
In one embodiment, sensor latency is measured and logged, and the platform compensates for latency to align sensor streams.

\paragraph{8.4 Deterministic replay artifacts.}
In one embodiment, each run produces a data bundle including sensor streams, configuration metadata, timestamps, and checksums/hashes for integrity.

\subsection*{9. Calibration and Uncertainty Accounting}

In one embodiment, the platform performs:
\begin{itemize}[leftmargin=*]
  \item load cell calibration with traceable masses;
  \item torsion pendulum calibration (torsion constant and damping);
  \item temperature sensor calibration (multi-point reference);
  \item EMI probe and magnetometer calibration (reference sources);
  \item accelerometer calibration and cross-axis characterization.
\end{itemize}

Uncertainty is computed as a combination of calibration uncertainty, drift uncertainty, and statistical noise.

\subsection*{10. Example Embodiments (Non-Limiting)}

\paragraph{Embodiment A: bench-top load cell metrology stack.}
A DUT is mounted on a load cell stage with thermal sensors, accelerometers, and near-field probes, all sampled on a common DAQ clock.

\paragraph{Embodiment B: vacuum chamber torsion pendulum stack.}
A torsion pendulum in a vacuum chamber measures force proxy while pressure, temperature, vibration, and EMI sensors are logged synchronously.

% ===========================================================================
% CLAIMS (DRAFT / PROVISIONAL-STYLE)
% ===========================================================================
\section*{Claims (Draft)}

\textbf{Note:} The following claims are an initial, non-limiting claim set intended to preserve multiple fallback positions. Final claim strategy should be reviewed by counsel.

\subsection*{Independent Claims}

\begin{enumerate}[leftmargin=*]
  \item \textbf{(System)} A metrology platform for a rotating-field device under test (DUT), the metrology platform comprising: a DUT mounting stage; a force measurement subsystem configured to output a force proxy; a plurality of confounder sensor subsystems comprising at least one of thermal sensors, EMI sensors, magnetic sensors, or vibration sensors; and a data acquisition subsystem configured to sample outputs of the force measurement subsystem and the plurality of confounder sensor subsystems on a common timebase.

  \item \textbf{(Method)} A method of measuring a rotating-field DUT, the method comprising: mounting the DUT on a mechanically isolated stage; measuring a force proxy while driving the DUT; simultaneously measuring at least one confounder variable selected from temperature, vibration, EMI, magnetic field, or environmental pressure; time-aligning the measured signals using a common timebase; and generating an output data product comprising the force proxy and at least one correlation or coherence metric between the force proxy and the confounder variable.

  \item \textbf{(Non-transitory medium)} A non-transitory computer-readable medium storing instructions that, when executed by one or more processors, cause the one or more processors to: receive synchronized sensor streams from a force sensor and at least one confounder sensor; apply calibration mappings to convert sensor outputs to physical units; compute drift and uncertainty metrics; and output a run artifact comprising calibrated sensor streams and integrity checksums.
\end{enumerate}

\subsection*{Dependent Claims (Examples; Non-Limiting)}

\begin{enumerate}[leftmargin=*]
  \setcounter{enumi}{3}
  \item The system of claim 1, wherein the force measurement subsystem comprises a load cell.
  \item The system of claim 1, wherein the force measurement subsystem comprises a torsion pendulum.
  \item The system of claim 1, further comprising a vacuum chamber configured to operate the DUT under reduced pressure.
  \item The system of claim 1, wherein the plurality of confounder sensor subsystems comprise near-field EMI probes and a magnetometer array.
  \item The system of claim 1, wherein the data acquisition subsystem comprises anti-aliasing filters and a common clock distributed to sensor modules.
  \item The method of claim 2, further comprising calibrating the force measurement subsystem using traceable masses.
  \item The method of claim 2, wherein generating the correlation or coherence metric comprises computing coherence between an accelerometer signal and the force proxy.
  \item The non-transitory medium of claim 3, wherein outputting the run artifact comprises storing configuration metadata and hashes for tamper evidence.
\end{enumerate}

% ===========================================================================
% FALLBACK POSITIONS / ADDITIONAL EMBODIMENTS
% ===========================================================================
\section*{Additional Embodiments and Fallback Positions (Non-Limiting)}

\begin{itemize}[leftmargin=*]
  \item Force measurement may include multi-axis force/torque sensing and tilt/level sensing for compensation.
  \item Thermal measurement may include infrared imaging, heat flux sensors, and calorimetry-based energy accounting.
  \item EMI measurement may include cable current probes, shield/ground integrity monitoring, and synchronized spectrum snapshots.
  \item The platform may include automated null-test runs and calibration runs as part of a runbook (method claims may be filed separately).
  \item The platform may output machine-readable data artifacts (CSV/JSON) and may include deterministic replay metadata.
\end{itemize}

\vspace{1em}
\hrule
\vspace{0.75em}
\noindent \textbf{End of Specification (Draft)}

\end{document}

