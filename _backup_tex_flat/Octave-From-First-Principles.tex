\documentclass[11pt]{article}

\usepackage[margin=1in]{geometry}
\usepackage{amsmath,amssymb}
\usepackage{microtype}
\usepackage{xcolor}
\usepackage{hyperref}

\hypersetup{
    colorlinks=true,
    linkcolor=blue,
    citecolor=blue,
    urlcolor=blue
}

\newcommand{\phiG}{\varphi}
\newcommand{\tauZero}{\tau_0}
\newcommand{\Ecoh}{E_{\mathrm{coh}}}
\newcommand{\QD}{Q_D}

% Claim hygiene tags + certificate pointers (see docs/OCTAVE_FIRST_PRINCIPLES_RIGOR_PLAN.md)
\newcommand{\RSTAG}[1]{\textcolor{gray}{\textbf{[#1]}}}
\newcommand{\CERT}[1]{\textcolor{gray}{\footnotesize\texttt{#1}}}

\title{\textbf{Octave From First Principles}\\[0.3em]
\large The Eight-Tick Cycle in Recognition Science (RS): minimal closure, phase, and its role in the forcing chain}

\author{Jonathan Washburn\\
Recognition Science Research Institute\\
Austin, Texas, USA\\
\texttt{washburn.jonathan@gmail.com}}

\date{\today}

\begin{document}
\maketitle

\section{Executive Summary}

\paragraph{What ``octave'' means here.} \RSTAG{MECH} \CERT{Spec: T6 / @FORCING\_CHAIN\_SUMMARY; Lean: IndisputableMonolith.Octave.Theorem + Patterns.grayCycle3}
In Recognition Science (RS), the \textbf{octave} is not introduced as a musical analogy first. It is the claim that a discrete, conservation-preserving ledger in three spatial dimensions naturally carries a \textbf{minimal closed phase cycle of length \(8\)}.
This is summarized in the RS architecture spec (\texttt{docs/Recognition-Science-Full-Theory.txt}) as:
\begin{quote}
``8-tick cycle — minimal ledger-compatible walk on \(Q_3\) hypercube; period \(=2^D\), \(D=3\) (T6).''
\end{quote}

\paragraph{Core derivation idea.} \RSTAG{MECH} \CERT{MECH: “ledger step = adjacency”; Lean: Octave/LedgerBridge.atomicTickStep\_implies\_grayAdj + Patterns.GrayCycleBRGC.brgcGrayCycle}
Model the ``ledger state'' as a \(D\)-bit parity register (equivalently, the vertex set of a \(D\)-dimensional hypercube graph). A \emph{ledger-compatible} step changes one bit (adjacent vertices). A full ``closure'' cycle requires covering all parity states. There are exactly \(2^D\) such states; therefore the minimal covering cycle has length \(2^D\). With \(D=3\), this yields \(8\) ticks.

\paragraph{What this document is (and is not).} \RSTAG{SCOPE NOTE} \CERT{Scope: theorem vs mechanism vs empirical; falsifier example: OCTAVE\_DNA\_OPERATIONAL\_PLAN.md}
This memo reconstructs the octave claim \emph{from first principles} using only:
discreteness, conservation/ledger accounting, and the combinatorics of hypercubes/Gray codes.
It \textbf{does not} assert that ``mod-8 structure'' must appear in any arbitrary biological length distribution.
Empirical claims require preregistered tests and can be falsified (the intron-length mod-8 gene-architecture hypothesis is an example of a falsified extrapolation in this workspace; see \texttt{OCTAVE\_DNA\_OPERATIONAL\_PLAN.md} and \texttt{FALSIFIER\_DATABASE.md}).

\section{Claim Hygiene and Scope (RS-aware)}

\paragraph{Theorem-level content (pure math).} \RSTAG{THEOREM} \CERT{Lean: Patterns/GrayCycle.lean (D=3) + Patterns/GrayCycleBRGC.lean (general D)}
The following are standard facts from discrete mathematics:
\begin{itemize}
  \item There are \(2^D\) binary parity states in \((\mathbb{Z}/2\mathbb{Z})^D\).
  \item The \(D\)-dimensional hypercube graph \(\QD\) has \(2^D\) vertices and edges corresponding to Hamming distance \(1\).
  \item \(\QD\) admits a Hamiltonian cycle (a Gray-code cycle) of length \(2^D\).
\end{itemize}

\paragraph{Mechanism-level RS assumptions (modeling choices).} \RSTAG{MECH} \CERT{Spec tags: @KERNEL / @FORCING\_CHAIN\_SUMMARY; MECH boundary: “ledger step = adjacency”}
To connect theorems to RS claims, RS assumes:
\begin{itemize}
  \item \textbf{Ledger state is discrete} (countable, serializable events).
  \item \textbf{Conservation is tracked by a ledger} (double-entry balance).
  \item \textbf{Ledger-compatible steps are adjacency steps} (single-bit updates; Gray-code realizability).
  \item \textbf{A ``full cycle'' means ``coverage''} (the phase closes only after the ledger has traversed all parity classes).
\end{itemize}
These assumptions are consistent with the narrative in \texttt{docs/Recognition-Science-Full-Theory.txt} (see \texttt{@KERNEL}, \texttt{@FORCING\_CHAIN\_SUMMARY}).

\paragraph{Certified / formal surface (Lean pointers).} \RSTAG{THEOREM} \CERT{Lean: import IndisputableMonolith.Octave}
In this workspace, the recommended single import path for the certified octave surface is:
\begin{quote}
\texttt{reality/IndisputableMonolith/Octave.lean} \quad (module \texttt{IndisputableMonolith.Octave})
\end{quote}
It aggregates:
\begin{itemize}
  \item \texttt{IndisputableMonolith.Octave.Theorem}: phase closure + an explicit 3-bit Gray-8 witness;
  \item \texttt{IndisputableMonolith.Octave.LedgerBridge}: a ledger-step bridge (posting model) implying one-bit parity adjacency;
  \item \texttt{IndisputableMonolith.Octave.LNALBridge}: alignment to LNAL's Gray-8 scheduler and neutrality exports.
\end{itemize}
As always, these are \emph{formal model facts} (THEOREM/MECH), not empirical confirmations.

\paragraph{Full certificate map (audit index).} \RSTAG{SCOPE NOTE} \CERT{Docs: docs/OCTAVE\_CERTIFICATE\_MAP.md}
For a publication/audit-ready mapping from each claim to its Lean certificate (or its MECH/EMPIRICAL protocol),
see \texttt{docs/OCTAVE\_CERTIFICATE\_MAP.md}.

\paragraph{Empirical scope (what cannot be claimed from the octave alone).} \RSTAG{SCOPE NOTE} \CERT{Audit boundary: intron mod-8 gene-architecture refuted in this workspace}
Any statement like ``DNA intron lengths prefer residue \(4 \bmod 8\)'' is \emph{not} a theorem of the octave. It is an empirical hypothesis that must survive counting-mode audits, preregistration, and falsifiers.

\section{Step 1: Discreteness + Conservation motivates a ledger}

\paragraph{From conservation to accounting.} \RSTAG{MECH} \CERT{Spec: @FORCING\_CHAIN\_SUMMARY; Lean modeling: Recognition/Ledger + LedgerPostingAdjacency}
If events are discrete and something is conserved across events, then conservation is naturally expressed as balanced accounting: every local change must be offset by an equal and opposite change (``debit = credit'').
This is the meaning of ``ledger'' in RS (see \texttt{@FORCING\_CHAIN\_SUMMARY} in \texttt{docs/Recognition-Science-Full-Theory.txt}).

\paragraph{Why discreteness matters.} \RSTAG{MECH} \CERT{Spec: T2/T3 (discreteness / serialization)}
If there were no discrete events, then the ledger is not a ledger but a continuum field; RS treats the discrete skeleton as essential for ``zero-parameter'' structure claims. (See the RS spec discussion of discreteness/serialization under T2/T3.)

\section{Step 2: A parity register yields \(2^D\) phase states}

\paragraph{Define the parity state space.} \RSTAG{THEOREM} \CERT{Lean: Patterns.Pattern d := Fin d → Bool (card = 2\^d)}
Let the ledger carry \(D\) binary degrees of freedom. Then its parity state is
\[
  \mathrm{State}(D) := (\mathbb{Z}/2\mathbb{Z})^D,
\]
which has cardinality \(|\mathrm{State}(D)| = 2^D\).

\paragraph{Hypercube graph model.} \RSTAG{THEOREM} \CERT{Lean: Patterns.OneBitDiff + Patterns.GrayCycle}
Define the hypercube graph \(\QD\) with vertex set \(\mathrm{State}(D)\) and an edge between two vertices iff they differ in exactly one coordinate (Hamming distance \(1\)).
This is the canonical discrete geometry for ``one-bit'' updates.

\section{Step 3: ``Ledger-compatible'' stepping implies Gray-code adjacency}

\paragraph{Adjacency as atomic change.} \RSTAG{MECH} \CERT{Lean bridge: LedgerPostingAdjacency.postingStep\_oneBitDiff; optional: LedgerPostingAdjacency.postingStep\_iff\_legalAtomicTick + LedgerPostingAdjacency.legalAtomicTick\_oneBitDiff; optional (proxy cost): LedgerPostingAdjacency.minCost\_monotoneStep\_implies\_postingStep; Octave/LedgerBridge.atomicTickStep\_implies\_grayAdj + Octave/LedgerBridge.postingStep\_iff\_legalAtomicTick + Octave/LedgerBridge.minCost\_monotoneStep\_implies\_postingStep; falsifier: docs/OCTAVE\_LEDGER\_BRIDGE\_FALSIFIER\_PROTOCOL.md}
RS motivates an atomic update as a minimal, local change that preserves ledger accounting constraints.
A common minimality model is ``change one bit per tick''; i.e.\ move along an edge of \(\QD\).
This is exactly the Gray-code adjacency condition.

\paragraph{Why Gray code is natural.} \RSTAG{THEOREM} \CERT{Lean: Patterns/GrayCycleBRGC.lean (general D), Patterns/GrayCycle.lean (D=3 witness)}
Gray codes are precisely sequences of binary states where each successive state differs by one bit. They give a way to enumerate all \(2^D\) parity states while keeping the step locally minimal.

\section{Step 4: The minimal closure length is \(2^D\)}

\paragraph{Coverage requirement.} \RSTAG{MECH} \CERT{MECH choice: “cycle = cover all parity classes”; Lean analog: GrayCover.complete}
RS uses the word ``cycle'' not merely as ``return to start,'' but as \textbf{complete coverage} of the ledger's parity state space (``prove closure by visiting every state class once per cycle'').

\paragraph{Lower bound.} \RSTAG{THEOREM} \CERT{Lean: Patterns.grayCover\_min\_ticks}
Any cycle that covers all \(2^D\) distinct parity states must have length at least \(2^D\), because it must contain at least one visit per state.

\paragraph{Achievability.} \RSTAG{THEOREM} \CERT{Lean: Patterns.GrayCycleBRGC.brgcGrayCycle (d>0)}
Since \(\QD\) has a Hamiltonian cycle (Gray code), there exists a length-\(2^D\) closed walk that visits every state exactly once and returns to the start.

\paragraph{Conclusion (T6 form).} \RSTAG{THEOREM} \CERT{THEOREM under MECH premises; Lean: GrayCover + GrayCycle witnesses}
Therefore, under the \emph{coverage + adjacency} model of a ledger cycle,
\[
  \text{minimal closure period} = 2^D.
\]

\paragraph{Important alignment note (Lean vs narrative).} \RSTAG{SCOPE NOTE} \CERT{Coverage vs adjacency: Patterns.lean vs Patterns/GrayCycle.lean; general D: Patterns/GrayCycleBRGC.lean}
This workspace now contains both:
\begin{itemize}
  \item \textbf{Coverage-only certificates} (\texttt{CompleteCover d}) in \texttt{reality/IndisputableMonolith/Patterns.lean};
  \item \textbf{Adjacency-aware certificates} for the octave case (\(D=3\)) in \texttt{reality/IndisputableMonolith/Patterns/GrayCycle.lean}.
\end{itemize}
The coverage layer is strong enough to certify the \emph{counting/coverage} statement ``\(2^D\) states exist and you need at least \(2^D\) ticks to cover them'', and it proves:
\begin{itemize}
  \item \texttt{Patterns.period\_exactly\_8}: \(\exists w:\texttt{CompleteCover 3},\ w.\texttt{period}=8\),
  \item \texttt{Patterns.eight\_tick\_min}: any surjective pass covering all 3-bit patterns has \(T\ge 8\).
\end{itemize}
For adjacency, this workspace now includes:
\begin{itemize}
  \item \texttt{Patterns.GrayCycle 3} and a concrete witness \texttt{Patterns.grayCycle3} (one-bit steps + wrap-around);
  \item \texttt{Patterns.GrayCover 3 8} via \texttt{Patterns.grayCover3} (surjective cover + one-bit steps);
  \item a ledger-shaped bridge: \texttt{reality/IndisputableMonolith/LedgerPostingAdjacency.lean} proves that a single ``posting'' step implies one-bit parity adjacency, and \texttt{reality/IndisputableMonolith/Octave/LedgerBridge.lean} exposes it under the octave namespace.
\end{itemize}
What remains ``beyond cardinality'' is mainly: general-\(D\) Gray-cycle existence proofs (not just \(D=3\)), and deriving the posting/atomic-step constraint from RS ledger legality and cost principles rather than assuming it. (Lightweight partial routes exist: \texttt{LedgerPostingAdjacency.LegalAtomicTick} formalizes ``monotone + unit L1 step'' $\Rightarrow$ posting, and \texttt{LedgerPostingAdjacency.ledgerJlogCost} provides a closer-to-RS proxy cost with bridges \texttt{postingStep\_of\_monotone\_and\_ledgerJlogCost\_le\_Jlog1} and \texttt{minJlogCost\_monotoneStep\_implies\_postingStep}. The full RS-cost derivation remains open.)

\section{Step 5: Specialize to \(D=3\) (the cube) to obtain \(8\)}

\paragraph{The cube as \(Q_3\).} \RSTAG{THEOREM} \CERT{Lean: Patterns.grayCycle3 (GrayCycle 3)}
For \(D=3\), \(\QD = Q_3\) is the graph of a cube. It has \(2^3 = 8\) vertices.

\paragraph{An explicit 8-cycle (Gray-code Hamiltonian cycle).} \RSTAG{THEOREM} \CERT{Lean: Patterns.grayCycle3Path / Patterns.grayCycle3\_oneBit\_step}
One valid length-8 Gray cycle is:
\[
000 \to 001 \to 011 \to 010 \to 110 \to 111 \to 101 \to 100 \to 000.
\]
Each transition flips exactly one bit, and all 8 vertices are covered once.

\paragraph{The octave.} \RSTAG{THEOREM} \CERT{Lean: Patterns.grayCycle3\_period (2\^3 = 8) + OctaveKernel phase closure}
Thus, if the ledger is 3-bit at the parity level and ``closure'' means a full adjacency-cover cycle, the minimal period is:
\[
  2^3 = 8.
\]
This is the RS octave: \textbf{the eight-tick cycle}.

\section{Step 6: Why RS emphasizes \(D=3\) (brief)}

\paragraph{RS ``dimensional rigidity'' summary.} \RSTAG{MECH} \CERT{Lean arithmetic: Verification/Dimension.lean + Gap45.lean; MECH protocol: docs/OCTAVE\_DIMENSIONAL\_RIGIDITY\_GAP45\_PROTOCOL.md}
The RS spec asserts that \(D=3\) is forced by a combination of:
\begin{itemize}
  \item \textbf{Topological linking}: non-trivial integer linking is unique to 3D (sketch: in 2D curves separate the plane; in \(D\ge4\) complements are simply connected; in 3D linking numbers exist).
  \item \textbf{Gap-45 synchronization}: a claimed unique synchronization identity \(\mathrm{lcm}(2^D,45)=360\) selects \(D=3\).
\end{itemize}
See \texttt{@DIMENSIONAL\_RIGIDITY} in \texttt{docs/Recognition-Science-Full-Theory.txt}.
This memo does not re-prove that rigidity; it only notes that, given \(D=3\), the octave is \(8\).
\paragraph{Certificate pointer (what is formal vs.\ what is meaning).} \RSTAG{SCOPE NOTE} \CERT{Lean: Dimension.onlyD3\_satisfies\_RSCounting\_Gap45\_Absolute; MECH: interpret “gap-45/360”}
The arithmetic filter \(\mathrm{lcm}(2^D,45)=360\Rightarrow D=3\) is formalized in Lean as
\texttt{reality/IndisputableMonolith/Verification/Dimension.lean} (e.g.\ \texttt{Dimension.onlyD3\_satisfies\_RSCounting\_Gap45\_Absolute}).
The key arithmetic facts behind the “360 sync” language are also certified in Lean:
\texttt{reality/IndisputableMonolith/Gap45.lean} (e.g.\ \texttt{Gap45.lcm\_8\_45\_eq\_360}, \texttt{Gap45.lcm\_9\_5\_eq\_45}).
The physical meaning of ``gap-45'' and why 360 is the synchronization target remain MECH-level obligations; we isolate them and attach preregistration templates/falsifiers in
\texttt{docs/OCTAVE\_DIMENSIONAL\_RIGIDITY\_GAP45\_PROTOCOL.md}.

\section{Step 7: From 8 ticks to phase and time}

\paragraph{Phase as a type.} \RSTAG{THEOREM} \CERT{Lean: OctaveKernel.Phase := Fin 8; Octave/Theorem.lean}
Once the minimal closure period is \(8\), it is natural to model phase as
\[
  \mathrm{Phase} := \mathbb{Z}/8\mathbb{Z},
\]
or (as in the RS Lean kernel) \(\mathrm{Phase} := \mathrm{Fin}\ 8\).

\paragraph{Time tick \(\tauZero\) (ledger units).} \RSTAG{MECH} \CERT{Spec: @UNITS\_AND\_SCALE (RS units first; SI anchor explicit)}
RS then defines an ``atomic tick'' \(\tauZero\) as the duration of one phase increment \(\mathrm{Phase}\to\mathrm{Phase}\).
Important audit note from the RS spec: \(\tauZero\) is first defined in RS units; \textbf{mapping to SI requires an explicit external anchor} (e.g.\ CODATA \(\hbar\)); see \texttt{@UNITS\_AND\_SCALE}.

\section{Step 8: The recognition operator and octave equivalence}

\paragraph{Octave-time update.} \RSTAG{MECH} \CERT{Spec: @RECOGNITION\_OPERATOR; Lean kernel: OctaveKernel step law}
The RS spec writes the fundamental discrete-time dynamics as
\[
  s(t + 8\tauZero) = \widehat{R}\bigl(s(t)\bigr),
\]
where \(\widehat{R}\) is the recognition operator and the evolution is cost-minimizing (J-minimizing), not energy-minimizing; see \texttt{@KERNEL} and \texttt{@RECOGNITION\_OPERATOR} in \texttt{docs/Recognition-Science-Full-Theory.txt}.

\paragraph{Octave equivalence (conceptual).} \RSTAG{MECH} \CERT{Concept: phase modulo 8; Lean: Octave/Theorem.lean phase\_add8}
Musically, octave equivalence means ``up one octave is the same pitch class.'' In RS, octave equivalence is ``advance one full closure cycle is the same phase class.'' Formally: phase is modulo 8; a full cycle returns to the same phase class.

\section{What the octave does \emph{not} license (audit constraint)}

\paragraph{Avoid numerology leaks.} \RSTAG{SCOPE NOTE} \CERT{Audit rule: no empirical extrapolation without preregistration + falsifiers}
The existence of an 8-tick phase type does not imply that arbitrary lengths in biology (intron lengths, gene lengths, etc.) must show mod-8 structure as a genome-architecture constraint. That is an empirical claim.

\paragraph{Empirical bridge principle.} \RSTAG{EMPIRICAL} \CERT{Protocol: preregister dataset+null+test+falsifier (see OCTAVE\_DNA\_OPERATIONAL\_PLAN.md)}
To claim an octave imprint in a biological observable \(X\), one must:
\begin{itemize}
  \item define \(X\) precisely (counting mode, unit, sampling frame),
  \item preregister the test and null model,
  \item show robustness under reasonable measurement conventions,
  \item and provide a falsifier protocol.
\end{itemize}
This is consistent with the RS spec's own ``claim hygiene'' policy: THEOREM vs HYPOTHESIS vs CERT.

\section{Assumptions (MECH) and boundary conditions (EMPIRICAL)}
\noindent \RSTAG{SCOPE NOTE} \CERT{This section is an explicit assumptions+falsifiers appendix for publication hygiene.}

\paragraph{MECH assumptions (explicit).} \RSTAG{MECH} \CERT{See also docs/OCTAVE\_CERTIFICATE\_MAP.md}
The octave theorem bundle is interpreted through the following modeling assumptions:
\begin{itemize}
  \item \textbf{Parity modeling}: a system admits a finite-dimensional parity state space \((\mathbb{Z}/2)^D\) (or \texttt{Pattern D}).
  \item \textbf{Atomic adjacency}: a “tick” corresponds to a single-coordinate ledger update (posting/atomicity), inducing one-bit parity adjacency.
  \item \textbf{Cycle meaning}: “closure” in the RS narrative is taken to mean a coverage-like cycle (visit all parity classes), not only “return to start.”
  \item \textbf{Dimension narrative}: any gap‑45/360 interpretation beyond arithmetic is MECH and must be separately justified.
\end{itemize}

\paragraph{EMPIRICAL boundary conditions (what is already falsified / scoped).} \RSTAG{EMPIRICAL} \CERT{Boundary: no numerology leaks}
\begin{itemize}
  \item \textbf{Refuted extrapolation}: genome-architecture intron mod‑8 constraints are refuted in human/mouse/fly under audit-grade counting (see \texttt{OCTAVE\_DNA\_OPERATIONAL\_PLAN.md} and \texttt{FALSIFIER\_DATABASE.md}).
  \item \textbf{Posting/atomic-step bridge falsifier}: if an intended mapping yields parity jumps \(>1\) per tick (or multi-coordinate \(\Delta\phi\)), the posting/atomic-step MECH is falsified for that mapping (see \texttt{docs/OCTAVE\_LEDGER\_BRIDGE\_FALSIFIER\_PROTOCOL.md}).
  \item \textbf{Gap‑45 / 360 mapping protocol}: any empirical use of gap‑45/360 requires a declared mapping + prereg test (see \texttt{docs/OCTAVE\_DIMENSIONAL\_RIGIDITY\_GAP45\_PROTOCOL.md}).
  \item \textbf{Downstream gate}: all domain claims must appear in the registry format (mapping → dataset → null → prereg test → falsifier) (see \texttt{docs/OCTAVE\_PREDICTION\_REGISTRY.md}).
\end{itemize}

\section{Pointers (where this lives in the RS architecture spec)}
\noindent \RSTAG{SCOPE NOTE} \CERT{Spec pointers for readers; not a proof step.}

\begin{itemize}
  \item \textbf{One-page forcing chain}: \texttt{@KERNEL} (\texttt{docs/Recognition-Science-Full-Theory.txt})
  \item \textbf{Eight-tick mechanism}: \texttt{@FORCING\_CHAIN\_SUMMARY} entries for ``eight\_tick'' (T6)
  \item \textbf{Dimensional rigidity}: \texttt{@DIMENSIONAL\_RIGIDITY}
  \item \textbf{Units and SI anchor policy}: \texttt{@UNITS\_AND\_SCALE}
  \item \textbf{Formal phase kernel}: Lean pointers in the RS spec (OctaveKernel / LNAL invariants)
\end{itemize}

\section{What it would take to make this ``fully developed'' (beyond cardinality)}
\noindent \RSTAG{SCOPE NOTE} \CERT{Roadmap summary; see docs/OCTAVE\_FIRST\_PRINCIPLES\_RIGOR\_PLAN.md.}

This memo is intentionally conservative: it separates (i) what is already certified in Lean,
from (ii) what is still a mechanistic bridge assumption in RS narratives.
To make the octave argument feel ``complete'' in the scientific sense, we need to close three gaps:

\subsection*{(A) Upgrade the Lean object from ``cover'' to ``ledger-compatible cycle''}
\noindent \RSTAG{SCOPE NOTE} \CERT{Lean: GrayCycle / GrayCover structures; D=3 witness already certified.}
This is now \emph{partially implemented}:
\begin{itemize}
  \item \texttt{Patterns.GrayCycle d} and \texttt{Patterns.GrayCover d T} exist as Lean structures;
  \item the full adjacency \(\Rightarrow\) 8-tick certificate is implemented for \(D=3\) (see \texttt{Patterns.grayCycle3} and \texttt{Patterns.grayCover3} in \texttt{reality/IndisputableMonolith/Patterns/GrayCycle.lean}).
\end{itemize}
What remains is the general-\(D\) development: prove Gray-cycle existence/minimality at arbitrary \(D\) without axioms, and then connect it to RS ledger constraints.
\paragraph{Status update (general \(D\), bounded + assumptions isolated).} \RSTAG{SCOPE NOTE} \CERT{Lean: Patterns/GrayCycleGeneral.lean (d≤64 + GrayCodeFacts)}
This workspace now includes a compiling general-\(D\) BRGC construction in
\texttt{reality/IndisputableMonolith/Patterns/GrayCycleGeneral.lean}.
It packages a general \texttt{GrayCycle d} / \texttt{GrayCover d (2\textasciicircum d)} for \(d \le 64\).
Both BRGC path injectivity and the wrap-around adjacency (last $\to$ 0) are now proved in Lean (under the stated Gray-code facts). What remains is to remove/relax the remaining scaffolding---notably the \(d \le 64\) bound and the reliance on \texttt{[GrayCodeAxioms.GrayCodeFacts]} for successor-adjacency and the 64-bit inverse---or else keep those assumptions explicitly labeled as non-core boundaries.
\paragraph{Status update (general \(D\), axiom-free).} \RSTAG{THEOREM} \CERT{Lean: Patterns/GrayCycleBRGC.lean (no axioms; d>0)}
This workspace now also includes an \emph{axiom-free} general-\(D\) BRGC construction in
\texttt{reality/IndisputableMonolith/Patterns/GrayCycleBRGC.lean}.
It defines a recursive BRGC path (via tuple \texttt{append} + \texttt{rev}) and proves:
(i) injectivity (no repeats) and (ii) one-bit adjacency \emph{including wrap-around} for all \(d>0\),
packaged as \texttt{GrayCycle d} / \texttt{GrayCover d (2\textasciicircum d)} with \emph{no} \texttt{[GrayCodeFacts]} and no \(d \le 64\) bound.

\subsection*{(B) Connect the ledger axioms to adjacency (why one-bit steps are forced)}
\noindent \RSTAG{MECH} \CERT{Open: derive atomicity/posting from RS legality/cost; partial: LedgerPostingAdjacency.postingStep\_iff\_legalAtomicTick + LedgerPostingAdjacency.legalAtomicTick\_implies\_PostingStep.}
Right now, the strongest RS-to-adjacency bridge is still MECH-level: ``ledger-compatible'' \(\Rightarrow\) ``single-bit update''.
To develop it fully, we need a theorem that derives adjacency from the ledger constraints and the cost-minimization principle,
e.g.\ show that multi-bit jumps increase cost or violate a neutrality/invariance property.
This is where the LNAL invariance chain (``neutral every 8th tick'') should be explicitly related to a bit-flip dynamics.
\paragraph{Status update.} \RSTAG{SCOPE NOTE} \CERT{Lean: LedgerPostingAdjacency + Octave/LedgerBridge; finite carriers admit AtomicTick (LedgerPostingAdjacency.accountRS\_atomicTick, d\textbackslash{}neq0)}
This workspace now contains a rigorous intermediate bridge (and a lightweight ``cost/legality'' characterization lemma):
\texttt{reality/IndisputableMonolith/LedgerParityAdjacency.lean} proves ``single-coordinate \(\pm1\) update \(\Rightarrow\) one-bit parity change'',
and \texttt{reality/IndisputableMonolith/LedgerPostingAdjacency.lean} packages a posting-style ledger tick implying that hypothesis.
Additionally, \texttt{LedgerPostingAdjacency.postingStep\_iff\_legalAtomicTick} shows that, within the model, the posting step predicate is equivalent to a simple legality predicate (\texttt{LegalAtomicTick} = monotone + unit L1 change).
And \texttt{LedgerPostingAdjacency.minCost\_monotoneStep\_implies\_postingStep} provides a small ``cost$\to$atomicity'' bridge: if a monotone nontrivial tick minimizes the proxy cost \texttt{ledgerL1Cost} among monotone nontrivial successors, then it must be a posting step.
We also provide a stronger proxy closer to RS cost: \texttt{LedgerPostingAdjacency.ledgerJlogCost} (based on \texttt{Cost.Jlog}) and the associated bridges \texttt{postingStep\_of\_monotone\_and\_ledgerJlogCost\_le\_Jlog1} and \texttt{minJlogCost\_monotoneStep\_implies\_postingStep}.
The remaining obligation is to justify (or explicitly assume) why RS legality/cost implies the posting model (or a comparably strong legality predicate such as \texttt{LegalAtomicTick}/\texttt{ledgerJlogCost}).

\subsection*{(C) Make dimensional rigidity more than an arithmetic filter}
\noindent \RSTAG{MECH} \CERT{Lean arithmetic exists; physical meaning still MECH (gap-45/360).}
The current Lean proof that \(D=3\) is forced uses the synchronization identity \(\mathrm{lcm}(2^D,45)=360\)
(\texttt{reality/IndisputableMonolith/Verification/Dimension.lean}), and the core 8/45 arithmetic lives in
\texttt{reality/IndisputableMonolith/Gap45.lean}. These certificates are now lightweight and buildable without
pulling in the full RS spec stack.
To make this ``first principles'', RS needs a derivation of:
\begin{itemize}
  \item why a 45-gap clock is forced (not postulated), and
  \item why 360 is the synchronization target (not chosen),
\end{itemize}
from earlier mechanisms (ledger, cost, and whatever ``gap-45'' is intended to represent physically).

\paragraph{Bottom line.} \RSTAG{SCOPE NOTE} \CERT{Next: deeper MECH derivations (optional) + maintain registry discipline; keep MECH boundaries explicit}
Lean now certifies both the counting core and an explicit adjacency-aware octave witness at \(D=3\) (Gray-8 cycle), plus a ledger-shaped posting bridge that implies one-bit parity adjacency under a clear step predicate.
To make the octave a fully developed \emph{first-principles dynamics} claim (and publish it as such), the remaining step is to (i) strengthen the remaining MECH bridges (notably the physical meaning of gap‑45/360 and any cost→atomicity derivation), while (ii) keeping downstream claims locked behind the registry/preregistration gate (\texttt{docs/OCTAVE\_PREDICTION\_REGISTRY.md}).

\section{Lean Certificate Map (snapshot)}

\paragraph{Canonical import.} \RSTAG{THEOREM} \CERT{Lean: import IndisputableMonolith.Octave}
\begin{quote}
\texttt{import IndisputableMonolith.Octave}
\end{quote}

\paragraph{Core theorems / symbols.} \RSTAG{THEOREM} \CERT{Certificate pointers for audit/hygiene}
\begin{itemize}
  \item \textbf{Phase closure (Fin 8)}: \texttt{reality/IndisputableMonolith/Octave/Theorem.lean} (\texttt{phase\_add8}, \texttt{phase\_add1\_iter8})
  \item \textbf{Gray-8 adjacency witness}: \texttt{reality/IndisputableMonolith/Patterns/GrayCycle.lean} (\texttt{grayCycle3}, \texttt{grayCover3})
  \item \textbf{General-\(D\) BRGC cycle (axiom-free)}: \texttt{reality/IndisputableMonolith/Patterns/GrayCycleBRGC.lean} (\texttt{GrayCycleBRGC.brgcGrayCycle}, \texttt{GrayCycleBRGC.brgcGrayCover})
  \item \textbf{General-\(D\) BRGC cycle (bounded; assumptions isolated)}: \texttt{reality/IndisputableMonolith/Patterns/GrayCycleGeneral.lean} (\texttt{GrayCycleGeneral.brgcGrayCycle}, \texttt{GrayCycleGeneral.brgcGrayCover})
  \item \textbf{Patterns observed by OctaveKernel}: \texttt{reality/IndisputableMonolith/OctaveKernel/Instances/PatternCover.lean} (\texttt{patternAtPhase}, \texttt{patternAtPhase\_oneBit\_step})
  \item \textbf{Ledger posting step $\Rightarrow$ one-bit parity adjacency}: \texttt{reality/IndisputableMonolith/LedgerPostingAdjacency.lean} (\texttt{postingStep\_oneBitDiff})
  \item \textbf{Octave-facing ledger bridge}: \texttt{reality/IndisputableMonolith/Octave/LedgerBridge.lean} (\texttt{postingStep\_implies\_grayAdj})
  \item \textbf{LNAL alignment / neutrality exports}: \texttt{reality/IndisputableMonolith/Octave/LNALBridge.lean} (\texttt{lnal\_gray8At\_eq\_patterns\_gray8At}, export of \texttt{LNAL.neutral\_every\_8th\_from0})
  \item \textbf{Gap-45 / 360 arithmetic certificates}: \texttt{reality/IndisputableMonolith/Gap45.lean} (\texttt{Gap45.lcm\_8\_45\_eq\_360}, \texttt{Gap45.lcm\_9\_5\_eq\_45})
  \item \textbf{Dimension filter (arithmetic certificate)}: \texttt{reality/IndisputableMonolith/Verification/Dimension.lean} (\texttt{Dimension.onlyD3\_satisfies\_RSCounting\_Gap45\_Absolute})
  \item \textbf{Single import bundle}: \texttt{reality/IndisputableMonolith/Octave.lean} (\texttt{IndisputableMonolith.Octave})
\end{itemize}

\section{Repro checklist (Lean)}
\noindent \RSTAG{SCOPE NOTE} \CERT{Commands that should succeed in a clean checkout of /reality.}

\paragraph{Build targets.} \RSTAG{SCOPE NOTE} \CERT{Use these to reproduce the octave certificates.}
\begin{itemize}
  \item \texttt{cd reality \&\& lake build IndisputableMonolith.Octave}
  \item \texttt{cd reality \&\& lake build IndisputableMonolith.Patterns.GrayCycle}
  \item \texttt{cd reality \&\& lake build IndisputableMonolith.Patterns.GrayCycleBRGC}
  \item \texttt{cd reality \&\& lake build IndisputableMonolith.Verification.Dimension}
  \item \texttt{cd reality \&\& lake build IndisputableMonolith.Gap45}
  \item \texttt{cd reality \&\& lake build IndisputableMonolith.Octave.LedgerBridge}
\end{itemize}

\paragraph{Octave core symbol set (recommended).} \RSTAG{SCOPE NOTE} \CERT{This is the canonical “Octave theorem bundle” surface.}
\begin{itemize}
  \item \texttt{IndisputableMonolith.Octave} (umbrella import)
  \item \texttt{Patterns.grayCycle3}, \texttt{Patterns.grayCover3}
  \item \texttt{Octave.Theorem.phase\_add8}, \texttt{Octave.Theorem.phase\_add1\_iter8}
  \item \texttt{Octave.LedgerBridge.postingStep\_implies\_grayAdj}, \texttt{Octave.LedgerBridge.atomicTickStep\_implies\_grayAdj}
  \item \texttt{Octave.LNALBridge.lnal\_gray8At\_eq\_patterns\_gray8At} (plus neutrality exports)
\end{itemize}

\paragraph{Optional minimal CI target.} \RSTAG{SCOPE NOTE} \CERT{If desired: restrict CI to the theorem bundle.}
\begin{itemize}
  \item \texttt{lake build IndisputableMonolith.Octave}
\end{itemize}

\end{document}


