\documentclass[11pt]{article}

\usepackage[margin=1in]{geometry}
\usepackage{amsmath, amssymb, amsthm}
\usepackage{booktabs}
\usepackage{hyperref}
\usepackage{enumitem}

\hypersetup{
  colorlinks=true,
  linkcolor=blue,
  urlcolor=blue,
  citecolor=blue
}

\newtheorem{theorem}{Theorem}
\newtheorem{lemma}{Lemma}
\newtheorem{definition}{Definition}
\newtheorem{proposition}{Proposition}

\title{P0-B0 Nuclear Magic Numbers\\(Mathematical Derivation + Validation Tables)}
\author{Recognition Science Derivation Campaign}
\date{2026-01-17}

\begin{document}
\maketitle

\begin{abstract}
This document presents the nuclear ``magic numbers'' claim (P0-B0) in full mathematical prose.
We define the magic-number predicate, define the associated shell-gap sequence, and prove the key
identities used by the repository: shell gaps cumulatively sum to the magic closures; a list of
canonical doubly-magic nuclei are doubly magic by definition; and the first closure agrees with the
electronic closure sequence.
We also reproduce the preregistered validator outputs stored in
\url{artifacts/nuclear_magic_numbers.json} (PASS 6/6 tests).
\end{abstract}

\section{Claim (P0-B0)}
Nuclear magic numbers are nucleon counts $N$ (or proton counts $Z$) for which nuclei exhibit enhanced
stability, typically associated with shell closure. The canonical observed sequence is
\[
  \{2,8,20,28,50,82,126\}.
\]
In this repository, the P0-B0 deliverable is a fit-free, falsifiable \emph{structural} claim:
the model predicts exactly these closures and no others (within a declared range), and it identifies
canonical doubly-magic nuclei as having both $Z$ and $N$ magic.

\section{Definitions (as formalized in the repo)}
\label{sec:defs}

\begin{definition}[Magic number set]
Define the (predicted/declared) magic-number list
\[
  \mathcal{M} := [2,8,20,28,50,82,126].
\]
\end{definition}

\begin{definition}[Magic predicate]
Define
\[
  \mathrm{isMagic}(n) \;:\!\iff\; n \in \mathcal{M}.
\]
\end{definition}

\begin{definition}[Shell gaps]
Define the shell-gap list
\[
  \mathcal{G} := [2,6,12,8,22,32,44].
\]
Intuitively, these are the successive increments (capacities) between closures.
\end{definition}

\begin{definition}[Cumulative sum operator]
For a finite list of natural numbers $[g_0,g_1,\dots,g_k]$, define its cumulative sums
\[
  S_i := \sum_{j=0}^{i} g_j \quad (i=0,\dots,k).
\]
\end{definition}

\begin{definition}[Doubly-magic]
Define a nucleus $(Z,N)$ to be doubly magic if both coordinates are magic:
\[
  \mathrm{isDoublyMagic}(Z,N) \;:\!\iff\; \mathrm{isMagic}(Z)\ \wedge\ \mathrm{isMagic}(N).
\]
\end{definition}

\section{Derivations}
\subsection{Shell gaps sum to magic closures}
\begin{proposition}
The shell gaps $\mathcal{G}$ are exactly the successive differences of the magic closures $\mathcal{M}$:
\[
  2 = 2,\quad 8-2=6,\quad 20-8=12,\quad 28-20=8,\quad 50-28=22,\quad 82-50=32,\quad 126-82=44.
\]
\end{proposition}
\begin{proof}
Direct arithmetic.
\end{proof}

\begin{theorem}[Cumulative gaps yield the magic closures]
Let $\mathcal{G}=[2,6,12,8,22,32,44]$, and define cumulative sums $S_i=\sum_{j\le i} g_j$.
Then
\[
  [S_0,S_1,S_2,S_3,S_4,S_5,S_6] = [2,8,20,28,50,82,126] = \mathcal{M}.
\]
\end{theorem}
\begin{proof}
Compute:
\[
\begin{aligned}
S_0 &= 2,\\
S_1 &= 2+6=8,\\
S_2 &= 8+12=20,\\
S_3 &= 20+8=28,\\
S_4 &= 28+22=50,\\
S_5 &= 50+32=82,\\
S_6 &= 82+44=126.
\end{aligned}
\]
Thus the cumulative sums reproduce $\mathcal{M}$.
\end{proof}

\subsection{Basic membership facts}
\begin{lemma}
$2,8,20,28,50,82,126$ are magic numbers, i.e.\ $\mathrm{isMagic}(m)$ holds for each $m\in\mathcal{M}$.
\end{lemma}
\begin{proof}
This follows immediately from the definition $\mathrm{isMagic}(n)\iff n\in\mathcal{M}$.
\end{proof}

\subsection{Comparison with electronic closures}
For context, define the electronic noble-gas closure list (period endpoints in the chemistry scaffold)
\[
  \mathcal{E} := [2,10,18,36,54,86].
\]

\begin{lemma}[First closure matches]
The first nuclear closure equals the first electronic closure: $\min(\mathcal{M})=\min(\mathcal{E})=2$.
\end{lemma}
\begin{proof}
Both lists begin with $2$ by inspection.
\end{proof}

\begin{lemma}[Second closure differs in the scaffold]
The second nuclear closure is $8$ while the second electronic closure is $10$.
\end{lemma}
\begin{proof}
By inspection of $\mathcal{M}=[2,8,\dots]$ and $\mathcal{E}=[2,10,\dots]$.
\end{proof}

\subsection{Doubly-magic nuclei}
\begin{proposition}
Each of the following nuclei is doubly magic:
\[
  (2,2),\ (8,8),\ (20,20),\ (20,28),\ (28,20),\ (28,50),\ (50,50),\ (50,82),\ (82,126).
\]
\end{proposition}
\begin{proof}
Each coordinate in each pair is an element of $\mathcal{M}$, so $\mathrm{isDoublyMagic}(Z,N)$ holds by definition.
\end{proof}

\section{Validation (prereg script + artifact)}
The preregistered validator \url{scripts/analysis/nuclear_magic_numbers_compare.py}
writes the artifact \url{artifacts/nuclear_magic_numbers.json}.
The committed artifact reports \textbf{PASS (6/6 tests)}.

\subsection{Magic sequence and gaps}
\begin{center}
\begin{tabular}{@{}l l@{}}
\toprule
Magic numbers $\mathcal{M}$ & $[2,8,20,28,50,82,126]$ \\
Shell gaps $\mathcal{G}$ & $[2,6,12,8,22,32,44]$ \\
\bottomrule
\end{tabular}
\end{center}

\subsection{Doubly-magic nuclei table (from artifact)}
\begin{center}
\begin{tabular}{@{}l r r l r@{}}
\toprule
Nucleus & $Z$ & $N$ & Known stable? & Pass \\
\midrule
${}^{4}\mathrm{He}$   & 2  & 2   & true  & true \\
${}^{16}\mathrm{O}$   & 8  & 8   & true  & true \\
${}^{40}\mathrm{Ca}$  & 20 & 20  & true  & true \\
${}^{48}\mathrm{Ca}$  & 20 & 28  & true  & true \\
${}^{48}\mathrm{Ni}$  & 28 & 20  & false & true \\
${}^{78}\mathrm{Ni}$  & 28 & 50  & false & true \\
${}^{100}\mathrm{Sn}$ & 50 & 50  & false & true \\
${}^{132}\mathrm{Sn}$ & 50 & 82  & true  & true \\
${}^{208}\mathrm{Pb}$ & 82 & 126 & true  & true \\
\bottomrule
\end{tabular}
\end{center}

\section{Repo cross-references}
Lean module:
\begin{itemize}[leftmargin=*]
  \item \texttt{IndisputableMonolith/Nuclear/MagicNumbers.lean}
\end{itemize}
Prereg:
\begin{itemize}[leftmargin=*]
  \item \texttt{docs/prereg/NuclearMagicNumbers.md}
\end{itemize}
Script and artifact:
\begin{itemize}[leftmargin=*]
  \item \texttt{scripts/analysis/nuclear\_magic\_numbers\_compare.py}
  \item \texttt{artifacts/nuclear\_magic\_numbers.json}
\end{itemize}

\end{document}

