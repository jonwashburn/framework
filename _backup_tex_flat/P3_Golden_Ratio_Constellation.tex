\documentclass[11pt,a4paper]{article}
\usepackage[utf8]{inputenc}
\usepackage[T1]{fontenc}
\usepackage{geometry}
\usepackage{hyperref}
\usepackage{enumitem}
\usepackage{amsmath}
\usepackage{amssymb}
\usepackage{graphicx}

\geometry{margin=1in}

\title{\textbf{PATENT APPLICATION}}
\author{}
\date{}

\begin{document}

\begin{center}
    \Large\textbf{GOLDEN RATIO CONSTELLATION MAPPING FOR OPTICAL COMMUNICATIONS}
\end{center}

\vspace{1cm}

\section*{FIELD OF THE INVENTION}
The present invention relates to modulation formats for coherent optical communication systems, and more specifically to geometric constellation shaping techniques that utilize the Golden Ratio ($\phi$) to minimize harmonic interference and maximize nonlinear tolerance.

\section*{BACKGROUND OF THE INVENTION}
The capacity of optical fiber networks is limited by the nonlinear Shannon limit. Standard modulation formats, such as 16-QAM and 64-QAM, utilize a Cartesian grid where symbol points are spaced by integer multiples of a fundamental distance. This regularity creates two significant problems:

\begin{enumerate}
    \item \textbf{Harmonic Interference:} The integer relationships between symbol amplitudes allow for the coherent buildup of Four-Wave Mixing (FWM) products. When signal frequencies mix ($f_{FWM} = f_i + f_j - f_k$), the resulting products often fall exactly on top of other valid symbol locations, causing severe crosstalk.
    \item \textbf{Suboptimal Packing:} Square grids do not minimize the average energy for a given minimum Euclidean distance ($d_{min}$). This results in a "shaping loss" of approximately 1.53 dB compared to a Gaussian distribution.
\end{enumerate}

Existing geometric shaping techniques often rely on complex iterative algorithms or probabilistic amplitude shaping (PAS), which increase DSP complexity and latency. There is a need for a deterministic, geometrically optimal constellation that naturally suppresses nonlinear interference.

\section*{SUMMARY OF THE INVENTION}
The present invention provides a modulation scheme called "$\phi$-QAM" (Phi-QAM). The core innovation is the use of the Golden Ratio ($\phi \approx 1.618$) to define the radial and angular spacing of the constellation points.

In one embodiment, the symbol amplitudes are defined by a geometric series $r_n = r_0 \cdot \phi^{n/2}$. Because $\phi$ is the "most irrational" number, the ratio of any two amplitudes is never a simple rational fraction. This prevents the coherent superposition of FWM products, effectively "detuning" the nonlinear interference.

In another embodiment, the angular separation of symbols is governed by the "Recognition Angle" $\theta_0 = \arccos(1/4) \approx 75.52^\circ$, which is derived from an energy-minimization principle. This angular spacing maximizes the Euclidean distance between points in the phase plane.

\section*{DETAILED DESCRIPTION}

\subsection*{Amplitude Rings}
The constellation consists of $M$ concentric rings. The radius of the $n$-th ring is given by:
\begin{equation}
    R_n = R_0 \cdot \phi^{n/2}, \quad n = 0, 1, \dots, M-1
\end{equation}
where $R_0$ is a scaling factor determined by the average power constraint. This scaling ensures that no three rings form an arithmetic progression ($R_a + R_b = 2R_c$), suppressing FWM efficiency.

\subsection*{Phase Distribution}
Points on each ring are distributed to maximize the minimum distance to neighbors on adjacent rings. In the preferred embodiment, the phase offset between ring $n$ and ring $n+1$ is the Golden Angle $\Psi = 360^\circ (1 - 1/\phi) \approx 137.5^\circ$. This phyllotactic distribution ensures uniform area coverage without rotational symmetries that could lead to phase-dependent nonlinear penalties.

\subsection*{16-Symbol Embodiment}
A specific embodiment for a 16-symbol constellation comprises:
\begin{itemize}
    \item \textbf{Inner Ring ($n=0$):} 4 symbols at radius $R_0$.
    \item \textbf{Middle Ring ($n=1$):} 4 symbols at radius $R_0 \sqrt{\phi}$.
    \item \textbf{Outer Ring ($n=2$):} 8 symbols at radius $R_0 \phi$.
\end{itemize}
This arrangement approximates a Gaussian distribution more closely than 16-QAM, providing a linear shaping gain of $\approx 0.8$ dB, in addition to the nonlinear tolerance benefits.

\section*{CLAIMS}

What is claimed is:

\begin{enumerate}
    \item A method for modulating an optical carrier, comprising:
    \begin{enumerate}
        \item receiving a stream of digital data;
        \item mapping said data to a set of complex symbols selected from a two-dimensional constellation;
        \item wherein said constellation comprises a plurality of concentric rings having radii $R_n$; and
        \item wherein the ratio of the radii of adjacent rings $R_{n+1} / R_n$ is substantially equal to the square root of the Golden Ratio ($\sqrt{\phi}$).
    \end{enumerate}

    \item The method of claim 1, wherein the angular positions of symbols on adjacent rings are offset by the Golden Angle ($\approx 137.5^\circ$).

    \item The method of claim 1, wherein the constellation comprises 16 symbols arranged in three rings having populations of 4, 4, and 8 symbols respectively.

    \item An optical transmitter comprising:
    \begin{enumerate}
        \item a digital signal processor (DSP) configured to map input bits to complex coordinates $(I, Q)$;
        \item a digital-to-analog converter (DAC) coupled to said DSP; and
        \item an optical modulator coupled to said DAC;
        \item wherein said DSP utilizes a look-up table defining a constellation where amplitude levels follow a geometric progression of powers of $\phi$.
    \end{enumerate}

    \item The transmitter of claim 4, wherein the minimum angular separation between any two symbols in the constellation is at least $\arccos(1/4)$.

    \item A system for optical communication utilizing the method of claim 1, further comprising a receiver configured to demodulate said complex symbols using a maximum likelihood sequence estimator (MLSE) or a symbol-by-symbol demapper adapted to the non-uniform grid.
\end{enumerate}

\end{document}
