\documentclass[11pt,a4paper]{article}
\usepackage[utf8]{inputenc}
\usepackage[T1]{fontenc}
\usepackage{geometry}
\usepackage{hyperref}
\usepackage{enumitem}
\usepackage{amsmath}
\usepackage{amssymb}
\usepackage{graphicx}

\geometry{margin=1in}

\title{\textbf{PATENT APPLICATION}}
\author{}
\date{}

\begin{document}

\begin{center}
    \Large\textbf{J-COST OPTIMIZED ROUTING IN OPTICAL NETWORKS}
\end{center}

\vspace{1cm}

\section*{FIELD OF THE INVENTION}
The present invention relates to optical network management and control planes, and specifically to routing and wavelength assignment (RWA) algorithms that optimize network capacity and stability by minimizing a physics-derived cost functional.

\section*{BACKGROUND OF THE INVENTION}
Optical networks are traditionally managed using algorithms that minimize metrics such as hop count, physical distance, or Optical Signal-to-Noise Ratio (OSNR). While effective for static demands, these metrics often lead to suboptimal resource allocation in dynamic, elastic optical networks.

Specifically, standard "Shortest Path First" (SPF) algorithms tend to congest central links, leading to nonlinear crosstalk and blocking. Load balancing algorithms attempt to spread traffic but often lack a rigorous physical basis for the "cost" of a link state.

There is a need for a routing metric that unifies the disparate physical constraints of an optical network (power, noise, nonlinearity, and spectral fragmentation) into a single, convex objective function that guarantees global stability and maximizes entropy.

\section*{SUMMARY OF THE INVENTION}
The present invention provides a method for "J-Cost Optimized Routing." The core innovation is the use of a specific cost functional $J(x) = \frac{1}{2}(x + 1/x) - 1$ to evaluate the "strain" or "defect" of a network link, where $x$ represents a normalized load or congestion factor.

Unlike linear cost functions ($Cost \propto Load$) or exponential barriers, the J-cost function is derived from the principle of "Recognition Composition," ensuring that the cost of a composite path is minimized when resources are distributed according to a geometric progression (specifically, the Golden Ratio $\phi$).

By routing traffic to minimize the aggregate $\sum J(x_i)$ across the network, the control plane naturally avoids both under-utilization ($x \ll 1$) and saturation ($x \gg 1$), driving the network towards a state of maximum entropy and thermodynamic stability.

\section*{DETAILED DESCRIPTION}

\subsection*{The J-Cost Functional}
The cost of a link $i$ with utilization $u_i$ and capacity $C_i$ is defined as:
\begin{equation}
    J_i = \frac{1}{2} \left( \frac{u_i}{C_{opt}} + \frac{C_{opt}}{u_i} \right) - 1
\end{equation}
where $C_{opt}$ is the optimal operating point (typically defined by the nonlinear threshold of the fiber).

This function has a unique minimum at $u_i = C_{opt}$ (where $J=0$). It penalizes both "noise-limited" weak signals ($u_i < C_{opt}$) and "nonlinearity-limited" strong signals ($u_i > C_{opt}$) symmetrically in log-space.

\subsection*{Routing Algorithm}
The control plane executes the following steps for each connection request:
\begin{enumerate}
    \item \textbf{State Discovery:} Collect real-time telemetry (power, OSNR) from all links.
    \item \textbf{Cost Calculation:} Compute the incremental $\Delta J$ for adding the new connection to every candidate path.
    \item \textbf{Path Selection:} Select the path $P$ that minimizes $\sum_{l \in P} \Delta J_l$.
\end{enumerate}
Because $J$ is strictly convex, this optimization problem has a unique global minimum, preventing routing oscillations (flapping).

\subsection*{Network Entropy Maximization}
It is a theorem of the invention that minimizing the aggregate J-cost is mathematically equivalent to maximizing the thermodynamic entropy of the network's power distribution, subject to the constraints of the fiber physics. This leads to a "self-organizing" network state that is robust to traffic bursts and component failures.

\section*{CLAIMS}

What is claimed is:

\begin{enumerate}
    \item A method for routing optical connections in a network, comprising:
    \begin{enumerate}
        \item determining a normalized load metric $x$ for each link in the network;
        \item calculating a cost value $J(x)$ for each link using a convex function of the form $f(x) + f(1/x)$; and
        \item selecting a path for a new connection that minimizes the sum of said cost values along the path.
    \end{enumerate}

    \item The method of claim 1, wherein said cost value is calculated as $J(x) = \frac{1}{2}(x + 1/x) - 1$.

    \item The method of claim 1, wherein said normalized load metric $x$ is the ratio of total channel power to the optimal nonlinear threshold power of the fiber link.

    \item A software-defined networking (SDN) controller comprising:
    \begin{enumerate}
        \item a telemetry interface for receiving link state data;
        \item a path computation element (PCE) configured to execute the method of claim 1; and
        \item a southbound interface for provisioning optical cross-connects (OXCs) according to the selected path.
    \end{enumerate}

    \item The SDN controller of claim 4, wherein the PCE is further configured to re-optimize existing connections to maintain the network in a state of minimum aggregate J-cost.
\end{enumerate}

\end{document}
