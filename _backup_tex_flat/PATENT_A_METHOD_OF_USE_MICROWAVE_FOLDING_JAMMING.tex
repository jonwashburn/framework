\documentclass[11pt]{article}

% Packages (keep minimal for broad TeX compatibility)
\usepackage[utf8]{inputenc}
\usepackage[T1]{fontenc}
\usepackage{amsmath, amssymb, amsfonts}
\usepackage{graphicx}
\usepackage{hyperref}
\usepackage{geometry}
\usepackage{microtype}

% Manual definitions for compatibility (avoid siunitx dependency)
\newcommand{\angstrom}{\text{\normalfont\AA}}
\newcommand{\SI}[2]{#1\,\text{#2}}
\newcommand{\code}[1]{\texttt{#1}}

\geometry{margin=1in}

\title{\textbf{PATENT A (Draft): Frequency-Selective Modulation of Protein Folding Using Microwave Irradiation}\\
\large Core Method-of-Use Specification and Starter Claim Set}

\author{
Jonathan Washburn\\
\texttt{jon@recognitionphysics.org}
}

\date{\today}

\begin{document}
\maketitle

\noindent\textbf{Status:} Technical draft for counsel; \textbf{not legal advice}.\\
\textbf{Related internal documents:} \code{docs/JAMMING\_PATENT\_OUTLINES.md}; \code{docs/JAMMING\_PROTOCOL.md}; \code{docs/RS\_JAMMING\_FREQUENCY\_PAPER.pdf}; \code{run\_e41\_jamming\_calc.py}.\\
\textbf{Note on examples:} any ``Example'' describing expected outcomes is \textbf{prophetic} unless explicitly stated as experimentally observed.

\section*{Abstract (Patent)}
Disclosed are methods for modulating protein folding kinetics and/or stability in an aqueous sample by applying narrowband microwave irradiation at or near a frequency-selective window, while maintaining substantially constant bulk sample temperature such that observed effects are distinguishable from dielectric heating. In embodiments, irradiation is applied in the Ku band (e.g., \SI{10}{GHz} to \SI{20}{GHz}) and may be centered near \SI{14.653}{GHz} with optional harmonics/subharmonics. Folding modulation is detected using one or more folding metrics including circular dichroism (CD) ellipticity, intrinsic fluorescence, Förster resonance energy transfer (FRET), NMR, or other spectroscopic readouts. Methods include matched-heating controls, off-resonance controls, and optional isotope-shift validation in D\(_2\)O.

\section{Field of the invention}
The present disclosure relates to biophysics and biotechnology, and more particularly to methods and protocols for frequency-selective modulation of biomolecular folding dynamics using electromagnetic irradiation with thermal compensation and control logic to separate resonance-like effects from bulk heating.

\section{Background}
Protein folding in aqueous media is commonly perturbed by temperature, denaturants, solvent composition, pressure, viscosity, and sequence mutations. Microwave irradiation in water is typically associated with dielectric heating, which broadly changes temperature-dependent kinetic rates and equilibrium constants. As a result, the art lacks reliable methods for producing reproducible, \emph{narrowband, frequency-specific} modulation of folding kinetics that is distinguishable from thermal effects.

If folding dynamics couple to specific resonant or clocked degrees of freedom---for example, hydration-shell or structured-water mechanisms---then external periodic forcing at particular frequencies may modulate folding in a way that cannot be explained by bulk heating alone. A practical method requires a protocol that (i) maintains constant bulk temperature, (ii) uses matched-heating controls and off-resonance controls, and (iii) provides a reproducible procedure for identifying and applying a frequency-selective window.

\section{Summary of the invention}
In one aspect, a method is provided for modulating folding of a protein in an aqueous sample. The method comprises: providing a sample containing a protein in an aqueous medium; irradiating the sample with narrowband electromagnetic radiation in a microwave frequency band; maintaining the bulk temperature of the sample within a tolerance during irradiation; and measuring a folding metric. The method further comprises comparing the folding metric to one or more control conditions (e.g., off-resonance frequency, or matched heating) such that a frequency-selective modulation of folding is determined to be present when the effect is localized in frequency and cannot be explained by bulk heating.

In embodiments, the microwave frequency is in the Ku band. In embodiments, the irradiation frequency is selected within a band around \SI{14.653}{GHz} (e.g., \SI{14.65}{GHz} $\pm$ \SI{0.1}{GHz}), optionally including harmonics (e.g., \SI{29.305}{GHz}) and/or subharmonics (e.g., \SI{1.832}{GHz}). In embodiments, the method includes an isotope-shift validation performed in D\(_2\)O.

\section{Brief description of drawings}
Drawings are not included in this draft. In typical embodiments, the method can be implemented with any suitable microwave applicator (waveguide, cavity, stripline, dielectric probe) and with temperature and readout instrumentation.

\section{Detailed description}

\subsection{Definitions}
Unless otherwise stated:
\begin{itemize}
    \item \textbf{Protein}: any polypeptide, peptide, or protein construct, including engineered sequences, labeled constructs, and fusion proteins.
    \item \textbf{Aqueous sample}: a liquid medium comprising water (H\(_2\)O and/or D\(_2\)O) and optionally buffer salts, co-solvents, stabilizers, denaturants, or crowding agents.
    \item \textbf{Folding metric}: any measurable observable correlated with folded fraction and/or folding kinetics, including but not limited to CD ellipticity (e.g., at 222 nm), intrinsic fluorescence (e.g., tryptophan emission), extrinsic dyes, FRET, light scattering/aggregation, NMR peaks, IR, or calorimetric signatures.
    \item \textbf{Narrowband irradiation}: electromagnetic radiation with a center frequency and bandwidth sufficiently narrow to support frequency-specific comparisons (e.g., bandwidth \(\le\) \SI{10}{MHz}, \SI{1}{MHz}, \SI{100}{kHz}, or narrower depending on apparatus).
    \item \textbf{Bulk-temperature-controlled}: the bulk sample temperature is held at a setpoint within a tolerance during each measurement (e.g., within $\pm 0.2\,^{\circ}\mathrm{C}$, $\pm 0.1\,^{\circ}\mathrm{C}$, or tighter).
    \item \textbf{Matched-heating control}: a control condition in which power/duty cycle is adjusted such that the bulk temperature trajectory (or absorbed power estimate) matches that of the test condition, while the frequency is different.
    \item \textbf{Off-resonance control}: a control condition at a frequency outside the resonant window (e.g., \SI{11.519}{GHz} or \SI{18.638}{GHz} in one embodiment), while matching temperature/absorbed power.
    \item \textbf{Frequency-selective modulation}: a statistically and practically significant change in a folding metric occurring in a bounded frequency interval, larger than off-resonance controls under matched heating.
\end{itemize}

\subsection{General method (core embodiment)}
In one embodiment, the method comprises:
\begin{enumerate}
    \item \textbf{Prepare sample.} Provide an aqueous sample containing a protein at a concentration suitable for the folding metric (e.g., $1\,\mu\mathrm{M}$ to $1\,\mathrm{mM}$ depending on assay).
    \item \textbf{Set baseline.} Measure the folding metric at a temperature setpoint with irradiation off.
    \item \textbf{Irradiate.} Apply narrowband microwave irradiation at a selected frequency \(f\) and power \(P\) (continuous-wave or pulsed) for a time interval \(T\).
    \item \textbf{Thermal control.} Maintain bulk temperature at the setpoint during irradiation using a feedback controller and/or power adjustment.
    \item \textbf{Measure effect.} Measure the folding metric during irradiation and/or after irradiation.
    \item \textbf{Compare to controls.} Compare the observed folding metric to at least one matched-heating control and/or off-resonance control.
\end{enumerate}

\subsection{Frequency selection embodiments}
\paragraph{Embodiment: fixed target near \SI{14.653}{GHz}.}
In one embodiment, the irradiation frequency \(f\) is selected within a narrow band around \SI{14.653}{GHz}. For example, \(f \in [\SI{14.55}{GHz}, \SI{14.75}{GHz}]\), with stepwise testing at increments of \SI{0.05}{GHz} or finer.

\paragraph{Embodiment: sweep and lock.}
In one embodiment, the method includes sweeping \(f\) across a band (e.g., \SI{14.0}{GHz} to \SI{15.2}{GHz}) to identify a frequency producing maximal modulation of the folding metric, then operating at that frequency for subsequent experiments.

\paragraph{Embodiment: harmonics and subharmonics.}
In one embodiment, the method includes applying irradiation at one or more related frequencies such as \(2f\), \(f/2\), or \(f/8\) and comparing effects across these frequencies under matched-heating controls.

\paragraph{Embodiment: off-rung controls.}
In one embodiment, the method includes explicit off-resonance controls at predetermined alternative frequencies (e.g., \SI{11.519}{GHz} and \SI{18.638}{GHz}) that are expected to show reduced effect relative to the primary window.

\subsection{Thermal discrimination embodiments}
\paragraph{Matched heating by temperature.}
In one embodiment, the bulk temperature trajectory is measured (e.g., thermistor/RTD near sample). For each frequency tested, the microwave duty cycle or incident power is adjusted to reproduce the same temperature trajectory, enabling direct subtraction of thermal effects.

\paragraph{Matched heating by absorbed power.}
In one embodiment, delivered and reflected microwave power are measured (e.g., via directional coupler) to estimate absorbed power. Controls are performed with equalized absorbed power across frequencies while recording bulk temperature.

\subsection{Readout embodiments}
\paragraph{CD readout.}
In one embodiment, CD ellipticity at 222 nm is measured during irradiation (or in alternating blocks) to estimate alpha-helical content or folded fraction.

\paragraph{Fluorescence readout.}
In one embodiment, intrinsic tryptophan fluorescence is measured as a kinetic proxy for folding/unfolding, including time-resolved measurements during irradiation.

\paragraph{Multi-readout.}
In one embodiment, two independent readouts are used (e.g., CD + fluorescence) to detect and reject artifacts such as scattering changes from aggregation.

\subsection{Isotope shift embodiment (D\(_2\)O)}
In one embodiment, the method is repeated in D\(_2\)O-buffered samples. A shift of the frequency-selective window compared to H\(_2\)O strengthens the interpretation that the modulation couples to hydration-shell or solvent-inertia mechanisms rather than purely to bulk heating.

\section{Examples (prophetic unless otherwise stated)}

\subsection*{Example 1: Fast folder (Trp-cage) frequency sweep}
Prepare Trp-cage (1L2Y) in phosphate buffer at pH 7.0. Set temperature to \(25\,^{\circ}\mathrm{C}\). Measure baseline fluorescence. Sweep frequency from \SI{14.0}{GHz} to \SI{15.2}{GHz} in \SI{0.05}{GHz} steps with constant-temperature control. For each step, record folding kinetics and compare to off-resonance controls at \SI{11.519}{GHz} and \SI{18.638}{GHz} under matched heating.

\subsection*{Example 2: D\(_2\)O shift}
Repeat Example 1 in D\(_2\)O buffer. Identify whether the frequency of maximal modulation shifts downward relative to H\(_2\)O.

\subsection*{Example 3: Duty-cycle to reduce heating}
Run the target frequency with pulsed irradiation (e.g., \(\le 10\%\) duty cycle) while maintaining constant bulk temperature. Compare effect sizes with continuous-wave irradiation at matched temperature.

\section{Claims (starter set; for counsel refinement)}
\noindent\textbf{What follows is a technical starter claim set to guide drafting. Counsel should rewrite for jurisdiction, support, and scope.}

\begin{enumerate}
    \item A method of modulating protein folding, comprising:
    \begin{enumerate}
        \item providing an aqueous sample comprising a protein;
        \item irradiating the aqueous sample with narrowband electromagnetic radiation in a microwave frequency band;
        \item maintaining a bulk temperature of the aqueous sample within a temperature tolerance during the irradiating; and
        \item measuring a folding metric of the protein,
    \end{enumerate}
    wherein the folding metric differs from a control condition by an amount not explained by bulk heating of the aqueous sample.

    \item The method of claim 1, wherein the microwave frequency band comprises a frequency in a range of \SI{10}{GHz} to \SI{20}{GHz}.

    \item The method of claim 1, wherein the narrowband electromagnetic radiation has a center frequency in a range of \SI{14.55}{GHz} to \SI{14.75}{GHz}.

    \item The method of claim 3, wherein the center frequency is \SI{14.65}{GHz} $\pm$ \SI{0.1}{GHz}.

    \item The method of claim 1, further comprising sweeping the center frequency across a band and selecting a frequency that maximizes an effect size of the folding metric.

    \item The method of claim 5, wherein the band is \SI{14.0}{GHz} to \SI{15.2}{GHz}.

    \item The method of claim 1, wherein maintaining the bulk temperature comprises maintaining temperature within $\pm 0.2\,^{\circ}\mathrm{C}$.

    \item The method of claim 1, further comprising performing a matched-heating control by adjusting a duty cycle or power such that a bulk temperature trajectory is substantially matched across two different irradiation frequencies.

    \item The method of claim 1, further comprising performing an off-resonance control at a frequency selected to be outside a resonant window.

    \item The method of claim 9, wherein the off-resonance control frequency comprises \SI{11.5}{GHz} or \SI{18.6}{GHz}.

    \item The method of claim 1, wherein irradiating comprises pulsed irradiation with a duty cycle selected to reduce bulk heating.

    \item The method of claim 1, wherein the folding metric comprises circular dichroism (CD) ellipticity at a wavelength near 222 nm.

    \item The method of claim 1, wherein the folding metric comprises intrinsic fluorescence of tryptophan.

    \item The method of claim 1, wherein the folding metric comprises a FRET signal of a labeled protein construct.

    \item The method of claim 1, further comprising measuring the folding metric at a plurality of frequencies and computing a frequency-response curve.

    \item The method of claim 15, further comprising identifying a peak or dip in the frequency-response curve corresponding to a resonant frequency window.

    \item The method of claim 1, wherein the aqueous sample comprises D\(_2\)O.

    \item The method of claim 17, further comprising comparing a resonant frequency window in D\(_2\)O to a resonant frequency window in H\(_2\)O and determining an isotope-dependent frequency shift.

    \item The method of claim 1, wherein the protein comprises a fast-folding miniprotein.

    \item The method of claim 19, wherein the protein comprises Trp-cage (1L2Y) or a ubiquitin construct.
\end{enumerate}

\end{document}


