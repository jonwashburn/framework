\documentclass[11pt]{article}

% Packages (keep minimal for broad TeX compatibility)
\usepackage[utf8]{inputenc}
\usepackage[T1]{fontenc}
\usepackage{amsmath, amssymb, amsfonts}
\usepackage{graphicx}
\usepackage{hyperref}
\usepackage{geometry}
\usepackage{microtype}

% Manual definitions for compatibility (avoid siunitx dependency)
\newcommand{\angstrom}{\text{\normalfont\AA}}
\newcommand{\SI}[2]{#1\,\text{#2}}
\newcommand{\code}[1]{\texttt{\detokenize{#1}}}

\geometry{margin=1in}

\title{\textbf{PATENT E (Draft): Use Cases for Frequency-Selective Microwave}\\
\textbf{Modulation of Protein Folding (Research Tools, Intermediate Trapping, and Bioprocess Control)}}

\author{
Jonathan Washburn\\
\texttt{jon@recognitionphysics.org}
}

\date{\today}

\begin{document}
\maketitle

\noindent\textbf{Status:} Technical draft for counsel; \textbf{not legal advice}.\\
\textbf{Related internal documents:} \code{docs/JAMMING_PATENT_OUTLINES.md}; \code{docs/JAMMING_PROTOCOL.md}; \code{docs/RS_JAMMING_FREQUENCY_PAPER.pdf}; \code{docs/RS_PROTEIN_FOLDING_BASELINE_PAPER.pdf}.\\
\textbf{Note on examples:} any ``Example'' describing expected outcomes is \textbf{prophetic} unless explicitly stated as experimentally observed.

\section*{Abstract (Patent)}
Disclosed are use-case methods and kits enabled by frequency-selective microwave modulation of protein folding in aqueous samples under temperature-controlled conditions. In embodiments, narrowband microwave irradiation at or near a resonant frequency window (e.g., in the Ku band and optionally near \SI{14.653}{GHz}) is applied to slow folding kinetics, induce controlled destabilization, or alter folding pathways beyond what is explained by bulk heating. Use cases include: (i) trapping and stabilizing folding intermediates for measurement or isolation; (ii) controlling aggregation and improving yield during in vitro refolding or bioprocess steps by modulating folding kinetics under thermal compensation; and (iii) standardized research kits comprising an irradiation system and analysis instructions for generating frequency-response curves and operating at an identified frequency window.

\section{Field of the invention}
The present disclosure relates to biotechnology and biophysical instrumentation. More particularly, it relates to applications of frequency-selective microwave irradiation for controlling and measuring protein folding dynamics, including methods for trapping intermediates, reducing aggregation, and standardized assay kits for laboratories and manufacturing settings.

\section{Background}
Protein folding and refolding are central to biochemical research and industrial bioprocessing. Many proteins fold through transient intermediate states, some of which are difficult to observe due to short lifetimes. Aggregation and misfolding are persistent problems in recombinant protein expression and in vitro refolding, often requiring extensive optimization of temperature, denaturant concentration, buffer composition, and stirring/flow conditions.

Conventional interventions (temperature ramps, chemical chaperones, denaturants) often act broadly and can perturb many properties simultaneously. If folding dynamics exhibit frequency-selective sensitivity to external forcing, then applying narrowband irradiation under thermal compensation may enable new classes of controlled interventions: slowing folding to trap intermediates, selectively destabilizing states, or tuning kinetics to reduce aggregation without changing chemical conditions.

\section{Summary of the invention}
In one aspect, a method is provided for trapping a folding intermediate of a protein, comprising initiating folding of a protein in an aqueous sample, applying narrowband microwave irradiation at or near an operating frequency window while maintaining bulk sample temperature within a tolerance, and measuring a folding metric to detect and optionally stabilize an intermediate state.

In another aspect, a method is provided for improving yield during in vitro refolding or bioprocess operations by applying temperature-controlled narrowband microwave irradiation to modulate folding kinetics and reduce aggregation relative to a control process.

In another aspect, a kit is provided comprising a microwave irradiation system, a sample cell, temperature control components, and instructions/software for performing a calibration sweep, identifying an operating frequency window, and running experiments or process steps at the identified window with matched-heating controls.

\section{Brief description of drawings}
Drawings are not included in this draft. Typical filing figures include:
\begin{itemize}
    \item Fig. 1: schematic timeline of folding initiation, irradiation window, and intermediate capture measurement.
    \item Fig. 2: frequency-response curve and selection of an operating window for intermediate trapping.
    \item Fig. 3: bioprocess/refolding setup showing irradiation stage, temperature control, and aggregation monitoring.
    \item Fig. 4: kit components and workflow (calibration \(\rightarrow\) lock mode).
\end{itemize}

\section{Detailed description}

\subsection{Definitions}
Unless otherwise stated:
\begin{itemize}
    \item \textbf{Intermediate state}: a partially folded conformational ensemble between unfolded and folded states, detectable by one or more folding metrics.
    \item \textbf{Folding initiation}: any operation that starts folding or refolding, including dilution from denaturant, temperature jump, pH jump, or buffer exchange.
    \item \textbf{Aggregation metric}: any measurement correlated with aggregation, including light scattering, turbidity, SEC yield, precipitation, or loss of soluble protein.
    \item \textbf{Yield}: amount of correctly folded, soluble, functional protein recovered from a process step.
    \item \textbf{Temperature-controlled}: bulk temperature maintained within a tolerance (e.g., \(\pm 0.2\,^{\circ}\mathrm{C}\) or tighter).
    \item \textbf{Operating frequency window}: a bounded frequency interval selected via calibration/verification where an effect is strongest and distinguishable from heating.
\end{itemize}

\subsection{Use case 1: Trapping and measuring folding intermediates}
\subsubsection{Overview}
In one embodiment, folding is initiated (e.g., by denaturant dilution), and microwave irradiation is applied during a selected time window to slow folding transitions. This can lengthen the lifetime of intermediate states, enabling measurement by spectroscopy or structural probes.

\subsubsection{Method outline}
In one embodiment, intermediate trapping comprises:
\begin{enumerate}
    \item Initiate folding of a protein in an aqueous sample.
    \item Apply narrowband microwave irradiation at a selected operating frequency window while maintaining constant bulk temperature.
    \item Measure one or more folding metrics during irradiation to detect an intermediate state.
    \item Optionally, perform a quench or stabilization step (e.g., rapid cooling, buffer change) to preserve the intermediate for downstream characterization.
\end{enumerate}

\subsubsection{Measurement embodiments}
In one embodiment, folding metrics include fluorescence kinetics (e.g., tryptophan emission), CD ellipticity (e.g., 222 nm), time-resolved FRET, hydrogen-deuterium exchange, limited proteolysis susceptibility, or other conformational probes.

\subsubsection{Selection of operating frequency window}
In one embodiment, the operating window is identified via a sweep that measures folding metric response versus frequency under matched-heating controls, then selecting a frequency band maximizing residual effect (see calibration/verification methods).

\subsection{Use case 2: Reducing aggregation and improving refolding yield}
\subsubsection{Overview}
In one embodiment, aggregation occurs when partially folded species expose hydrophobic surfaces and collide faster than productive folding completes. By modulating folding kinetics (slowing a problematic step or reducing nonproductive transitions), irradiation can reduce aggregation under controlled conditions.

\subsubsection{Method outline}
In one embodiment, a refolding yield improvement method comprises:
\begin{enumerate}
    \item Provide an unfolded or partially unfolded protein solution (e.g., after expression inclusion body solubilization).
    \item Initiate refolding by dilution, buffer exchange, or controlled temperature change.
    \item Apply narrowband irradiation at an operating frequency window while maintaining constant bulk temperature and optionally matching absorbed power across controls.
    \item Monitor an aggregation metric and/or yield metric.
    \item Compare to a control refolding process without irradiation or at off-resonance irradiation under matched heating.
\end{enumerate}

\subsubsection{Process embodiments}
In one embodiment, irradiation is applied in a batch reactor. In another embodiment, irradiation is applied in a flow cell during refolding (continuous manufacturing), enabling fixed residence times and thermal management.

\subsubsection{Aggregation monitoring embodiments}
In one embodiment, aggregation is monitored by light scattering, turbidity, or inline optical density. In another embodiment, aggregation and yield are measured by SEC, DLS, or activity assay.

\subsection{Use case 3: Standardized research kit and workflow}
\subsubsection{Kit components}
In one embodiment, a kit comprises:
\begin{itemize}
    \item a microwave irradiation module (source + applicator compatible with aqueous samples),
    \item a sample cell (e.g., thin-path or microfluidic),
    \item temperature control components (sensor + thermal stage),
    \item software/instructions for frequency sweeps, matched-heating controls, and operating window selection.
\end{itemize}

\subsubsection{Kit workflow}
In one embodiment, the workflow comprises:
\begin{enumerate}
    \item calibration sweep to identify an operating frequency window for a target protein/solvent,
    \item lock mode operation at the window for kinetics or intermediate trapping,
    \item optional D\(_2\)O shift verification.
\end{enumerate}

\section{Examples (prophetic unless otherwise stated)}

\subsection*{Example 1: Intermediate trapping in a fast folder}
Initiate folding of a fast-folding miniprotein and apply irradiation at a selected window during the transition. Measure fluorescence kinetics and compare intermediate lifetime with and without irradiation under matched heating.

\subsection*{Example 2: Refolding yield improvement}
Perform in vitro refolding of a recombinant protein from denaturant and apply irradiation during the early refolding stage under temperature control. Monitor aggregation via scattering and measure soluble yield. Compare to off-resonance controls.

\subsection*{Example 3: Kit-based frequency selection}
Run a calibration sweep with matched heating to identify a frequency window showing the strongest residual response in a folding metric, then run subsequent experiments at that frequency in lock mode.

\section{Claims (starter set; for counsel refinement)}
\noindent\textbf{What follows is a technical starter claim set to guide drafting. Counsel should rewrite for jurisdiction, support, and scope.}

\begin{enumerate}
    \item A method of trapping a folding intermediate of a protein, comprising:
    \begin{enumerate}
        \item initiating folding of a protein in an aqueous sample;
        \item irradiating the aqueous sample with narrowband microwave radiation at an operating frequency window while maintaining a bulk temperature of the aqueous sample within a temperature tolerance; and
        \item measuring a folding metric to detect an intermediate state.
    \end{enumerate}

    \item The method of claim 1, further comprising performing a quench operation after irradiating to preserve the intermediate state for downstream characterization.

    \item The method of claim 1, wherein measuring the folding metric comprises measuring intrinsic fluorescence or a FRET signal.

    \item The method of claim 1, wherein measuring the folding metric comprises measuring circular dichroism ellipticity near 222 nm.

    \item A method of improving yield of correctly folded protein during an in vitro refolding process, comprising:
    \begin{enumerate}
        \item initiating refolding of a protein in an aqueous sample; and
        \item irradiating the aqueous sample with narrowband microwave radiation at an operating frequency window while maintaining bulk temperature within a tolerance,
    \end{enumerate}
    wherein an aggregation metric is reduced or a soluble yield metric is increased relative to a control process.

    \item The method of claim 5, wherein the control process comprises off-resonance irradiation under matched heating.

    \item The method of claim 5, wherein irradiating is performed in a flow cell during continuous refolding.

    \item A kit for performing frequency-selective modulation of protein folding, comprising:
    \begin{enumerate}
        \item a microwave irradiation module configured to output narrowband microwave radiation;
        \item a sample cell configured to contain an aqueous protein sample;
        \item a temperature control subsystem configured to maintain bulk temperature within a tolerance; and
        \item instructions for performing a frequency sweep to identify an operating frequency window and for operating at the operating frequency window.
    \end{enumerate}

    \item The kit of claim 8, further comprising software configured to compute an operating frequency window based on folding metric measurements and matched-heating controls.

    \item The method of claim 1 or claim 5, wherein the operating frequency window is in a Ku-band range.

    \item The method of claim 1 or claim 5, wherein the operating frequency window comprises a band around \SI{14.653}{GHz}.

    \item The method of claim 1 or claim 5, further comprising repeating the method in an aqueous sample comprising D\(_2\)O and determining an isotope-dependent shift of the operating frequency window.
\end{enumerate}

\end{document}


