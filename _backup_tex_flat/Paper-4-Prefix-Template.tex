\documentclass[11pt]{article}

\usepackage{amsmath,amssymb,amsthm}

\title{Prefix-Coherent Template Bookkeeping for Mesh Assemblies of Calibrated Sheets}
\author{
Jonathan Washburn\\
Recognition Science\\
Recognition Physics Institute\\
\texttt{jon@recognitionphysics.org}\\
Austin, Texas, USA
}
\date{}

% --- Theorem environments ---
\newtheorem{theorem}{Theorem}
\newtheorem{lemma}{Lemma}
\newtheorem{proposition}{Proposition}
\newtheorem{corollary}{Corollary}
\theoremstyle{definition}
\newtheorem{definition}{Definition}
\newtheorem{remark}{Remark}

% --- Recognition Geometry macros (compatible, optional framing) ---
\newcommand{\config}{\mathcal{C}}
\newcommand{\events}{\mathcal{E}}

% --- Notation ---
\newcommand{\R}{\mathbb{R}}
\newcommand{\N}{\mathbb{N}}
\newcommand{\Z}{\mathbb{Z}}
\newcommand{\dist}{\operatorname{dist}}
\newcommand{\Lip}{\operatorname{Lip}}
\newcommand{\supp}{\operatorname{supp}}
\newcommand{\diam}{\operatorname{diam}}
\newcommand{\floorround}[1]{\left\lfloor #1 \right\rceil}

% Euclidean norm
\newcommand{\norm}[1]{\left\lVert #1 \right\rVert}

\begin{document}
\maketitle

\begin{abstract}
We develop a global bookkeeping mechanism that produces face-to-face coherence for large families of localized calibrated sheets assembled on a cubical mesh. The key idea is a prefix-template selection rule: fix an ordered master list of transverse translation parameters for each direction label, and in each cell choose the active sheets as an initial prefix of that list.

Under a slow-variation hypothesis on the integer prefix lengths across adjacent cells, we prove that the mismatch across any interior face is confined to short tails, producing an $O(h)$ face-edit regime in which the unmatched boundary mass is a controlled fraction of the matched boundary mass. We then show that the induced discrete transverse measures admit integral optimal couplings, enabling facewise pairings of sheets by integer transport plans.

Finally, we package the construction into a global coherence theorem simultaneously across all direction labels produced by a stable dictionary decomposition of a cone-valued form field. This provides a deterministic alternative to solving large global assignment problems, and is designed specifically to interface with weighted flat-norm gluing estimates for microstructured calibrated currents.
\end{abstract}

\section{Introduction}

In mesh-based constructions built from many local pieces (``sheets''), the local geometry is not the hard part; the hard part is \emph{global coherence}.
Even if each cell contains a nicely controlled family of calibrated sheets, the union typically fails to glue across interior faces:
the families on the two sides do not automatically match sheet-by-sheet, so boundaries appear on internal interfaces.

A naive fix is to solve a large global assignment problem: decide, for every interior face, which sheet on one side should be paired with which sheet on the other.
But this becomes combinatorially expensive and unstable when multiplicities vary from cell to cell.

This paper isolates a deterministic alternative:
\begin{quote}
\emph{Fix one ordered master template of transverse parameters, and in each cell activate only an initial segment (a prefix) of that master list.}
\end{quote}
The ``prefix'' rule makes mismatch across a face \emph{explicit}: it can only occur in a terminal tail.
If neighboring prefix lengths vary slowly, that tail is a small fraction (an $O(h)$ fraction in the regime relevant for cubical meshes of size $h$).

The second ingredient is integer transport.
The facewise matching problem can be phrased as an optimal transport problem between two atomic measures with \emph{integer} weights.
We show such transport problems admit \emph{integral} optimal couplings, so an optimal plan can be realized as an honest pairing of unit masses (i.e.\ of sheets).

\subsection*{Recognition Geometry framing (optional)}
In Recognition Geometry terms, one can view this as a recognizer that turns a continuous ``mass budget'' configuration into a discrete event:
the event is the set of activated template atoms.
The prefix rule is a particularly rigid recognizer that makes mismatch structure transparent.

\section{Discrete template model}

We fix a mesh scale $h\in(0,1)$ and work with a cubical mesh $\{Q\}$ (in a chart or in a local Euclidean model).
Two cubes $Q,Q'$ are \emph{neighbors} if they share an interior face $F=Q\cap Q'$.

We isolate one direction label (one sheet family) first, then treat multiple labels in a later section.

\subsection{Master template and prefixes}

Fix a transverse parameter space $\R^{d_\perp}$ (in calibrated complex codimension-$p$ applications one has $d_\perp=2p$).
Fix constants $C_0\ge 1$ and $\varrho\in(0,1)$.
A \emph{master template} is an ordered sequence
\[
\mathbf{y}=(y_a)_{a\ge 1}\subset B_{C_0\varrho h}(0)\subset \R^{d_\perp}.
\]

\begin{definition}[Prefix measures]
For each $N\in\N$, define the atomic prefix measure
\[
\nu^{(N)} := \sum_{a=1}^{N}\delta_{y_a}.
\]
For a cube $Q$, an \emph{integer prefix length} $N_Q\in\N$ activates the prefix $\{y_1,\dots,y_{N_Q}\}$, equivalently $\nu^{(N_Q)}$.
\end{definition}

\begin{remark}[Why the template radius scales like $\varrho h$]
In the downstream geometric application, the transverse parameters encode small translations of a template at scale comparable to the cell diameter.
Keeping $\norm{y_a}\lesssim \varrho h$ ensures that when face parameterizations vary by $O(h)$ between neighboring cells, the induced face displacement is $O(h^2)$.
This is the scale that couples correctly with flat-norm and filling estimates.
\end{remark}

\subsection{Face restriction maps}

A purely combinatorial prefix rule does not yet define how a sheet ``hits'' a face.
For that we introduce \emph{face maps} that convert a transverse parameter $y$ into a face parameter $u$.

Fix an interior face $F=Q\cap Q'$.
Let $\Omega_F\subset\R^{d_F}$ denote the parameter domain on the face (the precise dimension $d_F$ is irrelevant for this paper).

\begin{definition}[Face maps]
A pair of face maps for the face $F=Q\cap Q'$ is a pair of maps
\[
\Phi_{Q,F},\,\Phi_{Q',F}: B_{C_0\varrho h}(0)\to \Omega_F
\]
that represent how the transverse parameter is seen on the face from the two sides.
\end{definition}

The only structural property we need is that adjacent face maps are close.

\begin{definition}[Uniform face-map control]
We say the face maps satisfy \emph{uniform control at mesh $h$} if there exist constants $C_{\Phi,0},C_\Phi\ge 1$ such that
\[
\norm{\Phi_{Q,F}}_{\mathrm{Lip}}+\norm{\Phi_{Q',F}}_{\mathrm{Lip}}\le C_{\Phi,0},
\qquad
\norm{\Phi_{Q,F}-\Phi_{Q',F}}_{\mathrm{Lip}}\le C_\Phi\, h,
\]
where $\norm{\cdot}_{\mathrm{Lip}}$ denotes the Lipschitz constant on $B_{C_0\varrho h}(0)$ with respect to the Euclidean norm.
\end{definition}

\begin{definition}[Induced face measures]
Given a cube prefix length $N_Q$, define the induced face measure (from the $Q$ side) by
\[
\mu_{Q\to F} := (\Phi_{Q,F})_\# \nu^{(N_Q)}.
\]
Similarly, from the $Q'$ side $\mu_{Q'\to F}:=(\Phi_{Q',F})_\#\nu^{(N_{Q'})}$.
\end{definition}

\section{Slow variation and prefix mismatch decomposition}

\subsection{Slow variation}

The prefix rule is useful only if neighboring cubes choose comparable prefix lengths.

\begin{definition}[Neighbor slow variation]
Fix $\theta\in[0,1]$. We say prefix lengths vary slowly at scale $h$ if for every neighbor pair $Q\sim Q'$,
\[
|N_Q-N_{Q'}|\le \theta\, h\, \min\{N_Q,N_{Q'}\}.
\]
\end{definition}

In applications, slow variation is typically produced by rounding a Lipschitz real target count.
The following lemma is a clean, standalone version.

\begin{lemma}[Slow variation under rounding of Lipschitz targets]
Let $\{Q\}$ be a cubical mesh of side $h$ inside a coordinate chart.
Let $f$ be a nonnegative Lipschitz function with $\Lip(f)\le L$ in that chart.
Fix $m\ge 1$ and define real target counts
\[
n_Q := m\,h^{d_\perp}\, f(x_Q),
\]
where $x_Q\in Q$ is a chosen basepoint.
Define integer counts by nearest-integer rounding $N_Q:=\floorround{n_Q}$.
Then for adjacent cubes $Q\sim Q'$,
\[
|N_Q-N_{Q'}|\le L\,m\,h^{d_\perp+1}+1.
\]
If moreover $f\ge f_0>0$ and $m\,h^{d_\perp+1}\ge 2/f_0$, then there is a constant $C=C(L,f_0)$ such that
\[
|N_Q-N_{Q'}|\le C\,h\,N_Q.
\]
\end{lemma}

\begin{proof}
Nearest-integer rounding satisfies $|N_Q-N_{Q'}|\le |n_Q-n_{Q'}|+1$.
By Lipschitz continuity, $|f(x_Q)-f(x_{Q'})|\le L\,\dist(x_Q,x_{Q'})\le Lh$ for neighbors, hence
$|n_Q-n_{Q'}|\le m\,h^{d_\perp}\cdot Lh = L\,m\,h^{d_\perp+1}$.

If $f\ge f_0$, then $n_Q\ge m\,h^{d_\perp}f_0$, so $N_Q\ge n_Q-1$.
Under $m\,h^{d_\perp+1}\ge 2/f_0$ one has $m\,h^{d_\perp}f_0\ge 2/h$, hence $N_Q\ge 1/h$.
Therefore $1\le hN_Q$, and
\[
|N_Q-N_{Q'}|\le L\,m\,h^{d_\perp+1}+1 \le \Bigl(\frac{L}{f_0}+1\Bigr)\,hN_Q.
\]
\end{proof}

\subsection{Prefix mismatch decomposition}

The key combinatorial point is that two prefixes differ only in a tail.

\begin{proposition}[Prefix mismatch decomposition]
Let $Q,Q'$ be neighboring cubes with prefix lengths $N_Q,N_{Q'}$.
Let $N_{\min}:=\min\{N_Q,N_{Q'}\}$ and $N_{\max}:=\max\{N_Q,N_{Q'}\}$.
Then the activated index sets decompose as a common prefix plus a tail:
\[
\{1,\dots,N_Q\}=\{1,\dots,N_{\min}\}\ \cup\ \{N_{\min}{+}1,\dots,N_Q\},
\]
\[
\{1,\dots,N_{Q'}\}=\{1,\dots,N_{\min}\}\ \cup\ \{N_{\min}{+}1,\dots,N_{Q'}\}.
\]
In particular, the symmetric difference of the two activated sets is exactly the tail index set
\[
\{N_{\min}{+}1,\dots,N_{\max}\},
\]
which has cardinality $|N_Q-N_{Q'}|$.
\end{proposition}

\begin{proof}
This is immediate from the definition of $N_{\min}$ and $N_{\max}$.
\end{proof}

\begin{corollary}[Tail size is an $O(h)$ fraction under slow variation]
If the neighbor slow-variation bound $|N_Q-N_{Q'}|\le \theta h\,N_{\min}$ holds, then the mismatch tail has cardinality at most $\theta h\,N_{\min}$.
Equivalently, the tail is a $\theta h$ fraction of the common prefix size.
\end{corollary}

\begin{proof}
By the previous proposition, the tail cardinality is $|N_Q-N_{Q'}|$. Divide by $N_{\min}$.
\end{proof}

\subsection{Vertex-based prefixes (used in corner-exit schemes)}
Some constructions attach templates at vertex stars rather than per-face.
The same prefix logic applies, but with counts $N_{Q,v}$ indexed by cube $Q$ and vertex $v$ of $Q$.
If adjacent cubes share an edge through $v$, requiring $|N_{Q,v}-N_{Q',v}|\le 1$ forces symmetric differences of size at most one,
because the active sets are prefixes of one ordered master list.
This reduces all local mismatch to adding or removing a single terminal atom, which is the simplest possible edit regime.

\section{Face-level coherence up to $O(h)$ edits}

Prefix mismatch becomes a geometric boundary mismatch only after one assigns a boundary ``weight'' to each activated index.
We keep this abstract and isolate a hypothesis that is exactly what downstream gluing estimates need.

Fix a face $F=Q\cap Q'$.
Suppose that each activated index $a$ corresponds to a sheet trace on the face, and define a nonnegative weight $b_a(F)$ representing
its boundary-mass contribution on $F$ (for example, the $(k-1)$-mass of a face slice current).

Define the total face boundary weights on each side by
\[
B_Q(F):=\sum_{a=1}^{N_Q} b_a(F),\qquad B_{Q'}(F):=\sum_{a=1}^{N_{Q'}} b_a(F),
\]
and define the unmatched tail weight by
\[
B_{\mathrm{tail}}(F):=\sum_{a=N_{\min}+1}^{N_{\max}} b_a(F).
\]

\begin{definition}[$O(h)$ face-edit regime]
We say the face $F$ is in the $O(h)$ face-edit regime if
\[
B_{\mathrm{tail}}(F)\le \theta_F\bigl(B_Q(F)+B_{Q'}(F)\bigr)
\qquad\text{for some }\theta_F\lesssim h.
\]
\end{definition}

The prefix scheme guarantees that \emph{all} mismatch is in the tail.
The remaining question is whether the tail carries only an $O(h)$ fraction of the total face boundary weight.
A sufficient condition is uniform comparability of per-sheet face weights.

\begin{proposition}[Uniform weights force $O(h)$ face edits]
Assume that for the face $F$ there exist constants $0<b_-\le b_+$ such that
\[
b_-\le b_a(F)\le b_+\qquad\text{for every activated index }a\le N_{\max}.
\]
Then
\[
\frac{B_{\mathrm{tail}}(F)}{B_Q(F)+B_{Q'}(F)}\ \le\ \frac{b_+}{2b_-}\cdot \frac{|N_Q-N_{Q'}|}{N_{\min}}.
\]
In particular, if neighbor slow variation holds with parameter $\theta$ then $F$ lies in the $O(h)$ face-edit regime with
$\theta_F\le \frac{b_+}{2b_-}\theta h$.
\end{proposition}

\begin{proof}
The tail contains exactly $|N_Q-N_{Q'}|$ indices, each with weight at most $b_+$, so $B_{\mathrm{tail}}(F)\le b_+|N_Q-N_{Q'}|$.
On the other hand, the combined weight of the common prefix part is at least $2b_-N_{\min}$ (each side contains at least the first $N_{\min}$ indices).
Thus $B_Q(F)+B_{Q'}(F)\ge 2b_-N_{\min}$, and the ratio bound follows.
\end{proof}

\begin{remark}[How corner-exit geometry supplies uniform weights]
In the corner-exit sliver regime, each face slice comes from a uniformly fat simplex facet, and small-slope graph control transfers this
to the realized sheets. The consequence is precisely a uniform comparability statement of the type assumed above.
This paper isolates the bookkeeping and does not assume any particular geometric mechanism for obtaining it.
\end{remark}

\section{Integer optimal transport for atomic measures}

We now address the matching step. Facewise coherence requires pairing sheet traces on $F$.
This is naturally phrased as optimal transport between two atomic measures.

\subsection{Kantorovich formulation with integer masses}

Let $\{x_i\}_{i=1}^I$ and $\{y_j\}_{j=1}^J$ be finite point sets in a metric space (in applications $\R^{d_F}$).
Let $m_i,n_j\in\Z_{\ge 0}$ satisfy $\sum_i m_i=\sum_j n_j$.
Define atomic measures
\[
\mu:=\sum_{i=1}^I m_i \delta_{x_i},\qquad \nu:=\sum_{j=1}^J n_j\delta_{y_j}.
\]
Fix a nonnegative cost matrix $c_{ij}=c(x_i,y_j)$, for example $c_{ij}=\norm{x_i-y_j}$ in Euclidean space.

A \emph{coupling} is a matrix $\pi=(\pi_{ij})$ with $\pi_{ij}\ge 0$ and
\[
\sum_{j=1}^J \pi_{ij}=m_i,\qquad \sum_{i=1}^I \pi_{ij}=n_j.
\]
The Kantorovich cost is $\sum_{i,j} c_{ij}\pi_{ij}$.
An \emph{optimal coupling} minimizes this cost among all couplings.

\begin{theorem}[Integral optimal couplings exist]
If $m_i,n_j\in\Z_{\ge 0}$ then there exists an optimal coupling $\pi^\star$ with \emph{integer} entries $\pi^\star_{ij}\in\Z_{\ge 0}$.
\end{theorem}

\begin{proof}
The feasible set is a polytope defined by the linear constraints above and inequalities $\pi_{ij}\ge 0$.
The objective is linear, hence achieves its minimum at an extreme point of the feasible polytope.

The constraint matrix is the node--arc incidence matrix of a bipartite flow network:
there is a source connected to each $x_i$ with capacity $m_i$, each $x_i$ connects to each $y_j$ with arc flow $\pi_{ij}$,
and each $y_j$ connects to a sink with demand $n_j$.
Incidence matrices of directed graphs are totally unimodular, and this property is preserved under the standard bipartite flow formulation.
With integer right-hand sides $(m_i)$ and $(n_j)$, every extreme point of the feasible polytope is integral.
Therefore there exists an extreme-point minimizer, hence an optimal coupling, with integer entries.
\end{proof}

\begin{corollary}[Unit masses yield a permutation matching]
If all masses are $m_i=n_j=1$ and $I=J=N$, then an integral coupling is exactly the indicator matrix of a permutation.
Equivalently, the optimal transport problem reduces to a minimum-cost matching between $\{x_i\}$ and $\{y_j\}$.
\end{corollary}

\begin{proof}
If each row and column sum equals $1$ and entries are nonnegative integers, each row and column contains exactly one $1$ and the rest $0$.
\end{proof}

\section{Facewise matched pairings and $W_1$ bounds from prefix coherence}

We now specialize transport to the prefix situation, where both sides are induced from the \emph{same} master atoms but through two nearby face maps.

\subsection{$W_1$ for unit-mass atomic measures}
For unit-mass atomic measures with equal total mass $N$,
\[
\mu=\sum_{a=1}^N\delta_{u_a},
\qquad
\nu=\sum_{a=1}^N\delta_{v_a},
\]
define the equal-weight $W_1$ cost by
\[
W_1(\mu,\nu):=\min_{\sigma\in S_N}\sum_{a=1}^N \norm{u_a-v_{\sigma(a)}}.
\]

\subsection{Explicit prefix-induced coupling}
Assume we have balanced the two sides of a face $F$ to a common prefix length $N_F$ (for example $N_F=N_{\min}$),
and define
\[
\mu_{Q\to F}=(\Phi_{Q,F})_\#\nu^{(N_F)},\qquad
\mu_{Q'\to F}=(\Phi_{Q',F})_\#\nu^{(N_F)}.
\]
Then there is a canonical \emph{integral} coupling obtained by matching each template index with itself:
\[
\pi_F:=\sum_{a=1}^{N_F}\delta_{\bigl(\Phi_{Q,F}(y_a),\,\Phi_{Q',F}(y_a)\bigr)}.
\]
This plan is integral and has marginals $\mu_{Q\to F}$ and $\mu_{Q'\to F}$.

\begin{proposition}[Template-index coupling gives an $O(h^2 N_F)$ $W_1$ bound]
Assume the uniform face-map control inequalities hold on $F=Q\cap Q'$.
Then
\[
W_1(\mu_{Q\to F},\mu_{Q'\to F})\ \le\ C_\Phi\,C_0\,\varrho\,h^2\,N_F.
\]
\end{proposition}

\begin{proof}
By definition of $W_1$, any coupling gives an upper bound. Using the explicit coupling $\pi_F$,
\[
W_1(\mu_{Q\to F},\mu_{Q'\to F})
\le \sum_{a=1}^{N_F}\norm{\Phi_{Q,F}(y_a)-\Phi_{Q',F}(y_a)}.
\]
By the Lipschitz bound on the difference and the template radius,
\[
\norm{\Phi_{Q,F}(y_a)-\Phi_{Q',F}(y_a)}\le \norm{\Phi_{Q,F}-\Phi_{Q',F}}_{\mathrm{Lip}}\,\norm{y_a}
\le (C_\Phi h)\,(C_0\varrho h)=C_\Phi C_0\varrho h^2.
\]
Summing over $a=1,\dots,N_F$ yields the stated bound.
\end{proof}

\begin{remark}[Integral optimal plans versus explicit plans]
The theorem on integral optimal couplings says an optimal plan can be chosen integral, hence realized as a genuine pairing.
The explicit index-matching plan above need not be optimal, but it is often sufficient because its cost is already at the correct asymptotic scale $O(h^2 N_F)$
under the face-map control hypotheses.
\end{remark}

\section{Simultaneous coherence across direction labels}

In applications one has finitely many direction labels (for example a finite calibrated-direction dictionary).
For each label $i\in\{1,\dots,M\}$ we fix:
\begin{itemize}
\item a master template $\mathbf{y}^{(i)}=(y^{(i)}_a)_{a\ge 1}\subset B_{C_0\varrho h}(0)\subset\R^{d_\perp}$,
\item integer prefix lengths $N_{Q,i}$ per cube $Q$,
\item and face maps $\Phi^{(i)}_{Q,F}$ on each face $F$ (these may depend on $i$ through the chosen template geometry).
\end{itemize}

\begin{theorem}[Global multi-label prefix coherence package]
Assume that for each label $i$ the neighbor slow-variation bound
\[
|N_{Q,i}-N_{Q',i}|\le \theta_i\, h\, \min\{N_{Q,i},N_{Q',i}\}
\]
holds on every neighbor pair $Q\sim Q'$.
Fix a face $F=Q\cap Q'$ and set $N_{F,i}:=\min\{N_{Q,i},N_{Q',i}\}$.
Then, for each label $i$:
\begin{enumerate}
\item (Matched part) the two sides share a common activated prefix of length $N_{F,i}$, and the mismatch is confined to a tail of size $|N_{Q,i}-N_{Q',i}|$.
\item (Face edits) if the face weights are uniformly comparable for label $i$ on $F$, then the unmatched tail weight is an $O(h)$ fraction of the total
face boundary weight (an $O(h)$ face-edit regime).
\item (Integer transport) if the face-map control inequalities hold for label $i$ on $F$, then the induced matched face measures satisfy
\[
W_1(\mu_{Q\to F,i},\mu_{Q'\to F,i})\ \le\ C_\Phi\,C_0\,\varrho\,h^2\,N_{F,i},
\]
and there exists an integral coupling between them.
\end{enumerate}
All constants are uniform in $Q,Q'$ and depend only on the fixed template/facemapping controls and the chosen slow-variation parameters $\theta_i$.
\end{theorem}

\begin{proof}
Items (1) and (2) are immediate from the prefix mismatch decomposition and the uniform-weight face-edit proposition applied label-by-label.
Item (3) is the template-index coupling bound combined with the general existence of integral optimal couplings.
\end{proof}

\begin{remark}[Where the integer counts $N_{Q,i}$ come from]
A common source is a stable dictionary decomposition of a cone-valued form field:
a real mass budget per label is computed in each cell and then rounded to an integer prefix length.
The rounding lemma above shows that Lipschitz budgets produce slow-variation integer counts in the ``many pieces'' regime.
This paper only needs the resulting slow-variation property, not the origin of the budgets.
\end{remark}

\section{What this module outputs (for downstream gluing)}

The prefix-template bookkeeping provides, for each interior face $F=Q\cap Q'$ and each direction label $i$, the following data:
\begin{itemize}
\item a \emph{matched} prefix index set $\{1,\dots,N_{F,i}\}$ shared by both sides,
\item an \emph{unmatched tail} index set of size $|N_{Q,i}-N_{Q',i}|$ (the only place edits can occur),
\item and an \emph{integral transport plan} (often an explicit index-matching plan) between the matched face measures
with $W_1$ cost bounded by $O(\varrho h^2 N_{F,i})$ under uniform face-map control.
\end{itemize}
This is exactly the type of input needed by weighted flat-norm gluing estimates: ``matched'' indices are transported with small displacement cost,
and ``unmatched'' indices are treated as controlled insertions/deletions in an $O(h)$ edit regime.

\section*{Conclusion}

A single global ordering of transverse template atoms, combined with per-cell prefix activation, makes interface mismatch combinatorially explicit:
it is always a tail. If prefix lengths vary slowly across neighbors, the tail is an $O(h)$ fraction. The resulting atomic measures on faces admit
integral optimal transport plans, so facewise matchings can be realized as honest pairings. This bookkeeping layer replaces unstable global assignment problems
with a deterministic, local, and quantitatively controlled scheme designed to plug into geometric-measure gluing estimates.

\end{document}