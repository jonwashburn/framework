\documentclass[12pt]{article}

% Packages (kept minimal for portability)
\usepackage[margin=1in]{geometry}
\usepackage{amsmath,amssymb,amsthm}
\usepackage{mathtools}
\usepackage{hyperref}
\usepackage{microtype}
\usepackage{booktabs}

\hypersetup{
  colorlinks=true,
  linkcolor=blue!70!black,
  citecolor=blue!70!black,
  urlcolor=blue!70!black
}

% Theorem environments
\theoremstyle{plain}
\newtheorem{theorem}{Theorem}[section]
\newtheorem{lemma}[theorem]{Lemma}
\theoremstyle{definition}
\newtheorem{definition}[theorem]{Definition}
\theoremstyle{remark}
\newtheorem{remark}[theorem]{Remark}

% Notation
\newcommand{\R}{\mathbb{R}}
\newcommand{\Rp}{\R_{>0}}
\newcommand{\phiG}{\varphi}
\newcommand{\J}{J}
\newcommand{\W}{\mathcal{W}}
\newcommand{\C}{\mathcal{C}}
\newcommand{\Lat}{\mathcal{L}}

\title{\textbf{Quantized Semantics}\\[0.25em]
\large The Golden Ratio as the Discrete Ladder of Reference}

\author{Jonathan Washburn}
\date{\today}

\begin{document}
\maketitle

\begin{abstract}
Why is meaning discrete (words, concepts, semantic atoms) rather than continuous?
And why do there exist specific ``modes'' of meaning---in particular, a finite periodic table
of 20 canonical WTokens---instead of an arbitrary continuum of symbols?
This paper proposes a cost-geometric answer: the semantic cost manifold is not flat.
Within Recognition Science, reference is governed by the universal cost functional
\(\J(x)=\tfrac12(x+x^{-1})-1\) on \(\Rp\), and ledger self-similarity forces a discrete
quantization grid at \(\phiG^0,\phiG^1,\phiG^2,\phiG^3\).
Continuous inputs are mapped to this grid by geodesic projection, which snaps to the
cost-minimizing rung.
The resulting ``ladder'' explains both the discreteness of meaning and the necessity of the
WToken periodic table: stable semantic atoms must live at lattice minima.
\end{abstract}

\section{The problem: why discrete meaning? why \(\phiG\)?}
If meaning were purely conventional, we might expect arbitrary semantic primitives.
Instead, natural languages and effective symbolic systems exhibit strong discreteness
(phonemes, words, concepts) and striking regularities.
Recognition Science treats this as a physical phenomenon: semantic stability is determined by
cost minimization.

The question ``why \(\phiG\)?'' is then sharpened: why does the stable semantic ladder align with
powers of the golden ratio rather than an arbitrary base?

\section{The universal cost functional}
\begin{definition}[Recognition cost]
Let \(x\in\Rp\). Define
\begin{equation}
  \J(x) \;:=\; \frac{1}{2}\bigl(x + x^{-1}\bigr) - 1.
\end{equation}
\end{definition}

This functional is nonnegative, symmetric under inversion, and has a unique zero-point at
\(x=1\). It is the unique cost compatible with the Recognition Composition Law (RCL) and the
ledger invariants (see the Recognition Science derivation chain).

\section{Self-similarity forces \(\phiG\)}
The Recognition Composition Law expresses a self-similar constraint on how costs compose under
multiplicative transformations.
In brief: a self-referential ledger must admit scale transformations that preserve the form of
cost interactions. The unique nontrivial fixed point of this self-similarity is the golden ratio.

\begin{remark}
This paper does not re-derive the full forcing chain; it uses the established RS result that
\(\phiG\) is forced by ledger self-similarity.
\end{remark}

\section{The \(\phiG\)-lattice: a discrete ladder of stable semantic amplitudes}
\begin{definition}[\(\phiG\)-lattice (levels 0--3)]
Define the discrete ladder of stable reference rungs
\begin{equation}
  \Lat := \{\phiG^0, \phiG^1, \phiG^2, \phiG^3\} \subset \Rp.
\end{equation}
\end{definition}

In the current formalization, these are exactly the stable target points used for WToken
projection:
\begin{equation}
  \texttt{wtokenLattice} = [1,\phiG,\phiG^2,\phiG^3].
\end{equation}

\subsection{Higher rungs cost more}
Define the intrinsic cost of rung \(k\) by \(\J(\phiG^k)\).
The key monotonicity fact is that higher rungs correspond to higher intrinsic cost:
\begin{equation}
  k < \ell \implies \J(\phiG^k) < \J(\phiG^\ell).
\end{equation}

This is proved in the codebase as \texttt{phiLevelCost\_mono}.
Intuitively, since \(\phiG>1\) we have \(\phiG^k < \phiG^\ell\), and \(\J\) is strictly increasing on
\((1,\infty)\).

\section{Geodesic projection: why continuous inputs snap to discrete levels}
A continuous input (a configuration ratio \(r\in\Rp\)) is not ``classified'' into a WToken.
Rather, it is \emph{projected} onto the lattice by minimizing reference mismatch cost.

Let the mismatch cost to rung \(k\) be
\begin{equation}
  \mathrm{Ref}(r,k) := \J\bigl(r/\phiG^k\bigr).
\end{equation}

\begin{definition}[Geodesic projection to the ladder]
Define
\begin{equation}
  \Pi(r) := \arg\min_{k\in\{0,1,2,3\}} \J\bigl(r/\phiG^k\bigr).
\end{equation}
\end{definition}

In the formalization, \(\Pi\) is implemented as \texttt{projectOntoPhiLattice}, and its optimality is
proved as \texttt{projection\_minimizes\_reference}.

\begin{remark}
Computationally, this minimization is equivalent to log-rounding:
\(k \approx \mathrm{round}(\log_{\phiG}(r))\), producing the observed ``snap'' of continuous inputs
into discrete semantic atoms.
\end{remark}

\section{Why 20 WTokens? the periodic table of meaning}
WTokens are the canonical semantic atoms of Recognition Science.
They combine:
\begin{itemize}
  \item a \emph{mode family} (Fourier/DFT structure of 8-tick meaning),
  \item a \emph{\(\phiG\)-level} (one of the four stable rungs),
  \item and a \emph{\(\tau\)} offset where applicable.
\end{itemize}

This yields a finite periodic table: 20 canonical WToken identities.
The discreteness is not imposed by fiat; it is the necessary outcome of minimizing reference
cost in a self-referential ledger.

\section{Conclusion}
Meaning is discrete because stable semantics must sit at local minima of the cost manifold.
Ledger self-similarity forces \(\phiG\), producing a natural quantization ladder
\(\phiG^0,\phiG^1,\phiG^2,\phiG^3\).
Geodesic projection maps continuous inputs to the nearest cost-minimizing rung, explaining why
semantic atoms cannot stably exist ``between'' levels.
Therefore the periodic table of meaning (the 20 WTokens) is not an invention; it is a necessary
discrete solution to the problem of minimizing cost in a self-referential universe.

\section*{Formal verification anchors}
\begin{itemize}
  \item \texttt{IndisputableMonolith/Foundation/WTokenReference.lean}
    \begin{itemize}
      \item \texttt{wtokenLattice}
      \item \texttt{projectOntoPhiLattice}
      \item \texttt{projection\_minimizes\_reference}
    \end{itemize}
  \item \texttt{IndisputableMonolith/LightLanguage/WTokenReferenceBridge.lean}
    \begin{itemize}
      \item \texttt{phiLevelCost\_mono}
    \end{itemize}
\end{itemize}

\begin{thebibliography}{9}
\bibitem{wigner}
E. P. Wigner,
\emph{The Unreasonable Effectiveness of Mathematics in the Natural Sciences},
Communications in Pure and Applied Mathematics, 13(1):1--14, 1960.
\end{thebibliography}

\end{document}
