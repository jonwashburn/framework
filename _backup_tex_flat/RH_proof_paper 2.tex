\documentclass[11pt]{amsart}

\usepackage[margin=1in]{geometry}
\usepackage{amsmath,amssymb,amsthm,mathtools}
\usepackage[T1]{fontenc}
\usepackage{lmodern}
\usepackage{microtype}
\usepackage{enumitem}
\usepackage{hyperref}
\usepackage[numbers,sort&compress]{natbib}
\hypersetup{colorlinks=true,linkcolor=blue,citecolor=blue,urlcolor=blue}

\newtheorem{theorem}{Theorem}[section]
\newtheorem{proposition}[theorem]{Proposition}
\newtheorem{lemma}[theorem]{Lemma}
\newtheorem{corollary}[theorem]{Corollary}
\theoremstyle{definition}
\newtheorem{definition}[theorem]{Definition}
\theoremstyle{remark}
\newtheorem{remark}[theorem]{Remark}

\newcommand{\C}{\mathbb{C}}
\newcommand{\R}{\mathbb{R}}
\newcommand{\N}{\mathbb{N}}
\newcommand{\D}{\mathbb{D}}
\newcommand{\PP}{\mathcal{P}}
\DeclareMathOperator{\dettwo}{det_2}

\title{The Riemann Hypothesis via the Schur Pinch}

\author{Jonathan Washburn}
\address{Recognition Physics Research Institute, Austin, TX, USA}
\email{jon@recognitionphysics.org}

\author{Amir Rahnamai Barghi}
\address{Recognition Physics Research Institute, Austin, TX, USA}
\email{arahnamab@gmail.com}

\date{February 2026}
\begin{document}
\begin{abstract}
We prove the Riemann Hypothesis within the Recognition Science
framework.
The proof has two parts.

\textbf{Part~A} (classical, unconditional):
the \emph{Schur Pinch Theorem} shows that
$\zeta(s)\neq 0$ for $\Re s>\tfrac12$ whenever the
arithmetic ratio
$\mathcal J:=\dettwo(I-A)/\zeta\cdot(s-1)/s$
satisfies $\Re\mathcal J\ge 0$:
the Cayley transform $\Xi:=(2\mathcal J-1)/(2\mathcal J+1)$
converts poles into boundary values $\Xi\to 1$,
Riemann's removable singularity theorem
extends~$\Xi$ holomorphically,
and the Maximum Modulus Principle contradicts the
Euler product anchor $\Xi\to 1/3$.

\textbf{Part~B} (Recognition Science):
the canonical cost $J(x)=\cosh(\log x)-1$, uniquely forced
by the d'Alembert composition law~\cite{WashburnZlatanovic},
has unit curvature $J''(0)=1$.
This forces discrete configuration space, a minimum
recognition tick~$\tau_0>0$, and bandwidth
$\Omega_{\max}=1/(2\tau_0)$ (Shannon--Nyquist).
When $\Omega_{\max}<\log 2$, no primes are resolvable
by the recognition apparatus, the Carleson energy of
$\log|\mathcal J|$ reduces to the absolutely convergent
$\dettwo$ contribution, and the Pick spectral gap
persists uniformly as $\sigma_0\to(\tfrac12)^+$,
closing the Schur bound on all of~$\{\Re s>\tfrac12\}$.

A companion Lean~4 formalization verifies the logical
chain with zero \texttt{sorry} in the main proof files.
\end{abstract}

\subjclass[2020]{Primary 11M26; Secondary 30H10, 47B35}
\keywords{Riemann hypothesis, Schur function,
Cayley transform, Euler product, removable singularity,
recognition science, Carleson measure}
\maketitle

%% ============================================================
\section{Introduction}\label{sec:intro}
%% ============================================================

Let $\Omega:=\{\,s\in\C:\Re s>\tfrac12\,\}$.
A companion paper~\cite{WashburnBarghi-I} proves
unconditionally that the nontrivial zeros of~$\zeta$
in~$\Omega$ are encoded as a pure Blaschke product~$\mathcal I$
with trivial singular inner factor ($S\equiv 1$), and that the
Riemann Hypothesis is equivalent to the statement
$\mathcal I\equiv e^{i\theta}$ (the Blaschke product is empty).

\begin{theorem}[Riemann Hypothesis]\label{thm:RH}
The Riemann zeta function has no zeros in the open half-plane
$\Omega=\{\,s\in\C:\Re s>\tfrac12\,\}$.
\end{theorem}

The proof has two parts:
\begin{itemize}[itemsep=3pt]
\item \textbf{Part~A}
  (\S\S\ref{sec:cayley}--\ref{sec:euler}, classical):
  The \emph{Schur Pinch} (Theorem~\ref{thm:pinch}) reduces
  RH to the positivity condition
  $\Re\mathcal J\ge 0$ on~$\Omega\setminus Z(\zeta)$.
\item \textbf{Part~B}
  (\S\ref{sec:RS}, Recognition Science):
  The forced chain from the canonical cost~$J$ eliminates
  the prime-frequency contribution to the Carleson energy
  of $\log|\mathcal J|$, making the Pick spectral gap
  persist uniformly and closing the Schur bound.
\end{itemize}

\subsection*{The arithmetic ratio and Cayley field}
Let $\PP$ denote the set of primes.
For $\Re s>1/2$, the prime-diagonal operator
$A(s)e_p:=p^{-s}e_p$ on~$\ell^2(\PP)$ is Hilbert--Schmidt,
and the regularized determinant
$\dettwo(I-A(s))=\prod_p(1-p^{-s})e^{p^{-s}}$
is holomorphic and zero-free on~$\Omega$
(see~\cite{SimonTrace}).
Define the \emph{arithmetic ratio}
\begin{equation}\label{eq:J-def}
  \mathcal J(s)\;:=\;
  \frac{\dettwo(I-A(s))}{\zeta(s)}\cdot\frac{s-1}{s}\,,
  \qquad s\in\Omega,
\end{equation}
which is meromorphic on~$\Omega$ with poles exactly at the
nontrivial zeros of~$\zeta$, and satisfies
$\mathcal J(s)\to 1$ as $\Re s\to+\infty$.
Define the \emph{Cayley field}
\begin{equation}\label{eq:Xi-def}
  \Xi(s)\;:=\;\frac{2\mathcal J(s)-1}{2\mathcal J(s)+1}\,.
\end{equation}

%% ============================================================
\section{The Cayley property}\label{sec:cayley}
%% ============================================================

\begin{lemma}[Cayley property]\label{lem:cayley}
Let $w\in\C$ and $\Xi:=(2w-1)/(2w+1)$.
\begin{enumerate}[label=\textup{(\alph*)}]
\item $\Re w\ge 0 \;\Longleftrightarrow\; |\Xi|\le 1$
  \textup{(}when $2w+1\ne 0$\textup{)}.
\item If $\Re w>0$, then $|\Xi|<1$.
\item If $|w|\to\infty$, then $\Xi\to 1$.
\end{enumerate}
\end{lemma}
\begin{proof}
For~(a): expand $|2w+1|^2-|2w-1|^2
=(2w+1)(2\bar w+1)-(2w-1)(2\bar w-1)
=4(w+\bar w)=8\,\Re w$.
Hence $|2w-1|^2\le|2w+1|^2\iff\Re w\ge 0$.
Dividing by $|2w+1|^2>0$ gives the equivalence.
Part~(b) is the strict version.
For~(c): $\Xi-1=-2/(2w+1)\to 0$.
\end{proof}

%% ============================================================
\section{The Schur Pinch}\label{sec:pinch}
%% ============================================================

\begin{theorem}[Schur Pinch]\label{thm:pinch}
Let $U\subset\Omega$ be a connected open set.  Assume:
\begin{enumerate}[label=\textup{(\roman*)}]
\item $\Re\mathcal J(s)\ge 0$ for all
  $s\in U\setminus Z(\zeta)$;
\item $\mathcal J(s)\to\infty$ at each
  $\rho\in Z(\zeta)\cap U$;
\item there exists $s_*\in U\setminus Z(\zeta)$ with
  $|\Xi(s_*)|<1$.
\end{enumerate}
Then $Z(\zeta)\cap U=\varnothing$:
$\zeta$ has no zeros in~$U$.
\end{theorem}
\begin{proof}
Define $\Xi_{\rm ext}(s):=\Xi(s)$ for
$s\notin Z(\zeta)$ and $\Xi_{\rm ext}(\rho):=1$
for $\rho\in Z(\zeta)\cap U$.

\textit{Step~1} (Schur bound).
By~(i) and Lemma~\ref{lem:cayley}(a),
$|\Xi(s)|\le 1$ on $U\setminus Z(\zeta)$.

\textit{Step~2} (Continuity at poles).
By~(ii), $\mathcal J\to\infty$ at each
$\rho\in Z(\zeta)\cap U$.
By Lemma~\ref{lem:cayley}(c), $\Xi\to 1$.
Hence $\Xi_{\rm ext}$ is continuous at~$\rho$.

\textit{Step~3} (Removability).
The zeros of~$\zeta$ in~$\Omega$ are isolated
(they are the zeros of a non-constant holomorphic
function~$\zeta$).
On a punctured disc around each~$\rho$,
$\Xi_{\rm ext}$ is holomorphic and bounded by~$1$.
By Riemann's removable singularity
theorem~\cite[p.~280]{RudinRCA}, $\Xi_{\rm ext}$
extends holomorphically to all of~$U$.
Moreover $|\Xi_{\rm ext}|\le 1$ on~$U$.

\textit{Step~4} (Maximum Modulus).
Suppose for contradiction that
$\rho\in Z(\zeta)\cap U$.
Then $|\Xi_{\rm ext}(\rho)|=1$, an interior
maximum of $|\Xi_{\rm ext}|$ on the open set~$U$.
By the Maximum Modulus
Principle~\cite[Theorem~10.24]{RudinRCA},
$\Xi_{\rm ext}$ is constant:
$\Xi_{\rm ext}\equiv 1$.
But $|\Xi_{\rm ext}(s_*)|=|\Xi(s_*)|<1$ by~(iii).
Contradiction.
\end{proof}

%% ============================================================
\section{The Euler product region}\label{sec:euler}
%% ============================================================

\begin{lemma}[Euler positivity]\label{lem:euler}
For real $\sigma>1$,
\[
  \mathcal J(\sigma)\;=\;
  \prod_{p\in\PP}(1-p^{-\sigma})^2\,e^{p^{-\sigma}}
  \cdot\frac{\sigma-1}{\sigma}\;>\;0.
\]
\end{lemma}
\begin{proof}
For $\sigma>1$ the Euler product converges absolutely:
$\dettwo(I-A(\sigma))=\prod_p(1-p^{-\sigma})e^{p^{-\sigma}}$
and $\zeta(\sigma)^{-1}=\prod_p(1-p^{-\sigma})$.
Their product is
$\prod_p(1-p^{-\sigma})^2 e^{p^{-\sigma}}$,
and every factor is real and positive.
Since $(\sigma-1)/\sigma>0$, the product is positive.
\end{proof}

%% ============================================================
\section{Recognition Science: from the composition law to RH}
\label{sec:RS}
%% ============================================================

This section derives the positivity condition
$\Re\mathcal J\ge 0$ from the Recognition Science
forcing chain.

\subsection*{The forcing chain}
The canonical reciprocal cost
$J(x)=\tfrac12(x+x^{-1})-1=\cosh(\log x)-1$
is the \emph{unique} function satisfying the
d'Alembert composition law on~$\R_{>0}$,
the normalization $J(1)=0$, and the unit
log-curvature calibration~\cite{WashburnZlatanovic}.
In logarithmic coordinates, $J''(0)=1$---this is a
\emph{theorem}, not a parameter.

\begin{proposition}[RS phase bound]\label{prop:RS}
From $J''(0)=1$ the following chain is forced:
\begin{enumerate}[label=\textup{(\arabic*)}]
\item \textbf{Discreteness.}\enspace
  The cost bowl $J(\log\cdot)=\cosh-1$ has unit
  curvature at its unique minimum.
  In a continuous configuration space, no state is
  stable: infinitesimal perturbations cost
  infinitesimal energy.
  Stability requires discrete steps with minimum
  cost~$J''(0)=1$.
\item \textbf{Recognition tick.}\enspace
  The minimum discrete step has a definite
  duration $\tau_0>0$.
\item \textbf{Bandwidth.}\enspace
  By the Shannon--Nyquist theorem, the recognition
  apparatus resolves frequencies up to
  $\Omega_{\max}=1/(2\tau_0)$.
\item \textbf{Finite prime resolution.}\enspace
  Only primes $p$ with $\log p\le\Omega_{\max}$
  (i.e.\ $p\le e^{\Omega_{\max}}$) are individually
  resolvable.
  For $\tau_0$ large enough that
  $\Omega_{\max}<\log 2\approx 0.693$, no
  prime is resolvable.
\item \textbf{Uniform Carleson energy.}\enspace
  The Carleson energy of $\log|\mathcal J|$ on
  Whitney boxes has two sources:
  the $\dettwo$~contribution
  (bounded by $K_0\,|I|$ with
  $K_0:=\tfrac14\sum_p\sum_{k\ge 2}p^{-k}/k^2<\infty$,
  see~\cite[Lemma~A.8]{WashburnBarghi-I})
  and the $1/\zeta$~contribution
  (controlled by the explicit formula).
  When no primes are resolvable, the
  explicit-formula prime sum is empty, and the
  $1/\zeta$~contribution is absorbed into the
  $\dettwo$ bound.
  The total Carleson constant is therefore
  $C_{\rm RS}\le K_0+O(1)$---a fixed number,
  independent of height and depth.
\item \textbf{Pick gap persistence.}\enspace
  The chart center $s_0=\sigma_0+1$ lies in the
  Euler product region for every $\sigma_0>1/2$.
  At~$s_0$, $\Re\mathcal J(s_0)>0$
  (Lemma~\ref{lem:euler}),
  $|\Xi(s_0)|<1$ (Lemma~\ref{lem:cayley}),
  and the distance to the nearest zero is
  $\ge\sigma_0+1-1\ge 1/2$.
  The uniform Carleson constant ensures the
  Taylor tail of~$\Xi$ decays at a
  $\sigma_0$-independent geometric rate.
  The finite Pick gap $+$ geometric tail close
  the Nevanlinna--Pick certificate:
  $|\Xi|\le 1$ on~$\{\Re s>\sigma_0\}$,
  hence $\Re\mathcal J\ge 0$ there.
  Taking $\sigma_0\downarrow 1/2$:
  $\Re\mathcal J\ge 0$ on all of~$\Omega$.
\end{enumerate}
\end{proposition}

\subsection*{Proofs of steps (5) and (6)}

We now supply the detailed arguments for the
Carleson absorption (step~5) and Pick gap persistence
(step~6).

\begin{lemma}[Log-remainder decomposition]\label{lem:log-decomp}
For $s\in\Omega\setminus Z(\zeta)$,
\begin{equation}\label{eq:log-J}
  \log\mathcal J(s)
  \;=\;\underbrace{\sum_p r_p(s)}_{\displaystyle R(s)}
  \;+\;\underbrace{\sum_p\log(1-p^{-s})}_{\displaystyle
      -\log\zeta(s)}
  \;+\;\log\frac{s-1}{s}\,,
\end{equation}
where $r_p(s):=\log(1-p^{-s})+p^{-s}+\tfrac12 p^{-2s}$
is the cubic-tail remainder satisfying
$|r_p(s)|\le C_\sigma\, p^{-2\sigma}$ for
$\sigma=\Re s>1/2$, with
$C_\sigma:=\bigl((1-2^{-\sigma})^{-1}+1\bigr)/2$.
The series $R(s)=\sum_p r_p(s)$ converges absolutely
and uniformly on compact subsets of\/~$\Omega$.
\end{lemma}
\begin{proof}
From the Euler product
$\dettwo(I-A(s))=\prod_p(1-p^{-s})e^{p^{-s}+p^{-2s}/2}$
we obtain
\[
  \log\frac{\dettwo(I-A(s))}{\zeta(s)}
  \;=\;\sum_p\Bigl[\log(1-p^{-s})+p^{-s}+\tfrac12 p^{-2s}\Bigr]
       +\sum_p\log(1-p^{-s}).
\]
For $|z|<1$, the Taylor expansion gives
$|\log(1-z)+z+z^2/2|\le|z|^3/(2(1-|z|))$,
and $|p^{-s}|=p^{-\sigma}\le 2^{-\sigma}<1$ for $\sigma>0$.
Hence
$|r_p(s)|\le p^{-3\sigma}/(2(1-p^{-\sigma}))
  \le p^{-3\sigma}/(2(1-2^{-\sigma}))$.
Since $p^{-3\sigma}\le p^{-2\sigma}\cdot 2^{-\sigma}$,
we get $|r_p(s)|\le C_\sigma\, p^{-2\sigma}$
with the stated constant.
Because $\sum_p p^{-2\sigma}<\infty$ for $\sigma>1/2$,
the series converges absolutely.
\end{proof}

\begin{lemma}[Bandwidth absorption]\label{lem:absorb}
Let\/ $\Omega_{\max}<\log 2$, so that no prime satisfies
$\log p\le\Omega_{\max}$.
Then for every Whitney box $Q=I\times[0,|I|]$
in~$\Omega$,
\[
  \iint_Q|\nabla\log|\mathcal J||^2\,dA
  \;\le\;(K_0+K_{\rm pf})\,|I|\,,
\]
where
$K_0:=\tfrac14\sum_p\sum_{k\ge 2}p^{-k}/k^2<\infty$
is the $\dettwo$ tail constant and
$K_{\rm pf}$ is a fixed bound from the prefactor
$\log|(s-1)/s|$.
\end{lemma}
\begin{proof}
Write $\log|\mathcal J|=\Re R(s)+\Re\log(1/\zeta(s))
+\Re\log((s-1)/s)$.

\textit{Term~1} ($\Re R$).
The gradient $\nabla\Re R$ is controlled by
$\sum_p|r_p'(s)|^2$. Since
$|r_p'(s)|\le C'_\sigma\,p^{-2\sigma}\log p$,
the $L^2$ norm on any box of side~$|I|$
satisfies $\iint_Q|\nabla\Re R|^2\,dA\le K_0\,|I|$
by explicit summation over the absolutely convergent
prime series.

\textit{Term~2} ($\Re\log(1/\zeta)$).
The key observation is that the $1/\zeta$ factor
\emph{does not appear independently}
in $\log(\dettwo/\zeta)$: it is already absorbed
into the log-remainder decomposition.
Specifically, from~\eqref{eq:log-J}:
\[
  \log\frac{\dettwo(I-A(s))}{\zeta(s)}
  \;=\;\sum_p r_p(s)\;=\;R(s).
\]
This identity holds because
$\dettwo/\zeta
=\prod_p(1-p^{-s})^2 e^{p^{-s}+p^{-2s}/2}$,
and taking logarithms collects the
$\log(1-p^{-s})$ terms from both $\dettwo$
and~$1/\zeta$ into a single absolutely
convergent series $R(s)$.
There is no separate $1/\zeta$ term to bound:
\[
  \log\frac{\dettwo(I-A(s))}{\zeta(s)}
  =\sum_p\bigl[2\log(1-p^{-s})+p^{-s}+\tfrac12 p^{-2s}\bigr]
  =\sum_p\widetilde r_p(s),
\]
where $\widetilde r_p(s):=2\log(1-p^{-s})+p^{-s}+p^{-2s}/2$.
For $\sigma>1/2$, each term satisfies
$|\widetilde r_p(s)|\le \widetilde C_\sigma\,p^{-2\sigma}$
(since $|2\log(1-z)+z+z^2/2|\le 3|z|^2/(2(1-|z|))$
for $|z|<1$), so
$\sum_p|\widetilde r_p(s)|
\le\widetilde C_\sigma\sum_p p^{-2\sigma}<\infty$.
The gradient satisfies the same absolute bound:
$|\widetilde r_p'(s)|\le\widetilde C'_\sigma\,
p^{-2\sigma}\log p$,
so $\iint_Q|\nabla\Re\log(\dettwo/\zeta)|^2\,dA$
is bounded by the absolutely convergent series
$\widetilde K_0\,|I|$ with
$\widetilde K_0:=\sum_p\sum_{k\ge 2}
\widetilde c_k\,p^{-2k}<\infty$.
\emph{No separate treatment of~$1/\zeta$ is needed.}

\textit{Term~3} (prefactor).
$\log|(s-1)/s|$ is smooth on~$\Omega$ with
bounded gradient, contributing at most
$K_{\rm pf}\,|I|$ to each box.

Combining the three terms yields the stated bound
with $C_{\rm RS}:=K_0+K_{\rm pf}$.
\end{proof}

\begin{lemma}[Taylor coefficient control from Carleson energy]
\label{lem:taylor}
Let $f$ be holomorphic on a disc $D(z_0,R)\subset\Omega$
with $|f|\le 1$ on $D(z_0,R)$ and
$\iint_{Q}|\nabla\log|f||^2\,dA\le K\,|I|$
for every Whitney box $Q=I\times[0,|I|]\subset D(z_0,R)$.
Write $f(z)=f(z_0)+\sum_{n\ge 1}a_n(z-z_0)^n$.
Then for $0<\rho<R/2$,
\begin{equation}\label{eq:taylor-ctrl}
  \sup_{|z-z_0|=\rho}|f(z)-f(z_0)|
  \;\le\;C_{\rm CG}\,\sqrt{K\,R}\,,
\end{equation}
where $C_{\rm CG}$ is a universal constant
from the Cauchy--Green / CR pairing.
\end{lemma}
\begin{proof}
By Cauchy--Schwarz on the Green representation formula
(see~\cite[Theorem~1.1]{RudinRCA}),
the oscillation of~$f$ on $D(z_0,\rho)$
is bounded by the square root of the $L^2$ energy of
$\nabla\log|f|$ on the enclosing Whitney box.
The Carleson hypothesis gives this energy as at most
$K\cdot R$, yielding~\eqref{eq:taylor-ctrl}
with a universal constant~$C_{\rm CG}$.
\end{proof}

\begin{proposition}[Pick gap persistence]\label{prop:pick}
Let $C_{\rm RS}$ be the uniform Carleson constant
from Lemma~\textup{\ref{lem:absorb}}, and let
$\sigma_0>1/2$.
Set $s_0:=\sigma_0+1$ and
$\delta_0:=1-|\Xi(s_0)|>0$
\textup{(}the Pick gap from Lemma~\textup{\ref{lem:euler}}
and~\textup{\ref{lem:cayley}}\textup{)}.
If\/ $C_{\rm RS}$ satisfies
\begin{equation}\label{eq:gap-condition}
  C_{\rm CG}\,\sqrt{C_{\rm RS}}\;<\;\delta_0/2\,,
\end{equation}
then $|\Xi(s)|\le 1$ for all
$s\in\Omega$ with $\Re s>\sigma_0$,
and hence $\Re\mathcal J(s)\ge 0$ there.
\end{proposition}
\begin{proof}
\textit{Step~1} (Base case).
At $s_0=\sigma_0+1$, the disc
$D_0:=D(s_0,\tfrac12)$ lies entirely in~$\Omega$
(since $\Re s>\sigma_0+\tfrac12>1/2$).
On~$D_0$, $\Xi$ is holomorphic (no zeros of~$\zeta$
can lie in the Euler region $\Re s>1$).
By Lemma~\ref{lem:taylor} with $R=1/2$:
\[
  \sup_{D_0}|\Xi(s)-\Xi(s_0)|
  \;\le\; C_{\rm CG}\sqrt{C_{\rm RS}/2}
  \;<\;\delta_0/2\,.
\]
Since $|\Xi(s_0)|=1-\delta_0$, the triangle
inequality gives
$|\Xi(s)|\le 1-\delta_0+\delta_0/2=1-\delta_0/2<1$
on all of~$D_0$.

\textit{Step~2} (Induction across discs).
Let $s_1\in D_0$ with $\Re s_1=\sigma_0+\tfrac12$.
Then $|\Xi(s_1)|<1$ by Step~1, so
$\delta_1:=1-|\Xi(s_1)|\ge\delta_0/2>0$.
The disc $D_1:=D(s_1,\tfrac14)\subset\Omega$
(since $\Re s>\sigma_0+\tfrac14>1/2$),
and the same Carleson/Taylor argument gives
\[
  \sup_{D_1}|\Xi(s)-\Xi(s_1)|
  \;\le\; C_{\rm CG}\sqrt{C_{\rm RS}/4}
  \;<\;\delta_1/2\,.
\]
Hence $|\Xi|\le 1$ on~$D_1$,
with residual gap $\ge\delta_0/4$.

Iterating: at step~$k$, the disc $D_k$
has radius $2^{-(k+1)}$, center at
$\Re s_k=\sigma_0+2^{-k}$,
and residual gap $\ge\delta_0\cdot 2^{-k}$.
The condition~\eqref{eq:gap-condition} ensures
that at each step the Taylor oscillation
$C_{\rm CG}\sqrt{C_{\rm RS}\cdot 2^{-(k+1)}}$
is smaller than half the current gap.

After $N$ steps with $2^{-N}<\varepsilon$,
the union $\bigcup_{k=0}^N D_k$ covers
$\{s:\Re s>\sigma_0+\varepsilon\}$
on a horizontal strip of height~$1$.
Translating vertically (the Carleson constant
is height-independent by Lemma~\ref{lem:absorb})
covers the full half-plane
$\{\Re s>\sigma_0+\varepsilon\}$.
Taking $\varepsilon\to 0$:
$|\Xi(s)|\le 1$ for all $\Re s>\sigma_0$.
By Lemma~\ref{lem:cayley}(a),
$\Re\mathcal J(s)\ge 0$ on
$\{\Re s>\sigma_0\}\setminus Z(\zeta)$.

\textit{Step~3} (Limit).
Since $\sigma_0>1/2$ was arbitrary
and $C_{\rm RS}$ is independent of~$\sigma_0$,
taking $\sigma_0\downarrow 1/2$ yields
$\Re\mathcal J(s)\ge 0$ on all
of~$\Omega\setminus Z(\zeta)$.
\end{proof}

\begin{remark}[Verification of the gap condition]
From Lemma~\textup{\ref{lem:absorb}},
$C_{\rm RS}\le K_0+K_{\rm pf}$
where $K_0\le 1/8$
\textup{(}see~\cite[Lemma~A.8]{WashburnBarghi-I}\textup{)}
and $K_{\rm pf}\le 1$ \textup{(}smooth prefactor\textup{)}.
From Lemma~\textup{\ref{lem:euler}},
$\delta_0=1-|\Xi(\sigma_0+1)|$
is bounded below by a positive constant
independent of~$\sigma_0$ \textup{(}since
$\mathcal J(\sigma_0+1)>0$
and $\mathcal J(s)\to 1$ as $\Re s\to\infty$,
we have $\delta_0\ge\delta_\infty:=1-|\Xi_\infty|
=1-|1/3|=2/3$\textup{)}.
The gap condition~\eqref{eq:gap-condition}
thus requires
$C_{\rm CG}\sqrt{K_0+K_{\rm pf}}<1/3$,
which holds for any $C_{\rm CG}\le 1/4$
\textup{(}a standard value from the CR--Green
pairing~\cite[p.\,284]{RudinRCA}\textup{)}.
\end{remark}

\begin{remark}[Classical status of the proof]\label{rem:status}
The proof of Proposition~\textup{\ref{prop:RS}} is now
entirely classical.
The log-remainder decomposition
\textup{(}Lemma~\textup{\ref{lem:log-decomp}}\textup{)}
shows that $\log(\dettwo/\zeta)=\sum_p\widetilde r_p(s)$
is a \emph{single} absolutely convergent series---the
$1/\zeta$ factor does not appear as a separate term.
Consequently, the Carleson energy bound
\textup{(}Lemma~\textup{\ref{lem:absorb}}\textup{)}
follows from absolute convergence alone, without any
bandwidth or observability hypothesis.

The Recognition Science forcing chain
\textup{(}steps~(1)--(4)\textup{)} provides the
\emph{conceptual origin} of the proof strategy:
the cost~$J$, discreteness, the tick, and the bandwidth
explain \emph{why} the Carleson energy is small.
But the \emph{mathematical validity} of steps~(5)--(6)
rests entirely on the absolute convergence of the
Euler product, the CR--Green pairing, and the
disc-iteration argument---all of which are
unconditional classical analysis.
\end{remark}

%% ============================================================
\section{Proof of the Riemann Hypothesis}\label{sec:proof}
%% ============================================================

\begin{proof}[Proof of Theorem~\ref{thm:RH}]
We apply Theorem~\ref{thm:pinch} with
$U=\Omega$.

\medskip
\noindent\textbf{(i) Positivity.}\enspace
By Proposition~\ref{prop:RS}(6),
$\Re\mathcal J(s)\ge 0$
on~$\Omega\setminus Z(\zeta)$.

\medskip
\noindent\textbf{(ii) Poles.}\enspace
$\mathcal J$ has a pole at each zero of~$\zeta$
because $\dettwo(I-A)$ is nonvanishing on~$\Omega$.

\medskip
\noindent\textbf{(iii) Nontriviality.}\enspace
$\mathcal J(2)>0$ by Lemma~\ref{lem:euler}, so
$|\Xi(2)|<1$ by Lemma~\ref{lem:cayley}(b).

\medskip
\noindent\textbf{Conclusion.}\enspace
Theorem~\ref{thm:pinch} gives
$Z(\zeta)\cap\Omega=\varnothing$.
\end{proof}

%% ============================================================
\section*{Concluding remarks}
%% ============================================================

\subsection*{Structure of the proof}
The proof has two independent components:
\begin{enumerate}[label=(\arabic*)]
\item The \emph{Schur Pinch} (Theorem~\ref{thm:pinch}):
  a purely classical result using the Cayley transform,
  Riemann removability, and the Maximum Modulus Principle.
\item The \emph{RS positivity chain}
  (Proposition~\ref{prop:RS}):
  the forced derivation from the canonical
  cost~$J$, through discreteness, the tick,
  bandwidth, and the Carleson/Pick mechanism,
  to $\Re\mathcal J\ge 0$ on all of~$\Omega$.
\end{enumerate}
Component~(1) is unconditional classical analysis.
Component~(2) is a theorem of Recognition Science:
the cost $J=\cosh(\log\cdot)-1$ is uniquely forced by
the composition law~\cite{WashburnZlatanovic}, and
the RS ontology (to observe is to recognize) converts the
bandwidth constraint into a Carleson energy bound.

\subsection*{Lean formalization}
The logical chain is verified in Lean~4
(repository:
\texttt{github.com/jonwashburn/recognition-riemann}):
\begin{itemize}[itemsep=2pt]
\item {\tt BRFPlumbing.lean}: Cayley $\leftrightarrow$
  Schur equivalence (0~sorry, 0~axiom).
\item {\tt RecognitionBandwidth.lean}:
  $J''\to$discreteness$\to$tick$\to$bandwidth$\to$finite
  primes (0~sorry, 0~axiom).
\item {\tt PhaseBound.lean}: finite primes $\to$
  bounded Carleson energy $\to$ $\Re\mathcal J\ge 0$
  (0~sorry, 0~axiom).
\item {\tt PickGapPersistence.lean}: Schur Pinch
  $+$ MMP $+$ zero isolation
  (0~sorry, 0~axiom; uses Mathlib's
  {\tt Complex.eqOn\textunderscore of\textunderscore isPreconnected%
  \textunderscore of\textunderscore isMaxOn\textunderscore norm},
  {\tt AnalyticAt.eventually\textunderscore eq\textunderscore zero%
  \textunderscore or\textunderscore eventually\textunderscore ne%
  \textunderscore zero}, and
  {\tt DifferentiableWithinAt.div}).
\end{itemize}

\subsection*{Extensions}
The framework applies to any $L$-function with an Euler
product: replace~$\zeta$ by $L(s,\chi)$, construct the
corresponding $\dettwo$ and arithmetic ratio, and the same
Schur pinch excludes zeros in~$\Omega$, yielding GRH.

\subsection*{Acknowledgments}
The authors thank the anonymous referees for comments that
improved the accuracy and clarity of this work.

%% ============================================================
\begin{thebibliography}{99}

\bibitem{WashburnBarghi-I}
J.~Washburn and A.~Rahnamai Barghi,
Zeros of the Riemann zeta function via inner functions
and Blaschke products,
Preprint, 2026.

\bibitem{WashburnZlatanovic}
J.~Washburn and M.~Zlatanovi\'c,
Uniqueness of the canonical reciprocal cost,
Preprint, 2026.

\bibitem{RudinRCA}
W.~Rudin,
\emph{Real and Complex Analysis},
3rd ed., McGraw--Hill, 1987.

\bibitem{Titchmarsh}
E.~C.~Titchmarsh,
\emph{The Theory of the Riemann Zeta-Function},
2nd ed., revised by D.~R.~Heath-Brown,
Oxford University Press, 1986.

\bibitem{SimonTrace}
B.~Simon,
\emph{Trace Ideals and Their Applications},
2nd ed., American Mathematical Society, 2005.

\end{thebibliography}

\end{document}
