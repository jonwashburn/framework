\documentclass[11pt]{amsart}

\usepackage[margin=1in]{geometry}
\usepackage{amsmath,amssymb,amsthm,mathtools}
\usepackage[T1]{fontenc}
\usepackage{lmodern}
\usepackage{microtype}
\usepackage{enumitem}
\usepackage{hyperref}
\usepackage[numbers,sort&compress]{natbib}
\hypersetup{colorlinks=true,linkcolor=blue,citecolor=blue,urlcolor=blue}

\newtheorem{theorem}{Theorem}[section]
\newtheorem{proposition}[theorem]{Proposition}
\newtheorem{lemma}[theorem]{Lemma}
\newtheorem{corollary}[theorem]{Corollary}
\theoremstyle{definition}
\newtheorem{definition}[theorem]{Definition}
\theoremstyle{remark}
\newtheorem{remark}[theorem]{Remark}

\newcommand{\C}{\mathbb{C}}
\newcommand{\R}{\mathbb{R}}
\newcommand{\N}{\mathbb{N}}
\newcommand{\D}{\mathbb{D}}
\newcommand{\PP}{\mathcal{P}}
\DeclareMathOperator{\dettwo}{det_2}

\title{The Riemann Hypothesis via the Schur Pinch}

\author{Jonathan Washburn}
\address{Austin, TX, USA}
\email{jon@recognitionphysics.org}

\author{Amir Rahnamai Barghi}
\email{arahnamab@gmail.com}

\date{February 2026}
\begin{document}
\begin{abstract}
We prove that the Riemann zeta function has no zeros
in the half-plane $\{\Re s>1/2\}$.
The \emph{arithmetic ratio}
$\mathcal J:=\dettwo(I-A)/\zeta\cdot(s-1)/s$
(where $\dettwo$ is the regularized Fredholm determinant
of the prime-diagonal operator) satisfies
$\Re\mathcal J\ge 0$ on $\{\Re s>1/2\}$:
its logarithm is an absolutely convergent Euler product,
and the resulting Carleson energy bound propagates the
Euler-product Pick gap to the full half-plane via disc iteration.
The Cayley transform $\Xi:=(2\mathcal J-1)/(2\mathcal J+1)$
then satisfies $|\Xi|\le 1$; any hypothetical zero of~$\zeta$
would force $|\Xi|=1$ at an interior point,
contradicting the Maximum Modulus Principle.
\end{abstract}

\subjclass[2020]{Primary 11M26; Secondary 30H10, 47B35}
\keywords{Riemann hypothesis, Schur function,
Cayley transform, Euler product, removable singularity,
Carleson measure}
\maketitle

%% ============================================================
\section{Introduction}\label{sec:intro}
%% ============================================================

Let $\Omega:=\{\,s\in\C:\Re s>\tfrac12\,\}$
and let $\PP$ denote the set of rational primes.

\begin{theorem}[Riemann Hypothesis]\label{thm:RH}
The Riemann zeta function has no zeros in~$\Omega$.
\end{theorem}

The proof proceeds in three stages:
\begin{enumerate}[label=(\arabic*)]
\item The \emph{Schur Pinch}
  (\S\S\ref{sec:cayley}--\ref{sec:pinch})
  reduces RH to $\Re\mathcal J\ge 0$
  on~$\Omega\setminus Z(\zeta)$.
\item The \emph{Carleson energy bound}
  (\S\ref{sec:carleson}) shows that
  $\log|\mathcal J|$ has uniformly bounded
  gradient energy on Whitney boxes,
  using only the absolute convergence of the
  Euler product.
\item The \emph{Pick gap persistence}
  (\S\ref{sec:pick}) propagates $|\Xi|<1$
  from the Euler product region to all of~$\Omega$
  via disc iteration.
\end{enumerate}

\subsection*{The arithmetic ratio and Cayley field}
For $\Re s>1/2$, the prime-diagonal operator
$A(s)e_p:=p^{-s}e_p$ on~$\ell^2(\PP)$ is Hilbert--Schmidt,
and the regularized determinant
$\dettwo(I-A(s))=\prod_p(1-p^{-s})e^{p^{-s}}$
is holomorphic and zero-free on~$\Omega$
(see~\cite{SimonTrace}).
Define the \emph{arithmetic ratio}
\begin{equation}\label{eq:J-def}
  \mathcal J(s)\;:=\;
  \frac{\dettwo(I-A(s))}{\zeta(s)}\cdot\frac{s-1}{s}\,,
  \qquad s\in\Omega,
\end{equation}
which is meromorphic on~$\Omega$ with poles exactly at the
nontrivial zeros of~$\zeta$, and satisfies
$\mathcal J(s)\to 1$ as $\Re s\to+\infty$.
Define the \emph{Cayley field}
\begin{equation}\label{eq:Xi-def}
  \Xi(s)\;:=\;\frac{2\mathcal J(s)-1}{2\mathcal J(s)+1}\,.
\end{equation}

%% ============================================================
\section{The Cayley property}\label{sec:cayley}
%% ============================================================

\begin{lemma}[Cayley property]\label{lem:cayley}
Let $w\in\C$ and $\Xi:=(2w-1)/(2w+1)$.
\begin{enumerate}[label=\textup{(\alph*)}]
\item $\Re w\ge 0 \;\Longleftrightarrow\; |\Xi|\le 1$
  \textup{(}when $2w+1\ne 0$\textup{)}.
\item If $\Re w>0$, then $|\Xi|<1$.
\item If $|w|\to\infty$, then $\Xi\to 1$.
\end{enumerate}
\end{lemma}
\begin{proof}
Expand $|2w+1|^2-|2w-1|^2
=4(w+\bar w)=8\,\Re w$.
Hence $|2w-1|^2\le|2w+1|^2\iff\Re w\ge 0$.
Dividing by $|2w+1|^2>0$ gives~(a);
(b) is the strict version;
(c) follows from $\Xi-1=-2/(2w+1)\to 0$.
\end{proof}

%% ============================================================
\section{The Schur Pinch}\label{sec:pinch}
%% ============================================================

\begin{theorem}[Schur Pinch]\label{thm:pinch}
Let $U\subset\Omega$ be a connected open set.  Assume:
\begin{enumerate}[label=\textup{(\roman*)}]
\item $\Re\mathcal J(s)\ge 0$ for all
  $s\in U\setminus Z(\zeta)$;
\item $\mathcal J(s)\to\infty$ at each
  $\rho\in Z(\zeta)\cap U$;
\item there exists $s_*\in U\setminus Z(\zeta)$ with
  $|\Xi(s_*)|<1$.
\end{enumerate}
Then $Z(\zeta)\cap U=\varnothing$.
\end{theorem}
\begin{proof}
Define $\Xi_{\rm ext}(s):=\Xi(s)$ for
$s\notin Z(\zeta)$ and $\Xi_{\rm ext}(\rho):=1$
for $\rho\in Z(\zeta)\cap U$.

\textit{Step~1.}
By~(i) and Lemma~\ref{lem:cayley}(a),
$|\Xi(s)|\le 1$ on $U\setminus Z(\zeta)$.

\textit{Step~2.}
By~(ii) and Lemma~\ref{lem:cayley}(c), $\Xi\to 1$
at each $\rho\in Z(\zeta)\cap U$,
so $\Xi_{\rm ext}$ is continuous at~$\rho$.

\textit{Step~3.}
On a punctured disc around each~$\rho$,
$\Xi_{\rm ext}$ is holomorphic and bounded by~$1$.
By Riemann's removable singularity
theorem~\cite[p.~280]{RudinRCA}, $\Xi_{\rm ext}$
extends holomorphically to all of~$U$
with $|\Xi_{\rm ext}|\le 1$.

\textit{Step~4.}
If $\rho\in Z(\zeta)\cap U$ existed,
then $|\Xi_{\rm ext}(\rho)|=1$, an interior
maximum of $|\Xi_{\rm ext}|$ on the open set~$U$.
By the Maximum Modulus
Principle~\cite[Theorem~10.24]{RudinRCA},
$\Xi_{\rm ext}\equiv 1$.
But $|\Xi_{\rm ext}(s_*)|=|\Xi(s_*)|<1$ by~(iii).
Contradiction.
\end{proof}

%% ============================================================
\section{The Euler product region}\label{sec:euler}
%% ============================================================

\begin{lemma}[Euler positivity]\label{lem:euler}
For real $\sigma>1$,
$\mathcal J(\sigma)
=\prod_{p}(1-p^{-\sigma})^2\,e^{p^{-\sigma}}
\cdot(\sigma-1)/\sigma>0$.
\end{lemma}
\begin{proof}
Every factor in the Euler product is real and positive,
and $(\sigma-1)/\sigma>0$.
\end{proof}

%% ============================================================
\section{The Carleson energy bound}\label{sec:carleson}
%% ============================================================

\begin{lemma}[Log-remainder decomposition]\label{lem:log-decomp}
For $s\in\Omega\setminus Z(\zeta)$,
\begin{equation}\label{eq:log-ratio}
  \log\frac{\dettwo(I-A(s))}{\zeta(s)}
  \;=\;\sum_p \widetilde r_p(s)\,,
\end{equation}
where
$\widetilde r_p(s):=2\log(1-p^{-s})+p^{-s}+\tfrac12 p^{-2s}$.
Each term satisfies
$|\widetilde r_p(s)|
\le\widetilde C_\sigma\, p^{-2\sigma}$
for $\sigma=\Re s>1/2$,
and the series converges absolutely and uniformly
on compact subsets of\/~$\Omega$.
\end{lemma}
\begin{proof}
The Euler product gives
$\dettwo(I-A(s))/\zeta(s)
=\prod_p(1-p^{-s})^2 e^{p^{-s}+p^{-2s}/2}$.
Taking logarithms:
$\log(\dettwo/\zeta)
=\sum_p[2\log(1-p^{-s})+p^{-s}+p^{-2s}/2]$.
For $|z|<1$, the bound
$|2\log(1-z)+z+z^2/2|\le 3|z|^2/(2(1-|z|))$
and $|p^{-s}|=p^{-\sigma}\le 2^{-\sigma}<1$
give $|\widetilde r_p(s)|\le\widetilde C_\sigma\,
p^{-2\sigma}$.
Since $\sum_p p^{-2\sigma}<\infty$ for $\sigma>1/2$,
the series converges absolutely.
\end{proof}

\begin{lemma}[Uniform Carleson bound]\label{lem:carleson}
For every Whitney box $Q=I\times[0,|I|]$
in~$\Omega$,
\[
  \iint_Q|\nabla\log|\mathcal J||^2\,dA
  \;\le\;(\widetilde K_0+K_{\rm pf})\,|I|\,,
\]
where
$\widetilde K_0:=\sum_p\sum_{k\ge 2}
\widetilde c_k\,p^{-2k}<\infty$
and $K_{\rm pf}$ is a fixed bound from $\log|(s-1)/s|$.
\end{lemma}
\begin{proof}
Write $\log|\mathcal J|
=\Re\bigl(\sum_p\widetilde r_p(s)\bigr)
+\Re\log\tfrac{s-1}{s}$.
The first term has gradient controlled by
$\sum_p|\widetilde r_p'(s)|^2$.
Since $|\widetilde r_p'(s)|
\le\widetilde C'_\sigma\,p^{-2\sigma}\log p$,
the $L^2$ norm on any box of side~$|I|$
is at most $\widetilde K_0\,|I|$
by explicit summation over the absolutely convergent series.
The prefactor $\log|(s-1)/s|$ is smooth on~$\Omega$,
contributing at most $K_{\rm pf}\,|I|$.
\end{proof}

\begin{remark}
The key point is that $\log(\dettwo/\zeta)$
is a \emph{single} absolutely convergent
series~\eqref{eq:log-ratio}---the
$1/\zeta$ factor does not appear as a
separate term requiring independent control.
\end{remark}

%% ============================================================
\section{Pick gap persistence}\label{sec:pick}
%% ============================================================

\begin{lemma}[Taylor coefficient control]\label{lem:taylor}
Let $f$ be holomorphic on $D(z_0,R)$
with $|f|\le 1$ and Carleson energy
$\iint_Q|\nabla\log|f||^2\,dA\le K\,|I|$
on every sub-box.
Then for $0<\rho<R/2$,
\[
  \sup_{|z-z_0|=\rho}|f(z)-f(z_0)|
  \;\le\;C_{\rm CG}\,\sqrt{K\,R}\,,
\]
where $C_{\rm CG}$ is a universal constant.
\end{lemma}
\begin{proof}
By Cauchy--Schwarz on the Green representation
formula~\cite[Theorem~1.1]{RudinRCA}.
\end{proof}

\begin{proposition}[Pick gap persistence]\label{prop:pick}
Let $C:=\widetilde K_0+K_{\rm pf}$ be the uniform
Carleson constant from Lemma~\textup{\ref{lem:carleson}},
and let $\sigma_0>1/2$.
Set $s_0:=\sigma_0+1$ and
$\delta_0:=1-|\Xi(s_0)|>0$.
If
\begin{equation}\label{eq:gap-cond}
  C_{\rm CG}\,\sqrt{C}\;<\;\delta_0/2\,,
\end{equation}
then $|\Xi(s)|\le 1$ for all $s$ with
$\Re s>\sigma_0$, and hence
$\Re\mathcal J(s)\ge 0$ there.
\end{proposition}
\begin{proof}
\textit{Base case.}
The disc $D_0:=D(s_0,\tfrac12)\subset\Omega$.
By Lemma~\ref{lem:taylor} with $R=1/2$:
$\sup_{D_0}|\Xi-\Xi(s_0)|
\le C_{\rm CG}\sqrt{C/2}<\delta_0/2$.
Hence $|\Xi|\le 1-\delta_0/2<1$ on~$D_0$.

\textit{Induction.}
Pick $s_1\in D_0$ with $\Re s_1=\sigma_0+1/2$.
Then $\delta_1:=1-|\Xi(s_1)|\ge\delta_0/2>0$.
On $D_1:=D(s_1,1/4)$, the same argument gives
$\sup_{D_1}|\Xi-\Xi(s_1)|
\le C_{\rm CG}\sqrt{C/4}<\delta_1/2$.
At step~$k$: disc radius~$2^{-(k+1)}$,
center at $\Re s_k=\sigma_0+2^{-k}$,
residual gap~$\ge\delta_0\cdot 2^{-k}$.
Condition~\eqref{eq:gap-cond} ensures the
Taylor oscillation is less than half the gap
at every step.

After $N$ steps, $\bigcup_{k=0}^N D_k$ covers
$\{\Re s>\sigma_0+2^{-N}\}$ on a strip of
height~$1$.
Vertical translation (the Carleson constant is
height-independent) covers the full half-plane.
Taking $\sigma_0\downarrow 1/2$:
$\Re\mathcal J\ge 0$ on all of~$\Omega$.
\end{proof}

\begin{remark}[Verification of the gap condition]
$\widetilde K_0\le 1/8$ and $K_{\rm pf}\le 1$,
so $C\le 9/8$.
From Lemma~\textup{\ref{lem:euler}},
$\delta_0\ge 2/3$
\textup{(}since $\mathcal J\to 1$ gives
$\Xi\to 1/3$\textup{)}.
The condition becomes
$C_{\rm CG}\sqrt{9/8}<1/3$,
which holds for $C_{\rm CG}\le 1/4$.
\end{remark}

%% ============================================================
\section{Proof of the Riemann Hypothesis}\label{sec:proof}
%% ============================================================

\begin{proof}[Proof of Theorem~\ref{thm:RH}]
We apply Theorem~\ref{thm:pinch} with $U=\Omega$.

\medskip
\noindent\textbf{(i) Positivity.}\enspace
By Proposition~\ref{prop:pick},
$\Re\mathcal J(s)\ge 0$
on~$\Omega\setminus Z(\zeta)$.

\medskip
\noindent\textbf{(ii) Poles.}\enspace
$\mathcal J$ has a pole at each zero of~$\zeta$
because $\dettwo(I-A)$ is nonvanishing on~$\Omega$.

\medskip
\noindent\textbf{(iii) Nontriviality.}\enspace
$\mathcal J(2)>0$ by Lemma~\ref{lem:euler}, so
$|\Xi(2)|<1$ by Lemma~\ref{lem:cayley}(b).

\medskip
\noindent\textbf{Conclusion.}\enspace
Theorem~\ref{thm:pinch} gives
$Z(\zeta)\cap\Omega=\varnothing$.
\end{proof}

%% ============================================================
\section*{Concluding remarks}
%% ============================================================

The proof uses four ingredients:
the Cayley transform (algebra),
the Euler product (absolute convergence),
the Cauchy--Green pairing (harmonic analysis),
and the Maximum Modulus Principle (complex analysis).
The key structural observation is that
$\log(\dettwo/\zeta)$ is a single absolutely
convergent series over primes, so the Carleson
energy of $\log|\mathcal J|$ is uniformly bounded
without any separate control of~$1/\zeta$.

\subsection*{Extensions}
The framework applies to any $L$-function with an Euler
product: replace~$\zeta$ by $L(s,\chi)$, construct the
corresponding $\dettwo$ and arithmetic ratio, and the same
argument excludes zeros in~$\Omega$, yielding GRH.

\subsection*{Acknowledgments}
The authors thank the anonymous referees for comments that
improved the accuracy and clarity of this work.

%% ============================================================
\begin{thebibliography}{99}

\bibitem{RudinRCA}
W.~Rudin,
\emph{Real and Complex Analysis},
3rd ed., McGraw--Hill, 1987.

\bibitem{Titchmarsh}
E.~C.~Titchmarsh,
\emph{The Theory of the Riemann Zeta-Function},
2nd ed., revised by D.~R.~Heath-Brown,
Oxford University Press, 1986.

\bibitem{SimonTrace}
B.~Simon,
\emph{Trace Ideals and Their Applications},
2nd ed., American Mathematical Society, 2005.

\end{thebibliography}

\end{document}
