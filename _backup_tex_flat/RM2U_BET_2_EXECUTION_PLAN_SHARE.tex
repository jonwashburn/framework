\documentclass[11pt]{article}
\usepackage[margin=1in]{geometry}
\usepackage{fontspec}
\setmainfont{Libertinus Serif}
\setsansfont{Libertinus Sans}
\setmonofont{DejaVu Sans Mono}
\usepackage{microtype}
\usepackage{amsmath,amssymb,mathtools}
\newcommand{\R}{\mathbb{R}}
\newcommand{\Sbb}{\mathbb{S}}
\usepackage{hyperref}
\usepackage{bookmark}
\hypersetup{colorlinks=true,linkcolor=blue,urlcolor=blue,citecolor=blue}
\usepackage{xcolor}
\usepackage{listings}
\lstset{
  basicstyle=\ttfamily\small,
  breaklines=true,
  columns=fullflexible,
  keepspaces=true,
  showstringspaces=false
}

\title{Reader’s guide for a co-author/editor (added for sharing)}
\author{(generated from Markdown; shareable working document)}
\date{\today}

\begin{document}
\maketitle
\tableofcontents
\newpage

This PDF is a shareable rendering of \texttt{\detokenize{docs/RM2U_BET_2_EXECUTION_PLAN.md}} intended for a mathematician co-author who will edit and help close the remaining gates.
It is written as an \textbf{execution plan} (not a finished paper): it turns a large proof idea into a sequence of checkable gates, each with a minimal acceptance test and an explicit dependency graph.


\subsection{What this document is (and is not)}


\begin{itemize}
\item \textbf{What it is}: a “proof engineering notebook” that keeps TeX, Lean, and the research dependency graph aligned.
\item \textbf{What it is not}: a polished exposition of Navier–Stokes, nor a complete proof. You’ll see many items phrased as \emph{Hypotheses}—those are the deliberate \emph{interfaces} we either prove later or explicitly declare as blockers.
\end{itemize}


\subsection{How to read it quickly}


\begin{itemize}
\item \textbf{Start at “0. Canonical anchors”} to see the fixed TeX and Lean reference points this plan is allowed to use.
\item \textbf{Then read “1. Definition of Done”} to understand the concrete mathematical and Lean end-state for Bet 2.
\item \textbf{Skim “4. Multi-session execution roadmap”} for the intended order of attack and what each session must output.
\item \textbf{Keep an eye on “6. Pivot triggers”}: these are “fail-fast” guardrails that prevent hidden circularity (e.g. accidentally re-importing C2/UEWE/RM2 as if it were new progress).
\end{itemize}


\subsection{Legend / conventions}


\begin{itemize}
\item \textbf{TeX anchors} like \texttt{\detokenize{lem:*}}, \texttt{\detokenize{cor:*}}, \texttt{\detokenize{thm:*}}, \texttt{\detokenize{rem:*}} refer to items in \texttt{\detokenize{navier-dec-12-rewrite.tex}}.
\item \textbf{Lean anchors} refer to concrete symbols under \texttt{\detokenize{IndisputableMonolith/NavierStokes/RM2U/}}.
\item \textbf{Hypothesis blocks} are \emph{interfaces} (specifications). If a hypothesis is marked “blocked,” the doc records exactly one blocking dependency (the smallest missing lemma we can currently identify).
\item \textbf{Acceptance tests} are short, mechanical checks meant to prevent “hand-wave drift.” If an acceptance test can’t be met, the plan demands we either (i) tighten the statement, or (ii) record a single blocker and stop.
\end{itemize}


\subsection{How to co-author with this document (practical editing workflow)}


\begin{itemize}
\item \textbf{Canonical source}: treat \texttt{\detokenize{docs/RM2U_BET_2_EXECUTION_PLAN.md}} as the source of truth; the PDF is generated for sharing/review.
\item \textbf{When you “prove a hypothesis”}: upgrade it to a Lemma/Proposition/Theorem block \emph{in the same doc} (or move it into the manuscript TeX), and update the dependency/acceptance-test lines so the plan stays executable.
\item \textbf{Single-blocker discipline}: if a step fails, record exactly one minimal missing lemma (not a list). This keeps the next action crisp.
\item \textbf{No hidden circularity}: if a step implicitly requires C2/UEWE/U-decay/RM2 compactness, label it as such and trigger the pivot logic rather than burying it.
\item \textbf{Lean is bookkeeping, not PDE}: the Lean files primarily encode the dependency graph via Prop-only interfaces and “bookkeeping lemmas” (no analytic content); this is meant to prevent logical drift as the TeX side evolves.
\end{itemize}


\subsection{What the workflow demonstrates (why this is useful to share)}


This document is meant to make the working style auditable:


\begin{itemize}
\item \textbf{Decomposition}: split a hard theorem into small gates with explicit input/output contracts.
\item \textbf{Bidirectional alignment}: every TeX gate has a Lean-facing signature (often Prop-only) so the dependency graph is enforced by the typechecker.
\item \textbf{No hidden imports}: if something needs C2/UEWE/U-decay/RM2 compactness, it must be stated as such (pivot trigger).
\item \textbf{One-blocker discipline}: each blocked gate records \textbf{one} minimal blocker to keep the next step crisp.
\end{itemize}


\subsection{Status snapshot (updated 2025-12-23)}

\noindent\textbf{High-level state of the art (this project).}
We now have a referee-checkable \emph{reduction} of essentially every local closure attempt to one global tail/tightness gate.
The document is therefore in “go/no-go” mode: we do not proceed by optimism; we proceed by \textbf{certifying the first unavoidable global estimate}.

\begin{itemize}
\item \textbf{What is solid (proved / classical in the current writeup)}:
  the two brick lemmas (\texttt{\detokenize{lem:tail_biot_savart_affine_core}}, \texttt{\detokenize{lem:l2_shell_tests_detect_l2_component}})
  and the forcing decompositions/IBP identities that expose the true obstruction (e.g.\ \texttt{\detokenize{lem:forcing_pairing_no_absvalues_pivot}}).

\item \textbf{Certified pivots (fail-fast diagnostics already executed)}:
  \begin{itemize}
  \item K-ODE route: the “approximate Riccati defect” derivation collapses at a pressure-Hessian/\(\sigma_+\) estimate
    (\texttt{\detokenize{lem:attempt_derive_approx_riccati_defect_pivot}}), i.e. into C2/U-decay.
  \item RM2U/Bet-1/Family-A coefficient-energy routes: forcing control and boundary-term integrability collapse into the same C2/U-decay bottleneck
    (\texttt{\detokenize{lem:familyA_pivot_to_C2}}, \texttt{\detokenize{lem:bet1_boundary_integrability_pivot_to_C2}}), and there is no hidden algebraic cancellation
    (\texttt{\detokenize{lem:forcing_pairing_no_absvalues_pivot}}).
  \item “\(\ell=2\) dynamical instability” and “\(\ell=2\) Lamb-vector depletion” as replacement theorems: both collapse into global finite-budget/tightness control
    (\texttt{\detokenize{lem:attempt_prop_l2_instability_pivot_to_C2}}, \texttt{\detokenize{lem:attempt_l2_lamb_vector_depletion_pivot_to_RTD}}).
  \item UEWE via log-weighted enstrophy (“\(\beta=2\) truth serum”): pivots at the outer transport/boundary flux terms as \(R\to\infty\)
    (\texttt{\detokenize{lem:UEWE_log_weighted_enstrophy_pivot_to_RTD}}).
  \end{itemize}

\item \textbf{Single remaining global blocker (the real bottleneck)}:
  a genuine \textbf{tightness / no-multi-bubble} mechanism in blow-up variables (RTD/UEWE/RM2U-tail),
  i.e. a theorem that upgrades “tail bounded” to “tail small” uniformly in the running-max blow-up sequence.
  This is explicitly \emph{not} implied by the automatic running-max bounds (two-bubble counterexample:
  \texttt{\detokenize{lem:RTD_not_from_runningmax_bounds}}).

\item \textbf{Next best step (if continuing)}:
  target RTD/UEWE directly as a standalone global theorem, in whichever equivalent form is most checkable in classical PDE terms:
  \texttt{\detokenize{hyp:relative_tail_depletion_blowup}} (RTD),
  the UEWE bound \(\sup_{t\le 0}\int_{|x|\ge 1}(|\omega|^2/|x|^2+|\nabla\omega|^2)<\infty\) (Theorem \texttt{\detokenize{thm:RM2U-weighted-enstrophy-target}} in the alternative manuscript),
  or uniform exterior \(L^2_x\) control of the tail velocity/Lamb vector.
\end{itemize}


\section{RM2U Bet 2 — Rigorous Execution Plan (multi‑session)}


\textbf{Bet 2 = self‑falsification / backward contradiction.}\\
We assume “parasitic export persists” (tail‑flux does \textbf{not} vanish) and drive a contradiction with the running‑max \textbf{finite budget} constraints.


This doc is deliberately TeX‑anchored and Lean‑signature‑ready: each step ends with a checkable deliverable.



\bigskip\hrule\bigskip


\subsection{0. Canonical anchors (do not drift)}


\subsubsection{0.1 TeX anchors in \texttt{navier-dec-12-rewrite.tex}}


\begin{itemize}
\item \textbf{RM2U coefficient PDE}: \texttt{\detokenize{lem:Ab-evolution}} (\texttt{\detokenize{eq:Ab-PDE}}, \texttt{\detokenize{eq:Ab-forcing}})
\item \textbf{Energy identity behind coercive bound}: \texttt{\detokenize{rem:Ab-energy-identity}}
\item \textbf{Tail strain moment is the obstruction}: \texttt{\detokenize{lem:tail-strain-formula}}, \texttt{\detokenize{cor:RM2-equivalence}}
\item \textbf{RM2U closure from coercive bound}: \texttt{\detokenize{thm:RM2U-closure-from-coercive}}
\item \textbf{The explicit “missing conversion” note}: \texttt{\detokenize{rem:l2-instability-open}}
\item \textbf{Running‑max finite budget inequality (maximizers)}: \texttt{\detokenize{rem:runningmax-injection-constraint}}
\item \textbf{“Payment engine” on \(\{\rho\approx 1\}\)} (signed injection / diffusion cost):
\item \texttt{\detokenize{lem:injection-damping-balance}}
\item \texttt{\detokenize{lem:superlevel-selection}}, \texttt{\detokenize{cor:superlevel-selection-simplified}}
\item \texttt{\detokenize{rem:superlevel-selection-block}}, \texttt{\detokenize{rem:superlevel-time-fraction-sign}}
\item \texttt{\detokenize{rem:sigma-minus-paid-by-diffusion}}, \texttt{\detokenize{rem:rho32-errors-small-scale}}
\item \textbf{Heuristic barrier vs tail forcing}: \texttt{\detokenize{rem:RM2U-spectral-gap-heuristic}}
\item \textbf{(Placeholder in the draft)}: Proposition \texttt{\detokenize{prop:l2-instability}} is referenced but not defined; this bet aims to supply it.
\end{itemize}


\subsubsection{0.2 Lean anchors (already in repo)}


\begin{itemize}
\item Bet‑2 interface: \texttt{\detokenize{RM2U.NonParasitism.Bet2SelfFalsificationHypothesis}}
\item Tail‑flux: \texttt{\detokenize{RM2U.Core.TailFluxVanish}}
\item Bet‑2 ⇒ non‑parasitism: \texttt{\detokenize{RM2U.NonParasitism.nonParasitism_of_bet2}}
\item Coercive ⇒ RM2Closed: \texttt{\detokenize{RM2U.RM2Closure.rm2Closed_of_coerciveL2Bound}}
\item End‑to‑end wrapper (once coercive step exists): \texttt{\detokenize{RM2U.NonParasitism.rm2Closed_of_bet2}}
\item Coercive-step interface (explicit energy/forcing assumptions) + proved theorem:
  \texttt{\detokenize{RM2U.EnergyIdentity.TailFluxVanishImpliesCoerciveHypothesis}} and
  \texttt{\detokenize{RM2U.EnergyIdentity.coerciveL2Bound_of_tailFluxVanish}}
\end{itemize}



\bigskip\hrule\bigskip


\subsection{1. “Definition of Done” for Bet 2}


\subsubsection{1.1 Mathematical done (TeX / community-facing)}


Add a referee‑checkable statement (and proof) that can replace the missing reference:


\begin{quote}
\textbf{Proposition \texttt{\detokenize{prop:l2-instability}} (Dynamical instability of the ℓ=2 tail).}\\
For a running‑max/vorticity‑normalized ancient element, any non‑zero persistent ℓ=2 tail moment forces a contradiction with the running‑max budget; therefore the ℓ=2 tail strain moment vanishes (or is forced into an integrable class strong enough to close RM2).
\end{quote}


We will treat “vanishes” vs “\(L^2_t\)” as two graded targets:


\begin{itemize}
\item \textbf{Target T‑strong}: \(S(0,t)\equiv 0\) for all \(t\le 0\).
\item \textbf{Target T‑weak (still closes RM2 subsequentially)}: \(S(0,\cdot)\in L^2((-\infty,0])\) (see \texttt{\detokenize{thm:RM2-from-tail-L2}}).
\end{itemize}


\subsubsection{1.2 Lean done (repo-facing)}


At the abstract time‑slice level:


\begin{itemize}
\item Provide a concrete instance of \texttt{\detokenize{Bet2SelfFalsificationHypothesis P}} for the coefficient profile \texttt{\detokenize{P}}
  associated to the RM2U ℓ=2 coefficient.
\end{itemize}


This implies \texttt{\detokenize{NonParasitismHypothesis P}} automatically.



\bigskip\hrule\bigskip


\subsection{1.3 Critical dependency hazard (avoid circularity)}


Some “running‑max ancient element” statements in the TeX (notably the pointwise identification
\texttt{\detokenize{|ω∞(0,t)| = 1}} for all \texttt{\detokenize{t}} in the proof of \texttt{\detokenize{lem:runningmax-sup-freeze-3d}}) explicitly note they
use a \textbf{local compactness step} that is \emph{itself} obstructed by RM2 (the affine/harmonic mode).


For Bet 2 to be logically clean, we must \textbf{not} assume pointwise-in-time, pointwise-in-space
facts that already require RM2. Therefore we will run Bet 2 in one of two logically safe modes:


\begin{itemize}
\item \textbf{Mode (M‑seq): “pre‑limit” contradiction on the running‑max rescaled sequence}\\
  Prove that persistent ℓ=2 export along the rescaled sequence forces a violation of the running‑max
  normalization \emph{before} taking any fixed‑frame limit. This avoids needing the blocked compactness step.
\item \textbf{Mode (M‑limit‑with‑affine): “limit” contradiction with affine mode kept explicit}\\
  Work with a limit object that includes the affine/harmonic correction as part of the state
  (i.e. do not quotient it out prematurely). Bet 2 then aims to show that this affine mode must be
  zero / integrable, thereby closing RM2.
\end{itemize}


\textbf{Rule:} every lemma in Sessions S1–S3 must declare which mode it is using.


\subsubsection{1.3.1 Mode cheat sheet (when to use which)}


\begin{itemize}
\item \textbf{Use Mode (M‑limit‑with‑affine)} if you want statements that match the manuscript language
  “affine/harmonic obstruction coefficient” directly. This is the most faithful route to a
  replacement for \texttt{\detokenize{prop:l2-instability}}.
\item \textbf{Use Mode (M‑seq)} if you want to avoid \emph{any} reliance on fixed‑frame compactness/limits.
  This is safest logically, but will require carrying “approximate maximizer” errors and
  uniform-in-\texttt{\detokenize{k}} bookkeeping.
\end{itemize}



\bigskip\hrule\bigskip


\subsection{2. What Bet 2 must \emph{actually} prove (the core logical move)}


Bet 2 is only real if we can produce a monotone/contradiction mechanism of the form:


\begin{quote}
\textbf{(B2‑Core)} If tail export persists in the ℓ=2 sector (formalized as \texttt{\detokenize{¬TailFluxVanish}} or an equivalent lower bound), then some \textbf{nonnegative functional} \(J(t)\) must grow backward in time at a rate incompatible with running‑max normalization.
\end{quote}


\texttt{\detokenize{rem:l2-instability-open}} identifies the gap precisely:


\begin{itemize}
\item identify the correct ℓ=2 component that generates the obstruction; and
\item upgrade “sign at one time/model” into \textbf{uniform backward‑time decay/integrability} for the tail moment.
\end{itemize}


We already have the identification mechanism in TeX:


\begin{itemize}
\item \texttt{\detokenize{lem:tail-strain-formula}} + \texttt{\detokenize{thm:RM2U-closure-from-coercive}} show
  \(\Sigma_b^{1,\infty}(t)=\int_1^\infty A_b(r,t)\,dr/r\) controls the tail strain \(S(0,t)\).
\end{itemize}


So Bet 2 is now entirely about the second bullet: finding \(J(t)\) and proving the evolution inequality.


\subsubsection{2.1 “Known identities” we are allowed to treat as fixed targets}


These are the concrete algebraic relations Bet 2 should lean on (they already appear explicitly in TeX):


\begin{itemize}
\item \textbf{Tail strain as the obstruction:} for tail vorticity \(\Omega\) supported in \(|w|>1\),
  the tail strain at the origin is (TeX \texttt{\detokenize{lem:tail-strain-formula}})
  \[
  S_{tail}(0)
  =-\frac{3}{8\pi}\int_{|w|>1}\frac{(w\times\Omega(w))\otimes w+w\otimes(w\times\Omega(w))}{|w|^5}\,dw.
  \]
\item \textbf{Directional contraction equals the log‑critical shell moment:} for \(b\in S^2\),
  \[
  b\cdot S_{tail}(0,t)b=\frac{3}{4\pi}\,\Sigma_b^{1,\infty}(t),
  \qquad
  \Sigma_b^{1,\infty}(t):=\int_1^\infty \frac{A_b(r,t)}{r}\,dr,
  \]
  and \(A_b(r,t)=\int_{S^2}\omega(r\theta,t)\cdot\Phi_b(\theta)\,d\theta\).
  (This computation is used in the proof of TeX \texttt{\detokenize{thm:RM2U-closure-from-coercive}}.)
\end{itemize}


Bet 2 should be written so these identities are “plugs” rather than re-derivations.



\bigskip\hrule\bigskip


\subsection{3. Candidate contradiction functionals (choose one; don’t multiply work)}


All candidates must be:


\begin{itemize}
\item \textbf{nonnegative}, and
\item \textbf{linked to the ℓ=2 tail moment} (via a quantitative inequality),
\item \textbf{constrained by running‑max} (directly or via a bounded “budget” quantity).
\end{itemize}


\subsubsection{3.1 Candidate family A: ℓ=2 coefficient energies (truncated in radius)}


Use the PDE \texttt{\detokenize{eq:Ab-PDE}} and the energy identity (\texttt{\detokenize{rem:Ab-energy-identity}}) on a finite interval \texttt{\detokenize{[1,R]}}:


\begin{itemize}
\item \(E_{b,R}(t) := \int_1^R |A_b(r,t)|^2 r^2\,dr\).
\end{itemize}


Then compute \(dE_{b,R}/dt\) and keep the boundary flux at \texttt{\detokenize{R}} explicit:


\begin{itemize}
\item the boundary term at \texttt{\detokenize{R}} is the “tail flux” object (the one Option‑A is built around).
\end{itemize}


\textbf{Bet 2 goal in this family:} show persistent nonzero boundary flux forces backward growth of \(E_{b,R}\) fast enough to contradict some a priori bound available under running‑max normalization.

\paragraph{Family A diagnostic (pivot check): where forcing control collapses into C2/U-decay.}

\begin{lstlisting}
\begin{lemma}[Attempt: Family A energy route collapses into C2/U-decay (pivot trigger)]\label{lem:familyA_pivot_to_C2}
Fix $b\in\Sbb^2$ and $R>1$. Let $A(r,t):=A_b^\infty(r,t)$ and $\mathcal F(r,t):=\mathcal F_b(r,t)$ satisfy the $A_b$ PDE
\(\eqref{eq:Ab-PDE}\)--\(\eqref{eq:Ab-forcing}\) in \texttt{navier-dec-12-rewrite.tex}.
Define
\[
E_{b,R}(t):=\int_1^R |A(r,t)|^2 r^2\,dr,
\qquad
B(r,t):=(-A(r,t))\,(r^2\partial_r A(r,t)).
\]
Then the finite-$R$ energy identity (see \texttt{book/docs/navier-dec-12-rewrite-alternative.tex},
Lemma~\texttt{\detokenize{\ref{lem:Ab-energy-identity-finiteR}}}) reads
\[
\frac12\frac{d}{dt}E_{b,R}(t)\;+\;\int_{1}^{R}\Bigl(|\partial_r A|^2\,r^2+6|A|^2\Bigr)\,dr
\;=\;\int_{1}^{R}\mathcal F\,A\,r^2\,dr\;+\;B(1,t)-B(R,t).
\tag{$\star$}
\]

\smallskip
\noindent\textbf{Attempted contradiction from nonvanishing tail flux.}
The ``tail flux'' event is precisely that $B(R,t)$ (or a normalized version of it) does not vanish as $R\to\infty$ on a time-thick set.
To turn such a persistence into a contradiction with boundedness of $E_{b,R}(t)$ (which is finite for each fixed $R$ under $\|\omega^\infty\|_\infty<\infty$),
one must control the \emph{forcing work term} in $(\star)$ in a way that allows it to be absorbed by the coercive dissipation:
\[
\left|\int_{1}^{R}\mathcal F\,A\,r^2\,dr\right|
\ \le\ \frac12\int_{1}^{R}\Bigl(|\partial_r A|^2\,r^2+6|A|^2\Bigr)\,dr\ +\ \mathsf{Err}_R(t),
\tag{$\dagger$}
\]
with $\mathsf{Err}_R(t)$ controlled uniformly in $R$ and (crucially) \emph{without} importing global tail decay.

\smallskip
\noindent\textbf{Where the pivot trigger fires.}
In the current manuscript architecture, the only available mechanism that yields a scale-consistent bound of the form $(\dagger)$
for the ancient element is the \textbf{$\sigma$-decomposition + tail/decay gate}, which quantitatively controls the positive stretching injection
\[
\iint_{Q_r}\rho^{3/2}\sigma_+\quad\text{as }r\downarrow 0
\]
(Theorem~C2-closure, conditional on U-decay/RTD).
Equivalently: controlling the forcing work term $\int \mathcal F\,A\,r^2$ without losing on large radii requires suppressing
the tail/harmonic/affine contributions to $\mathcal F$, and the only known way to do that in the current framework is via
the C2/U-decay mechanism.

Therefore the Family A route hits the same bottleneck as the K-ODE route: it collapses into controlling
\(\iint \rho^{3/2}\sigma_+\) (hence into C2/U-decay/RTD). This certifies the \textbf{P-C2 pivot trigger}.
\end{lemma}
\end{lstlisting}

\paragraph{No-absolute-values diagnostic: the forcing pairing has no hidden structural cancellation.}

\begin{lstlisting}
\begin{lemma}[Attempt: no-absolute-values rewrite of the forcing work term still exposes the same tail/tightness gate]\label{lem:forcing_pairing_no_absvalues_pivot}
Fix $t\le 0$, $b\in\Sbb^2$, and write $A(r):=A_b^\infty(r,t)$ and $\mathcal F(r):=\mathcal F_b(r,t)$
from \(\eqref{eq:Ab-PDE}\)--\(\eqref{eq:Ab-forcing}\).
Let $Y_b(\theta):=(b\cdot\theta)^2-\frac13$ and define the (time-slice) $\ell=2$ Lamb-vector coefficients
\[
G(r):=\int_{\Sbb^2}(u^\infty\times\omega^\infty)(r\theta,t)\cdot\nabla_{\!S}Y_b(\theta)\,d\theta,
\qquad
H(r):=\int_{\Sbb^2}(u^\infty\times\omega^\infty)(r\theta,t)\cdot\theta\;Y_b(\theta)\,d\theta.
\]
Then the RM2U forcing admits the exact representation (see \texttt{navier-dec-12-rewrite.tex}, Remark \texttt{\detokenize{rem:curl-coupling-RM2U}}):
\[
\mathcal F(r)=\frac{1}{2r}\frac{d}{dr}\bigl(r\,G(r)\bigr)-\frac{3}{r}\,H(r).
\tag{$\Diamond$}
\]
Consequently, the forcing work term in the $A_b$ energy identity has the exact integration-by-parts formula
(see \texttt{book/docs/navier-dec-12-rewrite-alternative.tex}, Lemma \texttt{\detokenize{\ref{lem:forcing-pairing-GH}}}):
\[
\int_{1}^{R}\mathcal F(r)\,A(r)\,r^2\,dr
=\frac12\Bigl[r\,A(r)\,(rG(r))\Bigr]_{r=1}^{r=R}
-\frac12\int_{1}^{R}\bigl(A(r)+rA'(r)\bigr)\,(rG(r))\,dr
-3\int_{1}^{R}r\,A(r)\,H(r)\,dr.
\tag{$\clubsuit\clubsuit$}
\]
Equivalently, absorbing the $H$ term into a Hardy gauge (Lemma \texttt{\detokenize{\ref{lem:forcing-pairing-Hardy-gauge}}})
one can rewrite $\mathcal F(r)=\frac{1}{2r}\frac{d}{dr}(r\widetilde G(r))$, hence
\[
\int_{1}^{R}\mathcal F\,A\,r^2\,dr
=\frac12\Bigl[r\,A(r)\,(r\widetilde G(r))\Bigr]_{r=1}^{r=R}
-\frac12\int_{1}^{R}\bigl(A(r)+rA'(r)\bigr)\,(r\widetilde G(r))\,dr.
\tag{$\clubsuit\clubsuit\clubsuit$}
\]

\smallskip
\noindent\textbf{Conclusion (fail-fast).}
The identities $(\clubsuit\clubsuit)$--$(\clubsuit\clubsuit\clubsuit)$ show that, beyond an explicit boundary contribution,
the forcing pairing is an \emph{indefinite bulk bilinear form} involving $(A+rA')$ paired with $rG$ (or $r\widetilde G$)
and $A$ paired with $rH$.
There is no remaining spherical-harmonic orthogonality or incompressibility identity that forces these bulk terms to vanish,
and there is no sign structure visible at the level of this exact decomposition.
Therefore, any attempt to control (or absorb) $\int_1^R \mathcal F A r^2dr$ even \emph{without} taking absolute values
still requires a genuine tail/tightness mechanism that controls the $\ell=2$ Lamb-vector coefficients
$rG,rH$ (equivalently, the degree-$2$ part of $u_{>1}^\infty\times\omega^\infty$ in an exterior $L^2_x$ norm).
In the current manuscript architecture, this is precisely the RTD/UEWE/C2-type global gate.
\end{lemma}
\end{lstlisting}

\paragraph{Candidate new theorem: $\ell=2$ Lamb-vector depletion (a direct replacement for RTD/U-decay at the forcing level).}

\begin{lstlisting}
\begin{definition}[Proposed $\ell=2$ Lamb-vector depletion gate]\label{def:l2_lamb_vector_depletion_gate}
Let $(u^\infty,p^\infty)$ be the running-max ancient element and write $\omega^\infty:=\curl u^\infty$.
Let $u_{\le 1}^\infty(\cdot,t)$ be the truncated Biot--Savart field generated by $\omega^\infty(\cdot,t)$ on $\{|y|\le 1\}$, and set
\[
u_{>1}^\infty(\cdot,t):=u^\infty(\cdot,t)-u_{\le 1}^\infty(\cdot,t)
\]
(cf. \texttt{book/docs/navier-dec-12-rewrite-alternative.tex}, Remark \texttt{\detokenize{\ref{rem:BS-core-vs-tail-RM2U}}}).

For each $b\in\Sbb^2$ let $Y_b(\theta):=(b\cdot\theta)^2-\frac13$ and define the \emph{tail Lamb-vector coefficients}
\[
G_b^{\mathrm{tail}}(r,t):=\int_{\Sbb^2}(u_{>1}^\infty\times\omega^\infty)(r\theta,t)\cdot\nabla_{\!S}Y_b(\theta)\,d\theta,
\qquad
H_b^{\mathrm{tail}}(r,t):=\int_{\Sbb^2}(u_{>1}^\infty\times\omega^\infty)(r\theta,t)\cdot\theta\;Y_b(\theta)\,d\theta.
\]

We say the \textbf{$\ell=2$ Lamb-vector depletion gate} holds if there exists $C<\infty$ such that
\[
\sup_{t\le 0}\ \sup_{b\in\Sbb^2}\ \int_{1}^{\infty}\Bigl(|r\,G_b^{\mathrm{tail}}(r,t)|^2+|r\,H_b^{\mathrm{tail}}(r,t)|^2\Bigr)\,dr\ \le\ C.
\]
\end{definition}

\begin{lemma}[If $\ell=2$ Lamb-vector depletion holds, the forcing pairing is uniformly form-bounded (hence no RTD is needed at this step)]\label{lem:l2_lamb_vector_depletion_implies_pairing_bound}
Assume the gate in Definition~\ref{def:l2_lamb_vector_depletion_gate} holds with constant $C$.
Then for each $t\le 0$ there exists a constant $C_\mathcal F(t)<\infty$ such that for all $b\in\Sbb^2$ and all $R>1$,
\[
\left|\int_{1}^{R}\mathcal F_b(r,t)\,A_b^\infty(r,t)\,r^2\,dr\right|\ \le\ C_\mathcal F(t),
\]
where $A_b^\infty$ and $\mathcal F_b$ are the projected coefficient and forcing from \(\eqref{eq:Ab-PDE}\)--\(\eqref{eq:Ab-forcing}\).
In particular, this supplies exactly the uniform-in-$R$ forcing-pairing bound required in the RM2U time-slice energy interface
(\texttt{IndisputableMonolith/NavierStokes/RM2U/ProjectedPDE.lean}, \texttt{ProjectedPDEPairingHypothesisAt.hForcePair}).
\end{lemma}

\begin{proof}[Proof sketch]
Fix $t,b$ and write $A(r):=A_b^\infty(r,t)$, $G(r):=G_b(r,t)$, $H(r):=H_b(r,t)$ as in Lemma~\ref{lem:forcing_pairing_no_absvalues_pivot}.
Decompose $G=G^{\mathrm{core}}+G^{\mathrm{tail}}$ and $H=H^{\mathrm{core}}+H^{\mathrm{tail}}$ with $u^\infty=u_{\le 1}^\infty+u_{>1}^\infty$.

The core contribution satisfies
\(\sup_b\int_1^\infty (|rG_b^{\mathrm{core}}|^2+|rH_b^{\mathrm{core}}|^2)\,dr\le C_0\)
by the purely classical Biot--Savart decay estimate (see \texttt{book/docs/navier-dec-12-rewrite-alternative.tex}, Lemma \texttt{\detokenize{\ref{lem:GH-core-L2}}}).
By the assumed tail gate, the same bound holds for the tail part; hence for this fixed time $t$,
\[
\sup_{b\in\Sbb^2}\int_1^\infty (|rG_b|^2+|rH_b|^2)\,dr\ \le\ C_0+C.
\]
Now apply the exact identity \((\clubsuit\clubsuit)\) from Lemma~\ref{lem:forcing_pairing_no_absvalues_pivot} and Cauchy--Schwarz:
\[
\left|\int_{1}^{R}(A+rA')\,(rG)\,dr\right|
\le \left(\int_{1}^{R}|A+rA'|^2\,dr\right)^{1/2}\left(\int_{1}^{R}|rG|^2\,dr\right)^{1/2},
\]
\[
\left|\int_{1}^{R}r\,A\,H\,dr\right|
\le \left(\int_{1}^{R}|A|^2\,dr\right)^{1/2}\left(\int_{1}^{R}|rH|^2\,dr\right)^{1/2}.
\]
Since \(\int_1^R |A+rA'|^2\lesssim \int_1^R (|A'|^2r^2+|A|^2)\,dr + |RA(R)|^2+|A(1)|^2\) by integration by parts,
and since \(\int_1^\infty (|rG|^2+|rH|^2)\,dr\) is finite uniformly in $b$, the right-hand side is bounded by a constant depending only on the
time-slice energy of $A$ and the gate constant. This yields a bound uniform in $R$ for each fixed $t$, as claimed.
\end{proof}

\begin{lemma}[Attempt: proving $\ell=2$ Lamb-vector depletion unconditionally collapses into the same global tail/tightness bottleneck]\label{lem:attempt_l2_lamb_vector_depletion_pivot_to_RTD}
There is no purely local (running-max) mechanism currently available in the program that proves Definition~\ref{def:l2_lamb_vector_depletion_gate}
for the running-max ancient element.
Any proof attempt reduces immediately to controlling an \emph{exterior} $L^2_x$ norm of the tail velocity (or tail Lamb vector),
which is exactly an RTD/UEWE-type tightness input.
\end{lemma}

\begin{proof}[Fail-fast proof attempt]
Fix $t\le 0$ and abbreviate $\omega:=\omega^\infty(\cdot,t)$ and $v:=u_{>1}^\infty(\cdot,t)$.
By Cauchy--Schwarz on $\Sbb^2$ and finite-dimensionality of $\ell=2$, one has for each $r\ge 1$ and each $b\in\Sbb^2$
\[
|G_b^{\mathrm{tail}}(r,t)|+|H_b^{\mathrm{tail}}(r,t)|\ \lesssim\ \|(v\times\omega)(r\cdot,t)\|_{L^2(\Sbb^2)}.
\]
Multiplying by $r$, squaring, and integrating in $r$ gives the classical implication
\[
\sup_{b\in\Sbb^2}\int_{1}^{\infty}\bigl(|rG_b^{\mathrm{tail}}|^2+|rH_b^{\mathrm{tail}}|^2\bigr)\,dr
\ \lesssim\ \int_{|x|\ge 1}|(v\times\omega)(x,t)|^2\,dx
\ \le\ \|\omega\|_{L^\infty}^2\int_{|x|\ge 1}|v(x,t)|^2\,dx
\]
(see \texttt{book/docs/navier-dec-12-rewrite-alternative.tex}, Lemmas \texttt{\detokenize{\ref{lem:rGH-L2-from-vxw}}} and \texttt{\detokenize{\ref{lem:rGH-L2-from-kinetic}}}).

Thus, to prove Definition~\ref{def:l2_lamb_vector_depletion_gate} it suffices to prove a uniform-in-time exterior kinetic energy bound
\(\sup_{t\le 0}\int_{|x|\ge 1}|u_{>1}^\infty(x,t)|^2dx<\infty\)
(or a uniform exterior $L^2_x$ bound on $u_{>1}^\infty\times\omega^\infty$).

\smallskip
\noindent
\textbf{Stop / pivot trigger:} no such exterior $L^2_x$ control follows from the running-max bounds
(\(\|\omega^\infty\|_{L^\infty}\le 1\) and scale-critical local \(L^{3/2}\) control), because these allow non-decaying affine/harmonic
velocity modes and translation bubbling in blow-up variables. In particular, rigid rotation shows that bounded vorticity alone does not control
the $r$-weighted coefficients (alternative manuscript, Remark \texttt{\detokenize{\ref{rem:rigid-rotation-breaks-rGH}}}).
Therefore, establishing the depletion gate requires a genuinely global tightness mechanism excluding such tail modes,
i.e. an RTD/UEWE/RM2U-type assumption (equivalently: tail boundedness \emph{must} be upgraded to tail smallness).
\end{proof}
\end{lstlisting}

\paragraph{UEWE attempt via a log-weighted enstrophy identity (the $\beta=2$ ``truth serum'').}

\begin{lstlisting}
\begin{lemma}[Attempt: log-weighted enstrophy identity exposes the $\beta=2$ defect and pivots at the outer transport/boundary terms]\label{lem:UEWE_log_weighted_enstrophy_pivot_to_RTD}
Let $(u,p)$ be a smooth incompressible Navier--Stokes solution on $\R^3\times(t_1,t_2)$ and let $\omega:=\curl u$.
Fix a basepoint $x_0\in\R^3$ and radii $0<\lambda<R$.
Let $\phi_{\lambda,R}(x)$ be the standard cutoff log-weight supported on the annulus $\{\lambda\le |x-x_0|\le 2R\}$:
\[
\phi_{\lambda,R}(x):=\chi_{\lambda,R}(|x-x_0|)\,\log\!\Bigl(\frac{|x-x_0|}{\lambda}\Bigr),
\]
with $\chi_{\lambda,R}\equiv 1$ on $2\lambda\le |x-x_0|\le R$ and supported on $\{\lambda\le |x-x_0|\le 2R\}$
(see \texttt{book/docs/navier-dec-12-rewrite-alternative.tex}, Lemma \texttt{\detokenize{\ref{lem:log-weighted-enstrophy-identity}}}).
Then the weighted enstrophy balance yields, on any $t_a<t_b$,
\[
\int_{t_a}^{t_b}\int_{2\lambda\le |x-x_0|\le R}\frac{|\omega(x,t)|^2}{|x-x_0|^2}\,dx\,dt
\ \lesssim\ 
\Delta_{\mathrm{time}}(\lambda,R)
\ +\ \mathcal I_{\mathrm{stretch}}(\lambda,R)
\ +\ \mathcal I_{\mathrm{transport}}(\lambda,R)
\ +\ \mathcal I_{\mathrm{bdry}}(\lambda,R),
\tag{$\spadesuit$}
\]
where:
\begin{itemize}
\item $\Delta_{\mathrm{time}}(\lambda,R)$ is the time-endpoint term
\(\int |\omega(x,t_b)|^2\phi_{\lambda,R}(x)dx-\int |\omega(x,t_a)|^2\phi_{\lambda,R}(x)dx\),
\item $\mathcal I_{\mathrm{stretch}}(\lambda,R):=\int_{t_a}^{t_b}\int (\omega\cdot\nabla u)\cdot\omega\ \phi_{\lambda,R}\,dx\,dt$,
\item $\mathcal I_{\mathrm{transport}}(\lambda,R):=\int_{t_a}^{t_b}\int |\omega|^2\ u\cdot\nabla\phi_{\lambda,R}\,dx\,dt$,
\item $\mathcal I_{\mathrm{bdry}}(\lambda,R)$ is an explicit boundary-layer error supported only on
\(\{\lambda\le |x-x_0|\le 2\lambda\}\cup\{R\le |x-x_0|\le 2R\}\).
\end{itemize}

\smallskip
\noindent\textbf{Fail-fast pivot.}
Even under the running-max bound $|\omega|\le 1$, there is no way to send $R\to\infty$ in $(\spadesuit)$ using only local scale-critical control:
the outer boundary-layer density obeys $|\mathcal E^{\mathrm{bdry}}_{\lambda,R}|\lesssim (1+\log(R/\lambda))R^{-2}\mathbf 1_{\{R\le r\le 2R\}}$,
so a trivial bound gives $\mathcal I_{\mathrm{bdry}}(\lambda,R)\lesssim (1+\log(R/\lambda))\,R\,(t_b-t_a)$, which diverges as $R\to\infty$ unless one has
genuine tail smallness of $|\omega|^2$ on $\{R\le |x-x_0|\le 2R\}$.
Similarly, controlling $\mathcal I_{\mathrm{transport}}(\lambda,R)$ requires a uniform control of the outer transport flux of $|\omega|^2$,
which is exactly a tightness/UEWE/RTD-type input in blow-up variables (``tail boundedness $\Rightarrow$ tail smallness'').

Therefore, the log-weighted identity provides a precise analytic reduction of UEWE to controlling the outer transport/boundary terms, but it does not
close UEWE unconditionally. Any successful proof must supply a genuinely new global mechanism to control these outer terms.
\end{lemma}
\end{lstlisting}

\paragraph{Bet 1 diagnostic (pivot check): boundary-term integrability collapses into the same C2/U-decay bottleneck.}

\begin{lstlisting}
\begin{lemma}[Attempt: Bet 1 boundary-term integrability collapses into C2/U-decay (pivot trigger)]\label{lem:bet1_boundary_integrability_pivot_to_C2}
Fix $b\in\Sbb^2$, $t\le 0$, and write $A(r):=A_b^\infty(r,t)$ and $\mathcal F(r):=\mathcal F_b(r,t)$ as in
\(\eqref{eq:Ab-PDE}\)--\(\eqref{eq:Ab-forcing}\) in \texttt{navier-dec-12-rewrite.tex}.
Define the boundary term
\[
B(r):=(-A(r))\,(r^2A'(r)).
\]

\smallskip
\noindent\textbf{Bet 1 goal.}
Bet 1 aims to prove $B\in L^1((1,\infty))$ and $B'\in L^1((1,\infty))$, so that $B(r)\to 0$ as $r\to\infty$ by the already-formalized
``zero-skew at infinity'' lemma (Lean: \texttt{SkewHarmGate.zeroSkewAtInfinity\_of\_integrable}).

\smallskip
\noindent\textbf{First obstruction: the $B'$ structure exposes tail interaction terms.}
Using the PDE \eqref{eq:Ab-PDE}, the alternative manuscript records the pointwise identity (Remark ``Structure of $B'$''):
\[
B'(r)
=-|A'(r)|^2\,r^2\;-\;6|A(r)|^2\;-\;r^2A(r)\,(\partial_tA)(r,t)\;+\;r^2A(r)\,\mathcal F(r).
\tag{$\clubsuit$}
\]
Hence, even if one has good control of the coercive integrands $|A'|^2r^2$ and $|A|^2$, proving the \emph{absolute} integrability
\(\int_1^\infty |B'(r)|dr<\infty\) forces control of the two tail-interaction terms
\[
\int_1^\infty r^2|A(r)|\,|\partial_tA(r,t)|\,dr,
\qquad
\int_1^\infty r^2|A(r)|\,|\mathcal F(r)|\,dr.
\tag{$\heartsuit$}
\]

\smallskip
\noindent\textbf{Where the pivot trigger fires.}
Controlling the forcing-pairing term in $(\heartsuit)$ (and similarly controlling $\partial_tA$ through the PDE) requires a scale-consistent
bound that suppresses the tail/harmonic/affine contribution to $\mathcal F$ at large radii.
This is exactly the same analytic bottleneck as in Lemma~\ref{lem:familyA_pivot_to_C2}:
the only mechanism available in the current manuscript framework that yields such a forcing suppression is the
$\sigma$-decomposition plus a global tail/decay gate, i.e. vanishing positive injection
\(\iint \rho^{3/2}\sigma_+\) (C2/U-decay/RTD).

Therefore, the Bet 1 route (boundary-term integrability) also collapses into controlling \(\iint \rho^{3/2}\sigma_+\),
certifying the \textbf{P-C2 pivot trigger}.
\end{lemma}
\end{lstlisting}


\subsubsection{3.2 Candidate family B: tail strain moment energy}


The obstruction is \(S(0,t)\) (matrix). Consider:


\begin{itemize}
\item \(J(t) := \|S(0,t)\|_{\mathrm{op}}^2\) or \(\|S(0,t)\|_{\mathrm{F}}^2\).
\end{itemize}


\textbf{Bet 2 goal in this family:} show \(J(t)\) cannot be persistently nonzero over \((-\infty,0]\) because it forces positive stretching at maximizers; combine with \texttt{\detokenize{rem:runningmax-injection-constraint}} + a scale‑critical “finite total cost” statement to contradict infinite history.


This family naturally ties to RS “finite budget over infinite history”.


\subsubsection{3.3 Candidate family C: self‑stretching functional \(I[\omega]\)}


Use:


\begin{itemize}
\item \(I[\omega] := \int_{\mathbb R^3} (\omega\cdot\nabla u)\cdot\omega\,dx\),
  whose sign is computed for a canonical ℓ=2 toroidal model (\texttt{\detokenize{prop:l2-selfstretch-example}}).
\end{itemize}


\textbf{Bet 2 goal in this family:} prove a \emph{robust} implication “nonzero ℓ=2 tail ⇒ \(I[\omega^\infty(t)]\ge c>0\) on many times”, then relate \(I\) to an evolution constraint coming from running‑max normalization.


This route is risky unless we can make the sign robust beyond one model profile.



\bigskip\hrule\bigskip


\subsection{3.4 Strawman commitment (default) — choose Family B first}


Unless we explicitly decide otherwise, we will treat \textbf{Family B (tail strain energy)} as the
default Bet‑2 path, because:


\begin{itemize}
\item it is exactly the RM2 obstruction object (via \texttt{\detokenize{lem:tail-strain-formula}} / \texttt{\detokenize{cor:RM2-equivalence}});
\item it is naturally “budget visible” through stretching at top vorticity level
  (\texttt{\detokenize{rem:runningmax-injection-constraint}});
\item it avoids relying on robustness of a model-profile sign computation.
\end{itemize}


This is a \emph{plan commitment}, not a claim that the proof works yet.



\bigskip\hrule\bigskip


\subsection{3.5 Mechanism skeleton (Family B): “either-sign costs money”}


This is the cleanest RS-shaped mechanism we can currently articulate in fully PDE terms:


\begin{enumerate}
\item \textbf{Export persists ⇒ tail strain persists.}\\
   In the fixed-frame language, “export” is exactly the affine/harmonic obstruction coefficient,
   i.e. the tail strain moment \(S_{tail}(0,t)\) from \texttt{\detokenize{lem:tail-strain-formula}}.
\item \textbf{Tail strain persists ⇒ stretching at top vorticity level has persistent size/sign.}\\
   The stretching scalar is \(\sigma=(S\xi\cdot\xi)\). Bet 2 needs a bridge from \texttt{\detokenize{S_tail}} to \texttt{\detokenize{σ}}
   on a set where \(\rho\approx 1\). This is the hard geometric step.
\item \textbf{Either sign of stretching on \(\{\rho\approx 1\}\) forces a compensating payment.}
\end{enumerate}


\begin{itemize}
\item If \(\sigma\) is persistently \textbf{positive} near top level, the running‑max maximizer inequality
     (\texttt{\detokenize{rem:runningmax-injection-constraint}}) forces payment by either \texttt{\detokenize{|\nabla ξ|^2}} or \texttt{\detokenize{-Δρ}}.
\item If \(\sigma\) is persistently \textbf{negative} near top level, the superlevel-set selection mechanism
     (\texttt{\detokenize{rem:sigma-minus-paid-by-diffusion}}) says cancellation by \(\sigma_-\) is only possible if one pays
     a \textbf{transition-band diffusion cost} in \texttt{\detokenize{|∇(ρ^{3/4})|^2}}.
\end{itemize}


\begin{enumerate}
\item \textbf{Infinite history + persistent export ⇒ infinite total payment.}\\
   This is the literal “finite budget over infinite history” punchline.
\item \textbf{We still need a finite-payment bound in the running‑max setting.}\\
   The manuscript currently obtains such a bound for C2 using an additional global gate (U‑decay).
   Bet 2 succeeds only if we can bound the relevant payment \emph{without} reimporting that same gate.
\end{enumerate}


\textbf{What this skeleton buys:} it makes the missing research lemma(s) explicit:


\begin{itemize}
\item a bridge from \texttt{\detokenize{S_tail}} to a persistent lower bound on \texttt{\detokenize{σ}} on top-level regions, and
\item a finite-budget bound that prevents “infinite payment” without assuming U‑decay.
\end{itemize}



\bigskip\hrule\bigskip


\subsection{3.6 Technical enabler: “maximizer” arguments on \(\mathbb R^3\) without decay}


Many key inequalities (e.g. \texttt{\detokenize{rem:runningmax-injection-constraint}}) are stated at \textbf{maximizers} of
\(\rho(\cdot,t)\). On \(\mathbb R^3\), even for smooth bounded \(\rho\), the supremum need not be attained.


To keep Bet 2 rigorous, we commit to one of these techniques whenever a “maximizer” is invoked:


\begin{itemize}
\item \textbf{(Max‑A) Penalized maximizer (attainment by coercive perturbation).}\\
  For fixed \(t\), define \(\rho_\varepsilon(x,t):=\rho(x,t)-\varepsilon|x|^2\). Then \(\rho_\varepsilon(\cdot,t)\)
  attains a maximum at some \(x_{\varepsilon,t}\). At that point:
  \[
    \nabla\rho(x_{\varepsilon,t},t)=2\varepsilon x_{\varepsilon,t},\qquad
    \Delta\rho(x_{\varepsilon,t},t)\le 6\varepsilon,
  \]
  so any “maximum-point” PDE inequality holds up to explicit \(O(\varepsilon)\) errors.
\item \textbf{(Max‑B) Approximate maximizers / viscosity selection.}\\
  Pick \(x_t\) with \(\rho(x_t,t)\ge \sup_x\rho(\cdot,t)-\varepsilon\) and apply a standard selection lemma
  to obtain approximate first/second derivative control at \(x_t\). This is slightly more abstract than (Max‑A)
  but equivalent in spirit.
\end{itemize}


\textbf{Why this matters for Bet 2:} it avoids silently assuming the (RM2‑blocked) pointwise statement
“\(\sup_x\rho(\cdot,t)\) is attained at some \(x_t\)” and keeps the running‑max budget inequalities usable
in Mode (M‑seq).


\textbf{Deliverable to add early (S1/S2):} a short TeX lemma formalizing (Max‑A) and stating the
approximate version of \texttt{\detokenize{rem:runningmax-injection-constraint}} with explicit \(\varepsilon\) errors.



\bigskip\hrule\bigskip


\subsection{3.7 What outputs actually suffice to close RM2 (don’t over-aim)}


The manuscript itself records multiple sufficiency thresholds:


\begin{itemize}
\item \textbf{RM2 (subsequence) suffices for extraction:}\\
\texttt{\detokenize{thm:RM2-from-tail-L2}} shows \texttt{\detokenize{S(0,·) ∈ L²_t}} implies boundedness of \texttt{\detokenize{S(0,t_k)}} along a subsequence
  of times \(t_k\uparrow 0\). For some contradiction schemes, this subsequence control is already enough.
\item \textbf{RM2 (uniform-in-time) is the cleanest closure:}\\
  uniform boundedness of \texttt{\detokenize{S(0,t)}} (or equivalently of \texttt{\detokenize{Σ_b^{1,∞}(t)}} uniformly in \(t\)) gives the
  crisp “fixed-frame compactness gate is closed” statement.
\end{itemize}


\textbf{Plan consequence:} we treat the “weak” outcome \(S\in L^2_t\) as a legitimate intermediate target,
and only upgrade to uniform-in-time if the argument naturally provides it.



\bigskip\hrule\bigskip


\subsection{3.8 Bridge menu (Family B): how \texttt{S\_tail} becomes “budget-visible”}


This section makes the \textbf{single hardest step} explicit: converting a persistent affine/tail strain
coefficient into a persistent \textbf{cost} that the running‑max framework can charge.


We treat two bridge routes as first‑class options (we may need both):


\subsubsection{Bridge Route (B‑amp): amplitude / stretching scalar route}


Goal: show persistent \texttt{\detokenize{S_tail}} forces persistent stretching magnitude on the top-level set,
so the maximizer inequality + superlevel-selection payment machinery triggers.


Key objects:


\begin{itemize}
\item \(\sigma = (S\xi\cdot\xi)\) (normal component of \(S\xi\))
\item running‑max maximizer constraint (TeX \texttt{\detokenize{rem:runningmax-injection-constraint}})
\item signed injection / diffusion payment machinery (TeX \texttt{\detokenize{lem:superlevel-selection}}, \texttt{\detokenize{rem:sigma-minus-paid-by-diffusion}})
\end{itemize}


Minimal deliverable for this route:


\begin{itemize}
\item a lemma of the form:
  \[
  \|S_{tail}(0,t)\|\ge \varepsilon\ \Rightarrow\ \exists\ \text{(top-level region)}\ \text{with}\ |\sigma|\ge c\varepsilon
  \]
  or else \texttt{\detokenize{|∇ξ|^2}} is large (which is itself a paid cost via the same machinery).
\end{itemize}


\subsubsection{Bridge Route (B‑dir): direction equation forcing route}


Observation: even if \(\sigma\) is tuned small by choosing \(\xi\), the tangential component
\texttt{\detokenize{H_sing = P_ξ(Sξ)}} can remain large. The direction equation is:


\[
\partial_t \xi - \Delta \xi + u\cdot\nabla\xi = |\nabla\xi|^2\,\xi + H,
\qquad H_{\mathrm{sing}} = P_\xi(S\xi).
\]


This suggests a different contradiction shape:


\begin{itemize}
\item persistent \texttt{\detokenize{S_tail}} forces persistent \texttt{\detokenize{H_sing}} (unless \(\xi\) aligns with a null eigenvector),
\item but the direction-locking mechanism demands small forcing at small scales,
\item therefore the tail/affine mode cannot persist.
\end{itemize}


\textbf{Why this route is attractive:} it attacks the “escape hatch” where \(\sigma\approx 0\) pointwise.


\textbf{Hard edge case:} if \(S_{tail}\) has an actual null eigenvector and \(\xi\) aligns with it,
then both \(\sigma\) and \(H_{\mathrm{sing}}\) contributed by that component can vanish. In that case,
we must argue we have effectively reduced to a 2D/degenerate scenario (and then other gates like \texttt{\detokenize{E}}
may become relevant). We treat this as a named case in Section 3.9.



\bigskip\hrule\bigskip


\subsection{3.9 Case-analysis scaffold (so the plan doesn’t stall on one geometry)}


Fix a time \(t\) and write \(S:=S_{tail}(0,t)\in\mathbb R^{3\times 3}_{sym,0}\).


We expect Bet 2 to split into cases depending on:


\begin{itemize}
\item eigenstructure of \(S\) (full rank vs one small/zero eigenvalue),
\item behavior of \(\xi\) on a top-level region \(\{\rho\ge 1-\eta\}\).
\end{itemize}


\subsubsection{Case C1: \(S\) has no small eigenvalue (uniformly nondegenerate)}


Heuristic: \(S\xi\) is never small, so either \(|\sigma|\) or \(\|P_\xi(S\xi)\|\) must be \(\gtrsim \|S\|\)
on any region where \(\xi\) is not fine-tuned. This feeds either Bridge (B‑amp) or (B‑dir).


\subsubsection{Case C2: \(S\) has a small/zero eigenvalue and \(\xi\) aligns to kill the effect}


This is the “stealth” scenario: if \(\xi\) is near the null eigenvector on \(\{\rho\approx 1\}\),
then both \(\sigma\) and \(H_{\mathrm{sing}}\) contributed by \(S\) can be small.


In this case, Bet 2 should explicitly attempt one of:


\begin{itemize}
\item (C2‑a) show that maintaining such alignment over a top-level region forces large \texttt{\detokenize{|∇ξ|^2}}
  (recognition strain payment), or
\item (C2‑b) show the flow has effectively reduced dimensionality (and redirect to \texttt{\detokenize{E}}-type Liouville input),
  or
\item (C2‑c) show the affine/tail coefficient must still decay in time (weak target \(L^2_t\)).
\end{itemize}


\subsubsection{Case C3: \(\xi\) varies to “track” a small-\(\sigma\) cone}


Even if \(S\) is full rank, the set \(\{\xi:\xi\cdot S\xi\approx 0\}\) is a codimension‑1 curve on \(S^2\).
If \(\xi\) tries to live near that set on \(\{\rho\approx 1\}\), it likely forces large \texttt{\detokenize{|∇ξ|^2}}.
This is the most RS‑native case: “avoid paying injection by paying recognition strain instead.”


\textbf{Plan point:} we should treat “large \texttt{\detokenize{|∇ξ|^2}} on top-level sets” as a \emph{successful outcome}
(it is a paid cost in the same budget ledger).



\bigskip\hrule\bigskip


\subsection{3.10 “Ledger” view (RS → PDE): what is the budget, what is the payment?}


Bet 2 becomes much easier to reason about if we keep an explicit accounting table.


\subsubsection{3.10.1 Budget invariants we actually have}


These are the items we can use without importing extra global decay gates:


\begin{itemize}
\item \textbf{Running‑max cap:} \(0\le \rho \le 1\) (and in the idealized statement, \(\sup_x\rho(\cdot,t)=1\)).
\item \textbf{Local smoothness:} bounded vorticity implies local regularity on compact cylinders (TeX references around \texttt{\detokenize{lem:Linfty-vort-smooth}}).
\item \textbf{Scale‑critical vorticity mass control:} the running‑max normalization gives uniform control of
  \(\iint_{Q_r} |\omega|^{3/2}\) at small scales (TeX \texttt{\detokenize{lem:omega32-runningmax-automatic}}), which makes various lower-order
  cutoff/time boundary errors \(O(r^3)\) (TeX \texttt{\detokenize{rem:rho32-errors-small-scale}}).
\end{itemize}


\subsubsection{3.10.2 Payment terms the PDE forces us to pay}


On top-level/superlevel regions, the localized \(\rho^{3/2}\) identities charge:


\begin{itemize}
\item \textbf{Recognition strain payment:} \(\iint \rho^{3/2}|\nabla\xi|^2\) (direction coherence cost)
\item \textbf{Magnitude diffusion payment:} \(\iint |\nabla(\rho^{3/4})|^2\) (band/transition cost)
\item \textbf{Concavity payment at peaks:} \(-\Delta\rho\) at (approximate) maximizers (sharp peak cost)
\end{itemize}


These are exactly the “cost channels” in \texttt{\detokenize{rem:runningmax-injection-constraint}} and \texttt{\detokenize{rem:sigma-minus-paid-by-diffusion}}.


\subsubsection{3.10.3 What Bet 2 must prove in ledger language}


To get a contradiction, Bet 2 needs two ingredients:


\begin{itemize}
\item \textbf{(L‑lower)} persistent export ⇒ a \emph{uniform lower bound} on payment over a time set of positive measure
  (e.g. per-cylinder payment ≥ c·ε).
\item \textbf{(L‑upper)} a \emph{finite budget bound} that prevents paying that amount over an infinite time interval.
\end{itemize}


The current manuscript closes analogous loops for C2 using U‑decay; Bet 2 is a search for an alternative
upper bound (or an alternative contradiction mechanism) that does not reimport U‑decay.


\subsubsection{3.10.4 Candidate “finite payment” upper bounds (L‑upper) (where Bet 2 can win or fail)}


This is the main place Bet 2 risks collapsing into C2/U‑decay. We keep candidate upper bounds explicit and
grade them by plausibility and circularity.


\textbf{Target shape:} an inequality of the form


\[
\int_{t_0-r^2}^{t_0}\int_{B_r(x_0)} \text{PaymentDensity} \;\le\; \mathbf{Budget}(r)
\]


with \(\mathbf{Budget}(r)\) scaling subcritically (e.g. \(O(r^3)\) or Carleson-small as \(r\downarrow 0\)),
or a bound that forces the payment to be large only on a small fraction of times.


Candidate sources:


\begin{itemize}
\item \textbf{(U‑1) “B‑gate controls payment” (dream outcome).}\\
  Show that large top-level payment (either \(\rho^{3/2}|\nabla\xi|^2\) or \(|\nabla(\rho^{3/4})|^2\) on the band)
  forces a comparably large scale-critical vorticity mass \(\iint_{Q_r}|\omega|^{3/2}\), contradicting the already‑automatic
  running‑max bound (\texttt{\detokenize{lem:omega32-runningmax-automatic}}).
\item \textbf{Pros:} would make Bet 2 truly independent of C2/U‑decay.
\item \textbf{Cons:} nontrivial; likely false without extra structure, but worth testing.
\item \textbf{(U‑2) “Direction equation damping bounds \texttt{\detokenize{|∇ξ|}} over infinite history.”}\\
  Use the ancient direction equation energy/Bernstein mechanism (see the proof style around \texttt{\detokenize{thm:global-directional-locking}})
  to obtain a genuine bound on \(\int_{-\infty}^0\!\!\int |\nabla\xi|^4\) (or similar), then convert it into a bound on
  the ledger term \(\rho^{3/2}|\nabla\xi|^2\) on top-level regions.
\item \textbf{Pros:} aligns with the RS idea “twist can’t persist in an ancient bounded profile.”
\item \textbf{Cons:} must avoid assuming tail smallness of forcing (otherwise circular with RM2U itself).
\item \textbf{(U‑3) “Band diffusion payment is controlled by a monotone quantity.”}\\
  Try to bound \(\iint |\nabla(\rho^{3/4})|^2\) on cylinders by a quantity that is automatically bounded in the running‑max setting
  (e.g. via the \(\rho^{3/2}\) identity plus small-scale error control in \texttt{\detokenize{rem:rho32-errors-small-scale}}).
\item \textbf{Pros:} directly targets the payment that appears in \texttt{\detokenize{rem:sigma-minus-paid-by-diffusion}}.
\item \textbf{Cons:} current TeX remarks indicate this is exactly where C2 gets stuck without extra global control.
\item \textbf{(U‑4) “Pre-limit finite-window contradiction.”}\\
  Work purely in Mode (M‑seq) on a fixed window \texttt{\detokenize{[-r^2,0]}}: show that persistent export forces a payment
  that is too large compared to what the \emph{pre-blow-up} solution can supply on that physical time window.
\item \textbf{Pros:} uses the fact the rescaled window corresponds to a vanishing physical time interval near \(t_k\).
\item \textbf{Cons:} requires careful scaling bookkeeping; but it may be the cleanest non-circular way to produce an upper bound.
\end{itemize}


\textbf{Default recommendation for the next execution attempt:} start with \textbf{(U‑4)}.


\begin{itemize}
\item It matches the safest logical mode (M‑seq),
\item it has a clear “finite-window” structure (one cylinder, one contradiction),
\item it makes circularity easier to detect early.
\end{itemize}


\textbf{Fast filter (to avoid circularity):}


\begin{itemize}
\item If your chosen upper bound requires controlling \(\iint \rho^{3/2}\sigma_+\), you’ve rediscovered \textbf{C2} (pivot trigger P‑C2).
\item If it requires UEWE/coercive \(\ell=2\) control, you’ve rediscovered \textbf{RM2U} (pivot trigger P‑UEWE).
\item If it requires the fixed-frame limit without affine bookkeeping, you’ve hit \textbf{RM2} again (pivot trigger P‑compactness).
\end{itemize}



\bigskip\hrule\bigskip


\subsection{3.11 Pre-limit formulation (Mode M‑seq): define the obstruction without taking the RM2-blocked limit}


To keep Bet 2 non-circular, we should be able to state it directly on the running‑max rescaled sequence
\(\omega^{(k)}\) without passing to a fixed-frame ancient limit.


\subsubsection{3.11.1 Define a “tail strain coefficient” for each rescaled solution}


Using the explicit Biot–Savart tail strain formula (TeX \texttt{\detokenize{lem:tail-strain-formula}}), for each rescaled vorticity
\(\omega^{(k)}(\cdot,t)\) define a tail strain moment at the origin:


\begin{itemize}
\item split \(\omega^{(k)} = \omega^{(k)}_{\le 1} + \omega^{(k)}_{>1}\) by a fixed radial cutoff,
\item let \(S^{(k)}_{tail}(0,t)\) be the symmetric gradient at the origin of the Biot–Savart velocity generated by \(\omega^{(k)}_{>1}\).
\end{itemize}


This definition is meaningful without any compactness assumption and matches the “affine/harmonic mode”
coefficient used in the fixed-frame compactness discussion.


\subsubsection{3.11.2 What “export persists” means in M‑seq form}


Instead of a statement about a limit object, we can formulate persistence as:


\begin{itemize}
\item there exists \(\varepsilon>0\), a scale \(r_0>0\), and infinitely many indices \(k\) such that
  on a time set \(E_k\subset[-r_0^2,0]\) of measure \(|E_k|\ge c_0 r_0^2\), we have
  \(\|S^{(k)}_{tail}(0,t)\|\ge \varepsilon\) for all \(t\in E_k\).
\end{itemize}


This is deliberately “finite-window” and scale-normalized, so we can run a contradiction on a single cylinder.


\subsubsection{3.11.3 Target contradiction in M‑seq form}


Show that such persistence forces (for \(k\) large):


\begin{itemize}
\item either the running‑max normalization is violated at some later time (sup \(ρ^{(k)}\) exceeds 1),
\item or a paid cost term becomes too large on that cylinder in a way that contradicts an available scale-critical bound.
\end{itemize}


This is the cleanest way to avoid using the RM2‑blocked fixed-frame limit.


\subsection{4. Multi‑session execution roadmap (rigorous, checkable)}


\subsubsection{Session S0 — Lock the configuration (this is the “last planning step”)}


\textbf{Goal:} eliminate ambiguity so the next sessions produce proofs, not more branching.


\textbf{Decisions to lock (with defaults):}


\begin{itemize}
\item \textbf{Mode}: default \textbf{(M‑seq)} (pre-limit contradiction; avoids RM2‑blocked compactness)
\item \textbf{Target strength}: default \textbf{Weak} (\texttt{\detokenize{S ∈ L²_t}} or finite-window “no persistence” contradiction)
\item \textbf{Export hypothesis}: default \textbf{E3 in finite-window form} (Section 3.11.2)
\item \textbf{Bridge route}: default \textbf{(B‑amp)} first; keep \textbf{(B‑dir)} as the explicit escape-hatch handler
\item \textbf{L‑upper candidate}: default \textbf{U‑4} (pre-limit finite-window contradiction)
\end{itemize}


\textbf{Acceptance test:} a 5-line “configuration block” is written at the top of the scratchpad/proof notes
for the next session, and we do not change it mid-session without triggering a pivot note.


\subsubsection{Session S1 — Specify the missing proposition and the export hypothesis precisely}


\textbf{Deliverables:}


\begin{itemize}
\item Write down a concrete TeX proposition statement to replace the missing \texttt{\detokenize{prop:l2-instability}},
  in \textbf{two versions} (strong/weak). Use the tail strain moment \texttt{\detokenize{S(0,t)}} from \texttt{\detokenize{lem:tail-strain-formula}}:
\item \textbf{(S1‑Strong, strawman):} If the running‑max ancient element has a nontrivial ℓ=2 tail strain
    moment at some time, then it violates the running‑max budget. Equivalently:
    \[
      \forall t\le 0,\quad S(0,t)=0.
    \]
\item \textbf{(S1‑Weak, strawman):} The ℓ=2 tail strain moment is forced into an integrability class:
    \[
      \int_{-\infty}^0 \|S(0,t)\|^2\,dt < \infty,
    \]
    which closes RM2 along a subsequence by \texttt{\detokenize{thm:RM2-from-tail-L2}}.
\end{itemize}


These are “strawman targets” that will likely need hypotheses; the point of S1 is to make those
  hypotheses explicit instead of implicit.


\paragraph{S1.1 Copy‑paste‑ready TeX proposition block (strawman)}


This is the exact “placeholder replacement” we want to end up with in the manuscript. It is
intentionally stated in a way that can be weakened/parameterized later.


\begin{lstlisting}
\begin{proposition}[Dynamical instability of the $\ell=2$ tail (strawman)]\label{prop:l2-instability}
Let $(u^\infty,p^\infty)$ be a running-max/vorticity-normalized ancient element on $\mathbb R^3\times(-\infty,0]$,
and let $S_{tail}(0,t)$ denote the $\ell=2$ tail strain moment at the origin defined by Lemma~\ref{lem:tail-strain-formula}
(equivalently, the affine/harmonic obstruction coefficient in the fixed-frame compactness step).
Then:
\begin{enumerate}
\item[(Strong)] $S_{tail}(0,t)\equiv 0$ for all $t\le 0$.
\item[(Weak)] Alternatively (sufficient for RM2 subsequential closure), one has
  $\int_{-\infty}^0 |S_{tail}(0,t)|^2\,dt < \infty$.
\end{enumerate}
\end{proposition}
\end{lstlisting}

\paragraph{S1.2 Fail-fast proof attempt (where the real bottleneck appears).}

\begin{lstlisting}
\begin{lemma}[Attempt: proving \texttt{prop:l2-instability} collapses into C2/U-decay (pivot trigger)]\label{lem:attempt_prop_l2_instability_pivot_to_C2}
We attempt to prove Proposition~\ref{prop:l2-instability} from the running-max axioms alone
(bounded vorticity + local scale-critical $L^{3/2}$ control), without assuming U-decay/RTD/UEWE.

\smallskip
\noindent\textbf{Step 0 (export hypothesis).}
Assume (E3) in measurable form: there exist $\varepsilon_0>0$ and a measurable set $E\subset(-\infty,0]$ with $|E|>0$ such that
\[
\|S_{tail}(0,t)\|_{\mathrm{op}}\ \ge\ \varepsilon_0\qquad\forall t\in E.
\tag{$\mathsf{Export}$}
\]

\smallskip
\noindent\textbf{Step 1 (tail strain is affine forcing on the core).}
By the brick lemma \texttt{lem:tail\_biot\_savart\_affine\_core}, for each fixed $t$ and each large $R$,
the far-field velocity induced by $\omega^\infty\mathbf 1_{\{|y|\ge R\}}$ is harmonic on $B_1$ and admits an affine approximation
$a(t)+A_R(t)x$ with $A_R(t)$ symmetric trace-free.
Moreover $A_R(t)$ converges (as $R\to\infty$) to the tail strain matrix $S_{tail}(0,t)$ in operator norm.
Thus $(\mathsf{Export})$ implies a persistent affine strain of size $\gtrsim\varepsilon_0$ acting on the core.

\smallskip
\noindent\textbf{Step 2 (bridge to a budget-visible lower bound).}
To contradict running-max, we need to convert persistent affine strain into a persistent lower bound on
the \emph{positive stretching injection} on a set where $\rho=|\omega|$ is near its top level:
\[
\iint_{Q_r(z_t)} \rho^{3/2}\,\sigma_+\ \gtrsim\ c(\varepsilon_0)\,r^5
\quad\text{for many }t\in E\text{ and some radii }r\downarrow 0.
\tag{$\mathsf{LB}$}
\]
The difficulty is geometric: $\sigma=\xi\cdot S\,\xi$, so the affine tail contribution is $\xi\cdot S_{tail}(0,t)\,\xi$.
A lower bound of the form $(\mathsf{LB})$ requires (at minimum) control of:
\begin{itemize}
\item alignment of $\xi$ on the top band with a positive-eigenvector direction of $S_{tail}(0,t)$; and
\item domination of the tail contribution over the near-field/stretching remainder on the relevant cylinders.
\end{itemize}
Any known way to guarantee these (e.g. small weighted direction energy on the top band, or vanishing tail/harmonic remainder in $\sigma$)
re-introduces the global tail/decay mechanism.

\smallskip
\noindent\textbf{Step 3 (finite budget upper bound is exactly C2).}
Even if one grants $(\mathsf{LB})$, turning it into a contradiction requires an \emph{upper} bound on the total available payment
(finite budget over infinite history). In the manuscript, the only mechanism that upgrades the pointwise maximizer constraint
\[
\sigma(x_t,t)\ \le\ |\nabla\xi(x_t,t)|^2-\Delta\rho(x_t,t)
\qquad\text{(Remark \texttt{rem:runningmax-injection-constraint})}
\]
into a uniform small-scale/infinite-history integral bound is Theorem C2-closure, which is conditional on U-decay/RTD.
Equivalently: bounding the accumulated positive injection
\(\sup_{z_0}\iint_{Q_r(z_0)}\rho^{3/2}\sigma_+\)
without U-decay is precisely the C2 bottleneck.

\smallskip
\noindent\textbf{Pivot trigger.}
The proof attempt therefore fires the same pivot trigger as the K-ODE and Bet-1/Family-A diagnostics:
proving \texttt{prop:l2-instability} from running-max alone collapses into controlling \(\iint \rho^{3/2}\sigma_+\),
i.e. into C2/U-decay/RTD (pivot trigger P-C2).
\end{lemma}
\end{lstlisting}


\textbf{Notes:}


\begin{itemize}
\item In the first draft, we may need to add hypotheses (e.g. a quantified “export persists” condition,
  or a mild time-regularity/integrability assumption). The key is that \emph{any} added hypotheses must
  be stated explicitly and recorded as “this is what Bet 2 really needs.”
\item If we operate in Mode (M‑seq), the proposition should be stated for the running‑max rescaled sequence
  (uniform in \texttt{\detokenize{k}}) and then passed to the limit only after the contradiction is obtained.
\item Pick the exact “export persists” hypothesis (choose one and stick to it):
\item (E1) \texttt{\detokenize{¬TailFluxVanish}} for the RM2U coefficient profile,
\item (E2) \texttt{\detokenize{limsup_{r→∞} |tailFlux(r,t)| ≥ ε}} on a set of times,
\item (E3) a lower bound on \texttt{\detokenize{‖S(0,t)‖}} on a time interval.
\end{itemize}


\textbf{Default choice for Family B:} use (E3) in a measurable form:
  there exists \texttt{\detokenize{ε>0}} and a time set \texttt{\detokenize{E ⊂ (-∞,0]}} of positive measure such that
  \texttt{\detokenize{‖S(0,t)‖ ≥ ε}} for \texttt{\detokenize{t ∈ E}}.


This avoids prematurely committing to a particular tail‑flux formulation while still encoding
  “persistent export.”


\textbf{Acceptance test:}


\begin{itemize}
\item New TeX snippet (can live initially in a doc) with a fully quantified proposition statement.
\item A clear mapping to Lean interface:
\item \texttt{\detokenize{Bet2SelfFalsificationHypothesis P}} corresponds to “(E1) implies False” at the time‑slice level.
\end{itemize}


\textbf{Additional acceptance (so Lean doesn’t drift):}


\begin{itemize}
\item Record the \emph{eventual} Lean-facing statement we want, even if it can’t be implemented yet:
  Bet 2 will ultimately need a \textbf{time-parameterized} profile (something like \texttt{\detokenize{A : ℝ → ℝ → ℝ}}),
  but we keep the current Lean interface as a guardrail:
  the final theorem should produce \texttt{\detokenize{Bet2SelfFalsificationHypothesis}} for each fixed time-slice profile.
\end{itemize}


\textbf{New S1 deliverable (to prevent hidden circularity):}


\begin{itemize}
\item Explicitly choose \texttt{\detokenize{Mode (M‑seq)}} or \texttt{\detokenize{Mode (M‑limit‑with‑affine)}} for the rest of Bet 2.
  Default recommendation: start in \textbf{(M‑seq)} until RM2 is closed.
\end{itemize}


\textbf{Refined default recommendation (after deeper audit):}


\begin{itemize}
\item Start in \textbf{Mode (M‑limit‑with‑affine)} if we can extract a limit \emph{modulo} the affine mode
  without needing RM2. This keeps the “obstruction coefficient” explicit and avoids
  “approximate maximum” bookkeeping.
\item Fall back to \textbf{Mode (M‑seq)} only if the limit-with-affine extraction turns out to still
  require the same missing estimates.
\end{itemize}


\subsubsection{Session S2 — Build the “bridge inequality” from export to budget‑visible quantity}


\textbf{Goal:} connect the ℓ=2 tail obstruction to something the running‑max budget controls.


\textbf{Candidate bridges (pick one):}


\begin{itemize}
\item (Br‑S) tail flux / Σ ⇒ tail strain:
\item use \texttt{\detokenize{lem:tail-strain-formula}} + the Σ representation (already explicit in TeX).
\item (Br‑σ) tail strain ⇒ positive stretching at maximizers:
\item quantify how a nonzero affine strain component contributes to \(\sigma(x_t,t)\) at maximizers.
\item (Br‑E) tail flux ⇒ growth of truncated energy \(E_{b,R}\):
\item compute the sign contribution of boundary flux in the \(d/dt\) identity.
\end{itemize}


\textbf{Acceptance test:}


\begin{itemize}
\item One inequality of the form:
  \[
  \text{(export)} \;\Rightarrow\; \text{(lower bound on a budget-visible term)}
  \]
  with explicit constants and hypotheses stated.
\end{itemize}


\textbf{Strawman sub-lemma list (Family B):}


\begin{itemize}
\item \textbf{(Br‑σ1)} If \texttt{\detokenize{‖S(0,t)‖ ≥ ε}} then there exists a direction \texttt{\detokenize{b}} such that
  \(b\cdot S(0,t)b \ge c\,\varepsilon\).
\item \textbf{(Br‑σ2)} The direction \texttt{\detokenize{b}} can be aligned with a maximizer direction for the vorticity direction field
  in a way that produces a lower bound on stretching scalar \texttt{\detokenize{σ(x_t,t)}} on a nontrivial region.
  (This is where RS/CPM “local test” design matters: pick the right “test direction” tied to the maximizer.)
  These are expected to be the hard geometric steps.
\end{itemize}


\textbf{New explicit sub-lemma (so we can actually use \texttt{\detokenize{rem:runningmax-injection-constraint}} rigorously):}


\begin{itemize}
\item \textbf{(Br‑max)} Replace “maximizer” by a penalized maximizer (Section 3.6, Max‑A) and show the running‑max
  injection inequality holds with controlled \(O(\varepsilon)\) errors. This is required if we operate on
  \(\mathbb R^3\) without decay/tightness assumptions.
\end{itemize}


\paragraph{S2.1 Copy‑paste TeX lemma (penalized maximizer version; first lemma to actually prove)}


\begin{lstlisting}
\begin{lemma}[Penalized maximizer version of the running-max injection constraint (proved)]\label{lem:runningmax-injection-penalized}
Let $(u,p)$ be smooth on $\mathbb R^3\times I$ and write $\omega=\rho\xi$ on $\{\rho>0\}$ with $\rho=|\omega|\le 1$.
Fix $t\in I$ and $\varepsilon>0$, and define $\rho_\varepsilon(x):=\rho(x,t)-\varepsilon|x|^2$.
Let $x_{\varepsilon,t}$ be a maximizer of $\rho_\varepsilon$.
\emph{Assume additionally} the time one-sided constraint
\[
\partial_t \rho(x_{\varepsilon,t},t)=\partial_t\rho_\varepsilon(x_{\varepsilon,t},t)\le 0
\tag{H-t}
\]
(e.g. $t$ is a running-max time for $\sup_x\rho_\varepsilon(\cdot,t)$).
Then evaluating the amplitude equation \eqref{eq:amplitude} at $(x_{\varepsilon,t},t)$ and using
$\nabla\rho(x_{\varepsilon,t},t)=2\varepsilon x_{\varepsilon,t}$ gives the precise bound
\[
\sigma(x_{\varepsilon,t},t)
\ \le\ |\nabla\xi(x_{\varepsilon,t},t)|^2
\ -\ \frac{\Delta\rho(x_{\varepsilon,t},t)}{\rho(x_{\varepsilon,t},t)}
\ +\ \frac{2\varepsilon\,u(x_{\varepsilon,t},t)\cdot x_{\varepsilon,t}}{\rho(x_{\varepsilon,t},t)}.
\tag{*}
\]
In particular, on a top-level set where $\rho(x_{\varepsilon,t},t)\ge 1-\eta$ \emph{and} under a drift-control hypothesis
\[
|u(x_{\varepsilon,t},t)\cdot x_{\varepsilon,t}|\le M
\tag{H-u}
\]
one obtains the schematic form
\[
\sigma(x_{\varepsilon,t},t)\ \le\ |\nabla\xi(x_{\varepsilon,t},t)|^2\ -\ \frac{\Delta\rho(x_{\varepsilon,t},t)}{\rho(x_{\varepsilon,t},t)}\ +\ C_{\eta,M}\,\varepsilon,
\]
which is the penalized-maximizer analogue of `rem:runningmax-injection-constraint`.
\end{lemma}

\begin{proof}
Because $\rho(\cdot,t)$ is continuous and bounded while $-\varepsilon|x|^2\to-\infty$ as $|x|\to\infty$,
the penalized function $\rho_\varepsilon(\cdot,t)$ attains a maximizer $x_{\varepsilon,t}$.
At $x_{\varepsilon,t}$ we have $\nabla\rho_\varepsilon=0$ and $\Delta\rho_\varepsilon\le 0$, hence
\[
\nabla\rho(x_{\varepsilon,t},t)=2\varepsilon x_{\varepsilon,t}.
\]
On $\{\rho>0\}$ the amplitude equation is
\[
\partial_t\rho + u\cdot\nabla\rho - \Delta\rho \;=\; \rho(\sigma-|\nabla\xi|^2).
\]
Evaluate at $(x_{\varepsilon,t},t)$ and use (H-t) to bound $\partial_t\rho(x_{\varepsilon,t},t)\le 0$:
\[
\rho\,\sigma \;=\; \rho|\nabla\xi|^2 + \partial_t\rho + u\cdot\nabla\rho - \Delta\rho
\ \le\ \rho|\nabla\xi|^2 + u\cdot\nabla\rho - \Delta\rho.
\]
Dividing by $\rho(x_{\varepsilon,t},t)>0$ and substituting $u\cdot\nabla\rho=2\varepsilon\,u\cdot x_{\varepsilon,t}$
gives (*). If additionally $\rho\ge 1-\eta$ and $|u\cdot x_{\varepsilon,t}|\le M$, then
\[
\Bigl|\frac{2\varepsilon\,u\cdot x_{\varepsilon,t}}{\rho}\Bigr|
\le \frac{2M}{1-\eta}\,\varepsilon,
\]
yielding the stated schematic bound.
\end{proof}
\end{lstlisting}


\textbf{Why this is the right “first lemma”:}


\begin{itemize}
\item it makes the maximizer argument honest on \(\mathbb R^3\) (no attainment assumptions),
\item it quantifies exactly what errors we must carry through in Mode (M‑seq),
\item it is purely local/PDE and should be provable once the amplitude equation is in place.
\end{itemize}


\paragraph{S2.1.1 Proof sketch (and what hypotheses are \emph{actually} needed)}


The clean way to prove a penalized-maximizer inequality is to apply the amplitude equation to the \textbf{penalized field}
\(\rho_\varepsilon(x,t):=\rho(x,t)-\varepsilon|x|^2\), not to \(\rho\) alone.


\begin{enumerate}
\item \textbf{Penalized PDE.} From \texttt{\detokenize{\eqref{eq:amplitude}}},
\[
(\partial_t + u\cdot\nabla - \Delta)\rho
  = \rho(\sigma-|\nabla\xi|^2).
\]
Since \((\partial_t + u\cdot\nabla - \Delta)(\varepsilon|x|^2)=2\varepsilon\,u\cdot x - 6\varepsilon\),
we obtain
\[
(\partial_t + u\cdot\nabla - \Delta)\rho_\varepsilon
  = \rho(\sigma-|\nabla\xi|^2)\ -\ 2\varepsilon\,u\cdot x\ +\ 6\varepsilon.
\]
\item \textbf{Spatial maximum identities.} At a spatial maximizer \(x_{\varepsilon,t}\) of \(\rho_\varepsilon(\cdot,t)\),
we have
\[
\nabla\rho_\varepsilon(x_{\varepsilon,t},t)=0,\qquad \Delta\rho_\varepsilon(x_{\varepsilon,t},t)\le 0.
\]
Equivalently, \(\nabla\rho(x_{\varepsilon,t},t)=2\varepsilon x_{\varepsilon,t}\) and
\(\Delta\rho(x_{\varepsilon,t},t)\le 6\varepsilon\).
\item \textbf{Where the time-sign enters.} The “running‑max injection constraint” (as used in
\texttt{\detokenize{rem:runningmax-injection-constraint}}) additionally needs a \textbf{one-sided time monotonicity} at the maximizer:
\[
\partial_t\rho_\varepsilon(x_{\varepsilon,t},t)\le 0.
\]
This is \emph{not automatic} from spatial maximality alone; it holds if \(t\) is chosen so that
\(\sup_x\rho_\varepsilon(\cdot,t)\) cannot increase for slightly later times (a “running‑max time” for \(\rho_\varepsilon\)),
or in the true running‑max ancient element setting where \(\sup_x\rho(\cdot,t)=1\) for all \(t\le 0\).
\item \textbf{Conclusion (with explicit error terms).} Under the extra hypothesis \(\partial_t\rho_\varepsilon(x_{\varepsilon,t},t)\le 0\),
evaluate the penalized PDE at \((x_{\varepsilon,t},t)\) and use \(\nabla\rho_\varepsilon=0\), \(\Delta\rho_\varepsilon\le 0\) to get
\[
0 \ge (\partial_t + u\cdot\nabla - \Delta)\rho_\varepsilon
  = \rho(\sigma-|\nabla\xi|^2) - 2\varepsilon\,u\cdot x_{\varepsilon,t} + 6\varepsilon.
\]
Rearranging yields
\[
\sigma(x_{\varepsilon,t},t)
\le |\nabla\xi(x_{\varepsilon,t},t)|^2
\ -\ \frac{\Delta\rho(x_{\varepsilon,t},t)}{\rho(x_{\varepsilon,t},t)}
\ +\ \frac{2\varepsilon\,u(x_{\varepsilon,t},t)\cdot x_{\varepsilon,t}}{\rho(x_{\varepsilon,t},t)}.
\]
To recover a bound of the schematic form in the lemma statement, we keep the concavity term
\(-\Delta\rho/\rho\) explicit, and we need one more input:
either (i) a bound on the drift term \(|u\cdot x_{\varepsilon,t}|\) at the penalized maximizer, or
(ii) a gauge choice/affine subtraction that removes the dangerous linear drift contribution.
\end{enumerate}


\textbf{Bottom line:} the \emph{real} lemma splits into three checkable sub-hypotheses:


\begin{itemize}
\item (H‑t) a time one-sided constraint \(\partial_t\rho_\varepsilon(x_{\varepsilon,t},t)\le 0\),
\item (H‑u) control of the drift correction \(u\cdot x_{\varepsilon,t}\) at penalized maximizers (often via affine gauge),
\item (H‑ρ) a lower bound on \(\rho(x_{\varepsilon,t},t)\) (automatic if we are on a top-level region \(\{\rho\ge 1-\eta\}\)).
\end{itemize}


In Bet 2 Mode (M‑seq), these are exactly the places the plan must be honest about what is “free” versus “gate”.


\paragraph{S2.2 Next lemma targets (beyond S2.1) — turn tail strain into a cost on a cylinder}


Once S2.1 is in place, the next step is to connect \texttt{\detokenize{S_tail}} to a \emph{paid} term on a small cylinder
in a way that is compatible with the ledger (Section 3.10).


\textbf{Lemma candidates (ordered by desirability):}


\begin{itemize}
\item \textbf{(L2‑A) Tail strain persistence ⇒ local strain persistence.}\\
  If \(\|S^{(k)}_{tail}(0,t)\|\ge\varepsilon\) at the origin, show that for \(|x|\le c r\) one has
  \(\|S^{(k)}_{tail}(x,t)\|\ge \varepsilon/2\) provided \(r\) is small enough (quantify using harmonicity of the tail velocity in \(B_1\)).
  This uses only bounded vorticity and the fact the tail kernel is smooth on \(B_1\).
\item \textbf{(L2‑B) Strain persistence ⇒ either stretching or tangential forcing persists on top-level sets.}\\
  On \(\{\rho\ge 1-\eta\}\cap Q_r\), either
  \(|\sigma|=(S\xi\cdot\xi)\) is \(\gtrsim \varepsilon\) on a positive-measure subset, or
  \(\|P_\xi(S\xi)\|\) is \(\gtrsim \varepsilon\) on a positive-measure subset (Bridge menu 3.8 + cases 3.9).
\item \textbf{(L2‑C) If \(\xi\) “tracks the null cone”, then \texttt{\detokenize{|∇ξ|^2}} pays.}\\
  Implement Case C3: if \(\xi\) stays near \(\{\xi:\xi\cdot S\xi\approx 0\}\) on a region, then \(|\nabla\xi|^2\) must be large there.
  This is the cleanest RS‑native “avoid injection ⇒ pay recognition strain” lemma.
\end{itemize}


\textbf{Acceptance test:} one of (L2‑A/B/C) is written as a fully quantified TeX lemma statement (even if hypotheses are “TODO”),
and it is explicitly mapped to which payment channel it triggers in the ledger.


\paragraph{Picked for U‑4 (default): L2‑A as the next concrete lemma statement}


Because U‑4 is a finite-window, pre-limit contradiction, the most “checkable” next lemma is L2‑A:
it upgrades persistence at one point into persistence on a small ball, enabling the cylinder-based payment estimates.


\subparagraph{L2‑A (fully quantified TeX lemma statement; template)}


\begin{lstlisting}
\begin{lemma}[Local persistence of the tail strain (template)]\label{lem:tail-strain-local-persistence}
Let $\Omega:\mathbb R^3\to\mathbb R^3$ be smooth, divergence-free, and supported in $\{|w|>1\}$.
Assume the Biot--Savart velocity
\[
u(x):=\frac{1}{4\pi}\int_{\mathbb R^3}\frac{(x-w)\times \Omega(w)}{|x-w|^3}\,dw
\]
is well-defined and smooth on $B_1(0)$, and let $S(x):=\tfrac12(\nabla u(x)+\nabla u(x)^T)$ be its symmetric gradient.
Assume moreover that $S$ is Lipschitz on $B_{1/2}(0)$ with Lipschitz constant $L$, i.e.
\[
\|S(x)-S(y)\|\le L|x-y| \qquad \forall x,y\in B_{1/2}(0).
\]
Then for any $\varepsilon>0$, if $\|S(0)\|\ge \varepsilon$ we have
\[
\|S(x)\|\ge \varepsilon/2 \qquad \forall x\in B_{\varepsilon/(2L)}(0)\cap B_{1/2}(0).
\]
\end{lemma}

\begin{proof}
Since $\xi$ is smooth and $\phi$ is time-independent, $t\mapsto \int_{B_\delta}\phi^2\xi(\cdot,t)$ is $C^1$ and hence $m$ is absolutely continuous.
Differentiate under the integral sign and use the PDE to write
\[
\partial_t m(t)=\frac1M\int\phi^2\,\partial_t\xi
=\frac1M\int\phi^2\Bigl(\Delta\xi-u\cdot\nabla\xi+|\nabla\xi|^2\xi+P_\xi(S\xi)+2(\nabla\log\rho)\cdot\nabla\xi\Bigr).
\]
For the diffusion term, integrate by parts (no boundary term since $\phi$ is compactly supported):
\[
\int\phi^2\Delta\xi
=-\int \nabla(\phi^2)\cdot\nabla\xi
=-\int 2\phi\,\nabla\phi\cdot\nabla\xi.
\]
For the transport term, use $\nabla\cdot u=0$ and integrate by parts:
\[
\int\phi^2\,u\cdot\nabla\xi
=-\int (u\cdot\nabla\phi^2)\,\xi.
\]
Collecting yields the claimed identity and the explicit formula for $\mathsf{Err}_{\mathrm{cut}}$.

For the bound, note that $M\sim \delta^3$ and $\|\nabla(\phi^2)\|_{L^2}\lesssim \delta^{1/2}\|\nabla\phi\|_\infty\lesssim \delta^{-1/2}$.
By Cauchy--Schwarz,
\[
\Bigl|\frac{1}{M}\int 2\phi\,\nabla\phi\cdot\nabla\xi\Bigr|
\lesssim \frac{1}{\delta^3}\,\|\nabla(\phi^2)\|_{L^2}\,\|\nabla\xi\|_{L^2(B_\delta)}
\lesssim \delta^{-5/2}\|\nabla\xi\|_{L^2(B_\delta)},
\]
and similarly
\[
\Bigl|\frac{1}{M}\int (u\cdot\nabla\phi^2)\,\xi\Bigr|
\lesssim \frac{1}{\delta^3}\,\|u\|_{L^2(B_\delta)}\,\|\nabla(\phi^2)\|_{L^2}
\lesssim \delta^{-5/2}\|u\|_{L^2(B_\delta)},
\]
using $|\xi|=1$.

Finally, on $E'$ we have $b=m/|m|$ with $|m|\ge 1/2$, so $b$ is absolutely continuous with derivative
\[
\partial_t b = \frac{\partial_t m}{|m|}-\frac{m(m\cdot \partial_t m)}{|m|^3}
 = (I-b\otimes b)\,\partial_t m/|m|,
\]
as claimed.
\end{proof}
\end{lstlisting}


\textbf{Acceptance test for L2‑A (what makes this nontrivial):}


\begin{itemize}
\item Provide an explicit bound for the Lipschitz constant \(L\) in terms of a \textbf{checkable tail size} of $\Omega$
  (e.g. an explicit weighted integral or decay assumption). This is where any missing tail hypothesis must be stated.
\end{itemize}


\textbf{Ledger mapping:} L2‑A is a bridge-enabler; it doesn’t itself “pay,” but it allows any pointwise lower bound on the tail strain
to be upgraded to a lower bound on a full ball, which then plugs into the superlevel/cylinder payment machinery.


\subparagraph{L2‑A acceptance: a concrete “tail size ⇒ Lipschitz constant” candidate (so this isn’t vague)}


Because the tail strain kernel has cancellations (as in TeX \texttt{\detokenize{lem:tail-strain-formula}}), one expects
\texttt{\detokenize{S(x)}} to be not just harmonic but \emph{Lipschitz} on \(B_{1/2}\) under very mild tail size control.


A concrete checkable tail size to try first is:


\[
\mathsf{Tail}_4(\Omega)\ :=\ \int_{|w|>1}\frac{|\Omega(w)|}{|w|^4}\,dw.
\]


Since \(|w|^{-4}\) is integrable in 3D, \(\mathsf{Tail}_4(\Omega)<\infty\) already holds under bounded vorticity:
\(\mathsf{Tail}_4(\Omega)\le \|\Omega\|_\infty\int_{|w|>1}|w|^{-4}dw <\infty\).


\textbf{Acceptance target:} prove a bound of the form


\[
\sup_{x\in B_{1/2}}|\nabla S(x)|\ \le\ C\,\mathsf{Tail}_4(\Omega),
\]


which implies the Lipschitz constant \(L\le C\,\mathsf{Tail}_4(\Omega)\) in \texttt{\detokenize{lem:tail-strain-local-persistence}}.


If this bound goes through with only \(\|\Omega\|_\infty\) (using \(\mathsf{Tail}_4(\Omega)\lesssim \|\Omega\|_\infty\)),
then L2‑A becomes essentially “free” under the running‑max bounded-vorticity input.


\textbf{Proof sketch (why the bound is plausible):}


\begin{itemize}
\item For \(|x|\le \tfrac12\) and \(|w|>1\), the Biot–Savart kernel and its derivatives are smooth, with
  \[
  |\nabla_x^m\bigl((x-w)/|x-w|^3\bigr)|\ \lesssim_m\ |w|^{-2-m}.
  \]
\item The strain \(S(x)=\tfrac12(\nabla u+\nabla u^T)\) is a first \(x\)-derivative of the Biot–Savart integral;
  hence \(\nabla S(x)\) is a \textbf{second} \(x\)-derivative of that integral.
\item Differentiating under the integral sign (justified by the support separation \(|w|>1\) from \(|x|\le 1/2\))
  gives an integrand bounded by \(C\,|\Omega(w)|\,|w|^{-4}\).
\item Taking the supremum over \(|x|\le 1/2\) yields
  \(\sup_{B_{1/2}}|\nabla S|\le C\int_{|w|>1}|\Omega(w)|\,|w|^{-4}\,dw = C\,\mathsf{Tail}_4(\Omega)\).
\end{itemize}


This is deliberately “estimate-only”: it does not use cancellations, so it should be robust.


\begin{lstlisting}
\begin{lemma}[Tail$_4$ controls Lipschitz of the tail strain on $B_{1/2}$ (proved)]\label{lem:tail4_controls_lipschitz_tail_strain}
Let $\Omega:\R^3\to\R^3$ be smooth and assume the weighted tail size is finite:
\[
\mathsf{Tail}_4(\Omega)\ :=\ \int_{|w|>1}\frac{|\Omega(w)|}{|w|^4}\,dw\ <\ \infty.
\]
Define the (tail) Biot--Savart velocity for $|x|\le\tfrac12$ by
\[
u(x):=\frac{1}{4\pi}\int_{|w|>1}\frac{(x-w)\times \Omega(w)}{|x-w|^3}\,dw,
\]
and let $S(x):=\tfrac12(\nabla u(x)+\nabla u(x)^T)$ be the symmetric gradient.
Then $u$ is smooth on $B_{1/2}$ and there exists an absolute constant $C$ such that
\[
\sup_{x\in B_{1/2}}|\nabla S(x)|\ \le\ C\,\mathsf{Tail}_4(\Omega).
\]
In particular, $S$ is Lipschitz on $B_{1/2}$ with constant $L\le C\,\mathsf{Tail}_4(\Omega)$:
\[
|S(x)-S(y)|\ \le\ C\,\mathsf{Tail}_4(\Omega)\,|x-y|\qquad\forall x,y\in B_{1/2}.
\]
\end{lemma}

\begin{proof}[Proof sketch]
For $|x|\le \tfrac12$ and $|w|>1$ one has $|x-w|\sim |w|$ uniformly, so the kernel
\(
K(x,w):=\frac{x-w}{|x-w|^3}
\)
and all its $x$-derivatives are smooth and satisfy pointwise bounds
\[
|\nabla_x^m K(x,w)|\ \lesssim_m\ |w|^{-2-m}.
\]
Since $u$ is the convolution of $K(x,w)$ with $\Omega(w)$ (up to the cross product), $S$ is a first $x$-derivative of $u$
and $\nabla S$ is a second $x$-derivative of $u$.
Differentiating under the integral sign gives an integrand bounded by $C|\Omega(w)|\,|w|^{-4}$ uniformly for $|x|\le 1/2$.
Integrating over $|w|>1$ yields
\(\sup_{B_{1/2}}|\nabla S|\le C\int_{|w|>1}|\Omega(w)|\,|w|^{-4}\,dw\),
as claimed.
\end{proof}
\end{lstlisting}


\subsubsection{Session S3 — Turn “budget-visible term” into an infinite-history contradiction}


\textbf{Goal:} make “finite budget over infinite history” a literal inequality.


Primary tool is \texttt{\detokenize{rem:runningmax-injection-constraint}} plus its propagation lemmas
(\texttt{\detokenize{lem:thick-maximum}}, \texttt{\detokenize{lem:maxpoint-to-injection-region}}, …), which explain how persistent
positive stretching at maximizers forces a positive‑measure injection region.


\textbf{Default L‑upper choice (from 3.10.4):} try \textbf{U‑4} first.


In Mode (M‑seq), U‑4 means:


\begin{itemize}
\item run the contradiction on a \emph{fixed rescaled cylinder} \(Q_r(0,0)\) (with \(r\le 1\)),
\item translate any would-be “infinite payment” mechanism into a statement that already contradicts
  available scale‑critical bounds on that cylinder (or contradicts the record-time normalization).
\end{itemize}


This avoids needing a global-in-time finite budget statement.


\textbf{Two outcomes (both useful):}


\begin{itemize}
\item (C‑strong) derive a direct contradiction: budget-visible term can’t persist at all ⇒ export impossible.
\item (C‑weak) derive an integrability class over time (e.g. \(L^2_t\)) strong enough to close RM2 subsequentially.
\end{itemize}


\textbf{Acceptance test:}


\begin{itemize}
\item A TeX lemma “persistent export ⇒ integral cost ≥ c·|I|” and a second lemma that the integral cost
  is bounded (or cannot persist) for a running‑max ancient element.
\end{itemize}


\textbf{Strawman structure (Family B):}


\begin{itemize}
\item Use \texttt{\detokenize{rem:runningmax-injection-constraint}} at maximizers:
  \[
  \sigma(x_t,t)\le |\nabla\xi(x_t,t)|^2 - \Delta\rho(x_t,t).
  \]
  If a bridge lemma yields \texttt{\detokenize{σ(x_t,t) ≥ c ε}} on a positive-measure set of times,
  then on those times either \texttt{\detokenize{|\nabla\xi|^2}} is large or \texttt{\detokenize{-Δρ}} is large.
\item Propagate “large at a point” to “large on a region” using the thick-maximum machinery
  (\texttt{\detokenize{lem:thick-maximum}} etc.) to obtain an integral lower bound on a cost density.
\item Contradict an a priori scale-critical bound (or derive \texttt{\detokenize{L²_t}} control of \texttt{\detokenize{S(0,t)}} as the weak outcome).
\end{itemize}


\paragraph{S3.1 Concrete lemma chain for U‑4 (pre-limit, single cylinder)}


In U‑4 we want a contradiction \textbf{inside one fixed rescaled cylinder} \(Q_r(0,0)\), avoiding any global-in-time
budget statement. The right structure is:


\begin{enumerate}
\item persistence of export (\texttt{\detokenize{S_tail}}) on a time set in \([-r^2,0]\),
\item select a “good” time/point where we can apply the penalized-max injection inequality (H‑t/H‑u/H‑ρ),
\item convert export persistence into a lower bound on a \textbf{paid} term (one of the ledger payments),
\item contradict a cylinder-scale upper bound (or declare it as a new explicit hypothesis and track it).
\end{enumerate}


Below are the next concrete lemmas to write/attempt.


\subparagraph{(S3‑T) Time/point selection lemma (discharges H‑t without circularity)}


\begin{lstlisting}
\begin{lemma}[Select a low-Dini-derivative time in a high-value slice (proved)]\label{lem:select_runningmax_time_in_persistence}
Let $I=[-r^2,0]$ and let $E\subset I$ be measurable with $|E|>0$.
Let $f:I\to\mathbb R$ be Lipschitz, and let $m:=\operatorname*{ess\,sup}_{t\in E} f(t)$.
Fix $\delta>0$ and set
\[
E_\delta \ :=\ \{t\in E:\ f(t)\ge m-\delta\}.
\]
Then $|E_\delta|>0$ and there exists $t_\ast\in E_\delta$ such that the upper right Dini derivative satisfies
\[
D^+ f(t_\ast)\ \le\ \frac1{|E_\delta|}\int_{E_\delta} (D^+ f(t))_+\,dt.
\]
In particular, if one has the bound
\[
\int_{E_\delta} (D^+ f(t))_+\,dt\ \le\ \delta\,|E_\delta|,
\]
then $D^+ f(t_\ast)\le \delta$.
\end{lemma}

\begin{proof}
Since $m$ is the essential supremum on $E$, the set $E_\delta$ has positive measure.
For Lipschitz $f$, Rademacher's theorem gives that $f$ is differentiable a.e. and in particular
$D^+f$ exists a.e. and $(D^+f)_+\in L^1(I)$.
Set $g(t):=(D^+f(t))_+$ (with any measurable representative).
If $g(t)>\frac1{|E_\delta|}\int_{E_\delta}g$ held for a.e. $t\in E_\delta$, then integrating over $E_\delta$ would yield a contradiction.
Hence there exists $t_\ast\in E_\delta$ with
$g(t_\ast)\le \frac1{|E_\delta|}\int_{E_\delta}g$, and then $D^+f(t_\ast)\le g(t_\ast)$.
The “in particular” clause is immediate.
\end{proof}
\end{lstlisting}


\textbf{How this discharges (H‑t) (a clean envelope inequality):}


If $f(t)=\sup_x \rho_\varepsilon(x,t)$ and for a given $t$ there exists a maximizer $x_{\varepsilon,t}$ with
$f(t)=\rho_\varepsilon(x_{\varepsilon,t},t)$ and $\partial_t\rho_\varepsilon(x_{\varepsilon,t},t)$ exists, then
\[
\partial_t\rho_\varepsilon(x_{\varepsilon,t},t)\ \le\ D^+ f(t),
\]
since $f(t+h)\ge \rho_\varepsilon(x_{\varepsilon,t},t+h)$ for all $h>0$ and one takes $\limsup_{h\downarrow 0}$.
Thus the conclusion $D^+f(t_\ast)\le \delta$ yields the approximate one-sided constraint
$\partial_t\rho_\varepsilon(x_{\varepsilon,t_\ast},t_\ast)\le \delta$.


\textbf{Proof sketch (one workable route):}


\begin{itemize}
\item If \(\rho\) is smooth on \(Q_r\) and the penalized maximizer \(x_{\varepsilon,t}\) stays in a fixed ball (true since
  \(\rho\le 1\Rightarrow \varepsilon|x_{\varepsilon,t}|^2\le 1\)), then \(t\mapsto f(t)=\sup_x(\rho-\varepsilon|x|^2)\)
  is locally Lipschitz on \([-r^2,0]\) with constant \(\lesssim \sup_{|x|\le 1/\sqrt\varepsilon}|\partial_t\rho(x,t)|\).
\item A Lipschitz function is absolutely continuous, hence \(f'\in L^1\) and the set where \(f'(t)>\delta\) has measure
  bounded by \(\operatorname{Var}(f)/\delta\).
\item Since \(E\subset[-r^2,0]\) has positive measure, choose \(\delta\) so that \(E\) is not contained in \(\{f'>\delta\}\);
  then pick \(t_\ast\in E\) where the upper Dini derivative of \(f\) is \(\le \delta\).
\item Finally, relate the Dini derivative of \(f\) to \(\partial_t\rho_\varepsilon(x_{\varepsilon,t_\ast},t_\ast)\) at a penalized maximizer
  (standard envelope derivative inequality), giving the desired approximate (H‑t).
\end{itemize}


\subparagraph{(S3‑U) Drift-control lemma (discharges H‑u in a local/affine gauge)}


\begin{lstlisting}
\begin{lemma}[Drift control from thickness + $L^2$ energy (proved)]\label{lem:drift_control_penalized}
Fix $r>0$ and $\theta\in(0,1)$.
Let $\tilde u(\cdot,t)\in L^2(B_r;\mathbb R^3)$ and let $G_t\subset B_r$ be measurable with $|G_t|\ge \theta |B_r|$.
Then there exists a point $x_t\in G_t$ such that
\[
|\tilde u(x_t,t)|\ \le\ \sqrt{\frac{2}{\theta |B_r|}}\ \|\tilde u(\cdot,t)\|_{L^2(B_r)}
\qquad\text{and hence}\qquad
|\tilde u(x_t,t)\cdot x_t|\ \le\ r\,\sqrt{\frac{2}{\theta |B_r|}}\ \|\tilde u(\cdot,t)\|_{L^2(B_r)}.
\]
In particular, if one has the scale-critical bound
\[
\int_{B_r}|\tilde u(x,t)|^2\,dx\ \le\ M\,r\qquad\text{for all relevant }t,
\]
then for the selected points $x_t$ above one gets a uniform drift bound
\[
|\tilde u(x_t,t)\cdot x_t|\ \le\ C(\theta)\,\sqrt{M},
\qquad C(\theta):=\sqrt{\frac{2}{\theta |B_1|}}.
\]
\end{lemma}

\begin{proof}
Let $\lambda:=\sqrt{\frac{2}{|G_t|}}\|\tilde u(\cdot,t)\|_{L^2(B_r)}$.
By Chebyshev,
\[
|\{x\in B_r:\ |\tilde u(x,t)|>\lambda\}|
\ \le\ \frac{1}{\lambda^2}\int_{B_r}|\tilde u(x,t)|^2\,dx
\ =\ \frac{|G_t|}{2}.
\]
Hence the complement set $\{|\tilde u|\le\lambda\}$ intersects $G_t$, so choose $x_t\in G_t$ with $|\tilde u(x_t,t)|\le\lambda$.
The dot-product bound follows from $|x_t|\le r$.
Finally, using $|G_t|\ge \theta |B_r|=\theta |B_1| r^3$ and $\|\tilde u\|_{L^2(B_r)}\le \sqrt{Mr}$ yields the stated uniform bound.
\end{proof}
\end{lstlisting}


\textbf{Acceptance test:} we must (i) specify the thickness mechanism that produces $G_t$ (usually a “thick maximum” lemma on the top band
$\{\rho\ge 1-\eta\}$), and (ii) specify the scale-critical bound that yields $\int_{B_r}|\tilde u|^2\lesssim r$ in the chosen affine gauge.
The lemma above itself is purely measure-theoretic + Cauchy–Schwarz/Chebyshev (“plumbing-clean”).


\subparagraph{(S3‑B) Bridge: tail export ⇒ “σ or twist” on the top band (hypothesis interface)}


This is the \textbf{first genuinely research-level bridge} in S3: it converts “tail strain persists” into a pointwise lower bound
on either (i) stretching injection \(\sigma\) or (ii) direction twisting \(|\nabla\xi|^2\), at points where \(\rho\) is near its top band.
We keep it isolated as a hypothesis interface so the rest of S3 is honest and modular.


\begin{lstlisting}
\begin{hypothesis}[Bridge: tail strain forces stretching-or-twist near the top band]\label{hyp:tail_strain_forces_sigma_or_twist}
Fix $\eta\in(0,1/8)$.
There exist constants $c_\sigma,c_\xi>0$ such that for any rescaled running-max cylinder solution on $Q_r(0,0)$
and any parameters $(\varepsilon,\delta)$, the following holds:

If for some time $t$ and some point $x\in B_\delta$ we have
\[
\rho(x,t)\ \ge\ 1-\eta
\qquad\text{and}\qquad
\|S^{(k)}_{\mathrm{tail}}(x,t)\|\ \ge\ \varepsilon/2,
\]
then at that same spacetime point either
\[
\sigma(x,t)\ \ge\ c_\sigma\,\varepsilon
\qquad\text{or}\qquad
|\nabla\xi(x,t)|^2\ \ge\ c_\xi\,\varepsilon^2.
\]
\end{hypothesis}
\end{lstlisting}


\textbf{Acceptance test:} we must pin down the precise definition of \(\sigma\) in our running-max variables (e.g. \(\sigma=\xi\cdot S\,\xi\)),
and confirm the constants are scale-consistent. If proving this bridge collapses into C2, trigger the pivot.


\textbf{Research focus (how to actually try to prove S3‑B):}


The bridge is the first place where we truly need a new geometric argument. Here is the cleanest way to structure it so we
don’t accidentally “solve C2 by stealth”.


\begin{itemize}
\item \textbf{Default shot (commitment for this execution run): start with Route (B‑ξ).}\\
  Rationale: (i) it targets the always-nonnegative payment channel \(\rho^{3/2}|\nabla\xi|^2\) directly, (ii) it is robust to sign issues
  in \(\sigma\), and (iii) if it collapses into a C2-type estimate we will detect that quickly (pivot trigger P‑C2).
  Route (B‑σ) becomes the fallback if we discover a clean alignment/sign mechanism at maximizers.
\item \textbf{(B‑0) Fix the exact definition of \(\sigma\)} (no drift):
\item In vorticity variables \(\omega=\rho\xi\) with \(|\xi|=1\), the amplitude equation is
    \[
    \partial_t\rho + u\cdot\nabla\rho - \Delta\rho = \rho(\sigma-|\nabla\xi|^2),
    \]
    where \(\sigma := \xi\cdot S\,\xi\) and \(S=\tfrac12(\nabla u+\nabla u^T)\) is the strain.
\item Decompose \(S = S_{\mathrm{core}} + S_{\mathrm{tail}}\) using the same core/tail split as in the manuscript’s tail strain moment.
    The assumption in S3 is about \(\|S_{\mathrm{tail}}\|\), but \(\sigma\) involves the \textbf{quadratic form} \(\xi\cdot S_{\mathrm{tail}}\xi\),
    which can be small even when \(\|S_{\mathrm{tail}}\|\) is large if \(\xi\) is aligned with a compressive/near-null direction.
\item \textbf{(B‑1) Two-way route to a usable statement} (pick one):
\item \textbf{Route (B‑σ): stretching case.} Prove that on a positive-measure subset of the top band, the vorticity direction cannot stay
    aligned with compressive directions of \(S_{\mathrm{tail}}\); hence \(\sigma\gtrsim\varepsilon\) somewhere in the top band.
    This is a \emph{sign + alignment persistence} problem.
\item \textbf{Route (B‑ξ): twisting case.} Prove that if \(\xi\) avoids the stretching directions of \(S_{\mathrm{tail}}\) across a region where
    \(S_{\mathrm{tail}}\) is large and slowly varying (L2‑A gives Lipschitz control), then \(\xi\) must rotate fast, forcing
    \(|\nabla\xi|^2\gtrsim\varepsilon^2\) in that region. This is a \emph{direction-forcing ⇒ gradient lower bound} problem.
\item \textbf{(B‑ξ.0) A pointwise trichotomy we can always use (pure algebra, no PDE):}
\end{itemize}


Even before any dynamics, there is a purely algebraic decomposition at a point:
for any symmetric matrix $S$ and any unit vector $\xi$,
\[
S\xi = (\xi\cdot S\xi)\,\xi \;+\; (I-\xi\otimes\xi)\,S\xi,
\qquad
|S\xi|^2 = (\xi\cdot S\xi)^2 + |(I-\xi\otimes\xi)S\xi|^2.
\]
Thus if $|S\xi|\gtrsim \varepsilon$ then either the quadratic form $\sigma=\xi\cdot S\xi$ is $\gtrsim\varepsilon$ in magnitude,
or the perpendicular forcing term $(I-\xi\otimes\xi)S\xi$ is $\gtrsim\varepsilon$ in magnitude.
The remaining obstruction is the “null alignment” case $|S\xi|\ll \varepsilon$, which corresponds to $\xi$ lying close to a null
direction of $S$ (e.g. the middle eigenvector when $\lambda_2\approx 0$).


\begin{itemize}
\item \textbf{(B‑ξ.1) Minimal interface to bypass the null-alignment case (hypothesis):}
\end{itemize}


\begin{lstlisting}
\begin{hypothesis}[No persistent null-alignment on the top band]\label{hyp:no_persistent_null_alignment}
Fix $\eta\in(0,1/8)$. There exists $c_*>0$ such that whenever $\rho(x,t)\ge 1-\eta$ and
$\|S_{\mathrm{tail}}^{(k)}(x,t)\|\ge \varepsilon/2$ on a set $B_\delta\times E$, then on a positive-measure subset
of $B_\delta\times E$ one has
\[
|S_{\mathrm{tail}}^{(k)}(x,t)\,\xi(x,t)|\ \ge\ c_*\,\varepsilon.
\]
\end{hypothesis}
\end{lstlisting}


\textbf{Lean symbol:} \texttt{\detokenize{Bet2U4.NoPersistentNullAlignmentHypothesis}}.


This hypothesis isolates exactly the scenario where tail strain is large but “misses” the vorticity direction everywhere on the top band.
If this hypothesis fails, we have discovered a new geometric regime (“vorticity rides the null cone of the tail strain”), which is itself
a plausible pivot-to-E/2D mechanism.


\textbf{Proof status (unconditional checklist item \#3):} currently \textbf{blocked}.
This is a genuinely geometric alignment statement about the interaction of $\xi$ with the eigen-structure of $S_{\mathrm{tail}}$ on a
time-thick top-band set. No known “bounded vorticity only” argument implies it; if it fails, we treat the resulting null-cone regime as an
explicit new absorption/degenerate class to be eliminated (likely via an E/2D pivot).


\textbf{Single blocker:} a proof (or sharp classification) of the null-cone alignment regime on time-thick top-band sets.


\begin{itemize}
\item \textbf{(B‑ξ.1a) Immediate algebraic consequence (proved; this is the handoff into Route (B‑σ) vs Route (B‑ξ)):}
\end{itemize}


\begin{lstlisting}
\begin{lemma}[No-null-alignment $\Rightarrow$ (stretching magnitude) or (perpendicular tail forcing) on a subset (algebra)]\label{lem:no_null_alignment_implies_sigma_or_perp}
Assume the hypotheses of \ref{hyp:no_persistent_null_alignment} at scale $r$ so that on a positive-measure subset
$U\subset B_\delta\times E$ one has
\(
|S_{\mathrm{tail}}^{(k)}(x,t)\,\xi(x,t)|\ge c_*\,\varepsilon
\).
Then on the same set $U$ one has the pointwise implication:
\[
|(\xi\cdot S_{\mathrm{tail}}^{(k)}\xi)(x,t)|\ \ge\ \frac{c_*}{\sqrt2}\,\varepsilon
\qquad\text{or}\qquad
|(I-\xi\otimes\xi)\,S_{\mathrm{tail}}^{(k)}(x,t)\,\xi(x,t)|\ \ge\ \frac{c_*}{\sqrt2}\,\varepsilon.
\]
\end{lemma}

\begin{proof}
At a point $(x,t)\in U$, apply the identity
\(
|S\xi|^2=(\xi\cdot S\xi)^2+|(I-\xi\otimes\xi)S\xi|^2
\)
with $S=S_{\mathrm{tail}}^{(k)}(x,t)$ and $\xi=\xi(x,t)$.
If both terms on the right were $<\frac12|S\xi|^2$, then their sum would be $<|S\xi|^2$, a contradiction. Taking square-roots gives the claim.
\end{proof}
\end{lstlisting}


\begin{itemize}
\item \textbf{(B‑ξ.2) Forcing ⇒ payment interface (what Route (B‑ξ) really needs):}
\end{itemize}


\begin{lstlisting}
\begin{hypothesis}[Perpendicular tail forcing pays for twist on the top band]\label{hyp:forcing_to_twist_payment_on_top_band}
Fix $\eta\in(0,1/8)$. There exists $c>0$ such that for any cylinder $Q_r(0,0)$,
if on a set $B_\delta\times E\subset B_r\times[-r^2,0]$ with $|E|\ge c_0 r^2$ one has $\rho\ge 1-\eta$ and
\[
|(I-\xi\otimes\xi)\,S_{\mathrm{tail}}^{(k)}(x,t)\,\xi(x,t)|\ \ge\ c_*\,\varepsilon
\qquad\text{for all }(x,t)\in B_\delta\times E,
\]
then the normalized **twist-or-band** payment obeys
\[
\mathsf{Pay}_\xi(r)\;+\;\mathsf{Pay}_\rho(r)\ \ge\ c\,\varepsilon^2,
\]
where
\[
\mathsf{Pay}_\xi(r):=\frac1{r^2}\iint_{Q_r}\rho^{3/2}|\nabla\xi|^2,
\qquad
\mathsf{Pay}_\rho(r):=\frac1{r^2}\,\eta^{-1}\iint_{Q_r\cap\{1-2\eta<\rho<1-\eta\}}|\nabla(\rho^{3/4})|^2.
\]
\end{hypothesis}
\end{lstlisting}


\textbf{Reality check (important): magnitude-only forcing does \emph{not} automatically force spatial twist.}
There is a genuine “rigid rotation mode” obstruction: if (locally) $\rho$ is essentially constant and $S_{\mathrm{tail}}$ is essentially
constant in space, one can have $\xi=\xi(t)$ spatially constant (so $\nabla\xi\equiv 0$ and $\mathsf{Pay}_\xi=\mathsf{Pay}_\rho=0$) while
the tangential forcing $P_\xi(S_{\mathrm{tail}}\xi)$ remains nonzero and is absorbed by a purely time-dependent rotation of $\xi$.
Therefore, \textbf{proving \texttt{\detokenize{\ref{hyp:forcing_to_twist_payment_on_top_band}}} as stated requires a specifically Navier–Stokes / Biot–Savart input}
that rules out this “pure-rotation cancellation” regime for the \emph{tail} forcing.


\textbf{Minimal missing input (explicit interface; this is what we must actually prove next):}


\begin{lstlisting}
\begin{hypothesis}[Tail forcing cannot be absorbed by a rigid-rotation mode on the top band]\label{hyp:no_rigid_rotation_absorption}
Fix $\eta\in(0,1/8)$. There exists $c>0$ such that for any rescaled running-max cylinder solution on $Q_r(0,0)$:
if on $B_\delta\times E$ (with $|E|\ge c_0 r^2$) one has $\rho\ge 1-\eta$ and
\[
|(I-\xi\otimes\xi)\,S_{\mathrm{tail}}^{(k)}(x,t)\,\xi(x,t)|\ \ge\ c_*\,\varepsilon,
\]
then either
\[
\mathsf{Pay}_\xi(r)+\mathsf{Pay}_\rho(r)\ \ge\ c\,\varepsilon^2
\]
or else the solution on $Q_r$ lies in an explicit “affine/rigid” structural class (time-dependent rotation / affine mode)
that is independently ruled out for running-max ancient elements (by a separate E/2D-type or affine-mode ODE contradiction).
\end{hypothesis}
\end{lstlisting}


\textbf{Lean symbol:} \texttt{\detokenize{Bet2U4.NoRigidRotationAbsorptionHypothesis}}.


\textbf{Acceptance test:} either (i) prove the original \texttt{\detokenize{\ref{hyp:forcing_to_twist_payment_on_top_band}}} directly, or (ii) prove the weaker but
still-useful \texttt{\detokenize{\ref{hyp:no_rigid_rotation_absorption}}} by \emph{classifying} the rigid-rotation absorption regime and then killing it with a clean
ODE/structure argument. If the classification reduces to “$\xi$ constant direction on the top band”, that is an E/2D pivot trigger.


\textbf{Proof status (unconditional checklist item \#4):} currently \textbf{blocked}.
We have the TeX engines for the classification lemma \texttt{\detokenize{\ref{lem:rigid_rotation_absorption_implies_structure}}}
(items (i)–(iii) and the averaged-direction ODE template), but closing this hypothesis unconditionally
requires completing the absorption-kill (K‑ODE) or an E/2D elimination for the resulting explicit class.


\textbf{Single blocker:} rule out the explicit absorption class output by Lemma~\texttt{\detokenize{\ref{lem:rigid_rotation_absorption_implies_structure}}} uniformly as $r\downarrow 0$
(K‑ODE affine-mode contradiction or K‑E quasi‑2D elimination).


\textbf{Concrete “classification” lemmas to shoot next (so this doesn’t stay hand-wavy):}


\begin{lstlisting}
\begin{lemma}[Small twist payment forces near-constancy of $\xi$ on the top band (time-slice form; proved)]\label{lem:small_pay_xi_implies_xi_near_constant}
Fix $\eta\in(0,1/8)$ and $\delta>0$. Let $t$ be a time such that $\rho(\cdot,t)\ge 1-\eta$ on $B_\delta$.
Then for any unit vector $b_t\in \Sbb^2$ minimizing $\int_{B_\delta}|\xi(\cdot,t)-b|^2$ one has
\[
\int_{B_\delta}|\xi(x,t)-b_t|^2\,dx\ \le\ C\,\delta^2\int_{B_\delta}|\nabla\xi(x,t)|^2\,dx,
\]
with a universal constant $C$.
In particular, if $\int_{B_\delta}\rho(x,t)^{3/2}|\nabla\xi(x,t)|^2\,dx$ is small, then $\xi(\cdot,t)$ is $L^2$-close to a constant
direction on $B_\delta$.
\end{lemma}

\begin{proof}
Let $B:=B_\delta$ and write $f(x):=\xi(x,t)\in \R^3$, so $|f|=1$ a.e. on $B$.
Let $\mu:=f_B:=\frac1{|B|}\int_B f$ be the spatial average, so $|\mu|\le 1$.

\smallskip
\noindent\textbf{Step 1 (Poincar\'e for the unconstrained best constant).}
By the (vector-valued) Poincar\'e inequality on balls,
\[
\int_B |f-\mu|^2 \ \le\ C_P\,\delta^2 \int_B |\nabla f|^2,
\]
with a universal constant $C_P$ depending only on the dimension.

\smallskip
\noindent\textbf{Step 2 (relate the best unit vector to the average).}
For any unit vector $b\in\Sbb^2$,
\[
\int_B |f-b|^2
=\int_B |f|^2 + \int_B|b|^2 - 2 b\cdot\int_B f
=2|B| - 2|B|\,(b\cdot \mu).
\]
Thus minimizing over $|b|=1$ is equivalent to maximizing $b\cdot\mu$, and a minimizer is $b_t=\mu/|\mu|$ if $\mu\neq 0$
(and any $b_t\in\Sbb^2$ if $\mu=0$).
In either case, the minimum value is
\[
\int_B |f-b_t|^2 \ =\ 2|B|\,(1-|\mu|).
\]
On the other hand,
\[
\int_B|f-\mu|^2
=\int_B|f|^2 - 2\mu\cdot\int_B f + \int_B|\mu|^2
=|B|\,(1-|\mu|^2).
\]
Since $0\le |\mu|\le 1$, we have $1-|\mu|\le 1-|\mu|^2$, hence
\[
\int_B|f-b_t|^2 = 2|B|\,(1-|\mu|)\ \le\ 2|B|\,(1-|\mu|^2)=2\int_B|f-\mu|^2.
\]
Combining with Step 1 gives
\[
\int_B|f-b_t|^2\ \le\ 2C_P\,\delta^2\int_B|\nabla f|^2,
\]
which is the claimed estimate with $C:=2C_P$.

\smallskip
\noindent\textbf{Weighted-to-unweighted.}
If $\rho(\cdot,t)\ge 1-\eta$ on $B$, then $\rho^{3/2}\ge (1-\eta)^{3/2}$ on $B$, hence
\[
\int_B|\nabla\xi|^2 \ \le\ (1-\eta)^{-3/2}\int_B \rho^{3/2}|\nabla\xi|^2,
\]
so small weighted twist energy implies small unweighted twist energy, and therefore $\xi(\cdot,t)$ is $L^2$-close to the constant
direction $b_t$.
\end{proof}
\end{lstlisting}


\begin{lstlisting}
\begin{lemma}[Rigid-rotation absorption regime implies an explicit “ODE + affine tail” structure (target)]\label{lem:rigid_rotation_absorption_implies_structure}
Fix $\eta\in(0,1/8)$, $r\le 1$, and a top-band persistence set $B_\delta\times E\subset B_r\times[-r^2,0]$ with $|E|\ge c_0 r^2$
on which $\rho\ge 1-\eta$.
Assume also the perpendicular tail forcing lower bound
\[
|(I-\xi\otimes\xi)\,S_{\mathrm{tail}}^{(k)}(x,t)\,\xi(x,t)|\ \ge\ c_*\,\varepsilon
\qquad\forall (x,t)\in B_\delta\times E.
\]
Assume finally that the normalized payments are small:
\[
\mathsf{Pay}_\xi(r)+\mathsf{Pay}_\rho(r)\ \le\ \kappa\,\varepsilon^2
\qquad\text{for some sufficiently small }\kappa=\kappa(\eta,c_0).
\]
Then there exists a subset $E'\subset E$ with $|E'|\ge \tfrac12|E|$ and a measurable map $b:E'\to\Sbb^2$ such that:
\begin{enumerate}
\item[(i)] (\textbf{Near-constancy in space}) For every $t\in E'$,
  \[
  \int_{B_\delta}|\xi(x,t)-b(t)|^2\,dx\ \le\ C\,\delta^2\int_{B_\delta}|\nabla\xi(x,t)|^2\,dx
  \ \lesssim_\eta\ \delta^2\int_{B_\delta}\rho(x,t)^{3/2}|\nabla\xi(x,t)|^2\,dx.
  \]
\item[(ii)] (\textbf{Rigid-rotation absorption ODE}) The direction $b(t)$ approximately satisfies (in a weak/distributional sense on $E'$)
  \[
  \partial_t b(t)\ \approx\ (I-b(t)\otimes b(t))\,S_{\mathrm{tail}}^{(k)}(0,t)\,b(t),
  \]
  i.e. the tail forcing is absorbed by a predominantly \emph{time-dependent} rotation of an almost-spatially-constant direction field.
\item[(iii)] (\textbf{Affine tail on the ball}) The tail velocity/strain field on $B_\delta$ is well-approximated by its affine Taylor
  polynomial determined by $S_{\mathrm{tail}}^{(k)}(0,t)$, uniformly for $t\in E'$ (this is where harmonicity of the tail velocity on
  $B_1$ is used).
\end{enumerate}
\end{lemma}

\begin{proof}[Proof sketch / reduction]
Assume $\mathsf{Pay}_\xi(r)+\mathsf{Pay}_\rho(r)\le \kappa\varepsilon^2$ on $Q_r$.

\smallskip
\noindent\textbf{Step 1 (choose good times).}
By Fubini + Chebyshev, there exists a subset $E'\subset E$ with $|E'|\ge \tfrac12|E|$ such that for every $t\in E'$
the time-slice direction energy on $B_\delta$ is small:
\[
\int_{B_\delta}\rho(x,t)^{3/2}|\nabla\xi(x,t)|^2\,dx\ \lesssim\ \kappa\,\varepsilon^2\,\delta^2
\qquad(\text{after calibrating }\delta\sim r).
\]
(If needed, intersect with the analogous “small band diffusion at time $t$” set coming from $\mathsf{Pay}_\rho$.)

\smallskip
\noindent\textbf{Step 2 (spatial near-constancy).}
For each $t\in E'$, define $b(t)\in\Sbb^2$ to be a minimizer of $\int_{B_\delta}|\xi(\cdot,t)-b|^2$.
Then Lemma~\ref{lem:small_pay_xi_implies_xi_near_constant} gives (i).

\smallskip
\noindent\textbf{Step 3 (average-ODE and absorption).}
Apply Lemma~\ref{lem:avg_direction_ode_template} on $B_\delta\times E'$ with a cutoff $\phi$ supported in $B_\delta$.
Use (i) to replace $\xi$ by $b(t)$ in the averaged forcing term, and bound the cutoff errors by the same time-slice energies
coming from $\mathsf{Pay}_\xi,\mathsf{Pay}_\rho$.
This yields an approximate ODE for $b(t)$ driven by the averaged tangential forcing.

\smallskip
\noindent\textbf{Step 4 (affine tail).}
Use the tail Lipschitz estimate (Lemma~\ref{lem:tail4_controls_lipschitz_tail_strain} or its cylinder version)
to replace $S_{\mathrm{tail}}^{(k)}(x,t)$ on $B_\delta$ by its center value $S_{\mathrm{tail}}^{(k)}(0,t)$ up to an error $O(\delta)$,
closing item (iii) and simplifying the RHS in the ODE to the stated “absorption” form in (ii).
\end{proof}
\end{lstlisting}


\textbf{Acceptance test (for \texttt{\detokenize{\ref{lem:rigid_rotation_absorption_implies_structure}}}):}
\begin{itemize}
\item (i) is now proved (Lemma~\texttt{\detokenize{\ref{lem:small_pay_xi_implies_xi_near_constant}}}) once we pick $E'$ as the set of times where the
      time-slice twist energy is small (Chebyshev/Fubini).
\item (iii) should be standard from harmonicity: on $B_\delta$, the tail velocity is harmonic so Taylor/Cauchy estimates bound the remainder
      by higher-derivative norms; we will state the needed derivative control as a hypothesis if it is not automatic from the tail truncation.
\item (ii) is the true “rigid-rotation absorption” step: integrate the direction equation against test functions on $B_\delta$,
      use (i) to replace $\xi$ by $b(t)$ in the forcing term, and bound drift/diffusion boundary errors by \(\mathsf{Pay}_\xi,\mathsf{Pay}_\rho\).
\end{itemize}


\textbf{Next concrete lemma (what we should try to prove immediately): the averaged-direction ODE.}


\begin{lstlisting}
\begin{lemma}[Averaged direction obeys an ODE driven by averaged tangential forcing (proved)]\label{lem:avg_direction_ode_template}
Fix $\delta>0$ and let $\phi\in C_c^\infty(B_\delta)$ satisfy $\phi\equiv 1$ on $B_{\delta/2}$ and $\|\nabla\phi\|_{L^\infty}\lesssim \delta^{-1}$.
Let $(\rho,\xi,u)$ be smooth on $B_\delta\times I$ with $\rho>0$, $|\xi|=1$, and $\nabla\cdot u=0$,
satisfying the exact projected direction equation
\[
\partial_t \xi + u\cdot\nabla\xi - \Delta\xi
= |\nabla\xi|^2\xi \;+\; P_\xi(S\xi)\;+\;2(\nabla\log\rho)\cdot\nabla\xi
\qquad\text{on }B_\delta\times I.
\]
Define the normalized weighted spatial average and its unit-normalization
\[
M:=\int_{B_\delta}\phi(x)^2\,dx,\qquad
m(t):=\frac{1}{M}\int_{B_\delta}\phi(x)^2\,\xi(x,t)\,dx,
\qquad
b(t):=\frac{m(t)}{|m(t)|}\in \Sbb^2\quad\text{whenever }m(t)\neq 0.
\]
Then $m$ is absolutely continuous on $I$ and for a.e. $t\in I$ one has the identity
\[
\partial_t m(t)
= \frac{1}{M}\int_{B_\delta}\phi^2\Bigl(|\nabla\xi|^2\xi + P_\xi(S\xi)+2(\nabla\log\rho)\cdot\nabla\xi\Bigr)\,dx
\;+\; \mathsf{Err}_{\mathrm{cut}}(t),
\]
where the cutoff/drift error is explicitly
\[
\mathsf{Err}_{\mathrm{cut}}(t)
= -\frac{1}{M}\int_{B_\delta} 2\phi\,\nabla\phi\cdot\nabla\xi\,dx
\;+\;\frac{1}{M}\int_{B_\delta}(u\cdot\nabla\phi^2)\,\xi\,dx,
\]
and satisfies the bound
\[
|\mathsf{Err}_{\mathrm{cut}}(t)|
\ \lesssim\ \delta^{-5/2}\Bigl(\|u(\cdot,t)\|_{L^2(B_\delta)}+\|\nabla\xi(\cdot,t)\|_{L^2(B_\delta)}\Bigr).
\]
Moreover, if $|m(t)|\ge 1/2$ on a measurable $E'\subset I$, then $b\in W^{1,1}(E')$ and for a.e. $t\in E'$,
\[
\partial_t b(t)
\;=\; (I-b(t)\otimes b(t))\,\partial_t m(t)\,/\,|m(t)|.
\]
\end{lemma}

\begin{proof}
Since $\xi$ is smooth and $\phi$ is time-independent, $t\mapsto \int_{B_\delta}\phi^2\xi(\cdot,t)$ is $C^1$ and hence $m$ is absolutely continuous.
Differentiate under the integral sign and use the PDE to write
\[
\partial_t m(t)=\frac1M\int_{B_\delta}\phi^2\,\partial_t\xi
=\frac1M\int_{B_\delta}\phi^2\Bigl(\Delta\xi-u\cdot\nabla\xi+|\nabla\xi|^2\xi+P_\xi(S\xi)+2(\nabla\log\rho)\cdot\nabla\xi\Bigr)\,dx.
\]
For the diffusion term, integrate by parts (no boundary term since $\phi$ is compactly supported):
\[
\int_{B_\delta}\phi^2\Delta\xi\,dx
=-\int_{B_\delta}\nabla(\phi^2)\cdot\nabla\xi\,dx
=-\int_{B_\delta} 2\phi\,\nabla\phi\cdot\nabla\xi\,dx.
\]
For the transport term, use $\nabla\cdot u=0$ and integrate by parts:
\[
\int_{B_\delta}\phi^2\,u\cdot\nabla\xi\,dx
=-\int_{B_\delta} (u\cdot\nabla\phi^2)\,\xi\,dx.
\]
Collecting yields the claimed identity and the explicit formula for $\mathsf{Err}_{\mathrm{cut}}$.

For the bound, note that $M\sim \delta^3$ and $\|\nabla(\phi^2)\|_{L^2(B_\delta)}\lesssim \delta^{1/2}\|\nabla\phi\|_\infty\lesssim \delta^{-1/2}$.
By Cauchy--Schwarz,
\[
\Bigl|\frac{1}{M}\int_{B_\delta} 2\phi\,\nabla\phi\cdot\nabla\xi\,dx\Bigr|
\lesssim \frac{1}{\delta^3}\,\|\nabla(\phi^2)\|_{L^2(B_\delta)}\,\|\nabla\xi\|_{L^2(B_\delta)}
\lesssim \delta^{-5/2}\|\nabla\xi\|_{L^2(B_\delta)}.
\]
Similarly,
\[
\Bigl|\frac{1}{M}\int_{B_\delta} (u\cdot\nabla\phi^2)\,\xi\,dx\Bigr|
\lesssim \frac{1}{\delta^3}\,\|u\|_{L^2(B_\delta)}\,\|\nabla(\phi^2)\|_{L^2(B_\delta)}
\lesssim \delta^{-5/2}\|u\|_{L^2(B_\delta)},
\]
using $|\xi|=1$.

Finally, on $E'$ we have $b=m/|m|$ with $|m|\ge 1/2$, so $b$ is absolutely continuous with derivative
\[
\partial_t b = \frac{\partial_t m}{|m|}-\frac{m(m\cdot \partial_t m)}{|m|^3}
 = (I-b\otimes b)\,\partial_t m/|m|,
\]
as claimed.
\end{proof}
\end{lstlisting}


\textbf{Acceptance test:} we should be able to prove Lemma~\texttt{\detokenize{\ref{lem:avg_direction_ode_template}}} purely from integration by parts and Poincar\'e,
with no global inputs. The only “nontrivial” step is bookkeeping the cutoff errors and keeping the scaling explicit.
This lemma is the engine behind item (ii) in Lemma~\texttt{\detokenize{\ref{lem:rigid_rotation_absorption_implies_structure}}}.


\begin{lstlisting}
\begin{lemma}[Dichotomy: persistent perpendicular tail forcing either pays or enters the absorption class]\label{lem:forcing_payment_or_absorption}
Fix $\eta\in(0,1/8)$ and assume the setup of Hypothesis~\ref{hyp:forcing_to_twist_payment_on_top_band} at scale $r\le 1$:
there exist $\delta\in(0,r)$ and a time set $E\subset[-r^2,0]$ with $|E|\ge c_0 r^2$ such that on $B_\delta\times E$,
\[
\rho\ge 1-\eta
\qquad\text{and}\qquad
|(I-\xi\otimes\xi)\,S_{\mathrm{tail}}^{(k)}\,\xi|\ \ge\ c_*\,\varepsilon.
\]
Then there exists a constant $c>0$ (depending only on $\eta,c_0$) such that either
\[
\mathsf{Pay}_\xi(r)+\mathsf{Pay}_\rho(r)\ \ge\ c\,\varepsilon^2
\]
or else the explicit absorption conclusion of Lemma~\ref{lem:rigid_rotation_absorption_implies_structure} holds
(i.e. existence of $E'\subset E$ and a map $b:E'\to\Sbb^2$ with properties (i)–(iii)).
\end{lemma}

\begin{proof}[One-line proof]
If $\mathsf{Pay}_\xi(r)+\mathsf{Pay}_\rho(r)\ge c\,\varepsilon^2$ we are done.
Otherwise choose $\kappa$ in Lemma~\ref{lem:rigid_rotation_absorption_implies_structure} and apply that lemma to enter the absorption class.
\end{proof}
\end{lstlisting}


At this point, the Bet2/U-4 route reduces to a single question:
can we rule out the absorption class uniformly as $r\downarrow 0$?


\begin{lstlisting}
\begin{hypothesis}[Absorption class is impossible for running-max ancient elements]\label{hyp:absorption_class_impossible}
Fix $\eta\in(0,1/8)$ and $\varepsilon_0>0$.
There exists $r_0=r_0(\eta,\varepsilon_0)>0$ such that for all $0<r\le r_0$,
the absorption conclusion of Lemma~\ref{lem:rigid_rotation_absorption_implies_structure} cannot hold at level $\varepsilon_0$
for the running-max cylinder (in particular, not on any time set $E$ with $|E|\ge c_0 r^2$).
\end{hypothesis}
\end{lstlisting}


\textbf{How to try to kill the absorption class (two concrete paths):}


\begin{itemize}
\item \textbf{(K‑ODE) ODE/affine-mode contradiction (preferred if possible):}
\item Use (ii) + (iii) to show the local flow is dominated by an explicit affine mode determined by $S_{\mathrm{tail}}^{(k)}(0,t)$.
\item Derive an ODE for the corresponding affine coefficient (compare to TeX Lemma \texttt{\detokenize{lem:linear-mode-ODE}}) and show it is incompatible with
    (a) ancientness, and/or (b) bounded vorticity + running-max normalization.
\item \textbf{Missing input to isolate if stuck:} a bound on the time-variation of $S_{\mathrm{tail}}^{(k)}(0,t)$ on $[-r^2,0]$
    (or a statement that it is approximately constant on short windows), so the ODE can be closed.
\item \textbf{(K‑E) Quasi‑2D pivot (safe fallback):}
\item Show that on sufficiently small cylinders $Q_r$, the absorption class forces $\xi$ to be close to a \emph{single}
    constant direction on a positive-measure subset (combine (i) with a short-time bound on $b(t)$).
\item Conclude the flow is locally quasi‑2D and invoke the E/2D Liouville elimination mechanism to rule it out.
\item \textbf{Missing input to isolate if stuck:} a quantitative “quasi‑2D on a cylinder ⇒ triviality” interface,
    which may depend on backward uniqueness / Liouville inputs (gate E).
\end{itemize}


\paragraph{K‑ODE — make the “affine mode” explicit (interfaces + targets)}


\begin{lstlisting}
\begin{hypothesis}[Time-variation control of the tail strain on a short window]\label{hyp:tail_strain_time_variation}
There exists a modulus $\omega:(0,1]\to(0,\infty)$ with $\omega(r)\to 0$ as $r\downarrow 0$ such that for every rescaled cylinder $Q_r$,
\[
\sup_{t,s\in[-r^2,0]} \bigl\|S_{\mathrm{tail}}^{(k)}(0,t)-S_{\mathrm{tail}}^{(k)}(0,s)\bigr\|
\ \le\ \omega(r).
\]
\end{hypothesis}
\end{lstlisting}


\begin{lstlisting}
\begin{hypothesis}[Bridge input for Hypothesis~\ref{hyp:tail_strain_time_variation}: uniform $L^2$ time-modulus for the tail vorticity]\label{hyp:tail_vorticity_L2_time_modulus}
For the running-max rescaled sequence (or ancient element) on $Q_1$, define the \emph{tail vorticity}
\[
\Omega^{(k)}(w,t):=\omega^{(k)}(w,t)\mathbf{1}_{\{|w|>1\}}.
\]
Assume:
\begin{enumerate}
\item[(i)] (\textbf{Well-definedness}) For every $k$ and every $t\in[-1,0]$, one has \(\Omega^{(k)}(\cdot,t)\in L^2(\R^3)\).
\item[(ii)] (\textbf{Scale-consistent time modulus}) There exists a modulus $\mu:(0,1]\to(0,\infty)$ with $\mu(r)\to 0$ as $r\downarrow 0$
such that for every $0<r\le 1$ and every $k$,
\[
\sup_{t,s\in[-r^2,0]}\ \|\Omega^{(k)}(\cdot,t)-\Omega^{(k)}(\cdot,s)\|_{L^2(\R^3)}\ \le\ \mu(r).
\]
\end{enumerate}
\end{hypothesis}
\end{lstlisting}


\textbf{Lean symbol:} \texttt{\detokenize{Bet2U4.TailVorticityL2TimeModulusHypothesis}} (alias of \texttt{\detokenize{Bet2U4.TailStrainTimeVariationHypothesis}} on the tail signal).


\textbf{Proof status (unconditional checklist item \#5):} currently \textbf{blocked}.
For the running-max rescaled sequence we have $\|\omega^{(k)}\|_{L^\infty}\lesssim 1$ and local smoothness on compact cylinders,
but converting that into a \emph{uniform-in-$k$} time-modulus in $L^2(\R^3;\{|w|>1\})$ is a real global/tail regularity statement.
If we can only prove a weaker time-modulus in a different norm, we should restate this hypothesis accordingly and redo
Lemma~\texttt{\detokenize{\ref{lem:tail_vorticity_L2_time_modulus_implies_tail_strain_time_variation}}} with the matching dual kernel norm.


\textbf{Single blocker:} prove a uniform-in-$k$ time modulus for the tail vorticity in an $L^2$-type tail norm (or restate the hypothesis in the weakest norm
that still controls the tail strain moment pairing).


\begin{lstlisting}
\begin{lemma}[Bridge: Hypothesis~\ref{hyp:tail_vorticity_L2_time_modulus} $\Rightarrow$ Hypothesis~\ref{hyp:tail_strain_time_variation}]\label{lem:tail_vorticity_L2_time_modulus_implies_tail_strain_time_variation}
Assume Hypothesis~\ref{hyp:tail_vorticity_L2_time_modulus}. Define $S_{\mathrm{tail}}^{(k)}(0,t)$ by the explicit tail moment formula
(Lemma~\ref{lem:tail-strain-formula} in \texttt{navier-dec-12-rewrite.tex}) applied to $\Omega^{(k)}(\cdot,t)$.
Then Hypothesis~\ref{hyp:tail_strain_time_variation} holds with modulus
\[
\omega(r):=C_{\mathrm{BS}}\mu(r),
\]
where $C_{\mathrm{BS}}$ is the $L^2$ norm of the kernel in Lemma~\ref{lem:tail_strain_time_variation_from_mildness} (universal).
\end{lemma}
\end{lstlisting}


\begin{lstlisting}
\begin{lemma}[Tail strain time-variation from a single $L^2$-continuity input (proved)]\label{lem:tail_strain_time_variation_from_mildness}
Let $\omega:\R^3\times[-1,0]\to\R^3$ be such that the truncated tail field
\[
\Omega(\cdot,t):=\omega(\cdot,t)\mathbf{1}_{\{|w|>1\}}
\]
lies in $L^2(\R^3)$ for every $t\in[-1,0]$, and assume the map $t\mapsto \Omega(\cdot,t)$ is uniformly continuous as a map
$[-1,0]\to L^2(\R^3)$.
Define the tail strain moment at the origin by the explicit Biot--Savart $\ell=2$ formula (Lemma~\ref{lem:tail-strain-formula} in
\texttt{navier-dec-12-rewrite.tex}):
\[
S_{\mathrm{tail}}(0,t)
:=-\frac{3}{8\pi}\int_{|w|>1}\frac{(w\times\Omega(w,t))\otimes w+w\otimes(w\times\Omega(w,t))}{|w|^5}\,dw.
\]
Then $t\mapsto S_{\mathrm{tail}}(0,t)$ is uniformly continuous on $[-1,0]$. In particular, for every $0<r\le 1$,
\[
\sup_{t,s\in[-r^2,0]}\bigl\|S_{\mathrm{tail}}(0,t)-S_{\mathrm{tail}}(0,s)\bigr\|
\ \le\ \omega_{\mathrm{tail}}(r)
\quad\text{with}\quad
\omega_{\mathrm{tail}}(r)\to 0\ \text{as }r\downarrow 0,
\]
so Hypothesis~\ref{hyp:tail_strain_time_variation} holds (with $S_{\mathrm{tail}}^{(k)}$ replaced by this $S_{\mathrm{tail}}$).
\end{lemma}

\begin{proof}
Write the moment as a linear pairing $S_{\mathrm{tail}}(0,t)=\int_{|w|>1}K(w)\,\Omega(w,t)\,dw$ where the matrix-valued kernel $K$ is the
linear map in $\Omega$ appearing in the displayed formula. Pointwise one has
\[
\|K(w)\|\ \lesssim\ |w|^{-3},
\]
hence $K\in L^2(\{|w|>1\})$ because $\int_{|w|>1}|w|^{-6}\,dw<\infty$.
Therefore Cauchy--Schwarz gives, for any $t,s\in[-1,0]$,
\[
\bigl\|S_{\mathrm{tail}}(0,t)-S_{\mathrm{tail}}(0,s)\bigr\|
\le \|K\|_{L^2(\{|w|>1\})}\,\|\Omega(\cdot,t)-\Omega(\cdot,s)\|_{L^2(\R^3)}.
\]
Uniform continuity of $t\mapsto\Omega(\cdot,t)$ in $L^2$ implies uniform continuity of $t\mapsto S_{\mathrm{tail}}(0,t)$.
Finally define
\[
\omega_{\mathrm{tail}}(r)
:=\|K\|_{L^2(\{|w|>1\})}\cdot
\sup_{\substack{t,s\in[-1,0]\\ |t-s|\le r^2}}\|\Omega(\cdot,t)-\Omega(\cdot,s)\|_{L^2(\R^3)}.
\]
Then $\omega_{\mathrm{tail}}(r)\to 0$ as $r\downarrow 0$ and the stated bound holds.
\end{proof}
\end{lstlisting}


\textbf{Acceptance test (unconditional-proof guard):}
the proof of Lemma~\texttt{\detokenize{\ref{lem:tail_strain_time_variation_from_mildness}}} must use only (a) time-continuity of $\omega$ in a norm that
dominates the \(|w|^{-4}\) tail pairing (e.g. $L^2$ works by Cauchy--Schwarz since \(\int_{|w|>1}|w|^{-8}<\infty\)),
and (b) the explicit tail moment formula.
If it silently imports higher regularity at $t=0$ or global Sobolev control, we keep it as a hypothesis interface and proceed with a different K‑ODE closure.


\begin{lstlisting}
\begin{hypothesis}[Absorption class implies an approximate 2.5D linear mode with a small Riccati defect (single blocker)]\label{hyp:absorption_implies_approx_riccati_defect}
Assume the absorption conclusion of Lemma~\ref{lem:rigid_rotation_absorption_implies_structure} holds on $Q_r$ at level $\varepsilon_0$.
After a \emph{fixed} rigid motion (rotation + Galilean shift), there exist functions
\[
\alpha_r,\ b_r \in W^{1,\infty}([-(r/2)^2,0])
\]
such that on the smaller cylinder $Q_{r/2}$ the third velocity component admits the decomposition
\[
u_3(x,t)=\alpha_r(t)+b_r(t)\,x_3 + w_r(x,t),
\]
and the following two estimates hold:
\begin{enumerate}[(i)]
\item (\textbf{Small remainder}) the remainder is small in a scale-invariant $H^1$ sense:
\[
\frac{1}{r}\iint_{Q_{r/2}}|\nabla w_r|^2\ \le\ \mathrm{Err}(r),
\]
for some rate function $\mathrm{Err}:(0,1]\to[0,\infty)$ with $\mathrm{Err}(r)\to 0$ as $r\downarrow 0$ along the absorption cylinders.
\item (\textbf{Approximate Riccati closure with controlled defect}) the linear-mode coefficient obeys
\[
\sup_{t\in[-(r/2)^2,0]}\bigl|\dot b_r(t)+b_r(t)^2\bigr|\ \le\ \mathrm{Err}(r),
\]
and the rate is controlled by already-paid quantities plus the tail-affine remainder:
\[
\mathrm{Err}(r)\ \le\ C\Bigl(\mathsf{Pay}_\xi(r)^{1/2}+\mathsf{Pay}_\rho(r)^{1/2}+r\Bigr),
\]
with a constant $C$ independent of $r$.
\end{enumerate}
\end{hypothesis}
\end{lstlisting}

\paragraph{Attempted derivation of the Riccati defect (pivot check; stop at first C2 dependence)}

\begin{lstlisting}
\begin{lemma}[Attempt: the Riccati defect reduces to a pressure-Hessian estimate (pivot trigger)]\label{lem:attempt_derive_approx_riccati_defect_pivot}
Assume the absorption conclusion of Lemma~\ref{lem:rigid_rotation_absorption_implies_structure} holds on $Q_r$ and assume we have
an approximate linear-mode decomposition of $u_3$ as in Hypothesis~\ref{hyp:absorption_implies_approx_riccati_defect}(i).
Let $b_r(t):=(\partial_3 u_3)(0,t)$ be the basepoint linear coefficient.

Then differentiating the third Navier--Stokes equation and evaluating at the basepoint gives the identity
\[
\dot b_r(t) + b_r(t)^2
= -(\partial_{33}p)(0,t)\;+\;\mathsf{Err}_{\mathrm{nl}}(r,t),
\]
where $\mathsf{Err}_{\mathrm{nl}}$ collects the non-2.5D remainder terms (transport/diffusion of $\partial_3 u_3$ and the mixed term
$(\partial_3 u)\cdot\nabla u_3 - b_r^2$).

If one could bound $|\mathsf{Err}_{\mathrm{nl}}(r,t)|$ purely by the already-paid quantities
$\mathsf{Pay}_\xi(r),\mathsf{Pay}_\rho(r)$ and the tail-affine remainder $O(r)$, then the remaining obstruction to
Hypothesis~\ref{hyp:absorption_implies_approx_riccati_defect}(ii) is controlling the pressure Hessian term $(\partial_{33}p)(0,t)$.

\smallskip
\noindent\textbf{Where the pivot trigger fires.}
By the pressure Poisson law $\Delta p = -\partial_i u_j\,\partial_j u_i$, the quantity $(\partial_{33}p)(0,t)$ is a Calder\'on--Zygmund singular integral
of the quadratic form $\partial_i u_j\,\partial_j u_i(\cdot,t)$ with kernel $\sim |x|^{-3}$.
Splitting into near-field and far-field contributions shows that to make $(\partial_{33}p)(0,t)$ small at scale $r$
one needs a scale-consistent smallness control of a local \emph{unweighted} budget term of the form
\[
\frac{1}{r^2}\iint_{Q_{2r}} \bigl(|\sigma|+|\nabla\xi|^2\bigr)\,dx\,dt
\quad\text{(or equivalently }\frac{1}{r^2}\iint_{Q_{2r}}|\nabla u|^2\,dx\,dt\text{ up to constants).}
\]
In the current running-max framework, the only available mechanism that controls this budget without additional global decay is the
\emph{vanishing positive injection} (C2/U-decay) gate:
\[
\frac{1}{r^2}\iint_{Q_{2r}} \rho^{3/2}\,\sigma_+ \ \to\ 0\qquad(r\downarrow 0),
\]
which is exactly the C2 pivot point (\texttt{\detokenize{hyp:C2_vanishing_injection}} / \texttt{\detokenize{hyp:U4_vanishing_injection_rate}}).

Therefore, the first nontrivial step in proving Hypothesis~\ref{hyp:absorption_implies_approx_riccati_defect}(ii) already collapses into controlling
\(\iint \rho^{3/2}\sigma_+\): the attempt certifies the \textbf{P-C2 pivot trigger}.
\end{lemma}
\end{lstlisting}

\begin{lstlisting}
\begin{lemma}[K-ODE gate: absorption $\Rightarrow$ local affine gauge + approximate Riccati mode (proved modulo one blocker)]\label{lem:absorption_implies_affine_ansatz}
Fix $r\in(0,1]$ and assume the absorption conclusion of Lemma~\ref{lem:rigid_rotation_absorption_implies_structure} holds on $Q_r$ at level $\varepsilon_0$.
Assume Hypothesis~\ref{hyp:absorption_implies_approx_riccati_defect}.

Define the affine gauge by the first-order Taylor polynomial at the basepoint:
\[
a_r(t):=u(0,t),\qquad A_r(t):=\nabla u(0,t),\qquad \ell_{0,r}(x,t):=a_r(t)+A_r(t)x.
\]
Then $\mathrm{tr}\,A_r(t)=0$ for all $t$ (since $\nabla\cdot u\equiv 0$).
Define the scalar linear mode (basepoint linear coefficient)
\[
b_r(t):=e_3\cdot A_r(t)\,e_3 = (\partial_3 u_3)(0,t).
\]
Then on $Q_{r/2}$:
\begin{enumerate}
\item[(i)] (Scale-critical small remainder) one has
\[
\frac{1}{r}\iint_{Q_{r/2}}|\nabla(u-\ell_{0,r})|^2 \ \le\ \delta(r),
\qquad
\delta(r):=C r^6 \sup_{Q_{r/2}}|\nabla^2 u|^2\ \xrightarrow{r\downarrow 0}\ 0,
\]
for an absolute constant $C$.
\item[(ii)] (Approximate Riccati mode) the scalar $b_r$ satisfies the approximate Riccati ODE on $[-(r/2)^2,0]$:
\[
\bigl|\dot b_r(t)+b_r(t)^2\bigr|\ \le\ \mathrm{Err}(r),
\]
where $\mathrm{Err}(r)\to 0$ as $r\downarrow 0$ and is controlled as in Hypothesis~\ref{hyp:absorption_implies_approx_riccati_defect}.
\end{enumerate}
\end{lemma}

\begin{proof}
\textbf{(i)} Fix $(x,t)\in Q_{r/2}$. Since $\ell_{0,r}(\cdot,t)$ is the first-order Taylor polynomial of $u(\cdot,t)$ at $x=0$,
Taylor’s theorem gives
\[
\nabla u(x,t)-\nabla u(0,t)
=\int_0^1 \nabla^2 u(\theta x,t)[x]\,d\theta.
\]
Hence
\[
|\nabla(u-\ell_{0,r})(x,t)|
=|\nabla u(x,t)-\nabla u(0,t)|
\le |x|\sup_{B_{r/2}}|\nabla^2 u(\cdot,t)|
\le \tfrac{r}{2}\sup_{Q_{r/2}}|\nabla^2 u|.
\]
Squaring and integrating over $Q_{r/2}$ (whose spacetime volume is $\sim r^5$) yields
\[
\iint_{Q_{r/2}}|\nabla(u-\ell_{0,r})|^2
\le C r^7 \sup_{Q_{r/2}}|\nabla^2 u|^2.
\]
Dividing by $r$ gives the stated bound. The limit $\delta(r)\to 0$ follows from smoothness of $u$ at the basepoint.

\smallskip
\textbf{(ii)} This is exactly the “approximate Riccati closure with controlled defect” conclusion in
Hypothesis~\ref{hyp:absorption_implies_approx_riccati_defect}(ii).
\end{proof}
\end{lstlisting}


\textbf{Lean symbol (spec interface):} \texttt{\detokenize{Bet2U4.AbsorptionImpliesAffineAnsatzHypothesis}}.


\textbf{Proof status (unconditional checklist item \#6):} currently \textbf{blocked}.\\
\textbf{Single blocker (K-ODE):} prove Hypothesis~\texttt{\detokenize{\ref{hyp:absorption_implies_approx_riccati_defect}}}
(approximate 2.5D linear mode + Riccati defect bound with $\mathrm{Err}(r)\to 0$).
Once the Riccati defect is uniformly small along arbitrarily small absorption cylinders, the ODE instability mechanism can be made quantitative
to rule out the absorption class (no C2/U-decay import required).


\begin{lstlisting}
\begin{lemma}[Affine-mode ODE contradiction (now a one-line corollary)]\label{lem:affine_mode_ode_contradiction}
Assume the conclusion of Lemma~\ref{lem:absorption_implies_affine_ansatz} holds for a running-max ancient cylinder at arbitrarily small scales $r\downarrow 0$.
Then the running-max ancient element is impossible (contradiction with vorticity normalization and/or ancientness), by applying the Riccati mode in
Lemma~\ref{lem:absorption_implies_affine_ansatz}(iii) and the ancient-solution sign constraints (compare \texttt{navier-dec-12-rewrite.tex},
Lemma~\texttt{\detokenize{\ref{lem:linear-mode-ODE}}} and Lemma~\texttt{\detokenize{\ref{lem:E1-b-negative-impossible}}}).
\end{lemma}
\end{lstlisting}


\textbf{Lean symbol (spec interface):} \texttt{\detokenize{Bet2U4.AffineAnsatzImpossibleHypothesis}}.


\textbf{Proof status (unconditional checklist item \#7):} this is no longer a separate blocker.\\
Once Lemma~\texttt{\detokenize{\ref{lem:absorption_implies_affine_ansatz}}} supplies a closed Riccati mode,
the contradiction reduces to the explicit ODE analysis already present in \texttt{\detokenize{navier-dec-12-rewrite.tex}}.


\textbf{Acceptance test for K‑ODE:} if we can’t prove Hypothesis~\texttt{\detokenize{\ref{hyp:tail_strain_time_variation}}} from bounded vorticity + tail control,
we should keep it as an explicit interface and move to K‑E.


\paragraph{K‑E — reduce absorption to the 2D/E gate}


\begin{lstlisting}
\begin{lemma}[Absorption class implies quasi-2D on a smaller cylinder (fully quantified target)]\label{lem:absorption_implies_quasi2d}
Fix $\eta\in(0,1/8)$ and $\varepsilon_0>0$.
There exist $\theta\in(0,1/8)$ and $\delta_{\mathrm{abs}}=\delta_{\mathrm{abs}}(\eta,\varepsilon_0,c_0,c_*)>0$ such that:
for any $r\in(0,1]$, if the absorption conclusion of Lemma~\ref{lem:rigid_rotation_absorption_implies_structure} holds on $Q_r$ at level
$\varepsilon_0$ (hence there is a time set $E'\subset[-r^2,0]$ with $|E'|\ge \tfrac12|E|\ge \tfrac{c_0}{2}r^2$ and a map
$b:E'\to\Sbb^2$ satisfying items (i)–(iii)),
then there exist a unit vector $b_0\in\Sbb^2$ and a time set $I_0\subset E'$ with $|I_0|\ge \tfrac{c_0}{4}r^2$ such that
\[
\int_{B_{\theta r}} \rho(x,t)^{3/2}\,|\xi(x,t)-b_0|^2\,dx\ \le\ \delta_{\mathrm{abs}}\,(\theta r)^3
\qquad\forall t\in I_0.
\]
Equivalently,
\[
\iint_{B_{\theta r}\times I_0} \rho^{3/2}\,|\xi-b_0|^2\ \le\ \delta_{\mathrm{abs}}\,(\theta r)^5.
\]
\end{lemma}
\end{lstlisting}


\begin{lstlisting}
\begin{hypothesis}[Backward uniqueness for bounded mild Navier--Stokes (Lei--Yang--Yuan, IMRN 2024)]\label{hyp:backward_uniqueness_NS_LYY}
Let $(u_1,p_1)$ and $(u_2,p_2)$ be bounded mild solutions on $\mathbb R^3\times[0,T]$ with bounded vorticities.
If $u_1(\cdot,T)=u_2(\cdot,T)$, then $(u_1,\nabla p_1)=(u_2,\nabla p_2)$ on $\mathbb R^3\times[0,T]$.
\end{hypothesis}
\end{lstlisting}


\begin{lstlisting}
\begin{hypothesis}[E-gate bridge (pure LYY/symmetry propagation): quasi-2D forces an exact symmetry of the final data]\label{hyp:E_quasi2d_to_exact_symmetry}
There exists $\delta_E>0$ such that if a running-max ancient element on $Q_1$ satisfies
\(\iint_{Q_1}\rho^{3/2}|\xi-b_0|^2\le \delta_E\)
for some $b_0\in\Sbb^2$, then there exists a \emph{nontrivial} Euclidean symmetry $g$ (translation/rotation/reflection)
such that the final-time data is exactly $g$-invariant:
\[
u(\cdot,0)\ =\ g_\# u(\cdot,0).
\]
Equivalently, if we define the symmetry-pushforward solution $u^g(x,t):=(g_\# u)(x,t)$ on $Q_1$,
then \(u^g(\cdot,0)=u(\cdot,0)\).

\smallskip
\noindent\textbf{Intended choice of $g$ (to force “2D”):}
take $g$ to be translation by a nonzero vector parallel to $b_0$, so $g$-invariance means the flow is independent of the $b_0$-direction.
\end{hypothesis}
\end{lstlisting}


\begin{lstlisting}
\begin{hypothesis}[E-gate elimination of the symmetry class]\label{hyp:E_symmetry_class_impossible}
No nontrivial running-max ancient element can be invariant (even at a single time-slice) under a nontrivial translation,
equivalently no nontrivial translation-invariant running-max ancient element exists.
\end{hypothesis}
\end{lstlisting}


\begin{lstlisting}
\begin{hypothesis}[E-gate (Lei--Yang--Yuan backward uniqueness): quasi-2D cylinder implies triviality]\label{hyp:E_quasi2d_elimination}
Assume Hypothesis~\ref{hyp:backward_uniqueness_NS_LYY}, Hypothesis~\ref{hyp:E_quasi2d_to_exact_symmetry}, and
Hypothesis~\ref{hyp:E_symmetry_class_impossible}.
Then no nontrivial running-max ancient element can satisfy the quasi-2D smallness event
\(\iint_{Q_1}\rho^{3/2}|\xi-b_0|^2\le \delta_E\) for any $b_0\in\Sbb^2$.
\end{hypothesis}
\end{lstlisting}


\textbf{Acceptance test for K‑E (LYY endpoint):}
- \texttt{\detokenize{hyp:backward_uniqueness_NS_LYY}} must be tied to a single concrete import target:
  \texttt{\detokenize{docs/lit/papers/2024-lei-yang-yuan-backward-uniqueness-final-data.md}} (Theorem 1.1, IMRN 2024).
- We must spell out (and either prove or isolate) the \textbf{solution-class bridge} needed to apply LYY:
  bounded mild on $\mathbb R^3\times[-1,0]$ + bounded vorticity (and any extra global bounds they require).
- The only genuinely hard geometric bridge is now localized to \texttt{\detokenize{\ref{hyp:E_quasi2d_to_exact_symmetry}}}:
  “small weighted direction variance on $Q_1$ forces an \emph{exact} translation symmetry at $t=0$”.
  \textbf{Reality check (unconditional-proof guard):} this is extremely strong and very likely false if the only input is an $L^2$-type
  smallness of $\xi-b_0$ on a cylinder. We keep it as an explicit interface (so the dependency is honest), but we do \textbf{not} plan to
  rely on it for an unconditional proof unless we later upgrade the absorption output to a genuinely rigidity-level statement at $t=0$.
  Accordingly, the main unconditional elimination effort should prioritize \textbf{K‑ODE}.
- \texttt{\detokenize{lem:absorption_implies_quasi2d}} must output exactly the smallness predicate used in
  \texttt{\detokenize{\ref{hyp:E_quasi2d_to_exact_symmetry}}} after rescaling to $Q_1$ (no hidden norms).


If proving (ii)+(iii) forces a C2-type bound (e.g. control of $\iint\rho^{3/2}\sigma_+$), we stop and pivot: this would certify that
Route (B-$\xi$) collapses into a C2-strength estimate.


This is the cleanest “Route (B‑ξ)” target: it only talks about the tail forcing term and the paid channel \(\rho^{3/2}|\nabla\xi|^2\).
If proving it requires exactly a C2-type estimate on \(\sigma_+\), we trigger the pivot.


\begin{itemize}
\item \textbf{(B‑ξ.1) Concrete first lemma to attempt (integrated version, avoids pointwise eigenvector language):}
\end{itemize}


\begin{lstlisting}
\begin{lemma}[Tail strain forcing implies twist payment on the top band (target for Route (B-$\xi$))]\label{lem:tail_strain_forcing_implies_twist_payment}
Fix $\eta\in(0,1/8)$ and $r\le 1$. Let $(\rho,\xi,u)$ be a rescaled running-max cylinder solution on $Q_r(0,0)$ with $\rho\le 1$.
Assume there is a time set $E\subset[-r^2,0]$ with $|E|\ge c_0 r^2$ and a radius $\delta\in(0,r)$ such that
\[
\rho(x,t)\ge 1-\eta \ \text{ and }\ \|S_{\mathrm{tail}}^{(k)}(x,t)\|\ge \varepsilon/2
\qquad\forall (x,t)\in (B_\delta\times E).
\]
Assume furthermore that on $\{\rho>0\}$ the direction equation holds in the exact form obtained from $\omega=\rho\xi$:
\[
\partial_t\xi + u\cdot\nabla\xi - \Delta\xi
\;=\; (I-\xi\otimes\xi)\,S\,\xi \;+\; \frac{2}{\rho}\,(\nabla\rho\cdot\nabla)\xi.
\]
Assume additionally that the perpendicular tail forcing is uniformly present on the top band:
\[
|(I-\xi\otimes\xi)\,S_{\mathrm{tail}}^{(k)}(x,t)\,\xi(x,t)|\ \ge\ c_*\,\varepsilon
\qquad\forall (x,t)\in (B_\delta\times E).
\]
Then there exists $c>0$ (depending only on $\eta,c_0$) such that the normalized twist-or-band payment obeys
\[
\mathsf{Pay}_\xi(r)\;+\;\mathsf{Pay}_\rho(r)\ \ge\ c\,\varepsilon^2.
\]
\end{lemma}
\end{lstlisting}


\textbf{Acceptance test:} either prove this lemma cleanly (best case), or isolate the \emph{single} missing analytic input as a hypothesis interface
(\texttt{\detokenize{hyp:forcing_to_twist_payment_on_top_band}}) and record whether it is equivalent to a known C2 endpoint.


\begin{itemize}
\item \textbf{(B‑2) The minimal “direction forcing ⇒ gradient” interface we may need} (keep it explicit):
  If we go Route (B‑ξ), we should isolate a single analytic lemma of the form:
  \[
  |(I-\xi\otimes\xi)S_{\mathrm{tail}}\xi|\ \gtrsim\ \varepsilon
  \quad\Longrightarrow\quad
  \iint_{Q_r}\rho^{3/2}|\nabla\xi|^2\ +\ \eta^{-1}\iint_{Q_r\cap\{1-2\eta<\rho<1-\eta\}}|\nabla(\rho^{3/4})|^2
  \ \gtrsim\ \varepsilon^2 r^2,
  \]
  i.e. the tail strain produces a nontrivial forcing term in the \(\xi\)-equation that must be “paid for” by the \(\rho^{3/2}|\nabla\xi|^2\)
  channel (and, if necessary, by the transition-band diffusion channel \(|\nabla(\rho^{3/4})|^2\)). If we can’t prove this without invoking
  a full C2-type estimate, we declare it as a hypothesis interface and track the pivot.
\item \textbf{(B‑3) Acceptance target for S3‑B (what “proved” means):}
  Provide either:
\item a \emph{time-slice pointwise} lemma: \(\|S_{\mathrm{tail}}\|\ge\varepsilon\) and \(\rho\ge 1-\eta\) imply \(\sigma\ge c\varepsilon\) or
    \(|\nabla\xi|^2\ge c\varepsilon^2\), \textbf{or}
\item a \emph{cylinder-integrated} lemma (often easier): persistence on \(E\) implies
    \[
    \frac1{r^2}\iint_{Q_r}\rho^{3/2}|\nabla\xi|^2 \ \ge\ c\,\varepsilon^2,
    \]
    unless there is a positive-measure subset of times where \(\sigma\ge c\varepsilon\) at top-band points.
  Either version is sufficient to feed into S3‑P.
\end{itemize}


\subparagraph{(S3‑P) Export ⇒ payment lower bound (the actual “self-falsification” step)}


This is the main new lemma Bet 2 needs. In U‑4 language:


\begin{lstlisting}
\begin{hypothesis}[S3‑P splitter (interface; Lean: \texttt{Bet2U4.ExportSplitHypothesis})]\label{hyp:export_split}
Fix abstract predicates
\(
\mathsf{ExportEvent},\mathsf{ForcingEvent},\mathsf{NegativeSigmaEvent},\mathsf{ConcavityEvent}:\R_{>0}\times(0,1]\to\{\text{true,false}\}.
\)
Assume that for all $\varepsilon>0$ and $r\in(0,1]$,
\[
\mathsf{ExportEvent}(\varepsilon,r)\ \Longrightarrow\
\mathsf{ForcingEvent}(\varepsilon,r)\ \lor\ \mathsf{NegativeSigmaEvent}(\varepsilon,r)\ \lor\ \mathsf{ConcavityEvent}(\varepsilon,r).
\]
\end{hypothesis}
\end{lstlisting}


\begin{lstlisting}
\begin{lemma}[Persistence of tail export forces cylinder payment (template)]\label{lem:export_forces_payment_on_cylinder}
Fix $r\le 1$. Assume that on a time set $E\subset[-r^2,0]$ of measure $|E|\ge c_0 r^2$ we have
$\|S^{(k)}_{tail}(0,t)\|\ge \varepsilon$.
Assume the local persistence lemma `lem:tail-strain-local-persistence` holds with a radius $\delta=\delta(\varepsilon)$,
so that $\|S^{(k)}_{tail}(x,t)\|\ge \varepsilon/2$ for all $|x|\le \delta$ and $t\in E$.

Assume also:
\begin{itemize}
\item the maximizer-selection + drift-control lemmas (S3‑T/S3‑U) needed to apply `lem:runningmax-injection-penalized`
      on a positive-measure subset of $E$;
\item the “no persistent null alignment” hypothesis \ref{hyp:no_persistent_null_alignment};
\item the “no rigid-rotation absorption” hypothesis \ref{hyp:no_rigid_rotation_absorption}
      (equivalently: apply Lemma~\ref{lem:forcing_payment_or_absorption} to reduce perpendicular forcing to
      payment-or-absorption);
\item (optional, if needed) a single hypothesis turning persistent *negative* injection on $\{\rho\approx 1\}$ into band payment:
\end{itemize}

\begin{hypothesis}[Negative injection on the top band forces band diffusion payment]\label{hyp:negative_sigma_forces_band_payment}
Fix $\eta\in(0,1/8)$. There exists $c>0$ such that for any $Q_r(0,0)$,
if on a set $B_\delta\times E$ with $|E|\ge c_0 r^2$ one has $\rho\ge 1-\eta$ and
\[
\sigma(x,t)\ \le\ -c_*\varepsilon \qquad\forall (x,t)\in B_\delta\times E,
\]
then the normalized band diffusion payment obeys
\[
\mathsf{Pay}_\rho(r)\ :=\ \frac1{r^2}\,\eta^{-1}\iint_{Q_r\cap\{1-2\eta<\rho<1-\eta\}}|\nabla(\rho^{3/4})|^2\ \ge\ c\,\varepsilon^2.
\]
\end{hypothesis}

Define the **normalized cylinder payments**
\[
\mathsf{Pay}_\xi(r)\ :=\ \frac1{r^2}\iint_{Q_r}\rho^{3/2}|\nabla\xi|^2,
\qquad
\mathsf{Pay}_\rho(r)\ :=\ \frac1{r^2}\,\eta^{-1}\iint_{Q_r\cap\{1-2\eta<\rho<1-\eta\}}|\nabla(\rho^{3/4})|^2.
\]

Then at least one of the following “outcomes” holds on the cylinder $Q_r$ (with constants independent of $r$):
\begin{enumerate}
\item[(P1)] (recognition strain) $\mathsf{Pay}_\xi(r)\ \ge\ c_1\,\varepsilon^2$,
\item[(P2)] (band diffusion) $\mathsf{Pay}_\rho(r)\ \ge\ c_2\,\varepsilon^2$,
\item[(A)]  (absorption class) the explicit absorption conclusion of Lemma~\ref{lem:rigid_rotation_absorption_implies_structure}
            holds (equivalently: the `AbsorptionClass` predicate holds at $(\varepsilon,r)$),
\item[(P3)] (peak concavity) there exists a time $t_\ast\in E$ such that at a (penalized) maximizer,
  $-\Delta\rho(\cdot,t_\ast)$ is $\gtrsim \varepsilon$ on a nontrivial ball (so its space integral on $B_r$ is $\gtrsim \varepsilon r^3$).
\end{enumerate}
\end{lemma}

\begin{proof}[Proof sketch]
Use Hypothesis~\ref{hyp:no_persistent_null_alignment} to promote $\|S_{\mathrm{tail}}\|\gtrsim\varepsilon$ on $B_\delta\times E$
to a positive-measure subset $U\subset B_\delta\times E$ where $|S_{\mathrm{tail}}\xi|\gtrsim \varepsilon$.
On $U$, decompose
\[
S_{\mathrm{tail}}\xi = (\xi\cdot S_{\mathrm{tail}}\xi)\,\xi + (I-\xi\otimes\xi)\,S_{\mathrm{tail}}\xi,
\]
so either $|\sigma|=|\xi\cdot S_{\mathrm{tail}}\xi|\gtrsim \varepsilon$ or $|(I-\xi\otimes\xi)S_{\mathrm{tail}}\xi|\gtrsim \varepsilon$.

If the perpendicular forcing is $\gtrsim\varepsilon$ on a time-thick top-band set, invoke
Hypothesis~\ref{hyp:no_rigid_rotation_absorption} (equivalently Lemma~\ref{lem:forcing_payment_or_absorption})
to obtain either a payment lower bound (P1/P2) or the absorption conclusion (A).

Otherwise, $|\sigma|\gtrsim\varepsilon$ on a time-thick top-band set. If $\sigma\le -c\varepsilon$ there, apply
Hypothesis~\ref{hyp:negative_sigma_forces_band_payment} to obtain (P2).
If not, then on a positive-measure subset of times one has $\sigma\ge c\varepsilon$ at selected near-max times/points (from S3‑T/S3‑U),
and the penalized maximizer inequality `lem:runningmax-injection-penalized` forces either large $|\nabla\xi|^2$ or large $-\Delta\rho$.
In the latter case, thick-maximum propagation yields the “peak concavity” outcome (P3).
\end{proof}
\end{lstlisting}


\textbf{Acceptance test:} the proof must be a \emph{three-case} bookkeeping argument:
(i) if the perpendicular forcing is large on the top band, invoke \texttt{\detokenize{\ref{hyp:no_rigid_rotation_absorption}}} (or Lemma~\texttt{\detokenize{\ref{lem:forcing_payment_or_absorption}}})
to get either (P1)/(P2) or (A);
(ii) if $\sigma$ is uniformly negative on the top band, invoke \texttt{\detokenize{\ref{hyp:negative_sigma_forces_band_payment}}} to get (P2);
(iii) otherwise the running-max injection inequality at selected near-max times forces (P3) via $-\Delta\rho$.
Any additional analytic input beyond these three named hypotheses is considered “hidden C2” and triggers a pivot.


\textbf{Proof status (unconditional checklist item \#2):} currently \textbf{blocked}.\\
\textbf{Single blocker:} prove the S3‑P splitter Hypothesis~\texttt{\detokenize{\ref{hyp:export_split}}} (Lean: \texttt{\detokenize{Bet2U4.ExportSplitHypothesis}}) for the running-max rescaled cylinder sequence,
without importing C2/UEWE.


\subparagraph{(S3‑Q) Cylinder-scale upper bound (the U‑4 contradiction point)}


Finally, U‑4 needs a cylinder-scale upper bound that rules out the payments (P1–P3) on \(Q_r\) for \(r\) small.
If we cannot prove it from already‑automatic running‑max bounds, we name it as a single explicit hypothesis:


\begin{lstlisting}
\begin{hypothesis}[U-4 cylinder payment bound (template)]\label{hyp:U4_payment_upper_bound}
Fix $\eta\in(0,1/8)$.
There exists a function $\beta_\eta(r)\to 0$ as $r\downarrow 0$ such that for every rescaled running-max solution,
the **normalized** cylinder payments obey
\[
\mathsf{Pay}_\xi(r)\ +\ \mathsf{Pay}_\rho(r)\ \le\ \beta_\eta(r).
\]
\end{hypothesis}
\end{lstlisting}


\textbf{Proof sketch status (honest):}
This is \emph{not} currently proved from bounded vorticity or from the running-max normalization alone.
It is exactly a \emph{vanishing Carleson} (VMO-type) statement for the nonnegative defect measure
\(\rho^{3/2}|\nabla\xi|^2\) plus the transition-band diffusion measure \(|\nabla(\rho^{3/4})|^2\).
Absent an external rigidity mechanism, it can fail even for smooth bounded-vorticity flows near vorticity zeros.


\textbf{Proof status (unconditional checklist item \#1):} currently \textbf{blocked}.\\
\textbf{Single blocker:} prove the U‑4 injection/payment-control package (Hypotheses~\texttt{\detokenize{\ref{hyp:U4_vanishing_injection_rate}}}
and \texttt{\detokenize{\ref{hyp:U4_payrho_controlled_by_injection_rate}}}) for the running-max rescaled sequence
(or else change the U‑4 payment normalization as discussed below).


\textbf{PayRho sub-status (we updated this gate):} \texttt{\detokenize{\ref{hyp:U4_payrho_controlled_by_injection_rate}}} is now written as a clean
interface with explicit definitions of $\mathsf{Pay}_\rho$ and $\mathsf{injRate}$, and its proof sketch reduces it (via
Corollary~\texttt{\detokenize{\ref{cor:band-payment-simplified}}} in \texttt{navier-dec-12-rewrite.tex}) to the single missing analytic step
\texttt{\detokenize{\ref{hyp:U4_band_budget_controlled_by_injection_rate}}}.  If \texttt{\detokenize{\ref{hyp:U4_band_budget_controlled_by_injection_rate}}} collapses into
C2/UEWE/U-decay, treat that as the explicit pivot trigger.


\begin{lstlisting}
\begin{hypothesis}[Vanishing positive injection on small cylinders (C2-stretch gate; scale-invariant form)]\label{hyp:C2_vanishing_injection}
There exists a function $\alpha:(0,1]\to[0,\infty)$ with $\alpha(r)\to 0$ as $r\downarrow 0$ such that for every rescaled running-max solution,
every cylinder center $z_0\in\R^3\times(-\infty,0]$, and every $0<r\le 1$,
\[
\iint_{Q_r(z_0)}\rho(x,t)^{3/2}\,\sigma_+(x,t)\,dx\,dt\ \le\ \alpha(r).
\]
\end{hypothesis}
\end{lstlisting}


\textbf{Cross-reference / scaling check:} in \texttt{\detokenize{navier-dec-12-rewrite.tex}} the (currently open) “C2 stretch hypothesis”
is recorded as \texttt{\detokenize{\eqref{eq:C2-stretch-hyp}}} in Lemma~\texttt{\detokenize{\ref{lem:C2-closure-from-stretch}}}:
\(\iint_{Q_r}\rho^{3/2}\sigma_+\le \alpha(r)\) with $\alpha(r)\to 0$. We have now matched that scale-invariant formulation here.
Note that our U‑4 “payments” were normalized by \(r^{-2}\); if we keep that normalization, then \texttt{\detokenize{\ref{hyp:C2_vanishing_injection}}} alone does
\emph{not} force \(\mathsf{Pay}_\xi(r)+\mathsf{Pay}_\rho(r)\to 0\) unless we additionally know the damping integrals are \(o(r^2)\).
Therefore, for an unconditional route, either:
(i) we redefine the U‑4 payments without the extra \(r^{-2}\) normalization (so vanishing follows from the scale-invariant C2 gate), or
(ii) we prove a genuinely stronger “rate” version of \texttt{\detokenize{\ref{hyp:C2_vanishing_injection}}}. We treat this normalization choice as a
first-order design decision (pivot trigger P‑C2).


\begin{lstlisting}
\begin{hypothesis}[Rate-strengthened vanishing positive injection (U-4 compatible form)]\label{hyp:U4_vanishing_injection_rate}
There exists a function $\gamma:(0,1]\to[0,\infty)$ with $\gamma(r)\to 0$ as $r\downarrow 0$ such that for every rescaled running-max solution,
every cylinder center $z_0\in\R^3\times(-\infty,0]$, and every $0<r\le 1$,
\[
\frac{1}{r^2}\iint_{Q_r(z_0)}\rho(x,t)^{3/2}\,\sigma_+(x,t)\,dx\,dt\ \le\ \gamma(r).
\]
\end{hypothesis}
\end{lstlisting}


\begin{lstlisting}
\begin{hypothesis}[U-4 payments are controlled by the normalized positive injection rate (interface)]\label{hyp:U4_payments_controlled_by_injection_rate}
Fix $\eta\in(0,1/8)$. There exist constants $C\ge 0$ and a function $\mathrm{Err}_\eta:(0,1]\to[0,\infty)$ with
$\mathrm{Err}_\eta(r)\to 0$ as $r\downarrow 0$ such that for every rescaled running-max solution and every $0<r\le 1$,
\[
\mathsf{Pay}_\xi(r)+\mathsf{Pay}_\rho(r)\ \le\
C\cdot\frac{1}{r^2}\iint_{Q_{2r}(0,0)}\rho(x,t)^{3/2}\,\sigma_+(x,t)\,dx\,dt\ +\ \mathrm{Err}_\eta(r).
\]
\end{hypothesis}
\end{lstlisting}


\textbf{Proof sketch status (honest):}\\
For $\mathsf{Pay}_\xi$, this is consistent with the weighted-coherence estimate
(\texttt{\detokenize{navier-dec-12-rewrite.tex}}, Lemma~\texttt{\detokenize{\ref{lem:weighted-coherence-bound}}}) plus running-max normalization $\rho\le 1$, which yields
\[
\mathsf{Pay}_\xi(r)\ \lesssim\ \frac{1}{r^2}\iint_{Q_{2r}} \rho^{3/2}\sigma_+\ +\ O(r).
\]
For $\mathsf{Pay}_\rho$, the known local band-payment inequalities (e.g. Corollary~\texttt{\detokenize{\ref{cor:band-payment-simplified}}} in
\texttt{\detokenize{navier-dec-12-rewrite.tex}}) currently control the band diffusion by terms involving $\iint |\sigma|$ and $\iint |\nabla\xi|^2$ on a larger cylinder,
and it is not yet proved that those can be reduced to the \emph{positive} weighted injection alone without importing the C2/U-decay gate.


Accordingly, for U‑4 we isolate the remaining missing piece as the band-payment control interface
Hypothesis~\texttt{\detokenize{\ref{hyp:U4_payrho_controlled_by_injection_rate}}}, and bundle it together with the injection-rate gate
Hypothesis~\texttt{\detokenize{\ref{hyp:U4_vanishing_injection_rate}}} into the single named blocker
Hypothesis~\texttt{\detokenize{\ref{hyp:U4_injection_payment_package}}}.


\begin{lstlisting}
\begin{lemma}[U-4 $\mathsf{Pay}_\xi$ control from weighted coherence (proved in \texttt{navier-dec-12-rewrite.tex})]\label{lem:U4_payxi_controlled_by_injection_rate}
Assume the setting of the running-max rescaled cylinder solution on $Q_{2r}(0,0)$ with $\rho\le 1$ and $0<r\le 1$.
Then there exists a universal constant $C$ such that
\[
\mathsf{Pay}_\xi(r)
\ :=\ \frac1{r^2}\iint_{Q_r(0,0)}\rho^{3/2}|\nabla\xi|^2
\ \le\ C\cdot\frac1{r^2}\iint_{Q_{2r}(0,0)}\rho^{3/2}\sigma_+\ +\ C\,r.
\]
\end{lemma}

\begin{proof}[Proof sketch]
Apply Lemma~\ref{lem:weighted-coherence-bound} in \texttt{navier-dec-12-rewrite.tex} with basepoint $(0,0)$.
The lower-order terms are $r^{-2}\iint_{Q_{2r}}\rho^{3/2}\lesssim r^{-2}\cdot r^5\lesssim r^3$ and
$\sup_t\int_{B_{2r}}\rho^{3/2}\lesssim r^3$ since $\rho\le 1$ and $|Q_{2r}|\sim r^5$.
Dividing by $r^2$ yields the stated $O(r)$ error.
\end{proof}
\end{lstlisting}


\textbf{Lean symbol (spec interface):} \texttt{\detokenize{Bet2U4.U4PayXiControlledByInjectionRateHypothesis}} (the Lean file keeps this as a Prop-only interface; the PDE proof lives in TeX).


\begin{lstlisting}
\begin{hypothesis}[U-4 $\mathsf{Pay}_\rho$ control by the normalized positive injection rate (interface)]\label{hyp:U4_payrho_controlled_by_injection_rate}
Fix $\eta\in(0,1/8)$ and consider a rescaled running-max cylinder solution on $Q_{2}(0,0)$ with vorticity
$\omega=\rho\,\xi$ (so $0\le \rho\le 1$ and $\rho(0,0)=1$) and stretching scalar $\sigma:=\xi\cdot S\xi$.
Define the **normalized positive injection rate**
\[
\mathsf{injRate}(r)\ :=\ \frac1{r^2}\iint_{Q_r(0,0)}\rho(x,t)^{3/2}\,\sigma_+(x,t)\,dx\,dt,
\]
and the **normalized band-diffusion payment**
\[
\mathsf{Pay}_\rho(r)
\ :=\ \frac1{r^2}\,\eta^{-1}\iint_{Q_r(0,0)\cap\{1-2\eta<\rho<1-\eta\}}|\nabla(\rho^{3/4})|^2\,dx\,dt.
\]
There exist a constant $C_\eta\ge 0$ and a function $\mathrm{Err}_{\rho,\eta}:(0,1]\to[0,\infty)$ with
$\mathrm{Err}_{\rho,\eta}(r)\to 0$ as $r\downarrow 0$ such that for every rescaled running-max solution and every $0<r\le 1$,
\[
\mathsf{Pay}_\rho(r)\ \le\ C_\eta\cdot \mathsf{injRate}(2r)\ +\ \mathrm{Err}_{\rho,\eta}(r).
\]
\end{hypothesis}
\end{lstlisting}


\textbf{Lean symbol (spec interface):} \texttt{\detokenize{Bet2U4.U4PayRhoControlledByInjectionRateHypothesis}}.


\textbf{Proof sketch (and honest status):} currently \textbf{blocked} on a single reduction step.\\
\textbf{Single blocker (for this PayRho gate):} \texttt{\detokenize{\ref{hyp:U4_band_budget_controlled_by_injection_rate}}}.\\
Apply Corollary~\texttt{\detokenize{\ref{cor:band-payment-simplified}}} in \texttt{navier-dec-12-rewrite.tex} at scale $r/2$ (so that the band integral is over $Q_r$)
to obtain
\[
\eta^{-1}\iint_{Q_r(0,0)\cap\{1-2\eta<\rho<1-\eta\}}|\nabla(\rho^{3/4})|^2
\ \lesssim_\eta\ r^3\ +\ \iint_{Q_{2r}(0,0)}\bigl(|\sigma|+|\nabla\xi|^2\bigr).
\]
Dividing by $r^2$ yields
\[
\mathsf{Pay}_\rho(r)\ \lesssim_\eta\ r\ +\ \frac1{r^2}\iint_{Q_{2r}(0,0)}\bigl(|\sigma|+|\nabla\xi|^2\bigr).
\]
Therefore, \texttt{\detokenize{\ref{hyp:U4_payrho_controlled_by_injection_rate}}} follows provided we can reduce the scale-critical “budget term”
\(\frac1{r^2}\iint_{Q_{2r}}(|\sigma|+|\nabla\xi|^2)\) to the \textbf{positive weighted injection} \(\mathsf{injRate}(2r)\) up to an $o(1)$ error,
without importing C2/UEWE/U-decay. We isolate that reduction as the single missing analytic input:


\begin{lstlisting}
\begin{hypothesis}[U-4 band-budget reduction to positive injection (single missing analytic step)]\label{hyp:U4_band_budget_controlled_by_injection_rate}
Fix $\eta\in(0,1/8)$. There exist constants $C_\eta\ge 0$ and a function $\mathrm{Err}_{\mathrm{bud},\eta}:(0,1]\to[0,\infty)$ with
$\mathrm{Err}_{\mathrm{bud},\eta}(r)\to 0$ as $r\downarrow 0$ such that for every rescaled running-max solution and every $0<r\le 1$,
\[
\frac1{r^2}\iint_{Q_{2r}(0,0)}\bigl(|\sigma|+|\nabla\xi|^2\bigr)\,dx\,dt
\ \le\ C_\eta\cdot \mathsf{injRate}(2r)\ +\ \mathrm{Err}_{\mathrm{bud},\eta}(r).
\]
\end{hypothesis}
\end{lstlisting}


\textbf{Lean symbol (spec interface):} \texttt{\detokenize{Bet2U4.U4BandBudgetControlledByInjectionRateHypothesis}} (added below).


\textbf{Proof status (honest):} currently \textbf{blocked}, and very likely a \textbf{pivot trigger}.\\
Controlling the unweighted scale-critical budget \(\iint(|\sigma|+|\nabla\xi|^2)\) by the \emph{positive} weighted injection
\(\iint\rho^{3/2}\sigma_+\) appears to require (at minimum) a mechanism that suppresses the tail and harmonic/affine
contributions to \(\sigma\) (cf.\ the $\sigma$-decomposition and Gate~S discussion in \texttt{navier-dec-12-rewrite.tex}),
and/or a separate argument eliminating the curl-free affine mode. If proving \texttt{\detokenize{\ref{hyp:U4_band_budget_controlled_by_injection_rate}}}
silently imports C2/UEWE/U-decay, treat that as the explicit “Bet2 collapses into that gate” pivot.


\subparagraph{RTD — Relative tail depletion in blow-up variables (global tail gate)}

\textbf{Why RTD is the earliest unconditional pivot trigger.}
The PayRho reduction \texttt{\detokenize{\ref{hyp:U4_band_budget_controlled_by_injection_rate}}} is blocked because the unweighted budget term
\(\frac1{r^2}\iint(|\sigma|+|\nabla\xi|^2)\) can contain borderline contributions from:
(i) the far-field Biot--Savart tail, and (ii) the curl-free harmonic/affine mode.
Any attempt to bound those terms purely by \(\mathsf{injRate}(r)=r^{-2}\iint\rho^{3/2}\sigma_+\) risks silently re-importing C2/U-decay/UEWE.

\smallskip
\noindent
\textbf{Single global gate (named): RTD.}
We therefore promote the missing global input to a \emph{standalone} hypothesis in the blow-up variables.

\begin{lstlisting}
\begin{hypothesis}[Relative tail depletion in blow-up variables (global gate)]\label{hyp:relative_tail_depletion_blowup}
Let $(u,p)$ be a smooth Navier--Stokes solution on $\R^3\times[0,T^*)$ with $T^*<\infty$.
Let $(x_k,t_k,\lambda_k)$ be a running-max blow-up sequence with $\lambda_k:=|\omega(x_k,t_k)|^{-1/2}$ and define the rescaled vorticities
\[
\omega^{(k)}(y,s):=\lambda_k^2\,\omega(x_k+\lambda_k y,\ t_k+\lambda_k^2 s)
\qquad (y\in\R^3,\ s\le 0).
\]
There exists a decreasing envelope $h:[1,\infty)\to(0,1]$ with $h(R)\to 0$ as $R\to\infty$ and
\[
\int_1^{\infty}\frac{h(R)}{R}\,dR\ <\ \infty
\]
such that for every $R\ge 1$,
\[
\sup_{k}\ \sup_{s\le 0}\ \sup_{|y|\ge R}\ |\omega^{(k)}(y,s)|\ \le\ h(R).
\tag{RTD}
\]
Equivalently (unwinding the rescaling), for every $R\ge 1$,
\[
\sup_{k}\ \sup_{t\le t_k}\ \sup_{|x-x_k|\ge \lambda_k R}\ \frac{|\omega(x,t)|}{|\omega(x_k,t_k)|}\ \le\ h(R).
\tag{RTD'}
\]
\end{hypothesis}
\end{lstlisting}

\smallskip
\noindent
\textbf{Immediate consequence (bookkeeping): RTD $\Rightarrow$ U-decay.}
This is exactly Lemma~\texttt{\detokenize{\ref{lem:decay-inheritance}}} / Eq.~\texttt{\detokenize{\eqref{eq:relative-tail-depletion}}} in
\texttt{navier-dec-12-rewrite.tex}.

\begin{lstlisting}
\begin{lemma}[RTD implies U-decay of the ancient limit (proved)]\label{lem:RTD_implies_U_decay}
Assume Hypothesis~\ref{hyp:relative_tail_depletion_blowup} and assume $\omega^{(k)}\to\omega^\infty$ locally uniformly on $\R^3\times(-\infty,0]$.
Then
\[
|\omega^\infty(y,s)|\ \le\ h(|y|)\qquad\forall (y,s)\in\R^3\times(-\infty,0],
\]
and the log-tail integrability condition in the U-decay gate holds with $g=h$.
\end{lemma}

\begin{proof}
Fix $(y,s)$ and set $R:=|y|$. By (RTD), $|\omega^{(k)}(y,s)|\le h(R)$ for all $k$; pass $k\to\infty$.
\end{proof}
\end{lstlisting}

\smallskip
\noindent
\textbf{Two classical brick lemmas (used to attack RTD).}
These are “off-the-shelf” harmonic analysis / spherical-harmonic linear algebra; they do not use any Navier--Stokes-specific input.

\smallskip
\noindent
\textbf{Sanity check (fail-fast): RTD is \emph{not} implied by the automatic running-max bounds.}
Even if we grant the two canonical “automatic” inheritances from running-max (bounded vorticity and the scale-critical local \(L^{3/2}\) vorticity budget),
RTD can still fail purely by a \emph{second bubble at infinity in blow-up variables}. This is why RTD/UEWE must be treated as a separate global gate.

\begin{lstlisting}
\begin{lemma}[Sanity check: bounded vorticity + local critical $L^{3/2}$ control do \emph{not} imply RTD (multi-bubble counterexample)]\label{lem:RTD_not_from_runningmax_bounds}
There exists a sequence of smooth, divergence-free vorticity fields
$\{\omega^{(k)}\}_{k\ge 1}$ on $\R^3\times(-\infty,0]$ such that:
\begin{enumerate}[(i)]
\item (Uniform $L^\infty$ bound) $\sup_k\|\omega^{(k)}\|_{L^\infty(\R^3\times(-\infty,0])}\le 1$.
\item (Uniform scale-critical local $L^{3/2}$ budget) There exists $K_0<\infty$ such that for every $k$,
\[
\sup_{z_0\in\R^3\times(-\infty,0]}\ \sup_{0<r\le 1}\ r^{-2}\iint_{Q_r(z_0)}|\omega^{(k)}|^{3/2}\,dx\,dt\ \le\ K_0.
\]
\item (Failure of RTD) For every decreasing envelope $h(R)\to 0$ as $R\to\infty$ one has
\[
\sup_{k}\ \sup_{s\le 0}\ \sup_{|y|\ge R}|\omega^{(k)}(y,s)|\ \not\le\ h(R)
\qquad\text{for all large }R.
\]
\end{enumerate}
\end{lemma}
%
\begin{proof}[Proof sketch (explicit two-bubble construction)]
Choose any smooth divergence-free $f:\R^3\to\R^3$ supported in $B_1(0)$ with $\|f\|_\infty\le 1$ and $|f(0)|=1$.
Let $R_k\to\infty$ and define $\omega^{(k)}(y,s):=f(y)+f(y-R_k e_1)$ (independent of $s$).
Since the supports are disjoint for $k$ large, $\|\omega^{(k)}\|_\infty=\|f\|_\infty\le 1$, proving (i).
For (ii), on each $Q_r$ with $r\le 1$, the integrand is bounded by $1$ and $\omega^{(k)}$ is supported in at most two unit balls, so
$\iint_{Q_r}|\omega^{(k)}|^{3/2}\lesssim r^5$, hence $r^{-2}\iint_{Q_r}\lesssim r^3\le 1$.
For (iii), given $R$ choose $k$ so that $R_k\ge R+1$; then $|\omega^{(k)}(R_k e_1,0)|=|f(0)|=1$.
\end{proof}
\end{lstlisting}

\smallskip
\noindent
\textbf{RTD: inheritance vs self-falsification (diagnostic).}
There are two natural ways one might hope to close RTD without assuming it.
The next two lemmas isolate the \emph{first} missing input in each route.

\begin{lstlisting}
\begin{lemma}[Attempt: inheriting RTD from physical-space decay fails; one needs a no-multi-bubble/tightness mechanism (pivot trigger)]\label{lem:RTD_inheritance_pivot}
Let $(u,p)$ be a smooth Navier--Stokes solution on $\R^3\times[0,T^*)$ with compactly supported $u_0$,
and let $(x_k,t_k,\lambda_k)$ be a running-max blow-up sequence with rescaled vorticities
\[
\omega^{(k)}(y,s)=\lambda_k^2\,\omega(x_k+\lambda_k y,t_k+\lambda_k^2 s).
\]
Even if one knows a strong physical-space decay statement at each fixed time
(\emph{e.g.} Gaussian decay $|\omega(x,t)|\lesssim e^{-|x|^2/(Ct)}$ for $t>0$),
this does \emph{not} imply RTD in blow-up variables, because for fixed $R$ the region
\(|y|\ge R\) corresponds to the physical annulus \(|x-x_k|\ge \lambda_k R\), which \emph{shrinks to the blow-up center} as $k\to\infty$.

Therefore, any inheritance proof of RTD must supply a genuinely new mechanism that rules out
``secondary bubbles'' at distance $O(\lambda_k R)$ from $x_k$, uniformly in $k$ and in backward times.
Equivalently: one must prove a blow-up-variable tightness statement, which is (up to reformulation) RTD itself.
\end{lemma}
\end{lstlisting}

\begin{lstlisting}
\begin{lemma}[Attempt: RTD via self-falsification reduces to a dynamical-instability / finite-budget lemma (pivot trigger)]\label{lem:RTD_self_falsification_pivot}
A self-falsification approach to RTD would argue:
``if a tail persists in blow-up variables (e.g. a persistent nonzero $\ell=2$ tail strain moment, or nonvanishing tail flux in the $A_b$ energy identity),
then some nonnegative functional $J(t)$ must grow backward and contradict the running-max constraints.''

In the current manuscript architecture, making this implication rigorous requires at least one of:
\begin{enumerate}[(i)]
\item a new \emph{dynamical instability} proposition in the $\ell=2$ sector that upgrades a model-profile self-stretching sign
into a coercive, robust monotonicity inequality for a tail functional (the missing `prop:l2-instability` discussed in
\texttt{book/docs/navier-dec-12-rewrite-alternative.tex}, Remark~\texttt{\detokenize{\ref{rem:l2-instability-open}}}); or
\item a \emph{finite-budget} bound that prevents ``infinite payment over infinite history'' for the running-max rescaled sequence
without assuming U-decay/RTD (in the current draft, the only established way to obtain such a bound is the C2 closure mechanism,
which is conditional on U-decay/RTD).
\end{enumerate}

Thus, absent a new instability/monotonicity lemma of type (i) or a new unconditional finite-budget lemma of type (ii),
self-falsification collapses back into the same global gate (C2/U-decay/RTD). This is the pivot trigger.
\end{lemma}
\end{lstlisting}

\begin{lstlisting}
\begin{lemma}[Attempt: an \emph{unconditional} C2 closure collapses into a tail/tightness gate (pivot trigger)]\label{lem:C2_unconditional_pivot_to_RTD}
Recall the C2 target (in blow-up variables / for the running-max ancient element):
there exists a modulus $\alpha(r)\to 0$ as $r\downarrow 0$ such that
\[
\sup_{z_0}\ \iint_{Q_r(z_0)}\rho^{3/2}\,\sigma_+\ \le\ \alpha(r).
\tag{C2}
\]

\smallskip
\noindent\textbf{Near-field is not the obstruction.}
In the current manuscript architecture (see \texttt{navier-dec-12-rewrite.tex}, Theorem~C2-closure),
the stretching scalar $\sigma$ is decomposed into a near-field part and a tail part.
The near-field contribution is Carleson-small at small scales under the running-max bound $\|\omega\|_\infty\le 1$
(this is the same mechanism as near-field forcing depletion: small scales + bounded vorticity $\Rightarrow$ $O(r^5)$).

\smallskip
\noindent\textbf{Tail boundedness is available, but tail smallness is not.}
The tail term is (up to harmless frozen-direction dependence) a Calder\'on--Zygmund truncation of $\omega$:
it admits $L^p$ boundedness uniformly in the truncation scale $r$,
but this yields only \emph{boundedness} of the tail contribution in critical norms, not \emph{smallness} as $r\downarrow 0$
(cf.\ \texttt{book/docs/navier-dec-12-rewrite-alternative.tex}, Lemma~\texttt{\detokenize{\ref{lem:tail-bounded}}} and surrounding discussion).

\smallskip
\noindent\textbf{Where the pivot trigger fires.}
To upgrade boundedness to the vanishing modulus $\alpha(r)\to 0$ in (C2), one must suppress the tail/harmonic/affine contribution
to $\sigma$ at small scales. In the current framework, every such suppression is (up to reformulation) a
\emph{tail/tightness gate in blow-up variables}:
relative tail depletion (RTD), U-decay, UEWE/RM2U, or an equivalent ``no multi-bubble / no export'' mechanism.

Therefore, an attempt to prove C2 \emph{unconditionally} does not bypass RTD:
it simply relocates the same missing global input to the step ``tail boundedness $\Rightarrow$ tail smallness''.
This certifies that the C2-vs-RTD choice is not two independent targets: closing C2 requires (at least) an RTD-type tail gate.
\end{lemma}
\end{lstlisting}

\smallskip
\noindent
\textbf{Verdict / next best step.}
Inheritance offers no leverage beyond restating RTD as a no-multi-bubble/tightness gate.
Therefore the best next step is to pursue the self-falsification direction by isolating and attempting the smallest checkable missing statement:
prove (or explicitly assume as a single named interface) the $\ell=2$ \emph{dynamical instability} proposition
(`prop:l2-instability` / Remark~\texttt{\detokenize{\ref{rem:l2-instability-open}}} in the alternative manuscript).
If that attempt again forces a C2-type estimate, we have a definitive certification that RTD/C2 is the true bottleneck.

\begin{lstlisting}
\begin{lemma}[Tail Biot--Savart is harmonic on the core and admits an affine approximation]\label{lem:tail_biot_savart_affine_core}
Fix $R\ge 2$ and let $\Omega:\R^3\to\R^3$ be smooth, divergence-free, and bounded.
Define the tail vorticity $\Omega_{\ge R}:=\Omega\,\mathbf 1_{\{|y|\ge R\}}$ and the associated Biot--Savart velocity
\[
u_{\ge R}(x):=\frac{1}{4\pi}\int_{|y|\ge R}\frac{(x-y)\times \Omega(y)}{|x-y|^3}\,dy.
\]
Then on the core ball $B_1(0)$:
\begin{enumerate}[(i)]
\item $u_{\ge R}$ is smooth, divergence-free, and curl-free. In particular $u_{\ge R}$ is harmonic on $B_1$.
\item Let $a:=u_{\ge R}(0)$ and $A:=\nabla u_{\ge R}(0)$. Then $A$ is symmetric and trace-free.
\item (Quantitative affine approximation.) There is an absolute constant $C$ such that for all $x\in B_1$,
\[
\bigl|u_{\ge R}(x) - (a + A x)\bigr|
\ \le\ C\,|x|^2 \int_{|y|\ge R}\frac{|\Omega(y)|}{|y|^4}\,dy.
\]
In particular, if $|\Omega|\le \|\Omega\|_\infty$, then
\[
\sup_{x\in B_1}\bigl|u_{\ge R}(x) - (a + A x)\bigr|
\ \le\ C\,\|\Omega\|_\infty \cdot R^{-1}.
\]
\end{enumerate}
\end{lemma}

\begin{proof}[Proof (sketch, fully classical)]
Because $\Omega_{\ge R}$ vanishes on $B_R$, we have $\curl u_{\ge R}=\Omega_{\ge R}=0$ and $\dv u_{\ge R}=0$ on $B_1$,
so $u_{\ge R}$ is harmonic there.
Support separation ($|x|\le 1<|y|$) makes the Biot--Savart kernel smooth in $x$ and allows differentiation under the integral sign
for $\nabla^2 u_{\ge R}$ with an integrand bounded by $C|\Omega(y)|\,|y|^{-4}$.
Thus $\sup_{B_1}|\nabla^2 u_{\ge R}|\le C\int_{|y|\ge R}|\Omega(y)|\,|y|^{-4}dy$.
Taylor’s theorem with remainder gives the affine approximation bound. Curl-free implies $\nabla u_{\ge R}$ is symmetric; divergence-free implies $\mathrm{tr}(\nabla u_{\ge R})=0$.
\end{proof}
\end{lstlisting}

\begin{lstlisting}
\begin{lemma}[$\ell=2$ transverse shell tests detect the $\ell=2$ transverse component]\label{lem:l2_shell_tests_detect_l2_component}
For $b\in \Sbb^2$ define the $\ell=2$ transverse test field on $\Sbb^2$
\[
\Phi_b(\theta):=(b\cdot\theta)(\theta\times b).
\]
Let $V:=\mathrm{span}\{\Phi_b:\ b\in\Sbb^2\}\subset L^2(\Sbb^2;\R^3)$ and let $P_V$ be the $L^2$ orthogonal projection onto $V$.
Then $V$ is finite-dimensional and there exists a constant $c_0>0$ such that for every $g\in L^2(\Sbb^2;\R^3)$,
\[
\sup_{b\in \Sbb^2}\left|\int_{\Sbb^2} g(\theta)\cdot \Phi_b(\theta)\,d\theta\right|
\ \ge\ c_0\,\|P_V g\|_{L^2(\Sbb^2)}.
\]
\end{lemma}

\begin{proof}[Proof (finite-dimensional norm equivalence)]
Since $V$ is finite-dimensional, choose finitely many directions $b_1,\dots,b_N$ so that $\{\Phi_{b_j}\}$ spans $V$.
All norms on $V$ are equivalent, so $\|f\|_{L^2}\le c_0^{-1}\max_j|\langle f,\Phi_{b_j}\rangle|$ for $f\in V$.
Apply this to $f=P_V g$ and use $\langle P_V g,\Phi_{b_j}\rangle=\langle g,\Phi_{b_j}\rangle$.
\end{proof}
\end{lstlisting}

\smallskip
\noindent
\textbf{Two closure routes (kill switches).}
\begin{itemize}
\item \textbf{Kill K‑ODE (preferred)}: persistent far-field vorticity $\Rightarrow$ persistent affine/tail forcing $\Rightarrow$ enter the absorption/affine class
      $\Rightarrow$ \emph{Riccati/ODE contradiction} (see the “K‑ODE” chain around Lemma~\texttt{\detokenize{\ref{lem:absorption_implies_affine_ansatz}}}).
\item \textbf{Kill Payment (fallback)}: tail forcing $\Rightarrow$ twist/band payment lower bound $\Rightarrow$ contradict a cylinder-scale payment upper bound.
      This route is currently \emph{treacherous}: the upper bound tends to collapse into C2/U-decay unless RTD (or an equivalent tail-suppression gate) is already available.
\end{itemize}

\smallskip
\noindent
\textbf{Payment route (lower-bound step) — best possible formulation + single blocker.}
The naive statement “persistent tail forcing $\Rightarrow$ large \(\iint \rho^{3/2}|\nabla\xi|^2\)” is false because tail forcing can be absorbed by a
purely time-dependent rotation of an almost-spatially-constant direction field (“rigid rotation absorption”).
Accordingly, the strongest honest interface is the dichotomy already isolated earlier as
\texttt{\detokenize{hyp:no_rigid_rotation_absorption}}:
\[
\text{(persistent perpendicular tail forcing on the top band)}
\ \Longrightarrow\
\text{(twist/band payment)}\ \lor\ \text{(enter an explicit absorption class)}.
\]
Thus the \emph{single blocker} for making the Payment lower bound unconditional is exactly the K‑ODE kill:
rule out the absorption class via \texttt{\detokenize{lem:absorption_implies_affine_ansatz}} (affine ansatz + Riccati contradiction).

\smallskip
\noindent
\textbf{Concrete “next lemma” (co-author-ready).}
Treat \texttt{\detokenize{lem:absorption_implies_affine_ansatz}} as the \emph{one} K‑ODE lemma to shoot:
it bundles the affine-gauge extraction and the Riccati-mode closure into a single checkable gate.
If this lemma can be proved (even under the mild auxiliary regularity stated there), the program becomes unconditional \emph{without} needing a separate
PayRho band-budget reduction.

\smallskip
\noindent
\textbf{Current best-route recommendation (co-author-facing).}
\emph{Prioritize Kill K‑ODE.} It targets an explicit structured class (affine/absorbing dynamics) and, if it closes, it removes the need to prove a
separate PayRho band-budget reduction. The Payment route is best treated as a diagnostic: if it forces C2/U-decay, that is the explicit pivot trigger.


\begin{lstlisting}
\begin{hypothesis}[U-4 injection/payment package (single named blocker)]\label{hyp:U4_injection_payment_package}
Assume Hypothesis~\ref{hyp:U4_vanishing_injection_rate} and Hypothesis~\ref{hyp:U4_payrho_controlled_by_injection_rate}.
(Together with Lemma~\ref{lem:U4_payxi_controlled_by_injection_rate}, this yields Hypothesis~\ref{hyp:U4_payment_upper_bound}.)
\end{hypothesis}
\end{lstlisting}


\begin{lstlisting}
\begin{lemma}[U-4 injection rate $\Rightarrow$ U-4 payment upper bound (template)]\label{lem:U4_payment_upper_bound_of_U4_vanishing_injection_rate}
Assume Hypotheses~\ref{hyp:U4_vanishing_injection_rate} and \ref{hyp:U4_payrho_controlled_by_injection_rate}.
Then Hypothesis~\ref{hyp:U4_payment_upper_bound} holds.
\end{lemma}

\begin{proof}[Proof sketch]
By Lemma~\ref{lem:U4_payxi_controlled_by_injection_rate} and Hypothesis~\ref{hyp:U4_vanishing_injection_rate} (applied at scale $2r$),
\[
\mathsf{Pay}_\xi(r)\ \lesssim\ \gamma(2r)\ +\ O(r).
\]
By Hypothesis~\ref{hyp:U4_payrho_controlled_by_injection_rate} and Hypothesis~\ref{hyp:U4_vanishing_injection_rate},
\[
\mathsf{Pay}_\rho(r)\ \lesssim_\eta\ \gamma(2r)\ +\ \mathrm{Err}_{\rho,\eta}(r).
\]
Adding yields \(\mathsf{Pay}_\xi(r)+\mathsf{Pay}_\rho(r)\to 0\) as \(r\downarrow 0\), which is exactly Hypothesis~\ref{hyp:U4_payment_upper_bound}.
\end{proof}
\end{lstlisting}


\textbf{Acceptance test (unconditional checklist item \#1):}
- Either (i) prove \texttt{\detokenize{\ref{hyp:U4_payment_upper_bound}}} directly from a scale-critical estimate already available for the running-max rescaled
  sequence (best case), or
- (ii) prove the two-input route: \texttt{\detokenize{\ref{hyp:U4_vanishing_injection_rate}}} plus the band-payment control interface
  \texttt{\detokenize{\ref{hyp:U4_payrho_controlled_by_injection_rate}}} (and use Lemma~\texttt{\detokenize{\ref{lem:U4_payxi_controlled_by_injection_rate}}} for $\mathsf{Pay}_\xi$),
  yielding \texttt{\detokenize{\ref{hyp:U4_payment_upper_bound}}} by Lemma~\texttt{\detokenize{\ref{lem:U4_payment_upper_bound_of_U4_vanishing_injection_rate}}}, or
- (iii) prove a weaker but sufficient version: for every fixed $\varepsilon_0>0$ there exists $r_0(\varepsilon_0,\eta)$ such that
  \(\mathsf{Pay}_\xi(r)+\mathsf{Pay}_\rho(r)\le \varepsilon_0\) for all $0<r\le r_0\),
  and track exactly which known “gate” it is equivalent to (C2/UEWE/RM2U).
If neither is possible, then U‑4 cannot close unconditionally and Bet2 collapses into that gate.


\textbf{Fast go/no-go for “unconditional”:} by Corollary~\texttt{\detokenize{\ref{cor:band-payment-simplified}}} in \texttt{navier-dec-12-rewrite.tex},
\texttt{\detokenize{\ref{hyp:U4_payrho_controlled_by_injection_rate}}} reduces to the single missing budget-reduction hypothesis
\texttt{\detokenize{\ref{hyp:U4_band_budget_controlled_by_injection_rate}}}. If proving \texttt{\detokenize{\ref{hyp:U4_band_budget_controlled_by_injection_rate}}}
forces C2/UEWE/U-decay, treat that as an explicit pivot trigger (Bet2 collapses into that gate).


\subparagraph{(S3‑C) One-cylinder contradiction lemma (U‑4 closure)}


This is the “end of Session S3” statement: combine (S3‑P) and (S3‑Q).


\begin{lstlisting}
\begin{lemma}[U-4 contradiction from persistent export (template)]\label{lem:U4_contradiction_from_export}
Assume the hypotheses of Lemma~\ref{lem:export_forces_payment_on_cylinder} and Hypothesis~\ref{hyp:U4_payment_upper_bound}
hold for a fixed $\eta\in(0,1/8)$ and some $\varepsilon_0>0$.
Then persistent export at level $\varepsilon_0$ is impossible: there is no $r\le 1$ and measurable $E\subset[-r^2,0]$ with
$|E|\ge c_0 r^2$ such that $\|S^{(k)}_{tail}(0,t)\|\ge \varepsilon_0$ for all $t\in E$.
\end{lemma}
\end{lstlisting}


\textbf{Acceptance test:} the proof is a one-line calibration argument:
take $\varepsilon\le \varepsilon_0$, choose $r$ small so that $\beta_\eta(r) < \min(c_1,c_2)\varepsilon^2$ and also $r\ge \delta(\varepsilon)$,
then (S3‑P) contradicts (S3‑Q). (We must explicitly check that \(\delta(\varepsilon)\to 0\) as \(\varepsilon\to 0\), which is true if
\(\delta(\varepsilon)=\varepsilon/(2L)\) with \(L\) independent of \(\varepsilon\).)


\textbf{Lean packaging note (checklist item \#8):}
the abstract U‑4 contradiction is now mirrored by a pure-bookkeeping lemma in
\texttt{\detokenize{IndisputableMonolith/NavierStokes/RM2U/NonParasitism.lean}}:
\texttt{\detokenize{Bet2U4.U4NoExportHypothesis.of_u4}} and \texttt{\detokenize{RM2U.bet2SelfFalsification_of_u4}}.
What remains (and is genuinely PDE/compactness) is the bridge from the time-slice profile statement
\texttt{\detokenize{¬TailFluxVanish}} to an \texttt{\detokenize{ExportEvent}} occurring at arbitrarily small scales (\texttt{\detokenize{Bet2U4.TailFluxNonVanishImpliesExportAtSmallScales}}).


\begin{lstlisting}
\begin{hypothesis}[Tail-flux nonvanishing $\Rightarrow$ export at arbitrarily small scales (bridge interface)]\label{hyp:tailflux_nonvanish_implies_export_at_small_scales}
Fix an abstract predicate $\mathsf{ExportEvent}:(0,\infty)\times(0,1]\to\{\text{true,false}\}$.
Assume that for the running-max rescaled sequence (or limiting ancient element),
\[
\neg\ \mathrm{TailFluxVanish}\ \Longrightarrow\
\exists \varepsilon_0>0\ \ \forall r_0\in(0,1]\ \ \exists r\in(0,r_0]\ \text{such that}\ \mathsf{ExportEvent}(\varepsilon_0,r).
\]
\end{hypothesis}
\end{lstlisting}


\textbf{Lean symbol:} \texttt{\detokenize{Bet2U4.TailFluxNonVanishImpliesExportAtSmallScales}}.


\textbf{Proof status (unconditional checklist item \#8):} Lean bookkeeping is \textbf{done}, but the PDE bridge is \textbf{blocked}.\\
\textbf{Single blocker:} prove \texttt{\detokenize{TailFluxNonVanishImpliesExportAtSmallScales}} for the running-max rescaled sequence.


\subsubsection{Session S4 — Package the result into the repo (docs + Lean signature)}


\textbf{Deliverables:}


\begin{itemize}
\item Update \texttt{\detokenize{docs/RM2U_BET_2_ATTACK.md}} to end with the exact lemma statement you proved (or are now targeting).
\item Update \texttt{\detokenize{docs/RM2U_NEXT_LEMMA.md}} if the “next lemma” has changed shape.
\item Optionally add a more refined Lean interface if we discover Bet 2 naturally splits into two sub‑lemmas
  (but keep \emph{all} hypotheses in \texttt{\detokenize{RM2U.NonParasitism}}).
\end{itemize}


\textbf{Acceptance test:}


\begin{itemize}
\item A single “shoot-at-this” Lean statement (signature only is OK) that corresponds 1‑to‑1 with the TeX proposition.
\end{itemize}



\bigskip\hrule\bigskip


\subsection{5. Risk register (how this can fail, explicitly)}


\begin{itemize}
\item \textbf{Circularity risk:} Bet 2 collapses into C2 (or UEWE) and therefore does not close RM2U independently.
  We will treat that as a discovery and update the dependency graph accordingly.
\item \textbf{Sign robustness risk:} any argument relying on \(I[\omega]\ge 0\) must be made stable beyond a toy profile.
\item \textbf{Control-at-maximizers risk:} converting a nonzero tail strain into uniform statements at maximizers may require
  extra regularity bounds along maximizers (not currently derived in the TeX, see remarks near C2).
\item \textbf{Time‑regularity escape hatch:} the endpoint maximal-regularity/time-regularity route (\texttt{\detokenize{rem:RM2-l2-moment}})
  might become the “real” missing theorem if Bet 2 can’t supply a clean monotone contradiction.
\item \textbf{Missing-notes risk:} the draft references external working notes/files (e.g. \texttt{\detokenize{P0_PLAN_ONE_CORE_DOMINANCE.md}},
  \texttt{\detokenize{WORK_L2_SELF_STRETCHING.tex}}) that are not currently in this repo. Bet 2 must remain executable without them:
  any dependency on those notes must be re-derived into this repository (as a lemma/proof or as a doc).
\end{itemize}



\bigskip\hrule\bigskip


\subsection{6. Pivot triggers (so we don’t burn months on the wrong target)}


Bet 2 is allowed to “fail fast” in a disciplined way.


We pivot away from the Family‑B self‑falsification route if, by the end of Sessions S1–S2, we discover:


\begin{itemize}
\item \textbf{(P‑C2) The bridge step is equivalent to C2:}\\
  our attempted bridge from \texttt{\detokenize{S_tail}} to a persistent lower bound on \texttt{\detokenize{σ_+}} on \(\{\rho\approx 1\}\) requires
  exactly the same missing control of \texttt{\detokenize{\iint ρ^{3/2} σ_+}} that the manuscript currently closes only using U‑decay.
\item \textbf{(P‑UEWE) The bridge step is equivalent to UEWE/RM2U already:}\\
  i.e. proving it would already imply the coercive ℓ=2 bound or UEWE in a circular way.
\item \textbf{(P‑compactness) The mode choice collapses:}\\
  neither Mode (M‑seq) nor Mode (M‑limit‑with‑affine) can be executed without assuming the very fixed‑frame
  compactness RM2 is supposed to provide.
\end{itemize}


If any pivot trigger fires, the plan is:


\begin{itemize}
\item switch to \textbf{Family A} (ℓ=2 coefficient energy / boundary flux) or to the \textbf{time‑regularity} route
  flagged in TeX \texttt{\detokenize{rem:RM2-l2-moment}}, and update the dependency graph in \texttt{\detokenize{docs/RM2U_NEXT_LEMMA.md}}.
\end{itemize}



\bigskip\hrule\bigskip


\subsection{7. Diminishing returns + how to prompt the AI to work on this}


\subsubsection{7.1 Should we keep fortifying the plan?}


We have now completed the \textbf{last high‑value fortification round}:


\begin{itemize}
\item default choices are recorded in \textbf{Session S0} (Mode/Target/Export/L‑upper),
\item the L‑upper candidates are explicit (Section 3.10.4),
\item S2 has a concrete next lemma list beyond the penalized-max lemma.
\item \textbf{Diminishing returns trigger (when to stop planning and start executing):}
\item we have a \emph{single} next lemma to prove (already present: \texttt{\detokenize{lem:runningmax-injection-penalized}}),
\item we have a \emph{single} candidate L‑upper bound selected (U‑1…U‑4),
\item we have a named pivot trigger if it collapses into C2/UEWE/RM2.
\end{itemize}


We’re now at that trigger: the best progress comes from \textbf{actually attempting the lemmas}, not adding more prose.


\subsubsection{7.2 Prompt template (copy/paste)}


When you want me to work on Bet 2, use a prompt that pins down: \textbf{mode}, \textbf{target strength}, and \textbf{deliverable}.


Here are three templates that work well.


\paragraph{Template A — “Work the next session”}


Use this when you want me to advance the plan by one session step.


\begin{quote}
Context: we are working on RM2U Bet 2. Read \texttt{\detokenize{docs/RM2U_BET_2_EXECUTION_PLAN.md}} and follow it.\\
Mode: (M‑seq) or (M‑limit‑with‑affine). Target: (Weak) or (Strong).\\
Task: execute Session S2 (or S3) and produce the deliverables + acceptance tests.\\
Constraints: keep all missing pieces isolated as hypothesis interfaces in \texttt{\detokenize{RM2U.NonParasitism}}; don’t introduce sorries/axioms.
\end{quote}


\paragraph{Template B — “Prove/clean up one lemma”}


Use this when you want a concrete proof artifact (TeX or Lean) rather than planning.


\begin{quote}
Prove (or at least fully derive) the lemma \texttt{\detokenize{lem:runningmax-injection-penalized}} from \texttt{\detokenize{docs/RM2U_BET_2_EXECUTION_PLAN.md}}.\\
Output: a clean TeX proof sketch plus (if feasible) the Lean skeleton signature we’d eventually implement.\\
Also update the execution plan doc with any missing hypotheses discovered.
\end{quote}


\paragraph{Template C — “Sanity check for circularity”}


Use this when you’re worried a route is secretly C2/UEWE.


\begin{quote}
Evaluate candidate upper bound (U‑X) from Section 3.10.4 of \texttt{\detokenize{docs/RM2U_BET_2_EXECUTION_PLAN.md}}.\\
Tell me whether it collapses into C2, UEWE, or RM2, and if so, trigger the pivot and propose the next-best route.
\end{quote}


\subsubsection{7.3 Minimal context to include if you want fast progress}


If you include just these file pointers, I can move quickly:


\begin{itemize}
\item \texttt{\detokenize{docs/RM2U_BET_2_EXECUTION_PLAN.md}}
\item \texttt{\detokenize{docs/RM2U_NEXT_LEMMA.md}}
\item \texttt{\detokenize{navier-dec-12-rewrite.tex}} (anchors listed in Section 0.1)
\item Lean: \texttt{\detokenize{IndisputableMonolith/NavierStokes/RM2U/{Core,EnergyIdentity,RM2Closure,NonParasitism}.lean}}
\end{itemize}


\end{document}
