\documentclass[11pt]{article}

\usepackage[T1]{fontenc}
\usepackage[utf8]{inputenc}
\usepackage{lmodern}
\usepackage{microtype}
\usepackage[margin=1in]{geometry}
\usepackage{amsmath,amssymb}
\usepackage[hidelinks]{hyperref}

\title{Recognition Science Track Strategy:\\
Inevitability \texorpdfstring{$\to$}{->} Exclusivity \texorpdfstring{$\to$}{->} Stability Audit}
\author{Jonathan Washburn}
\date{\today}

\begin{document}
\maketitle

\section*{Purpose}
This short note explains why the following three papers are grouped as a
strategic track in Recognition Science (RS), and how each independently advances
both RS and the broader scientific method:
\begin{enumerate}
  \item \textit{D'Alembert Inevitability}
  \item \textit{Model-Independent Exclusivity on the Quotient State Space}
  \item \textit{The Recognition Stability Audit (RSA)}
\end{enumerate}
Together they turn RS from a proposed model into a forced, observable, and
auditable framework.

\section*{1. D'Alembert Inevitability}
\textbf{Role in the track (strategy).} This paper removes a modeling choice at
the base of the RS stack: the form of the composition law for the cost
functional. It shows the law is forced (up to a single scalar) by symmetry and
polynomial consistency, then reduces the law to classical d'Alembert structure.
With calibration, the canonical cost follows.

\textbf{RS advance.} The key RS move is that the Recognition Composition Law is
no longer a postulate. It becomes a theorem that locks the J-cost pipeline and
secures the cost-first foundation of the ledger dynamics.

\textbf{General scientific advance.} This is a classification result for
functional equations: under mild structure, only the bilinear family survives.
It answers ``why this law and not another?'' without parameter tuning.

\textbf{Anchor excerpt.} The abstract states the inevitability claim and the
forced bilinear family (see \texttt{DAlembert\_Inevitability.tex}, Abstract).

\section*{2. Model-Independent Exclusivity on the Quotient State Space}
\textbf{Role in the track (strategy).} This paper elevates RS from a specific
model to a ``no alternatives'' theorem, but phrased correctly: at the level of
observational equivalence. It shows that any zero-parameter framework with the
minimal RS structure collapses to RS on the quotient.

\textbf{RS advance.} With zero parameters, self-similarity, and the Recognition
Composition Law, the cost is forced to $J$ and the preferred scale to
$\varphi$. Thus all admissible states are observationally equivalent on the
quotient, making RS inevitable at the interface level.

\textbf{General scientific advance.} It provides a model-independent template
for exclusivity claims in physics: compare theories by their observable
quotients rather than internal representations.

\textbf{Anchor excerpt.} The abstract records the forced $J$ and $\varphi$ and
the quotient collapse (see
\texttt{Model-Independent-Exclusivity-Quotient.tex}, Abstract).

\section*{3. The Recognition Stability Audit (RSA)}
\textbf{Role in the track (strategy).} This paper turns the RS cost principle
into a practical auditing pipeline. It is the ``impossibility compiler'' that
converts existence claims into finite certificates using a canonical sensor,
Cayley transforms, and Schur/Pick control.

\textbf{RS advance.} RSA makes RS operational: it provides a reusable mechanism
to rule out candidate states that would require infinite recognition cost. It
grounds the audit in the 8-tick realizability class so the certificates are
finite and checkable.

\textbf{General scientific advance.} It introduces a cross-disciplinary method
linking complex analysis, control theory, and computational certificates into a
decision-style pipeline for impossibility.

\textbf{Anchor excerpt.} The abstract describes the compiler structure and the
three-layer audit (see \texttt{Recognition\_Stability\_Audit.tex}, Abstract).

\section*{How they fit together}
The track forms a tight progression:
\begin{itemize}
  \item \textbf{Inevitability} fixes the lawful cost structure.
  \item \textbf{Exclusivity} proves that any zero-parameter framework with the
  same structural commitments is observationally equivalent to RS.
  \item \textbf{RSA} supplies a concrete auditing tool that makes the framework
  testable by finite certificates.
\end{itemize}
This sequence converts RS from ``a model with choices'' into a forced and
auditable theory.

\end{document}
