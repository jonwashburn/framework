\documentclass[11pt]{article}
\usepackage[margin=1in]{geometry}
\usepackage[hidelinks]{hyperref}
\usepackage{parskip}
\setlength{\parindent}{0pt}
\title{Recognition Science Papers\\(193 Papers)}
\author{}
\date{Generated February 04, 2026}
\begin{document}
\maketitle
\vspace{1em}
\textbf{1. The Cost of Existence: A First-Principles Derivation of Physical Law from the Recognition Composition Law}: \texttt{The\_Cost\_of\_Existence.tex} (Authorship: Feb 03, 2026)
\\\textit{Why}: Standard physical theories typically postulate the existence of a manifold, a set of logical axioms, and initial conditions as irreducible priors.
\vspace{0.8em}

\textbf{2. Dimensional Rigidity as a Selection Principle in Recognition Geometry}: \texttt{papers/tex/Draft\_1\_Jan\_29.tex} (Authorship: Feb 02, 2026)
\\\textit{Why}: blue Why is physical space three-dimensional? We show that is singled out when observable space is constructed from measurement processes rather than assumed a priori.
\vspace{0.8em}

\textbf{3. The Golden Ratio as a Universal Coherence Eigenvalue: Bridging Penrose Aperiodic Order and Information-Theoretic Comparison}: \texttt{papers/tex/Penrose\_golden\_ratio\_and\_ledger\_structure.tex} (Authorship: Feb 02, 2026)
\\\textit{Why}: The golden ratio ( =1+ 52) occupies a distinguished position in mathematics, appearing across diverse domains from number theory and dynamical systems to geometric tilings and quasicrystal physics.
\vspace{0.8em}

\textbf{4. Full First Principles Mass Derivation}: \texttt{papers/tex/Full\_First\_Principles\_Mass\_Derivation.tex} (Authorship: Jan 31, 2026)
\\\textit{Why}: The Standard Model of particle physics is remarkably successful but structurally incomplete: it requires the masses of fermions to be inserted as free parameters (Yukawa couplings).
\vspace{0.8em}

\textbf{5. Charged Fermion Masses from Octave Closure and -Ladder Geometry A Recognition Science Framework with Single-Anchor Phenomenological Validation (editable reconstruction from PDF)}: \texttt{papers/tex/allahyarov\_integrated\_corrected.tex} (Authorship: Jan 31, 2026)
\\\textit{Why}: The Standard Model treats the nine charged fermion masses as empirical inputs.
\vspace{0.8em}

\textbf{6. Full Inevitability Paper}: \texttt{Full\_Inevitability\_Paper.tex} (Authorship: Jan 29, 2026)
\\\textit{Why}: We prove the complete inevitability theorem for the Recognition Composition Law (RCL).
\vspace{0.8em}

\textbf{7. The Geometric Necessity of a Recognition Blind Cone from}: \texttt{Recognition-Blind-Cone-arXiv.tex} (Authorship: Jan 29, 2026)
\\\textit{Why}: We prove that finite-cost recognition in imposes a strictly positive minimal angle between two compared directions, inducing a budget-dependent geometric blind cone around exact collinearity.
\vspace{0.8em}

\textbf{8. Geometric Necessity of Recognition Angle}: \texttt{papers/tex/Geometric-Necessity-Recognition-Angle.tex} (Authorship: Jan 27, 2026)
\\\textit{Why}: We prove that the recognition angle is forced in the highest sense: it is the unique value consistent with minimal axioms, with no free parameters.
\vspace{0.8em}

\textbf{9. Simulated Efficacy of Coherence-Controlled Fusion Upgrades to National Ignition Facility (NIF) Parameters}: \texttt{fusion/papers/tex/NIF\_Upgrade\_Simulation\_Report.tex} (Authorship: Jan 26, 2026)
\\\textit{Why}: This report presents a proxy-model sensitivity study of potential yield enhancement at the National Ignition Facility (NIF) under the Recognition Science (RS) Coherence Control hypothesis (Patents ...
\vspace{0.8em}

\textbf{10. Eight-Component Quasi-Periodic Train and a Candidate \$ 5}: \texttt{papers/FRB\_Ledger\_Signatures\_20190122C.tex} (Authorship: Jan 26, 2026)
\\\textit{Why}: Using the published component-level pulse table for FRB 20190122C (Xiao et al.
\vspace{0.8em}

\textbf{11. Attractor Dynamics in Stellar Nucleosynthesis}: \texttt{fusion/papers/Attractor\_Dynamics\_Stellar\_Nucleosynthesis.tex} (Authorship: Jan 25, 2026)
\\\textit{Why}: We present a graph-theoretic analysis of stellar nucleosynthesis that explains observed abundance patterns without parameter fitting.
\vspace{0.8em}

\textbf{12. Gibbs Sensor Fusion}: \texttt{fusion/papers/Gibbs\_Sensor\_Fusion.tex} (Authorship: Jan 25, 2026)
\\\textit{Why}: We present a principled framework for multi-sensor fusion based on Gibbs weighting from Recognition Science.
\vspace{0.8em}

\textbf{13. Robustness of Golden-Ratio Pulse Sequencing in Noisy Environments}: \texttt{fusion/papers/Golden\_Ratio\_Pulse\_Sequencing\_Robustness.tex} (Authorship: Jan 25, 2026)
\\\textit{Why}: We present a rigorous mathematical analysis of pulse sequencing using Golden Ratio () interval timing in pulsed energy systems.
\vspace{0.8em}

\textbf{14. Nuclear Magic Numbers from Ledger Topology: A Recognition Science Derivation}: \texttt{fusion/papers/Nuclear\_Magic\_Numbers\_RS\_Derivation.tex} (Authorship: Jan 25, 2026)
\\\textit{Why}: We derive the nuclear magic numbers from Recognition Science (RS) first principles, demonstrating that these stability markers emerge from the same 8-tick ledger topology that forces noble gas clos...
\vspace{0.8em}

\textbf{15. Nuclear Magic Numbers from First Principles: New Understanding of Nuclear Stability and Fusion Pathways}: \texttt{fusion/papers/Nuclear\_Magic\_Numbers\_Stability\_Fusion.tex} (Authorship: Jan 25, 2026)
\\\textit{Why}: We present a first-principles derivation of the nuclear magic numbers from Recognition Science (RS) ledger topology.
\vspace{0.8em}

\textbf{16. Topological Origins of Nuclear Binding Energy Corrections}: \texttt{fusion/papers/Topological\_Origins\_Nuclear\_Binding\_Energy.tex} (Authorship: Jan 25, 2026)
\\\textit{Why}: We present a novel derivation of nuclear shell corrections to the semi-empirical mass formula (SEMF) from first principles, based on a discrete topological structure we call the ``8-tick ledger.
\vspace{0.8em}

\textbf{17. CPM Method Closure:}: \texttt{CPM\_Method\_Closure.tex} (Authorship: Jan 24, 2026)
\\\textit{Why}: The Coercive Projection Method (CPM) is a reusable proof kernel that converts three quantitative hypotheses into global, domain-independent consequences.
\vspace{0.8em}

\textbf{18. Optimization-Based Reference:}: \texttt{Optimization\_Based\_Reference\_Symbol\_Grounding.tex} (Authorship: Jan 24, 2026)
\\\textit{Why}: The Symbol Grounding Problem asks how symbols can be about things without an external interpreter.
\vspace{0.8em}

\textbf{19. The Universal Light Language: Periodic Table of Meaning}: \texttt{papers/tex/ULL-Periodic-Table-Meaning.tex} (Authorship: Jan 24, 2026)
\\\textit{Why}: We present the Universal Light Language (ULL), a zero-parameter semantic code that assigns canonical, short descriptions to multi-modal signals based not on their surface statistics but on the reco...
\vspace{0.8em}

\textbf{20. Meaning is Forced:}: \texttt{planning/papers/Meaning\_Is\_Forced.tex} (Authorship: Jan 24, 2026)
\\\textit{Why}: Recognition Science (RS) and its closure certificates constrain admissible structure, invariants, and canonical representations, but (as a common caveat notes) closure alone does not yet imply mean...
\vspace{0.8em}

\textbf{21. Bergman-Scale Holomorphic Manufacturing of Prescribed Tangent Templates in Projective K"ahler Manifolds}: \texttt{papers/tex/Paper-2-Bergman-Scale-Holomorphic.tex} (Authorship: Jan 20, 2026)
\\\textit{Why}: Let be a smooth complex projective manifold with an ample line bundle whose curvature form is a K"ahler form.
\vspace{0.8em}

\textbf{22. Corner-Exit Slivers for Calibrated Sheet Constructions: Deterministic Face Incidence and Uniform Boundary Control}: \texttt{papers/tex/Paper-3-Corner-Exit-Slivers.tex} (Authorship: Jan 20, 2026)
\\\textit{Why}: We introduce corner-exit slivers: local calibrated template pieces inside a cube whose footprint is a uniformly fat simplex meeting only a prescribed set of boundary faces.
\vspace{0.8em}

\textbf{23. Weighted Flat-Norm Gluing for Sliver Microstructures and Vanishing-Mass Boundary Correction}: \texttt{papers/tex/Paper-5-Weighted-Flat.tex} (Authorship: Jan 20, 2026)
\\\textit{Why}: We prove a quantitative gluing estimate for mesh-based assemblies of many small calibrated pieces (``slivers'') in a compact Riemannian manifold.
\vspace{0.8em}

\textbf{24. Cohomology Quantization for Microstructured Calibrated Currents via Discrepancy Rounding}: \texttt{papers/tex/Paper-6-Cohomology.tex} (Authorship: Jan 20, 2026)
\\\textit{Why}: We address the global integrality constraint in microstructured constructions of calibrated currents: producing a closed integral current in an exact prescribed homology class with fixed , while lo...
\vspace{0.8em}

\textbf{25. Stable Direction Dictionaries for Strongly Positive -Forms via Regularized Simplex Fits}: \texttt{papers/tex/paper-1-stable-direction-dictionaries.tex} (Authorship: Jan 20, 2026)
\\\textit{Why}: Let be a compact K"ahler manifold of complex dimension , and let denote the cone of strongly positive -covectors at.
\vspace{0.8em}

\textbf{26. Reciprocal Convex Costs for Ratio Matching: 0.3em Functional-Equation Characterization and Decision Geometry}: \texttt{papers/Algebra\_of\_Aboutness\_Amir\_final-vv.tex} (Authorship: Jan 19, 2026)
\\\textit{Why}: We study ratio-induced mismatch costs of the form built from positive scale maps and and a penalty.
\vspace{0.8em}

\textbf{27. Uniqueness Of The Canonical Reciprocal Cost}: \texttt{papers/UNIQUENESS OF THE CANONICAL RECIPROCAL COST.tex} (Authorship: Jan 19, 2026)
\\\textit{Why}: Cont... 1.mm Keywords: 1.mm Mathematics Subject Classifications (2010):
\vspace{0.8em}

\textbf{28. Dimensional Rigidity: D=3 from Linking of Loops, Kepler Stability, and Minimal Dyadic Synchronization}: \texttt{papers/pdf/Dimensional\_Rigidity\_D3-b.tex} (Authorship: Jan 19, 2026)
\\\textit{Why}: We give three mathematically precise constraints that each single out the spatial dimension.
\vspace{0.8em}

\textbf{29. A Cost-Minimization Theory of Reference: 0.3em Aboutness from Balance and Compression}: \texttt{papers/tex/Algebra\_of\_Aboutness\_Amir\_final-v1 (1).tex} (Authorship: Jan 19, 2026)
\\\textit{Why}: We present an axiomatic and checkable model of reference in which a symbol and a candidate referent are compared through positive scale maps and.
\vspace{0.8em}

\textbf{30. D'Alembert Inevitability:; Polynomial Consistency Forces the Canonical Composition Law on}: \texttt{papers/tex/DAlembert\_Inevitability.tex} (Authorship: Jan 19, 2026)
\\\textit{Why}: Let be a real-valued functional on multiplicative ratios.
\vspace{0.8em}

\textbf{31. Universal Light Language: A Zero-Parameter Periodic Table of Meaning}: \texttt{papers/tex/New-ULL-Periodic-Table-Meaning.tex} (Authorship: Jan 19, 2026)
\\\textit{Why}: Recognition Geometry provides a measurement-first axiomatic setting in which geometric structure is derived from constraints on observables.
\vspace{0.8em}

\textbf{32. Reality-Native Measurements with a Single-Anchor SI Bridge}: \texttt{papers/RSNative-Measurement-Framework.tex} (Authorship: Jan 17, 2026)
\\\textit{Why}: Claims that a theory has no free parameters are only as strong as the measurement and reporting layer connecting the theory to the world.
\vspace{0.8em}

\textbf{33. P0-A0 Noble Gas Closure Theorem(Mathematical Derivation + Validation Tables)}: \texttt{papers/tex/P0\_A0\_Noble\_Gas\_Closure\_Derivation.tex} (Authorship: Jan 17, 2026)
\\\textit{Why}: This document rewrites the P0-A0 ``noble gas closure'' result as a complete mathematical derivation.
\vspace{0.8em}

\textbf{34. P0-A2 Ionization Energy Sawtooth(Mathematical Derivation + NIST Validation Tables)}: \texttt{papers/tex/P0\_A2\_Ionization\_Energy\_Sawtooth\_Derivation.tex} (Authorship: Jan 17, 2026)
\\\textit{Why}: This document rewrites the P0-A2 ionization ``sawtooth'' result as a full mathematical derivation.
\vspace{0.8em}

\textbf{35. P0-B0 Nuclear Magic Numbers(Mathematical Derivation + Validation Tables)}: \texttt{papers/tex/P0\_B0\_Nuclear\_Magic\_Numbers\_Derivation.tex} (Authorship: Jan 17, 2026)
\\\textit{Why}: This document presents the nuclear ``magic numbers'' claim (P0-B0) in full mathematical prose.
\vspace{0.8em}

\textbf{36. THE CANON}: \texttt{planning/RECOGNITION\_SCIENCE\_CANONICAL\_LIBRARY\_STRATEGY.tex} (Authorship: Jan 17, 2026)
\\\textit{Why}: This document describes a plan to build a unified, machine-verified library of physics.
\vspace{0.8em}

\textbf{37. The Mathematical Zero-Point}: \texttt{papers/tex/Mathematical\_Zero\_Point.tex} (Authorship: Jan 16, 2026)
\\\textit{Why}: Wigner asked why mathematics is so effective in describing the natural sciences.
\vspace{0.8em}

\textbf{38. Model-Independent Exclusivity on the Quotient State Space Recognition Science as an Inevitability Theorem for Zero-Parameter Frameworks}: \texttt{papers/tex/Model-Independent-Exclusivity-Quotient.tex} (Authorship: Jan 16, 2026)
\\\textit{Why}: We prove a model-independent exclusivity theorem for Recognition Science (RS) on the quotient state space: states are identified when they are observationally indistinguishable (i.
\vspace{0.8em}

\textbf{39. Quantized Semantics}: \texttt{papers/tex/Quantized\_Semantics.tex} (Authorship: Jan 16, 2026)
\\\textit{Why}: Why is meaning discrete (words, concepts, semantic atoms) rather than continuous? And why do there exist specific ``modes'' of meaning---in particular, a finite periodic table of 20 canonical WToke...
\vspace{0.8em}

\textbf{40. WTokens as Compression}: \texttt{papers/tex/WToken\_Compression\_Recognition\_Addendum.tex} (Authorship: Jan 16, 2026)
\\\textit{Why}: This short addendum isolates the new WToken-specific consequences of the Algebra of Aboutness.
\vspace{0.8em}

\textbf{41. The Projection Operator : Active Enforcement of Information Conservation in Recognition Science}: \texttt{papers/tex/projection\_operator.tex} (Authorship: Jan 12, 2026)
\\\textit{Why}: Standard physics often treats conservation laws as passive constraints that systems naturally obey.
\vspace{0.8em}

\textbf{42. The Algebra Of Reality Paper}: \texttt{papers/The\_Algebra\_of\_Reality\_Paper.tex} (Authorship: Jan 11, 2026)
\\\textit{Why}: We present a mathematics-first derivation of discrete structure from a single cost-theoretic primitive.
\vspace{0.8em}

\textbf{43. Phantom Light: Future Neutrality Constraints as Present-Time Structure in Recognition Science}: \texttt{papers/PhantomLight\_Paper.tex} (Authorship: Jan 10, 2026)
\\\textit{Why}: Standard dynamical systems typically privilege initial conditions, evolving states forward in time from.
\vspace{0.8em}

\textbf{44. Coherent Comparison Costs from the d'Alembert Composition Law: Discrete Ledger Structure with a Lean 4 Formalization}: \texttt{papers/tex/January 8.tex} (Authorship: Jan 10, 2026)
\\\textit{Why}: We study a calibrated multiplicative d'Alembert functional equation arising from the requirement that comparison costs compose coherently on ratios.
\vspace{0.8em}

\textbf{45. Tau Step Coefficient: Exclusivity and First-Principles Derivation}: \texttt{papers/tex/tau\_step\_exclusivity.tex} (Authorship: Jan 09, 2026)
\\\textit{Why}: A reviewer correctly noted that the tau-generation -correction coefficient numerically equals in and that many different expressions can reproduce the same value.
\vspace{0.8em}

\textbf{46. Zero-Parameter Galaxy Rotation Curves from Information-Limited Gravity: A Lean-Verified Test Against 99 SPARC Galaxies}: \texttt{papers/ILG\_Galaxy\_Rotation\_Curves.tex} (Authorship: Jan 08, 2026)
\\\textit{Why}: We present the first formally-verified, zero-parameter numerical test of a modified gravity theory against empirical galaxy rotation curves.
\vspace{0.8em}

\textbf{47. Convergence of Empirical Optimization and First-Principles Derivation in Galactic Dynamics: A Unified Validation of Recognition Science}: \texttt{papers/ILG\_Validation\_Synthesis.tex} (Authorship: Jan 08, 2026)
\\\textit{Why}: We present a unified analysis of two independent tests of the Information-Limited Gravity (ILG) framework against the SPARC galaxy rotation curve database.
\vspace{0.8em}

\textbf{48. The Inevitability of Existence: A Cost-First Derivation of Physical Law, Semantics, and Ethics}: \texttt{papers/root\_papers/The\_Inevitability\_of\_Existence.tex} (Authorship: Jan 08, 2026)
\\\textit{Why}: Standard cosmology encounters a fundamental singularity at , forcing the assumption of initial conditions and physical laws as axiomatic "givens.
\vspace{0.8em}

\textbf{49. The Pre-Big-Bang Universe: Complete Account from Recognition Science}: \texttt{papers/root\_papers/The\_Pre\_Big\_Bang\_Universe.tex} (Authorship: Jan 08, 2026)
\\\textit{Why}: This document presents a comprehensive prose account of the universe before the Big Bang according to Recognition Science, as formalized in the IndisputableMonolith Lean repository.
\vspace{0.8em}

\textbf{50. The Recognition Operator}: \texttt{papers/root\_papers/The\_Recognition\_Operator.tex} (Authorship: Jan 08, 2026)
\\\textit{Why}: Defines the discrete cost-minimizing operator.
\vspace{0.8em}

\textbf{51. The Pre--Big Bang Origin of Law:}: \texttt{papers/root\_papers/pre\_big\_bang\_origin\_paper.tex} (Authorship: Jan 08, 2026)
\\\textit{Why}: The phrase ``before the Big Bang'' is usually treated as either a poetic question or a category error: in general relativity, ``before'' presupposes time, but classical time is defined by a spaceti...
\vspace{0.8em}

\textbf{52. Quark Masses from the Quarter-Integer -Ladder}: \texttt{papers/tex/quark\_masses\_quarter\_ladder.tex} (Authorship: Jan 08, 2026)
\\\textit{Why}: The single-anchor mass framework (Paper 1) successfully predicts charged lepton masses using integer rungs on the -ladder.
\vspace{0.8em}

\textbf{53. Structural Resolution of the Tau Generation Step}: \texttt{papers/tex/tau\_step\_resolution.tex} (Authorship: Jan 08, 2026)
\\\textit{Why}: Recent review of the charged lepton mass pipeline identified a potential "numerology risk" in the muon-to-tau generation step formula.
\vspace{0.8em}

\textbf{54. Fourgates Inevitability Paper}: \texttt{papers/root\_papers/FourGates\_Inevitability\_Paper.tex} (Authorship: Jan 05, 2026)
\\\textit{Why}: We study multiplicatively consistent comparison costs , i.
\vspace{0.8em}

\textbf{55. Full Unconditional Inevitability}: \texttt{papers/root\_papers/Full\_Unconditional\_Inevitability.tex} (Authorship: Jan 04, 2026)
\\\textit{Why}: We isolate the inevitability argument into two logically distinct components.
\vspace{0.8em}

\textbf{56. Threegates Inevitability Paper}: \texttt{papers/root\_papers/ThreeGates\_Inevitability\_Paper.tex} (Authorship: Jan 04, 2026)
\\\textit{Why}: We study multiplicatively consistent comparison costs , i.
\vspace{0.8em}

\textbf{57. Dalembert Inevitability Paper}: \texttt{papers/DAlembert\_Inevitability\_Paper.tex} (Authorship: Jan 03, 2026)
\\\textit{Why}: We prove that the Recognition Composition Law (RCL)---the functional equation J(xy) + J(x/y) = 2J(x)J(y) + 2J(x) + 2J(y) is not an arbitrary mathematical choice but is transcendentally necessary.
\vspace{0.8em}

\textbf{58. The Ultimate Inevitability of the Recognition Composition Law}: \texttt{papers/root\_papers/Ultimate\_RCL\_Inevitability.tex} (Authorship: Jan 03, 2026)
\\\textit{Why}: We present the strongest possible statement regarding the Recognition Composition Law (RCL).
\vspace{0.8em}

\textbf{59. Closing the Foundational Gaps:0.5em First-Principles Derivation of \$ \textasciicircum{}-1}: \texttt{papers/tex/Closing-The-Gaps.tex} (Authorship: Jan 03, 2026)
\\\textit{Why}: We provide complete, first-principles derivations for two foundational claims of Recognition Science (RS): (1) the fine-structure constant (with and ) arises from combinatorial topology of the cubi...
\vspace{0.8em}

\textbf{60. The Derivation of Physical Constants from the Meta-Principle:0.5em A Complete Chain of Custody from Logic to Cosmology}: \texttt{papers/tex/Formalized-Derivations-T1-T8.tex} (Authorship: Jan 03, 2026)
\\\textit{Why}: We present a rigorous derivation of the fundamental constants of physics starting from a single logical axiom: the Meta-Principle (MP) stating that ``Nothing cannot recognize itself.
\vspace{0.8em}

\textbf{61. Recognition Science: A Zero-Parameter Framework 0.5em Deriving Fundamental Constants from Logical Necessity}: \texttt{papers/tex/RS-Foundations.tex} (Authorship: Jan 03, 2026)
\\\textit{Why}: We present Recognition Science (RS), a theoretical framework that derives all fundamental physical constants---the speed of light , Planck's constant , Newton's gravitational constant , and the fin...
\vspace{0.8em}

\textbf{62. Recognition Science}: \texttt{papers/tex/Recognition\_Science\_Compendium.tex} (Authorship: Jan 03, 2026)
\\\textit{Why}: Contributes to the Recognition Science framework.
\vspace{0.8em}

\textbf{63. A Mechanized Proof of Reality's Architecture from a Minimal Axiom}: \texttt{papers/tex/mechanized-proof-arXiv.tex} (Authorship: Jan 03, 2026)
\\\textit{Why}: We present a machine-checked mathematical proof that a single, parameter-free framework both describes physical reality and is uniquely determined by it.
\vspace{0.8em}

\textbf{64. The Inevitability of the Recognition Composition Law}: \texttt{planning/papers/Unconditional\_RCL\_Inevitability.tex} (Authorship: Jan 03, 2026)
\\\textit{Why}: We isolate and formalize a precise ``unconditional'' statement about the Recognition Composition Law (RCL).
\vspace{0.8em}

\textbf{65. The Recognition Composition Law}: \texttt{papers/tex/Recognition\_Composition\_Law\_Primer.tex} (Authorship: Jan 02, 2026)
\\\textit{Why}: We present the Recognition Composition Law, the foundational axiom of Recognition Science from which all physical structure emerges.
\vspace{0.8em}

\textbf{66. Standard-Model Masses from Octave Closure and Integer Baselines}: \texttt{papers/tex/SM\_MASSES\_SINGLE\_ANCHOR\_PAPER.tex} (Authorship: Jan 02, 2026)
\\\textit{Why}: The Standard Model treats fermion masses as inputs (Yukawa couplings) rather than outputs.
\vspace{0.8em}

\textbf{67. The Algebra of Aboutness: 0.3em Reference as Cost-Minimizing Compression}: \texttt{papers/tex/Algebra\_of\_Aboutness.tex} (Authorship: Jan 01, 2026)
\\\textit{Why}: We develop a mathematical theory of reference---the semantic relation by which symbols ``point to'' objects---grounded in cost-minimization principles.
\vspace{0.8em}

\textbf{68. The Cumulative Density Argument: Global Energy Constraints on Off-Line Zeros}: \texttt{papers/tex/CUMULATIVE\_DENSITY\_PROOF.tex} (Authorship: Jan 01, 2026)
\\\textit{Why}: We develop a global energy argument showing that the total ``off-line cost'' of all zeros is bounded by the prime layer energy.
\vspace{0.8em}

\textbf{69. Cost Is Not a Dial: A Self-Contained Uniqueness Theorem for the Canonical Reciprocal Cost on \$ \_>0}: \texttt{papers/tex/Cost-uniqueness.tex} (Authorship: Jan 01, 2026)
\\\textit{Why}: Many mathematical and physical frameworks introduce a cost (or action, penalty, divergence, or energy) to quantify change, then proceed as if that choice were canonical.
\vspace{0.8em}

\textbf{70. A Cost-Theoretic Foundation for Data Compression}: \texttt{papers/tex/Cost\_Compression\_Theory.tex} (Authorship: Jan 01, 2026)
\\\textit{Why}: We develop a mathematical framework for analyzing data compression based on the cost functional , uniquely characterized by the d'Alembert composition law with natural boundary conditions.
\vspace{0.8em}

\textbf{71. Foundations of Recognition Science: From the Recognition Composition Law to Physical Constants}: \texttt{papers/tex/Dic 22-jon.tex} (Authorship: Jan 01, 2026)
\\\textit{Why}: We present an informational framework, termed Recognition Science (RS), aimed at recovering familiar physical structures with minimal parameters.
\vspace{0.8em}

\textbf{72. The Energy Separation Principle: A Rigorous Proof from Recognition Science Axioms}: \texttt{papers/tex/ENERGY\_SEPARATION\_PROOF.tex} (Authorship: Jan 01, 2026)
\\\textit{Why}: We prove the Energy Separation Principle rigorously within the Recognition Science (RS) axiomatic framework.
\vspace{0.8em}

\textbf{73. The Geometry of Decision: A Cost-Theoretic Framework for Attention, Choice, and Agency}: \texttt{papers/tex/Geometry\_of\_Decision.tex} (Authorship: Jan 01, 2026)
\\\textit{Why}: We develop a mathematical framework for decision-making based on the universal cost functional.
\vspace{0.8em}

\textbf{74. The Geometry of Inquiry: Cost-Theoretic Framework for Questions}: \texttt{papers/tex/Geometry\_of\_Inquiry.tex} (Authorship: Jan 01, 2026)
\\\textit{Why}: We develop a mathematical framework where questions are equipped with cost functions over their answer spaces.
\vspace{0.8em}

\textbf{75. The Grammar of Possibility: Cost-Theoretic Foundation for Modal Logic}: \texttt{papers/tex/Grammar\_of\_Possibility.tex} (Authorship: Jan 01, 2026)
\\\textit{Why}: We present a novel foundation for modal logic grounded in cost minimization rather than abstract possible-worlds semantics.
\vspace{0.8em}

\textbf{76. The Harmonic Structure of Particle Interactions: the CKM Hierarchy from the Golden Ratio}: \texttt{papers/tex/Particle\_Harmony\_CKM.tex} (Authorship: Jan 01, 2026)
\\\textit{Why}: The Standard Model of particle physics contains approximately 19 free parameters, a significant fraction of which reside in the flavor sector (masses and mixing angles).
\vspace{0.8em}

\textbf{77. The Physics of Narrative: A Geometric Formalization of Story Structure in Recognition Science}: \texttt{papers/tex/Physics\_of\_Narrative.tex} (Authorship: Jan 01, 2026)
\\\textit{Why}: We present a mathematical formalization of narrative structure within Recognition Science (RS).
\vspace{0.8em}

\textbf{78. The Physics of Reference: 0.3em A Cost-Theoretic Foundation for Semantics}: \texttt{papers/tex/Physics\_of\_Reference.tex} (Authorship: Jan 01, 2026)
\\\textit{Why}: We develop a mathematical theory of reference---the semantic relation by which configurations ``point to'' one another---grounded in cost-minimization principles.
\vspace{0.8em}

\textbf{79. The Placebo Operator: Recognition Science Framework for Mind-Body Coupling}: \texttt{papers/tex/Placebo\_Operator\_RRF\_Somatic\_Coupling.tex} (Authorship: Jan 01, 2026)
\\\textit{Why}: The placebo effect---whereby belief produces measurable physiological change---lacks a principled coupling mechanism in conventional medicine.
\vspace{0.8em}

\textbf{80. The Recognition Composition Law for Zeta Zeros: A New Mathematical Framework}: \texttt{papers/tex/RECOGNITION\_COMPOSITION\_LAW.tex} (Authorship: Jan 01, 2026)
\\\textit{Why}: We introduce a new mathematical structure---the Recognition Composition Law---that connects the d'Alembert functional equation governing the RS cost function to constraints on the zero distribution...
\vspace{0.8em}

\textbf{81. The Energetic Necessity of the Riemann Hypothesis: the Prime Distribution from the Law of Existence}: \texttt{papers/tex/RS\_Axiomatic\_Proof\_RH.tex} (Authorship: Jan 01, 2026)
\\\textit{Why}: The Riemann Hypothesis (RH) remains unproven in standard Zermelo-Fraenkel set theory (ZFC) because ZFC treats all logically consistent objects as equally existent, regardless of their complexity or...
\vspace{0.8em}

\textbf{82. Recognitionscience Navierstokes Prose}: \texttt{papers/tex/RecognitionScience\_NavierStokes\_prose.tex} (Authorship: Jan 01, 2026)
\\\textit{Why}: Assuming Recognition Science (RS) is an accurate architecture of reality, this note explains---in prose, with a minimal amount of classical PDE notation---what RS would predict about the remaining ...
\vspace{0.8em}

\textbf{83. The Explicit Formula Obstruction: Why Off-Line Zeros Violate the Prime Number Theorem}: \texttt{papers/tex/STRUCTURAL\_OBSTRUCTION\_PROOF.tex} (Authorship: Jan 01, 2026)
\\\textit{Why}: We prove that any off-line zero of the Riemann zeta function creates oscillations in the explicit formula that violate the known error term in the Prime Number Theorem.
\vspace{0.8em}

\textbf{84. Response to Referee Comments on T5 (Cost Uniqueness): Clarification of the d'Alembert Functional Equation Requirement}: \texttt{papers/tex/T5\_uniqueness\_response.tex} (Authorship: Jan 01, 2026)
\\\textit{Why}: We respond to the referee's valid critique regarding the uniqueness theorem T5 for the cost functional.
\vspace{0.8em}

\textbf{85. The Topology of Self-Reference: A Positive Characterization of Stable Consciousness in Recognition Science}: \texttt{papers/tex/Topology\_of\_Self\_Reference.tex} (Authorship: Jan 01, 2026)
\\\textit{Why}: We present a complete topological characterization of stable self-reference within the Recognition Science (RS) framework.
\vspace{0.8em}

\textbf{86. Coulomb Fusion Closure of the Riemann Hypothesis: Unconditional Elimination of the Height-Dependent Gap}: \texttt{fusion/papers/COULOMB\_FUSION\_CLOSURE.tex} (Authorship: Dec 31, 2025)
\\\textit{Why}: We close the remaining height-dependent gap in the energy-barrier proof of the Riemann Hypothesis.
\vspace{0.8em}

\textbf{87. Rigorous Coulomb Fusion: The Separation Principle and Unconditional RH}: \texttt{fusion/papers/COULOMB\_FUSION\_RIGOROUS.tex} (Authorship: Dec 31, 2025)
\\\textit{Why}: We provide a rigorous formulation of the Coulomb Fusion argument for the Riemann Hypothesis.
\vspace{0.8em}

\textbf{88. Applied Algebra of Aboutness}: \texttt{papers/tex/Aboutness\_Applications.tex} (Authorship: Dec 31, 2025)
\\\textit{Why}: We present practical applications of the cost-theoretic theory of reference developed in ``The Algebra of Aboutness.
\vspace{0.8em}

\textbf{89. The Blaschke-Prime Constraint: New Rigorous Theorems on Zero Positioning}: \texttt{papers/tex/BLASCHKE\_PRIME\_CONSTRAINT.tex} (Authorship: Dec 31, 2025)
\\\textit{Why}: We prove several new rigorous theorems connecting the Blaschke product structure of zeta zeros to constraints from the prime distribution.
\vspace{0.8em}

\textbf{90. Free Energy Principles for Optimal Resource Allocation}: \texttt{papers/tex/Free\_Energy\_Resource\_Allocation.tex} (Authorship: Dec 31, 2025)
\\\textit{Why}: We present a unified framework for resource allocation based on free energy minimization from Recognition Science.
\vspace{0.8em}

\textbf{91. Golden Ratio Scheduling}: \texttt{papers/tex/Golden\_Ratio\_Scheduling.tex} (Authorship: Dec 31, 2025)
\\\textit{Why}: We present a principled framework for time allocation and scheduling based on the golden ratio.
\vspace{0.8em}

\textbf{92. Market Thermodynamics}: \texttt{papers/tex/Market\_Thermodynamics.tex} (Authorship: Dec 31, 2025)
\\\textit{Why}: We develop a thermodynamic theory of financial markets based on Recognition Science.
\vspace{0.8em}

\textbf{93. A Cost-Function Approach to Musical Consonance: Deriving Interval Hierarchy from a Symmetry Principle}: \texttt{papers/tex/Music\_Theory\_from\_Recognition\_Science.tex} (Authorship: Dec 31, 2025)
\\\textit{Why}: We investigate a cost-function approach to musical consonance based on the function , which arises from requiring inversion symmetry () and normalization ().
\vspace{0.8em}

\textbf{94. The Prime Stiffness Theorem and the Riemann Hypothesis}: \texttt{papers/tex/RH\_Prime\_Stiffness\_Proof-alt.tex} (Authorship: Dec 31, 2025)
\\\textit{Why}: We present a framework for proving the Riemann Hypothesis from the discrete nature of prime numbers.
\vspace{0.8em}

\textbf{95. The Statistical Mechanics of Recognition: Thermodynamic Foundations for Cost-Based Physics}: \texttt{papers/tex/Recognition\_Thermodynamics.tex} (Authorship: Dec 31, 2025)
\\\textit{Why}: We develop a thermodynamic extension of Recognition Science (RS), a framework in which physical existence is characterized by minimization of the universal cost functional.
\vspace{0.8em}

\textbf{96. Symmetry Fusion: An Unconditional Proof of the Riemann Hypothesis via Coulomb Minimization}: \texttt{papers/tex/SYMMETRY\_FUSION\_PROOF.tex} (Authorship: Dec 31, 2025)
\\\textit{Why}: We present a proof of the Riemann Hypothesis based on a variational principle for the ``zero gas'' of the zeta function.
\vspace{0.8em}

\textbf{97. The Symmetry Resonance Theorem: A Novel Characterization of the Critical Line}: \texttt{papers/tex/SYMMETRY\_RESONANCE\_THEOREM.tex} (Authorship: Dec 31, 2025)
\\\textit{Why}: We introduce the concept of symmetry resonance for zeta zeros and prove that the critical line is uniquely characterized as the locus where two fundamental symmetries---the functional equation and ...
\vspace{0.8em}

\textbf{98. The Anchor Scale \$}: \texttt{papers/tex/Anchor-Scale-Derivation.tex} (Authorship: Dec 29, 2025)
\\\textit{Why}: We present a formal, non-circular derivation of the anchor-scale principle used by the Recognition Science mass framework.
\vspace{0.8em}

\textbf{99. -0.5em}: \texttt{papers/tex/CPM-Gravity.tex} (Authorship: Dec 29, 2025)
\\\textit{Why}: We articulate a single, universal principle that governs gravitational inference under finite information: the coercive projection law.
\vspace{0.8em}

\textbf{100. Machine-Verified Emergence of General Relativity and Standard Model Symmetries from the Recognition Octave}: \texttt{papers/tex/GRAVITATIONAL\_EMERGENCE\_PAPER.tex} (Authorship: Dec 29, 2025)
\\\textit{Why}: Recognition Science (RS) posits that physical reality is the emergent stationary configuration of a self-consistent, cost-minimizing recognition ledger.
\vspace{0.8em}

\textbf{101. Pressure Gravity}: \texttt{papers/tex/Pressure-Gravity.tex} (Authorship: Dec 29, 2025)
\\\textit{Why}: We recast Information-Limited Gravity (ILG) as classical gravity sourced by an effective pressure field.
\vspace{0.8em}

\textbf{102. Parameter-Free Sector Constants: From Cube Geometry to Mass Yardsticks}: \texttt{papers/tex/Sector-Constants-Derivation.tex} (Authorship: Dec 29, 2025)
\\\textit{Why}: We document and audit the derivation of the sector constants and that define the mass yardsticks in the Recognition Science framework.
\vspace{0.8em}

\textbf{103. Sector Constants Now Fully Derived: A Formal Verification Milestone}: \texttt{papers/tex/Sector-Constants-Now-Derived.tex} (Authorship: Dec 29, 2025)
\\\textit{Why}: We report a verification milestone in the Recognition Science Lean 4 codebase: all sector constants ( and ) are now computed from an explicit counting layer rather than declared as unexplained lite...
\vspace{0.8em}

\textbf{104. Zero-Parameter Derivation of Standard Model Fermion Masses}: \texttt{papers/tex/Zero-Parameter-Mass-Derivation.tex} (Authorship: Dec 29, 2025)
\\\textit{Why}: The Standard Model of particle physics contains 22 arbitrary parameters related to fermion masses and mixing.
\vspace{0.8em}

\textbf{105. Foundations of Recognition Science: A Theory of Countability, Recognition, and Cost Minimization}: \texttt{papers/tex/Dic\_23\_refereed.tex} (Authorship: Dec 28, 2025)
\\\textit{Why}: We present an informational framework, termed Recognition Science (RS), aimed at recovering familiar physical structures with minimal parameters.
\vspace{0.8em}

\textbf{106. Why Particle Masses Have Structure}: \texttt{papers/tex/MassFramework\_PlainProse.tex} (Authorship: Dec 28, 2025)
\\\textit{Why}: This paper explains, in plain language, where the mass formula comes from and why each piece exists.
\vspace{0.8em}

\textbf{107. What Is Recognition Science?}: \texttt{papers/tex/WhatIsRecognitionScience.tex} (Authorship: Dec 28, 2025)
\\\textit{Why}: This document explains Recognition Science (RS) at a conceptual level, before any equations.
\vspace{0.8em}

\textbf{108. A Geometric Framework for Finite Multi-Loop Calculations in QFT}: \texttt{papers/tex/voxel-arXiv.tex} (Authorship: Dec 28, 2025)
\\\textit{Why}: Multi-loop calculations in quantum field theory traditionally require evaluating hundreds of divergent Feynman integrals with complex regularization schemes.
\vspace{0.8em}

\textbf{109. Recognition Science, Prime Numbers, and the Riemann Hypothesis: A Standalone Roadmap of What We Know, What We Built, and What Still Blocks Us}: \texttt{papers/tex/RecognitionScience\_Primes\_RH\_Blockers.tex} (Authorship: Dec 27, 2025)
\\\textit{Why}: This note is a standalone ``state-of-the-art'' writeup for a specific research codebase (riemann-geometry-rs) and a specific guiding narrative (``Recognition Science'').
\vspace{0.8em}

\textbf{110. Recognition Science closes the far-field attachment gate (Paper B in a two-paper Recognition Science proof of the Riemann Hypothesis)}: \texttt{papers/tex/Riemann-PaperB\_RS\_Kxi.tex} (Authorship: Dec 27, 2025)
\\\textit{Why}: Paper\textasciitilde{}A gives a two-regime route to the Riemann Hypothesis (RH).
\vspace{0.8em}

\textbf{111. Whitney box energy for and weighted microscopic variation of}: \texttt{papers/tex/Riemann-pro-unconditional.tex} (Authorship: Dec 26, 2025)
\\\textit{Why}: We study the harmonic field (U\_ ( ,t)= ( 12+ +it)) on the right half-plane and its Whitney-box Dirichlet energy at the microscopic scale (L 1/ T ).
\vspace{0.8em}

\textbf{112. The Octave System and the Particle Mass Spectrum}: \texttt{papers/tex/OCTAVE\_MASSES\_PAPER.tex} (Authorship: Dec 24, 2025)
\\\textit{Why}: The Standard Model accurately predicts particle interactions, yet it does not explain why fermion masses take their observed values: the Yukawa couplings are inserted as free parameters.
\vspace{0.8em}

\textbf{113. A Weighted Diagonal Operator, Regularised Determinants, and a Critical--Line Criterion for the Riemann Zeta Function0.5em An Operator--Theoretic Approach Inspired by Recognition Science}: \texttt{papers/tex/Recognition-Riemann-Final.tex} (Authorship: Dec 24, 2025)
\\\textit{Why}: We realise as a -regularised Fredholm determinant of , where the arithmetic Hamiltonian acts on the weighted space with.
\vspace{0.8em}

\textbf{114. A Proof of the Riemann Hypothesis: Via Transfer Operator Spectral Analysis}: \texttt{papers/tex/Riemann-July-7.tex} (Authorship: Dec 24, 2025)
\\\textit{Why}: We present a proof of the Riemann Hypothesis through construction of a transfer operator whose Fredholm determinant equals and whose spectral gap off the critical line forces all zeros to.
\vspace{0.8em}

\textbf{115. Prime-Tail Schur-Covering in the Bounded-Real Framework: Unconditional Bridges B--C and a Certified Covering}: \texttt{papers/tex/Riemann-Strongest.tex} (Authorship: Dec 24, 2025)
\\\textit{Why}: We develop unconditional operator tools for a bounded-real (Herglotz/Schur) program on the right half-plane ( = s> 12).
\vspace{0.8em}

\textbf{116. \% The Voxel as Meaning}: \texttt{papers/tex/VoxelMeaning.tex} (Authorship: Dec 24, 2025)
\\\textit{Why}: We present a mathematical formalization of how meaning emerges from the fundamental structure of light.
\vspace{0.8em}

\textbf{117. A Lean-Referenced Derivation of the Electromagnetic Fine-Structure Constant from Recognition Ledger Geometry}: \texttt{papers/tex/alpha\_derivation.tex} (Authorship: Dec 24, 2025)
\\\textit{Why}: This paper documents the derivation of the electromagnetic fine-structure constant in the exact form implemented in this repository's Lean\textasciitilde{}4 development.
\vspace{0.8em}

\textbf{118. Internal Memo: Resolving the ``Single-Anchor'' Mass Questions}: \texttt{papers/tex/anil\_dec\_23.tex} (Authorship: Dec 24, 2025)
\\\textit{Why}: Derives particle masses from the recognition framework.
\vspace{0.8em}

\textbf{119. The Geometrodynamics of Consciousness: Light-Field Saturation and the Bio-Clocking Mechanism}: \texttt{papers/tex/geometry\_of\_consciousness.tex} (Authorship: Dec 24, 2025)
\\\textit{Why}: We extend the Zero-Parameter Framework from fundamental physics to the mesoscale dynamics of biology and consciousness.
\vspace{0.8em}

\textbf{120. The Thermodynamics of the Massless State: Phase Saturation, Bio-Clocking, and the Geometric Necessity of Existence}: \texttt{papers/tex/light-field-saturation.tex} (Authorship: Dec 24, 2025)
\\\textit{Why}: We present a rigorous derivation of the thermodynamic constraints governing massless information storage within Recognition Science (RS), a zero-parameter framework where all physical constants are...
\vspace{0.8em}

\textbf{121. Navier Dec 12 Rewrite Alternative}: \texttt{papers/tex/navier-dec-12-rewrite-alternative.tex} (Authorship: Dec 24, 2025)
\\\textit{Why}: (under audit). This manuscript records a running-max blow-up framework and a proposed route to global regularity by reducing a hypothetical singularit...
\vspace{0.8em}

\textbf{122. Navier Dec 12 Rewrite}: \texttt{papers/tex/navier-dec-12-rewrite.tex} (Authorship: Dec 24, 2025)
\\\textit{Why}: (under audit). This manuscript records a running-max blow-up framework and a proposed route to global regularity by reducing a hypothetical singularit...
\vspace{0.8em}

\textbf{123. Recognition Geometry}: \texttt{papers/tex/recognition-geometry-dec-23.tex} (Authorship: Dec 24, 2025)
\\\textit{Why}: Recognition Geometry is a new geometric framework that inverts the traditional relationship between space and measurement.
\vspace{0.8em}

\textbf{124. T5 Cost Uniqueness and the Certificate Circle What Completing ``T5'' Certifies in the reality}: \texttt{papers/tex/T5\_Cost\_Uniqueness\_Certificate\_Circle.tex} (Authorship: Dec 19, 2025)
\\\textit{Why}: This note explains the mathematical and engineering content of completing ``T5'' in the reality repository's Lean formalization workflow.
\vspace{0.8em}

\textbf{125. Gravity Formalization in the Reality Lean Repository0.5em A Complete Inventory of Machine-Verified Content}: \texttt{papers/tex/GRAVITY\_LEAN\_FORMALIZATION\_SUMMARY.tex} (Authorship: Dec 18, 2025)
\\\textit{Why}: This document provides a complete inventory of the gravity-related Lean formalizations in the reality repository.
\vspace{0.8em}

\textbf{126. The Riemann Hypothesis via Recognition Geometry}: \texttt{tmp/riemann-rs-geometry/riemann-rs-geometry-main/riemann-recognition-geometry.tex} (Authorship: Dec 18, 2025)
\\\textit{Why}: We present a proof of the Riemann Hypothesis using a geometric approach we term ``Recognition Geometry.
\vspace{0.8em}

\textbf{127. Formalized Properties of the Display Function}: \texttt{papers/tex/GapProperties-Formalization.tex} (Authorship: Dec 16, 2025)
\\\textit{Why}: We present a collection of Lean 4 formalized results concerning the display function (or structural residue) (Z) ;=; (1 + Z/ ) , where is the golden ratio.
\vspace{0.8em}

\textbf{128. Recognition Meta-Theory and the Single-Anchor Mass Residue Function}: \texttt{papers/tex/Single-Anchor-Function-f-Lean-Appendix.tex} (Authorship: Dec 15, 2025)
\\\textit{Why}: This note is a standalone, Lean-backed appendix for collaborators working on the particle-mass ``single-anchor'' program in this repository.
\vspace{0.8em}

\textbf{129. Rg 9 12}: \texttt{papers/tex/RG-9-12.tex} (Authorship: Dec 10, 2025)
\\\textit{Why}: Recognition geometry is.
\vspace{0.8em}

\textbf{130. Response to Recognition Geometry Comments}: \texttt{papers/tex/RG-Response-Dec11.tex} (Authorship: Dec 10, 2025)
\\\textit{Why}: Establishes geometric foundations of recognition.
\vspace{0.8em}

\textbf{131. The Geometry of Evil0.5em How Recognition Science Defines Wrongdoing}: \texttt{papers/tex/The\_Geometry\_of\_Evil.tex} (Authorship: Dec 10, 2025)
\\\textit{Why}: Evil is not a mysterious supernatural force.
\vspace{0.8em}

\textbf{132. Recognition Science: A Zero-Parameter Framework Deriving Fundamental Constants from Logical Necessity}: \texttt{papers/tex/RS\_Foundations\_Outline.tex} (Authorship: Dec 09, 2025)
\\\textit{Why}: We present Recognition Science (RS), a theoretical framework that derives all fundamental physical constants (, , , ) and resolves outstanding empirical tensions (including the Hubble tension) from...
\vspace{0.8em}

\textbf{133. Why Gravity Exists}: \texttt{papers/tex/why\_gravity\_exists.tex} (Authorship: Dec 09, 2025)
\\\textit{Why}: Physics tells us how gravity behaves---masses attract, light bends, time slows near heavy objects.
\vspace{0.8em}

\textbf{134. Gravity Questions Response}: \texttt{papers/tex/GRAVITY\_QUESTIONS\_RESPONSE.tex} (Authorship: Dec 08, 2025)
\\\textit{Why}: This document has three parts.
\vspace{0.8em}

\textbf{135. Formalized Elements of Reality}: \texttt{papers/tex/lean\_formalization\_report.tex} (Authorship: Dec 08, 2025)
\\\textit{Why}: This document provides a comprehensive catalog of all elements of physical reality, mathematics, consciousness, ethics, and biological systems that have been formally verified in the Lean 4 theorem...
\vspace{0.8em}

\textbf{136. Protein Folding from First Principles}: \texttt{papers/tex/protein-dec-6.tex} (Authorship: Dec 06, 2025)
\\\textit{Why}: The protein folding problem has been approached primarily through data-driven methods, with recent breakthroughs from AlphaFold and ESMFold achieving remarkable accuracy by learning from millions o...
\vspace{0.8em}

\textbf{137. Riemann Hypothesis via Nyman--Beurling/B'aez--Duarte and Recognition Science: CPM Energy, Group Averaging, and a No Physical Gap Principle}: \texttt{papers/tex/RH\_NBBD\_CPM.tex} (Authorship: Dec 05, 2025)
\\\textit{Why}: We record a concrete proof track for the Riemann Hypothesis (RH) that marries the Nyman--Beurling/B'aez--Duarte (NB/BD) criterion with Recognition Science (RS).
\vspace{0.8em}

\textbf{138. The Photonic Nature of Love: Deriving Social Coherence from U(1) Gauge Invariance}: \texttt{papers/tex/love.tex} (Authorship: Dec 05, 2025)
\\\textit{Why}: We present a derivation from the Recognition Science (RS) framework proving that the ethical virtue of ``Love'' is not a subjective emotion, but a precise geometric operator mathematically isomorph...
\vspace{0.8em}

\textbf{139. The Logical Derivation of Fundamental Physical Constants from the Internal Consistency of a Zero-Parameter Framework}: \texttt{papers/tex/foundation\_theory\_outline.tex} (Authorship: Dec 04, 2025)
\\\textit{Why}: Draft: We derive the fine structure constant () and electron mass () as necessary eigenvalues of a zero-parameter system constrained by information conservation.
\vspace{0.8em}

\textbf{140. Coercive Projection Method: Rigorous Derivation of Constants from First Principles0.5em Supporting Technical Document}: \texttt{papers/tex/CPM\_Constants\_Derivation.tex} (Authorship: Dec 01, 2025)
\\\textit{Why}: This document provides rigorous mathematical derivations of all constants appearing in the Coercive Projection Method (CPM) and its gravitational instantiation (CPM-Gravity / ILG).
\vspace{0.8em}

\textbf{141. The DREAM Theorem: Virtues as Generators of Ethical Symmetry}: \texttt{papers/tex/dream\_theorem.tex} (Authorship: Nov 30, 2025)
\\\textit{Why}: We prove that virtues are the complete, minimal generating set for all admissible ethical transformations in Recognition Science.
\vspace{0.8em}

\textbf{142. Theoretical Bounds on Global-Only Rotation Curve Fits}: \texttt{external/gravity/Theoretical-Bounds.tex} (Authorship: Nov 22, 2025)
\\\textit{Why}: We estimate the theoretical lower bound for the reduced statistic () in galaxy rotation curve fits under a strict "Global-Only" policy.
\vspace{0.8em}

\textbf{143. Gravity Derived: Emergence from Information Constraints}: \texttt{external/gravity/active/paper/Gravity-derived.tex} (Authorship: Nov 22, 2025)
\\\textit{Why}: Gravity emerges from finite information–bandwidth constraints on the substrate that maintains gravitational fields.
\vspace{0.8em}

\textbf{144. Galaxy Rotation Curves from an Information-Limited Gravitational Model}: \texttt{external/gravity/active/paper/dark-matter-galaxy-rotation.tex} (Authorship: Nov 22, 2025)
\\\textit{Why}: Gravity may appear modified on galactic scales if the exchange of dynamical information is limited by finite propagation and processing rates.
\vspace{0.8em}

\textbf{145. The Universal Light Language: A Physically Forced Protocol for Cross-Modal and Telepathic Communication}: \texttt{papers/tex/universal-light.tex} (Authorship: Nov 20, 2025)
\\\textit{Why}: We present the Universal Light Language (ULL), a zero-parameter semantic code derived entirely from the axioms of Recognition Science.
\vspace{0.8em}

\textbf{146. The Theory of Us}: \texttt{papers/tex/Us.tex} (Authorship: Nov 16, 2025)
\\\textit{Why}: Contributes to the Recognition Science framework.
\vspace{0.8em}

\textbf{147. The Coercive Projection Method: Axioms, Theorems, and Applications}: \texttt{light-language/CPM.tex} (Authorship: Nov 15, 2025)
\\\textit{Why}: The Coercive Projection Method (CPM) is a reusable proof template that converts quantitative distance-to-structure control into global positivity or existence statements.
\vspace{0.8em}

\textbf{148. Axiomatic Completeness of the Light Language}: \texttt{light-language/LightLanguageFormalization.tex} (Authorship: Nov 15, 2025)
\\\textit{Why}: We derive and formalize the Light Language, the unique zero-parameter semantic calculus enforced by Recognition Science (RS).
\vspace{0.8em}

\textbf{149. Collatz via Finite Window--Funnel Certificates (CPM Form)}: \texttt{papers/tex/collatz-conjecture.tex} (Authorship: Nov 15, 2025)
\\\textit{Why}: Applies the Coercive Projection Method.
\vspace{0.8em}

\textbf{150. The Universal Light Language: Formal Foundations, Implementation, and Truth Certification}: \texttt{papers/tex/light-language-2.tex} (Authorship: Nov 14, 2025)
\\\textit{Why}: We present the Universal Light Language (ULL), a zero-parameter semantic calculus discovered by enforcing the gates of Recognition Science.
\vspace{0.8em}

\textbf{151. How Meaning is Derived in the Universal Light Language}: \texttt{papers/tex/meaning-derivation.tex} (Authorship: Nov 14, 2025)
\\\textit{Why}: The Universal Light Language (ULL) is a zero-parameter way to encode recognition-ledger patterns at the Recognition Science (RS) bridge.
\vspace{0.8em}

\textbf{152. What the Universal Light Language Is: A Philosophical and Structural Overview}: \texttt{papers/tex/universal-light-language.tex} (Authorship: Nov 14, 2025)
\\\textit{Why}: The Universal Light Language (ULL) is not a toy coding scheme and not merely a clever representation for signals.
\vspace{0.8em}

\textbf{153. Recognition Science: Derivation Chain}: \texttt{papers/tex/DERIVATION\_CHAIN.tex} (Authorship: Nov 06, 2025)
\\\textit{Why}: We present the complete derivation chain for Recognition Science, showing rigorously that from the Meta Principle and zero-parameter constraint, all of RS structure is mathematically forced.
\vspace{0.8em}

\textbf{154. How We Proved Recognition Science Is Inevitable: A Plain-Language Explanation}: \texttt{papers/tex/Recognition\_Science\_Inevitability\_Explained.tex} (Authorship: Nov 06, 2025)
\\\textit{Why}: This document explains, without mathematics, how we formally proved that Recognition Science is the inevitable consequence of demanding a complete explanation of reality.
\vspace{0.8em}

\textbf{155. The Inevitability of Recognition Science: Proving Completeness Forces Structure}: \texttt{papers/tex/inevitability.tex} (Authorship: Nov 06, 2025)
\\\textit{Why}: We recently proved that Recognition Science (RS) is the unique zero-parameter framework deriving observables from first principles, establishing RS uniqueness among parameter-free theories.
\vspace{0.8em}

\textbf{156. A CPM Companion for Protein Folding(Coercive Projection Method}: \texttt{papers/tex/CPM-Folding-Companion-arXiv.tex} (Authorship: Nov 04, 2025)
\\\textit{Why}: We provide a concise companion to ``Protein folding as phase recognition'' that instantiates the Coercive Projection Method (CPM) for protein folding.
\vspace{0.8em}

\textbf{157. The Law of Existence:0.3em Proven Uniqueness of -Cost Minimization0.2em Across Mathematics, Physics, Biology, and Consciousness}: \texttt{papers/tex/Law-of-Existence-arXiv.tex} (Authorship: Nov 04, 2025)
\\\textit{Why}: The Coercive Projection Method (CPM) is the universal algorithm by which possibilities collapse to actuality---the same minimum-description-length optimization Darwin discovered for biological fitn...
\vspace{0.8em}

\textbf{158. The Minimal Complexity Functional: A Parameter,}: \texttt{papers/tex/Minimal-Complexity-Functional-arXiv.tex} (Authorship: Nov 04, 2025)
\\\textit{Why}: We formulate a minimal, ,complexity principle for physical evolution built on a single convex, symmetric, normalized functional on the positive reals, ( (x)= 12(x+1/x)-1).
\vspace{0.8em}

\textbf{159. Recognition Abiogenesis Arxiv}: \texttt{papers/tex/Recognition-Abiogenesis-arXiv.tex} (Authorship: Nov 04, 2025)
\\\textit{Why}: Addresses origin of life via RS.
\vspace{0.8em}

\textbf{160. Crystallography Selection Rules from Eight-Window Neutrality and Legal Triads}: \texttt{papers/tex/Crystallography-SelectionRules.tex} (Authorship: Nov 03, 2025)
\\\textit{Why}: We propose selection rules for reciprocal-space motifs based on (i) an eight-window neutrality diagnostic and (ii) a legal-triad parity constraint.
\vspace{0.8em}

\textbf{161. \% -1em}: \texttt{papers/tex/EightAxiomsForced.tex} (Authorship: Nov 03, 2025)
\\\textit{Why}: We show that a single logical tautology—the Meta,Principle (MP), “nothing cannot recognize itself”—forces eight core theorems (T1–T8) that pin down the recognition ledger, the unique convex symmetr...
\vspace{0.8em}

\textbf{162. Environment Display Rescale: \$E' = E, \textasciicircum{} P}: \texttt{papers/tex/Env-Pressure-Display.tex} (Authorship: Nov 03, 2025)
\\\textit{Why}: We propose a display-only rescale for environment-dependent observables, , that leaves the integer scaffolding intact.
\vspace{0.8em}

\textbf{163. Information-Limited Gravity as a Pressure Display (Algebraic Equivalence)}: \texttt{papers/tex/ILG-Pressure-Form.tex} (Authorship: Nov 03, 2025)
\\\textit{Why}: We rewrite the ILG effective source term, (4 G a\textasciicircum{}2, ,w(k,a), ), using a pressure variable (p:= ,w, ), yielding the identical display (4 G a\textasciicircum{}2,p).
\vspace{0.8em}

\textbf{164. A Fit-Free Periodic Table Engine from Eight-Tick Cadence and -Tiers}: \texttt{papers/tex/Periodic-Table-Engine.tex} (Authorship: Nov 03, 2025)
\\\textit{Why}: We present a zero-parameter engine for periodic-table trends based on two Recognition Science invariants: (i) a forced eight-tick cadence that induces eight-window neutrality ``rests'' at closures,...
\vspace{0.8em}

\textbf{165. Pressure Gravity Arxiv}: \texttt{papers/tex/Pressure-Gravity-arXiv.tex} (Authorship: Nov 03, 2025)
\\\textit{Why}: We recast Information-Limited Gravity (ILG) as classical gravity sourced by an effective pressure field.
\vspace{0.8em}

\textbf{166. Spiral Wavefields from -Cost with -Scaling and Eight-Gate Neutrality}: \texttt{papers/tex/Spiral-Wavefields.tex} (Authorship: Nov 03, 2025)
\\\textit{Why}: We propose a fit-free spiral-field variational ansatz: under the unique convex cost and -scaling, with an eight-phase neutrality gate, the stationary in-plane flow takes a logarithmic-spiral form w...
\vspace{0.8em}

\textbf{167. Virtues as Generators: A Zero-Parameter, Auditable Ethics from Recognition Science}: \texttt{papers/tex/Virtues-As-Generators.tex} (Authorship: Nov 03, 2025)
\\\textit{Why}: We present a companion to ``Morality as a Conservation Law'', extending the Recognition Science (RS) framework from feasibility to operation: fourteen virtues are formalized as the complete set of ...
\vspace{0.8em}

\textbf{168. Copenhagen Interpretation: , Inconsistencies, and Circularities}: \texttt{papers/tex/interpretation.tex} (Authorship: Nov 01, 2025)
\\\textit{Why}: Contributes to the Recognition Science framework.
\vspace{0.8em}

\textbf{169. The Mathematical Foundations of the 14 Virtues}: \texttt{papers/tex/virtues.tex} (Authorship: Nov 01, 2025)
\\\textit{Why}: This document provides a detailed mathematical exposition of the 14 virtues as formalized in the ledger-ethics Lean 4 repository.
\vspace{0.8em}

\textbf{170. Recognition Science Baryogenesis: A Parameter-Free Resolution of the Matter-Antimatter Asymmetry}: \texttt{papers/tex/Baryogenesis-HubbleTensionSet.tex} (Authorship: Oct 29, 2025)
\\\textit{Why}: . The observed baryon-to-photon ratio, , demands efficient violation and a controlled departure from equilibrium in the early universe. Standard Model...
\vspace{0.8em}

\textbf{171. C=2A: Unifying Quantum Measurement, Gravitational Collapse, and Consciousness}: \texttt{papers/tex/C2A-arXiv.tex} (Authorship: Oct 29, 2025)
\\\textit{Why}: Problem: Quantum measurement, gravity-driven reduction, and the definiteness of conscious experience are usually treated as separate puzzles.
\vspace{0.8em}

\textbf{172. Information-Limited Gravity: Source-Side Kernel Tests Against Distances, Growth, and Lensing}: \texttt{papers/tex/Dark-Energy-HubbleTensionSet.tex} (Authorship: Oct 29, 2025)
\\\textit{Why}: We present a fixed-constant, no-fit, source-side modification to gravitational sourcing—the information-limited gravity (ILG) kernel—and confront it with cosmological observables that usually motiv...
\vspace{0.8em}

\textbf{173. Late-time Recognition-Weighted Growth and the Hubble Tension}: \texttt{papers/tex/Hubble-Tension-Resolution-HubbleTensionSet.tex} (Authorship: Oct 29, 2025)
\\\textit{Why}: Background. Late-time structure probes and CMB inferences yield discrepant values of the Hubble constant when analyzed under standard GR growth kernel...
\vspace{0.8em}

\textbf{174. Light as Consciousness: A Bi-Interpretability Theorem with Mechanical Verification (Alternate title: Photonic Equivalence of Operational Consciousness at the RS Bridge)}: \texttt{papers/tex/Light-Consciousness-Theorem-arXiv.tex} (Authorship: Oct 29, 2025)
\\\textit{Why}: We convert the slogan ``Light = Consciousness'' into a formal theorem by proving a bi-interpretability result at the Recognition Science bridge.
\vspace{0.8em}

\textbf{175. Meta Principle Arxiv}: \texttt{papers/tex/Meta-Principle-arXiv.tex} (Authorship: Oct 29, 2025)
\\\textit{Why}: Contributes to the Recognition Science framework.
\vspace{0.8em}

\textbf{176. Morality as a Conservation Law in a Recognition-Structured Universe}: \texttt{papers/tex/Morality-As-Conservation-Law.tex} (Authorship: Oct 29, 2025)
\\\textit{Why}: This paper derives a parameter-free moral law from the same physical invariants that fix the Recognition Science (RS) bridge between the discrete ledger and the continuum.
\vspace{0.8em}

\textbf{177. Recognition Compliant Perovskite on Silicon Tandems: A Parameter Free Stability Framework That Survives Damp Heat + UV}: \texttt{papers/tex/Perovskite-arXiv.tex} (Authorship: Oct 29, 2025)
\\\textit{Why}: We demonstrate a manufacturing and certification framework---grounded in recognition science invariants---that enables perovskite on silicon tandem modules to achieve efficiency while preserving pe...
\vspace{0.8em}

\textbf{178. Protein folding as phase recognition: a formal framework and executable pipeline with testable IR signatures}: \texttt{papers/tex/Protein Folding as Phase Recognition.tex} (Authorship: Oct 29, 2025)
\\\textit{Why}: We recast protein folding as a fast, instrument-coupled phase–recognition process rather than a slow, combinatorial search.
\vspace{0.8em}

\textbf{179. Zero-Parameter Quantum Gravity from Discrete Recognition Calculus}: \texttt{papers/tex/Quantum-Gravity-New-HubbleTensionSet.tex} (Authorship: Oct 29, 2025)
\\\textit{Why}: We derive classical and quantum gravity from a minimal information-theoretic axiom—a recognition event requires non-empty data—with parameter-fixed, gauge-rigid displays (dimensionless quantities i...
\vspace{0.8em}

\textbf{180. Parameter Free Synthetic Conductivity: A Recognition Science Bridge to Room Temperature Emulation}: \texttt{papers/tex/RS-Conductor-arXiv.tex} (Authorship: Oct 29, 2025)
\\\textit{Why}: We present a parameter free synthetic conductor that emulates two hallmark features of superconductivity near zero voltage drop and band limited magnetic field expulsion without invoking a new ther...
\vspace{0.8em}

\textbf{181. Recognition Abiogenesis}: \texttt{papers/tex/Recognition-Abiogenesis.tex} (Authorship: Oct 29, 2025)
\\\textit{Why}: Addresses origin of life via RS.
\vspace{0.8em}

\textbf{182. Beyond the Hamiltonian: The Recognition Operator as Fundamental Dynamics}: \texttt{papers/tex/Recognition-Operator-arXiv.tex} (Authorship: Oct 29, 2025)
\\\textit{Why}: For four centuries, the Hamiltonian has been treated as fundamental, with dynamics derived from energy minimization.
\vspace{0.8em}

\textbf{183. Recognition Architecture (Integrated)}: \texttt{papers/tex/Recognition\_Architecture-arXiv.tex} (Authorship: Oct 29, 2025)
\\\textit{Why}: This paper presents a complete, parameter-free recognition architecture whose proof layer is strictly dimensionless and whose empirical layer is reduced to a small set of layered falsifiability gates.
\vspace{0.8em}

\textbf{184. Bootstrap Origin without Singularity: Emergent Spacetime from a Discrete Conserved Network}: \texttt{papers/tex/Universe-Origin-HubbleTensionSet.tex} (Authorship: Oct 29, 2025)
\\\textit{Why}: We present a singularity-free origin scenario in which spacetime emerges from a discrete, conserved adjacency network.
\vspace{0.8em}

\textbf{185. Goldbach via a Mod-8 Kernel: Density-One and Short-Interval Positivity}: \texttt{papers/tex/goldbach\_rs-arXiv.tex} (Authorship: Oct 29, 2025)
\\\textit{Why}: We present a purely classical framework for Goldbach's conjecture based on a mod-8 periodic kernel and the circle method.
\vspace{0.8em}

\textbf{186. Light as Consciousness: A Universal Information-Cost Identity from a Unique Convex Functional}: \texttt{papers/tex/light-consiousness-arXiv.tex} (Authorship: Oct 29, 2025)
\\\textit{Why}: We show that a single, uniquely determined information-cost functional governs quantum measurement, photonic operations, and operational (measurement-like) conscious selection, establishing an iden...
\vspace{0.8em}

\textbf{187. Exclusivity of Recognition Science (RS): The Unique Zero -}: \texttt{papers/tex/exclusivity.tex} (Authorship: Oct 28, 2025)
\\\textit{Why}: We prove an exclusivity theorem for Recognition Science (RS)Also referred to as the Recognition Physics framework in some artifacts; we standardize on “Recognition Science (RS)” per brand policy.
\vspace{0.8em}

\textbf{188. A Universal Register Mapping for the Light-Native Assembly Language -Start Guide for Multi-Domain Ledger Initialisation}: \texttt{papers/tex/LNAL-Register-Mapping.tex} (Authorship: Oct 22, 2025)
\\\textit{Why}: The Light-Native Assembly Language (LNAL) offers a 16-opcode, cost-balanced instruction set that, in principle, can compile any physical process into an executable ledger of recognition moves.
\vspace{0.8em}

\textbf{189. Quantum Coherence as Gated Recognition: An Eight-Tick Mechanism with Parameter-Free Bridges}: \texttt{papers/tex/Quantum-Coherence-Theory.tex} (Authorship: Oct 22, 2025)
\\\textit{Why}: We address the core question of quantum coherence: when is phase information preserved and when does it dephase under realistic readout? We show that coherence is an operational property of a discr...
\vspace{0.8em}

\textbf{190. No Per Flavor Tuning}: \texttt{papers/tex/No-Per-Flavor-Tuning.tex} (Authorship: Oct 13, 2025)
\\\textit{Why}: We present four theory results that require no experimental input.
\vspace{0.8em}

\textbf{191. No Per Flavor Tuning Massespapers}: \texttt{papers/tex/No-Per-Flavor-Tuning-MassesPapers.tex} (Authorship: Oct 12, 2025)
\\\textit{Why}: We present four theory results that require no experimental input.
\vspace{0.8em}

\textbf{192. Information-Limited Quantum Gravity: A Parameter-Free, Audit-Gated Scaffold with GR-Limit Derivations}: \texttt{papers/tex/ILG-GPT5.tex} (Authorship: Sep 30, 2025)
\\\textit{Why}: We present a parameter-free framework for quantum gravity built from a single measurement principle and an information-limited action.
\vspace{0.8em}

\textbf{193. Information-Limited Gravity: A Mechanized, Covariant, Quantum-Consistent Framework with Observational Gates}: \texttt{papers/tex/QG\_PRD.tex} (Authorship: Sep 30, 2025)
\\\textit{Why}: We present a covariant, quantum‑consistent gravitational framework built from information‑limited principles and verified end‑to‑end, while keeping the narrative human‑first.
\vspace{0.8em}

\end{document}