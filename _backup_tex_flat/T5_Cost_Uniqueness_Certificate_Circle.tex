\documentclass[11pt]{article}

\usepackage[margin=1in]{geometry}
\usepackage{amsmath, amssymb, amsthm}
\usepackage{mathtools}
\usepackage{microtype}
\usepackage{hyperref}
\usepackage{xcolor}

\hypersetup{
  colorlinks=true,
  linkcolor=blue!60!black,
  citecolor=blue!60!black,
  urlcolor=blue!60!black
}

\newtheorem{theorem}{Theorem}
\newtheorem{definition}{Definition}
\newtheorem{remark}{Remark}

\newcommand{\R}{\mathbb{R}}
\newcommand{\Rp}{\R_{>0}}

\title{T5 Cost Uniqueness and the Certificate Circle\\
\large What Completing ``T5'' Certifies in the \texttt{reality} Repository}
\author{}
\date{\today}

\begin{document}
\maketitle

\begin{abstract}
This note explains the mathematical and engineering content of completing ``T5'' in the
\texttt{reality} repository's Lean formalization workflow. In this codebase, T5 is packaged as a
\emph{certificate} (\texttt{IndisputableMonolith/Verification/T5UniqueCert.lean}) asserting a
uniqueness theorem for the Recognition Science cost function \(J\): any function satisfying the
certified ``JensenSketch'' obligations agrees with \(J\) on all positive reals. We describe (i) the
certificate-circle methodology used to expand the machine-checked surface while enforcing
non-circularity, (ii) the explicit definition and key identities of \(J\), (iii) the precise statement
of the certified T5 theorem, and (iv) what this theorem does and does not guarantee about the
overall theory stack.
\end{abstract}

\section{The certificate circle: what it is and why it matters}

The repository uses a disciplined pattern to grow what it calls the \emph{certified surface}:
the set of claims that are not merely written down, but \emph{machine-checked} and imported by a
top-level certificate module.

\begin{definition}[Certificate module (repo convention)]
A certificate module is a Lean file that defines a record \texttt{\dots Cert} together with:
\begin{itemize}
  \item a predicate \texttt{\dots Cert.verified : Prop} that expresses the intended claim; and
  \item a theorem \texttt{\dots Cert.verified\_any : \dots Cert.verified} proved with no \texttt{sorry}
        and no vacuous placeholders.
\end{itemize}
In practice, the record often carries no data; its role is to package a named proposition and a
named proof in a stable import path.
\end{definition}

This pattern supports a ``certificate circle'': each completed step enlarges the certified surface,
and future steps are required to import and build on prior certificates rather than reintroducing
informal assumptions.

\subsection{Where T5 sits in the certificate chain}

The current top-level certificate anchor is
\texttt{IndisputableMonolith/URCGenerators/UltimateCPMClosureCert.lean}. This module imports the
audit certificate \texttt{IndisputableMonolith/Verification/NonCircularityCert.lean}. Since T5 is a
conjunct of \texttt{NonCircularityCert.verified}, completing T5 places the T5 theorem inside the
transitive import closure of the top-level certificate.

\subsection{Non-circularity as an explicit invariant}

The same workflow imposes a crucial constraint: \emph{non-circularity}. The repo explicitly audits
against smuggling patterns such as:
\begin{itemize}
  \item proving ``matches'' by hard-coding empirical numerals and using \texttt{rfl};
  \item \texttt{True} stubs or vacuous existentials (\(\exists c,\ \texttt{True}\));
  \item hidden axioms introduced via typeclass instances.
\end{itemize}

At the Lean level, the main audit anchor is
\texttt{IndisputableMonolith/Verification/NonCircularityCert.lean}, which is a large conjunction of
certified facts (defaults, invariances, bridges, cost facts, etc.). T5 appears as one conjunct in this
audit certificate, meaning it is explicitly part of the ``what is certified'' story for the project.

\section{The cost function \(J\) in the formalization}

The Recognition Science cost used throughout the certified surface is the function
\(\,J:\Rp\to\R\,\) defined in Lean as \texttt{Jcost}. Its definition is explicit and algebraic:

\begin{definition}[J-cost]
For \(x\in\R\), define
\[
J(x)\ :=\ \frac{x + x^{-1}}{2} - 1.
\]
In the certified statements, the relevant domain is \(x>0\).
\end{definition}

Several basic properties are proved in the cost layer. Two identities are especially important:

\begin{remark}[Symmetry and squared form]
For \(x>0\),
\[
J(x)=J(x^{-1}),
\qquad
J(x)=\frac{(x-1)^2}{2x}\ge 0.
\]
The squared form makes non-negativity and the ``unique minimum at \(x=1\)'' intuition immediate.
\end{remark}

\subsection{Log-coordinates and the hyperbolic connection}

The cost layer also defines a log-coordinate representation
\[
J_{\log}(t)\ :=\ J(e^t).
\]
In Lean, this is \texttt{Jlog}. A key certified identity is:
\[
J_{\log}(t)=\cosh(t)-1.
\]
This ties the cost geometry to a canonical hyperbolic function and supports analytic statements
about derivatives, convexity, and normalization (e.g.\ \(J_{\log}''(0)=1\)).

\section{The T5 theorem that was completed}

\subsection{The certified interface: \texttt{JensenSketch}}

In the codebase, T5 is phrased as a uniqueness result relative to a small interface called
\texttt{JensenSketch}. Informally, \texttt{JensenSketch F} asserts that \(F\) is:
\begin{itemize}
  \item symmetric under inversion (\(F(x)=F(1/x)\) for \(x>0\));
  \item normalized at \(1\) (\(F(1)=0\));
  \item ``squeezed'' on the exponential axis by \(J\), i.e.\ for all \(t\in\R\),
        \(F(e^t)\le J(e^t)\) and \(J(e^t)\le F(e^t)\).
\end{itemize}
The squeeze is logically equivalent to the pointwise equality \(F(e^t)=J(e^t)\), but it is written
as two inequalities to match how such equalities are typically obtained in analysis
(upper and lower bounds derived independently).

\subsection{T5 (as certified)}

The certificate \texttt{IndisputableMonolith/Verification/T5UniqueCert.lean} packages the following
statement:

\begin{theorem}[T5 cost uniqueness on \(\Rp\)]
Let \(F:\R\to\R\). If \(F\) satisfies the \texttt{JensenSketch} obligations, then for all \(x>0\),
\[
F(x)=J(x).
\]
\end{theorem}

\begin{remark}[Proof idea]
The proof is conceptually simple: the \texttt{JensenSketch} bounds give equality on the exponential
axis, and every \(x>0\) can be written as \(x=e^{\log x}\), so equality transfers from
\(\{e^t:t\in\R\}\) to all of \(\Rp\). In Lean this is implemented as
\texttt{T5\_cost\_uniqueness\_on\_pos}.
\end{remark}

\section{A ``full'' (classical) uniqueness route exists in the repo}

While the certified T5 theorem is phrased at the \texttt{JensenSketch} interface, the repository also
contains a more classical uniqueness theorem (\texttt{IndisputableMonolith/CostUniqueness.lean}).
That theorem shows how one can force \(F=J\) from more recognizable analytic hypotheses such as:
symmetry, strict convexity on \(\Rp\), calibration in log-coordinates, continuity, and a cosh-type
functional identity.

This matters for interpretation:
\begin{itemize}
  \item The T5 certificate says: \emph{once the JensenSketch obligations are met}, the cost is forced.
  \item The deeper uniqueness theorem suggests: there are principled analytic roads to proving those
        obligations without hard-coding \(J\) into the assumptions.
\end{itemize}

\section{What completing T5 means for the certificate circle}

Completing T5 has two distinct meanings: a mathematical meaning and a certification meaning.

\subsection{Mathematical meaning}

Mathematically, T5 asserts that \(J\) is not an arbitrary choice once the project commits to the
invariances and normalizations encoded by \texttt{JensenSketch}. Downstream theorems that depend on
these obligations cannot silently ``swap in'' a different cost function to fit desired outcomes: the
interface forces the same \(J\) everywhere on \(\Rp\).

\subsection{Certification meaning}

Engineering-wise, T5 is now part of the auditable, import-stable certified surface:
\begin{itemize}
  \item \texttt{T5UniqueCert} provides a named proposition (\texttt{verified}) and a named proof
        (\texttt{verified\_any}) with no \texttt{sorry};
  \item \texttt{NonCircularityCert} imports \texttt{T5UniqueCert} and includes it as one conjunct in its
        verified predicate, so the audit certificate explicitly depends on T5.
\end{itemize}

In other words: T5 is not only proven somewhere in the library; it is a \emph{declared checkpoint}
in the repo's audit certificate, and thus a stable part of what the project is willing to call
``certified''.

\section{What T5 does \emph{not} claim}

It is equally important to state the boundary:
\begin{itemize}
  \item T5 does not claim that empirical physics selects \(J\) from nothing; it claims uniqueness
        \emph{conditional on} the stated obligations.
  \item T5 does not certify any CODATA/SI numerics. The repo treats hard-coded empirical numerics as
        quarantined from the certified surface.
  \item T5 does not automatically certify every downstream ``closure'' claim; it is one component in
        a broader non-circularity story that also includes explicit match evaluators, calibration
        witnesses, and rescaling invariances.
\end{itemize}

\section{Reproducibility (artifact map)}

The relevant Lean artifacts are:
\begin{itemize}
  \item \texttt{IndisputableMonolith/URCGenerators/UltimateCPMClosureCert.lean} (top-level certificate
        anchor that imports \texttt{NonCircularityCert});
  \item \texttt{IndisputableMonolith/Cost.lean} (definition of \texttt{Jcost}, \texttt{Jlog}, and
        \texttt{JensenSketch}, plus \texttt{T5\_cost\_uniqueness\_on\_pos});
  \item \texttt{IndisputableMonolith/Verification/T5UniqueCert.lean} (packages T5 as a cert);
  \item \texttt{IndisputableMonolith/Verification/NonCircularityCert.lean} (imports and asserts T5 as
        part of the audit certificate);
  \item \texttt{IndisputableMonolith/CostUniqueness.lean} (a stronger, explicit-hypothesis uniqueness
        theorem that explains one non-circular route to deriving \(J\)).
\end{itemize}

Typical local verification commands (from repo root) are of the form:
\begin{center}
\texttt{lake build IndisputableMonolith.URCGenerators.UltimateCPMClosureCert}\\
\texttt{lake build IndisputableMonolith.Verification.T5UniqueCert}\\
\texttt{lake build IndisputableMonolith.Verification.NonCircularityCert}
\end{center}

\end{document}


