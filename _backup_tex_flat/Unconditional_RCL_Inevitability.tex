\documentclass[12pt,a4paper]{article}

% Minimal package set (portable TeX Live)
\usepackage{amsmath,amssymb,amsthm}
\usepackage{geometry}
\usepackage{hyperref}
\geometry{margin=1in}

\hypersetup{colorlinks=true, linkcolor=blue, citecolor=blue, urlcolor=blue}

\theoremstyle{plain}
\newtheorem{theorem}{Theorem}[section]
\newtheorem{lemma}[theorem]{Lemma}
\newtheorem{corollary}[theorem]{Corollary}
\theoremstyle{definition}
\newtheorem{definition}[theorem]{Definition}
\theoremstyle{remark}
\newtheorem{remark}[theorem]{Remark}

\newcommand{\R}{\mathbb{R}}
\newcommand{\Rp}{\mathbb{R}_{>0}}

\title{\textbf{The Inevitability of the Recognition Composition Law}\\[0.5em]
\large Unconditional forcing of the combiner from the canonical cost}
\author{Jonathan Washburn\\[0.25em]
Recognition Science Research Institute\\[0.5em]
\small Proof artifacts in Lean 4 (\texttt{IndisputableMonolith})}
\date{January 2026}

\begin{document}
\maketitle

\begin{abstract}
We isolate and formalize a precise ``unconditional'' statement about the Recognition Composition Law (RCL).
Let
\[
J(x)\;:=\;\frac{1}{2}\!\left(x+\frac{1}{x}\right)-1,\qquad x>0,
\]
and suppose there exists \emph{any} function $P:\R\times\R\to\R$ such that
\[
J(xy)+J(x/y)=P(J(x),J(y))\qquad\forall x,y>0.
\]
Then $P$ is uniquely forced on the whole first quadrant:
\[
P(u,v)=2uv+2u+2v\qquad\forall u,v\ge 0.
\]
No regularity, algebraic form, continuity, or measurability of $P$ is assumed; only existence.
This ``computed-combiner'' theorem is machine-verified in Lean 4
(\texttt{IndisputableMonolith/Foundation/DAlembert/Unconditional.lean}, theorem \texttt{rcl\_unconditional}).
We also explain what this result does \emph{not} claim (it does not by itself force an arbitrary cost $F$ to equal $J$).
\end{abstract}

\tableofcontents

\section{Statement}

\begin{definition}[Canonical reciprocal cost]
Define the canonical cost $J:\Rp\to\R$ by
\[
J(x)=\frac{1}{2}\!\left(x+\frac{1}{x}\right)-1.
\]
\end{definition}

\begin{definition}[Cost-combiner consistency]\label{def:combiner}
Let $F:\Rp\to\R$ be a cost function. A \emph{combiner} for $F$ is any function $P:\R\times\R\to\R$ such that
\[
F(xy)+F(x/y)=P(F(x),F(y))\qquad\forall x,y>0.
\]
\end{definition}

\begin{definition}[RCL polynomial]
Define $P_{\mathrm{RCL}}:\R\times\R\to\R$ by
\[
P_{\mathrm{RCL}}(u,v)=2uv+2u+2v.
\]
\end{definition}

\begin{theorem}[Unconditional forcing of the combiner]\label{thm:unconditional}
Let $P:\R\times\R\to\R$ satisfy the combiner consistency equation for the canonical cost $J$, i.e.
\[
J(xy)+J(x/y)=P(J(x),J(y))\qquad\forall x,y>0.
\]
Then for all $u,v\ge 0$,
\[
P(u,v)=P_{\mathrm{RCL}}(u,v)=2uv+2u+2v.
\]
\end{theorem}

\begin{remark}[What ``unconditional'' means here]
The theorem is unconditional with respect to the \emph{form} of $P$:
we assume nothing about $P$ except that it exists and satisfies the functional equation with $J$.
In particular, $P$ need not be polynomial, continuous, or even measurable.
\end{remark}

\section{Lean formalization}

\subsection{Key file}

All statements in Theorem~\ref{thm:unconditional} are verified in Lean 4 in:
\begin{itemize}
  \item \texttt{IndisputableMonolith/Foundation/DAlembert/Unconditional.lean}
\end{itemize}

\subsection{Key theorems (Lean names)}

The proof chain is implemented via the following theorems (names as in Lean):
\begin{itemize}
  \item \texttt{J\_computes\_P}: the d'Alembert/RCL identity holds for \texttt{Cost.Jcost}.
  \item \texttt{J\_surjective\_nonneg}: $J$ is surjective onto $[0,\infty)$.
  \item \texttt{P\_determined\_nonneg}: any combiner agreeing with $J$ is determined on $[0,\infty)^2$.
  \item \texttt{rcl\_unconditional}: the final statement (Theorem~\ref{thm:unconditional}).
\end{itemize}

\section{Proof sketch (mathematics)}

The Lean proof follows a clean three-step strategy.

\subsection{Step 1: $J$ satisfies the RCL identity}

\begin{lemma}[RCL identity for $J$]\label{lem:rclJ}
For all $x,y>0$,
\[
J(xy)+J(x/y)=2J(x)J(y)+2J(x)+2J(y).
\]
\end{lemma}

This is a direct algebraic identity (proved in Lean as \texttt{J\_computes\_P} by converting to log-coordinates and using a verified cosh-add identity).

\subsection{Step 2: $J$ hits every value in $[0,\infty)$}

\begin{lemma}[Surjectivity]\label{lem:surj}
For every $u\ge 0$ there exists $x>0$ such that $J(x)=u$.
\end{lemma}

One explicit solution is $x=u+1+\sqrt{u^2+2u}$; Lean proves this as \texttt{J\_surjective\_nonneg}.

\subsection{Step 3: compute $P$ on the whole first quadrant}

Assume $P$ satisfies the combiner consistency equation with $J$.
Given any $u,v\ge 0$, choose $x,y>0$ with $J(x)=u$ and $J(y)=v$ (Lemma~\ref{lem:surj}).
Then
\[
P(u,v)=P(J(x),J(y))=J(xy)+J(x/y)=2uv+2u+2v,
\]
using the defining equation for $P$ and Lemma~\ref{lem:rclJ}.
This is exactly \texttt{P\_determined\_nonneg} in Lean, and yields Theorem~\ref{thm:unconditional}.

\section{Scope and peer-review notes}

\begin{remark}[What this theorem does \emph{not} prove]
Theorem~\ref{thm:unconditional} does \emph{not} by itself show that an arbitrary cost function $F$
must equal $J$.
It establishes a different rigidity fact: \emph{once the canonical $J$ is fixed}, the combiner $P$
compatible with \eqref{def:combiner} is forced and unique on $[0,\infty)^2$.
\end{remark}

\begin{remark}[How this connects to broader ``inevitability'' claims]
To promote ``the combiner for $J$ is forced'' to ``the RCL is forced for all admissible costs,''
one needs (in addition) a separate theorem identifying $J$ as the unique admissible cost.
The Lean repository contains a fully formal cost-uniqueness development (T5) with explicit hypotheses
in \texttt{IndisputableMonolith/CostUniqueness.lean}.
\end{remark}

\section{Conclusion}

The unconditional Lean theorem \texttt{rcl\_unconditional} proves a strong and audit-friendly statement:
\emph{if any function $P$ composes $J$-costs via the product/quotient identity, then $P$ must equal the RCL polynomial on the entire first quadrant}.
Thus, the RCL combiner is not a modeling choice once $J$ is fixed; it is mathematically forced.

\end{document}


