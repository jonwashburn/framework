\documentclass[11pt,a4paper]{article}

% ===== Packages =====
\usepackage{amsmath,amssymb,amsthm}
\usepackage{mathtools}
\usepackage{graphicx}
\usepackage{hyperref}
\usepackage{booktabs}
\usepackage{array}

% ===== Theorem environments =====
\newtheorem{theorem}{Theorem}[section]
\newtheorem{lemma}[theorem]{Lemma}
\newtheorem{proposition}[theorem]{Proposition}
\newtheorem{corollary}[theorem]{Corollary}
\newtheorem{definition}[theorem]{Definition}
\newtheorem{example}[theorem]{Example}
\newtheorem{remark}[theorem]{Remark}

% ===== Custom commands =====
\newcommand{\C}{\mathbb{C}}
\newcommand{\R}{\mathbb{R}}
\newcommand{\Z}{\mathbb{Z}}
\newcommand{\N}{\mathbb{N}}
\newcommand{\ph}{\varphi}
\newcommand{\DFT}{\mathrm{DFT}_8}
\newcommand{\iDFT}{\mathrm{iDFT}_8}
\newcommand{\WToken}{\mathcal{W}}
\newcommand{\Voxel}{\mathcal{V}}

% ===== Title =====
\title{%
  \textbf{The Voxel as Meaning}\\[0.5em]
  \large How Eight Photons Create Semantic Content\\
  via Discrete Fourier Decomposition
}

\author{
  Recognition Science\\
  \texttt{IndisputableMonolith/OctaveKernel/VoxelMeaning.lean}
}

\date{December 2024}

\begin{document}

\maketitle

\begin{abstract}
We present a mathematical formalization of how meaning emerges from the fundamental structure of light. A \emph{voxel}---the minimal unit of spatial reality in Recognition Science---is shown to be not a container holding particles, but rather an \emph{eight-phase chord}: eight photons at different phase positions sounding simultaneously. Through the 8-point Discrete Fourier Transform (DFT-8), we decompose any voxel into frequency modes that correspond to distinct semantic qualities. The neutrality constraint (ledger balance) forces the DC mode to vanish, leaving exactly four independent mode pairs. Quantizing amplitudes on the golden-ratio ($\ph$) ladder and classifying by mode activation yields precisely 20 minimal semantic atoms---the \emph{WTokens}---which form the Periodic Table of Meaning. All results are formalized in Lean 4 with machine-checked proofs.
\end{abstract}

\tableofcontents

%=============================================================================
\section{Introduction}
\label{sec:intro}
%=============================================================================

What is meaning? In Recognition Science, meaning is not an emergent property layered on top of physics---meaning \emph{is} physics, structured at the most fundamental level. The central object of this structure is the \emph{voxel}: the minimal unit of ``location in reality.''

\subsection{The Core Insight}

Standard intuition treats a voxel as empty space through which particles pass. This is wrong. A voxel \textbf{is} eight phases co-present---a chord, not a point. When a recognition event enters a voxel, it takes eight ticks to complete its cycle. In steady state, there are always eight tokens at different phases simultaneously. These eight tokens are not separate particles; they are eight aspects of a single recognition event \emph{being}.

\medskip
\fbox{\parbox{0.95\textwidth}{%
\textbf{Central Thesis:} \emph{A voxel is a chord of eight photons. Meaning is the frequency spectrum of that chord.}
}}
\medskip

\subsection{Physical Analogy}

Consider eight piano strings tuned to eight notes. When struck together, they produce a \emph{chord}---a superposition of frequencies. The identity of the chord is not in the individual strings but in the \emph{combination}. Similarly, a voxel's meaning is not in its individual phase slots but in the \emph{frequency spectrum} obtained by Fourier decomposition.

\subsection{Outline}

Section~\ref{sec:voxel} defines the voxel structure. Section~\ref{sec:dft} introduces the DFT-8 transform. Section~\ref{sec:meaning} shows how modes map to meaning. Section~\ref{sec:wtokens} derives the 20 WTokens. Section~\ref{sec:formalization} summarizes the Lean formalization.

%=============================================================================
\section{The Voxel Structure}
\label{sec:voxel}
%=============================================================================

\subsection{Photons at Phase Positions}

\begin{definition}[Photon]
A \emph{photon} is a light quantum characterized by:
\begin{itemize}
  \item \textbf{Amplitude} $a \in \R_{\geq 0}$: the energy content
  \item \textbf{Phase offset} $\theta \in [0, 2\pi)$: fine phase modulation within the slot
\end{itemize}
The complex representation is $\gamma = a \cdot e^{i\theta} \in \C$.
\end{definition}

\begin{definition}[Voxel]
A \emph{voxel} $\Voxel$ is a function from phase positions to photons:
\[
\Voxel : \{0, 1, 2, 3, 4, 5, 6, 7\} \to \{\text{Photons}\}
\]
We write $\gamma_k = \Voxel(k)$ for the photon at phase position $k$.
\end{definition}

\subsection{The Pipeline Model}

A voxel is not a static container; it is a \emph{pipeline} through which recognition events flow. At each tick, a new token enters at phase 0, all others advance by one, and the token at phase 7 exits:

\begin{center}
\begin{tabular}{ccccccccccc}
& $k{=}0$ & $k{=}1$ & $k{=}2$ & $k{=}3$ & $k{=}4$ & $k{=}5$ & $k{=}6$ & $k{=}7$ & \\
enter $\to$ & $[\gamma_0]$ & $[\gamma_1]$ & $[\gamma_2]$ & $[\gamma_3]$ & $[\gamma_4]$ & $[\gamma_5]$ & $[\gamma_6]$ & $[\gamma_7]$ & $\to$ exit
\end{tabular}
\end{center}

At steady state, all eight slots are occupied. The voxel \emph{is} these eight co-present phases.

\subsection{Physical Quantities}

\begin{definition}[Total Energy]
The total energy of a voxel is
\[
E[\Voxel] = \sum_{k=0}^{7} |\gamma_k|^2 = \sum_{k=0}^{7} a_k^2
\]
where $a_k$ is the amplitude of the photon at position $k$.
\end{definition}

\begin{definition}[Complex Signal]
The \emph{complex signal} of a voxel is the sequence
\[
x = (x_0, x_1, \ldots, x_7) \in \C^8, \quad x_k = \gamma_k = a_k e^{i\theta_k}
\]
\end{definition}

%=============================================================================
\section{The DFT-8 Transform}
\label{sec:dft}
%=============================================================================

\subsection{Definition}

The 8-point Discrete Fourier Transform maps the time-domain signal (photons at phase positions) to the frequency domain (modes of oscillation).

\begin{definition}[DFT-8]
For a signal $x \in \C^8$, the DFT-8 is
\[
X_k = \sum_{n=0}^{7} x_n \cdot \omega^{nk}, \quad k = 0, 1, \ldots, 7
\]
where $\omega = e^{-2\pi i / 8}$ is the primitive 8th root of unity.
\end{definition}

\begin{definition}[Inverse DFT-8]
The inverse transform recovers the time-domain signal:
\[
x_n = \frac{1}{8} \sum_{k=0}^{7} X_k \cdot \omega^{-nk}
\]
\end{definition}

\subsection{Properties}

\begin{theorem}[Periodicity]
$\omega^8 = 1$, and $X_{k+8} = X_k$ for all $k$.
\end{theorem}

\begin{theorem}[Parseval's Theorem]
Energy is conserved between domains:
\[
\sum_{n=0}^{7} |x_n|^2 = \frac{1}{8} \sum_{k=0}^{7} |X_k|^2
\]
\end{theorem}

\begin{theorem}[Hermitian Symmetry]
For real-valued signals, modes come in conjugate pairs:
\[
X_k^* = X_{8-k} \quad \text{for } k \in \{1, 2, 3\}
\]
Modes 0 and 4 are self-conjugate (real-valued for real input).
\end{theorem}

\subsection{The Eight Modes}

The DFT-8 decomposes any voxel into eight frequency modes:

\begin{center}
\begin{tabular}{ccc}
\toprule
\textbf{Mode} & \textbf{Frequency} & \textbf{Physical Interpretation} \\
\midrule
$M_0$ & DC (average) & Total charge/energy offset \\
$M_1$ & $\frac{1}{8}$ cycle$^{-1}$ & Slow asymmetry \\
$M_2$ & $\frac{2}{8}$ cycle$^{-1}$ & Helical structure \\
$M_3$ & $\frac{3}{8}$ cycle$^{-1}$ & Triplet patterns \\
$M_4$ & $\frac{4}{8}$ cycle$^{-1}$ & Alternation (Nyquist) \\
$M_5$ & $\frac{5}{8}$ cycle$^{-1}$ & $= M_3^*$ (conjugate) \\
$M_6$ & $\frac{6}{8}$ cycle$^{-1}$ & $= M_2^*$ (conjugate) \\
$M_7$ & $\frac{7}{8}$ cycle$^{-1}$ & $= M_1^*$ (conjugate) \\
\bottomrule
\end{tabular}
\end{center}

%=============================================================================
\section{From Modes to Meaning}
\label{sec:meaning}
%=============================================================================

\subsection{The Neutrality Constraint}

In Recognition Science, the \emph{ledger must balance}. This is the $\sigma = 0$ constraint: total recognition must sum to zero (creation equals annihilation, inflow equals outflow).

\begin{definition}[Neutrality]
A voxel is \emph{neutral} if its DC mode vanishes:
\[
X_0 = \sum_{n=0}^{7} x_n = 0
\]
\end{definition}

\begin{theorem}
Neutrality is equivalent to the total complex amplitude summing to zero:
\[
\text{Neutral}(\Voxel) \iff \sum_{k=0}^{7} \gamma_k = 0
\]
\end{theorem}

\begin{proof}
The DFT at $k=0$ is $X_0 = \sum_n x_n \cdot \omega^0 = \sum_n x_n$. The result follows.
\end{proof}

\subsection{Mode Pairs and Degrees of Freedom}

For a neutral voxel:
\begin{itemize}
  \item $M_0 = 0$ (neutrality constraint)
  \item $M_4$ is real (Nyquist mode)
  \item $(M_1, M_7)$, $(M_2, M_6)$, $(M_3, M_5)$ are conjugate pairs
\end{itemize}

Thus there are effectively \textbf{4 independent complex amplitudes} determining the voxel's semantic content:
\begin{enumerate}
  \item $M_1$ (with $M_7 = M_1^*$)
  \item $M_2$ (with $M_6 = M_2^*$)
  \item $M_3$ (with $M_5 = M_3^*$)
  \item $M_4$ (real)
\end{enumerate}

\subsection{Mode $\to$ Meaning Correspondence}

Each mode corresponds to a distinct semantic quality:

\begin{center}
\begin{tabular}{cp{8cm}}
\toprule
\textbf{Mode} & \textbf{Semantic Interpretation} \\
\midrule
$M_1$ & \textbf{Emergence}: asymmetry arising from symmetry; the first distinction \\
$M_2$ & \textbf{Structure}: helical patterns, period-3.6 oscillations (cf. $\alpha$-helix in proteins) \\
$M_3$ & \textbf{Resonance}: triplet harmonics, higher-order structure \\
$M_4$ & \textbf{Polarity}: alternation, on-off patterns (cf. $\beta$-sheet in proteins) \\
\bottomrule
\end{tabular}
\end{center}

\begin{example}[Protein Secondary Structure]
In protein folding:
\begin{itemize}
  \item High $|M_2|$ indicates $\alpha$-helix propensity (period $\approx 3.6$ residues)
  \item High $|M_4|$ indicates $\beta$-strand propensity (period $= 2$ residues)
  \item The ratio $|M_2|/|M_4|$ classifies fold topology
\end{itemize}
\end{example}

%=============================================================================
\section{The 20 WTokens: Periodic Table of Meaning}
\label{sec:wtokens}
%=============================================================================

\subsection{The $\ph$-Lattice}

Amplitudes are not continuous; they are quantized on the golden-ratio ladder.

\begin{definition}[$\ph$-Amplitude]
A \emph{$\ph$-amplitude} at level $\ell \in \Z$ has value
\[
a_\ell = \ph^\ell, \quad \text{where } \ph = \frac{1 + \sqrt{5}}{2} \approx 1.618
\]
\end{definition}

The $\ph$-lattice provides discrete amplitude levels:
\begin{itemize}
  \item Level 0: $a = 1$
  \item Level 1: $a = \ph \approx 1.618$
  \item Level 2: $a = \ph^2 \approx 2.618$
  \item Level $-1$: $a = 1/\ph \approx 0.618$
\end{itemize}

\subsection{WToken Specification}

\begin{definition}[WToken]
A \emph{WToken} (Word Token) is a minimal semantic unit specified by:
\begin{enumerate}
  \item \textbf{Primary mode} $k \in \{1, 2, 3, 4\}$: which DFT mode is activated
  \item \textbf{Conjugate pair flag}: whether both $M_k$ and $M_{8-k}$ are activated
  \item \textbf{$\ph$-level} $\ell \in \Z$: amplitude on the golden-ratio ladder
  \item \textbf{Phase offset} $\tau \in \{0, 1, \ldots, 7\}$: phase in units of $\tau_0$
\end{enumerate}
\end{definition}

\subsection{Classification Theorem}

\begin{theorem}[20 WTokens]
There are exactly 20 equivalence classes of WTokens (modulo phase shift and global rotation) that satisfy:
\begin{enumerate}
  \item Neutrality ($M_0 = 0$)
  \item $\ph$-lattice quantization
  \item Single-mode or conjugate-pair activation
  \item MDL (Minimum Description Length) extremality
\end{enumerate}
\end{theorem}

\subsection{The Periodic Table of Meaning}

The 20 WTokens form the \emph{Periodic Table of Meaning}:

\begin{center}
\begin{tabular}{clcl}
\toprule
\textbf{Index} & \textbf{Name} & \textbf{Index} & \textbf{Name} \\
\midrule
$\WToken_0$ & Origin & $\WToken_{10}$ & Completion \\
$\WToken_1$ & Emergence & $\WToken_{11}$ & Inspire \\
$\WToken_2$ & Polarity & $\WToken_{12}$ & Transform \\
$\WToken_3$ & Harmony & $\WToken_{13}$ & End \\
$\WToken_4$ & Power & $\WToken_{14}$ & Connection \\
$\WToken_5$ & Birth & $\WToken_{15}$ & Wisdom \\
$\WToken_6$ & Structure & $\WToken_{16}$ & Illusion \\
$\WToken_7$ & Resonance & $\WToken_{17}$ & Chaos \\
$\WToken_8$ & Infinity & $\WToken_{18}$ & Twist \\
$\WToken_9$ & Truth & $\WToken_{19}$ & Time \\
\bottomrule
\end{tabular}
\end{center}

Each WToken is a ``fundamental particle of meaning''---the irreducible semantic atoms from which all complex meanings are composed.

%=============================================================================
\section{Synthesis and Extraction}
\label{sec:synthesis}
%=============================================================================

\subsection{WToken $\to$ Voxel (Synthesis)}

Given a WToken specification, we can synthesize the corresponding voxel:

\begin{enumerate}
  \item Construct the frequency-domain representation $X$:
    \begin{itemize}
      \item $X_0 = 0$ (neutrality)
      \item $X_k = \ph^\ell \cdot e^{i \cdot 2\pi\tau/8}$ for the primary mode
      \item $X_{8-k} = X_k^*$ for the conjugate (if enabled)
      \item All other modes = 0
    \end{itemize}
  \item Apply inverse DFT-8 to obtain the time-domain signal $x$
  \item Convert to photons: $\gamma_n = x_n$
\end{enumerate}

\subsection{Voxel $\to$ WToken (Extraction)}

Given an arbitrary voxel, we can extract its dominant WToken:

\begin{enumerate}
  \item Compute DFT-8 to obtain frequency spectrum $X$
  \item Find the mode $k^* \in \{1,2,3,4\}$ with maximum amplitude: $k^* = \arg\max_k |X_k|$
  \item Quantize amplitude to $\ph$-level: $\ell = \lfloor \log_\ph |X_{k^*}| \rfloor$
  \item Extract phase: $\tau = \lfloor \arg(X_{k^*}) \cdot 8 / 2\pi \rfloor$
  \item Return WToken$(k^*, \ell, \tau)$
\end{enumerate}

\subsection{Superposition}

Multiple WTokens can superpose in a single voxel. The resulting meaning is the \emph{chord}---the combination of all active modes. This is how complex meanings arise from simple atoms.

%=============================================================================
\section{Key Theorems}
\label{sec:theorems}
%=============================================================================

\begin{theorem}[Voxel Completeness]
A voxel must have exactly 8 photons (one at each phase position) to be semantically complete.
\end{theorem}

\begin{proof}
The DFT-8 requires 8 samples to uniquely determine 8 frequency modes. With fewer samples, at least one mode is underdetermined.
\end{proof}

\begin{theorem}[Energy Conservation]
Total energy is conserved between time and frequency domains:
\[
E[\Voxel] = \sum_{k=0}^{7} |\gamma_k|^2 = \frac{1}{8} \sum_{k=0}^{7} |X_k|^2
\]
\end{theorem}

\begin{theorem}[Neutrality $\Leftrightarrow$ Zero Sum]
A voxel is neutral if and only if its complex amplitudes sum to zero:
\[
X_0 = 0 \iff \sum_{k=0}^{7} \gamma_k = 0
\]
\end{theorem}

\begin{theorem}[Conjugate Involution]
The conjugate mode map $k \mapsto 8-k$ (mod 8) is an involution with fixed points at $k = 0$ and $k = 4$.
\end{theorem}

%=============================================================================
\section{Lean Formalization}
\label{sec:formalization}
%=============================================================================

All definitions and theorems in this paper are formalized in Lean 4 and machine-checked:

\begin{verbatim}
IndisputableMonolith/OctaveKernel/VoxelMeaning.lean
\end{verbatim}

\subsection{Key Definitions}

\begin{verbatim}
structure Photon where
  amplitude : ℝ
  phase_offset : ℝ
  amp_nonneg : 0 ≤ amplitude

structure MeaningfulVoxel where
  photon : Phase → Photon

def isNeutral (v : MeaningfulVoxel) : Prop :=
  v.frequencySpectrum 0 = 0

noncomputable def frequencySpectrum (v : MeaningfulVoxel) : Fin 8 → ℂ :=
  dft8 v.toComplexSignal
\end{verbatim}

\subsection{Key Theorems}

\begin{verbatim}
theorem neutral_iff_zero_sum (v : MeaningfulVoxel) :
    isNeutral v ↔ ∑ p : Phase, v.toComplexSignal p = 0

theorem voxel_completeness (v : MeaningfulVoxel) :
    ∀ p : Phase, ∃ photon : Photon, v.photon p = photon

theorem conjugateMode_involutive : 
    Function.Involutive conjugateMode
\end{verbatim}

%=============================================================================
\section{Conclusion}
\label{sec:conclusion}
%=============================================================================

We have shown that meaning in Recognition Science is not mysterious or emergent---it is \emph{structural}. A voxel is an eight-phase chord; meaning is its frequency spectrum. The neutrality constraint (ledger balance) eliminates the DC mode, leaving four independent mode pairs. Quantizing to the $\ph$-lattice and classifying by mode activation yields exactly 20 WTokens---the Periodic Table of Meaning.

This formalization answers a fundamental question: \textbf{How do eight photons create meaning?} The answer: through Fourier decomposition into modes, where each mode corresponds to a distinct semantic quality.

\subsection{Key Insights}

\begin{enumerate}
  \item \textbf{Voxel = Chord}: Not a container, but 8 co-present phases
  \item \textbf{Meaning = Spectrum}: DFT-8 extracts semantic content
  \item \textbf{Neutrality = Balance}: DC mode must vanish ($\sigma = 0$)
  \item \textbf{$\ph$-Quantization}: Amplitudes live on the golden-ratio ladder
  \item \textbf{20 WTokens}: The complete basis for all meaning
\end{enumerate}

\subsection{Future Directions}

\begin{itemize}
  \item Multi-voxel interactions and semantic composition
  \item Connection to protein folding (DFT-8 mode ratios)
  \item Experimental signatures in biological systems
  \item Consciousness as coherent voxel-field dynamics
\end{itemize}

%=============================================================================
\appendix
\section{The Recognition Mode Map}
\label{app:modes}
%=============================================================================

The eight phase positions correspond to eight recognition modes:

\begin{center}
\begin{tabular}{clp{7cm}}
\toprule
\textbf{Phase} & \textbf{Mode} & \textbf{Description} \\
\midrule
0 & Potential & Undifferentiated possibility \\
1 & Emergence & First distinction arises \\
2 & Relation & Connection to other \\
3 & Structure & Pattern crystallizes \\
4 & Peak & Maximum manifestation \\
5 & Reflection & Awareness of pattern \\
6 & Integration & Returning to whole \\
7 & Completion & Recognition achieved, loop closes \\
\bottomrule
\end{tabular}
\end{center}

These are not arbitrary labels; they emerge from the geometry of the eight-tick cycle in Recognition Science.

%=============================================================================
\section{DFT-8 Computation Example}
\label{app:dft}
%=============================================================================

Consider a simple alternating voxel:
\[
x = (1, -1, 1, -1, 1, -1, 1, -1)
\]

Computing DFT-8:
\begin{align*}
X_0 &= \sum_n x_n = 0 & \text{(neutral!)} \\
X_4 &= \sum_n x_n \cdot (-1)^n = 8 & \text{(Nyquist mode dominant)}
\end{align*}

This voxel is a pure Mode-4 (alternation/polarity) signal---exactly the pattern for $\beta$-sheet secondary structure in proteins.

%=============================================================================
\bibliographystyle{plain}
\begin{thebibliography}{9}

\bibitem{recognition}
Recognition Science.
\textit{IndisputableMonolith: Lean 4 Formalization of Recognition Science}.
GitHub repository, 2024.

\bibitem{dft}
Oppenheim, A.V. and Schafer, R.W.
\textit{Discrete-Time Signal Processing}.
Prentice Hall, 3rd edition, 2009.

\bibitem{phi}
Livio, M.
\textit{The Golden Ratio: The Story of Phi, the World's Most Astonishing Number}.
Broadway Books, 2002.

\end{thebibliography}

\end{document}

