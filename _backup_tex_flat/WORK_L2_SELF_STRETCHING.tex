\documentclass{article}
\usepackage{amsmath,amssymb,amsthm,geometry}
\geometry{margin=1in}

\title{Explicit self-stretching sign computation for a concrete $\ell=2$ toroidal vorticity profile}
\author{(under audit)}
\date{\today}

\begin{document}
\maketitle

\section{Scope and objective}
This note performs an \emph{explicit} computation of the enstrophy-production / self-stretching functional
\[
I[\omega]\;:=\;\int_{\mathbb{R}^3} (\omega\cdot\nabla u)\cdot\omega \,dx,
\qquad u=\curl(-\Delta)^{-1}\omega,
\]
for a concrete axisymmetric $\ell=2$ \emph{toroidal} vorticity mode (azimuthal vorticity, poloidal velocity).

\medskip
\noindent\textbf{Important:} This computation \emph{does not} by itself prove any global-regularity statement.  Its role is to pin down the \emph{sign} of $I$ for one canonical $\ell=2$ profile as an input to the broader ``Symmetry Attack'' investigation.

\section{A concrete $\ell=2$ toroidal vorticity ansatz}
Work in spherical coordinates $(r,\theta,\phi)$ and consider the axisymmetric azimuthal vorticity field
\begin{equation}\label{eq:omega-ansatz}
\omega(r,\theta)\;=\;\omega_\phi(r,\theta)\,\hat\phi,
\qquad
\omega_\phi(r,\theta)\;=\;3\,f(r)\,\sin(2\theta),
\end{equation}
where $f:[0,\infty)\to\mathbb R$ is a radial amplitude (assume $f$ is smooth and compactly supported to avoid all integrability issues).
This is the $\ell=2$ toroidal vector spherical harmonic associated with $Y_{2,0}$.

\section{Recovering the induced velocity}
Introduce a (poloidal) vector potential of the form $A=\psi(r,\theta)\,\hat\phi$ and define
\[
u:=\curl(\psi\,\hat\phi).
\]
For axisymmetric $\psi$ one has the standard formulas
\begin{equation}\label{eq:u-from-psi}
u_r=\frac{1}{r\sin\theta}\,\partial_\theta(\psi\sin\theta),
\qquad
u_\theta=-\frac1r\,\partial_r(r\psi),
\qquad
u_\phi=0.
\end{equation}
Now make the $\ell=2$ separation ansatz
\begin{equation}\label{eq:psi-ansatz}
\psi(r,\theta)=g(r)\,\sin(2\theta).
\end{equation}
Then \eqref{eq:u-from-psi} gives
\begin{equation}\label{eq:u-components}
u_r(r,\theta)=\frac{2g(r)}{r}\,(3\cos^2\theta-1),
\qquad
u_\theta(r,\theta)=-\frac{(rg(r))'}{r}\,\sin(2\theta).
\end{equation}
A direct curl computation for axisymmetric $(u_r,u_\theta,0)$ yields
\[
(\curl u)_\phi=\frac1r\bigl(\partial_r(r u_\theta)-\partial_\theta u_r\bigr)
\;=\;\Bigl(-g''-\frac{2}{r}g'+\frac{6}{r^2}g\Bigr)\,\sin(2\theta).
\]
Matching $(\curl u)_\phi=\omega_\phi=3f(r)\sin(2\theta)$ shows that $g$ must solve the radial ODE
\begin{equation}\label{eq:g-ode}
g''(r)+\frac{2}{r}g'(r)-\frac{6}{r^2}g(r)=-3\,f(r),
\end{equation}
with the physical boundary conditions ``regular at $0$'' and ``decaying at $\infty$''.

\section{Computing $(\omega\cdot\nabla)u$ and reducing $I$}
Since $\omega=\omega_\phi\hat\phi$, we have
\[
\omega\cdot\nabla=\frac{\omega_\phi}{r\sin\theta}\,\partial_\phi.
\]
Even though $u$ is axisymmetric (no $\phi$ dependence in components), $\partial_\phi u\neq 0$ because the basis vectors rotate:
\[
\partial_\phi \hat r=\sin\theta\,\hat\phi,
\qquad
\partial_\phi \hat\theta=\cos\theta\,\hat\phi.
\]
Therefore
\[
(\omega\cdot\nabla)u
=\frac{\omega_\phi}{r\sin\theta}\,\partial_\phi(u_r\hat r+u_\theta\hat\theta)
=\frac{\omega_\phi}{r}\bigl(u_r+u_\theta\cot\theta\bigr)\,\hat\phi,
\]
and hence
\[
(\omega\cdot\nabla u)\cdot\omega
=\frac{\omega_\phi^2}{r}\bigl(u_r+u_\theta\cot\theta\bigr).
\]
Substituting \eqref{eq:u-components} and $\omega_\phi=3f\sin(2\theta)$ yields a separable angular integral.  Writing $u=\cos\theta$ and using
\[
\int_{-1}^1 u^2(1-u^2)\,du=\frac{4}{15},
\qquad
\int_{-1}^1 u^4(1-u^2)\,du=\frac{4}{35},
\]
one obtains the exact reduction
\begin{equation}\label{eq:I-reduction}
I[\omega]
\;=\;-\frac{192\pi}{35}\int_0^\infty f(r)^2\bigl(g(r)+3r g'(r)\bigr)\,dr,
\end{equation}
where $g$ is the unique solution of \eqref{eq:g-ode} with the regular/decay boundary conditions.

\medskip
\noindent\textbf{Correction vs. earlier scratch:} note that $g(r)+3r g'(r)$ is \emph{not} equal to $r^{-2}\frac{d}{dr}(r^3 g(r))$ (the latter equals $3g+rg'$).

\section{A fully explicit concrete profile with $I>0$}
Take the toy shell profile
\begin{equation}\label{eq:f-shell}
f(r)=\mathbf 1_{[1,2]}(r),
\end{equation}
and interpret \eqref{eq:omega-ansatz} in a weak sense (or replace $\mathbf 1_{[1,2]}$ by a smooth nonnegative cutoff supported near $[1,2]$; the sign is stable under small perturbations).

On the interval $r\in[1,2]$ we have
\[
\int_0^r s^4 f(s)\,ds=\int_1^r s^4\,ds=\frac{r^5-1}{5},
\qquad
\int_r^\infty \frac{f(s)}{s}\,ds=\int_r^2 \frac{ds}{s}=\log\frac{2}{r}.
\]
Solving \eqref{eq:g-ode} with the standard $\ell=2$ Green function gives, for $r\in[1,2]$,
\[
g(r)=-\frac{3}{5}\Bigl(r^{-3}\!\int_0^r s^4 f(s)\,ds+r^2\!\int_r^\infty \frac{f(s)}{s}\,ds\Bigr)
=-\frac{3}{25}\Bigl(r^2-r^{-3}+5r^2\log\frac{2}{r}\Bigr).
\]
Differentiating yields
\[
g(r)+3r g'(r)
=\frac{24}{25}\bigl(r^2-r^{-3}\bigr)-\frac{21}{5}r^2\log\frac{2}{r}.
\]
Thus the integral in \eqref{eq:I-reduction} reduces to an elementary computation on $[1,2]$:
\[
\int_1^2 \bigl(g(r)+3r g'(r)\bigr)\,dr
=\frac{105\log 2-104}{75}\;<\;0,
\]
and therefore
\begin{equation}\label{eq:I-positive-shell}
I[\omega]
=\frac{192\pi}{35}\cdot\frac{104-105\log 2}{75}
=\frac{64\pi}{875}\,\bigl(104-105\log 2\bigr)\;>\;0.
\end{equation}

\section{Remark on the poloidal $\ell=2$ vorticity case}
For an axisymmetric \emph{poloidal} vorticity mode (with $\omega$ in the $(r,\theta)$ plane and induced $u$ purely azimuthal), one has $(\omega\cdot\nabla)u$ purely azimuthal while $\omega$ is poloidal, hence $(\omega\cdot\nabla u)\cdot\omega\equiv 0$ pointwise. This does \emph{not} address non-axisymmetric modes or mixed configurations.

\section{What remains open for the larger program}
\begin{itemize}
\item The computation above is \emph{one mode}. Extending sign information to the full $\ell=2$ subspace (and identifying which components control the RM2 obstruction) remains open.
\item Even a positive sign for $I$ does not automatically imply backward-time decay of the relevant tail moment for bounded ancient solutions; turning \eqref{eq:I-reduction} into a usable spectral/dynamical inequality is a separate step.
\end{itemize}

\end{document}

