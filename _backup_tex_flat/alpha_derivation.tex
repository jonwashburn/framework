\documentclass[aps,preprint,12pt]{revtex4-2}

% ============================
% Packages (keep minimal/robust)
% ============================
\usepackage[T1]{fontenc}
\usepackage[utf8]{inputenc}
\usepackage{amsmath,amssymb,mathtools}
\usepackage{bm}
\usepackage{booktabs}
\usepackage{hyperref}
\usepackage{xcolor}
\usepackage{listings}

\hypersetup{
  colorlinks=true,
  linkcolor=blue,
  citecolor=blue,
  urlcolor=blue
}

% ============================
% Convenience macros
% ============================
% Make `\Lean{...}` safe in both text and math mode.
\newcommand{\Lean}[1]{\ifmmode\text{\texttt{#1}}\else\texttt{#1}\fi}
\newcommand{\RS}{\textsc{Recognition Science}}
\newcommand{\phiG}{\phi}
\newcommand{\alphainv}{\alpha^{-1}}

\lstset{
  basicstyle=\ttfamily\footnotesize,
  breaklines=true,
  frame=single,
  columns=fullflexible,
  keepspaces=true
}

\begin{document}

\title{A Lean-Referenced Derivation of the Electromagnetic Fine-Structure Constant from Recognition Ledger Geometry}

\author{Recognition Science Collaboration}
\affiliation{Recognition Physics Institute}

\date{\today}

\begin{abstract}
This paper documents the \RS{} derivation of the electromagnetic fine-structure constant in the exact form implemented in this repository's Lean~4 development.
The derived inverse constant is:
\[
\alpha^{-1} = 4\pi\cdot 11 \;-\; f_{\mathrm{gap}} \;-\; \delta_{\kappa},
\quad\text{where}\quad
\delta_{\kappa} = -\frac{103}{102\,\pi^5}.
\]
We explain each factor: \emph{what} enters this formula, \emph{why} it is structurally forced by the ledger/cube interface, and \emph{where} each component lives in the Lean codebase.
Integer/combinatorial components (11, 102, 103), the solid-angle factor ($4\pi$), and the $\pi^5$ power are proved in Lean; the gap weight $w_8$ is currently an external certificate, with a DFT-8 derivation in progress.
\end{abstract}

\maketitle

%==============================================================================
\section{What ``alpha'' means in this repo}
%==============================================================================

This repository contains \emph{two} distinct ``alpha'' quantities:
\begin{itemize}
  \item \textbf{Electromagnetic fine-structure constant:} $\alpha_{\mathrm{EM}} \approx 1/137$, implemented as \texttt{alpha} $:= 1/$\texttt{alphaInv} in \texttt{Constants.Alpha}, with $\alpha^{-1}\approx 137.036$.
  This is the subject of this paper.

  \item \textbf{Locked kernel exponent:} $\alpha_{\mathrm{lock}} := (1-1/\phi)/2 \approx 0.191$, implemented as \texttt{alphaLock} and used as an exponent in the ILG kernel.
  It is \emph{not} equal to $\alpha_{\mathrm{EM}}$.
\end{itemize}

Throughout, ``\(\alpha\)'' or ``fine-structure constant'' refers to \(\alpha_{\mathrm{EM}}\).

%==============================================================================
\section{The Lean top-level definition}
%==============================================================================

The \emph{symbolic} definition of $\alpha_{\mathrm{EM}}$ is in \texttt{Constants/Alpha.lean}.
The core definitions are:
\begin{align*}
\texttt{alpha\_seed} &:= 4\cdot\pi\cdot 11,\\
\texttt{delta\_kappa} &:= -103/(102\cdot\pi^5),\\
\texttt{alphaInv} &:= \texttt{alpha\_seed} - (\texttt{f\_gap} + \texttt{delta\_kappa}),\\
\texttt{alpha} &:= 1/\texttt{alphaInv}.
\end{align*}

For reference, here is the exact Lean snippet (abridged):
\begin{lstlisting}
-- IndisputableMonolith/Constants/Alpha.lean
@[simp] def alpha_seed : Real := 4 * Real.pi * 11
@[simp] def delta_kappa : Real := -(103 : Real) / (102 * Real.pi ^ 5)
@[simp] def alphaInv : Real := alpha_seed - (f_gap + delta_kappa)
@[simp] def alpha : Real := 1 / alphaInv
\end{lstlisting}

Everything else in this paper explains why these specific factors appear.

%==============================================================================
\section{First-principles structure: seed--gap--curvature}
%==============================================================================

In \RS{}, the coupling constant is not introduced as a measured parameter; it is assembled as a \emph{dimensionless closure balance} across the discrete ledger-to-continuum interface:
\[
\alphainv = \underbrace{\text{(geometric seed)}}_{\text{baseline closure}}
\;-\;
\underbrace{\text{(gap cost)}}_{\text{8-tick spectral deficit}}
\;-\;
\underbrace{\text{(curvature correction)}}_{\text{tiling/phase-space mismatch}}.
\]

Lean implements this as:
\[
\alpha^{-1} = \texttt{alpha\_seed} - (\texttt{f\_gap} + \texttt{delta\_kappa}),
\]
where $\texttt{delta\_kappa}<0$. Thus subtracting it adds a small positive curvature correction.

%==============================================================================
\section{Why \texorpdfstring{$D=3$}{D=3} (brief): the Gap45 synchronization forcing}
%==============================================================================

In \RS{}, the cube geometry used in the $\alpha_{\mathrm{EM}}$ pipeline is not an arbitrary modeling choice: $D=3$ is structurally forced by the eight-tick closure and the ``Gap45'' synchronization constraint.
The certificate is \texttt{Verification/Gap45DimensionCert.lean}.

In that development, the main idea is:
\begin{itemize}
  \item The eight-tick period is \(2^D\). For $D=3$, \(2^D=8\).
  \item A closure/fibonacci construction yields a ``gap'' of 45:
  \[
  45 = (8+1)\cdot 5,
  \]
  where $8+1$ is a wrap-around closure factor and $5=\mathrm{fib}(4)$ is the smallest nontrivial Fibonacci factor coprime to 8.
  \item A global synchronization period is imposed:
  \[
    \mathrm{lcm}(2^D,45)=360.
  \]
  Since $\gcd(2^D,45)=1$ (no factor of 2 in 45), this reduces to \(2^D\cdot 45=360\), hence \(2^D=8\) and therefore \(D=3\).
\end{itemize}

For the $\alpha_{\mathrm{EM}}$ derivation, \texttt{AlphaDerivation.lean} sets $D:=3$ directly; the Gap45 certificate is the upstream justification.

%==============================================================================
\section{Why the number 11: passive edges of the \texorpdfstring{$D{=}3$}{D=3} cube}
%==============================================================================

The integer $\mathbf{11}$ is the passive-edge count of the 3-cube $Q_3$.
This is formalized in \texttt{AlphaDerivation.lean} and certified in \texttt{CubeGeometryCert.lean}.

\subsection{Cube combinatorics in Lean}

Lean defines the hypercube edge/face/vertex counts:
\begin{align}
V(D) &= 2^D, \\
E(D) &= D\cdot 2^{D-1}, \\
F(D) &= 2D,
\end{align}
as \texttt{cube\_vertices}, \texttt{cube\_edges}, \texttt{cube\_faces}.
With $D=3$, Lean proves: $V(3)=8$, $E(3)=12$, $F(3)=6$.

\subsection{Active vs.\ passive edges}

The ledger dynamics require one ``active'' edge transition per atomic tick.
Lean encodes:
\[
\texttt{active\_edges\_per\_tick} := 1,\qquad
\texttt{passive\_field\_edges}(D) := E(D) - 1.
\]
For $D=3$, the key theorem is $\texttt{passive\_field\_edges}\ 3 = 11$.

Interpretation: the single active edge realizes the local transition; the remaining 11 edges represent the ``field dressing'' paths around that transition---this is the structural origin of 11 in $\texttt{alpha\_seed}$.

%==============================================================================
\section{Why \texorpdfstring{$4\pi$}{4pi}: isotropic solid-angle closure in \texorpdfstring{$D{=}3$}{D=3}}
%==============================================================================

The factor $4\pi$ appears because the seed is a \emph{spherical closure} over directions in three spatial dimensions.
In Lean, the derivation is in \texttt{Constants/SolidAngleExclusivity.lean}.

\subsection{Sphere surface area in arbitrary $D$}

Lean defines the surface area of the unit \((D-1)\)-sphere embedded in \(\mathbb{R}^D\) by the standard Gamma-function formula:
\[
S_{D-1} = \frac{2\pi^{D/2}}{\Gamma(D/2)}.
\]
This is \texttt{unitSphereSurface}.

\subsection{Specializing to $D=3$}

Lean proves $S_2 = 4\pi$ via \texttt{unitSphereSurface\_D3}.

\subsection{Seed assembly}

With the passive-edge count 11 and isotropic direction measure $4\pi$, Lean defines:
\[
\texttt{alpha\_seed} := 4\pi\cdot 11.
\]
This is the ``baseline'' inverse coupling before spectral gap and curvature corrections.

%==============================================================================
\section{The curvature correction: why \texorpdfstring{$103/102$}{103/102} and why \texorpdfstring{$\pi^5$}{pi^5}}
%==============================================================================

The curvature correction is:
\[
\texttt{delta\_kappa} = -\frac{103}{102\pi^5}.
\]

The integer provenance is proved in \texttt{Constants/AlphaDerivation.lean}; the $\pi^5$ power is explained in \texttt{Constants/CurvatureSpaceDerivation.lean}.

\subsection{Why 102 and 103: voxel seam counting}

Lean defines the denominator as:
\[
\texttt{seam\_denominator}(D) := F(D)\cdot W,
\]
where $W:=17$ is the number of wallpaper groups (plane symmetry groups).
For $D=3$, $F(3)=6$, so $6\cdot 17 = 102$.

Lean then adds an explicit closure increment:
\[
\texttt{euler\_closure}:=1,\quad
\texttt{seam\_numerator}(D):=\texttt{seam\_denominator}(D)+1,
\]
so for $D=3$: $102+1=103$.

\subsection{Why $\pi^5$: 5D configuration-space angular measure}

Lean models the curvature integral as living on an effective configuration space of dimension:
\[
3\ \text{(space)} + 1\ \text{(time/phase)} + 1\ \text{(dual-balance)} = 5,
\]
encoded as $\texttt{configSpaceDim}:=5$.
Each dimension contributes a factor of $\pi$ under the angular normalization, yielding $\pi^5$ in the denominator.

\subsection{Sign of the correction}

Lean encodes the curvature term as negative ($\texttt{delta\_kappa} < 0$).
In the top-level assembly, subtracting a negative value adds a small positive correction.
The seam mismatch is a \emph{defect} term; the sign convention treats it as a negative correction inside the parenthesis.

%==============================================================================
\section{The gap term \texorpdfstring{$f_{\mathrm{gap}} = w_8\ln(\phi)$}{f\_gap = w8 ln(phi)}}
%==============================================================================

The gap term is the information/spectral deficit arising from the eight-tick periodicity and the $\phi$-lattice scaling at the discrete--continuous interface.
In Lean, two layers exist:
\begin{itemize}
  \item \textbf{Pipeline constant:} \texttt{Constants/GapWeight.lean} (currently used).

  \item \textbf{DFT-8 derivation:} \texttt{Constants/GapWeightDerivation.lean}, with positivity certified in \texttt{Verification/GapWeightDerivationCert.lean}.
\end{itemize}

\subsection{Why $\ln(\phi)$}

\RS{} distinguishes multiplicative self-similar scaling (natural on the $\phi$-lattice) from additive bookkeeping cost (natural for ledger balance).
The logarithm is the unique bridge between the two:
\[
\ln(\phi^n) = n\ln(\phi).
\]
Thus a unit ``scale step'' carries additive cost proportional to $\ln\phi$.
Lean uses \texttt{Real.log phi} for this quantity.

\subsection{What $w_8$ is}

In the DFT-8 derivation module, Lean defines the canonical $\phi$-pattern on eight ticks:
\[
p(t) = \phi^t,\qquad t\in\{0,1,\dots,7\},
\]
as \texttt{phiPattern}.
It then defines DFT coefficients $c_k$ and mode energies $|c_k|^2$.
The weight $w_8$ is defined as a weighted sum over the neutral modes $k\neq 0$:
\[
w_8 = \sum_{k=1}^{7} |c_k|^2\,g_k(\phi),
\]
where $g_k(\phi)$ are geometric weights from the 8-tick frequency alignment and $\phi$-scaling.
This is \texttt{w8\_computed}.

\subsection{DFT-8 coefficients and closed form}

The DFT-8 coefficient implemented in \texttt{GapWeightDerivation.lean} is:
\[
c_k \;:=\; \sum_{t\in \mathrm{Fin}\,8} \overline{\mathrm{DFT}_{t,k}}\;\phi^t,
\]
encoded as:
\begin{lstlisting}
-- IndisputableMonolith/Constants/GapWeightDerivation.lean
noncomputable def phiDFTCoeff (k : Fin 8) : Complex :=
  Finset.univ.sum fun t => star (dft8_entry t k) * (phi ^ t.val : Complex)
\end{lstlisting}

Two key proved facts are:
\begin{itemize}
  \item \textbf{Nontriviality:} the mode-1 coefficient is nonzero (\texttt{phiDFTCoeff\_one\_ne\_zero}).
  The proof uses the geometric series identity
  $\sum_{t=0}^{7} r^t = (r^8-1)/(r-1)$ with $r := \overline{\omega_8}\phi$, together with
  $\phi^8 = 21\phi+13$, to show $r^8\neq 1$.

  \item \textbf{Positivity:} $w_8>0$ (\texttt{w8\_computed\_pos}), since every summand $|c_k|^2 g_k(\phi)$ is nonnegative and at least $k{=}1$ is strictly positive.
\end{itemize}

\subsection{The geometric weights $g_k(\phi)$}

The per-mode weight is:
\[
g_k(\phi) =
\begin{cases}
0, & k=0,\\[4pt]
\sin^2\!\bigl(\tfrac{\pi k}{8}\bigr)\,\phi^{-k}, & k\in\{1,\dots,7\}.
\end{cases}
\]

\subsection{Current pipeline status}

The top-level \texttt{alphaInv} currently uses:
\[
\texttt{w8\_from\_eight\_tick} := 2.488254397846,
\quad
\texttt{f\_gap} := \texttt{w8\_from\_eight\_tick}\cdot\ln(\phi),
\]
documented as a quarantined certificate in \texttt{GapWeight.lean}.
The JSON file \texttt{w8.json} records $w_8$, $\phi$, $\ln\phi$, $f_{\mathrm{gap}}$, $\delta_\kappa$, and $\alpha^{-1}$.

Separately, \texttt{GapWeightDerivation.lean} proves positivity of $w_8$ and non-vanishing of the mode-1 coefficient, and records \texttt{w8\_computed\_eq\_abstract} as a named proposition pending final integration.

%==============================================================================
\section{Assembling \texorpdfstring{$\alphainv$}{alpha^-1} and numerical bounds}
%==============================================================================

\subsection{Assembly theorem}

The symbolic assembly is in \texttt{Alpha.lean}, connected to the ingredient derivations by \texttt{alphaInv\_derived\_eq\_formula}.

\subsection{Numeric evaluation (quarantined)}

Numeric checks are quarantined in \texttt{AlphaNumericsScaffold.lean}; the JSON certificate records $\alpha^{-1} \approx 137.036$.

\subsection{Rigorous interval bounds (proved)}

Rigorous inequality bounds are in \texttt{AlphaBounds.lean}.
That module proves:
\[
137.031 < \texttt{alphaInv} < 137.040,
\]
using certified bounds on $\pi$ and coarse bounds on $\ln\phi$.
Tightening the logarithm bounds correspondingly tightens the $\alpha^{-1}$ interval.

%==============================================================================
\section{Lean cross-reference map (where each factor comes from)}
%==============================================================================

\begin{table}[ht]
\centering
\small
\begin{tabular}{@{}lll@{}}
\toprule
\textbf{Factor} & \textbf{Meaning} & \textbf{Lean module} \\
\midrule
$D=3$       & dimension forcing      & \texttt{Verification/Gap45DimensionCert} \\
$11$        & passive edges ($12{-}1$) & \texttt{Constants/AlphaDerivation} \\
$4\pi$      & solid angle in 3D      & \texttt{Constants/SolidAngleExclusivity} \\
$17$        & wallpaper groups       & \texttt{Constants/AlphaDerivation} \\
$102$       & $6\times 17$ (seam denom.) & \texttt{Constants/AlphaDerivation} \\
$103$       & $102+1$ (Euler closure)   & \texttt{Constants/AlphaDerivation} \\
$\pi^5$     & 5D config.\ space      & \texttt{Constants/CurvatureSpaceDerivation} \\
$w_8$       & 8-tick gap weight      & \texttt{Constants/GapWeight[Derivation]} \\
$\ln\phi$   & additive scale cost    & Mathlib \texttt{Real.log} \\
\bottomrule
\end{tabular}
\caption{Provenance map for each element of the $\alpha^{-1}$ derivation.}
\end{table}

%==============================================================================
\section{Conclusion}
%==============================================================================

The \RS{} fine-structure pipeline implemented in Lean decomposes \(\alpha_{\mathrm{EM}}^{-1}\) into:
\begin{itemize}
  \item a geometric seed \(4\pi\cdot 11\) forced by $D{=}3$ cube combinatorics and isotropy,
  \item an eight-tick gap cost \(f_{\mathrm{gap}} = w_8\ln(\phi)\) capturing the neutral spectral deficit of the $\phi$-pattern under DFT-8,
  \item and a curvature/tiling correction \(\delta_\kappa = -103/(102\pi^5)\) from seam topology over a 5D effective configuration space.
\end{itemize}

Lean makes the dependency structure explicit and (for most components) machine-checkable.
The remaining open step is to replace the pipeline constant \texttt{w8\_from\_eight\_tick} with the derived DFT-8 expression \texttt{w8\_computed} (or prove their equality), closing the last derivation gap inside the certified surface.

%==============================================================================
\begin{thebibliography}{99}

\bibitem{Lean}
Lean~4 theorem prover, \url{https://leanprover.github.io/}.

\bibitem{Mathlib}
Mathlib: the Lean mathematical library, \url{https://github.com/leanprover-community/mathlib}.

\bibitem{Fedorov1891}
E.~S.~Fedorov, ``Symmetry of regular systems of figures,'' \emph{Zapiski Imp.\ S.-Peterburgskogo Mineral.\ Obshchestva} \textbf{28}, 1--146 (1891).

\bibitem{Conway}
J.~H.~Conway, H.~Burgiel, C.~Goodman-Strauss, \emph{The Symmetries of Things}, A K Peters (2008).

\bibitem{Repo}
Lean modules referenced: \texttt{Constants/\{Alpha, AlphaDerivation, SolidAngleExclusivity, CurvatureSpaceDerivation, GapWeight, GapWeightDerivation\}.lean}, \texttt{Numerics/Interval/AlphaBounds.lean}, and \texttt{data/certificates/w8.json}.

\end{thebibliography}

\end{document}


