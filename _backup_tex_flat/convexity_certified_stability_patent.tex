\documentclass[12pt]{article}
\usepackage[margin=1in]{geometry}
\usepackage{amsmath,amssymb,amsthm}
\usepackage{graphicx}
\usepackage{enumitem}
\usepackage{array}
\usepackage{hyperref}

% Simple page style
\pagestyle{plain}

\newtheorem{theorem}{Theorem}
\newtheorem{lemma}[theorem]{Lemma}
\newtheorem{definition}{Definition}
\newtheorem{corollary}[theorem]{Corollary}

\begin{document}

\begin{center}
\textbf{\LARGE PATENT APPLICATION}\\[0.5cm]
\textbf{\Large Method and System for Certifying Stability in Control Systems\\Using a Strictly Convex Ledger Functional}\\[1cm]

\begin{tabular}{rl}
\textbf{Application Type:} & Utility Patent \\
\textbf{Filing Date:} & January 25, 2026 \\
\textbf{Inventor:} & Jonathan Washburn \\
\textbf{Technology Field:} & Control Theory / Formal Verification / Safety-Critical Systems \\
\textbf{International Class:} & G05B 13/02; G06F 17/11; H04L 9/00 \\
\end{tabular}
\end{center}

\vspace{1cm}
\hrule
\vspace{0.5cm}

\section*{ABSTRACT}

A method and system for generating an auditable convexity certificate for a ratio-space ledger objective used in control systems. The system computes a weighted ledger \(L=\sum_i w_i\,J(r_i)\) on positive ratios \(r_i>0\), where \(J(x)=\frac12(x+x^{-1})-1\) (J-cost). Lean formalizes that \(J\) is symmetric and strictly convex on \(x>0\); the implementation computes ledger values and emits a hash-based certificate bundle binding declared ratios/weights and an explicit curvature witness (e.g., \(w_i\,J''(r_i)\) with \(J''(x)=x^{-3}\)). The certificate attests to objective convexity in ratio-space and can be used as a deployment gate; any closed-loop stability guarantee for a specific plant/controller is an explicit seam unless separately validated.

\vspace{0.5cm}
\hrule
\vspace{0.5cm}

\section{BACKGROUND OF THE INVENTION}

\subsection{Technical Field}

This invention relates generally to the design and verification of stable control systems, and specifically to the use of strictly convex objectives to support convergence analysis and auditability in high-dimensional, safety-critical environments under explicit assumptions.

\subsection{Description of Related Art}

Stability is the most critical property of any control system. A stable system will return to its setpoint after a disturbance; an unstable one may oscillate or diverge, leading to catastrophic failure.

Standard control theory relies on Lyapunov functions $V(x)$ to prove stability. If $\dot{V}(x) < 0$ along trajectories, the system is stable. However, constructing a valid Lyapunov function for complex, non-linear systems is difficult. Engineers often default to simple quadratic costs (Mean Squared Error, MSE), i.e., $V(x) = (x-1)^2$.

While MSE is convex locally, it lacks symmetry in ratio space (an error of $0.5$ is not treated equivalently to an error of $2.0$, despite being the multiplicative inverse). In some applications (including fusion diagnostics and other multiplicative processes), engineers prefer ratio-symmetric objectives; any physical ``law'' interpretation is conceptual background and not asserted as a facility-independent fact.

There is a need for a cost function that is strictly convex on a declared domain (ensuring a unique minimizer) and symmetric for ratio-based errors, providing a rigorous basis for auditability and for stability analysis under explicit plant/controller assumptions.

\section{SUMMARY OF THE INVENTION}

The present invention provides a \textbf{Convexity-Certified Stability System} based on the J-cost functional.

The core innovation is the application of the function:
\[
J(x) = \frac{1}{2}\left(x + \frac{1}{x}\right) - 1
\]
as the fundamental control objective. This function has properties proven in the accompanying formal verification (Lean 4):
\begin{enumerate}
    \item \textbf{Strict Convexity (ratio-space):} $J''(x) = x^{-3} > 0$ for all $x > 0$. This implies \(J\) has a unique minimizer at \(x=1\) on \(x>0\) and no spurious local minima in ratio-space.
    \item \textbf{Symmetry:} $J(x) = J(1/x)$. This ensures that a parameter being ``half'' its target is penalized exactly as much as being ``double'' its target, aligning with multiplicative physical processes.
\end{enumerate}

The system uses this functional to:
\begin{itemize}
    \item Aggregate multi-dimensional errors into a single scalar ``Ledger'' $L = \sum w_i J(r_i)$.
    \item Verify objective convexity in ratio-space by checking declared domain conditions (\(r_i>0\), \(w_i>0\)) and emitting a curvature witness (e.g., diagonal entries \(w_i\,J''(r_i)\)).
    \item Issue a hash-based ``Convexity Certificate'' (certificate bundle) that binds inputs and outputs for audit/replay; external signing is an optional integration seam.
\end{itemize}

\section{BRIEF DESCRIPTION OF THE DRAWINGS}

\begin{itemize}
    \item \textbf{FIG. 1} compares the J-cost function with standard quadratic loss, showing the divergence of J-cost at zero.
    \item \textbf{FIG. 2} illustrates the unique global minimum of the aggregate Ledger surface.
    \item \textbf{FIG. 3} is a block diagram of the stability certification pipeline.
\end{itemize}

\section{DETAILED DESCRIPTION OF EMBODIMENTS}

\subsection{Definitions}

\begin{itemize}
    \item \textbf{State Ratio ($x$):} A dimensionless variable representing the ratio of a measured value to its target setpoint (\(x := y/y^*\)); requires \(y^*\neq 0\) (integration seam).
    \item \textbf{J-Cost ($J(x)$):} The cost function defined as $\frac{1}{2}(x + x^{-1}) - 1$ with domain \(x>0\).
    \item \textbf{Ledger ($L$):} a weighted sum \(L=\sum_i w_i\,J(r_i)\) with weights \(w_i\ge 0\) and ratios \(r_i>0\).
    \item \textbf{Lyapunov candidate (seam-labeled)}: using \(L\) as a Lyapunov function for a particular plant/controller is a separate, explicit assumption; the certificate in this disclosure attests only to ratio-space convexity of the objective.
    \item \textbf{Strict Convexity:} A property of a function $f$ where $f'' > 0$, ensuring that any line segment connecting two points on the graph lies strictly above the graph.
    \item \textbf{Convexity Certificate (ratio-space):} A digital artifact (certificate bundle) attesting that the ledger objective is strictly convex in ratio coordinates under declared ratios/weights (does not by itself certify closed-loop plant stability).
    \item \textbf{Certificate bundle}: a hash-based auditable record including an input hash, outputs, and theorem identifier references; external signing optional seam.
\end{itemize}

\subsection{Theoretical Foundation}

The invention relies on the theorem (formally proven in \texttt{IndisputableMonolith/Cost/Convexity.lean}) that $J(x)$ is strictly convex on $(0, \infty)$.
\[
\frac{d^2}{dx^2} J(x) = \frac{d}{dx} \left( \frac{1 - x^{-2}}{2} \right) = x^{-3}
\]
Since $x^{-3}$ is strictly positive for positive $x$, the curvature is always upward. This precludes the existence of ``traps'' (local minima) where a controller might get stuck away from the target.

\subsection{System Architecture}

The control loop comprises:
\begin{enumerate}
    \item \textbf{Ratio Computer:} Converts raw sensor data $y_i$ and targets $y_i^*$ into ratios $r_i = y_i / y_i^*$, requiring \(y_i^*\neq 0\). (Integration seam)
    \item \textbf{Ledger Engine:} Computes the aggregate cost $L = \sum w_i J(r_i)$.
    \item \textbf{Convexity Monitor (ratio-space):} Computes curvature witnesses for the ledger with respect to ratio coordinates \(r_i\). For \(L=\sum_i w_i\,J(r_i)\), the ratio-space Hessian is diagonal with entries \(w_i\,J''(r_i)=w_i/r_i^3\) (for \(r_i>0\)).
    \item \textbf{Certifier:} Checks declared domain conditions (e.g., \(r_i>0\), \(w_i>0\)) and emits a hash-based certificate bundle including the computed ledger value and curvature witnesses. Any actuation gating/fallback behavior is an integration seam.
\end{enumerate}

\subsection{Application to Fusion}

In fusion, ratio variables may represent normalized mode amplitudes or other diagnostic-derived ratios. J-cost penalizes both ``overshoot'' and ``undershoot'' symmetrically in ratio-space and diverges as \(x\to 0^+\). Any mapping from this model-layer objective to physical safety or yield is an explicit facility seam.

\subsection{Seams}

The mathematical convexity of $J(x)$ on \(x>0\) is Lean-proved. The primary \textbf{seam} is the mapping from physical actuators to ratios (the plant model) and the controller update law. In a nonlinear plant, the composite objective \(L(r(u))\) need not be convex in control-input space without additional assumptions; any such plant-level convexity/stability claim is treated as an explicit seam unless separately validated.

\section{CLAIMS}

\begin{enumerate}
    \item \textbf{A computer-implemented method for generating a ratio-space convexity certificate for a control objective, comprising:}
    \begin{enumerate}
        \item mapping a plurality of system state variables to dimensionless ratios relative to a target operating point;
        \item defining a Lyapunov candidate function as a weighted sum of convex costs, wherein each cost is defined by the function $J(x) = \frac{1}{2}(x + x^{-1}) - 1$;
        \item calculating, for each ratio coordinate, a curvature witness value \(w_i\,J''(r_i)\) with \(J''(x)=x^{-3}\);
        \item verifying declared domain conditions including \(r_i>0\) and \(w_i>0\), and verifying \(w_i\,J''(r_i)>0\); and
        \item generating a hash-based certificate bundle binding the declared inputs and the computed outputs, including theorem identifier references.
    \end{enumerate}

    \item The method of claim 1, wherein the cost function satisfies symmetry \(J(x)=J(1/x)\) on \(x>0\) and has a unique minimizer at \(x=1\).

    \item The method of claim 1, further comprising using the ledger objective in a controller that computes a control action intended to decrease the ledger; any closed-loop stability guarantee is an explicit seam unless separately validated for a given plant and update law.

    \item The method of claim 1, wherein the digital certificate includes a reference to a machine-checked proof of the strict convexity of $J(x)$.

    \item \textbf{A control system comprising:}
    \begin{enumerate}
        \item a state estimator configured to output normalized error ratios;
        \item an objective function generator configured to construct a cost surface using the function $J(x) = \frac{1}{2}(x + x^{-1}) - 1$;
        \item a convexity monitor configured to evaluate ratio-space curvature witnesses \(w_i\,J''(r_i)\) at a current operating point; and
        \item a certificate generator configured to emit a hash-based certificate bundle recording inputs, outputs, and theorem identifier references.
    \end{enumerate}

    \item The system of claim 5, wherein the objective function generator aggregates multiple error ratios into a single scalar ledger value.

    \item \textbf{A non-transitory computer-readable medium storing instructions that, when executed by a processor, cause a system to:}
    \begin{enumerate}
        \item compute a control error metric using a symmetric convex functional;
        \item verify that the second derivative of the metric is positive;
        \item determine a control action intended to reduce the metric (e.g., a descent step under a declared update rule); and
        \item log a certificate bundle containing an input hash, computed outputs, and theorem identifier references associated with the control action.
    \end{enumerate}
\end{enumerate}

\section*{APPENDIX: Implementation Evidence}

The core logic of this invention is implemented in the accompanying software artifacts:
\begin{itemize}
    \item \textbf{Python Implementation:} \texttt{fusion/simulator/foundations/jcost.py} implements the J-cost function and ratio-space curvature witness \texttt{Jcost.second\_derivative}.
    \item \textbf{Certificate emission (hash-based):} \texttt{fusion/simulator/fusion/certificate.py} implements \texttt{generate\_jcost\_convexity\_certificate} and \texttt{CertificateBundle}/\texttt{compute\_input\_hash}.
    \item \textbf{Formal Verification:} The strict convexity of J is proven in Lean 4 in \texttt{IndisputableMonolith/Cost/Convexity.lean}, specifically \texttt{Jcost\_strictConvexOn\_pos}.
\end{itemize}

\end{document}
