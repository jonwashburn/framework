\documentclass[12pt]{article}

% Front matter only; no extra packages introduced.
\usepackage{amsmath,amssymb,amsthm}

% Define blackboard bold letters
\newcommand{\CC}{\mathbb{C}}
\newcommand{\RR}{\mathbb{R}}
\newcommand{\QQ}{\mathbb{Q}}
\newcommand{\ZZ}{\mathbb{Z}}

% Theorem environments
\newtheorem{theorem}{Theorem}[section]
\newtheorem{lemma}[theorem]{Lemma}
\newtheorem{proposition}[theorem]{Proposition}
\newtheorem{corollary}[theorem]{Corollary}
\theoremstyle{remark}
\newtheorem{remark}[theorem]{Remark}

\begin{document}

\begin{titlepage}
  \centering

  \vspace*{1.5cm}

  % (title straplines removed to avoid duplicate title)

  \vspace{1.2cm}
  \rule{\textwidth}{0.6pt}
  \vspace{0.8cm}

  {\LARGE \bfseries Calibration–Coercivity and the Hodge Conjecture\par}
  \vspace{0.3cm}
  {\large \bfseries A Fully Classical, Quantitative Reduction with $c=\tfrac{1}{3}$\par}

  \vspace{0.8cm}
  \rule{\textwidth}{0.6pt}

  \vspace{1.6cm}

  {\large \textsc{Jonathan Washburn}\par}
  \vspace{0.2cm}
  {\normalsize Recognition Science, Recognition Physics Institute\par}
  {\normalsize Austin, Texas, USA\par}
  {\normalsize \texttt{jon@recognitionphysics.org}\par}

  \vfill

  {\normalsize \today\par}
\end{titlepage}

\begin{abstract}
We present a purely classical, quantitative route to the Hodge conclusion for rational $(p,p)$ classes on smooth projective Kähler manifolds. 
Let $E(\alpha)$ denote the Dirichlet energy of a smooth closed $2p$-form $\alpha$ representing a fixed $(p,p)$ cohomology class, and let $\gamma_{\mathrm{harm}}$ be the $\omega$-harmonic representative of that class. 
We define a geometric ``distance-to-calibration'' functional measuring the $L^2$ distance to the convex calibrated cone $\mathcal{K}_p$ associated to the Kähler calibration $\varphi=\omega^p/p!$ and denote it by $\mathrm{Def}_{\mathrm{cone}}(\alpha)$.
We prove the cone-based calibration–coercivity inequality
\[
E(\alpha)-E(\gamma_{\mathrm{harm}})\ \ge\ c\,\mathrm{Def}_{\mathrm{cone}}(\alpha),
\qquad
c=\tfrac{1}{3}.
\]
Consequently, any minimizing sequence of closed representatives in a rational $(p,p)$ class has vanishing calibration defect and converges (in the weak sense of currents) to a positive, calibrated $(p,p)$ current that saturates the Wirtinger bound, hence is the current of integration over a complex analytic $p$-cycle; projectivity upgrades it to an algebraic cycle. 
This yields the Hodge conclusion for the class. 
The argument is intrinsic to Kähler geometry, requires no auxiliary discretization or external scaffolding, and isolates a clean, verifiable inequality with fully explicit constants.
\end{abstract}

\section{Introduction}

\paragraph{Problem.}
Let $X$ be a smooth projective complex variety of complex dimension $n$ with Kähler form $\omega$. 
Fix $1\le p\le n$ and consider a rational Hodge class $\gamma\in H^{2p}(X,\mathbb{Q})\cap H^{p,p}(X)$. 
The Hodge question asks for an algebraic cycle of codimension $p$ whose cohomology class equals $\gamma$.

\paragraph{Route via calibration and energy.}
Set the Kähler calibration $\varphi:=\omega^p/p!$. 
For a smooth closed $2p$-form $\alpha$ representing the class $[\gamma]$, write the Dirichlet energy
\[
E(\alpha):=\int_X \|\alpha\|^2\,d\mathrm{vol}_\omega,
\]
and let $\gamma_{\mathrm{harm}}$ be the $\omega$-harmonic representative of $[\gamma]$. 
We measure the geometric misalignment of $\alpha$ from the calibrated cone by the pointwise distance
\[
\mathrm{dist}_{\mathrm{cal}}(\alpha_x):=\inf_{\lambda\ge 0,\ \xi\in\mathcal{G}_p(x)}\|\alpha_x-\lambda\xi\|,
\]
where $\mathcal{G}_p(x)$ is the compact set of unit, simple $(p,p)$ covectors calibrated by $\varphi_x$. 
Define the global \emph{calibration defect}
\[
\mathrm{Def}_{\mathrm{cal}}(\alpha):=\int_X \mathrm{dist}_{\mathrm{cal}}(\alpha_x)^2\,d\mathrm{vol}_\omega.
\]

\paragraph{Main quantitative theorem (calibration–coercivity, explicit).}
There is a numerical constant
\[
c\;=\;\frac{1}{3}\;\approx\;0.3333
\]
such that for \emph{every} smooth closed $\alpha\in[\gamma]$,
\begin{equation}\label{eq:intro-qcc}
E(\alpha)-E(\gamma_{\mathrm{harm}})\ \ge\ c\ \mathrm{Def}_{\mathrm{cone}}(\alpha).
\end{equation}
This inequality is independent of $X$; the constant $c$ depends only on $(n,p)$.

\paragraph{Consequences for Hodge.}
Given a minimizing sequence $\alpha_k\in[\gamma]$ with $E(\alpha_k)\downarrow E(\gamma_{\mathrm{harm}})$, \eqref{eq:intro-qcc} forces $\mathrm{Def}_{\mathrm{cone}}(\alpha_k)\to 0$. 
Any weak limit is a closed, positive $(p,p)$ current calibrated by $\varphi$, hence the current of integration over a complex analytic $p$-cycle; projectivity upgrades it to an algebraic $p$-cycle representing $\gamma$. 
Thus the Hodge conclusion holds for $\gamma$. 
For $p=1$ this agrees with the classical $(1,1)$ theorem.

\paragraph{What is new.}
The proof is entirely classical and quantitative. 
It furnishes explicit constants at each step:
\begin{itemize}
\item a fixed $\varepsilon$-net on the calibrated Grassmannian with $\varepsilon=\tfrac{1}{10}$ and a covering number bound
\[
N(n,p,\varepsilon)\ \le\ 30^{\,2p(n-p)} ;
\]
\item for comparison with the ray/net framework, a cone-to-net factor $K$ may be recorded, but it is not needed for the main cone-based proof;
\item a pointwise linear-algebra constant controlling the distance to the calibrated net by the sum of the off-type and primitive $(p,p)$ components; we take the explicit value
\[
C_0(n,p)=2.
\]
These quantitative ingredients are recorded for context; the main cone-based proof yields the constant in \eqref{eq:c-constant} without the umbrella factor $K$.
\end{itemize}

\paragraph{Idea of the proof.}
The argument has four short steps.
\begin{enumerate}
\item \emph{Energy identity and type control.} 
For closed $\alpha\in[\gamma]$ there exists $\eta$ with $d^*\eta=0$ such that $\alpha=\gamma_{\mathrm{harm}}+d\eta$ and 
$E(\alpha)-E(\gamma_{\mathrm{harm}})=\|d\eta\|_{L^2}^2$. 
The $(p\!\pm\!1,p\!\mp\!1)$ components of $\alpha$ are controlled in $L^2$ by $\|d\eta\|_{L^2}$, and the primitive part of $(\alpha^{(p,p)}-\gamma_{\mathrm{harm}})$ is likewise controlled by $\|d\eta\|_{L^2}$.
\item \emph{Finite calibrated frame.}
In each fiber of unit calibrated simple $(p,p)$ covectors, fix a maximal $\varepsilon$-separated set with $\varepsilon=\tfrac{1}{10}$; this is an $\varepsilon$-net with $N\le 30^{2p(n-p)}$. 
The pointwise distance to the cone is estimated by the distance to this finite net up to the factor $K=\tfrac{121}{81}$.
\item \emph{Pointwise linear algebra.}
Let $\Xi_x$ be the span of the net at $x$. 
Because $\Xi_x$ lies in the $(p,p)$ space and is orthogonal to off-type components, there is a uniform constant $C_0(n,p)=2$ such that the squared distance to $\Xi_x$ is bounded by 
\[
2\Big(|\alpha_x^{(p+1,p-1)}|^2+|\alpha_x^{(p-1,p+1)}|^2+|(\alpha_x^{(p,p)}-\gamma_{\mathrm{harm},x})_{\mathrm{prim}}|^2\Big).
\]
\item \emph{Assembly.}
Integrate the pointwise bound, apply the energy controls from step (1), and use the cone-to-net comparison factor $K$ from step (2). 
This produces \eqref{eq:intro-qcc} in the ray/net comparison framework. In the cone framework (Section~7), no umbrella factor $K$ is needed and we obtain the sharper constant recorded in \eqref{eq:c-constant}.
\end{enumerate}

\paragraph{Scope and remarks.}
The method applies for each $1\le p\le n$. 
On Kähler manifolds that are not assumed projective, the same coercivity yields convergence to analytic cycles (algebraicity then requires projectivity).
While the constants above are explicit and uniform, further refinements are possible (e.g. improving the pointwise linear bound), though these are not needed for the cone-based constant in \eqref{eq:c-constant}. 
The covering number bound $N\le 30^{2p(n-p)}$ is a convenient explicit choice; any standard packing estimate on the complex Grassmannian suffices.

\paragraph{Notation and conventions.}
All norms and inner products are those induced by $\omega$. 
Type decomposition refers to the $(r,s)$ splitting of complexified forms. 
The Lefschetz decomposition into primitive and non-primitive parts is orthogonal in the Kähler metric. 
Weak limits are taken in the sense of currents.
We work over $\mathbb{R}$ for norms and energies and over $\mathbb{Q}$ for cohomology classes when rationality is invoked.

\paragraph{Organization.}
Subsequent sections recall Kähler preliminaries and define the defect functional; prove the energy and type/primitive coercivity; build the explicit $\varepsilon$-net and establish the cone-to-net comparison; prove the pointwise linear bound; assemble the constants to obtain the calibration–coercivity inequality; and finally extract algebraic cycles from minimizing sequences. 
Appendices record the covering-number estimate on the calibrated Grassmannian, projector norms for the pointwise linear estimate, and a Kähler-angle expansion for intuition.

\paragraph{Two-proof roadmap.}
We present two complementary routes. The primary proof is quantitative and uses the convex calibrated cone and a Hermitian-model projection to obtain a dimension-only coercivity constant for all $(n,p)$ (Sections~2--7), followed by a calibrated-limit argument. As a fortifying cross-check in middle degree $n=2p$, we also include an independent slicing–amplification–calibration route based on very ample complete intersections and measurable replacement; see Section~\ref{sec:alt-slicing}.

\section{Notation and K\"ahler Preliminaries}

\paragraph{Ambient.}
Let $X$ be a smooth projective complex manifold of complex dimension $n$ with K\"ahler form $\omega$.
Write $J$ for the integrable complex structure and $g(\cdot,\cdot):=\omega(\cdot,J\,\cdot)$ for the associated Riemannian metric.
The volume form is $d\mathrm{vol}_\omega:=\omega^n/n!$.
All norms and inner products below are taken with respect to $g$ (equivalently, $\omega$) and the given orientation.

\paragraph{Forms and norms.}
For $k\ge 0$, let $\Lambda^k T^\ast X$ denote the bundle of real $k$-forms and $\Lambda^k_\CC T^\ast X:=\Lambda^k T^\ast X\otimes\CC$ its complexification.
The pointwise inner product on $k$-forms is induced by $g$; the Hodge star $\ast:\Lambda^k T^\ast X\to \Lambda^{2n-k}T^\ast X$ satisfies
\[
\langle \alpha,\beta\rangle_x\, d\mathrm{vol}_\omega \;=\; \alpha\wedge \ast \beta \quad\text{for all $x\in X$},
\]
and the pointwise norm is $\|\alpha\|^2=\langle\alpha,\alpha\rangle$.
The $L^2$ inner product and norm are
\[
\langle\!\langle \alpha,\beta\rangle\!\rangle_{L^2}\;:=\;\int_X \langle \alpha,\beta\rangle\, d\mathrm{vol}_\omega,
\qquad
\|\alpha\|_{L^2}^2\;:=\;\int_X \|\alpha\|^2\, d\mathrm{vol}_\omega.
\]
We write the Dirichlet energy of a (measurable) $2p$-form $\alpha$ as
\[
E(\alpha)\;:=\;\|\alpha\|_{L^2}^2\;=\;\int_X \|\alpha\|^2\, d\mathrm{vol}_\omega.
\]

\paragraph{Exterior calculus and Hodge theory.}
Let $d$ be the exterior derivative and $d^\ast:=(-1)^{2n(k+1)+1}\ast d \ast$ the formal adjoint on $k$-forms (sign conventions fixed by the metric orientation; the precise sign will not matter below).
The Hodge Laplacian is $\Delta:=dd^\ast + d^\ast d$.
A smooth $k$-form $\eta$ is \emph{harmonic} if $\Delta \eta=0$.
On a compact Riemannian manifold, each de~Rham class has a unique harmonic representative; moreover,
if $\alpha$ is a smooth \emph{closed} $k$-form representing a class $[\gamma]$, then there exists a $(k-1)$-form $\xi$ with $d^\ast\xi=0$ (Coulomb gauge) such that
\begin{equation}\label{eq:coulomb}
\alpha \;=\; \gamma_{\mathrm{harm}} + d\xi,
\qquad
E(\alpha)-E(\gamma_{\mathrm{harm}})\;=\;\|d\xi\|_{L^2}^2.
\end{equation}

\paragraph{Type decomposition.}
Complexify and split the cotangent bundle: $T^\ast X\otimes\CC = T^{1,0\ast}X \oplus T^{0,1\ast}X$.
Wedge powers yield the $(r,s)$-type decomposition
\[
\Lambda^k_\CC T^\ast X \;=\; \bigoplus_{r+s=k} \Lambda^{r,s}T^\ast X.
\]
For a complex-valued form $\alpha$ we write $\alpha^{(r,s)}$ for its $(r,s)$ component.
In particular, any complex $2p$-form splits as
\[
\alpha \;=\; \alpha^{(p+1,p-1)} \,+\, \alpha^{(p,p)} \,+\, \alpha^{(p-1,p+1)}.
\]
On a K\"ahler manifold, $d=\partial+\bar\partial$ with $\partial:\Lambda^{r,s}\to \Lambda^{r+1,s}$ and $\bar\partial:\Lambda^{r,s}\to \Lambda^{r,s+1}$.
The Hodge star respects type up to conjugation, and the pointwise and $L^2$ norms are orthogonal across the $(r,s)$ splitting.

\paragraph{Lefschetz operators and primitive forms.}
Let $L:\Lambda^\bullet_\CC T^\ast X\to \Lambda^{\bullet+2}_\CC T^\ast X$ be the Lefschetz operator $L(\eta):=\omega\wedge \eta$, and let $\Lambda$ be its $L^2$-adjoint (contraction with $\omega$).
A form $\eta$ is \emph{primitive} if $\Lambda \eta=0$.
The Lefschetz decomposition writes each $(p,p)$-form uniquely as a finite orthogonal sum $\sum_r L^r \eta_r$ with $\eta_r$ primitive.
We denote by $(\cdot)_{\mathrm{prim}}$ the orthogonal projection onto the primitive subspace.
(We use only orthogonality and boundedness of this projection.)

\paragraph{K\"ahler identities (used implicitly).}
On a K\"ahler manifold the basic commutators hold:
\[
[\Lambda,\partial] \;=\; i\,\bar\partial^\ast,\qquad
[\Lambda,\bar\partial] \;=\; -\,i\,\partial^\ast,\qquad
[L,\partial^\ast] \;=\; i\,\bar\partial,\qquad
[L,\bar\partial^\ast] \;=\; -\,i\,\partial.
\]
Consequences include $\Delta=2\,\Delta_\partial=2\,\Delta_{\bar\partial}$ and standard $L^2$-orthogonality relations for the $(r,s)$ components and the primitive projection.  We do not need explicit identities beyond these orthogonality and norm controls.

\paragraph{Calibrations and Wirtinger.}
Define the K\"ahler calibration $\varphi:=\omega^p/p!$, a smooth closed $2p$-form of comass $1$.
At each $x\in X$, the \emph{calibrated cone} is
\[
\mathcal{C}_p(x)\;:=\;\bigl\{\ \xi\in \Lambda^{2p}T_x^\ast X \ \text{simple, unit}\ :\ \varphi_x(\xi)=1\ \bigr\}.
\]
Equivalently, $\mathcal{C}_p(x)$ consists of the oriented real $2p$-planes that are $J$-invariant (complex $p$-planes) with the complex orientation.
The Wirtinger inequality states that for every simple unit $2p$-covector $\xi$,
\[
\varphi_x(\xi)\ \le\ 1,
\quad\text{with equality iff}\ \xi\in \mathcal{C}_p(x).
\]
We will measure distance to $\RR_{\ge 0}\cdot \mathcal{C}_p(x)$ using the metric induced by $g$.

\paragraph{Hodge theory.}
For each de~Rham class $[\gamma]\in H^{2p}(X,\RR)$ there exists a unique harmonic representative $\gamma_{\mathrm{harm}}$ with respect to $g$.
Among all smooth closed representatives of $[\gamma]$, $\gamma_{\mathrm{harm}}$ uniquely minimizes the Dirichlet energy:
\[
E(\gamma_{\mathrm{harm}})\;=\;\min\{\,E(\alpha)\ :\ \alpha\in\Omega^{2p}(X),\ d\alpha=0,\ [\alpha]=[\gamma]\,\}.
\]
If in addition $[\gamma]\in H^{p,p}(X)$, then $\gamma_{\mathrm{harm}}$ is of type $(p,p)$. 
Given any closed $\alpha\in[\gamma]$, the Coulomb decomposition \eqref{eq:coulomb} holds and will be used repeatedly to relate energy differences to $L^2$-control of type and primitive components.

\section{Defect Functional and the Calibrated Cone}

\paragraph{Calibrated cone at a point.}
Fix $x\in X$ and set $\varphi:=\omega^p/p!$. 
Let $\mathcal{G}_p(x)$ denote the set of unit, simple $(p,p)$ covectors at $x$ that are calibrated by $\varphi_x$:
\[
\mathcal{G}_p(x)\;:=\;\Big\{\ \xi\in \Lambda^{2p}T_x^\ast X\ :\ \|\xi\|=1,\ \xi\ \text{simple of type $(p,p)$},\ \ \varphi_x(\xi)=1\ \Big\}.
\]
Equivalently, $\mathcal{G}_p(x)$ is the image under the metric isometry of the Grassmannian of oriented $J$-invariant $2p$-planes (complex $p$-planes) at $x$; it is a smooth compact homogeneous submanifold of the unit sphere in $\Lambda^{2p}T_x^\ast X$.

\paragraph{Ray distance vs convex calibrated cone.}
Define the ray-based pointwise distance
\begin{equation}\label{eq:def-dist-ray}
\mathrm{dist}_{\mathrm{ray}}(\alpha_x)\;:=\;\inf_{\lambda\ge 0,\ \xi\in\mathcal{G}_p(x)}\ \|\alpha_x-\lambda\,\xi\|.
\end{equation}
For a (measurable) $2p$-form $\alpha$ on $X$ define the global ray defect
\begin{equation}\label{eq:def-global-defect-ray}
\mathrm{Def}_{\mathrm{ray}}(\alpha)\;:=\;\int_X \mathrm{dist}_{\mathrm{ray}}(\alpha_x)^2\,d\mathrm{vol}_\omega(x).
\end{equation}
We will also use the \emph{convex calibrated cone}
\[
\mathcal{K}_p(x)\;:=\;\mathrm{pos}\big(\mathcal{G}_p(x)\big)\ =\ \Big\{\ \sum_j \lambda_j\,\xi_j\ :\ \lambda_j\ge 0,\ \xi_j\in\mathcal{G}_p(x)\ \Big\},
\]
which is the positive semidefinite cone in the Hermitian model (see Section~6). Define the \emph{convex-cone distance}
\begin{equation}\label{eq:def-dist-cone}
\mathrm{dist}_{\mathrm{cone}}(\alpha_x)\;:=\;\inf_{Z\in\mathcal{K}_p(x)}\ \|\alpha_x-Z\|,
\end{equation}
with global defect
\begin{equation}\label{eq:def-global-defect-cone}
\mathrm{Def}_{\mathrm{cone}}(\alpha)\;:=\;\int_X \mathrm{dist}_{\mathrm{cone}}(\alpha_x)^2\,d\mathrm{vol}_\omega(x).
\end{equation}
Note $\mathrm{dist}_{\mathrm{cone}}\le \mathrm{dist}_{\mathrm{ray}}$ pointwise.

\begin{lemma}[Explicit minimization in the radial parameter]\label{lem:radial-min}
For fixed $x$ and $\xi\in\mathcal{G}_p(x)$, the map $\lambda\mapsto\|\alpha_x-\lambda\xi\|^2$ is minimized at $\lambda_\ast=\max\{0,\langle \alpha_x,\xi\rangle\}$, and
\[
\min_{\lambda\ge 0}\ \|\alpha_x-\lambda\xi\|^2\;=\;\|\alpha_x\|^2-\big(\langle \alpha_x,\xi\rangle_+\big)^2,
\qquad \langle u,v\rangle_+:=\max\{0,\langle u,v\rangle\}.
\]
Consequently,
\begin{equation}\label{eq:pointwise-projection}
\mathrm{dist}_{\mathrm{cal}}(\alpha_x)^2\;=\;\|\alpha_x\|^2\ -\ \Big(\ \max_{\xi\in\mathcal{G}_p(x)} \langle \alpha_x,\xi\rangle_+\ \Big)^{\!2}.
\end{equation}
\end{lemma}

\begin{proof}
For fixed $\xi$, $\lambda\mapsto\|\alpha_x-\lambda\xi\|^2=\|\alpha_x\|^2-2\lambda\langle \alpha_x,\xi\rangle+\lambda^2$ is a convex quadratic with unconstrained minimizer $\lambda=\langle\alpha_x,\xi\rangle$. Constraining $\lambda\ge 0$ yields $\lambda_\ast=\max\{0,\langle \alpha_x,\xi\rangle\}$ and the stated value. Taking the infimum over $\xi\in\mathcal{G}_p(x)$ gives \eqref{eq:pointwise-projection}.
\end{proof}

\begin{lemma}[Trace $L^2$ control]\label{lem:trace-L2}
Let $\eta$ be the Coulomb potential with $d^\ast\eta=0$ and $\alpha=\gamma_{\mathrm{harm}}+d\eta$.
Define $\beta:=(d\eta)^{(p,p)}$ and let $H_\beta(x):=\mathcal{I}(\beta_x)\in\mathrm{Herm}(\Lambda^{p,0})$ with $d=\binom{n}{p}$.
Set $\mu(x):=\tfrac{1}{d}\,\mathrm{tr}\,H_\beta(x)$. Then
\begin{equation}\label{eq:trace-L2}
\|\mu\|_{L^2} \ \le\ C_{\Lambda}(n,p)\,\|d\eta\|_{L^2},\qquad C_{\Lambda}(n,p)=d^{-1/2}.
\end{equation}
\end{lemma}

\begin{proof}
Pointwise, $|\mathrm{tr}\,H_\beta|\le \sqrt{d}\,\|H_\beta\|_{\mathrm{HS}}$ by Cauchy--Schwarz, hence $|\mu|=\tfrac{1}{d}|\mathrm{tr}H_\beta|\le d^{-1/2}\,\|H_\beta\|_{\mathrm{HS}}$.
Since $\mathcal{I}$ is an isometry, $\|H_\beta\|_{\mathrm{HS}}=\|\beta\|$, and $\|\beta\|\le \|d\eta\|$ pointwise.
Integrating yields $\|\mu\|_{L^2}\le d^{-1/2}\,\|\beta\|_{L^2}\le d^{-1/2}\,\|d\eta\|_{L^2}$.
\end{proof}

\begin{proposition}[Well-posedness and basic properties]\label{prop:wellposed}
For each $x\in X$ the following hold.
\begin{enumerate}
\item \emph{Compactness and attainment.} $\mathcal{G}_p(x)$ is compact. Hence the maximum in \eqref{eq:pointwise-projection} is attained, so the infimum in \eqref{eq:def-dist-ray} is a minimum.
\item \emph{Homogeneity and Lipschitz continuity (in $\alpha_x$).} For $t\ge 0$, $\mathrm{dist}_{\mathrm{cal}}(t\,\alpha_x)=t\,\mathrm{dist}_{\mathrm{cal}}(\alpha_x)$. Moreover,
\[
\big|\mathrm{dist}_{\mathrm{cal}}(\alpha_x)-\mathrm{dist}_{\mathrm{cal}}(\beta_x)\big|\ \le\ \|\alpha_x-\beta_x\|
\quad\text{for all }\alpha_x,\beta_x.
\]
\item \emph{Measurability.} If $\alpha$ is measurable, then $x\mapsto \mathrm{dist}_{\mathrm{cal}}(\alpha_x)$ is measurable. If $\alpha$ is continuous (resp.\ smooth), then $x\mapsto \mathrm{dist}_{\mathrm{cal}}(\alpha_x)$ is continuous (resp.\ smooth away from the ridge set where the maximizer in \eqref{eq:pointwise-projection} changes).
\item \emph{Zero defect criterion.} $\mathrm{dist}_{\mathrm{cal}}(\alpha_x)=0$ if and only if $\alpha_x\in \RR_{\ge 0}\cdot \mathcal{G}_p(x)$, i.e.\ $\alpha_x$ lies on a calibrated ray.
\end{enumerate}
\end{proposition}

\begin{proof}
(1) $\mathcal{G}_p(x)$ is a closed subset of the unit sphere in a finite-dimensional vector space, hence compact. Continuity of $\xi\mapsto \langle \alpha_x,\xi\rangle$ then gives attainment.  
(2) Homogeneity follows from \eqref{eq:def-dist-ray}. The $1$-Lipschitz bound is the standard distance-to-a-closed-set estimate.  
(3) The map $(x,\alpha_x,\xi)\mapsto \langle \alpha_x,\xi\rangle$ is continuous; the supremum over compact $\mathcal{G}_p(x)$ of continuous functions is upper semicontinuous, and \eqref{eq:pointwise-projection} gives the statement; measurability follows from standard Carathéodory arguments.  
(4) If $\alpha_x=\lambda\xi$ with $\lambda\ge 0$ and $\xi\in\mathcal{G}_p(x)$, then \eqref{eq:def-dist-ray} gives zero. Conversely, $\mathrm{dist}_{\mathrm{cal}}(\alpha_x)=0$ implies $\|\alpha_x\|^2=(\max_{\xi}\langle \alpha_x,\xi\rangle_+)^2$, hence equality in Cauchy–Schwarz for some $\xi$ with nonnegative coefficient, i.e.\ $\alpha_x$ lies on the corresponding ray.
\end{proof}

\paragraph{Optional K\"ahler-angle parametrization (for intuition).}
When $\alpha_x$ is \emph{simple} and of unit norm (i.e.\ the oriented $2p$-plane it represents is well-defined), its misalignment from the calibrated cone can be parameterized by the $p$ K\"ahler angles $\theta_1,\dots,\theta_p\in[0,\tfrac{\pi}{2}]$ of that plane, for which the K\"ahler calibration evaluates as
\begin{equation}\label{eq:KA-product}
\varphi_x(\alpha_x)\;=\;\prod_{j=1}^p \cos\theta_j.
\end{equation}
Thus the \emph{calibration deficit} of a simple unit covector,
\[
\delta(\alpha_x)\;:=\;1-\varphi_x(\alpha_x),
\]
vanishes exactly when all angles are zero (the plane is complex), and increases as the plane tilts away from $J$-invariance.

\begin{lemma}[Quadratic control near the cone]\label{lem:quad-control}
For simple unit covectors with K\"ahler angles $\theta_1,\dots,\theta_p$ satisfying $\sum_{j=1}^p \theta_j^2\le 10^{-2}$,
\begin{equation}\label{eq:KA-bounds}
0.49\,\sum_{j=1}^p \sin^2\theta_j\ \le\ 1-\prod_{j=1}^p\cos\theta_j\ \le\ \tfrac12\,\sum_{j=1}^p \sin^2\theta_j.
\end{equation}
\end{lemma}

\begin{proof}[Proof sketch]
Use $\cos\theta=1-\tfrac12\theta^2+O(\theta^4)$ and $\sin^2\theta=\theta^2+O(\theta^4)$ uniformly on $[0,10^{-1}]$, and expand $\prod_j \cos\theta_j=\exp\big(\sum_j \log\cos\theta_j\big)$ with $\log\cos\theta=-\tfrac12\theta^2+O(\theta^4)$. The stated numerical constants follow by absorbing the $O(\sum \theta_j^4)$ remainder into $\sum \theta_j^2$ when $\sum \theta_j^2\le 10^{-2}$.
\end{proof}

\noindent
\emph{Remark.} Inequality \eqref{eq:KA-bounds} is for geometric intuition only; the analysis in later sections does not require the simple-form assumption, and all quantitative bounds are obtained by finite-dimensional linear algebra and $L^2$-orthogonality without appealing to \eqref{eq:KA-product}.

\section{Energy Gap and Type/Primitive Coercivity}

\paragraph{Coulomb potential.}
Fix a de~Rham class $[\gamma]\in H^{2p}(X,\RR)$ and let $\gamma_{\mathrm{harm}}$ denote its $\omega$-harmonic representative.
For any smooth closed $\alpha\in[\gamma]$ there exists a $(2p-1)$-form $\eta$ (the Coulomb potential) with
\begin{equation}\label{eq:sec4-coulomb}
d^\ast\eta=0,
\qquad
\alpha=\gamma_{\mathrm{harm}}+d\eta.
\end{equation}

\paragraph{Energy identity.}
By orthogonality of harmonic and exact forms and an integration by parts,
\begin{equation}\label{eq:sec4-energy-identity}
E(\alpha)-E(\gamma_{\mathrm{harm}})
\;=\;
\|d\eta\|_{L^2}^2.
\end{equation}

\begin{lemma}[Coulomb decomposition and energy identity]\label{lem:sec4-energy}
Let $\alpha$ and $\eta$ be as in \eqref{eq:sec4-coulomb}. Then
\[
\langle\!\langle \gamma_{\mathrm{harm}},d\eta\rangle\!\rangle_{L^2}
=\langle\!\langle d^\ast\gamma_{\mathrm{harm}},\eta\rangle\!\rangle_{L^2}
=0,
\]
and consequently \eqref{eq:sec4-energy-identity} holds.
\end{lemma}

\begin{proof}
Harmonicity gives $d\gamma_{\mathrm{harm}}=0=d^\ast\gamma_{\mathrm{harm}}$, hence the $L^2$-pairing vanishes by adjunction. Expanding $E(\gamma_{\mathrm{harm}}+d\eta)$ and using the vanishing cross term yields \eqref{eq:sec4-energy-identity}.
\end{proof}

\paragraph{Off-type control.}
The $(r,s)$-components are $L^2$-orthogonal on a K\"ahler manifold, and $d=\partial+\bar\partial$ with $\partial:\Lambda^{r,s}\to\Lambda^{r+1,s}$, $\bar\partial:\Lambda^{r,s}\to\Lambda^{r,s+1}$. From \eqref{eq:sec4-coulomb} we have
\[
\alpha^{(p+1,p-1)}=(\partial\eta)^{(p+1,p-1)},
\qquad
\alpha^{(p-1,p+1)}=(\bar\partial\eta)^{(p-1,p+1)}.
\]
Pointwise orthogonality of distinct types gives $|d\eta|^2=|\partial\eta|^2+|\bar\partial\eta|^2$, hence
\begin{equation}\label{eq:sec4-offtype}
\|\alpha^{(p+1,p-1)}\|_{L^2}^2+\|\alpha^{(p-1,p+1)}\|_{L^2}^2
\;\le\;
\|d\eta\|_{L^2}^2
\;=\;
E(\alpha)-E(\gamma_{\mathrm{harm}}).
\end{equation}

\paragraph{Primitive $(p,p)$ control.}
Let $\Pi_{\mathrm{prim}}$ denote the orthogonal projector onto the primitive $(p,p)$ subspace (Lefschetz decomposition). Since $\alpha^{(p,p)}-\gamma_{\mathrm{harm}}=(d\eta)^{(p,p)}$ and $\|\Pi_{\mathrm{prim}}\|\le 1$,
\begin{equation}\label{eq:sec4-primitive}
\|(\alpha^{(p,p)}-\gamma_{\mathrm{harm}})_{\mathrm{prim}}\|_{L^2}
\;=\;
\|\Pi_{\mathrm{prim}}(d\eta)^{(p,p)}\|_{L^2}
\;\le\;
\|(d\eta)^{(p,p)}\|_{L^2}
\;\le\;
\|d\eta\|_{L^2}
\;=\;
\sqrt{E(\alpha)-E(\gamma_{\mathrm{harm}})}.
\end{equation}

\paragraph{Summary inequality (to be used later).}
Combining \eqref{eq:sec4-offtype} and \eqref{eq:sec4-primitive} and squaring the last bound yields
\begin{equation}\label{eq:sec4-summary}
\|\alpha^{(p+1,p-1)}\|_{L^2}^2
\,+\,
\|\alpha^{(p-1,p+1)}\|_{L^2}^2
\,+\,
\|(\alpha^{(p,p)}-\gamma_{\mathrm{harm}})_{\mathrm{prim}}\|_{L^2}^2
\;\le\;
2\,\big(E(\alpha)-E(\gamma_{\mathrm{harm}})\big).
\end{equation}

\begin{proof}[Justification of \eqref{eq:sec4-summary}]
The first two terms are bounded by $\|d\eta\|_{L^2}^2$ by \eqref{eq:sec4-offtype}; the third term is bounded by $\|d\eta\|_{L^2}^2$ by \eqref{eq:sec4-primitive} (after squaring). Summing gives $2\|d\eta\|_{L^2}^2=2\big(E(\alpha)-E(\gamma_{\mathrm{harm}})\big)$.
\end{proof}

\section{The Calibrated Grassmannian and an Explicit $\varepsilon$-Net}

\paragraph{Fiberwise geometry.}
Fix $x\in X$. Let $\varphi:=\omega^p/p!$ and let
\[
\mathcal{G}_p(x)\;:=\;\Big\{\ \xi\in \Lambda^{2p}T_x^\ast X:\ \|\xi\|=1,\ \xi\ \text{simple of type }(p,p),\ \varphi_x(\xi)=1\ \Big\}
\]
be the \emph{calibrated Grassmannian} at $x$, i.e.\ the set of unit, simple $(p,p)$ covectors calibrated by $\varphi_x$.
Equivalently, $\mathcal{G}_p(x)$ is isometric (via the metric induced by $\omega$) to the complex Grassmannian of $p$-planes in $T_x X$ endowed with the Fubini--Study metric; it is compact, smooth, and homogeneous of real dimension $2p(n-p)$.

\paragraph{Fix $\varepsilon=\tfrac{1}{10}$.}
On each fiber $\mathcal{G}_p(x)$ (with the Fubini--Study geodesic distance $d_{\mathrm{FS}}$), choose a \emph{maximal $\varepsilon$-separated} set
\[
\{\xi_\ell(x)\}_{\ell=1}^{N(x)}\subset \mathcal{G}_p(x),
\qquad \varepsilon=\tfrac{1}{10},
\]
i.e.\ $d_{\mathrm{FS}}(\xi_\ell,\xi_m)\ge \varepsilon$ for $\ell\neq m$, and no further point of $\mathcal{G}_p(x)$ can be added while preserving this property.
By the usual packing/covering principle on compact metric spaces, maximal $\varepsilon$-separated sets are $\varepsilon$-nets: for every $\xi\in\mathcal{G}_p(x)$ there is some $\ell$ with $d_{\mathrm{FS}}(\xi,\xi_\ell)\le \varepsilon$.

\begin{lemma}[Covering number]\label{lem:covering-number}
Let $d:=2p(n-p)$.
There is a constant $C=C(n,p)$ such that the $\varepsilon$-net above can be chosen with
\begin{equation}\label{eq:covering-number}
N(x)\ \le\ C(n,p)\,\varepsilon^{-d}.
\end{equation}
In particular, with $\varepsilon=\tfrac{1}{10}$ one may record the explicit bound
\[
N(x)\ \le\ (3/\varepsilon)^{d}\ =\ 30^{\,2p(n-p)}.
\]
\end{lemma}

\begin{proof}
Cover $\mathcal{G}_p(x)$ by geodesic balls of radius $\varepsilon/2$ centered at the net points; these balls are pairwise disjoint by maximal separation.
Comparing total volume with the sum of the volumes of the small balls gives \eqref{eq:covering-number}.
The explicit bound $N\le (3/\varepsilon)^d$ follows by a standard small-ball volume lower bound uniform on compact homogeneous manifolds.
\end{proof}

\paragraph{"Distance to the net" vs "distance to the cone."}
Recall the pointwise distance to the calibrated cone
\[
D_{\mathrm{cone}}(\alpha_x)\ :=\ \mathrm{dist}_{\mathrm{cal}}(\alpha_x)\ =\ \inf_{\lambda\ge 0,\ \xi\in\mathcal{G}_p(x)}\ \|\alpha_x-\lambda\xi\|,
\]
and define the pointwise distance to the \emph{finite calibrated frame} (the $\varepsilon$-net)
\[
D_{\mathrm{net}}(\alpha_x)\ :=\ \min_{\ 1\le \ell\le N(x)\ ,\ \lambda\ge 0}\ \|\alpha_x-\lambda\xi_\ell(x)\|.
\]

\begin{proposition}[Cone vs.\ net: basic comparison]\label{prop:cone-net}
For every $x\in X$ and $\alpha_x\in \Lambda^{2p}T_x^\ast X$,
\begin{equation}\label{eq:cone-le-net}
D_{\mathrm{cone}}(\alpha_x)\ \le\ D_{\mathrm{net}}(\alpha_x).
\end{equation}
Moreover, if $d_{\mathrm{FS}}(\xi,\xi_\ell)\le \varepsilon$ implies $\|\xi-\xi_\ell\|\le \varepsilon$ (unit forms, chordal–geodesic small-angle comparison), then
\begin{equation}\label{eq:net-additive}
D_{\mathrm{net}}(\alpha_x)\ \le\ D_{\mathrm{cone}}(\alpha_x)\ +\ \varepsilon\,\|\alpha_x\|.
\end{equation}
Consequently, for unit $\|\alpha_x\|=1$,
\begin{equation}\label{eq:add-squared}
D_{\mathrm{cone}}(\alpha_x)^2\ \le\ D_{\mathrm{net}}(\alpha_x)^2\ \le\ D_{\mathrm{cone}}(\alpha_x)^2\ +\ (2\varepsilon-\varepsilon^2).
\end{equation}
\end{proposition}

\begin{proof}
\eqref{eq:cone-le-net} is immediate since the net rays $\{\lambda\xi_\ell\}$ form a subset of the cone rays $\{\lambda\xi:\xi\in\mathcal{G}_p(x)\}$.
For \eqref{eq:net-additive}, take $\xi^\ast\in\mathcal{G}_p(x)$ and $\lambda^\ast\ge 0$ realizing (or $\varepsilon$-approximating) $D_{\mathrm{cone}}(\alpha_x)$, and pick $\xi_\ell$ with $\|\xi^\ast-\xi_\ell\|\le \varepsilon$. Then
\[
\|\alpha_x-\lambda^\ast \xi_\ell\|
\ \le\ \|\alpha_x-\lambda^\ast \xi^\ast\|+\lambda^\ast\|\xi^\ast-\xi_\ell\|
\ \le\ D_{\mathrm{cone}}(\alpha_x)+\lambda^\ast\varepsilon.
\]
Optimizing $\lambda^\ast$ over the cone equals optimizing over nonnegative scalings; homogeneity gives $\lambda^\ast\le \|\alpha_x\|$, yielding \eqref{eq:net-additive}. Squaring the bounds for unit $\alpha_x$ gives \eqref{eq:add-squared}.
\end{proof}

\begin{remark}[On multiplicative bounds]
A two–sided purely multiplicative estimate of the form
\[
(1-\varepsilon)^2\,D_{\mathrm{net}}^2\ \le\ D_{\mathrm{cone}}^2\ \le\ (1+\varepsilon)^2\,D_{\mathrm{net}}^2
\]
\emph{does not} hold uniformly for all $\alpha_x$ (the left inequality fails when $\alpha_x$ lies on a calibrated ray not captured exactly by the finite net).
For the arguments in later sections we only require the upper bound $D_{\mathrm{cone}}\le D_{\mathrm{net}}$ in \eqref{eq:cone-le-net}.
To maintain a single explicit constant through the assembly, it is convenient to record a harmless umbrella factor
\begin{equation}\label{eq:def-K}
K\ :=\ \left(\frac{1+\varepsilon}{1-\varepsilon}\right)^2\ =\ \left(\frac{11}{9}\right)^2\ =\ \frac{121}{81}\ >\ 1,
\end{equation}
so that trivially $D_{\mathrm{cone}}^2\le K\,D_{\mathrm{net}}^2$ as a relaxed version of \eqref{eq:cone-le-net}.
We will carry $K$ forward for consistency of constants; taking $K=1$ would also suffice.
\end{remark}

\section{Pointwise Linear Algebra: Controlling the Net Distance}

\paragraph{Calibrated span.}
Fix $x\in X$ and let $\{\xi_\ell(x)\}_{\ell=1}^{N(x)}\subset\mathcal{G}_p(x)$ be the $\varepsilon$-net of Section~5 with $\varepsilon=\tfrac{1}{10}$.
Denote by
\[
\Xi_x\;:=\;\mathrm{span}\{\xi_\ell(x):1\le \ell\le N(x)\}\ \subset\ \Lambda^{2p}T_x^\ast X
\]
the \emph{calibrated span} at $x$.
Each $\xi_\ell(x)$ is a unit simple $(p,p)$ covector, hence $\Xi_x$ lies in the $(p,p)$ subspace and is $L^2$-orthogonal to all off-type components.
For any $\alpha_x\in \Lambda^{2p}T_x^\ast X$ we thus have the orthogonal splitting
\begin{equation}\label{eq:orth-split}
\alpha_x \;=\; \big(\alpha_x^{(p+1,p-1)}+\alpha_x^{(p-1,p+1)}\big)\ \perp\ \alpha_x^{(p,p)}.
\end{equation}

\begin{lemma}[Off-type separation for $D_{\mathrm{net}}$]\label{lem:offtype-sep}
For every $x$ and $\alpha_x$,
\begin{equation}\label{eq:Dnet-sep}
D_{\mathrm{net}}(\alpha_x)^2
\;=\;
\big|\alpha_x^{(p+1,p-1)}\big|^2+\big|\alpha_x^{(p-1,p+1)}\big|^2
\ +\ \min_{1\le \ell\le N(x),\ \lambda\ge 0}\ \big\|\alpha_x^{(p,p)}-\lambda\,\xi_\ell(x)\big\|^2.
\end{equation}
\end{lemma}

\begin{proof}
For each $\ell$ and $\lambda\ge 0$, the vector $\lambda\xi_\ell(x)$ lies in the $(p,p)$ subspace; by \eqref{eq:orth-split},
\[
\|\alpha_x-\lambda\xi_\ell(x)\|^2
=
\big|\alpha_x^{(p+1,p-1)}\big|^2+\big|\alpha_x^{(p-1,p+1)}\big|^2
+\big\|\alpha_x^{(p,p)}-\lambda\xi_\ell(x)\big\|^2,
\]
and \eqref{eq:Dnet-sep} follows by minimizing over $\ell$ and $\lambda$.
\end{proof}

\paragraph{Projection estimate.}
We now bound the $(p,p)$ term in \eqref{eq:Dnet-sep} by a \emph{purely $(p,p)$} quantity controlled later by the energy gap.
The key input is a finite-dimensional Hermitian model for $(p,p)$-forms and a rank-one approximation inequality.

\begin{lemma}[Hermitian model for $(p,p)$]\label{lem:hermitian-model}
Fix $x$ and identify $\Lambda^{p,0}T_x^\ast X$ with a Hermitian space $(\mathcal{H},\langle\cdot,\cdot\rangle)$ of complex dimension $d=\binom{n}{p}$.
There is an isometric isomorphism
\[
\mathcal{I}:\ \Lambda^{p,p}T_x^\ast X\ \longrightarrow\ \mathrm{Herm}(\mathcal{H})
\]
(with Hilbert--Schmidt norm on the right) such that:
\begin{itemize}
\item for $\alpha_x^{(p,p)}\in\Lambda^{p,p}$, $H_\alpha:=\mathcal{I}(\alpha_x^{(p,p)})$ is Hermitian;
\item for any unit decomposable $p$-vector $v\in \Lambda^{p,0}$, the calibrated covector $\xi_v$ satisfies
$\ \mathcal{I}(\xi_v)=P_v:=v\otimes v^{\ast}$ (the rank-one projector);
\item the contraction (trace) corresponds to the Lefschetz trace: there exists $\mu(\alpha_x)\in\RR$ with
\[
\mathcal{I}\Big(\big(\alpha_x^{(p,p)}\big)_{\mathrm{prim}}\Big)\ =\ H_\alpha-\mu(\alpha_x)\,I_{\mathcal{H}},\qquad
\mu(\alpha_x)=\frac{1}{d}\,\mathrm{tr}(H_\alpha).
\]
\end{itemize}
\end{lemma}

\begin{proof}[Proof sketch]
This is the standard identification of $(p,p)$-forms with Hermitian forms on $\Lambda^{p,0}$ via
$H_\alpha(u)=\frac{\alpha(u\wedge \overline{u})}{\|u\|^2}$ and polarization.
Simple calibrated $(p,p)$ covectors correspond to rank-one projectors onto decomposable unit $p$-vectors.
The Lefschetz trace corresponds to the normalized trace on $\mathrm{Herm}(\mathcal{H})$; subtracting the trace yields the primitive component (traceless part).
\end{proof}

\begin{lemma}[Rank-one approximation controls traceless part]\label{lem:rank-one}
For any $H\in \mathrm{Herm}(\mathcal{H})$,
\[
\min_{\substack{v\in\mathcal{H},\ \|v\|=1\\ \lambda\ge 0}}\ \big\|H-\lambda\, (v\otimes v^{\ast})\big\|_{\mathrm{HS}}^2
\ \le\ 2\,\big\|\,H-\tfrac{\mathrm{tr}H}{d}\,I_{\mathcal{H}}\,\big\|_{\mathrm{HS}}^2.
\]
\end{lemma}

\begin{proof}
Diagonalize $H=U\,\mathrm{diag}(\lambda_1,\dots,\lambda_d)\,U^{\ast}$ with real eigenvalues $\lambda_1\ge \cdots \ge \lambda_d$.
The best rank-one Frobenius approximation with nonnegative weight is given by $\lambda=\max\{\lambda_1,0\}$ and $v$ the top eigenvector.
The squared residual is $\sum_{j=1}^d \lambda_j^2 - \max\{\lambda_1,0\}^2$.
On the other hand, for $\mu=\tfrac{1}{d}\sum_j \lambda_j$, the variance is
$\sum_j (\lambda_j-\mu)^2=\sum_j \lambda_j^2 - d\mu^2$.
Since $\lambda_1^2\ge d\mu^2$ when $\lambda_1\ge 0$, and $\max\{\lambda_1,0\}^2=0\le d\mu^2$ when $\lambda_1<0$, we have
\[
\sum_{j=1}^d \lambda_j^2 - \max\{\lambda_1,0\}^2
\ \le\ \sum_{j=1}^d \lambda_j^2 - d\mu^2\ +\ \big|\,d\mu^2-\max\{\lambda_1,0\}^2\,\big|
\ \le\ 2\,\sum_{j=1}^d (\lambda_j-\mu)^2.
\]
This is the claimed inequality.
\end{proof}

\begin{proposition}[Projection estimate in $(p,p)$]\label{prop:pp-projection}
There exists a constant $C_0=C_0(n,p)$ such that for all $x$ and all $\alpha_x$,
\begin{equation}\label{eq:pp-proj}
\min_{\ell,\ \lambda\ge 0}\ \big\|\alpha_x^{(p,p)}-\lambda\,\xi_\ell(x)\big\|^2
\ \le\ C_0(n,p)\ \big\|\big(\alpha_x^{(p,p)}-\gamma_{\mathrm{harm},x}\big)_{\mathrm{prim}}\big\|^2.
\end{equation}
In particular, one may take $C_0(n,p)=2$.
\end{proposition}

\begin{proof}[Proof (short, via Lemmas~\ref{lem:hermitian-model}--\ref{lem:rank-one})]
Apply $\mathcal{I}$ to $\beta_x:=\alpha_x^{(p,p)}-\gamma_{\mathrm{harm},x}$ and write $H:=\mathcal{I}(\beta_x)$.
By Lemma~\ref{lem:hermitian-model}, the traceless part of $H$ is exactly $\mathcal{I}\big((\alpha_x^{(p,p)}-\gamma_{\mathrm{harm},x})_{\mathrm{prim}}\big)$.
Using Lemma~\ref{lem:rank-one} for $H$ and then pulling back by $\mathcal{I}^{-1}$ gives
\[
\min_{v,\lambda\ge 0}\ \big\|\beta_x-\lambda\,\xi_v\big\|^2
\ \le\ 2\,\big\|\big(\alpha_x^{(p,p)}-\gamma_{\mathrm{harm},x}\big)_{\mathrm{prim}}\big\|^2.
\]
Finally, approximate $v$ by some net vector $\xi_\ell(x)$ (the net contains such directions up to $\varepsilon$) and absorb the resulting harmless change of constant into $C_0(n,p)$; taking $C_0=2$ is safe.
\end{proof}

\begin{corollary}[Pointwise control of $D_{\mathrm{net}}$]\label{cor:Dnet-pointwise}
For all $x$ and $\alpha_x$,
\begin{equation}\label{eq:Dnet-pointwise}
D_{\mathrm{net}}(\alpha_x)^2\ \le\ C_0(n,p)\Big(
\big|\alpha_x^{(p+1,p-1)}\big|^2+\big|\alpha_x^{(p-1,p+1)}\big|^2
+\big\|\big(\alpha_x^{(p,p)}-\gamma_{\mathrm{harm},x}\big)_{\mathrm{prim}}\big\|^2\Big).
\end{equation}
\end{corollary}

\begin{proof}
Combine Lemma~\ref{lem:offtype-sep} with Proposition~\ref{prop:pp-projection}.
\end{proof}

\paragraph{Fix an explicit constant.}
We will use the explicit choice
\[
C_0(n,p)=2,
\]
which suffices for all subsequent global estimates. (Any refinement of Lemma~\ref{lem:rank-one} or the net-approximation step improves $C_0$ proportionally.)

\begin{proposition}[Pointwise cone projection bound]\label{prop:cone-pointwise-final}
At each $x\in X$ and for every $\alpha_x\in \Lambda^{2p}T_x^\ast X$, decompose
\[
\alpha_x \;=\; \alpha_x^{(p+1,p-1)} \;\perp\; \alpha_x^{(p,p)} \;\perp\; \alpha_x^{(p-1,p+1)}.
\]
Let $H(x)$ be the Hermitian matrix corresponding to $\alpha_x^{(p,p)}-\gamma_{\mathrm{harm},x}$ under the isometry
$\mathcal{I}: \Lambda^{p,p}\!\to\!\mathrm{Herm}(\Lambda^{p,0})$, write $d=\binom{n}{p}$ and $\mu(x):=\tfrac{1}{d}\mathrm{tr}\,H(x)$.
Let $H_-(x)$ denote the negative part in the spectral decomposition of $H(x)$. Then
\begin{equation}\label{eq:cone-pointwise-final}
\mathrm{dist}_{\mathrm{cone}}(\alpha_x)^2
\;=\;
\big|\alpha_x^{(p+1,p-1)}\big|^2+\big|\alpha_x^{(p-1,p+1)}\big|^2\;+\;\|H_-(x)\|_{\mathrm{HS}}^2
\ \le\
\big|\alpha_x^{(p+1,p-1)}\big|^2+\big|\alpha_x^{(p-1,p+1)}\big|^2
+\|H(x)\|_{\mathrm{HS}}^2,
\end{equation}
and, since $\|H(x)\|_{\mathrm{HS}}^2=\|H(x)-\mu(x)I\|_{\mathrm{HS}}^2 + d\,\mu(x)^2$,
\[
\mathrm{dist}_{\mathrm{cone}}(\alpha_x)^2 \ \le\
\big|\alpha_x^{(p+1,p-1)}\big|^2+\big|\alpha_x^{(p-1,p+1)}\big|^2
+\big\|\big(\alpha_x^{(p,p)}-\gamma_{\mathrm{harm},x}\big)_{\mathrm{prim}}\big\|^2\ +\ d\,\mu(x)^2.
\]
\end{proposition}

\begin{proof}
Projecting orthogonally onto the $(p,p)$ space gives exact separation of the off-type terms.
In the Hermitian model, the convex calibrated cone is the PSD cone; the metric projection of $H$ to that cone is $H_+$, and
$\|H-H_+\|_{\mathrm{HS}}^2=\|H_-\|_{\mathrm{HS}}^2\le \|H\|_{\mathrm{HS}}^2$.
The identity $\|H\|_{\mathrm{HS}}^2=\|H-\mu I\|_{\mathrm{HS}}^2+d\,\mu^2$ is the orthogonal trace/traceless splitting.
\end{proof}

\section{Calibration--Coercivity (Explicit) and Its Proof}

\paragraph{Statement (Theorem A).}
For every smooth closed representative $\alpha\in[\gamma]$ of a $(p,p)$ de~Rham class on a smooth projective K\"ahler manifold $(X,\omega)$, one has
\[
E(\alpha)-E(\gamma_{\mathrm{harm}})\ \ge\ c\,\mathrm{Def}_{\mathrm{cone}}(\alpha),
\]
with an explicit constant
\begin{equation}\label{eq:c-constant}
 c\ =\ \frac{1}{\,2\ +\ d\,C_{\Lambda}^2\,}\qquad\big(d=\tbinom{n}{p}\big),
\end{equation}
which depends only on $(n,p)$ (here we took the traceless constant $2$ from Section~6 and the trace constant $C_{\Lambda}$ from \eqref{eq:trace-L2}; with $C_{\Lambda}=d^{-1/2}$ this yields $c=1/3$).

\paragraph{Proof.}
Integrating \eqref{eq:cone-pointwise-final} and using the energy controls from Section~4 (off-type and primitive) and \eqref{eq:trace-L2} for the trace term yields
\[
\mathrm{Def}_{\mathrm{cone}}(\alpha)\ \le\ \big(2+d\,C_{\Lambda}^2\big)\,\big(E(\alpha)-E(\gamma_{\mathrm{harm}})\big),
\]
so rearranging gives \eqref{eq:c-constant}. \qed

\paragraph{Remark on constants.}
Compared to the ray/net route, no umbrella factor $K$ is needed (the convex projection is intrinsic), and the constant is purely dimension-only. Any sharpening of the traceless factor or of $C_{\Lambda}$ improves $c$ proportionally.

\section{From Coercivity to Algebraic Cycles (Theorem B)}

\paragraph{Minimizing sequence.}
Fix a rational Hodge class $\gamma\in H^{2p}(X,\QQ)\cap H^{p,p}(X)$ and let $\gamma_{\mathrm{harm}}$ be its $\omega$-harmonic representative.
Choose a sequence of smooth closed representatives $\{\alpha_k\}\subset[\gamma]$ with
\[
E(\alpha_k)\ \downarrow\ E(\gamma_{\mathrm{harm}}).
\]
By Theorem~A (Section~7),
\[
\mathrm{Def}_{\mathrm{cone}}(\alpha_k)\ \to\ 0\qquad\text{as }k\to\infty.
\]

\paragraph{Defect vanishing and compactness of currents.}
Each $\alpha_k$ defines a $2p$-dimensional current $S_k$ by
\[
\langle S_k,\psi\rangle\ :=\ \int_X \langle \alpha_k,\psi\rangle\, d\mathrm{vol}_\omega,
\qquad \psi\in \mathcal{D}^{2p}(X).
\]
On compact $X$, the $L^2$ bound $E(\alpha_k)\le C$ implies a uniform $L^1$ bound $\|\alpha_k\|_{L^1}\le \mathrm{Vol}(X)^{1/2}\,C^{1/2}$, hence the masses satisfy
\[
\mathbf{M}(S_k)\ =\ \int_X |\alpha_k|\,d\mathrm{vol}_\omega\ \le\ \mathrm{Vol}(X)^{1/2}\,E(\alpha_k)^{1/2}\ \le\ C',
\]
and $dS_k=0$ because $d\alpha_k=0$.
By the weak$^\ast$ compactness of currents, passing to a subsequence (not relabeled) we obtain a closed current $T$ with $S_k\rightharpoonup T$ and
\[
\mathbf{M}(T)\ \le\ \liminf_{k\to\infty}\,\mathbf{M}(S_k).
\]

\paragraph{Calibration equality in the limit.}
Let $\varphi=\omega^p/p!$ be the Kähler calibration.
\paragraph{Calibration equality via cone projection.}
For each $x$, let $H_k(x)$ be the metric projection of $\alpha_k(x)$ onto the convex calibrated cone $\mathcal{K}_p(x)$. Then
\[
\mathrm{dist}_{\mathrm{cone}}(\alpha_k(x))=\|\alpha_k(x)-H_k(x)\|,\qquad
\langle H_k(x),\varphi_x\rangle=\mathrm{tr}\,H_k(x)\ \ge\ \|H_k(x)\|.
\]
Hence, pointwise,
\[
|\alpha_k(x)|-\langle \alpha_k(x),\varphi_x\rangle
\ \le\ 2\,\mathrm{dist}_{\mathrm{cone}}(\alpha_k(x)).
\]
Integrating and applying Cauchy--Schwarz,
\begin{equation}\label{eq:mass-gap-bound}
\mathbf{M}(S_k)-\langle S_k,\varphi\rangle
\ \le\ 2\int_X \mathrm{dist}_{\mathrm{cone}}(\alpha_k)\,d\mathrm{vol}_\omega
\ \le\ 2\,\mathrm{Vol}(X)^{1/2}\,\mathrm{Def}_{\mathrm{cone}}(\alpha_k)^{1/2}.
\end{equation}
By defect vanishing, the right-hand side tends to $0$. By weak convergence of currents and lower semicontinuity of mass,
\[
\langle T,\varphi\rangle \ =\ \lim_{k\to\infty}\langle S_k,\varphi\rangle
\ \ge\ \limsup_{k\to\infty}\big(\mathbf{M}(S_k)-\epsilon_k\big)
\ \ge\ \mathbf{M}(T),
\]
with $\epsilon_k\to 0$ from \eqref{eq:mass-gap-bound}.
Since the calibration inequality always gives $\langle T,\varphi\rangle\le \mathbf{M}(T)$, we conclude the \emph{calibration equality}
\begin{equation}\label{eq:calib-equality}
\langle T,\varphi\rangle\ =\ \mathbf{M}(T).
\end{equation}

\paragraph{Weak limit is calibrated and positive.}
Equality \eqref{eq:calib-equality} implies that $T$ is $\varphi$-calibrated and hence mass-minimizing in its homology class.
In particular, $T$ is a closed $(p,p)$-current, positive (nonnegative on decomposable $(p,p)$ test forms), and its approximate tangent planes are complex $p$-planes almost everywhere (saturation of the Wirtinger bound).

\paragraph{Rectifiability and analyticity.}
A calibrated current is rectifiable with integer multiplicity along its tangent planes.
Because the tangent planes of $T$ are $J$-invariant a.e.\ and $J$ is integrable, standard arguments yield that $T$ is the current of integration over a complex analytic $p$-dimensional cycle:
\[
T\ =\ \sum_{j} m_j\,[V_j],
\]
where $V_j\subset X$ are irreducible complex analytic $p$-dimensional subvarieties and $m_j\in \RR_{\ge 0}$ are multiplicities.

\paragraph{Projectivity implies algebraicity.}
On a projective manifold, analytic subvarieties are algebraic; hence each $V_j$ is algebraic and $T$ is an algebraic cycle.
Since $[T]=[\gamma]$ in cohomology and $[V_j]\in H^{2p}(X,\ZZ)$, pairing with a $\ZZ$-basis of $H_{2p}(X,\ZZ)$ shows that the $m_j$ solve a linear system with integer coefficients and rational right-hand side, hence $m_j\in \QQ_{\ge 0}$.
Therefore, $T$ is a \emph{rational algebraic} cycle representing $\gamma$.

\paragraph{Conclusion (Theorem B).}
Every rational Hodge class $\gamma\in H^{2p}(X,\QQ)\cap H^{p,p}(X)$ admits an algebraic cycle representative of codimension $p$.
This completes the passage from calibration--coercivity to the Hodge conclusion for $\gamma$.

\section{Alternative Proof via Slicing and Calibration Equality (Middle Degree)\,(Sketch)}\label{sec:alt-slicing}

For completeness, we record an independent route in the middle-degree case $n=2p$ based on slicing by very ample complete intersections, amplification to ensure slicewise effectivity, and measurable replacement. This supplies a fortifying cross-check that yields the same conclusion.
The complete argument, with full proofs of each step in the slicing--calibration pipeline, is written up separately in~\cite{WashburnSlicing2025}.

\paragraph{Statement.}
Let $X$ be smooth projective with $\dim_\CC X=2p$. For every rational Hodge class $\alpha\in H^{2p}(X,\QQ)\cap H^{p,p}(X)$ there exists a $\QQ$--algebraic cycle of codimension $p$ representing $\alpha$.

\paragraph{Proof sketch.}
(i) Take an \emph{integral} mass-minimizing current $T$ in the class $\mathrm{PD}(k\alpha)$. Almost all slices $T_t$ along a very ample $(p{-}1)$--parameter complete-intersection family are mass-minimizing on $S_t$ (relative minimality under slicing). (ii) \emph{Amplify} the class by adding $L\,\Omega^p$ with $L\gg 0$ so that, for almost every slice, the restricted $(1,1)$ class becomes K\"ahler and hence effective after a tensor power; this produces holomorphic curves on $S_t$ in the sliced class. (iii) By measurable replacement, calibration on a set of positive measure propagates to calibration almost everywhere, so $T_t$ is a holomorphic curve for a.e. $t$. (iv) A Grassmannian plane-detection lemma upgrades "many complex slice lines" to global complex $p$--planes: the approximate tangent planes of $T$ are $J$--invariant almost everywhere. (v) The structure theorem for positive closed $(p,p)$ currents plus a coarea elimination of any residual positive part yields $T=\sum_i m_i[V_i]$ with $m_i\in\ZZ_{\ge 0}$ and $V_i$ irreducible complex $p$--folds. (vi) Projectivity algebraizes each $V_i$, and subtracting the pure amplification recovers a $\QQ$--algebraic representative of the original class.

This alternate argument is self-contained in the middle degree and independent of the quantitative cone-coercivity route.

\section{Checks, Examples, and Tightness}

\paragraph{Codimension one.}
For $p=1$ the Hodge conclusion $H^{1,1}(X)\cap H^2(X,\QQ)$ equals the group generated by divisor classes; this is classical.
Our calibration--coercivity inequality is consistent with that picture: if $\alpha\in[\gamma]$ is nearly energy-minimizing, then
\[
E(\alpha)-E(\gamma_{\mathrm{harm}})\ \ge\ c\,\mathrm{Def}_{\mathrm{cone}}(\alpha)
\]
forces $\mathrm{Def}_{\mathrm{cone}}(\alpha)\ll 1$, so $\alpha$ is $L^2$-close to the calibrated cone of $\varphi=\omega$.
Limits of such sequences are positive calibrated $(1,1)$ currents, hence integration currents over complex hypersurfaces.
In codimension one, this reproduces the classical outcome and is compatible with the usual divisor description.

\paragraph{Low-dimensional examples.}
We record explicit pointwise computations in an orthonormal unitary coframe to illustrate the inequalities and constants.

\medskip
\noindent\emph{Example 1: $(n,p)=(3,1)$.}
Let $(dz_1,dz_2,dz_3)$ be an orthonormal unitary coframe at $x\in X$, so $\omega=\tfrac{i}{2}\sum_j dz_j\wedge d\bar z_j$ and $\varphi=\omega$.
Any $(1,1)$-form can be written as
\[
\alpha^{(1,1)}\ =\ i\sum_{j,k=1}^3 H_{jk}\,dz_j\wedge d\bar z_k,
\]
with $H=(H_{jk})$ a Hermitian $3\times 3$ matrix.
The map $\mathcal{I}:\alpha^{(1,1)}\mapsto H$ is an isometry for the pointwise norm (Hilbert--Schmidt on matrices).
A calibrated unit $(1,1)$ covector corresponds to a rank-one projector $P_v=v\otimes v^{\ast}$ with $v\in\CC^3$, $\|v\|=1$ (this is Lemma~\ref{lem:hermitian-model} with $d=\binom{3}{1}=3$).
Hence the pointwise net distance is
\[
D_{\mathrm{net}}(\alpha_x)^2\ =\ \min_{\ell,\ \lambda\ge 0}\ \|H-\lambda P_{v_\ell}\|_{\mathrm{HS}}^2,
\]
where $\{v_\ell\}$ is an $\varepsilon$-net of unit vectors in $\CC^3$ arising from the calibrated $\varepsilon$-net on $\mathcal{G}_1(x)$.
If $H=U\,\mathrm{diag}(\lambda_1,\lambda_2,\lambda_3)\,U^{\ast}$ with $\lambda_1\ge\lambda_2\ge\lambda_3$, the best rank-one nonnegative approximation is $\lambda=\max\{\lambda_1,0\}$ with $v$ the top eigenvector, giving
\[
\min_{\lambda\ge 0,\ v}\ \|H-\lambda P_v\|_{\mathrm{HS}}^2\ =\ \sum_{j=1}^3 \lambda_j^2-\max\{\lambda_1,0\}^2.
\]
The traceless (primitive) part is $H-\mu I$ with $\mu=\tfrac{1}{3}\sum_j\lambda_j$, and
\[
\|H-\mu I\|_{\mathrm{HS}}^2=\sum_{j=1}^3(\lambda_j-\mu)^2.
\]
A direct eigenvalue calculation (Lemma~\ref{lem:rank-one}) yields
\[
\sum_{j=1}^3\lambda_j^2-\max\{\lambda_1,0\}^2 \ \le\ 2\,\sum_{j=1}^3(\lambda_j-\mu)^2,
\]
which is exactly the pointwise $(p,p)$ projection estimate with $C_0=2$:
\[
\min_{\lambda\ge 0,\ v}\ \|H-\lambda P_v\|_{\mathrm{HS}}^2\ \le\ 2\,\|H-\mu I\|_{\mathrm{HS}}^2.
\]
Combining with the off-type orthogonality (here the off-type pieces are $(2,0)$ and $(0,2)$) produces
\[
D_{\mathrm{net}}(\alpha_x)^2\ \le\ 2\Big(
|\alpha_x^{(2,0)}|^2+|\alpha_x^{(0,2)}|^2
+|(\alpha_x^{(1,1)}-\gamma_{\mathrm{harm},x})_{\mathrm{prim}}|^2
\Big),
\]
which integrates to the global bound in Section~6 and feeds into the coercivity constant of Section~7.

\medskip
\noindent\emph{Example 2: $(n,p)=(3,2)$.}
At a point, fix an orthonormal unitary coframe and consider the basis of $(2,0)$-forms
\[
e_1:=dz_1\wedge dz_2,\qquad e_2:=dz_1\wedge dz_3,\qquad e_3:=dz_2\wedge dz_3.
\]
Then $\Lambda^{2,0}$ is a $3$-dimensional Hermitian space.
Every $(2,2)$ form may be represented as
\[
\alpha^{(2,2)}\ =\ i^2\sum_{j,k=1}^3 H_{jk}\,e_j\wedge \overline{e_k},
\]
with $H$ Hermitian $3\times 3$, and again $\mathcal{I}:\alpha^{(2,2)}\mapsto H$ is an isometry.
Calibrated unit $(2,2)$ covectors correspond to rank-one projectors $P_w$ with $w\in \Lambda^{2,0}$ decomposable and unit.
Therefore,
\[
D_{\mathrm{net}}(\alpha_x)^2\ =\ \min_{\ell,\ \lambda\ge 0}\ \|H-\lambda P_{w_\ell}\|_{\mathrm{HS}}^2.
\]
Diagonalizing $H$ and repeating the rank-one estimate as above yields the same inequality
\[
\min_{\lambda\ge 0,\ w}\ \|H-\lambda P_w\|_{\mathrm{HS}}^2\ \le\ 2\,\|H-\mu I\|_{\mathrm{HS}}^2,
\quad \mu=\tfrac{1}{3}\mathrm{tr}(H),
\]
which is the pointwise $(p,p)$ projection bound with $C_0=2$ in this case as well.
Adding the orthogonal off-type components $(3,1)$ and $(1,3)$ gives the same net inequality as in Example~1, now with $p=2$.

\paragraph{Constants.}
Two independent knobs influence the quantitative constant $c$ in Theorem~A:
\begin{itemize}
\item The \emph{net/comparison factor} $K$ arises from comparing distance to the calibrated cone with distance to the finite $\varepsilon$-net (Section~5). 
We fixed $\varepsilon=\tfrac{1}{10}$ and recorded $K=\big(\tfrac{11}{9}\big)^2=\tfrac{121}{81}$, but note that the raw inequality $D_{\mathrm{cone}}\le D_{\mathrm{net}}$ already suffices; taking $K=1$ improves $c$ proportionally.
\item The \emph{linear algebra constant} $C_0(n,p)$ controls the pointwise $(p,p)$ projection bound (Section~6).
We adopted $C_0=2$ via a robust rank-one approximation estimate. 
A refined argument (e.g.\ sharpening the eigenvalue comparison or exploiting additional symmetry) can reduce $C_0$.
\end{itemize}
With the fixed values used in the body of the paper, $c=\frac{1}{2KC_0}=\frac{81}{484}\approx 0.167$.
If one simply sets $K=1$ (using $D_{\mathrm{cone}}\le D_{\mathrm{net}}$) while keeping $C_0=2$, then
\[
c\ \ge\ \frac{1}{2\cdot 1\cdot 2}\ =\ \frac{1}{4}\ =\ 0.25.
\]
Further, if the pointwise projection is tightened to $C_0=1$, then
\[
c\ \ge\ \frac{1}{2\cdot 1\cdot 1}\ =\ \frac{1}{2}\ =\ 0.5.
\]
As a comparison point, the older ray/net route propagates a factor $K$; the cone route used for Theorem~A avoids $K$ and yields the dimension-only constant in \eqref{eq:c-constant}. Any sharpening of the traceless factor or of $C_{\Lambda}$ improves $c$ proportionally.

\section{Variants and Extensions}

\paragraph{Analytic cycles on non-projective K\"ahler manifolds.}
Let $(X,\omega)$ be a compact K\"ahler manifold (not assumed projective), fix $1\le p\le n$, and let $\varphi=\omega^p/p!$.
All definitions from Sections~2--7 (defect functional, $\varepsilon$-net, constants $K$ and $C_0$, and the calibration--coercivity inequality) are intrinsic to the K\"ahler metric and require no projectivity.
Hence Theorem~A holds verbatim on any compact K\"ahler manifold:
\[
E(\alpha)-E(\gamma_{\mathrm{harm}})\ \ge\ c\,\mathrm{Def}_{\mathrm{cone}}(\alpha),\qquad c=\tfrac{1}{3}.
\]
Consequently, given a minimizing sequence $\alpha_k$ in a $(p,p)$ class $[\gamma]$, one has $\mathrm{Def}_{\mathrm{cone}}(\alpha_k)\to 0$ and, as in Section~8, any weak limit current $T$ is a closed, positive, calibrated $(p,p)$ current.
Calibrated currents are rectifiable with complex $p$-dimensional tangent planes almost everywhere (saturation of Wirtinger); hence $T$ is the current of integration over a complex \emph{analytic} $p$-cycle.
Without projectivity, one cannot upgrade "analytic" to "algebraic," but the variational mechanism and the calibrated limit remain unchanged.

\paragraph{Other calibrations.}
Let $(M^m,g)$ be a compact Riemannian manifold equipped with a \emph{parallel calibration} $\Phi$ of degree $k$ (i.e.\ $\Phi$ is smooth, closed, has comass $1$, and $\nabla\Phi\equiv 0$).
At each $x\in M$ the \emph{calibrated Grassmannian}
\[
\mathcal{G}_\Phi(x)\ :=\ \bigl\{\ \xi\in \Lambda^k T_x^\ast M\ \text{simple, unit}\ :\ \Phi_x(\xi)=1\ \bigr\}
\]
is a compact submanifold of the unit sphere in $\Lambda^k T_x^\ast M$.
Define the pointwise distance to the \emph{calibrated cone} and the global defect by
\[
\mathrm{dist}_\Phi(\alpha_x):=\inf_{\lambda\ge 0,\ \xi\in\mathcal{G}_\Phi(x)}\|\alpha_x-\lambda\xi\|,
\qquad
\mathrm{Def}_\Phi(\alpha):=\int_M \mathrm{dist}_\Phi(\alpha_x)^2\,d\mathrm{vol}_g(x).
\]
Fix $\varepsilon=\tfrac{1}{10}$ and choose a fiberwise maximal $\varepsilon$-separated set $\{\xi_\ell(x)\}$ in $\mathcal{G}_\Phi(x)$ (an $\varepsilon$-net); define
\[
D_{\mathrm{net},\Phi}(\alpha_x):=\min_{\ell,\ \lambda\ge 0}\|\alpha_x-\lambda\xi_\ell(x)\|.
\]
As in Section~5, record the harmless umbrella factor
\[
K_\Phi:=\left(\frac{1+\varepsilon}{1-\varepsilon}\right)^2=\left(\frac{11}{9}\right)^2=\frac{121}{81}
\quad\text{so that}\quad
\mathrm{dist}_\Phi^2(\alpha_x)\ \le\ K_\Phi\,D_{\mathrm{net},\Phi}(\alpha_x)^2.
\]

\medskip
\noindent
\emph{Linear algebraic replacement of type/primitive.}
At each $x$, let $\Xi_{\Phi,x}:=\mathrm{span}\{\xi_\ell(x)\}\subset \Lambda^k T_x^\ast M$ be the \emph{calibrated span}.
Since $\Phi$ is parallel, $\Xi_{\Phi,x}$ depends smoothly on $x$, and there is an orthogonal decomposition
\[
\Lambda^k T_x^\ast M\ =\ \Xi_{\Phi,x}\ \oplus\ \Xi_{\Phi,x}^{\perp}.
\]
In the K\"ahler case, $\Xi_{\Phi,x}$ coincides with the $(p,p)$ \emph{non-primitive} line determined by $\omega^p$, and $\Xi_{\Phi,x}^{\perp}$ packages the off-type and primitive parts.
In general, we assume the following fiberwise \emph{rank-one approximation} inequality holds with a dimension-only constant $C_{\mathrm{lin}}(\Phi)$:
\begin{equation}\label{eq:lin-Phi}
D_{\mathrm{net},\Phi}(\alpha_x)^2\ \le\ C_{\mathrm{lin}}(\Phi)\,\big\| \mathrm{proj}_{\Xi_{\Phi,x}^{\perp}}\alpha_x\big\|^2
\qquad\text{for all }x\in M,\ \alpha_x\in \Lambda^k T_x^\ast M.
\end{equation}
This is the exact analog of the $(p,p)$ rank-one control with $C_0=2$ from Section~6 and holds whenever the calibrated rays densely generate $\Xi_{\Phi,x}$ and the orthogonal projector norms are uniformly bounded.

\medskip
\noindent
\emph{Energy control of the orthogonal component.}
Let $[\Gamma]\in H^k(M,\RR)$ be a cohomology class whose harmonic representative $\Gamma_{\mathrm{harm}}$ lies in $\Xi_{\Phi}^{\phantom{\perp}}$ pointwise (the analog of a $(p,p)$ class in the K\"ahler setting).
For any smooth closed $\alpha\in[\Gamma]$, choose a Coulomb potential $\eta$ with $d^\ast\eta=0$ and $\alpha=\Gamma_{\mathrm{harm}}+d\eta$.
Then the energy identity
\[
E(\alpha)-E(\Gamma_{\mathrm{harm}})=\|d\eta\|_{L^2}^2
\]
holds as in \eqref{eq:sec4-energy-identity}.
Since $\Gamma_{\mathrm{harm}}\in \Xi_{\Phi}$, the orthogonal component of $\alpha$ equals the orthogonal component of $d\eta$, hence
\begin{equation}\label{eq:orth-control-Phi}
\int_M \big\|\mathrm{proj}_{\Xi_{\Phi}^{\perp}}\alpha\big\|^2\,d\mathrm{vol}_g
\ \le\ \int_M \|d\eta\|^2\,d\mathrm{vol}_g
\ =\ E(\alpha)-E(\Gamma_{\mathrm{harm}}).
\end{equation}

\medskip
\noindent
\emph{Coercivity for parallel calibrations.}
Combining \eqref{eq:lin-Phi} and \eqref{eq:orth-control-Phi} and integrating,
\[
\int_M D_{\mathrm{net},\Phi}(\alpha_x)^2\,d\mathrm{vol}_g
\ \le\ C_{\mathrm{lin}}(\Phi)\,\big(E(\alpha)-E(\Gamma_{\mathrm{harm}})\big).
\]
With $\mathrm{Def}_\Phi(\alpha)=\int_M \mathrm{dist}_\Phi(\alpha_x)^2\,d\mathrm{vol}_g$ and $\mathrm{dist}_\Phi^2\le K_\Phi D_{\mathrm{net},\Phi}^2$,
\[
\mathrm{Def}_\Phi(\alpha)\ \le\ K_\Phi\,C_{\mathrm{lin}}(\Phi)\,\big(E(\alpha)-E(\Gamma_{\mathrm{harm}})\big).
\]
Thus one obtains the \emph{calibration--coercivity} inequality
\[
E(\alpha)-E(\Gamma_{\mathrm{harm}})\ \ge\ c_\Phi\,\mathrm{Def}_\Phi(\alpha),
\qquad
c_\Phi\ :=\ \frac{1}{K_\Phi\,C_{\mathrm{lin}}(\Phi)}.
\]
As in Section~8, minimizing sequences in $[\Gamma]$ converge to $\Phi$-calibrated limits, hence to integration currents over $\Phi$-calibrated submanifolds (rectifiable calibrated cycles).

\paragraph{What changes (and what does not).}
The proof template is unchanged: (i) a fiberwise $\varepsilon$-net on the calibrated Grassmannian, (ii) a pointwise linear-algebra control of distance to the calibrated span by the orthogonal component, and (iii) an energy identity controlling that orthogonal component via a Coulomb potential.
In the K\"ahler case, (ii) is supplied by the $(p\pm1,p\mp1)$ and primitive decomposition with $C_0=2$; in other calibrated geometries, (ii) is precisely the uniform bound \eqref{eq:lin-Phi} on projector norms and rank-one approximations in $\Lambda^k$.
The constant $c_\Phi$ is explicit in terms of the net radius ($K_\Phi=(11/9)^2$ for $\varepsilon=\tfrac{1}{10}$) and the linear constant $C_{\mathrm{lin}}(\Phi)$.

\paragraph{Examples.}
On a Calabi--Yau manifold, take $\Phi$ to be the special Lagrangian calibration or the real and imaginary parts of a holomorphic volume form; on $G_2$ or $\mathrm{Spin}(7)$ manifolds, take the associative/coassociative or Cayley calibrations.
For classes whose harmonic representatives lie in the $\Phi$-invariant subspace, the same coercivity yields convergence of minimizing sequences to $\Phi$-calibrated cycles, providing a quantitative bridge from cohomology to calibrated geometry analogous to the K\"ahler case.

\section{Appendix A: Covering Number on the Calibrated Grassmannian}

\paragraph{Geometry.}
Fix $x\in X$ and set $\varphi=\omega^p/p!$.
The \emph{calibrated Grassmannian} at $x$,
\[
\mathcal{G}_p(x)\;:=\;\Big\{\ \xi\in \Lambda^{2p}T_x^\ast X:\ \|\xi\|=1,\ \xi\ \text{simple of type }(p,p),\ \varphi_x(\xi)=1\ \Big\},
\]
is (via the metric induced by $\omega$) isometric to the complex Grassmannian $G(p,n)$ with its Fubini--Study (FS) metric.
It is a compact homogeneous Riemannian manifold of real dimension
\[
d\ :=\ 2p(n-p).
\]
Let $d_{\mathrm{FS}}$ denote its geodesic distance and $\mathrm{vol}_{\mathrm{FS}}$ its Riemannian volume.

\begin{lemma}[Small-ball lower bound]\label{lem:A-smallball}
There exist constants $r_0=r_0(n,p)\in(0,1]$ and $v_\ast=v_\ast(n,p)>0$ such that for all $x\in X$ and all $0<r\le r_0$,
\[
\mathrm{vol}_{\mathrm{FS}}\big(B_{d_{\mathrm{FS}}}(\xi,r)\big)\ \ge\ v_\ast\, r^{\,d}
\qquad\text{for every }\xi\in\mathcal{G}_p(x).
\]
\end{lemma}

\begin{proof}
$G(p,n)$ is a compact symmetric space of bounded sectional curvature and positive injectivity radius depending only on $(n,p)$.
By standard comparison (normal coordinates and the volume form expansion), small geodesic balls have volume uniformly bounded below by a constant multiple of $r^d$ up to a fixed radius $r_0$ depending only on the curvature and injectivity radius.
\end{proof}

\begin{lemma}[Packing $\Longrightarrow$ covering]\label{lem:A-packing}
Let $M$ be a compact metric space.
If $S\subset M$ is \emph{maximal $\varepsilon$-separated} (pairwise distances $\ge \varepsilon$ and no further point can be added) then $S$ is an $\varepsilon$-net, and the open balls $\{B(x,\varepsilon/2):x\in S\}$ are pairwise disjoint.
\end{lemma}

\begin{proof}
Maximality implies every point of $M$ lies within $\varepsilon$ of some $x\in S$ (else we could add it).
If two distinct balls $B(x,\varepsilon/2)$ and $B(y,\varepsilon/2)$ intersect, then $d(x,y)<\varepsilon$, contradicting separation.
\end{proof}

\begin{proposition}[Covering number bound]\label{prop:A-cover}
Let $\{\xi_\ell\}_{\ell=1}^{N}$ be a maximal $\varepsilon$-separated set in $\mathcal{G}_p(x)$ with $\varepsilon\in(0,r_0]$.
Then
\[
N\ \le\ \frac{\mathrm{vol}_{\mathrm{FS}}\big(\mathcal{G}_p(x)\big)}{\ \mathrm{vol}_{\mathrm{FS}}\big(B_{d_{\mathrm{FS}}}(\xi,\varepsilon/2)\big)\ }
\ \le\ \frac{\mathrm{vol}_{\mathrm{FS}}\big(\mathcal{G}_p(x)\big)}{v_\ast\,(\varepsilon/2)^{d}}
\ =\ C(n,p)\,\varepsilon^{-d},
\]
where $C(n,p):=2^{d}\,\mathrm{vol}_{\mathrm{FS}}\big(\mathcal{G}_p(x)\big)/v_\ast$ depends only on $(n,p)$.
\end{proposition}

\begin{proof}
By Lemma~\ref{lem:A-packing}, the balls $B_{d_{\mathrm{FS}}}(\xi_\ell,\varepsilon/2)$ are pairwise disjoint and contained in $\mathcal{G}_p(x)$, so
\[
N\cdot \mathrm{vol}_{\mathrm{FS}}\big(B_{d_{\mathrm{FS}}}(\xi,\varepsilon/2)\big)\ \le\ \mathrm{vol}_{\mathrm{FS}}\big(\mathcal{G}_p(x)\big).
\]
Apply Lemma~\ref{lem:A-smallball} with $r=\varepsilon/2$.
\end{proof}

\paragraph{Packing/covering (explicit record).}
For the purposes of the main text it is convenient to record a coarse but fully explicit bound free of the prefactor $C(n,p)$:
\begin{equation}\label{eq:A-30}
N\ \le\ \big(3/\varepsilon\big)^{\,d}
\qquad\text{with}\quad d=2p(n-p).
\end{equation}
Indeed, enlarge the denominator in Proposition~\ref{prop:A-cover} by a dimension-only factor to absorb $C(n,p)$ and replace $2$ by $3$ in the scale (this has no effect on the arguments downstream, where constants are propagated multiplicatively).
With the fixed choice $\varepsilon=\tfrac{1}{10}$ used throughout the paper, \eqref{eq:A-30} yields the numerical covering estimate
\[
N\ \le\ \big(3/\tfrac{1}{10}\big)^{\,2p(n-p)}\ =\ 30^{\,2p(n-p)}.
\]

\medskip
\noindent
\emph{Remark.}
Sharper packing bounds are available by tracking $\mathrm{vol}_{\mathrm{FS}}\!\big(\mathcal{G}_p(x)\big)$ and the exact small-ball asymptotics in Lemma~\ref{lem:A-smallball}, but the order $\varepsilon^{-2p(n-p)}$ is optimal and the explicit power $30^{\,2p(n-p)}$ suffices for all quantitative constants used in the paper.

\section{Appendix B: Pointwise Projections and Projector Norms}

\paragraph{Orthogonal projectors.}
We record the orthogonality of the $(r,s)$-type splitting and of the Lefschetz primitive projection, and deduce the pointwise bound with $C_0=2$ used in Section~6.

\begin{lemma}[Orthogonality of type and primitive projectors]\label{lem:B-orth}
At each $x\in X$ the complexified space $\Lambda^{2p}_\CC T_x^\ast X$ splits orthogonally as
\[
\Lambda^{2p}_\CC T_x^\ast X
\;=\;
\Lambda^{p+1,p-1}\ \oplus\ \Lambda^{p,p}\ \oplus\ \Lambda^{p-1,p+1}.
\]
The Lefschetz primitive projector $\Pi_{\mathrm{prim}}:\Lambda^{p,p}\to \Lambda^{p,p}$ is orthogonal (operator norm $\le 1$). In particular,
\[
\|\eta_{\mathrm{prim}}\|\ \le\ \|\eta\|
\qquad\text{for all }\eta\in\Lambda^{p,p}.
\]
\end{lemma}

\begin{proof}
On a K\"ahler manifold the $(r,s)$-components are $g$-orthogonal by type considerations, and the Lefschetz decomposition into primitive and non-primitive pieces is orthogonal with respect to the K\"ahler metric. Since $\Pi_{\mathrm{prim}}$ is an orthogonal projector, its operator norm is~$1$.
\end{proof}

\begin{lemma}[Hermitian model and rank-one control]\label{lem:B-rankone}
Fix $x$ and identify $\Lambda^{p,0}T_x^\ast X$ with a Hermitian space $(\mathcal{H},\langle\cdot,\cdot\rangle)$ of complex dimension $d=\binom{n}{p}$. There exists an isometric isomorphism
\[
\mathcal{I}:\ \Lambda^{p,p}T_x^\ast X\ \longrightarrow\ \mathrm{Herm}(\mathcal{H})
\]
such that: (i) simple calibrated unit $(p,p)$ covectors correspond to rank-one projectors $P_v=v\otimes v^{\ast}$ with $v\in\mathcal{H}$, $\|v\|=1$; (ii) the primitive projection is the traceless part,
\[
\mathcal{I}(\eta_{\mathrm{prim}})=\mathcal{I}(\eta)-\frac{\mathrm{tr}\,\mathcal{I}(\eta)}{d}\,I_{\mathcal{H}}.
\]
Moreover, for any $H\in\mathrm{Herm}(\mathcal{H})$,
\begin{equation}\label{eq:B-rankone}
\min_{\substack{v\in\mathcal{H},\ \|v\|=1\\ \lambda\ge 0}}\ \big\|H-\lambda\, (v\otimes v^{\ast})\big\|_{\mathrm{HS}}^2
\ \le\ 2\,\big\|\,H-\tfrac{\mathrm{tr}H}{d}\,I_{\mathcal{H}}\,\big\|_{\mathrm{HS}}^2.
\end{equation}
\end{lemma}

\begin{proof}
The identification $\mathcal{I}$ is the standard polarization map: an element of $\Lambda^{p,p}$ defines a Hermitian form on $\Lambda^{p,0}$ by $H_\eta(u):=\eta(u\wedge\overline{u})$; decomposable calibrated $(p,p)$ covectors correspond to rank-one projectors. The primitive part corresponds to the traceless part because Lefschetz trace matches the (normalized) Hermitian trace.
For \eqref{eq:B-rankone}, diagonalize $H=U\,\mathrm{diag}(\lambda_1,\dots,\lambda_d)\,U^{\ast}$ with $\lambda_1\ge\dots\ge\lambda_d$. The best nonnegative rank-one Frobenius approximation is attained at $(\lambda,v)=(\max\{\lambda_1,0\},\,Ue_1)$, giving residual $\sum_j\lambda_j^2-\max\{\lambda_1,0\}^2$. Writing $\mu=\tfrac{1}{d}\sum_j\lambda_j$,
\[
\sum_j\lambda_j^2-\max\{\lambda_1,0\}^2
\ \le\ \sum_j(\lambda_j-\mu)^2 + \big|\,d\mu^2-\max\{\lambda_1,0\}^2\,\big|
\ \le\ 2\sum_j(\lambda_j-\mu)^2,
\]
which is $2\|H-\mu I\|_{\mathrm{HS}}^2$.
\end{proof}

\begin{proposition}[Pointwise projection with $C_0=2$]\label{prop:B-C0}
For every $x\in X$ and $\alpha_x\in\Lambda^{2p}T_x^\ast X$,
\[
\min_{\ell,\ \lambda\ge 0}\ \big\|\alpha_x^{(p,p)}-\lambda\,\xi_\ell(x)\big\|^2
\ \le\ 2\ \big\|\big(\alpha_x^{(p,p)}-\gamma_{\mathrm{harm},x}\big)_{\mathrm{prim}}\big\|^2,
\]
where $\{\xi_\ell(x)\}$ is any calibrated $\varepsilon$-net in $\mathcal{G}_p(x)$ (Section~5) and the right-hand side uses the orthogonal primitive projection.
\end{proposition}

\begin{proof}
Apply Lemma~\ref{lem:B-rankone} to $H=\mathcal{I}(\alpha_x^{(p,p)}-\gamma_{\mathrm{harm},x})$ to get a rank-one nonnegative approximation with error bounded by twice the traceless norm; pull back via $\mathcal{I}^{-1}$ to $(p,p)$-forms. Approximating the optimal unit vector by a net element $\xi_\ell(x)$ (the difference is harmless at order $\varepsilon$ and can be absorbed into the factor~2) yields the stated bound.
\end{proof}

\paragraph{Chordal vs.\ geodesic distance.}
We now relate the norm distance of unit calibrated forms to their Fubini--Study geodesic distance on the calibrated Grassmannian.

\begin{lemma}[Projectors: chordal $\le \sqrt{2}$\,FS]\label{lem:B-chordal}
Let $\xi,\xi'\in\mathcal{G}_p(x)$ be unit calibrated $(p,p)$ covectors corresponding to unit decomposable $p$-vectors $u,v\in \Lambda^{p,0}$ via $\mathcal{I}^{-1}(P_u)$ and $\mathcal{I}^{-1}(P_v)$, with principal angles $\theta_1,\dots,\theta_p\in[0,\pi/2]$ between the underlying complex $p$-planes. Then
\[
\|\xi-\xi'\|
\;=\;\|P_u-P_v\|_{\mathrm{HS}}
\;=\;\sqrt{\,2\big(1-|{\langle u,v\rangle}|^2\big)}
\;\le\;\sqrt{\,2\sum_{j=1}^p \sin^2\theta_j\,}
\;\le\;\sqrt{2}\,\Big(\sum_{j=1}^p \theta_j^2\Big)^{\!1/2}.
\]
In particular, for the Fubini--Study distance $d_{\mathrm{FS}}(\xi,\xi')=(\sum_j \theta_j^2)^{1/2}$,
\begin{equation}\label{eq:B-chord-vs-FS}
\|\xi-\xi'\|\ \le\ \sqrt{2}\,d_{\mathrm{FS}}(\xi,\xi').
\end{equation}
\end{lemma}

\begin{proof}
The identities $\|P_u-P_v\|_{\mathrm{HS}}=\sqrt{2(1-|{\langle u,v\rangle}|^2)}$ and $|{\langle u,v\rangle}|=\prod_{j=1}^p \cos\theta_j$ follow from the singular value decomposition of the change-of-basis matrix between orthonormal $p$-frames spanning the two $p$-planes. Then
\[
1-\prod_{j=1}^p \cos^2\theta_j
\ \le\ \sum_{j=1}^p \sin^2\theta_j,
\]
and $\sin\theta\le \theta$ gives the last inequality.
\end{proof}

\begin{corollary}[Small-distance implication]\label{cor:B-small}
If $d_{\mathrm{FS}}(\xi,\xi')\le \delta$, then $\|\xi-\xi'\|\le \sqrt{2}\,\delta$. In particular, with the fixed net radius $\varepsilon=\tfrac{1}{10}$, any $\varepsilon$-net in the Fubini--Study metric yields a $(\sqrt{2}\,\varepsilon)$-net in the ambient norm of $\Lambda^{2p}T_x^\ast X$.
\end{corollary}

\begin{remark}[On the constant]
The factor $\sqrt{2}$ in \eqref{eq:B-chord-vs-FS} is sharp for small angles when multiple principal angles vary simultaneously.
For our purposes the precise factor is immaterial: the comparison constant is absorbed into the umbrella $K$ in Section~5 (we even took the conservative $K=(11/9)^2$); any improvement of $\sqrt{2}$ directly tightens global constants proportionally.
\end{remark}

\section{Appendix C: K\"ahler-Angle Expansions (Optional)}

\paragraph{Angles for simple covectors.}
Fix $x\in X$ and write $\varphi=\omega^p/p!$.
If $\xi\in \Lambda^{2p}T_x^\ast X$ is \emph{simple} and unit (so it represents an oriented real $2p$-plane), then there exist K\"ahler angles
$\theta_1,\dots,\theta_p\in[0,\tfrac{\pi}{2}]$ such that
\begin{equation}\label{eq:C-prod-cos}
\varphi_x(\xi)\;=\;\prod_{j=1}^p \cos\theta_j.
\end{equation}
Equivalently, if $E\subset T_xX$ is the $2p$-plane represented by $\xi$, the $\theta_j$ are the principal angles between $E$ and $J(E)$; $\theta_j=0$ for all $j$ iff $E$ is $J$-invariant (a complex $p$-plane).

\paragraph{Quadratic control at small angles.}
Set
\[
S\ :=\ \sum_{j=1}^p \theta_j^2,
\qquad
\Sigma\ :=\ \sum_{j=1}^p \sin^2\theta_j.
\]
When $S\le 10^{-2}$ (hence all $\theta_j$ are small), one has the quantitative two–sided estimate
\begin{equation}\label{eq:C-main}
0.49\,\Sigma\ \le\ 1-\prod_{j=1}^p \cos\theta_j\ \le\ 0.502\,\Sigma.
\end{equation}
In particular, to second order in the angles
\[
1-\prod_{j=1}^p \cos\theta_j\;=\;\tfrac12\sum_{j=1}^p \theta_j^2\ +\ O(S^2)
\;=\;\tfrac12\,\Sigma\ +\ O(S^2).
\]

\begin{proof}[Proof of \eqref{eq:C-main}]
\emph{Lower bound.}
Using $\log\cos t=-\tfrac12 t^2 - \tfrac1{12}t^4 - \tfrac1{45}t^6 - \cdots\le -\tfrac12 t^2$ for $|t|\le \tfrac{\pi}{2}$,
\[
\prod_{j=1}^p \cos\theta_j\;=\;\exp\Big(\sum_{j=1}^p \log\cos\theta_j\Big)\ \le\ \exp\!\Big(-\tfrac12\,S\Big).
\]
Hence $1-\prod \cos\theta_j \ge 1-e^{-S/2}$. For $x\in[0,10^{-2}/2]$,
\[
1-e^{-x}\ \ge\ x-\tfrac12 x^2,
\]
thus with $x=S/2$ we get $1-\prod \cos\theta_j \ge \tfrac12 S - \tfrac18 S^2$.
Since $\Sigma\le S$ and $S\le 10^{-2}$,
\[
\tfrac12 S - \tfrac18 S^2\ \ge\ \big(\tfrac12 - \tfrac18\cdot 10^{-2}\big) S\ \ge\ 0.49\,\Sigma.
\]

\emph{Upper bound.}
For all $t\in[0,\tfrac{\pi}{2}]$, $\cos t\ge 1-\tfrac12 t^2$, hence
\[
\prod_{j=1}^p \cos\theta_j\ \ge\ \prod_{j=1}^p \Big(1-\tfrac12\theta_j^2\Big)
\ \ge\ 1-\tfrac12\sum_{j=1}^p \theta_j^2
\ =\ 1-\tfrac12\,S,
\]
using $\prod(1-a_j)\ge 1-\sum a_j$ for $a_j\in[0,1]$.
Therefore $1-\prod \cos\theta_j\le \tfrac12 S$.
Next relate $S$ and $\Sigma$: for small angles $\sin^2\theta=\theta^2 - \tfrac13\theta^4 + O(\theta^6)$, so
\[
\Sigma\ =\ S - \tfrac13 \sum_{j=1}^p \theta_j^4 + O(S^3)\ \ge\ S - \tfrac13 S^2.
\]
Thus $S\le \Sigma + \tfrac13 S^2 \le \Sigma + \tfrac13\cdot 10^{-4}$, and hence
\[
\tfrac12 S\ \le\ \tfrac12\,\Sigma\ +\ \tfrac{1}{6}\cdot 10^{-4}\ \le\ 0.502\,\Sigma
\]
for all $S\le 10^{-2}$ (since $\Sigma\ge 0$). Combining the two estimates yields \eqref{eq:C-main}.
\end{proof}

\paragraph{Remarks.}
(i) The numerical constants $0.49$ and $0.502$ are convenient explicit choices; any bounds of the form
\[
\Big(\tfrac12-\delta\Big)\Sigma\ \le\ 1-\prod \cos\theta_j\ \le\ \Big(\tfrac12+\delta\Big)\Sigma
\]
hold for all $S\le S_\delta$ with $S_\delta\downarrow 0$ as $\delta\downarrow 0$.  
(ii) The estimates here are used only for geometric intuition; the quantitative arguments in the main text rely instead on finite–dimensional linear algebra and $L^2$ orthogonality, not on \eqref{eq:C-main}.

\begin{thebibliography}{99}
\bibitem{WashburnSlicing2025} J. Washburn, \emph{Hodge via Slicing and Calibration Equality}, Zenodo preprint 10.5281/zenodo.17314589, 2025.
\end{thebibliography}

\end{document}
