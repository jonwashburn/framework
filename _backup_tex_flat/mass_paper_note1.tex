\documentclass[11pt]{article}
\usepackage{amsmath,amssymb,amsthm,mathtools}
\usepackage[margin=1in]{geometry}

\newtheorem{definition}{Definition}
\newtheorem{lemma}{Lemma}
\newtheorem{proposition}{Proposition}
\newtheorem{theorem}{Theorem}
\newtheorem{corollary}{Corollary}
\theoremstyle{remark}
\newtheorem{remark}{Remark}

\begin{document}

\section*{A note on non-uniqueness and hand-specificity in a lepton mass chain}

\vspace{4mm}
%\section{Setup}

From the Recognition Science, we have 
\begin{equation}
    \varphi=\frac{1+\sqrt{5}}{2}, \qquad \varphi^2=\varphi+1,
\end{equation}
and \(\alpha\) denote the fine structure constant (a fixed real number). The counting-layer integers read as
\begin{equation}
    W=17,\quad D=3,\quad F=6,\quad E_{\text{total}}=12,\quad E_{\text{passive}}=11.
\end{equation}
Note that one has the exact counting identities \(F=2D\) and \(E_{\text{total}}=2F=4D\), and also
\(E_{\text{total}}-E_{\text{passive}}=1\).

The lepton chain under investigation can be summarized as follows:
%
\begin{itemize}
    \item {\bf Electron break and generation steps:}
    The electron prediction of the RS has the form
\begin{equation}\label{eq:electron}
m^{\text{pred}}_e
=
m_{\text{skel}}(e)\,\varphi^{\mathrm{gap}(1332)-\delta_e},
\end{equation}
with \emph{electron break exponent} $\delta_e$ given by
\begin{equation}\label{eq:deltae}
\delta_e
:=
2W+\frac{W+E_{\text{total}}}{4E_{\text{passive}}}+\alpha^2+E_{\text{total}}\alpha^3.
\end{equation}
Then the theory defines the generation-step exponents to predict the 2nd and 3rd generation lepton masses as
\begin{align}
S_{e\to\mu} &:= E_{\text{passive}}+\frac{1}{4\pi}-\alpha^2,
\label{eq:SeMu}\\
S_{\mu\to\tau} &:= F-\frac{2W+3}{2}\,\alpha.
\label{eq:SMuTau}
\end{align}
The corresponding chain predictions then can be written as
\begin{align}
m_\mu^{\text{pred}} &= m_{\text{skel}}(e)\,\varphi^{\mathrm{gap}(1332)-\delta_e+S_{e\to\mu}},
\label{eq:mu}\\
m_\tau^{\text{pred}} &= m_{\text{skel}}(e)\,\varphi^{\mathrm{gap}(1332)-\delta_e+S_{e\to\mu}+S_{\mu\to\tau}}.
\label{eq:tau}
\end{align}
%%%
\item {\bf Numerical consistency check of the arithmetic} 
The internal arithmetic of \eqref{eq:deltae}--\eqref{eq:SMuTau} is straightforward and consistent:
\begin{equation}
    \delta_e
=
2\cdot 17+\frac{17+12}{4\cdot 11}+\alpha^2+12\alpha^3
=
34+\frac{29}{44}+\alpha^2+12\alpha^3.
\end{equation}
Using \(\alpha\approx 1/137.036\) gives \(\alpha^2\approx 5.3\times 10^{-5}\) and
\(12\alpha^3\approx 4.6\times 10^{-6}\), hence
\begin{equation}
    \delta_e \approx 34+0.6590909+0.000053+0.0000046 \approx 34.659148.
\end{equation}
Likewise,
\begin{equation}
    S_{e\to\mu}
=
11+\frac{1}{4\pi}-\alpha^2
\approx 11+0.0795775-0.000053 \approx 11.079524,
\end{equation}
and
\begin{equation}
    S_{\mu\to\tau}
=
6-\frac{37}{2}\alpha
=
6-18.5\,\alpha
\approx 6-0.1350 \approx 5.864999.
\end{equation}
Thus, at the purely algebraic/arithmetic level, there is no contradiction in the definitions.

This note is not about arithmetic mistakes.
It is about \emph{identifiability} and \emph{uniqueness}: whether the above specific functional forms
can be claimed as a law rather than as one among many hand-chosen formulas.
\end{itemize}



\subsection*{}


\section{What a parameter-free mass law must mean}

\paragraph{Parameter-free law, identifiability}\label{def:law}
Fix a list of universal constants \(\mathcal{C}\) (e.g.\ \(\varphi,\alpha\) and a finite set of
counting integers).
A \emph{parameter-free mass law} for a class of particles is a mapping
\[
\text{particle label } i \longmapsto m_i^{\text{pred}} = \mathcal{F}_i(\mathcal{C})
\]
such that:
\begin{enumerate}
\item \(\mathcal{F}_i\) is \emph{uniquely determined} by stated axioms/derivation rules,
not chosen ad hoc per particle.
\item There are no hidden degrees of freedom in the \emph{choice of functional form}.
(Discrete choices of which expression to use are degrees of freedom even if no real-valued
fit-parameter appears.)
\item The mapping is \emph{identifiable}: the same axioms/derivation rules cannot generate multiple
inequivalent formula families that agree on the fitted data but disagree elsewhere.
\end{enumerate}

\begin{remark}
Saying ``no per-species fitting'' is \emph{not} equivalent to saying ``no per-species parameters.''
One may avoid explicit numeric fit-parameters while still fitting by (i) selecting functional forms
after looking at the data, (ii) selecting which constants to combine, or (iii) selecting different
formulas for different transitions.
Those are discrete degrees of freedom, hence fitting in the sense of identifiability.
\end{remark}

\section{The generation steps are just logarithms of mass ratios}

Assume only the multiplicative chain structure \eqref{eq:electron}, \eqref{eq:mu}, \eqref{eq:tau}.
Then
\begin{align}
\frac{m_\mu^{\text{pred}}}{m_e^{\text{pred}}} &= \varphi^{S_{e\to\mu}},\label{eq:ratio1}\\
\frac{m_\tau^{\text{pred}}}{m_\mu^{\text{pred}}} &= \varphi^{S_{\mu\to\tau}}.\label{eq:ratio2}
\end{align}
Consequently,
\begin{align}
S_{e\to\mu} &= \log_\varphi\!\left(\frac{m_\mu^{\text{pred}}}{m_e^{\text{pred}}}\right),\label{eq:SeMuLog}\\
S_{\mu\to\tau} &= \log_\varphi\!\left(\frac{m_\tau^{\text{pred}}}{m_\mu^{\text{pred}}}\right).\label{eq:SMuTauLog}
\end{align}

One can get these relations by dividing \eqref{eq:mu} by \eqref{eq:electron}. The factor
\(m_{\text{skel}}(e)\,\varphi^{\mathrm{gap}(1332)-\delta_e}\) cancels exactly, leaving
\(\varphi^{S_{e\to\mu}}\), proving \eqref{eq:ratio1}.
Similarly, divide \eqref{eq:tau} by \eqref{eq:mu} to obtain \eqref{eq:ratio2}.
Taking \(\log_\varphi\) gives \eqref{eq:SeMuLog} and \eqref{eq:SMuTauLog}.

\paragraph{Two empirical numbers always exist}
For any positive target masses \((m_e,m_\mu,m_\tau)\), there exist \emph{unique} real numbers
\((S_{e\to\mu},S_{\mu\to\tau})\) that make \eqref{eq:ratio1}--\eqref{eq:ratio2} hold exactly, namely
\begin{equation}
    S_{e\to\mu}=\log_\varphi(m_\mu/m_e),\qquad S_{\mu\to\tau}=\log_\varphi(m_\tau/m_\mu).
\end{equation}

\underline{Why this is devastating for ``law'' claims?}
\emph{Introducing the symbols \(S_{e\to\mu}\) and \(S_{\mu\to\tau}\) already introduces two
free real degrees of freedom.}
Writing down any closed-form expressions for them is meaningful only if those expressions are
\emph{derived independently} of the masses, and (crucially) \emph{uniquely forced} by the theory.
Otherwise the procedure is logically indistinguishable from fitting the two ratios.


Take the values quoted in the lepton-chain table (or any other external targets).
The empirical ratios determine:
\begin{equation}
    S_{e\to\mu}^{\text{data}}=\log_\varphi(m_\mu/m_e),\qquad
S_{\mu\to\tau}^{\text{data}}=\log_\varphi(m_\tau/m_\mu).
\end{equation}
The proposed forms \eqref{eq:SeMu}--\eqref{eq:SMuTau} simply assert that these two empirical numbers
happen to equal
\begin{equation}
    11+\frac{1}{4\pi}-\alpha^2,\qquad 6-\frac{37}{2}\alpha,
\end{equation}
respectively.
But without a uniqueness derivation, this is \emph{numerical representation}, not a law.

\section{Exact non-uniqueness from identities}

There are two distinct non-uniqueness mechanisms:
\begin{itemize}
\item[A)] \textbf{ Exact symbolic non-uniqueness:} infinitely many different-looking expressions can
evaluate to \emph{exactly} the same number, whenever the constant set satisfies any identity.
\item[B)] \textbf{Approximate non-uniqueness:} infinitely many different expressions can match any
target to any tolerance, in any sufficiently rich expression class.
\end{itemize}

This section provides a rigorous algebraic proof of (A).

\begin{lemma}[Identity inflation]\label{lem:identity}
Let \(\mathcal{C}\) be constants and suppose there exists a nonzero expression
\(P(\mathcal{C})\) such that \(P(\mathcal{C})=0\) exactly.
Then for any expression \(E(\mathcal{C})\) and any expression \(Q(\mathcal{C})\),
\begin{equation}
    E(\mathcal{C}) \equiv E(\mathcal{C}) + P(\mathcal{C})\,Q(\mathcal{C})
\end{equation}
are \emph{distinct formulas} (unless \(Q\equiv 0\)) that evaluate to \emph{exactly the same number}.
In particular, there are infinitely many such alternatives by varying \(Q\).
\end{lemma}


Proof: Since \(P(\mathcal{C})=0\), we have \(P(\mathcal{C})Q(\mathcal{C})=0\) for all \(Q\).
Hence the two expressions are equal as real numbers. Distinctness is formal/syntactic.

\paragraph{Identities available in the constant set:}
The constant set used here contains multiple exact identities:
\begin{equation}
    \varphi^2-\varphi-1=0,\qquad E_{\text{total}}-2F=0,\qquad F-2D=0,\qquad
E_{\text{total}}-E_{\text{passive}}-1=0,
\end{equation}
all of which are true by definition of \(\varphi\) and by counting of the 3-cube.

\paragraph{Infinite families of alternative ``laws'' for the same numbers:}
By Lemma~\ref{lem:identity}, the following are all \emph{exactly equal} to the proposed steps,
yet are formally different:
\begin{align}
S_{e\to\mu}^{(k)} &:= E_{\text{passive}}+\frac{1}{4\pi}-\alpha^2
+ k(\varphi^2-\varphi-1),\\
S_{\mu\to\tau}^{(k)} &:= F-\frac{2W+3}{2}\alpha
+ k(E_{\text{total}}-2F)\alpha,\\
\delta_e^{(k)} &:= 2W+\frac{W+E_{\text{total}}}{4E_{\text{passive}}}+\alpha^2+E_{\text{total}}\alpha^3
+ k(F-2D),
\end{align}
for any integer \(k\).
There are infinitely many alternatives for each object \(\delta_e,S_{e\to\mu},S_{\mu\to\tau}\).

\begin{remark}
If a claim is ``this formula is the uniquely derived mass law,'' then the claim is dead on arrival
unless the theory supplies a canonical normal form or a uniqueness theorem that rules out
all alternative expressions.
Otherwise, the formula is not unique even in principle, because equality-preserving identities
generate infinite degeneracy.
\end{remark}

\section{The tau-step ``resolution'' does not resolve non-uniqueness}

The note attempts to remove the appearance of an arbitrary integer in
\(\frac{2W+3}{2}\) by rewriting \(3\) as \(D\), giving \(C_\tau=W+\frac{D}{2}\) and hence
\begin{equation}
    S_{\mu\to\tau}=F-\left(W+\frac{D}{2}\right)\alpha.
\end{equation}
Even if one accepts that reinterpretation, it does not produce uniqueness.

\underline{Many decompositions of the same coefficient}
Let \(C:=\frac{37}{2}=18.5\).
Using only the exact identities among \(\{W,D,F,E_{\text{total}},E_{\text{passive}}\}\),
\(C\) has multiple distinct decompositions, e.g.
\begin{equation}
    C = W+\frac{D}{2}
= W+\frac{F}{4}
= W+\frac{E_{\text{total}}}{8}
= \frac{2W+D}{2}
= \frac{4E_{\text{total}}-E_{\text{passive}}}{2},
\end{equation}
and infinitely many more can be generated by adding zero-identities.


Proof: Each equality is a one-line consequence of the cube relations:
\begin{equation}
    \frac{D}{2}=\frac{2D}{4}=\frac{F}{4}\quad \text{since }F=2D,
\qquad
\frac{F}{4}=\frac{2F}{8}=\frac{E_{\text{total}}}{8}\quad \text{since }E_{\text{total}}=2F.
\end{equation}
Also \(1=E_{\text{total}}-E_{\text{passive}}\) implies
\begin{equation}
    \frac{37}{2}=\frac{3E_{\text{total}}+1}{2}
=\frac{3E_{\text{total}}+(E_{\text{total}}-E_{\text{passive}})}{2}
=\frac{4E_{\text{total}}-E_{\text{passive}}}{2}.
\end{equation}
Finally, infinite families follow by Lemma~\ref{lem:identity}, e.g.\
\(C = \left(W+\frac{D}{2}\right) + k(E_{\text{total}}-2F)\) for any integer \(k\).


\begin{remark}
Replacing ``\(3\)'' by ``\(D\)'' merely chooses one interpretation among many equivalent ones.
It does not answer the uniqueness question:
\emph{why is the tau step linear in \(\alpha\) with precisely this coefficient rather than some
other structure (e.g.\ involving \(\alpha^2\), \(1/(4\pi)\), etc.)?}
Nor does it provide a derivation rule that would force this expression and forbid the others.
\end{remark}

\section{Approximate formula-mining lets us hit arbitrary targets}

So far, the non-uniqueness was exact and purely algebraic. Now I prove a stronger point: if one allows integer coefficients multiplying a single irrational constant, one can approximate any target number to arbitrary accuracy. This is a standard theorem; its consequence is that ``nice-looking combinations'' can be found for essentially any desired correction term.

\begin{theorem}[Density of integer translates of an irrational (using GPT)]\label{thm:dense} 
Let \(c\in\mathbb{R}\setminus\mathbb{Q}\) be irrational.
Then the set
\begin{equation}
    \{m+nc : m,n\in\mathbb{Z}\}
\end{equation}
is dense in \(\mathbb{R}\).
Equivalently, for any \(x\in\mathbb{R}\) and any \(\varepsilon>0\), there exist integers \(m,n\)
such that
\begin{equation}
    \bigl|x-(m+nc)\bigr|<\varepsilon.
\end{equation}
\end{theorem}

Proof: Sketch with the pigeonhole principle. 
Consider the fractional parts of \(0,c,2c,\dots,Nc\) modulo \(1\).
Partition \([0,1)\) into \(N\) equal intervals of length \(1/N\).
Two of the \(N+1\) fractional parts must lie in the same interval, hence their difference is less
than \(1/N\) modulo \(1\):
\begin{equation}
    \| (k-\ell)c \|_{\,\mathbb{R}/\mathbb{Z}} < \frac{1}{N}.
\end{equation}
Let \(n:=k-\ell\neq 0\). Then \(nc\) is within \(1/N\) of an integer \(p\).
Thus \(nc-p\) can approximate any desired fractional offset by choosing \(N\) large.
Finally, choose \(m\) to correct the integer part of \(x\).

\begin{remark}[Application to the expression class used in the chain]
The chain explicitly uses \(1/(4\pi)\), and \(\pi\) is irrational, hence \(c:=1/(4\pi)\) is irrational.
Therefore, even with \(\alpha\) held fixed, one already has a mechanism to build arbitrarily close
approximations to any desired correction term by using integer combinations of \(1/(4\pi)\).
If one further allows \(\alpha\) and \(\alpha^2\) and rational combinations of counting integers,
the space of approximants explodes.
\end{remark}

\subsection*{Examples of ``many alternatives''}
Let the target correction be the non-integer part of a step exponent.
For the muon step, the correction beyond \(11\) is about \(0.079524...\).
One candidate is \(c-\alpha^2\) where \(c=1/(4\pi)\).
But there are many other equally ``structural-looking'' candidates, for instance:
\begin{equation}
    \frac{E_{\text{total}}}{48\pi}-\alpha^2
=
\frac{1}{4\pi}-\alpha^2,
\qquad
\frac{F}{24\pi}-\alpha^2
=
\frac{1}{4\pi}-\alpha^2,
\qquad
\frac{D}{12\pi}-\alpha^2
=
\frac{1}{4\pi}-\alpha^2,
\end{equation}
all different-looking yet identical numerically because \(E_{\text{total}}=12,F=6,D=3\).

More importantly, if one were allowed to choose integer coefficients freely, then
Theorem~\ref{thm:dense} implies that for any desired correction \(\Delta\) and any tolerance
\(\varepsilon\), there exist integers \(m,n\) such that
\begin{equation}
    \left|\Delta - \left(m+\frac{n}{4\pi}\right)\right|<\varepsilon.
\end{equation}
This is the mathematical statement behind ``with \(\pi\) in your toolbox you can cook up
essentially any correction to any accuracy, unless you impose strict derivation constraints.''

\section{Why uniqueness is required to claim a mass law }

\begin{definition}[Non-identifiability]\label{def:nonident}
A mass construction is \emph{non-identifiable} if there exist two distinct formula families
\(\mathcal{F}\neq \mathcal{G}\) (distinct functional forms, not merely renamings)
such that
\[
\mathcal{F}_i(\mathcal{C})=\mathcal{G}_i(\mathcal{C})
\quad\text{for all tested particles } i,
\]
but the two families are not provably equivalent under the axioms.
\end{definition}


(Using GPT:) If a framework is non-identifiable in the sense of Definition~\ref{def:nonident}, then
from the tested data alone one cannot claim that a specific displayed formula is ``the mass law,''
because the same data are compatible with multiple inequivalent formula choices.
To claim a law, one must provide (at minimum) a derivation rule plus a uniqueness theorem
that selects a canonical form and rules out alternatives.



If \(\mathcal{F}\) and \(\mathcal{G}\) produce the same tested outputs, the data cannot distinguish
them.
Absent a theorem that \(\mathcal{F}\equiv \mathcal{G}\) under the axioms, choosing \(\mathcal{F}\)
over \(\mathcal{G}\) is underdetermined.
Therefore the statement ``the law is \(\mathcal{F}\)'' is not established by the evidence.
Uniqueness is exactly what would be required to make such a claim meaningful.

The chain demonstrates that one can \emph{represent} the empirical mass ratios as
\(\varphi\)-exponents.
It does not demonstrate that the specific expressions \eqref{eq:SeMu}--\eqref{eq:SMuTau}
are uniquely forced by the theory.
Therefore it cannot be claimed as a parameter-free fermion mass law in the sense of
Definition~\ref{def:law}.

%\section{Comment on the tau-step ``exclusivity'' with hypothesis}

\end{document}
