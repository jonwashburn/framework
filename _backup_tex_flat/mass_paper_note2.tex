\documentclass[11pt]{article}
\usepackage{amsmath,amssymb,amsthm,mathtools}
\usepackage[margin=1in]{geometry}

\newtheorem{definition}{Definition}
\newtheorem{lemma}{Lemma}
\newtheorem{proposition}{Proposition}
\newtheorem{theorem}{Theorem}
\newtheorem{corollary}{Corollary}
\theoremstyle{remark}
\newtheorem{remark}{Remark}

\begin{document}

\section*{Tau-Step ``Exclusivity'' Does Not Establish a Fermion Mass Law}

Here I evaluate the claim that the revised muon-to-tau step coefficient,
\begin{equation}
   C_\tau = W + \frac{D}{2}, 
\end{equation}
derived in \texttt{tau\_step\_exclusivity.tex}, resolves the criticism that the lepton-generation
formulas are hand-specific and admit many alternatives.

I show that while the algebra is correct, the exclusivity claim fails at a structural level. The failure is not numerical but logical: the construction does not uniquely determine the functional form of the tau step from prior axioms and therefore cannot support a mass \emph{law} claim.

Most of the sections below are already done in the previous note, here I again briefly give it to be self contained. 

\section{What must be proven to claim a mass law}
As mentioned in the earlier note:
A fermion mass law is a mapping
\[
\mathcal{L}:\{\text{fermion species}\}\to\mathbb{R}_{>0}
\]
that is:
\begin{enumerate}
\item \emph{Uniquely derived} from stated axioms or mechanisms,
\item \emph{Identifiable}: no alternative inequivalent constructions exist that reproduce the same
validated data,
\item \emph{Predictive}: functional forms are fixed \emph{before} comparison to experimental masses.
\end{enumerate}

Numerical agreement alone is insufficient. What is required is uniqueness of the derivation.

\section{Summary of the tau-step revision}

We have
\begin{equation}
   S_{\mu\to\tau} = F - \frac{2W+3}{2}\alpha,
\end{equation}
with $W=17$, giving a coefficient $C_\tau=18.5$.

The exclusivity note rewrites
\begin{equation}
    \frac{2W+3}{2} = W + \frac{D}{2},
\end{equation}
interpreting the integer $3$ as the spatial dimension $D=3$.

This removes an \emph{aesthetic} arbitrariness but does not establish necessity.

\section{Rewriting a number is not a derivation}

Re-expressing a numerically fitted constant in terms of other constants does not constitute a
derivation unless the expression is uniquely forced by prior axioms.


Proof: The exclusivity note explicitly states that the value $C_\tau=18.5$ was \emph{numerically determined}
first. The expression $W + D/2$ is chosen afterward to reproduce this value.

Formally, let $C$ be a real number determined from data. Any identity of the form
\begin{equation}
    C = f(c_1,c_2,\dots)
\end{equation}
is a \emph{representation}, not a derivation, unless the theory proves:
\begin{equation}
    \text{(axioms)} \;\Rightarrow\; C = f(c_1,c_2,\dots)
\end{equation}
and proves that no other $f'$ is admissible.

The note does not provide such a proof.

\begin{remark}
Lean verification of $17 + 3/2 = 18.5$ certifies arithmetic, not physical necessity.
\end{remark}

\section{Explicit non-uniqueness using the same constants}

Even fixing the counting-layer constants
\begin{equation}
    W=17,\quad D=3,\quad F=6,\quad E_{\text{total}}=12,\quad E_{\text{passive}}=11,
\end{equation}
there are many inequivalent formulas that evaluate to $18.5$.

\underline{Exact degeneracy at $D=3$}:
The coefficient $C_\tau=18.5$ admits multiple distinct expressions built solely from the
same counting-layer constants.


Proof: Using cube identities $F=2D$, $E_{\text{total}}=2F$, and $E_{\text{total}}-E_{\text{passive}}=1$:
\begin{align}
C_\tau
&= W + \frac{D}{2},\\
&= W + \frac{F}{4},\\
&= W + \frac{E_{\text{total}}}{8},\\
&= \frac{2W + D}{2},\\
&= \frac{4E_{\text{total}} - E_{\text{passive}}}{2}.
\end{align}
Each expression equals $18.5$ at $D=3$, yet they are algebraically distinct.


\begin{remark}
No principle in the exclusivity note rules out the alternatives above.
Selecting $D/2$ is therefore a choice, not a consequence.
\end{remark}

\section{Additivity assumptions merely restrict the hypothesis class}

The exclusivity note enforces a condition of the form
\begin{equation}
    \Delta(D_1 + D_2) = \Delta(D_1) + \Delta(D_2),
\end{equation}
from which linearity $\Delta(D)=kD$ follows.


Imposing additivity does not derive the tau-step coefficient; it restricts the space of allowed
functions after the fact.

Proof:
On $\mathbb{N}$, additivity implies $\Delta(D)=D\Delta(1)$. Fixing $\Delta(3)=3/2$ yields
$\Delta(D)=D/2$.

However:
\begin{enumerate}
\item Additivity is not derived from the mass framework axioms,
\item Many non-additive functions agree at $D=3$,
\item RS already fixes $D=3$, so cross-$D$ behavior is untestable.
\end{enumerate}

Thus additivity functions as an \emph{exclusion rule}, not a derivation.


\section{Collapse of cross-dimensional arguments}

The exclusivity argument implicitly appeals to behavior across dimensions $D$.
But RS independently claims that only $D=3$ is physically realized.


If only $D=3$ is physically meaningful, then functional uniqueness in $D$ cannot be inferred.

Proof:
Let $f(D)$ be any function such that $f(3)=3/2$.
Define
\begin{equation}
    g(D) := \frac{D}{2} + (D-3)^2 h(D),
\end{equation}
for arbitrary $h(D)$. Then $g(3)=3/2$ exactly, but $g\neq f$ as a function.

Since RS forbids testing at $D\neq 3$, all such functions are observationally indistinguishable.


\begin{remark}
Thus dimensional rigidity removes, rather than enforces, functional uniqueness.
\end{remark}

\section{General non-identifiability of the lepton chain}

\begin{theorem}[Non-identifiability of the lepton mass pipeline]
The lepton mass chain cannot uniquely determine its step formulas from lepton masses alone.
\end{theorem}

\begin{proof}
The chain has the structure
\[
\frac{m_\mu}{m_e} = \varphi^{S_{e\to\mu}}, \qquad
\frac{m_\tau}{m_\mu} = \varphi^{S_{\mu\to\tau}}.
\]
Hence
\[
S_{e\to\mu} = \log_\varphi(m_\mu/m_e), \quad
S_{\mu\to\tau} = \log_\varphi(m_\tau/m_\mu).
\]

These two real numbers always exist for any positive masses.
Therefore:
\begin{itemize}
\item Introducing $S_{e\to\mu}$ and $S_{\mu\to\tau}$ introduces two degrees of freedom,
\item Writing them in terms of $\alpha$, $W$, $D$, etc.\ does not remove those degrees unless the
functional forms are uniquely derived,
\item No such uniqueness proof is provided.
\end{itemize}
\end{proof}

\section{Conclusion}

The revised tau-step formula is:
\begin{itemize}
\item Algebraically correct,
\item Aestheticly cleaner,
\item Formally verifiable in Lean.
\end{itemize}

However, it does \emph{not}:
\begin{itemize}
\item Eliminate functional non-uniqueness,
\item Derive the coefficient from prior axioms,
\item Prevent alternative constructions using the same constants,
\item Establish identifiability required for a mass law.
\end{itemize}

\end{document}
