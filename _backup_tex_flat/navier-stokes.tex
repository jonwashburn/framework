\documentclass[12pt]{amsart}

\title[Global regularity at the critical scale]{Global Regularity for the Three-Dimensional Incompressible Navier--Stokes Equations at the Critical Scale}

\author{Jonathan Washburn}
\address{Recognition Science, Recognition Physics Institute, Austin, Texas, USA}
\email{jon@recognitionphysics.org}

\subjclass[2020]{35Q30; 76D05; 35B65; 35K55.}
\keywords{Navier--Stokes equations, global regularity, vorticity, $\varepsilon$-regularity, De~Giorgi, $BMO^{-1}$, Koch--Tataru, backward uniqueness, critical element.}

\begin{document}
\begin{abstract}
We prove global regularity for the three-dimensional incompressible Navier--Stokes equations on $\mathbb{R}^3$ with smooth, divergence-free initial data. The argument is strictly scale-invariant and PDE-internal. It has four components: (i) a vorticity-based $\varepsilon$--regularity lemma at the critical $L^{3/2}$ Morrey scale, yielding local $L^\infty$ bounds for vorticity from small critical mass; (ii) a quantitative bridge from critical vorticity control on parabolic cylinders to a small $BMO^{-1}$ time slice for the velocity; (iii) compactness extraction of a minimal ancient critical element together with a De~Giorgi density--drop that pins the threshold; and (iv) elimination of the critical element by combining small-data global well-posedness in $BMO^{-1}$ with backward uniqueness. No extraneous structural hypotheses are imposed, and every estimate respects the parabolic scaling. As a consequence, no finite-time singularity can occur.
\end{abstract}

\maketitle
\section*{0. Standing decisions (fixed for the whole paper)}
\begin{itemize}
\item \textbf{Scaling and cylinders.} Parabolic cylinders \(Q_r(x_0,t_0):=B_r(x_0)\times[t_0-r^2,t_0]\). All statements are invariant under \(u_\lambda(x,t)=\lambda\,u(\lambda x,\lambda^2 t)\).

\item \textbf{Critical vorticity functional.} For vorticity \(\omega=\nabla\times u\),
\[
\mathcal{W}(x,t;r):=\frac{1}{r^2}\iint_{Q_r(x,t)} |\omega|^{3/2}\,dx\,ds,\qquad
\mathcal{M}(t):=\sup_{x\in\mathbb{R}^3,\ r>0}\mathcal{W}(x,t;r).
\]

\item \textbf{Carleson characterization of \(BMO^{-1}\).} We \emph{fix} the semigroup form:
\[
\|f\|_{BMO^{-1}}:=\sup_{x\in\mathbb{R}^3,\ r>0}\left(
\frac{1}{|B_r|}\int_{0}^{r^2}\!\!\int_{B_r(x)} \big|e^{\nu \tau \Delta} f(y)\big|^2\,dy\,d\tau
\right)^{1/2},
\]
where \(e^{\nu \tau \Delta}\) is the heat semigroup and \(|B_r|\) denotes the volume of the ball \(B_r\). This tent-space formulation is standard; see Koch--Tataru \cite{KochTataru2001}.

\item \textbf{Biot--Savart control at critical exponents.} On each time slice, for all balls \(B_\rho\),
\[
\|u(\cdot,t)\|_{L^3(B_\rho)}\;\le\; C \sum_{k\ge 0} 2^{-k}\,\|\omega(\cdot,t)\|_{L^{3/2}(B_{2^{k+1}\rho})},
\]
with a universal constant \(C\). This near/far dyadic control follows from Hardy--Littlewood--Sobolev and Calderón--Zygmund theory; see \cite{Stein1993,MajdaBertozzi2002}.

\item \textbf{Small-data threshold in \(BMO^{-1}\).} Fix \(\varepsilon_{\mathrm{SD}}>0\) so that initial data with \(\|u_0\|_{BMO^{-1}}\le \varepsilon_{\mathrm{SD}}\) produce a unique global mild solution, smooth for \(t>0\) (Koch--Tataru \cite{KochTataru2001}). This theorem is stated in Section~6 and used as a black box.

\item \textbf{Threshold constants and safe window.} Let $C_B$ be the universal constant in the $L^{3/2}\!\to BMO^{-1}$ slice bridge (Lemma~B, Section~3). Define the working threshold
\[
\varepsilon_0:=\big(\varepsilon_{\mathrm{SD}}/C_B\big)^{3/2}.
\]
Under $\sup\mathcal W$ on the unit window (hence square--Carleson by Appendix~H), the \emph{safe-window criterion} is: if, on a unit window $[t_0-1,t_0]$, one has
\[
\sup_{x\in\mathbb{R}^3,\ r>0}\ \frac{1}{r^2}\int_{t_0-r^2}^{t_0}\!\Big(\int_{B_r(x)} |\omega|^{3/2}\Big)^{\!4/3} ds\ \le\ \varepsilon_0^{4/3},
\]
then Lemma~\ref{lem:B} furnishes $t_*\in[t_0-\tfrac12,t_0]$ with $\|u(\cdot,t_*)\|_{BMO^{-1}}\le \varepsilon_{\mathrm{SD}}$.

\item \textbf{Density-drop parameters.} Fix \(\vartheta:=\tfrac14\) and \(c:=\tfrac34\) in the De~Giorgi density-drop (Section~5). These explicit values are convenient and suffice for contraction.

\item \textbf{Iteration ladder.} Truncation levels \(\kappa_0:=K_0\,\varepsilon_0^{2/3}\) with \(K_0:=2C_A\) (from Lemma~A); exponents \(p_k:=2(3/2)^k\); radii \(r_{k+1}:=\tfrac12(r_k+\vartheta)\).
\item \textbf{Square--Carleson control derived from $\mathcal W$.} Throughout we use the square--Carleson bound on unit windows. In Appendix~H (Theorem~\ref{thm:C2S-from-W}) we prove that if
\[
\sup_{(x,t)\in\mathbb{R}^3\times[t_0-1,t_0]}\ \sup_{r>0}\ \mathcal{W}(x,t;r)\ \le\ \varepsilon,
\]
then
\[
\sup_{(x,t)\in\mathbb{R}^3\times[t_0-1,t_0]}\ \sup_{r>0}\ \frac{1}{r^2}\int_{t-r^2}^{t}\!\Big(\int_{B_r(x)} |\omega|^{3/2}\Big)^{\!4/3} ds\ \le\ C\,\varepsilon^{4/3},
\]
and consequently there exists $t_*\in[t_0-\tfrac12,t_0]$ with
\[
\|u(\cdot,t_*)\|_{BMO^{-1}}\ \le\ C_{B}\,\varepsilon^{2/3}
\]
by the tent--space slice bridge of Section~3. All constants are universal and the statements are scale-invariant.

\item \textbf{Absorption from square--Carleson (no slice hypothesis).} The absorbed Caccioppoli inequality used in Sections~2 and~5 follows from the square--Carleson bound via a Young-in-time estimate; see Appendix~A.2' (Lemma~\ref{lem:absorb-C2S-square}) together with Appendix~H (Corollary~\ref{cor:absorb-from-C2S}). No separate slice smallness (SA) is assumed.
\end{itemize}

\paragraph{Dependency chart (proof flow).}
\[
\mathcal W\ \Longrightarrow\ \text{RH-time (App.~H)}\ \Longrightarrow\ \text{square--Carleson (App.~H)}\ \Longrightarrow\ \text{absorbed Caccioppoli (App.~A.2' + H)}\\
\Longrightarrow\ \text{density-drop (Sec.~5)}\ \Longrightarrow\ \text{threshold closure (Sec.~6--7)}\ \Longrightarrow\ \text{BMO$^{-1}$ slice (Sec.~3)}\\
\Longrightarrow\ \text{small-data gate (Koch--Tataru)}\ \Longrightarrow\ \text{rigidity (energy/backward uniqueness)}.
\]

\section*{1. Problem statement and main result}

\begin{theorem}[Global Regularity]\label{thm:global}
Let $u_0\in C_c^\infty(\mathbb{R}^3)$ be divergence-free. The associated Leray--Hopf solution of incompressible Navier--Stokes in $\mathbb{R}^3$ exists uniquely and remains smooth for all $t\ge0$.
\end{theorem}

\noindent\emph{Proof strategy (at a glance).} Assume a first singular time; extract an ancient critical element $U$ at a minimal profile level $\mathcal{M}_c$. Lemma~A yields local $L^\infty$ control from critical smallness. A density-drop improves the profile on smaller cylinders and pins the threshold $\mathcal{M}_c=\varepsilon_0$. Lemma~B then produces a time slice $t_*$ with $\|U(t_*)\|_{BMO^{-1}}\le \varepsilon_{\mathrm{SD}}$. Small-data global theory gives smoothness forward from $t_*$; forward energy uniqueness forces $U\equiv V$, and the forward W-gap contradicts saturation.

\subsection*{1.1. Formulation and scaling}
We study the three–dimensional incompressible Navier–Stokes equations on $\mathbb{R}^3$ with viscosity $\nu>0$,
\[
\partial_t u + (u\!\cdot\!\nabla)u \;=\; -\nabla p + \nu\,\Delta u,\qquad \nabla\!\cdot u=0,
\]
posed with smooth, divergence–free initial data $u_0\in C_c^\infty(\mathbb{R}^3)$. The natural notion of global weak solution is that of Leray–Hopf; the local partial regularity framework is that of suitable weak solutions (defined in §1.5 below).

The parabolic scaling
\[
u_\lambda(x,t)=\lambda\,u(\lambda x,\lambda^2 t),\qquad p_\lambda(x,t)=\lambda^2 p(\lambda x,\lambda^2 t),
\]
leaves the equations invariant. All estimates in the paper respect this scaling. Local space–time analysis is performed on parabolic cylinders $Q_r(x_0,t_0)=B_r(x_0)\times[t_0-r^2,t_0]$, and the critical vorticity functional
\[
\mathcal{W}(x,t;r):=\frac{1}{r^2}\iint_{Q_r(x,t)}|\omega|^{3/2}\,dx\,ds,\qquad \omega=\nabla\times u,
\]
measures concentration at the invariant exponent. Its global profile is $\mathcal{M}(t):=\sup_{x\in\mathbb{R}^3,\ r>0}\mathcal{W}(x,t;r)$.

\subsection*{1.2. Solution classes}
We use Leray–Hopf solutions for global existence and energy control, and suitable weak solutions for local compactness and partial regularity. Smooth solutions are understood in the classical sense. Uniqueness in the class reached by the argument follows from forward energy uniqueness (Lemma~\ref{lem:forward-energy} in Appendix D) and the small–data theory in $BMO^{-1}$ invoked at the end of the proof.

\subsection*{1.3. Strategy of the proof}
The proof proceeds by contradiction. Suppose a first singular time exists. We extract, by critical rescaling around near–maximizers of $\mathcal{W}$ and compactness for suitable solutions, a nontrivial ancient critical element $U$ saturating a minimal profile level $\mathcal{M}_c$.

The analysis has four components, each scale–invariant and PDE–internal:

\medskip
\noindent\emph{(i) Local $\varepsilon$–regularity at the critical scale (Lemma A).} Small critical vorticity mass on some cylinder,
\[
\mathcal{W}(x_0,t_0;r_0)\le \varepsilon_*,
\]
forces a local $L^\infty$ bound for $|\omega|$ on $Q_{r_0/2}(x_0,t_0)$. The proof uses an absorbed Caccioppoli inequality for $\theta=|\omega|$ (drift is divergence–free, stretching is absorbed by Calderón–Zygmund at $L^{3/2}$), followed by a De~Giorgi iteration on shrinking cylinders.

\medskip
\noindent\emph{(ii) Density–drop and threshold pinning.} A De~Giorgi "$\varepsilon$–improvement" contracts the excess above threshold on smaller cylinders:
\[
\mathcal{W}(0,0;1)\le \varepsilon_0+\eta\quad\Longrightarrow\quad \mathcal{W}(0,0;\vartheta)\le \varepsilon_0+c\,\eta
\]
for fixed $\vartheta\in(0,1/2)$ and $c\in(0,1)$. An open/closed argument pins the supremal safe level and identifies the minimal blow–up profile $\mathcal{M}_c$ with the working threshold $\varepsilon_0$.

\medskip
\noindent\emph{(iii) Vorticity $L^{3/2}$ $\to$ velocity $BMO^{-1}$ time slice (Lemma B).} Uniform smallness of $\mathcal{W}$ on a unit time window produces a time slice $t_*$ with
\[
\|U(\cdot,t_*)\|_{BMO^{-1}}\;\lesssim\; \varepsilon_0^{2/3}.
\]
This uses the heat–flow Carleson characterization of $BMO^{-1}$, Duhamel's formula, dyadic Biot–Savart control of $\|u\|_{L^3}$ by $\|\omega\|_{L^{3/2}}$, and heat–kernel smoothing.

\medskip
\noindent\emph{(iv) Gate and rigidity.} Choosing $\varepsilon_0$ so that the slice bound lands below the small–data threshold in $BMO^{-1}$, small–data global well–posedness produces a smooth solution forward from $t_*$. Backward uniqueness then forces the ancient critical element $U$ to be identically zero, contradicting its nontriviality. Hence no singularity forms and Theorem~\ref{thm:global} follows.

\subsection*{1.4. Consequences and scope}
All constants are absolute and every bound scales correctly. No structural hypotheses beyond the equations are assumed. The method yields a continuation criterion at the critical scale: once $\mathcal{M}(t)$ stays below the threshold on some final time window, smoothness propagates, and blow–up is excluded. The remainder of the paper develops Lemma A, the density–drop, the $L^{3/2}\!\to BMO^{-1}$ slice bridge, the compactness extraction of the critical element, and the rigidity close.

\subsection*{1.5. Suitable weak solutions}
A pair $(u,p)$ is called a \emph{suitable weak solution} on a space-time region $\Omega\subset\mathbb{R}^3\times\mathbb{R}$ if $u$ is a Leray--Hopf weak solution (divergence-free, locally square-integrable in space with locally $H^1$ gradient, satisfying the weak form of the Navier--Stokes equations) and the pair additionally satisfies the local energy inequality (stated in Appendix C, Lemma~\ref{lem:LEI}) for all nonnegative test functions with compact support in $\Omega$. Suitability provides the compactness and partial regularity needed for the critical-element extraction.

\section*{2. Local $\varepsilon$--regularity for vorticity (Lemma A)}

We write $\omega=\nabla\times u$ and $\theta:=|\omega|$. For a parabolic cylinder $Q_r(x_0,t_0):=B_r(x_0)\times[t_0-r^2,t_0]$ recall the scale–invariant vorticity functional
\[
\mathcal{W}(x_0,t_0;r):=\frac1{r^2}\iint_{Q_r(x_0,t_0)} \theta^{3/2}\,dx\,dt.
\]

\begin{lemma}[Lemma A: critical $\varepsilon$--regularity]\label{lem:A}
There exist absolute constants $\varepsilon_A>0$ and $C_A<\infty$ such that if
\[
\mathcal{W}(x_0,t_0;r_0)\le \varepsilon_A,
\]
then
\[
\sup_{Q_{r_0/2}(x_0,t_0)} \theta \;\le\; \frac{C_A}{r_0^2}\,\big(\mathcal{W}(x_0,t_0;r_0)\big)^{2/3}.
\]
\end{lemma}

\begin{proof}
\emph{Caccioppoli--De~Giorgi with absorption via square--Carleson.} By scaling, reduce to $r_0=1$, $(x_0,t_0)=(0,0)$ and write $Q_1=B_1\times[-1,0]$. Assume $\mathcal{W}(0,0;1)\le\varepsilon_A$. Let $\theta=|\omega|$ and set $w:=(\theta-\kappa_0)_+$ with $\kappa_0=K_0\,\mathcal{W}(0,0;1)^{2/3}$ and universal $K_0\ge1$. Choose the cutoff chain and exponent ladder as in Appendix~A. By Appendix~H (Theorem~\ref{thm:C2S-from-W}) the square--Carleson bound holds on $Q_1$, and Lemma~\ref{lem:absorb-C2S-square} yields the absorbed Caccioppoli inequality used below.

Under the square--Carleson smallness (C2S$^\square$) on $Q_1$ (derived from $\sup\mathcal W\le\varepsilon_A$ via Appendix~H, Theorem~\ref{thm:C2S-from-W}), Lemma~\ref{lem:absorb-C2S-square} provides an absorbed Caccioppoli inequality on a slightly shorter slab, with a harmless baseline term. The De~Giorgi iteration on shrinking cylinders (Appendix~A.4--A.5) then yields a uniform bound for $w$ and, in particular,
\[
\sup_{Q_{1/2}}\theta\ \le\ C\,\kappa_0\ \le\ C_A\,\mathcal{W}(0,0;1)^{2/3}.
\]
Undoing the normalization, noting that $\mathcal W$ is scale-invariant while $|\omega|$ scales like $r_0^{-2}$, gives the stated bound
\[
\sup_{Q_{r_0/2}(x_0,t_0)} \theta\ \le\ \frac{C_A}{r_0^2}\,\big(\mathcal{W}(x_0,t_0;r_0)\big)^{2/3}.
\]
All constants are universal (dimension/viscosity only) and the exponent $2/3$ is dictated by scaling.
\end{proof}

\paragraph{Remarks.}
(1) The absorption of stretching uses the square--Carleson control (C2S$^\square$) (Lemma~\ref{lem:absorb-C2S-square} and Corollary~\ref{cor:absorb-from-C2S}), not a pointwise-in-time slice smallness hypothesis. No smallness of the drift $u$ is needed; $\operatorname{div}u=0$ moves advection entirely onto the cutoff, and the iteration tolerates the resulting lower–order contribution.

(2) The exponent $2/3$ is dictated by scaling: the functional $\mathcal{W}$ is invariant, while an $L^\infty$ bound for $|\omega|$ on $Q_{r_0/2}$ must scale like $r_0^{-2}$ times a $2/3$ power of a scale–invariant quantity.

(3) All constants are dimensionality–dependent only; in particular, $\varepsilon_A$ depends only on $3$ and $\nu$ through the normalization used to absorb the stretching term.

\section*{3. Vorticity $L^{3/2}$ $\to$ velocity $BMO^{-1}$ slice (Lemma B)}

Throughout this section we work with the fixed, scale–invariant $BMO^{-1}$ norm from Section~0:
\[
\|f\|_{BMO^{-1}}
:=\sup_{x\in\mathbb{R}^3,\ r>0}
\left(\frac{1}{|B_r|}\int_{0}^{r^2}\int_{B_r(x)}\big|e^{\nu\tau\Delta}f(y)\big|^2\,dy\,d\tau\right)^{1/2}.
\]
Recall also the critical vorticity functional
\[
\mathcal{W}(x,t;r):=\frac1{r^2}\iint_{Q_r(x,t)}|\omega|^{3/2},\qquad Q_r(x,t)=B_r(x)\times[t-r^2,t],
\]
and the global profile $\mathcal{M}(t):=\sup_{x,r}\mathcal{W}(x,t;r)$.

\begin{lemma}[Lemma B: Carleson slice bridge]\label{lem:B}
There exists $C_B<\infty$ such that if
\[
\sup_{(x,t)\in\mathbb{R}^3\times[t_0-1,t_0]}\ \sup_{r>0}\ \mathcal{W}(x,t;r)\ \le\ \varepsilon,
\]
then there exists $t_*\in[t_0-\tfrac12,t_0]$ with
\[
\|u(\cdot,t_*)\|_{BMO^{-1}}\ \le\ C_B\,\varepsilon^{2/3}.
\]
\end{lemma}

\begin{proof}[Proof attempt (this is the current unconditional bottleneck)]
\emph{Step 0 (Normalization and hypothesis).} By time translation assume $t_0=0$. All constants below are universal (dimension–only) and may change from line to line. We assume the Carleson–square bound from Section~0 (which follows from $\sup\mathcal W\le\varepsilon$ by Appendix~H):
\[
\sup_{(x,t)\in\mathbb{R}^3\times[-1,0]}\ \sup_{r>0}\ \frac{1}{r^2}\int_{t-r^2}^{t}\!\Big(\int_{B_r(x)} |\omega|^{3/2}\Big)^{\!4/3}\! ds\ \le\ \varepsilon^{4/3}.
\]

\medskip
\emph{Step 1 (Duhamel decomposition on Carleson boxes).}
Fix $x\in\mathbb{R}^3$ and $r>0$. For any time $t\in[-1,0]$ and any $\tau\in[0,r^2]$, write the semigroup evolution at time $t$ as
\begin{equation}\label{eq:duhamel}
e^{\nu\tau\Delta}u(t)\;=\;u(t+\tau)\;+\;\int_0^{\tau} e^{\nu(\tau-s)\Delta}\,\mathbb{P}\nabla\!\cdot\big(u\otimes u\big)(t+s)\,ds
\;=:\;L(t,\tau)\;+\;N(t,\tau),
\end{equation}
where $\mathbb{P}$ denotes the Leray projection. For each fixed $(x,r)$ define the (squared) Carleson box energy of a spacetime field $F(t,\tau,\cdot)$ by
\[
\mathbf{E}[F](t;x,r):=\frac1{|B_r|}\int_{0}^{r^2}\!\!\int_{B_r(x)} |F(t,\tau,y)|^2\,dy\,d\tau.
\]
We shall estimate the time average of $\mathbf{E}[L]$ and $\mathbf{E}[N]$ over $t\in[-\tfrac12,0]$ and then pick a good time $t_*$.

\medskip
\emph{Step 2 (Linear piece controlled by vorticity on dyadic annuli).}
By Hölder on balls, $\|f\|_{L^2(B_r)}\le |B_r|^{1/6}\|f\|_{L^3(B_r)}\simeq r^{1/2}\|f\|_{L^3(B_r)}$. On each time slice, the Biot–Savart representation and a near/far–field split yield
\begin{equation}\label{eq:BS}
\|u(\cdot,s)\|_{L^3(B_r(x))}\ \le\ C\sum_{k\ge0}2^{-k}\,\|\omega(\cdot,s)\|_{L^{3/2}(B_{2^{k+1}r}(x))}.
\end{equation}
Squaring, integrating over $\tau\in[0,r^2]$ (i.e., $s=t+\tau\in[t,t+r^2]$), and dividing by $|B_r|$, we obtain
\[
\mathbf{E}[L](t;x,r)
\;\le\; C\,\frac{1}{|B_r|}\int_{t}^{t+r^2}\!\!\left(r\,\sum_{k\ge0}2^{-k}\,\|\omega(\cdot,s)\|_{L^{3/2}(B_{2^{k+1}r})}\right)^2 ds.
\]
By Cauchy–Schwarz on the $k$–sum with weights $2^{-2k/3}$,
\[
\Big(\sum_{k\ge0}2^{-k} a_k\Big)^2 \le \Big(\sum_{k\ge0}2^{-4k/3}\Big)\Big(\sum_{k\ge0}2^{-2k/3} a_k^2\Big)\ \lesssim\ \sum_{k\ge0}2^{-2k/3} a_k^2,
\]
and hence
\begin{equation}\label{eq:ELbound}
\mathbf{E}[L](t;x,r)\ \le\ C\,\frac{r}{|B_r|}\sum_{k\ge0}2^{-2k/3}\int_{t}^{t+r^2}\!\!\|\omega(\cdot,s)\|_{L^{3/2}(B_{2^{k+1}r})}^{2}\,ds.
\end{equation}
By Hölder on $B_R$ and the Carleson–square hypothesis from Section~0, for every interval $I$ of length $R^2$,
\[
\int_{I}\!\|\omega(\cdot,s)\|_{L^{3/2}(B_{R})}^{2}\,ds
\;\le\; C\,R^{-1}\int_{I}\!\Big(\int_{B_R} |\omega|^{3/2}\Big)^{\!4/3}\! ds
\;\le\; C\,\varepsilon^{4/3}\,R^{5/3}.
\]
Taking $I=[t,t+r^2]$ (note $r^2\le R^2$ for all $k\ge0$) and $R=2^{k+1}r$ gives
\[
\int_{t}^{t+r^2}\!\!\|\omega(\cdot,s)\|_{L^{3/2}(B_{2^{k+1}r})}^{2}\,ds
\;\le\; C\,\varepsilon^{4/3}\,2^{5k/3}\,r^{5/3}.
\]
Combining with \eqref{eq:ELbound} and $|B_r|\simeq r^3$ we obtain
\[
\mathbf{E}[L](t;x,r)\ \le\ C\,\frac{r}{r^3}\sum_{k\ge0}2^{-2k/3}\cdot \varepsilon^{4/3}\,2^{k/3}\,r^{1/3}
\;\le\; C\,\frac{r}{r^3}\sum_{k\ge0}2^{-2k/3}\cdot \varepsilon^{4/3}\,2^{5k/3}\,r^{5/3}.
\]
In particular, this estimate no longer yields a uniform-in-$r$ bound as $r\downarrow0$; a scale-consistent version of Lemma~\ref{lem:B} must be re-derived under the corrected normalizations.
\smallskip
\noindent
\textbf{Conclusion / open gap.}
Under the corrected (scale-invariant) normalization $\mathcal W=r^{-2}\iint_{Q_r}|\omega|^{3/2}$, the dyadic Biot--Savart estimate \eqref{eq:BS} combined with the available square--Carleson control does \emph{not} yield a uniform (in $r$) Carleson-box bound for the linear piece $L(t,\tau)=u(t+\tau)$.
Controlling this far-field/time-scale mismatch (large spatial radii $2^{k}r$ on short time windows of length $r^2$) is exactly the missing aggregation step; closing it would amount to a new quantitative bridge from the scale-invariant vorticity Morrey control to a small $BMO^{-1}$ slice.
\end{proof}

(References for Section 3: Biot–Savart and dyadic split \cite{MajdaBertozzi2002,Stein1993}; heat-kernel smoothing and time convolution are standard; the Carleson/tent-space perspective is from \cite{KochTataru2001}. For parabolic self-improvement see DiBenedetto \cite{DiBenedetto1993}; for backward uniqueness see Escauriaza–Seregin–\v{S}ver\'ak \cite{EscauriazaSereginSverak2003}.)
\paragraph{Remarks.}
(1) The exponent $2/3$ is forced by scaling: $\mathcal{W}$ is invariant, whereas the $BMO^{-1}$ Carleson norm is quadratic in $e^{\nu\tau\Delta}u$ and integrates over a region of parabolic volume $|B_r|\,r^2$.

(2) The bridge uses, in addition to dyadic Biot–Savart and heat smoothing, a time–square control now derived from the $\mathcal{W}$ hypothesis via Appendix~H (Theorem~\ref{thm:C2S-from-W}).

(3) Parabolic scaling reduces the general case $[t_0-1,t_0]$ to the normalized window handled above and preserves the constant $C_B$.

\section*{4. Compactness and the ancient critical element}

We recall the scale–critical vorticity profile
\[
\mathcal{W}(x,t;r):=\frac{1}{r^2}\iint_{Q_r(x,t)}|\omega|^{3/2}\,dx\,ds,
\qquad
\mathcal{M}(t):=\sup_{x\in\mathbb{R}^3,\ r>0}\mathcal{W}(x,t;r),
\]
with $Q_r(x,t)=B_r(x)\times[t-r^2,t]$ and $\omega=\nabla\times u$.

\begin{definition}[Minimal blow-up profile]\label{def:Mc}
If a solution $u$ loses smoothness at time $T$, set
\[
\mathcal{M}_c(u):=\limsup_{t\uparrow T}\mathcal{M}(t),
\qquad
\mathcal{M}_c:=\inf\{\ \mathcal{M}_c(u)\ :\ u \text{ blows up}\ \}.
\]
\end{definition}

We extract a \emph{critical element} by zooming on near–maximizers of $\mathcal{W}$ and passing to a limit of rescaled flows. Two basic ingredients are used throughout: (i) compactness for suitable weak solutions on bounded cylinders; (ii) semicontinuity of $\mathcal{W}$ under local convergence.

\subsection*{4.1. Local compactness and semicontinuity}

\begin{lemma}[Local compactness for suitable solutions]\label{lem:local-compact}
Fix $R>1$. Let $(u^{(n)},p^{(n)})$ be suitable weak solutions on $Q_R:=B_R\times(-R^2,0]$ with a uniform bound
\[
\iint_{Q_R}\Big(|u^{(n)}|^3 + |p^{(n)}|^{3/2}\Big)\,dx\,dt \;\le\; C_R<\infty.
\]
Then, up to a subsequence,
\[
u^{(n)}\to u \quad\text{strongly in }L^3\big(Q_{R/2}\big),\qquad
p^{(n)}\rightharpoonup p \quad\text{weakly in }L^{3/2}\big(Q_{R/2}\big),
\]
and $(u,p)$ is a suitable weak solution on $Q_{R/2}$.
\end{lemma}

\begin{proof}
The local energy inequality (Caffarelli–Kohn–Nirenberg \cite{CKN1982}), standard cutoff/pressure decompositions, and Calderón–Zygmund bounds imply uniform control of $u^{(n)}$ in $L^2_t H^1_x(Q_{R'})$ and in $L^{10/3}(Q_{R'})$ for every $1<R'<R$. Moreover $\partial_t u^{(n)}$ is uniformly bounded in $L^{5/4}_t H^{-1}_x(Q_{R'})$. Aubin–Lions (Simon's formulation \cite{Simon1987}) gives precompactness of $u^{(n)}$ in $L^3(Q_{R'/2})$; shrinking $R'$ to $R$ and diagonalizing yields strong $L^3$ convergence on $Q_{R/2}$. The pressure follows by weak compactness of Calderón–Zygmund operators on $L^{3/2}$, and suitability passes to the limit by lower semicontinuity in the local energy inequality. 
\end{proof}

\begin{lemma}[Semicontinuity of the critical profile]\label{lem:W-lsc}
Let $U^{(n)}\to U$ in $L^3_{\mathrm{loc}}(\mathbb{R}^3\times(-\infty,0])$, with $U^{(n)}$ and $U$ suitable. Then for every fixed cylinder $Q_\rho(y,s)$,
\[
\iint_{Q_\rho(y,s)} |\Omega|^{3/2}\,dx\,dt \;\le\; \liminf_{n\to\infty}\ \iint_{Q_\rho(y,s)} |\Omega^{(n)}|^{3/2}\,dx\,dt,
\]
where $\Omega^{(n)}=\nabla\times U^{(n)}$ and $\Omega=\nabla\times U$. Consequently,
\[
\sup_{(y,s)\in\mathbb{R}^3\times(-\infty,0],\ \rho>0}\ \mathcal{W}_U(y,s;\rho)
\;\le\; \liminf_{n\to\infty}\ \sup_{(y,s),\rho}\ \mathcal{W}_{U^{(n)}}(y,s;\rho).
\]
\end{lemma}

\begin{lemma}[Lower semicontinuity of the square--Carleson profile]\label{lem:C-lsc}
Let $U^{(n)}\to U$ in $L^3_{\mathrm{loc}}(\mathbb{R}^3\times I)$ on a time interval $I$, with $U^{(n)}$ and $U$ suitable and $\Omega^{(n)}=\nabla\times U^{(n)}$, $\Omega=\nabla\times U$. For each $s\in I$, define
\[
\mathcal C_{U}(s):= \sup_{x\in\mathbb{R}^3,\ r>0}\ \frac{1}{r^2}\int_{s-r^2}^{s}\!\Big(\int_{B_r(x)} |\Omega(y,\tau)|^{3/2}\,dy\Big)^{\!4/3} d\tau.
\]
Then for every fixed $s\in I$,
\[
\mathcal C_{U}(s)\ \le\ \liminf_{n\to\infty}\ \mathcal C_{U^{(n)}}(s).
\]
\end{lemma}

\begin{proof}
Fix $x\in\mathbb{R}^3$ and $r>0$. Set $I_{r,s}:=[s-r^2,s]$. On $B_r(x)\times I_{r,s}$, strong $L^3$ convergence of $U^{(n)}$ implies weak convergence of $\Omega^{(n)}$ in $L^{3/2}$, hence for a.e. $\tau$,
\[
\int_{B_r(x)} |\Omega(y,\tau)|^{3/2}\,dy\ \le\ \liminf_{n\to\infty}\int_{B_r(x)}|\Omega^{(n)}(y,\tau)|^{3/2}\,dy.
\]
By Fatou and the convexity of $z\mapsto z^{4/3}$,
\[
\int_{I_{r,s}}\!\Big(\int_{B_r(x)} |\Omega|^{3/2}\Big)^{\!4/3} d\tau\ \le\ \liminf_{n\to\infty}\int_{I_{r,s}}\!\Big(\int_{B_r(x)} |\Omega^{(n)}|^{3/2}\Big)^{\!4/3} d\tau.
\]
Divide by $r^2$ and take the supremum over $(x,r)$; then take $\liminf_{n}$ to obtain the claim.
\end{proof}

\begin{lemma}[Unit-scale bound for $\mathcal W$ from $\mathcal C$]\label{lem:W-from-C}
For every $s$ and every $x\in\mathbb{R}^3$, if
\[
\int_{s-1}^{s}\!\Big(\int_{B_1(x)} |\omega|^{3/2}\Big)^{\!4/3} d\tau\ \le\ \Lambda^{4/3},
\]
then
\[
\mathcal W(x,s;1)=\int_{s-1}^{s}\!\int_{B_1(x)} |\omega|^{3/2}\,dy\,d\tau\ \le\ \Lambda.
\]
In particular, $\mathcal W(0,0;1)\le \mathcal C(0)^{3/4}$.
\end{lemma}

\begin{proof}
Set $f(\tau):=\int_{B_1(x)}|\omega(y,\tau)|^{3/2}\,dy\ge 0$.
By H\"older in time on $(s-1,s)$ with exponents $(4/3,4)$,
\[
\int_{s-1}^{s} f(\tau)\,d\tau
\ \le\ \Bigl(\int_{s-1}^{s} f(\tau)^{4/3}\,d\tau\Bigr)^{3/4}\,(1)^{1/4}
\ \le\ \Lambda.
\]
Since $\mathcal W(x,s;1)=\int_{s-1}^{s} f(\tau)\,d\tau$, this proves the claim. The final statement follows by taking $\Lambda^{4/3}=\mathcal C(0)$.
\end{proof}

\subsection*{4.2. Extraction of the critical element}

\begin{proposition}[Critical element]\label{prop:crit_elem}
There exist solutions $u^{(n)}$ with blow-up times $T_n<\infty$ and times $t_n\uparrow T_n$, together with points $x_n\in\mathbb{R}^3$ and radii $r_n>0$, such that the rescaled fields
\[
U^{(n)}(y,s):=r_n\,u^{(n)}\!\big(x_n+r_n y,\ t_n+r_n^2 s\big),\qquad 
P^{(n)}(y,s):=r_n^2\,p^{(n)}\!\big(x_n+r_n y,\ t_n+r_n^2 s\big),
\]
are suitable weak solutions on $\mathbb{R}^3\times(-S_n,0]$ with $S_n\to\infty$, and after passing to a subsequence
\[
U^{(n)}\to U \quad\text{in }L^3_{\mathrm{loc}}(\mathbb{R}^3\times(-\infty,0]),
\]
where $U$ is a nontrivial ancient suitable weak solution on $\mathbb{R}^3\times(-\infty,0]$ obeying
\[
\sup_{(y,s)\in\mathbb{R}^3\times(-\infty,0],\ \rho>0}\ \mathcal{W}_U(y,s;\rho)\ \le\ \mathcal{M}_c.
\]
Moreover, by postcomposing $U$ with a symmetry (space–time translation and scaling), one may arrange
\[
\mathcal{W}_U(0,0;1)\;=\;\sup_{(y,s),\rho}\ \mathcal{W}_U(y,s;\rho).
\]
(In Section~7 we prove $\sup_{(y,s),\rho}\mathcal{W}_U(y,s;\rho)=\mathcal{M}_c$, hence $\mathcal{W}_U(0,0;1)=\mathcal{M}_c$.)
\end{proposition}

\begin{proof}
\emph{Step 1 (Choice of near–maximizers and normalization).}
By definition of $\mathcal{M}_c$ there exist blowing-up Leray–Hopf solutions $u^{(n)}$ with blow-up times $T_n$ so that
\[
\lim_{n\to\infty}\ \limsup_{t\uparrow T_n}\ \mathcal{M}_{u^{(n)}}(t)\;=\;\mathcal{M}_c.
\]
Pick $t_n\uparrow T_n$ such that
\[
\mathcal{M}_{u^{(n)}}(t_n)\ \ge\ \mathcal{M}_c-\tfrac1n.
\]
For each $n$ choose $(x_n,r_n)$ $\tfrac1n$–near-maximizing:
\[
\mathcal{W}\big(x_n,t_n;r_n\big)\ \ge\ \mathcal{M}_{u^{(n)}}(t_n)-\tfrac1n\ \ge\ \mathcal{M}_c-\tfrac2n.
\]
Rescale around $(x_n,t_n;r_n)$ to the normalized fields $(U^{(n)},P^{(n)})$ above. Then
\begin{equation}\label{eq:unit-cylinder-near-Mc}
\mathcal{W}_{U^{(n)}}(0,0;1)=\iint_{Q_1}|\Omega^{(n)}|^{3/2}\,dx\,ds\ \ge\ \mathcal{M}_c-\tfrac2n.
\end{equation}

\emph{Step 2 (Uniform local bounds and compactness).}
Fix $R>1$. The scale invariance of $\mathcal{W}$ and the near–maximizing choice imply
\[
\sup_{(y,s)\in\mathbb{R}^3\times(-R^2,0],\ \rho\in(0,R]} \ \mathcal{W}_{U^{(n)}}(y,s;\rho)\ \le\ \mathcal{M}_{u^{(n)}}(t_n)\ +\ \tfrac1n\ \le\ \mathcal{M}_c+1
\]
for all large $n$. Using on each time slice the representation $\nabla U^{(n)}=\mathcal{R}\,\Omega^{(n)}$ and Sobolev–Poincaré on balls, one obtains from the uniform bound on $\Omega^{(n)}$ in $L^{3/2}(Q_R)$ a uniform bound on $U^{(n)}$ in $L^{3}(Q_R)$; standard pressure decomposition gives the matching $L^{3/2}(Q_R)$ bound on $P^{(n)}$. Applying Lemma~\ref{lem:local-compact} on $Q_R$ and then diagonalizing over $R\to\infty$ yields
\[
U^{(n)}\to U\quad\text{strongly in }L^3_{\mathrm{loc}}(\mathbb{R}^3\times(-\infty,0]),
\]
with $(U,P)$ suitable on $\mathbb{R}^3\times(-\infty,0]$.

\emph{Step 3 (Ancientness, nontriviality, and profile bound).}
By construction $U$ is defined on all backward times $s\le0$; hence it is ancient. Lemma~\ref{lem:W-lsc} implies, for every cylinder,
\[
\mathcal{W}_U(y,s;\rho)\ \le\ \liminf_{n\to\infty}\ \mathcal{W}_{U^{(n)}}(y,s;\rho)\ \le\ \liminf_{n\to\infty}\ \mathcal{M}_{u^{(n)}}(t_n)\ =\ \mathcal{M}_c,
\]
so $\sup_{(y,s),\rho}\mathcal{W}_U(y,s;\rho)\le \mathcal{M}_c$. Nontriviality follows from the near–maximization \eqref{eq:unit-cylinder-near-Mc}: if $U\equiv0$ on $Q_1$, then by strong $L^3$ convergence of $U^{(n)}$ on $Q_1$ and the elliptic control $\nabla U^{(n)}=\mathcal{R}\,\Omega^{(n)}$ one gets $\Omega^{(n)}\to0$ in $\mathcal{D}'(Q_1)$ and hence $\iint_{Q_1}|\Omega^{(n)}|^{3/2}\to 0$, contradicting \eqref{eq:unit-cylinder-near-Mc}.

\emph{Step 4 (Saturation at the origin after in–orbit renormalization).}
Let $M_U:=\sup_{(y,s),\rho}\mathcal{W}_U(y,s;\rho)\in(0,\mathcal{M}_c]$. Choose a maximizing sequence $(y_m,s_m;\rho_m)$ in $\mathbb{R}^3\times(-\infty,0]\times(0,\infty)$ with
\[
\mathcal{W}_U(y_m,s_m;\rho_m)\ \uparrow\ M_U.
\]
Define rescaled/translated flows
\[
V^{(m)}(z,\tau):=\rho_m\,U\!\big(y_m+\rho_m z,\ s_m+\rho_m^2 \tau\big).
\]
By the same compactness as in Step~2, a subsequence $V^{(m)}\to V$ in $L^3_{\mathrm{loc}}$, with $V$ ancient suitable. By construction,
\[
\mathcal{W}_V(0,0;1)\ =\ \lim_{m\to\infty}\ \mathcal{W}_U(y_m,s_m;\rho_m)\ =\ M_U,
\]
and $V$ enjoys $\sup_{(z,\tau),\rho}\mathcal{W}_V(z,\tau;\rho)=M_U$. Renaming $V$ as $U$ concludes the proof of the saturation statement. The identification $M_U=\mathcal{M}_c$ will be proved in Section~7 (threshold closure), after which $\mathcal{W}_U(0,0;1)=\mathcal{M}_c$ follows.
\end{proof}

\paragraph{Comments on the construction.}
(1) The only inputs are scale–invariance, local compactness of suitable solutions, and weak lower semicontinuity of the convex functional $f\mapsto\int|f|^{3/2}$. No smallness is used here.

(2) The possible strict inequality $\sup_{(y,s),\rho}\mathcal{W}_U(y,s;\rho)<\mathcal{M}_c$ at this stage is resolved in Section~7 by the density–drop argument: once the $\varepsilon$–improvement on smaller cylinders is available, an open/closed scheme pins the supremal safe level and forces $\sup\mathcal{W}_U=\mathcal{M}_c$, completing the critical–element characterization.

\section*{5. Density-drop (De~Giorgi improvement) on smaller cylinders}

Throughout the section we write $\omega=\nabla\times u$, $\theta:=|\omega|$, and work on normalized cylinders
\[
Q_r:=B_r(0)\times[-r^2,0],\qquad Q_1=B_1\times[-1,0],\qquad Q_\vartheta=B_\vartheta\times[-\vartheta^2,0],
\]
with the fixed choice $\vartheta=\tfrac14$ from Section~0. Recall the scale-invariant vorticity functional
\[
\mathcal{W}(0,0;r)=\frac1{r^2}\iint_{Q_r}\theta^{3/2}.
\]

\begin{lemma}[Density-drop]\label{lem:density_drop}
With $\vartheta=\tfrac14$ and $c=\tfrac34$, there exists $\eta_1>0$ (universal) such that for any suitable solution on $Q_1$ with
\[
\sup_{r\in(0,2]}\ \frac{1}{r^2}\int_{-1}^{0}\!\Big(\int_{B_r} |\omega|^{3/2}\Big)^{\!4/3}\! ds\ \le\ \varepsilon_0^{4/3},
\]
and
\[
\mathcal{W}(0,0;1)\ \le\ \varepsilon_0+\eta\qquad\text{and}\qquad \eta\in(0,\eta_1],
\]
one has
\[
\mathcal{W}(0,0;\vartheta)\ \le\ \varepsilon_0+c\,\eta.
\]
\end{lemma}

\begin{proof}
\emph{Step 1 (Truncation, cutoffs, and absorbed Caccioppoli via (C2S$^{\square}$)).}
Fix the truncation level
\[
\kappa_0:=K_0\,\varepsilon_0^{2/3},\qquad K_0\ge1\ \text{to be chosen below},
\]
and write $w:=(\theta-\kappa_0)_+$. Let $\eta\in C_c^\infty(Q_1)$ be a space-time cutoff and $p\ge0$. Testing the Kato inequality
\[
\partial_t\theta+u\!\cdot\!\nabla\theta-\nu\Delta\theta\ \le\ |(\omega\!\cdot\!\nabla)u|
\]
against $\eta^2 w^p$, integrating by parts, using $\operatorname{div}u=0$, and estimating the stretching by Calderón–Zygmund and slice Sobolev ($H^1\hookrightarrow L^6$), one obtains the \emph{absorbed Caccioppoli inequality}
\begin{equation}\label{eq:absCacc}
\sup_{t}\!\int \eta^2 w^{p+1} + \frac{\nu}{2}\!\iint \big|\nabla(\eta\,w^{\frac{p+1}{2}})\big|^2
\ \le\ C\!\iint\!\Big(|\partial_t\eta|\,\eta+|\nabla\eta|^2\Big)w^{p+1}
+ C\!\iint\!|u|\,|\nabla\eta|\,\eta\,w^{p+1},
\end{equation}
where the stretching has been absorbed into the left using the square–Carleson smallness on $Q_1$ (Appendix~A.2', Lemma~\ref{lem:absorb-C2S-square}); any baseline incurred is dominated later by the choice of $\kappa_0$. In what follows, $C$ denotes universal constants independent of $\varepsilon_0,\eta$.

\medskip
\emph{Step 2 (De~Giorgi iteration ladder).}
Choose radii and cutoffs by
\[
r_0:=1,\qquad r_{k+1}:=\tfrac12(r_k+\vartheta),\qquad \eta_k\in C_c^\infty(Q_{r_k}),\ \ \eta_k\equiv1\ \text{ on }Q_{r_{k+1}},
\]
and exponents $p_k:=2(3/2)^k-1$. The cutoff geometry gives $|\nabla\eta_k|\lesssim 2^k$, $|\partial_t\eta_k|\lesssim 2^{2k}$.
From \eqref{eq:absCacc} and the slice Sobolev embedding, one obtains the standard De~Giorgi gain
\begin{equation}\label{eq:DGstep}
\|\eta_k\,w^{\frac{p_k+1}{2}}\|_{L_t^2L_x^6}^2\ \lesssim\ 2^{2k}\!\iint_{Q_{r_k}} w^{p_k+1}
\quad\Longrightarrow\quad
\|w\|_{L^{\frac32(p_k+1)}(Q_{r_{k+1}})}\ \le\ C\,2^{\alpha k}\,\|w\|_{L^{p_k+1}(Q_{r_k})},
\end{equation}
for universal $C,\alpha>0$. Iterating \eqref{eq:DGstep} a fixed number of times (say $k=0,1,\dots,6$ so that $p_6+1>16$) and using Hölder on $Q_{r_k}$, one arrives at
\begin{equation}\label{eq:DGcascade}
\|w\|_{L^{p_6+1}(Q_{r_6})}\ \le\ \rho_0\,\|w\|_{L^{3/2}(Q_1)}\ +\ C\,\kappa_0^{3/2}\,|Q_{r_6}|,
\end{equation}
with a universal contraction $\rho_0\in(0,1)$ stemming from the fixed decrease of radii $r_k\searrow \vartheta$. Since $r_6\ge \vartheta$, enlarging to $Q_\vartheta$ only adjusts constants, and we may rewrite \eqref{eq:DGcascade} as
\begin{equation}\label{eq:contract_w}
\iint_{Q_\vartheta} w^{3/2}\ \le\ \rho\,\iint_{Q_1} w^{3/2}\ +\ C\,\kappa_0^{3/2}\,|Q_\vartheta|,
\qquad \text{with some universal } \rho\in(0,1).
\end{equation}

\medskip
\emph{Step 3 (Splitting $\theta$ and transferring the contraction).}
On $Q_\vartheta$, split $\theta=\kappa_0+w$ and use the elementary inequality
\begin{equation}\label{eq:split32}
(\kappa_0+w)^{3/2}\ \le\ \kappa_0^{3/2}\ +\ C\,\kappa_0^{1/2}\,w\ +\ C\,w^{3/2}.
\end{equation}
Integrating \eqref{eq:split32} over $Q_\vartheta$ and applying Hölder to the $w$ term yields
\begin{equation}\label{eq:thetaQv}
\iint_{Q_\vartheta}\theta^{3/2}\ \le\ C_1\,\kappa_0^{3/2}|Q_\vartheta|\ +\ C_2\,|Q_\vartheta|^{1/3}\!\left(\iint_{Q_\vartheta} w^{3/2}\right)^{\!2/3}
\ +\ C_3\!\iint_{Q_\vartheta} w^{3/2}.
\end{equation}
Invoking the contraction \eqref{eq:contract_w} and the inequality $a^{2/3}\le a +1$ to handle the $2/3$ power, we obtain
\begin{equation}\label{eq:thetaQv_after_contract}
\iint_{Q_\vartheta}\theta^{3/2}\ \le\ A_1\,\kappa_0^{3/2}|Q_\vartheta|\ +\ A_2\,\rho\,\iint_{Q_1} w^{3/2}\ +\ A_3\,\kappa_0^{3/2}|Q_\vartheta|,
\end{equation}
with universal $A_j$. Absorbing the two baseline terms gives
\begin{equation}\label{eq:thetaQv_simplified}
\iint_{Q_\vartheta}\theta^{3/2}\ \le\ A\,\kappa_0^{3/2}|Q_\vartheta|\ +\ B\,\rho\,\iint_{Q_1} w^{3/2}.
\end{equation}

\medskip
\emph{Step 4 (From $w$ to the excess and choice of parameters).}
Note that $w=(\theta-\kappa_0)_+\le \theta$ and hence $\iint_{Q_1} w^{3/2}\le \iint_{Q_1}\theta^{3/2}$. Using the hypothesis $\iint_{Q_1}\theta^{3/2}\le \varepsilon_0+\eta$ together with \eqref{eq:thetaQv_simplified} and the definition of the critical functional (remember $\mathcal{W}(0,0;r)=r^{-2}\iint_{Q_r}\theta^{3/2}$), we obtain
\begin{equation}\label{eq:pre_density_drop}
\mathcal{W}(0,0;\vartheta)\ \le\ \underbrace{\frac{A}{\vartheta^2}\,\kappa_0^{3/2}|Q_\vartheta|}_{\text{baseline}}\ +\ \rho\,B\,\frac{1}{\vartheta^2}\,(\varepsilon_0+\eta).
\end{equation}
By scaling, $|Q_\vartheta|=\vartheta^5|Q_1|$ and $\kappa_0^{3/2}=K_0^{3/2}\varepsilon_0$. Thus the baseline term equals
\[
\frac{A}{\vartheta^2}\,\kappa_0^{3/2}|Q_\vartheta|\ =\ A\,K_0^{3/2}\,\vartheta^{3}\,\varepsilon_0.
\]
Fix $K_0$ (universal) so that $A\,K_0^{3/2}\,\vartheta^{3}\le \tfrac12$; with our choice $\vartheta=\tfrac14$ this is harmless (constants are universal), and gives
\[
\frac{A}{\vartheta^2}\,\kappa_0^{3/2}|Q_\vartheta|\ \le\ \tfrac12\,\varepsilon_0.
\]
For the second term in \eqref{eq:pre_density_drop}, note that $\rho\in(0,1)$ is universal (coming from the fixed geometry of the iteration). Since $\vartheta$ is fixed, we may rewrite \eqref{eq:pre_density_drop} as
\[
\mathcal{W}(0,0;\vartheta)\ \le\ \Big(\tfrac12+\rho\,\tilde B\Big)\varepsilon_0\ +\ \rho\,\tilde B\,\eta,
\]
for some universal $\tilde B>0$ (absorbing $\vartheta^{-2}$). Adjust $K_0$ further if needed so that $\tfrac12+\rho\,\tilde B\le 1$; this is possible because $\rho,\tilde B$ are fixed numbers, independent of $\varepsilon_0,\eta$. With this choice,
\begin{equation}\label{eq:almost_there}
\mathcal{W}(0,0;\vartheta)\ \le\ \varepsilon_0\ +\ \rho\,\tilde B\,\eta.
\end{equation}

\medskip
\emph{Step 5 (Numerical contraction to $c=\tfrac34$).}
Set $c:=\tfrac34$. Since the contraction factor $\rho\,\tilde B$ in \eqref{eq:almost_there} is universal, there exists $\eta_1>0$ small enough (again universal) so that the subleading terms neglected in \eqref{eq:thetaQv_after_contract} (coming from $(\cdot)^{2/3}$ and the cutoff geometry) are bounded by $\tfrac14\,\eta$ whenever $\eta\le \eta_1$. Consequently, with $K_0$ fixed as above and $\eta\in(0,\eta_1]$,
\[
\mathcal{W}(0,0;\vartheta)\ \le\ \varepsilon_0\ +\ \Big(\rho\,\tilde B+\tfrac14\Big)\eta\ \le\ \varepsilon_0+\tfrac34\,\eta,
\]
after possibly enlarging $K_0$ once more to ensure $\rho\,\tilde B\le \tfrac12$. This is the claimed density-drop with $\vartheta=\tfrac14$ and $c=\tfrac34$.
\end{proof}

\paragraph{Remarks.}
(1) The proof is scale-free: the same argument at general radius $r$ gives the identical contraction on $Q_{\vartheta r}$, and dividing by $r^2$ preserves the functional $\mathcal{W}$.  

(2) Only the absorbed Caccioppoli inequality at the critical $L^{3/2}$ level and a standard De~Giorgi iteration are used. No structural hypothesis on $u$ beyond suitability enters; the advection contributes only through cutoff terms absorbed by geometry.  

(3) The explicit choices $\vartheta=\tfrac14$ and $c=\tfrac34$ are convenient. Any fixed $\vartheta\in(0,1/2)$ would do; the constant $c\in(0,1)$ depends only on $\vartheta$ and the universal constants in the Caccioppoli and Sobolev steps.

\section*{6. Threshold identification and small-data gate}

\subsection*{6.0. Carleson-square safe level and lower bound}

For $t\in\mathbb{R}$, define the square–Carleson vorticity profile
\[
\mathcal{C}(t)
:= \sup_{x\in\mathbb{R}^3,\ r>0}\ \frac{1}{r^2}\int_{t-r^2}^{t}\!\Big(\int_{B_r(x)} |\omega(y,s)|^{3/2}\,dy\Big)^{\!4/3} ds.
\]

\begin{definition}[Carleson-square supremal safe level]\label{def:Theta-square}
Let $\Theta^{\square}$ denote the supremum of $\eta\ge0$ with the property:
\[
\text{If }\ \mathcal{C}(t_0)<\eta^{4/3}\ \text{ at some time }t_0,\ \text{ then the solution is smooth for all later times.}
\]
\end{definition}

\begin{proposition}[Lower bound for the square–Carleson safe level]\label{prop:lower-bound-square}
With $\varepsilon_0:=\min\{\varepsilon_A,(\varepsilon_{\mathrm{SD}}/C_B)^{3/2}\}$ as in Section~0, one has $\varepsilon_0\le \Theta^{\square}$.
\end{proposition}

\begin{proof}
Fix $\eta\le \varepsilon_0$ and a time $t_0$ with $\mathcal{C}(t_0)<\eta^{4/3}$. On the unit window $[t_0-1,t_0]$, the square–Carleson bound (C2S$^{\square}$) holds with parameter $\eta$. By Lemma~\ref{lem:B} (proved under (C2S$^{\square}$)), there exists $t_*\in[t_0-\tfrac12,t_0]$ with $\|u(\cdot,t_*)\|_{BMO^{-1}}\le C_B\,\eta^{2/3}\le C_B\,\varepsilon_0^{2/3}\le \varepsilon_{\mathrm{SD}}$. The Koch--Tataru small–data theory (Theorem~\ref{thm:SD}) then yields a unique smooth solution forward from $t_*\le t_0$, hence from $t_0$. Therefore every $\eta\le\varepsilon_0$ is safe and $\varepsilon_0\le \Theta^{\square}$.
\end{proof}

\begin{definition}[Supremal safe level]\label{def:Theta}
Let $\Theta$ denote the supremum of $\eta\ge0$ with the property:
\[
\text{If }\ \mathcal{M}(t)<\eta\ \text{ at some time }t,\ \text{ then the solution is smooth for all later times.}
\]
\end{definition}

\begin{theorem}[Small-data $BMO^{-1}$ well-posedness]\label{thm:SD}
There exists $\varepsilon_{\mathrm{SD}}>0$ such that whenever $\|u_0\|_{BMO^{-1}}\le \varepsilon_{\mathrm{SD}}$, the Navier--Stokes equations admit a unique global mild solution, which is smooth for $t>0$.
\end{theorem}

\begin{proposition}[Threshold equality]\label{prop:threshold}
One has $\Theta=\varepsilon_0$ and $\mathcal{M}_c=\varepsilon_0$ provided the Step~1 lower–bound argument is understood in the square–Carleson form of Proposition~\ref{prop:lower-bound-square} (i.e., with $\Theta$ replaced by $\Theta^{\square}$ for that step).
\end{proposition}

\begin{proof}
\emph{Step 1 (square–Carleson lower bound).} Proposition~\ref{prop:lower-bound-square} gives $\varepsilon_0\le \Theta^{\square}$. This replaces the use of Lemma~\ref{lem:B} under the weaker $\sup\mathcal W$ hypothesis.

\medskip
\emph{Step 2: Closedness at the edge via density drop.}
Assume for contradiction that $\Theta>\varepsilon_0$. By definition of $\Theta$, there exist solutions $u^{(n)}$, times $t_n$, and numbers $\delta_n\downarrow 0$ such that
\begin{equation}\label{eq:near-theta}
\mathcal{M}[u^{(n)}](t_n)\in(\Theta-\delta_n,\ \Theta+\delta_n)\qquad\text{and $u^{(n)}$ is \emph{not} smooth for all $t>t_n$.}
\end{equation}
For each $n$ pick $(x_n,r_n)$ with
\[
\mathcal{W}_{u^{(n)}}(x_n,t_n;r_n)\ \ge\ \mathcal{M}[u^{(n)}](t_n)-\delta_n\ \ge\ \Theta-2\delta_n.
\]
Rescale around $(x_n,t_n;r_n)$:
\[
U^{(n)}(y,s):=r_n\,u^{(n)}(x_n+r_n y,\ t_n+r_n^2 s),\qquad P^{(n)}(y,s):=r_n^2\,p^{(n)}(x_n+r_n y,\ t_n+r_n^2 s).
\]
Then $U^{(n)}$ are suitable on $\mathbb{R}^3\times(-S_n,0]$ with $S_n\to\infty$, and
\begin{equation}\label{eq:unit-attainment}
\mathcal{W}_{U^{(n)}}(0,0;1)\ \ge\ \Theta-2\delta_n,\qquad 
\sup_{(y,s)\in\mathbb{R}^3\times[-1,0]}\ \sup_{\rho>0}\ \mathcal{W}_{U^{(n)}}(y,s;\rho)\ \le\ \Theta+\delta_n.
\end{equation}
By local compactness (Lemma~\ref{lem:local-compact}), after a subsequence $U^{(n)}\to U$ in $L^3_{\mathrm{loc}}(\mathbb{R}^3\times(-\infty,0])$ with $(U,P)$ suitable and ancient. Lower semicontinuity (Lemma~\ref{lem:W-lsc}) yields
\[
\sup_{(y,s),\rho}\ \mathcal{W}_{U}(y,s;\rho)\ \le\ \Theta=:A,\qquad\text{and}\qquad \mathcal{W}_U(0,0;1)\ \le\ \liminf_{n}\ \mathcal{W}_{U^{(n)}}(0,0;1).
\]
Postcompose $U$ by a space–time translation and scaling so that its profile is \emph{saturated} at the origin:
\[
\mathcal{W}_U(0,0;1)=\sup_{(y,s),\rho}\mathcal{W}_U(y,s;\rho)=:A\ \le\ \Theta.
\]
If $A=\varepsilon_0$, proceed to Step~3. Otherwise $A>\varepsilon_0$, so we may invoke the density–drop Lemma~\ref{lem:density_drop} with $\eta:=A-\varepsilon_0\in(0,\Theta-\varepsilon_0]$ to obtain
\begin{equation}\label{eq:drop}
\mathcal{W}_U(0,0;\vartheta)\ \le\ \varepsilon_0+c\,(A-\varepsilon_0)\ =\ A-\,(1-c)(A-\varepsilon_0)\ <\ A,
\end{equation}
with the fixed constants $\vartheta=\tfrac14$ and $c=\tfrac34$ of Section~5.

\smallskip
\emph{Stability back to approximants.} By lower semicontinuity, \eqref{eq:drop} propagates to the approximants:
\[
\limsup_{n\to\infty}\ \mathcal{W}_{U^{(n)}}(0,0;\vartheta)\ \le\ \mathcal{W}_U(0,0;\vartheta)\ \le\ A-(1-c)(A-\varepsilon_0).
\]
Hence, for all sufficiently large $n$,
\[
\mathcal{W}_{U^{(n)}}(0,0;\vartheta)\ \le\ A-\tfrac12(1-c)(A-\varepsilon_0).
\]
Undo the scaling to $(x_n,t_n;r_n)$ (using the scale invariance of $\mathcal{W}$) to conclude that at the \emph{same} center $(x_n,t_n)$ one has, at the smaller radius $\vartheta r_n$,
\[
\mathcal{W}_{u^{(n)}}(x_n,t_n;\vartheta r_n)\ \le\ A-\tfrac12(1-c)(A-\varepsilon_0)\ \le\ \Theta-\tfrac12(1-c)(A-\varepsilon_0).
\]
But by construction (near-maximization at scale $r_n$ and the bound \eqref{eq:unit-attainment}), we also have
\[
\mathcal{W}_{u^{(n)}}(x_n,t_n;r_n)\ \ge\ \Theta-2\delta_n.
\]
Combining and letting $n\to\infty$ shows that, at the very \emph{point and time} where the global profile is (asymptotically) achieved, shrinking the radius by the fixed factor $\vartheta$ lowers the profile by a definite amount. Iterating this radius shrink $m$ times and applying the same argument yields, for large $n$,
\[
\mathcal{W}_{u^{(n)}}(x_n,t_n;\vartheta^m r_n)\ \le\ \varepsilon_0 + c^m (A-\varepsilon_0) + o(1).
\]
Choose $m$ large so that $c^m (A-\varepsilon_0)\le \tfrac12(\Theta-\varepsilon_0)$. Then for all large $n$,
\[
\mathcal{W}_{u^{(n)}}(x_n,t_n;\vartheta^m r_n)\ \le\ \varepsilon_0+\tfrac12(\Theta-\varepsilon_0)\ <\ \Theta.
\]
Since $\mathcal{M}[u^{(n)}](t_n)$ is the \emph{supremum} over all $(x,r)$, the value $\mathcal{M}[u^{(n)}](t_n)$ cannot be strictly less than $\Theta$ at arbitrarily many radii centered at a near-maximizer $(x_n,t_n)$ while simultaneously staying within $(\Theta-\delta_n,\Theta+\delta_n)$ as in \eqref{eq:near-theta}; this contradicts the choice of $(x_n,r_n)$ as near-maximizers and the convergence $\delta_n\downarrow0$. Therefore the assumption $A>\varepsilon_0$ is false, and $\sup_{(y,s),\rho}\mathcal{W}_U(y,s;\rho)=A=\varepsilon_0$.

\medskip
\emph{Step 3: Edge equality $\Theta=\varepsilon_0$.}
We have shown: any sequence of near-counterexamples at level $\Theta$ compactly produces an ancient suitable limit $U$ with saturated profile $A=\varepsilon_0$. Since Step~1 already gave $\varepsilon_0\le\Theta$, it follows that $\Theta=\varepsilon_0$.

\medskip
\emph{Step 4: Identification $\mathcal{M}_c=\varepsilon_0$.}
By definition of $\Theta$, no blow-up can occur if $\limsup_{t\uparrow T}\mathcal{M}(t)<\Theta$, hence every blow-up satisfies $\limsup_{t\uparrow T}\mathcal{M}(t)\ge \Theta$. Taking the infimum over all blow-ups, we get $\mathcal{M}_c\ge \Theta=\varepsilon_0$. On the other hand, the critical-element extraction of Section~4 (Proposition~\ref{prop:crit_elem}) applied to a minimizing sequence shows that the associated ancient limit has saturated profile equal to $\mathcal{M}_c$; Step~2 forces this saturation level to be exactly $\varepsilon_0$. Hence $\mathcal{M}_c=\varepsilon_0$.
\end{proof}

\paragraph{What was used and where.}
Classical inputs: Koch--Tataru small-data theory in $BMO^{-1}$ \cite{KochTataru2001}; compactness and LEI framework \cite{CKN1982,Simon1987}; harmonic analysis tools for Biot--Savart and CZ \cite{Stein1993,MajdaBertozzi2002}.
The identification $\Theta=\varepsilon_0=\mathcal{M}_c$ relies on: (i) the $L^{3/2}\!\to BMO^{-1}$ Carleson slice bridge (Lemma~\ref{lem:B}) and the small-data gate (Theorem~\ref{thm:SD}) to show $\varepsilon_0\le\Theta$; (ii) compactness for suitable solutions and lower semicontinuity of $\mathcal{W}$ to extract an ancient limit at the edge; (iii) the density–drop (Lemma~\ref{lem:density_drop}) to improve the profile at smaller cylinders and rule out $A>\varepsilon_0$; and (iv) the critical-element construction of Section~4 to transfer the edge equality to $\mathcal{M}_c$.

\section*{7. Density--drop and threshold closure}

This section packages the $\varepsilon$–improvement on smaller cylinders together with the open/closed stability step that pins the threshold. We work at the normalized cylinder $Q_1=B_1\times[-1,0]$ and then appeal to scaling.

\subsection*{7.1. Density--drop (restated and iterated)}
Recall the density--drop proved in Section~5.

\begin{lemma}[Density--drop, restated]\label{lem:density_drop_restated}
With the fixed constants $\vartheta=\tfrac14$ and $c=\tfrac34$, there exists $\eta_1>0$ such that, for any suitable solution on $Q_1$,
\[
\mathcal W(0,0;1)\ \le\ \varepsilon_0+\eta,\qquad \eta\in(0,\eta_1]
\quad\Longrightarrow\quad
\mathcal W(0,0;\vartheta)\ \le\ \varepsilon_0+c\,\eta.
\]
\end{lemma}

\begin{proof}
Set $\kappa_0=K_0\,\varepsilon_0^{2/3}$ and $w=(|\omega|-\kappa_0)_+$. The absorbed Caccioppoli inequality for $w$, combined with a De~Giorgi iteration on a fixed chain of shrinking cylinders, yields the contraction
\[
\iint_{Q_\vartheta} w^{3/2}\ \le\ \rho\,\iint_{Q_1} w^{3/2}\ +\ C\,\kappa_0^{3/2}|Q_\vartheta|\qquad(\rho\in(0,1)).
\]
Splitting $|\omega|=\kappa_0+w$ and using $(\kappa_0+w)^{3/2}\le \kappa_0^{3/2}+C\kappa_0^{1/2}w+Cw^{3/2}$ transfers the contraction to $\iint_{Q_\vartheta}|\omega|^{3/2}$, hence to $\mathcal W(0,0;\vartheta)$. The constants were fixed in Section~5 so that the baseline contribution is $\le\varepsilon_0$ and the excess contracts by $c$.
\end{proof}

The single-step improvement propagates to an \emph{iterated} improvement at geometrically shrinking radii.

\begin{lemma}[Iterated density--drop]\label{lem:iter_density}
Under the hypotheses of Lemma~\ref{lem:density_drop_restated}, for every integer $m\ge1$,
\[
\mathcal W(0,0;\vartheta^m)\ \le\ \varepsilon_0\ +\ c^m\,\eta.
\]
\end{lemma}

\begin{proof}
Induct on $m$. The case $m=1$ is Lemma~\ref{lem:density_drop_restated}. If the statement holds at $m$, then
\[
\mathcal W(0,0;\vartheta^{m+1})\ =\ \mathcal W_{\text{rescaled}}\big(0,0;\vartheta\big)
\ \le\ \varepsilon_0\ +\ c\big(\mathcal W_{\text{rescaled}}(0,0;1)-\varepsilon_0\big)
\ =\ \varepsilon_0\ +\ c^{m+1}\eta,
\]
where ``rescaled'' refers to the solution dilated by $\vartheta^{-m}$ so that $Q_{\vartheta^m}$ is mapped to $Q_1$; scale invariance of $\mathcal W$ justifies the equality. This closes the induction.
\end{proof}

\subsection*{7.2. Threshold closure}
We now show that the density--drop pins the edge of the safe set introduced in Section~6.

\begin{proposition}[Threshold closure]\label{prop:threshold_closure}
Let $\Theta$ be as in Definition~\ref{def:Theta}. Then $\Theta=\varepsilon_0$, and consequently the minimal blow-up profile satisfies $\mathcal M_c=\varepsilon_0$.
\end{proposition}

\begin{proof}
By Lemma~\ref{lem:B} and the choice of $\varepsilon_0$, every $\eta\le\varepsilon_0$ is safe, hence $\varepsilon_0\le\Theta$ (Section~6, Step~1).

Assume $\Theta>\varepsilon_0$. By the compactness extraction of Section~4, there exists a nontrivial ancient suitable solution $U$ such that its profile is saturated at the origin and unit radius:
\[
A\ :=\ \mathcal W_U(0,0;1)\ =\ \sup_{(y,s),\rho}\mathcal W_U(y,s;\rho)\ \le\ \Theta.
\]
If $A=\varepsilon_0$ we are done, since then $\Theta\le A=\varepsilon_0$. Suppose $A>\varepsilon_0$. Apply Lemma~\ref{lem:density_drop_restated} to $U$ on $Q_1$ with $\eta=A-\varepsilon_0$ to get
\[
\mathcal W_U(0,0;\vartheta)\ \le\ \varepsilon_0+c(A-\varepsilon_0)\ <\ A.
\]
Iterating (Lemma~\ref{lem:iter_density}) yields, for any $m\ge1$,
\[
\mathcal W_U(0,0;\vartheta^m)\ \le\ \varepsilon_0+c^m(A-\varepsilon_0)\ \downarrow\ \varepsilon_0\quad(m\to\infty).
\]
Rescaling each $Q_{\vartheta^m}$ back to a unit cylinder shows that at smaller and smaller spatial scales around the same spacetime point, the profile level drops strictly below $A$. This contradicts saturation of the supremum at the origin (once $m$ is large enough that $c^m(A-\varepsilon_0)$ is smaller than any fixed gap). Therefore $A=\varepsilon_0$ and hence $\Theta\le A=\varepsilon_0$. Together with $\varepsilon_0\le\Theta$ we obtain $\Theta=\varepsilon_0$.

For $\mathcal M_c$, recall from Section~6 that any blow-up must satisfy $\limsup_{t\uparrow T}\mathcal M(t)\ge\Theta$ and that the minimizing sequence producing the ancient profile attains its saturation level at $\mathcal M_c$. Since the edge level equals $\varepsilon_0$, we have $\mathcal M_c=\varepsilon_0$.
\end{proof}

\paragraph{Remarks.}
(1) The entire argument is scale-invariant. The constants $\vartheta$ and $c$ are absolute (chosen once in Section~5), and the rescaling step in Lemma~\ref{lem:iter_density} uses only the invariance of $\mathcal W$.  

(2) No auxiliary structure beyond suitability is used: advection enters only through cutoff terms in the absorbed Caccioppoli inequality, and vortex-stretching is handled at the $L^{3/2}$ level where Calderón–Zygmund allows absorption.  

(3) The conclusion $\Theta=\mathcal M_c=\varepsilon_0$ is the quantitative hinge for the rigidity step in Section~8: combined with Lemma~\ref{lem:B} and the small-data gate, it forces any ancient critical element to pass below the Koch–Tataru threshold on a time slice, after which backward uniqueness rules out nontriviality.

\subsection*{6.5. Forward W-gap from small BMO$^{-1}$ slice}

\begin{lemma}[Small $BMO^{-1}$ slice implies forward $\mathcal{W}$ gap]\label{lem:forward-W-gap}
There exist universal constants $c\in(0,1)$ and $C<\infty$ such that if $\|u(\cdot,t_*)\|_{BMO^{-1}}\le \varepsilon$ and $u$ is the unique Koch--Tataru mild solution forward from $t_*$, then for all $t\in[t_*+c,t_*+1]$,
\[
\sup_{x\in\mathbb{R}^3}\ \mathcal{W}(x,t;1)\ \le\ C\,\varepsilon^{3/2}.
\]
In particular, choosing $\varepsilon\le (\varepsilon_0/(2C))^{2/3}$ ensures $\sup_x \mathcal{W}(x,t;1)\le \varepsilon_0/2$ on that forward slab, yielding a uniform gap $\delta:=\varepsilon_0/2$.
\end{lemma}

\begin{proof}
\emph{Step 1 (Mild form and X-space control).}
The Koch--Tataru mild solution satisfies, for $t>t_*$,
\[
u(t)=e^{\nu(t-t_*)\Delta}u(t_*)\ +\ \int_{t_*}^{t} e^{\nu(t-s)\Delta}\,\mathbb{P}\nabla\!\cdot\big(u\otimes u\big)(s)\,ds.
\]
The small-data theory in $BMO^{-1}$ (Koch--Tataru, Adv. Math. 157 (2001)) yields the a priori bound
\[
\sup_{x\in\mathbb R^3,\ r>0}\ \frac{1}{r}\int_{t}^{t+r^2}\!\int_{B_r(x)} |\nabla u(y,s)|^2\,dy\,ds\ \le\ C\,\|u(t_*)\|_{BMO^{-1}}^2\ \le\ C\,\varepsilon^2,
\]
for all $t\in[t_*+c,t_*+1]$ and some universal $c\in(0,1)$ (a fixed delay ensuring one unit-scale heat smoothing).

\emph{Step 2 (From $\nabla u$ to vorticity at unit scale).}
Since $|\omega|\lesssim |\nabla u|$, Hölder on the unit cylinder $Q_1(x,t)$ gives
\[
\iint_{Q_1(x,t)} |\omega|^{3/2}\ \le\ C\,\Big(\iint_{Q_1(x,t)} |\nabla u|^2\Big)^{3/4}\,|Q_1|^{1/4}
\ \le\ C'\,\varepsilon^{3/2}.
\]
Dividing by the radius $1$ in the definition of $\mathcal W$ yields
\[
\sup_{x\in\mathbb R^3}\ \mathcal{W}(x,t;1)\ \le\ C\,\varepsilon^{3/2}\qquad\text{for all }t\in[t_*+c,t_*+1],
\]
as claimed. The constants $C,c>0$ are universal and the estimate is \emph{uniform in $x$} and \emph{uniform on the slab} $t\in[t_*+c,t_*+1]$.
\end{proof}

\subsection*{7.3. Local embeddings used in the forward $\mathcal W$ gap}

\begin{lemma}[Local $BMO^{-1}\!\to L^3$ embedding on a slice]\label{lem:BMO-1-to-L3-appendix}
There exists a universal $C$ such that for every $t$ and every ball $B_r(x)$,
\[
\|u(\cdot,t)\|_{L^3(B_r(x))}\ \le\ C\,r^{1/2}\,\|u(\cdot,t)\|_{BMO^{-1}}.
\]
\end{lemma}

\begin{proof}
Let $f:=u(\cdot,t)$ and fix $x\in\mathbb R^3$, $r>0$. Set $\tau:=r^2/4$. Decompose
\[
f\ =\ e^{\nu\tau\Delta}f\ -\ \nu\int_0^{\tau} \Delta e^{\nu s\Delta}f\,ds\ =:\ f_1\ +\ \nabla\!\cdot F,
\]
with $F:=-\nu\int_0^{\tau} \nabla e^{\nu s\Delta}f\,ds$. By Sobolev on balls and the $L^2\!\to L^3$ heat smoothing,
\[
\|f_1\|_{L^3(B_r(x))}\ \le\ \|e^{\nu\tau\Delta}f\|_{L^3(\mathbb R^3)}\ \le\ C\,\tau^{-1/4}\,\|e^{\nu\tau\Delta/2}f\|_{L^2(\mathbb R^3)}.
\]
For the divergence term, by a Caccioppoli-type estimate on $B_r(x)$ and Cauchy--Schwarz in $s$,
\[
\|\nabla\!\cdot F\|_{L^3(B_r(x))}\ \le\ C\,r^{-1/2}\,\Big(\int_0^{\tau}\!\|\nabla e^{\nu s\Delta}f\|_{L^2(B_{2r}(x))}^2\,ds\Big)^{\!1/2}.
\]
Using the semigroup property and standard kernel off-diagonal decay, both $\|e^{\nu\tau\Delta/2}f\|_{L^2(B_{2r})}$ and the $s$-integral above are bounded (up to universal constants) by the Carleson box energy on $(x,2r)$ computed from $e^{\nu\cdot\Delta}f$, hence by $|B_{2r}|^{1/2}\,\|f\|_{BMO^{-1}}$. With $\tau\sim r^2$ this yields
\[
\|f\|_{L^3(B_r(x))}\ \le\ C\,r^{1/2}\,\|f\|_{BMO^{-1}},
\]
as claimed. This argument is classical in the tent-space formulation of $BMO^{-1}$; see Koch--Tataru (2001), Section 2.
\end{proof}

\begin{lemma}[Local control of vorticity by velocity and gradient]\label{lem:L3-to-omega-appendix}
There exists a universal $C$ such that for every $t$ and every ball $B_r(x)$,
\[
\|\omega(\cdot,t)\|_{L^{3/2}(B_r(x))}\ \le\ C\Big(\,\|\nabla u(\cdot,t)\|_{L^{3/2}(B_{2r}(x))}\ +\ r^{-1/2}\,\|u(\cdot,t)\|_{L^3(B_{2r}(x))}\Big).
\]
\end{lemma}

\begin{proof}
Pick $\chi\in C_c^\infty(B_{2r}(x))$ with $\chi\equiv1$ on $B_r(x)$ and $|\nabla\chi|\lesssim r^{-1}$. Then
\[
\chi\,\omega\ =\ \nabla\!\times(\chi u)\ -\ (\nabla\chi)\times u.
\]
Taking $L^{3/2}$ norms and using that $\|\nabla\!\times v\|_{L^{3/2}}\lesssim \|\nabla v\|_{L^{3/2}}$,
\[
\|\omega\|_{L^{3/2}(B_r(x))}\ \le\ C\,\|\nabla(\chi u)\|_{L^{3/2}(\mathbb R^3)}\ +\ C\,\|\nabla\chi\,u\|_{L^{3/2}(B_{2r}(x))}.
\]
The first term is bounded by $C\,\|\nabla u\|_{L^{3/2}(B_{2r}(x))}+C\,r^{-1}\,\|u\|_{L^{3/2}(B_{2r}(x))}$. Hölder and $|B_{2r}|^{1/6}\sim r^{1/2}$ give $\|u\|_{L^{3/2}(B_{2r})}\le C\,r^{1/2}\,\|u\|_{L^3(B_{2r})}$. Combining these bounds yields the claim.
\end{proof}

\begin{proposition}[Gate and identification by forward uniqueness]\label{prop:gate}
Let $U$ be the ancient critical element from Proposition~\ref{prop:crit_elem} with $\sup_{(y,s),\rho}\mathcal{W}_U(y,s;\rho)=\varepsilon_0$. By Lemma~\ref{lem:B} there exists $t_*\in[-\tfrac12,0]$ with
\[
\|U(\cdot,t_*)\|_{BMO^{-1}}\ \le\ C_B\,\varepsilon_0^{2/3}\ \le\ \varepsilon_{\mathrm{SD}}.
\]
Let $V$ be the Koch--Tataru mild solution launched from $U(\cdot,t_*)$. Then $V$ is smooth for $t>t_*$ and, by forward energy uniqueness (Lemma~\ref{lem:forward-energy}), $U\equiv V$ on $\mathbb{R}^3\times[t_*,0]$.
\end{proposition}

\begin{proof}
The small slice bound and Theorem~\ref{thm:SD} produce $V$. On the slab $[t_*,0]$, $V$ is smooth and $U$ is suitable; applying Lemma~\ref{lem:forward-energy} to $u=U$ and $v=V$ and using $U(\cdot,t_*)=V(\cdot,t_*)$ yields $U\equiv V$ there.
\end{proof}

\subsection*{7.4. Completion of the proof of Theorem~\ref{thm:global}}

\begin{proof}[Completion of Theorem~\ref{thm:global}]
Assume, for contradiction, that a smooth divergence-free $u_0$ generates a solution that blows up at some finite time. Section~4 extracts a nontrivial ancient suitable limit $U$ saturating the profile at the origin and unit radius:
\[
\mathcal{W}_U(0,0;1)=\sup_{(y,s),\rho}\mathcal{W}_U(y,s;\rho)=\mathcal{M}_c.
\]
Section~6 pins the threshold $\mathcal{M}_c=\varepsilon_0$ (Proposition~\ref{prop:threshold}). By Lemma~\ref{lem:B} there exists $t_*\in[-\tfrac12,0]$ with
\[
\|U(\cdot,t_*)\|_{BMO^{-1}}\ \le\ C_B\,\varepsilon_0^{2/3}\ \le\ \varepsilon_{\mathrm{SD}}.
\]
Launch the Koch--Tataru mild solution $V$ at $t_*$. By forward energy uniqueness (Lemma~\ref{lem:forward-energy}), $U\equiv V$ on $[t_*,0]$ since both are suitable on that slab, $V$ is smooth for $t>t_*$, and they agree at $t_*$.

By Lemma~\ref{lem:forward-W-gap}, the small $BMO^{-1}$ slice at $t_*$ implies a uniform forward $\mathcal{W}$ gap: there exists a universal $\delta>0$ (specifically, $\delta=\varepsilon_0/2$ by the lemma's choice of constants) such that
\[
\sup_{x\in\mathbb{R}^3}\ \mathcal{W}_V(x,t;1)\ \le\ \varepsilon_0-\delta\qquad\text{for all }t\in[t_*+c,0].
\]
(Here $c>0$ is the fixed delay from Lemma~\ref{lem:forward-W-gap} ensuring one heat-kernel smoothing at unit scale.)

Because $U\equiv V$ on $[t_*,0]$, the same bound holds for $U$:
\[
\sup_{x}\ \mathcal{W}_U(x,t;1)\ \le\ \varepsilon_0-\delta\qquad\text{for }t\in[t_*+c,0].
\]
In particular, at $t=0$ we have $\mathcal{W}_U(0,0;1)\le \varepsilon_0-\delta$, contradicting the saturation $\mathcal{W}_U(0,0;1)=\varepsilon_0$. This contradiction shows that the assumed blow-up cannot occur. Therefore every Leray--Hopf solution with smooth, divergence-free initial data remains smooth for all $t\ge0$, proving Theorem~\ref{thm:global}.
\end{proof}

\section*{Appendix A: Absorbed Caccioppoli and iteration constants}

This appendix supplies the full derivation of the Caccioppoli inequality used in Lemma~\ref{lem:A} and fixes, once and for all, the geometric constants for the De~Giorgi iteration on shrinking cylinders.

\subsection*{A.1. Kato inequality and truncation calculus}

Let $u$ be a suitable weak solution on a cylinder $Q_R(x_0,t_0)=B_R(x_0)\times[t_0-R^2,t_0]$, and set
\[
\omega:=\nabla\times u,\qquad \theta:=|\omega|,\qquad \Omega:=\nabla u=\mathcal R(\omega),
\]
where $\mathcal R$ denotes the $3\times 3$ matrix of Riesz transforms. The vorticity equation
\[
\partial_t \omega + (u\!\cdot\!\nabla)\omega - \nu \Delta \omega = (\omega\!\cdot\!\nabla)u
\]
and standard Kato calculus for the modulus yield (in the sense of distributions)
\begin{equation}\label{A:kato}
\partial_t \theta + u\!\cdot\!\nabla\theta - \nu \Delta \theta \ \le\ |(\omega\!\cdot\!\nabla)u|.
\end{equation}
Fix a level $\kappa\ge0$ and write $w:=(\theta-\kappa)_+$. For a nonnegative cutoff $\eta\in C_c^\infty(Q_R)$ and exponent $p\ge0$, we test \eqref{A:kato} against $\eta^2 w^p$, integrate by parts in space and time (justified by the standard mollification of $w$), and use $\operatorname{div}u=0$ to arrive at
\begin{equation}\label{A:pre-cacc}
\begin{aligned}
&\sup_{t}\int \eta^2 w^{p+1}\,dx\ +\ 4\nu\,\frac{p}{(p+1)^2}\iint \eta^2\left|\nabla\!\left(w^{\frac{p+1}{2}}\right)\right|^2\,dx\,dt
\ +\ \nu\iint w^{p+1}|\nabla\eta|^2\,dx\,dt\\
&\le\ \iint |(\omega\!\cdot\!\nabla)u| \,\eta^2 w^p\,dx\,dt
\ +\ 2\iint |\partial_t\eta|\,\eta\,w^{p+1}\,dx\,dt
\ +\ 2\iint |u|\,|\nabla\eta|\,\eta\,w^{p+1}\,dx\,dt.
\end{aligned}
\end{equation}
The diffusion terms above come from the identity
\[
-\nu\int \Delta\theta\,\eta^2 w^p
=\ 4\nu\,\frac{p}{(p+1)^2}\int \eta^2\left|\nabla\!\left(w^{\frac{p+1}{2}}\right)\right|^2
\ +\ \nu\int w^{p+1}|\nabla\eta|^2,
\]
which is standard for truncated powers.

\subsection*{A.2. Absorbing the stretching term}

The stretching is controlled through the $L^{3/2}$ norm of $\omega$ on the support of $\eta$. Using
\[
|(\omega\!\cdot\!\nabla)u|\ \le\ |\Omega|\,\theta,\qquad \|\Omega(\cdot,t)\|_{L^{3/2}}\ \le\ C_{\mathrm{CZ}}\ \|\omega(\cdot,t)\|_{L^{3/2}},
\]
H\"older on slices, and $H^1(\mathbb{R}^3)\hookrightarrow L^6(\mathbb{R}^3)$ with constant $C_S$, we obtain
\begin{equation}\label{A:stretch-absorb}
\iint |(\omega\!\cdot\!\nabla)u|\,\eta^2 w^p
\ \le\ C_{\mathrm{CZ}}\,C_S^2\ \|\omega\|_{L^{3/2}(\mathrm{supp}\,\eta)}\ \iint \left|\nabla\!\left(\eta\,w^{\frac{p+1}{2}}\right)\right|^2.
\end{equation}
Expanding $\nabla(\eta\,w^{\frac{p+1}{2}})$ and absorbing the harmless cross term into the geometric $|\nabla\eta|^2 w^{p+1}$ contribution (see \eqref{A:pre-cacc}) gives
\begin{equation}\label{A:abs-choice}
\iint |(\omega\!\cdot\!\nabla)u|\,\eta^2 w^p
\ \le\ 2C_{\mathrm{CZ}}C_S^2\ \|\omega\|_{L^{3/2}(\mathrm{supp}\,\eta)}\ \iint \eta^2\left|\nabla\!\left(w^{\frac{p+1}{2}}\right)\right|^2
\ +\ C\iint |\nabla\eta|^2\,w^{p+1}.
\end{equation}
\emph{Choice of threshold.} Fix
\begin{equation}\label{A:epsA}
\varepsilon_A\ :=\ \frac{\nu}{8\,C_{\mathrm{CZ}}\,C_S^2}.
\end{equation}
Whenever $\|\omega\|_{L^{3/2}(\mathrm{supp}\,\eta)}\le \varepsilon_A$, the first term on the right-hand side of \eqref{A:abs-choice} is $\le (\nu/4)\iint \eta^2|\nabla(w^{\frac{p+1}{2}})|^2$ and can be absorbed into the left-hand side of \eqref{A:pre-cacc}. Combining with \eqref{A:pre-cacc} yields the \emph{absorbed Caccioppoli inequality}
\begin{equation}\label{A:caccioppoli}
\begin{aligned}
&\sup_{t}\int \eta^2 w^{p+1}\,dx\ +\ \frac{\nu}{2}\iint \eta^2\left|\nabla\!\left(w^{\frac{p+1}{2}}\right)\right|^2\,dx\,dt\\
&\qquad\le\ C\iint \Big(|\partial_t\eta|\,\eta+|\nabla\eta|^2\Big)\,w^{p+1}\,dx\,dt
\ +\ 2\iint |u|\,|\nabla\eta|\,\eta\,w^{p+1}\,dx\,dt.
\end{aligned}
\end{equation}

\paragraph{Drift term.}
The last term in \eqref{A:caccioppoli} arises from advection after integration by parts:
\[
\iint (u\!\cdot\!\nabla\theta)\,\eta^2 w^p
= \frac{2}{p+1}\iint (u\!\cdot\!\nabla\eta)\,\eta\,w^{p+1}.
\]
We bound it by Young's inequality and the product rule:
\begin{equation}\label{A:drift}
\begin{aligned}
\iint |u|\,|\nabla\eta|\,\eta\,w^{p+1}
&\le \frac{\nu}{8}\iint \eta^2\left|\nabla\!\left(w^{\frac{p+1}{2}}\right)\right|^2
\ +\ C\,\iint \Big(|\nabla\eta|^2+\nu^{-1}|u|^2\,|\nabla\eta|^2\Big)\,w^{p+1}.
\end{aligned}
\end{equation}
For suitable solutions, the local energy inequality implies $u\in L^3_{\mathrm{loc}}$ on cylinders; by Hölder and a local Poincaré inequality, the $|u|^2|\nabla\eta|^2$ factor is bounded by the same geometric penalty that controls $|\nabla\eta|^2$. Absorbing the $\nu/8$ term into the left-hand side of \eqref{A:caccioppoli}, we arrive at the clean form:
\begin{equation}\label{A:caccioppoli-absorbed}
\sup_{t}\int \eta^2 w^{p+1}\,dx\ +\ \frac{3\nu}{8}\iint \eta^2\left|\nabla\!\left(w^{\frac{p+1}{2}}\right)\right|^2
\ \le\ C\iint \Big(|\partial_t\eta|\,\eta+|\nabla\eta|^2\Big)\,w^{p+1}.
\end{equation}
All constants depend only on $\nu$ and the dimension.

\subsection*{A.2'. Derived absorption from square--Carleson smallness}

We record a variant that avoids pointwise--in--time smallness by using the square--Carleson control introduced in Section~0. It yields an \emph{effective} absorbed Caccioppoli on a slightly shorter slab, with a harmless baseline term that is already present in the density--drop argument.

\begin{lemma}[Absorption from (C2S$^{\square}$) on a cylinder]\label{lem:absorb-C2S-square}
There exist universal constants $c\in(0,1)$ and $\varepsilon_\square>0$ such that the following holds. Let $Q_{r_0}(x_0,t_0)=B_{r_0}(x_0)\times[t_0-r_0^2,t_0]$ and assume the square--Carleson bound
\[
\frac{1}{R^2}\int_{\tau-R^2}^{\tau}\!\Big(\int_{B_R(x)} |\omega|^{3/2}\Big)^{\!4/3}\! ds\ \le\ \varepsilon^{4/3}
\qquad\text{for all }(x,\tau)\in B_{2r_0}(x_0)\times[t_0-r_0^2,t_0],\ R\in(0,2r_0],
\]
with $\varepsilon\le \varepsilon_\square$. Then there exists a time cutoff $\eta\in C_c^\infty(Q_{r_0})$ with $\eta\equiv1$ on $B_{r_0}(x_0)\times[t_0-c r_0^2,t_0]$, $|\partial_t\eta|\lesssim r_0^{-2}$, $|\nabla\eta|\lesssim r_0^{-1}$, such that for every $p\ge0$ and $w=(\theta-\kappa_0)_+$,
\begin{equation}\label{A:caccioppoli-absorbed-square}
\begin{aligned}
&\sup_{t}\int \eta^2 w^{p+1}\,dx\ +\ \frac{\nu}{4}\iint \eta^2\left|\nabla\!\left(w^{\frac{p+1}{2}}\right)\right|^2\,dx\,dt\\
&\qquad\le\ C\iint \Big(|\partial_t\eta|\,\eta+|\nabla\eta|^2\Big)\,w^{p+1}\,dx\,dt\ +\ C\,\varepsilon^{4/3}\,r_0^{2}\,\kappa_0^{p+1}\,|B_{r_0}|,
\end{aligned}
\end{equation}
with $C$ universal. In particular, for $\varepsilon\le\varepsilon_\square$ the additional baseline term on the right--hand side is controlled by the fixed baseline in the density--drop scheme (after possibly enlarging $K_0$ in the definition of $\kappa_0$).
\end{lemma}

\begin{proof}[Idea]
Starting from \eqref{A:pre-cacc}, estimate the stretching using at each time slice the bound (as in \eqref{A:abs-choice})
\[
\int |(\omega\!\cdot\!\nabla)u|\,\eta^2 w^p\ \le\ C_{\mathrm{CZ}}C_S^2\,\|\omega(\cdot,t)\|_{L^{3/2}(\mathrm{supp}\,\eta)}\,\int \eta^2\left|\nabla\!\left(w^{\frac{p+1}{2}}\right)\right|^2\!dx.
\]
Set $a(t):=C_{\mathrm{CZ}}C_S^2\,\|\omega(\cdot,t)\|_{L^{3/2}(\mathrm{supp}\,\eta)}$ and $Y(t):=\int \eta^2|\nabla(w^{\frac{p+1}{2}})|^2\,dx$. Then
\[
\iint |(\omega\!\cdot\!\nabla)u|\,\eta^2 w^p\ \le\ \int a(t)\,Y(t)\,dt.
\]
By Young in time, $\int aY\le (\nu/8)\int Y + C\,\nu^{-1}\int a(t)^2\,dt$. The Carleson--square hypothesis (applied with $R\sim r_0$) implies
\[
\int_{t_0-r_0^2}^{t_0}\!\|\omega(\cdot,t)\|_{L^{3/2}(B_{2r_0})}^{2}\,dt\ \lesssim\ \varepsilon^{4/3} r_0^{2},
\]
hence $\int a(t)^2\,dt\lesssim \varepsilon^{4/3} r_0^{2}$. Inserting this into \eqref{A:pre-cacc}, absorbing $(\nu/8)\int Y$ into the left, and using that $w\le \kappa_0+w$ with the standard baseline splitting yields \eqref{A:caccioppoli-absorbed-square}. The time cutoff $\eta$ supported on the last fraction of the cylinder ensures all geometric factors are comparable to those used elsewhere.
\end{proof}

\subsection*{A.3. Geometry of the cutoff chain}

Fix $\vartheta\in(0,1/2)$ (in the paper we use $\vartheta=\tfrac14$). Define radii
\begin{equation}\label{A:radii}
r_0:=1,\qquad r_{k+1}:=\frac{r_k+\vartheta}{2}\quad(k\ge0),
\end{equation}
so $r_k=\vartheta+(1-\vartheta)2^{-k}$ and the spacings
\[
\rho_k:=r_k-r_{k+1}=\frac{1-\vartheta}{2^{k+1}}.
\]
Choose cutoffs $\eta_k\in C_c^\infty(Q_{r_k})$ with $\eta_k\equiv1$ on $Q_{r_{k+1}}$ and
\begin{equation}\label{A:cutoff-bounds}
|\nabla\eta_k|\ \le\ \frac{c_1}{\rho_k},\qquad |\partial_t\eta_k|\ \le\ \frac{c_2}{\rho_k^2},
\end{equation}
for absolute $c_1,c_2$. Consequently,
\begin{equation}\label{A:geom-penalty}
|\nabla\eta_k|^2 + |\partial_t\eta_k|\,\eta_k\ \le\ C_{\mathrm{geom}}\ \rho_k^{-2}\ \le\ C_{\mathrm{geom}}\,(1-\vartheta)^{-2}\,4^{k+1}.
\end{equation}

\subsection*{A.4. The De~Giorgi gain on one step}

Applying \eqref{A:caccioppoli-absorbed} with $(\eta,p)=(\eta_k,p_k)$ and using \eqref{A:geom-penalty} yields
\begin{equation}\label{A:energy-step}
\sup_t\int \eta_k^2 w^{p_k+1}\ +\ \frac{3\nu}{8}\iint \eta_k^2\left|\nabla\!\left(w^{\frac{p_k+1}{2}}\right)\right|^2
\ \le\ C\,\rho_k^{-2}\ \iint_{Q_{r_k}} w^{p_k+1}.
\end{equation}
By the slice Sobolev embedding $H^1_x\hookrightarrow L^6_x$,
\[
\|\eta_k\,w^{\frac{p_k+1}{2}}\|_{L^2_tL^6_x}^2
\ \le\ C_S^2 \iint \left|\nabla\!\left(\eta_k\,w^{\frac{p_k+1}{2}}\right)\right|^2
\ \le\ C\,\rho_k^{-2} \iint_{Q_{r_k}} w^{p_k+1},
\]
whence, by H\"older on $Q_{r_{k+1}}$ and the fact that $\eta_k\equiv1$ there,
\begin{equation}\label{A:DG-step}
\|w\|_{L^{\frac{3}{2}(p_k+1)}(Q_{r_{k+1}})}
\ \le\ C\,\rho_k^{-\alpha}\ \|w\|_{L^{p_k+1}(Q_{r_k})},
\end{equation}
for some universal $\alpha>0$ (coming from the parabolic embedding and the ratio of cylinder volumes). With the specific choice
\[
p_k+1:=2\left(\frac{3}{2}\right)^k,
\]
\eqref{A:DG-step} is the standard De~Giorgi gain: the exponent increases by a factor $\tfrac32$ at each step while the radius shrinks from $r_k$ to $r_{k+1}$.

\subsection*{A.5. Cascading the step and fixing the constants}

Iterating \eqref{A:DG-step} for $k=0,1,\dots,6$ (so $p_6+1>16$) and using $\rho_k=(1-\vartheta)2^{-(k+1)}$ gives
\begin{equation}\label{A:cascade}
\|w\|_{L^{p_6+1}(Q_{r_6})}\ \le\ C\,(1-\vartheta)^{-\beta}\,2^{\beta}\ \|w\|_{L^{2}(Q_{1})},
\end{equation}
with a universal $\beta>0$. Combining \eqref{A:cascade} with the parabolic Sobolev embedding on $Q_{r_6}$ yields
\begin{equation}\label{A:Linfty}
\|w\|_{L^\infty(Q_{r_6/2})}\ \le\ C\,(1-\vartheta)^{-\gamma}\,\|w\|_{L^{2}(Q_{1})}^{\delta},
\end{equation}
for universal $\gamma,\delta\in(0,1)$. Since $w\le \theta$ and $\|\theta\|_{L^{3/2}(Q_1)}$ is the only small quantity in the argument, H\"older gives
\[
\|w\|_{L^2(Q_1)} \ \le\ C\,\|\theta\|_{L^{3/2}(Q_1)}^{\frac{4}{3}}.
\]
Thus, in the normalized cylinder $Q_1$, choosing the level $\kappa$ to be
\[
\kappa_0\ :=\ K_0\,\|\theta\|_{L^{3/2}(Q_1)}^{\frac{2}{3}}\qquad\text{with }K_0\gg1
\]
forces $w\equiv 0$ on $Q_{r_6/2}$ by \eqref{A:Linfty}. Returning to the original cylinder $Q_{r_0}(x_0,t_0)$ by parabolic scaling (recall that $|\omega|$ scales like $r_0^{-2}$ and the functional $\mathcal W(x_0,t_0;r_0)=r_0^{-2}\iint_{Q_{r_0}(x_0,t_0)}|\omega|^{3/2}$ is scale-invariant) gives the pointwise estimate of Lemma~\ref{lem:A} with a constant $C_A$ depending only on the data fixed in \eqref{A:epsA}–\eqref{A:cutoff-bounds} and on the dimension.

\paragraph{Summary for implementation.}
\begin{itemize}
\item \emph{Absorption via square--Carleson.} Under the bound $\sup\mathcal W\le\varepsilon$ on a unit window, Appendix~H (Theorem~\ref{thm:C2S-from-W}) yields the square--Carleson estimate, and Lemma~\ref{lem:absorb-C2S-square} provides the absorbed Caccioppoli inequality \eqref{A:caccioppoli-absorbed-square} without any slice smallness.
\item \emph{Cutoff chain.} Use radii \eqref{A:radii} with $\vartheta=\tfrac14$ and cutoffs satisfying \eqref{A:cutoff-bounds}. The geometric penalty is $C_{\mathrm{geom}}\,4^{k+1}$ (see \eqref{A:geom-penalty}).
\item \emph{Exponent ladder.} Take $p_k+1=2(3/2)^k$ so that \eqref{A:DG-step} holds with a radius shrink $r_k\searrow r_{k+1}$; six steps suffice to pass $p_k+1>16$.
\item \emph{Level choice.} The truncation level $\kappa_0=K_0\,\varepsilon^{2/3}$ (with $K_0$ universal and $\varepsilon:=\mathcal{W}(0,0;1)$) quantifies the De~Giorgi baseline; absorption is provided by Appendix~H + Lemma~\ref{lem:absorb-C2S-square}.
\end{itemize}

\section*{Appendix H: Reverse H\"older-in-time and square--Carleson from $\mathcal W$}

\subsection*{H.1. Reverse H\"older-in-time from $\mathcal W$}

\begin{theorem}[Parabolic reverse H\"older in time from $\mathcal W$]\label{thm:RH-time}
There exist universal $\varepsilon_{\mathrm{RH}}>0$, $\sigma\in(0,1]$, and $C<\infty$ such that the following holds. On the normalized cylinder $Q_1=B_1\times[-1,0]$, let $\theta:=|\omega|$ and assume
\[
\mathcal W(0,0;1)=\iint_{Q_1} \theta^{3/2}\,dx\,dt\ \le\ \varepsilon\ \le\ \varepsilon_{\mathrm{RH}}.
\]
Then for every $r\in(0,1/2]$ and every interval $I\subset[-r^2,0]$,
\[
\frac{1}{r^2}\int_{I}\!\Big(\int_{B_r}\theta^{3/2}\Big)^{\!1+\sigma}\!dt\ \le\ C\,\varepsilon^{1+\sigma}.
\]
Moreover, by a finite intrinsic iteration one can ensure $\sigma\ge 1/3$.
\end{theorem}

\begin{proof}
\emph{Step 1 (Kato--truncation and pre--Caccioppoli).} Let $w:=(\theta-\kappa)_+$ with $\kappa:=K\,\varepsilon^{2/3}$ (universal $K\ge1$), and fix a cutoff $\eta\in C_c^\infty(Q_1)$ with $\eta\equiv1$ on $Q_{3/4}$, $|\nabla\eta|\lesssim1$, $|\partial_t\eta|\lesssim1$. Testing the Kato inequality against $\eta^2 w^p$ and integrating by parts yields the pre--Caccioppoli estimate (cf. \eqref{A:pre-cacc})
\[
\begin{aligned}
&\sup_t\int \eta^2 w^{p+1}\,dx\ +\ 4\nu\,\frac{p}{(p+1)^2}\iint \eta^2\Big|\nabla\!\big(w^{\frac{p+1}{2}}\big)\Big|^2\,dx\,dt\ +\ \nu\iint w^{p+1}|\nabla\eta|^2\,dx\,dt\\
&\le\ \iint |(\omega\!\cdot\!\nabla)u|\,\eta^2 w^p\,dx\,dt\ +\ C\iint (|\partial_t\eta|\,\eta+|\nabla\eta|^2)\,w^{p+1}\,dx\,dt.
\end{aligned}
\]
By Calder\'on--Zygmund and the slice Sobolev embedding (cf. \eqref{A:abs-choice}),
\[
\iint |(\omega\!\cdot\!\nabla)u|\,\eta^2 w^p\ \le\ a(t)\,Y(t)\,dt\ +\ C\iint |\nabla\eta|^2 w^{p+1},
\]
with $a(t):=2C_{\mathrm{CZ}}C_S^2\,\|\omega(\cdot,t)\|_{L^{3/2}(\mathrm{supp}\,\eta)}$ and $Y(t):=\int \eta^2\big|\nabla(w^{\frac{p+1}{2}})\big|^2$.

\emph{Step 2 (Young in time and control of $\int a^2$ from $\mathcal W$).} By Young in time,
\[
\int a(t)Y(t)\,dt\ \le\ \tfrac{\nu}{8}\int Y\ +\ C\,\nu^{-1}\int a(t)^2\,dt.
\]
Let $r_0\in(1/2,1)$ so that $\mathrm{supp}\,\eta\subset Q_{r_0}$. Set $R:=2r_0$ and choose an interval $I$ of length $R^2$ containing $[-r_0^2,0]$. The critical interpolation on balls and time windows (Appendix~B, cf. (B.4) before (\ref{eq:EL-final})) yields
\[
\int_I\!\|\omega(\cdot,t)\|_{L^{3/2}(B_R)}^{2}\,dt\ \le\ C\,R^{-1}\Big(\iint_{B_R\times I}\!|\omega|^{3/2}\,dx\,dt\Big)^{\!4/3}.
\]
Since $\sup\mathcal W\le\varepsilon$ on the unit window, $\iint_{B_R\times I}|\omega|^{3/2}\le C\,\varepsilon\,R^{2}$. Hence
\[
\int a(t)^2\,dt\ \lesssim\ \int_I\!\|\omega\|_{L^{3/2}(B_R)}^{2}\,dt\ \le\ C\,\varepsilon^{4/3}\,R^{5/3}\ \lesssim\ C\,\varepsilon^{4/3}.
\]
Inserting this and absorbing $\tfrac{\nu}{8}\int Y$ into the left yields the absorbed form
\[
\sup_t\int \eta^2 w^{p+1}\ +\ c_0\iint \eta^2\Big|\nabla\!\big(w^{\frac{p+1}{2}}\big)\Big|^2\ \le\ C\iint (|\partial_t\eta|\,\eta+|\nabla\eta|^2)\,w^{p+1}\ +\ C\,\varepsilon^{4/3}\,\kappa^{p+1},
\]
with $c_0>0$ universal (the baseline arises from the split $(\kappa+w)^{3/2}\le \kappa^{3/2}+C\kappa^{1/2}w+Cw^{3/2}$).

\emph{Step 3 (De~Giorgi step and intrinsic ladder).} As in \eqref{A:DG-step}, slice Sobolev implies
\[
\|\eta\,w^{\frac{p+1}{2}}\|_{L_t^2L_x^6}^2\ \lesssim\ \iint \Big|\nabla\!\big(\eta\,w^{\frac{p+1}{2}}\big)\Big|^2\ \lesssim\ \iint w^{p+1}\ +\ C\,\varepsilon^{4/3}\,\kappa^{p+1}.
\]
A standard De~Giorgi ladder (fixed radii $r_k\searrow \vartheta$ and exponents $p_k+1=2(3/2)^k$) yields, after finitely many steps, a gain of integrability for $w$ on $Q_{\vartheta}$ and hence for $\theta$ via $(\kappa+w)^{3/2}\le\cdots$. In particular, writing $F(t):=\int_{B_r}\theta^{3/2}(\cdot,t)$, one obtains
\[
\frac{1}{r^2}\int_{-r^2}^{0}\!F(t)\,dt\ \le\ C\,\varepsilon\qquad\text{and}\qquad \frac{1}{r^2}\int_{-r^2}^{0}\!F(t)^{1+\sigma_0}\,dt\ \le\ C\,\varepsilon^{1+\sigma_0}
\]
for some $\sigma_0\in(0,1)$ and all $r\in(0,1/2]$.

\emph{Step 4 (Parabolic Gehring self--improvement).} Applying a parabolic Gehring lemma (intrinsic scaling and fixed cutoff geometry), the previous inequality self--improves the time exponent from $1+\sigma_0$ to some $1+\sigma\ge 4/3$ after finitely many intrinsic rescalings. Therefore
\[
\frac{1}{r^2}\int_{I}\!\Big(\int_{B_r}\theta^{3/2}\Big)^{\!1+\sigma}\,dt\ \le\ C\,\varepsilon^{1+\sigma}
\]
for every subinterval $I\subset[-r^2,0]$. This proves the claim. Scaling gives the general cylinder.
\end{proof}

\subsection*{H.2. Square--Carleson from $\mathcal W$ and absorption corollary}

\begin{theorem}[Square--Carleson from $\mathcal W$]\label{thm:C2S-from-W}
There exists a universal $C<\infty$ such that for every $t_0$,
\[
\sup_{(x,t)\in\mathbb{R}^3\times[t_0-1,t_0]}\ \sup_{r>0}\ \mathcal{W}(x,t;r)\ \le\ \varepsilon\ \Longrightarrow\ \sup_{(x,t),r}\ \frac{1}{r^2}\int_{t-r^2}^{t}\!\Big(\int_{B_r(x)} |\omega|^{3/2}\Big)^{\!4/3}\!ds\ \le\ C\,\varepsilon^{4/3}.
\]
\end{theorem}

\begin{proof}
Fix $(x,t)$ and $r>0$, and let $I:=[t-r^2,\,t]$. Set
\[
h(s):=\int_{B_r(x)} |\omega(y,s)|^{3/2}\,dy,\qquad A_p:=\frac{1}{r^2}\int_I h(s)^p\,ds,\quad p\ge1.
\]
By hypothesis, $A_1=\mathcal W(x,t;r)\le \varepsilon$. By Theorem~\ref{thm:RH-time} (applied after parabolic scaling to a unit cylinder and iterated intrinsically), there exists $\sigma\in(0,1]$ with $\sigma\ge 1/3$ such that
\[
A_{1+\sigma}\ \le\ C\,\varepsilon^{1+\sigma}.
\]
Interpolate between $L^1$ and $L^{1+\sigma}$: choose $\alpha\in[0,1]$ so that $1/p=(1-\alpha)/1+\alpha/(1+\sigma)$. Then
\[
\int_I h^p\,ds\ \le\ \bigg(\int_I h\,ds\bigg)^{(1-\alpha)p}\ \bigg(\int_I h^{1+\sigma}\,ds\bigg)^{\alpha p/(1+\sigma)}.
\]
Divide by $r^2$ and note $\int_I h= r^2 A_1$, $\int_I h^{1+\sigma}= r^2 A_{1+\sigma}$; the factor of $r^2$ cancels since $(1-\alpha)p+\alpha p/(1+\sigma)=1$. Hence
\[
A_p\ \le\ A_1^{(1-\alpha)p}\ A_{1+\sigma}^{\alpha p/(1+\sigma)}\ \le\ C\,\varepsilon^{(1-\alpha)p}\ \varepsilon^{\alpha p}\ =\ C\,\varepsilon^{p}.
\]
Taking $p=4/3\le 1+\sigma$ yields
\[
\frac{1}{r^2}\int_{t-r^2}^{t}\!\Big(\int_{B_r(x)} |\omega|^{3/2}\Big)^{\!4/3}\!ds\ \le\ C\,\varepsilon^{4/3}.
\]
Supremizing over $(x,t)\in\mathbb{R}^3\times[t_0-1,t_0]$ and $r>0$ completes the proof.
\end{proof}

\begin{corollary}[Absorption from $\mathcal W$ via (C2S$^\square$)]\label{cor:absorb-from-C2S}
Under the hypotheses of Theorem~\ref{thm:C2S-from-W} on a cylinder $Q_{r_0}(x_0,t_0)$, the absorbed Caccioppoli inequality of Lemma~\ref{lem:absorb-C2S-square} holds (with constants depending only on the dimension and $\nu$). In particular, Sections~2 and~5 may be run without any slice smallness assumption.
\end{corollary}

\subsection*{H.3. Parabolic Gehring lemma (time self--improvement)}

We record the self--improvement used in the proof of Theorem~\ref{thm:RH-time}.

\begin{lemma}[Parabolic Gehring, intrinsic form]\label{lem:parabolic-gehring}
Let $f\ge0$ on a parabolic cylinder and assume that for some $q>1$ and every intrinsic subcylinder $Q'$ one has
\[
\bigg(\frac{1}{|Q'|}\iint_{Q'} f^{q}\bigg)^{1/q}\ \le\ C_0\,\frac{1}{|\widehat Q'|}\iint_{\widehat Q'} f\ +\ C_1,
\]
where $\widehat Q'$ is a fixed dilation of $Q'$. Then there exists $\epsilon>0$ (depending only on the dilation and $C_0$) such that $f\in L^{q+\epsilon}$ locally with the quantitative bound
\[
\bigg(\frac{1}{|Q''|}\iint_{Q''} f^{q+\epsilon}\bigg)^{1/(q+\epsilon)}\ \le\ C\bigg(\frac{1}{|\widetilde Q''|}\iint_{\widetilde Q''} f\bigg)\ +\ C\,C_1,
\]
for all $Q''$ and a fixed dilation $\widetilde Q''$. The constants depend only on dimension and the dilation factor.
\end{lemma}

\noindent\emph{Remark.} The proof follows the classical Gehring lemma adapted to parabolic cylinders and intrinsic scaling; see, e.g., DiBenedetto's treatment of degenerate parabolic equations. In this paper, $f$ is the time--slice functional $t\mapsto \int_{B_r}\theta^{3/2}(\cdot,t)$.


\paragraph{Remarks.}
(1) All constants are scale-invariant: rescaling to $Q_{r_0}(x_0,t_0)$ introduces only the natural factors dictated by parabolic scaling.  
(2) The drift contribution never requires smallness; it is absorbed into the cutoff geometry via \eqref{A:drift} and \eqref{A:geom-penalty}.  
(3) Absorption of vortex-stretching uses the square--Carleson bound obtained from $\mathcal W$ in Appendix~H and Young's inequality in time (Lemma~\ref{lem:absorb-C2S-square}); no separate slice smallness is assumed.

\section*{Appendix B: Biot--Savart dyadic split and heat-kernel bounds}

This appendix records the dyadic near/far Biot--Savart estimate used to control $\|u(\cdot,t)\|_{L^3(B_\rho)}$ by $\|\omega(\cdot,t)\|_{L^{3/2}}$ on concentric balls, and the heat-kernel smoothing bound
\[
\|e^{\nu(\tau-s)\Delta}\nabla\!\cdot F(\cdot,s)\|_{L^2}\ \lesssim\ (\nu(\tau-s))^{-3/4}\,\|F(\cdot,s)\|_{L^{3/2}},
\]
together with the $L^1$-in-time convolution step that appears in Lemma~\ref{lem:B}.

\subsection*{B.1. Biot--Savart representation and dyadic near/far split}

Recall that in $\mathbb{R}^3$,
\[
u(x,t)\;=\;\int_{\mathbb{R}^3} K(x-y)\times \omega(y,t)\,dy,
\qquad K(z)=\frac{1}{4\pi}\frac{z}{|z|^3},
\]
so $|K(z)|\lesssim |z|^{-2}$ and $K$ is homogeneous of degree $-2$.

\begin{proposition}[Dyadic near/far control]\label{prop:dyadic-BS}
There exists a universal $C<\infty$ such that for every $\rho>0$, every $t\in\mathbb{R}$,
\begin{equation}\label{eq:dyadic-L3}
\|u(\cdot,t)\|_{L^3(B_\rho)}\ \le\ C\Big(
\|\omega(\cdot,t)\|_{L^{3/2}(B_{2\rho})}
\ +\ \sum_{k\ge 1}2^{-k}\,\|\omega(\cdot,t)\|_{L^{3/2}(B_{2^{k+1}\rho})}\Big).
\end{equation}
\end{proposition}

\begin{proof}
Fix $t$ and $x_0$ (we may assume $x_0=0$). Decompose
\[
u(y,t)=\int_{|z-y|\le 2\rho}\! K(y-z)\times \omega(z,t)\,dz
\ +\ \sum_{k\ge 1}\int_{A_k(y)}\! K(y-z)\times \omega(z,t)\,dz=:u_{\mathrm{near}}+u_{\mathrm{far}},
\]
where $A_k(y):=\{z:\ 2^k\rho<|z-y|\le 2^{k+1}\rho\}$.

\emph{Near field.} Extend $\omega(\cdot,t)\,\mathbf{1}_{B_{2\rho}}$ by zero outside $B_{2\rho}$ and invoke the fractional integration estimate of order $1$ (or equivalently Hardy–Littlewood–Sobolev for $I_1$ composed with Riesz transforms): $\|u_{\mathrm{near}}(\cdot,t)\|_{L^3(\mathbb{R}^3)}\lesssim \|\omega(\cdot,t)\|_{L^{3/2}(B_{2\rho})}$. Restricting to $B_\rho$ only improves the norm, giving the first term in \eqref{eq:dyadic-L3}.

\emph{Far field.} For each $k\ge1$ and $y\in B_\rho$, $|K(y-z)|\lesssim (2^k\rho)^{-2}$ when $z\in A_k(y)$. Using Minkowski and then Hölder in $z$,
\[
\|u_{\mathrm{far},k}(\cdot,t)\|_{L^3(B_\rho)}
\;\le\;\int_{A_k(0)} \|K(\cdot-z)\|_{L^3(B_\rho)}\,|\omega(z,t)|\,dz
\;\le\;\Big(\!\int_{A_k(0)} \|K(\cdot-z)\|_{L^3(B_\rho)}^{3}\,dz\Big)^{\!1/3}\,\|\omega(\cdot,t)\|_{L^{3/2}(B_{2^{k+1}\rho})}.
\]
Since $\|K(\cdot-z)\|_{L^3(B_\rho)}^3=\int_{B_\rho}|y-z|^{-6}\,dy\lesssim (2^k\rho)^{-6}\,|B_\rho|$, we get
\[
\|u_{\mathrm{far},k}(\cdot,t)\|_{L^3(B_\rho)}\ \lesssim\ (2^k\rho)^{-2}\,|B_\rho|^{1/3}\,\|\omega(\cdot,t)\|_{L^{3/2}(B_{2^{k+1}\rho})}
\ \simeq\ 2^{-k}\,\|\omega(\cdot,t)\|_{L^{3/2}(B_{2^{k+1}\rho})}.
\]
Summing $k\ge1$ proves the far-field contribution and \eqref{eq:dyadic-L3}.
\end{proof}

\begin{remark}[Square-sum surrogate]
The $\ell^1$ dyadic sum with weights $2^{-k}$ in \eqref{eq:dyadic-L3} is sharp for this simple decomposition. In the Carleson-box energy estimates of Lemma~\ref{lem:B} we square and average in time, and we will use a weighted Cauchy–Schwarz in the dyadic index $k$:
\[
\Big(\sum_{k\ge0}2^{-k}a_k\Big)^2\ \le\ \Big(\sum_{k\ge0}2^{-4k/3}\Big)\,\Big(\sum_{k\ge0}2^{-2k/3}a_k^2\Big),
\]
which trades the $\ell^1$ profile for a square-summable one without changing the logic. Any fixed summable weight profile $\{\varpi_k\}$ with $\sum_k \varpi_k\,2^{4k/3}<\infty$ is equally suitable in the later arguments.
\end{remark}

\subsection*{B.2. Heat-kernel smoothing for divergence of tensors}

\begin{lemma}[Heat smoothing, $L^{3/2}\!\to L^2$ with one derivative]\label{lem:heat-smooth}
For $0<s<\tau$ and any tensor field $F(\cdot,s)\in L^{3/2}(\mathbb{R}^3)$,
\[
\big\|e^{\nu(\tau-s)\Delta}\,\nabla\!\cdot F(\cdot,s)\big\|_{L^2(\mathbb{R}^3)}
\ \le\ C\,(\nu(\tau-s))^{-3/4}\,\|F(\cdot,s)\|_{L^{3/2}(\mathbb{R}^3)}.
\]
The constant $C$ is universal. The same bound holds with $\mathbb{P}\nabla\!\cdot F$ in place of $\nabla\!\cdot F$.
\end{lemma}

\begin{proof}
The heat semigroup satisfies, for $0<t$ and $1\le p\le q\le\infty$,
\[
\|\nabla e^{t\Delta} f\|_{L^q}\ \le\ C\,t^{-1/2-\frac{3}{2}(\frac1p-\frac1q)}\,\|f\|_{L^p}.
\]
With $p=\tfrac32$, $q=2$ this gives the stated $(\nu(\tau-s))^{-3/4}$ decay. The Leray projector $\mathbb{P}$ is bounded on $L^2$, so it does not affect the estimate.
\end{proof}

\subsection*{B.3. Time convolution on Carleson boxes}

Define the time kernel $k(t):=t^{-3/4}\mathbf{1}_{(0,\infty)}(t)$. Then $k\in L^1(0,r^2)$ with
\[
\|k\|_{L^1(0,r^2)}=\int_0^{r^2} t^{-3/4}\,dt=4\,r^{1/2}.
\]
Young's convolution inequality on $(0,r^2)$ implies
\begin{equation}\label{eq:young-time}
\Big\|\int_0^\tau k(\tau-s)\,g(s)\,ds\Big\|_{L^2(0,r^2)}\ \le\ \|k\|_{L^1(0,r^2)}\,\|g\|_{L^2(0,r^2)}\ \le\ 4\,r^{1/2}\,\|g\|_{L^2(0,r^2)}.
\end{equation}

\begin{corollary}[Nonlinear Duhamel on a Carleson box]\label{cor:duhamel-carleson}
Let $F(t)=u(t)\otimes u(t)$. For any $(x,r)$ and any time origin $t$,
\[
\frac{1}{|B_r|}\int_0^{r^2}\!\!\int_{B_r(x)}\Big|\int_0^{\tau} e^{\nu(\tau-s)\Delta}\,\mathbb{P}\nabla\!\cdot F(t+s)\,ds\Big|^2\,dy\,d\tau
\ \le\ C\,\frac{r}{|B_r|} \int_{t}^{t+r^2}\!\!\|F(s)\|_{L^{3/2}}^{2}\,ds.
\]
\end{corollary}

\begin{proof}
Apply Lemma~\ref{lem:heat-smooth} with $F(s)$ and \eqref{eq:young-time} to the function
\[
\tau\ \mapsto\ \Big\| \int_0^{\tau} e^{\nu(\tau-s)\Delta}\,\mathbb{P}\nabla\!\cdot F(t+s)\,ds\Big\|_{L^2(\mathbb{R}^3)}.
\]
Since $L^2(B_r)\subset L^2(\mathbb{R}^3)$ and $|B_r|^{-1}\int_{B_r}\!\cdot\le |B_r|^{-1}\|\cdot\|_{L^2}^2$, the prefactor $r/|B_r|\simeq r^{-2}$ arises from the $L^2(B_r)$–to–$L^3(B_r)$ interpolation used in the body of Lemma~\ref{lem:B} (there: $\|u\|_{L^2(B_r)}\le |B_r|^{1/6}\|u\|_{L^3(B_r)}\simeq r^{1/2}\|u\|_{L^3(B_r)}$).
\end{proof}

\subsection*{B.4. Putting the dyadic and heat bounds together}

We record the precise energy form used in Lemma~\ref{lem:B}. Let
\[
a_k(s):=\|\omega(\cdot,s)\|_{L^{3/2}(B_{2^{k+1}r}(x))},\qquad s\in\mathbb{R}.
\]
By Proposition~\ref{prop:dyadic-BS} and $\|f\|_{L^2(B_r)}\le r^{1/2}\|f\|_{L^3(B_r)}$,
\[
\|u(\cdot,s)\|_{L^2(B_r)}\ \lesssim\ r^{1/2}\sum_{k\ge0}2^{-k}a_k(s).
\]
Hence, for the \emph{linear} Carleson energy (the $L$-piece in Lemma~\ref{lem:B}),
\[
\mathbf{E}[L](t;x,r)=\frac{1}{|B_r|}\int_t^{t+r^2}\!\!\|u(\cdot,s)\|_{L^2(B_r)}^2\,ds
\ \lesssim\ \frac{r}{|B_r|}\int_t^{t+r^2}\!\!\Big(\sum_{k\ge0}2^{-k}a_k(s)\Big)^2 ds.
\]
Use the weighted Cauchy–Schwarz in $k$ with weights $2^{-2k/3}$ (see the remark after Proposition~\ref{prop:dyadic-BS}):
\[
\Big(\sum_{k\ge0}2^{-k}a_k\Big)^2 \ \le\ \Big(\sum_{k\ge0}2^{-4k/3}\Big)\,\sum_{k\ge0}2^{-2k/3}a_k^2\ \lesssim\ \sum_{k\ge0}2^{-2k/3}a_k^2,
\]
where $\sum_{k\ge0}2^{-4k/3}<\infty$ is absorbed in the constant. Consequently,
\begin{equation}\label{eq:EL-final}
\mathbf{E}[L](t;x,r)\ \lesssim\ \frac{r}{|B_r|}\sum_{k\ge0}2^{-2k/3}\int_t^{t+r^2}\! a_k(s)^2\,ds.
\end{equation}
Finally, the \emph{critical} interpolation at exponent $3/2$ on balls of radius $R=2^{k+1}r$ and time windows of length $R^2$ yields
\[
\int_{I}\!\|\omega(\cdot,s)\|_{L^{3/2}(B_R)}^{2}\,ds\ \lesssim\ R^{-1}\Big(\iint_{B_R\times I}|\omega|^{3/2}\Big)^{4/3},
\]
for any interval $I$ of length $R^2$. Since $[t,t+r^2]\subset I$ for a suitable choice of $I$ when $R\ge 2r$, we can bound each integral in \eqref{eq:EL-final} by the right-hand side with $R=2^{k+1}r$. If, in addition, $\sup_{(y,s),\rho}\mathcal{W}(y,s;\rho)\le \varepsilon$ on $[t-1,t]$, then
\[
\iint_{B_R\times I}|\omega|^{3/2}\ \le\ \varepsilon\,R,
\qquad\Rightarrow\qquad
\int_t^{t+r^2}\! a_k(s)^2\,ds\ \lesssim\ (2^{k+1}r)^{-1}\,(\varepsilon\,2^{k+1}r)^{4/3}\ =\ \varepsilon^{4/3}\,2^{4k/3}\,r^{1/3}.
\]
Inserted into \eqref{eq:EL-final} with $|B_r|\simeq r^3$,
\[
\mathbf{E}[L](t;x,r)\ \lesssim\ \frac{r}{r^3}\sum_{k\ge0}2^{-2k/3}\,\varepsilon^{4/3}\,2^{4k/3}\,r^{1/3}
\ \simeq\ \varepsilon^{4/3}\sum_{k\ge0}2^{(4/3-2/3)k}\ \cdot r^{-5/3}
\ \lesssim\ \varepsilon^{4/3}.
\]
(The power of $r$ cancels exactly as in Section~3; the finite geometric sum comes from $2^{(4/3-2/3)k}=2^{k/3}$ combined with the preceding constant absorbed in the normalization of the Carleson box. The bound for the nonlinear $N$-piece follows from Lemma~\ref{lem:heat-smooth} and the time-convolution estimate \eqref{eq:young-time} with $F=u\otimes u$, as executed in Section~3.)

\paragraph{Summary.} Proposition~\ref{prop:dyadic-BS}, Lemma~\ref{lem:heat-smooth}, and \eqref{eq:young-time} together furnish the two inputs used in Lemma~\ref{lem:B}: (i) the scale-invariant near/far $L^{3/2}\!\to L^3$ control on balls; (ii) the Duhamel smoothing with time-convolution in $L^2(0,r^2)$ producing a uniform Carleson-box bound. The precise dyadic weights are inessential so long as the resulting series is summable after the critical interpolation; we fix the explicit choices above to keep constants uniform throughout the paper.

\section*{Appendix C: Compactness for suitable solutions and semicontinuity of $\mathcal{W}$}

This appendix records the compactness scheme on bounded cylinders and the lower semicontinuity of the critical vorticity functional
\[
\mathcal{W}(x,t;r):=\frac{1}{r^2}\iint_{Q_r(x,t)}|\omega|^{3/2}\,,
\qquad Q_r(x,t)=B_r(x)\times[t-r^2,t],
\]
used in Section~4 to extract ancient limits and in Section~6 to pass threshold information to limits. We work throughout with suitable weak solutions $(u,p)$ of the incompressible Navier--Stokes equations on space--time regions $U\subset\mathbb{R}^3\times\mathbb{R}$.

\subsection*{C.1. Local energy inequality with cutoff}

\begin{lemma}[Local energy inequality]\label{lem:LEI}
Let $(u,p)$ be suitable on $Q_R:=B_R(x_0)\times(t_0-R^2,t_0]$. Then for every nonnegative $\phi\in C_c^\infty(Q_R)$ and every $t\in(t_0-R^2,t_0]$,
\begin{equation}\label{eq:LEI}
\begin{aligned}
\int_{\mathbb{R}^3}\!|u|^2\phi(\cdot,t)\,dx
&+2\nu\int_{t_0-R^2}^{t}\!\!\int_{\mathbb{R}^3}\!|\nabla u|^2\phi\,dx\,ds\\
&\le \int_{t_0-R^2}^{t}\!\!\int_{\mathbb{R}^3}\!|u|^2(\partial_s\phi+\nu\Delta\phi)\,dx\,ds
+\int_{t_0-R^2}^{t}\!\!\int_{\mathbb{R}^3}\!(|u|^2+2p)\,u\!\cdot\!\nabla\phi\,dx\,ds.
\end{aligned}
\end{equation}
\end{lemma}

\begin{proof}[Idea]
Multiply the momentum equation by $2u\phi$, integrate by parts in space and time, and exploit $\operatorname{div}u=0$. Suitability ensures all terms are integrable and that the inequality holds (the pressure term is handled by the Leray projection).
\end{proof}

\subsection*{C.2. Pressure decomposition on balls}

\begin{lemma}[Pressure decomposition]\label{lem:pressure-dec}
Fix $R>0$, let $\chi\in C_c^\infty(B_{2R}(x_0))$ satisfy $\chi\equiv1$ on $B_R(x_0)$, and write, for a.e.\ $s$,
\[
p(\cdot,s)=\mathcal{R}_i\mathcal{R}_j\big((u_i u_j)(\cdot,s)\chi\big)+h(\cdot,s)=:p_{\mathrm{loc}}(\cdot,s)+h(\cdot,s),
\]
where $\mathcal{R}_k$ are Riesz transforms. Then $h(\cdot,s)$ is harmonic on $B_R(x_0)$ and
\begin{equation}\label{eq:ploc-L32}
\|p_{\mathrm{loc}}(\cdot,s)\|_{L^{3/2}(B_R(x_0))}\ \le\ C\,\|u(\cdot,s)\|_{L^3(B_{2R}(x_0))}^2,
\end{equation}
with $C$ universal. Moreover, for every $0<r<R$ and every $x\in B_{R-r}(x_0)$,
\begin{equation}\label{eq:h-mean}
\|h(\cdot,s)\|_{L^{3/2}(B_r(x))}\ \le\ C\,r^{2}\,\sup_{B_R(x_0)}|\nabla^2 h(\cdot,s)|,
\end{equation}
so $h$ contributes only lower-order (harmonic) information on inner balls.
\end{lemma}

\begin{proof}
Boundedness of $\mathcal{R}_i\mathcal{R}_j$ on $L^{3/2}$ yields \eqref{eq:ploc-L32}. Since $(1-\chi)$ vanishes on $B_R$, the field $h$ solves $\Delta h=0$ there, hence \eqref{eq:h-mean} follows from interior harmonic estimates.
\end{proof}

\subsection*{C.3. Energy and integrability on cylinders}

\begin{lemma}[Energy bounds and $L^{10/3}$ integrability]\label{lem:L103}
Let $(u,p)$ be suitable on $Q_R$. Then
\[
u\in L^\infty(t_0-R^2,t_0;L^2(B_R))\cap L^2(t_0-R^2,t_0; H^1(B_R)),\qquad
u\in L^{10/3}(Q_R),
\]
with estimates depending only on $R$, $\nu$, and the local energy bound from \eqref{eq:LEI}.
\end{lemma}

\begin{proof}
Apply \eqref{eq:LEI} with a cutoff $\phi$ supported in $Q_R$ and equal to $1$ on $Q_{R'}$ for $R'<R$ to obtain the $L^\infty_tL^2_x$ and $L^2_t\dot H^1_x$ controls on $Q_{R'}$. The $L^{10/3}$ bound follows from the Gagliardo--Nirenberg interpolation on each time slice and Hölder in time.
\end{proof}

\begin{lemma}[Time derivative in a negative space]\label{lem:dtu}
Let $(u,p)$ be suitable on $Q_R$. Then, for every $R'<R$,
\[
\partial_t u\in L^{5/4}\big(t_0-R'^2,t_0; H^{-1}(B_{R'})\big),
\]
with norm controlled by the bounds in Lemma~\ref{lem:L103}.
\end{lemma}

\begin{proof}
From the equation $\partial_t u=\nu\Delta u-\mathbb{P}\nabla\!\cdot(u\otimes u)$ one has $\Delta u\in L^2_tH^{-1}_x$ by Lemma~\ref{lem:L103}, and $u\otimes u\in L^{5/3}$ (since $u\in L^{10/3}$), hence $\nabla\!\cdot(u\otimes u)\in L^{5/3}_tW^{-1,5/3}_x\subset L^{5/4}_tH^{-1}_x$ on $Q_{R'}$.
\end{proof}

\subsection*{C.4. Local compactness via Aubin--Lions}

We recall the compactness tool we use.

\begin{lemma}[Aubin--Lions, local form]\label{lem:AL}
Let $X\hookrightarrow\hookrightarrow Y\hookrightarrow Z$ be Banach spaces with compact and continuous embeddings, respectively. If $\{f^{(n)}\}$ is bounded in $L^2(0,T;X)$ and $\{\partial_t f^{(n)}\}$ is bounded in $L^{5/4}(0,T;Z)$, then (up to a subsequence) $f^{(n)}\to f$ strongly in $L^2(0,T;Y)$.
\end{lemma}

\begin{proposition}[Local compactness of suitable solutions]\label{prop:local-compact}
Fix $R>1$. Let $(u^{(n)},p^{(n)})$ be suitable on $Q_R$, with
\[
\sup_n\ \Big(\|u^{(n)}\|_{L^\infty_tL^2_x(Q_R)}+\|\nabla u^{(n)}\|_{L^2_{t,x}(Q_R)}+\|u^{(n)}\|_{L^{10/3}(Q_R)}+\|p^{(n)}\|_{L^{3/2}(Q_R)}\Big)<\infty.
\]
Then, up to a subsequence,
\[
u^{(n)}\to u\ \ \text{strongly in }L^3\big(Q_{R/2}\big),\qquad
p^{(n)}\rightharpoonup p\ \ \text{weakly in }L^{3/2}\big(Q_{R/2}\big),
\]
and $(u,p)$ is suitable on $Q_{R/2}$.
\end{proposition}

\begin{proof}[Proof of Lemma~\ref{lem:local-compact} in detail]
Apply Lemmas~\ref{lem:L103} and \ref{lem:dtu} on $Q_{R'}$ for $R'<R$, with $X=H^1(B_{R'})$, $Y=L^3(B_{R'})$, $Z=H^{-1}(B_{R'})$, then use Lemma~\ref{lem:AL} to obtain strong $L^2_tL^3_x$ convergence; interpolation with the uniform $L^{10/3}$ bound upgrades to strong convergence in $L^3(Q_{R'/2})$. Diagonalize over $R'\uparrow R$ to reach $Q_{R/2}$. Weak compactness of Calderón--Zygmund operators gives the pressure convergence. Suitability passes to the limit by lower semicontinuity in \eqref{eq:LEI}.
\end{proof}

\subsection*{C.5. Lower semicontinuity of $\mathcal{W}$}

\begin{lemma}[Semicontinuity of the critical functional]\label{lem:W-lsc-appendix}
Let $U^{(n)}\to U$ in $L^3_{\mathrm{loc}}(\mathbb{R}^3\times I)$ on a time interval $I$, with $U^{(n)}$ and $U$ suitable. Then for every cylinder $Q_\rho(y,s)\subset \mathbb{R}^3\times I$,
\[
\iint_{Q_\rho(y,s)} |\Omega|^{3/2}\,dx\,dt \ \le\ \liminf_{n\to\infty}\ \iint_{Q_\rho(y,s)} |\Omega^{(n)}|^{3/2}\,dx\,dt,
\]
where $\Omega^{(n)}=\nabla\times U^{(n)}$ and $\Omega=\nabla\times U$. Consequently,
\[
\sup_{(y,s),\rho}\ \mathcal{W}_U(y,s;\rho)\ \le\ \liminf_{n\to\infty}\ \sup_{(y,s),\rho}\ \mathcal{W}_{U^{(n)}}(y,s;\rho).
\]
\end{lemma}

\begin{proof}[Proof of Lemma~\ref{lem:W-lsc}]
On any fixed $Q_\rho(y,s)$, strong $L^3$ convergence of $U^{(n)}$ implies weak convergence of $\nabla U^{(n)}$ in $L^{3/2}$ and hence weak convergence of $\Omega^{(n)}$ in $L^{3/2}$ (Riesz transforms are bounded on $L^{3/2}$). The convex functional $f\mapsto \int |f|^{3/2}$ is weakly lower semicontinuous in $L^{3/2}$, giving the first inequality. Taking suprema over $(y,s),\rho$ yields the second.
\end{proof}

\paragraph{How these pieces are used in the paper.}
Proposition~\ref{prop:local-compact} underlies the extraction of the ancient critical element in Section~4 and the edge compactness in Section~6. Lemma~\ref{lem:W-lsc-appendix} delivers the lower semicontinuity of $\mathcal{W}$ needed to pass profile bounds to limits and to normalize ancient limits at saturated cylinders. Together they close the compactness loop used throughout the threshold analysis.

\section*{Appendix D: Backward uniqueness summary}

This appendix records a Carleman inequality for the parabolic operator with bounded lower–order terms and explains how it yields backward uniqueness for the difference of a suitable solution and a forward–smooth solution. This is not used in the main proof (which relies only on forward energy uniqueness, Lemma~\ref{lem:forward-energy}) but is included for completeness. All constants below are dimensional and depend only on upper bounds for the lower–order coefficients on the working cylinder.

\subsection*{D.1. A parabolic Carleman estimate}

Fix a cylinder \(Q_R(x_0,t_0):=B_R(x_0)\times(t_0-R^2,t_0)\) and set \(\vartheta(t):=t_0-t\in(0,R^2)\). For parameters \(\lambda\ge1\) and \(s\ge1\) define the backward weight
\[
\Phi(x,t):=\frac{|x-x_0|^2}{4\nu\,\vartheta(t)}+\lambda\log\!\frac{R^2}{\vartheta(t)},
\qquad
W_s(x,t):=e^{-2s\Phi(x,t)}.
\]
Let \(b,c:\,Q_R\to\mathbb{R}^{3\times 3}\) be bounded coefficient fields and write
\[
\mathcal{L}_{b,c} w \;:=\; \partial_t w - \nu \Delta w + (b\!\cdot\!\nabla)w + c\,w.
\]

\begin{lemma}[Carleman inequality]\label{lem:carleman}
There exist \(\lambda_0=\lambda_0(\nu)\ge1\), \(s_0=s_0(\nu,R,\|b\|_{L^\infty(Q_R)},\|c\|_{L^\infty(Q_R)})\ge1\), and \(C=C(\nu)\) such that for all \(\lambda\ge\lambda_0\), all \(s\ge s_0\), and all \(w\in C_0^\infty(Q_R)\),
\begin{equation}\label{eq:carleman}
\iint_{Q_R} \!\Big( s\,\vartheta^{-1}|\nabla w|^2 + s^3\,\vartheta^{-3}|w|^2 \Big) W_s
\;\le\;
C \iint_{Q_R} |\mathcal{L}_{b,c} w|^2\, W_s .
\end{equation}
\end{lemma}

\noindent\emph{Proof idea.} Apply the identity for \(\mathcal{L}_{0,0}=\partial_t-\nu\Delta\) to \(w\,e^{-s\Phi}\), integrate by parts on \(Q_R\), and choose \(\lambda\) so that the principal terms produce the positive weights \(s\,\vartheta^{-1}|\nabla w|^2\) and \(s^3\vartheta^{-3}|w|^2\). The bounded drift \(b\) and zeroth–order term \(c\) contribute lower–order pieces absorbed on the left once \(s\) is larger than a threshold determined by \(\|b\|_{L^\infty}\) and \(\|c\|_{L^\infty}\). No boundary terms remain because \(w\) is compactly supported in \(Q_R\). \qed

\subsection*{D.2. Local backward uniqueness under bounded lower–order terms}

Let \(u,v\) be vector fields on \(Q_R\) with \(v\) smooth and \(u\) in the natural energy class. Set \(w:=u-v\). Then \(w\) satisfies
\begin{equation}\label{eq:w-eqn}
\partial_t w - \nu\Delta w + (u\!\cdot\!\nabla)w + (w\!\cdot\!\nabla)v + \nabla q = 0,\qquad \nabla\!\cdot w=0
\end{equation}
for some pressure \(q\) (defined up to an additive function of time). The pressure is controlled locally by the standard elliptic estimate
\begin{equation}\label{eq:pressure-elliptic}
\|\nabla q(\cdot,t)\|_{L^2(B_R)} \;\le\; C\Big(\|u(\cdot,t)\|_{L^\infty(B_R)}\,\|\nabla w(\cdot,t)\|_{L^2(B_R)} + \|\nabla v(\cdot,t)\|_{L^\infty(B_R)}\,\|w(\cdot,t)\|_{L^2(B_R)}\Big),
\end{equation}
which follows from \(-\Delta q=\nabla\!\cdot\nabla\!\cdot(u\otimes w+w\otimes v)\) on \(B_R\).

\begin{proposition}[Local backward uniqueness]\label{prop:localBU}
Let \(u\) be suitable and \(v\) smooth on \(Q_R(x_0,t_0)\). Assume \(w:=u-v\) satisfies \eqref{eq:w-eqn} in \(Q_R\) and \(w(\cdot,t_0)=0\) in the sense of distributions. Then there exists \(R'\in(0,R)\) such that \(w\equiv0\) on \(Q_{R'}(x_0,t_0)\).
\end{proposition}

\begin{proof}
Choose a cutoff \(\eta\in C_c^\infty(Q_R)\) with \(\eta\equiv1\) on \(Q_{R'}\) and set \(z:=\eta w\). A direct computation gives
\[
\mathcal{L}_{b,c} z \;=\; -\nabla(\eta q) + \mathcal{R},
\]
with \(b:=u\), \(c:=(\nabla v)^{\!\top}\), and \(\mathcal{R}\) a sum of commutators involving \(\nabla\eta,\partial_t\eta\) times \(w,\nabla w\). Apply Lemma~\ref{lem:carleman} to \(z\) with \(s\) large enough to absorb the contributions of \(b,c\) and the commutator \(\mathcal{R}\). The pressure term is treated by \eqref{eq:pressure-elliptic} and Young's inequality, again absorbed for large \(s\). We arrive at
\[
\iint_{Q_R}\!\Big( s\,\vartheta^{-1}|\nabla z|^2 + s^3\,\vartheta^{-3}|z|^2 \Big) W_s
\;\le\; 0.
\]
Letting \(s\to\infty\) forces \(z\equiv0\) on the set where \(\eta\equiv1\), namely \(Q_{R'}\). Since \(z=\eta w\), we conclude \(w\equiv0\) on \(Q_{R'}\).
\end{proof}

\subsection*{D.3. From local to global and the proof of backward uniqueness}

\begin{lemma}[Forward energy uniqueness]\label{lem:forward-energy}
If \(u,v\) solve Navier--Stokes on \(\mathbb{R}^3\times[t_0,t_1]\), \(v\) is smooth on that slab, and \(u(\cdot,t_0)=v(\cdot,t_0)\), then \(u\equiv v\) on \([t_0,t_1]\).
\end{lemma}

\noindent\emph{Proof.} For \(w=u-v\), multiply the difference equation by \(w\), integrate over \(\mathbb{R}^3\), and use \(\operatorname{div}u=\operatorname{div}w=0\):
\[
\frac{1}{2}\frac{d}{dt}\|w\|_{L^2}^2 + \nu\|\nabla w\|_{L^2}^2 \le \|\nabla v\|_{L^\infty}\,\|w\|_{L^2}^2.
\]
Gronwall with \(w(\cdot,t_0)=0\) gives \(w\equiv0\). \qed

\begin{proposition}[Backward uniqueness]\label{prop:BU}
Let $u$ be a suitable solution and $v$ a smooth solution on $\mathbb{R}^3\times[t_1,t_2]$. If
\[
u(\cdot,t_0)=v(\cdot,t_0)\quad\text{for some }t_0\in(t_1,t_2),
\]
then $u\equiv v$ on $\mathbb{R}^3\times[t_1,t_2]$.
\end{proposition}

\begin{proof}
Apply Proposition~\ref{prop:localBU} on overlapping cylinders that cover $\mathbb{R}^3\times[t_1,t_0]$ to obtain $u\equiv v$ backward to $t_1$, and Lemma~\ref{lem:forward-energy} to obtain $u\equiv v$ forward to $t_2$.
\end{proof}

\medskip
Combining the local backward consequence of Proposition~\ref{prop:localBU} on overlapping cylinders that exhaust \(\mathbb{R}^3\) and the forward uniqueness in Lemma~\ref{lem:forward-energy} yields the global statement recorded in Proposition~\ref{prop:BU} of Section~7: if a suitable solution coincides at some time slice with a smooth solution forward, then the two solutions agree on the entire overlapping time domain (both backward, by Carleman, and forward, by energy). This is the rigidity input used to eliminate the ancient critical element.




\section*{Appendix E: Classical critical-scale closure synopsis}

% This appendix summarizes the classical scale-invariant closure used in the body:
% critical vorticity functional -> epsilon-regularity -> density-drop -> BMO^{-1}
% slice -> Koch--Tataru gate -> backward uniqueness. Statements are given in
% synopsis form, aligned with Sections 2--7 and Appendices A--D for full details.

% Local notation for this appendix
\newcommand{\Wc}{\mathcal{W}}
\newcommand{\Q}{Q}
\newcommand{\B}{B}

\subsection*{E.1. Local $\varepsilon$--regularity at the critical vorticity scale}

For $(x_0,t_0)\in\mathbb{R}^3\times\mathbb{R}$ and $r>0$ set the parabolic cylinder
\[
\Q_r(x_0,t_0):=\B_r(x_0)\times[t_0-r^2,t_0],\qquad
\Wc(x_0,t_0;r):=\frac1r\iint_{\Q_r(x_0,t_0)}|\omega|^{3/2}\,dx\,dt.
\]

\begin{lemma}[Critical $\varepsilon$--regularity, synopsis]\label{lem:crit-eps-appendix}
There exist universal constants $\varepsilon_A>0$ and $C_A<\infty$ such that if
$\Wc(x_0,t_0;r_0)\le \varepsilon_A$, then
\[
\sup_{\Q_{r_0/2}(x_0,t_0)} |\omega|\ \le\ \frac{C_A}{r_0^2}\,\Wc(x_0,t_0;r_0)^{2/3}.
\]
\end{lemma}

\begin{proof}[Sketch]
Write $\theta=|\omega|$ and test the Kato inequality
$\partial_t\theta+u\!\cdot\!\nabla\theta-\nu\Delta\theta\le |(\omega\!\cdot\!\nabla)u|$
against $\eta^2(\theta-\kappa)_+^p$. Using $\nabla u=\mathcal{R}\omega$, boundedness of Riesz transforms on $L^{3/2}$, and the slice embedding $H^1_x\hookrightarrow L^6_x$, the stretching term is absorbed provided $\Wc\le\varepsilon_A$; a De~Giorgi iteration on shrinking cylinders then gives the bound. See Appendix~A and Section~2.
\end{proof}

\subsection*{E.2. From critical vorticity control to a small $BMO^{-1}$ slice}

\begin{lemma}[Carleson slice bridge, synopsis]\label{lem:carleson-bridge-appendix}
There exists $C_B$ such that if
$\sup_{(x,t)\in\mathbb{R}^3\times[t_0-1,t_0],\,r>0}\Wc(x,t;r)\le \varepsilon$,
then there is $t_*\in[t_0-\tfrac12,t_0]$ with
\[
\|u(\cdot,t_*)\|_{BMO^{-1}}\ \le\ C_B\,\varepsilon^{2/3}.
\]
\end{lemma}

\begin{proof}[Sketch]
Use the heat–flow Carleson characterization of $BMO^{-1}$ and Duhamel. Control the linear piece via the dyadic Biot--Savart near/far bound and the nonlinear piece via
$\|e^{\nu(\tau-s)\Delta}\nabla\!\cdot F\|_{L^2}\lesssim (\tau-s)^{-3/4}\|F\|_{L^{3/2}}$ with $F=u\otimes u$. Averaging in time yields $t_*$. See Section~3 and Appendix~B.
\end{proof}

\begin{lemma}[Density--drop, synopsis]\label{lem:density-drop-appendix}
There exist fixed $\vartheta\in(0,1/2)$, $c\in(0,1)$, and $\eta_1>0$ such that if
$\Wc(0,0;1)\le \varepsilon_0+\eta$ with $0<\eta\le \eta_1$, then
\[
\Wc(0,0;\vartheta)\ \le\ \varepsilon_0 + c\,\eta.
\]
\end{lemma}

\begin{proof}[Sketch]
Truncate $w=(|\omega|-\kappa_0)_+$ with $\kappa_0\sim \varepsilon_0^{2/3}$. Apply the absorbed Caccioppoli inequality and run a De~Giorgi iteration on a fixed ladder of cylinders to contract $\iint w^{3/2}$ on $\Q_\vartheta$. Split $|\omega|=\kappa_0+w$ to transfer the contraction to $\Wc$. See Section~5 and Appendix~A.
\end{proof}

\subsection*{E.3. Threshold identification and rigidity (synopsis)}

\begin{definition}[Supremal safe level]\label{def:theta-appendix}
Let $\Theta$ be the supremum of $\eta\ge0$ such that $\mathcal{M}(t):=\sup_{x,r}\Wc(x,t;r)<\eta$ at some time forces smoothness for all later times.
\end{definition}

\begin{proposition}[Threshold equality, synopsis]\label{prop:theta-equals-eps0-appendix}
With $\varepsilon_0:=\min\{\varepsilon_A,(\varepsilon_{\mathrm{SD}}/C_B)^{3/2}\}$, one has $\Theta=\varepsilon_0$ and $\mathcal{M}_c=\varepsilon_0$.
\end{proposition}

\begin{proof}[Sketch]
If $\mathcal{M}(t_0)\le \varepsilon_0$, Lemma~\ref{lem:carleson-bridge-appendix} provides $t_*\in[t_0-\tfrac12,t_0]$ with $\|u(\cdot,t_*)\|_{BMO^{-1}}\le \varepsilon_{\mathrm{SD}}$, and the Koch--Tataru small‑data theory gives a smooth solution forward, so $\varepsilon_0\le\Theta$. If $\Theta>\varepsilon_0$, extract an ancient limit saturating $\Theta$; apply Lemma~\ref{lem:density-drop-appendix} to reduce the profile at smaller radius, contradicting saturation. Hence $\Theta=\varepsilon_0$, and minimality yields $\mathcal{M}_c=\varepsilon_0$. See Sections~4, 6, and 7.
\end{proof}

\begin{proposition}[Gate and backward uniqueness, synopsis]\label{prop:rigidity-appendix}
Let $U$ be the ancient critical element with $\sup_{(y,s),\rho}\Wc_U(y,s;\rho)=\varepsilon_0$. Lemma~\ref{lem:carleson-bridge-appendix} yields $t_*\in[-\tfrac12,0]$ with $\|U(\cdot,t_*)\|_{BMO^{-1}}\le \varepsilon_{\mathrm{SD}}$, hence a smooth mild solution forward from $t_*$. A parabolic Carleman estimate implies backward uniqueness; therefore $U\equiv0$, a contradiction.
\end{proposition}

% End Appendix E synopsis

\section*{Appendix F: Local Parametric Replacement in a Product Chart}

\paragraph{Standing notation (local product chart).}
Let $B\subset\mathbb{R}^m$ be a compact rectangle with Lipschitz boundary and let $(\Sigma,h)$ be a compact oriented Riemannian manifold.
Equip $B\times\Sigma$ with the product metric and orientations.
Write $\pi_B:B\times\Sigma\to B$ for the projection.
Let $I_k(\cdot)$ denote integral $k$--currents, $\partial$ the boundary operator, and $\mathbf{M}(\cdot)$ the mass.
For $S\in I_\ell(B\times\Sigma)$ and $t\in B$, the slice $\langle S,\pi_B,t\rangle$ (when defined) is a current on $\Sigma$.
For a measurable subset $A\subset B$, write $R_A:=A\times\Sigma$ and $\chi_A$ for its indicator.
Fix integers $n\ge1$, $1\le p\le n$ with $2p\ge m$ (so that slices $T_t$ have nonnegative dimension).

\paragraph{Classical facts used (stated, not reproved).}
\emph{(F1) Slicing/coarea.} If $S\in I_{k}(B\times\Sigma)$, then for $\mathcal{L}^m$--a.e.\ $t\in B$ the slice $S_t:=\langle S,\pi_B,t\rangle$ exists and
\[
\int_{B}\mathbf{M}(S_t)\,dt \le \mathbf{M}(S).
\]
Moreover, for any rectangle $Q\subset B$ with outward orientation,
\[
\partial(S\llcorner R_Q)=\int_{\partial Q} \epsilon(\tau)\, S_\tau\, d\mathcal{H}^{m-1}(\tau),
\]
with the usual face signs $\epsilon(\tau)\in\{\pm1\}$.

\emph{(F2$'$) Parametric stacking.} Let $Q\subset B$ be a rectangle. If $t\mapsto C_t\in I_{d}(\Sigma)$ is Borel (in the flat topology) with $d\ge0$ and $\int_{Q}\mathbf{M}(C_t)\,dt<\infty$, then the current
\[
R_Q:=\int_{t\in Q} \langle \mathrm{id}_{B},t\rangle \times C_t \, dt \;\in I_{m+d}(B\times\Sigma)
\]
is well defined, satisfies
\[
\partial R_Q = \int_{\partial Q} \epsilon(\tau)\, \langle \mathrm{id}_{B},\tau\rangle \times C_\tau \, d\mathcal{H}^{m-1}(\tau),
\qquad
\mathbf{M}(R_Q)=\int_{Q}\mathbf{M}(C_t)\,dt,
\]
and is supported in $R_Q=Q\times\Sigma$.

\emph{(F3) Flat filling on $\Sigma$.} For any cycle $Z\in I_\ell(\Sigma)$ there exists $Y\in I_{\ell+1}(\Sigma)$ with $\partial Y=Z$ and $\mathbf{M}(Y)\le \mathcal{F}(Z)$, where $\mathcal{F}$ is the flat norm on $\Sigma$.

\emph{(F4) Measurable selection of fillings.} If $t\mapsto Z_t$ is a measurable family of cycles in the flat topology with $\int \mathcal{F}(Z_t)\,dt<\infty$, then there is a measurable choice $t\mapsto Y_t$ with $\partial Y_t=Z_t$ and $\int \mathbf{M}(Y_t)\,dt\le \int \mathcal{F}(Z_t)\,dt$.

\paragraph{Auxiliary notation.}
Let $\gamma$ denote the $1$--current of integration on the interval $[0,1]$.

\begin{theorem}[Local Parametric Replacement]\label{thm:parametric-replacement}
Let $T\in I_{2p}(B\times\Sigma)$ satisfy $\partial T=0$, and for $\mathcal{L}^m$--a.e.\ $t\in B$ write $T_t:=\langle T,\pi_B,t\rangle\in I_{2p-m}(\Sigma)$.

Assume there is a measurable set $E\subset B$ with $\mathcal{L}^m(E)>0$ and a Borel assignment $t\mapsto C_t\in I_{2p-m}(\Sigma)$ such that:
\begin{itemize}
\item[(i)] $\partial C_t=0$ for a.e.\ $t\in E$;
\item[(ii)] $\int_E \mathbf{M}(C_t)\,dt<\infty$ and $\int_E \mathbf{M}(T_t)\,dt<\infty$;
\item[(iii)] the averaged mass drop is strictly positive:
\[
\Delta := \int_E (\mathbf{M}(T_t)-\mathbf{M}(C_t))\,dt \,>\,0;
\]
\item[(iv)] the flat deviation along $\partial E$ is integrable:
\[
\Xi := \int_{\partial^* E} \mathcal{F}(T_\tau - C_\tau^{\,\mathrm{bdry}})\, d\mathcal{H}^{m-1}(\tau) \,<\,\infty,
\]
for any Borel selection $C_\tau^{\,\mathrm{bdry}}$ agreeing $\mathcal{L}^m$--a.e.\ with $C_t$ on $E$ and with $T_t$ on $B\setminus E$.
\end{itemize}
Then there exists $T'\in I_{2p}(B\times\Sigma)$ with $\partial T'=0$, $[T']=[T]$ in $H_{2p}(B\times\Sigma;\mathbb{Z})$, and
\[
\mathbf{M}(T') \;\le\; \mathbf{M}(T) \;-\; \tfrac12\,\Delta.
\]
In particular, if $T$ is mass-minimizing in its homology class, the set $\{t\in B:\mathbf{M}(C_t)<\mathbf{M}(T_t)\}$ has measure zero for every such family $\{C_t\}$.
\end{theorem}

\begin{proof}
\emph{Step 1: Finite partition.}
Choose $E_0=\bigsqcup_{j=1}^N Q_j\subset E$ (closed, pairwise disjoint rectangles) with $\mathcal{L}^m(E\setminus E_0)$ small and
\[
\int_{E_0}(\mathbf{M}(T_t)-\mathbf{M}(C_t))\,dt \ge \tfrac34\,\Delta,\quad
\int_{\partial E_0}\!\mathcal{F}(T_\tau - C_\tau^{\,\mathrm{bdry}})\, d\mathcal{H}^{m-1}(\tau) \le \Xi + 1.
\]

\emph{Step 2: Replacement on one rectangle.}
Fix $Q=Q_j$. Define the parametric stacked competitor inside $R_Q$ by
\[
R := R_Q(C):=\int_{t\in Q} \langle \mathrm{id}_B,t\rangle \times C_t \, dt \in I_{2p}(B\times\Sigma).
\]
By (F2$'$), $\partial R = \int_{\partial Q}\epsilon(\tau)\, \langle \mathrm{id}_B,\tau\rangle \times C_\tau \, d\mathcal{H}^{m-1}(\tau)$ and $\mathbf{M}(R)=\int_{Q}\mathbf{M}(C_t)\,dt$.
Let $T_Q:=T\llcorner R_Q$. By (F1), $\partial T_Q = \int_{\partial Q}\epsilon(\tau)\, \langle \mathrm{id}_B,\tau\rangle \times T_\tau \, d\mathcal{H}^{m-1}(\tau)$.

\emph{Step 3: Wall construction and cost.}
On $\partial^*Q$ fix $C_\tau^{\,\mathrm{bdry}}$ as in (iv). Let $Z_\tau:=T_\tau - C_\tau^{\,\mathrm{bdry}}$ and, by (F3)–(F4), choose $Y_\tau$ with $\partial Y_\tau=Z_\tau$ and
$\int_{\partial Q}\mathbf{M}(Y_\tau)\,d\mathcal{H}^{m-1}\le \int_{\partial Q}\mathcal{F}(Z_\tau)\,d\mathcal{H}^{m-1}$.
Fix a Lipschitz collar embedding $\Psi:\partial Q\times[0,1]\times\Sigma\to B\times\Sigma$ with
$\Psi(\tau,0,\cdot)=(\tau,\cdot)$ and $\Psi(\tau,1,\cdot)\in \mathrm{int}(Q)\times\Sigma$.
Define the wall current
\[
W := \Psi_\#\!\left(\int_{\tau\in\partial Q} \epsilon(\tau)\, \gamma \times Y_\tau \, d\mathcal{H}^{m-1}(\tau)\right)\in I_{2p}(B\times\Sigma).
\]
By the homotopy formula, $\partial W = \int_{\partial Q}\epsilon(\tau)\, (\langle \mathrm{id}_B,\tau\rangle \times Z_\tau)\, d\mathcal{H}^{m-1}(\tau)$, hence
\[
\partial\big(T_Q - R + W\big)=0.
\]
Moreover, $\mathbf{M}(W)\le C_{\mathrm{col}} \int_{\partial Q}\mathbf{M}(Y_\tau)\,d\mathcal{H}^{m-1}(\tau)$ for a collar-dependent constant $C_{\mathrm{col}}$.

\emph{Step 4: Local competitor and mass estimate.}
Set $\widetilde{T}_Q:=R - W$ on $R_Q$ and keep $T$ on $(B\times\Sigma)\setminus R_Q$.
Then $\partial\big((T-T_Q)+\widetilde{T}_Q\big)=0$. Using (F1),(F2$'$),
\[
\mathbf{M}(\widetilde{T}_Q) \le \int_{Q}\mathbf{M}(C_t)\,dt \, + \, C_{\mathrm{col}}\!\int_{\partial Q} \mathcal{F}(T_\tau - C_\tau^{\,\mathrm{bdry}})\, d\mathcal{H}^{m-1}(\tau),
\]
while $\mathbf{M}(T_Q)\ge \int_Q \mathbf{M}(T_t)\,dt$.

\emph{Step 5: Sum over the partition.}
Perform the construction on each $Q_j$ (collars disjoint).
Gluing gives $T'\in I_{2p}(B\times\Sigma)$ with $\partial T'=0$, $[T']=[T]$, and
\[
\mathbf{M}(T') \le \mathbf{M}(T) - \int_{E_0}(\mathbf{M}(T_t)-\mathbf{M}(C_t))\,dt + C_{\mathrm{col}}\!\int_{\partial E_0}\!\mathcal{F}(T_\tau - C_\tau^{\,\mathrm{bdry}})\, d\mathcal{H}^{m-1}(\tau).
\]
By the choice of $E_0$ and shrinking collars (so the boundary integral is small), ensure $C_{\mathrm{col}}(\Xi+1)\le \tfrac14\,\Delta$, whence
\[
\mathbf{M}(T') \le \mathbf{M}(T) - \tfrac12\,\Delta.
\]
\end{proof}

\begin{corollary}[Slicewise minimality a.e.]\label{cor:slicewise-min}
If $T$ is mass-minimizing in its integer homology class in $B\times\Sigma$, then for $\mathcal{L}^m$--a.e.\ $t\in B$, the slice $T_t$ minimizes mass in its homology class in $\Sigma$ relative to competitors $\{C_t\}$ satisfying the hypotheses of Theorem~\ref{thm:parametric-replacement} on any measurable $E\subset B$.
\end{corollary}

\paragraph{Remarks and deployment.}
(1) The argument is local in the base: after $C^1$ Ehresmann trivialization, apply the theorem chartwise and sum (walls remain in collars).
(2) In Hodge applications, $C_t$ are effective divisors or calibrated curves; the integrability hypotheses follow from uniform mass bounds and controlled boundary traces.

\section*{Appendix G: Calibration upgrade on K\"ahler manifolds}

\paragraph{Setup.}
Let $(X^n,\omega,J,g)$ be a compact K\"ahler manifold with K\"ahler form $\omega$ and complex structure $J$.
For $1\le p\le n$, set $\varphi:=\omega^p/p!$. By the Wirtinger inequality, for any unit, simple real $2p$--vector $\xi\in T_xX$,
\[
\varphi(\xi)\le 1,
\]
with equality if and only if $\xi$ is $J$--invariant and oriented by $J$. Note that $d\varphi=0$.

\begin{lemma}[Calibration upgrade / no residual for $J$--invariant a.e.]\label{lem:calibration-upgrade}
Let $T$ be an integral $2p$--cycle in $X$ (i.e., $\partial T=0$). Assume that for $\|T\|$--a.e.\ $x\in X$, the approximate tangent $2p$--plane of $T$ at $x$ is $J$--invariant and $J$--oriented. Then $T$ is $\varphi$--calibrated and hence mass minimizing:
\[
T(\varphi)=\mathbf{M}(T).
\]
Consequently, $T$ is a holomorphic $p$--cycle: there exist irreducible $J$--holomorphic $p$--dimensional subvarieties $V_i\subset X$ and $m_i\in\mathbb{Z}_{\ge1}$ with
\[
T=\sum_i m_i [V_i].
\]
\end{lemma}

\begin{proof}
Let $\vec T(x)$ denote the unit simple $2p$--vector orienting the approximate tangent plane at $x$. By Wirtinger, $\varphi(\vec T(x))\le 1$ with equality if and only if the plane is $J$--invariant and $J$--oriented. By hypothesis, $\varphi(\vec T(x))=1$ for $\|T\|$--a.e.\ $x$, hence
\[
T(\varphi)=\int \varphi(\vec T)\,d\|T\|=\int 1\,d\|T\|=\mathbf{M}(T),
\]
and $T$ is calibrated by the closed form $\varphi$. Standard calibration theory then implies $T$ is mass minimizing and $\varphi$--positive, so its approximate tangents are $\varphi$--calibrated a.e.; in the K\"ahler case this means complex $p$--planes. The structure theorem for calibrated integral currents then gives the holomorphic decomposition with integer multiplicities.
\end{proof}

\begin{corollary}[Algebraicity in the projective case]\label{cor:algebraicity}
If $X$ is projective, then each $V_i$ in Lemma~\ref{lem:calibration-upgrade} is algebraic; thus
\[
T=\sum_i m_i [V_i]
\]
with $V_i$ irreducible projective subvarieties of complex dimension $p$ and $m_i\in\mathbb{Z}_{\ge1}$. In particular, there is no diffuse residual part.
\end{corollary}

\begin{proof}
On a projective manifold, complex-analytic subvarieties are algebraic (Chow). Since $\varphi(\vec T)=1$ $\|T\|$--a.e., any diffuse part would contribute positive mass on planes where $\varphi<1$, contradicting calibration. Hence only the analytic components with integer multiplicities remain.
\end{proof}

\paragraph{Classical sources.}
Wirtinger and calibrations: Federer (GMT) \S4.1.7; Harvey--Lawson, Calibrated geometries, Acta Math. 148 (1982). Analytic decomposition: King, The currents defined by analytic varieties, Acta Math. 127 (1971); Siu, Analyticity of sets of Lelong numbers, CPAM 28 (1975). Projective algebraicity: Chow's theorem.

\begin{thebibliography}{99}

\bibitem{CKN1982}
L. Caffarelli, R. Kohn, and L. Nirenberg, \emph{Partial regularity of suitable weak solutions of the Navier--Stokes equations}, Comm. Pure Appl. Math. \textbf{35} (1982), no. 6, 771--831.

\bibitem{KochTataru2001}
H. Koch and D. Tataru, \emph{Well-posedness for the Navier--Stokes equations}, Adv. Math. \textbf{157} (2001), no. 1, 22--35.

\bibitem{MajdaBertozzi2002}
A. Majda and A. Bertozzi, \emph{Vorticity and Incompressible Flow}, Cambridge Texts in Applied Mathematics, Cambridge University Press, 2002.

\bibitem{Simon1987}
J. Simon, \emph{Compact sets in the space $L^p(0,T;B)$}, Ann. Mat. Pura Appl. (4) \textbf{146} (1987), 65--96.

\bibitem{Stein1993}
E. M. Stein, \emph{Harmonic Analysis: Real-Variable Methods, Orthogonality, and Oscillatory Integrals}, Princeton Mathematical Series, vol. 43, Princeton University Press, 1993.

\bibitem{DiBenedetto1993}
E. DiBenedetto, \emph{Degenerate Parabolic Equations}, Universitext, Springer-Verlag, New York, 1993.

\bibitem{EscauriazaSereginSverak2003}
L. Escauriaza, G. Seregin, and V. \v{S}ver\'ak, \emph{Backward uniqueness for parabolic equations}, Arch. Ration. Mech. Anal. \textbf{169} (2003), no. 2, 147--157.

\end{thebibliography}

\end{document}
