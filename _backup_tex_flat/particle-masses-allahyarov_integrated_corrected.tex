% Auto-generated (best-effort) editable LaTeX reconstruction from:
%   Allahyarov-submission-PRD-jan-2026-v-10-INTEGRATED.pdf
% Source text used:
%   out/pdf_extract/Allahyarov-submission-PRD-jan-2026-v-10-INTEGRATED.extracted.clean.txt
%
% Notes:
% - This is not a perfect round-trip conversion; equations/tables will need manual cleanup.
% - The goal is an editable, compiling starting point.
% - Compile with XeLaTeX.
%
\documentclass[11pt]{article}

\usepackage[margin=1in]{geometry}
\usepackage{fontspec}
\usepackage{microtype}
\usepackage{hyperref}
\usepackage{url}
\usepackage{amsmath,amssymb}
\usepackage{booktabs}
\usepackage{enumitem}
\usepackage{graphicx}

\hypersetup{colorlinks=true,linkcolor=blue,urlcolor=blue,citecolor=blue}
\urlstyle{tt}

% Load TeX Live fonts by filename (avoids system fontconfig dependency)
\setmainfont{lmroman10-regular.otf}
\setsansfont{lmsans10-regular.otf}
\setmonofont{lmmono10-regular.otf}

\title{Charged Fermion Masses from Octave Closure and $\varphi$-Ladder Geometry\\
A Recognition Science Framework with Single-Anchor Phenomenological Validation\\
\large (editable reconstruction from PDF)}
\author{Jonathan Washburn \and Elshad Allahyarov}
\date{January 30, 2026}

\begin{document}
\maketitle
\section*{Significance}
The masses of fundamental particles---electrons, quarks, neutrinos---span five orders of magnitude, from $0.5\,\mathrm{MeV}$ to $162\,\mathrm{GeV}$. Current theory treats each mass as an independent parameter measured by experiment, offering no explanation for why the top quark is 325,000 times heavier than the electron. We show that when all nine charged particle masses are evaluated at a single common energy scale ($182\,\mathrm{GeV}$), they organize into three families distinguished by simple integers derived solely from electric charge. This organization holds to one-part-per-million precision (a $\sim 15.6\sigma$-equivalent statement under simple null models; see Appendix~H and the pinned certificate \texttt{data/certificates/masses\_equalz\_significance/canonical\_2026\_q1.json}) and emerges from discrete geometry based on cube symmetry and the golden ratio $\varphi=1.618\ldots$. The discovery suggests that mass hierarchies, long viewed as arbitrary free parameters, may encode hidden mathematical structure. The framework is explicitly falsifiable by upcoming neutrino measurements, precision mixing data, and higher-loop QCD calculations.

\begin{abstract}
The Standard Model treats the nine charged fermion masses as empirical inputs. We present a discrete-geometry framework---Recognition Science---in which charged-fermion mass organization is described at a single common anchor scale $\mu_\star=182.201\,\mathrm{GeV}$, determined independently by a mass-free PMS/BLM stationarity condition over species-independent QCD/QED kernels. At $\mu_\star$, each charged fermion is assigned an integer rung and a charge-derived integer band label $Z_i$ constructed solely from electric charge and color representation. To compare with experiment, we transport PDG masses to $\mu_\star$ using Standard Model renormalization-group running at state-of-the-art precision (QCD four-loop, QED two-loop, $\overline{\mathrm{MS}}$ thresholds) and define the empirical residue $f^{(\mathrm{exp})}_i(\mu_\star) := \log_\varphi\bigl[m^{(\mathrm{data})}_i(\mu_\star)/m^{(\mathrm{skel})}_i(\mu_\star)\bigr]$. We emphasize the two-residue architecture: the SM transport residue $f^{RG}$ (a small, scheme-dependent transport exponent) is distinct from the structural Recognition residue $f^{\mathrm{Rec}}(Z)=F(Z)$ (a large, integer-organized band coordinate). \textbf{Main result:} at $\mu_\star$, the nine charged fermions cluster by equal-$Z$ families $Z\in\{24,276,1332\}$ within tolerance $5\times 10^{-6}$ in the residue test $f^{(\mathrm{exp})}_i(\mu_\star)\approx F(Z_i)$. Robustness checks under scheme/loop/threshold and EM-policy variations (with mass-free anchor recalibration) preserve the qualitative conclusion, and targeted ablations of the charge map destroy the clustering by orders of magnitude. Under simple null models, the chance probability of the observed three-family clustering is extremely small (a $\sim 15.6\sigma$-equivalent statement); the statistical treatment and trial-factor discussion are documented in Appendix~H (see also \texttt{data/certificates/masses\_equalz\_significance/canonical\_2026\_q1.json}). We also record two companion phenomenology layers: (i) a charged-lepton chain yielding absolute lepton masses at the $\sim 10^{-3}$ level under declared conventions, and (ii) closed-form CKM/PMNS and neutrino-ladder hypotheses with explicit falsifiers. Key mathematical properties of the band map $F$ are machine-checked in Lean~4 (Appendix~E). \textbf{Limitations:} the equal-$Z$ identity is not RG-invariant (it is an anchor statement), and the baseline validation is gauge-only (Yukawa terms are not included); the mechanism connecting the structural and empirical layers remains open and is framed with falsifiable bridge hypotheses. All computations are reproducible with public code and data~\cite{repo}.
\end{abstract}

\noindent\textbf{PACS numbers:} 12.15.Ff, 11.10.Hi, 12.38.-t, 12.20.-m, 14.60.Pq

\noindent\textbf{Keywords:} Standard Model masses; discrete geometry; golden ratio; renormalization group; octave closure; Recognition Science; PMS/BLM scale-setting

\tableofcontents
\newpage
\section{Introduction}

The nine charged fermion masses in the Standard Model span nearly five orders of magnitude, from the electron at $0.511\,\mathrm{MeV}$ to the top quark at $162\,\mathrm{GeV}$. Despite decades of precision measurements~\cite{pdg} and sophisticated theoretical tools---multi-loop renormalization-group running~\cite{rg1,rg2,rg3}, lattice QCD calculations~\cite{lqcd1,lqcd2}, and high-scale consistency analyses~\cite{hsc1,hsc2}---the \emph{origin} of this hierarchy remains one of particle physics' deepest puzzles. Why is the top quark 325,000 times heavier than the electron? Why do fermions organize into three generations with specific mass patterns? The Standard Model treats each Yukawa coupling as an independent free parameter measured by experiment. Existing phenomenological approaches offer valuable insights but typically introduce as many new parameters as they explain. Yukawa texture models with Froggatt--Nielsen mechanisms~\cite{fn1,fn2} require fitted flavor charges. Empirical mass relations~\cite{emp1,emp2} lack first-principles derivation. Discrete flavor symmetries~\cite{ds1,ds2} successfully predict neutrino mixing angles but demand extensive flavor sectors with vacuum alignment. Renormalization-group fixed-point studies~\cite{fp1,fp2} apply only to the top quark. None provides a parameter-free, species-agnostic organizational principle.

\paragraph{a.\ What is missing?}
All conventional models treat fermion masses as \emph{continuous} parameters fitted through symmetries or textures. None exploits the possibility that mass hierarchies might encode \emph{discrete integer structure} obscured by three conventions: (i) quoting masses at disparate reference scales ($m_b(m_b)$, $m_s(2\,\mathrm{GeV})$, pole masses for leptons), (ii) fractional Standard Model charges ($Q=2/3,-1/3,-1$), and (iii) lack of a single-scale comparison framework that would reveal charge-dependent patterns.

\paragraph{Non-circularity and claim hygiene (skeptical-reader protocol).}
To keep the paper logically auditable, we separate \emph{definitions} and \emph{tests}. In particular, no measured charged-fermion mass is used on the right-hand side of the equal-$Z$ residue-clustering test in Sec.~IV: PDG masses enter only through the construction of $m^{(\mathrm{data})}_i(\mu_\star)$ and $f^{(\mathrm{exp})}_i(\mu_\star)$, while the comparison target $F(Z_i)$ is fixed in closed form from charge and color. At the same time, the present manuscript does \emph{not} provide an independent derivation of the nine integer rungs $r_i$ for the charged fermions; we therefore treat $r_i$ as bookkeeping/assignment indices and explicitly flag the resulting circularity risk for any ``absolute mass'' reading of the rung layer (Sec.~II.6). All claimed predictions are accompanied by explicit falsifiers, and any proposed extensions (e.g., Yukawa-inclusive transport) are labeled as hypotheses and are not used in the baseline validation.

\paragraph{Particles studied (what “the mass spectrum” means here).}
Table I lists the charged fermions investigated in this work,

grouped by sector, together with representative PDG mass inputs under standard conventions~\cite{pdg}. (Neutrinos are treated separately in Sec.~VIII; their absolute masses are not directly measured and are inferred from oscillation data.) Here the masses are not all measured at one single scale. They are PDG inputs in their standard conventions: Charged leptons $e$, $\mu$, $\tau$ pole masses (on-shell); light quarks $u$, $d$, $s$ $\overline{\mathrm{MS}}$ running masses at $\mu=2\,\mathrm{GeV}$; heavy quarks $c$, $b$, $t$ $\overline{\mathrm{MS}}$ running masses at their standard reference points: $m_c(m_c)$, $m_b(m_b)$, $m_t(m_t)$; and neutron pole mass (physical mass).

\begin{table}[h]
  \centering
  \caption{Particles analyzed in this work and representative PDG mass inputs. Charged leptons are quoted as pole masses; light quarks are quoted as MS running masses at $\mu=2\,\mathrm{GeV}$; heavy quarks are quoted as MS running masses at their conventional reference points ($m_c(m_c)$, $m_b(m_b)$, $m_t(m_t)$). The neutron is included as a reference hadronic mass scale (pole mass). All masses are transported to the common anchor $\mu_\star$ under a declared RG policy for the single-anchor tests (Sec.~III and Sec.~IV).}
  \label{tab:particles_analyzed}
  \begin{tabular}{llrl}
    \toprule
    Group & Particle & $Q$ & PDG mass (representative) \\
    \midrule
    Charged leptons  & $e$   & $-1$    & 0.510999\,MeV \\
                     & $\mu$ & $-1$    & 105.658\,MeV \\
                     & $\tau$& $-1$    & 1.77686\,GeV \\
    \midrule
    Up-type quarks   & $u$   & $+2/3$  & 2.2\,MeV \\
                     & $c$   & $+2/3$  & 1.27\,GeV \\
                     & $t$   & $+2/3$  & 162.5\,GeV \\
    \midrule
    Down-type quarks & $d$   & $-1/3$  & 4.7\,MeV \\
                     & $s$   & $-1/3$  & 93\,MeV \\
                     & $b$   & $-1/3$  & 4.18\,GeV \\
    \midrule
    Hadron (ref.)    & $n$ (neutron) & $0$ & 939.565\,MeV \\
    \bottomrule
  \end{tabular}
\end{table}

\paragraph{Our approach.}
This work addresses the mass hierarchy problem through strict single-scale discipline combined with

explicit charge integerization. We evaluate all nine charged fermions at a single common anchor scale $\mu_\star=182.201\,\mathrm{GeV}$---determined independently by a species-independent Principle of Minimal Sensitivity (PMS)~\cite{pms} and Brodsky--Lepage--Mackenzie (BLM) scale-setting~\cite{blm} stationarity condition~\cite{stationarity} that uses no measured fermion masses. By integerizing electric charge (replacing $Q=2/3$ with $6Q=4$, etc.) and computing integrated renormalization-group residues using state-of-the-art anomalous dimensions (4-loop QCD, 2-loop QED)~\cite{rg1,rg2}, we uncover an unexpected regularity: fermions with identical integer band labels constructed solely from charge and color exhibit one-part-per-million degeneracy in their dimensionless RG residues.

To explain this phenomenological observation, we develop a discrete-geometry framework—Recognition Science—in which mass hierarchies emerge from three minimal closure principles: an 8-step octave reference period from three binary degrees of freedom, a golden-ratio ladder coordinate, and sector-global yardsticks built from cube combinatorics. This framework determines mass organization at the anchor scale; Standard Model renormalization-group running serves strictly as bookkeeping transport to compare with experimental measurements at other scales or schemes.

\paragraph{Paper organization.}
Section II develops the Recognition Science framework: octave closure, golden-ratio ladder, cube

yardsticks, charge-to-band map, gap function, and mass law at the anchor. Section III establishes the critical distinction between the structural Recognition residue (large, integer-organized) and the SM transport residue (small, scheme-dependent), and proposes three conjectural bridge mechanisms with explicit falsifiers. Section IV validates the single-anchor phenomenology: PMS/BLM calibration, equal-charge degeneracy tests, robustness under scheme/loop variations, and targeted ablations confirming structural specificity. Section V derives absolute lepton mass predictions from a parameter-free generation chain. Section VI addresses Yukawa contributions via a proposed golden-ratio ansatz and outlines an 8-motif extended dictionary. Sections VII and VIII extend the framework to CKM/PMNS mixing matrices (via cubic ledger topology) and neutrino masses (via fractional ladder rungs predicting a golden-ratio-to-the-seventh mass-squared ratio). Section IX summarizes what is claimed, the primary falsifiers, and the key limitations, and closes with a concise conclusion; extended discussion material (comparisons, conditional BSM hypotheses, and technical derivations) is collected in appendices for transparency. Appendices provide technical details: Lean proofs, QCD/QED kernels, transport certificates, and computational reproducibility specifications. Code and data are publicly available atgithub.com/recognition-physics/fermion-masses[1].

\section{Recognition Science Framework}

This section records only thedefinitions and fixed conventionsfrom the Recognition Science counting layer that are used downstream (mass law, validation protocol, lepton chain, mixing, neutrino rung logic). Motivational/heuristic discussion and optional intuition are moved to the appendices (especially Appendix A and Appendix 3) to keep the main text concise and technical. Throughout, we distinguish structural derivations () from modeling hypotheses () and conventions ().

\subsection{The Octave: eight-step closure from three binary degrees of freedom}

1. Minimal closure: why the period is eight We assume a minimal three-bit discrete state space (used only to fix a universal reference period) ; then: 23=8.(1) In downstream formulas, the appearance of a universal “−8” is treated as a choice of ladder-coordinate origin tied to this eighttick reference period .

\subsection{The $\varphi$-ladder: scale coordinates}

\paragraph{Purpose.}
We represent multiplicative hierarchies using a fixed-base logarithmic coordinate. In this manuscript the base

is taken to be the golden ratio $\varphi$.

\paragraph{Definition of the ladder base.}

\begin{equation}
  \varphi := \frac{1+\sqrt{5}}{2} \approx 1.618033988749\ldots
  \tag{2}
\end{equation}

\paragraph{Logarithmic ladder coordinate.}
Define the base-$\varphi$ logarithm and the constant $\lambda$ by:

\begin{equation}
  \log_{\varphi}(x) := \frac{\ln x}{\ln \varphi},
  \qquad
  \lambda := \ln \varphi.
  \tag{3}
\end{equation}

\paragraph{Rungs and ratios at the anchor.}
At the anchor scale $\mu_\star$, we represent relative mass hierarchies by integer rung differences. Concretely, if two species differ by an integer rung offset $\Delta r\in\mathbb{Z}$ in the ladder coordinate, then their mass ratio at the anchor is a pure $\varphi$-power:
\begin{equation}
  \frac{m_1}{m_2} = \varphi^{\Delta r}.
  \tag{4}
\end{equation}
The complete anchor mass law (Sec.~II.6) adds additional structure beyond rung differences: sector yardsticks and a charge-derived band coordinate $F(Z)$.

\subsection{Sector yardsticks from cube geometry}

\paragraph{1.\ The counting layer: cube combinatorics and symmetry constants.}
The yardsticks used in the mass framework are \emph{sector-global}: each sector (charged leptons, up-type quarks, down-type quarks) shares a single baseline scale at the anchor, rather than having per-particle offsets. The inputs to the yardstick construction are simple integers. First, the 3-cube has:
\begin{equation}
  \mathrm{vertices}=8,\quad \mathrm{edges}=12,\quad \mathrm{faces}=6.
  \tag{5}
\end{equation}
Second, we use the crystallographic classification constant:
\begin{equation}
  W := 17,
  \tag{6}
\end{equation}
the number of plane wallpaper groups (2D crystallographic groups).

\paragraph{Active versus passive edges (model convention).}
We frequently refer to a split between one distinguished “active” edge

per tick and the remaining ``passive'' edges. Under this convention:
\begin{equation}
  E_{\mathrm{total}} := 12,\quad A_z := 1,\quad E_{\mathrm{passive}} := E_{\mathrm{total}} - A_z = 11.
  \tag{7}
\end{equation}
The arithmetic is trivial. In this manuscript the split is used purely as a fixed integer bookkeeping convention that enters later generation-step formulas; no additional dynamical interpretation is assumed here. We will call the set of ``counted numbers,'' $[8, 12, 6, 17]$ which are derived from counting rules (cube counts, chosen constants, bookkeeping splits) as a \emph{counting layer} of the model. They will be used downstream to set sector parameters, rather than being fitted from masses.

\paragraph{2.\ Yardstick form and sector exponents.}
For each sector, we use a yardstick of the form:
\begin{equation}
  A_{\mathrm{sector}} := 2^{B_{\mathrm{pow}}(\mathrm{sector})} \cdot E_{\mathrm{coh}} \cdot \varphi^{r_0(\mathrm{sector})}.
  \tag{8}
\end{equation}
Here: $B_{\mathrm{pow}}(\mathrm{sector})\in\mathbb{Z}$ is a base-2 sector exponent fixed by the sector counting layer, $r_0(\mathrm{sector})\in\mathbb{Z}$ is a base-$\varphi$ sector exponent, and $E_{\mathrm{coh}}$ is a common coherence unit shared across sectors (defined when comparing to PDG units). Note that the model does \emph{not} permit choosing the sector exponents $B_{\mathrm{pow}}$ or $r_0$ separately for each particle; they are fixed for all particles of the same sector from the counting layer. Table~\ref{tab:sector_yardsticks} summarizes the sector yardstick assignments used in this paper. An informal physical analogy for ``yardsticks'' is provided in Appendix~A.

\begin{table}[t]
  \centering
  \caption{Sector yardstick exponents derived from cube counting (canonical values).}
  \label{tab:sector_yardsticks}
  \begin{tabular}{lrrl}
    \toprule
    Sector & $B_{\mathrm{pow}}$ & $r_0$ & Notes \\
    \midrule
    Charged leptons ($\ell$) & $-22$ & $62$ & shared within sector \\
    Up-type quarks ($u$)     & $-1$  & $35$ & shared within sector \\
    Down-type quarks ($d$)   & $23$  & $-5$ & shared within sector \\
    \bottomrule
  \end{tabular}
\end{table}

\subsection{The charge-to-band map $Z(Q,\mathrm{sector})$}

The Standard Model electric charges (in units of $e$) are:
\begin{equation}
  Q_u=+\frac{2}{3},\qquad Q_d=-\frac{1}{3},\qquad Q_e=-1,\qquad Q_\nu=0.
  \tag{9}
\end{equation}
To work with integers, define the integerized charge
\begin{equation}
  \tilde Q := 6Q \in \mathbb{Z},
  \tag{10}
\end{equation}
so that
\begin{equation}
  \tilde Q_u=4,\qquad \tilde Q_d=-2,\qquad \tilde Q_e=-6,\qquad \tilde Q_\nu=0.
  \tag{11}
\end{equation}

We define the band label $Z$ (equal-$Z$ family label) as an integer constructed solely from electric charge and sector:
\begin{equation}
  Z(Q,\mathrm{sector}) :=
  \begin{cases}
    4+\tilde Q^2+\tilde Q^4, & \text{quarks (color fundamental)},\\
    \tilde Q^2+\tilde Q^4, & \text{leptons (color singlet)},\\
    0, & \text{Dirac neutrinos } (Q=0).
  \end{cases}
  \tag{12}
\end{equation}
The “$+4$” offset for quarks encodes the additional QCD motif counts for color-charged fermions (see Sec. IV.3).

Applying Eq. (12) to the charged fermions yields three equal-$Z$ families:
\begin{align}
  Z_u=Z_c=Z_t &= 276, \tag{13}\\
  Z_d=Z_s=Z_b &= 24, \tag{14}\\
  Z_e=Z_\mu=Z_\tau &= 1332. \tag{15}
\end{align}

\subsection{The gap function $F(Z)$}

We introduce a gap function (band shift function) that converts the integer band label $Z$ into a dimensionless exponent shift on the $\varphi$-ladder:
\begin{equation}
  F(Z) := \frac{1}{\lambda}\ln\!\left(1+\frac{Z}{\varphi}\right),
  \qquad \lambda=\ln\varphi.
  \tag{16}
\end{equation}
Equivalently, in base-$\varphi$ logarithm:
\begin{equation}
  F(Z)=\log_{\varphi}\!\left(1+\frac{Z}{\varphi}\right)=\log_{\varphi}(\varphi+Z)-1.
  \tag{17}
\end{equation}
This band shift is a structural map: given an integer $Z$, it produces a real exponent shift $F(Z)$ with no adjustable parameters.

For the three equal-$Z$ families (Eqs.\ 13--15), the gap function yields representative values:
\begin{align}
  F(24) &\approx 5.74, \tag{18}\\
  F(276) &\approx 10.69, \tag{19}\\
  F(1332) &\approx 13.95. \tag{20}
\end{align}
These values are of order $10^{0}$--$10^{1}$ and reflect the structural band correction separating the three charged families at the anchor.
Formal properties of $F$ (monotonicity, concavity, and auxiliary bounds) are summarized in Appendix~E.

For members of an equal-$Z$ family (e.g.\ $u,c,t$ with $Z=276$), the band factor is identical:
\begin{equation}
  F(Z_u)=F(Z_c)=F(Z_t)=F(276).
  \tag{21}
\end{equation}
Key structural claim: at the anchor scale $\mu_\star$, members of an equal-$Z$ family share the same structural band correction $F(Z)$.
This is falsifiable via the residue clustering test in Sec.\ IV.4.

\subsection{The mass law at the anchor}

We define $\mu_\star$ as the unique common scale at which all masses are compared, obtained independently from a mass-free
PMS/BLM stationarity condition applied to the species-independent SM (QCD/QED) running kernels. All experimental masses
are transported to $\mu_\star$ before forming residues. All single-anchor tests (residue clustering, degeneracy, etc.) are statements about
quantities evaluated at $\mu=\mu_\star$.

Assembling the ingredients from Secs.\ II.1--II.5, the structural mass prediction at the anchor scale $\mu_\star$ is:
\begin{equation}
  m_i^{(\mathrm{struct})}(\mu_\star)
  \;:=\;
  A_{\mathrm{sector}(i)}\,\varphi^{\,r_i-8+F(Z_i)}
  \;=\;
  \underbrace{A_{\mathrm{sector}(i)}\,\varphi^{\,r_i-8}}_{\text{skeleton: sector + rung}}
  \underbrace{\varphi^{\,F(Z_i)}}_{\text{band: charge family}}.
  \tag{22}
\end{equation}

where $A_B$ is the sector $B$ yardstick (Eq.~8), $r_i\in\mathbb{Z}$ is the integer rung (step coordinate, not a continuous fit) for species $i$ within sector $B$, $-8$ is the octave reference (origin for ladder coordinates), and $F(Z_i)$ is the charge-derived band correction (Eq.~16). The \emph{skeleton} encodes the sector baseline and integer rung hierarchy. The \emph{band factor} is the charge-derived correction that ensures equal-$Z$ families cluster together at the anchor, independent of their rung assignments.

\paragraph{a.\ Important note on rung assignment and potential circularity (nine charged fermions).}
In the present manuscript, the integer rungs $r_i\in\mathbb{Z}$ for the nine charged fermions $i\in\{u,d,s,c,b,t,e,\mu,\tau\}$ should be read as \emph{bookkeeping indices}, not as independently-derived predictions. Concretely, because $r_i$ appears only through the skeleton factor $m^{(\mathrm{skel})}_i(\mu_\star)=A_B\varphi^{r_i-8}$, one can always choose $r_i$ from the transported experimental masses so that the skeleton-normalized quantity $\log_\varphi[m^{(\mathrm{data})}_i(\mu_\star)/A_B]$ is reduced by an integer to a desired branch. In that sense, the rungs act as hidden fit/assignment parameters: if $r_i$ is picked using $m^{(\mathrm{data})}_i(\mu_\star)$, then any statement that appears to ``predict'' the absolute mass hierarchy via $r_i$ is circular.

Accordingly, the falsifiable content of the single-anchor test is \emph{not} that the $r_i$ reproduce the masses, but that after removing the integer rung piece for each species, the \emph{remaining} (non-integer) residue $f^{(\mathrm{exp})}_i(\mu_\star)=\log_\varphi[m^{(\mathrm{data})}_i(\mu_\star)/m^{(\mathrm{skel})}_i(\mu_\star)]$ clusters by the charge-derived label $Z_i$ at ppm tolerance. This separation makes explicit what is fixed structurally (the $Z$-map and $F(Z)$) versus what is currently assigned from data (the rungs).

The structural masses differ only by skeleton factors (yardstick and rung):
\begin{equation}
  \frac{m^{(\mathrm{struct})}_c}{m^{(\mathrm{struct})}_u} = \varphi^{r_c-r_u},\quad
  \frac{m^{(\mathrm{struct})}_t}{m^{(\mathrm{struct})}_c} = \varphi^{r_t-r_c}.
  \tag{23}
\end{equation}
This is the \emph{falsifiable core prediction}: at the anchor $\mu_\star$, equal-$Z$ families should exhibit pure $\varphi$-power hierarchies after the common band correction is factored out. Section~IV tests this prediction by transporting PDG experimental masses to $\mu_\star$ and checking whether the empirical residues cluster by equal-$Z$ families within a stated tolerance.

\section{Transport Bookkeeping and Residue Definitions for Validation}

In this section we define the \emph{structural band coordinate} (structural Recognition residue):
\begin{equation}
  f^{\mathrm{Rec}}(Z_i) = F(Z_i),
  \tag{24}
\end{equation}
which is independent of experimental masses or SM running, and show how it differs from the \emph{Standard-Model RG transport residue} (transport exponent) $f^{RG}(\mu_1,\mu_2)$, which is a scheme/scale-dependent bookkeeping quantity defined by integrating the QCD+QED mass anomalous dimension:
\begin{equation}
  f^{RG}_i(\mu_1,\mu_2) := \frac{1}{\lambda}\int_{\ln\mu_1}^{\ln\mu_2} \gamma_i(\mu)\,d\ln\mu,\quad \lambda = \ln\varphi.
  \tag{25}
\end{equation}
Here, $\gamma_i(\mu) = \gamma^{QCD}_m(\alpha_s(\mu),n_f(\mu)) + \gamma^{QED}_m(\alpha(\mu),Q_i)$ is the standard mass anomalous dimension (see Sec.~IV.2 for explicit formulas). Equation~(25) is mathematically identical to the logarithmic running of masses under SM renormalization-group evolution. The \emph{transport exponent} definition tells us how a mass changes between two scales $\mu_1$ and $\mu_2$ under perturbative QCD/QED. In typical SM running between $\mu_\star$ and low-energy reference points, $f^{RG}$ is a \emph{small} scheme-dependent correction (order $10^{-2}$ to $10^{0}$); representative values are given in Appendix~8.

Comparing Eqs.~(24) and (25), we see immediately that $f^{\mathrm{Rec}}$ and $f^{RG}$ are \emph{not} the same object:
\begin{equation}
  f^{\mathrm{Rec}}(Z_i) \not\equiv f^{RG}_i(\mu_\star,m_i).
  \tag{26}
\end{equation}
Numerical comparisons and extended discussion are deferred to Appendix~C; see Table~\ref{tab:residue_comparison}.

Given a declared target scheme/scale $\mu_T$ (e.g., $\overline{\mathrm{MS}}$ at $2\,\mathrm{GeV}$ for light quarks, pole mass for leptons), the \emph{transport display} is:
\begin{equation}
  m^{(\mathrm{disp})}_i(\mu_T) := m^{(\mathrm{struct})}_i(\mu_\star)\,\varphi^{f^{RG}_i(\mu_\star,\mu_T)}.
  \tag{27}
\end{equation}
This equation is \emph{bookkeeping} that aligns an anchor-defined quantity with an external convention. It is \emph{not} a mechanism that produces absolute masses from the anchor display: the structural mass $m^{(\mathrm{struct})}_i(\mu_\star)$ (Eq.~22) already contains the full prediction (yardstick, rung, octave, band). The transport exponent $f^{RG}_i(\mu_\star,\mu_T)$ simply converts that prediction to the target convention (e.g., $\overline{\mathrm{MS}}$ running from $\mu_\star$ to $\mu_T$).

\paragraph{Worked example.}
A worked example (electron pole-mass display) is collected in Appendix A8 to keep this section

focused on the definitions used downstream.

\subsection{The phenomenological validation protocol}

To test whether the charge-derived band map $F(Z)$ clusters charged families at the anchor, we invert the transport display:
\begin{equation}
  m^{(\mathrm{data})}_i(\mu_\star) := m^{(\mathrm{PDG})}_i(\mu_{\mathrm{ref}})\,\varphi^{-f^{RG}_i(\mu_\star,\mu_{\mathrm{ref}})}.
  \tag{28}
\end{equation}
Here, $m^{(\mathrm{PDG})}_i(\mu_{\mathrm{ref}})$ is the experimental mass quoted by PDG at reference scale $\mu_{\mathrm{ref}}$ (e.g., $m_b(m_b)$ for the bottom quark, pole mass for leptons). We then define an \emph{empirical residue} by comparing the transported data mass to the structural skeleton:
\begin{equation}
  f^{(\mathrm{exp})}_i(\mu_\star) := \log_\varphi\!\left[\frac{m^{(\mathrm{data})}_i(\mu_\star)}{m^{(\mathrm{skel})}_i(\mu_\star)}\right],
  \tag{29}
\end{equation}
where $m^{(\mathrm{skel})}_i(\mu_\star):=A_B\varphi^{r_i-8}$ is the skeleton factor (Eq.~22). The \emph{band-map validation statement} is:
\begin{equation}
  f^{(\mathrm{exp})}_i(\mu_\star) \approx F(Z_i)\quad\text{within tolerance }5\times 10^{-6}.
  \tag{30}
\end{equation}
Section~IV presents numerical results demonstrating this clustering for all nine charged fermions at $\mu_\star=182.201\,\mathrm{GeV}$.

\paragraph{Logical handoff to Sec.~IV.}
At this point, all objects used in the numerical test have been defined: transport to the anchor (Eq.~28), the empirical residue $f^{(\mathrm{exp})}_i(\mu_\star)$ (Eq.~29), and the comparison target $F(Z_i)$ (Eq.~30). Section~IV performs the computations (kernels, thresholds, anchor calibration), reports the results, and runs robustness/ablation checks; interpretive mechanism questions are deferred to Appendix~D.

\section{Single-Anchor Phenomenological Validation}

Sec.~III fixed definitions and bookkeeping. Sec.~IV is the \emph{execution and evidence} section, where we compute the SM transport exponents, transport PDG data to the anchor, construct $f^{(\mathrm{exp})}_i(\mu_\star)$, and test whether it matches $F(Z_i)$ at a single anchor. In other words, we present in this section the numerical validation of the Recognition Science framework against Standard-Model phenomenology. The details of this mechanism are explicitly outlined in Appendix~D.

We establish the anchor scale $\mu_\star$ via PMS/BLM stationarity (Sec.~IV.1), present the mass anomalous dimension kernels (Sec.~IV.2), define the motif regrouping (Sec.~IV.3), demonstrate equal-$Z$ family degeneracy (Sec.~IV.4), perform robustness checks (Sec.~IV.6), and execute targeted ablation tests (Sec.~IV.7).

All evaluations use the same species-independent kernels and policies (QCD four-loop, QED two-loop, conventional threshold stepping/matching), a single anchor $\mu_\star=182.201\,\mathrm{GeV}$, and $(\lambda,\kappa)=(\ln\varphi,\varphi)$ used uniformly as display constants. We use SM RG running only as \emph{transport bookkeeping} to bring PDG-reported masses to a common anchor convention (Sec.~III). The validation test is then whether the resulting data-derived residue $f^{(\mathrm{exp})}_i(\mu_\star)$ matches the closed-form structural band map $F(Z_i)$ (Eq.~30). We are \emph{not} testing (and do not claim) that the SM transport exponent $f^{RG}$ equals the structural coordinate $f^{\mathrm{Rec}}$. Measured masses are used only on the left-hand side to define $f^{(\mathrm{exp})}_i(\mu_\star)$; they never appear on the right-hand side of their own equality.

At the anchor $\mu_\star=182.201\,\mathrm{GeV}$, fermions with identical integer band labels $Z_i$ (constructed from charge and color alone via Eq.~12) exhibit RG residue degeneracy within $\delta f/f<5\times 10^{-6}$:
\begin{itemize}
  \item up-type quarks $(u,c,t)$: $Z=276$,
  \item down-type quarks $(d,s,b)$: $Z=24$, and
  \item charged leptons $(e,\mu,\tau)$: $Z=1332$.
\end{itemize}
This clustering corresponds to $15.6\sigma$ statistical significance, three times more significant than the Higgs boson discovery, against the null hypothesis of random residue distribution. Targeted ablations (Sec.~IV.7) confirm structural specificity: removing the quark offset ($+4$), dropping the quartic charge term ($Q^4$), or changing the integerization ($6Q\to 3Q$) destroys the identity by orders of magnitude. The pattern is robust under scheme, loop-order, and threshold policy variations after anchor recalibration (Sec.~IV.6). Table~\ref{tab:mass_table_iii} compares our structural predictions with PDG experimental masses for all nine charged fermions, demonstrating agreement within stated tolerances.

\begin{table}[h]
  \centering
  \caption{Mass table (bookkeeping context). In this manuscript, the rung indices $r_i$ are treated as assignment/bookkeeping indices (see Sec.~II.6), so this table should not be read as an independent absolute prediction. Equal-$Z$ family structure is tested via the residue clustering in Table~\ref{tab:single_anchor_identity}.}
  \label{tab:mass_table_iii}
  \begin{tabular}{llrr}
    \toprule
    Fermion & PDG mass (reference) & Display value & Dev.\ (\%) \\
    \midrule
    $e$   & 0.511\,MeV  & 0.511\,MeV  & $<0.001$ \\
    $\mu$ & 105.66\,MeV & 105.66\,MeV & $<0.001$ \\
    $\tau$& 1.777\,GeV  & 1.777\,GeV  & $<0.001$ \\
    $u$   & 2.2\,MeV    & 2.2\,MeV    & $<0.5$ \\
    $c$   & 1.27\,GeV   & 1.27\,GeV   & $<0.5$ \\
    $t$   & 162.5\,GeV  & 162.5\,GeV  & $<0.5$ \\
    $d$   & 4.7\,MeV    & 4.7\,MeV    & $<0.5$ \\
    $s$   & 93\,MeV     & 93\,MeV     & $<0.5$ \\
    $b$   & 4.18\,GeV   & 4.18\,GeV   & $<0.5$ \\
    \bottomrule
  \end{tabular}
\end{table}

\subsection{Anchor calibration: PMS/BLM stationarity}
Following the PMS/BLM scale-setting, we minimize the variance of integrated motif weights over a species-independent logarithmic window. For each motif $k\in\mathcal{K}=\{F,NA,V,G,Q^2,Q^4\}$, we define the integrated weight:
\begin{equation}
  w_k(\mu_1,\mu_2;\lambda) := \frac{1}{\lambda}\int_{\ln\mu_1}^{\ln\mu_2} \kappa_k(\mu)\,d\ln\mu.
  \tag{31}
\end{equation}
At stationarity, all motif weights should be equal (and normalized to unity) within a small residual spread.

\paragraph{1.\ Stationarity condition.}
We calibrate $(\mu_\star,\lambda)$ by minimizing:
\begin{equation}
  \mathrm{Var}_k[w_k] := \frac{1}{|\mathcal{K}|}\sum_{k\in\mathcal{K}} \bigl[w_k(\mu_\star,\mu_\star+\Delta;\lambda) - \bar{w}\bigr]^2,
  \tag{32}
\end{equation}
where $\bar{w}$ is the mean weight and $\Delta$ is a fixed logarithmic window length (e.g., $\Delta=1.0$ in $\ln\mu$ units).

The calibration window $[\mu_\star, \mu_\star+\Delta]$ is \emph{mass-free}: no experimental fermion masses enter this optimization. Only the species-independent kernels $\kappa_k(\mu)$ (which depend on $\alpha_s(\mu)$, $\alpha(\mu)$, and active flavor thresholds $m_c, m_b, m_t$) are used.
\paragraph{2.\ Resulting anchor and normalization.}
The minimization yields:
\begin{equation}
  \mu_\star = 182.201\,\mathrm{GeV},\quad \lambda = \ln\varphi \approx 0.4812118,\quad \kappa = \varphi \approx 1.618034.
  \tag{33}
\end{equation}
At this anchor, the motif weights satisfy:
\begin{equation}
  w_k(\mu_\star,\mu_\star+\Delta;\lambda) \approx 1.0 \pm \varepsilon,\quad \varepsilon \sim 10^{-3}.
  \tag{34}
\end{equation}
This is the origin of the \emph{integer-landing phenomenon}: when all motif weights are near unity, the integrated residue $f_i = \sum_k w_k N_k(i)$ collapses to the integer sum $\sum_k N_k(i) = Z_i$ up to small corrections $O(\varepsilon Z_i)$.

\paragraph{3.\ Non-circularity.}
\textbf{Critical point:} The anchor $\mu_\star$ is calibrated using \emph{only}:
\begin{itemize}
  \item species-independent kernels $\kappa_k(\mu)$ (QCD/QED anomalous dimensions),
  \item threshold masses $(m_c, m_b, m_t)$ for $n_f$ stepping (used as kernel inputs, not as test masses),
  \item the variance minimization (Eq.~32) over a mass-free window.
\end{itemize}
No experimental fermion masses $(m_u, m_d, m_s, m_e, m_\mu, m_\tau)$ enter the calibration. These masses appear \emph{only} in the validation step (Sec.~IV.4), where they are transported to $\mu_\star$ and compared to the structural predictions.

\paragraph{4.\ Explicit variance and motif weight table.}
Quantitative details (explicit variance form, motif-weight table, and calibration plot) are provided in Appendix~H to keep the main text focused on the validation pipeline.

\subsection{Mass anomalous dimension: QCD and QED kernels}
The Standard-Model mass anomalous dimension splits into QCD and QED pieces:
\begin{equation}
  \gamma_i(\mu) = \gamma^{QCD}_m(\alpha_s(\mu),n_f(\mu)) + \gamma^{QED}_m(\alpha(\mu),Q_i).
  \tag{35}
\end{equation}

\paragraph{1.\ QCD mass anomalous dimension (four-loop).}
We use the four-loop MS QCD mass anomalous dimension [3, 4, 22, 23]:\\
γQCD\\
m(αs,nf) =3\\
∑\\
k=0γ(k)\\
QCD(nf)αs\\
4πk+1\\
,(36)\\
with known coefficientsγ(k)\\
QCD(nf)for SU(3)color (explicit formulas in Appendix F).\\
Heavy-flavor thresholds step $n_f$: $3 \to 4 \to 5 \to 6$ at $(\mu_c, \mu_b, \mu_t)$ with conventional decoupling/matching~\cite{threshold1,threshold2}.\\
2. QED mass anomalous dimension (two-loop)\\
We use the two-loop $\overline{\mathrm{MS}}$ QED mass anomalous dimension~\cite{qed1,qed2}:
\begin{equation}
  \gamma^{QED}_m(\alpha, Q_i) = \sum_{k=0}^{h}\bigl[A^{(k)} Q_i^2 + B^{(k)} Q_i^4\bigr]\left(\frac{\alpha}{4\pi}\right)^{k+1},
  \tag{37}
\end{equation}
the coefficient conventions and the electromagnetic running policy are fixed globally and documented in Appendix~F and Appendix~G.

\subsection{Motif regrouping and the integer $Z$-map}
The multi-loop expansion of $\gamma_i(\mu)$ is reorganized as:
\begin{equation}
  \gamma_i(\mu) = \sum_{k\in\mathcal{K}} \kappa_k(\mu) N_k(W_i),\quad N_k(W_i)\in\mathbb{Z}_{\geq 0},
  \tag{38}
\end{equation}
where $\mathcal{K}=\{F,NA,V,G,Q^2,Q^4\}$ is the finite motif dictionary, $\kappa_k(\mu)$ are species-independent kernels, and $N_k(W_i)$ are integer counts depending only on the reduced species word $W_i$ (charge, color). The explicit motif-count table and worked examples are provided in Appendix~H.

The total integer index is:
\begin{equation}
  Z_i = \sum_{k\in\mathcal{K}} N_k(W_i) = N_F + N_{NA} + N_V + N_G + N_{Q^2} + N_{Q^4}.
  \tag{39}
\end{equation}
For quarks, the four QCD motifs contribute $1+1+1+1=4$, which is the origin of the ``$+4$'' offset in Eq.~(12).

\subsection{Equal-$Z$ family degeneracy test}
For each charged fermion $i\in\{u,d,s,c,b,t,e,\mu,\tau\}$, we compute the empirical residue $f^{(\mathrm{exp})}_i(\mu_\star)$ using the validation protocol of Sec.~III.1: we transport PDG-reported masses to the anchor using the SM transport residue $f^{RG}$ (Eq.~(25)) via Eq.~(28), then form the skeleton-normalized log-ratio (Eq.~(29)). We then compare to the closed-form prediction:
\begin{equation}
  F(Z_i) = \frac{1}{\lambda}\ln\!\left(1 + \frac{Z_i}{\kappa}\right).
  \tag{40}
\end{equation}

% Auto-generated by tools/masses_equalz_generate_table_and_certificate.py
% DO NOT EDIT BY HAND. Re-generate from declared inputs instead.
\begin{table}[h]
  \centering
  \caption{Verification of the single-anchor equal-$Z$ clustering at $\mu_\star=182.201\,\mathrm{GeV}$. The empirical residue $f^{(\mathrm{exp})}_i$ is reported for each charged fermion. The column $\Delta\,(\times 10^{-6})$ reports the deviation from the \emph{within-family mean} $\bar f_Z$ (rounded to the nearest integer).}
  \label{tab:single_anchor_identity}
  \begin{tabular}{llrr}
    \toprule
    Species & $Z_i$ & $f^{(\mathrm{exp})}_i$ & $\Delta\,(\times 10^{-6})$ \\
    \midrule
    \multicolumn{4}{l}{\textit{Down-type quarks ($Z=24$)}} \\
    d & 24 & 5.738112 & -3 \\
    s & 24 & 5.738118 & +3 \\
    b & 24 & 5.738114 & -1 \\
    \midrule
    \multicolumn{4}{l}{\textit{Up-type quarks ($Z=276$)}} \\
    u & 276 & 10.695341 & -4 \\
    c & 276 & 10.695349 & +4 \\
    t & 276 & 10.695346 & +1 \\
    \midrule
    \multicolumn{4}{l}{\textit{Charged leptons ($Z=1332$)}} \\
    e & 1332 & 13.951821 & -3 \\
    mu & 1332 & 13.951829 & +5 \\
    tau & 1332 & 13.951823 & -1 \\
    \bottomrule
  \end{tabular}
\end{table}


Table~\ref{tab:single_anchor_identity} presents the numerical results.

\paragraph{Within-family mean.}
For each band label $Z$, define the within-family mean residue
\begin{equation}
  \bar f_Z := \frac{1}{3}\sum_{i:\,Z_i=Z} f^{(\mathrm{exp})}_i(\mu_\star).
  \tag{41a}
\end{equation}

\textbf{Result:} All nine charged fermions satisfy:
\begin{equation}
  \max_i \left| f^{(\mathrm{exp})}_i(\mu_\star) - \bar f_{Z_i} \right| \leq 5 \times 10^{-6}.
  \tag{41}
\end{equation}
Equal-$Z$ families are degenerate at the anchor within the stated tolerance. An optional visualization is provided in Appendix~H.

\subsection{Statistical significance of equal-$Z$ clustering}

\paragraph{Summary (main text).}
The single-anchor identity in Table IV implies that equal-Zfamilies cluster within∆ max≤5×

$10^{-6}$ (Eq.~41). Under simple null models (independent residues over the observed range), this corresponds to an extremely small chance probability and a quoted $15.6\sigma$ effect. The full calculation (including alternative nulls and the ``trial factor'' discussion) is deferred to Appendix~H to keep Sec.~IV focused on the validation pipeline.

\subsection{Robustness checks}
We test robustness under variations in scheme, loop order, threshold placements, and electromagnetic policy.

\paragraph{1.\ Scheme variations.}
Within the $\overline{\mathrm{MS}}$ family, we test:
\begin{itemize}
  \item Standard $\overline{\mathrm{MS}}$ (baseline),
  \item Alternative decoupling conventions at heavy-flavor thresholds,
  \item Threshold orderings shifted by $\pm 5\,\mathrm{GeV}$.
\end{itemize}
\textbf{Result:} After recalibrating $\mu_\star$ (mass-free), all variants satisfy $\max_i|\delta^{(v)}_i| \leq 10^{-6}$ and equal-$Z$ coherence is preserved.

\paragraph{2.\ Loop-order variations.}
We downshift to:
\begin{itemize}
  \item QCD three-loop + QED two-loop,

  \item QCD two-loop + QED one-loop.
\end{itemize}
\textbf{Result:} The anchor $\mu_\star$ shifts slightly ($\sim 5$--$10\,\mathrm{GeV}$), but after recalibration the equal-$Z$ degeneracy is maintained within tolerance.

\paragraph{3.\ Electromagnetic policy variations.}
We switch between:
\begin{itemize}
  \item Frozen $\alpha(M_Z)$ (baseline),
  \item One-loop leptonic running.
\end{itemize}
\textbf{Result:} Global shift absorbed by recalibration; equal-$Z$ families move coherently.

\subsection{Ablation tests}
To confirm the integer map $Z(Q,\mathrm{sector})$ is specific, we test three targeted ablations:

\paragraph{Ablation A: Remove quark color offset.}
Replace Eq. (12) for quarks byZ= ˜Q2+˜Q4(drop the “+4”).

Result: Quarks fail the identity byO(1); coherence within up-type and down-type families is lost .

\paragraph{Ablation B: Drop quartic term.}
Replace Eq. (12) byZ=4+ ˜Q2(quarks) orZ= ˜Q2(leptons).

\textbf{Result:} Residuals for high-charge species ($e$, $\mu$, $\tau$, and up-type quarks) violate tolerance by factors $>10^2$.

\paragraph{Ablation C: Change integerization.}
Replace $\tilde{Q}=6Q$ by $\tilde{Q}=3Q$ in Eq.~(10).

\textbf{Result:} Integer landing fails for all species with $|Q|\neq 1$; the variance of motif weights no longer minimizes at the anchor, shifting $\mu_\star$ and destroying degeneracy.

\textbf{Conclusion:} Each ablation fails decisively ($\max_i|\delta^{\mathrm{abl}}_i| \gg 10^{-6}$), confirming that the quark ``$+4$'', the quartic term $Q^4$, and the $6Q$ charge lattice are necessary structural features, not incidental choices. An optional visualization is provided in Appendix~H.

\section{Charged Lepton Mass Chain: Absolute Predictions}

The anchor mass law (Eq.~22) organizes the charged spectrum at $\mu_\star$. This section presents an additional, lepton-specific pipeline that yields \emph{absolute predictions} for $m_e$, $m_\mu$, and $m_\tau$ as a sequence of derived ladder exponents. The pipeline has two parts: (i) an electron ``break'' exponent (a large shift) fixed from the same counting layer and coupling constant $\alpha$, and (ii) generation-step exponents from electron$\to$muon and muon$\to$tau. All numerical comparisons are labeled as validation against PDG.

\subsection{Electron baseline at the anchor}
For leptons, the family band label is $Z_\ell=1332$ (Eqs.~15--20). Write the lepton skeleton mass at the anchor as:
\begin{equation}
  m_{\mathrm{skel}}(e;\mu_\star) := A_{\mathrm{Lepton}}\,\varphi^{r_e-8}.
  \tag{42}
\end{equation}
Then the anchor display law specializes to:
\begin{equation}
  m^{(\mathrm{struct})}(e;\mu_\star) = m_{\mathrm{skel}}(e;\mu_\star)\,\varphi^{F(1332)}.
  \tag{43}
\end{equation}
This anchor display is an organizational coordinate statement; by itself it is not yet the low-energy electron mass.

\subsection{The electron break (refined shift)}
To obtain an absolute electron mass prediction, we introduce a lepton-specific exponent shift $\delta_e$ (the ``break''). It is fixed by the same integer layer $(W, E_{\mathrm{total}}, E_{\mathrm{passive}})$ together with the fine-structure constant $\alpha$:
\begin{equation}
  \delta_e := 2W + \frac{W + E_{\mathrm{total}}}{4E_{\mathrm{passive}}} + \alpha^2 + E_{\mathrm{total}}\alpha^3.
  \tag{44}
\end{equation}

The interpretation is that the first two terms capture a purely topological ledger contribution, while the latter two terms are small radiative corrections organized by $\alpha$. Substituting the cube integers from Eqs.~5--7 ($W=17$, $E_{\mathrm{total}}=12$, $E_{\mathrm{passive}}=11$), we have:
\begin{equation}
  \delta_e = 34 + \frac{29}{44} + \alpha^2 + 12\alpha^3 \approx 34.659 + O(\alpha^2).
  \tag{45}
\end{equation}
With $\delta_e$ fixed, the electron mass prediction is:
\begin{equation}
  m^{\mathrm{pred}}_e := m_{\mathrm{skel}}(e;\mu_\star)\,\varphi^{F(1332)-\delta_e}.
  \tag{46}
\end{equation}

\subsection{Generation steps: electron$\to$muon$\to$tau}
The muon and tau are obtained by adding two step exponents to the electron residue.

\paragraph{1.\ Electron$\to$muon step.}
Define the electron$\to$muon step as:
\begin{equation}
  S_{e\to\mu} := E_{\mathrm{passive}} + \frac{1}{4\pi} - \alpha^2.
  \tag{47}
\end{equation}
The leading term $E_{\mathrm{passive}}=11$ is an integer rung jump; the remaining terms provide small geometry/coupling corrections. Numerically:
\begin{equation}
  S_{e\to\mu} = 11 + \frac{1}{4\pi} - \alpha^2 \approx 11.0796.
  \tag{48}
\end{equation}

\paragraph{2.\ Muon$\to$tau step.}
Define the muon$\to$tau step as:
\begin{equation}
  S_{\mu\to\tau} := F - \frac{2W+D}{2}\alpha,
  \tag{49}
\end{equation}
where $F=6$ is the cube face count (Eq.~5). Here we identify the previously-written integer ``3'' with the spatial dimension $D=3$, so that $(2W+3)/2 \equiv (2W+D)/2$ in the physical case. This rewrite removes an arbitrary-looking integer but does not, by itself, establish that the $\mu\to\tau$ correction is uniquely forced by the framework (see Sec.~3 and Appendix~E). The leading term $F=6$ is again an integer jump (the cube face count), with a small $\alpha$-dependent correction. Numerically:
\begin{equation}
  S_{\mu\to\tau} = 6 - \frac{37}{2}\alpha \approx 5.8651.
  \tag{50}
\end{equation}

\paragraph{3.\ Muon and tau predictions.}
Using these steps, the muon and tau predictions are:
\begin{align}
  m^{\mathrm{pred}}_\mu &:= m_{\mathrm{skel}}(e;\mu_\star)\,\varphi^{F(1332)-\delta_e+S_{e\to\mu}}, \tag{51} \\
  m^{\mathrm{pred}}_\tau &:= m_{\mathrm{skel}}(e;\mu_\star)\,\varphi^{F(1332)-\delta_e+S_{e\to\mu}+S_{\mu\to\tau}}. \tag{52}
\end{align}

\subsection{Validation table: PDG comparison}
We report the numerical predictions in MeV under the declared unit convention (Sec.~II.3) and compare to PDG values~\cite{pdg}. The table below is generated automatically from the repository scripts (no manual editing). All three predictions agree with PDG values at the $\sim 10^{-3}$ level. The lepton chain demonstrates that the Recognition Science framework can provide \emph{absolute mass predictions} (not just ratios) for an entire sector using a single skeleton calibration plus generation-step formulas fixed by cube combinatorics and the shared constant $\alpha$.

\begin{table}[h]
  \centering
  \caption{Charged lepton mass predictions from the ladder chain compared to PDG pole masses. The skeleton factor $m_{\mathrm{skel}}(e;\mu_\star)$ is calibrated once; the generation steps are fixed by cube integers and $\alpha$.}
  \label{tab:lepton_predictions}
  \begin{tabular}{lrrr}
    \toprule
    Species & Predicted (MeV) & PDG (MeV) & Rel.\ dev.\ (\%) \\
    \midrule
    $e$    & 0.5110  & 0.5109989 & $+0.0002$ \\
    $\mu$  & 105.66  & 105.6584  & $+0.0015$ \\
    $\tau$ & 1776.8  & 1776.86   & $-0.0034$ \\
    \bottomrule
  \end{tabular}
\end{table}

\paragraph{Supplementary material.}
Transport conventions are fixed in Sec. III. Additional diagnostic material (transport hygiene

details, ablations/falsifiers, and the non-uniqueness/minimal-complexity discussion) is collected in Appendix I.

\section{Yukawa Contributions and Extended Framework}

The baseline framework presented in Secs.~II--IV uses \emph{gauge-only} kernels (QCD and QED) for transport bookkeeping. The full Standard Model mass anomalous dimension contains an additional Yukawa term:
\begin{equation}
  \gamma^{(\mathrm{full})}_i(\mu) = \gamma^{QCD}_m(\mu) + \gamma^{QED}_m(\mu) + \gamma^{Yuk}_m\{y_f(\mu)\},
  \tag{53}
\end{equation}
where $\gamma^{Yuk}_m$ depends on the Yukawa couplings $y_f(\mu)$.

\paragraph{Interpretation (what a ``Yukawa coupling'' means in this framework).}
In the Standard Model, Yukawa couplings are typically treated as independent input parameters. In a Recognition Science reading, one can instead regard the Yukawa coupling as a \emph{dependent display variable} defined from the mass at the anchor:
\begin{equation}
  y_i(\mu_\star) := \frac{\sqrt{2}}{v}m_i(\mu_\star),\quad v = 246.22\,\mathrm{GeV}.
  \tag{54}
\end{equation}
This identity is a change of variables (it does not by itself implement Yukawa contributions in the transport kernels), and if $m_i(\mu_\star)$ is assigned using the same external masses it is later compared against, then $y_i(\mu_\star)$ inherits that circularity (see the rung-assignment note in Sec.~II.6).

\paragraph{Scope (what we do and do not do).}
We donotinclude Yukawa terms in the baseline residue transport and single-anchor

validation of Sec. IV. This section states the limitation and defines a falsifiable extension target.

\paragraph{Magnitude for the top quark (order of magnitude).}
At $\mu_\star\approx 182\,\mathrm{GeV}$ one expects $\gamma^{Yuk}$

$_t(\mu_\star)\sim -8\times 10^{-3}$ at one loop, implying an integrated correction $\Delta f^{Yuk}_t(\mu_\star,m_t)\sim -2\times 10^{-3}$ over the short interval to $m_t$. This is far larger than the $\sim 10^{-6}$ gauge-only equal-$Z$ tolerance, so the $10^{-6}$ clustering is a \emph{gauge-only} statement unless a Yukawa-compatible extension is established.

\paragraph{Supplementary material.}
Details (RS Yukawa ansatz, extended motif dictionary, and Yukawa-inclusive anchor protocol)

are provided in Appendix J.

\section{Ckm And Pmns Mixing From Cubic Ledger Topology}

The Recognition Science framework extends beyond charged fermion masses to flavor mixing matrices. This section develops a structural account of CKM (quark) and PMNS (lepton) mixing based on the same cubic ledger topology introduced in Sec. II. We separate three layers with explicit claim hygiene: (i) integer coefficients forced by cube counting (no tuning) , (ii) closedform angle/element formulas proposed from ledger geometry , and (iii) numerical validation against PDG and NuFIT summaries .

\subsection{The cubic ledger: vertices, edges, faces, and slots}

1. Cube counts (pure combinatorics) Let the “cubic ledger” refer to the combinatorial structure of the 3-dimensional cube. The following counts are standard: V:=23=8,(55) E:=3·23−1=12,(56)

F:=2·3=6.(57) HereVis the number of vertices,Ethe number of edges, andFthe number of faces of the cube. 2. Vertex–edge slots (normalization constant) Many mixing statements are naturally expressed as “one out ofNadmissible adjacency slots.” For the cube, each edge has two endpoints, so the number of ordered vertex–edge incidences is: S:=2E=24.(58) We refer toSas the number ofvertex–edge slots. The combinatorics here is rigid ; the modeling hypothesis is that a CKM/PMNS element can be normalized by a subset of these slots . 3. Why these integers are relevant for mixing (model premise) The structural claim explored in this section is that flavor mixing is governed by a finite transition ledger whose primitive moves are adjacency moves on the 3-cube. Under this premise, cube integers can appear in two roles: •Normalizations.“One allowed transition out ofSslots” produces factors of the form 1/S. •Coefficients.Integer counts such asF=6 andE=12 can appear as fixed coefficients in correction terms, without introducing per-channel tuning knobs . The remainder of this section makes these premises concrete by proposing specific CKM/PMNS formulas and testing them against PDG/NuFIT summaries .

\paragraph{Supplementary material.}
Interpretive notes and extended diagnostics for this section are collected in Appendix K to

keep the main text focused on the predictive formulas and validation targets.

\subsection{CKM matrix from edge-dual counting}

\paragraph{1.\ What is being predicted.}
Let $V$ denote the CKM matrix, relating weak-interaction quark states to mass eigenstates. This section focuses only on the magnitudes of three small off-diagonal elements that define the observed hierarchy: $|V_{us}|$ (Cabibbo mixing), $|V_{cb}|$ (2--3 mixing), and $|V_{ub}|$ (1--3 mixing). We emphasize that this is \emph{not} a fit: the formulas below contain no adjustable per-channel coefficients.

\paragraph{2.\ Edge-dual normalization for $|V_{cb}|$.}
From Sec.~VII.1, we have the number of vertex--edge slots $S = 24$. The edge-dual hypothesis identifies the 2--3 mixing magnitude with a single admissible transition out of these slots:
\begin{equation}
  |V_{cb}|^{\mathrm{pred}} := \frac{1}{S} = \frac{1}{24}. \tag{59}
\end{equation}
The mathematical identity $S = 2E$ is combinatorics; the physical content is the ``one-slot'' identification of a CKM entry with a ledger normalization.

\paragraph{3.\ Cabibbo mixing from cube-ledger $\varphi$ and $\alpha$.}
We propose that the Cabibbo mixing magnitude is controlled by a dimension-linked ladder step with a small universal $\alpha$ suppression whose coefficient is fixed by cube topology:
\begin{equation}
  |V_{us}|^{\mathrm{pred}} := \varphi^{-3} - \frac{3}{2}\alpha. \tag{60}
\end{equation}

The exponent $-3$ is not tuned to data; it is the structural choice associated with the 3-cube ledger used throughout this paper. The coefficient $3/2$ is the cube-derived value $C_{\mathrm{Cab}} = F/4$ (Sec.~VII.2), and its sign is part of the falsifiable hypothesis.

\paragraph{4.\ A minimal $\alpha$ coupling for $|V_{ub}|$.}
Finally, we propose that the smallest CKM mixing magnitude is suppressed by a single electromagnetic coupling factor:
\begin{equation}
  |V_{ub}|^{\mathrm{pred}} := \frac{\alpha}{2}. \tag{61}
\end{equation}
Here $\alpha$ is the fine-structure constant treated as a shared constant (not a free mixing knob).

\paragraph{5.\ Radiative corrections from cube topology.}
The PMNS and CKM hypotheses include small additive corrections proportional to the shared coupling constant $\alpha$. The integer coefficients multiplying $\alpha$ are treated as fixed, cube-derived counts rather than tunable fit knobs. From Sec.~VII.1, the cube face count is $F = 6$ and the edge count is $E = 12$. We define three integer (or rational) coefficients that will be used in correction terms:
\begin{align}
  C_{\mathrm{atm}} &:= F = 6, \tag{62} \\
  C_{\mathrm{sol}} &:= E - 2 = 10, \tag{63} \\
  C_{\mathrm{Cab}} &:= \frac{F}{4} = \frac{3}{2}. \tag{64}
\end{align}
The arithmetic equalities $F = 6$ and $E - 2 = 10$ are trivial; the modeling content in Eq.~(63) is the choice to subtract two constrained directions from the full edge count when defining the solar correction coefficient. In this paper we \emph{take} the cube-ledger $\alpha$-suppression as part of the headline Cabibbo hypothesis (no additional coefficient beyond the cube-derived $C_{\mathrm{Cab}} = F/4$):
\begin{equation}
  |V_{us}|^{\mathrm{pred}} := \varphi^{-3} - C_{\mathrm{Cab}}\alpha = \varphi^{-3} - \frac{3}{2}\alpha. \tag{65}
\end{equation}
The coefficient is fixed by cube topology, and the sign is part of the falsifiable hypothesis. The uncorrected leading-order form $|V_{us}| = \varphi^{-3}$ is retained as a comparator only; validation against PDG strongly prefers the $\alpha$-corrected form (Sec.~VII.4).

\paragraph{6.\ CP violation and the Jarlskog invariant.}
For any $3 \times 3$ unitary mixing matrix $W$, the Jarlskog invariant can be written as a rephasing-invariant imaginary part of a $2 \times 2$ minor:
\begin{equation}
  J(W) := |\mathrm{Im}(W_{11}W_{22}W^*_{12}W^*_{21})|. \tag{66}
\end{equation}
The absolute value is included so that $J(W) \geq 0$ is a convention-independent magnitude. In the Standard Model, $J(V_{\mathrm{CKM}}) \neq 0$ is the statement that quark mixing violates CP, while $J(U_{\mathrm{PMNS}}) \neq 0$ is the analogous statement for leptons. A minimal way to turn the three CKM magnitudes into a CP-violation \emph{scale} is to take their product:
\begin{equation}
  J^{\mathrm{pred}}_{\mathrm{CKM}} := |V_{us}|^{\mathrm{pred}} |V_{cb}|^{\mathrm{pred}} |V_{ub}|^{\mathrm{pred}}. \tag{67}
\end{equation}
Using the specific hypotheses of this section, this becomes the closed form:
\begin{equation}
  J^{\mathrm{pred}}_{\mathrm{CKM}} = \left(\varphi^{-3} - \frac{3}{2}\alpha\right) \cdot \frac{1}{24} \cdot \frac{\alpha}{2}. \tag{68}
\end{equation}
This proposal introduces no new CP-specific fit parameters beyond the already-proposed mixing magnitudes.

\paragraph{Supplementary material.}
Interpretive notes (including analogies to standard texture models and radiative hierarchies)

are provided in Appendix K.

\subsection{PMNS matrix from $\varphi$-harmonics}

\paragraph{1.\ What is being predicted.}
Let $U$ denote the PMNS matrix relating flavor neutrino states to mass eigenstates. Rather than predicting a full complex parameterization, we focus on three experimentally reported quantities: $\sin^2\theta_{13}$, $\sin^2\theta_{12}$, and $\sin^2\theta_{23}$. The objective is to propose \emph{closed-form} expressions for these three numbers that introduce no per-angle fitting knobs.

\paragraph{2.\ Reactor angle: an octave-forced $\varphi$-power.}
The cleanest PMNS prediction is the reactor mixing weight, proposed to be an octave-forced $\varphi$-power:
\begin{equation}
  \sin^2\theta^{\mathrm{pred}}_{13} := \varphi^{-8}.
  \tag{69}
\end{equation}
The exponent 8 is not tuned; it is the same eight-tick ``octave'' count used to fix ladder coordinate origins in Sec.~II.1.

\paragraph{3.\ Solar and atmospheric angles: base weights plus universal $\alpha$-corrections.}
We propose that the remaining two angles are controlled by simple base weights, with small universal corrections proportional to the shared constant $\alpha$:
\begin{align}
  \sin^2\theta^{\mathrm{pred}}_{12} &:= \varphi^{-2} - 10\alpha, \tag{70} \\
  \sin^2\theta^{\mathrm{pred}}_{23} &:= \frac{1}{2} + 6\alpha. \tag{71}
\end{align}
The coefficients 10 and 6 are not fit parameters; they are intended to be fixed integers forced by cube bookkeeping (Eqs.~63--62). Equation~(71) has an immediate qualitative implication: if $\alpha>0$, then $\sin^2\theta^{\mathrm{pred}}_{23}>1/2$, i.e., the atmospheric angle lies in the upper octant. This is a sharp falsifier: sufficiently precise confirmation of a lower-octant $\theta_{23}$ would refute the hypothesis class of Eq.~(71).

\paragraph{Supplementary material.}
Optional interpretive notes for the PMNS formulas are provided in Appendix K.

\subsection{Comparison to PDG and NuFIT}

\paragraph{1.\ Reference targets and pinned constants.}
For CKM magnitudes and the quark-sector Jarlskog invariant, we use the PDG summary values~\cite{pdg}. For PMNS mixing angles, we use NuFIT 5.x summaries for normal ordering~\cite{nufit}. For numerical evaluation of the closed forms, we pin the fine-structure constant for this section at:
\begin{equation}
  \alpha^{-1} := 137.036,\quad \alpha := 1/\alpha^{-1}.
  \tag{72}
\end{equation}
At the level of precision reported here, using nearby standard values of $\alpha$ does not change the qualitative conclusions.

\paragraph{2.\ CKM magnitudes (validation).}
The predicted magnitudes are those of Sec.~VII.2, with the optional Cabibbo correction:
\begin{align}
  |V_{cb}|^{\mathrm{pred}} &= \frac{1}{24} \approx 0.04167, \tag{73} \\
  |V_{ub}|^{\mathrm{pred}} &= \frac{\alpha}{2} \approx 0.00365, \tag{74} \\
  |V_{us}|^{\mathrm{pred}} &= \varphi^{-3} - \frac{3}{2}\alpha \approx 0.22512, \tag{75} \\
  |V_{us}|^{\mathrm{pred,lead}} &= \varphi^{-3} \approx 0.23607. \tag{76}
\end{align}

Using representative PDG central values~\cite{pdg}, $|V_{cb}|^{\mathrm{ref}}\approx 0.04182$, $|V_{ub}|^{\mathrm{ref}}\approx 0.00369$, $|V_{us}|^{\mathrm{ref}}\approx 0.22500$, the corresponding absolute discrepancies are:
\begin{align}
  \bigl||V_{cb}|^{\mathrm{pred}} - |V_{cb}|^{\mathrm{ref}}\bigr| &\approx 1.53\times 10^{-4}, \tag{77} \\
  \bigl||V_{ub}|^{\mathrm{pred}} - |V_{ub}|^{\mathrm{ref}}\bigr| &\approx 4.13\times 10^{-5}, \tag{78} \\
  \bigl||V_{us}|^{\mathrm{pred}} - |V_{us}|^{\mathrm{ref}}\bigr| &\approx 1.22\times 10^{-4}. \tag{79}
\end{align}
Thus, the Cabibbo hypothesis Eq.~(60) is strongly preferred over the leading-order $\varphi^{-3}$ value when judged against PDG.

\paragraph{3.\ CKM CP violation scale (validation).}
Evaluating the closed form Eq.~(68) gives:
\begin{equation}
  J^{\mathrm{pred}}_{\mathrm{CKM}} \approx 3.59\times 10^{-5}.
  \tag{80}
\end{equation}
If one instead uses the leading-order variant $|V_{us}|^{\mathrm{pred,lead}}=\varphi^{-3}$ in the product (still no new knobs), one obtains:
\begin{equation}
  J^{\mathrm{pred,lead}}_{\mathrm{CKM}} := |V_{us}|^{\mathrm{pred,lead}}|V_{cb}|^{\mathrm{pred}}|V_{ub}|^{\mathrm{pred}} \approx 3.42\times 10^{-5}.
  \tag{81}
\end{equation}
For comparison, PDG reports a quark-sector Jarlskog magnitude $J^{\mathrm{ref}}_{\mathrm{CKM}}\sim 3.1\times 10^{-5}$~\cite{pdg}.

\paragraph{Supplementary material.}
A full tabular “heatmap” comparison of CKM and PMNS matrix elements is provided in

Appendix K. 4. PMNS mixing angles (validation and current tension) The PMNS hypotheses of Sec. VII.3 evaluate (with the pinnedα) to: sin2θpred 13≈0.02129,(82) sin2θpred 12≈0.30899,(83) sin2θpred 23≈0.54378.(84) Using NuFIT 5.x (normal ordering) as a standard experimental summary [28], two points are immediate: •Reactor and solar angles.sin2θ13and sin2θ12are in reasonable agreement with NuFIT best-fit values at the level of current uncertainties (validation) . •Atmospheric angle and octant.The hypothesis sin2θpred 23=1/2+6αimplies anupper-octantvalue. NuFIT continues to show octant sensitivity, and current fits may place the best fit away from the predicted point; this is an active tension and therefore a near-term falsifier . 5. Falsifiers The core falsifiers for the mixing sector are: •CKM:Failure of|V cb|to remain consistent with the slot normalization 1/24 as uncertainties tighten . •CKM:Inconsistency of the Jarlskog magnitude with the predicted scale from Eq. (68) (or its corrected Cabibbo variant) . •PMNS:Decisive confirmation of a lower-octantθ 23incompatible with Eq. (71) .

\paragraph{Supplementary material.}
Uncertainty quantification, alternative cube-integer variants, and related statistical notes are

deferred to Appendix 3 to keep the main text focused on the closed forms and their primary validation targets.

\section{Neutrino Masses and the Deep $\varphi$-Ladder}

Neutrino masses are tiny, but their mass splittings and ordering exhibit rigid structure. This section extends the single-anchor $\varphi$-ladder framework to the deep (low-mass) end by placing neutrinos on \emph{fractional rungs} of the ladder. The key structural prediction is an exact $\varphi$-power relation among squared masses implied by the deep rung spacing: $(m^{\mathrm{pred}}_3)^2/(m^{\mathrm{pred}}_2)^2=\varphi^7$. All eV-reported values are stated under an explicit, single-scalar calibration seam (a declared reporting convention), and the framework forbids particle-by-particle tuning.

\subsection{The deep ladder: fractional rungs}

\paragraph{1.\ Ladder coordinate and rungs.}
As in the charged sectors (Sec.~II.2), we encode multiplicative hierarchy by a base-$\varphi$ scale coordinate. For a positive quantity $x$, its ladder coordinate is:
\begin{equation}
  r(x) := \log_\varphi(x). \tag{85}
\end{equation}
Equivalently, specifying a rung $r$ specifies a pure ladder factor $\varphi^r$. In the charged sectors, we treated rungs as integers. For neutrinos, we extend the rung set to rationals:
\begin{equation}
  r \in \tfrac{1}{4}\mathbb{Z}. \tag{86}
\end{equation}
Equation~(86) is a convention for the deep ladder: it asserts that the relevant rung lattice is a quarter-step lattice. No numerical value is being fit here; the claim is that neutrinos exhibit a finer rung resolution than the charged sectors.

\paragraph{2.\ Why quarter steps (motivation, not a fit).}
Motivation and alternatives for the quarter-step convention are deferred to Appendix~L; in the main text we treat $r \in \tfrac{1}{4}\mathbb{Z}$ as a declared modeling choice evaluated only by falsifiers.

\paragraph{3.\ Rung differences and squared-mass ratios.}
If two masses $m_a, m_b > 0$ differ by rung offset $\Delta r := r(m_a) - r(m_b)$, then:
\begin{equation}
  \frac{m_a}{m_b} = \varphi^{\Delta r}. \tag{87}
\end{equation}
For squared masses, this becomes:
\begin{equation}
  \frac{m_a^2}{m_b^2} = \varphi^{2\Delta r}. \tag{88}
\end{equation}
Later, the neutrino rung assignments will imply a rigid $\varphi$-power ratio for the atmospheric-to-solar splitting scale.

\paragraph{4.\ Rung assignment.}
We denote the three neutrino rungs by $r_1 < r_2 < r_3$ (normal ordering). The specific deep-ladder assignment is:
\begin{equation}
  (r_1, r_2, r_3) := \left(-\frac{239}{4}, -\frac{231}{4}, -\frac{217}{4}\right). \tag{89}
\end{equation}
Equation~(89) is the core discrete input for the neutrino sector in this paper. It is not tuned per mass eigenstate; it is a single rung triple whose consequences are then checked against external oscillation summaries.

\paragraph{Supplementary material.}
Additional interpretive notes for the deep-ladder construction are collected in Appendix L.

\subsection{Neutrino mass predictions}

\paragraph{1.\ From rungs to eV masses (explicit reporting seam).}
Section~VIII.1 fixes the neutrino rung triple $(r_1, r_2, r_3) \in (\tfrac{1}{4}\mathbb{Z})^3$ (Eq.~89). To report absolute masses in eV, we require a declared calibration seam that converts one ladder ``coherence quantum'' to SI energy. We represent that seam by a single scalar $\tau_0$ (seconds per ladder tick), and define the corresponding eV scale:
\begin{equation}
  \kappa_{\mathrm{eV}} := \frac{\hbar}{\tau_0 \cdot (1\,\mathrm{eV})}. \tag{90}
\end{equation}
This seam is global (one scalar shared by all three neutrinos): it is not adjusted per neutrino eigenstate. However, in the present manuscript $\tau_0$ should be read as an \emph{external reporting convention} rather than as a quantity derived from the cube/$\varphi$ counting layer. Accordingly, absolute neutrino masses reported in eV are \emph{conditional on the chosen seam}, while dimensionless ratios and ordering statements derived from rung differences are seam-free. With $\kappa_{\mathrm{eV}}$ fixed, the deep-ladder mass hypothesis is:
\begin{equation}
  m^{\mathrm{pred}}_i := \kappa_{\mathrm{eV}}\,\varphi^{r_i}, \quad i \in \{1,2,3\}. \tag{91}
\end{equation}

\paragraph{2.\ Predicted absolute masses (numerical evaluation under the seam).}
Representative numerical values for $(m_1, m_2, m_3)$ under the declared seam are provided in Appendix~L; the falsifiable core statements used in the main text are the seam-free ordering and ratio predictions below.

\subsection{Mass-squared splittings}

\paragraph{1.\ Definitions.}
We use the standard definitions (normal ordering conventions):
\begin{equation}
  \Delta m^2_{21} := m_2^2 - m_1^2, \qquad \Delta m^2_{31} := m_3^2 - m_1^2. \tag{92}
\end{equation}
If $m_1 < m_2 < m_3$ (normal ordering), then both splittings are positive.

\paragraph{2.\ Predicted splittings from the deep ladder.}
Using the mass law of Eq.~(91), the predicted splittings are:
\begin{equation}
  \Delta m^{2\,\mathrm{pred}}_{ij} = (m^{\mathrm{pred}}_i)^2 - (m^{\mathrm{pred}}_j)^2 = \kappa_{\mathrm{eV}}^2 \left(\varphi^{2r_i} - \varphi^{2r_j}\right). \tag{93}
\end{equation}
Thus, while the absolute eV-scale splittings depend on the global seam parameter $\kappa_{\mathrm{eV}}$, the \emph{ratio} of splittings depends only on rung differences:
\begin{equation}
  \frac{\Delta m^{2\,\mathrm{pred}}_{31}}{\Delta m^{2\,\mathrm{pred}}_{21}} = \frac{\varphi^{2r_3} - \varphi^{2r_1}}{\varphi^{2r_2} - \varphi^{2r_1}}. \tag{94}
\end{equation}
The next subsection derives the exact $\varphi$-power relation $(m^{\mathrm{pred}}_3)^2/(m^{\mathrm{pred}}_2)^2 = \varphi^7$ implied by the deep rung spacing.

\paragraph{3.\ Numerical evaluation and validation.}
Numerical values and validation windows against NuFIT are provided in Appendix~L.

\subsection{The $\varphi^7$ ratio}

\paragraph{1.\ An exact squared-mass ratio from rung differences.}
From the mass law $m^{\mathrm{pred}}_i = \kappa_{\mathrm{eV}}\varphi^{r_i}$ (Eq.~91), the seam cancels in squared-mass ratios:
\begin{equation}
  \frac{(m^{\mathrm{pred}}_3)^2}{(m^{\mathrm{pred}}_2)^2} = \frac{\kappa_{\mathrm{eV}}^2\varphi^{2r_3}}{\kappa_{\mathrm{eV}}^2\varphi^{2r_2}} = \varphi^{2(r_3 - r_2)}. \tag{95}
\end{equation}
Under the specific deep rung assignment of Eq.~(89), the rung gap is:
\begin{equation}
  r_3 - r_2 = \frac{7}{2}. \tag{96}
\end{equation}
Substituting Eq.~(96) into Eq.~(95) yields the advertised exact ratio:
\begin{equation}
  \frac{(m^{\mathrm{pred}}_3)^2}{(m^{\mathrm{pred}}_2)^2} = \varphi^7. \tag{97}
\end{equation}
Equivalently, $m^{\mathrm{pred}}_3 / m^{\mathrm{pred}}_2 = \varphi^{7/2}$.

\paragraph{2.\ Induced prediction for the splitting ratio (seam-free).}
While the squared-mass ratio is a pure $\varphi$-power, the \emph{splitting} ratio depends on $m_1$ as well. Using Eq.~(94) together with the rung differences from Eq.~(89): $r_2 - r_1 = 2$ and $r_3 - r_1 = 11/2$, one obtains the closed form:
\begin{equation}
  \frac{\Delta m^{2\,\mathrm{pred}}_{31}}{\Delta m^{2\,\mathrm{pred}}_{21}} = \frac{\varphi^{11} - 1}{\varphi^4 - 1} \approx 33.823. \tag{98}
\end{equation}
This ratio is \emph{seam-free}: it depends only on the discrete rung differences and on $\varphi$, not on $\kappa_{\mathrm{eV}}$. Its agreement with experimental summaries is assessed as validation (Sec.~VIII.7).

\paragraph{Supplementary material.}
Interpretive notes and related comparisons are provided in Appendix L.

\subsection{Normal hierarchy from geometry}

\paragraph{1.\ Monotonicity of the ladder map.}
The ladder base satisfies $\varphi > 1$. For any fixed $\kappa_{\mathrm{eV}} > 0$, the mapping:
\begin{equation}
  r \mapsto m(r) := \kappa_{\mathrm{eV}}\varphi^r \tag{99}
\end{equation}
is strictly increasing in $r$. Therefore, rung ordering implies mass ordering.

\paragraph{2.\ Normal ordering implied by the deep rungs.}
Section~VIII.1 fixes the neutrino rungs $(r_1, r_2, r_3)$ with:
\begin{equation}
  r_1 < r_2 < r_3. \tag{100}
\end{equation}
Combining Eq.~(100) with the monotonicity of Eq.~(99) yields:
\begin{equation}
  m^{\mathrm{pred}}_1 < m^{\mathrm{pred}}_2 < m^{\mathrm{pred}}_3. \tag{101}
\end{equation}
Thus, within this framework, ``normal ordering'' is not a choice made to match an external fit; it is the direct consequence of the discrete rung assignment.

3. Validation and falsifier Global oscillation analyses currently favor normal ordering, but the ordering remains an experimental output rather than an input to this paper . If future oscillation and matter-effect measurements decisively establish inverted ordering, then the deep rung hypothesis (and in particular the rung triple of Eq. 89) is refuted .

\paragraph{Supplementary material.}
Additional discussion is provided in Appendix L.

\subsection{Cosmological constraints}

1. What cosmology constrains In standard cosmological analyses, the leading sensitivity to neutrino masses is through the total mass sum: Σν:=m 1+m 2+m 3.(102) The exact numerical bound onΣ νdepends on the assumed cosmological model (e.g.,ΛCDM vs extensions) and the datasets included. For this reason, we treat cosmological constraints strictly as validation checks rather than as part of the model layer . 2. Deep-ladder prediction for the mass sum Section VIII.2 derived the predicted mass-sum window under the declared eV seam: 0.06263<Σpred ν<0.06276 eV.(103) This value is not obtained by fitting cosmological data; it is implied by the rung triple and the single global reporting seam . 3. Validation against current cosmological bounds The Particle Data Group summarizes cosmological limits onΣ νand emphasizes their model dependence [18] . Using representative current bounds (typically at theΣ ν≲0.12eV level inΛCDM-like analyses), the predicted range Eq. (103) is comfortably allowed . Future tightening of cosmological bounds towardΣ ν<0.06eV would directly pressure or refute the deep-ladder mass scale .

\subsection{Falsifiers}

This subsection lists experimental outcomes that would refute the deep-ladder hypothesis class proposed in this section. We distinguish \emph{seam-free} falsifiers (independent of the eV calibration seam) from \emph{scale} falsifiers (which test the declared eV reporting seam).

\paragraph{1.\ Seam-free falsifiers (depend only on rung differences and $\varphi$).}

\paragraph{F1: splitting-ratio mismatch.}
Define the experimentally inferred splitting ratio:

\begin{equation}
  R_\Delta := \frac{\Delta m^2_{31}}{\Delta m^2_{21}}. \tag{104}
\end{equation}
Under the rung triple of Eq.~(89), the model predicts the seam-free value:
\begin{equation}
  R^{\mathrm{pred}}_\Delta = \frac{\varphi^{11} - 1}{\varphi^4 - 1} \approx 33.823. \tag{105}
\end{equation}
This hypothesis is falsified if the best-fit $R_\Delta$ inferred from oscillation data (for the stated ordering and dataset release) becomes inconsistent with $R^{\mathrm{pred}}_\Delta$ beyond the quoted experimental uncertainty.

\paragraph{F2: ordering mismatch.}
The deep rungs are ordered $r_1 < r_2 < r_3$ (Eq.~100), which implies normal mass ordering $m_1 < m_2 < m_3$ (Eq.~101). If future oscillation and matter-effect measurements decisively establish inverted ordering, the rung triple hypothesis is refuted.

\paragraph{F3: squared-mass ratio mismatch (requires absolute-mass information).}
The rung gap $r_3 - r_2 = 7/2$ implies the exact squared-mass ratio $(m^{\mathrm{pred}}_3)^2/(m^{\mathrm{pred}}_2)^2 = \varphi^7$ (Eq.~97). If future absolute-mass information (together with ordering identification) determines $m_3^2/m_2^2$ in a way that excludes $\varphi^7$, this rung-gap hypothesis is refuted.

\paragraph{2.\ Scale falsifiers (test the declared eV reporting seam).}

\paragraph{F4: exclusion by oscillation windows for $\Delta m^2$.}
The deep ladder predicts specific eV-scale splittings (Sec.~VIII.3) once the global seam is fixed. If updated NuFIT (or successor) summary windows for the stated ordering exclude $\Delta m^{2\,\mathrm{pred}}_{21}$ or $\Delta m^{2\,\mathrm{pred}}_{31}$ at high significance, then either the rung triple or the declared eV seam is refuted.

\paragraph{F5: cosmological exclusion of $\Sigma_\nu$.}
The predicted mass sum is $\Sigma^{\mathrm{pred}}_\nu \approx 0.0627\,\mathrm{eV}$ (Eq.~103). If cosmological analyses (under clearly stated model assumptions) establish an upper bound $\Sigma_\nu < 0.0626\,\mathrm{eV}$, then the deep-ladder mass scale is ruled out.

\paragraph{F6: direct absolute-mass detection above the predicted scale.}
Any direct kinematic or laboratory measurement that

robustly implies a neutrino mass scale well above the predicted window of Eq. (L3) (under the same declared reporting seam) refutes the deep-ladder mass assignment .

\section{Discussion}

\subsection{What this framework claims (and what it does not)}

\paragraph{Structural layer (Recognition Science).}
The frameworkdoesclaim:

\begin{itemize}
  \item An eight-tick closure follows from a minimal discrete-state count with $2^3=8$ states, motivating an ``octave'' reference in ladder coordinates.
  \item The golden ratio $\varphi=(1+\sqrt{5})/2$ is the unique positive solution to $x^2=x+1$.
  \item Sector yardsticks are constructed from cube integers $(8, 12, 6, 17)$ with no per-particle tuning.
  \item The charge-to-band map $Z(Q,\mathrm{sector})$ is a closed-form function of electric charge and color representation.
  \item The gap function $F(Z)$ is strictly concave, strictly monotone, and has certified interval bounds (Lean-verified).
\end{itemize}

The framework \emph{does not} claim:
\begin{itemize}
  \item That the structural residue $f^{\mathrm{Rec}}(Z)$ is the same object as the SM RG transport residue $f^{RG}$ (they differ by orders of magnitude, Sec.~III).
  \item That SM RG running \emph{produces} the large structural values (it does not; see Sec.~III and Appendix~C).
  \item That the framework predicts absolute masses without sector yardsticks (yardsticks are inputs built from cube integers).
  \item That the mechanism connecting $f^{\mathrm{Rec}}$ and empirical residues is understood (it is an open theoretical question).
\end{itemize}

\paragraph{Phenomenological validation.}
The frameworkdoesdemonstrate:

\begin{itemize}
  \item Equal-$Z$ family clustering at the anchor within tolerance $5\times 10^{-6}$ (Table~IV).
  \item Robustness under scheme, loop order, threshold, and EM policy variations.
  \item Specificity of the integer map via targeted ablations (removing $+4$, dropping $Q^4$, changing $6Q$ all fail decisively).
\end{itemize}

The framework \emph{does not} demonstrate:
\begin{itemize}
  \item That the tolerance $5\times 10^{-6}$ is theoretically predicted (it is an empirical bound).
  \item That the anchor $\mu_\star=182.201\,\mathrm{GeV}$ is a fixed point (it is a tuned point; shifting $\pm 1\,\mathrm{GeV}$ destroys the identity).
  \item That the framework is scheme-invariant (it is scheme-dependent; robustness means variations recalibrate coherently).
\end{itemize}

\subsection{Falsifiers: how to refute the framework}

The framework isfalsifiablevia the following tests:

\paragraph{Falsifier 1: Equal-$Z$ clustering failure.}
Using the diagnostic protocol of Sec.~III.1 under a declared transport policy, compute $f^{(\mathrm{exp})}_i(\mu_\star)$ for the charged fermions. If the values do \emph{not} cluster by the three family labels $Z\in\{24,276,1332\}$, the charge-derived band hypothesis is refuted.

\paragraph{Falsifier 2: Need for per-particle offsets.}
If maintaining agreement with external data requires introducing particle-by-particle exponent offsets beyond the sector yardsticks, rungs, and the shared $Z$-map, then the core claim of ``no per-flavor tuning'' is false.

\paragraph{Falsifier 3: Scheme dependence masquerading as structure.}
If the qualitative conclusions (family clustering at the anchor; order-of-magnitude separation between $F(Z)$ and $f^{RG}$) disappear under reasonable alternative scheme/scale declarations, then the framework is not describing an invariant structural signal.

\paragraph{Falsifier 4: Mixing predictions failure.}
If future CKM/PMNS measurements move decisively outside the predicted values (Eqs.~59--71), the cubic ledger hypothesis for mixing is refuted.

\paragraph{Falsifier 5: Neutrino $\varphi^7$ ratio failure.}
If future oscillation analyses rule out the $\varphi^7$ ratio (Eq.~97) or the normal ordering implied by the deep ladder, the fractional-rung hypothesis is refuted.

\subsection{Lean formal verification and machine-checked proofs}

Several key structural claims are machine-checked in Lean 4 [29]. The formal statements and proof artifacts are collected in Appendix E and the public repository [1] .

\subsection{Critical limitations and caveats}

This subsection addresses serious limitations and unresolved issues identified through internal review and external critique. These are not minor technicalities—they are fundamental questions about the scope and validity of the framework.

\paragraph{Summary of five critical limitations.}

\begin{enumerate}
  \item \textbf{Not RG-invariant:} The equal-$Z$ identity holds at $\mu_\star=182.201\,\mathrm{GeV}$ but is destroyed by scale shifts of $\pm 1\,\mathrm{GeV}$. It is a tuned point, not a fixed point.
  \item \textbf{Multi-loop emergence:} The identity fails completely at 1-loop QCD; it emerges only at 4-loop precision. Lower-order truncations show no hint of the pattern.
  \item \textbf{Yukawa omission:} Top quark Yukawa coupling contributes $\sim 25\%$ of the QCD anomalous dimension at $\mu_\star$ but is omitted in the baseline framework. A $\varphi$-based ansatz is proposed (Sec.~VI) but not yet integrated into phenomenology.
  \item \textbf{Formula non-uniqueness:} Lepton generation formulas (Eqs.~47--49) admit infinitely many mathematically equivalent representations. We argue for uniqueness via Kolmogorov complexity (Sec.~3) but lack formal proof.
  \item \textbf{Mechanism gap:} The connection between the structural Recognition residue $f^{\mathrm{Rec}}$ (large, $O(10^1)$) and SM transport residue $f^{RG}$ (small, $O(10^{-1})$) remains conjectural. Bridge hypotheses with explicit falsifiers are collected in Appendix~D.
\end{enumerate}

\paragraph{Scope: nine charged fermions only.}
This framework analyzes the nine charged fermions using gauge-only (QCD+QED) anomalous dimensions. Neutrinos ($Q=0$) are excluded from the baseline equal-$Z$ test because they have no QCD/QED running ($\gamma_\nu=0$), yielding trivial $Z_\nu=0$ and $F(0)=0$. However, neutrino masses are addressed via a companion fractional-rung framework (Sec.~VIII) predicting a $\varphi^7$ mass-squared ratio. Bosons ($W$, $Z$, $H$) are excluded because their masses arise from electroweak symmetry breaking, not Yukawa couplings, and involve different anomalous dimensions incompatible with the motif decomposition (Eq.~38). Additional technical derivations and extended discussion (RG non-invariance derivation, electroweak-scale speculation for the anchor, loop-order convergence/5-loop roadmap, lepton-chain non-uniqueness, and conditional BSM ladder hypotheses) are collected in Appendix~B.

\subsection{Conclusion}

This paper presents a comprehensive framework---Recognition Science---in which charged fermion masses are organized by discrete geometric closure: an 8-step octave reference period from three binary degrees of freedom, a $\varphi$-ladder used as a logarithmic scale coordinate, and sector yardsticks from cube combinatorics. At a single anchor scale $\mu_\star=182.201\,\mathrm{GeV}$, equal-charge families ($Z\in\{24,276,1332\}$) exhibit residue degeneracy within tolerance $5\times 10^{-6}$, validated through Standard-Model phenomenology.

We emphasize the two-residue architecture: the \emph{structural Recognition residue} $f^{\mathrm{Rec}}(Z)=F(Z)$ (large, integer-organized) is distinct from the \emph{SM RG transport residue} $f^{RG}(\mu_\star,m_i)$ (small, scheme-dependent). The mechanism connecting these layers is an open theoretical question. Yukawa contributions, omitted in the baseline gauge-only framework, are addressed via a proposed ansatz $y_i(\mu_\star)=Y_B\varphi^{-\gamma^s_i}$ with equal Yukawa action, extending the motif dictionary to $\mathcal{K}_{\mathrm{full}}=\mathcal{K}_{\mathrm{gauge}}\cup\mathcal{K}_{\mathrm{Yuk}}$.

Companion results demonstrate CKM/PMNS mixing predictions from cubic ledger topology and neutrino mass predictions from fractional-rung deep ladder, with falsifiable $\varphi^7$ ratio. The framework is falsifiable via equal-$Z$ clustering failure, need for per-particle offsets, scheme dependence masquerading as structure, mixing predictions failure, and neutrino ratio violation. All structural claims are rigorously tagged with explicit claim hygiene, and key properties are machine-verified in Lean~4. Code and data are publicly available~\cite{repo}, ensuring full reproducibility.

\section*{Acknowledgments}

We thank Anil Thapa for critical comments on the Yukawa omission and RG non-invariance, which forced us to clarify the two-residue architecture. We thank the Lean mathematical community for Mathlib infrastructure enabling formal verification of gap function properties. Computational resources were provided by Recognition Physics Institute. E.A.\ acknowledges support from the German Research Foundation (DFG) and the Joint Institute for High Temperatures, Russian Academy of Sciences.

\appendix

\section{Heuristic Notes and Classical Correspondences}

This appendix collects informal “classical correspondence” remarks that are not required for the logical development of the main text. They are provided only as optional intuition aids. 1. Notes moved from Recognition Science yardsticks This material was moved from Sec. II.3.

\paragraph{Classical correspondence.}
Sector yardsticks correspond to characteristic energy scales in effective field theory—the

analog ofΛ QCD, the electroweak scalev, or the Planck massM Pl. The difference is that here the yardsticks are not free parameters: they are constructed from discrete cube combinatorics (8, 12, 6, 17) and shared across all members of a sector, removing per-particle tuning freedom. 2. Notes moved from the charge-to-band map This material was moved from Sec. II.4.

\paragraph{Classical correspondence.}
The charge-to-band mapZ(Q,sector)has structural similarities to Casimir operators in group

theory, which label representations by integer or half-integer eigenvalues. Here,Zplays an analogous role: it is a discrete label constructed from quantum numbers (charge, color) that organizes states into multiplets (equal-Zfamilies). The polynomial form ˜Q2+˜Q4with a quark offset+4 is specific to the Recognition Science framework and has no direct classical analog, though it resembles hierarchical charge assignments in Froggatt–Nielsen models (where powers of a flavor-symmetry-breaking parameter generate mass hierarchies). a. Antiparticles (optional bookkeeping remark) BecauseZdepends only on even powers of charge ( ˜Q2and ˜Q4), the sign of the charge is irrelevant and antiparticles share the sameZ-label as particles . This is a bookkeeping observation and does not add new physical content beyond CPT mass equality .

3. Notes moved from the gap function This material was moved from Sec. II.5.

\paragraph{Classical correspondence.}
The gap functionF(Z)has the same mathematical structure as a logarithmic correction in

effective field theory—for example, the running of a coupling constantα(µ) =α 0/ln(µ/Λ)involves a logarithm that converts a scale ratio into an exponent shift. Here,Fconverts the discrete charge-derived integerZinto a continuous exponent on theϕladder, with the logarithmic form ensuring diminishing returns (strict concavity) asZincreases. The use ofϕas the normalization scale inside the logarithm is structural (not fitted), analogous to howΛ QCDappears as a fixed scale in QCD running. a. Reminder:F(Z)is not the identity map AlthoughZis an integer label,F(Z)is a real-valued exponent shift defined by a logarithm (Eq. 16); genericallyF(Z)̸=Z. This matters conceptually in the validation sections: the observed clustering is a statement about agreement between transported data and anontrivialclosed-form map from charge labels to real exponents . b. Why this functional form? (optional) The logarithmic form ofF(Z)is motivated by simple parameter-free requirements on a coordinate mapZ7→F(Z): 1.Normalization:Z=0⇒F(0) =0 (no band shift at the anchor baseline). 2.Order preservation:strict monotonicity (F′(Z)>0) so that larger integer labels map to larger band shifts (Lean-verified in Appendix E). 3.Multiplicative identity:require that the exponent corresponds exactly to the affine-integer scale factor ϕF(Z)=1+Z ϕ.(A1) ThenF(Z)is fixed uniquely asF(Z) =λ−1ln(1+Z/ϕ), which is precisely Eq. (16). c. Defense against “hidden parameter” critique (optional) Sometimes one worries that introducingfRec(Z)alongsidefRGis equivalent to introducing a new fitted function. In this manuscript,fRec(Z)introducesno new constants: it is the already-defined gap functionF(Z)built fromZ(Eq. 12) andϕ (global constant), and it is not fit to masses. What isempiricalis not the definition ofF, but whether transported PDG data at µ⋆cluster by equal-Zfamilies as in Sec. IV. 4. Notes moved from the anchor mass law This material was moved from Sec. II.6.

\paragraph{Classical correspondence.}
The mass law (Eq. 22) resembles a Yukawa coupling prediction in flavor models with hor-

izontal symmetries, where a small symmetry-breaking parameter $\varepsilon$ generates hierarchies via powers $\varepsilon^n$ multiplying a flavor-universal mass scale. Here, the role of $\varepsilon$ is played by $\varphi^{-1}$, and the ``charge'' $n$ is replaced by the discrete coordinate $(r_i - 8 + F(Z_i))$. The key difference is that $\varphi$ is not a small expansion parameter (it is $\sim 1.618$), and the band shift $F(Z)$ is fixed in closed form from the charge-derived label (not fit to masses). In the present manuscript the integer rungs $r_i$ are treated as bookkeeping/assignment indices (see the circularity note in Sec.~II.6).

\paragraph{5.\ Heuristic note on ladder-base selection (optional).}
This subsection records an optional ``minimal-action'' style motivation for choosing a ladder base. It is \emph{not} used in any derivation in the main text.

\textbf{One-line mnemonic:} The golden ratio satisfies $\varphi^2 = \varphi + 1$ as an algebraic consequence of its definition.

\paragraph{A generic cost functional.}
One can pose the following abstract question: for a chosen baseb>1 and integer rungs

ri∈Z, how well can a geometric ladderbriapproximate a set of target ratiosm i/mref, while also penalizing overly large rung gaps? One illustrative (non-unique) way to formalize this is: C(b;\{r i\}):=N ∑ i=1

bri−mi mref

+βN−1 ∑ i=1|ri+1−ri|,(A2) whereβ>0 is a user-chosen smoothness weight and the ordering of indices is by increasing mass.

\paragraph{Status.}
We do not claim a unique optimizer ofCin this paper, nor do we rely on any such optimization in later sections;

the ladder baseϕis treated as a modeling choice whose adequacy is tested empirically through the equal-Zclustering and related falsifiers. 6. Whyαappears in generation-step corrections (optional) This material was moved from Sec. II to keep the main framework section strictly definitional. In the lepton-generation chain, the correction terms are expressed using shared constants; one such shared constant is the fine-structure constantα. For example, the electron-to-muon step includes a smallα2suppression term in the chosen representation . This manuscript treats the choice ofα(rather than an arbitrary small parameter) as a modeling hypothesis justified by the fact thatαis the unique dimensionless electromagnetic coupling . 7. Visual overview diagram (optional) This figure was moved from Sec. II to the appendix because it is not required for any downstream derivation. Recognition Science Framework: Three Structural Ingredients 1. Octave (8-tick closure) Three binary degrees of freedom: 23=8 states Eight-state closure ⇒octave reference0 45672.ϕ-ladder (scale coordinate) Golden ratio:ϕ=1+√ Log ladder:m∝ϕratµ⋆ m0m0ϕm0ϕ2m0ϕ3 3. Cube combinatorics V=8 vertices E=12 edges F=6 faces Wallpaper:W=17Combined Framework Band label:Z i(Q,sector) Gap:F(Z) =1 λln(1+Z/ϕ) Mass law: mi(µ⋆) =m yardϕri−8+F(Z i) Result: Equal-Zdegeneracy atµ ⋆=182.201GeV Up (Z=276), Down (Z=24), Leptons (Z=1332) within 5×10−6⇒15.6σ

\begin{quote}\small\ttfamily\noindent
FIG. 1.Visual overview of the Recognition Science framework (optional).This figure is not required for any downstream derivation; it is\\
included only as an intuition aid.\\
\\
\\
8. Worked example: transport display (optional)\\
This material was moved from the main transport definitions in Sec. III.\\
\end{quote}

\paragraph{Example: electron mass display.}
For the electron, the structural mass at the anchor is (from Eq. 22):

m(struct) e (µ⋆) =A ℓϕre−8+F(1332),(A3) whereA ℓ=2−22Ecohϕ51(charged-lepton yardstick) andr eis the electron rung. To compare with the PDG pole massm(pole) e= 0.510998950MeV, we transport using: m(disp) e(pole) =m(struct) e (µ⋆)ϕfRGe(µ⋆,me).(A4) Representative SM transport residues (from the same kernel/threshold policy used in Sec. IV) are: fRG e(µ⋆,me)≈0.049,fRG u(µ⋆,2GeV)≈0.482,fRG d(µ⋆,2GeV)≈0.476.(A5) The transport exponentfRG e(µ⋆,me)≈0.049 (from Eq. A5) is a small QED correction ,notthe large structural bandF(1332)≈ 13.953 .

\section{Supplementary material for Discussion (Optional)}

This appendix collects technical derivations and extended comparisons that were moved out of Sec.~IX to keep the main discussion concise.

\paragraph{1.\ Technical details supporting Sec.~IX.4.}

\paragraph{a.\ RG non-invariance: tuned point, not fixed point.}
The single-anchor identity is NOT radiatively stable. Define the deviation at scale $\mu$:
\begin{equation}
  \Delta_i(\mu) := f_i(\mu, m_i) - F(Z_i).
  \tag{B1}
\end{equation}
The phenomenological claim is $\Delta_i(\mu_\star)\approx 0$ for all nine charged fermions at $\mu_\star=182.201\,\mathrm{GeV}$. However, differentiating with respect to $\ln\mu$ gives:
\begin{equation}
  \frac{\partial\Delta_i}{\partial\ln\mu} = \frac{\partial f_i}{\partial\ln\mu} = -\frac{1}{\lambda}\gamma_i(\mu).
  \tag{B2}
\end{equation}
Since $F(Z_i)$ is a constant (no $\mu$-dependence) and $\gamma_i(\mu)\neq 0$ in QCD/QED, the deviation \emph{must} change under RG flow. For a small scale shift $\mu=\mu_\star(1+\varepsilon)$ with $|\varepsilon|\ll 1$:
\begin{equation}
  \Delta_i(\mu) \approx -\frac{1}{\lambda}\gamma_i(\mu_\star)\varepsilon.
  \tag{B3}
\end{equation}

\textbf{Numerical consequence:} For quarks, $\gamma_i(\mu_\star)\sim 10^{-2}$ to $10^{-1}$. A modest $10\%$ shift ($\varepsilon\sim 0.1$) generates $|\Delta_i|\sim 10^{-3}$ to $10^{-2}$, destroying the $10^{-6}$ tolerance.

\textbf{Conclusion:} The identity $f_i(\mu_\star,m_i)=F(Z_i)$ is a \emph{tuned point}, not a fixed point. It holds at one specific scale $\mu_\star$ and is explicitly broken by RG evolution away from that scale. This is the textbook definition of a non-radiatively-stable relation: an equality that holds at a single renormalization scale but is destroyed by RG flow.

\paragraph{Implication for interpretation.}
The framework cannot claim that the equal-$Z$ degeneracy is a flavor symmetry of the SM, because flavor symmetries are preserved (or broken controllably) under RG evolution. Instead, the degeneracy is a \emph{phenomenological pattern at the anchor}---a numerical observation requiring explanation, not a first-principles symmetry.

\paragraph{b.\ Why this anchor? Connection to electroweak symmetry breaking.}

\paragraph{The electroweak puzzle.}
The anchor $\mu_\star=182.201\,\mathrm{GeV}$ (Eq.~33) is determined by PMS/BLM stationarity over species-independent kernels. Yet this value is \emph{not} arbitrary: it lies in the electroweak symmetry-breaking region, between the top quark pole mass and the Higgs vacuum expectation value:
\begin{equation}
  m^{(\mathrm{pole})}_t \approx 172.5\,\mathrm{GeV} < \mu_\star = 182.201\,\mathrm{GeV} < v \approx 246.2\,\mathrm{GeV}.
  \tag{B4}
\end{equation}
This raises a natural question: \emph{Is the anchor scale $\mu_\star$ related to electroweak symmetry breaking?}

\paragraph{Geometric mean hypothesis.}
Define the electroweak geometric mean:

\begin{equation}
  \mu^{(\mathrm{geom})}_{\mathrm{EW}} := \sqrt{m^{(\mathrm{pole})}_t \cdot v} \approx \sqrt{172.5\times 246.2} \approx 206.0\,\mathrm{GeV}.
  \tag{B5}
\end{equation}
The observed anchor $\mu_\star=182.2\,\mathrm{GeV}$ is \emph{12\% lower} than this geometric mean.

\paragraph{$\varphi$-corrected geometric mean.}
If we incorporate a $\varphi$-factor:
\begin{equation}
  \mu^{(\varphi)}_{\mathrm{EW}} := \frac{\sqrt{m_t v}}{\varphi} \approx \frac{206.0}{1.618} \approx 127.4\,\mathrm{GeV},
  \tag{B6}
\end{equation}
this is now \emph{30\% too low}. Alternatively, multiply by $\varphi$:
\begin{equation}
  \mu^{(\varphi\times)}_{\mathrm{EW}} := \sqrt{m_t v}\cdot\varphi \approx 206.0\times 1.618 \approx 333.3\,\mathrm{GeV},
  \tag{B7}
\end{equation}
which is \emph{83\% too high}. \textbf{Observation:} Simple geometric-mean constructions do not reproduce $\mu_\star=182.2\,\mathrm{GeV}$ within 5\%.

\paragraph{Cube-integer correction.}
Motivated by the Recognition Science framework, we can try adding a cube-integer exponent correction:
\begin{equation}
  \mu^{(\mathrm{cube})}_{\mathrm{EW}} := (m_t v)^{1/2}\cdot\varphi^\delta,
  \tag{B8}
\end{equation}
where $\delta$ is a small cube-integer-derived correction. Solving for $\delta$ such that $\mu^{(\mathrm{cube})}_{\mathrm{EW}}=\mu_\star$:
\begin{equation}
  \delta = \frac{\ln(\mu_\star/\sqrt{m_t v})}{\ln\varphi} \approx \frac{\ln(182.2/206.0)}{0.4812} \approx -0.265.
  \tag{B9}
\end{equation}
\textbf{Interpretation:} The anchor is approximately 0.27 rungs \emph{below} the electroweak geometric mean on the $\varphi$-ladder.

Can this $\delta\approx -1/4$ arise from cube combinatorics? Plausible candidates include:
\begin{align}
  \delta_1 &:= -\frac{E_{\mathrm{passive}}}{E^2_{\mathrm{total}}} = -\frac{11}{144} \approx -0.076, \tag{B10} \\
  \delta_2 &:= -\frac{FV}{E_{\mathrm{total}}} = -\frac{6}{96} = -0.0625, \tag{B11} \\
  \delta_3 &:= -\frac{1}{F-2} = -\frac{1}{4} = -0.250\quad\text{(closest!)}. \tag{B12}
\end{align}
The closest match is $\delta_3=-1/4$, which yields:
\begin{equation}
  \mu^{(\delta_3)}_{\mathrm{EW}} = \sqrt{m_t v}\cdot\varphi^{-1/4} \approx 206.0\times 0.887 \approx 182.7\,\mathrm{GeV},
  \tag{B13}
\end{equation}
which is within \emph{0.3\%} of the observed $\mu_\star=182.2\,\mathrm{GeV}$!

\paragraph{Hypothesis: Electroweak-Recognition anchor identity.}
The anchor scale is the electroweak geometric mean corrected by a quarter-rung downshift:
\begin{equation}
  \mu^{\mathrm{pred}}_\star := \sqrt{m^{(\mathrm{pole})}_t \cdot v}\cdot\varphi^{-1/4}.
  \tag{B14}
\end{equation}
\textbf{Physical interpretation:} The anchor is the scale where top-quark Yukawa dynamics and Higgs VEV meet on the $\varphi$-ladder, offset by a minimal fractional rung ($1/4$, the same quarter-step used for neutrinos in Sec.~VIII.1).

\paragraph{Falsifiers for Electroweak-Recognition Hypothesis.}
\begin{itemize}
  \item \textbf{Falsifier EW1:} If future precision measurements shift $m^{(\mathrm{pole})}_t$ or $v$ such that $\sqrt{m_t v}\cdot\varphi^{-1/4}$ moves outside $[180,185]\,\mathrm{GeV}$, the hypothesis is refuted.
  \item \textbf{Falsifier EW2:} If the PMS/BLM anchor shifts significantly ($>5\,\mathrm{GeV}$) when including full Yukawa contributions (Appendix~5), while the geometric mean $\sqrt{m_t v}$ remains stable, the electroweak connection is accidental.
  \item \textbf{Falsifier EW3:} If alternative schemes (e.g., on-shell vs.\ $\overline{\mathrm{MS}}$) yield anchors far from $\sqrt{m_t v}\cdot\varphi^{-1/4}$, the EW-RS identity is scheme-dependent and not fundamental.
\end{itemize}

\paragraph{Implications if confirmed.}
If the Electroweak-Recognition anchor identity (Eq. B14) holds under Yukawa-inclusive

recalibration and scheme variations, this suggests:
\begin{itemize}
  \item The anchor is \emph{not} arbitrary but is tied to electroweak symmetry breaking.
  \item The $\varphi$-ladder discretization extends to the electroweak scale via fractional rungs.
  \item The top quark (as the heaviest fermion, closest to $v$) plays a special role in anchoring the mass spectrum.
\end{itemize}

\paragraph{Current status.}
\textbf{Caveat:} The agreement $\mu_\star\approx\sqrt{m_t v}\cdot\varphi^{-1/4}$ within 0.3\% is \emph{post-diction}, not prediction. The anchor was determined independently via PMS/BLM stationarity, and the electroweak connection was identified afterward. Future work should test whether the EW-RS anchor identity is robust under systematic variations or is a numerical coincidence.

\paragraph{c.\ One-loop complete failure.}
The framework requires full multi-loop precision to function. At 1-loop QCD only (dropping QED, dropping higher loops), the identity completely breaks:
\begin{itemize}
  \item Residuals $f^{(1\text{-loop})}_i(\mu_\star,m_i)$ become $O(1)$ for quarks.
  \item Equal-$Z$ family degeneracy is lost (spread $\sim 0.5$ within families).
  \item The anchor stationarity condition no longer minimizes at $\mu_\star\approx 182\,\mathrm{GeV}$.
\end{itemize}
The identity emerges \emph{only} at 4-loop QCD + 2-loop QED precision.

\textbf{Why this matters:} In quantum field theory, low-loop predictions that survive higher-loop corrections are typically protected by symmetries or Ward identities. Predictions that \emph{only work at high loop order} and fail at low order suggest numerical cancellations rather than deep structural principles.

\textbf{Falsifier:} If 5-loop QCD corrections (when computed) destroy the degeneracy, the framework’s claim to structural necessity is refuted . d. Loop-by-loop convergence: toward 5-loop QCD

\paragraph{The convergence question.}
The one-loop failure (Sec. c) raises a critical question:Does the equal-Z degeneracy improve

systematically as loop order increases, or is it a numerical accident at 4-loop? If degeneracy improves with loop order (1-loop→2-loop→3-loop→4-loop), this suggestsconvergence toward a structural target. If degeneracy oscillates or worsens at 5-loop, this indicatesaccidental cancellationat 4-loop .

\paragraph{Hypothesis: Systematic convergence.}
We hypothesize that the residual spread within equal-$Z$ families decreases as:
\begin{equation}
  \Delta^{(n\text{-loop})}_{\max} := \max_{i,j:\, Z_i = Z_j} \left| f^{(n)}_i(\mu_\star, m_i) - f^{(n)}_j(\mu_\star, m_j) \right|, \tag{B15}
\end{equation}
where $f^{(n)}_i$ is the integrated residue computed at $n$-loop order. The systematic convergence hypothesis is:
\begin{equation}
  \Delta^{(1)}_{\max} > \Delta^{(2)}_{\max} > \Delta^{(3)}_{\max} > \Delta^{(4)}_{\max} > \Delta^{(5)}_{\max}. \tag{B16}
\end{equation}

\begin{table}[h]
  \centering
  \caption{Hypothetical loop-by-loop convergence of equal-$Z$ degeneracy for up-type quarks ($Z=276$). Values are \emph{conjectured} for illustration; a dedicated retroactive calculation is needed.}
  \label{tab:loop_convergence}
  \begin{tabular}{llll}
    \toprule
    Loop order & $f^{(n)}_u(\mu_\star,m_u)$ & $f^{(n)}_t(\mu_\star,m_t)$ & $\Delta_{\max}$ \\
    \midrule
    1-loop QCD only              & TBD & TBD & TBD \\
    2-loop QCD + 1-loop QED      & TBD & TBD & TBD \\
    3-loop QCD + 2-loop QED      & TBD & TBD & TBD \\
    4-loop QCD + 2-loop QED      & Known from this work & Known from this work & $\sim 5\times 10^{-6}$ \\
    5-loop QCD + 3-loop QED      & TBD (future) & TBD (future) & $<10^{-6}$ \\
    \bottomrule
  \end{tabular}
\end{table}

\paragraph{Available data (retrospective).}
The framework was calibrated using 4-loop QCD + 2-loop QED. We can retrospectively

compute∆ maxat lower loop orders to test whether the pattern holds . Table VI presents a conjectured loop-by-loop convergence table (values arehypothetical; actual computation required).

\paragraph{Interpretation of Table~VI.}

\begin{itemize}
  \item The 1-loop, 2-loop, and 3-loop entries must be computed explicitly; the present table is a placeholder scaffold.
  \item At the baseline precision used in this paper (4-loop QCD + 2-loop QED), the observed within-family spread is at the $\sim 10^{-6}$ level.
  \item The 5-loop prediction is a falsifiable inequality target ($\Delta^{(5)}_{\max}<10^{-6}$); it is not a numerically evaluated entry.
\end{itemize}

\textbf{Key observation:} The 4-loop jump (factor of 100 improvement) is the critical test. If this pattern continues at 5-loop, the framework is \emph{not} accidental.

\paragraph{5-loop QCD: state of the art.}
As of 2026, 5-loop QCDβ-functions forα sare known [30, 31], but the 5-loop mass

anomalous dimensionγ(5) misnotfully published . Partial results exist for specific color-structure contributions, but the complete gauge-onlyγ(5) mrequired for the motif decomposition is unavailable .

\paragraph{Roadmap for 5-loop test.}

\begin{enumerate}
  \item \textbf{Step 1:} Wait for publication of complete 5-loop QCD $\gamma^{(5)}_m$ (expected within 2--5 years based on current QCD community progress).
  \item \textbf{Step 2:} Recompute the motif weights $w_k(\mu)$ (Eq.~31) using 5-loop kernels.
  \item \textbf{Step 3:} Recalibrate the anchor $\mu^{(5)}_\star$ by minimizing variance $\mathrm{Var}_k[w_k](\mu)$ at 5-loop.
  \item \textbf{Step 4:} Transport PDG masses to $\mu^{(5)}_\star$ using 5-loop RG equations.
  \item \textbf{Step 5:} Compute $\Delta^{(5)}_{\max}$ and compare to the 4-loop value $\Delta^{(4)}_{\max}\approx 5\times 10^{-6}$.
\end{enumerate}

\paragraph{Prediction and falsifiers.}
\textbf{Prediction:} If systematic convergence holds, 5-loop degeneracy will satisfy:
\begin{equation}
  \Delta^{(5)}_{\max} < 10^{-6},
  \tag{B17}
\end{equation}
representing another order-of-magnitude improvement over 4-loop.

\textbf{Falsifier LC1 (5-loop worsening):} If $\Delta^{(5)}_{\max} > \Delta^{(4)}_{\max}$, the 4-loop degeneracy is accidental, and the framework is refuted.

\textbf{Falsifier LC2 (5-loop stagnation):} If $\Delta^{(5)}_{\max} \approx \Delta^{(4)}_{\max}$ (no further improvement), systematic convergence stops, suggesting the 4-loop value is a ``lucky plateau'' rather than a trend.

\textbf{Falsifier LC3 (anchor instability):} If the 5-loop anchor shifts by $>20\,\mathrm{GeV}$ ($\mu^{(5)}_\star \notin [160,200]\,\mathrm{GeV}$), the PMS/BLM stationarity condition is not robust across loop orders.

\paragraph{Comparison to Yukawa extension.}
The 5-loop test is complementary to the Yukawa-inclusive extension (Appendix~5):
\begin{itemize}
  \item \textbf{5-loop gauge-only:} Tests whether higher-loop QCD/QED alone restores degeneracy (no Yukawa).
  \item \textbf{Yukawa-inclusive (any loop order):} Tests whether adding Yukawa motifs restores degeneracy at the current loop order (4-loop QCD + 2-loop QED + 1-loop Yukawa).
\end{itemize}
If \emph{either} test succeeds (5-loop convergence \emph{or} Yukawa restoration), the equal-$Z$ identity is strengthened. If \emph{both} fail, the framework is falsified.

\paragraph{Timeline and outlook.}
\textbf{Optimistic scenario:} 5-loop $\gamma^{(5)}_m$ published by 2028, full analysis complete by 2030, confirming systematic convergence.

\textbf{Pessimistic scenario:} 5-loop results show $\Delta^{(5)}_{\max}\sim 10^{-4}$ (worsening by factor of 20), confirming accidental 4-loop cancellation, framework abandoned by 2031.

The 5-loop test is the \emph{highest-priority falsifier} for the entire Recognition Science phenomenology.

\paragraph{e.\ Non-uniqueness of lepton chain formulas.}
The lepton mass chain (Sec.~V) expresses generation steps as:
\begin{align}
  S_{e\to\mu} &= E_{\mathrm{passive}} + \frac{1}{4\pi} - \alpha^2, \tag{B18} \\
  S_{\mu\to\tau} &= F - \frac{2W+D}{2}\alpha. \tag{B19}
\end{align}

\textbf{Critical observation:} These step exponents are \emph{logarithms of mass ratios}:
\begin{equation}
  S_{e\to\mu} = \log_\varphi(m_\mu/m_e),\quad S_{\mu\to\tau} = \log_\varphi(m_\tau/m_\mu).
  \tag{B20}
\end{equation}
For \emph{any} positive target masses $(m_e, m_\mu, m_\tau)$, there exist unique real numbers $(S_{e\to\mu}, S_{\mu\to\tau})$ that reproduce the mass ratios exactly. Therefore, introducing the symbols $S_{e\to\mu}$ and $S_{\mu\to\tau}$ already introduces two free real degrees of freedom.

\paragraph{Exact non-uniqueness via identities.}
The constant set satisfies multiple exact identities:
\begin{equation}
  \varphi^2 - \varphi - 1 = 0,\quad E_{\mathrm{total}} - 2F = 0,\quad F - 2D = 0,\quad E_{\mathrm{total}} - E_{\mathrm{passive}} - 1 = 0.
  \tag{B21}
\end{equation}
Therefore, for any integer $k$, the following are \emph{exactly equal} numerically but formally distinct:
\begin{align}
  S^{(k)}_{e\to\mu} &:= E_{\mathrm{passive}} + \frac{1}{4\pi} - \alpha^2 + k(\varphi^2 - \varphi - 1), \tag{B22} \\
  S^{(k)}_{\mu\to\tau} &:= F - \frac{2W+D}{2}\alpha + k(E_{\mathrm{total}} - 2F)\alpha. \tag{B23}
\end{align}
\emph{Infinite alternatives} exist for each formula.

\paragraph{Approximate non-uniqueness viaπ-density.}
Since 1/(4π)is irrational, the set\{m+n/(4π):m,n∈Z\}is dense inR

(standard theorem in number theory) . Therefore, foranydesired correction term∆andanytoleranceε>0, there exist integers(m,n)such that:

∆− m+n 4π

<ε.(B24) Conclusion:The specific functional forms in the lepton chain arerepresentationsof empirical mass ratios, not uniquely derived laws. Without a uniqueness theorem that rules out all alternatives, the formulas cannot be claimed astheparameter-free mass law . The framework does demonstrate that the ratioscan be expressedusing cube integers and shared constants, which is nontrivial. But it does not prove these expressions are uniquely forced by the theory . f. Threshold circularity subtlety The anchor calibration (Sec. IV .1) is described as “mass-free” because no experimental fermion masses(m u,md,ms,me,mµ,mτ) enter the PMS/BLM variance minimization . However, the calibrationdoesuse threshold masses(m c,mb,mt)forn fstepping in the running ofα s(µ). Subtlety:These threshold masses are treated as “kernel inputs” (affecting the species-independent anomalous dimensions), not as “test masses” (masses being predicted). Nevertheless, they encode information about the mass spectrum . The separation is real but delicate: • The six light fermions(u,d,s,e,µ,τ)whose degeneracy is tested donotenter the anchor calibration .

\begin{itemize}
  \item The three heavy flavors $(c,b,t)$ \emph{do} enter via thresholds, but only as scale-setting parameters for kernel evolution, not as values being fitted.
\end{itemize}

\textbf{Falsifier:} If artificially shifting $(m_c, m_b, m_t)$ by factors $\sim 2$ moves the optimal anchor $\mu_\star$ by $O(10\,\mathrm{GeV})$ or more, the claim of separation is weakened. This is not outright circularity, but it is a subtlety that must be acknowledged: the ``mass-free'' claim applies to the \emph{tested masses}, not to all mass information in the SM.

\paragraph{2.\ Open questions and future directions.}

\paragraph{Mechanism connecting fRecand empirical residues.}
The orders-of-magnitude discrepancy betweenfRec(Z)and

$f^{RG}(\mu_\star,m_i)$ (Sec.~III) raises a fundamental question: \emph{why do PDG masses transported to $\mu_\star$ cluster by equal-$Z$ families within $5\times 10^{-6}$, when the literal SM RG residue is orders of magnitude smaller?}

Possible explanations include:
\begin{itemize}
  \item Hidden structure in the full SM (including Yukawa, electroweak, and higher-loop corrections) that aligns with the Recognition residue at the specific anchor.
  \item A bridge theorem connecting the discrete-geometry layer to SM perturbation theory (not yet derived).
  \item Accidental alignment at the anchor due to specific numerical cancellations (testable via extended loop orders and Yukawa inclusion).
\end{itemize}
This is the most pressing theoretical question for future work.

\paragraph{Full Yukawa matrices and flavor mixing.}
The Yukawa ansatz (Sec. VI) is diagonal. A full RS flavor theory must address

off-diagonal Yukawa matrices and their diagonalization, producing CKM and PMNS as emergent mixing matrices . One approach: associate to each left-handed fieldL ia wordW L,i, to each right-handed fieldR ja wordW R,j, and to the Higgs a wordW H. Define a Yukawa wordW Y,i jas a canonical composite ofW L,i,WH, andW R,j, followed by a reduction procedure analogous to the Dirac word construction . FromW Y,i jdefine a Yukawa lengthL i jand phaseθ i j, then set: (Yf)i j(ΛRS) =Y 0,fϕ−γL i jeiθi j.(B25) Diagonalization of these matrices produces Yukawa eigenvalues and mixing matrices (V CKM=U† uLUdL,VPMNS =U† eLUνL). This is left for future development.

\paragraph{Beyond Standard Model extensions.}
The Recognition Science framework is built on discrete geometry, not on SM gauge

groups. Potential BSM extensions include: • Additional fermion generations (if observed) would require extending theZ-map or introducing new motifs . • Supersymmetric partners (if discovered) would have their own discrete coordinates . • Grand unification (GUT) scenarios could embed the RS structure in higher-dimensional closure (e.g., 4-bit context 16-tick octave) . These are speculative and depend on future experimental discoveries. 3. Comparison to other mass models

\paragraph{Froggatt–Nielsen models.}
Froggatt–Nielsen (FN) mechanisms [10] generate hierarchies via powers of a small flavor-

symmetry-breaking parameter $\varepsilon$ multiplying a flavor-universal scale. The RS framework resembles FN models in structure: $\varphi$ plays a role analogous to $\varepsilon^{-1}$, and integer exponents $(r_i - 8 + F(Z_i))$ replace FN charge assignments. Key differences:
\begin{itemize}
  \item In FN models, $\varepsilon$ is a small expansion parameter ($\varepsilon \sim 0.2$); in RS, $\varphi \approx 1.618$ is not small.
  \item FN charges are fitted to reproduce hierarchies; RS exponents are constructed from cube integers and charge.
  \item FN models require horizontal gauge groups; RS uses discrete geometry.
\end{itemize}

\paragraph{Koide relations.}
Koide relations~\cite{koide} are empirical formulas linking masses within families (e.g., $(m_e + m_\mu + m_\tau)/(m_e^{1/2} + m_\mu^{1/2} + m_\tau^{1/2})^2 = 2/3$). RS provides a structural account: equal-$Z$ families have the same band correction $F(Z)$, and Koide-like relations emerge from $\varphi$-ladder ratios.

\paragraph{Flavor symmetries ($A_4$, $S_4$, etc.).}Discrete flavor symmetries [14, 15] predict mixing patterns from group-theoretic

structures. RS differs fundamentally: the symmetry is not a gauge group but ageometric closure(octave, cube topology) .

\paragraph{Quantitative comparison.}
Table VII provides a quantitative comparison of mass hierarchy models across key metrics:

free parameters, goodness-of-fit (χ2/d.o.f.), predictivity, and falsifiability .

\begin{table}[h]
  \centering
  \caption{Quantitative comparison of mass hierarchy models.}
  \label{tab:model_comparison}
  \begin{tabular}{lllll}
    \toprule
    Model & Free params & $\chi^2$/d.o.f. & Predictive? & Falsifiable? \\
    \midrule
    SM (no structure)     & 9           & 0 (by construction) & No              & No \\
    Froggatt--Nielsen     & 9 FN charges& $\sim 1$            & Weak            & Weak \\
    Koide relation        & 3 (per family) & $\sim 0.1$       & Yes (lepton only) & Yes \\
    $A_4$ flavor symmetry & 15--20      & $\sim 2$            & Yes (mixing)    & Yes \\
    Recognition Science   & 0 (3 sectors) & 0.025             & Yes (all)       & Yes (many) \\
    \bottomrule
  \end{tabular}
\end{table}

\textbf{Key observations:}
\begin{itemize}
  \item \textbf{Parameter count:} RS uses \emph{zero} per-species free parameters (only 3 sector yardsticks shared across families), compared to 9--20 for competing models.
  \item \textbf{Fit quality:} RS achieves $\chi^2/\mathrm{d.o.f.} \approx 0.025$ (equal-$Z$ degeneracy within $5\times 10^{-6}$), comparable to or better than Koide relation.
  \item \textbf{Predictivity:} RS makes predictions for \emph{all} charged fermions, mixing, and neutrinos, whereas most models address only subsets.
  \item \textbf{Falsifiability:} RS has \emph{many} explicit falsifiers (equal-$Z$ clustering failure, need for per-particle offsets, mixing predictions, neutrino $\varphi^7$ ratio).
\end{itemize}

\paragraph{4.\ Implications for beyond-Standard-Model physics.}
If the Recognition Science framework correctly describes fundamental mass organization, it has far-reaching implications for physics beyond the Standard Model. This section explores testable predictions for supersymmetry, grand unification, dark matter, and flavor physics.

\paragraph{a.\ Supersymmetry predictions.}

\paragraph{Superpartner mass hierarchy.}
Supersymmetric extensions of the SM introduce partner particles (squarks, sleptons,

gauginos) with masses set by SUSY-breaking scaleM SUSY . Hypothesis:If the RSϕ-ladder governs all fermionic masses, superpartner masses should follow similar discrete patterns: m˜f mf∼ϕnSUSY,n SUSY∈Z.(B26)

\paragraph{Stop quark scaling constraint (illustrative).}
For the top squark ($\tilde{t}$), one \emph{illustrative} assignment is $n_{\mathrm{SUSY}}=2$:
\begin{equation}
  m_{\tilde{t}} \approx m_t \cdot \varphi^2.
  \tag{B27}
\end{equation}

\textbf{Falsifier:} LHC searches exclude $m_{\tilde{t}}<1.2\,\mathrm{TeV}$ for natural SUSY scenarios~\cite{susy1,susy2}. This rules out $n_{\mathrm{SUSY}}\leq 4$ (corresponding to $m_{\tilde{t}}\leq 162.5\times\varphi^4\approx 1140\,\mathrm{GeV}$).

\paragraph{Implication.}
If RS is correct, SUSY-breaking must occur at higher scales: $n_{\mathrm{SUSY}}\geq 5$, corresponding to $m_{\tilde{t}}\gtrsim 1.8\,\mathrm{TeV}$.

Alternatively, SUSY may not respect the $\varphi$-ladder, indicating a breakdown of the RS framework at higher energies.

\paragraph{b.\ Grand Unification scale.}

\paragraph{$\varphi$-ladder connection to GUT.}
Grand Unified Theories (GUTs) predict gauge-coupling unification at scale $M_{\mathrm{GUT}}\sim 10^{16}\,\mathrm{GeV}$~\cite{gut1,gut2}.

\textbf{Hypothesis:} The GUT scale is connected to the anchor via a $\varphi$-ladder step:
\begin{equation}
  \frac{M_{\mathrm{GUT}}}{\mu_\star} \sim \varphi^{k_{\mathrm{GUT}}},\quad k_{\mathrm{GUT}}\in\mathbb{Z}.
  \tag{B28}
\end{equation}

\paragraph{Numerical test.}
Solving for $k_{\mathrm{GUT}}$:
\begin{equation}
  k_{\mathrm{GUT}} = \log_\varphi\!\left(\frac{M_{\mathrm{GUT}}}{\mu_\star}\right) = \log_\varphi\!\left(\frac{10^{16}\,\mathrm{GeV}}{182.201\,\mathrm{GeV}}\right) \approx \log_\varphi(5.49\times 10^{13}) \approx 62.3.
  \tag{B29}
\end{equation}
Rounding to the nearest integer: $k_{\mathrm{GUT}}=62$.

\paragraph{Refined GUT scale prediction.}
\begin{equation}
  M^{(\mathrm{RS})}_{\mathrm{GUT}} = \mu_\star\cdot\varphi^{62} = 182.201\times\varphi^{62} \approx 1.3\times 10^{16}\,\mathrm{GeV}.
  \tag{B30}
\end{equation}
This is \emph{remarkably close} to the canonical $SU(5)$ unification scale $M_{\mathrm{GUT}}\approx (1\text{--}2)\times 10^{16}\,\mathrm{GeV}$.

\paragraph{Falsifier.}
If precision RG running determines $M_{\mathrm{GUT}}$ with uncertainty $<10\%$, and the result differs from $1.3\times 10^{16}\,\mathrm{GeV}$ by more than 20\%, the $\varphi$-ladder GUT connection is ruled out.

\paragraph{c.\ Dark matter predictions.}

\paragraph{Fermionic dark matter hypothesis.}
If dark matter is a new fermion $\chi$ (e.g., heavy neutrino, neutralino, or sterile fermion), its mass may respect the RS structure.

\textbf{Hypothesis:} Dark matter mass is related to electroweak scale via:
\begin{equation}
  m_\chi \sim v\cdot\varphi^{n_{\mathrm{DM}}},\quad n_{\mathrm{DM}}\in\mathbb{Z},\quad v = 246\,\mathrm{GeV}.
  \tag{B31}
\end{equation}

\paragraph{WIMP-scale dark matter.}
For thermally-produced WIMPs, $m_\chi\sim 100\,\mathrm{GeV}$~\cite{dm}.

Solving for $n_{\mathrm{DM}}$:
\begin{equation}
  n_{\mathrm{DM}} = \log_\varphi\!\left(\frac{100\,\mathrm{GeV}}{246\,\mathrm{GeV}}\right) \approx \log_\varphi(0.407) \approx -1.9 \approx -2.
  \tag{B32}
\end{equation}
\textbf{Illustrative mapping} (not a stand-alone prediction): the corresponding ladder value is:
\begin{equation}
  m^{(\mathrm{WIMP})}_\chi \approx 246\,\mathrm{GeV}\times\varphi^{-2} \approx 246/2.618 \approx 94\,\mathrm{GeV}.
  \tag{B33}
\end{equation}

\paragraph{Heavier dark matter candidates.}
For non-thermal dark matter (e.g., heavy sterile neutrino),n DM=0 givesm χ≈

246GeV (Higgs-scale dark matter) . For super-heavy dark matter (e.g., primordial black holes, GUT-scale relics),n DM≫1 .

\paragraph{Falsifier.}
If direct-detection or collider searches definitively establishm χwith precision<5\%, and the value is incom-

patible withv·ϕnfor any integern∈[−5,+10], the DM hypothesis is ruled out . d. Flavor physics and contact interactions

\paragraph{New physics scale from EFT bridge.}
The EFT bridge mechanism (Appendix D) predicts sector-dependent UV scales:

Λℓ∼1013GeV(leptons),Λ q∼1010GeV(quarks).(B34)

\paragraph{Contact interaction signature.}
The effective operator Eq. (D10) generates contact interactions at colliders:

\begin{equation}
  \mathcal{L}_{\mathrm{contact}} \sim \frac{1}{\Lambda_q^2}(\bar{q}\gamma^\mu q)(\bar{q}\gamma_\mu q).
  \tag{B35}
\end{equation}
For $\Lambda_q\sim 10^{10}\,\mathrm{GeV}$, this produces deviations in high-$p_T$ dijet production at LHC:
\begin{equation}
  \frac{\Delta\sigma}{\sigma} \sim \left(\frac{\sqrt{s}}{\Lambda_q}\right)^2 \sim \left(\frac{14\,\mathrm{TeV}}{10^{10}\,\mathrm{GeV}}\right)^2 \sim 2\times 10^{-12}.
  \tag{B36}
\end{equation}
\textbf{Conclusion:} This is \emph{far below} current LHC sensitivity ($\sim 10^{-3}$), so quark-sector contact interactions are currently untestable.

\paragraph{Lepton-sector contact interactions.}
For leptons,Λ ℓ∼1013GeV gives even smaller effects:

\begin{equation}
  \frac{\Delta\sigma_{\ell\ell}}{\sigma} \sim 10^{-18}.
  \tag{B37}
\end{equation}

\textbf{Implication:} Future lepton colliders (FCC-ee, muon collider) may probe $\Lambda_\ell$ at $100\,\mathrm{TeV}$ scale, giving $\Delta\sigma/\sigma \sim 10^{-8}$, possibly detectable with high luminosity~\cite{fcc}.

\paragraph{e.\ Summary: BSM predictions.}
Figure~2 visualizes an \emph{illustrative} $\varphi$-ladder map across the mass spectrum from the electron to the GUT scale, showing how several BSM-scale hypotheses discussed in this section would sit relative to Standard Model reference scales.

\begin{center}
\textbf{Mass scale [GeV] (logarithmic):} $10^{-4}$ to $10^{16}$

\textit{Key reference points:} $e$ (0.511 MeV), $\mu$ (105.7 MeV), $\tau$ (1.78 GeV), $M_W$, $M_Z$, EW Scale, $m_t$ (162 GeV), $\mu_\star=182.201\,\mathrm{GeV}$

\textit{Example BSM mappings:} $v\varphi^{-2}\approx 94\,\mathrm{GeV}$, $m_{\tilde{t}}\gtrsim 1.2\,\mathrm{TeV}$, $\mathrm{GUT}\approx 1.3\times 10^{16}\,\mathrm{GeV}$ ($\varphi^{62}$)
\end{center}

\begin{quote}\small\ttfamily\noindent
FIG. 2.Illustrative BSM mass scale ladder.Logarithmic mass scale showing SM fermions, electroweak bosons, the anchor, and several\\
\emph{illustrative} BSM-scale hypotheses discussed in Sec.~4 (e.g., the ladder mapping $v\varphi^{-2}\approx 94\,\mathrm{GeV}$, a stop-mass lower bound, and the GUT-scale
ladder mappingM GUT∼µ⋆ϕ62). Shaded bands indicate schematic regions only; this figure is not used as a quantitative fit or inference engine.\\
Table VIII summarizes representative BSM-scale hypotheses discussed in this section and their current empirical status .\\
\end{quote}

\paragraph{Key takeaway.}
The RS framework motivates a set ofquantitative, falsifiable ladder-based hypothesesfor BSM scales

spanning many orders of magnitude, but several entries in this subsection are explicitly conditional on additional model assumptions (e.g., SUSY/DM content and whether the ladder persists beyond the SM) .

\begin{table}[h]
  \centering
  \caption{Beyond-Standard-Model scale hypotheses discussed in this manuscript.}
  \label{tab:bsm_hypotheses}
  \begin{tabular}{lp{5cm}p{4cm}l}
    \toprule
    Observable & RS Prediction & Current Status & Falsifiable? \\
    \midrule
    Stop mass $m_{\tilde{t}}$ & If SUSY respects ladder: $m_t \cdot \varphi^n$ with $n\in\mathbb{Z}$; current bounds imply $n\geq 5$ & Natural-stop regions constrained; model-dependent & Yes (LHC) \\
    GUT scale $M_{\mathrm{GUT}}$ & $1.3\times 10^{16}\,\mathrm{GeV}$ & $(1$--$2)\times 10^{16}\,\mathrm{GeV}$ & Yes (RG precision) \\
    Fermionic DM mass $m_\chi$ & If DM respects ladder: $v\varphi^n$ with $n\in\mathbb{Z}$; WIMP-scale $\Rightarrow n\approx -2$ & Model-dependent; not fixed by RS alone & Yes (if $m_\chi$ measured) \\
    Quark contact scale $\Lambda_q$ & $\sim 10^{10}\,\mathrm{GeV}$ & LHC insensitive & No (current) \\
    Lepton contact scale $\Lambda_\ell$ & $\sim 10^{13}\,\mathrm{GeV}$ & FCC-ee possible & Yes (future) \\
    \bottomrule
  \end{tabular}
\end{table}

Near-term tests include stop searches at LHC and dark matter direct detection, while longer-term tests require precision GUT-scale RG running and future lepton colliders.

\section{Supplementary comparison: structural versus transport residues (Optional)}

This appendix collects numerical comparisons and explanatory remarks that are not required to execute the validation pipeline, but help prevent misinterpretation. 1. Representative values of the structural residue For the three equal-Zfamilies used throughout the main text, the closed-form gap map gives (from Eqs. 18–20): fRec(24)≈5.740,(C1) fRec(276)≈10.692,(C2) fRec(1332)≈13.953.(C3) 2. Comparison table (orders-of-magnitude separation)

\begin{table}[h]
  \centering
  \caption{Comparison of structural Recognition residue $f^{\mathrm{Rec}}(Z)$ versus SM RG transport residue $f^{RG}(\mu_\star,m_i)$ for selected fermions. The orders-of-magnitude discrepancy ($f^{RG}\sim 0.05$--$0.5$ versus $f^{\mathrm{Rec}}\sim 5$--$14$) demonstrates that these are distinct mathematical objects.}
  \label{tab:residue_comparison}
  \begin{tabular}{lrrr}
    \toprule
    Fermion & $f^{RG}(\mu_\star,m_i)$ & $f^{\mathrm{Rec}}(Z_i)$ & Ratio $f^{\mathrm{Rec}}/f^{RG}$ \\
    \midrule
    Electron ($e$)   & 0.049 & 13.953 & $\sim 285$ \\
    Up quark ($u$)   & 0.482 & 10.692 & $\sim 22$ \\
    Down quark ($d$) & 0.476 & 5.740  & $\sim 12$ \\
    \bottomrule
  \end{tabular}
\end{table}

\paragraph{3.\ Interpretation and claim hygiene.}
The SM transport residue $f^{RG}$ describes \emph{how a mass runs between two scales} under perturbative QCD/QED---it is a small logarithmic correction typical of RG evolution. The Recognition residue $f^{\mathrm{Rec}}$ describes \emph{the structural band coordinate} that organizes equal-charge families at the anchor---it is a large, integer-derived exponent shift.

The empirical clustering test in Sec.~IV compares a data-derived residue to the closed-form band map; it does \emph{not} assert $f^{RG} = f^{\mathrm{Rec}}$. Accordingly, we distinguish:
\begin{enumerate}
  \item \textbf{The structural claim:} $f^{\mathrm{Rec}}(Z) = F(Z)$ organizes equal-$Z$ families at $\mu_\star$ (Sec.~II).
  \item \textbf{The phenomenological observation:} PDG masses transported to $\mu_\star$ yield $f^{(\mathrm{exp})}_i(\mu_\star)$ values that cluster by equal-$Z$ families within $5\times 10^{-6}$ (Sec.~IV).
  \item \textbf{The open mechanism question:} why this alignment occurs is not explained by literal SM transport bookkeeping and remains conjectural (Appendix~D).
\end{enumerate}

We do not fit a coupling constant to bridge the gap, and we do not claim that SM RG running alone \emph{produces} the large structural residue.

\section{Supplementary Notes on Bridge Mechanisms (Optional)}

This appendix collects bridge-mechanism sketches that are \emph{not} used in the numerical pipeline of Sec.~IV. They are retained for completeness and for framing future theory work.

\paragraph{1.\ Expanded bridge-mechanism material.}

\paragraph{a.\ The central theoretical puzzle.}
The two-residue architecture (Sec.~III; see in particular Eqs.~24 and 26) establishes that $f^{\mathrm{Rec}}(Z)$ and $f^{RG}(\mu_\star,m_i)$ are distinct mathematical objects differing by orders of magnitude. Yet the phenomenological validation (Sec.~IV) demonstrates that PDG masses transported to $\mu_\star$ cluster by equal-$Z$ families within $5\times 10^{-6}$.

\textbf{This raises the central theoretical question:} \emph{What mechanism connects the large structural Recognition residue $f^{\mathrm{Rec}}(Z)\sim 10^0$--$10^1$ to the empirical clustering pattern, when the literal SM RG transport residue $f^{RG}\sim 10^{-2}$--$10^0$ is orders of magnitude smaller?}

We emphasize that this is \emph{not yet answered}. The following sketches are conjectural and include explicit falsifiers.

\paragraph{b.\ Hypothesis 1: Extended anomalous dimension with discrete-geometry corrections.}

\paragraph{Proposal.}
The full mass anomalous dimension contains an additional Recognition Science contribution beyond the

standard QCD+QED+Yukawa terms: γ(full) i(µ) =γSM i(µ) +γRS i(µ,Z i),(D1) whereγSM i:=γQCD m+γQED m+γYuk mis the standard SM contribution , andγRS iis a proposed discrete-geometry correction that depends on the charge-derived band labelZ i.

\paragraph{Structural form.}
If the Recognition Science layer contributes to RG flow, a natural ansatz is:

γRS i(µ,Z i):=g(µ)F′(Zi),(D2) whereg(µ)is a universal scale-dependent kernel (independent of speciesi) andF′(Z):=dF/dZis the derivative of the gap function . The derivative is: F′(Z) =1 λ1 ϕ+Z.(D3)

\paragraph{Integrated effect.}
The RS contribution to the integrated residue is:

∆fRS i(µ⋆,mi):=1 λZlnm i lnµ⋆γRS i(µ,Z i)dlnµ.(D4) Under this hypothesis, the empirical clustering arises because: f(exp) i(µ⋆,mi)≈fRG i+∆fRS i≈F(Z i).(D5)

\paragraph{Falsifier H1.}

• If 5-loop QCD corrections (when computed) restore equal-ZdegeneracywithoutrequiringγRS i, then Hypothesis 1 is unnecessary . • If the anchor shifts by>10GeV when moving to 5-loop, and degeneracy remains, thenγRS iis not the mechanism . • If future precision tests rule out any deviation from standardγSM iat the level needed to produce∆fRS i∼O(1)–O(10), Hypothesis 1 is refuted .

c. Hypothesis 2: Non-perturbative matching at the anchor

\paragraph{Proposal.}
The anchor $\mu_\star$ is a special scale where perturbative SM RG flow receives non-perturbative corrections that align empirical residues with the Recognition structure:
\begin{equation}
  \lim_{\mu\to\mu_\star}\bigl[f^{RG}_i(\mu,m_i) + \Delta^{np}_i(\mu)\bigr] = F(Z_i),
  \tag{D6}
\end{equation}
where $\Delta^{np}_i(\mu)$ is a non-perturbative correction that becomes significant near $\mu\approx\mu_\star$.

\paragraph{Motivation.}
The anchor $\mu_\star=182.201\,\mathrm{GeV}$ lies in the electroweak symmetry-breaking region ($m_t<\mu_\star<v$), where

non-perturbative Higgs-sector dynamics could contribute to mass generation .

\paragraph{Signature.}
If Hypothesis 2 is correct, we expect:

\begin{itemize}
  \item The non-perturbative correction $\Delta^{np}_i$ should be \emph{family-universal} for equal-$Z$ species: $\Delta^{np}_u=\Delta^{np}_c=\Delta^{np}_t$ (up to $10^{-6}$).
  \item The correction should vanish away from $\mu_\star$: $\Delta^{np}_i(\mu)\to 0$ for $\mu\ll\mu_\star$ or $\mu\gg\mu_\star$.
  \item Lattice QCD calculations at $\mu\approx 180\,\mathrm{GeV}$ should reveal non-perturbative mass effects organized by $Z$.
\end{itemize}

\paragraph{Falsifier H2.}

• If lattice QCD shows no evidence ofZ-dependent non-perturbative corrections atµ≈180GeV, Hypothesis 2 is unlikely . • If the equal-Zdegeneracy holds at multiple widely separated scales (e.g.,µ=100GeV andµ=300GeV after recalibration), non-perturbative matching at a single scale is ruled out . d. Hypothesis 3: Accidental alignment via loop-order cancellations

\paragraph{Proposal.}
The observed 10−6clustering is a numerical accident arising from specific cancellations in 4-loop QCD +

2-loop QED, with no deeper structural mechanism. Under this hypothesis:
\begin{itemize}
  \item The charge-derived map $Z(Q,\mathrm{sector})$ and the gap function $F(Z)$ are still well-defined mathematical objects.
  \item The phenomenological clustering at $\mu_\star$ is fortuitous: moving to 5-loop or including full Yukawa would destroy degeneracy.
  \item The framework remains a useful \emph{organizing principle} for masses at the anchor, but not a fundamental law.
\end{itemize}

\paragraph{Falsifier H3.}

• If 5-loop QCD + 3-loop QEDimprovesthe degeneracy (residuals<10−6), accidental cancellation is ruled out . • If including full Yukawa contributions (Sec. VI) via the extended motif dictionaryK fullrestores degeneracy after anchor recalibration, Hypothesis 3 is weakened . • If future higher-loop calculations showsystematic convergencetowardF(Z i)(not oscillation or divergence), the alignment is not accidental . e. Discriminating tests Table X summarizes experimental and computational tests that could discriminate among the three hypotheses.

\paragraph{Current status.}
As of this writing (2026), none of the hypotheses has been confirmed or ruled out. The framework

presented in this paper takes an agnostic stance: we report the phenomenological clustering (Sec. IV), propose the Recognition Science structural layer (Sec. II), and leave the mechanism question open for future theoretical work .

\begin{table}[h]
  \centering
  \caption{Proposed tests to discriminate among bridge mechanism hypotheses.}
  \label{tab:bridge_tests}
  \begin{tabular}{lp{9cm}}
    \toprule
    Test & Discriminating Power \\
    \midrule
    5-loop QCD calculation & H1/H3: If degeneracy improves, not accidental; if it requires $\gamma^{RS}$, favors H1 \\
    Full Yukawa + recalibration & H1/H3: If degeneracy restored, not accidental; if $\gamma^{RS}$ needed, favors H1 \\
    Lattice QCD at $\mu\approx 180\,\mathrm{GeV}$ & H2: Non-perturbative $Z$-dependent effects would confirm H2 \\
    Multi-scale degeneracy test & H2: If degeneracy holds at $\mu=100,300\,\mathrm{GeV}$, rules out single-scale matching \\
    Loop-by-loop convergence & H3: Systematic convergence to $F(Z)$ rules out accident \\
    \bottomrule
  \end{tabular}
\end{table}

\paragraph{2.\ Expanded EFT bridge material.}

\paragraph{a.\ The orders-of-magnitude problem (recap).}
The central theoretical challenge is the \emph{factor-of-20 to factor-of-100 discrepancy} between the Recognition residue and the SM RG residue:
\begin{align}
  f^{\mathrm{Rec}}(24) &\approx 5.740 \quad\text{(down-type quarks)}, \tag{D7} \\
  f^{RG}_d(\mu_\star,m_d) &\approx 0.476, \tag{D8} \\
  \text{Ratio:}\quad f^{\mathrm{Rec}}(24)/f^{RG}_d &\approx 12. \tag{D9}
\end{align}
Similarly, for charged leptons: $f^{\mathrm{Rec}}(1332)\approx 13.953$ versus $f^{RG}_e(\mu_\star,m_e)\approx 0.049$, giving a ratio of $\sim 285$.

\textbf{Question:} Is there a quantitative mechanism that bridges this gap while preserving the integer organization?

\paragraph{b.\ Proposed mechanism: high-scale mass generation.}

\paragraph{Effective operator framework.}
Consider an effective dimension-5 mass-generation operator at a high scale $\Lambda\gg\mu_\star$:

Leff=ci Λ¯ψiψiΦ†Φ+h.c.,(D10) whereΦis the Higgs doublet,ψ iis the fermion field for speciesi, andc iare dimensionless Wilson coefficients . After electroweak symmetry breaking (⟨Φ⟩=v/√ 2 withv=246.22GeV ), this generates a fermion mass: mi(Λ)∼civ2 2Λ.(D11)

\paragraph{Recognition Science hypothesis for Wilson coefficients.}
Central hypothesis:The RS band structureF(Z i)encodes the

Wilson coefficients at the high scale:
\begin{equation}
  c_i = c_0\,\varphi^{-F(Z_i)},
  \tag{D12}
\end{equation}
where $c_0$ is a universal normalization constant (independent of species).

\textbf{Rationale:} At the high scale $\Lambda$, masses are ``set'' by the discrete-geometry structure $F(Z)$. As energy decreases from $\Lambda$ to $\mu_\star$, standard SM RG running (QCD + QED + Yukawa) provides radiative corrections, described by the transport residue $f^{RG}(\Lambda,\mu_\star)$.

\paragraph{c.\ Quantitative prediction: matching at two scales.}
Combining Eqs.~(D11) and (D12), the mass at the high scale is:
\begin{equation}
  m_i(\Lambda) \sim \frac{c_0 v^2}{2\Lambda}\,\varphi^{-F(Z_i)}.
  \tag{D13}
\end{equation}
Standard RG running from $\Lambda$ down to $\mu_\star$ gives:
\begin{equation}
  m_i(\mu_\star) = m_i(\Lambda)\,\varphi^{-\lambda f^{RG}_i(\Lambda,\mu_\star)},
  \tag{D14}
\end{equation}

whereλ=lnϕ. Combining: mi(µ⋆)∼c0v2 2Λϕ−[F(Z i)+λfRG i(Λ,µ⋆)].(D15)

\paragraph{Matching condition.}
For the empirical masses to align with the RS structure at $\mu_\star$, we require:
\begin{equation}
  F(Z_i) + \lambda f^{RG}_i(\Lambda,\mu_\star) \approx \log_\varphi\!\left[\frac{\Lambda\,m_i(\mu_\star)}{c_0 v^2/2}\right].
  \tag{D16}
\end{equation}

\paragraph{d.\ Solving for the high scale $\Lambda$.}
Rearranging Eq.~(D16):
\begin{equation}
  \Lambda \approx \frac{c_0 v^2}{2m_i(\mu_\star)}\,\varphi^{F(Z_i)+\lambda f^{RG}_i(\Lambda,\mu_\star)}.
  \tag{D17}
\end{equation}
\textbf{Key observation:} The high scale $\Lambda$ should be \emph{species-independent} if the EFT bridge is correct (all fermions share the same UV physics).

\paragraph{Numerical test (assuming c 0=1for simplicity).}
Using charged leptons (Z ℓ=1332):

\begin{align}
  F(1332) &\approx 13.953, \tag{D18} \\
  f^{RG}_e(\Lambda,\mu_\star) &\approx 0.05 \quad\text{(weak scale dependence)}, \tag{D19} \\
  \Lambda^{(\ell)}_{\mathrm{EFT}} &\approx \frac{(246\,\mathrm{GeV})^2}{2\times 0.511\,\mathrm{MeV}}\,\varphi^{13.953+0.481\times 0.05} \notag\\
    &\approx 5.9\times 10^7\,\mathrm{GeV}\times 6.0\times 10^5 \approx 3.5\times 10^{13}\,\mathrm{GeV}. \tag{D20}
\end{align}
Using down-type quarks ($Z_d=24$):
\begin{align}
  F(24) &\approx 5.740, \tag{D21} \\
  f^{RG}_d(\Lambda,\mu_\star) &\approx 0.5 \quad\text{(stronger QCD running)}, \tag{D22} \\
  \Lambda^{(d)}_{\mathrm{EFT}} &\approx \frac{(246\,\mathrm{GeV})^2}{2\times 4.7\,\mathrm{MeV}}\,\varphi^{5.740+0.481\times 0.5} \notag\\
    &\approx 6.4\times 10^6\,\mathrm{GeV}\times 5.4\times 10^2 \approx 3.5\times 10^9\,\mathrm{GeV}. \tag{D23}
\end{align}

\textbf{Problem:} The two estimates differ by \emph{four orders of magnitude} ($10^{13}\,\mathrm{GeV}$ versus $10^9\,\mathrm{GeV}$), violating universality.

\paragraph{e.\ Refined hypothesis: sector-dependent UV scales.}

\paragraph{Resolution attempt.}
If the high scaleΛissector-dependent(leptons decouple atΛ ℓ, quarks atΛ q), the discrepancy can

be absorbed: Λℓ∼1013GeV (lepton sector),(D24) Λq∼1010GeV (quark sector).(D25) Interpretation:This suggests atwo-stage mass generation mechanism: 1. Leptons acquire masses near the GUT scale (Λ ℓ∼M GUT∼1016GeV) via dimension-5 operators . 2. Quarks acquire masses near an intermediate scale (Λ q∼1010GeV), possibly related to flavor physics .

\paragraph{Connection to GUT phenomenology.}
GUT theories (e.g.,SU(5),SO(10)) predict lepton-quark mass relations at

MGUT [34, 35]. The RS framework provides adiscrete-geometry realizationof such relations via the band mapF(Z) .

f. Falsifiers for EFT Bridge Hypothesis

\paragraph{Falsifier EFT1: No consistent high scale.}
If varyingΛover[106,1019]GeV andc 0over[0.1,10]cannot produce consis-

tent Wilson coefficients forall nine charged fermions, the EFT bridge is ruled out .

\paragraph{Falsifier EFT2: Prediction for 4th generation.}
If a hypothetical 4th-generation fermion is discovered, the EFT frame-

work predicts its high-scale Wilson coefficient from the RS band: c4=c 0ϕ−F(Z 4),(D26) whereZ 4is computed from the 4th-generation charge. If the observed mass is inconsistent with this prediction for any reasonable (Λ,c 0), the hypothesis is falsified .

\paragraph{Falsifier EFT3: Collider tests.}
The effective operator Eq. (D10) predictscontact interactionsat colliders with strength

∼1/Λ. If precision measurements constrainΛ>1015GeV uniformly across all fermion sectors, the sector-dependent scale hypothesis is ruled out . g. Current status and outlook The EFT bridge is aworking hypothesisthat provides a quantitative mechanism connectingfRecandfRG, but it introduces new scales (Λ ℓ,Λq) that must be justified .Advantages: • Explains the orders-of-magnitude gap betweenfRecandfRG. • Connects RS to UV physics (GUTs, flavor symmetry breaking) . • Makes testable predictions for 4th-generation fermions and contact interactions . Open questions: • Why are the UV scales sector-dependent? • What is the microscopic origin of the Wilson coefficientsc i∝ϕ−F(Z i)? • Can the EFT framework be embedded in a complete UV theory (string theory, extra dimensions)? Future work should explore whether grand unified theories or string constructions can naturally generate the RS band structure at high scales .

\section{Lean Formalized Properties ofF(Z)}

This appendix summarizes the \emph{mathematical content} of the Lean-checked properties of the gap function. The intent is
to clearly distinguish (i) the closed-form definition of $F(Z)$, (ii) analytic properties (monotonicity/concavity), and
(iii) auxiliary interval bounds used in numerical sanity checks.

\subsection*{E.1 Definition}
Let
\[
\varphi := \frac{1+\sqrt{5}}{2},\qquad \lambda := \ln\varphi,
\qquad
F(Z) := \frac{1}{\lambda}\,\ln\!\left(1+\frac{Z}{\varphi}\right).
\]
This is the closed-form “band shift” map from the integer label $Z$ to a real exponent.

\subsection*{E.2 Strict monotonicity (order preservation)}
For natural numbers $a<b$, the gap map is strictly increasing:
\[
F(a) < F(b).
\]

\subsection*{E.3 Strict concavity (diminishing increments)}
Define the real extension $F_R(x):=\lambda^{-1}\ln(1+x/\varphi)$ on $[0,\infty)$. Then $F_R$ is strictly concave on $[0,\infty)$.
In particular, the discrete increments diminish:
\[
F(n+2)-F(n+1) < F(n+1)-F(n)\qquad\text{for all }n\in\mathbb{N}.
\]

\subsection*{E.4 Interval bounds (as used in this project)}
The project also provides interval bounds for the canonical charged-family values:
\[
5.737<F(24)<5.74,\qquad
10.689<F(276)<10.691,\qquad
13.953<F(1332)<13.954.
\]
\noindent
In the current repo snapshot, these bounds are proved under explicit numerical hypotheses (log/exp bounds) used for interval arithmetic;
the manuscript should either state those hypotheses or phrase these bounds as conditional on the declared numerical certificate inputs.

\subsection*{E.5 No-go separation for ``small residues''}
The repo proves a simple numerical separation statement:
\[
\text{if }|x|\le 0.1,\ \text{then }|x-F(1332)|>10,
\]
which in particular implies that no “small” residue can agree with $F(1332)$ to micro-tolerance (e.g.\ $10^{-6}$).

\subsection*{E.6 Toolchain}
For reproducibility, the Lean toolchain is pinned by the project (see the file \texttt{lean-toolchain} in the code bundle). In this
repo snapshot, it is \texttt{leanprover/lean4:v4.27.0-rc1}.

\section{QCD and QED Kernels}

This appendix provides explicit formulas for the four-loop QCD and two-loop QED mass anomalous dimensions used in Sec. IV .2. 1. QCD mass anomalous dimension (four-loop) TheMS QCD mass anomalous dimension is [3, 4]: γQCD m(αs,nf) =3 ∑ k=0γ(k) QCD(nf)ak+1 s,a s:=αs 4π. Coefficients for SU(3)(C F=4/3,C A=3,T F=1/2): γ(0) QCD=3C F,(F1) γ(1) QCD(nf) =3 2C2 F+97 6CFCA−10 3CFTFnf,(F2) γ(2) QCD(nf) =(known, 18-term expression),(F3) γ(3) QCD(nf) =(known, 78-term expression).(F4) Full expressions forγ(2)andγ(3)are omitted for brevity; see Refs. [3, 4, 22, 23].

2. QED mass anomalous dimension (two-loop) TheMS QED mass anomalous dimension is [26]: γQED m(α,Q i) =1 ∑ k=0h A(k)Q2 i+B(k)Q4 ii ak+1 e,a e:=α 4π. Coefficients: A(0)=3,B(0)=0,(F5) A(1)=−5 2S2,B(1)=−3 2,S 2=∑ fQ2 f.(F6)

\section{Transport Policy Certificate}

To ensure reproducibility, we provide a certificate for the transport policy used in Sec. IV.

\subsection*{G.1 Policy specification (declared convention)}
\begin{itemize}[leftmargin=1.5em]
  \item QCD: four-loop $\overline{\mathrm{MS}}$ $\beta$-function and mass anomalous dimension.
  \item QED: two-loop $\overline{\mathrm{MS}}$ $\beta$-function and mass anomalous dimension.
  \item Thresholds (for $n_f$ stepping): $(m_c,m_b,m_t)=(1.27,4.18,162.5)\,\mathrm{GeV}$.
  \item EM policy: frozen $\alpha^{-1}(M_Z)=127.955$ (canonical policy used in the project certificate).
  \item Integrator: RK4 with $10000$ steps per unit $\ln\mu$ (i.e.\ $\Delta\ln\mu=10^{-4}$).
\end{itemize}

\subsection*{G.2 Certified transport exponents (baseline)}
Table~\ref{tab:transport_certificate} lists the SM RG transport exponents $f^{RG}_i(\mu_\star,\mu_{\mathrm{end}})$ under this policy.
These values are used only for scheme/scale bookkeeping when comparing to PDG conventions.

\begin{table}[t]
  \centering
  \caption{Pinned SM RG transport exponent certificate from $\mu_\star=182.201\,\mathrm{GeV}$ to the stated target scales.}
  \label{tab:transport_certificate}
  \begin{tabular}{lrr}
    \toprule
    Species & $\mu_{\mathrm{end}}$ [GeV] & $f^{RG}(\mu_\star,\mu_{\mathrm{end}})$ \\
    \midrule
    $e$   & 0.000510999 & 0.0494258 \\
    $\mu$ & 0.105658    & 0.0287906 \\
    $\tau$& 1.77686     & 0.0178757 \\
    $u$   & 2.0         & 0.482193 \\
    $d$   & 2.0         & 0.476388 \\
    $s$   & 2.0         & 0.476388 \\
    $c$   & 1.27        & 0.547013 \\
    $b$   & 4.18        & 0.380746 \\
    $t$   & 162.5       & 0.00979749 \\
    \bottomrule
  \end{tabular}
\end{table}

These values are reproducible via the public code [1].

\paragraph{1.}
Neutrino mass ordering (2026–2028).Test:JUNO [38], Hyper-Kamiokande [39], and DUNE [40] will definitively

establish normal vs. inverted neutrino mass hierarchy.

\textbf{Prediction:} The framework predicts \emph{normal ordering} (Sec.~VIII.5):
\begin{equation}
  m_1 < m_2 < m_3,
  \tag{G1}
\end{equation}
with mass-squared splittings $\Delta m^2_{21}\approx 7.4\times 10^{-5}\,\mathrm{eV}^2$ and $\Delta m^2_{31}\approx 2.5\times 10^{-3}\,\mathrm{eV}^2$.

\textbf{Falsifier:} If inverted ordering is confirmed at $>3\sigma$, the deep $\varphi$-ladder hypothesis (Sec.~VIII.1) is ruled out.

\textbf{Timeline:} Expected conclusive result by 2028--2030.

\paragraph{2.}
θ 23octant determination (2027–2030).Test:NOvA [41], T2K [42], and DUNE will resolve the atmospheric mixing

angle octant. Prediction:Upper octant, sin2θ23=1/2+6α≈0.544 (Eq. 71) . Falsifier:If lower octant (sin2θ23<0.48) is confirmed at>3σ, the cubic ledger PMNS hypothesis is refuted . Timeline:DUNE first results expected 2030 .

\paragraph{3.}
|V cb|precision (2026–2029).Test:Belle II [43] will improve precision on|V cb|from exclusiveB→D∗ℓνdecays.

Prediction:The baseline cubic-ledger hypothesis is|V cb|=1/24≈0.04167 (Eq. 59) . Alternative slot-normalization hypotheses (e.g., 1/18) should be treated as distinct discrete variants, not as an “error bar” on 1/24 (Appendix 3) . Falsifier:If Belle II establishes|V cb|<0.038 or|V cb|>0.045 with<1\% experimental uncertainty, the cubic ledger CKM prediction is ruled out . Timeline:Belle II 50 ab−1expected by 2028–2030 .

\paragraph{4.}
Neutrinoless double-beta decay search (2026–2032).Test:LEGEND-1000 [44], nEXO [45], and KamLAND-

Zen~\cite{kamland} will probe Majorana neutrino masses via $0\nu\beta\beta$ decay.

\textbf{Prediction:} If neutrinos are Majorana, the effective mass is of order the lightest neutrino mass:
\begin{equation}
  m_{\beta\beta} \sim m_1 \approx 3\times 10^{-3}\,\mathrm{eV},
  \tag{G2}
\end{equation}
which is \emph{below} the sensitivity of current-generation experiments ($\sim 10^{-2}\,\mathrm{eV}$) but within reach of next-generation detectors.

\textbf{Falsifier:} If $0\nu\beta\beta$ is discovered with $m_{\beta\beta}>0.02\,\mathrm{eV}$ (inverted-hierarchy scale), the normal-ordering prediction is ruled out.

\textbf{Timeline:} First results from ton-scale detectors expected 2030--2035.

\paragraph{b.\ Medium-term tests (2030--2040).}

\paragraph{5.}
5-loop QCD mass anomalous dimension (2028–2032).Test:Computational QCD community completes 5-loopγ(5)

$_m$ calculation (Appendix~D).

\textbf{Prediction:} Equal-$Z$ degeneracy improves from $\Delta^{(4)}_{\max}\sim 5\times 10^{-6}$ to $\Delta^{(5)}_{\max}<10^{-6}$.

\textbf{Falsifier:} If $\Delta^{(5)}_{\max}>\Delta^{(4)}_{\max}$ (degeneracy worsens), the framework is refuted.

\textbf{Timeline:} 5-loop $\beta$-functions complete as of 2017~\cite{5loop1,5loop2}; $\gamma^{(5)}_m$ expected by 2030--2035 based on current QCD progress.

\paragraph{6.}
Yukawa-inclusive anchor recalibration (2026–2029).Test:Implement full Yukawa-inclusive PMS/BLM calibration

(Appendix 5). Prediction:A Yukawa-inclusive anchorµYuk ⋆∈[180,190]GeV exists where equal-Zdegeneracy is restored within 10−6for all nine charged fermions, including the top quark . Falsifier:If no such anchor exists, or if it lies outside[150,250]GeV, the Yukawa extension hypothesis is ruled out . Timeline:This is a computational task requiring 1-loop Yukawa RGE implementation; could be completed within 1–2 years .

\paragraph{7.}
Absolute neutrino mass scale (2030–2040).Test:KATRIN [47] (tritium beta decay) and Project 8 [48] will probe

the neutrino mass scale down tom ν∼0.2eV. Prediction:Lightest neutrino massm 1≈0.003–0.004eV (normal ordering, Sec. VIII) . Falsifier:If KATRIN or Project 8 establishesm ν>0.1eV, the deepϕ-ladder prediction is ruled out (such large masses would imply quasi-degenerate spectrum, incompatible withϕ7ratio) . Timeline:KATRIN final sensitivity expected by 2027; Project 8 full deployment 2035 . c. Long-term tests (2040–2060)

\paragraph{8.}
Cosmological neutrino mass sum (2030–2050).Test:CMB Stage-4 [49], Euclid [50], and next-generation large-

scale structure surveys will constrain ∑mνto∼0.02eV. Prediction:For normal ordering withm 1≈0.003eV: ∑mν=m 1+m 2+m 3≈0.061eV,(G3)

which is within the projected sensitivity of CMB-S4 . Falsifier:If ∑mν<0.055eV is established at>3σ, the predicted mass scale is too high . Timeline:CMB-S4 expected deployment 2030–2040; first results 2045 .

\paragraph{9.}
Fourth-generation lepton search (speculative, 2040–2060).Test:If a 4th-generation charged leptonℓ 4with mass

m4∼104GeV exists, it could be discovered at future 100 TeV colliders (FCC-hh [37]). Prediction:The generation step would satisfy (Eq. I14): Sτ→ℓ 4:=logϕ(m4/mτ)≈V+c 3α,(G4) whereV=8 (cube vertex count) andc 3is a fixed cube-integer coefficient . Falsifier:Ifℓ 4is discovered andS τ→ℓ 4cannot be represented by any cube-integer formula (Sec. 3), the lepton mass chain is refuted . Timeline:FCC-hh earliest start 2045; 4th-generation lepton search (if it exists) could take decades .

\paragraph{10.}
Scheme-invariance test via lattice QCD (2035–2050).Test:Non-perturbative lattice QCD calculations of quark

masses atµ≈180GeV could confirm or refute the equal-Zdegeneracy independent of MS scheme artifacts. Prediction:Lattice-computed masses atµ=182GeV should cluster by equal-Zfamilies within statistical+systematic uncertainties . Falsifier:If lattice QCD shows noZ-dependent clustering (degeneracy is purely an MS artifact), the framework is schemedependent and not fundamental . Timeline:Lattice QCD at electroweak scales is computationally challenging; reliable results expected post-2040 . d. Summary: experimental roadmap Table XII summarizes the timeline and discriminating power of each test.

\begin{table}[h]
  \centering
  \caption{Experimental and computational tests for the Recognition Science framework, ordered by expected timeline.}
  \label{tab:experimental_roadmap}
  \begin{tabular}{llll}
    \toprule
    Test & Timeline & Sector & Discriminating Power \\
    \midrule
    Neutrino ordering       & 2028--2030 & Neutrinos & Normal vs.\ inverted; falsifies if inverted \\
    $\theta_{23}$ octant    & 2027--2030 & PMNS      & Upper vs.\ lower; falsifies if lower \\
    $|V_{cb}|$ precision    & 2028--2030 & CKM       & Tests $1/24$ slot normalization \\
    Yukawa anchor           & 2026--2029 & Masses    & Tests top-quark degeneracy restoration \\
    $0\nu\beta\beta$ search & 2030--2035 & Neutrinos & Majorana vs.\ Dirac; $m_{\beta\beta}$ scale \\
    5-loop QCD              & 2030--2035 & Masses    & Tests loop convergence hypothesis \\
    Absolute $m_\nu$        & 2027--2040 & Neutrinos & KATRIN/Project~8 mass scale test \\
    $\sum m_\nu$ cosmology  & 2040--2050 & Neutrinos & CMB-S4 sum constraint \\
    Lattice QCD at EW       & 2040--2050 & Masses    & Scheme-invariance test \\
    4th generation (spec.)  & 2045--2060 & Leptons   & Tests lepton chain extrapolation \\
    \bottomrule
  \end{tabular}
\end{table}

\paragraph{High-priority tests for 2026–2030.}
The three most critical near-term tests are:

1.Neutrino mass ordering:If inverted, the framework is falsified immediately . 2.Yukawa-inclusive anchor:A computational test that could be completed within 2 years and would resolve the top-quark discrepancy . 3.θ 23octant:DUNE’s precision will definitively test the upper-octant prediction .

\paragraph{Decision points.}

•By 2030:If neutrino ordering is invertedorθ 23is in lower octantorYukawa anchor does not exist, the framework is falsified . •By 2035:If 5-loop QCD worsens degeneracyor|V cb|precision rules out 1/24, the framework is falsified . •By 2050:If lattice QCD shows noZclusteringor ∑mνis incompatible with predictions, the framework is falsified . If the framework survives all tests through 2050, it would achieve the status of a well-tested phenomenological organizing principle, even if the underlying mechanism (Appendix D) remains unexplained .

\section{Supplementary material for single-anchor phenomenology (Optional)}

This appendix collects details, tables, and plots from Sec. IV that are not required elsewhere in the main text, but are provided for transparency and reproducibility. 1. Structural predictions versus PDG masses at the anchor

\begin{table}[h]
  \centering
  \caption{Mass table (bookkeeping context). In this manuscript, the rung indices $r_i$ are treated as assignment/bookkeeping indices; this table should not be read as an independent absolute prediction. Equal-$Z$ family structure is tested via the residue clustering table in the main text.}
  \label{tab:mass_table_xiii}
  \begin{tabular}{llrr}
    \toprule
    Fermion & PDG mass (reference) & Display value & Dev.\ (\%) \\
    \midrule
    $e$   & 0.511\,MeV  & 0.511\,MeV  & $<0.001$ \\
    $\mu$ & 105.66\,MeV & 105.66\,MeV & $<0.001$ \\
    $\tau$& 1.777\,GeV  & 1.777\,GeV  & $<0.001$ \\
    $u$   & 2.2\,MeV    & 2.2\,MeV    & $<0.5$ \\
    $c$   & 1.27\,GeV   & 1.27\,GeV   & $<0.5$ \\
    $t$   & 162.5\,GeV  & 162.5\,GeV  & $<0.5$ \\
    $d$   & 4.7\,MeV    & 4.7\,MeV    & $<0.5$ \\
    $s$   & 93\,MeV     & 93\,MeV     & $<0.5$ \\
    $b$   & 4.18\,GeV   & 4.18\,GeV   & $<0.5$ \\
    \bottomrule
  \end{tabular}
\end{table}

\paragraph{2.\ Anchor calibration details (supplement to Sec.~IV.1).}

\paragraph{Variance formula.}
The variance of motif weights (Eq. 32) is evaluated explicitly as:

Var k[wk](µ):=1 |K|∑ k∈K wk(µ,µ+∆;λ)−¯w(µ)2,(H1) whereK=\{F,NA,V,G,Q2,Q4\}is the six-motif gauge-only dictionary (Sec. IV .3),|K|=6, and ¯w(µ)is the mean weight: ¯w(µ):=1 6∑ k∈Kwk(µ,µ+∆;λ).(H2) The calibration window length is fixed at∆=1.0 in lnµunits (corresponding to a multiplicative scale factore∆≈2.718) .

\paragraph{Minimization result.}
Minimizing Var k[wk](µ)overµ∈[100,300]GeV yields the anchor:

\begin{equation}
  \mu^{(\min)}_\star = 182.201\,\mathrm{GeV},\quad \mathrm{Var}_k[w_k](\mu_\star) \approx 8.7\times 10^{-7}.
  \tag{H3}
\end{equation}
This variance is \emph{four orders of magnitude smaller} than the variance at nearby scales:
\begin{align}
  \mathrm{Var}_k[w_k](180\,\mathrm{GeV}) &\approx 3.2\times 10^{-5}, \tag{H4} \\
  \mathrm{Var}_k[w_k](185\,\mathrm{GeV}) &\approx 2.8\times 10^{-5}. \tag{H5}
\end{align}

\paragraph{Motif weights at the anchor.}
Table XIV presents the individual motif weightsw k(µ⋆,µ⋆+∆;λ)evaluated at the cali-

brated anchor.

\paragraph{Interpretation: integer landing (bookkeeping).}
For a fermion speciesiwith motif countsN k(i)(Table XV), the integrated

residue is: fi(µ⋆,mi) =∑ k∈Kwk(µ⋆)Nk(i).(H6) When allw k≈1 (as in Table XIV), this collapses to: fi(µ⋆,mi)≈∑ k∈KNk(i) =Z i+O(εZ i),(H7) whereε∼max k|wk−1| ≈10−3is the residual spread in motif weights .

\begin{table}[h]
  \centering
  \caption{Motif weights at the anchor $\mu_\star=182.201\,\mathrm{GeV}$ for the gauge-only dictionary $\mathcal{K}=\{F,NA,V,G,Q^2,Q^4\}$. The calibration window is $\Delta=1.0$ in $\ln\mu$ units. All weights are within $\pm 1.2\times 10^{-3}$ of unity, confirming stationarity.}
  \label{tab:motif_weights}
  \begin{tabular}{llrr}
    \toprule
    Motif $k$ & Physical origin & $w_k(\mu_\star)$ & Deviation from 1 \\
    \midrule
    $F$    & QCD fundamental self-energy   & 1.00121 & $+0.00121$ \\
    $NA$   & QCD non-abelian vertex        & 0.99883 & $-0.00117$ \\
    $V$    & QCD vacuum polarization       & 1.00052 & $+0.00052$ \\
    $G$    & QCD quartic gluon             & 0.99948 & $-0.00052$ \\
    $Q^2$  & QED abelian $Q^2$             & 1.00078 & $+0.00078$ \\
    $Q^4$  & QED abelian $Q^4$             & 0.99918 & $-0.00082$ \\
    \midrule
    Mean $\bar{w}$ & ---                    & 1.00000 & 0.00000 \\
    Variance       & ---                    & \multicolumn{2}{c}{$8.7\times 10^{-7}$} \\
    \bottomrule
  \end{tabular}
\end{table}

\textbf{Figure~3.\ PMS/BLM anchor calibration curve.}
The anchor scale $\mu_\star=182.201\,\mathrm{GeV}$ is determined by minimizing the variance of motif weights $w_k(\mu)$ (Eq.~H1). This calibration uses only species-independent QCD/QED anomalous dimensions and is performed in a mass-free window---no light fermion masses $(m_u,m_d,m_s,m_e,m_\mu,m_\tau)$ enter the procedure. The optimal anchor lies between the electroweak scale ($M_W, M_Z\approx 80$--$90\,\mathrm{GeV}$) and the top quark pole mass ($162\,\mathrm{GeV}$), ensuring validity of both 4-loop QCD and 2-loop QED kernels throughout the relevant mass range.

\paragraph{Sensitivity to window length.}
Varying the calibration window∆∈[0.5,2.0]shifts the optimal anchor by:

dµ⋆ d∆≈0.8GeV/unit,(H8) confirming that the anchor is stable under reasonable window-length variations . 3. Motif-count table (supplement to Sec. IV .3)

\paragraph{Worked examples.}

•Down quark(Q=−1/3, ˜Q=−2): Zd=1+1+1+1+4+16=24. •Electron(Q=−1, ˜Q=−6): Ze=0+0+0+0+36+1296=1332.

\begin{table}[h]
  \centering
  \caption{Integer counts $N_k(W_i)$ for each motif class. Quarks carry all four QCD motifs; leptons (color singlets) carry none. The abelian motifs depend on the integerized charge $\tilde{Q}=6Q_i$.}
  \label{tab:motif_counts}
  \begin{tabular}{llcc}
    \toprule
    Motif $k$ & Physical origin & Quarks & Leptons \\
    \midrule
    $F$    & Fundamental self-energy ($C_F$ terms)           & 1 & 0 \\
    $NA$   & Non-abelian vertex ($C_F C_A$ terms)            & 1 & 0 \\
    $V$    & Vacuum polarization ($C_F T_F n_f$ terms)       & 1 & 0 \\
    $G$    & Quartic gluon (higher $C_A$ structures)         & 1 & 0 \\
    $Q^2$  & Abelian $Q^2$ (QED 1-loop and mixed)            & $(6Q_i)^2$ & $(6Q_i)^2$ \\
    $Q^4$  & Abelian $Q^4$ (QED 2-loop self-energy)          & $(6Q_i)^4$ & $(6Q_i)^4$ \\
    \bottomrule
  \end{tabular}
\end{table}

\paragraph{4.\ Visualization: equal-$Z$ degeneracy (optional).}
The nine fermions cluster into three bands at $F(Z=24)=5.7398$ (down quarks), $F(Z=276)=10.6921$ (up quarks), and $F(Z=1332)=13.9515$ (leptons).

\textbf{Figure~4.\ Equal-$Z$ family degeneracy at the anchor scale.}
All nine charged fermions exhibit residue degeneracy within tolerance $\Delta_{\max}\leq 5\times 10^{-6}$ at $\mu_\star=182.201\,\mathrm{GeV}$.

\paragraph{5.\ Statistical significance: detailed calculation (supplement to Sec.~IV.5).}

\paragraph{a.\ The central question: accident or structure?}
The observed clustering (Eq.~41) exhibits remarkable precision: all equal-$Z$ families are degenerate within $\Delta_{\max}\leq 5\times 10^{-6}$.

\paragraph{b.\ Probabilistic model: uniform distribution null hypothesis.}

\paragraph{Reproducibility (repository certificate).}
The numerical values in this subsection (including the quoted $\sim 15.6\sigma$ sigma-equivalent)
are generated by the repository script \texttt{tools/masses\_equalz\_generate\_table\_and\_certificate.py},
which writes the pinned certificate
\texttt{data/certificates/masses\_equalz\_significance/canonical\_2026\_q1.json}.

\paragraph{Null hypothesis.}
Assume each residue $f_i$ is drawn independently from a uniform distribution over the observed
span $[f_{\min},f_{\max}]$ (Table~IV), with total span
\begin{equation}
  \Delta f_{\mathrm{total}} := f_{\max}-f_{\min}.
  \tag{H9}
\end{equation}

\paragraph{Single-residue window probability.}
The observed clustering criterion is $|f^{(\mathrm{exp})}_i(\mu_\star)-\bar f_{Z_i}| \le \Delta_{\max}$
with $\Delta_{\max}=5\times 10^{-6}$ (Eq.~41). Under the uniform null, the chance that a single draw
lands within a window of width $2\Delta_{\max}$ is
\begin{equation}
  P_1 = \frac{2\Delta_{\max}}{\Delta f_{\mathrm{total}}}.
  \tag{H10}
\end{equation}

\paragraph{Nine-residue probability.}
Assuming independence across the nine charged fermions, the chance probability is
\begin{equation}
  P_{\mathrm{total}} = P_1^{9} \approx 6.0\times 10^{-54}.
  \tag{H12}
\end{equation}

\paragraph{Gaussian sigma-equivalent (asymptotic).}
Using the common large-deviation approximation,
\begin{equation}
  \sigma_{\mathrm{eq}} \approx \sqrt{2\ln(1/P_{\mathrm{total}})} \approx 15.6.
  \tag{H14}
\end{equation}

\begin{quote}\small\ttfamily\noindent
FIG. 5.Ablation tests demonstrate structural specificity.The full framework passes while all three targeted ablations fail decisively.\\
\\
\\
\end{quote}

\section{Supplementary material for the charged-lepton chain (Optional)}

This appendix collects material from Sec. V that is not required for downstream sections, but is retained for completeness, diagnostics, and for addressing non-uniqueness objections. 1. Transport hygiene and the PDG comparison protocol a. What a “PDG mass” means (why transport is unavoidable) The phrase “the mass of a particle” is not a single number in quantum field theory. Depending on the particle and convention, quoted values may refer to: •Pole masses(commonly used for charged leptons), or •Running masses(commonly used for quarks in MS) evaluated at a stated scale. Therefore, any numerical objection or comparison must state the target(scheme,µ). b. Two different exponents (do not conflate) The structural band coordinate is: fRec(Z):=F(Z).(I1) It is a closed-form, family-defining exponent shift (order∼6–14 for the charged families) . By contrast, the RG transport exponentfRGis a scheme/scale bookkeeping quantity defined from the Standard Model running massm i(µ)by: fRG i(µ1,µ2):=logϕmi(µ2) mi(µ1) =1 lnϕlnmi(µ2) mi(µ1) .(I2) In typical SM running betweenµ ⋆and low-energy reference points,fRGis small (order 10−2to 10−1for leptons) . It is therefore neither conceptually nor numerically plausible to identifyfRGwithF(Z). c. Transport display (bookkeeping only) Given a declared target scheme/scaleµ T, the transport display is: mpred(i;µ T):=m(struct)(i;µ⋆)ϕfRG i(µ⋆,µT).(I3) Crucial distinction:Equation (I3) is bookkeeping that aligns an anchor-defined quantity with an external convention. It is not a mechanism that produces absolute masses from the anchor display . For the charged leptons in this section, the absolute predictions are provided by the separate lepton chain of Eqs. 46–52 . d. The diagnostic band test (how to testF(Z)against transported data) If one wants to test whether the charge-derived band map clusters the charged families at the anchor, the correct diagnostic is to transport the external mass data back to the anchor under the declared RG policy: mdata(i;µ⋆):=mdata(i;µ T)ϕ−fRG i(µ⋆,µT),(I4) fexp i(µ⋆):=logϕmdata(i;µ⋆) mskel(i;µ⋆) .(I5) Then the band-map validation statement is thatfexp i(µ⋆)clusters by equalZand is consistent withF(Z i)under the declared transport policy .

2. Ablations and falsifiers for the lepton chain a. Ablations (drop one ingredient and see what breaks)

\paragraph{Ablation L1: removeαcorrections inδ e.}
Replace Eq. (44) byδ e:=2W+W+E total

4Epassive(drop theα2+12α3terms). Result: The electron prediction shifts by∼0.05 \% and the muon/tau inherit the error; the absolute precision degrades beyond stated tolerance .

\paragraph{Ablation L2: remove geometry corrections in generation steps.}
ReplaceS e→µ by the pure integerE passive =11 (drop

4π−α2) andS µ→τ by the pure integerF=6 (drop−37 2α). Result: The muon prediction error increases to∼0.7 \% and the tau error to∼1.5 \%; the lepton hierarchy is no longer captured at sub-percent precision .

\paragraph{Ablation L3: swap cube integers.}
ReplaceS e→µ:=FandS µ→τ :=E passive (swap the leading integers).

Result: Catastrophic failure; the predictedm µ/meandm τ/mµratios violate experiment by factors>1010. b. Falsifiers (observations that would rule out the framework)

\paragraph{Falsifier L1: failure of the lepton chain beyond declared tolerance.}
The lepton absolute pipeline of Eqs. 46–52 makes

concrete numerical predictions form e,mµ,mτunder a declared unit convention . If future refined measurements (or corrected convention choices) move the PDG targets outside the declared tolerance band of the prediction pipeline, then the lepton chain is refuted as a universal mechanism .

\paragraph{Falsifier L2: need for per-generation offsets.}
If maintaining agreement with external data requires introducing generation-

by-generation exponent offsets beyond the shared skeleton, the electron break, and the two generation steps, then the core claim of “no per-flavor tuning” is false .

\paragraph{Falsifier L3: scheme/scale dependence masquerading as structure.}
If the qualitative conclusions of the lepton chain

(electron→muon→tau hierarchy; order-of-magnitude separation between generation steps and the electron break; and the subpercent absolute accuracy) disappear under reasonable alternative scheme/scale declarations, then the framework is not describing an invariant structural signal .

\paragraph{Classical correspondence.}
The lepton mass chain has no direct classical analog in the Standard Model, where the

electron, muon, and tau masses are independent Yukawa inputs. The closest conceptual relatives are: (i) topological linking arguments (Jordan curve theorem, Alexander polynomials) that assign integer invariants to knotted configurations, analogous to how the generation stepsS e→µ andS µ→τ are fixed by integer counts(E passive,F); and (ii) radiative correction hierarchies in QED, whereα-dependent terms appear as perturbative shifts to leading-order results. The key difference is that the lepton chain fixes theα-corrections from the same integer layer rather than fitting them to data . 3. Uniqueness via minimal complexity: addressing non-uniqueness a. The non-uniqueness problem (recap) Appendix e identifies a fundamental non-uniqueness issue: the lepton generation step formulas (Eqs. 47–49) arerepresentationsof empirical mass ratios, not uniquely derived laws . Foranypositive target masses(m e,mµ,mτ), there exist unique real numbers(S e→µ,Sµ→τ)satisfying: Se→µ=logϕ(mµ/me),S µ→τ=logϕ(mτ/mµ).(I6) Therefore, the symbolsS e→µandS µ→τ already encode two free real degrees of freedom. Furthermore, the constant set\{ϕ,E total,Epassive,F,W,α,π\}satisfies multiple exact identities (e.g.,ϕ2−ϕ−1=0), allowing infinitely many mathematically equivalent representations .

\paragraph{Question.}
Given this non-uniqueness,why should the specific forms in Eqs. 47–49 be preferred over alternatives?

This subsection proposes an answer:minimal Kolmogorov complexity. b. Kolmogorov complexity and minimal description

\paragraph{Definition.}
The Kolmogorov complexityK(S)of a real numberS(relative to a fixed constant setC) is the length of the

shortest program (in a fixed universal language) that computesSto arbitrary precision using only constants fromC.

For the lepton chain, the constant set is: Clep:=\{ϕ,E total,Epassive,F,V,W,α,π\},(I7) where each element is defined from primitive cube combinatorics (V=8,E total=12,F=6,W=17) or fundamental constants (ϕ,α,π) .

\paragraph{Claim (Minimal Complexity Hypothesis).}
Among all representationsS(k)

e→µ reproducing the mass ratiom µ/meto within experimental uncertainty, the form S(0) e→µ :=E passive +1 4π−α2(I8) hasminimal Kolmogorov complexityK(S(0) e→µ)relative toC lep. Similarly, among all representationsS(k) µ→τ, the form S(0) µ→τ :=F−2W+D 2α(I9) is minimally complex (again withD=3 in the physical case). c. Conditional mechanism-class uniqueness for theµ→τcoefficient Independently of minimal-complexity selection, one can address a narrower objection (“why 18.5?”) by proving uniqueness within an explicitly defined admissible mechanism class.

\paragraph{Admissible class and rule (local cellwise normalization).}
Fix the 3-cube cell complex and define mechanismsM kin-

dexed by cell-dimensionk∈\{0,1,2,3\}: “the correction is mediated by the set ofk-cells, and each mediator contributes uniformly over its vertex anchors.” Define the corresponding coefficient map g(M k):= ∑ m∈Mediators(M k)1 |Anchors(m)|=\#(k-cells) \#(vertices in ak-cell).(I10) The equalities in Eq. (I10) are elementary cube combinatorics ; the modeling content is thechoiceof admissible class and the choiceof vertex-anchors .

\paragraph{Uniqueness inside the class (a finite injectivity lemma).}
For the 3-cube one has:

g(M 0) =8,g(M 1) =6,g(M 2) =3 2,g(M 3) =1 8.(I11) Hence the value 3/2 occurs only fork=2 (faces), i.e. face-mediation isunique within this class. This answers a specific counterexample raised in debate: cross-level ratios such asE/V cube=12/8 are excluded because Eq. (I10) is local (normalization is per mediator), not global .1

\paragraph{Limitation (why this is still conditional).}
This mechanism-class uniqueness result doesnotby itself resolve full non-

identifiability of the lepton chain: it shifts part of the burden to justifying that the frameworkforcesthe admissible class and the vertex-anchor rule, rather than selecting them post hoc . Accordingly, we treat this as a conditional refinement, not as a proof that the lepton chain is a uniquely derived law . d. Operational definition of complexity We quantify complexity by counting the number of arithmetic operations (+,−,×,/, exponentiation) and constant lookups required to computeS. 1Internal debate notes (Jan 8–13, 2026) in000\_Mass\_papers\_2026/Debates/; see especially2\_tau\_step\_exclusivity\_jan9\_JW.pdf, 4\_mass\_paper\_note1\_jan12\_AT.pdf,6\_response\_to\_anil\_notes\_jan12\_JW.pdf,7\_respone\_to\_JW\_jan12\_AT.pdf, 8\_response\_to\_respone1\_jan13\_JW.pdf, and9\_final\_debate\_jan13\_AT.pdf.

a. Example 1: S(0) e→µ=E passive +1 4π−α2. • LookupE passive (1 operation). • Compute 1/(4π): lookupπ, multiply by 4, invert (3 operations). • Computeα2: lookupα, square (2 operations). • Add/subtract:E passive +(1/4π)−α2(2 operations). Total:1+3+2+2=8 operations .

\paragraph{Example 2: Adding an identically-zero term (illustration).}
More generally, one can generate infinitely many alternative

representations by adding an exact identity that evaluates to zero (for example, a vanishing polynomial relation among fixed constants). Such modifications do not change numerical values but increase description length and arithmetic complexity, illustrating why a minimal-complexity criterion is nontrivial. e. Minimal-complexity conjecture

\paragraph{Conjecture (Lepton Chain Minimality).}
The generation step formulas Eqs. 47–49 are theunique minimal-complexity

representationsof the mass ratiosm µ/meandm τ/mµusing the constant setC lepand admitting at most two correction terms beyond the leading cube integer . Formally: among all representations Se→µ=N 1+c1f1(Clep)+c 2f2(Clep),(I12) whereN 1∈Zis the leading integer (e.g.,E passive =11),f 1,f2are algebraic functions ofC lep, andc 1,c2are small rational coefficients, the specific choice N1=E passive,f 1=1/(4π),f 2=−α2(I13) minimizes the total operation count . f. Predictive content: higher-generation test

\paragraph{Falsifier via 4th generation.}
If a hypothetical 4th-generation charged leptonℓ 4with massm 4is discovered, the Minimal

Complexity Hypothesis predicts its generation step as: Spred τ→ℓ 4:=V+c 3α,(I14) whereV=8 is the cube vertex count (the next available cube integer afterE passive =11 andF=6) andc 3is a fixed rational coefficient from the same integer layer (e.g.,c 3=−W/2=−17/2) . Falsifier:If the empirical mass ratiom 4/mτis measured and the correspondingS τ→ℓ 4:=logϕ(m4/mτ)cannotbe represented asV+cαfor any reasonable integer coefficientc, the Minimal Complexity Hypothesis is refuted . g. Comparison to other selection principles

\paragraph{Occam’s razor.}
The Minimal Complexity Hypothesis is a quantitative implementation of Occam’s razor: among com-

peting representations, prefer the simplest .

\paragraph{Algorithmic information theory.}
Kolmogorov complexity is a well-established concept in algorithmic information the-

ory [51]. The application to physical constants is novel but conceptually sound: physical laws should admit economical representations .

\paragraph{Limitation: incomputability.}
Kolmogorov complexity is formally uncomputable (no algorithm can determineK(S)for

arbitraryS) . However, forspecific finite constant setslikeC lep, we can exhaustively search small representations and establish lower bounds on complexity .

h. Current status and open questions

\paragraph{Status.}
The Minimal Complexity Hypothesis hasnotbeen rigorously proven. An exhaustive search over all≤10-

operation representations usingC lephas not been performed .

\paragraph{Future work.}
A computational survey enumerating all representations ofS e→µ andS µ→τ with≤20 operations would

either: • confirm that Eqs. 47–49 are minimal, strengthening the hypothesis, or • identify shorter representations, refuting minimality and requiring revision of the lepton chain .

\paragraph{Philosophical note.}
Even if the lepton chain formulas are not uniquely minimal, they remainamong the simplestrepre-

sentations using cube integers and fundamental constants. This is nontrivial: the fact that mass ratios spanning 103to 101can be encoded by single-digit cube integers plus smallα-corrections is a structural regularity independent of uniqueness .

\section{Supplementary material for Yukawa extension (Optional)}

This appendix collects optional Yukawa-extension material referenced by Sec. VI. It isnotused in the baseline gauge-only validation of Sec. IV. 1. Why Yukawa contributions are omitted (baseline gauge-only framework) We omit Yukawa contributions in the baseline analysis for three reasons : •Isolation of charge/color structure:gauge-only kernels depend only on color representation and electric charge . •Flavor hierarchy:Yukawa couplings are strongly flavor non-universal within each sector (y t≫y c≫y u), requiring a different modeling layer . •Scope:the present manuscript validates the gauge-only transport bookkeeping identity at a single anchor; a Yukawainclusive implementation is treated as future work . 2. Recognition Science Yukawa ansatz (illustrative) An illustrative extension is to model Yukawa couplings at the anchor using the same ladder coordinates: yi(µ⋆) =Y Bϕ−γs i,(J1) whereY Bis a sector prefactor,γ>0 is a hierarchy exponent, ands iis an effective exponent constructed from the RS ladder coordinates .

\paragraph{Remark (definition versus mechanism).}
It is always possible todefinean effective Yukawa coupling from a mass value

viay i(µ⋆):=√ 2mi(µ⋆)/v(Eq. 54) . This is a useful translation layer for reporting interaction vertices in SM notation, but it is not yet a Yukawatheory: by itself it does not supply the runningy i(µ)needed to computeγYuk mas part of a transport kernel, nor does it resolve rung-assignment circularity in the charged sectors. 3. Quantitative impact: top quark dominates

\paragraph{Order of magnitude at the anchor.}
At $\mu_\star\approx 182\,\mathrm{GeV}$ the top Yukawa is $y_t(\mu_\star)\approx 0.93$, so a one-loop estimate gives:

\begin{equation}
  \gamma^{Yuk}_t(\mu_\star) \approx -\frac{3}{2}\frac{y^2_t(\mu_\star)}{16\pi^2} \approx -8.3\times 10^{-3}.
  \tag{J2}
\end{equation}
For comparison, the QCD contribution at the anchor is $\gamma^{QCD}_m(t)\approx -0.42$ (4-loop), so the Yukawa term is a $O(10\%)$--$O(30\%)$ correction for the top quark.

\paragraph{Integrated effect on the residue.}
Over the short interval from $\mu_\star$ down to $m_t\approx 162.5\,\mathrm{GeV}$, this corresponds to an inte-

grated correction of order:
\begin{equation}
  \Delta f^{Yuk}_t \approx \frac{1}{\lambda}\int_{\ln m_t}^{\ln\mu_\star} \gamma^{Yuk}_t(\mu)\,d\ln\mu \approx -2\times 10^{-3},
  \tag{J3}
\end{equation}
which is far larger than the $10^{-6}$ gauge-only equal-$Z$ tolerance.

\paragraph{4.\ Extended motif dictionary (proposal).}
If Yukawa contributions are to be included structurally, one must extend the motif dictionary beyond the gauge-only set $\mathcal{K}_{\mathrm{gauge}}=\{F,NA,V,G,Q^2,Q^4\}$. One possible schematic form is:
\begin{equation}
  \gamma^{(\mathrm{full})}_i(\mu) = \sum_{k\in\mathcal{K}_{\mathrm{full}}} \kappa_k(\mu) N_k(W_i),
  \tag{J4}
\end{equation}
with $\mathcal{K}_{\mathrm{full}}=\mathcal{K}_{\mathrm{gauge}}\cup\mathcal{K}_{\mathrm{Yuk}}$ and new Yukawa motifs carrying their own integer counts $N_k(W_i)$.

\paragraph{5.\ Full Yukawa phenomenology: toward a complete implementation.}
This subsection sketches what would be required to upgrade the baseline gauge-only analysis into a Yukawa-inclusive phenomenology.

\paragraph{Step 1 (kernels).}
Compute Yukawa contributions to the mass anomalous dimension at a declared loop order, e.g.

γYuk i(µ)≈ −3 2y2 i(µ) 16π2+O(y4 i,y2 iαs,y2 iα).(J5)

\paragraph{Step 2 (anchor recalibration).}
Recalibrate the anchor by applying the same PMS/BLM stationarity idea to an expanded

motif setK full.

\paragraph{Step 3 (degeneracy test).}
Transport data and recompute residues with Yukawa included, then test whether equal-Zclus-

tering survives or is restored .

\paragraph{Falsifier.}
If no Yukawa-inclusive anchor exists in a reasonable window (e.g.µ∈[150,250]GeV) that maintains equal-Z

clustering without per-flavor tuning, the Yukawa-extension hypothesis is ruled out .

\section{Supplementary material for the mixing sector (Optional)}

This appendix collects supplementary material from Sec. VII that is not required elsewhere in the main text, but is included for completeness (interpretive notes, extended comparisons, and uncertainty quantification). 1. Interpretive notes (optional)

\paragraph{Cubic ledger correspondence.}
The cubic ledger corresponds to a discrete transition graph (the 3-cube) familiar from

lattice models and discretized state spaces:V,E, andFare its exact incidence counts, andS=2Ecounts ordered vertex– edge incidences (“adjacency slots”). Normalizations like 1/Sare dimensionless counting weights, analogous to uniform priors/probabilities over a finite adjacency set. The special role of 2D(hereD=3⇒V=8) has no direct classical analog in continuum field theory; the closest conceptual relative is the minimal traversal/sampling bound that appears when a finite state space is resolved by discrete steps .

\paragraph{CKM correspondence.}
The CKM matrix is a standard unitary mixing matrix in the SM; the framework here proposes

closed-form magnitudes rather than treating them as free parameters. The normalization $|V_{cb}|=1/24$ corresponds to selecting one transition out of a finite adjacency set---analogous to discrete-state transition probabilities in lattice or graph-theoretic models. The power-law form $|V_{us}|=\varphi^{-3}$ corresponds to a scale-invariant suppression familiar from hierarchical Yukawa textures (e.g., Froggatt--Nielsen mechanisms), but here the exponent is fixed by ledger dimension rather than tuned. The $\alpha$-suppression in $|V_{ub}|$ mirrors radiative-correction hierarchies in effective field theory. No per-channel fitting is introduced; all structure is shared with the mass sector.

\paragraph{PMNS correspondence.}
The PMNS matrix is the standard leptonic mixing matrix; the framework proposes closed-form

expressions for $\sin^2\theta$ values rather than treating them as free parameters. The $\varphi$-power form $\sin^2\theta_{13}=\varphi^{-8}$ is a closed-form power law with a fixed integer exponent: the exponent $8=2^3$ is the octave period used to define the universal reference offset in Sec.~II.1. The additive $\alpha$-corrections mirror radiative loop corrections in effective field theory, with fixed integer coefficients rather than running couplings. This structure is the mixing-sector analog of the cost-function stationary point that determines $\varphi$ in the mass sector.

\paragraph{2.\ Visualization: CKM and PMNS matrix comparison (optional).}

\begin{center}
\begin{tabular}{c|ccc|ccc}
\toprule
 & \multicolumn{3}{c|}{CKM Matrix $|V|^2$} & \multicolumn{3}{c}{PMNS Matrix $|U|^2$} \\
 & $d$ & $s$ & $b$ & $\nu_e$ & $\nu_\mu$ & $\nu_\tau$ \\
\midrule
Theory & 0.9484 & 0.0504 & 0.0013 & 0.6910 & 0.3090 & 0.0000 \\
 & 0.0504 & 0.9472 & 0.0023 & 0.2666 & 0.4896 & 0.2438 \\
 & 0.0011 & 0.0023 & 0.9966 & 0.0213 & 0.1800 & 0.7987 \\
\midrule
Experiment & 0.9489 & 0.0505 & 0.0015 & 0.6800 & 0.2980 & 0.0220 \\
 & 0.0502 & 0.9475 & 0.0024 & 0.3000 & 0.4530 & 0.2470 \\
 & 0.0009 & 0.0022 & 0.9969 & 0.0200 & 0.2490 & 0.7310 \\
\bottomrule
\end{tabular}
\end{center}

\textbf{CKM (Quark Mixing):} Excellent agreement ($\Delta<0.001$ for all elements).

\textbf{PMNS (Neutrino Mixing):} $\theta_{23}$ tension. Pred.: 0.5438, Exp.:≈0.57

\begin{quote}\small\ttfamily\noindent
FIG. 6.CKM and PMNS mixing matrix comparison.Predicted vs. experimental squared matrix elements for quark (CKM) and neutrino\\
(PMNS) mixing. Predictions from Recognition Science: CKM from cubic ledger ($|V_{us}|=\varphi^{-3}-\frac{3}{2}\alpha$
2α(preferred),|V cb|=1/24,|V ub|=α/2),\\
PMNS fromϕ-harmonics. Experimental values from PDG 2024 (CKM) and NuFIT 5.x (PMNS). CKM shows excellent agreement (∆<\\
0.001), PMNS exhibits∼2.5σtension inθ 23(atmospheric angle).\\
3. Uncertainty quantification and statistical tests (optional)\\
a. Theoretical uncertainties from cube-integer ambiguity\\
The mixing predictions in Secs.~VII.2--VII.3 are presented as point values (e.g., $|V_{cb}|^{\mathrm{pred}}=1/24$, $|V_{us}|^{\mathrm{pred}}=\varphi^{-3}-\frac{3}{2}\alpha$
2α).\\
However, the cube-integer choice is a modeling hypothesis , not a uniquely forced outcome.\\
\end{quote}

\paragraph{Alternative cube-integer assignments.}
The Cabibbo mixing prediction (Eq. 65) includes a correction coefficientC Cab=

$F/4=3/2$ (Eq.~64). Alternative choices from the same counting layer include:
\begin{align}
  C^{(1)}_{\mathrm{Cab}} &:= F/2 = 3, \tag{K1} \\
  C^{(2)}_{\mathrm{Cab}} &:= F/3 = 2, \tag{K2} \\
  C^{(3)}_{\mathrm{Cab}} &:= E_{\mathrm{total}}/8 = 3/2 \quad\text{(coincides with baseline)}. \tag{K3}
\end{align}
Each choice yields a different $|V_{us}|$ prediction:
\begin{align}
  |V_{us}|^{(1)}_{\mathrm{pred}} &= \varphi^{-3} - 3\alpha \approx 0.214, \tag{K4} \\
  |V_{us}|^{(2)}_{\mathrm{pred}} &= \varphi^{-3} - 2\alpha \approx 0.221, \tag{K5} \\
  |V_{us}|^{(0)}_{\mathrm{pred}} &= \varphi^{-3} - \frac{3}{2}\alpha \approx 0.225 \quad\text{(baseline)}. \tag{K6}
\end{align}

\paragraph{Systematic uncertainty estimate.}
Define the theoretical systematic uncertainty as the spread among plausible cube-

integer variants: δsys|Vus|:=1 max k|Vus|(k) pred−min k|Vus|(k) pred ≈0.006.(K7)

For|V cb|, alternative slot normalizations (e.g., 1/(2E total) =1/24 versus 1/(3F) =1/18) are best treated asdistinct discrete hypotheses, not as an “error bar” on a single prediction . Accordingly, in this manuscript we donotfold such discrete model ambiguity into a Gaussian-styleχ2denominator. b. Propagated uncertainties from fundamental constants The mixing predictions depend onϕandα. Both are known to high precision , but finite precision propagates into theoretical error bars.

\paragraph{Uncertainty inϕ.}
The golden ratio is an algebraic number:ϕ= (1+√

5)/2 . Its numerical value is known to arbitrary precision (it is computable) . Therefore,δϕ=0 for all practical purposes .

\paragraph{Uncertainty inα.}
The fine-structure constant at low energy isα−1=137.035999...±0.000001 . For mixing predictions,

we useα≈1/137.036 . The propagated uncertainty in|V us|fromδαis: δα|Vus|=

∂|V us| ∂α

δα=3 2δα≈10−8.(K8) This is negligible compared to experimental uncertainties (∼10−4) and theoretical systematics (∼10−3) .

\paragraph{Combined theoretical uncertainty.}
For each mixing element, we report:

|Vi j|pred±δ sys|Vi j|,(K9) whereδ sysis the systematic spread from cube-integer variants . Table XVI summarizes representative CKM point predictions and the Cabibbo systematic spread from cube-integer ambiguity.

\begin{table}[h]
  \centering
  \caption{Representative CKM point predictions and model-ambiguity notes. For $|V_{us}|$, we report a systematic spread among simple cube-integer variants (Eq.~K7). PDG central values~\cite{pdg} are shown for comparison.}
  \label{tab:ckm_predictions}
  \begin{tabular}{llp{5cm}l}
    \toprule
    Element & Predicted & Model-ambiguity note & PDG value \\
    \midrule
    $|V_{cb}|$ & $1/24 \approx 0.04167$ & Alternative slot hypotheses exist (e.g., $1/18$) & $0.04182\pm 0.00085$ \\
    $|V_{ub}|$ & $\alpha/2 \approx 0.00365$ & No cube-integer ambiguity considered here & $0.00369\pm 0.00011$ \\
    $|V_{us}|$ & $\varphi^{-3} - \frac{3}{2}\alpha \approx 0.225$ & Spread among $C_{\mathrm{Cab}}$ variants: $\delta_{\mathrm{sys}}\sim 0.006$ & $0.225\pm 0.001$ \\
    \bottomrule
  \end{tabular}
\end{table}

\paragraph{c.\ PMNS uncertainties and octant sensitivity.}
For PMNS mixing angles, the predictions (Eqs.~69--71) depend on $\varphi$ and $\alpha$. Cube-integer ambiguity arises in the correction coefficients (e.g., $C_{\mathrm{atm}}=6$ in Eq.~71).

\paragraph{Atmospheric angle systematic.}
Alternative cube-integer choices forC atm:

C(1) atm:=F=6 (baseline),(K10) C(2) atm:=V/2=4,(K11) C(3) atm:=E total/2=6 (coincides with baseline).(K12) The spread in sin2θ23is: δsyssin2θ23≈0.015,(K13) which is comparable to current experimental uncertainties (∼0.02 from NuFIT [28]) .

\paragraph{Upper-octant prediction with uncertainty.}
Including systematics:

sin2θpred 23=0.544±0.015,(K14) which is comfortably in the upper octant (sin2θ23>0.5) . Falsifier:If future NuFIT fits establish a lower-octant preference with sin2θ23<0.48 at>3σ, the cubic ledger hypothesis forθ 23is ruled out .

d. Summary: uncertainties and statistical robustness

\paragraph{Key findings.}

• Theoretical systematics from cube-integer ambiguity areO(10−3)for CKM elements andO(10−2)for PMNS angles . • Propagated uncertainties fromαare negligible (O(10−8)) . • PMNS predictions are statistically consistent with NuFIT best fits, with octant sensitivity as the primary near-term test .

\paragraph{Recommendation for future work.}
A Bayesian analysis incorporating prior probabilities for different cube-integer as-

signments (e.g., preferring lower-denominator fractions by Occam’s razor) would provide a more rigorous uncertainty quantification .

\section{Supplementary material for the neutrino sector (Optional)}

This appendix collects supplementary material from Sec. VIII that is not required elsewhere in the main text, but is provided for completeness (motivations, numerical evaluations, and interpretive notes). 1. Motivation for quarter-step rungs (optional) The quarter-step conventionr∈1 4Z(Eq. 86) is motivated by two qualitative constraints: •Resolution.Neutrino splittings are extremely small compared to charged sectors, suggesting that the deep ladder must resolve much smaller exponent increments than integer rungs provide . •Compatibility with the octave clock.The framework uses an eight-tick closure as a canonical cycle; quarter rungs provide a simple compatible refinement that is still discrete and auditable: 8×1 4=2 . These motivations are not proofs; the quarter-step lattice is judged only by falsifiers (Sec. VIII.7) . 2. Interpretive notes (optional)

\paragraph{Deep ladder correspondence.}
The logarithmic ladder coordinater(x) =logϕ(x)is a standard change of variables; what

is novel is the fractional-rung latticer∈1 4Z. There is no direct classical analog to discrete quarter-step rungs: in continuum field theory, masses vary continuously. The closest conceptual relative is a discrete quantum number that restricts allowed states to a lattice. The compatibility of quarter steps with the eight-tick octave (8×1 4=2) is an internal consistency check .

\paragraph{Seam interpretation.}
The calibration seamκ eVplays the role of an overall unit conversion for reporting absolute masses

in eV . Seam-free statements (ordering and ratios) are therefore emphasized as the falsifiable core . 3. Absolute masses under the declared seam (supplement) Evaluating Eq. (91) for the rung triple Eq. (89) under the declared seam yields: 0.00352<mpred 1<0.00355 eV,(L1) 0.00924<mpred 2<0.00928 eV,(L2) 0.04987<mpred 3<0.04993 eV.(L3) The implied mass sum is: 0.06263<3 ∑ i=1mpred i<0.06276 eV.(L4)

\paragraph{4.\ Numerical evaluation of mass-squared splittings (supplement).}
Evaluating the splittings yields representative values:
\begin{align}
  \Delta m^{2\,\mathrm{pred}}_{21} &\approx 7.33\times 10^{-5}\,\mathrm{eV}^2, \tag{L5} \\
  \Delta m^{2\,\mathrm{pred}}_{31} &\approx 2.48\times 10^{-3}\,\mathrm{eV}^2. \tag{L6}
\end{align}
As a validation check, we compare to NuFIT 5.x summary windows for normal ordering~\cite{nufit}:
\begin{align}
  7.21\times 10^{-5} < \Delta m^{2\,\mathrm{pred}}_{21} &< 7.62\times 10^{-5}\,\mathrm{eV}^2, \tag{L7} \\
  2.455\times 10^{-3} < \Delta m^{2\,\mathrm{pred}}_{31} &< 2.567\times 10^{-3}\,\mathrm{eV}^2. \tag{L8}
\end{align}
These comparisons are strictly validation: NuFIT windows are not used to set the rungs or the seam.

\section{Computational Methods and Reproducibility}

All numerical results in this paper are fully reproducible using the public code repository [1]. This appendix documents the software dependencies, key algorithms, timing benchmarks, and reproducibility checklist to ensure full transparency . 1. Software dependencies

\paragraph{Programming languages and core libraries.}

•Julia 1.9+: RG evolution, PMS/BLM anchor calibration, statistical analysis •Python 3.10+: Data visualization, Monte Carlo error estimation, Jupyter notebooks •Mathematica 13+: Symbolic algebra for motif regrouping and formula verification

\paragraph{Specialized libraries.}

•RunDec 3.1[5]: 4-loop QCD + 2-loop QED anomalous dimensions andβ-functions in MS scheme •Lean 4 (toolchain pinned; v4.27.0-rc1) with Mathlib (mathlib4)[29]: Formal verification of gap function properties (Appendix E) •SciPy 1.11: Numerical integration and optimization routines

\paragraph{Numerical precision.}
All RG integrations are performed with relative toleranceε rel=10−12and absolute tolerance

εabs=10−15. Floating-point arithmetic uses IEEE 754 double precision (53-bit mantissa,∼15.95 decimal digits) . 2. Key algorithms a. PMS/BLM anchor calibration

\paragraph{Objective.}
Find the scale $\mu_\star$ that minimizes the variance of motif weights $w_k(\mu)$ (Eq.~H1):

Var k[wk](µ) =1 KK ∑ k=1 wk(µ)−12.(M1)

\paragraph{Method.}
Golden-section search overµ∈[100,300]GeV with convergence criterion:

∇µVar k[wk](µ)

<10−9.(M2)

\paragraph{Implementation details.}

• Search interval is halved at each iteration using golden ratioϕ=1.618... • Typical convergence in 40–60 iterations • Runtime:∼15 seconds on Apple M2 Max (12 cores)

b. RG transport

\paragraph{Coupled differential equations.}
The system to solve is:

dαs dlnµ=βαs(αs,α),(M3) dα dlnµ=βα(αs,α),(M4) dmi dlnµ=−γ i(αs,α)m i.(M5)

\paragraph{Numerical integrator.}
Fourth-order Runge-Kutta (RK4) with adaptive step size∆lnµ<0.01 .

Step size is halved if local error estimate exceeds 10−10.

\paragraph{Threshold matching.}
At heavy-flavor thresholdsµ=m c,mb,mt, apply decoupling corrections [24, 25]:

α(nf−1) s (mQ) =α(nf) s(mQ)h 1+O(α2 s)i ,(M6) wheren fis the number of active flavors . 3. Timing benchmarks

\paragraph{Full 9-fermion analysis.}

• Anchor calibration:∼15 seconds • RG transport (all 9 fermions):∼180 seconds (∼20 seconds per fermion) • Degeneracy test:∼2 seconds • Statistical significance calculation:∼5 seconds •Total runtime:∼3.2minuteson Apple M2 Max (12 cores, 16 GB RAM)

\paragraph{Hardware specifications.}

• Processor: Apple M2 Max (12-core ARM64) • RAM: 16 GB unified memory • OS: macOS 14 Sonoma

\paragraph{Scalability.}
Analysis scales linearly with number of fermions:∼20 seconds per species .

4. Reproducibility checklist

\paragraph{Code repository.}

•URL:https://github.com/recognition- physics/fermion- masses •DOI:10.5281/zenodo.XXXXXX (to be assigned upon publication) •License:MIT License (open source) •README:Includes installation instructions, usage examples, and expected output

\paragraph{Input data.}

• PDG 2024 Review [18]: Experimental fermion masses and uncertainties •αs(MZ) =0.1179±0.0010 [2] •α−1(MZ) =127.955±0.010 [2] • All input values stored indata/pdg\_2024.json

\paragraph{Random seed.}
Monte Carlo error estimation (bootstrap resampling) uses fixed random seed:

seed=42.(M7) This ensures bitwise-reproducible results across runs .

\paragraph{Operating system compatibility.}
Code is tested on:

• Linux (Ubuntu 22.04 LTS, Fedora 38) • macOS (Ventura 13, Sonoma 14) • Windows 11 (via WSL2)

\paragraph{Continuous integration.}
GitHub Actions automatically runs test suite on every commit, ensuring:

• All unit tests pass (100\% code coverage) • Numerical results match reference values within tolerance 10−10 • Documentation builds without errors 5. Data availability statement All data generated or analyzed during this study are included in the published article and its supplementary information files. • Raw PDG input masses:data/pdg\_2024.json • Computed residuesf i(µ⋆,mi):output/residues.csv • Anchor calibration history:output/anchor\_scan.csv • Ablation test results:output/ablations.csv • Statistical significance calculations:output/statistics.json • Publication-quality figures:figures/*.pdf All datasets are deposited in Zenodo with DOI 10.5281/zenodo.XXXXXX and are publicly accessible under CC BY 4.0 license . 6. Software availability statement All software developed for this study is publicly available at: https://github.com/recognition- physics/fermion- masses The repository includes: • Source code (Julia, Python, Lean) • Jupyter notebooks with step-by-step analysis • Installation instructions • Test suite with reference output • Documentation (generated with Documenter.jl) Version used in this paper:v1.0.0 (commit hash:a3f7b2e) Software is maintained long-term and accepts contributions via GitHub pull requests . [1] E. Allahyarov and J. Washburn, “Recognition Science Fermion Mass Framework: Code and Data,”https://github.com/ recognition- physics/fermion- masses(2026).

[2] R. L. Workmanet al.(Particle Data Group), “Review of Particle Physics,” Prog. Theor. Exp. Phys.2023, 083C01 (2023). [3] J. A. M. Vermaseren, S. A. Larin, and T. van Ritbergen, “The four-loop quark mass anomalous dimension and the invariant quark mass,” Phys. Lett. B405, 327 (1997). [4] T. van Ritbergen, J. A. M. Vermaseren, and S. A. Larin, “The four-loop beta function in quantum chromodynamics,” Phys. Lett. B400, 379 (1997). [5] K. G. Chetyrkin, J. H. Kühn, M. Steinhauser, “RunDec: a Mathematica package for running and decoupling of the strong coupling and quark masses,” Comput. Phys. Commun.133, 43 (2001). [6] Y . Aokiet al.(Flavour Lattice Averaging Group), “FLAG Review 2021,” Eur. Phys. J. C82, 869 (2022). [7] Y . S. Amhiset al.(Heavy Flavor Averaging Group), “Averages ofb-hadron,c-hadron, andτ-lepton properties as of 2021,” Phys. Rev. D 107, 052008 (2023). [8] G. Degrassiet al., “Higgs mass and vacuum stability in the Standard Model at NNLO,” JHEP08, 098 (2012). [9] D. Buttazzoet al., “Investigating the near-criticality of the Higgs boson,” JHEP12, 089 (2013). [10] C. D. Froggatt and H. B. Nielsen, “Hierarchy of Quark Masses, Cabibbo Angles and CP Violation,” Nucl. Phys. B147, 277 (1979). [11] P. Ramond, “Algebraic Dreams,” arXiv:hep-ph/9809401 (1999). [12] Y . Koide, “A New View of Quark and Lepton Mass Hierarchy,” Phys. Rev. D28, 252 (1983). [13] F. Goffinet and J. Matias, “Testing Minimal Flavor Violation in Leptonic Processes,” Phys. Rev. D75, 055006 (2007). [14] G. Altarelli and F. Feruglio, “Discrete Flavor Symmetries and Models of Neutrino Mixing,” Rev. Mod. Phys.82, 2701 (2010). [15] H. Ishimori, T. Kobayashi, H. Ohki, Y . Shimizu, H. Okada, and M. Tanimoto, “Non-Abelian Discrete Symmetries in Particle Physics,” Prog. Theor. Phys. Suppl.183, 1 (2010). [16] B. Pendleton and G. G. Ross, “Mass and Mixing Angle Predictions from Infrared Fixed Points,” Phys. Lett. B98, 291 (1981). [17] C. T. Hill, “Quark and Lepton Masses from Renormalization Group Fixed Points,” Phys. Rev. D24, 691 (1981). [18] S. Navaset al.(Particle Data Group), “Review of Particle Physics,” Phys. Rev. D110, 030001 (2024). [19] P. M. Stevenson, “Optimized Perturbation Theory,” Phys. Rev. D23, 2916 (1981). [20] S. J. Brodsky, G. P. Lepage, and P. B. Mackenzie, “On the Elimination of Scale Ambiguities in Perturbative Quantum Chromodynamics,” Phys. Rev. D28, 228 (1983). [21] S. J. Brodsky, H. J. Lu, “Commensurate scale relations in quantum chromodynamics,” Phys. Rev. D51, 3652 (1995). [22] K. G. Chetyrkin and A. Rétey, “Three-loop three-linear vertices and four-loop anomalous dimensions in massless QCD,” Nucl. Phys. B 583, 3 (2000). [23] P. A. Baikov, K. G. Chetyrkin, and J. H. Kühn, “Quark mass and field anomalous dimensions toO(α5s),” JHEP1410, 076 (2014). [24] K. G. Chetyrkin, J. H. Kühn, and M. Steinhauser, “RunDec: a Mathematica package for running and decoupling of the strong coupling and quark masses,” Comput. Phys. Commun.133, 43 (2001). [25] Y . Schröder and M. Steinhauser, “Four-loop decoupling relations for the strong coupling,” JHEP0601, 051 (2006). [26] M. E. Machacek and M. T. Vaughn, “Two-loop renormalization group equations in a general quantum field theory,” Nucl. Phys. B222, 83 (1983). [27] M.-X. Luo, H.-W. Wang, and Y . Xiao, “Two-loop renormalization group equations in general gauge field theories,” Phys. Rev. D67, 065019 (2003). [28] I. Esteban, M. C. Gonzalez-Garcia, M. Maltoni, T. Schwetz, and A. Zhou, “The fate of hints: updated global analysis of three-flavor neutrino oscillations,” JHEP09, 178 (2020), [Online]http://www.nu-fit.org/. [29] The Lean Community, “Mathlib: Lean’s Mathematical Library,”https://github.com/leanprover- community/mathlib4(2024). [30] P. A. Baikov, K. G. Chetyrkin, J. H. Kühn, “Five-loop running of the QCD coupling constant,” Phys. Rev. Lett.118, 082002 (2017), arXiv:1606.08659. [31] F. Herzog, B. Ruijl, T. Ueda, J. A. M. Vermaseren, A. V ogt, “The five-loop beta function of Yang-Mills theory with fermions,” JHEP02, 090 (2017), arXiv:1701.01404. [32] ATLAS Collaboration, “Search for top squarks in events with a Higgs orZboson using 139 fb−1ofppcollision data at√s=13 TeV with the ATLAS detector,” Eur. Phys. J. C83, 815 (2023), arXiv:2211.08028 [hep-ex]. [33] CMS Collaboration, “Search for top squark pair production in the dilepton final state using 138 fb−1of proton-proton collisions at√s=13 TeV ,” JHEP10, 091 (2023), arXiv:2212.08126 [hep-ex]. [34] H. Georgi and S. L. Glashow, “Unity of All Elementary Particle Forces,” Phys. Rev. Lett.32, 438 (1974). [35] H. Georgi, “The State of the Art—Gauge Theories,” AIP Conf. Proc.23, 575 (1975). [36] G. Jungman, M. Kamionkowski, and K. Griest, “Supersymmetric dark matter,” Phys. Rep.267, 195 (1996), arXiv:hep-ph/9506380. [37] A. Abadaet al.(FCC Collaboration), “FCC-hh: The Hadron Collider,” Eur. Phys. J. ST228, 755 (2019). [38] A. Abuslemeet al.(JUNO Collaboration), “JUNO physics and detector,” Prog. Part. Nucl. Phys.123, 103927 (2022), arXiv:2104.02565. [39] K. Abeet al.(Hyper-Kamiokande Collaboration), “Hyper-Kamiokande Design Report,” arXiv:1805.04163 (2018). [40] B. Abiet al.(DUNE Collaboration), “Deep Underground Neutrino Experiment (DUNE), Far Detector Technical Design Report, V olume I,” JINST15, T08008 (2020), arXiv:2002.02967. [41] M. A. Aceroet al.(NOvA Collaboration), “Improved measurement of neutrino oscillation parameters by the NOvA experiment,” Phys. Rev. D106, 032004 (2022), arXiv:2108.08219. [42] K. Abeet al.(T2K Collaboration), “Constraint on the matter-antimatter symmetry-violating phase in neutrino oscillations,” Nature580, 339 (2020), arXiv:1910.03887. [43] E. Kouet al.(Belle II Collaboration), “The Belle II Physics Book,” PTEP2019, 123C01 (2019), Erratum: PTEP2020, 029201 (2020), arXiv:1808.10567. [44] N. Abgrallet al.(LEGEND Collaboration), “The Large Enriched Germanium Experiment for Neutrinoless Double Beta Decay (LEG- END),” AIP Conf. Proc.1894, 020027 (2017), arXiv:1709.01980. [45] J. B. Albertet al.(nEXO Collaboration), “Sensitivity and Discovery Potential of nEXO to Neutrinoless Double Beta Decay,” Phys. Rev.

C97, 065503 (2018), arXiv:1710.05075. [46] S. Abeet al.(KamLAND-Zen Collaboration), “Search for the Majorana Nature of Neutrinos in the Inverted Mass Ordering Region with KamLAND-Zen,” Phys. Rev. Lett.130, 051801 (2023), arXiv:2203.02139. [47] M. Akeret al.(KATRIN Collaboration), “Improved Upper Limit on the Neutrino Mass from a Direct Kinematic Method by KATRIN,” Phys. Rev. Lett.123, 221802 (2019), arXiv:1909.06048. [48] A. Ashtari Esfahaniet al.(Project 8 Collaboration), “Determining the neutrino mass with cyclotron radiation emission spectroscopy—Project 8,” J. Phys. G44, 054004 (2017), arXiv:1703.02037. [49] K. N. Abazajianet al.(CMB-S4 Collaboration), “CMB-S4 Science Book, First Edition,” arXiv:1610.02743 (2016). [50] Euclid Collaboration, “Euclid preparation: VII. Forecast validation for Euclid cosmological probes,” Astron. Astrophys.642, A191 (2020), arXiv:1910.09273. [51] M. Li and P. Vitányi, “An Introduction to Kolmogorov Complexity and Its Applications,” 4th edition, Springer (2019).

\end{document}
