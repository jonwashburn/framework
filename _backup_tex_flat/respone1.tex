\documentclass[11pt]{article}
\usepackage{amsmath,amssymb,amsthm,mathtools}
\usepackage[margin=1in]{geometry}

\newtheorem{definition}{Definition}
\newtheorem{lemma}{Lemma}
\newtheorem{proposition}{Proposition}
\newtheorem{theorem}{Theorem}
\theoremstyle{remark}
\newtheorem{remark}{Remark}

\title{
\large Reply: Why ``Structural Mechanism'' Does Not Resolve Non-Identifiability}
\author{}
\date{}

\begin{document}
\maketitle

The response under critique concedes the purely algebraic point that many expressions evaluate to
the same numerical constant (e.g.\ $1.5$ can be written as $F/4$, $E/8$, $D/2$, etc.). It then claims
this objection is irrelevant because a \emph{mechanism} uniquely forces one form:
\[
\Delta(3) = \frac{F}{V} = \frac{6}{4} = 1.5,
\]
where $F$ is the number of cube faces and $V$ is the number of vertices of a face, interpreted as a
``discrete solid angle'' normalization. The response asserts that alternative expressions are ruled out
because they allegedly correspond to the wrong ``mechanism'' (edge-mediated vs face-mediated, etc.).

This reply fails as a resolution for three independent reasons:
\begin{enumerate}
\item It introduces new principles (``Inverse Measure Rule,'' ``Discrete--Continuous Duality,'' ``discrete solid angle = vertex count'') without derivation and then uses them as if they were theorems. This is not a derivation; it is a new hypothesis layer.
\item Even if these principles are granted, the coefficient remains \emph{non-identifiable}: multiple \emph{distinct} count/measure mechanism pairs produce the same value $1.5$ using the same counting-layer data. Therefore the mechanism-to-number map is not injective.
\item The proposed ``mechanism'' is not uniquely specified even internally: in the cube case the integer $4$ is simultaneously (i) vertices-per-face and (ii) edges-per-face, so the chosen ``discrete measure'' is not uniquely determined.
\end{enumerate}

\subsection*{1. Minimal criterion for a resolution?}

\begin{definition}[Resolution of the ``infinite formulas'' objection]
Let $\mathcal{M}$ be the set of admissible model mechanisms and let $g:\mathcal{M}\to\mathbb{R}$ map each
mechanism to the predicted coefficient (here, the tau-step correction coefficient or sub-coefficient).
A genuine resolution must supply:
\begin{enumerate}
\item a \emph{precise definition} of $\mathcal{M}$ (not narrative labels),
\item a \emph{precise rule} $g$ (not analogy),
\item and a \emph{uniqueness theorem} showing $g$ is injective on admissible mechanisms for the data used.
\end{enumerate}
Absent injectivity, ``uniqueness of derivation'' is false in a literal mathematical sense.
\end{definition}

\subsection*{2. Failure 1: New axioms are asserted, not derived}

The response introduces:
\begin{itemize}
\item ``Inverse Measure Rule: Contribution $=$ Count / Measure,''
\item ``Discrete--Continuous Duality Principle,'' and
\item ``discrete solid angle of a facet equals its vertex count $V$.'' 
\end{itemize}

\underline{Postulating a selector is not a derivation:}
If the only reason a formula is ``unique'' is that one postulates an additional selector rule that
excludes other formulas, then uniqueness is merely conditional on that selector and not a derived fact.

Proof:
By definition, a derivation must show:
\[
\text{(prior axioms)}\ \Rightarrow\ \Delta(3)=F/V.
\]
But if instead one states new principles designed to make $\Delta(3)$ equal $F/V$, then the actual logical structure is:
\[
\text{(prior axioms)} + \text{(new selector postulate)}\ \Rightarrow\ \Delta(3)=F/V.
\]
This is hypothesis revision, not derivation. The critique was precisely about freedom in choosing the
form of the correction; moving that freedom into a new selector postulate does not remove it.

\begin{remark}
A typical fallback is ``the selector rule is physically motivated.'' That is not enough.
One must still prove that the selector is \emph{forced} by the framework, or provide an independent falsifier.
Otherwise it remains an unconstrained degree of freedom.
\end{remark}

\subsection*{3. Failure 2: Even granting the rule, the mechanism is not unique (explicit counterexample)}

The response argues: ``Only $F/V$ respects the face-mediated mechanism; alternatives like $E/8$ imply
edge mediation and are excluded.'' This is false even on its own terms because the same numerical
correction arises from distinct Count/Measure pairs using the same cube data.

\underline{Concrete degeneracy under Count/Measure}
Using only cube counts:
\[
F=6,\quad E=12,\quad V_{\mathrm{cube}}=8,\quad V_{\mathrm{face}}=4,
\]
the value $1.5$ arises from at least two distinct Count/Measure pairs:
\[
\frac{F}{V_{\mathrm{face}}} = \frac{6}{4} = 1.5
\qquad\text{and}\qquad
\frac{E}{V_{\mathrm{cube}}} = \frac{12}{8} = 1.5.
\]


Proof:
Compute:
\[
\frac{6}{4}=\frac{3}{2}=1.5,\qquad
\frac{12}{8}=\frac{3}{2}=1.5.
\]
These are distinct mechanism candidates if one takes the response's own narrative seriously:
``faces distributed over face-anchors'' versus ``edges distributed over vertex-anchors.'' 
Both satisfy the same abstract template \emph{Contribution = Count / Measure}.


\underline{Non-identifiability persists under the proposed mechanism template:}
Under the response's own template class
\[
\Delta = \frac{\text{Count}}{\text{Measure}},
\]
the tau-step correction is \emph{not identifiable} from cube data because there exist multiple distinct
(Count, Measure) pairs producing the same value.


Proof:
Identifiability would require uniqueness of the generating pair:
\[
\frac{c_1}{m_1}=\frac{c_2}{m_2} \implies (c_1,m_1)=(c_2,m_2)
\quad \text{for admissible pairs.}
\]
But the lemma exhibits $(c_1,m_1)=(F,V_{\mathrm{face}})$ and $(c_2,m_2)=(E,V_{\mathrm{cube}})$
with equal ratios and unequal pairs. Therefore the mapping is not injective, hence not identifiable.


\begin{remark}
The response tries to rescue uniqueness by declaring that one pair is ``wrong mechanism.'' But that is
precisely the contested point: the framework has not provided a theorem that selects (faces, face-vertices)
and forbids (edges, cube-vertices). Without such a theorem, the choice is discretionary.
\end{remark}

\subsection*{4. Failure 3: The ``discrete measure'' is not uniquely defined even for faces}

Even restricting to ``face-mediated'' reasoning, the denominator $4$ is ambiguous for a square face:
it is both the number of vertices per face and the number of edges per face.

\underline{Internal ambiguity of the alleged discrete measure}
For the cube, the rule ``normalize by vertex count of the mediating object'' does not uniquely fix the
normalization because multiple inequivalent structural quantities coincide numerically:
\[
V_{\mathrm{face}} = E_{\mathrm{face}} = 4.
\]


Proof:
A square face has 4 vertices and 4 edges. Therefore the proposed normalization by ``vertex anchors''
cannot be distinguished, at $D=3$, from an equally plausible normalization by ``edge anchors,'' while
yielding the same numeric correction. The response does not provide a theorem that picks vertices rather
than edges as the correct ``discrete measure.'' 

\begin{remark}
This is an important point: the response claims it has eliminated the arbitrariness of the integer $4$.
It has not. It has only renamed the $4$ after selecting it.
\end{remark}

\subsection*{5. ``Structural'' vs ``numerical'' is not a valid separation here}

The response asserts that algebraic non-uniqueness applies only to ``numerical representations,'' not
to ``structural derivations.''

\underline{If structure is not independently formalized, it collapses to numerical relabeling:}
If a proposed ``mechanism'' is not encoded as a formal object with independent constraints and falsifiers,
then any selection of a formula can be rebranded as a ``mechanism,'' and the distinction from numerical
representation evaporates.\\


Proof:
Given a desired value $x$, one can always define a post-hoc mechanism label $\mathsf{Mech}_x$ whose
rule is ``output $x$.'' If mechanisms are not independently constrained, the mapping
``mechanism $\to$ number'' provides no explanatory restriction.

Therefore, for a mechanism claim to do work, it must be formalized in a way that makes alternative
mechanisms \emph{impossible} (or at least testably false). The response supplies neither a formal definition
of admissible mechanisms nor a falsifier that would distinguish the proposed mechanism from rivals.


\subsection*{6. Fixing one coefficient cannot establish a fermion mass law}

Even if one accepted the tau-step story, the lepton pipeline still contains additional hand-specified
hypothesis formulas (e.g.\ the electron break $\delta_e$ and the $e\to\mu$ step correction terms),
so the system remains underdetermined at the level required to claim a universal mass law.

\paragraph{Conclusion.}
The response does not solve the ``infinite formulas'' objection. It replaces it with a new unproven selector
postulate and then asserts uniqueness by narrative mechanism labeling. Even granting its template, explicit
degeneracy remains.

\subsection*{What would actually close the loophole? (Non-negotiable checklist)}
To legitimately claim the mechanism resolves the objection, one must provide:
\begin{enumerate}
\item Formal definitions of ``edge-mediated'' and ``face-mediated'' as distinct objects in the theory,
\item A theorem that the tau-step must be face-mediated (not stipulated),
\item A theorem that the discrete normalization for face mediation is \emph{vertex count} (not edges-per-face,
not total vertices, not any other coincident count),
\item A uniqueness theorem proving injectivity of the mechanism-to-coefficient map,
\item An empirical falsifier that would refute the rule if the universe were different.
\end{enumerate}
Until then, the proposed resolution is not a derivation; it is an after-the-fact explanation.

\end{document}
