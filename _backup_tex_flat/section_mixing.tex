\section{The Cubic Ledger: Vertices, Edges, Faces}
\noindent\fbox{\parbox{0.97\linewidth}{%
\textbf{Section summary.}
This paper models flavor mixing as constrained by a finite combinatorial ``ledger'' associated with the 3-cube.
In this section we record the relevant cube counts and define the normalization objects that will appear in later mixing formulas.
The cube counts themselves are elementary combinatorics (\PROVED); the premise that they control mixing is a modeling hypothesis (\HYP).}}

\subsection{Cube counts (pure combinatorics)}
Let the ``cubic ledger'' refer to the combinatorial structure of the 3-dimensional cube.
The following counts are standard:
\begin{equation}
  V := 2^3 = 8, \PROVED
\end{equation}
\begin{equation}
  E := 3\cdot 2^{3-1} = 12, \PROVED
\end{equation}
\begin{equation}
  F := 2\cdot 3 = 6. \PROVED
\end{equation}
Here \(V\) is the number of vertices, \(E\) the number of edges, and \(F\) the number of faces of the cube. \PROVED

\subsection{Vertex--edge slots (the key normalization)}
Many mixing statements are naturally expressed as ``one out of \(N\) admissible adjacency slots.''
For the cube, each edge has two endpoints, so the number of ordered vertex--edge incidences is
\begin{equation}
  S := 2E = 24. \PROVED
\end{equation}
We will refer to \(S\) as the number of \emph{vertex--edge slots}.
The combinatorics here is rigid; the modeling hypothesis is that a CKM/PMNS element can be normalized by a subset of these slots. \HYP

\subsection{Why these integers are relevant for mixing (model premise)}
The structural claim explored in this paper is that flavor mixing is governed by a finite transition ledger whose primitive moves are adjacency moves
on the 3-cube. \HYP
Under this premise, cube integers can appear in two roles:
\begin{itemize}
  \item \textbf{Normalizations.} ``One allowed transition out of \(S\) slots'' produces factors of the form \(1/S\). \HYP
  \item \textbf{Coefficients.} Integer counts such as \(F=6\) and \(E=12\) can appear as fixed coefficients in correction terms, without introducing
  per-channel tuning knobs. \HYP
\end{itemize}
The remainder of the paper makes this premise concrete by proposing specific CKM/PMNS formulas and then testing them against PDG/NuFIT summaries. \VAL

\paragraph{Classical correspondence.}
The cubic ledger corresponds to a discrete transition graph (the 3-cube) familiar from lattice models and discretized state spaces: \(V\), \(E\), and \(F\) are its exact incidence counts, and \(S=2E\) counts ordered vertex--edge incidences (``adjacency slots''). Normalizations like \(1/S\) are dimensionless counting weights, analogous to uniform priors/probabilities over a finite adjacency set. The special role of \(2^D\) (here \(D=3\Rightarrow V=8\)) has no direct classical analog in continuum field theory; the closest conceptual relative is the minimal traversal/sampling bound that appears when a finite state space is resolved by discrete steps (e.g.\ Hamiltonian-path and Nyquist--Shannon style bounds). \HYP

\section{CKM from Edge-Dual Counting}
\noindent\fbox{\parbox{0.97\linewidth}{%
\textbf{Section summary.}
We propose simple closed-form expressions for three CKM magnitudes using cube-ledger normalizations and shared constants.
The cube counts are fixed (\PROVED); the identification of particular CKM entries with those normalizations is a falsifiable model hypothesis (\HYP).
Numerical agreement is assessed later against PDG and labeled as validation (\VAL).}}

\subsection{What is being predicted}
Let \(V\) denote the CKM matrix, relating weak-interaction quark states to mass eigenstates.
This section focuses only on the magnitudes of three small off-diagonal elements that define the observed hierarchy:
\(|V_{us}|\) (Cabibbo mixing), \(|V_{cb}|\) (2--3 mixing), and \(|V_{ub}|\) (1--3 mixing).
We emphasize that this is not a fit: the formulas below contain no adjustable per-channel coefficients. \PROVED

\subsection{Edge-dual normalization for \texorpdfstring{\(|V_{cb}|\)}{Vcb}}
From Sec.~2 we have the number of vertex--edge slots \(S=24\). \PROVED
The edge-dual hypothesis identifies the 2--3 mixing magnitude with a single admissible transition out of these slots:
\begin{equation}
  |V_{cb}|_{\mathrm{pred}}
  \;:=\;
  \frac{1}{S}
  \;=\;
  \frac{1}{24}.
  \HYP
  \label{eq:Vcb_pred}
\end{equation}
The mathematical identity \(S=2E\) is combinatorics; the physical content is the ``one-slot'' identification of a CKM entry with a ledger normalization. \HYP

\subsection{\texorpdfstring{\(\phig\)}{phi}-power Cabibbo mixing for \texorpdfstring{\(|V_{us}|\)}{Vus}}
We propose that the Cabibbo mixing magnitude is controlled by a dimension-linked ladder step and therefore takes a pure \(\phig\)-power form:
\begin{equation}
  |V_{us}|_{\mathrm{pred}}
  \;:=\;
  \phig^{-3}.
  \HYP
  \label{eq:Vus_pred}
\end{equation}
The exponent \(-3\) is not tuned to data; it is the structural choice associated with the 3-cube ledger used throughout this paper. \HYP

\subsection{A minimal \texorpdfstring{\(\alpha\)}{alpha} coupling for \texorpdfstring{\(|V_{ub}|\)}{Vub}}
Finally, we propose that the smallest CKM mixing magnitude is suppressed by a single electromagnetic coupling factor:
\begin{equation}
  |V_{ub}|_{\mathrm{pred}}
  \;:=\;
  \frac{\alpha}{2}.
  \HYP
  \label{eq:Vub_pred}
\end{equation}
Here \(\alpha\) is the fine-structure constant treated as a shared constant (not a free mixing knob). \CERT

\subsection{How these hypotheses will be tested}
The validation test is direct: the predicted magnitudes in Eqs.~\eqref{eq:Vcb_pred}--\eqref{eq:Vub_pred} are compared to PDG values,
and any claimed agreement is labeled as validation rather than derivation. \VAL
Future improvements in CKM global fits tighten these tests without changing the proposed formulas. \VAL

\paragraph{Classical correspondence.}
The CKM matrix is a standard unitary mixing matrix in the SM; the framework here proposes closed-form magnitudes rather than treating them as free parameters.
The normalization \(|V_{cb}|=1/24\) corresponds to selecting one transition out of a finite adjacency set---analogous to discrete-state transition probabilities in lattice or graph-theoretic models.
The power-law form \(|V_{us}|=\phig^{-3}\) corresponds to a scale-invariant suppression familiar from hierarchical Yukawa textures (e.g.\ Froggatt--Nielsen mechanisms), but here the exponent is fixed by ledger dimension rather than tuned.
The \(\alpha\)-suppression in \(|V_{ub}|\) mirrors radiative-correction hierarchies in effective field theory.
No per-channel fitting is introduced; all structure is shared with the mass sector. \HYP

\section{PMNS from $\phig$-Harmonics}
\noindent\fbox{\parbox{0.97\linewidth}{%
\textbf{Section summary.}
We propose parameter-free closed-form expressions for the three PMNS mixing angles.
The proposal is that PMNS weights are controlled by $\phig$-harmonic ladder structure (including an octave-forced exponent) with small, universal corrections
proportional to the shared constant $\alpha$.
These are falsifiable hypotheses (\HYP); numerical agreement is assessed later and labeled as validation (\VAL).}}

\subsection{What is being predicted}
Let \(U\) denote the PMNS matrix relating flavor neutrino states to mass eigenstates.
Rather than predicting a full complex parameterization in this section, we focus on three experimentally reported quantities:
\(\sin^2\theta_{13}\), \(\sin^2\theta_{12}\), and \(\sin^2\theta_{23}\).
The objective is to propose \emph{closed-form} expressions for these three numbers that introduce no per-angle fitting knobs. \PROVED

\subsection{Reactor angle: an octave-forced $\phig$-power}
The cleanest PMNS prediction is the reactor mixing weight, proposed to be an octave-forced $\phig$-power:
\begin{equation}
  \sin^2\theta_{13}^{\mathrm{pred}}
  \;:=\;
  \phig^{-8}.
  \HYP
  \label{eq:pmns_theta13_pred}
\end{equation}
The exponent \(8\) is not tuned; it is the same eight-tick ``octave'' count used to fix ladder coordinate origins in Paper~1. \HYP

\subsection{Solar and atmospheric angles: base weights plus universal $\alpha$-corrections}
We propose that the remaining two angles are controlled by simple base weights, with small universal corrections proportional to the shared constant \(\alpha\):
\begin{align}
  \sin^2\theta_{12}^{\mathrm{pred}}
  &:= \phig^{-2} \;-\; 10\,\alpha,
  \HYP
  \label{eq:pmns_theta12_pred}\\
  \sin^2\theta_{23}^{\mathrm{pred}}
  &:= \frac{1}{2} \;+\; 6\,\alpha.
  \HYP
  \label{eq:pmns_theta23_pred}
\end{align}
The coefficients \(10\) and \(6\) are not fit parameters; they are intended to be fixed integers forced by cube bookkeeping.
Their geometric origin is addressed in the next section. \HYP

\subsection{Immediate qualitative consequences (falsifiable)}
Equation~\eqref{eq:pmns_theta23_pred} has an immediate qualitative implication: if \(\alpha>0\), then
\(\sin^2\theta_{23}^{\mathrm{pred}} > 1/2\), i.e.\ the atmospheric angle lies in the upper octant. \HYP
This is a sharp falsifier: sufficiently precise confirmation of a lower-octant \(\theta_{23}\) would refute the hypothesis class of~\eqref{eq:pmns_theta23_pred}. \VAL

For orientation only, substituting a fixed \(\alpha\) value and evaluating \(\phig\)-powers yields concrete numerical targets; these are checked against NuFIT
in Sec.~7 and are labeled as validation rather than derivation. \VAL

\paragraph{Classical correspondence.}
The PMNS matrix is the standard leptonic mixing matrix; the framework proposes closed-form expressions for \(\sin^2\theta\) values rather than treating them as free parameters.
The \(\phig\)-power form \(\sin^2\theta_{13}=\phig^{-8}\) corresponds to a discrete self-similar (fixed-point) scaling: the exponent \(8=2^3\) is the octave period, analogous to how renormalization-group fixed points generate power-law scaling in critical phenomena.
The additive \(\alpha\)-corrections mirror radiative loop corrections in effective field theory, with fixed integer coefficients rather than running couplings.
This structure is the mixing-sector analog of the cost-function stationary point (T5) that determines \(\phig\) in the mass sector. \HYP

\section{Radiative Corrections from Cube Topology}
\noindent\fbox{\parbox{0.97\linewidth}{%
\textbf{Section summary.}
The PMNS and CKM hypotheses proposed in Secs.~3--4 include small additive corrections proportional to the shared coupling constant \(\alpha\).
In this section we explain why the \emph{integer coefficients} multiplying \(\alpha\) are treated as fixed, cube-derived counts rather than tunable fit knobs.
The cube-count identities are elementary (\PROVED); the assignment of those counts to specific correction terms is a falsifiable modeling hypothesis (\HYP).}}

\subsection{Why ``radiative'' corrections appear in a structural model}
The leading-order terms in the mixing hypotheses are purely geometric: powers of \(\phig\) and ledger normalizations such as \(1/S\). \HYP
Corrections proportional to \(\alpha\) play the role of a universal small parameter that perturbs these geometric weights without introducing
new channel-by-channel degrees of freedom. \HYP
The core non-negotiable constraint is that coefficients multiplying \(\alpha\) must be fixed \emph{integers} supplied by the same cube ledger,
not new parameters tuned per observable. \PROVED

\subsection{Three cube-derived coefficients}
From Sec.~2, the cube face count is \(F=6\) and the edge count is \(E=12\). \PROVED
We define three integer (or rational) coefficients that will be used in later correction terms:
\begin{align}
  C_{\mathrm{atm}} &:= F = 6, \PROVED \label{eq:Catm_def}\\
  C_{\mathrm{sol}} &:= E - 2 = 10, \HYP \label{eq:Csol_def}\\
  C_{\mathrm{Cab}} &:= \frac{F}{4} = \frac{3}{2}. \PROVED \label{eq:Ccab_def}
\end{align}
The arithmetic equalities \(F=6\) and \(E-2=10\) are trivial; the modeling content in~\eqref{eq:Csol_def} is the choice to subtract two constrained directions
from the full edge count when defining the solar correction coefficient. \HYP

\subsection{How the coefficients enter PMNS}
The PMNS hypotheses of Sec.~4 can be summarized as ``base weight + coefficient\(\times \alpha\)'':
\begin{align}
  \sin^2\theta_{23}^{\mathrm{pred}} &= \frac{1}{2} + C_{\mathrm{atm}}\,\alpha, \HYP \\
  \sin^2\theta_{12}^{\mathrm{pred}} &= \phig^{-2} - C_{\mathrm{sol}}\,\alpha. \HYP
\end{align}
The claim is not that \(\alpha\) is adjusted to fit each angle; rather, \(\alpha\) is shared and fixed, and only the cube-derived integers
\(C_{\mathrm{atm}},C_{\mathrm{sol}}\) appear. \PROVED

\subsection{A Cabibbo correction option (no new knobs)}
Section~3 introduced \(|V_{us}|_{\mathrm{pred}} := \phig^{-3}\) as a leading-order Cabibbo weight. \HYP
If a universal \(\alpha\)-suppression is included for Cabibbo mixing without introducing a new coefficient, the cube-ledger choice is
to use \(C_{\mathrm{Cab}}=F/4\):
\begin{equation}
  |V_{us}|_{\mathrm{pred,corr}}
  \;:=\;
  \phig^{-3} \;-\; C_{\mathrm{Cab}}\,\alpha.
  \HYP
  \label{eq:Vus_pred_corr}
\end{equation}
This is an optional refinement within the same contract: the coefficient is fixed by cube topology, and the sign is part of the falsifiable hypothesis.
Whether the leading-order or corrected form is preferred is decided only by validation against PDG in Sec.~7. \VAL

\paragraph{Classical correspondence.}
The integer coefficients \(C_{\mathrm{atm}}=6\), \(C_{\mathrm{sol}}=10\), and \(C_{\mathrm{Cab}}=3/2\) play the role of ``constructor integers'' in the SM bookkeeping sense: they are fixed combinatorial counts (faces, edges, face-quarter) that discretize species dependence.
This mirrors how loop-order and group-theoretic factors appear in SM radiative corrections (e.g.\ Casimir coefficients, multiplicity factors), but here the integers are cube-derived rather than gauge-group-derived.
The constraint that coefficients must be integers or simple fractions from ledger combinatorics is analogous to the Buckingham \(\Pi\)-theorem requirement that dimensionless predictions depend only on pure numbers. \PROVED

\section{CP Violation and the Jarlskog Invariant}
\noindent\fbox{\parbox{0.97\linewidth}{%
\textbf{Section summary.}
CP violation in three-generation mixing is measured by a convention-invariant quantity: the Jarlskog invariant.
We define this invariant and then propose a simple, zero-parameter magnitude scale for CKM CP violation built from the same three CKM magnitudes
already predicted in Sec.~3.
The definition is mathematical (\PROVED); the proposed structural magnitude and sign conventions are falsifiable hypotheses (\HYP).}}

\subsection{The Jarlskog invariant (definition and invariance)}
For any \(3\times 3\) unitary mixing matrix \(W\), the Jarlskog invariant can be written as a rephasing-invariant imaginary part of a $2\times 2$ minor:
\begin{equation}
  J(W)
  \;:=\;
  \left|\operatorname{Im}\!\left(W_{11}W_{22}W_{12}^\ast W_{21}^\ast\right)\right|.
  \PROVED
  \label{eq:J_def}
\end{equation}
The absolute value is included so that \(J(W)\ge 0\) is a convention-independent magnitude.
In the Standard Model, \(J(V_{\mathrm{CKM}})\neq 0\) is the statement that quark mixing violates CP, while \(J(U_{\mathrm{PMNS}})\neq 0\) is the analogous statement for leptons. \PROVED

\subsection{A minimal CKM CP scale from the ledger hierarchy}
Section~3 proposes three CKM magnitudes with no per-channel tuning:
\(|V_{us}|_{\mathrm{pred}}\), \(|V_{cb}|_{\mathrm{pred}}\), and \(|V_{ub}|_{\mathrm{pred}}\). \HYP
A minimal way to turn these into a CP-violation \emph{scale} is to take their product:
\begin{equation}
  J_{\mathrm{CKM}}^{\mathrm{pred}}
  \;:=\;
  |V_{us}|_{\mathrm{pred}}\;|V_{cb}|_{\mathrm{pred}}\;|V_{ub}|_{\mathrm{pred}}.
  \HYP
  \label{eq:Jckm_pred}
\end{equation}
Using the specific hypotheses of Sec.~3, this becomes the closed form
\begin{equation}
  J_{\mathrm{CKM}}^{\mathrm{pred}}
  \;=\;
  \left(\phig^{-3}\right)\left(\frac{1}{24}\right)\left(\frac{\alpha}{2}\right).
  \HYP
  \label{eq:Jckm_pred_closed}
\end{equation}
This proposal has two important features:
(i) it introduces no new CP-specific fit parameters beyond the already-proposed mixing magnitudes, and
(ii) it predicts the correct \emph{order of magnitude} scale for CKM CP violation if the Sec.~3 magnitudes are correct. \HYP

\subsection{Sign conventions and falsifiers}
Because \(J(W)\) in Eq.~\eqref{eq:J_def} is a magnitude, it does not encode a sign.
A sign can be attached only after fixing a generation ordering and phase convention; in this paper we reserve all such sign tests for the validation section. \CERT

The falsifiers attached to this section are therefore magnitude-based:
if the measured CKM Jarlskog invariant \(J(V_{\mathrm{CKM}})\) is inconsistent with the predicted scale \(J_{\mathrm{CKM}}^{\mathrm{pred}}\) in~\eqref{eq:Jckm_pred_closed},
then the ``ledger-hierarchy'' CP hypothesis is refuted. \VAL
All numerical comparisons are carried out explicitly against PDG in Sec.~7. \VAL

\paragraph{Classical correspondence.}
The Jarlskog invariant \(J(W)\) is mathematically identical to the standard SM rephasing-invariant measure of CP violation; the definition in Eq.~\eqref{eq:J_def} is the same as in the SM literature.
The only structural addition is the proposal that \(J_{\mathrm{CKM}}^{\mathrm{pred}}\) takes a closed form built from already-proposed mixing magnitudes, rather than being an independent fit parameter.
This is a ``Twin'' correspondence: the mathematical object is the standard one, and the framework proposes its value rather than leaving it free. \PROVED

\section{Comparison to PDG and NuFIT}
\noindent\fbox{\parbox{0.97\linewidth}{%
\textbf{Section summary.}
This section validates the CKM/PMNS hypotheses of Secs.~3--6 against standard experimental summaries.
All numerical targets are external (PDG for CKM, NuFIT for PMNS) and all agreement statements are labeled as validation (\VAL).
No parameter is tuned per observable; in particular, the only small parameter is the shared constant \(\alpha\), and all integer coefficients are fixed by cube bookkeeping.}}

\subsection{Reference targets and pinned constants}
For CKM magnitudes and the quark-sector Jarlskog invariant we use the PDG summary values \cite{PDG2024}. \VAL
For PMNS mixing angles we use NuFIT 5.x summaries for normal ordering \cite{NuFIT}. \VAL

For numerical evaluation of the closed forms, we pin the fine-structure constant for this section at
\begin{equation}
  \alpha^{-1} := 137.036,
  \qquad
  \alpha := 1/\alpha^{-1}.
  \CERT
  \label{eq:alpha_pin_p2}
\end{equation}
At the level of precision reported here, using nearby standard values of \(\alpha\) does not change the qualitative conclusions. \CERT

\subsection{CKM magnitudes (validation)}
The predicted magnitudes are those of Sec.~3, with the optional Cabibbo correction of Sec.~5:
\begin{align}
  |V_{cb}|_{\mathrm{pred}} &= \frac{1}{24} \approx 0.04167, \VAL \\
  |V_{ub}|_{\mathrm{pred}} &= \frac{\alpha}{2} \approx 0.00365, \VAL \\
  |V_{us}|_{\mathrm{pred}} &= \phig^{-3} \approx 0.23607, \VAL \\
  |V_{us}|_{\mathrm{pred,corr}} &= \phig^{-3} - \frac{3}{2}\alpha \approx 0.22512. \VAL
\end{align}
Using representative PDG central values \cite{PDG2024},
\(|V_{cb}|_{\mathrm{ref}}\approx 0.04182\),
\(|V_{ub}|_{\mathrm{ref}}\approx 0.00369\),
\(|V_{us}|_{\mathrm{ref}}\approx 0.22500\),
the corresponding absolute discrepancies are
\begin{align}
  \bigl||V_{cb}|_{\mathrm{pred}} - |V_{cb}|_{\mathrm{ref}}\bigr| &\approx 1.53\times 10^{-4}, \VAL \\
  \bigl||V_{ub}|_{\mathrm{pred}} - |V_{ub}|_{\mathrm{ref}}\bigr| &\approx 4.13\times 10^{-5}, \VAL \\
  \bigl||V_{us}|_{\mathrm{pred}} - |V_{us}|_{\mathrm{ref}}\bigr| &\approx 1.11\times 10^{-2}, \VAL \\
  \bigl||V_{us}|_{\mathrm{pred,corr}} - |V_{us}|_{\mathrm{ref}}\bigr| &\approx 1.22\times 10^{-4}. \VAL
\end{align}
Thus, the corrected Cabibbo hypothesis Eq.~\eqref{eq:Vus_pred_corr} is strongly preferred over the leading-order \(\phig^{-3}\) value when judged against PDG. \VAL

\subsection{CKM CP violation scale (validation)}
Section~6 proposes the CKM CP scale
\(J_{\mathrm{CKM}}^{\mathrm{pred}} := |V_{us}|\;|V_{cb}|\;|V_{ub}|\) (no additional parameters). \HYP
Evaluating the closed form Eq.~\eqref{eq:Jckm_pred_closed} gives
\begin{equation}
  J_{\mathrm{CKM}}^{\mathrm{pred}}
  \approx 3.59\times 10^{-5}.
  \VAL
\end{equation}
If one instead uses the corrected Cabibbo variant \(|V_{us}|_{\mathrm{pred,corr}}\) in the product (still no new knobs), one obtains
\begin{equation}
  J_{\mathrm{CKM}}^{\mathrm{pred,corr}}
  :=
  |V_{us}|_{\mathrm{pred,corr}}\;|V_{cb}|_{\mathrm{pred}}\;|V_{ub}|_{\mathrm{pred}}
  \approx 3.42\times 10^{-5}.
  \VAL
  \label{eq:Jckm_pred_corr}
\end{equation}
For comparison, PDG reports a quark-sector Jarlskog magnitude \(J_{\mathrm{CKM}}^{\mathrm{ref}}\sim 3.1\times 10^{-5}\) \cite{PDG2024}. \VAL

\subsection{PMNS mixing angles (validation and current tension)}
The PMNS hypotheses of Sec.~4 evaluate (with the pinned \(\alpha\)) to
\begin{align}
  \sin^2\theta_{13}^{\mathrm{pred}} &\approx 0.02129, \VAL \\
  \sin^2\theta_{12}^{\mathrm{pred}} &\approx 0.30899, \VAL \\
  \sin^2\theta_{23}^{\mathrm{pred}} &\approx 0.54378. \VAL
\end{align}

Using NuFIT 5.x (normal ordering) as a standard experimental summary \cite{NuFIT}, two points are immediate:
\begin{itemize}
  \item \textbf{Reactor and solar angles.} \(\sin^2\theta_{13}\) and \(\sin^2\theta_{12}\) are in reasonable agreement with NuFIT best-fit values
  at the level of current uncertainties (validation). \VAL
  \item \textbf{Atmospheric angle and octant.} The hypothesis \(\sin^2\theta_{23}^{\mathrm{pred}} = 1/2 + 6\alpha\) implies an \emph{upper-octant} value.
  NuFIT continues to show octant sensitivity, and current fits may place the best fit away from the predicted point; this is an active tension and therefore a near-term falsifier. \VAL
\end{itemize}

\subsection{Referee checklist for any comparison}
Any objection or alternative comparison should specify:
\begin{itemize}
  \item the target summary (PDG vs a specific global-fit release),
  \item the target parameterization (e.g.\ whether \(\sin^2\theta_{ij}\) or \(\theta_{ij}\) is being reported),
  \item for PMNS, the ordering and the octant convention being assumed,
  \item and the exact hypothesis being tested (leading-order vs corrected Cabibbo; which PMNS correction terms are included).
\end{itemize}

