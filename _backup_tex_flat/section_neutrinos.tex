\section{The Deep Ladder: Fractional Rungs}
\noindent\fbox{\parbox{0.97\linewidth}{%
\textbf{Section summary.}
We define the ladder coordinate used for neutrinos and introduce the ``deep'' regime where rungs are taken to be fractional.
The ladder mathematics is elementary (\PROVED); the physical claim is the fractional-rung assignment and its specific step size (\HYP).
This section fixes notation so that later sections can state masses and splittings without hidden fitting knobs.}}

\subsection{Ladder coordinate and rungs}
As in Papers~1--2, we encode multiplicative hierarchy by a base-\(\phig\) scale coordinate.
For a positive quantity \(x\), define its ladder coordinate by
\begin{equation}
  r(x) \;:=\; \log_{\phig}(x). \PROVED
  \label{eq:r_of_x_def}
\end{equation}
Equivalently, specifying a rung \(r\) specifies a pure ladder factor \(\phig^{r}\). \PROVED

In the charged sectors we treated rungs as integers.
For neutrinos we extend the rung set to rationals:
\begin{equation}
  r \in \tfrac14\mathbb{Z}. \HYP
  \label{eq:quarter_rung_convention}
\end{equation}
Equation~\eqref{eq:quarter_rung_convention} is a convention for the deep ladder: it asserts that the relevant rung lattice is a quarter-step lattice.
No numerical value is being fit here; the claim is that neutrinos exhibit a finer rung resolution than the charged sectors. \HYP

\subsection{Why quarter steps (motivation, not a fit)}
The quarter-step convention is motivated by two qualitative constraints:
\begin{itemize}
  \item \textbf{Resolution.} Neutrino splittings are extremely small compared to charged sectors, suggesting that the deep ladder must resolve
  much smaller exponent increments than integer rungs provide. \HYP
  \item \textbf{Compatibility with the octave clock.} The series uses an eight-tick closure as a canonical cycle; quarter rungs provide a simple compatible refinement
  that is still discrete and auditable. \HYP
\end{itemize}
These motivations are not proofs; the quarter-step lattice is judged by falsifiers (Sec.~8). \VAL

\subsection{Rung differences and squared-mass ratios}
A key reason to use a ladder coordinate is that ratios become differences.
If two masses \(m_a,m_b>0\) differ by rung offset \(\Delta r := r(m_a)-r(m_b)\), then
\begin{equation}
  \frac{m_a}{m_b} = \phig^{\Delta r}. \PROVED
  \label{eq:mass_ratio_from_dr}
\end{equation}
For squared masses this becomes
\begin{equation}
  \frac{m_a^2}{m_b^2} = \phig^{2\Delta r}. \PROVED
  \label{eq:sq_mass_ratio_from_dr}
\end{equation}
Later, the neutrino rung assignments will imply a rigid $\phig$-power ratio for the atmospheric-to-solar splitting scale. \PROVED

\subsection{Rung assignment (to be used in later sections)}
We denote the three neutrino rungs by \(r_1<r_2<r_3\) (normal ordering). \HYP
In later sections we will use the specific deep-ladder assignment
\begin{equation}
  (r_1,r_2,r_3) := \left(-\frac{239}{4},-\frac{231}{4},-\frac{217}{4}\right). \HYP
  \label{eq:nu_rungs}
\end{equation}
Equation~\eqref{eq:nu_rungs} is the core discrete input for the neutrino sector in this paper.
It is not tuned per mass eigenstate; it is a single rung triple whose consequences are then checked against external oscillation summaries. \HYP

\paragraph{Classical correspondence.}
The logarithmic ladder coordinate \(r(x)=\log_\phig(x)\) is a standard change of variables; what is novel is the fractional-rung lattice \(r\in\tfrac14\mathbb{Z}\).
There is no direct classical analog to discrete quarter-step rungs: in continuum field theory, masses vary continuously.
The closest conceptual relative is a discrete quantum number (like spin projection or isospin component) that restricts allowed states to a lattice.
The compatibility of quarter steps with the eight-tick octave (\(8\times\tfrac14=2\)) is an internal consistency check, analogous to requiring that lattice refinements divide evenly into fundamental periods (T6 bridge). \HYP

\section{Neutrino Mass Predictions}
\noindent\fbox{\parbox{0.97\linewidth}{%
\textbf{Section summary.}
Given a discrete rung triple \((r_1,r_2,r_3)\) and a single global calibration seam for eV reporting, the deep ladder yields three absolute neutrino masses.
The rung triple is the only neutrino-sector discrete input (\HYP); the eV scale is fixed once for the overall framework (\CERT);
the numerical values quoted here are the consequence of those declarations (and are later compared to external constraints as validation).}}

\subsection{From rungs to eV masses (explicit reporting seam)}
Section~2 fixes the neutrino rung triple \((r_1,r_2,r_3)\in(\tfrac14\mathbb{Z})^3\) (Eq.~\eqref{eq:nu_rungs}). \HYP
To report absolute masses in eV, we require a declared calibration seam that converts one ladder ``coherence quantum'' to SI energy.
We represent that seam by a single scalar \(\tau_0\) (seconds per ladder tick), and define the corresponding eV scale
\begin{equation}
  \kappa_{\mathrm{eV}}
  \;:=\;
  \frac{\hbar}{\tau_0}\Big/\!\left(1\,\mathrm{eV}\right).
  \CERT
  \label{eq:kappa_ev}
\end{equation}
This seam is global: it is fixed once for the framework and is not adjusted per neutrino eigenstate. \CERT

With \(\kappa_{\mathrm{eV}}\) fixed, the deep-ladder mass hypothesis is:
\begin{equation}
  m_i^{\mathrm{pred}}
  \;:=\;
  \kappa_{\mathrm{eV}}\,\phig^{r_i},
  \qquad i\in\{1,2,3\}.
  \HYP
  \label{eq:nu_mass_pred}
\end{equation}

\subsection{Predicted absolute masses (numerical evaluation under the seam)}
Evaluating Eq.~\eqref{eq:nu_mass_pred} for the rung triple~\eqref{eq:nu_rungs} under the declared seam yields the absolute mass window:
\begin{align}
  0.00352 &< m_1^{\mathrm{pred}} < 0.00355\ \mathrm{eV}, \CERT \\
  0.00924 &< m_2^{\mathrm{pred}} < 0.00928\ \mathrm{eV}, \CERT \\
  0.04987 &< m_3^{\mathrm{pred}} < 0.04993\ \mathrm{eV}. \CERT
  \label{eq:nu_mass_bounds}
\end{align}
The implied mass sum is therefore
\begin{equation}
  0.06263 < \sum_{i=1}^3 m_i^{\mathrm{pred}} < 0.06276\ \mathrm{eV}. \CERT
  \label{eq:nu_mass_sum_bounds}
\end{equation}
Compatibility with cosmological and kinematic constraints is assessed later as validation, not used to set \(\tau_0\) or the rungs. \VAL

\paragraph{Classical correspondence.}
The mass prediction \(m_i^{\mathrm{pred}}=\kappa_{\mathrm{eV}}\,\phig^{r_i}\) is an instance of the single-anchor mass law used throughout the series: a global scale factor times a pure \(\phig\)-power determined by a discrete rung.
This corresponds to the phenomenological ``quantum ladder'' structure observed in many hierarchical mass spectra, where ratios between adjacent states follow a geometric progression.
The calibration seam \(\kappa_{\mathrm{eV}}\) plays the role of an overall unit conversion (Bridge to SI), analogous to fixing \(\hbar\) or \(c\) when reporting energies in eV rather than inverse seconds.
No per-eigenstate fitting is introduced; the entire mass hierarchy is encoded in the rung triple. \HYP

\section{Mass-Squared Splittings}
\noindent\fbox{\parbox{0.97\linewidth}{%
\textbf{Section summary.}
Oscillation experiments measure mass-squared splittings rather than absolute masses.
Given the deep-ladder masses from Sec.~3, we define the two independent splittings and evaluate them under the declared eV reporting seam.
The resulting values are then compared to NuFIT summary windows as validation.}}

\subsection{Definitions}
We use the standard definitions (normal ordering conventions are discussed later):
\begin{equation}
  \Delta m^2_{21} \;:=\; m_2^2 - m_1^2,
  \qquad
  \Delta m^2_{31} \;:=\; m_3^2 - m_1^2.
  \PROVED
  \label{eq:dm2_defs}
\end{equation}
If \(m_1<m_2<m_3\) (normal ordering), then both splittings are positive. \PROVED

\subsection{Predicted splittings from the deep ladder}
Using the mass law of Sec.~3, Eq.~\eqref{eq:nu_mass_pred}, the predicted splittings are
\begin{equation}
  \Delta m^2_{ij}{}^{\mathrm{pred}}
  \;=\;
  \left(m_i^{\mathrm{pred}}\right)^2 - \left(m_j^{\mathrm{pred}}\right)^2
  \;=\;
  \kappa_{\mathrm{eV}}^2\!\left(\phig^{2r_i}-\phig^{2r_j}\right).
  \PROVED
  \label{eq:dm2_pred_from_rungs}
\end{equation}
Thus, while the absolute eV-scale splittings depend on the global seam parameter \(\kappa_{\mathrm{eV}}\), the \emph{ratio} of splittings depends only on rung differences:
\begin{equation}
  \frac{\Delta m^2_{31}{}^{\mathrm{pred}}}{\Delta m^2_{21}{}^{\mathrm{pred}}}
  \;=\;
  \frac{\phig^{2r_3}-\phig^{2r_1}}{\phig^{2r_2}-\phig^{2r_1}}.
  \PROVED
  \label{eq:dm2_ratio_seam_cancels}
\end{equation}
The next section derives the exact $\phig$-power relation \((m_3^{\mathrm{pred}})^2/(m_2^{\mathrm{pred}})^2=\phig^7\) implied by the deep rung spacing,
and records the resulting closed-form (seam-free) prediction for the splitting ratio as a fixed function of \(\phig\). \HYP

\subsection{Numerical evaluation and validation}
Evaluating the splittings using the mass bounds from Sec.~3 (Eq.~\eqref{eq:nu_mass_bounds}) yields the representative values
\begin{align}
  \Delta m^2_{21}{}^{\mathrm{pred}} &\approx 7.33\times 10^{-5}\ \mathrm{eV}^2, \CERT \\
  \Delta m^2_{31}{}^{\mathrm{pred}} &\approx 2.48\times 10^{-3}\ \mathrm{eV}^2. \CERT
  \label{eq:dm2_pred_values}
\end{align}

As a validation check, we compare to NuFIT 5.x summary windows for normal ordering \cite{NuFIT}.
At the level of precision used in this paper, the predictions satisfy:
\begin{align}
  7.21\times 10^{-5}
  &< \Delta m^2_{21}{}^{\mathrm{pred}} < 7.62\times 10^{-5}\ \mathrm{eV}^2, \VAL \\
  2.455\times 10^{-3}
  &< \Delta m^2_{31}{}^{\mathrm{pred}} < 2.567\times 10^{-3}\ \mathrm{eV}^2. \VAL
  \label{eq:dm2_in_nufit}
\end{align}
These comparisons are strictly validation: the NuFIT windows are not used to set the rungs or the calibration seam. \VAL

\paragraph{Classical correspondence.}
The mass-squared splittings \(\Delta m^2_{21}\) and \(\Delta m^2_{31}\) are mathematically identical to the standard oscillation observables used in NuFIT and PDG summaries (Twin status).
The key structural addition is Eq.~\eqref{eq:dm2_ratio_seam_cancels}: the \emph{ratio} of splittings is seam-free because the global scale \(\kappa_{\mathrm{eV}}^2\) cancels.
This mirrors how mass-ratio predictions in QFT are often more robust than absolute-scale predictions, since renormalization-scale dependence cancels in ratios.
The framework predicts concrete numerical values for both splittings (Bridge to oscillation phenomenology), but the falsifiable core is the seam-free ratio. \HYP

\section{The $\phig^7$ Ratio}
\noindent\fbox{\parbox{0.97\linewidth}{%
\textbf{Section summary.}
The deep rung spacing implies an \emph{exact} $\phig$-power relation among the neutrino squared masses.
This exact ratio is independent of the eV calibration seam.
We also record the induced (seam-free) closed form for the ratio of the measured mass-squared splittings.}}

\subsection{An exact squared-mass ratio from rung differences}
From the mass law \(m_i^{\mathrm{pred}}=\kappa_{\mathrm{eV}}\phig^{r_i}\) (Eq.~\eqref{eq:nu_mass_pred}), the seam cancels in squared-mass ratios:
\begin{equation}
  \frac{\left(m_3^{\mathrm{pred}}\right)^2}{\left(m_2^{\mathrm{pred}}\right)^2}
  \;=\;
  \frac{\kappa_{\mathrm{eV}}^2\phig^{2r_3}}{\kappa_{\mathrm{eV}}^2\phig^{2r_2}}
  \;=\;
  \phig^{2(r_3-r_2)}.
  \PROVED
  \label{eq:nu_sq_mass_ratio_general}
\end{equation}

Under the specific deep rung assignment of Eq.~\eqref{eq:nu_rungs}, the rung gap is
\begin{equation}
  r_3-r_2 = \frac{7}{2}. \HYP
  \label{eq:dr32}
\end{equation}
Substituting~\eqref{eq:dr32} into~\eqref{eq:nu_sq_mass_ratio_general} yields the advertised exact ratio:
\begin{equation}
  \frac{\left(m_3^{\mathrm{pred}}\right)^2}{\left(m_2^{\mathrm{pred}}\right)^2}
  \;=\;
  \phig^{7}.
  \HYP
  \label{eq:nu_sq_mass_ratio_phi7}
\end{equation}
Equivalently, \(m_3^{\mathrm{pred}}/m_2^{\mathrm{pred}}=\phig^{7/2}\). \HYP

\subsection{Induced prediction for the splitting ratio (seam-free)}
While the squared-mass ratio is a pure $\phig$-power, the \emph{splitting} ratio depends on \(m_1\) as well.
Using Eq.~\eqref{eq:dm2_ratio_seam_cancels} together with the rung differences from Eq.~\eqref{eq:nu_rungs}:
\(r_2-r_1=2\) and \(r_3-r_1=11/2\), one obtains the closed form
\begin{equation}
  \frac{\Delta m^2_{31}{}^{\mathrm{pred}}}{\Delta m^2_{21}{}^{\mathrm{pred}}}
  \;=\;
  \frac{\phig^{11}-1}{\phig^{4}-1}
  \;\approx\;
  33.823.
  \HYP
  \label{eq:dm2_ratio_closed_form}
\end{equation}
This ratio is \emph{seam-free}: it depends only on the discrete rung differences and on \(\phig\), not on \(\kappa_{\mathrm{eV}}\). \PROVED
Its agreement with experimental summaries is assessed as validation (Sec.~4 and Sec.~8). \VAL

\paragraph{Classical correspondence.}
The exact relation \((m_3^{\mathrm{pred}})^2/(m_2^{\mathrm{pred}})^2=\phig^7\) is the neutrino-sector analog of the mass-family-ratio predictions in Paper~1: when two species share the same ladder base, their mass ratio is a pure \(\phig\)-power determined by the rung difference.
This corresponds to the general phenomenological observation that hierarchical mass spectra often exhibit geometric-progression structure (Bridge to Yukawa texture models).
The crucial feature is that the seam cancels: the ratio is a dimensionless, convention-free prediction testable by oscillation experiments alone.
This is the falsifiable core of the deep-ladder hypothesis. \HYP

\section{Normal Hierarchy from Geometry}
\noindent\fbox{\parbox{0.97\linewidth}{%
\textbf{Section summary.}
The ordering of neutrino masses (normal vs inverted) is an experimental question, but in the deep-ladder framework the ordering is not an independent fit knob:
it is fixed by the rung ordering together with monotonicity of the $\phig$-ladder map.
We record this implication explicitly and state what would falsify it.}}

\subsection{Monotonicity of the ladder map}
The ladder base satisfies \(\phig>1\). \PROVED
For any fixed \(\kappa_{\mathrm{eV}}>0\), the mapping
\begin{equation}
  r \mapsto m(r) := \kappa_{\mathrm{eV}}\,\phig^{r}
  \PROVED
  \label{eq:ladder_monotone_map}
\end{equation}
is strictly increasing in \(r\). \PROVED
Therefore rung ordering implies mass ordering. \PROVED

\subsection{Normal ordering implied by the deep rungs}
Section~2 fixes the neutrino rungs \((r_1,r_2,r_3)\) with
\begin{equation}
  r_1 < r_2 < r_3. \HYP
  \label{eq:r_ordering}
\end{equation}
Combining Eq.~\eqref{eq:r_ordering} with the monotonicity of Eq.~\eqref{eq:ladder_monotone_map} yields
\begin{equation}
  m_1^{\mathrm{pred}} < m_2^{\mathrm{pred}} < m_3^{\mathrm{pred}}.
  \PROVED
  \label{eq:normal_ordering_pred}
\end{equation}
Thus, within this framework, ``normal ordering'' is not a choice made to match an external fit; it is the direct consequence of the discrete rung assignment. \HYP

\subsection{Validation and falsifier}
Global oscillation analyses currently favor normal ordering, but the ordering remains an experimental output rather than an input to this paper. \VAL
If future oscillation and matter-effect measurements decisively establish inverted ordering, then the deep rung hypothesis
(and in particular the rung triple of Eq.~\eqref{eq:nu_rungs}) is refuted. \VAL

\paragraph{Classical correspondence.}
The statement ``rung ordering implies mass ordering'' has no direct classical analog: in continuum field theory, mass eigenvalues can be permuted freely by relabeling.
The closest conceptual relative is the constraint that quantum numbers (e.g.\ principal quantum number in atomic physics) order energy levels.
Here the rung triple is a discrete topological input (T9), and the monotonicity of \(\phig^r\) enforces a rigid mass ordering without additional fitting.
This is a \emph{Novel} structural claim: the ordering is not a choice but a consequence of the deep-ladder assignment. \HYP

\section{Cosmological Constraints}
\noindent\fbox{\parbox{0.97\linewidth}{%
\textbf{Section summary.}
Cosmological data constrain neutrino masses primarily through the sum \(\sum m_\nu\).
In the deep-ladder framework, \(\sum m_\nu\) is predicted once the rung triple and the global eV reporting seam are fixed.
Cosmological bounds are model-dependent and are used only for validation, never as an input.}}

\subsection{What cosmology constrains}
In standard cosmological analyses, the leading sensitivity to neutrino masses is through the total mass sum
\begin{equation}
  \Sigma_\nu \;:=\; m_1 + m_2 + m_3. \PROVED
  \label{eq:sigma_nu_def}
\end{equation}
The exact numerical bound on \(\Sigma_\nu\) depends on the assumed cosmological model (e.g.\ $\Lambda$CDM vs extensions) and the datasets included.
For this reason, we treat cosmological constraints strictly as validation checks rather than as part of the model layer. \VAL

\subsection{Deep-ladder prediction for the mass sum}
Section~3 derived the predicted mass-sum window under the declared eV seam:
\begin{equation}
  0.06263 < \Sigma_\nu^{\mathrm{pred}} < 0.06276\ \mathrm{eV}.
  \CERT
  \label{eq:sigma_nu_pred_bounds}
\end{equation}
This value is not obtained by fitting cosmological data; it is implied by the rung triple and the single global reporting seam. \CERT

\subsection{Validation against current cosmological bounds}
The Particle Data Group summarizes cosmological limits on \(\Sigma_\nu\) and emphasizes their model dependence \cite{PDG2024}. \VAL
Using representative current bounds (typically at the \(\Sigma_\nu \lesssim 0.12\,\mathrm{eV}\) level in $\Lambda$CDM-like analyses),
the predicted range~\eqref{eq:sigma_nu_pred_bounds} is comfortably allowed. \VAL
Future tightening of cosmological bounds toward \(\Sigma_\nu < 0.06\,\mathrm{eV}\) would directly pressure or refute the deep-ladder mass scale. \VAL

\section{Falsifiers}
This section lists experimental outcomes that would refute the deep-ladder hypothesis class proposed in this paper.
We distinguish \emph{seam-free} falsifiers (independent of the eV calibration seam) from \emph{scale} falsifiers (which test the declared eV reporting seam). \PROVED

\subsection{Seam-free falsifiers (depend only on rung differences and $\phig$)}
\paragraph{F1: splitting-ratio mismatch.}
Define the experimentally inferred splitting ratio
\begin{equation}
  R_{\Delta} \;:=\; \frac{\Delta m^2_{31}}{\Delta m^2_{21}}. \PROVED
  \label{eq:Rdelta_def}
\end{equation}
Under the rung triple of Eq.~\eqref{eq:nu_rungs}, the model predicts the seam-free value
\begin{equation}
  R_{\Delta}^{\mathrm{pred}}
  \;=\;
  \frac{\phig^{11}-1}{\phig^{4}-1}
  \;\approx\;
  33.823.
  \HYP
  \label{eq:Rdelta_pred}
\end{equation}
This hypothesis is falsified if the best-fit \(R_{\Delta}\) inferred from oscillation data (for the stated ordering and dataset release)
becomes inconsistent with \(R_{\Delta}^{\mathrm{pred}}\) beyond the quoted experimental uncertainty. \VAL

\paragraph{F2: ordering mismatch.}
The deep rungs are ordered \(r_1<r_2<r_3\) (Eq.~\eqref{eq:r_ordering}), which implies normal mass ordering \(m_1<m_2<m_3\) (Eq.~\eqref{eq:normal_ordering_pred}). \HYP
If future oscillation and matter-effect measurements decisively establish inverted ordering, the rung triple hypothesis is refuted. \VAL

\paragraph{F3: squared-mass ratio mismatch (requires absolute-mass information).}
The rung gap \(r_3-r_2=7/2\) implies the exact squared-mass ratio
\((m_3^{\mathrm{pred}})^2/(m_2^{\mathrm{pred}})^2=\phig^{7}\) (Eq.~\eqref{eq:nu_sq_mass_ratio_phi7}). \HYP
If future absolute-mass information (together with ordering identification) determines \(m_3^2/m_2^2\) in a way that excludes \(\phig^7\),
this rung-gap hypothesis is refuted. \VAL

\subsection{Scale falsifiers (test the declared eV reporting seam)}
\paragraph{F4: exclusion by oscillation windows for \(\Delta m^2\).}
The deep ladder predicts specific eV-scale splittings (Sec.~4) once the global seam is fixed. \CERT
If updated NuFIT (or successor) summary windows for the stated ordering exclude \(\Delta m^2_{21}{}^{\mathrm{pred}}\) or \(\Delta m^2_{31}{}^{\mathrm{pred}}\)
at high significance, then either the rung triple or the declared eV seam is refuted. \VAL

\paragraph{F5: cosmological exclusion of \(\Sigma_\nu\).}
The predicted mass sum is \(\Sigma_\nu^{\mathrm{pred}}\approx 0.0627\,\mathrm{eV}\) (Eq.~\eqref{eq:sigma_nu_pred_bounds}). \CERT
If cosmological analyses (under clearly stated model assumptions) establish an upper bound \(\Sigma_\nu < 0.0626\,\mathrm{eV}\),
then the deep-ladder mass scale is ruled out. \VAL

\paragraph{F6: direct absolute-mass detection above the predicted scale.}
Any direct kinematic or laboratory measurement that robustly implies a neutrino mass scale well above the predicted window
of Eq.~\eqref{eq:nu_mass_bounds} (under the same declared reporting seam) refutes the deep-ladder mass assignment. \VAL

