\documentclass[11pt]{article}

\usepackage[margin=1in]{geometry}
\usepackage{amsmath,amssymb}
\usepackage{microtype}
\usepackage{xcolor}
\usepackage{hyperref}

\hypersetup{
  colorlinks=true,
  linkcolor=blue,
  citecolor=blue,
  urlcolor=blue
}

% --- Notation (match repo conventions) ---
\newcommand{\phiG}{\varphi}
\newcommand{\muStar}{\mu_\star}
\newcommand{\lnphi}{\ln\phiG}
\newcommand{\Fgap}{\mathcal F}

\newcommand{\fRec}{f^{\mathrm{Rec}}}
\newcommand{\fRG}{f^{\mathrm{RG}}}

\newcommand{\Ecoh}{E_{\mathrm{coh}}}
\newcommand{\mPole}{m^{\mathrm{pole}}}
\newcommand{\mStruct}{m^{\mathrm{struct}}}

\title{\textbf{Internal Memo: Resolving the ``Single-Anchor'' Mass Questions}\\[0.35em]
\large Anchor-specific identity, RG transport vs.\ recognition residue, and non-circularity}

\author{Jonathan Washburn\\
\texttt{jon@recognitionphysics.org}}

\date{\today}

\begin{document}
\maketitle

\noindent\textbf{To:} A.\ Thapa (and internal mass-paper contributors)\\
\textbf{Subject:} Response to concerns raised in \texttt{sm\_running\_mass\_note\_anil.pdf}

\vspace{0.5em}

\section*{Executive Summary (what changes, what does not)}

\begin{itemize}
  \item \textbf{Anchor-specific, not RG-invariant.}
  The mass regularity we report is \emph{defined at a single common scale} \(\muStar\).
  We do \emph{not} claim that
  \(\Delta_i(\mu)\equiv f_i(\mu,m_i)-\Fgap(Z_i)\)
  is constant under RG flow for arbitrary \(\mu\).
  Off-anchor drift is expected and is not a contradiction.

  \item \textbf{Two distinct exponents (do not conflate).}
  The framework separates a \emph{recognition/geometric residue} \(\fRec(Z)\) (large, closed form)
  from a \emph{Standard-Model RG transport} \(\fRG_i\) (small, loop-computed transport factor).
  The large band values such as \(\Fgap(1332)\approx 13.95\) cannot be the same object as the
  small SM transport exponent for leptons.

  \item \textbf{Non-circularity is an operational rule.}
  Any use of ``structural mass'' must be specified so that \(\mStruct_i\) is fixed
  \emph{without using} the same measured \(m_i\) that the comparison is meant to test.
  Sector-level calibration and hold-out checks are acceptable; per-species tuning is not.
\end{itemize}

\section{Definitions (minimal, consistent with repo)}

\paragraph{Golden ratio and fixed constants.}
We use \(\phiG=(1+\sqrt5)/2\) and the canonical normalization
\[
  \lambda=\ln\phiG,\qquad \kappa=\phiG.
\]
We also use a single common reference scale
\[
  \muStar = 182.201~\mathrm{GeV}
  \quad\text{(the common ``anchor'' scale).}
\]

\paragraph{Charge-to-integer map \(Z(Q,\mathrm{sector})\).}
Define the integerized charge \(\widetilde Q:=6Q\in\mathbb Z\) for Standard-Model charges.
Then define
\begin{equation}
  Z \;=\;
  \begin{cases}
    4+\widetilde Q^2+\widetilde Q^4, & \text{quarks},\\[2pt]
    \widetilde Q^2+\widetilde Q^4,   & \text{charged leptons},\\[2pt]
    0,                               & \text{(Dirac) neutrinos (conditional statement).}
  \end{cases}
  \label{eq:Zmap}
\end{equation}

\paragraph{Closed-form band function.}
Define the closed form
\begin{equation}
  \Fgap(Z)\;:=\;\frac{1}{\ln\phiG}\,\ln\!\Bigl(1+\frac{Z}{\phiG}\Bigr).
  \label{eq:Fgap}
\end{equation}
For reference, the three equal-\(Z\) charged-fermion families correspond to
\[
  Z_{u,c,t}=276,\quad Z_{d,s,b}=24,\quad Z_{e,\mu,\tau}=1332,
\]
hence \(\Fgap(276)\approx 10.69\), \(\Fgap(24)\approx 5.74\), \(\Fgap(1332)\approx 13.95\).

\paragraph{SM RG transport exponent (small).}
Let \(m_i(\mu)\) denote the usual \(\overline{\mathrm{MS}}\) running mass of species \(i\).
Define the SM transport exponent between two scales (e.g.\ \(\muStar\) and a reference/pole point) by
\begin{equation}
  \fRG_i(\mu_1,\mu_2)
  \;:=\;
  \frac{1}{\ln\phiG}\,\ln\!\Bigl(\frac{m_i(\mu_2)}{m_i(\mu_1)}\Bigr)
  \;=\;
  \frac{1}{\ln\phiG}\int_{\ln\mu_1}^{\ln\mu_2}\gamma_i(\mu)\,d\ln\mu,
  \label{eq:fRG}
\end{equation}
with \(\gamma_i(\mu)=\gamma^{\mathrm{QCD}}_m+\gamma^{\mathrm{QED}}_m(+\text{EW if included})\)
evaluated under a declared loop order, threshold, and coupling policy.
Empirically and by scale/loop size, \(\fRG_i\) for leptons is \(O(10^{-2}\text{--}10^{-1})\)
for the \(\muStar\leftrightarrow m_\ell\) interval.

\paragraph{Recognition residue (band coordinate).}
We use the term ``recognition residue'' for the large band coordinate and set
\begin{equation}
  \fRec(Z) \;:=\; \Fgap(Z).
  \label{eq:fRec}
\end{equation}
This is a \emph{separate object} from \(\fRG_i\) in \eqref{eq:fRG}.

\section{What the anchor claim is (and is not)}

\subsection*{Not RG invariance}
Your note correctly observes that if one defines
\[
  \Delta_i(\mu) := f_i(\mu,m_i)-\Fgap(Z_i),
\]
then generically \(\Delta_i(\mu)\) will vary with \(\mu\) under RG flow.
We agree.
The claim we intend to report is instead:

\medskip
\noindent\textbf{Anchor-specific posture.}
\emph{There exists a single common scale \(\muStar\) at which the nine charged fermions organize
into three equal-\(Z\) bands described by the closed form \(\Fgap(Z)\), within declared policy bands.}
Off-anchor, standard SM RG applies; the anchor is a distinguished point, not a symmetry.

\subsection*{The key bookkeeping: transport vs.\ band coordinate}
The mismatch highlighted in your Table~1 comes from treating the large band value \(\Fgap(Z)\)
as though it were the SM transport integral itself.
That identification cannot be correct for leptons because the transport exponent is small while
\(\Fgap(1332)\) is \(O(10)\).
Accordingly, the framework uses \(\fRG_i\) only as a \emph{transport factor} (scheme/scale alignment),
and \(\fRec(Z)\) as the \emph{recognition-side coordinate} that carries the large band value.

\section{Where ``structural mass'' enters, and how to avoid tautology}

\subsection*{Mass-factorization display (what gets compared to data)}
The RS mass display splits a predicted ``structural'' prefactor from multiplicative exponents:
\begin{equation}
  \mPole_i
  \;=\;
  \mStruct_i\ \phiG^{\,\fRec(Z_i)}\ \phiG^{\,\fRG_i},
  \label{eq:mass-factorization}
\end{equation}
where the SM part is only \(\fRG_i\) (computed from declared RG kernels/policies),
and the large band coordinate is \(\fRec(Z_i)=\Fgap(Z_i)\).

\subsection*{The tautology failure mode}
If \(\mStruct_i\) (or any per-species offsets entering it) is extracted from the same measured
\(\mPole_i\) being ``tested'', then \eqref{eq:mass-factorization} can be made to hold by construction,
and the comparison carries no evidentiary content.
This is the circularity you flagged, and we agree it must be avoided.

\subsection*{Operational non-circularity rule (what we require)}
\noindent\textbf{Rule (no self-use).}
\emph{No measured mass \(m_i\) may appear on the right-hand side of its own prediction.
Any calibration must be sector-global (or done on a declared hold-out subset) and then frozen.}

\paragraph{Practically.}
There are two acceptable approaches:
\begin{enumerate}
  \item \textbf{Sector-global calibration (minimal knobs).}
  Choose \(\mStruct_i\) from a sector-wide form
  \(\mStruct_i = B_{B(i)}\,\Ecoh\,\phiG^{r_i+r_0(B(i))}\)
  with \emph{one} \(r_0\) per sector (and fixed \(B_B,\Ecoh\)),
  and fix those sector parameters once (from theory-side constraints or a declared small calibrator set),
  then predict the remaining species out-of-sample.

  \item \textbf{Strict hold-out.}
  Declare a subset used for any unavoidable calibration, freeze all constants, and then report
  all remaining species as a hold-out audit with no retuning.
\end{enumerate}

\paragraph{What your Eq.\ \(f^{\mathrm{Rec}}=\Delta_{\mathrm{obs}}-\fRG\) means.}
Rearranging \eqref{eq:mass-factorization} at the anchor gives
\[
  \fRec_i
  \;=\;
  \frac{1}{\ln\phiG}\ln\!\Bigl(\frac{m_i(\muStar)}{\mStruct_i}\Bigr),
  \qquad
  \text{and the test is}\quad
  \fRec_i \stackrel{?}{\approx} \Fgap(Z_i).
\]
This is valid \emph{only if} \(\mStruct_i\) is fixed independently of \(m_i\) (except via a declared calibration protocol).

\section{Direct response to \emph{Section 4} of your note (the serious point)}
\label{sec:section4}

\paragraph{Your Section 4 observation is correct.}
If one \emph{defines} a ``recognition residue'' by rearranging the measured masses,
e.g.\ from an equation of the form
\(\mPole_i=\mStruct_i\,\phiG^{\fRec_i+\fRG_i}\),
then
\[
  \fRec_i
  =
  \frac{1}{\ln\phiG}\ln\!\Bigl(\frac{m_i(\muStar)}{\mStruct_i}\Bigr)
\]
is just bookkeeping.
If \(\mStruct_i\) is itself chosen using \(m_i\) (explicitly or implicitly), the test becomes tautological.
We agree with this critique.

\paragraph{Resolution (what we actually mean by \(\fRec\)).}
In our framework, \(\fRec\) is \emph{not} an ``extracted from PDG'' residue.
It is a \emph{theory-side coordinate} fixed \emph{before} looking at masses:
\[
  \fRec(Z)\equiv \Fgap(Z),\qquad Z=Z(Q,\mathrm{sector})\ \text{from \eqref{eq:Zmap}}.
\]
This is exactly why we insist on the two-exponent split: \(\fRG_i\) is SM transport (small),
while \(\fRec(Z)\) is the band coordinate (large).

\paragraph{Resolution (what we mean by \(\mStruct\)).}
Likewise, \(\mStruct_i\) must be specified so it does \emph{not} import \(m_i\) species-by-species.
The intended structural form is sector-global, e.g.
\[
  \mStruct_i := B_{B(i)}\,\Ecoh\,\phiG^{r_i+r_0(B(i))},
\]
where \(B_{B}\) and \(r_0(B)\) are fixed once per sector (not per species) and \(r_i\in\mathbb Z\) is an integer rung.
Whether those sector-global constants come from independent anchors or from a declared hold-out calibration is a
separate methodological choice, but \emph{per-species} tuning is disallowed.

\paragraph{What a real (non-tautological) check looks like.}
Once the structural inputs are frozen, the comparison becomes meaningful:
\begin{enumerate}
  \item Fix \(Z_i\) from charge/sector \eqref{eq:Zmap} and set \(\fRec_i:=\Fgap(Z_i)\) (no fit).
  \item Fix \(\mStruct_i\) from sector-global parameters and integer rungs (no per-species tuning).
  \item Compute \(\fRG_i\) from SM RG kernels/policies (transport only).
  \item Compare the predicted \(\mPole_i\) (or \(m_i(\mu)\) in a declared scheme) to the experimental value after transport to like-for-like conventions.
\end{enumerate}
In this protocol, \(\fRec\) is not being \emph{solved for} from the same mass being tested, and \(\fRG\) does not
have to ``explain'' a band-sized exponent.

\paragraph{On introducing extra shifts \(\delta\).}
Your conclusion is right: a species-dependent \(\delta_i\) is just fitting.
Only a universal \(\delta\) or a sector-global \(\delta_B\) (fixed once and then frozen) can be admissible,
and it must be declared as part of the structural model, not silently adjusted to chase \(\Fgap(Z)\).

\section{``Not RG invariant'' vs.\ ``unstable under small perturbations''}
\label{sec:stability}

\paragraph{What you are worried about (and why it is reasonable).}
If an equality holds only at one chosen scale, it can feel like a brittle coincidence:
change the scheme, move thresholds, change loop order, or nudge the scale, and the equality disappears.
That concern is valid \emph{unless} the anchor is a \emph{stationary/plateau point} where the observable is
first-order insensitive to small changes in the calibration choices.

\paragraph{What we mean by stability (not invariance).}
We are \emph{not} asserting an RG symmetry.
We are asserting that a single globally chosen anchor \(\muStar\) behaves like a \emph{preferred measurement point}:
near \(\muStar\), small coherent perturbations in the policy (threshold placements within PDG bands, loop-order truncation,
minor scheme variants) move \emph{all} residues coherently, so that equal-\(Z\) family structure is preserved to first order.
In plain terms: the bands shift together; they do not fragment.

\paragraph{Where the ``plateau'' enters.}
In the mass-paper posture, \(\muStar\) is selected by a species-independent stationarity objective
on regrouped kernel/motif weights (a PMS/BLM-like ``minimal sensitivity'' choice).
This is different from claiming that each species satisfies a separate stationarity condition.
The testable consequence is:
\begin{itemize}
  \item equal-\(Z\) family \emph{differences} are comparatively insensitive to \(\mu\) near \(\muStar\),
  \item changes in global inputs shift families coherently (a band moves as a band).
\end{itemize}
This is the precise sense in which the identity can be non-invariant yet still not be ``fine-tuned.''

\paragraph{Operational evidence (what the artifacts are supposed to show).}
The intended numerical audit is not ``it works at one point, period'' but:
\begin{itemize}
  \item a central run at declared kernels/policies meets the tolerance at \(\muStar\),
  \item global variations (loop order, threshold policy, \(\alpha(\mu)\) policy) shift residuals by a small,
        coherent band (reported as a systematic),
  \item structural ablations (drop the quark \(+4\), drop \(Q^4\), replace \(6Q\to 3Q\)) break the equality by orders of magnitude.
\end{itemize}

\section{Suggested wording for the manuscript (to match our intended posture)}

\begin{itemize}
  \item \textbf{Be explicit about scope.}
  ``The band identity is an empirical regularity at one common anchor \(\muStar\); off-anchor, SM RG applies.''

  \item \textbf{Name the two exponents distinctly.}
  Use \( \fRec(Z)=\Fgap(Z) \) for the band coordinate and reserve \( \fRG_i \) for SM transport.
  Avoid calling \(\Fgap(Z)\) an ``RG residue'' without qualifiers.

  \item \textbf{State the non-circularity rule in the main text.}
  One sentence plus a checklist is enough: sector-global calibration allowed; no per-species tuning.
\end{itemize}

\section*{Repo pointers (for internal verification)}
\begin{itemize}
  \item Definitions and policy notes: \texttt{Recognition-Science-Full-Theory.txt} (blocks \texttt{@SM\_MASSES}, \texttt{@RG\_METHODS}).
  \item One-file narrative source: \texttt{book/papers/txt/Source-Super.txt} (entries \texttt{@SM\_MASSES}).
  \item Formal ``do-not-conflate'' sanity check: \texttt{IndisputableMonolith/Physics/MassResidueNoGo.lean}.
  \item Mass-law display split: \texttt{docs/Mass-From-Light-Memo.tex} (Eq.~\eqref{eq:mass-factorization} style).
\end{itemize}

\appendix
\section{Lean formalization: what is defined/proved vs.\ what is an interface}
\label{app:lean}

\paragraph{Purpose of this appendix.}
This is not meant as a ``Lean proof of the SM 4-loop kernels.''
It is a map of what we have actually formalized and how it supports the paper's bookkeeping clarity
(separation of objects; non-conflation; structural definitions frozen before comparison).

\subsection*{Core structural definitions (model layer)}
\begin{itemize}
  \item \texttt{IndisputableMonolith/Masses/Anchor.lean}:
  defines \(\Ecoh=\phi^{-5}\), sector yardsticks \((B\_\mathrm{pow},r_0)\), rung tables \(r_i\), and the integer charge map \(Z(Q,\mathrm{sector})\).
\end{itemize}

\subsection*{Closed-form band function \(\Fgap(Z)\) (proved properties)}
\begin{itemize}
  \item \texttt{IndisputableMonolith/RSBridge/GapProperties.lean}:
  defines \texttt{gap} and proves monotonicity, concavity, diminishing increments, and ordering of the certified bands (e.g.\ \(Z=24<276<1332\)).
\end{itemize}

\subsection*{Non-conflation sanity check (proved separation)}
\begin{itemize}
  \item \texttt{IndisputableMonolith/Physics/MassResidueNoGo.lean}:
  proves that any ``small'' quantity cannot equal the large lepton band value \(\texttt{gap 1332}\),
  formalizing the repo's ``do not set \(\fRG=\fRec\)'' warning.
\end{itemize}

\subsection*{RG transport as a mathematical interface (not the SM kernel implementation)}
\begin{itemize}
  \item \texttt{IndisputableMonolith/Physics/RGTransport.lean}:
  defines the transport integral interface (\texttt{AnomalousDimension}, \texttt{integratedResidue}) and relates it to mass ratios.
  It is intentionally abstract; the QCD/QED kernel coefficients are treated as external inputs in the phenomenology artifacts.
\end{itemize}

\subsection*{Example: interval-checked electron mass derivation (internal RS-side derivation)}
\begin{itemize}
  \item \texttt{IndisputableMonolith/Physics/ElectronMass/Necessity.lean} and \texttt{IndisputableMonolith/Physics/ElectronMass.lean}:
  contain an interval-arithmetic derivation pipeline for the electron mass within the RS model layer, and track what precision is currently provable.
\end{itemize}

\subsection*{Additional mass modules (context)}
\begin{itemize}
  \item \texttt{IndisputableMonolith/Physics/QuarkMasses.lean}: a ``quarter-ladder'' quark derivation scaffold (includes some numerical axioms; labeled as such).
  \item \texttt{IndisputableMonolith/RRF/Physics/ParticleMass.lean}: an RRF-level structural mass framework and ratio identities.
\end{itemize}

\section{``Mass from light'' derivation (where to find it)}
\label{app:mass-from-light}

For a self-contained standard SR/GR derivation of ``mass from light'' (invariant mass from four-momentum,
two-photon example, box-of-light, Einstein two-pulse argument), see the companion memo:
\texttt{docs/Mass-From-Light-Memo.tex} (compiled PDF recommended for sharing).

\vspace{0.5em}
\noindent\textbf{Bottom line.}
Your diagnosis in the note is right: treating the large band numbers as literal SM transport residues creates
RG-invariance and magnitude contradictions.
The resolution is to keep the two residues distinct (band coordinate vs.\ SM transport),
state the claim as anchor-specific, and enforce a strict non-circularity protocol for any structural prefactors.

\end{document}


