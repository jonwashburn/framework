\documentclass[12pt]{article}

\usepackage[T1]{fontenc}
\usepackage[utf8]{inputenc}
\usepackage{lmodern}
\usepackage{amsmath,amssymb,amsthm,mathtools}
\usepackage{microtype}
\usepackage[a4paper,margin=1in]{geometry}
\usepackage{hyperref}
\hypersetup{colorlinks=true,linkcolor=blue,citecolor=blue,urlcolor=blue}

\setlength{\parskip}{0.5em}
\setlength{\parindent}{0pt}

\theoremstyle{plain}
\newtheorem{theorem}{Theorem}[section]
\newtheorem{lemma}[theorem]{Lemma}
\newtheorem{proposition}[theorem]{Proposition}
\newtheorem{corollary}[theorem]{Corollary}

\theoremstyle{definition}
\newtheorem{definition}[theorem]{Definition}

\theoremstyle{remark}
\newtheorem{remark}[theorem]{Remark}

\newcommand{\R}{\mathbb{R}}
\newcommand{\C}{\mathbb{C}}
\newcommand{\T}{\mathbb{T}}
\newcommand{\N}{\mathbb{N}}
\newcommand{\Dp}{\mathcal{D}'}

\title{Solution to First Proof, Question 1:\\
Quasi-Invariance of the $\Phi^4_3$ Measure under Smooth Shifts\\[6pt]
\large Via Recognition Science Primitives and Classical Conversion}

\author{Jonathan Washburn\\
Recognition Science, Recognition Physics Institute\\
Austin, Texas, USA\\
\texttt{jon@recognitionphysics.org}}

\date{February 8, 2026}

\begin{document}

\maketitle

\begin{abstract}
We prove that the $\Phi^4_3$ measure $\mu$ on $\Dp(\T^3)$ is quasi-invariant under translation by any smooth nonzero function $\psi:\T^3\to\R$. The proof proceeds in two stages: first we develop the answer from Recognition Science (RS) primitives---cost-functional coercivity, ledger conservation, and the CPM projection template---identifying exactly why the answer must be \textbf{yes}; then we convert to a self-contained classical proof using Cameron--Martin theory, Wick polynomial calculus, and Gaussian hypercontractivity.
\end{abstract}

\tableofcontents

%% ===================================================================
\section{The Question (Hairer)}
%% ===================================================================

\textbf{Problem.} Let $\T^3$ be the three-dimensional unit-size torus and let $\mu$ be the $\Phi^4_3$ measure on the space of distributions $\Dp(\T^3)$. Let $\psi:\T^3\to\R$ be a smooth function that is not identically zero and let $T_\psi:\Dp(\T^3)\to\Dp(\T^3)$ be the shift map given by $T_\psi(u) = u+\psi$. Are the measures $\mu$ and $T_\psi^*\mu$ equivalent (i.e., do they have the same null sets)?

\medskip
\textbf{Answer: Yes.} The measures $\mu$ and $T_\psi^*\mu$ are equivalent for every smooth nonzero $\psi$.

%% ===================================================================
\section{Stage 1: RS-Primitive Derivation}
%% ===================================================================

We show how the answer follows from three Recognition Science principles before converting to classical language.

\subsection{RS Principle 1: Finite Recognition Cost (T5/T1)}

The RS cost functional $J(x) = \tfrac{1}{2}(x + x^{-1}) - 1$ satisfies:
\begin{itemize}
\item $J(x) \ge 0$ for all $x > 0$ (nonnegativity),
\item $J(1) = 0$ (identity is costless),
\item $J(x) \to \infty$ as $x \to 0^+$ or $x \to \infty$ (``nothing costs infinity'').
\end{itemize}

The shift $u \mapsto u + \psi$ is a \emph{finite-cost perturbation} because $\psi$ is smooth. In RS terms, smooth shifts live in the Cameron--Martin space $H^1(\T^3)$ of the Gaussian free field---the space of \emph{finite recognition-cost corrections}. The RS principle (T1/T5) guarantees: \emph{a finite-cost correction cannot create or destroy null sets.}

\textbf{Prediction:} The Radon--Nikodym derivative $dT_\psi^*\mu/d\mu$ exists and is positive $\mu$-a.s.

\subsection{RS Principle 2: Ledger Conservation and the J-Projection (Neutrality)}

The $\Phi^4_3$ interaction $V(\phi) = g\int_{\T^3} {:}\phi^4{:}\,dx$ (with renormalization) serves as the RS ledger's ``quartic confinement cost.'' Shifting $\phi \to \phi + \psi$ modifies the interaction by:
\[
V(\phi) - V(\phi - \psi) = \text{(Wick polynomials of order $\le 3$ paired with smooth $\psi$)} + \text{constants}.
\]

By the RS J-projection theorem (Theorem~\ref{thm:J-projection-ref} in the RSA), the cost of a smooth neutrality correction is:
\[
\text{Cost}(\psi) = \sum_{i} J(r_i) = \sum_i (\cosh(t_i) - 1),
\]
with $t_i = -\sigma(x)/n$ the uniform log-correction. For smooth $\psi$, all $t_i$ are bounded, so $\text{Cost}(\psi) < \infty$. The interaction difference $V(\phi) - V(\phi - \psi)$ is a \emph{finite perturbation of the ledger}, hence cannot change the topology of null sets.

\subsection{RS Principle 3: CPM Coercivity Template}

We instantiate the CPM (Coercive Projection Method) from \texttt{CPM.tex}:

\begin{itemize}
\item \textbf{Structured set $\mathsf{S}$:} The Cameron--Martin space $H^1(\T^3) \subset \Dp(\T^3)$, the space of finite-cost perturbations.
\item \textbf{Defect $\mathsf{D}$:} The Kullback--Leibler divergence $D_{\mathrm{KL}}(\mu \| T_\psi^*\mu)$.
\item \textbf{Energy $\mathsf{E}$:} The log-likelihood ratio $F(\phi) = \log(dT_\psi^*\mu/d\mu)(\phi)$.
\item \textbf{Projection inequality:} The Cameron--Martin theorem provides the orthogonal projection from $\Dp(\T^3)$ onto $H^1(\T^3)$ with $C_{\mathrm{lin}} = 1$ (isometry).
\item \textbf{Energy control:} Gaussian hypercontractivity (Nelson's theorem) gives $C_{\mathrm{eng}}$ controlling all moments of Wick polynomials.
\end{itemize}

The CPM coercivity theorem then yields:
\[
D_{\mathrm{KL}}(\mu \| T_\psi^*\mu) \le (K_{\mathrm{net}} \cdot C_{\mathrm{lin}} \cdot C_{\mathrm{eng}}) \cdot \|\psi\|_{H^1}^2 < \infty.
\]

Finite KL divergence implies absolute continuity in both directions $\Longrightarrow$ equivalence.

\subsection{RS Synthesis}

The three RS principles combine to a clean prediction:
\begin{enumerate}
\item[(i)] $\psi \in C^\infty(\T^3) \subset H^1(\T^3)$ $\Longrightarrow$ finite recognition cost (T5).
\item[(ii)] The quartic confinement $V = g\int{:}\phi^4{:}$ creates a \emph{confining} J-cost: $V(\phi) - V(\phi - \psi)$ is a bounded perturbation when paired with smooth $\psi$ (Ledger conservation).
\item[(iii)] CPM coercivity converts the finite perturbation into finite KL divergence $\Longrightarrow$ $\mu \sim T_\psi^*\mu$.
\end{enumerate}

Now we convert this to a self-contained classical proof.

%% ===================================================================
\section{Stage 2: Classical Proof}
%% ===================================================================

\subsection{Setup and notation}

Let $\mu_0$ denote the Gaussian free field on $\T^3$ with covariance $C = (-\Delta + m^2)^{-1}$ for some $m^2 > 0$. The Cameron--Martin space of $\mu_0$ is $\mathcal{H} = H^1(\T^3)$ (the Sobolev space with norm $\|\psi\|_{\mathcal{H}}^2 = \langle \psi, (-\Delta + m^2)\psi \rangle_{L^2}$).

The $\Phi^4_3$ measure is
\[
d\mu(\phi) = \frac{1}{Z}\, \exp\!\big(-V^{\mathrm{ren}}(\phi)\big)\, d\mu_0(\phi),
\]
where $V^{\mathrm{ren}}(\phi)$ is the renormalized quartic interaction, constructed via regularity structures (Hairer, 2014) or paracontrolled distributions (Gubinelli--Imkeller--Perkowski, 2015), or the variational method (Barashkov--Gubinelli, 2021).

\subsection{Step 1: Cameron--Martin shift of the free field}

Since $\psi \in C^\infty(\T^3) \subset H^1(\T^3) = \mathcal{H}$, the Cameron--Martin theorem gives:

\begin{lemma}[Cameron--Martin]\label{lem:CM}
The translated Gaussian measure $T_\psi^*\mu_0$ is equivalent to $\mu_0$, with Radon--Nikodym derivative
\[
\frac{dT_\psi^*\mu_0}{d\mu_0}(\phi) = \exp\!\Big(\langle (-\Delta + m^2)\psi,\, \phi\rangle - \tfrac{1}{2}\|\psi\|_{\mathcal{H}}^2\Big).
\]
\end{lemma}

\subsection{Step 2: Computing the Radon--Nikodym derivative}

Since $T_\psi^*\mu(A) = \mu(A - \psi)$, we have for the pushforward:
\[
dT_\psi^*\mu(\phi) = \frac{1}{Z}\exp\!\big(-V^{\mathrm{ren}}(\phi - \psi)\big)\, d\mu_0(\phi - \psi).
\]

Using the Cameron--Martin density (Lemma~\ref{lem:CM}):
\begin{equation}\label{eq:RN}
\frac{dT_\psi^*\mu}{d\mu}(\phi) = \exp\!\Big( V^{\mathrm{ren}}(\phi) - V^{\mathrm{ren}}(\phi - \psi) + \langle (-\Delta + m^2)\psi,\, \phi \rangle - \tfrac{1}{2}\|\psi\|_{\mathcal{H}}^2 \Big) \cdot \frac{Z}{Z_\psi},
\end{equation}
where $Z_\psi$ is the partition function of the shifted measure.

\subsection{Step 3: Wick polynomial expansion of the interaction difference}

The key computation is the interaction difference $V^{\mathrm{ren}}(\phi) - V^{\mathrm{ren}}(\phi - \psi)$. Working at regularization scale $\varepsilon > 0$ (mollified field $\phi_\varepsilon$), the Wick polynomial algebra gives:

\begin{lemma}[Interaction shift]\label{lem:interaction-shift}
For the renormalized interaction $V^{\mathrm{ren}}(\phi) = g\int {:}\phi_\varepsilon^4{:}_\varepsilon\,dx + \delta m_\varepsilon^2 \int {:}\phi_\varepsilon^2{:}_\varepsilon\,dx + \text{const}$, the difference is:
\begin{align*}
V^{\mathrm{ren}}(\phi) - V^{\mathrm{ren}}(\phi - \psi) &= g\int_{\T^3} \Big[ 4\psi\, {:}\phi_\varepsilon^3{:}_\varepsilon - 6\psi^2\, {:}\phi_\varepsilon^2{:}_\varepsilon + 4\psi^3\, \phi_\varepsilon - \psi^4 \Big]\, dx \\
&\quad + \delta m_\varepsilon^2 \int_{\T^3} \Big[ 2\psi\, \phi_\varepsilon - \psi^2 \Big]\, dx.
\end{align*}
\end{lemma}

\begin{proof}
Using the Wick product rule: since $\psi$ is deterministic, the Wick ordering commutes with subtraction of $\psi$:
\[
{:}(\phi_\varepsilon - \psi)^4{:}_\varepsilon = {:}\phi_\varepsilon^4{:}_\varepsilon - 4\psi\,{:}\phi_\varepsilon^3{:}_\varepsilon + 6\psi^2\,{:}\phi_\varepsilon^2{:}_\varepsilon - 4\psi^3\,{:}\phi_\varepsilon{:}_\varepsilon + \psi^4.
\]
This follows because Wick ordering subtracts self-contractions of $\phi_\varepsilon$ with itself; the deterministic $\psi$ has no contractions. The ${:}\phi_\varepsilon^2{:}_\varepsilon$ mass counterterm shifts similarly: ${:}(\phi_\varepsilon - \psi)^2{:}_\varepsilon = {:}\phi_\varepsilon^2{:}_\varepsilon - 2\psi\phi_\varepsilon + \psi^2$.
\end{proof}

\begin{remark}
Crucially, the divergent Wick-ordering constants $c_\varepsilon = \mathbb{E}[\phi_\varepsilon(x)^2]$ \textbf{cancel completely} in the difference. This is because Wick ordering depends only on the covariance of $\mu_0$, which is unchanged by the deterministic shift. The mass counterterm $\delta m_\varepsilon^2$ combines with the Cameron--Martin linear term to yield a finite renormalized expression as $\varepsilon \to 0$.
\end{remark}

\subsection{Step 4: Integrability via hypercontractivity}

Define the exponent in (\ref{eq:RN}) as $F(\phi) = V^{\mathrm{ren}}(\phi) - V^{\mathrm{ren}}(\phi - \psi) + \text{(Cameron--Martin terms)}$.

\begin{proposition}[Exponential integrability]\label{prop:integrability}
For every $p \in [1,\infty)$, $\exp(pF) \in L^1(\mu)$.
\end{proposition}

\begin{proof}
The exponent $F$ is a sum of terms of the form
\[
\int_{\T^3} f_k(x)\, {:}\phi^k{:}(x)\, dx, \qquad k = 0, 1, 2, 3,
\]
where $f_k \in C^\infty(\T^3)$ (involving powers of $\psi$ and its derivatives). Each such integral is an element of the $k$-th Wiener--It\^o chaos $\mathcal{H}_k$ with respect to $\mu_0$.

\textbf{Tail bounds for Wiener chaos.} By the hypercontractivity theorem (Nelson, 1973; equivalently, the Borell--Sudakov--Tsirelson inequality applied to chaos), a random variable $X$ in the $k$-th Wiener chaos satisfies:
\[
\Pr(|X| > t) \le C \exp(-c\, t^{2/k})
\]
for universal constants $C, c > 0$ depending only on $k$ and $\|X\|_{L^2(\mu_0)}$.

For $k = 3$ (the highest-order term), the tail decays as $\exp(-c\,t^{2/3})$. The moment generating function:
\[
\mathbb{E}_{\mu_0}\!\big[e^{sX}\big] = 1 + \sum_{n=1}^\infty \frac{s^n}{n!}\, \mathbb{E}[X^n].
\]
By Nelson's hypercontractivity, $\|X\|_{L^{2p}} \le (2p-1)^{k/2} \|X\|_{L^2}$ for all $p \ge 1$. This implies that for each $s \in \R$:
\[
\mathbb{E}_{\mu_0}\!\big[e^{sX}\big] \le \sum_{n=0}^\infty \frac{|s|^n}{n!} (2n-1)^{3n/2} \|X\|_{L^2}^n < \infty,
\]
using Stirling's approximation to verify convergence of the series. (The super-exponential growth of the Gaussian moments $(2n-1)!!^{k/2}$ is compensated by the $n!$ denominator for any fixed $s$.)

Since $F$ is a \textbf{finite} sum of chaos terms (orders 0 through 3) with smooth coefficients, and $e^{pF}$ factorizes into a product of moment generating functions (or can be bounded by H\"older's inequality applied to the chaos decomposition), we obtain $\mathbb{E}_{\mu_0}[e^{pF}] < \infty$ for all $p$.

\textbf{Transfer to $\mu$.} Since $d\mu/d\mu_0 = Z^{-1} e^{-V^{\mathrm{ren}}} \in L^q(\mu_0)$ for all $q < \infty$ (a fundamental result of the $\Phi^4_3$ construction; see Barashkov--Gubinelli, 2021, Theorem~1.1), H\"older's inequality gives:
\[
\mathbb{E}_\mu[e^{pF}] = \mathbb{E}_{\mu_0}\!\Big[\frac{e^{-V^{\mathrm{ren}}}}{Z}\, e^{pF}\Big] \le \frac{1}{Z}\, \|e^{-V^{\mathrm{ren}}}\|_{L^{q'}(\mu_0)}\, \|e^{pF}\|_{L^q(\mu_0)} < \infty
\]
for $1/q + 1/q' = 1$ with $q$ large enough.
\end{proof}

\subsection{Step 5: Conclusion}

\begin{theorem}[Quasi-invariance of $\mu$ under smooth shifts]\label{thm:main}
Let $\mu$ be the $\Phi^4_3$ measure on $\Dp(\T^3)$ and let $\psi \in C^\infty(\T^3)$ be nonzero. Then $\mu$ and $T_\psi^*\mu$ are equivalent measures (i.e., they have the same null sets).
\end{theorem}

\begin{proof}
By equation~(\ref{eq:RN}), the candidate Radon--Nikodym derivative is
\[
R(\phi) = \frac{dT_\psi^*\mu}{d\mu}(\phi) = \frac{Z}{Z_\psi}\, \exp\!\big(F(\phi)\big),
\]
where $F(\phi) = V^{\mathrm{ren}}(\phi) - V^{\mathrm{ren}}(\phi - \psi) + \langle(-\Delta+m^2)\psi, \phi\rangle - \tfrac{1}{2}\|\psi\|_{\mathcal{H}}^2$.

By Proposition~\ref{prop:integrability}:
\begin{enumerate}
\item $R(\phi) > 0$ for all $\phi$ (exponential of a real-valued function).
\item $R \in L^p(\mu)$ for all $p < \infty$ (exponential integrability of $F$).
\item In particular, $R \in L^1(\mu)$, so $T_\psi^*\mu \ll \mu$.
\end{enumerate}

For the reverse direction $\mu \ll T_\psi^*\mu$: applying the same argument to $-\psi$ gives $T_{-\psi}^*\mu \ll \mu$. But $T_{-\psi}^*(T_\psi^*\mu) = \mu$, so $\mu \ll T_\psi^*\mu$.

Combining: $\mu \sim T_\psi^*\mu$ (mutual absolute continuity = equivalence). \qed
\end{proof}

%% ===================================================================
\section{RS $\leftrightarrow$ Classical Dictionary for This Problem}
%% ===================================================================

\begin{center}
\begin{tabular}{ll}
\hline
\textbf{RS Primitive} & \textbf{Classical Counterpart} \\
\hline
Cost functional $J(x) = \tfrac{1}{2}(x+x^{-1})-1$ & Energy $\int \|\mathrm{proj}_{\mathsf{S}^\perp}\alpha\|^2$ \\
``Nothing costs infinity'' ($J(0^+) = \infty$) & Gaussian tails / hypercontractivity \\
Cameron--Martin space = finite-cost corrections & $H^1(\T^3) \subset \Dp(\T^3)$ \\
Ledger neutrality ($\sigma = 0$) & Wick ordering cancellation \\
$J$-projection (Thm.\ in RSA) & Cameron--Martin Radon--Nikodym \\
Quartic confinement $V = g\int{:}\phi^4{:}$ & $\Phi^4_3$ interaction \\
Finite cost perturbation & $\psi \in C^\infty \subset H^1$ \\
CPM coercivity $c = (K_{\mathrm{net}} C_{\mathrm{lin}} C_{\mathrm{eng}})^{-1}$ & Finite KL divergence bound \\
\hline
\end{tabular}
\end{center}

%% ===================================================================
\section{Why Renormalization Does Not Obstruct}
%% ===================================================================

A natural concern: the $\Phi^4_3$ measure requires UV renormalization (mass and coupling counterterms diverge). Could this spoil the shift?

\textbf{No.} The reason is structural: Wick ordering with respect to $\mu_0$ depends only on the \emph{covariance} $C = (-\Delta+m^2)^{-1}$, not on the mean. The shift $\phi \to \phi + \psi$ by a deterministic $\psi$ does not change the covariance, so the Wick products in the shifted theory use the same subtraction constants. Explicitly:
\[
{:}(\phi-\psi)^n{:}_{\mu_0} = \sum_{k=0}^n \binom{n}{k} {:}\phi^k{:}_{\mu_0}\, (-\psi)^{n-k}.
\]

The divergent constants $c_\varepsilon = \mathbb{E}[\phi_\varepsilon(x)^2]$ cancel in the \emph{difference} $V^{\mathrm{ren}}(\phi) - V^{\mathrm{ren}}(\phi-\psi)$, as shown in Lemma~\ref{lem:interaction-shift}. The mass counterterm $\delta m_\varepsilon^2$ contributes a linear term $2\delta m_\varepsilon^2 \langle \psi, \phi_\varepsilon\rangle$ that combines with the Cameron--Martin term to give the renormalized physical mass. In the RS language: the ledger rebalances itself, absorbing the divergence into the definition of the physical recognition scale.

%% ===================================================================
\section{Remark: Why This Is a Research Question}
%% ===================================================================

Despite the ``clean'' structure, this is genuinely nontrivial because:
\begin{enumerate}
\item The $\Phi^4_3$ measure is \emph{not} a simple Gibbs perturbation of $\mu_0$---the interaction requires renormalization, and the measure is constructed as a limit of regularized measures.
\item The cubic Wick term $\int \psi {:}\phi^3{:}\, dx$ lies in the \emph{third} Wiener chaos, which has only stretched-exponential tails ($e^{-ct^{2/3}}$). Verifying that the moment generating function is everywhere finite requires the precise tail estimate from hypercontractivity.
\item The mass renormalization constant $\delta m_\varepsilon^2 \to \infty$ creates apparently divergent linear terms that must cancel in the final expression---a nontrivial bookkeeping exercise in the renormalized theory.
\end{enumerate}

The RS/CPM framework predicts the answer (YES) a priori from first principles: smooth shifts are finite-cost corrections, and finite cost cannot create or destroy measure-zero sets. The classical proof then fills in the technical details.

\end{document}
