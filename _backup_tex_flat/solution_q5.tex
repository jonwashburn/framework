\documentclass[11pt]{article}

\usepackage{amsmath,amssymb,amsthm}
\usepackage[a4paper,margin=1in]{geometry}
\usepackage{hyperref}
\hypersetup{colorlinks=true,linkcolor=blue,citecolor=blue,urlcolor=blue}

\setlength{\parskip}{0.5em}
\setlength{\parindent}{0pt}

\theoremstyle{plain}
\newtheorem{theorem}{Theorem}
\newtheorem{lemma}[theorem]{Lemma}
\newtheorem{proposition}[theorem]{Proposition}

\theoremstyle{definition}
\newtheorem{definition}[theorem]{Definition}

\theoremstyle{remark}
\newtheorem{remark}[theorem]{Remark}

\newcommand{\SpG}{\mathrm{Sp}^G}
\newcommand{\Sp}{\mathrm{Sp}}
\newcommand{\Loc}{\mathrm{Loc}}
\newcommand{\Res}{\mathrm{Res}}
\newcommand{\Ind}{\mathrm{Ind}}
\newcommand{\PhiH}{\Phi}
\newcommand{\R}{\mathbb{R}}
\newcommand{\Z}{\mathbb{Z}}
\newcommand{\ceil}[1]{\left\lceil #1 \right\rceil}

\title{Solution to First Proof, Question~5:\\
$\mathcal{O}$-Slice Filtration and Geometric\\
Fixed-Point Characterization\\[6pt]
\large Via Recognition Science Primitives and Classical Conversion}

\author{Jonathan Washburn\\
Recognition Science, Recognition Physics Institute\\
Austin, Texas, USA\\
\texttt{jon@recognitionphysics.org}}

\date{February 9, 2026}

\begin{document}

\maketitle

\begin{abstract}
We define the slice filtration on the $G$-equivariant stable category adapted to an incomplete transfer system $\mathcal{O}$ associated to an $N_\infty$ operad, and prove that a connective $G$-spectrum $X$ is $\mathcal{O}$-slice $\geq n$ if and only if $\Phi^H X$ is $\lceil n/d_{\mathcal{O}}(H)\rceil$-connective for every subgroup $H \leq G$, where $d_{\mathcal{O}}(H) = [H : K_{\mathcal{O}}(H)]$ is the index of the minimal $\mathcal{O}$-admissible subgroup. The proof uses induction on $|G|$ via the isotopy separation cofiber sequence.
\end{abstract}

\tableofcontents

%% ===================================================================
\section{The Question (Blumberg)}
%% ===================================================================

Fix a finite group $G$. Let $\mathcal{O}$ be an incomplete transfer system associated to an $N_\infty$ operad. Define the slice filtration on the $G$-equivariant stable category adapted to $\mathcal{O}$ and state and prove a characterization of the $\mathcal{O}$-slice connectivity of a connective $G$-spectrum in terms of geometric fixed points.

%% ===================================================================
\section{Stage 1: Recognition Science Primitives View}
%% ===================================================================

Recognition Science (RS) treats ``what is allowed'' as a \emph{primitive} (an admissibility structure) and treats ``what can be stably detected'' as \emph{vantage-dependent recognition}. In this problem:
\begin{itemize}
\item The admissibility primitive is the incomplete transfer system $\mathcal{O}$ (which inclusions $K\le H$ are \emph{recognition-valid} as transfers/norm-orbits).
\item The vantage primitive is geometric fixed points $\Phi^H(-)$: what an $H$-observer can stably recognize after removing proper-isotropy artifacts.
\item The ``engineering'' quantity is a \emph{cost multiplier} $d_{\mathcal{O}}(H)$: how many ambient degrees of freedom must be paid to get one unit of $H$-recognizable degree, given only the transfers allowed by $\mathcal{O}$.
\end{itemize}
The RSA-style move is: replace a hard global statement (``$X$ has no low slices'') by a family of \emph{audits} (``no $H$-vantage sees low-degree signal''), and prove these are equivalent for connective inputs.

%% ===================================================================
\section{Stage 2: Classical Construction and Proof}
%% ===================================================================

\subsection{Setup: the $\mathcal{O}$-cost multiplier and $\mathcal{O}$-regular representation}

\begin{definition}[Incomplete transfer system data]
Write $K\le_{\mathcal{O}} H$ to mean the inclusion $K\le H$ is \emph{$\mathcal{O}$-admissible} (equivalently, the orbit $H/K$ is an admissible $H$-set for the indexing system of the $N_\infty$ operad).
\end{definition}

A basic closure property of transfer systems is that if $K_1\le_{\mathcal{O}} H$ and $K_2\le_{\mathcal{O}} H$, then $K_1\cap K_2\le_{\mathcal{O}} H$. (This follows from the restriction axiom---$K_1 \leq_{\mathcal{O}} H$ and $K_2 \leq H$ gives $K_1 \cap K_2 \leq_{\mathcal{O}} K_2$---combined with transitivity---$K_1 \cap K_2 \leq_{\mathcal{O}} K_2$ and $K_2 \leq_{\mathcal{O}} H$ gives $K_1 \cap K_2 \leq_{\mathcal{O}} H$.)

\begin{definition}[Minimal $\mathcal{O}$-source and cost multiplier]
For each subgroup $H\le G$, define the \emph{minimal $\mathcal{O}$-source} subgroup
\[
K_{\mathcal{O}}(H)\;:=\;\bigcap\{\,K\le H \mid K\le_{\mathcal{O}} H\,\}.
\]
Define the \emph{$\mathcal{O}$-cost multiplier} (an integer)
\[
d_{\mathcal{O}}(H)\;:=\;[H:K_{\mathcal{O}}(H)].
\]
\end{definition}

\begin{definition}[$\mathcal{O}$-regular permutation representation]
Define a real $H$-representation
\[
\rho^{\mathcal{O}}_H \;:=\;\R[\,H/K_{\mathcal{O}}(H)\,],
\]
the permutation representation on the transitive $H$-set $H/K_{\mathcal{O}}(H)$.
Then $\dim(\rho^{\mathcal{O}}_H)=d_{\mathcal{O}}(H)$ and $(\rho^{\mathcal{O}}_H)^H\cong \R$ is $1$-dimensional.
\end{definition}

\subsection{Definition: the $\mathcal{O}$-slice filtration}
Let $\SpG$ denote the genuine $G$-equivariant stable homotopy category.

\begin{definition}[$\mathcal{O}$-slice cells and $\mathcal{O}$-dimension]
For $H\le G$ and $m\in\Z_{\ge 0}$, define the \emph{$\mathcal{O}$-slice cells}
\[
G/H_+\wedge S^{m\rho_H^{\mathcal{O}}}
\qquad\text{and}\qquad
G/H_+\wedge S^{m\rho_H^{\mathcal{O}}-1}.
\]
Define their \emph{$\mathcal{O}$-dimension} by
\[
\dim_{\mathcal{O}}\!\bigl(G/H_+\wedge S^{m\rho_H^{\mathcal{O}}}\bigr)=m\,d_{\mathcal{O}}(H),
\qquad
\dim_{\mathcal{O}}\!\bigl(G/H_+\wedge S^{m\rho_H^{\mathcal{O}}-1}\bigr)=m\,d_{\mathcal{O}}(H)-1.
\]
\end{definition}

\begin{definition}[$\mathcal{O}$-slice filtration]
For each integer $n$, define $\tau^{\mathcal{O}}_{\ge n}\subset \SpG$ to be the localizing subcategory generated by all $\mathcal{O}$-slice cells of $\mathcal{O}$-dimension $\ge n$:
\[
\tau^{\mathcal{O}}_{\ge n}
\;:=\;
\Loc\Bigl\{
G/H_+\wedge S^{m\rho_H^{\mathcal{O}}},\, G/H_+\wedge S^{m\rho_H^{\mathcal{O}}-1}
\;\Big|\;
H\le G,\; m\ge 0,\; m\,d_{\mathcal{O}}(H)\ge n
\Bigr\}.
\]
A $G$-spectrum $X$ is \emph{$\mathcal{O}$-slice $\ge n$} (or \emph{$\mathcal{O}$-slice $n$-connective}) if $X\in\tau^{\mathcal{O}}_{\ge n}$.
\end{definition}

\subsection{Geometric fixed points and connectivity}

Let $\Phi^H:\SpG\to \Sp$ denote the geometric fixed points functor.

\begin{definition}[Non-equivariant connectivity]
A (non-equivariant) spectrum $Y$ is \emph{$r$-connective} if $\pi_k(Y)=0$ for all $k<r$.
\end{definition}

\begin{definition}[Connective $G$-spectrum]
A $G$-spectrum $X$ is \emph{connective} if its underlying non-equivariant spectrum is connective (i.e.\ $\pi_k(\Phi^{e}X)=0$ for $k<0$).
\end{definition}

\subsection{Key combinatorial lemma}

\begin{lemma}[Fixed subspace inequality]\label{lem:fixed-ineq}
Let $H\le K\le G$. Then
\[
\dim\!\bigl((\rho^{\mathcal{O}}_K)^H\bigr)
\;\ge\;
\frac{d_{\mathcal{O}}(K)}{d_{\mathcal{O}}(H)}.
\]
\end{lemma}

\begin{proof}
By definition $\rho^{\mathcal{O}}_K=\R[K/K_{\mathcal{O}}(K)]$.
Restricting to $H$, the $H$-fixed subspace has dimension equal to the number of $H$-orbits on the set $K/K_{\mathcal{O}}(K)$.

Each $H$-orbit on $K/K_{\mathcal{O}}(K)$ has size $[H : H \cap gK_{\mathcal{O}}(K)g^{-1}]$ for the appropriate coset representative $g$. Since $K_{\mathcal{O}}(K) \leq_{\mathcal{O}} K$ and conjugation within $K$ preserves $\mathcal{O}$-admissibility, we have $gK_{\mathcal{O}}(K)g^{-1} \leq_{\mathcal{O}} K$. Restricting to $H \leq K$:
\[
H \cap gK_{\mathcal{O}}(K)g^{-1} \leq_{\mathcal{O}} H,
\]
hence $K_{\mathcal{O}}(H) \leq H \cap gK_{\mathcal{O}}(K)g^{-1}$ by minimality of $K_{\mathcal{O}}(H)$.

Therefore every $H$-orbit has size
\[
[H : H \cap gK_{\mathcal{O}}(K)g^{-1}] \;\leq\; [H : K_{\mathcal{O}}(H)] \;=\; d_{\mathcal{O}}(H).
\]
Since $|K/K_{\mathcal{O}}(K)| = d_{\mathcal{O}}(K)$ and each orbit has size $\leq d_{\mathcal{O}}(H)$:
\[
\text{(\# orbits)} \;\geq\; \frac{d_{\mathcal{O}}(K)}{d_{\mathcal{O}}(H)}.\qedhere
\]
\end{proof}

\subsection{Main characterization theorem}

\begin{theorem}[$\mathcal{O}$-slice connectivity detected by geometric fixed points]\label{thm:main}
Let $X\in\SpG$ be connective and let $n\in\Z$.
Then $X\in \tau^{\mathcal{O}}_{\ge n}$ if and only if for every subgroup $H\le G$ the spectrum $\Phi^H X$ is $\ceil{n/d_{\mathcal{O}}(H)}$-connective, i.e.
\[
\pi_k\bigl(\Phi^H X\bigr)=0
\qquad\text{for all}\qquad
k<\ceil{\frac{n}{d_{\mathcal{O}}(H)}}.
\]
\end{theorem}

\begin{proof}
Write $r_H:=\ceil{n/d_{\mathcal{O}}(H)}$.

\medskip\noindent
\textbf{($\Rightarrow$)}
Assume $X\in\tau^{\mathcal{O}}_{\ge n}$.
By construction, $\tau^{\mathcal{O}}_{\ge n}$ is generated under colimits and cofibers by $\mathcal{O}$-slice cells of $\mathcal{O}$-dimension $\ge n$, and $\Phi^H$ preserves colimits and cofibers.

So it suffices to check the connectivity claim on a generator cell
\[
C \in \Bigl\{
G/K_+\wedge S^{m\rho_K^{\mathcal{O}}},\;
G/K_+\wedge S^{m\rho_K^{\mathcal{O}}-1}
\Bigr\}
\quad\text{with}\quad
m\,d_{\mathcal{O}}(K)\ge n.
\]
Fix $H\le G$.
If $H$ is not subconjugate to $K$, then $\Phi^H(C)$ is contractible, hence $r_H$-connective.
Otherwise conjugate so that $H\le K$.

Using standard behavior of geometric fixed points on induced cells, $\Phi^H(C)$ is a wedge of spheres whose suspension degrees are determined by the $H$-fixed subspace of the representation.
In particular, for $C=G/K_+\wedge S^{m\rho_K^{\mathcal{O}}}$ we have
\[
\Phi^H(C)\simeq \bigvee S^{\,m\cdot \dim((\rho_K^{\mathcal{O}})^H)},
\]
and similarly for the $-1$ shift.
Thus $\Phi^H(C)$ is at least $m\cdot \dim\!\bigl((\rho_K^{\mathcal{O}})^H\bigr)$-connective.
By Lemma~\ref{lem:fixed-ineq},
\[
m\cdot \dim\!\bigl((\rho_K^{\mathcal{O}})^H\bigr)
\;\ge\;
m\cdot \frac{d_{\mathcal{O}}(K)}{d_{\mathcal{O}}(H)}
\;\ge\;
\frac{n}{d_{\mathcal{O}}(H)}.
\]
Therefore $\Phi^H(C)$ is $r_H$-connective, and hence so is $\Phi^H(X)$ because $X$ is obtained from such cells by colimits and cofibers.

\medskip\noindent
\textbf{($\Leftarrow$)}
Assume $X$ is connective and every $\Phi^H X$ is $r_H$-connective.
We prove $X\in\tau^{\mathcal{O}}_{\ge n}$ by induction on $|G|$.
The case $G=e$ is ordinary connectivity: $\tau^{\mathcal{O}}_{\ge n}$ is generated by $S^m$ with $m\ge n$, so $X\in\tau^{\mathcal{O}}_{\ge n}$ iff $X$ is $n$-connective.

Now assume $G$ is nontrivial and the statement holds for all proper subgroups.
Let $\mathcal{P}$ be the family of proper subgroups of $G$, and let $E\mathcal{P}$ be the universal $G$-space for $\mathcal{P}$.
There is a cofiber sequence of $G$-spectra
\[
E\mathcal{P}_+\wedge X \longrightarrow X \longrightarrow \widetilde{E\mathcal{P}}\wedge X.
\]

\medskip\noindent
\emph{Step 1: $E\mathcal{P}_+\wedge X\in\tau^{\mathcal{O}}_{\ge n}$.}
The $G$-CW structure on $E\mathcal{P}$ has cells $G/H$ with $H\in\mathcal{P}$, so $E\mathcal{P}_+\wedge X$ lies in the localizing subcategory generated by $\Ind_H^G(\Res_H^G X)$ with $H\in\mathcal{P}$.
For each proper $H<G$, the restriction $\Res_H^G X$ is connective and satisfies the same geometric-fixed-point bounds for all $K\le H$ (because $\Phi^K(\Res_H^G X)\cong \Phi^K X$).
By the inductive hypothesis applied to the group $H$, we conclude $\Res_H^G X\in \tau^{\mathcal{O}|_H}_{\ge n}$.
Induction $\Ind_H^G$ sends $\mathcal{O}|_H$-slice cells to $\mathcal{O}$-slice cells of the same $\mathcal{O}$-dimension, hence carries $\tau^{\mathcal{O}|_H}_{\ge n}$ into $\tau^{\mathcal{O}}_{\ge n}$.
Therefore each $\Ind_H^G(\Res_H^G X)$ lies in $\tau^{\mathcal{O}}_{\ge n}$, and hence so does $E\mathcal{P}_+\wedge X$.

\medskip\noindent
\emph{Step 2: $\widetilde{E\mathcal{P}}\wedge X\in\tau^{\mathcal{O}}_{\ge n}$.}
The defining property of $\widetilde{E\mathcal{P}}$ is that $\Phi^H(\widetilde{E\mathcal{P}})$ is contractible for $H<G$ and is $S^0$ for $H=G$.
Hence $\Phi^H(\widetilde{E\mathcal{P}}\wedge X)$ is contractible for $H<G$ and equals $\Phi^G X$ for $H=G$.
In other words, $\widetilde{E\mathcal{P}}\wedge X$ has only $G$-isotropy, so within this ``geometric'' part the $\mathcal{O}$-slice filtration is generated only by the $G$-slice cells
\[
S^{m\rho_G^{\mathcal{O}}}\quad \text{and}\quad S^{m\rho_G^{\mathcal{O}}-1}
\quad\text{with}\quad m\,d_{\mathcal{O}}(G)\ge n.
\]
Since $(\rho_G^{\mathcal{O}})^G$ is $1$-dimensional, applying $\Phi^G$ identifies these generators with ordinary spheres $S^m$ (and $S^{m-1}$), so membership is equivalent to ordinary $r_G$-connectivity of $\Phi^G X$.
By assumption $\Phi^G X$ is $r_G$-connective, hence $\widetilde{E\mathcal{P}}\wedge X\in\tau^{\mathcal{O}}_{\ge n}$.

\medskip\noindent
\emph{Step 3: conclude for $X$.}
The subcategory $\tau^{\mathcal{O}}_{\ge n}$ is localizing and therefore closed under extensions/cofibers.
Since both ends of the cofiber sequence lie in $\tau^{\mathcal{O}}_{\ge n}$, the middle term $X$ lies in $\tau^{\mathcal{O}}_{\ge n}$ as well.

This completes the induction and the proof.
\end{proof}

%% ===================================================================
\section{Verification Notes}
%% ===================================================================

\begin{remark}[Steps verified]
\begin{enumerate}
\item \emph{Closure of transfer systems under intersection}: $K_1 \leq_{\mathcal{O}} H$ and $K_2 \leq_{\mathcal{O}} H$ $\Rightarrow$ $K_1 \cap K_2 \leq_{\mathcal{O}} H$ via restriction + transitivity. \checkmark
\item \emph{$(\rho^{\mathcal{O}}_H)^H \cong \R$}: transitive $H$-set $\Rightarrow$ invariant functions are constant. \checkmark
\item \emph{Fixed-subspace inequality}: max orbit size $\leq d_{\mathcal{O}}(H)$ because conjugates of $K_{\mathcal{O}}(K)$ are still $\mathcal{O}$-admissible, so stabilizers contain $K_{\mathcal{O}}(H)$. \checkmark
\item \emph{Forward direction}: generators produce correct connectivity via the inequality. \checkmark
\item \emph{Backward direction, Step 1}: restriction preserves geometric fixed points; induction on $|G|$ for proper subgroups. \checkmark
\item \emph{Backward direction, Step 2}: $\widetilde{E\mathcal{P}} \wedge X$ has only $G$-isotropy; $(\rho^{\mathcal{O}}_G)^G$ is 1-dimensional $\Rightarrow$ $\Phi^G$ converts to ordinary connectivity. \checkmark
\item \emph{Backward direction, Step 3}: localizing subcategory closed under cofibers. \checkmark
\item \emph{Ceiling function}: connectivity $\geq n/d_{\mathcal{O}}(H) \Rightarrow$ $\lceil n/d_{\mathcal{O}}(H) \rceil$-connective (integer homotopy groups). \checkmark
\end{enumerate}
\end{remark}

\subsection*{Classical summary (RS words removed)}

From an $N_\infty$ operad you get an incomplete transfer system $\le_{\mathcal{O}}$.
For each $H\le G$ define $K_{\mathcal{O}}(H)=\bigcap\{K\le H: K\le_{\mathcal{O}}H\}$ and $d_{\mathcal{O}}(H)=[H:K_{\mathcal{O}}(H)]$.
Define $\rho_H^{\mathcal{O}}=\R[H/K_{\mathcal{O}}(H)]$ and define $\tau^{\mathcal{O}}_{\ge n}$ as the localizing subcategory generated by the cells $G/H_+\wedge S^{m\rho_H^{\mathcal{O}}}$ and $G/H_+\wedge S^{m\rho_H^{\mathcal{O}}-1}$ with $m\,d_{\mathcal{O}}(H)\ge n$.
Then a connective $G$-spectrum $X$ lies in $\tau^{\mathcal{O}}_{\ge n}$ if and only if $\Phi^H X$ is $\ceil{n/d_{\mathcal{O}}(H)}$-connective for every $H\le G$.

\begin{thebibliography}{9}
\bibitem{AbouzaidEtAl2026}
M.~Abouzaid et al.
\newblock First Proof.
\newblock \emph{arXiv:2602.05192}, February 2026.
\end{thebibliography}

\end{document}
