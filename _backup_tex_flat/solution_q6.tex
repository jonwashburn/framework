\documentclass[11pt]{article}

\usepackage{amsmath,amssymb,amsthm}
\usepackage[a4paper,margin=1in]{geometry}
\usepackage{hyperref}
\hypersetup{colorlinks=true,linkcolor=blue,citecolor=blue,urlcolor=blue}

\setlength{\parskip}{0.5em}
\setlength{\parindent}{0pt}

\theoremstyle{plain}
\newtheorem{theorem}{Theorem}
\newtheorem{lemma}[theorem]{Lemma}

\theoremstyle{remark}
\newtheorem{remark}[theorem]{Remark}

\title{Solution to First Proof, Question~6:\\
Existence of $\varepsilon$-Light Vertex Subsets\\[6pt]
\large Via Recognition Science Primitives and Classical Conversion\\[6pt]
\normalsize Status: Upper bound proved; existence direction has an identified gap}

\author{Jonathan Washburn\\
Recognition Science, Recognition Physics Institute\\
Austin, Texas, USA\\
\texttt{jon@recognitionphysics.org}}

\date{February 9, 2026}

\begin{document}

\maketitle

\begin{abstract}
We address the question of whether every graph $G$ contains an $\varepsilon$-light vertex subset $S$ (meaning $\varepsilon L - L_S \succeq 0$) of size $|S| \geq c\varepsilon|V|$ for a universal constant $c > 0$. We prove the sharp upper bound $c \leq 1/2$ via the complete graph, and reduce the existence direction to a vertex-paving lemma. The paving lemma itself requires advanced spectral graph theory techniques beyond what we derive from first principles here; we clearly mark this gap.
\end{abstract}

\tableofcontents

%% ===================================================================
\section{The Question (Spielman)}
%% ===================================================================

Let $G=(V,E)$ be an (unweighted) graph with Laplacian $L$.
For $S\subseteq V$, let $G_S=(V,E(S,S))$ keep only edges with both endpoints in $S$,
with Laplacian $L_S$.
Call $S$ \emph{$\varepsilon$-light} if $\varepsilon L - L_S \succeq 0$.

Does there exist a universal constant $c>0$ so that for every graph $G$ and every $\varepsilon\in(0,1)$,
$V$ contains an $\varepsilon$-light $S$ with $|S|\ge c\,\varepsilon\,|V|$?

\medskip
\textbf{Expected answer: Yes}, with $c = 1/2$ being optimal.

\textbf{Status: Upper bound $c \leq 1/2$ proved; existence direction reduced to a paving lemma but not completed from first principles.}

%% ===================================================================
\section{RS Formulation}
%% ===================================================================

\begin{itemize}
\item \textbf{State space:} Vertices $V$ are sites; a recognition field is any $\phi:V\to\mathbb{R}$.
\item \textbf{Recognition cost:} $\mathcal{J}_G(\phi) := \phi^\top L\phi = \sum_{(u,v)\in E}(\phi_u-\phi_v)^2$.
\item \textbf{Restricted cost:} $\mathcal{J}_{G_S}(\phi) := \phi^\top L_S\phi = \sum_{(u,v)\in E(S,S)}(\phi_u-\phi_v)^2$.
\item \textbf{Audit defect:} $\Delta_S(\phi) := \mathcal{J}_{G_S}(\phi) - \varepsilon\,\mathcal{J}_G(\phi)$.
Then $S$ is $\varepsilon$-light iff $\Delta_S(\phi)\le 0$ for all $\phi$.
\item \textbf{Objective:} Maximize $|S|$ subject to passing the audit.
\end{itemize}

%% ===================================================================
\section{Sharp Upper Bound}
%% ===================================================================

\begin{theorem}[Upper bound on the constant]\label{thm:upper}
No universal constant $c$ can exceed $1/2$.
\end{theorem}

\begin{proof}
Consider the complete graph $K_n$.
For $S$ with $|S|=k$, the induced subgraph $G_S = K_k$.
For any $x\in\mathbb{R}^n$ supported on $S$ with $\sum_{i\in S}x_i=0$:
\[
x^\top L_{K_k}x = k\|x\|^2,\qquad x^\top L_{K_n}x = n\|x\|^2.
\]
So $\varepsilon L_{K_n} - L_{K_k}\succeq 0$ requires $\varepsilon n \geq k$, i.e., $k\leq \varepsilon n$.

The largest $\varepsilon$-light set has size $\lfloor\varepsilon n\rfloor$.
Take $\varepsilon$ just below $2/n$: then $\lfloor\varepsilon n\rfloor=1$ but $\varepsilon n\approx 2$.
Any universal $|S|\ge c\varepsilon n$ forces $1\ge c\cdot(2-\text{tiny})$, hence $c\le 1/2$.
\end{proof}

\begin{remark}
If the answer is yes, the optimal constant is $c = 1/2$ (up to integer rounding).
\end{remark}

%% ===================================================================
\section{Existence Direction: Reduction to Paving}
%% ===================================================================

The existence direction reduces to the following:

\begin{lemma}[Vertex paving — target statement]\label{lem:paving}
Fix an integer $r\ge 2$. There exists a partition $V = S_1\sqcup \cdots \sqcup S_r$ such that for some $i$:
\[
L_{S_i}\preceq\frac{2}{r}\,L
\quad\text{and}\quad
|S_i|\ge\frac{|V|}{r}.
\]
\end{lemma}

\textbf{Assuming the paving lemma:} choose $r=\lceil 2/\varepsilon\rceil$. Then $2/r\le\varepsilon$ and there is a part $S_i$ with $L_{S_i}\preceq\varepsilon L$ and $|S_i|\ge|V|/r\ge(\varepsilon/2)|V|$, giving $c=1/2$.

\medskip
\textbf{Gap:} A self-contained proof of Lemma~\ref{lem:paving} from first principles is not provided. The lemma is a vertex-paving statement related to the Marcus--Spielman--Srivastava resolution of the Kadison--Singer problem (2015), but adapting their operator-paving results to this specific vertex-paving setting requires additional technical work (matrix concentration for dependent random variables arising from vertex coloring, or a direct paving argument for graph Laplacians).

%% ===================================================================
\section{Verification Notes}
%% ===================================================================

\begin{remark}[What is proved vs.\ what is open]
\begin{enumerate}
\item \emph{Upper bound $c \leq 1/2$}: complete, via $K_n$. \checkmark
\item \emph{RS formulation}: clean translation to audit defect. \checkmark
\item \emph{Reduction to paving}: correct — if the lemma holds, $c = 1/2$ follows. \checkmark
\item \emph{Paving lemma}: NOT proved from first principles. This is the hard part. \ding{55}
\end{enumerate}
The paving lemma is a statement about spectral control of random vertex partitions.
A proof would likely use matrix concentration (matrix Azuma or Freedman inequalities
for vertex-exposed martingales) or connections to the Kadison--Singer / Weaver framework.
\end{remark}

\begin{remark}[Why this problem is hard]
The difficulty is that the PSD order $L_{S_i} \preceq (2/r)L$ must hold for ALL vectors $x$ simultaneously.
Point-wise (for a fixed $x$), averaging gives $\mathbb{E}[x^\top L_{S_i}x] = (1/r^2)x^\top Lx$, which is much smaller than $(2/r)x^\top Lx$.
But upgrading this expectation bound to a uniform (operator-norm) bound requires controlling the supremum over all directions, which is a spectral concentration problem with dependent random variables (edges sharing vertices).
\end{remark}

\begin{thebibliography}{9}
\bibitem{AbouzaidEtAl2026}
M.~Abouzaid et al.
\newblock First Proof.
\newblock \emph{arXiv:2602.05192}, February 2026.

\bibitem{MSS2015}
A.~W.~Marcus, D.~A.~Spielman, and N.~Srivastava.
\newblock Interlacing families II: Mixed characteristic polynomials and the Kadison--Singer problem.
\newblock \emph{Ann. of Math.}, 182(1):327--350, 2015.
\end{thebibliography}

\end{document}
