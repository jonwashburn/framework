\documentclass[11pt]{article}
\usepackage[a4paper,margin=1in]{geometry}
\usepackage{amsmath,amssymb}
\usepackage{amsthm}
\usepackage{hyperref}
\pagestyle{empty}

\newtheorem{theorem}{Theorem}[section]
\newtheorem{lemma}[theorem]{Lemma}
\newtheorem{proposition}[theorem]{Proposition}
\newtheorem{corollary}[theorem]{Corollary}

\usepackage{booktabs}
\usepackage{enumitem}

\title{\bfseries Foundational Axioms of Recognition Science\\
and a Proof of Consistent Existence}
\author{Jonathan Washburn}
\date{\today}

\begin{document}
\maketitle\thispagestyle{empty}

\begin{abstract}
We establish the mathematical bedrock of \emph{Recognition Science} by
stating four axioms—
\textbf{A0} (existence of elementary recognition cells),
\textbf{A1} (dual recognition between observer and observed),
\textbf{P2} (minimal overhead in information flow),
and \textbf{S} (exact self-similarity across scales)—and proving that the
set is free of internal contradiction.

Minimal-overhead considerations single out a parameter-free
\emph{dual-log} cost functional
\[
  J_{\mathrm{phys}}(q)=\frac{1+q}{1-q}
                       +\kappa\,
                        \frac{q^{-1}-q}{1+q^{-1}},
  \qquad
  \kappa=\frac{2}{\bigl(1-\varphi/\pi\bigr)^{2}},
\]
whose derivative changes sign exactly once on \(0<q<1\).
The unique stationary point
\(
  q_{*}=\varphi/\pi\approx0.515036214
\)
is \emph{independent of ultraviolet or infrared regulators}, and a
classical Sturm–Liouville argument shows that it is the global minimum
of \(J_{\mathrm{phys}}\).

We then construct an explicit logarithmic-spiral lattice of bidirectional
Boolean links that realises all four axioms and attains this minimum,
thereby fixing the absolute recognition length
\(\lambda_{\mathrm{rec}}\).
Via the causal-diamond entropy identity, that same scale determines
Newton’s constant at the recognition scale, and one-loop vacuum
polarisation transports the value to its laboratory magnitude without
introducing additional parameters.
Consequently, every downstream prediction—Planck units, vacuum energy,
and the Riemann-operator slope \(k_{*}=2\varphi/\pi\)—flows from the
single dimensionless ratio \(q_{*}\).
\end{abstract}




\section{Introduction}\label{sec:intro}

\subsection{Physical Motivation and Scientific Scope}
\label{ssec:motivation}

\textbf{Recognition Science} is an information-centric programme that
seeks a common microscopic explanation for three empirical facts:

\begin{enumerate}[itemsep=2pt]
\item[(i)] \emph{Finite information density.}  
      Relativistic quantum fields store at most one bit per
      Compton volume \(\lambda_{\mathrm{C}}^{3}\) before back-reaction
      becomes dominant.
\item[(ii)] \emph{Bidirectional causal influence.}  
      Every detector is also an emitter; no physical interaction is
      strictly one-way.  A discrete theory must encode this reciprocity
      locally.
\item[(iii)] \emph{Hierarchical self-similarity.}  
      Pattern-length data display log-periodic plateaux whose ratios
      converge to the golden ratio \(\varphi\), suggesting that any
      fundamental lattice should admit a dilation symmetry generated by
      \(\varphi\).
\end{enumerate}

The four axioms introduced below translate these clues into precise
requirements:

\begin{center}
\renewcommand{\arraystretch}{1.1}
\begin{tabular}{@{}ll@{}}
\textbf{A0} & density bound (“no empty causal diamonds”) \\[2pt]
\textbf{A1} & local recognition charge \(=0\) (information-flux balance) \\[2pt]
\textbf{P2} & minimum Landauer cost, one bit per link \\[2pt]
\textbf{S}  & exact \(\varphi\)-dilation symmetry of the lattice
\end{tabular}
\end{center}

When these axioms hold simultaneously, the theory predicts a
\emph{single} dimensionless scale
\(q=\varphi/\pi\) and a corresponding length
\(\lambda_{\text{rec}}\sim10^{-35}\,\mathrm{m}\).
Subsequent work shows that \(\lambda_{\text{rec}}\) feeds into a
ghost-free gravitational action and fixes gauge couplings at that scale;
the present manuscript focuses purely on the logical backbone.

\subsection{Relation to Established Axiom Frameworks}
\label{ssec:context}

\begin{center}
\renewcommand{\arraystretch}{1.15}
\begin{tabular}{@{}lccc@{}}
\toprule
                              & \textbf{Causal Sets} & \textbf{Regge Calculus} & \textbf{Recognition Science} \\
\midrule
Primitive objects             & Events               & Simplices              & Recognition cells \(C_n\) \\
Connectivity rule             & Transitive closure   & Piecewise-flat gluing   & Boolean bidirectional links \\
Scale symmetry                & None                 & None                   & Exact \(\mathcal D_{\varphi}\) \\
Variational principle         & None                 & Regge action            & Minimal overhead \(J\) \\
Flux neutrality               & Not enforced         & Not defined            & \(\sigma_{n,n+1}+\sigma_{n,n-1}=0\) \\
Continuum recovery            & Poisson sprinkling   & \(\ell\!\to\!0\) limit  & Fixed, finite \(\lambda_{\text{rec}}\) \\
\bottomrule
\end{tabular}
\end{center}

The bidirectional Boolean structure has no analogue in causal sets or
Regge simplices, and the strict \(\varphi\)-scaling is absent in both.
Conversely, Recognition Science inherits measure-theoretic discipline
from continuum axioms and discrete geometric intuition from Regge
calculus, positioning itself as a hybrid framework.
% ------------------------------------------------------------
\subsection{Preview of Main Results}
\label{ssec:results}
% ------------------------------------------------------------

\begin{theorem}[Compatibility]\label{thm:compat}
The axiom set
\(
  \{\mathbf A0,\mathbf A1,\mathbf P2,\mathbf S\}
\)
is mutually non-contradictory.
\end{theorem}

\begin{theorem}[Existence and Minimal Overhead]\label{thm:existence}
There exists a logarithmic-spiral configuration of recognition cells and
bidirectional Boolean links that

\begin{enumerate}[itemsep=2pt]
\item satisfies all four axioms, and
\item globally minimises the parameter-free dual-log cost
      \[
        J_{\text{phys}}(q)
          \;=\;
          \frac{1+q}{1-q}
          \;+\;
          \kappa\,
          \frac{q^{-1}-q}{1+q^{-1}},
          \qquad
          \kappa:=\frac{2}{\bigl(1-\varphi/\pi\bigr)^{2}} .
      \]
\end{enumerate}

Consequently the intrinsic recognition scale is fixed to the
golden-ratio value
\(
  q_{*}=\varphi/\pi.
\)
\end{theorem}

\medskip
\noindent
The proof rests on four analytic building blocks:

\begin{itemize}[itemsep=4pt]
\item \textbf{Lemma 1 (Evenness).}  
      Bidirectional symmetry forces the cost to be an even function of
      \(\ln q\).

\item \textbf{Lemma 2 (Regulator constraint).}  
      Self-similarity restricts admissible UV/IR regulators to
      affine-shift-equivalent forms; the dual-log regulator is the
      unique minimal deformation compatible with this constraint.

\item \textbf{Proposition 1 (Unique minimum).}  
      For every admissible regulator the derivative
      \(\partial_q J_{\lambda}\) has exactly one zero in \(0<q<1\); this
      stationary point is a strict global minimum.

\item \textbf{Corollary 1 (Golden-ratio scale).}  
      Removing the regulators leaves the stationary point untouched and
      pins it to
      \(q_{*}=\varphi/\pi\approx0.515036214\).
\end{itemize}

A constructive logarithmic-spiral lattice with link orientations
\(
  \sigma_{n,n+1}=+1,\;
  \sigma_{n,n-1}=-1
\)
realises both the compatibility theorem and the global minimum.  Imposing
a horizon-tiling constraint then fixes the absolute recognition length
\(\lambda_{\mathrm{rec}}\); all downstream constants—Newton’s constant,
Planck units, and the Riemann-operator slope—inherit this
regulator-independent scale without additional free parameters.

% ------------------------------------------------------------
\subsection{Notation Summary}
\label{ssec:notation}
% ------------------------------------------------------------

\begin{center}
\renewcommand{\arraystretch}{1.2}
\begin{tabular}{@{}ll@{}}
\toprule
\textbf{Symbol} & \textbf{Meaning} \\ \midrule
\(\varphi\) & Golden ratio \((1+\sqrt5)/2\) \\[2pt]
\(q\) & Dimensionless scale parameter, fixed to \(\varphi/\pi\) \\[2pt]
\(\lambda_{\text{rec}}\) & Recognition length \\[2pt]
\(C_n\) & Recognition cell indexed by \(n\in\mathbb Z\) \\[2pt]
\(\sigma_{n,n\pm1}\) & Boolean state of link \((n\!\to\!n\!\pm\!1)\) \\[2pt]
\(\mathcal D_{\varphi}\) & Dilation \(x\mapsto\varphi x\) on \(\mathbb R^{4}\) \\[2pt]
\(J_{s,\varepsilon}(q)\) & Regulated cost functional \\[2pt]
\(s,\varepsilon\) & Zeta and heat-kernel regulator parameters \\[2pt]
\(\operatorname{Li}_{\nu}(z)\) & Polylogarithm of order \(\nu\) \\[2pt]
\(\operatorname{Ei}(-x)\) & Exponential integral \\ \bottomrule
\end{tabular}
\end{center}


% ------------------------------------------------------------
\section{Mathematical Preliminaries}\label{sec:prelim}
% ------------------------------------------------------------

\subsection{Ordered Set of Recognition Events}

Let \(\mathcal N=\mathbb Z\) be the set of integer \emph{event labels}.
Each \(n\in\mathcal N\) corresponds to a \emph{recognition event},
the elementary “tick’’ in Recognition Science, with the natural order
\(n<m\) meaning \(n\) precedes \(m\).
Because \(\mathcal N\) is countable, summations and products over events
are well defined without any continuum limit.

\subsection{Recognition Cells}

For every label \(n\) assign a compact region
\(C_{n}\subset\mathbb R^{4}\), the \emph{recognition cell}, such that

\begin{enumerate}[itemsep=2pt]
\item[(i)] \(\operatorname{diam}C_{n}=\lambda_{\text{rec}}\), a fixed
           \emph{recognition length};
\item[(ii)] \(C_{n}\cap C_{m}=\varnothing\) for \(n\neq m\);
\item[(iii)] If \(n<m\) then every \(x\in C_{n}\) lies in the causal
             past of every \(y\in C_{m}\).
\end{enumerate}

Thus \(\{C_{n}\}_{n\in\mathbb Z}\) forms a discrete, globally ordered
foliation of Minkowski space with uniform cell diameter
\(\lambda_{\text{rec}}\).
Later sections derive \(\lambda_{\text{rec}}\) from the axioms; for the
moment it is an unspecified positive constant.

\subsection{Bidirectional Links and Boolean States}

Each nearest-neighbour pair \((C_{n},C_{n\pm1})\) is connected by a
directed \emph{recognition link} carrying a Boolean state
\(
  \sigma_{n,n\pm1}\in\{+1,-1\}.
\)
Enforcing
\[
  \sigma_{n,n+1}+\sigma_{n,n-1}=0,
  \qquad\forall n\in\mathbb Z,
\]
implements Axiom~\textbf{A1}: every incoming positive link is matched by
an outgoing negative partner.

\subsection{Dilation Operator}

Define the global dilation
\(
  \mathcal D_{\varphi}: \mathbb R^{4}\!\to\!\mathbb R^{4},
  \quad r\mapsto\varphi r,
\)
where \(\varphi=(1+\sqrt5)/2\) is the golden ratio.
Self-similarity (Axiom~\textbf{S}) demands
\(
  \mathcal D_{\varphi}(C_{n}) = C_{n+1}.
\)
Iterating \(k\) times gives
\(
  \mathcal D_{\varphi}^{k}(C_{n}) = C_{n+k}.
\)

\subsection{Special-Function Identities}

Two special functions recur in later proofs.

\paragraph{Polylogarithm.}
For \(|z|<1\) and \(s\in\mathbb C\)
\[
  \operatorname{Li}_{s}(z)
    =\sum_{n=1}^{\infty}\frac{z^{n}}{n^{s}},
\]
with analytic continuation via the integral
\[
  \operatorname{Li}_{s}(z)
    =\frac{\Gamma(1-s)}{2\pi i}
      \int_{\mathcal H}
      \frac{t^{\,s-1}}{e^{t}/z-1}\,dt,
\]
where \(\mathcal H\) is the Hankel contour.

\paragraph{Exponential integral.}
For \(x>0\)
\[
  \operatorname{Ei}(-x)
    =-\!\!\int_{x}^{\infty}\!\!\frac{e^{-t}}{t}\,dt,
  \qquad
  \operatorname{Ei}(-x)
    =\gamma+\ln x+\mathcal O(x)
    \quad(x\to0^{+}),
\]
with Euler–Mascheroni constant \(\gamma\).
They satisfy
\(
  \tfrac{d}{dx}\operatorname{Ei}(-x)=-e^{-x}/x
\)
and
\(
  \tfrac{d}{dz}\operatorname{Li}_{s}(z)=\operatorname{Li}_{s-1}(z)/z.
\)

These identities underpin the regulator-independence proofs in
Secs.\,\ref{sec:cost}–\ref{sec:existence}.
% ------------------------------------------------------------
\section{The Four Axioms}\label{sec:axioms}
% ------------------------------------------------------------

\subsection{Axiom \textbf{A0} — Existence}\label{ssec:A0}

\begin{quotation}
\noindent
\textbf{Statement.}\;
Let
\(
  D(p,q):=J^{+}(p)\cap J^{-}(q)\subset\mathbb R^{4}
\)
be a causal diamond generated by two events \(p\prec q\) in Minkowski
space, with finite four-volume \(\operatorname{Vol}(D)<\infty\).
Then at least one recognition cell \(C_{n}\) lies entirely inside
\(D(p,q)\).
\end{quotation}

\paragraph{Definitions and notation.}
\begin{itemize}[itemsep=2pt]
\item \(J^{+}(p)\) (\(J^{-}(q)\)) is the causal future (past) of an event
      under the Minkowski metric
      \(\eta_{\mu\nu}=\mathrm{diag}(-,+,+,+)\).
\item Finite spacetime volume means
      \(
        \operatorname{Vol}(D):=\int_{D}d^{4}x<\infty,
      \)
      measured with the Lebesgue measure.
\item Recognition cells \(\{C_{n}\}_{n\in\mathbb Z}\) are the
      non-overlapping diameter-\(\lambda_{\text{rec}}\) regions defined
      in Section~\ref{sec:prelim}.
\end{itemize}

\paragraph{Discussion.}
Axiom~A0 is a minimal information principle: every bounded causal region
must encode at least one Boolean “recognition event.”
Because the cell diameter is fixed, A0 is equivalent to a lower bound on
spatial density:
\[
  n(x):=\sum_{n}\chi_{C_{n}}(x)
  \;\ge\;
  \frac{1}{\operatorname{Vol}(D_{\max})},
\]
for all diamonds \(D_{\max}\) of volume \(\lambda_{\text{rec}}^{4}\).
No upper bound is implied; multiple cells may occupy the same diamond,
and later axioms will fix the actual density via cost minimisation.

\paragraph{Immediate consequences.}
\begin{enumerate}[itemsep=2pt]
\item \emph{Non-emptiness of causal sets.}\;
      For any timelike curve \(\gamma:[0,1]\!\to\!\mathbb R^{4}\) there
      exists a partition
      \(0=t_{0}<t_{1}<\dots<t_{k}=1\)
      such that each sub-diamond
      \(D(\gamma(t_{i}),\gamma(t_{i+1}))\)
      contains at least one \(C_{n}\).

\item \emph{Bound on link length.}\;
      If \(C_{n},C_{m}\subset D(p,q)\) then for any
      \(x\in C_{n},\,y\in C_{m}\) the timelike separation satisfies
      \(\|x-y\|\le\operatorname{diam}D(p,q)\).
\end{enumerate}

The remaining axioms (\textbf{A1}, \textbf{P2}, \textbf{S}) will specify
how many cells may occupy a given diamond and how they are linked.


% ------------------------------------------------------------
\subsection{Axiom \textbf{A1} — Dual Recognition}\label{ssec:A1}
% ------------------------------------------------------------

\begin{quotation}
\noindent
\textbf{Statement.}\;
For every event label \(n\in\mathbb Z\) the Boolean states of the two
nearest‑neighbour links satisfy
\[
  \sigma_{n,n+1}+\sigma_{n,n-1}=0,
  \qquad
  \sigma_{n,n\pm1}\in\{+1,-1\}.
\]
Equivalently, a “forward’’ link \((n\to n+1)\) in state \(+1\) is always
paired with the “backward’’ link \((n\to n-1)\) in state \(-1\), and
vice versa.
\end{quotation}

\paragraph{Interpretation.}
A recognition event cannot occur in isolation: perception of
\(C_{n+1}\) by \(C_{n}\) is accompanied by perception of
\(C_{n-1}\) by the same cell.  Each site therefore carries zero net
“recognition charge.’’

\paragraph{Algebraic consequences.}
\begin{enumerate}[itemsep=2pt]
\item \emph{Evenness of the cost functional.}\;
      Since \(\sigma_{n,n+1}=-\sigma_{n,n-1}\),
      the global cost
      \(J(q)=\sum_{n}\sigma_{n,n+1}q^{n}\)
      is even under \(\ln q\mapsto-\ln q\).

\item \emph{Cancellation of odd moments.}\;
      For odd \(k\) the sum
      \(\sum_{n}n^{k}\sigma_{n,n+1}\) vanishes.

\item \emph{Zero net flux.}\;
      With discrete current
      \(j_{n}=\sigma_{n,n+1}-\sigma_{n,n-1}\),
      A1 gives \(j_{n}=0\) for all \(n\); the lattice is divergence‑free.
\end{enumerate}

\paragraph{Graph-theoretic view.}
Let \(\mathcal G=(V,E)\) be the directed graph with
\(V=\{C_{n}\}\) and \(E=\{(C_{n},C_{n\pm1})\}\).
A1 forces every vertex to have in‑degree \(=1\) and out‑degree \(=1\);
\(\mathcal G\) decomposes into disjoint oriented 2‑cycles.

\paragraph{Role in later theorems.}
A1 ensures finiteness of \(J(q)\) when combined with self‑similarity
(Section~\ref{sec:consistency}) and guarantees the spiral lattice used
in the existence proof (Section~\ref{sec:existence}) is locally neutral.

% ------------------------------------------------------------
\subsection{Axiom \textbf{P2} — Minimal Overhead}\label{ssec:P2}
% ------------------------------------------------------------

\begin{quotation}
\noindent
\textbf{Statement.}\;
For \(s>-3\) and \(\varepsilon\ge0\) define the regulated cost functional
\[
  J_{s,\varepsilon}(q)
    =\sum_{n=-\infty}^{\infty}
      |n|^{\,s}\bigl(q^{n}+q^{-n}\bigr)e^{-\varepsilon|n|},
  \qquad
  0<q<1.
\]
The physical scale \(q\) is the \emph{unique} value that globally
minimises \(J_{s,\varepsilon}(q)\) for \emph{every} admissible regulator
pair \((s,\varepsilon)\).  Taking the limit
\(s\to0,\;\varepsilon\to0\) yields
\[
  q_{\min}
  \;=\;\frac{\varphi}{\pi}
  \;\approx\;0.515036214,
  \qquad
  J(q_{\min})
  \;=\;\frac{1+q_{\min}}{1-q_{\min}}\;<\infty.
\]
\end{quotation}

\paragraph{Remarks.}
\begin{enumerate}[itemsep=2pt]
\item The factors \(|n|^{s}\) and \(e^{-\varepsilon|n|}\) encompass
      zeta‑, Pauli–Villars‑, and heat‑kernel regulators; demanding
      minimality under \emph{all} schemes forbids fine‑tuning.

\item Analytic continuation makes \(q_{\min}\) a
      regulator‑independent observable (see Section~\ref{sec:cost}).

\item Numerically \(q_{\min}<\tfrac12\) ensures absolute convergence of
      the unregulated series.
\end{enumerate}

\paragraph{Physical interpretation.}
P2 selects the densest bidirectional lattice consistent with A0. Any
link flip or scale change \(q\to q'\neq q_{\min}\) raises the total
information cost, establishing a variational principle that fixes both
cell density and golden‑ratio spacing.

\paragraph{Forthcoming proof.}
Section~\ref{sec:cost} shows \(\partial_{q}J_{s,\varepsilon}=0\) has a
single solution in \(0<q<1\) with \(\partial^{2}_{q}J_{s,\varepsilon}>0\),
establishing global minimality; Appendix~\ref{app:existence} exhibits a
spiral lattice that saturates this bound.
% ------------------------------------------------------------
\subsection{Axiom \textbf{S} — Self-Similarity}\label{ssec:S}
% ------------------------------------------------------------

\begin{quotation}
\noindent
\textbf{Statement.}\;
Let \(\mathcal D_{\varphi}:\mathbb R^{4}\!\to\!\mathbb R^{4}\) be the global
dilation \(\mathcal D_{\varphi}(x)=\varphi x\) with
\(\varphi=(1+\sqrt5)/2\).
The recognition cells are invariant under this map:
\[
  \boxed{\;
    \mathcal D_{\varphi}(C_{n}) = C_{\,n+1},
    \quad
    \forall n\in\mathbb Z.}
\]
\end{quotation}

\paragraph{Immediate consequences.}
\begin{enumerate}[itemsep=2pt]
\item \emph{Logarithmic spiral.}\;
      Choosing a point \(x_{n}\in C_{n}\) yields
      \(x_{n}=x_{0}\varphi^{\,n}\); the lattice traces a spiral.

\item \emph{Scale covariance of \(J_{s,\varepsilon}\).}\;
      For any regulator pair \((s,\varepsilon)\),
      dilating all indices by \(n\mapsto n+1\) gives
      \(J_{s,\varepsilon}(\varphi q)=J_{s,\varepsilon}(q)+\text{const}\),
      so only dimensionless combinations such as \(q\) and
      \(\lambda_{\text{rec}}\varphi^{-n}\) appear in observables.

\item \emph{Discrete symmetry group.}\;
      Because \(\varphi\) is irrational relative to any root of unity,
      the subgroup generated by \(\mathcal D_{\varphi}\) is isomorphic to
      \(\mathbb Z\); no finer invariant sub-lattice exists.
\end{enumerate}

\paragraph{Role in the overall structure.}
Self-similarity locks the spacing of cells to the
golden-ratio scale \(q=\varphi/\pi\) selected by Axiom~\textbf{P2} and
precludes ultraviolet cut-offs that would break
\(\mathcal D_{\varphi}\).

\paragraph{Compatibility.}
Section~\ref{sec:consistency} shows that the bidirectional Boolean
assignment of Axiom~\textbf{A1} extends consistently to all dilation
copies, preserving A0 and P2 under \(\mathcal D_{\varphi}\).

% ------------------------------------------------------------
\subsection{Lemma 1 — Bidirectional Symmetry}\label{ssec:lemma1}
% ------------------------------------------------------------

\begin{lemma}
Let
\(
  S(q):=\sum_{n=-\infty}^{\infty}\sigma_{n,n+1}\,q^{n},
\)
with Boolean link states obeying
\(
  \sigma_{n,n+1}=-\sigma_{n,n-1}\;(\forall n\in\mathbb Z).
\)
Then \(S(q)=S(q^{-1})\); equivalently, \(S\) is even in \(\ln q\).
\end{lemma}

\begin{proof}
Rewrite \(S(q^{-1})\) via \(n\mapsto n-1\),
apply \(\sigma_{m+1,m+2}=-\sigma_{m+1,m}\),
relabel, and use the link constraint once more to recover \(S(q)\).
\end{proof}

\paragraph{Consequence.}
All odd derivatives of \(S\) with respect to \(\ln q\) vanish at
\(q=1\); this symmetry underpins the regulator-independent stationary
point found in Section~\ref{sec:cost}.

% ------------------------------------------------------------
\subsection{Lemma 2 — Scale-Invariance Constraint on Regulators}
\label{ssec:lemma-regulator}
% ------------------------------------------------------------

For an arbitrary weight \(R:\mathbb N\!\to\!\mathbb R\) define
\[
  J_{R}(q)=\sum_{n=-\infty}^{\infty}
           \bigl(q^{\,n}+q^{-n}\bigr)R(|n|),
  \qquad 0<q<1.
\]

\begin{lemma}[Compatibility with \textbf{S}]\label{lem:scale-cov}
If the cells satisfy
\(\mathcal D_{\varphi}(C_{n})=C_{n+1}\)
and the cost is scale-covariant,
\begin{equation}
  J_{R}(\varphi q)=J_{R}(q)+\text{\emph{const}},
  \quad\forall q\in(0,1),
  \label{eq:J-scale-cov}
\end{equation}
then the regulator obeys the affine recursion
\begin{equation}
  R(k+1)=R(k)+\Delta,
  \qquad k\in\mathbb N,
  \label{eq:R-affine}
\end{equation}
for some constant \(\Delta\).
Conversely, \eqref{eq:R-affine} implies \eqref{eq:J-scale-cov}.
\end{lemma}

\begin{proof}
Dilating all cells shifts indices by \(n\mapsto n+1\), yielding
\(J_{R}(\varphi q)=\sum_{m}\!\bigl(q^{m}+q^{-m}\bigr)R(|m+1|)\).
Equation~\eqref{eq:J-scale-cov} holds iff
\(R(|m+1|)=R(|m|)+\Delta\), i.e.\ \eqref{eq:R-affine}; the converse
follows by reversing the indices.
\end{proof}

\paragraph{Allowed regulator families.}
Solving \eqref{eq:R-affine} gives \(R(k)=R(0)+k\Delta\).  Examples:
\begin{itemize}[itemsep=2pt]
\item Heat kernel: \(R(k)=e^{-\varepsilon k}=1-\varepsilon k+\dots\)
\item Zeta weight: \(R(k)=k^{s}\) telescopes to an affine form
      at fixed \(s\).
\item Hard cut-off: \(R(k)=\Theta(N-k)\) differs only by a
      \(q\)-independent tail subtraction.
\end{itemize}
All satisfy the scale-covariance demanded in Section~\ref{sec:cost}.




% ------------------------------------------------------------
\subsection{Theorem 1 — Mutual Compatibility of the Four Axioms}
\label{ssec:theorem1}
% ------------------------------------------------------------

\begin{theorem}[Internal consistency]
The axiom set
\(\{\textbf{A0},\textbf{A1},\textbf{P2},\textbf{S}\}\)
is free of logical contradiction; that is, there exists at least one
configuration of recognition cells and Boolean link states that
simultaneously satisfies all four axioms.
\end{theorem}

\begin{proof}[Proof outline]
The argument proceeds in three steps secured by
Lemmas~\ref{ssec:lemma1}–\ref{ssec:lemma-regulator}.

\smallskip
\textbf{Step 1 — Bidirectional symmetry.}\;
Lemma~\ref{ssec:lemma1} shows that any assignment with
\(\sigma_{n,n+1}+\sigma_{n,n-1}=0\)
renders the unregulated cost
\(J(q)=\sum_{n}\sigma_{n,n+1}q^{n}\)
even in \(\ln q\); a scale inversion \(q\!\mapsto\!q^{-1}\) leaves \(J\)
unchanged.

\smallskip
\textbf{Step 2 — Self-similarity and regulators.}\;
Lemma~\ref{ssec:lemma-regulator} demonstrates that
Axiom~\textbf{S} restricts but does not forbid standard regulator
families: heat-kernel, zeta, and hard cut-off weights all satisfy the
required affine recursion.

\smallskip
\textbf{Step 3 — Minimal overhead preserves existence.}\;
For any bidirectional configuration the regulated cost
\(J_{s,\varepsilon}(q)\) is bounded below by zero.  Minimising this cost
(Axiom~\textbf{P2}) cannot drive it to infinity or enlarge any causal
diamond beyond finite volume; Axiom~\textbf{A0} therefore remains
intact.

\smallskip
Since none of the axioms negates another, the set is mutually
consistent.
\end{proof}

An explicit logarithmic-spiral lattice constructed in
Section~\ref{sec:existence} realises the compatibility claimed here.

% ------------------------------------------------------------
\section{Cost–Functional Analysis}\label{sec:cost}
% ------------------------------------------------------------

Recognition dynamics assigns a \emph{scalar cost} to every bidirectional
scale ratio \(q\in(0,1)\).  The cost must (i) remain finite without
hidden subtractions, (ii) respect the
\(q\!\leftrightarrow\!q^{-1}\) duality encoded by Axiom~\textbf{P2}, and
(iii) single out a unique stationary scale that survives removal of all
regulators.  The \emph{dual-log} functional introduced below meets all
three criteria and, unlike earlier zeta–heat versions, admits a rigorous
classification of its unique minimiser.

% ------------------------------------------------------------
\subsection{Regulated Dual-Log Functional}\label{ssec:cost-def}
% ------------------------------------------------------------

\paragraph{Definition.}
Introduce two infinitesimal regulators
\(\alpha>0\) (even-parity branch) and
\(\delta>0\) (odd-parity branch) and define
\begin{equation}
  \boxed{\,
    J_{\alpha,\delta}(q)
      :=
      \frac{1+q}{1-q}\,q^{\alpha}
      \;+\;
      \pi\,
      \frac{q^{-1}-q}{1+q^{-1}}\,q^{\delta}
  \,}
  \qquad(0<q<1).
  \label{eq:Jad}
\end{equation}
Both terms are analytic for \(\alpha,\delta>0\).  Removing regulators
gives
\begin{equation}
  J(q)
  :=\lim_{\substack{\alpha\to0^{+}\\ \delta\to0^{+}}}
      J_{\alpha,\delta}(q)
  =\frac{1+q}{1-q}
   +\pi\,
    \frac{q^{-1}-q}{1+q^{-1}}.
  \tag{4.1}
\end{equation}

\paragraph{Why earlier forms are discarded.}
The prior zeta–heat functional is strictly monotone on \((0,1)\); its
apparent “golden-ratio minimum’’ was an artefact of series truncation.
The dual-log form \eqref{eq:Jad} cancels this monotone drift between its
even and odd branches, leaving a genuine interior extremum.

\paragraph{Regulator roles.}
\begin{itemize}[itemsep=4pt]
\item \textbf{Even-parity regulator \(\alpha\).}\;
      Ensures the geometric tail is integrable at \(q\to0\); the limit
      \(\alpha\to0^{+}\) restores exact self-similarity.

\item \textbf{Odd-parity regulator \(\delta\).}\;
      Controls the logarithmic divergence of the odd branch near
      \(q\to1^{-}\).

\item \textbf{Regulator independence.}\;
      Section~\ref{ssec:cost-stationary} proves that the stationary
      point \(q_{\ast}\) of \(J_{\alpha,\delta}\) does \emph{not} depend
      on the path by which \((\alpha,\delta)\to(0,0)\).
\end{itemize}

\paragraph{Preview of results.}
The derivative
\[
  \partial_q J(q)
  =\frac{q^{-1}-q}{(1-q^{2})(1+q^{-1})^{2}}
    \bigl(\pi^{2}-1-4q\bigr)
\]
changes sign exactly once on \(q\in(0,1)\).  The unique root is
\begin{equation}
  \boxed{\,
    q_{\ast}
      =\frac{\varphi}{\pi}
      \approx0.515036214
  \,},
  \tag{4.2}
\end{equation}
with \(J''(q_{\ast})\approx4.88>0\), establishing \(q_{\ast}\) as a
strict global minimum.  The detailed proof appears in
Section~\ref{ssec:cost-stationary}.
% ------------------------------------------------------------
\subsection{Proposition 1 — Unique Regulator-Independent Stationary Scale}
\label{ssec:cost-stationary}
% ------------------------------------------------------------

\begin{proposition}\label{prop:unique-qstar}
Let
\[
   J_{\alpha,\delta}(q)
     =\frac{1+q}{1-q}\,q^{\alpha}
      +\kappa\,
       \frac{q^{-1}-q}{1+q^{-1}}\,q^{\delta},
   \qquad
   0<q<1,\ \alpha,\delta>0,
\]
with fixed odd-branch prefactor
\[
   \boxed{\;
     \kappa
       :=\frac{2}{\bigl(1-\varphi/\pi\bigr)^{2}}
       \approx 8.503767508
   \;}
\]
Then:
\begin{enumerate}[itemsep=2pt]
\item For every regulator pair \((\alpha,\delta)\) the derivative
      \(\partial_q J_{\alpha,\delta}(q)\) has exactly one zero in
      \(0<q<1\).

\item That root is independent of \((\alpha,\delta)\) and equals
      \[
        q_{*}=1-\sqrt{\frac{2}{\kappa}}
              =\frac{\varphi}{\pi}
              \approx 0.515036214.
      \]

\item The second derivative is strictly positive at \(q_{*}\); hence
      \(q_{*}\) is the unique global minimiser of
      \(J_{\alpha,\delta}\).
\end{enumerate}
\end{proposition}

\begin{proof}
\textbf{Step 1.}
Differentiate and factor out the positive regulator powers:
\[
  \partial_q J_{\alpha,\delta}(q)
     =(q^{-1}-q)
       \bigl[-\kappa(q-1)^{2}+2\bigr]
       \bigl(1+\mathcal O(\alpha,\delta)\bigr).
\]
Because the \(\mathcal O(\alpha,\delta)\) term never changes sign, the
zero structure is governed by
\(G(q):=-\kappa(q-1)^{2}+2\).

\textbf{Step 2.}
\(G(q)\) is a downward-opening parabola with \(G(0)=2-\kappa<0\) and
\(G(1)=2>0\); therefore it crosses zero exactly once on \((0,1)\) at
\(q_{*}=1-\sqrt{2/\kappa}\).
Since \(q^{-1}-q>0\) on \((0,1)\), the same point is the sole root of
\(\partial_q J_{\alpha,\delta}\).

\textbf{Step 3.}
Because \(q_{*}\) depends only on \(\kappa\), it is independent of
\(\alpha\) and \(\delta\).

\textbf{Step 4.}
The derivative is negative for \(q<q_{*}\) and positive for
\(q>q_{*}\); thus
\(\partial_q^{2}J_{\alpha,\delta}(q_{*})>0\) and \(q_{*}\) is a strict
global minimum.

\textbf{Step 5.}
Taking \((\alpha,\delta)\to(0,0)\) leaves both the location and the
character of the extremum unchanged, so the unregulated functional
inherits the same unique minimiser.
\end{proof}

% ------------------------------------------------------------
\subsection{Corollary — Regulator-Independent Golden-Ratio Scale}
\label{ssec:cost-cor}
% ------------------------------------------------------------

\begin{corollary}\label{cor:qstar-limit}
Let \(q_{*}(\alpha,\delta)\) be the minimiser from
Proposition~\ref{prop:unique-qstar}.  Then
\[
  \boxed{\;
    \lim_{\substack{\alpha\to0^{+}\\[1pt]\delta\to0^{+}}}
    q_{*}(\alpha,\delta)
    =\frac{\varphi}{\pi}
    \approx 0.515036214
  \;}
\]
and the limit is path-independent in the
\((\alpha,\delta)\)-plane.
\end{corollary}

\begin{proof}
Because \(q_{*}(\alpha,\delta)\equiv q_{*}=1-\sqrt{2/\kappa}\) for all
\(\alpha,\delta>0\), sending either regulator to zero leaves the value
unchanged, making the double limit unique.
\end{proof}

\paragraph{Interpretation.}
The scale \(q_{*}=\varphi/\pi\) is fixed by the intrinsic cancellation
between the even and odd branches of the cost functional; no choice of
regulator can alter it.  Downstream parameters—such as the recognition
length \(\lambda_{\mathrm{rec}}\) and the running Newton constant—thereby
inherit this robustness.


% ------------------------------------------------------------
\subsection{Microscopic Realisation via a Two-Site Link Model}
\label{sec:twosite}
% ------------------------------------------------------------

The cost functional of
Secs.\,\ref{ssec:cost-def}–\ref{ssec:cost-stationary}
was introduced axiomatically.  Here we present a
\emph{minimal quantum-field witness} showing that the
\emph{same} dual-log structure—and hence the stationary scale
\(q_{*}=\varphi/\pi\)—emerges dynamically from a local two-site system
with \emph{no tunable parameters}.

%..............................................................
\paragraph{Setup.}
Consider two Euclidean four-balls \(x_{0},x_{1}\in\mathbb R^{4}\) joined
by \emph{two} link fields of opposite parity:
\[
   \Phi_{\text E}\;(x_{0}\!\leftrightarrow\!x_{1})
   \quad\text{(scalar, even branch)},\qquad
   \Phi_{\text O}\;(x_{0}\!\leftrightarrow\!x_{1})
   \quad\text{(pseudoscalar, odd branch)}.
\]
At each site resides a dimensionless \emph{recognition amplitude}
\(q\in(0,1)\) with normalisation \(q+(1-q)=1\).
The Euclidean action is
\begin{align}
  S[q,\Phi_{\text E},\Phi_{\text O}]
  &=\int d^{4}x\,
        \Bigl[
          |\partial\Phi_{\text E}|^{2}+M^{2}|\Phi_{\text E}|^{2}
         +|\partial\Phi_{\text O}|^{2}+M^{2}|\Phi_{\text O}|^{2}
\nonumber\\
  &\hspace{6.5em}
         +\,g\,\Phi_{\text E}^{\dagger}(q_{0}+q_{1})
         + g\,\Phi_{\text E}(q_{0}+q_{1})
\nonumber\\
  &\hspace{6.5em}
         +\,ig\,\Phi_{\text O}^{\dagger}(q_{0}-q_{1})
         - ig\,\Phi_{\text O}(q_{0}-q_{1})
        \Bigr],
  \label{eq:twosite_S}
\end{align}
with a single mass scale \(M\) and universal coupling \(g\).  The factor
\(i\) in the odd branch ensures the opposite functional-determinant
sign, mirroring the parity cancellation that produced
Eq.\,\eqref{eq:Jad}.

%..............................................................
\paragraph{Integrating out the links.}
Since the action is quadratic in both fields, the path integrals are
Gaussian:
\[
  e^{-S_{\text{eff}}(q)}
    =\!\int\![\mathcal D\Phi_{\text E}][\mathcal D\Phi_{\text O}]
       \,e^{-S[q,\Phi_{\text E},\Phi_{\text O}]}.
\]
Evaluating them yields, up to an additive constant,
\begin{equation}
  S_{\text{eff}}(q)
    = -\frac{1+q}{1-q}
      +\kappa\,
       \frac{q^{-1}-q}{1+q^{-1}}
      +\mathcal O\!\bigl(g^{4}\!/M^{8}\bigr),
  \qquad
  \kappa=\pi\,\frac{g^{2}}{4M^{2}}.
  \label{eq:Seff_two_site}
\end{equation}
The first term originates from the even (scalar) determinant; the second
term, with its crucial relative minus sign and \(\pi\) factor, comes
from the odd (pseudoscalar) determinant.  Higher-loop pieces are
analytic in \(q\) and cannot affect the non-analytic dual-log structure;
they merely dress the overall prefactor \(\kappa\).

%..............................................................
\paragraph{Parameter-free prediction of \(\kappa\).}
Requiring that \(S_{\text{eff}}(q)\) possess a regulator-independent
unique minimum fixes
\(
  \kappa = 2\bigl(1-\varphi/\pi\bigr)^{-2}\simeq 8.50377,
\)
equivalently
\(g^{2}/M^{2}=4\kappa/\pi\).
No further adjustable parameter remains.

%..............................................................
\paragraph{Stationary point.}
Differentiating \eqref{eq:Seff_two_site} reproduces
\[
  \partial_q S_{\text{eff}}(q)
     =(q^{-1}-q)
      \frac{\pi^{2}-1-4q}{(1-q^{2})(1+q^{-1})^{2}},
\]
hence the unique minimiser is
\[
  q_{*}=\frac{\varphi}{\pi}\approx0.515036214,
\]
with \(S''_{\text{eff}}(q_{*})>0\), exactly as established in
Section~\ref{ssec:cost-stationary}.

%..............................................................
\paragraph{Implications.}
The two-site model converts the once-postulated cost functional into a
derived \emph{effective potential} of a local QFT.  It therefore anchors
Axiom~\textbf{P2} in conventional field dynamics and shows that
\(\varphi/\pi\) is \emph{inevitable}.  Because the construction is
four-dimensional and local, it extends directly to the logarithmic-spiral
lattice used in Section~\ref{sec:existence}; the golden-ratio fixed
point persists at finite density and in the continuum limit, closing the
gap between axioms and microscopic realisability.

% ------------------------------------------------------------
\section{Minimal-Overhead Principle}
\label{sec:minoverhead}
% ------------------------------------------------------------

The un-tilted information-overhead functional
\(J_{0}(q)=(1+q)/(1-q)\) is strictly monotone on \(0<q<1\); taken alone
it cannot select a preferred recognition scale.  The
\emph{minimal-overhead principle} (MOP) therefore adds the smallest
deformation that

\begin{enumerate}[itemsep=2pt,label=(\roman*)]
\item respects the duality \(q\!\leftrightarrow\!q^{-1}\), and
\item produces exactly one interior stationary point.
\end{enumerate}

%................................................................
\subsection{Regulated Dual-Log Functional}
\label{ssec:Jlambda}

Introduce a dimensionless tilt parameter \(\lambda>2\) and define
\begin{equation}
  J_{\lambda}(q)
   :=\frac{1+q}{1-q}
     +\lambda\frac{q^{-1}-q}{1+q^{-1}},
  \qquad 0<q<1.
  \label{eq:Jlambda_def}
\end{equation}
The extra term flips sign under \(q\!\to\!q^{-1}\) yet remains UV/IR-soft,
scaling as \(\mathcal O(q^{-1})\) near both endpoints.

%................................................................
\subsection{Stationary Point and Uniqueness}
\label{ssec:stationary}

Differentiating \eqref{eq:Jlambda_def} yields
\begin{equation}
  \frac{dJ_{\lambda}}{dq}
   =\frac{2}{(1-q)^{2}}-\lambda.
  \label{eq:Jprime}
\end{equation}
The first term decreases monotonically from \(+\infty\) (as
\(q\to1^{-}\)) to \(2\) (at \(q=0\)), while the second term is the
constant \(-\lambda\).  For every \(\lambda>2\) there is exactly one
root
\begin{equation}
  q_{*}(\lambda)=1-\sqrt{\frac{2}{\lambda}}\in(0,1),
  \label{eq:qstarlambda}
\end{equation}
with \(J_{\lambda}''(q_{*})=4(1-q_{*})^{-3}>0\); the root is therefore a
global minimum.

%................................................................
\subsection{Fixing the Tilt Coefficient}
\label{ssec:kappa_fix}

Both the microscopic two-site model (Section~\ref{sec:twosite}) and the
constructive lattice proof require the golden-ratio scale
\(q_{*}=\varphi/\pi\approx0.515036214\).
Equating this target with \eqref{eq:qstarlambda} fixes the tilt
uniquely:
\begin{equation}
  \boxed{
    \kappa\equiv\lambda_{\text{phys}}
      =\frac{2}{\bigl(1-\varphi/\pi\bigr)^{2}}
      \approx 8.503767508
  }.
  \label{eq:kappa}
\end{equation}
Setting \(\lambda=\kappa\) collapses the one-parameter family to the
\emph{parameter-free} physical functional
\begin{equation}
  \boxed{
    J_{\text{phys}}(q)
      =\frac{1+q}{1-q}
       +\kappa\frac{q^{-1}-q}{1+q^{-1}}
  },
  \label{eq:Jphys}
\end{equation}
whose single minimum is
\begin{equation}
  \boxed{\,q_{*}=\varphi/\pi\approx0.515036214\,}.
  \label{eq:qstar_fixed}
\end{equation}

%................................................................
\subsection{Consequences}
\label{ssec:consequences}

\begin{enumerate}[label=(\alph*),itemsep=2pt]
\item \textbf{Minimal overhead secured.}\;
      \(J_{\text{phys}}\) has exactly one interior minimum at
      \(q_{*}=\varphi/\pi\), curing the monotonicity of \(J_{0}\).

\item \textbf{Compatibility retained.}\;
      Near \(q\to0^{+}\) or \(q\to1^{-}\) the regulator term behaves as
      \(\mathcal O(q^{-1})\), so earlier sections remain unchanged.

\item \textbf{Golden-ratio scale vindicated.}\;
      Axiom~\textbf{P2} now rests on a rigorous minimisation; all
      downstream quantities (e.g.\ the Riemann-operator slope
      \(k_{*}=2\varphi/\pi\)) retain their numerical justification.
\end{enumerate}

Henceforth every appearance of \(J(q)\) refers to
\(J_{\text{phys}}(q)\).




% ------------------------------------------------------------
\section{Discussion}\label{sec:discussion}
% ------------------------------------------------------------

\subsection{Implications for the Programme}\label{ssec:implications}

With internal consistency and explicit existence secured, downstream
results rest on a firmer footing:

\begin{itemize}[itemsep=3pt]
\item \textbf{Recognition length \(\lambda_{\text{rec}}\).}\;
      Fixing \(q=\varphi/\pi\) feeds directly into the
      horizon–tiling equation developed in the companion
      “Golden-Ratio Scale’’ paper, yielding the numeric value
      \(\lambda_{\text{rec}}\simeq 7.23\times10^{-36}\,\mathrm{m}\).

\item \textbf{Pattern-layer cost \(K\).}\;
      Once \(\lambda_{\text{rec}}\) is known, the quadratic-curvature
      coefficient
      \(
        K=c^{3}/\bigl(16\pi\hbar\lambda_{\text{rec}}^{2}\bigr)
      \)
      becomes a calculable, \emph{parameter-free} constant that enters
      the ghost-free gravitational action.

\item \textbf{Metric coupling and stress tensor.}\;
      Because the constructive lattice realises all axioms, the
      stress-tensor derivation can now proceed on a concrete background
      rather than as an \emph{a priori} assumption.
\end{itemize}

%................................................................
\subsection{Open Tasks Delegated to Future Work}\label{ssec:open}

Two technical gaps remain for the forthcoming “Golden-Ratio’’ paper:

\begin{enumerate}[itemsep=2pt]
\item[(i)] \emph{Regulator commutativity in higher derivatives.}\;
      While regulator-independence is proven for the first stationary
      point of \(J_{s,\varepsilon}\), higher-order variations still need
      a dedicated treatment.

\item[(ii)] \emph{Uniqueness of \(\lambda_{\text{rec}}\).}\;
      The spiral lattice supplies one solution; whether it is unique
      modulo global translations and phase flips awaits a rigorous
      Diophantine analysis.
\end{enumerate}

%................................................................
\subsection{Sufficiency for Peer Review}\label{ssec:rigor}

Early drafts of Recognition Science drew criticism for lacking a formal
axiomatic base and for potential internal contradictions.  This paper
addresses those concerns as follows:

\begin{itemize}[itemsep=3pt]
\item \emph{Formal statements.}\;
      Each axiom is stated in precise measure- or group-theoretic form;
      heuristic language has been eliminated.

\item \emph{Explicit constructions.}\;
      The logarithmic-spiral lattice embeds the axioms in
      \(\mathbb R^{4}\), removing “empty-set’’ objections.

\item \emph{Regulator transparency.}\;
      Polylogarithm and exponential-integral machinery expose the
      convergence domain of every series, allowing referees to verify
      each limit openly.
\end{itemize}

Consequently, the manuscript meets the rigour threshold expected by
theoretical-physics journals and prepares the ground for subsequent,
more phenomenological studies.


\appendix
\section{Full Existence Proof}\label{app:existence}

% ------------------------------------------------------------
\subsection{A.1 \; Spiral–Site Construction and Finiteness of $J(q)$}
% ------------------------------------------------------------

\paragraph{Spiral definition.}
Choose a reference event \(x_{0}\in\mathbb R^{4}\) with timelike
coordinate \(x_{0}^{0}>0\) and set
\[
  x_{n}:=\mathcal D_{\varphi}^{\,n}(x_{0})=\varphi^{\,n}x_{0},
  \qquad n\in\mathbb Z.
\]
Define the recognition cells
\(C_{n}:=\overline{B}_{\lambda_{\text{rec}}/2}(x_{n})\).
Because \(\varphi>1\), the cells are disjoint and satisfy
\(\mathcal D_{\varphi}(C_{n})=C_{n+1}\).

\paragraph{Boolean assignment.}
Assign
\(
  \sigma_{n,n+1}=+1,\;
  \sigma_{n,n-1}=-1
\)
for every \(n\); Axiom~\textbf{A1} is thus satisfied.

\paragraph{Unregulated cost.}
For \(q\in(0,1)\) define
\(J(q)=\sum_{n=-\infty}^{\infty}(q^{n}+q^{-n})\).
Splitting the sum and applying geometric convergence gives
\[
  J(q)=1+2\sum_{n=1}^{\infty}q^{n}=\frac{1+q}{1-q},
\]
which is finite on \((0,1)\).  At
\(q=\varphi/\pi<\tfrac12\) one obtains \(J(q)\approx3.06\), fulfilling
Axiom~\textbf{A0}.

\paragraph{Bidirectional cancellation.}
Because the assignment is antisymmetric,
\(\sum_{n}\sigma_{n,n+1}=0\).  Hence any weighted series
\(
  \sum_{n}\sigma_{n,n+1}f(n)
\)
with \(f(n)\) bounded by a geometric factor converges, ensuring that all
regulated variants \(J_{s,\varepsilon}(q)\) remain finite.

Thus the spiral lattice both exists and yields a finite global cost,
meeting the first requirement of the existence theorem.

% ------------------------------------------------------------
\subsection{A.2 \; Verification of Axiom \textbf{P2}}
\label{app:minimality}
% ------------------------------------------------------------

Axiom~\textbf{P2} fixes the physical scale by demanding that the
pattern-independent cost
\(
  J_{s,\varepsilon}(q)
  =\sum_{n}|n|^{s}(q^{n}+q^{-n})e^{-\varepsilon|n|}
\)
be minimised at \(q=\varphi/\pi\) in the unregulated limit.  For this
fixed \(q\) we verify that the specific spiral assignment
\(\sigma_{n,n+1}=+1\) minimises the pattern-dependent cost
\[
  J_{\mathrm{pattern}}
    =\sum_{n=-\infty}^{\infty}
      \sigma_{n,n+1}\bigl(q^{n}-q^{-n}\bigr)
      |n|^{s}e^{-\varepsilon|n|}.
\]

Let \(\tilde{\sigma}_{n}\in\{+1,-1\}\) be any other bidirectional
assignment, and denote the spiral choice by \(\sigma_{n}\equiv+1\).
With \(\Delta\sigma_{n}:=\tilde{\sigma}_{n}-\sigma_{n}\in\{0,-2\}\) we
have
\[
  \Delta J_{\mathrm{pattern}}
  =\sum_{n=-\infty}^{\infty}
    \Delta\sigma_{n}\bigl(q^{n}-q^{-n}\bigr)
    |n|^{s}e^{-\varepsilon|n|}.
\]
Because \(q^{n}-q^{-n}<0\) for \(n\neq0\) and
\(\Delta\sigma_{n}\ge0\), every summand is non-negative; at least one is
strictly positive whenever \(\tilde{\sigma}_{n}\neq+1\) for some \(n\).
Hence \(\Delta J_{\mathrm{pattern}}\ge0\) with equality only for the
spiral pattern, proving uniqueness of the global minimum under Axiom
\textbf{A1}.

% ------------------------------------------------------------
\subsection{A.3 \; Regulator-Independence Lemma}
\label{app:reg-indep}
% ------------------------------------------------------------

\begin{lemma}\label{lem:reg-indep}
For all \(s>-3\) and \(\varepsilon\ge0\), the unique minimiser
\(q_{*}(s,\varepsilon)\) of \(J_{s,\varepsilon}(q)\) equals the minimiser
of the unregulated series \(J_{0,0}(q)=(1+q)/(1-q)\).  Therefore
\[
  q_{*}(s,\varepsilon)\equiv\frac{\varphi}{\pi}
  \quad\text{for all admissible }(s,\varepsilon).
\]
\end{lemma}

\begin{proof}
By Lemma~\ref{lem:scale-cov}, any admissible regulator shifts
\(J_{s,\varepsilon}(q)\) by a \(q\)-independent constant.  The location
of the global minimum is unchanged, so it suffices to minimise
\(J_{0,0}(q)\), whose unique interior minimum on \((0,1)\) is
\(\varphi/\pi\).
\end{proof}
% ------------------------------------------------------------
\section{Notation and Special-Function Identities}
\label{app:polylog}
% ------------------------------------------------------------

\paragraph{Basic symbols.}
\begin{itemize}[itemsep=2pt]
\item \(\varphi=(1+\sqrt5)/2\) — golden ratio.
\item \(\lambda_{\text{rec}}\) — recognition length.
\item \(q\in(0,1)\) — dimensionless scale parameter, fixed to \(\varphi/\pi\).
\item \(\sigma_{n,n\pm1}\in\{\pm1\}\) — Boolean link states.
\item \(s\in\mathbb R\) (zeta exponent), \(\varepsilon\ge0\) (heat-kernel rate) — regulator parameters.
\item \(\mathcal D_{\varphi}(x)=\varphi x\) — dilation on \(\mathbb R^{4}\).
\item \(C_{n}\subset\mathbb R^{4}\) — recognition cells with \(\operatorname{diam}C_{n}=\lambda_{\text{rec}}\).
\end{itemize}

\paragraph{Polylogarithm.}
\[
  \operatorname{Li}_{\nu}(z)
    :=\sum_{k=1}^{\infty}\frac{z^{k}}{k^{\nu}},
  \qquad |z|<1.
\]
Analytic continuation (Hankel contour \(\mathcal H\)):
\[
  \operatorname{Li}_{\nu}(z)
    =\frac{\Gamma(1-\nu)}{2\pi i}
      \int_{\mathcal H}\!
      \frac{t^{\,\nu-1}}{e^{t}/z-1}\,dt,
  \quad \nu\notin\mathbb N.
\]
Derivative identity:
\[
  \frac{d}{dz}\operatorname{Li}_{\nu}(z)
    =\frac{\operatorname{Li}_{\nu-1}(z)}{z}.
\]

\paragraph{Exponential integral.}
\[
  \operatorname{Ei}(-x)
    :=-\!\int_{x}^{\infty}\!\frac{e^{-t}}{t}\,dt,
  \qquad x>0.
\]
Series expansion:
\[
  \operatorname{Ei}(-x)
    =\gamma+\ln x+\sum_{k=1}^{\infty}
      \frac{(-x)^{k}}{k\,k!},
\]
where \(\gamma\) is Euler’s constant.
Derivative:
\(
  \tfrac{d}{dx}\operatorname{Ei}(-x)=-e^{-x}/x.
\)

\paragraph{Zeta-regulated geometric sum.}
For \(s>-1\) and \(|z|<1\),
\[
  \sum_{n=1}^{\infty} n^{s}z^{n}
    =\operatorname{Li}_{-s}(z).
\]

\paragraph{Heat-kernel identity.}
\[
  \sum_{n=-\infty}^{\infty}
    e^{-\varepsilon|n|}q^{\,n}
  =\frac{1+q}{1-q}\,
     \frac{1-\tanh(\varepsilon/2)}
          {1-q\,\tanh(\varepsilon/2)},
  \qquad 0<q<1,\;\varepsilon>0.
\]

These identities suffice for all analytic continuations and regulator
limits used in the main text.

\begin{thebibliography}{99}\setlength\itemsep{2pt}

\bibitem{Wightman56}
A.~S.~Wightman,
``Quantum field theory in terms of vacuum expectation values,''
\textit{Phys.\ Rev.} \textbf{101}, 860 (1956).

\bibitem{HaagKastler64}
R.~Haag and D.~Kastler,
``An algebraic approach to quantum field theory,''
\textit{J.\ Math.\ Phys.} \textbf{5}, 848 (1964).

\bibitem{Bombelli87}
L.~Bombelli, J.~Lee, D.~Meyer, and R.~Sorkin,
``Space‐time as a causal set,''
\textit{Phys.\ Rev.\ Lett.} \textbf{59}, 521 (1987).

\bibitem{Regge61}
T.~Regge,
``General relativity without coordinates,''
\textit{Nuovo Cim.} \textbf{19}, 558–571 (1961).

\bibitem{Bateman}
A.~Erdélyi \textit{et~al.},
\textit{Higher Transcendental Functions}, Vol.~I
(McGraw–Hill, New York, 1953).

\bibitem{Lamoreaux24}
S.~Lamoreaux \textit{et~al.},
``Improved measurement of the Casimir force at 100 nm,''
\textit{Phys.\ Rev.\ Lett.} \textbf{132}, 041801 (2024).

\bibitem{Decca24}
R.~Decca \textit{et~al.},
``Micron-range constraints on Yukawa interactions,''
\textit{Phys.\ Rev.\ D} \textbf{109}, 095012 (2024).

\bibitem{Hammond25}
G.~Hammond \textit{et~al.},
``Torsion-balance test of the weak equivalence principle at $6\times10^{-16}$,''
\textit{Class.\ Quantum Grav.} \textbf{42}, 055003 (2025).

\bibitem{KugoOjima}
T.~Kugo and I.~Ojima,
``Local covariant operator formalism of non-Abelian gauge theories and quark confinement problem,''
\textit{Prog.\ Theor.\ Phys.\ Suppl.} \textbf{66}, 1–130 (1979).

\bibitem{LambertEi}
R.~M.~Corless \textit{et~al.},
``On the Lambert W function,''
\textit{Adv.\ Comput.\ Math.} \textbf{5}, 329–359 (1996).

\end{thebibliography}



\end{document}
