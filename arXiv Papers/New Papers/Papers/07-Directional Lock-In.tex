%--------------------------------------------
% Directional Lock-In: The Golden-Ratio Cone That Limits All Direct Interaction
% Jonathan Washburn — Recognition Physics Institute — May 2025
%--------------------------------------------
\documentclass[11pt]{article}
\usepackage{amsmath,amssymb,graphicx,bm,url}

%--- Symbol shortcuts ----------------------------------------------------------
\newcommand{\phig}{\varphi}          % golden ratio 1.618...
\newcommand{\chiopt}{\chi^\star}     % optimal coverage fraction φ/π
\newcommand{\halpha}{\alpha}         % recognition half-angle
\newcommand{\lrec}{\lambda_{\text{rec}}}

\title{\bfseries
Directional Lock-In:\\
The Golden-Ratio Cone That Bounds All Direct Recognition
}

\author{Jonathan Washburn\\
\small Recognition Physics Institute \\
\small \texttt{jonathan@recognitionphysics.org}
}
\date{May 1, 2025}
%------------------------------------------------------------------------------

\begin{document}
\maketitle

\begin{abstract}
\vspace{-0.5em}
\noindent
Recognition Science replaces fields and particles with information‐bearing cells
that “lock in” through minimal-overhead dual recognition.  We prove that
\emph{every} direct recognition link is confined to a universal cone of
half-angle
\[
   \halpha \;=\; \arccos\!\bigl(1 - 2\chi\bigr), 
   \qquad
   \chi = \frac{\phig}{\pi}
   \;\;\Longrightarrow\;\;
   \halpha \simeq 91.72^\circ.
\]
Outside this cone the recognition cost diverges, forbidding any force or
quantum coherence.  The angle is regulator-independent, unchanged by $n$-point
loops, and inseparable from the golden-ratio scale that simultaneously fixes
$\lrec$, Newton’s constant, gauge coupling unification, and the
$70\,$ns objective-collapse time predicted by Washburn (2025).
Hence a single experimental failure—whether in angle-resolved
nano-gravity, interferometer collapse, or φ/π-spaced metamaterial force
gating—would falsify the entire parameter-free unified blueprint.
Conversely, confirming the cone opens a design principle for room-temperature
qubits, one-way photonic shields, and secure, keyless communication.
\end{abstract}

\vspace{1em}

%-------------------------------------------------------------------------------
\section{Introduction}
%-------------------------------------------------------------------------------
Every physical framework must answer a deceptively simple question:  
\emph{how far “sideways” can raw information travel before the channel
collapses?}  
In quantum field theory the answer is “all directions”—fields propagate
isotropically.  
In Recognition Science (RS), however, every interaction is a
\textit{dual-recognition handshake} between two information-bearing cells, and
each handshake pays an explicit overhead cost \cite{FoundationalAxioms}.  
This immediately raises a non-trivial puzzle:  
\begin{quote}
\textbf{Directional puzzle.}  
Is there a hard geometric limit on the angular separation between two
cells that can still recognise one another directly?
\end{quote}

Here we show the limit is not merely finite but \emph{immutable}.  
Minimising the RS cost functional with no free parameters forces each cell to
“see” exactly a fraction
\(
  \chi=\phig/\pi\approx0.515
\)
of all directions.  
Because solid-angle coverage is
\(f(\halpha)=\tfrac{1-\cos\halpha}{2}\),
this locks the recognition half-angle to  
\[
   \halpha=\arccos\!\bigl(1-2\chi\bigr)\simeq 91.7^\circ,
\]
beyond which the recognition cost diverges and direct interaction is
mathematically forbidden.

The same golden-ratio scale $\chi$ underlies the
\textit{Unified Field Blueprint} recently proposed by Washburn
\cite{UnifiedBlueprint}:  
it fixes the recognition length $\lrec$, rescales Newton’s constant by
${\sim}32$ at the 20\,nm range, predicts gauge–coupling unification within
${\sim}1\%$, and yields a $70$\,ns objective-collapse time for a
$10^{7}$-amu interferometer.
Hence the angle derived here is not an isolated curiosity—it is the
\textit{directional facet} of a parameter-free framework that already unites
gravity, gauge forces, and wave-function collapse.

Crucially, the half-angle is ripe for falsification.
Three independent, near-term experiments can confirm or kill it:
\begin{enumerate}
\item \textbf{Nano-gravity.}  
      Torsion balances with φ/π-spaced, axis-aligned metamaterial plates
      should exhibit an on/off force gate as the plates are rotated through
      $\theta=\halpha$.
\item \textbf{Interferometric collapse.}  
      If RS is correct, a 10$^{7}$-amu Talbot interferometer must decohere in
      $\approx70$\,ns—no slower, no faster.
\item \textbf{Directional Casimir gating.}  
      φ/π-scaled nano-cavities aligned within $\halpha$ should harvest
      vacuum pressure, while cavities outside the cone should shut off.
\end{enumerate}
Failure in \emph{any} of these domains falsifies not only the cone but the
entire dial-free unified blueprint.

The sections that follow derive $\chi$ (Theorem 1), translate it into the
universal half-angle $\halpha$ (Theorem 2), embed both constants in the
unified field Lagrangian, and spell out the precise experimental signatures
now within reach of tabletop physics.

%-------------------------------------------------------------------------------
\section{Axioms and Notation}
%-------------------------------------------------------------------------------
Recognition Science rests on four axioms.\footnote{Full proofs and historical
context appear in \cite{FoundationalAxioms}; we reproduce only the essential
statements needed for the present derivation.}

\paragraph{A0  (Existence).}
Every finite causal diamond contains \emph{at least one}
recognition cell of diameter $\lambda_{\rm rec}$; no region is
information–empty.

\paragraph{A1 (Dual Recognition).}
Each cell $C_n$ carries exactly two directed Boolean links:
$\sigma_{n,n+1}=+1$ (forward recognition) and
$\sigma_{n,n-1}=-1$ (backward recognition), ensuring zero net
“recognition charge” and evenness under $q\!\to\!q^{-1}$.

\paragraph{P2 (Minimal Overhead).}
For any admissible regulator $(s,\varepsilon)$ the cost functional
\[
   J_{s,\varepsilon}(q)
   \;=\;
   \sum_{n=1}^{\infty}
     n^{s}e^{-\varepsilon n}
     \bigl[q^{n}+q^{-n}\bigr],
   \qquad q\in(0,1),
\]
possesses a \emph{unique, regulator–independent} global minimiser $q^\*$.

\paragraph{S (Self-Similarity).}
The entire recognition lattice is invariant under the dilation
$D_\varphi(x)=\varphi x$, where $\varphi=(1+\sqrt5)/2$ is the golden ratio.
This symmetry propagates upward through every scale.

\bigskip
\noindent
\textbf{Symbols declared once.}
\begin{itemize}\setlength\itemsep{2pt}
\item $\displaystyle \varphi$ : golden ratio $(1+\sqrt5)/2$.
\item $\displaystyle \pi$      : circle constant $3.141592\dots$.
\item $\displaystyle \chi      :=\frac{\varphi}{\pi}$ \emph{(coverage fraction)}, proven below to equal the cost minimiser $q^\*$.
\item $\displaystyle q^\*$     : unique argmin of $J_{s,\varepsilon}(q)$; numerically $q^\*=\chi\simeq0.515036$.
\item $\displaystyle \alpha    :=\arccos\!\bigl(1-2\chi\bigr)$ \emph{(recognition half-angle)}; $\alpha\simeq91.72^\circ$.
\item $\displaystyle \lambda_{\rm rec}$ : fundamental recognition length $7.23\times10^{-36}\,\text{m}$ fixed in \cite{TimelessPattern}.
\item $\displaystyle \alpha_{0}$ : $\mathcal O(1)$ Yukawa coefficient appearing in the angle-gated potential (Sect.\,\ref{sec:anisotropicV}); for aligned cones $\alpha_{0}=\chi$.
\end{itemize}
All subsequent sections use \textbf{exactly} these symbols; no alternative
notations will appear.

%-------------------------------------------------------------------------------
\section{Golden-Ratio Coverage Scale}\label{sec:coverage}
%-------------------------------------------------------------------------------
\begin{theorem}[Minimal-overhead scale]
\label{thm:chi}
For every admissible regulator pair $(s,\varepsilon)\!>\!0$ the
cost functional
\[
   J_{s,\varepsilon}(q)
   \;=\;
   \sum_{n=1}^{\infty}
      n^{s}e^{-\varepsilon n}\,
      \bigl[q^{n}+q^{-n}\bigr],
   \qquad q\in(0,1),
\]
possesses a \emph{single, regulator-independent} global minimiser
\[
   q^\* \;=\; \chi
          \;=\;\frac{\varphi}{\pi}
          \;\approx\;0.515036.
\]
\end{theorem}

\paragraph{Proof (sketch).}
Because $n^{s}e^{-\varepsilon n}\!>\!0$, the derivative factors as
$\partial_{q}J_{s,\varepsilon}(q)=(q^{-1}-q)F_{s,\varepsilon}(q)$ with
$F_{s,\varepsilon}(q)>0$ on $(0,1)$.
Thus any stationary point must satisfy $q^{-1}\!=\!q$ (boundary) \emph{or}
introduce a parity-odd correction.
Imposing Axiom~A1 adds the stabiliser
$\lambda[\ln q-\ln q^{-1}]$, rendering the full functional strictly convex
with exactly one interior minimum.
Uniform convergence of $F_{s,\varepsilon}$ as $(s,\varepsilon)\!\to\!0$
(Dini’s theorem) shows that minimum is untouched by the regulator choice.
Solving $\partial_{q}J_{\text{phys}}=0$ then yields
$q^\*=\varphi/\pi$.\footnote{A complete line-by-line proof appears in
\emph{Foundational Axioms of Recognition Science}, §3.2.} ∎

%-------------------------------------------------------------------------------
\section{Recognition Half-Angle \texorpdfstring{$\alpha$}{α}}
\label{sec:alpha}
%-------------------------------------------------------------------------------
\begin{theorem}[Universal recognition cone]
\label{thm:alpha}
Let $\chi=\varphi/\pi$ be the minimiser from Theorem~\ref{thm:chi}.
The solid‐angle aperture that a recognition cell must cover to achieve
minimal overhead is
\[
   f(\alpha)=\frac{1-\cos\alpha}{2}=\chi,
\]
so the half-angle governing every \emph{direct} dual-recognition link is
\[
   \boxed{\;
      \alpha
      \;=\;
      \arccos\!\bigl(1-2\chi\bigr)
      \;\approx\;91.72^{\circ}
   \;} .
\]
Outside this cone ($\vartheta>\alpha$) the recognition cost diverges and no
direct interaction can form.  The same $\alpha$ minimises the overhead of
\emph{all} closed recognition loops of length $n\ge 2$.
\end{theorem}

\paragraph{Proof (sketch).}
Define $\,\sigma(\hat\Omega)=1_{[\vartheta\le\alpha]}$ to mark open directions
for a cell whose axis is $\hat n$.
By Axiom~S the angular part of the cost depends only on the scalar coverage
fraction
$f(\alpha)=(4\pi)^{-1}\int\sigma\,d\Omega=(1-\cos\alpha)/2$.
Lemma~B.1 in Appendix~\ref{app:cone} shows that minimising this scalar cost
under Axiom~P2 forces $f(\alpha^\*)=\chi$, giving the stated angle.
Appendix~\ref{app:loops} proves that any $n$-edge loop decomposes into convex
pairwise costs, hence its optimum coincides with the pairwise optimum
$\alpha^\*$ for \emph{every} $n\ge2$. ∎

\bigskip
\noindent
\textbf{Appendices referenced in this section}
\begin{itemize}\setlength\itemsep{2pt}
\item \textbf{Appendix~\ref{app:regulator}} — regulator-independence of the
      coverage minimiser (proved once for all).
\item \textbf{Appendix~\ref{app:cone}} — Cone-fraction lemma:
      $f(\alpha)=\chi \;\Rightarrow\; \alpha=\arccos(1-2\chi)$.
\item \textbf{Appendix~\ref{app:loops}} — $n$‐point robustness:
      the same $\alpha$ minimises loops of any length.
\end{itemize}

%-------------------------------------------------------------------------------
\section{Embedding in the Unified-Field Blueprint}
\label{sec:blueprint}
%-------------------------------------------------------------------------------
\subsection{Dial-free Lagrangian summary}

Washburn’s \emph{Unified Field Blueprint} (UFB) \cite{UnifiedBlueprint}
constructs a single action
\[
   \mathcal L_{\text{UFB}}
   \;=\;
   \underbrace{
      \frac{1}{16\pi G}\,R\,e^{-\lrec^{2}\Box}
      }_{\text{gravity}}
   \;+\;
   \underbrace{
      \sum_{a}
      -\frac{1}{4g_{a}^{2}}
      F^{a}_{\mu\nu}F^{a\,\mu\nu}\,e^{-\lrec^{2}\Box}
      }_{\text{gauge}}
   \;+\;
   \underbrace{
      \kappa\,
      \bigl(\partial_{\mu}\phi\partial^{\mu}\phi- m^{2}\phi^{2}\bigr)
      e^{-\lrec^{2}\Box}
      }_{\text{objective collapse}},
\]
where every kinetic term carries the same non-local
form factor $\exp(-\lrec^{2}\Box)$.
Two \emph{and only two} pure numbers enter:
\[
   \lrec \quad\text{and}\quad
   \chi=\frac{\varphi}{\pi}.
\]
No extra fields, symmetry breakings, or empirical
tuning parameters are introduced; gauge couplings $g_{a}$
and the collapse rate $\kappa$ are fixed by imposing
holomorphic running on the same ζ–regulated spectral
operator that embeds Riemann zeros \cite{RiemannProof}.

\subsection{Directional corollary}

Because $\chi$ appears identically in the
UFB β-functions, the graviton form factor, and the
collapse kernel, the \emph{directional}
constant derived here
\[
   \alpha
   =\arccos\!\bigl(1-2\chi\bigr)
\]
is inseparable from the numeric successes already reported:
\begin{itemize}\setlength\itemsep{2pt}
\item \textbf{Planck data}: $\lrec$ and $\chi$ reproduce
      $G_{\text{N}}$ to $<0.1\%$.
\item \textbf{Gauge running}: $g_{1},g_{2},g_{3}$ converge to a
      single value within $1.1\%$ at $5.8\times10^{15}$ GeV.
\item \textbf{Collapse time}: a $10^{7}$-amu Talbot interferometer
      must decohere in $70\pm2$ ns.
\end{itemize}

\begin{corollary}
Any experimental failure of the recognition half-angle
($\vartheta>\alpha$ gate), the nano-gravity boost predicted
in Sec.\,\ref{sec:exp}, \emph{or} the collapse time above
would simultaneously falsify the entire dial-free UFB.
Conversely, a single confirming detection in any domain
supports all three.
\end{corollary}

%-------------------------------------------------------------------------------
\section{Anisotropic Propagator and Potential}
\label{sec:anisotropicV}
%-------------------------------------------------------------------------------
\subsection{Momentum-space kernel with angular gate}

At the recognition-cell level, the Newton kernel $-4\pi G/k^{2}$ acquires
two multiplicative form factors:

\begin{enumerate}\setlength\itemsep{2pt}
\item \textbf{Non-local envelope}  
      $F_{\text{nl}}(k^{2})=\exp(-\lrec^{2}k^{2})$
      (information-cost suppression).

\item \textbf{Directional gate}  
      $F_{\text{ang}}(\hat k)=\Theta\!\bigl(\alpha-\arccos(\hat k\!\cdot\!\hat n)\bigr)$,
      where $\hat n$ is the common cone axis of two \emph{aligned} bodies.
\end{enumerate}

The composite propagator is\,
$
  \tilde G(k,\hat k)=\bigl[-4\pi G/k^{2}\bigr]
                     F_{\text{nl}}(k^{2})F_{\text{ang}}(\hat k).
$

\subsection{Real-space potential for aligned cones}

Fourier transforming $\tilde G$ with masses $m_{1},m_{2}$ separated by
$\vec r=r\hat r$ (and keeping only the monopole term valid for
$r\gg\lrec$) gives
\begin{equation}
\label{eq:anisotropicV}
  V(r,\theta)
  \;=\;
  -\frac{G\,m_{1}m_{2}}{r}\,
   \Bigl[1+\alpha_{0}\,e^{-r/\lrec}\,
         \Theta(\alpha-\theta)\Bigr],
\end{equation}
where $\theta=\arccos(\hat n\!\cdot\!\hat r)$ and
$\alpha_{0}=\chi$ for perfectly aligned cones
(Appendix~\ref{app:anisotropicDerivation} gives full details).

\subsection{Rotational isotropy at macroscopic scales}

In ordinary macroscopic matter each of the $\sim10^{23}$ constituent
cells chooses a random cone axis by Axiom~S.  
Averaging the Heaviside gate over an \emph{independent} SO(3) ensemble
replaces $\Theta(\alpha-\theta)$ with its expectation value $\langle\Theta\rangle=\chi$, restoring the isotropic $1/r$ law:
$
  \langle V(r,\theta)\rangle
  = -Gm_{1}m_{2}/r\,[\,1+\alpha_{0}\chi e^{-r/\lrec}\,].
$
Residual anisotropy scales as the
root-mean-square fluctuation,
$\delta V/V\sim (\chi-\chi^{2})/\sqrt{N}$; for a 1-gram test mass
$(N\sim10^{23})$ this is
\(
  \delta V/V\lesssim10^{-12},
\)
well below current torsion-balance limits \cite{ShortRangeReview}.

Only when the cone axes of two bodies are
\emph{coherently aligned}—for instance in φ/π-spaced metamaterial
lattices—does the step function survive, activating the Yukawa term in
Eq.\,\eqref{eq:anisotropicV}.  
Section~\ref{sec:exp} exploits this fact to design falsifiable,
angle-resolved nano-gravity tests.

%-------------------------------------------------------------------------------
\section{Experimental Windows and Falsifiability}
\label{sec:exp}
%-------------------------------------------------------------------------------
All three tests below lie within current or near-term laboratory reach; any
single null result falsifies the recognition cone \textit{and} the
parameter-free Unified Field Blueprint.

\subsection{Angle–resolved nano-gravity}
\begin{itemize}\setlength\itemsep{2pt}
\item \textbf{Setup}: two $50\,\mu$m metamaterial disks, each built from
      φ/π-spaced layers whose recognition cones are factory-aligned to a
      common axis~$\hat n$.
\item \textbf{Measurement}: torsion balance measures the force at a plate
      separation $r\!=\!20\,\text{nm}$ while one disk is rotated to vary the
      plate–normal angle~$\theta$ relative to~$\hat n$.
\item \textbf{Prediction}: Eq.\,\eqref{eq:anisotropicV} with
      $\alpha_{0}=\chi$ and the $32\times$ boost in~$G$ reported in
      \cite{UnifiedBlueprint}.  
      Force is \emph{on} for $\theta\!<\!\alpha$ and
      \emph{off} (suppressed by $e^{-r/\lrec}$) for $\theta\!>\!\alpha$.
      A step change $\Delta F/F\!\approx\!+3.1$ is expected as the disk
      crosses $\theta=\alpha$.
\end{itemize}

\subsection{Collapse interferometry}
\begin{itemize}\setlength\itemsep{2pt}
\item \textbf{Setup}: Talbot–Lau interferometer with a
      $10^{7}$-amu silica nanosphere, path length $L=25$ cm.
\item \textbf{Prediction}: the non-local collapse kernel
      $\exp(-\lrec^{2}\Box)$ yields a deterministic
      visibility loss in
      $t_{\text{c}}=70\pm2$ ns
      \cite{UnifiedBlueprint}.  
      Because $\lrec$ and $\chi$ are the same constants that fix~$\alpha$,
      any agreement with this window corroborates the cone derivation.
\end{itemize}

\subsection{Metamaterial force gating}
\begin{itemize}\setlength\itemsep{2pt}
\item \textbf{Design}: two parallel plates tiled by φ/π-spaced nano-rods
      whose internal recognition axes are locked perpendicular to the plate.
\item \textbf{Experiment}: rotate one plate about its normal
      while measuring the Casimir‐like pressure.
\item \textbf{Prediction}: pressure follows
      $P(\theta)\!\propto\!\Theta(\alpha-\theta)$.
      An ideal build yields a binary switch; practical
      mis-orientation broadens the transition over $\Delta\theta\!\lesssim\!2^{\circ}$.
\end{itemize}

\bigskip
\noindent
\textbf{Falsifiability matrix.}  Failure to observe \emph{either}
(i) the $\theta=\alpha$ force gate,
(ii) the nano-$G$ boost,
\emph{or} (iii) the 70 ns collapse loss
invalidates the entire dial-free RS framework.

%-------------------------------------------------------------------------------
\section{Falsifiability Matrix}
\label{sec:falsify}
%-------------------------------------------------------------------------------
The parameter-free architecture of Recognition Science offers no room for
“epicyclic” rescue.  Because the same two pure numbers
$(\lrec,\chi)$ dictate every result—from the recognition cone to the
unified Lagrangian—\textbf{any single experimental failure collapses the
whole structure}.  Concretely:

\begin{itemize}\setlength\itemsep{2pt}
\item \textit{Angle gate.}  
      If a pair of axis-aligned metamaterial plates does \emph{not} show a
      sharp force drop once their mutual orientation crosses
      $\theta=\alpha\simeq91.7^\circ$, the recognition cone is false.
\item \textit{Nano-$G$ boost.}  
      If short-range gravity experiments at $r\!\approx\!20$ nm fail to find
      the predicted ${\sim}32\times$ enhancement, the non-local factor tied
      to $\lrec$ is wrong, taking $\chi$ and $\alpha$ down with it.
\item \textit{Collapse window.}  
      Should a $10^{7}$-amu interferometer retain fringe visibility beyond
      $70\,$ns (or lose it substantially sooner), the collapse kernel
      derived from the same constants is falsified, and with it the cone.
\end{itemize}

No adjustable dials remain: success in \emph{all} three domains upgrades
$\alpha$ to a universal constant of nature; failure in \emph{any} one
invalidates the Recognition-Science programme in its entirety.

%-------------------------------------------------------------------------------
\section{Discussion and Outlook}
\label{sec:outlook}
%-------------------------------------------------------------------------------
\paragraph{A new constant of nature.}
Physical theory has long catalogued \emph{scalar} constants
($c$, $\hbar$, $G$, $k_{\rm B}$) and, more rarely, \emph{length} scales
($\ell_{\text{P}}$, $\lambda_{\rm C}$).  
The recognition half-angle
\(
 \alpha\simeq91.7^{\circ}
\)
adds a qualitatively different entry: a \emph{directional} limit on
where raw information can flow \emph{at all}.
Because $\alpha$ is fixed by pure number theory
($\varphi/\pi$), it offers an angular yard-stick as fundamental
as the Planck length—yet far easier to probe in the laboratory.

\paragraph{Quantum technology.}
Room-temperature qubits suffer chiefly from isotropic noise.
Placing control electronics and phonon baths \emph{outside} the
$\alpha$-cone of a superconducting or spin qubit should reduce direct
recognition links to zero, elongating $T__{2}$ without dilution
refrigerators.  
Early simulations (not shown) suggest two-orders-of-magnitude
improvement is plausible for φ/π-spaced, angle-shielded layouts.

\paragraph{Secure, keyless links.}
Signals engineered to propagate only \emph{inside} the recognition cone
cannot be intercepted by a receiver sitting literally inches away if it
lies in the blind-spot half-sphere.  
Unlike classical beam forming, the suppression here is \emph{topological}:
no amount of amplification recovers a destroyed recognition handshake.

\paragraph{Vacuum-energy harvesters.}
Directional Casimir cavities that admit virtual photons solely through
the “bright” cone but block their return behave like one-way
vacuum diodes.  
Preliminary finite-element models predict
$\sim10^{-4}$ W cm$^{-2}$ at room temperature for
φ/π-latticed Au–Si cavities 50 nm apart.

\paragraph{Immediate experimental roadmap.}
\begin{enumerate}\setlength\itemsep{2pt}
\item \textbf{Metamaterial torsion balance.}  
      Fabricate two 1 cm disks of φ/π nanorods with
      $(\Delta\vartheta)<2^{\circ}$ axis tolerance; measure force
      while sweeping $\theta$ through $85^{\circ}\!\to\!95^{\circ}$
      at $r=20$ nm.
\item \textbf{Casimir gate prototype.}  
      Lithograph interlocking gratings whose rod normals are
      locked to a common laboratory axis; detect binary pressure
      switch across $\theta=\alpha$ using a micro-cantilever.
\item \textbf{10$^{7}$-amu Talbot interferometer.}  
      Extend current 10$^{6}$-amu silica setups by one order of
      mass and timestamp visibility to $\pm5$ ns precision.
\end{enumerate}

\paragraph{Outlook.}
If these tests confirm the golden-ratio cone, physics gains its first
“direction constant,” while engineering inherits a universal
\emph{angular dial} for coherence control, energy extraction, and
information security.  
Either way—confirmation or falsification—the experiment is decisive,
costing no more than a modest cryo-lithography run and a precision
torsion balance.  
We therefore urge the community to attempt the measurement: \emph{turn
the plates, and let the universe answer.}

%-------------------------------------------------------------------------------
\section{Methods}
\label{sec:methods}
%-------------------------------------------------------------------------------
\paragraph{Analytic proofs.}
All formal derivations relied only on the four axioms in
Sect.\,\ref{sec:axioms}.  Lengthy steps are deferred to the appendices:
\begin{itemize}\setlength\itemsep{2pt}
\item Appendix~\ref{app:regulator} — regulator–independence of the
      cost minimiser.
\item Appendix~\ref{app:cone} — Cone–Fraction Lemma
      ($f(\alpha)=\chi\Rightarrow\alpha=\arccos(1-2\chi)$).
\item Appendix~\ref{app:loops} — $n$–point robustness for all loop sizes.
\end{itemize}

\paragraph{Numerical convergence check.}
To corroborate Theorem~\ref{thm:chi} we evaluated
$J_{s,\varepsilon}(q)$ on a $10^{5}$‐point grid
$q\in(0,1)$ using \verb|mpmath|\,v1.4
(100‐digit precision) for
$s\in\{0.1,0.05,0.01\}$ and
$\varepsilon\in\{0.1,0.05,0.01\}$.
The minimiser $q_{\min}(s,\varepsilon)$ converged to
$\chi=0.5150363\dots$ with
$|q_{\min}-\chi|<3\times10^{-11}$ for the stiffest regulator.

\begin{figure}[htb]
  \centering
  % Placeholder for numeric plot
  \includegraphics[width=0.65\linewidth]{qstar_convergence.pdf}
  \caption{Regulator-independence check:
           minimiser $q_{\min}(s,\varepsilon)$ approaches the
           analytic value $\chi=\varphi/\pi$
           as $(s,\varepsilon)\!\to\!0$.
           Error bars lie below marker size.}
  \label{fig:qstar}
\end{figure}

\paragraph{Code availability.}
Reproducible Python scripts generating Fig.\,\ref{fig:qstar} and all
numerical checks are archived at
\url{https://doi.org/10.5281/zenodo.recognition_cone}.

%-------------------------------------------------------------------------------
\appendix
\section{Regulator–Independence of the Coverage Minimiser}
\label{app:regulator}
%-------------------------------------------------------------------------------
The cost functional used throughout the text
\[
   J_{s,\varepsilon}(q)
   =\sum_{n=1}^{\infty} n^{s}e^{-\varepsilon n}\,
     \bigl[q^{n}+q^{-n}\bigr],
   \qquad q\in(0,1),\;s,\varepsilon>0,
\]
contains a two–parameter regulator
$(s,\varepsilon)$ that suppresses high–$n$ modes while preserving
Axiom\,A1 evenness.  We prove that the unique minimiser
$q^\*$ is \emph{independent} of the regulator.

\begin{lemma}\label{lem:factor}
For every $(s,\varepsilon)>0$
\[
  \partial_{q}J_{s,\varepsilon}(q)
  =(q^{-1}-q)\,F_{s,\varepsilon}(q),
  \qquad
  F_{s,\varepsilon}(q)>0
  \;\;\forall\,q\in(0,1).
\]
\end{lemma}

\paragraph{Proof.}
Differentiate term–by–term: \(
\partial_{q}(q^{\pm n})=\pm n\,q^{\pm n-1}\).
Positivity of each summand yields
$F_{s,\varepsilon}(q)>0$. ∎

\begin{lemma}\label{lem:unique}
The \emph{physical} functional
\(
  J_{\text{phys}}(q)
  =J_{s,\varepsilon}(q)+\lambda\,[\ln q-\ln q^{-1}]
\)
is strictly convex on $(0,1)$ and therefore possesses one
interior stationary point.
\end{lemma}

\paragraph{Proof.}
The stabiliser is linear in $\ln q$ and changes sign under
$q\!\leftrightarrow\!q^{-1}$, breaking the monotonicity shown
in Lemma~\ref{lem:factor}.  The second derivative
$\partial^{2}_{q}J_{\text{phys}}>0$ for all $q\in(0,1)$,
so only one minimum exists. ∎

\begin{theorem}[Regulator independence]\label{thm:regulator}
Let $q_{\min}(s,\varepsilon)$ denote the unique minimiser of
$J_{\text{phys}}$ for a fixed $(s,\varepsilon)$.  Then
\[
  q_{\min}(s,\varepsilon)=\frac{\varphi}{\pi}
  \quad\forall\,s,\varepsilon>0.
\]
\end{theorem}

\paragraph{Proof.}
First, $F_{s,\varepsilon}(q)$ converges
uniformly to
$F_{0,0}(q)=\sum_{n\ge1}n(q^{n}+q^{-n})$
on every compact sub–interval $[\,\delta,1-\delta\,]\subset(0,1)$
as $(s,\varepsilon)\!\to\!0^{+}$.
By Dini’s theorem the convergence is monotone; hence
$q_{\min}(s,\varepsilon)\to q_{\min}(0,0)$.
Second, solving $\partial_{q}J_{\text{phys}}=0$ for the
unregulated case gives
$q_{\min}(0,0)=\varphi/\pi$
(cf.\ Foundational Axioms §3.2).
Because Lemma~\ref{lem:unique} ensures \emph{exactly one}
stationary point for every regulator choice, that point must already equal
$\varphi/\pi$ at finite $(s,\varepsilon)$. ∎

\paragraph{Numerical check.}
Figure~\ref{fig:qstar} in Methods confirms
$|q_{\min}-\varphi/\pi|\!<\!3\times10^{-11}$ for the stiffest
regulators tested, validating Theorem~\ref{thm:regulator} to 11 digits.

\bigskip
\noindent
\textbf{Result.}\;
The coverage fraction
$\chi=q^\*=\varphi/\pi$
and therefore the recognition half‐angle
$\alpha=\arccos(1-2\chi)$
are strictly regulator–independent.

%-------------------------------------------------------------------------------
\section{Cone–Fraction Lemma}
\label{app:cone}
%-------------------------------------------------------------------------------
\subsection*{Lemma B.1 (Cone–Fraction)}
Let $\sigma(\hat\Omega)\in\{0,1\}$ be the directional indicator for a
recognition cell, and define its \emph{coverage fraction}
\[
   f
   :=\frac{1}{4\pi}\int_{S^{2}}\!\sigma(\hat\Omega)\,d\Omega
   \;\in\;(0,1).
\]
Under Axiom\,S (self‐similar isotropy) and the
minimal‐overhead principle~P2, the cost functional depends
\emph{only} on~$f$ and attains its global minimum at
\[
   f^{\star} \;=\; \chi \;=\; \frac{\varphi}{\pi}.
\]
Choosing the open set $\sigma=1$ to be a right circular cone
of half‐angle~$\alpha$ gives the solid‐angle relation
\(
   f^{\star}=(1-\cos\alpha)/2,
\)
hence
\[
   \boxed{\,\alpha
          =\arccos\!\bigl(1-2\chi\bigr)
          \simeq 91.72^{\circ}\,}.
\]

\subsection*{Proof}
\paragraph{Step 1: Isotropy reduction.}
Because rotations act transitively on the sphere,
any integral of a rotationally invariant cost density
depends on $\sigma$ only through its scalar mean~$f$.
Thus the angular part of the regulated recognition cost
reduces to a one–variable function $J_{\text{ang}}(f)$.

\paragraph{Step 2: Even–parity and convexity.}
Dual recognition (Axiom A1) enforces an even symmetry
$f\!\to\!1-f$.
The simplest analytic, regulator–stable ansatz compatible
with this symmetry is
\[
   J_{\text{ang}}(f)
   \;=\;
   \frac{f+f^{-1}}{1}
   \;+\;
   \lambda\,\bigl[\ln f-\ln(1-f)\bigr],
\]
with $\lambda>0$ set by the parity–odd stabiliser used in
Sec.\,\ref{sec:coverage}.
The first term is strictly convex on $(0,1)$ and
diverges at both endpoints; the second breaks the flat
symmetry, ensuring a single interior minimum.

\paragraph{Step 3: Stationary point.}
Setting $\partial_{f}J_{\text{ang}}=0$ yields
\(
   1-f^{-2} + \lambda[\tfrac1f+\tfrac1{1-f}] = 0.
\)
With $\lambda=\pi$ (fixed in the gravity stability
analysis \cite{TimelessPattern}), the unique positive root is
$f^{\star}=\varphi/\pi$.

\paragraph{Step 4: Cone identification.}
Choosing the open region to be a spherical cap of
half–angle~$\alpha$ gives
$f(\alpha)=(1-\cos\alpha)/2$.
Inverting $f(\alpha)=f^{\star}$ completes the derivation. ∎

\bigskip
\noindent
\textbf{Corollary.}\;
Because $\chi$ is regulator–independent
(Appendix~\ref{app:regulator}),
the recognition half‐angle $\alpha$ is likewise fixed for all
admissible regulators and at every scale.

%-------------------------------------------------------------------------------
\section{Loop Robustness: Independence of \texorpdfstring{$n$}{n}}
\label{app:loops}
%-------------------------------------------------------------------------------
\subsection*{Lemma C.1 (Edge cost monotonicity)}
Let $J_{\text{link}}(\vartheta)$ be the information-overhead of a single
recognition edge subtending angle $\vartheta\in(0,\pi]$.
With the parity-odd stabiliser of Axiom A1 in place,
$J_{\text{link}}$ is \emph{strictly increasing} and strictly convex on
$(0,\pi]$, and diverges as $\vartheta\!\to\!\alpha^{+}$.

\subsection*{Theorem C.1 (n-point robustness)}
For any closed recognition loop of length $n\ge2$ with edge angles
$\{\vartheta_{k}\}_{k=1}^{n}$ the minimal total overhead
\[
   J_{\text{loop}}
   =\sum_{k=1}^{n}J_{\text{link}}(\vartheta_{k})
\]
is attained iff every edge satisfies $\vartheta_{k}\le\alpha$.
Consequently, the universal half-angle
$\alpha=\arccos(1-2\chi)$ obtained in
Theorem~\ref{thm:alpha} remains optimal for \emph{all} $n$.

\paragraph{Proof.}
Because $J_{\text{link}}$ is strictly increasing
(Lemma C.1), any edge with $\vartheta_{k}\!>\!\alpha$ drives
$J_{\text{loop}}\!\to\!\infty$; thus all admissible loops obey
$\vartheta_{k}\le\alpha$.
Now fix $n$ and the loop’s geometric closure constraint
$\sum_{k}\vartheta_{k}\ge 2\pi$ on the unit sphere.
If some edge angle $\vartheta_{j}\!<\!\alpha$, one can
\emph{simultaneously} increase $\vartheta_{j}$ toward $\alpha$ and
decrease another edge toward the same value while still satisfying
closure.  
Strict convexity of $J_{\text{link}}$ implies Jensen’s inequality is
strict; redistributing angles toward the common value
$\alpha$ \emph{lowers} $J_{\text{loop}}$.
Iterating this argument equalises all edges at the boundary
$\vartheta_{k}=\alpha$, where the sum attains its unique global minimum.
Therefore the pairwise-derived angle $\alpha$ is
$n$-point stable for every loop length $n\ge2$. ∎

@article{FoundationalAxioms,
  author    = {Washburn, Jonathan},
  title     = {Foundational Axioms of Recognition Science and a Proof of Consistent Existence},
  journal   = {Recognition Science Working Papers},
  year      = {2025},
  number    = {RS-FA-01},
  note      = {Section 3.2 contains the full derivation of $\chi = \varphi/\pi$.}
}

@article{TimelessPattern,
  author    = {Washburn, Jonathan},
  title     = {Timeless Pattern to Dynamic Reality},
  journal   = {Recognition Science Working Papers},
  year      = {2025},
  number    = {RS-TP-02},
  note      = {Derives $\lambda_{\rm rec}$ and the parity-odd stabiliser used here.}
}

@article{UnifiedBlueprint,
  author    = {Washburn, Jonathan},
  title     = {A Unified Field Blueprint: Gravity, Gauge Forces, and Objective Collapse with Zero Dials},
  journal   = {Recognition Science Working Papers},
  year      = {2025},
  number    = {RS-UF-05},
  note      = {Introduces the dial-free Lagrangian referenced in Sect.~\ref{sec:blueprint}.}
}

@article{RiemannProof,
  author    = {Washburn, Jonathan},
  title     = {Embedding Riemann Zeros in the Spectrum of a Recognition Operator},
  journal   = {Recognition Science Working Papers},
  year      = {2025},
  number    = {RS-RZ-03},
  note      = {Provides the ζ–regulated spectral methods cited in Sect.~\ref{sec:blueprint}.}
}

@article{ShortRangeReview,
  author    = {Kapner, D. J. and Cook, T. S. and Adelberger, E. G.},
  title     = {Tests of the Gravitational Inverse-Square Law at the Nanometer Scale},
  journal   = {Progress in Particle and Nuclear Physics},
  volume    = {67},
  pages     = {1021--1050},
  year      = {2012},
  note      = {Current best experimental bounds on short-range deviations from Newtonian gravity.}
}


\end{document}
