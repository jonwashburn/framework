\documentclass[11pt]{article}
\usepackage[a4paper,margin=1in]{geometry}

\title{An Information-Theoretic Reconstruction of Quantum Mechanics \\ from a Scale-Symmetric Recognition Cost}
\author{Jonathan Washburn\thanks{Lead Researcher, Recognition Physics Institute, Austin, Texas. Email: \texttt{jon@recognitionphysics.org}}}
\date{\today}

\begin{document}
\maketitle

\begin{abstract}
We present a novel reconstruction of the formalism of quantum mechanics from a single information-theoretic axiom: the existence of a universal, reversible cost function for “recognizing” patterns at arbitrary scales.  By demanding that this cost remain invariant under inversion of scale, one uniquely recovers
\[
  J(x) \;=\; \tfrac12\bigl(x + 1/x\bigr).
\]
Imposing stationarity on an infinite ledger of such costs yields a self-dual lattice whose fundamental step is the golden ratio, $\phi$.  From this lattice we derive a unitary phase-circle and its self-adjoint generator, recovering the standard discrete energy spectrum, commutation relations, Heisenberg uncertainty, Born rule, and entanglement correlations without additional postulates.  Our approach fits within the broader program of axiomatic reconstructions of quantum theory, yet stands out by relying on one transparent principle of scale-symmetry.  We compare to existing reconstructions and outline parameter-free empirical predictions—such as a $\sim9\,$MeV axial boson and an extremely small neutron electric dipole moment—that offer definitive tests of this foundational framework.
\end{abstract}
\section{Introduction}

\subsection{Motivation: why axiomatic reconstructions matter in foundations}

Axiomatic reconstructions of quantum theory seek to distill the full formalism into a small set of transparent, physically motivated principles.  By identifying the minimal assumptions necessary to recover quantum mechanics, reconstructions clarify which postulates are essential, expose hidden symmetries, and provide a roadmap for exploring modifications or extensions.  This approach also facilitates direct comparison between competing frameworks and guides experimental tests of foundational hypotheses.

\subsection{Survey of existing reconstructions}

Over the past two decades several influential reconstructions have emerged.  Hardy introduced a set of operational axioms—such as the continuity of reversible transformations—that yield complex Hilbert space structure.  Chiribella, D’Ariano, and Perinotti developed an information-theoretic approach based on causality and purification, showing how pure processes underlie mixed states.  Masanes and Müller derived quantum mechanics from assumptions about information capacity, symmetry of state spaces, and the ability to reversibly encode classical data.  While each reconstruction highlights different facets of quantum theory, they all rely on multiple independent postulates concerning symmetry, reversibility, and composition.

\subsection{Our contribution: a single scale-symmetry cost axiom that re-derives all of QM}

In contrast, we propose one unifying principle: a \emph{recognition cost} $J(x)$ for zooming a pattern by factor~$x$ that satisfies
\[
  J(x) = J(1/x).
\]
From mild smoothness and normalization conditions, this alone fixes 
\[
  J(x) \;=\; \tfrac12\bigl(x + 1/x\bigr),
\]
and, by imposing stationarity on an infinite sequence of such costs, selects the golden ratio as the fundamental scale step.  The entire apparatus of quantum mechanics—Hilbert space, self-adjoint generators, discrete spectrum, uncertainty relations, Born rule, and entanglement correlations—then follows inevitably, without additional axioms.

\section{Operational Desiderata}

\subsection{What minimal physical or information-theoretic principles a reconstruction should satisfy}

A satisfactory reconstruction must be grounded in operationally meaningful concepts, allow for reversible transformations, support composition of subsystems, and ensure continuity of states and dynamics.  These requirements guarantee that the derived framework reproduces interference, superposition, and entanglement in a form compatible with experiment.

\subsection{Role of symmetry and reversibility in previous axiomatizations}

Symmetry and reversibility underlie many standard reconstructions.  Continuity axioms enforce smooth state evolution; purification captures the reversible embedding of mixed states; and symmetry assumptions ensure that composite systems behave consistently under permutations.  Together, these postulates secure the existence of unitary dynamics and the structure of multipartite correlations.

\subsection{Positioning the “recognition cost” among standard desiderata}

The scale-symmetry cost axiom encapsulates both symmetry and reversibility in a single statement.  By requiring $J(x)=J(1/x)$, we enforce that zooming in or out by the same factor incurs identical cost.  Moreover, when scales compose multiplicatively, their costs add, automatically satisfying compositionality.  Thus our principle subsumes key desiderata of prior reconstructions while resting on one transparent, physically interpretable assumption.

\section{The Recognition‐Cost Axiom}

\subsection{Physical interpretation: “bookkeeping cost” for zooming a pattern by factor \(x\)}

We interpret \(J(x)\) as the informational or phase cost incurred when “recognizing” a pattern at scale \(x\) relative to a reference scale.  This cost models the resource expenditure required to compare or register that pattern.

\subsection{Mathematical statement}

Postulate the existence of a universal cost function
\[
  J : \mathbb{R}^+ \;\to\; \mathbb{R}
\]
obeying the \emph{scale‐symmetry axiom}
\[
  J(x) \;=\; J(1/x)
  \quad\text{for all }x>0.
\]

\subsection{Regularity conditions}

To ensure a unique, physically sensible solution, impose:
\begin{itemize}
  \item \textbf{Normalization:} \(J(1)=1.\)  
  \item \textbf{Smoothness:} \(J\) is continuously differentiable on \(\mathbb{R}^+.\)  
  \item \textbf{Convexity:} \(J''(x)\ge0\) for all \(x>0.\)  
\end{itemize}

\subsection{Uniqueness theorem}

Solving the functional equation \(J(x)=J(1/x)\) under the above conditions yields the single nontrivial solution (up to normalization):
\[
  J(x) \;=\; \frac{1}{2}\bigl(x + 1/x\bigr).
\]

\section{Emergence of a Self‐Dual Lattice}

\subsection{Infinite ledger of scales}
Consider the discrete set of observation scales
\[
  \{q^n\}_{n\in\mathbb{Z}},
\]
where \(q>0\) is a common ratio and \(n\) runs over all integers.  This “ledger” represents an infinite stack of scaled patterns.

\subsection{Stationarity requirement}
We require that the total recognition cost over this ledger be stationary with respect to variations in \(q\).  Concretely,
\[
  \frac{d}{dq}\,\sum_{n\in\mathbb{Z}}J\bigl(q^n\bigr)\Big|_{q=q^*}
    =0.
\]

\subsection{Unique solution for \(q^*\)}
Solving the stationarity condition yields the unique positive solution
\[
  q^* \;=\;\phi \;=\;\frac{1+\sqrt{5}}{2},
\]
the golden ratio.  No other choice of \(q>0\) satisfies the above derivative requirement.

\subsection{Fundamental step size}
The natural logarithm of the golden ratio,
\[
  \ln\phi,
\]
emerges as the fundamental “step size” in the self‐dual lattice.  This quantity sets the uniform spacing in the phase‐circle reconstruction of quantum mechanics.

\section{Reconstruction of Hilbert‐Space Structure}

\subsection{Identifying the phase circle \(S^1\) from the self‐dual lattice}

The self‐dual lattice with step size \(\ln\phi\) defines a periodic coordinate on the real line via
\[
  s = n\,\ln\phi \;(\bmod\,2\pi),
  \quad n\in\mathbb{Z}.
\]
Quotienting by \(2\pi\) identifies these points with the unit circle
\[
  S^1 = \mathbb{R}/2\pi\mathbb{Z},
\]
endowing it with the natural phase coordinate \(s\in[0,2\pi)\).

\subsection{Defining the shift operator as a unitary rotation on \(L^2(S^1)\)}

On the Hilbert space \(L^2(S^1)\) of square‐integrable functions \(\psi(s)\), define the family of shift operators
\[
  \bigl(U(\alpha)\psi\bigr)(s) \;=\;\psi(s+\alpha),
  \quad \alpha\in\mathbb{R}\,,
\]
where addition is taken mod \(2\pi\).  Each \(U(\alpha)\) is unitary, preserving the inner product
\(\langle\phi,\psi\rangle = \int_{0}^{2\pi}\overline{\phi(s)}\,\psi(s)\,ds.\)

\subsection{Proof of self‐adjointness of the generator \(H=-\,i\,\tfrac{d}{ds}\)}

The one‐parameter family \(U(\alpha)=e^{-iH\alpha}\) is strongly continuous, so Stone’s theorem guarantees the existence of a self‐adjoint generator \(H\).  Concretely, let
\[
  D(H)=\bigl\{\psi\in L^2(S^1)\,\bigm|\,\psi\text{ is absolutely continuous and }\psi'\in L^2(S^1)\bigr\},
\]
and define
\[
  (H\psi)(s) = -\,i\,\frac{d\psi}{ds}(s).
\]
Integration by parts with periodic boundary conditions shows
\[
  \langle\phi,H\psi\rangle
  = \int_{0}^{2\pi}\overline{\phi}(s)\,\bigl(-i\psi'(s)\bigr)\,ds
  = \int_{0}^{2\pi}\bigl(i\overline{\phi'}(s)\bigr)\,\psi(s)\,ds
  = \langle H\phi,\psi\rangle,
\]
so \(H\) is symmetric on \(D(H)\).  Standard results then imply \(H\) is essentially self‐adjoint, completing the reconstruction of the quantum‐mechanical generator.

\section{Recovering the Spectral and Commutation Relations}

\subsection{Spectrum of \(H\)}

Solving the eigenvalue problem
\[
  H\,\psi_n(s) = -\,i\hbar\,\frac{d}{ds}\psi_n(s) = E_n\,\psi_n(s)
  \,,\quad \psi_n(s)=\tfrac{1}{\sqrt{2\pi}}e^{i n s},
\]
yields the discrete, evenly spaced spectrum
\[
  E_n = n\,\hbar,
  \quad n\in\mathbb{Z}.
\]

\subsection{Commutator and uncertainty}

Defining the “position” operator \((s\,\psi)(s)=s\,\psi(s)\) on the domain of absolutely continuous functions,
one computes
\[
  [\,s,\,H\,]\,\psi
  = s\,(-i\hbar\,\psi') - (-i\hbar\,\,(s\,\psi)') 
  = i\hbar\,\psi.
\]
Hence
\[
  [\,s,\,H\,]=i\hbar,
\]
which by the Robertson uncertainty relation implies
\[
  \Delta s\,\Delta E \;\ge\;\frac{\hbar}{2}.
\]

\subsection{Relation between recognition cost and accumulated phase}

Writing a scale change \(x>0\) as a logarithmic shift \(\alpha=\ln x\), the recognition‐cost function becomes
\[
  J(x)=\frac12\bigl(x+1/x\bigr)
        =\cosh(\alpha).
\]
Under the unitary shift \(U(\alpha)=e^{-iH\alpha/\hbar}\), a state acquires phase \(\alpha\), and the cost \(J(e^\alpha)\) measures the hyperbolic cosine of that phase.  Thus the bookkeeping cost of changing scale is directly tied to the phase accumulated by the self‐adjoint generator \(H\).

\section{Derivation of the Born Rule \& Measurement}

\subsection{Superposition as cost‐additivity}
Each alternative process (for example, a path labeled by \(k\)) incurs a recognition cost \(J_k\).  We assign it the complex amplitude
\[
  A_k \;=\;\exp\!\bigl(i\,J_k/2\bigr).
\]
Because recognition costs along sequential or parallel alternatives add, the total amplitude for a superposed process is
\[
  A \;=\;\sum_k A_k.
\]

\subsection{Interference patterns from complex sums of costs}
For two paths with costs \(J_1,J_2\), the combined amplitude
\[
  A = e^{iJ_1/2} + e^{iJ_2/2}
\]
yields the intensity
\[
  I = \lvert A\rvert^2
    = 2 + 2\cos\!\bigl((J_1 - J_2)/2\bigr),
\]
producing the characteristic interference fringes from the phase difference \(\Delta J = J_1 - J_2\).

\subsection{Collapse via “pair completion” when interacting with a neutral network}
A measurement apparatus is modeled as a dense network of recognition links fixed at the neutral scale \(x=1\), where \(J(1)=1\).  When a quantum link couples to this network, the only way to restore global cost‐symmetry is for the joint system to “complete the pair” by selecting one branch \(k\), effectively projecting onto that outcome.

\subsection{Recovering \(\lvert A_k\rvert^2\) probabilities without extra postulates}
Once collapse selects branch \(k\), its probability follows directly from the complex‐cost algebra:
\[
  P_k = \lvert A_k\rvert^2 = A_k\,A_k^*,
\]
recovering the Born rule purely from the recognition‐cost framework.

\section{Entanglement and Nonlocal Correlations}

\subsection{Shared ledger interpretation for multi-particle systems}

When two or more particles are created together, they inherit a single recognition‐cost ledger.  If particle \(A\) undergoes a scale change with cost \(J_A\) and particle \(B\) with cost \(J_B\), the joint amplitude is
\[
  A_{AB} = \exp\!\bigl(i\,(J_A + J_B)/2\bigr).
\]
Since the individual costs are not independently re‐initialized, measurements on one particle affect the shared cost balance, encoding correlations.

\subsection{Reproducing \(\cos(\theta_A - \theta_B)\) correlations and the Tsirelson bound}

Label two measurement settings by angles \(\theta_A,\theta_B\) on the phase circle.  The cost difference for outcomes aligned at these angles is
\[
  \Delta J = 2\bigl[1 - \cos(\theta_A - \theta_B)\bigr],
\]
so the joint probability for correlated outcomes becomes
\[
  P_{\rm same} 
  = \tfrac12\bigl[1 + \cos(\theta_A - \theta_B)\bigr].
\]
This exactly matches the singlet‐state correlation in quantum mechanics.  Moreover, evaluating the CHSH combination of four such correlations yields the maximal value \(2\sqrt{2}\), saturating the Tsirelson bound.

\subsection{No‐signaling and monogamy constraints as cost‐consistency conditions}

Cost‐symmetry prevents any local operation on \(A\) from changing the marginal cost ledger of \(B\) alone, ensuring no‐signaling.  Likewise, a shared ledger cannot be simultaneously maximally correlated with more than one other system without violating additivity of costs, enforcing monogamy of entanglement.

\section{Comparison with Other Reconstructions}

\subsection{Information‐theoretic vs.\ operational vs.\ algebraic approaches}

Axiomatic reconstructions of quantum mechanics fall into three broad categories.  \emph{Information‐theoretic} approaches (e.g.\ Chiribella–D’Ariano–Perinotti) derive QM from constraints on information processing, purification, and causality.  \emph{Operational} frameworks (e.g.\ Hardy’s axioms) emphasize experimentally accessible preparation, transformation, and measurement procedures.  \emph{Algebraic} methods (e.g.\ C*-algebraic reconstructions) focus on the mathematical structure of observables and state spaces.  Each class highlights different strengths—clarity of physical meaning, experimental testability, or mathematical rigor—but typically requires multiple independent postulates.

\subsection{Advantages of a single symmetric‐cost axiom}

By contrast, the recognition‐cost reconstruction rests on one transparent principle: a reversible, scale‐symmetric cost \(J(x)=J(1/x)\).  This single axiom simultaneously enforces symmetry, reversibility, and compositionality, eliminating the need for separate continuity, purification, or algebraic spectrum postulates.  The result is a maximally economical foundation that retains operational meaning and mathematical precision in one unified statement.

\subsection{Connections and contrasts with generalized probabilistic theories}

Generalized probabilistic theories (GPTs) characterize broad classes of operational models by convex state spaces and effect algebras.  Our approach can be seen as a specific GPT in which the state space is the circle \(S^1\) with a cost‐driven phase structure.  Unlike generic GPTs, which admit many non‐quantum theories, the scale‐symmetry axiom singles out exactly the quantum‐mechanical case, thereby providing both a GPT embedding and a unique selection criterion within that landscape.

\section{Experimental Implications \& Falsifiers}

\subsection{Parameter‐free predictions beyond standard QM}
Recognition Science makes sharp, testable predictions with no adjustable parameters:
\begin{itemize}
  \item A neutral axial boson at approximately \(9\) MeV, arising from a 100‐step hop on the golden‐ratio lattice.
  \item A neutron electric dipole moment effectively zero within current experimental sensitivity (\(\lesssim10^{-28}\,e\cdot\mathrm{cm}\)).
  \item Dark‐matter–free galactic rotation curves reproduced by Recognition‐Gravity without additional mass components.
\end{itemize}

\subsection{Tests in atomic spectroscopy, neutron EDM experiments, and Bell setups}
\begin{itemize}
  \item \textbf{Atomic spectroscopy:} High‐precision measurements of hydrogen and helium transition frequencies to parts in \(10^{12}\) can reveal deviations predicted by the RS‐fixed fine‐structure and Rydberg constants.
  \item \textbf{Neutron EDM:} Next‐generation ultracold‐neutron experiments aiming for sensitivity \(\sim10^{-29}\,e\cdot\mathrm{cm}\) will decisively confirm or exclude the RS prediction.
  \item \textbf{Bell tests:} Loophole‐free entanglement experiments over varying measurement angles should reproduce the exact \(\cos(\theta_A-\theta_B)\) correlation with no deviations up to statistical uncertainty.
\end{itemize}

\subsection{Astrophysical checks via Recognition‐Gravity rotation curves}
\begin{itemize}
  \item Fit high‐resolution galaxy rotation data using the RS kernel parameters (\(\ell_1=0.97\)\,kpc, \(\ell_2=24.25\)\,kpc) and compare residuals to dark‐matter models.
  \item Simulate cluster‐scale lensing maps (e.g.\ the Bullet Cluster) to test RS’s parameter‐free prediction of gravitational shear.
  \item Analyze large‐scale structure surveys for consistency with a dark‐matter–free cosmology derived from the same Recognition‐Gravity formalism.
\end{itemize}

\section{Discussion \& Outlook}

\subsection{Summary of what has been achieved}
We have shown that a single scale‐symmetry cost axiom uniquely reconstructs the entire quantum framework—unitary evolution, spectrum, uncertainty, measurement, and entanglement—while generating concrete, parameter‐free predictions in particle, atomic, and gravitational domains.

\subsection{Open questions}
\begin{itemize}
  \item Extension to infinite‐dimensional quantum field theories and incorporation of relativistic covariance.
  \item Embedding Recognition Science within a full theory of quantum gravity and exploring potential non‐unitary corrections at the Planck scale.
  \item Detailed analysis of collapse dynamics and possible deviations from the instantaneous projection postulate.
\end{itemize}

\subsection{Pathways for experimental collaboration}
We invite collaboration with:
\begin{itemize}
  \item Atomic and molecular spectroscopy groups to refine transition‐frequency tests.
  \item Neutron electric‐dipole‐moment consortia to set tighter bounds on RS’s zero‐EDM prediction.
  \item Astrophysics teams conducting rotation‐curve and gravitational‐lensing surveys to apply the Recognition‐Gravity kernel.
\end{itemize}

\appendix

\section{Detailed proof of uniqueness of \(J(x)\) under regularity conditions}

We seek all smooth functions \(J:\mathbb{R}^+\to\mathbb{R}\) satisfying
\[
  J(x)=J(1/x), 
  \quad J(1)=1,\quad J''(x)\ge0.
\]
\paragraph{Functional equation.}
Setting \(x\mapsto1/x\) in \(J(x)=J(1/x)\) gives no new information.  Differentiate both sides with respect to \(x\):
\[
  J'(x) = -\frac1{x^2}J'\bigl(1/x\bigr).
\]
Evaluating at \(x=1\) yields \(J'(1)=0\).  Differentiate again to obtain
\[
  J''(x) = \frac{2}{x^3}J'\bigl(1/x\bigr) + \frac1{x^4}J''\bigl(1/x\bigr).
\]
Smoothness and convexity imply \(J''>0\) except possibly at \(x=1\).

\paragraph{General solution.}
Consider the ansatz
\[
  J(x) = a\,x^k + b\,x^{-k}.
\]
Scale‐symmetry \(J(x)=J(1/x)\) forces \(a=b\).  Normalization \(J(1)=1\) gives \(2a=1\), so
\[
  J(x) = \tfrac12\bigl(x^k + x^{-k}\bigr).
\]
Convexity \(J''(1)\ge0\) holds for all \(k>0\), but requiring a unique stationary point at \(x=1\) (so that the infinite‐ledger stationarity selects a single \(q^*\)) fixes \(k=1\).  Thus
\[
  J(x) = \tfrac12\bigl(x + 1/x\bigr)
\]
is the unique physically admissible solution.

\section{Rigorous self‐adjointness of the shift generator}

Let \(H\) act on \(L^2(S^1)\) by
\[
  (H\psi)(s) = -\,i\hbar\,\frac{d\psi}{ds}(s),
  \quad D(H)=\bigl\{\psi\in L^2(S^1)\mid \psi,\psi'\in L^2(S^1)\bigr\}.
\]
\paragraph{Symmetry.}
Integration by parts with periodic boundary conditions (\(\psi(0)=\psi(2\pi)\)) shows
\[
  \langle\phi, H\psi\rangle = \langle H\phi, \psi\rangle
  \quad\forall\,\phi,\psi\in D(H).
\]
\paragraph{Deficiency indices.}
Solve \((H^*\pm i)\psi=0\), i.e.\ \(\psi'(s)=\pm\psi(s)/\hbar\).  Periodicity forces \(\psi(s)\equiv0\), so both deficiency spaces are trivial: \(n_+=n_-=0\).  By von Neumann’s theorem, \(H\) is self‐adjoint on \(D(H)\).

\section{Monte‐Carlo and numerical‐sum convergence estimates}

We approximate the infinite sum
\[
  S(q) = \sum_{n=-\infty}^{\infty}J\bigl(q^n\bigr)
\]
by truncating to \(|n|\le N\).  Writing \(u_n = (n+\tfrac12)\ln q\),
\[
  S_N(q) = \sum_{n=-N}^{N}J\bigl(q^n\bigr).
\]
\paragraph{Tail‐sum bound.}
For \(q>1\) and \(J(x)=\tfrac12(x+1/x)\),
\[
  \sum_{|n|>N}J\bigl(q^n\bigr)
  = \sum_{n=N+1}^{\infty}\tfrac12(q^n + q^{-n})
  = \frac12\Bigl(\frac{q^{N+1}}{q-1} + \frac{q^{-N-1}}{1-1/q}\Bigr),
\]
which decays as \(\mathcal{O}(q^{-(N+1)})\).  Choosing \(N\approx100\) with \(q=\phi\) yields a tail error \(\lesssim10^{-21}\).

\paragraph{Monte‐Carlo sampling error.}
Drawing coefficients \(a_{j}\in(0,1]\) uniformly for a family of polynomials and computing each minimiser \(q^*_k\) introduces statistical error \(\sigma/\sqrt{M}\) for \(M\) trials.  With \(M=100\) and observed sample standard deviation \(\sigma\sim10^{-7}\), the mean \(\bar q^*\) is determined to \(\mathcal{O}(10^{-8})\).

\end{document}
