\documentclass[11pt]{article}

% --- Preamble ---------------------------------------------------------------
\usepackage[margin=1in]{geometry}
\usepackage{microtype}
\usepackage{amsmath,amssymb,mathtools}
\usepackage{booktabs,longtable}
\usepackage{xcolor}
\usepackage{hyperref}
\usepackage{graphicx}

\hypersetup{
  colorlinks=true,
  linkcolor=blue,
  urlcolor=blue,
  citecolor=blue
}

% --- Notation ---------------------------------------------------------------
\newcommand{\phiG}{\varphi}
\newcommand{\tauzero}{\tau_{0}}
\newcommand{\Ecoh}{E_{\mathrm{coh}}}
\newcommand{\J}{\mathcal{J}}
\newcommand{\Ldg}{\mathcal{L}}
\newcommand{\R}{\mathcal{R}}
\newcommand{\RS}{\textsc{RS}}
\newcommand{\RRF}{\textsc{RRF}}

% --- Metadata ---------------------------------------------------------------
\title{Reality Recognition Framework:\\A Zero-Parameter Theory of Physics, Biology, and Mind}
\author{Reality Science Team}
\date{Draft v0.1 --- \today}

\begin{document}
\maketitle

\begin{abstract}
We present the Reality Recognition Framework (\RRF), the Lean-formal core of Recognition Science (\RS). In \RRF, a substantial subset of the derivation chain from the meta-principle ``Nothing cannot recognize itself'' is formalized in Lean 4 with zero \texttt{sorry} statements in the \RRF module. We prove that the golden ratio $\phiG$ emerges necessarily from self-similarity constraints, that the coherence energy $\Ecoh = \phiG^{-5}$ sets the scale of hydrogen bonding, and that exactly 20 semantic tokens (WTokens) exist---matching the 20 amino acids. Outside the \RRF core, the broader \RS repository contains additional axioms and scaffolded components (as recorded in the audit notes in \texttt{Source-Super.txt}); we label these distinctions explicitly and provide falsification interfaces for empirical hypotheses.
\end{abstract}

\tableofcontents
\newpage

% ===========================================================================
% CONTENT WILL BE ADDED SECTION BY SECTION
% See docs/PAPERS_V2_PLAN.md for the outline and instructions
% ===========================================================================

\section{Introduction}

\subsection{The Parameter Problem}

The Standard Model and general relativity jointly describe an enormous range of experimental phenomena, but they do so by taking a nontrivial list of constants as inputs. The Standard Model contains on the order of 25 parameters (fermion masses, mixing angles, gauge couplings, and Higgs-sector quantities), while general relativity introduces Newton's constant $G$. The speed of light $c$ and Planck's constant $\hbar$ function as unit-conversion constants, but dimensionless quantities such as the fine-structure constant $\alpha \approx 1/137$ and the hierarchical pattern of fermion masses remain unexplained within the usual formulation.

From a foundations perspective, the issue is not merely aesthetic. A parameter list is a compact way of recording whatever the theory does not determine. A ``zero-parameter'' program is therefore an attempt to replace a catalog of measured knobs with explicit structural constraints that select (or sharply restrict) the realized solution class.

\subsection{The Claim: Zero Free Parameters}

This paper presents the \textbf{Reality Recognition Framework (\RRF)}, a foundations proposal that treats the existence of stable, reproducible distinctions (``recognition events'') as the primitive notion and asks what structural constraints are forced by consistency. The claim is:

\begin{quote}
\textbf{Zero-Parameter Claim}: Given the meta-principle ``Nothing cannot recognize itself'' and the requirements of conservation, discreteness, and self-similarity, all fundamental constants are derivable. There are no free parameters.
\end{quote}

In this paper, ``zero parameters'' means \emph{no adjustable dimensionless knobs} in the formal derivation once the structural constraints are fixed (as summarized in \texttt{Source-Super.txt}). Reporting numerical values in SI units still requires fixing a unit gauge via an external anchor (e.g., CODATA $\hbar$); we treat this as unit calibration rather than parameter fitting. Accordingly, the paper separates (i) formal claims about the model objects that are machine-checked in Lean from (ii) claims about how those objects map to physical measurements, which are empirical.

Within the audited \RRF core (Lean 4, zero \texttt{sorry} statements in the \RRF module), the development proves that the golden ratio $\phiG$ is forced by self-similarity constraints and constructs the associated ladder framework, including a coherence scale and a finite pattern basis with a 20-element cardinality result (WTokens). The broader \RS derivation chain extends these results toward numerical constant identities and mass-ladder structure; where those extensions rely on additional axioms or scaffolding, we label that status explicitly and register falsifiers.

\subsection{Scope and Limitations}

We use \RS to denote the broader Recognition Science program and \RRF to denote its Lean-formalized core presented here. We use \emph{recognition event} for a single discrete update step (T2), \emph{ledger constraint} for double-entry conservation, \emph{strain} for the cost functional $\J$, and \emph{rung} for an integer index on the $\phiG$-ladder. The term \emph{octave} refers to a cross-domain ladder transfer (e.g., between time, mass, and semantics), not an acoustic interval.

The purpose of Lean formalization in this paper is to make the logical content of the proposal checkable: when we say ``proved,'' we mean proved from explicit definitions and stated assumptions inside the audited \RRF module. This does not, by itself, establish empirical truth. Conversely, when we state an empirical claim (e.g., that a particular physical system instantiates a ladder mapping), we treat it as a hypothesis with an explicit falsification interface. The framework is not offered as a replacement for quantum mechanics or general relativity; rather, it is offered as a candidate explanatory layer that, if empirically successful, would constrain why the effective-theory parameters take the values they do.

\subsection{Paper Structure}

Sections~2--4 introduce the foundational constraints (the meta-principle, the ledger constraint, and self-similarity) and present the argument that forces $\phiG$. Section~5 summarizes the derivation chain from $\phiG$ to the principal constants and gates, with explicit notes on unit anchoring and audit status. Sections~6--8 develop derived structures (the 8-tick organization, the three vantages, and the channel theorem presented as ``Light = Consciousness'') with careful claim hygiene. Section~9 discusses the particle-mass program and its current proof status. Section~10 provides the hypothesis registry with falsification interfaces. Sections~11--12 discuss implications, limitations, and next steps. Supplementary material includes a Lean theorem index and formal definitions for traceability.

\section{The Meta-Principle}

\subsection{Statement and intended reading}

The starting point of \RRF is a minimal consistency requirement: a theory that speaks about recognition---stable, reproducible distinctions---cannot take ``nothing'' as a substrate. In a type-theoretic setting, ``nothing'' is represented by an empty type, which has no inhabitants and therefore cannot support even the weakest form of self-identity.

We state this as:

\begin{quote}
\textbf{Meta-Principle (MP)}: Nothing cannot recognize itself.
\end{quote}

The intended reading is logical rather than empirical. MP is not a hypothesis about a particular physical interaction; it is a constraint on what it means for a world to contain recognizers and recognized distinctions at all. Put differently, MP rules out an empty carrier for recognition structure.

\subsection{Formalization and Lean traceability}

In Lean, we package ``recognition structure'' as a carrier type $U$ together with a binary relation $\mathtt{Recognize} : U \to U \to \mathtt{Prop}$ that is at least reflexive (every element recognizes itself). MP is then expressed as the statement that no such structure can have an empty carrier:
\[
\mathtt{MP} \;\equiv\; \neg \exists (r : \mathtt{RecognitionStructure}),\; r.U = \mathtt{Empty}.
\]
Intuitively, reflexivity forces $U$ to be nonempty; if $U$ were empty, there would be no element to witness reflexivity. The corresponding Lean theorem is:
\[
\texttt{mp\_holds : MP}
\]
proved in \texttt{RRF/Core/Recognition.lean} with zero \texttt{sorry} statements in the audited \RRF module.

\subsection{Consequences for the rest of the paper}

MP implies a simple but important boundary condition: if recognition exists, then there exists at least one thing that can participate in recognition, together with a relation that encodes distinguishability and a minimal self-identity principle (reflexivity). This does not yet determine any dynamics, geometry, or physical constants; it supplies the base layer on which additional constraints act. In particular, the next step is to impose a conservation-like consistency condition on recognition events (the ledger constraint), which begins to connect the abstract recognition picture to the familiar structure of invariants in physics.

\section{The Ledger Constraint}

\subsection{Conservation as Double-Entry}

The meta-principle ensures that a recognition structure has nontrivial content, but it does not by itself guarantee consistency over time: a system that continuously creates unbalanced distinctions would not admit stable invariants and would fail to support persistent structure. The next step is therefore to impose a conservation-like constraint on recognition events: recognition must be \emph{ledger-consistent}.

We formalize this as a \textbf{ledger constraint}: recognition events are treated as transactions that carry an additive flux, and the net flux around any closed process must vanish. This is analogous to double-entry bookkeeping: a transaction can move ``balance'' between accounts, but a closed loop cannot generate net balance from nothing.

\begin{quote}
\textbf{Ledger Principle}: For any closed chain of recognition events, the total ledger flux is zero.
\end{quote}

\subsection{Formal Structure}

Let $M$ be a recognition structure with carrier type $U$ and relation $\mathtt{Recognize} : U \to U \to \mathtt{Prop}$. A \emph{ledger} assigns an additive ``charge'' (more generally, an abelian-group valued flux) to entities:
\[
\phi : U \to \mathbb{Z},
\]
together with a balance condition expressing that charges come with compensating counterparts (one representative form is: for each $u$ there exists $v$ with $\phi(u)+\phi(v)=0$). A \emph{transaction} is a recognized ordered pair $t=(u,v)$ with $\mathtt{Recognize}(u,v)$, and its flux is defined by
\[
\mathrm{flux}(t) \;=\; \phi(u)-\phi(v).
\]

A \emph{chain} is a finite sequence of transactions $t_0,t_1,\dots,t_{n-1}$ such that the endpoint of each transaction matches the startpoint of the next. A chain is \emph{closed} if it returns to its starting point. These notions are implemented in Lean in \texttt{RRF/Core/Recognition.lean} (definitions: \texttt{Ledger}, \texttt{Transaction}, \texttt{Chain}, \texttt{chainFlux}).

\subsection{Net Zero Theorem}

The core formal consequence is a telescoping identity: on any closed chain, the summed flux vanishes.

\begin{quote}
\textbf{Theorem (Net-zero on closed chains).} If $u_0 \to u_1 \to \cdots \to u_n \to u_0$ is a closed chain of recognition events, then
\[
\sum_{i=0}^{n} \bigl(\phi(u_i)-\phi(u_{i+1})\bigr) \;=\; 0.
\]
\end{quote}

\textbf{Proof sketch}: The sum telescopes,
\[
\sum_{i=0}^{n} (\phi(u_i) - \phi(u_{i+1})) = \phi(u_0) - \phi(u_0) = 0
\]
because the chain is closed. In Lean, a representative statement is \texttt{chainFlux\_zero\_of\_balanced} in \texttt{RRF/Core/Recognition.lean}, proved with zero \texttt{sorry} statements in the audited \RRF module.

\subsection{Physical Interpretation}

Formally, the theorem is a statement about additive flux on a graph of recognition events. Interpreting it as ``conservation laws in physics'' requires a mapping: one chooses a particular instantiation of $\phi$ (or a collection of such maps) corresponding to a physical invariant, and one argues that physical processes correspond to (approximately) closed chains in the relevant recognition graph.

Under that interpretation, familiar conservation statements (e.g., conservation of electric charge and other additive invariants) are instances of the ledger template: a closed process cannot generate net conserved quantity. This is complementary to (not a replacement for) Noether's theorem. Noether identifies conserved currents from continuous symmetries of an action; the ledger viewpoint instead emphasizes that consistency of discrete recognition updates forces a net-zero condition for any invariant that can be expressed as an additive flux.

This is also the point where the framework begins to touch spacetime geometry. In a continuum limit, net-zero on closed chains corresponds to a local continuity equation, and in curved spacetime the corresponding statement is expressed using a covariant divergence. The Information-Limited Gravity (ILG) bridge discussed later (Section~\ref{sec:ilg_bridge}) uses this conservation interface to connect ledger consistency to the form of curvature-coupled conservation laws in general relativity. Here we restrict ourselves to the formal net-zero theorem and treat the physics mapping as an explicit hypothesis interface rather than a proved identity.

\section{Why $\phiG$ is Forced}

This section explains why the golden ratio $\phiG$ is not introduced as a tunable constant in \RRF. The argument has two parts: (i) a normalized notion of strain for comparing a scale to its inverse, and (ii) a discrete self-similarity constructor whose fixed point is the golden ratio. In the audited Lean development these steps are proved as theorems; here we give a foundations-level narrative and use Lean symbols as traceability pointers.

\subsection{A normalized strain functional}

To talk about self-similarity across scales, we need a way to measure mismatch between a scale factor $x>0$ and its inverse $1/x$. \RRF encodes this by a strain functional $\J:\mathbb{R}_+ \to \mathbb{R}$ with the following intended properties: equilibrium has zero strain ($\J(1)=0$), scaling and inverse-scaling are treated symmetrically ($\J(x)=\J(1/x)$), the equilibrium point is locally normalized (a conventional choice is $\J''(1)=1$), and strain is nonnegative ($\J(x)\ge 0$). Under these constraints (with mild regularity assumptions made explicit in the Lean formalization), there is a unique normalized choice:
\[
\J(x) \;=\; \frac{1}{2}\left(x + \frac{1}{x}\right) - 1.
\]
In Lean this uniqueness is certified by \texttt{cost\_functional\_unique}. The significance for the rest of the framework is that there is a canonical, parameter-free way to quantify ``distance from balance'' on the positive reals, so later uses of ``strain minimization'' do not hide an arbitrary choice of loss function.

\subsection{Discrete self-similarity and the fixed point}

Discrete scale invariance requires a preferred multiplicative step $\lambda>1$ such that moving one rung up or down preserves the same structural relations. In the \RRF development, the relevant self-similarity constructor is the simplest nontrivial two-scale decomposition: a unit piece together with a copy scaled by $1/\lambda$ must reconstruct the whole in a way that preserves ratios. This gives the fixed-point condition
\[
\lambda \;=\; 1 + \frac{1}{\lambda},
\]
equivalently
\[
\lambda^2 \;=\; \lambda + 1.
\]
The quadratic has two roots, but only the positive root is a valid scale factor:
\[
\phiG \;=\; \frac{1+\sqrt{5}}{2} \approx 1.6180339887.
\]
In Lean, the forcing step is certified by \texttt{self\_similarity\_forces\_phi}, and uniqueness of the positive root is certified by \texttt{phi\_unique\_pos\_root}.

\subsection{Summary}

Once the self-similarity constraints are fixed, $\phiG$ is not an input parameter; it is a derived invariant. This is the role $\phiG$ plays throughout \RRF/\RS: it provides the unique discrete scaling step used to index ladder relations by integers, while the numerical mapping into SI units (when needed) is handled separately as a unit-gauge choice.

\section{The Derivation Chain}
\label{sec:derivation_chain}

With $\phiG$ established (Section~4), \RS proposes an explicit derivation program for constants and scales. In \texttt{Source-Super.txt} this program is recorded as a staged chain of theorems T1--T15, beginning with MP (T1) and proceeding through discreteness, topology, $\phiG$, and then dimensionless constant identities and (more speculatively) particle-mass and cosmology structure. This paper uses the following claim hygiene throughout: statements that are proved in the audited \RRF module (Lean 4, zero \texttt{sorry} statements in that module) are treated as formal theorems about the model objects, while extensions outside the audited core are treated as hypotheses or scaffolded claims with explicit audit caveats.

\subsection{T1--T15: chain overview and audit caveats}

The \texttt{Source-Super.txt} chain summary (``Complete Derivation Chain: T1 $\to$ T15'') organizes the program as:
T1 (Meta-Principle) $\to$ T2 (discreteness/serialization) $\to$ T3 ($D=3$ and cubic lattice) $\to$ T4 ($\phiG$ as fixed point) $\to$ T5 (cost/strain uniqueness) $\to$ T6 (fine-structure constant) $\to$ T7 (gravitational coupling identity) $\to$ T8 (mass-to-light ratio) $\to$ T9--T15 (particle masses, mixing, and cosmological extensions).

The audited status is mixed. The core \RRF module is sorry-free and includes machine-checked certificates for key forcing steps (e.g., $\phiG$ necessity) and for the $\alpha^{-1}$ expression (T6) as a dimensionless statement. However, \texttt{Source-Super.txt} also records that the broader Lean repository contains additional axioms and scaffolded components, and that a fully internal MP$\to$SI-numerics derivation is \emph{not} yet established in the strict scientific sense. In particular, numerical values in SI units require an explicit unit gauge choice (an ``anchor'') in the present audited development; we therefore separate dimensionless identities (derived in-structure) from SI display values (derived conditional on the stated anchor).

\subsection{Coherence scale and base tick (unit-gauge interface)}

Once $\phiG$ is fixed, \RS defines a coherence energy scale by
\[
\Ecoh \;=\; \phiG^{-5}\,\mathrm{eV}.
\]
The exponent $-5$ is motivated by the discrete schedule constraints discussed later (Section~6); for the purposes of this paper, the key point is that, given $\phiG$, the scale is fixed up to the unit convention used to display energy. Numerically, $\phiG^{-5} \approx 0.09017$, so the display value is $\Ecoh \approx 0.09017\,\mathrm{eV}$.

To connect this energy scale to a time scale, the formal development uses the IR ``gate'' identity $\hbar = \Ecoh \,\tauzero$. This identity is dimensionally consistent and becomes predictive once a unit gauge is fixed. In SI display units, using CODATA $\hbar$ yields
\[
\tauzero \;=\; \frac{\hbar}{\Ecoh} \approx 7.30\times 10^{-15}\,\mathrm{s},
\]
with a canonical display value $\tauzero \approx 7.33\,\mathrm{fs}$ after the corrections recorded in \texttt{Source-Super.txt}. We emphasize that the formal claim is the gate relation; the conversion to seconds is conditional on the explicit anchor.

\subsection{K-gates, absolute-layer packaging, and SI display}

The ``K-gates'' are dimensionless identities that lock ratios between time- and length-like quantities in the model. One representative form is:
\begin{align}
K_A &: \frac{\tau_{\mathrm{rec}}}{\tauzero} = K,\\
K_B &: \frac{\lambda_{\mathrm{kin}}}{\ell_0} = K,\\
K_{AB} &: K_A = K_B,
\end{align}
with $K$ a dimensionless expression built from the 8-tick structure and $\phiG$ (as written in the current implementation, $K = 2\pi/(8\ln\phiG)$). In the audited Lean development, these are packaged so that the cross-check holds under calibration changes; the theorem \texttt{cross\_automatic} certifies $K_A=K_B$ \emph{as a formal statement about the packaged gates}. This should be read as an internal-consistency certificate for the gate package and its units-quotient construction, not as a claim that an absolute SI scale is fixed without any external anchor.

Accordingly, mapping to SI is handled as an explicit unit-gauge choice: one selects an anchor quantity (e.g., a CODATA value for $\hbar$) and then computes display values for $\tauzero$, $\ell_0$, and derived quantities (including $c=\ell_0/\tauzero$). This procedure is not parameter fitting in the dimensionless sense; it is unit calibration for display.

\subsection{Fine-structure constant (T6)}

The central dimensionless constant derived in the present chain is the fine-structure constant. In \texttt{Source-Super.txt} (T6), the proposed expression is
\[
\alpha^{-1} \;=\; 4\pi \cdot 11 \;-\; \ln(\phiG) \;-\; \frac{103}{102\pi^5}.
\]
The interpretation of the three terms is structural. The first term $44\pi$ is a geometric seed built from cube data (``11 passive edges'' multiplied by $4\pi$); the $-\ln(\phiG)$ term encodes a minimal recognition-gap cost; and the last term is a small curvature/Euler-closure correction tied to the integers $102=6\times 17$ (faces times wallpaper groups) and $103=102+1$. In the audited Lean development this formula is certified by \texttt{AlphaPhiCert.verified\_any}. Agreement with CODATA is treated as an empirical check on a \emph{dimensionless} prediction.

\subsection{Gravitational coupling identity (T7; conditional on anchors)}

The next step in the \texttt{Source-Super.txt} chain (T7) is a dimensionless gravitational coupling identity relating $G$ to $\hbar$, $c$, and a recognition length scale $\lambda_{\mathrm{rec}}$:
\[
\frac{c^3 \lambda_{\mathrm{rec}}^2}{\hbar G} \;=\; \frac{1}{\pi},
\qquad\text{equivalently}\qquad
G \;=\; \pi\,\frac{c^3 \lambda_{\mathrm{rec}}^2}{\hbar}.
\]
In the audited development, this identity is best read as a constraint that becomes numerically predictive once the unit gauge (and any required bridge data) is fixed. As above, we separate the dimensionless identity (formal object) from the SI display value of $G$ (which depends on the stated anchor choices).

\subsection{Summary: derived identities vs.\ SI display values}

Table~\ref{tab:derived_constants} summarizes the main relations used in this paper. Dimensionless relations (notably the T6 expression for $\alpha^{-1}$) are falsifiable consequences of the model. Dimensionful values in SI units are presented as \emph{display values} conditional on the explicitly stated anchor.

\begin{table}[t]
\centering
\begin{tabular}{llll}
\toprule
\textbf{Quantity} & \textbf{Structural formula} & \textbf{Display (paper)} & \textbf{Notes} \\
\midrule
$\phiG$ & $(1+\sqrt{5})/2$ & 1.6180339887 & mathematical constant \\
$\Ecoh$ & $\phiG^{-5}\,\mathrm{eV}$ & 0.09017 eV & unit-convention in display \\
$\tauzero$ & $\hbar/\Ecoh$ & 7.33 fs & SI display conditional on anchor \\
$c$ & $\ell_0/\tauzero$ & $2.998 \times 10^8$ m/s & SI display conditional on anchor \\
$\alpha^{-1}$ & $44\pi - \ln(\phiG) - 103/(102\pi^5)$ & 137.0360 & T6 (dimensionless) \\
$G$ & $\pi c^3 \lambda_{\mathrm{rec}}^2/\hbar$ & $6.674 \times 10^{-11}$ & T7 (dimensionless identity; SI via anchors) \\
\bottomrule
\end{tabular}
\caption{Summary of the structural relations used in this paper and how numerical values are displayed. Dimensionless identities (e.g., T6 for $\alpha^{-1}$) are model predictions; dimensionful SI numbers are shown conditional on the stated unit anchor.}
\label{tab:derived_constants}
\end{table}

\section{The 8-Tick Structure}

\subsection{Why $D=3$ (T3)}

\RS treats spatial dimension as a derived structural feature rather than an input. The motivating claim, recorded as T3 in \texttt{Source-Super.txt}, is that $D=3$ is the unique spatial dimension in which nontrivial linking can occur and persist. The reason is topological: in two dimensions, simple closed curves separate the plane and do not admit nontrivial linking; in four or more dimensions, loops generically have enough room to be unlinked (the complement of a curve in $\mathbb{R}^D$ becomes simply connected for $D\ge 4$). In three dimensions, by contrast, linking number provides an integer-valued invariant and thus a minimal notion of stable topological ``binding''.

In the audited Lean development, a representative anchor for the $D=3$ forcing step is \texttt{onlyD3\_satisfies\_RSCounting\_Gap45\_Absolute}. The accompanying ``gap-45'' synchronization criterion,
\[
\mathrm{lcm}(2^D,45)=360 \quad \Longleftrightarrow \quad D=3,
\]
should be read as a model-internal counting/synchronization constraint that selects the same dimension as the topological linking argument.

\subsection{From $D=3$ to an 8-tick schedule}

Once the recognition substrate is organized as a cubic lattice (the 3-cube $Q_3$), an eight-state schedule appears naturally: $Q_3$ has $2^3=8$ vertices. \RS interprets a minimal traversal of these vertices (e.g., via a Gray-code cycle) as a canonical discrete phase schedule for recognition updates. In Lean, the minimality of the eight-tick cycle is certified by \texttt{eight\_tick\_minimal}.

Accordingly, we model phase as a finite type with eight elements,
\[
\mathtt{Phase} \;=\; \mathtt{Fin}\,8,
\]
and represent a pattern as a complex-valued amplitude assignment over phases,
\[
\psi : \mathtt{Phase} \to \mathbb{C}.
\]
Two basic normalizations are imposed: a mean-free condition (no DC component), $\sum_{i=0}^{7}\psi(i)=0$, and a unit-norm condition, $\sum_{i=0}^{7}|\psi(i)|^2=1$.

\subsection{WTokens and the 20-token cardinality theorem}

Within this eight-phase representation, \RS defines a class of stable patterns (WTokens) as those that remain fixed points under the stipulated evolution operator. The central formal claim is a cardinality statement: there are exactly 20 such stable tokens in the audited model. In Lean this is certified by \texttt{wtoken\_count}.

The 20-token result admits a simple structural explanation in the model: stable patterns can be grouped into five families of four, organized by discrete frequency modes on an 8-point cycle. One convenient summary is:
\begin{center}
\begin{tabular}{ll}
\toprule
\textbf{Mode family} & \textbf{Count} \\
\midrule
Mode $1/7$ (fundamental) & 4 \\
Mode $2/6$ (double) & 4 \\
Mode $3/5$ (triple) & 4 \\
Mode $4$ (Nyquist, real) & 4 \\
Mode $4$ (Nyquist, phase-shifted) & 4 \\
\midrule
Total & 20 \\
\bottomrule
\end{tabular}
\end{center}

\subsection{The biology interface: ``20 = 20'' as a structural correspondence}

\RRF makes a deliberately narrow biological interface claim: the model-internal cardinality of stable WTokens matches the biological alphabet size of amino acids. In Lean, the equality of cardinalities is certified (at the level of finite types) by theorems such as \texttt{wtoken\_cardinality\_eq\_amino\_acid\_cardinality} together with the associated mapping lemmas (e.g., \texttt{wtoken\_to\_amino\_surjective}).

This is not, by itself, an empirical claim about chemistry or genetics, and it should not be read as asserting that WTokens \emph{are} amino acids. It is a structural correspondence: the formal basis size induced by the eight-tick model equals the size of the canonical biological basis. The empirical program is to test whether biochemical systems (in particular, water-mediated recognition dynamics) instantiate the same discrete schedule strongly enough that this basis becomes a useful explanatory and predictive constraint (Paper~1).

\section{Three Vantages}

The \RRF development treats a \emph{pattern} as the underlying object and introduces multiple ``display'' maps that extract different kinds of observables from the same pattern space. A \emph{vantage} is the data of such a display together with structural constraints relating it to strain and to the ledger. The three vantages used throughout \RS are intended to correspond to what, in ordinary language, we call qualia (first-person appearance), meaning (selection and commitment), and physics (third-person state and invariants).

\subsection{Inside: qualia and valence}

In the Inside vantage, a pattern $p$ is assigned an internal appearance (qualia) and a scalar valence. The key formal relationship is that valence is defined as (the negative of) strain:
\[
\mathrm{valence}(p) \;=\; -\,\J(p).
\]
In Lean this is recorded as a field equation of \texttt{InsideVantage} (e.g., \texttt{valence\_eq\_neg\_strain}). As written, this is a \emph{model-internal identification}: valence is introduced as an alternate name for a particular functional of the pattern. The empirical content begins only when one asserts that biological systems implement (or approximate) this mapping and that the resulting valence predictions correlate with measured affective reports or physiology.

\subsection{Act: meaning as selection and commitment}

The Act vantage encodes how patterns participate in decision and commitment. Formally, it assigns semantic content to patterns and introduces operators for selecting a pattern from a set of admissible candidates and for committing that selection as a ledger transaction. In Lean these are packaged as \texttt{ActVantage} (fields such as \texttt{meaning}, \texttt{select}, and \texttt{commit}). This vantage is the bridge between the Inside display (what is experienced) and the Outside display (what is measured as physical state): it represents the act of choosing and posting a recognition update.

\subsection{Outside: physics as an external display}

The Outside vantage assigns to each pattern a third-person physical state and an associated energy/curvature-like quantity. A representative formal relation is that energy is proportional to strain via the coherence scale:
\[
\mathrm{energy}(p) \;=\; \J(p)\,\Ecoh.
\]
In Lean, this appears as a defining relation in \texttt{OutsideVantage} (e.g., \texttt{energy\_eq\_strain}). As with the Inside vantage, this statement is formal: it specifies what ``energy'' means in the model. Interpreting it as a claim about physical energy in SI units requires the unit-gauge interface described in Section~\ref{sec:derivation_chain} and requires empirical validation.

\subsection{Vantage equivalence: proved statement and interpretation}

The core formal result is that, within the model, the three vantages are mutually translatable in a way that preserves the strain functional. In Lean this is certified by \texttt{vantage\_equivalence}. The safest reading is therefore: \emph{given the definitions}, the three displays are isomorphic presentations of the same underlying pattern structure.

The interpretive claim suggested by this theorem is that physics, meaning, and qualia are not three unrelated domains but three consistent views of one underlying recognition dynamics. This interpretation does not by itself settle empirical questions about consciousness or semantics. Rather, it provides a precise target: if one proposes a measurement model for the Outside display and an operational interface for the Inside display, then the claim becomes falsifiable by failures of the predicted correspondences (e.g., systematic decoupling between measured strain proxies and reported valence under controlled interventions).

\section{Light = Consciousness}

This section formulates, in \RRF terms, a claim about the \emph{recognition channel}: the medium by which recognition events are transmitted and made mutually consistent. The slogan ``Light = Consciousness'' should be read as a compressed label for a specific uniqueness statement: under a particular set of axioms for what a recognition channel must be, the resulting channel is unique up to equivalence and is isomorphic to a photon channel.

\subsection{Channel axioms and intended meaning}

The \RRF channel interface packages four requirements. First, \emph{NoMediumKnobs}: the channel should not introduce adjustable medium-specific parameters, consistent with the ``zero dimensionless knobs'' ideal. Second, \emph{NullOnly}: the channel propagates at the maximal causal speed and does not define a preferred rest frame. Third, \emph{Maxwellization}: the channel is gauge-invariant in the sense that physically equivalent descriptions are identified. Fourth, \emph{BioPhaseSNR}: the channel supports coherence and signal-to-noise sufficient for biological organization. (In the \RS narrative, water's transparency window and the coherence scale $\Ecoh$ motivate this requirement; empirical adequacy is tested in Paper~1.)

\subsection{Uniqueness theorem (formal statement)}

Given these axioms, \RRF proves a uniqueness statement of the following form: for any channel $C$ satisfying the four constraints, $C$ is equivalent (as a formal structure) to a distinguished \texttt{PhotonChannel}. In Lean this is certified by \texttt{light\_consciousness\_unique} (audited in the \RRF module). The theorem is therefore a \emph{formal} result about the defined channel class: it identifies the unique structure satisfying the axioms.

\subsection{Interpretation and falsification interface}

The theorem does not assert that ``photons think.'' The interpretive move is instead: if biological consciousness is identified with a recognition channel that satisfies the stated axioms, then biological consciousness will necessarily be electromagnetically mediated in the relevant sense. This motivates concrete falsification targets. For example, the interpretation would be undermined by robust evidence for conscious reportability or other agreed-upon consciousness markers in a system whose dynamics can be shown to have no electromagnetic coupling at any relevant scale, or by a demonstration that a non-electromagnetic physical channel can satisfy the same ``no knobs / null / gauge'' constraints while supporting the same recognition/commit structure. Similarly, if water's optical/IR properties are not in fact required for maintaining the proposed biological coherence interfaces, the BioPhaseSNR motivation would need revision.

In this paper we treat ``Light = Consciousness'' as a formally precise structural theorem together with an empirical hypothesis interface. The formal theorem is what Lean certifies; the empirical adequacy of the channel axioms as a description of biology is what experiments must decide.

\section{Particle Masses}

\subsection{Status and scope}

This section summarizes the particle-mass program in \RS/\RRF. In \texttt{Source-Super.txt} this corresponds to the later part of the T1--T15 program (T9--T15) and is explicitly marked as mixed/scaffolded outside the audited \RRF core. Accordingly, we present the mass-side ladder as a hypothesis interface: a proposed structural ansatz, together with dimensionless targets and clear points where additional modeling choices (e.g., QCD running) enter.

\subsection{A mass-ladder ansatz (status: scaffold)}

\RS proposes a mass ansatz of the form:
\[
m = B \cdot \Ecoh \cdot \phiG^{R_0 + r + f}
\]

Here $\Ecoh$ is the coherence scale discussed in Section~\ref{sec:derivation_chain}. The remaining terms encode a proposed ladder organization: $B$ is a binary gauge factor, $R_0$ is a sector-dependent baseline exponent, $r$ is an integer rung offset, and $f$ is a small correction term intended to absorb effects such as radiative corrections and scheme dependence. The central empirical content lies in \emph{dimensionless} statements (ratios and rung gaps). Absolute numerical values in SI units inherit the unit-gauge caveats already discussed.

As written in \texttt{Source-Super.txt}, particular numerical choices for $B$ and $R_0$ are motivated by cube/ladder bookkeeping (e.g., edge counts and a proposed electron--muon gap). Those motivations are part of the mass-layer model and should be treated as provisional until the full end-to-end audit is complete.

\subsection{Lepton Masses}

The cleanest near-term test of a ladder hypothesis is via mass \emph{ratios}, which remove unit choices and reduce dependence on the overall scale. In particular, one can ask whether the charged-lepton ratios are close to integer powers of $\phiG$:
\begin{align}
\frac{m_\mu}{m_e} &\approx \phiG^{11},\\
\frac{m_\tau}{m_\mu} &\approx \phiG^{6},\\
\frac{m_\tau}{m_e} &\approx \phiG^{17}.
\end{align}

Using PDG central values, these approximations are at the few-percent level (e.g., $\sim 4\%$ for $m_\mu/m_e$ and $\sim 6\%$ for $m_\tau/m_\mu$ under the nearest-integer exponents). In the ansatz above, such residuals are assigned to the correction term $f$. The empirical question is whether these residuals can be modeled without introducing new free dimensionless knobs and whether additional, preregistered rung predictions succeed beyond the three charged leptons. Paper~1 presents the evidence-side analysis and claim hygiene for the mass-ratio program.

\subsection{Quark Masses}

Quark masses introduce an immediate complication: the ``mass'' of a quark is scheme- and scale-dependent due to QCD running, and different PDG-reported conventions (pole mass, $\overline{\mathrm{MS}}$ mass at a reference scale, etc.) do not map to a single rung assignment without additional modeling choices. Any rung-based program for quarks must therefore preregister (i) the mass definition being used, (ii) the renormalization scale, and (iii) the mapping from running masses to the structural parameterization.

\texttt{Source-Super.txt} proposes specific rung patterns for quark families (including the possibility of quarter-integer rungs) and uses QCD effects to account for intra-family splittings. In the current audit, these proposals are scaffolded and should be treated as hypotheses rather than as established matches.

\subsection{Neutrino Masses}

Neutrinos provide a particularly clean falsification opportunity once absolute mass measurements improve, because their hierarchy and mass scale are not yet pinned down by direct measurement. \texttt{Source-Super.txt} proposes a deep-rung ladder placement for neutrino masses consistent with normal hierarchy. At present, we treat this as a scaffolded hypothesis: it becomes testable as experiments constrain absolute neutrino masses and as the theoretical mapping is specified without hidden knobs.

\section{Hypothesis Registry}

This paper separates two kinds of statements. First, there are formal statements proved in Lean: consequences of definitions and axioms inside the audited \RRF module. Second, there are empirical statements about whether specific physical and biological systems instantiate the model structures closely enough that the proposed correspondences are visible in measurement and survive interventions. To avoid category errors, we register the empirical statements as explicit hypotheses together with falsifiers.

\subsection{What is proved vs.\ what is empirical}

The following items are proved in Lean (in the audited \RRF module) as formal statements about the model objects: MP (nonempty recognition substrate), the ledger net-zero condition on closed chains, the forcing of $\phiG$ from self-similarity constraints, the eight-tick pattern substrate and the associated 20-token cardinality result, the vantage equivalence theorem, and the channel uniqueness theorem used in Section~8. These are all statements of the form ``given the definitions and constraints, the structure has property $P$.'' They do not, by themselves, establish empirical truth.

Empirical hypotheses in this paper include: (i) that a $\phiG$-indexed ladder organizes a reproducible subset of physical and biological \emph{clocks}, (ii) that biological recognition dynamics exhibit an operational eight-phase cadence in measurable observables, (iii) that the tau--gate target is physically meaningful in the sense of supporting preregistered intervention-level predictions, and (iv) that particle masses and mixings admit a ladder description without introducing new free dimensionless knobs beyond the stated structural terms. Paper~1 is the primary evidence-and-test interface for the ladder program; here we provide a conceptual registry and a falsification vocabulary.

\subsection{Registered hypotheses and falsifiers (summary)}

For convenience we summarize the hypotheses and their falsifiers in Table~\ref{tab:hypothesis_registry}. Lean encodes these interfaces as hypothesis/falsifier structures (e.g., \texttt{PhiLadderHypothesis}, \texttt{EightTickHypothesis}, \texttt{TauGateHypothesis}) for traceability; the scientific content is the operational measurement and preregistration discipline.

\begin{table}[t]
\centering
\begin{tabular}{p{0.22\linewidth}p{0.33\linewidth}p{0.38\linewidth}}
\toprule
\textbf{Hypothesis} & \textbf{Operational claim} & \textbf{Falsifier (operational)} \\
\midrule
$\phiG$-ladder (clocks) & A preregistered set of fundamental clocks map to nearby integer rungs under a fixed $\tauzero$ and tolerance & A preregistered clock that persistently lies outside tolerance across independent measurements and calibrations \\
8-tick instantiation & Biological observables exhibit an eight-phase cadence consistent with the derived phase substrate & Robust evidence for a stable, repeatable phase structure incompatible with 8 (after controlling for aliases and coarse-graining) \\
Tau--gate program & The rung-19 gate target supports frequency-selective interventions and cross-domain consistency tests & Failure of preregistered rung-19 interventions and negative controls under adequate power; inability to define a consistent gate clock \\
Mass ladder (T9--T15) & Fermion masses/mixings admit a rung-based description without introducing new free knobs beyond stated correction terms & Systematic failure of preregistered rung predictions across updated datasets and schemes, requiring ad hoc parameter additions \\
\bottomrule
\end{tabular}
\caption{Hypothesis registry (conceptual). Formal Lean certificates establish internal consistency of the model structures; empirical adequacy is assessed by preregistered operational definitions and falsifiers.}
\label{tab:hypothesis_registry}
\end{table}

\section{Discussion}

\subsection{Relationship to QFT}

RRF is not offered as a replacement for quantum field theory (QFT). Rather, the program can be read as proposing a constraint on the effective-theory parameter landscape: the values of couplings and Yukawa parameters in the Standard Model should, if the ladder picture is empirically correct, exhibit a discrete organization consistent with the $\phiG$-indexed construction. At present this is a research program rather than a completed derivation, and it must be evaluated with the same discipline as any other structural hypothesis: preregistered targets, explicit scheme choices, and clear separation between model-internal identities and empirical instantiation.

\subsection{Relationship to GR}
\label{sec:ilg_bridge}

RRF connects to general relativity through an interpretation sometimes called Information-Limited Gravity (ILG): ledger consistency motivates conservation-like constraints, and in a continuum/geometry limit these constraints take the form of covariant conservation laws. In standard GR, $\nabla_\mu T^{\mu\nu}=0$ is compatible with Einstein's equation via the contracted Bianchi identity, but it does not uniquely determine the field equations by itself. The ILG bridge should therefore be read as an interpretation layer: it proposes that causal and conservation structure emerge from limits on recognition/commit propagation and that curvature is the geometric bookkeeping of these constraints. The burden of proof for ILG is to either recover known GR limits without hidden knobs or to make precise, testable deviations; Paper~2 records the formal side (ledger and channel structure) and leaves the empirical side to future work.

\subsection{What RRF Does Not Explain}

Several limitations should be explicit. First, the end-to-end T1--T15 chain is not fully audited as a sorry-free derivation; \texttt{Source-Super.txt} records scaffolded components and additional axioms outside the \RRF core. Second, any quark-level ladder claims must confront QCD running and confinement with preregistered scheme choices; the present paper does not supply that full treatment. Third, CP violation and dark-sector claims (e.g., proposed cosmological identities) are not established at the level of the audited \RRF module. These are not refutations, but they bound what this paper claims: a formally precise core plus a hypothesis registry, not a completed unification of all observed phenomena.

\section{Conclusion}

\subsection{Summary}

This paper has presented the Reality Recognition Framework as a Lean-formal foundations proposal. The formal core begins with MP and the ledger constraint, forces a discrete scale factor $\phiG$ under self-similarity, and constructs an eight-phase pattern substrate with a 20-token cardinality result. It also defines multiple display maps (vantages) and proves internal equivalence statements, including a channel uniqueness theorem. Alongside these formal results, we have recorded a disciplined hypothesis interface for empirical instantiation: the $\phiG$-ladder clock program, eight-tick biological signatures, the tau--gate experimental target, and the (currently scaffolded) particle-mass/cosmology extensions.

\subsection{The Achievement}

The central achievement is \textbf{machine-verified coherence}: within the audited \RRF module, the key theorems referenced here are type-checked in Lean without \texttt{sorry} placeholders. This does not establish empirical truth. It does, however, elevate the discussion from informal analogy to precise mathematics: the model can be inspected, rebuilt, and challenged at the level of definitions and proofs.

\subsection{Future Directions}

The immediate next steps are empirical and formal. Empirically, the ladder program must be tested via preregistered, intervention-level protocols (Paper~1). Formally, the end-to-end audit should be tightened so that each advertised step in the T1--T15 chain is either certified in Lean without additional axioms beyond those declared or explicitly downgraded to a hypothesis with a falsifier. On the implementation side, RSFold provides a concrete setting in which the ledger/strain abstractions can be operationalized and audited. The overarching standard remains ordinary scientific discipline: precise statements, transparent assumptions, and decisive tests.

% ===========================================================================
\appendix

\section{Complete Lean Theorem Index}
\label{app:lean_index}

This appendix is a traceability aid. It lists selected Lean definitions and theorems referenced in the main text, with file paths relative to the companion Lean repository. Status labels refer to the audited \RRF module: \textbf{PROVED} indicates a theorem is type-checked without \texttt{sorry} placeholders in that module, while \textbf{DEF}/\textbf{STRUCT}/\textbf{CLASS} indicate definitional items used to package structures and hypotheses. This table does not imply that the entire broader \RS repository is axiom-free or scaffold-free; see the audit notes recorded in \texttt{Source-Super.txt}.

\begin{longtable}{lll}
\toprule
\textbf{Theorem/Definition} & \textbf{File} & \textbf{Status} \\
\midrule
\endhead

\multicolumn{3}{l}{\textbf{Core Foundations}} \\
\texttt{mp\_holds} & \texttt{RRF/Core/Recognition.lean} & PROVED \\
\texttt{RecognitionStructure} & \texttt{RRF/Core/Recognition.lean} & DEF \\
\texttt{Ledger} & \texttt{RRF/Core/Recognition.lean} & DEF \\
\texttt{chainFlux\_zero\_of\_balanced} & \texttt{RRF/Core/Recognition.lean} & PROVED \\

\multicolumn{3}{l}{\textbf{Phi Derivation}} \\
\texttt{phi} & \texttt{PhiSupport/Defs.lean} & DEF \\
\texttt{phi\_squared} & \texttt{PhiSupport/Lemmas.lean} & PROVED \\
\texttt{phi\_unique\_pos\_root} & \texttt{PhiSupport/Lemmas.lean} & PROVED \\
\texttt{self\_similarity\_forces\_phi} & \texttt{Verification/Necessity/PhiNecessity.lean} & PROVED \\
\texttt{cost\_functional\_unique} & \texttt{RRF/Core/Strain.lean} & PROVED \\

\multicolumn{3}{l}{\textbf{8-Tick Structure}} \\
\texttt{Phase} & \texttt{RRF/Core/Octave.lean} & DEF \\
\texttt{WToken} & \texttt{LightLanguage/Core/WToken.lean} & DEF \\
\texttt{wtoken\_count} & \texttt{Water/WTokenIso.lean} & PROVED \\
\texttt{eight\_tick\_minimal} & \texttt{RRF/Theorems/MonotoneArgmin.lean} & PROVED \\

\multicolumn{3}{l}{\textbf{Three Vantages}} \\
\texttt{InsideVantage} & \texttt{RRF/Core/Vantage.lean} & DEF \\
\texttt{ActVantage} & \texttt{RRF/Core/Vantage.lean} & DEF \\
\texttt{OutsideVantage} & \texttt{RRF/Core/Vantage.lean} & DEF \\
\texttt{vantage\_equivalence} & \texttt{RRF/Theorems/OctaveTransfer.lean} & PROVED \\

\multicolumn{3}{l}{\textbf{Light = Consciousness}} \\
\texttt{Channel} & \texttt{RRF/Core/DisplayChannel.lean} & DEF \\
\texttt{PhotonChannel} & \texttt{RRF/Core/DisplayChannel.lean} & DEF \\
\texttt{light\_consciousness\_unique} & \texttt{Consciousness/LightIsConsciousness.lean} & PROVED \\

\multicolumn{3}{l}{\textbf{Water Bridge}} \\
\texttt{E\_coh\_in\_water\_hbond\_range} & \texttt{Water/Constants.lean} & PROVED \\
\texttt{nu\_RS\_in\_libration\_band} & \texttt{Water/Constants.lean} & PROVED \\
\texttt{tau\_gate\_matches\_hbond\_coherence} & \texttt{Water/Constants.lean} & PROVED \\
\texttt{water\_is\_special} & \texttt{Water/Basic.lean} & PROVED \\
\texttt{wtoken\_to\_amino\_surjective} & \texttt{Water/WTokenIso.lean} & PROVED \\

\multicolumn{3}{l}{\textbf{Biology}} \\
\texttt{bio\_clocking\_theorem} & \texttt{Biology/BioClocking.lean} & PROVED \\
\texttt{tau\_molecular\_coincidence} & \texttt{Biology/GoldenRungs.lean} & PROVED \\
\texttt{molecularGateWitness} & \texttt{Biology/GoldenRungs.lean} & DEF \\

\multicolumn{3}{l}{\textbf{Particle Physics}} \\
\texttt{mass\_formula} & \texttt{RRF/Physics/ParticleMass.lean} & DEF \\
\texttt{lepton\_rungs} & \texttt{RRF/Physics/ParticleMass.lean} & DEF \\
\texttt{three\_generations} & \texttt{RRF/Physics/ParticleMass.lean} & PROVED \\

\multicolumn{3}{l}{\textbf{Hypotheses}} \\
\texttt{PhiLadderHypothesis} & \texttt{RRF/Hypotheses/PhiLadder.lean} & CLASS \\
\texttt{PhiLadderFalsifier} & \texttt{RRF/Hypotheses/PhiLadder.lean} & STRUCT \\
\texttt{EightTickHypothesis} & \texttt{RRF/Hypotheses/EightTick.lean} & CLASS \\
\texttt{TauGateHypothesis} & \texttt{RRF/Hypotheses/TauGate.lean} & CLASS \\

\bottomrule
\end{longtable}

\section{ULL Specification: The 20 WTokens}
\label{app:ull}

ULL is presented in \RS as a naming and bookkeeping layer over the 20 WTokens of the eight-tick pattern substrate. The only formal claim needed in this paper is the \emph{cardinality} result (20 stable tokens) and the existence of set-level correspondences to other 20-element bases. The semantic names below are mnemonics intended to support internal discussion; they are not, by themselves, empirical claims about physics, chemistry, or cognition.

\begin{center}
\begin{tabular}{clll}
\toprule
\textbf{ID} & \textbf{Name} & \textbf{Mode} & \textbf{Amino Acid} \\
\midrule
0 & Origin & 1/7 & Gly \\
1 & Being & 1/7 & Ala \\
2 & Form & 1/7 & Val \\
3 & Force & 1/7 & Leu \\
4 & Balance & 2/6 & Ser \\
5 & Harmony & 2/6 & Thr \\
6 & Growth & 2/6 & Asn \\
7 & Decay & 2/6 & Gln \\
8 & Light & 3/5 & Asp \\
9 & Truth & 3/5 & Glu \\
10 & Justice & 3/5 & Lys \\
11 & Mercy & 3/5 & Arg \\
12 & Creation & 4-real & His \\
13 & Destruction & 4-real & Phe \\
14 & Connection & 4-real & Tyr \\
15 & Wisdom & 4-real & Trp \\
16 & Illusion & 4-imag & Pro \\
17 & Chaos & 4-imag & Cys \\
18 & Twist & 4-imag & Met \\
19 & Time & 4-imag & Ile \\
\bottomrule
\end{tabular}
\end{center}

The semantic names are mnemonic, not physical claims. The amino acid correspondence is structural (cardinality match + bijection theorem).

\section{Falsification Interface Summary}
\label{app:falsifiers}

This table is a quick reference. The authoritative statement of hypothesis scope and falsifiers appears in Section~10 and Table~\ref{tab:hypothesis_registry}; the entries below should be read as operational targets, not as logically decisive single-point criteria.

\begin{center}
\begin{tabular}{lll}
\toprule
\textbf{Hypothesis} & \textbf{Falsifier} & \textbf{Evidence Required} \\
\midrule
$\phiG$-Ladder & $>$50\% deviation at any rung & Precision timing data \\
8-Tick & Phase period $\neq 8$ & Biological oscillation data \\
Tau-Gate & No shared rung exists & Mass + timing measurements \\
ULL mapping (beyond cardinality) & No preregistered predictive mapping & Cross-validated biochemical/semantic tests \\
Light = Consciousness & Consciousness without plausible EM coupling & Neuro/AI evidence under EM isolation controls \\
\bottomrule
\end{tabular}
\end{center}

Each hypothesis has a Lean-encoded falsification interface. Experimental evidence triggering the falsifier would require revision of the framework.

\bibliographystyle{unsrt}
\bibliography{RESONANCE_PAPERS}

\end{document}

