\documentclass[11pt,a4paper]{article}
\usepackage[margin=1in]{geometry}
\usepackage[T1]{fontenc}
\usepackage{lmodern}
\usepackage{microtype}
\usepackage{amsmath,amssymb,amsthm}
\usepackage{mathtools}
\usepackage{booktabs}
\usepackage{enumitem}
\usepackage{xcolor}
\usepackage[hidelinks]{hyperref}
\usepackage{tikz}
\usetikzlibrary{arrows.meta,positioning,calc}
\newtheorem{theorem}{Theorem}[section]
\newtheorem{proposition}[theorem]{Proposition}
\newtheorem{lemma}[theorem]{Lemma}
\newtheorem{corollary}[theorem]{Corollary}
\newtheorem{definition}[theorem]{Definition}
\newtheorem{remark}[theorem]{Remark}
\newcommand{\phig}{\varphi}
\newcommand{\Jcost}{J}
\newcommand{\Rhat}{\hat{R}}
\newcommand{\Ecoh}{E_{\mathrm{coh}}}
\newcommand{\muStar}{\mu_{\star}}
\newcommand{\mRS}{m^{\mathrm{RS}}}
\newcommand{\Epass}{E_{\mathrm{passive}}}
\newcommand{\RS}{Recognition Science}
\newcommand{\SM}{Standard Model}
\newcommand{\RCL}{Recognition Composition Law}

\title{\textbf{The Origin of Mass in Recognition Science:\\Cost Geometry, Recognition Boundaries, and the $\phig$-Ladder}\\[0.5em]
\large Paper I of VI: Mechanism}
\author{Jonathan Washburn\\\small Recognition Science Research Institute, Austin, Texas\\\small \texttt{washburn.jonathan@gmail.com}}
\date{\today}

\begin{document}
\maketitle

\begin{abstract}
In the \SM{}, fermion masses are free parameters encoded by Yukawa couplings to the Higgs field. This paper develops an alternative ontology of mass within \RS{} (RS), a framework in which all physical structure is derived from a single functional equation---the \RCL{}. We show that mass emerges as a geometric property of \emph{recognition boundaries}: self-sustaining patterns on a discrete ledger whose persistence is governed by cost minimization. The unique cost functional $\Jcost(x)=\tfrac{1}{2}(x+x^{-1})-1$, forced by the \RCL{} together with normalization and calibration, selects the golden ratio $\phig=(1+\sqrt{5})/2$ as the unique self-similar scaling base. Mass hierarchies are encoded by integer positions on a $\phig$-ladder, while sector-level scales are fixed by cube combinatorics ($D=3$). We derive the recognition operator $\Rhat$ that replaces the Hamiltonian, show how the eight-tick closure cycle ($2^3=8$) provides a canonical period, and demonstrate that interactions between recognition boundaries reduce to cost-weighted adjacency moves on the cubic ledger. The Higgs mechanism is reinterpreted as the low-energy effective description of a fundamentally discrete process. Companion papers develop phenomenological predictions (II), the neutrino sector (III), transport discipline (IV), the fine-structure constant (V), and the generation problem (VI).
\end{abstract}

\tableofcontents
\newpage

\section{Introduction}
\subsection{The mass problem}
The \SM{} contains nine charged fermion masses spanning nearly five orders of magnitude, from the electron ($0.511\,\mathrm{MeV}$) to the top quark ($173\,\mathrm{GeV}$). These masses enter as free Yukawa couplings---the SM tells us \emph{how} particles acquire mass (electroweak symmetry breaking) but not \emph{why} they have the particular values they do.

\subsection{The RS approach}
RS begins from a single primitive: the \RCL{},
\begin{equation}
  \Jcost(xy) + \Jcost(x/y) = 2\,\Jcost(x)\,\Jcost(y) + 2\,\Jcost(x) + 2\,\Jcost(y),
  \label{eq:RCL}
\end{equation}
together with normalization $\Jcost(1)=0$ and calibration $\Jcost''_{\log}(0)=1$. These three conditions uniquely determine $\Jcost(x) = \tfrac{1}{2}(x + x^{-1}) - 1$, proved in Lean~4 via ODE uniqueness for the d'Alembert functional equation.

From this cost functional, a chain of forced consequences (T0--T8) derives: logic from cost minimization (T0), the Meta-Principle ``nothing costs infinity'' (T1), discreteness (T2), a double-entry ledger (T3), recognition events (T4), $\Jcost$ uniqueness (T5), the golden ratio $\phig$ (T6), the eight-tick period (T7), and three spatial dimensions (T8). Within this architecture, mass is not a separate concept---it is a coordinate on a discrete multiplicative ladder whose base $\phig$, period~8, and sector structure are all forced.

\section{The Cost Functional}
\subsection{Uniqueness (T5)}
\begin{theorem}[Cost uniqueness]
Let $F:\mathbb{R}_+\to\mathbb{R}$ satisfy the \RCL{}, $F(1)=0$, and $\lim_{t\to 0} 2F(e^t)/t^2 = 1$. Then $F(x) = \Jcost(x) := \frac{1}{2}(x + x^{-1}) - 1$ for all $x > 0$.
\end{theorem}
The proof converts the \RCL{} to a d'Alembert equation via $H(t):=F(e^t)+1$, yielding $H(t+u)+H(t-u)=2H(t)H(u)$. By Acz\'el's theorem, continuous solutions are $\cosh(\lambda t)$; calibration fixes $\lambda=1$. Machine-verified: \texttt{IndisputableMonolith.Cost.FunctionalEquation}.

\subsection{Key properties}
$\Jcost$ has: reciprocal symmetry $\Jcost(x)=\Jcost(1/x)$; non-negativity with equality iff $x=1$; strict convexity on $\mathbb{R}_+$; divergence $\Jcost(0^+)=+\infty$, $\Jcost(+\infty)=+\infty$.

\subsection{The Law of Existence}
\begin{theorem}
$\mathrm{defect}(x):=\Jcost(x)=0$ if and only if $x=1$. The Meta-Principle ($\Jcost(0^+)\to\infty$: ``nothing costs infinity'') is a \emph{derived theorem}, not an axiom.
\end{theorem}

\section{Recognition Boundaries and the $\phig$-Ladder}
\subsection{What is a particle?}
A \emph{recognition boundary} is a localized, self-sustaining pattern on the cubic ledger $\mathbb{Z}^3$ with finite nonzero cost, invariant under the recognition operator $\Rhat$ (up to phase/translation), and satisfying eight-tick neutrality.

\subsection{Mass as a ladder coordinate}
\begin{definition}
The \emph{mass} of boundary $b$ at anchor $\muStar$ is:
\begin{equation}
  \mRS(b;\muStar) = A_{\mathrm{sector}(b)}\cdot\phig^{\,r_b - 8 + \mathrm{gap}(Z_b)},
\end{equation}
where $A_{\mathrm{sector}}$ is the sector yardstick, $r_b\in\mathbb{Z}$ the rung, $-8$ the octave reference, and $\mathrm{gap}(Z_b)=\log_\phig(1+Z_b/\phig)$ the charge-derived band function.
\end{definition}
Mass is a geometric coordinate, not an intrinsic property conferred by a field. The ladder base $\phig$ is forced by self-similarity ($x^2=x+1$, T6); the origin $-8$ by the eight-tick closure (T7); the sector structure by cube combinatorics.

\subsection{The $\phig$-forcing (T6)}
\begin{theorem}
The unique positive solution to $x^2=x+1$ is $\phig=(1+\sqrt{5})/2$.
\end{theorem}

\section{Cube Geometry and the Counting Layer}
\subsection{Three dimensions forced (T8)}
$D=3$ is the unique dimension with non-trivial linking AND gap-45 synchronization ($\mathrm{lcm}(8,45)=360$ iff $D=3$).

\subsection{The 3-cube}
$V=2^3=8$ vertices, $E=3\cdot 2^2=12$ edges, $F=2\cdot 3=6$ faces. With the crystallographic constant $W=17$ (wallpaper groups) and one active edge per tick ($A=1$), we get $\Epass=E-A=11$.

\subsection{Sector yardsticks}
Each sector has $A_s=2^{B_{\mathrm{pow}}(s)}\cdot\Ecoh\cdot\phig^{r_0(s)}$ where $\Ecoh=\phig^{-5}$:
\begin{center}\begin{tabular}{lrrl}\toprule Sector & $B_{\mathrm{pow}}$ & $r_0$ & Formula \\\midrule Lepton & $-22$ & $62$ & $-2\Epass$; $4W-6$ \\ Up quark & $-1$ & $35$ & $-A$; $2W+A$ \\ Down quark & $23$ & $-5$ & $2E-1$; $E-W$ \\ Electroweak & $1$ & $55$ & $A$; $3W+4$ \\\bottomrule\end{tabular}\end{center}
All derived in Lean: \texttt{IndisputableMonolith.Masses.Anchor}.

\subsection{Generation torsion and the charge-band map}
Generation torsion $\tau_g\in\{0,\Epass,W\}=\{0,11,17\}$ is universal across sectors (see Paper~VI for derivation). Charge integerization $\tilde{Q}:=6Q$ and the $Z$-map yield three families: $Z_\ell=1332$, $Z_u=276$, $Z_d=24$.

\section{The Recognition Operator and Dynamics}
The fundamental dynamical law is $s(t+8\tau_0)=\Rhat(s(t))$, where $\Rhat$ minimizes $\Jcost$ (not energy). The Hamiltonian emerges in the quadratic regime $\Jcost(x)\approx\frac{1}{2}(x-1)^2$ for $|x-1|\ll 1$, where cost minimization reduces to stationary action.

\section{Interactions and the Yukawa Bridge}
Interactions are cost-weighted adjacency moves on the ledger. The SM Yukawa coupling at the anchor is:
\begin{equation}
  y_f(\muStar) = \frac{\sqrt{2}}{v}\cdot A_{\mathrm{sector}(f)}\cdot\phig^{\,r_f - 8 + \mathrm{gap}(Z_f)}.
\end{equation}
Yukawa couplings are effective parameters encoding $\phig$-ladder positions, not fundamental.

\section{Relation to the Higgs Mechanism}
The Higgs field is the continuum effective description of discrete $\phig$-ladder structure. The VEV $v\approx 246\,\mathrm{GeV}$ corresponds to the electroweak yardstick $A_{\mathrm{EW}}=2\cdot\Ecoh\cdot\phig^{55}$. The Goldstone mechanism remains intact as an effective description.

\section{Falsifiers}
(1)~Equal-$Z$ clustering failure at $\muStar$; (2)~Generation ratios inconsistent with $\phig^{11}$, $\phig^6$; (3)~Octave reference $-8$ replaceable by another integer; (4)~Alternative ladder base outperforming $\phig$; (5)~Sector yardstick formulas achievable from different counting-layer inputs.

\section{Conclusions}
Mass in RS is a geometric coordinate on a $\phig$-ladder forced by the cost functional. The ladder base, period, sector structure, and generation torsion are all consequences of the \RCL{} and $D=3$ cube geometry. The recognition operator $\Rhat$ replaces the Hamiltonian; the Higgs mechanism is an effective description.

\begin{thebibliography}{99}
\bibitem{PDG2024} R.~L.~Workman \textit{et al.} [PDG], PTEP \textbf{2022}, 083C01.
\bibitem{Aczel1966} J.~Acz\'el, \textit{Lectures on Functional Equations}, Academic Press (1966).
\bibitem{Washburn2025} J.~Washburn, \textit{Axioms} \textbf{15}(2), 90 (2025).
\end{thebibliography}
\end{document}
