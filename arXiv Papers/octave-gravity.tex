\documentclass[aps,prd,amsmath,amssymb,superscriptaddress,nofootinbib,preprint]{revtex4-2}

\usepackage[utf8]{inputenc}
\usepackage[T1]{fontenc}
\usepackage{lmodern}
\usepackage{microtype}
\usepackage{mathtools}
\usepackage{graphicx}
\usepackage{xcolor}
\usepackage{amsthm}
\usepackage{hyperref}
\hypersetup{colorlinks=true,allcolors=blue}

% Simple helpers (keep minimal; narrative-first paper)
\newcommand{\defterm}[1]{\textbf{#1}}
\newcommand{\RS}{\textsc{Recognition Science}}

% Math helpers
\newcommand{\Z}{\mathbb{Z}}
\newcommand{\R}{\mathbb{R}}
\newcommand{\C}{\mathbb{C}}
\newcommand{\Fin}{\mathrm{Fin}}
\newcommand{\Zmod}[1]{\Z/#1\Z}
\newcommand{\abs}[1]{\left\lvert #1 \right\rvert}
\newcommand{\norm}[1]{\left\lVert #1 \right\rVert}
\newcommand{\dd}{\,\mathrm{d}}
\newcommand{\PsiRS}{\Psi_{\text{RS}}}

% Theorem environments (keep light; narrative-first but precise)
\newtheorem{definition}{Definition}
\newtheorem{assumption}{Assumption}
\newtheorem{theorem}{Theorem}
\newtheorem{proposition}{Proposition}
\theoremstyle{remark}
\newtheorem{remark}{Remark}

\begin{document}

\title{Octave Gravity: Why an 8-Step Update Cycle Produces Geometric Gravity}

\author{Jonathan Washburn}
\affiliation{Recognition Physics Institute, Austin, Texas, USA}

\date{\today}

\begin{abstract}
Why does gravity look like curved spacetime? This paper develops a concrete, testable answer: gravity is the macroscopic expression of a deeper loop-closure requirement on a discrete update ledger, and the smallest complete closure cycle has eight steps (the \emph{Octave}).

We (i) formalize the Octave as a $\Zmod{8}$ clock with a rigorously constructed 3-bit Gray cycle that visits all $2^3=8$ local states with one-bit adjacency per tick, (ii) show how the Octave's canonical shift symmetry forces a DFT-8 spectral basis and a unique discrete-derivative energy with weights $4\sin^2(\pi k/8)$, and (iii) explain how ``closure on loops'' becomes, in the continuum limit, conservation laws and curvature, leading naturally to the geometric field description of General Relativity.

The paper is written to be readable first and precise second: every technical term is defined before use, and we separate machine-verified discrete statements (Octave/Gray-cycle/DFT facts) from continuum-limit hypotheses and from the GR variational bridge that is still being formalized in the Lean 4 theorem prover.
\end{abstract}

\maketitle

\tableofcontents

\newpage

% ==============================================================================
\section{The question this paper answers}
% ==============================================================================

Gravity is strange. Unlike other forces, it cannot be shielded. Everything couples to it. It appears to encode global rules---energy conservation, causal structure, the shape of spacetime itself---yet physics is supposed to be local: what happens here should not require a cosmic referee checking what happens everywhere else.

How do you get global consistency from purely local rules?

This paper proposes an answer: \emph{loop closure}. If reality updates locally, contradictions can hide at single points. But contradictions cannot hide on closed loops. If you walk around a loop and return to your starting point, any mismatch reveals itself. So if you want a physics that cannot harbor hidden contradictions, you enforce consistency on loops.

Once you commit to loop closure, a question arises: what is the smallest nontrivial loop that can close consistently? This paper claims the answer is eight steps---what we call the \emph{Octave}. And from that single structural fact, the geometric character of gravity follows.

The rest of this paper unpacks that claim in plain language, backed by formal results from the machine-verified project core.

% ==============================================================================
\section{How to read this paper}
% ==============================================================================

\subsection{What this paper is}
This is a \emph{conceptual foundation} paper. It explains \textbf{why} gravity should be geometric, \textbf{what} the Octave is, and \textbf{how} discrete loop-closure rules become, at large scales, the curved-spacetime picture of General Relativity.

\subsection{What this paper is not}
This paper does not fit galaxy rotation curves. It does not run cosmological parameter pipelines. Those are separate papers that \emph{use} the framework established here. This paper's job is to make the framework itself clear.

\subsection{The one-sentence version}
\textbf{Thesis:} If reality updates locally but must remain globally consistent, then the smallest non-contradictory closed update loop matters; an 8-step loop is the minimal closure that supports a stable 3D neighborhood, and its closure rules become, in the continuum limit, the conservation laws and curvature that define geometric gravity.

\subsection{What you should understand after reading}
\begin{itemize}
  \item Why ``geometry'' is the natural language for gravity in this framework (not an arbitrary choice).
  \item What the Octave is, and why eight steps is forced rather than chosen.
  \item How discrete closure becomes continuous conservation and curvature.
  \item What is proven, what is formalized but unproven, and what remains hypothesis.
\end{itemize}

% ==============================================================================
\section{Terms we will use (defined before first use)}
% ==============================================================================

We introduce every technical term here, in plain language, before we use it in an argument. No term will appear undefined.

\subsection{Tick}
A \defterm{tick} is one atomic update of the underlying system. We do not assume continuous time at the deepest level. We assume there is a minimal indivisible unit of change. A tick is not ``one second''; it is simply ``one step'' of whatever the reality-update process is.

Think of it like a clock's smallest possible tick---except this clock is not measuring human time; it is counting the fundamental steps by which the universe updates itself.

\subsection{RS-native units and constants (\(\tau_0\), \(\phi\), and \(J_{\mathrm{bit}}\))}
To avoid hidden parameters, the core Recognition Science (RS) development is expressed in \emph{RS-native units}. In the Lean codebase, the fundamental time quantum is defined as one tick:
\[
  \tau_0 \equiv 1\ \text{tick}.
\]

RS also uses the \defterm{golden ratio} \(\phi\), defined as
\[
  \phi \coloneqq \frac{1+\sqrt{5}}{2}.
\]
In the Lean codebase this is \texttt{Constants.phi} (see \path{IndisputableMonolith/Constants.lean}).

From \(\phi\) one defines the \defterm{elementary ledger bit cost}
\[
  J_{\mathrm{bit}} \coloneqq \ln \phi,
\]
which appears throughout the RS ``no free parameters'' accounting as the fixed cost per discrete scale step (Lean: \texttt{Constants.J\_bit}).

\subsection{Ledger}
A \defterm{ledger} is a bookkeeping model of local interactions. Every update has a matching record, so that when you examine any closed loop of interactions, nothing is mysteriously created or destroyed.

The analogy is double-entry accounting: if something is credited somewhere, it is debited somewhere else. The books always balance. A ledger is not literally money---it is a model for how physical updates can be recorded in a way that prevents contradictions from hiding.

\subsection{Closure}
\defterm{Closure} means: if you follow a closed loop of local updates, the loop adds up to zero net inconsistency. You cannot walk around a loop and return with an accounting mismatch.

Closure is the core stability requirement. It is what prevents ``energy from nowhere'' or ``influence without source.'' If closure fails anywhere, the theory is internally broken.

\subsection{Octave}
The \defterm{Octave} is a specific claim about closure: that the minimal \emph{complete} closed cycle---the smallest loop that visits a full local neighborhood and returns consistently---has length eight ticks.

This is not yet a statement about gravity. It is a statement about the structure of consistent discrete updating. Gravity enters later, when we ask what this structure looks like at large scales.

\subsection{Recognition Reality Field (RRF)}
The \defterm{Recognition Reality Field} ($\PsiRS$) is a scalar field representing recognition potential across spacetime. In the Lean formalization used by this project, a spacetime ``point'' is modeled as a 4-tuple \(x:\Fin 4\to\R\), and the RRF is a function
\[
  \PsiRS : (\Fin 4\to\R) \to \R.
\]
Intuitively: the discrete ledger is the microscopic state; in a smooth limit, \(\PsiRS(x)\) is a coarse-grained field that lets us write the strain in familiar continuum form.

\subsection{J-cost (Strain)}
The \defterm{J-cost} is a nonnegative function that measures ``how far a ratio is from unity.'' In the Lean codebase it is defined for a real ratio \(x\) by
\[
  J(x) \coloneqq \frac{x + x^{-1}}{2} - 1,
\]
which is symmetric under inversion (\(J(x)=J(x^{-1})\)), satisfies \(J(1)=0\), and is nonnegative for \(x>0\) (AM--GM inequality).

\begin{remark}[Stationarity principle (physics postulate)]
The physical content of RS is not just the definition of \(J\), but a selection rule: realized configurations are those that minimize (or make stationary) an appropriate \emph{total} strain functional built from local ledger-consistency constraints. In continuum language, this becomes a variational principle (Sections~\ref{sec:discrete-closure}--\ref{sec:gr-bridge}).
\end{remark}

\subsection{Emergence}
When we say gravity \defterm{emerges}, we mean:
\begin{itemize}
  \item The underlying rules are discrete and local.
  \item The simplest and most accurate \emph{macroscopic} description is continuous and geometric.
  \item The geometric field equations are the best ``compressed'' summary of the large-scale behavior of the discrete closure rules.
\end{itemize}

Emergence does not mean ``approximate'' or ``illusory.'' It means the geometric description is genuinely the right language at large scales---just as fluid dynamics is genuinely the right language for water, even though water is made of molecules.

\subsection{Geometry}
By \defterm{geometry} we mean a rule that tells you what distances and times mean locally, and therefore what paths are ``straightest'' (least cost, least strain). In General Relativity, this rule is encoded by a metric field $g_{\mu\nu}$. In this framework, the metric is the tensor that captures local variations of the RRF such that total $J$-cost is stationary.

\subsection{Curvature}
\defterm{Curvature} measures how local rules fail to be globally flat. If you parallel-transport a vector around a closed loop and it comes back rotated, the region enclosed is curved. Curvature is the continuum version of ``the loop did not close trivially.''

% ==============================================================================
\section{Postulates and mathematical model}
% ==============================================================================

This paper has two jobs at once: explain the Octave-to-gravity narrative, and pin down a minimal mathematical model for the parts of the story that can be stated cleanly.
We therefore separate:
\begin{itemize}
  \item \textbf{postulates} (the physical assumptions this paper starts from),
  \item \textbf{discrete theorems} (finite statements that can be proved exactly),
  \item \textbf{continuum-limit hypotheses} (how the discrete model is assumed to coarse-grain).
\end{itemize}

\subsection{Clock and phases}
\begin{definition}[Tick clock]
We model a single ``Octave window'' of time as the cyclic group $\Zmod{8}$ (equivalently, \(\Fin 8\)). An element \(t\in\Zmod{8}\) is called a \emph{phase}.
\end{definition}

\begin{definition}[Shift symmetry]
The one-tick time-translation on phases is the map \(t\mapsto t+1\) (mod \(8\)). On an 8-tuple \(x=(x_t)_{t\in\Zmod{8}}\), the induced shift operator is
\[
  (Sx)_t \coloneqq x_{t+1}.
\]
\end{definition}

\subsection{Local state space as binary patterns}
\begin{definition}[Local pattern space]
For an integer \(d\ge 1\), define the \(d\)-bit \emph{pattern space}
\[
  \mathrm{Pattern}(d) \coloneqq \{0,1\}^d.
\]
Concretely, an element \(p\in\mathrm{Pattern}(d)\) is a function \(p:\{0,\dots,d-1\}\to\{0,1\}\).
\end{definition}

\begin{definition}[One-bit adjacency]
Two patterns \(p,q\in\mathrm{Pattern}(d)\) are \emph{one-bit adjacent} if they differ in exactly one coordinate:
\[
  \mathrm{OneBitDiff}(p,q)\; \Longleftrightarrow\; \exists!\,k\in\{0,\dots,d-1\}\;\text{such that}\; p(k)\ne q(k).
\]
This corresponds to a single-bit flip or unit Hamming distance.
\end{definition}

\begin{definition}[Gray cover and Gray cycle]
Let \(T\ge 1\). A \emph{Gray cover} of \(\mathrm{Pattern}(d)\) with period \(T\) is a map \(\gamma:\Zmod{T}\to \mathrm{Pattern}(d)\) that is surjective and satisfies one-bit adjacency \(\mathrm{OneBitDiff}(\gamma(t),\gamma(t+1))\) for all \(t\).

A \emph{Gray cycle} is a Gray cover that is also injective (hence bijective, hence visits every pattern exactly once).
\end{definition}

\subsection{Physical postulates used by this paper}
\begin{assumption}[Discrete local update]
At sufficiently fine scale, the system evolves in discrete ticks, and there exists a physically distinguished closure period of eight ticks (an Octave window), so that phase can be modeled by \(\Zmod{8}\).
\end{assumption}

\begin{assumption}[Local completeness + adjacency]
Over one Octave window, the system's local ``neighborhood state'' is fully explored: there exists a Gray cycle \(\gamma:\Zmod{8}\to \mathrm{Pattern}(3)\).
\end{assumption}

\begin{remark}
The \emph{existence} of such a Gray cycle for \(d=3\) is a pure finite theorem (we give an explicit construction in Section~\ref{sec:octave-gray}).
The \emph{physical interpretation} of that cycle as ``what a local neighborhood is'' is the postulate.
\end{remark}

\begin{assumption}[Simplicial Nyquist surjection (explicit hypothesis seam)]
In the simplicial-ledger bridge layer, a ``recognition loop'' is modeled as a closed cycle of adjacent 3-simplices. We assume (as an explicit hypothesis) that any such loop admits a phase-indexed map into \(\mathrm{Pattern}(3)\) that is surjective; equivalently, any closed recognition loop must have length at least \(8\).

In Lean this is tracked as an explicit hypothesis (not yet a theorem) because the ledger-to-pattern identification and loop adjacency are still scaffolded (\path{IndisputableMonolith/Foundation/SimplicialLedger.lean}).
\end{assumption}

\begin{assumption}[Ledger closure (conservativity)]
There exists a conservative ledger representation of evolution such that for any contractible closed loop of admissible updates, the net ledger mismatch is zero. (In differential-geometric language: the fundamental bookkeeping constraint is a closure constraint on loops.)
\end{assumption}

\begin{remark}
Sections~\ref{sec:discrete-closure}--\ref{sec:gr-bridge} explain how this closure postulate becomes conservation laws and curvature in the continuum limit, and what additional regularity assumptions are needed to make the bridge rigorous.
\end{remark}

% ==============================================================================
\section{Why expect gravity to be geometric at all?}
% ==============================================================================

Before explaining the Octave, we need to understand what problem it solves. The problem is: \emph{how does global consistency arise from local rules?}

\subsection{Gravity looks global}
Gravity appears to encode global constraints. Energy-momentum is conserved. Causal structure is respected everywhere. Spacetime has a coherent geometry, not a patchwork of unrelated local rules.

But physics is supposed to be local: an event here should be determined by conditions here, not by a cosmic referee checking conditions everywhere simultaneously.

\subsection{Local rules can hide contradictions---unless you check loops}
If you only check consistency point by point, contradictions can hide. A local rule might seem fine at every location but produce a mismatch when you trace a path that returns to its starting point.

Loops are where contradictions reveal themselves. If a set of local rules is inconsistent, you can often detect the inconsistency by going around a closed path and finding that you do not return to the same state.

\subsection{Therefore: enforce closure on loops}
If you want a physics that cannot harbor hidden contradictions, you do not just check points---you enforce that \emph{every closed loop} must close consistently. This is the loop-closure principle.

The loop-closure principle is not an extra assumption bolted onto physics. It is the minimal requirement for a locally-defined system to avoid internal contradictions.

% ==============================================================================
\section{Why the minimal loop length matters}
% ==============================================================================

Once you commit to loop closure, a structural question arises: what is the \emph{smallest} nontrivial loop that can close?

\subsection{The minimal loop sets the fundamental scale}
If there is a smallest consistent loop, its length becomes a foundational ``clock.'' It influences:
\begin{itemize}
  \item How many independent directions a local region can support.
  \item What symmetries are forced when the loop is analyzed as a repeating structure.
  \item What kinds of neighborhoods can exist without contradiction.
\end{itemize}

The minimal loop is not just a technical detail. It is the seed of dimensionality and structure.

\subsection{Smaller loops would be ``too small''}
A loop shorter than the minimal length cannot visit enough distinct states to cover a neighborhood. It would close trivially (going nowhere) or inconsistently (missing states and creating mismatches).

\subsection{Larger loops are built from the minimal one}
Larger loops can be decomposed into combinations of minimal loops. So the minimal loop is fundamental; everything else is composite.

% ==============================================================================
\section{The Octave theorem: a 3-bit Gray cycle on eight phases}
\label{sec:octave-gray}
% ==============================================================================

This section states the Octave claim in the cleanest mathematical form we currently have: an explicit 8-step, one-bit-adjacent cycle that visits all \(2^3\) local binary states exactly once.

\subsection{Why eight is the ``first spatial'' size}
\begin{proposition}[Counting]\label{prop:pattern-card}
\(\abs{\mathrm{Pattern}(d)} = 2^d\). In particular, \(\abs{\mathrm{Pattern}(3)}=8\).
\end{proposition}

\begin{remark}
This is the precise sense in which ``three independent binary partitions'' corresponds to ``eight local states.'' In the Octave story, those three independent binary partitions are what later become the three independent spatial directions.
\end{remark}

\subsection{Existence: an explicit Hamiltonian cycle on the 3-cube}
\begin{theorem}[3-bit Gray cycle of period 8]\label{thm:gray8}
There exists a map \(\gamma:\Zmod{8}\to \mathrm{Pattern}(3)\) such that:
\begin{enumerate}
  \item \(\gamma\) is bijective (every 3-bit pattern occurs exactly once),
  \item \(\mathrm{OneBitDiff}(\gamma(t),\gamma(t+1))\) for all \(t\in\Zmod{8}\) (one-bit adjacency, including wrap-around).
\end{enumerate}
Equivalently, \(\gamma\) is a Hamiltonian cycle on the 3-dimensional hypercube graph \(Q_3\).
\end{theorem}

\begin{remark}[A concrete witness]
One explicit choice is the standard binary-reflected Gray order on 3 bits:
\[
  0,\,1,\,3,\,2,\,6,\,7,\,5,\,4,
\]
where \(\gamma(t)\) is the 3-bit binary expansion of the listed integer at phase \(t\). Consecutive codewords differ in exactly one bit, and the final codeword \(4\) differs from \(0\) in exactly one bit, so the cycle closes.

This exact witness is implemented and machine-verified in the repository as a function \(\Fin 8\to (\Fin 3\to \mathrm{Bool})\) together with proofs of bijectivity and one-bit adjacency.
\end{remark}

\begin{remark}[Lean ``witness layer'' packaging]
The Lean codebase deliberately separates the \emph{finite theorem} from any physics interpretation. It packages the Gray-cycle witness into a minimal \texttt{OctaveKernel.Layer} instance whose state is just the 8-phase clock:
\begin{itemize}
  \item \textbf{Observation}: \texttt{patternAtPhase : Phase → Pattern 3} is defined as the Gray-cycle path.
  \item \textbf{Layer}: \texttt{PatternCoverLayer} has \texttt{State := Phase} and \texttt{step s := s + 1}.
\end{itemize}
This is a conservative artifact that lets later bridge theorems talk about ``an 8-phase clock whose observations cover 3-bit patterns adjacently'' without asserting that the layer is already ``physics'' (see \path{IndisputableMonolith/OctaveKernel/Instances/PatternCover.lean}).
\end{remark}

\subsection{Minimality: you cannot cover 3-bit space in fewer than eight ticks}
\begin{theorem}[Eight-tick lower bound]\label{thm:eight-min}
Let \(T\ge 1\). If a map \(\gamma:\Zmod{T}\to \mathrm{Pattern}(3)\) is surjective, then \(T\ge 8\).
\end{theorem}

\begin{remark}
This is the simplest ``no free lunch'' fact behind the Octave: if you really want to visit all \(2^3\) local states, you need at least eight visits. The nontrivial content of Theorem~\ref{thm:gray8} is that you can do so while only changing one bit per tick (local adjacency).
\end{remark}

\subsection{Interpretation: why this is a gravity-relevant statement}
The Octave is not yet gravity; it is the \emph{minimal stable local clock+neighborhood structure}. Gravity enters when we add the ledger closure postulate and ask: what is the macroscopic language of a system whose evolution must be globally consistent under this minimal local structure?

In the next sections we add two additional ingredients:
\begin{itemize}
  \item a \textbf{canonical quadratic ``variation energy''} on the 8-tick cycle (forced by shift symmetry), and
  \item a \textbf{closure constraint on loops} (ledger conservativity), whose continuum limit becomes curvature and conservation laws.
\end{itemize}

% ==============================================================================
\section{Octave spectral structure: DFT-8 and the canonical discrete derivative}
\label{sec:dft8}
% ==============================================================================

The Octave is not only a counting fact (eight phases) and not only a combinatorial fact (a Gray cycle on the cube). It is also a \emph{spectral} fact: an 8-tick clock has a canonical shift symmetry, and shift symmetry forces a canonical Fourier basis and a canonical notion of ``variation energy'' on the cycle.

\subsection{DFT-8 is forced by shift symmetry}
Let \(x:\Zmod{8}\to \C\) be any complex-valued signal on the eight phases. Define the one-tick shift operator \(S\) by \((Sx)_t=x_{t+1}\).
The irreducible unitary representations of the cyclic group \(\Zmod{8}\) are one-dimensional characters, so the eigenbasis of \(S\) is the discrete Fourier basis.

Define \(\omega\coloneqq e^{-2\pi i/8}\). The unitary DFT-8 matrix entries are
\[
  B_{t,k} \coloneqq \frac{\omega^{tk}}{\sqrt{8}},\qquad t,k\in\Zmod{8}.
\]
Fourier coefficients are \(c_k \coloneqq \sum_{t\in\Zmod{8}} \overline{B_{t,k}}\,x_t\). In this basis, the shift acts diagonally:
\[
  S:\; c_k \mapsto \omega^k\,c_k.
\]
The \(k=0\) mode is the DC component. Modes \(k\neq 0\) are mean-free (``neutral'') in the sense that \(\sum_{t} B_{t,k}=0\) for \(k\neq 0\).

\begin{remark}[Alignment with the Lean artifact]
In the Lean formalization, the primitive 8th root is defined as
\[
  \omega_8 \coloneqq e^{-i\pi/4},
\]
and the DFT entry is defined (in normalized form) by
\[
  \mathrm{dft8\_entry}(t,k) \;=\; \frac{\omega_8^{\,t k}}{\sqrt{8}}
\]
(see \nolinkurl{IndisputableMonolith/LightLanguage/Basis/DFT8.lean}).
The diagonalization statement (that the DFT basis diagonalizes the cyclic shift matrix) is proved as \texttt{dft8\_diagonalizes\_shift}.
The stronger statement ``any unitary basis that diagonalizes shift agrees with DFT-8 up to phase and permutation'' is currently tracked as an explicit hypothesis (\texttt{dft8\_unique\_up\_to\_phase\_hypothesis}); the narrative in this paper uses the standard mathematical fact, but we keep the proof-status distinction explicit.
\end{remark}

\subsection{A unique local quadratic energy on the Octave}
Among quadratic energies on \(x\) that are local and shift-invariant, the simplest choice is the discrete derivative energy built from the one-step difference:
\[
  (Dx)_t \coloneqq (Sx)_t - x_t = x_{t+1}-x_t,
  \qquad
  E[x]\coloneqq \sum_{t\in\Zmod{8}} \abs{(Dx)_t}^2.
\]
In the Fourier basis this energy is diagonal. Since \(D\) has eigenvalues \(\omega^k-1\), we obtain
\[
  E[x] = \sum_{k\in\Zmod{8}} \abs{c_k}^2\,\abs{\omega^k-1}^2
  = 4\sum_{k\in\Zmod{8}} \abs{c_k}^2 \sin^2\!\Bigl(\frac{\pi k}{8}\Bigr).
\]
The factor \(4\sin^2(\pi k/8)\) are the eigenvalues of the discrete Laplacian \(D^\ast D\). This is not an ad hoc weight; it is the spectral footprint of the unique nearest-neighbor, shift-invariant quadratic variation energy on the Octave.

\begin{remark}[Connection to RS ``gap'' accounting]
In Recognition Science, the same DFT-8 backbone and the same Laplacian weights appear in the parameter-free ``gap'' bookkeeping on an eight-tick window: forcing a self-similar scaling pattern onto a discrete 8-tick clock generically produces neutral (non-DC) spectral content, and the ``gap'' measures the weighted cost of that neutral content.

The relevant self-similarity premise is the \emph{\(\phi\)-ladder hypothesis} (explicitly marked as a hypothesis in Lean; see \path{IndisputableMonolith/RRF/Hypotheses/PhiLadder.lean}).

Concretely, the Lean support documents define the canonical 8-tick \(\phi\)-pattern \(p(t)=\phi^t\) for \(t\in\Fin 8\), compute its DFT-8 coefficients, weight the neutral modes by the canonical Laplacian spectrum, and then apply an explicit normalization/projection step (including the \(64 = 8 \times 8\) ``ticks \(\times\) vertices'' measure choice).

The repository exposes:
\begin{itemize}
  \item a raw DFT-based candidate \texttt{w8\_dft\_candidate} in \path{IndisputableMonolith/Constants/GapWeight/Formula.lean},
  \item an explicit normalized projection operator \texttt{w8\_projected} in \path{IndisputableMonolith/Constants/GapWeight/Projection.lean},
  \item and a closed-form constant used by the main \(\alpha\) pipeline, \texttt{w8\_from\_eight\_tick}, in \path{IndisputableMonolith/Constants/GapWeight.lean}.
\end{itemize}

Numerically, the pipeline constant is
\[
  w_8 = \frac{348 + 210\sqrt{2} - (204 + 130\sqrt{2})\phi}{7} \approx 2.4905...
\]
(see \path{docs/internal_memo_w8_derivation.tex}). Proving the equality between the transparent projected definition and the closed form is tracked as an internal follow-up theorem.
\end{remark}

% ==============================================================================
\section{Discrete closure: from loops to conservation}
\label{sec:discrete-closure}
% ==============================================================================

We now trace the path from a discrete closure constraint on loops to the conservation laws that any viable continuum limit must satisfy. This is the first half of the Octave-to-gravity bridge.

\subsection{A discrete calculus: boundaries and the identity ``boundary of a boundary is zero''}
To make ``closure on loops'' precise, we need a language for loops, surfaces, and volumes on a discrete substrate. The standard tool is a cell complex (specifically a 3-simplicial complex of tetrahedra) together with its boundary operator (a discrete exterior calculus viewpoint) \cite{hirani_dec,desbrun_dec}.

Let \(K\) be an oriented cell complex approximating a region of space(-time). Let \(C_k(K)\) denote formal integer combinations of oriented \(k\)-cells (chains). There is a boundary operator
\[
  \partial: C_k(K)\to C_{k-1}(K),
\]
that sends an oriented cell to its oriented boundary (e.g.\ a triangle to its three directed edges). A fundamental identity of this calculus is
\[
  \partial\circ \partial = 0,
\]
often read as ``the boundary of a boundary is empty.'' This identity is the discrete, topology-level ancestor of the continuum identities that later become Bianchi constraints.

\begin{remark}[Plain language]
If you take the boundary of a patch, you get its edge. If you then take the boundary of that edge, you get nothing: edges do not have edges. This ``nothing'' is exactly what prevents contradictions from hiding when you sum around closed loops.
\end{remark}

\subsection{Closure implies conservation (discrete divergence-free condition)}
A conserved flow can be represented discretely as fluxes through faces. For example, in 3D take a 2-cochain \(J\) assigning a signed flux \(J(f)\) to each oriented face \(f\). Conservation in a volume cell \(V\) is the statement
\[
  \sum_{f\in \partial V} J(f) = 0,
\]
meaning: net flux out of any closed volume is zero. This is the discrete divergence-free condition.

In the continuum limit (when cells become small and sums become integrals), this becomes the familiar divergence law. Writing a continuum current density \(\mathbf{j}\),
\[
  \oint_{\partial \Omega} \mathbf{j}\cdot \mathbf{n}\,\dd A = 0
  \quad \Longrightarrow\quad
  \nabla\cdot \mathbf{j}=0
\]
in regions without sources/sinks.

\subsection{Closure forces compatibility constraints on dynamics}
The conservation law is not just an observational fact; it is a \emph{compatibility constraint} that any macroscopic field equation must respect. If a continuum field equation claims ``field = source,'' then taking a divergence of both sides must be consistent. In successful physical theories, this is guaranteed by an identity (not by tuning):
\begin{itemize}
  \item In electromagnetism, gauge structure implies a differential identity that makes charge conservation automatic.
  \item In General Relativity, the Bianchi identity implies \(\nabla_\mu G^{\mu\nu}=0\), forcing \(\nabla_\mu T^{\mu\nu}=0\).
\end{itemize}
In this framework, those identities are the continuum shadow of the discrete fact \(\partial^2=0\): closure at the discrete level becomes ``divergence of the left-hand side is identically zero'' at the continuum level.

\subsection{Why a variational principle appears}
Once closure is a hard constraint, the system is not free to evolve arbitrarily. Given sources and boundary conditions, there is typically a family of admissible configurations; the physical configuration is selected by an optimization principle: the system chooses the configuration of \emph{least strain}.

Mathematically, the least-strain statement is naturally expressed by a functional \(J[\text{fields}]\) whose stationary points are the realized configurations:
\[
  \delta J = 0.
\]
The job of the next section is to identify what this functional must look like in the continuum limit if we also require locality and covariance.

\subsection{Preview: where curvature enters}
The discrete closure constraint controls \emph{holonomy}: how local frames compare after transport around loops. In the continuum, holonomy is measured by curvature. In discrete geometric gravity (e.g.\ Regge calculus), curvature is concentrated on lower-dimensional ``hinges'' and is read off from deficit angles around loops \cite{regge1961}. This is why curvature is the natural ``storage location'' for loop bookkeeping in a geometric theory of gravity.

% ==============================================================================
\section{Continuum limit: geometric gravity and Einstein dynamics}
\label{sec:gr-bridge}
% ==============================================================================

We now move from discrete closure language to the standard continuum language of gravitational physics \cite{wald,carroll}. The point of this section is not to re-teach GR, but to show exactly \emph{where} GR sits in the Octave story: GR is the cleanest covariant continuum closure of a loop-based consistency constraint.

\subsection{Kinematics: metric, connection, curvature}
A Lorentzian metric \(g_{\mu\nu}(x)\) assigns local spacetime intervals. The Levi--Civita connection \(\nabla\) (the unique torsion-free connection compatible with \(g\)) tells us how to compare vectors at nearby points.

Curvature measures the failure of transporting around a small loop to return a vector to itself. Formally, for a vector field \(V^\rho\),
\[
  [\nabla_\mu,\nabla_\nu]V^\rho = R^{\rho}{}_{\sigma\mu\nu}V^\sigma,
\]
where \(R^{\rho}{}_{\sigma\mu\nu}\) is the Riemann curvature tensor. Contracting indices gives the Ricci tensor \(R_{\mu\nu}\) and scalar curvature \(R\).

\subsection{Dynamics: the Einstein--Hilbert action}
The simplest local, generally covariant action for a metric is the Einstein--Hilbert action (plus a cosmological constant term):
\[
  S_{\mathrm{EH}}[g] = \frac{c^3}{16\pi G}\int (R - 2\Lambda)\sqrt{-g}\,\dd^4x.
\]
Matter fields contribute an additional action \(S_m[g,\psi]\), and define stress-energy by
\[
  T_{\mu\nu} \coloneqq -\frac{2}{\sqrt{-g}}\frac{\delta S_m}{\delta g^{\mu\nu}}.
\]
Stationarity \(\delta(S_{\mathrm{EH}}+S_m)=0\) yields the Einstein field equations
\[
  G_{\mu\nu} + \Lambda g_{\mu\nu} = \frac{8\pi G}{c^4}T_{\mu\nu},
  \qquad
  G_{\mu\nu}\coloneqq R_{\mu\nu}-\tfrac12 g_{\mu\nu}R.
\]

\subsection{Why conservation is automatic (Bianchi identity)}
The differential-geometric analogue of \(\partial^2=0\) is the Bianchi identity, which implies
\[
  \nabla_\mu G^{\mu\nu} = 0.
\]
Therefore any matter source coupled consistently to gravity must satisfy
\[
  \nabla_\mu T^{\mu\nu}=0.
\]
In the Octave story, this is not an afterthought: it is the continuum form of ledger closure. The metric field equations must be built so that a divergence identity is automatic, not tuned.

\subsection{Why GR is (nearly) unique as a local covariant closure}
There is a standard uniqueness theorem behind the slogan ``if you want a local covariant theory of a metric with second-order field equations, you essentially get GR.'' In four spacetime dimensions, the Lovelock theorem \cite{lovelock1971} states that the most general symmetric, divergence-free rank-2 tensor built from \(g_{\mu\nu}\) and up to its second derivatives is a linear combination of \(G_{\mu\nu}\) and \(g_{\mu\nu}\) (the cosmological constant term).

\begin{remark}[What this means here]
If the macroscopic limit of a ledger-closure theory is:
\begin{itemize}
  \item local (no explicit long-range kernels in the fundamental covariant law),
  \item generally covariant (no preferred coordinates),
  \item metric-based (gravity is encoded in \(g_{\mu\nu}\)),
  \item second-order (to avoid extra propagating ghost degrees of freedom),
\end{itemize}
then the closure/compatibility requirement forces the GR form. The remaining task is to show that the RS ledger strain functional really produces \(S_{\mathrm{EH}}\) (or an equivalent local covariant action) in its continuum limit.
\end{remark}

\subsection{Where RS sits: ``least strain'' as ``stationary action''}
At the narrative level, the Octave story says:
\begin{align}
\text{discrete closure + least strain}
&\Longrightarrow \text{covariant continuum action}, \nonumber\\
&\Longrightarrow \text{automatic divergence identity}.
\end{align}
At the mathematical level, the current Lean bridge defines a \textbf{field cost density} for the RRF:
\[
  \mathcal{J}_{\text{RRF}}(\Psi, g) = \frac{1}{2} g^{\mu\nu} \partial_\mu \Psi \partial_\nu \Psi.
\]

The \textbf{field-cost/curvature bridge} is then expressed as an equality of \emph{metric functional derivatives}: when \(\Psi\) encodes the emergent metric degrees of freedom, varying \(\mathcal{J}_{\text{RRF}}\) with respect to \(g^{\mu\nu}\) produces the same Euler--Lagrange response as varying the Ricci scalar term.
In the Lean codebase this appears as the named lemma
\[
  \frac{\delta}{\delta g^{\mu\nu}(x)}\Bigl(\mathcal{J}_{\text{RRF}}(\Psi,g)\Bigr)
  \;=\;
  \frac{\delta}{\delta g^{\mu\nu}(x)}\bigl(R(g)\bigr),
\]
recorded as \texttt{field\_cost\_equals\_curvature} in \path{IndisputableMonolith/Relativity/Dynamics/RecognitionField.lean}. As of this writing it is a formalized interface with explicit proof debt (\texttt{sorry}); we therefore treat it as an explicit hypothesis seam in the gravity story, tracked and audited in \path{docs/GR_EMERGENCE_PLAN.md} and \path{docs/GRAVITATIONAL_EMERGENCE_PAPER.tex}.

% ==============================================================================
\section{What gravity is, in this picture}
% ==============================================================================

We can now say plainly what gravity is.

\subsection{Gravity is the consistency field}
Gravity is the field that enforces consistent global bookkeeping of local updates. It is not primarily a force. It is the rule that determines which configurations of motion and influence can exist without contradiction.

When you feel gravity pulling you toward the Earth, what you are experiencing is this: configurations in which you hover motionless require more bookkeeping strain than configurations in which you fall. The geometry is telling you which paths are least strained.

\subsection{Matter sources gravity because matter creates demands}
Matter is where updates happen---where changes are being attempted. Updates create demands on the ledger: ``something changed here; the books must still balance.''

The geometry is the system's solution that satisfies closure while accommodating those demands. In the GR continuum language, those demands are summarized by a stress-energy tensor \(T_{\mu\nu}\): a local accounting object that encodes energy density, momentum flux, and pressure/stress.
The statement ``matter sources gravity'' is the statement that the curvature of the metric is constrained by \(T_{\mu\nu}\) through the Einstein field equations (Section~\ref{sec:gr-bridge}).

\subsection{Free fall is the path of least bookkeeping strain}
If geometry encodes consistency and the system chooses least strain, then test bodies move along paths that are easiest to keep consistent.

That is why free fall appears as the ``straightest path'' in curved geometry. It is not that curved spacetime exerts a force. It is that curved spacetime \emph{is} the consistency solution, and free-fall paths are the paths along which consistency is maintained with least effort.

In GR this is encoded by the geodesic equation. Writing a worldline \(x^\mu(\tau)\) with tangent \(u^\mu=\dd x^\mu/\dd\tau\),
\[
  u^\mu \nabla_\mu u^\nu = 0,
\]
or in coordinates,
\[
  \frac{\dd^2 x^\mu}{\dd \tau^2} + \Gamma^\mu_{\alpha\beta}\frac{\dd x^\alpha}{\dd\tau}\frac{\dd x^\beta}{\dd\tau}=0,
\]
where \(\Gamma^\mu_{\alpha\beta}\) are the Christoffel symbols of the Levi--Civita connection.

\subsection{Weak-field limit (what reduces to Newton's law)}
Any candidate continuum limit of Octave gravity must reproduce the weak-field, slow-motion regime tested in the solar system. In GR this appears as the Newtonian limit of the metric:
\[
  g_{00} \approx -\Bigl(1+\frac{2\Phi}{c^2}\Bigr),
\]
where \(\Phi\) is the Newtonian gravitational potential. In this limit, the Einstein field equations reduce to Poisson's equation
\[
  \nabla^2 \Phi = 4\pi G\rho,
\]
for nonrelativistic mass density \(\rho\).
This is the regime in which ``gravity looks like a force.'' In the geometric view, the ``force'' is an approximation to geodesic motion in a weakly curved metric.

\subsection{Why gravity cannot be shielded}
Gravity couples to everything because the ledger tracks everything. You cannot opt out of bookkeeping. Any update, anywhere, creates demands that the geometry must accommodate. There is no configuration of matter that erases its own ledger entries.

This is why gravity is universal: it is not a force carried by a special particle that some things might not interact with. It is the consistency requirement itself, and nothing escapes the requirement to be consistent.

% ==============================================================================
\section{Finite-information closure: Information-Limited Gravity (ILG) as an effective display}
\label{sec:ilg}
% ==============================================================================

Up to this point, we have described the \emph{ideal} closure limit: exact ledger closure, exact continuum limit, and a local covariant macroscopic law (GR).
Recognition Science also contains a second, empirically motivated layer: what happens when closure is \emph{information-limited}. In that regime the macroscopic law can remain geometric, but the \emph{effective sourcing} of gravity by observed matter is modified because the system is not permitted to condition on (or ``recognize'') arbitrary fine-grained information.

\subsection{The basic idea}
In plain terms: if the gravitational inference process cannot fully resolve a source at all scales, it must use a coarse-grained source. Coarse-graining does not merely ``blur'' the field; it can rescale the effective source strength in a scale-dependent way.
In RS this effect is organized by the Octave clock: the same eight-tick structure that forces DFT-8 spectral rigidity also supplies a canonical way to talk about ``what information is available'' per update cycle.

\subsection{Cosmology (quasi-static limit): a source-side kernel}
In the Newtonian/quasi-static regime of cosmological perturbations, gravity is often summarized by a Poisson-type constraint relating the potential \(\Phi\) to the matter density perturbation. Information-Limited Gravity modifies this \emph{on the source side} by a dimensionless kernel \(w\) that depends on scale and time.

In Fourier space one writes schematically
\[
  -k^2 \Phi(k,a) \;=\; 4\pi G\,a^2\, w(k,a)\,\delta\rho_b(k,a),
\]
where \(\delta\rho_b\) is the baryonic matter perturbation (the ``observed'' source). The ILG proposal fixes \(w\) to a simple scale/time form
\[
  w(k,a) = 1 + C\Bigl(\max(\varepsilon, \frac{a}{k\tau_0})\Bigr)^{\alpha},
\]
where \(\varepsilon\) is a small positive regulator (in the Lean formalization \(\varepsilon=0.01\)) used as a guard against division-by-zero edge cases; physically one considers \(k>0\) and drops the regulator.

In the RS-canonical parameterization formalized in Lean, the exponent and amplitude are fixed by \(\phi\):
\[
  \alpha = \frac{1 - 1/\phi}{2},\qquad C = \phi^{-3/2},
\]
and \(\tau_0\) is the fundamental tick (RS-native time quantum).
The detailed motivation, parameter policy, and linear-regime predictions are developed in the dedicated ILG cosmology paper(s); the corresponding Lean formalization lives in \path{IndisputableMonolith/ILG/Kernel.lean} and \path{IndisputableMonolith/ILG/PoissonKernel.lean}.

\subsection{Galaxies (phenomenological display): a time-lag kernel}
At galaxy scales, one convenient ``display'' of information-limited sourcing is a causal-response/time-lag kernel that rescales the baryonic contribution to rotation curves.
A representative form used in the repository's ILG weak-field display is
\[
  w_t(T_{\mathrm{dyn}},\tau_0)=1 + C_{\mathrm{lag}}\Bigl(\max(\varepsilon_t, T_{\mathrm{dyn}}/\tau_0)\Bigr)^{\alpha},
\]
where \(\varepsilon_t\) is a small positive floor (in the current Lean formalization it is set to \(0.01\); see \nolinkurl{IndisputableMonolith/Relativity/ILG/WeakField.lean}).
leading to a multiplicative prediction for squared circular speed \(v^2\):
\[
  v^2_{\mathrm{model}}(r) = w_t\bigl(T_{\mathrm{dyn}}(r),\tau_0\bigr)\,v^2_b(r),
\]
where \(v_b\) is the baryonic contribution and \(T_{\mathrm{dyn}}(r)\) is a local dynamical time. This is an \emph{effective} description; its job is to make the information-limited modification measurable in data analysis. A full covariant embedding is a separate paper-level task.

\begin{remark}[Why include ILG here at all?]
Because it answers the reader's natural next question: ``Fine, GR is the local covariant closure---so where do the dark-matter-like and dark-energy-like effects come from in your theory?'' In this program the answer is: they are \emph{not} new substances; they are effective consequences of information-limited sourcing on top of the GR baseline.
\end{remark}

% ==============================================================================
\section{Predictions and falsification}
% ==============================================================================

A good theory makes predictions that could be wrong. What does Octave gravity predict?

\subsection{Structural predictions}
\begin{itemize}
  \item \textbf{Rigidity of form:} The theory strongly constrains what macroscopic field equations are allowed. Not any geometric theory will do---only theories compatible with the Octave closure structure.
  \item \textbf{Dimensional structure:} The same 8-step structure that forces 3D geometry should constrain other sectors (symmetry groups, particle content). This is tested by checking whether the broader Recognition Science program correctly predicts Standard Model structure from the same Octave.
  \item \textbf{Auditability:} The derivation is intended to be machine-checkable. If formal verification reveals a gap, the theory must either close the gap or retract the claim.
\end{itemize}

\subsection{Where the theory touches data (GR baseline + ILG departures)}
Octave gravity makes contact with observation in two layers:
\begin{itemize}
  \item \textbf{GR baseline (ideal closure):} recover the standard successes of GR: gravitational redshift, light bending, time delay, perihelion precession, strong equivalence principle tests, and gravitational waves propagating at speed \(c\) (to observational accuracy).
  \item \textbf{ILG departures (finite-information closure):} introduce scale/time-dependent effective sourcing encoded by kernels like \(w(k,a)\) (Section~\ref{sec:ilg}). This produces distinctive signatures in galaxies (rotation curves, lensing) and in cosmology (growth, lensing, ISW).
\end{itemize}

\subsection{Hard falsifiers (examples)}
Beyond the structural falsifiers, the information-limited layer has sharp empirical failure modes. Examples (to be quantified in the dedicated ILG papers):
\begin{itemize}
  \item \textbf{Wrong-sign ISW prediction:} if the kernel predicts a suppression (or sign change) of a large-scale late-time effect and the opposite is robustly measured, the kernel form is ruled out.
  \item \textbf{No \(k\)-dependence where required:} if the model predicts a scale-dependent growth/lensing relation and data show scale-independence over the predicted regime, the kernel is ruled out.
  \item \textbf{GW-sector inconsistency:} any covariant completion that changes GW propagation in conflict with multimessenger constraints is ruled out.
\end{itemize}

\subsection{What would falsify this story}
\begin{itemize}
  \item \textbf{Alternative minimal cycle:} Showing that a cycle length other than eight can support the same closure and completeness requirements would weaken or falsify the Octave uniqueness claim.
  \item \textbf{Broken bridge:} Showing that loop closure does not force the required compatibility identities would break the bridge from discrete rules to geometric field equations.
  \item \textbf{Wrong continuum limit:} Showing that the continuum limit of the discrete structure does not produce GR-like dynamics would falsify the emergence claim.
\end{itemize}

These are not vague worries. They are specific checks that the theory must survive.

% ==============================================================================
\section{Relationship to other papers}
% ==============================================================================

This paper is one piece of a larger program. Here is how it fits.

\subsection{This paper provides the conceptual foundation}
The job of this paper is to make the ``why geometry?'' story clear and self-contained. A reader should finish this paper understanding why the Octave matters and how it leads to gravity.

\subsection{Other papers build on this foundation}
\begin{itemize}
  \item A \textbf{foundations paper} formalizes the definitions here and provides machine-verified proofs of the structural claims (Octave minimality, closure-to-conservation bridge, etc.).
  \item A \textbf{GR emergence paper} shows explicitly how the discrete structure becomes the Einstein field equations in the continuum limit.
  \item \textbf{Phenomenology papers} (galaxy rotation curves, cosmological tests) use the framework to make predictions and compare with data.
\end{itemize}

This paper is the first one a skeptical reader should read. The others assume you already understand why the framework might be worth taking seriously.

% ==============================================================================
\section{Proof status: what is proven, what is not}
% ==============================================================================

Honesty requires separating what is established from what is conjectured.

\subsection{Machine-verified discrete facts (Lean)}
\begin{itemize}
  \item \textbf{8-phase clock closure}: \(\Fin 8\) arithmetic and ``add 8 gives identity'' facts:
    \begin{itemize}
      \item \nolinkurl{IndisputableMonolith/Octave/Theorem.lean}: \texttt{phase\_add8}, \texttt{phase\_add1\_iter8}.
    \end{itemize}

  \item \textbf{3-bit pattern coverage and Gray adjacency}: complete coverage of \(\mathrm{Pattern}(3)\) at period 8, plus a concrete Gray-cycle witness with one-bit steps:
    \begin{itemize}
      \item \nolinkurl{IndisputableMonolith/Patterns.lean}: \texttt{card\_pattern}, \texttt{eight\_tick\_min}.
      \item \nolinkurl{IndisputableMonolith/Patterns/GrayCycle.lean}: \texttt{grayCycle3} witness, \texttt{grayCover\_eight\_tick\_min}.
      \item \nolinkurl{IndisputableMonolith/Octave/Theorem.lean}: bundles the witness as \texttt{patternAtPhase} and proves one-bit step facts.
    \end{itemize}

  \item \textbf{Octave witness-layer packaging (no physics claim)}:
    \begin{itemize}
      \item \nolinkurl{IndisputableMonolith/OctaveKernel/Instances/PatternCover.lean}: \texttt{PatternCoverLayer} + \texttt{patternAtPhase} channel.
    \end{itemize}

  \item \textbf{DFT-8 backbone (shift diagonalization and neutral modes)}:
    \begin{itemize}
      \item \nolinkurl{IndisputableMonolith/LightLanguage/Basis/DFT8.lean}: \texttt{dft8\_diagonalizes\_shift}, \texttt{dft8\_mode\_neutral}.
      \item \nolinkurl{IndisputableMonolith/Constants/GapWeight/Projection.lean}: \texttt{diffEnergy8\_mode} (canonical spectral footprint of the discrete derivative).
    \end{itemize}

  \item \textbf{Gap weight \(w_8\) (parameter-free constant + transparent projection operator)}:
    \begin{itemize}
      \item \nolinkurl{IndisputableMonolith/Constants/GapWeight.lean}: \texttt{w8\_from\_eight\_tick}, \texttt{w8\_pos}.
      \item \nolinkurl{IndisputableMonolith/Constants/GapWeight/Formula.lean}: \texttt{w8\_dft\_candidate} (raw weighted neutral sum).
      \item \nolinkurl{IndisputableMonolith/Constants/GapWeight/Projection.lean}: \texttt{w8\_projected} and \texttt{projectionScale\_eq} (explicit normalization choice).
    \end{itemize}

  \item \textbf{ILG kernel definitions (effective, finite-information display)}:
    \begin{itemize}
      \item \nolinkurl{IndisputableMonolith/ILG/Kernel.lean}: \texttt{kernel}, \texttt{rsKernelParams} (RS-canonical \(\alpha\), \(C\)).
      \item \nolinkurl{IndisputableMonolith/ILG/PoissonKernel.lean}: modified Poisson operator, enhancement identity.
      \item \nolinkurl{IndisputableMonolith/Relativity/ILG/WeakField.lean} and \nolinkurl{IndisputableMonolith/Relativity/ILG/KernelForm.lean}: the time-kernel \(w_t\) and its basic invariances.
    \end{itemize}
\end{itemize}

\subsection{Formalized but still incomplete / proof debt remains}
\begin{itemize}
  \item \textbf{Ledger-to-manifold bridge}: a simplicial-ledger topology layer exists but includes explicit hypotheses/scaffolds:
    \begin{itemize}
      \item \textbf{Lean scaffold}: \path{IndisputableMonolith/Foundation/SimplicialLedger.lean} (hypothesis \texttt{H\_SimplicialNyquistSurjection}).
    \end{itemize}
  \item \textbf{GR variational bridge}: the repository contains an audited GR-emergence manuscript and a Lean roadmap; several key lemmas (functional derivatives, Palatini identities, boundary-term handling) remain to be completed (see \path{docs/GRAVITATIONAL_EMERGENCE_PAPER.tex} and \path{docs/GR_EMERGENCE_PLAN.md}).
  \item \textbf{Field Cost Isomorphism}: The theorem identifying RRF field cost variation with Ricci scalar variation is currently a named lemma with a \texttt{sorry} (e.g.\ \path{IndisputableMonolith/Relativity/Dynamics/RecognitionField.lean}, theorem \texttt{field\_cost\_equals\_curvature}).
  \item \textbf{Covariant completion of ILG}: a full covariant embedding of the effective source-side kernel (consistent with GW constraints) is a planned, not-yet-finished deliverable.
\end{itemize}

\subsection{Explicit hypotheses (not yet formalized)}
\begin{itemize}
  \item The existence and uniqueness of the continuum limit (regularity assumptions, interchange of limits, and the precise sense in which discrete sums converge to continuum integrals).
  \item The identification of the RS ledger strain functional with a local covariant continuum action equivalent (up to boundary terms) to \(S_{\mathrm{EH}}+S_m\).
  \item The regime map between ideal closure (GR) and finite-information closure (ILG): where the effective kernels apply, and what the covariant completion must look like.
\end{itemize}

% ==============================================================================
\section{Conclusion}
% ==============================================================================

Gravity, in this framework, is the macroscopic face of a deeper requirement: local updates must compose without contradiction.

The Octave---the minimal 8-step closed cycle---matters because it is the smallest complete loop that can cover a 3D neighborhood consistently. It sets the dimensional structure of space. Its closure rules become, in the continuum limit, the conservation laws and curvature that geometric gravity describes.

We do not assume geometry and then derive predictions. We assume closure on loops and find that geometry is the inevitable macroscopic description. Curvature is the ledger's record of loop nontriviality; matter is the source of ledger demands; and free fall is the path along which ledger consistency is maintained with minimum strain.

This paper has aimed to explain that story in plain language: why eight, why loops, why geometry, why gravity. The formal verification and phenomenological tests belong to companion papers. But the conceptual core is here: gravity is consistency, and consistency closes on loops.

\begin{thebibliography}{99}
\bibitem{wald}
R.~M.~Wald,
\emph{General Relativity},
University of Chicago Press (1984).

\bibitem{carroll}
S.~M.~Carroll,
\emph{Spacetime and Geometry: An Introduction to General Relativity},
Addison-Wesley (2004).

\bibitem{lovelock1971}
D.~Lovelock,
``The Einstein tensor and its generalizations,''
\emph{Journal of Mathematical Physics} \textbf{12}, 498 (1971).

\bibitem{hirani_dec}
A.~N.~Hirani,
``Discrete Exterior Calculus,''
Ph.D. thesis, California Institute of Technology (2003).

\bibitem{desbrun_dec}
M.~Desbrun, E.~Kanso, and Y.~Tong,
``Discrete Differential Forms for Computational Modeling,''
in \emph{Discrete Differential Geometry}, Oberwolfach Seminars, vol.~38,
Birkh\"auser (2008).

\bibitem{regge1961}
T.~Regge,
``General Relativity Without Coordinates,''
\emph{Il Nuovo Cimento} \textbf{19}, 558--571 (1961).
\end{thebibliography}

\end{document}
