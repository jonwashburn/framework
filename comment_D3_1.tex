\documentclass[11pt]{article}

\usepackage{amsmath,amssymb,amsthm,mathtools}
\usepackage[margin=1in]{geometry}

\newtheorem{definition}{Definition}
\newtheorem{lemma}{Lemma}
\newtheorem{proposition}{Proposition}
\newtheorem{corollary}{Corollary}

\title{The ``three independent constraints pick $D=3$''!! (with help of AI) }
\author{}
\date{}

\begin{document}
\maketitle

Let $D\in \mathbb{N}$ denote the spatial dimension of an ambient Euclidean space $\mathbb{R}^D$.
Suppose one proposes three criteria/constraints, labeled as
\[
(T),\qquad (K),\qquad (S),
\]
and the claim is that they are ``independent/orthogonal'' but all ``converge'' to select $D=3$.

I think as currently stated, each of $(T),(K),(S)$ already \emph{encodes} $D=3$ (or at best
selects the smallest admissible $D$), so the intersection-of-evidence narrative is not credible.
I first try to give feedback, and then a mathematically (with GPT) give way to fix it.

\section*{1.\ What ``independent constraints'' should mean.}

For each criterion $(X)\in\{(T),(K),(S)\}$, define the \emph{allowed dimension set}
\[
\mathcal{A}_X \;:=\;\{D\in \mathbb{N}: \text{criterion $(X)$ holds in dimension $D$}\}.
\]


\noindent A family of constraints $(X_1),\dots,(X_n)$ is \emph{convergently independent} for selecting $D^\star$
if:
\begin{enumerate}
\item Nontriviality: Each $\mathcal{A}_{X_i}$ contains \emph{more than one} candidate dimension (so no single constraint
is already a disguised restatement of $D=D^\star$).
\item Convergence: The intersection is a singleton:
\begin{equation}
 \bigcap_{i=1}^n \mathcal{A}_{X_i} \;=\;\{D^\star\}.  
\end{equation}
\end{enumerate}

The ideal pattern should be: each constraint carves out a \emph{set} of dimensions and only the intersection collapses to $\{3\}$.

\medskip
\noindent
\textbf{Key point.}
If \emph{any} $\mathcal{A}_{X_i}=\{3\}$ already, then that one constraint alone selects $D=3$ and
the ``convergent independent'' claim is automatically weak (and can look circular), even if the other
constraints also happen to be true in $D=3$.

\section*{2.\ Analysis of (S): minimizing $45\cdot 2^D$ under $D\ge 3$}

Assume $(S)$ is:

\begin{quote}
\emph{Choose $D$ to minimize the cost function $C(D)=45\cdot 2^D$ subject to $D\ge 3$.}
\end{quote}


\underline{Statement 1:} The function $C(D)=45\cdot 2^D$ is strictly increasing in $D\in\mathbb{N}$.


\quad Proof:
For any integer $D$,
\begin{equation}
    C(D+1)=45\cdot 2^{D+1}=45\cdot 2\cdot 2^D = 2\,C(D) > C(D).
\end{equation}
So $C$ is strictly increasing on $\mathbb{N}$.\\


\underline{Statement 2:}
Under the constraint $D\ge 3$, the unique minimizer of $C(D)=45\cdot 2^D$ over $D\in\mathbb{N}$ is $D=3$.


\quad Proof:
Since $C(D)$ is strictly increasing, the minimum over the feasible set $\{3,4,5,\dots\}$ is attained at the
smallest feasible integer, i.e.\ $D=3$. \\

So the point is that $(S)$ is not ``physics selecting $D=3$''. 
$\mathcal{A}_S=\{3\}$ because $D=3$ is the \emph{smallest allowed input} to a strictly increasing function.
Thus $(S)$ does not derive $D\ge 3$; it \emph{assumes} it, then chooses the minimum.

So (S) is not an independent physical restriction on $D$; it is an Occam-style tie-breaker that always picks the
smallest permitted dimension. I think a good reviewer will read this as ``we selected $3$ by decree.''

\section*{3.\ Analysis of (T): ``loops have integer linking invariants''}

Here one must be very precise, because the statement can be vacuous if not formulated carefully.

\subsection*{3.1.\ Linking number is dimension-sensitive}

A standard (homological) fact:

\underline{Statement 3:}
Let $A\subset S^D$ and $B\subset S^D$ be disjoint, closed, oriented submanifolds of dimensions $p$ and $q$.
A $\mathbb{Z}$-valued \emph{linking number} $\mathrm{lk}(A,B)\in\mathbb{Z}$ is canonically defined whenever
\begin{equation}
   p+q = D-1. 
\end{equation}


\noindent
\emph{Sketch of definition (with all dimension bookkeeping).}
If $p+q=D-1$, then $q=D-1-p$. Under mild hypotheses, $A$ bounds some oriented $(p+1)$-chain $W$ in $S^D$
(e.g.\ if $H_p(S^D)=0$ for $0<p<D$), and one sets
\begin{equation}
    \mathrm{lk}(A,B) := W\cdot B,
\end{equation}
the oriented intersection number. This is an integer precisely because
\begin{equation}
    \dim(W)+\dim(B)=(p+1)+q=(p+1)+(D-1-p)=D,
\end{equation}
so the intersection is $0$-dimensional and can be counted with signs.

\subsection*{3.2.\ Specialize to \emph{loops} vs \emph{loops}}

Now take ``loop'' to mean an embedded circle $L\subset \mathbb{R}^D$ (or in the one-point compactification $S^D$),
so $p=q=1$. The condition for loop--loop linking number to exist is:
\begin{equation}
p+q = D-1
\quad\Longleftrightarrow\quad
1+1 = D-1
\quad\Longleftrightarrow\quad
D=3.    
\end{equation}

So the precise mathematical content is:

\begin{proposition}
A nontrivial $\mathbb{Z}$-valued \emph{loop--loop} linking number exists only in $D=3$.
\end{proposition}

Proof:
As above: loop--loop linking requires $p=q=1$ and $p+q=D-1$, hence $D=3$.


\subsection*{3.3.\ Stronger statement via Alexander duality: in $D\ge 4$ there is no nontrivial linking for loops}

Let $L\subset S^D$ be an embedded circle (a ``loop''). Alexander duality says
\begin{equation}
    \widetilde H_i(S^D\setminus L) \cong \widetilde H^{D-i-1}(L).
\end{equation}
Since $L\simeq S^1$ has cohomology $H^0(S^1)\cong \mathbb{Z}$, $H^1(S^1)\cong \mathbb{Z}$, and $H^k(S^1)=0$ for $k\ge 2$,
we obtain for $i=1$:
\begin{equation}
    \widetilde H_1(S^D\setminus L) \cong \widetilde H^{D-2}(S^1).
\end{equation}
Thus:
\begin{equation}
    \widetilde H_1(S^D\setminus L)
=
\begin{cases}
\mathbb{Z}, & D-2=1 \;\;\Longleftrightarrow\;\; D=3,\\
0, & D-2\ge 2 \;\;\Longleftrightarrow\;\; D\ge 4.
\end{cases}
\end{equation}
In $D=3$, the nontrivial $H_1$ is exactly what underwrites the usual linking number.
In $D\ge 4$, $H_1(S^D\setminus L)=0$, so there is no nontrivial homological obstruction for another loop to be ``linked'' with $L$.

\subsection*{3.4.\ Consequence for the independence claim}

If $(T)$ is intended to mean:
\begin{quote}
\emph{There exist nontrivial integer-valued ambient-isotopy invariants measuring how two loops link, i.e.\ not all values are $0$.}
\end{quote}
then it is essentially equivalent to $D=3$.

Formally:
Under the intended (non-vacuous) reading of $(T)$,
\begin{equation}
    \mathcal{A}_T=\{3\}.
\end{equation}

So $(T)$ is not an ``independent constraint that happens to point toward $3$'': it \emph{is} a re-encoding of $D=3$
once ``loops'' are assumed to be the objects whose linking is to be classified.

\section*{4.\ Analysis of (K): ``Green kernel + nonprecession''}

\subsection*{4.1.\ The Green kernel in $D$ dimensions forces a $D$-dependent power law}

Assume $(K)$ includes:

\begin{quote}
\emph{The potential $V$ of a point source is the (radial) fundamental solution of the $D$-dimensional Laplacian:
$\Delta V = -\delta$.}
\end{quote}

For a radial function $V(r)$ with $r=\|x\|$, the Laplacian in $\mathbb{R}^D$ is
\begin{equation}
    \Delta V(r)=V''(r)+\frac{D-1}{r}V'(r)\qquad (r>0).
\end{equation}
Away from the source ($r\neq 0$), $\Delta V=0$, so $V$ satisfies
\begin{equation}
    V''(r)+\frac{D-1}{r}V'(r)=0.
\end{equation}
Let $W(r):=V'(r)$. Then
\begin{equation}
    W'(r)+\frac{D-1}{r}W(r)=0
\quad\Longrightarrow\quad
\frac{W'(r)}{W(r)}=-(D-1)\frac{1}{r}.
\end{equation}
Integrating,
\begin{equation}
    \ln|W(r)|=-(D-1)\ln r + \ln C
\quad\Longrightarrow\quad
W(r)=C\,r^{1-D}.
\end{equation}
Integrating again,
\begin{equation}
    V(r)=\int C\,r^{1-D}\,dr
=
\begin{cases}
\displaystyle \frac{C}{2-D}\,r^{2-D} + C_0, & D\neq 2,\\[6pt]
C\ln r + C_0, & D=2.
\end{cases}
\end{equation}
For $D\ge 3$, the Green-kernel potential therefore has the form
\begin{equation}
    V_D(r)=\kappa\, r^{2-D} \;=\; \frac{\kappa}{r^{D-2}}
\qquad (D\ge 3).
\end{equation}

\subsection*{4.2.\ Central-force orbits and why $D=3$ is built in once we add ``no precession''}

Now add the second part of $(K)$:

\begin{quote}
\emph{Bounded orbits exhibit no perihelion precession (equivalently: the apsidal angle is exactly $\pi$, giving closed Keplerian ellipses rather than rosettes).}
\end{quote}

If we use the \emph{3D} Green kernel $V(r)\propto 1/r$ when stating $(K)$, then $D=3$ was assumed. To avoid circularity, we must start with $V_D(r)\propto r^{2-D}$ and \emph{derive} $D=3$ from the no-precession requirement.\\

\underline{Here is the derivation.}\\

In any $\mathbb{R}^D$, a central force $F(r)\,\hat r$ conserves angular momentum in the sense that the motion remains in the
two-dimensional plane spanned by the initial position and velocity. Thus the orbit analysis reduces to planar polar coordinates $(r,\theta)$.

\subsubsection*{Orbit equation for a power-law potential}
Take an attractive potential
\begin{equation}
    V(r)=-\frac{k}{r^{n}},\qquad n:=D-2>0.
\end{equation}
Then
\begin{equation}
    F(r)=-\frac{dV}{dr}=-\left(-k\cdot (-n) r^{-n-1}\right)=-\frac{k n}{r^{n+1}}.
\end{equation}
Let $\ell$ be the conserved angular momentum magnitude:
\begin{equation}
    \ell = m r^2 \dot \theta.
\end{equation}
Introduce $u(\theta):=1/r(\theta)$. A standard computation gives the orbit ODE
\begin{equation}
   u''(\theta)+u(\theta) = -\frac{m}{\ell^2 u(\theta)^2}\,F\!\left(\frac{1}{u(\theta)}\right). 
\end{equation}
Substitute $F(r)=-k n/r^{n+1}$ and $r=1/u$:
\begin{equation}
   F\!\left(\frac{1}{u}\right)=-k n\,u^{n+1}. 
\end{equation}
Therefore
\begin{equation}
    u''+u
=
-\frac{m}{\ell^2 u^2}\left(-k n\,u^{n+1}\right)
=
\frac{m k n}{\ell^2}\,u^{n-1}.
\end{equation}
Recall $n=D-2$, so $n-1=D-3$:
\begin{equation}\label{eq:orbit}
u''+u = \alpha\,u^{D-3},
\qquad \alpha:=\frac{m k (D-2)}{\ell^2}.
\end{equation}

\subsubsection*{The ``no precession'' condition forces $D=3$}
Observe the key structural fact:

\begin{itemize}
\item If $D=3$, then $D-3=0$, so \eqref{eq:orbit} becomes \emph{linear}:
\begin{equation}
   u''+u=\alpha, 
\end{equation}
whose solutions are conic sections (ellipses/parabolas/hyperbolas), and bounded ones are closed ellipses: no precession.

\item If $D\neq 3$, then $u^{D-3}$ is \emph{nonlinear} in $u$. In general, nonlinear central-force orbit equations produce
apsidal precession; closed non-precessing ellipses for \emph{all} bounded orbits do not occur.
\end{itemize}

To make this quantitative (and not just ``handwavy''), linearize around a circular orbit.
A circular orbit corresponds to $u(\theta)\equiv u_0$ constant, so \eqref{eq:orbit} gives
\begin{equation}
    u_0 = \alpha u_0^{D-3}
\quad\Longrightarrow\quad
\alpha = u_0^{4-D}.
\end{equation}
Now perturb $u(\theta)=u_0+\delta(\theta)$ with $|\delta|\ll 1$. Expand:
\begin{equation}
    u^{D-3} = (u_0+\delta)^{D-3} \approx u_0^{D-3} + (D-3)u_0^{D-4}\delta.
\end{equation}
Plug into \eqref{eq:orbit} and use $\alpha u_0^{D-3}=u_0$ and $\alpha u_0^{D-4}=1$ (since $\alpha=u_0^{4-D}$):
\begin{equation}
    (u_0+\delta)'' + (u_0+\delta)
\approx
\alpha\left(u_0^{D-3} + (D-3)u_0^{D-4}\delta\right)
=
u_0 + (D-3)\delta.
\end{equation}
Cancel $u_0$ and obtain the linearized ODE
\begin{equation}
    \delta'' + \delta = (D-3)\delta
\quad\Longleftrightarrow\quad
\delta'' + (4-D)\,\delta = 0.
\end{equation}
Thus (as a function of $\theta$) the radial perturbation oscillates with frequency
\begin{equation}
   \omega = \sqrt{4-D}. 
\end{equation}

\underline{No-precession criterion.}
For no precession, the angular advance between successive periapses must be exactly $\pi$. In the linearized regime, successive periapses occur once per half-period of $\delta$, so the apsidal angle is
\begin{equation} \Delta\theta_{\text{apsis}}=\frac{\pi}{\omega}=\frac{\pi}{\sqrt{4-D}}.
\end{equation}
No precession means $\Delta\theta_{\text{apsis}}=\pi$, hence $\sqrt{4-D}=1$, i.e.
\begin{equation}
    4-D=1\quad\Longleftrightarrow\quad D=3.
\end{equation}


Under the ``Green kernel + no precession'' package $(K)$, one derives $D=3$. Equivalently, as a constraint on dimension,
\begin{equation}
    \mathcal{A}_K=\{3\}\qquad (\text{for }D\ge 3).
\end{equation}


\medskip
\noindent
The point is: Once we adopt (i) the Laplacian Green kernel in $D$ and (ii) the Keplerian ``no precession'' orbit property,
the dimension is forced to be $3$. So $(K)$ is \emph{already a $D=3$ selection mechanism}, not an ``independent check.''

\section*{5.\ The core logical problem: each constraint already gives $\{3\}$}

From Sections 2--4, under the intended (non-vacuous) readings:
\begin{equation}
    \mathcal{A}_T=\{3\},\qquad \mathcal{A}_K=\{3\},\qquad \mathcal{A}_S=\{3\}.
\end{equation}

Because each allowed set is already a singleton, the triple-intersection argument
\begin{equation}
    \mathcal{A}_T\cap \mathcal{A}_K\cap \mathcal{A}_S = \{3\}
\end{equation}
is not evidentially meaningful. It does not exhibit \emph{convergent independent constraints};
it exhibits the same conclusion re-stated three times in three disguises.

\section*{6.\ How to fix ?}

We have two options.

\subsection*{6.1.\ Option A: stop claiming independence}
We keep $(T),(K),(S)$, but re-label them as:
\begin{quote}
\emph{``Three internal consistency checks (topological, dynamical, and complexity/Occam) that are all satisfied when $D=3$.''}
\end{quote}
Then we present \emph{one} of them as the actual dimension-selection argument, and the others as cross-checks.
This removes the circularity claim because we no longer pretend the three are independent selectors.

\subsection*{6.2.\ Option B (stronger): replace $(T),(K),(S)$ by three constraints with nontrivial allowed sets whose intersection is $\{3\}$}

\medskip
\noindent\textbf{Constraint A (topological; yields odd $D$).}
Do \emph{not} hard-code ``loops'' ($p=1$). Instead assume only:

\begin{quote}
There exists a single species of $p$-dimensional extended object whose pairwise linking is classified by a nontrivial integer linking number
between \emph{two objects of the same dimension $p$}.
\end{quote}

As reviewed in \S3.1, same-dimension linking requires
\begin{equation}
   p+p = D-1 \quad\Longleftrightarrow\quad D = 2p+1, 
\end{equation}
hence $D$ must be odd. Therefore
\begin{equation}
   \mathcal{A}_A = \{1,3,5,7,\dots\}. 
\end{equation}

\medskip
\noindent\textbf{Constraint B (dynamical stability; yields $D\le 3$).}

\begin{quote}
A long-range attractive potential generated by the Laplacian Green kernel admits \emph{stable} bound (e.g.\ stable circular) orbits.
\end{quote}

Using $V(r)=-k/r^{n}$ with $n=D-2$ and the effective potential
\begin{equation}
    U_{\mathrm{eff}}(r)=\frac{\ell^2}{2mr^2}-\frac{k}{r^n},
\end{equation}
a circular orbit occurs at a critical point $U'_{\mathrm{eff}}(r_0)=0$ and is stable iff $U''_{\mathrm{eff}}(r_0)>0$.
Compute:
\begin{equation}
    U'_{\mathrm{eff}}(r)= -\frac{\ell^2}{m r^3} + k n\, r^{-n-1}.
\end{equation}
Setting $U'(r_0)=0$ gives
\begin{equation}
    \frac{\ell^2}{m} = k n\, r_0^{2-n}.
\end{equation}
Next,
\begin{equation}
   U''_{\mathrm{eff}}(r)= \frac{3\ell^2}{m r^4} - k n(n+1)\,r^{-n-2}. 
\end{equation}
Substitute $\ell^2/m = k n r_0^{2-n}$:
\begin{equation}
  U''_{\mathrm{eff}}(r_0)
=
\frac{3k n r_0^{2-n}}{r_0^4} - k n(n+1)\,r_0^{-n-2}
=
k n(3-(n+1))\,r_0^{-n-2}
=
k n(2-n)\,r_0^{-n-2}.  
\end{equation}
For an attractive force, $k>0$ and $n>0$, so stability requires $2-n>0$, i.e.
\begin{equation}
    n<2 \quad\Longleftrightarrow\quad D-2<2 \quad\Longleftrightarrow\quad D<4.
\end{equation}
Hence
\begin{equation}
    \mathcal{A}_B = \{1,2,3\}.
\end{equation}

\medskip
\noindent\textbf{Constraint C (geometric; yields $D\ge 3$).}
A simple, mathematically sharp requirement that excludes $D=1,2$ without hard-coding $D=3$ is:

\begin{quote}
The rotation group $SO(D)$ is non-abelian (so that ``independent'' rotations do not commute).
\end{quote}

It is standard that $SO(1)$ is trivial and $SO(2)\cong S^1$ is abelian, while $SO(D)$ is non-abelian for $D\ge 3$.
Thus
\begin{equation}
    \mathcal{A}_C = \{3,4,5,\dots\}.
\end{equation}

\medskip
\noindent\textbf{Now the convergence is real:}
\[
\mathcal{A}_A\cap \mathcal{A}_B\cap \mathcal{A}_C
=
\{1,3,5,\dots\}\cap\{1,2,3\}\cap\{3,4,5,\dots\}
=
\{3\}.
\]


Constraints A, B, C are each individually non-singleton, but their intersection is $\{3\}$.
This is exactly the ``independent sets whose intersection is $D=3$'' pattern any referee will recognize as genuinely convergent.


\subsection*{6.3.\ What to do with the original $(T),(K),(S)$}
After adopting A, B, C as the \emph{independent} selectors, we can still keep the original $(T),(K),(S)$ in an appendix as:

\begin{itemize}
\item a specialization of A to the case $p=1$ (``loops''),
\item a strengthening of B (stable orbits) to the sharper no-precession condition (which then singles out $D=3$),
\item and an Occam/complexity tie-breaker that is only meaningful \emph{after} the admissible set has already been reduced to $\{3,5,7,\dots\}$.
\end{itemize}

That re-framing preserves the intuition of our narrative while removing the logical circularity.
It also matches the structure: each independent constraint yields a \emph{set} of candidate dimensions; only the intersection forces $3$.

\end{document}
