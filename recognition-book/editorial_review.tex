\documentclass[11pt]{article}
\usepackage[utf8]{inputenc}
\usepackage[T1]{fontenc}
\usepackage[margin=1in]{geometry}
\usepackage{enumitem}
\usepackage{xcolor}
\usepackage{titlesec}
\usepackage{hyperref}
\usepackage{longtable}
\usepackage{booktabs}

\hypersetup{
    colorlinks=true,
    linkcolor=blue,
    urlcolor=blue
}

\titleformat{\section}{\Large\bfseries}{\thesection}{1em}{}
\titleformat{\subsection}{\large\bfseries}{\thesubsection}{1em}{}
\titleformat{\subsubsection}{\normalsize\bfseries}{\thesubsubsection}{1em}{}

\definecolor{excellent}{RGB}{0,100,0}
\definecolor{good}{RGB}{0,0,139}
\definecolor{concern}{RGB}{178,34,34}
\definecolor{aimarker}{RGB}{128,0,128}

\newcommand{\excellent}[1]{\textcolor{excellent}{\textbf{EXCELLENT:} #1}}
\newcommand{\good}[1]{\textcolor{good}{\textbf{GOOD:} #1}}
\newcommand{\concern}[1]{\textcolor{concern}{\textbf{CONCERN:} #1}}
\newcommand{\aimarker}[1]{\textcolor{aimarker}{\textbf{AI MARKER:} #1}}

\title{\textbf{Editorial Review}\\[0.5em]\Large \textit{Recognition: A Brief History of Us}\\[0.5em]\normalsize by Jonathan Washburn}
\author{Reader Report\\Pre-Roughdraft Assessment}
\date{December 2025}

\begin{document}

\maketitle

\tableofcontents
\newpage

% ============================================
\section{Executive Summary}
% ============================================

\subsection{Overall Assessment}

This manuscript attempts something extraordinarily ambitious: deriving all of physics, consciousness, morality, and meaning from a single axiom (``Nothing cannot recognize itself''). At its best, it achieves moments of genuine intellectual excitement and accessible science writing that recalls Carl Sagan or Brian Greene. At its weakest, it slides into repetitive structures, self-help platitudes, and prose patterns characteristic of AI-assisted writing.

\subsection{Key Strengths}
\begin{itemize}[leftmargin=2em]
    \item \textbf{Narrative voice}: When the book is working, it has a distinctive, confident voice that balances accessibility with intellectual seriousness
    \item \textbf{Historical integration}: The Pacioli/Venice section, the karma parallels, the Pauli/137 mystery---these humanize the physics beautifully
    \item \textbf{Conceptual clarity}: Complex ideas (ledgers, Gray codes, phase coherence) are explained with effective analogies
    \item \textbf{Moral philosophy sections}: The ethics material (love, justice, forgiveness) is genuinely philosophically interesting
    \item \textbf{Ambition}: The sheer scope is admirable and will attract readers seeking unified theories of everything
\end{itemize}

\subsection{Key Concerns}
\begin{itemize}[leftmargin=2em]
    \item \textbf{AI writing patterns}: Repetitive structures, formulaic transitions, parallel constructions that signal machine-assisted composition
    \item \textbf{Tonal inconsistency}: The book shifts from rigorous derivation to wellness/self-help advice, especially in Part V
    \item \textbf{Overuse of signature lines}: ``This is not a metaphor. This is the physics.'' appears too frequently
    \item \textbf{Bullet-list creep}: Many sections break into lists where prose would be more appropriate for a trade book
    \item \textbf{Repetition of core concepts}: The same ideas are restated multiple times without adding new insight
\end{itemize}

\subsection{Recommendation}

\textbf{Conditional acceptance for further development.} The manuscript has strong bones and several brilliant sections. It requires significant revision to:
\begin{enumerate}
    \item Eliminate AI writing markers through prose variation
    \item Resolve the tension between physics book and spiritual self-help
    \item Cut repetition (estimate 15--20\% reduction needed)
    \item Either substantially revise or relocate Part V (Practices)
\end{enumerate}

\newpage

% ============================================
\section{Detailed Chapter-by-Chapter Analysis}
% ============================================

\subsection{Front Matter \& Introduction (Lines 1--400)}

\subsubsection{What Works}

\excellent{The opening epigraphs (John 1:1, Upanishads, Meta-Principle) create an intellectual triptych that signals the book's ambition without being pretentious.}

\good{The dedication (``The answer was always inside you. Now we can prove it.'') is memorable and sets the tone perfectly.}

\good{``A Note to the Reader'' is excellently crafted---honest about what the book is and isn't, inviting without being cloying.}

\subsubsection{What Needs Work}

\concern{The numbered list at lines 869--876 (``We will: 1) Build... 2) Show... 3) Open... 4) Prove...'') reads like an outline. Weave this into prose.}

\concern{Early bullet points (lines 897--902, 927--931) break the narrative flow too soon. Save lists for reference material.}

\subsubsection{Pacing}

Strong opening. The reader is engaged and oriented. Pacing: \textbf{GOOD}.

\subsection{Part I: The Architecture (Lines 400--2000)}

\subsubsection{What Works}

\excellent{The Pacioli/Venice section (lines 1054--1065) is the book at its best. This is engaging, humanizing, historically grounded science writing. It transforms abstract accounting principles into living history.}

\excellent{The karma/double-entry parallel (lines 1068--1077) is unexpected and illuminating. This will resonate with readers across traditions.}

\good{The ``cube of lights'' metaphor for the Gray code (line 1974) makes abstract mathematics visual and accessible.}

\good{The Escher reference (``hands drawing themselves'') is effective and returns productively.}

\subsubsection{What Needs Work}

\concern{The technical sections on Gray code walks (lines 1970--2000) get dense. Consider moving some detail to a ``For the Mathematically Curious'' box or appendix.}

\concern{``Conservation'' is explained multiple times in slightly different ways. Trust the reader to remember after the first clear explanation.}

\aimarker{Heavy use of ``Why must this be so? Because...'' structure (appears 7+ times in Part I).}

\subsubsection{Pacing}

Uneven. Historical sections: EXCELLENT. Technical derivations: SLOW. Consider alternating more systematically between narrative and derivation.

\subsection{Part II: Constants \& Particles (Lines 2000--3500)}

\subsubsection{What Works}

\excellent{The 137/Pauli conclusion (lines 2938--2952) is masterful. ``The number that haunted Pauli was not hiding a secret. It was announcing a structure.'' This is the kind of line readers will remember.}

\excellent{The cumulative rhetoric at lines 2949--2951 (``If you accept X, you accept Y...'') builds genuine philosophical momentum.}

\good{The anthropic principle discussion (2942--2945) is fair, nuanced, and doesn't strawman the opposition.}

\good{``The particle zoo will look less like a random menagerie and more like a census of allowed addresses'' (line 2988)---strong metaphor.}

\subsubsection{What Needs Work}

\concern{The ``ladder'' metaphor, while effective, becomes repetitive. After the third explanation, trust that readers understand.}

\concern{The consciousness threshold section (lines 3036--3039) is asserted rather than derived. This is a significant claim that needs more support or explicit acknowledgment of its speculative status.}

\aimarker{``Let us take it apart, word by word'' (line 2996)---mechanical scaffolding that could be cut.}

\subsubsection{Pacing}

The Pauli/137 payoff is worth the buildup. The particles chapter opens strong. Pacing: \textbf{GOOD to EXCELLENT} in narrative sections.

\subsection{Part III: Ethics as Physics (Lines 3500--5100)}

\subsubsection{What Works}

\excellent{The Love section (lines 3928--3976) is philosophically rich and emotionally resonant. ``When you love, you are not merely following your heart. You are participating in the geometry of existence.''---this will be quoted.}

\excellent{``Love sometimes hurts'' explanation (line 3939) is psychologically astute and shows the framework can handle complexity.}

\excellent{``Justice and mercy are not opposites'' (line 4035)---genuinely insightful reframing.}

\good{The forgiveness-as-skew-transfer section (4048--4100) grounds an abstract concept in memorable physics.}

\good{The practical audit walkthrough (lines 4910--5000) demonstrates the framework's applicability. Skeptical readers need this.}

\subsubsection{What Needs Work}

\concern{The justice section opens with a strawman: ``This sounds wrong. Where are the scales?'' This ``common understanding is backwards'' move is becoming a pattern.}

\concern{Lines 4901--4907 feel defensive rather than confident. Let the framework speak for itself.}

\concern{The Plan A/B/B-prime/C walkthrough is useful but mechanical. Consider making the example more vivid and specific.}

\aimarker{``This is why X is fundamental'' appears as a paragraph closer 5+ times in Part III.}

\subsubsection{Pacing}

The ethics material is among the strongest in the book. Pacing: \textbf{GOOD to EXCELLENT}.

\subsection{Part IV: The Soul (Lines 5100--6100)}

\subsubsection{What Works}

\excellent{The ``Can a Machine Have a Soul?'' bigquestion box (lines 5095--5101) is provocative and well-placed. ``They are smart, but there is no one home.''---sharp and memorable.}

\excellent{The Death chapter opening (lines 5940--5946) is genuinely moving and well-written.}

\excellent{The photon/flame analogy (lines 6032--6040) is brilliant physics communication---scientifically grounded and poetic.}

\excellent{``Death is the photon'' (line 6050)---a line people will remember.}

\excellent{The superconductor analogy (lines 6058--6062) is excellent and will resonate with readers who have science background.}

\excellent{The shoebox of letters passage (lines 6090--6104) is genuinely moving and humanizes the abstract claims about what survives death.}

\subsubsection{What Needs Work}

\concern{``The Tibetans knew'' (lines 6022--6026) is too credulous. Acknowledge this is interpretive, not established fact.}

\concern{Lines 6004--6011 acknowledge uncertainty (``we do not know for certain'') but then proceed as if we do. This tension undermines credibility.}

\concern{``Light Memory state'' terminology feels invented/New Age-ish and may alienate skeptical readers. Consider whether you need this term or can describe the concept without naming it.}

\concern{``Zero cost'' persistence is asserted but the physics of WHY is hand-wavy. This is a central claim that needs more support.}

\subsubsection{Pacing}

Strong when narrative, weaker when making unsupported claims. Pacing: \textbf{UNEVEN}.

\subsection{Part V: Practices (Lines 6100--7900)}

\subsubsection{What Works}

\good{The breath/spirit etymology section (lines 6905--6910) is well-researched.}

\good{The Buddha/meditation opening (line 6915) is graceful.}

\good{The tai chi description (lines 6983--6985) is evocative.}

\good{The ``Om'' opening (lines 7047--7051) is dramatic.}

\good{The resonance physics explanation (lines 7055--7062) is solid.}

\subsubsection{What Needs Work}

\concern{This entire section reads like a different book. The rigor of Parts I--III gives way to wellness guidance that could appear in any meditation manual.}

\concern{Lines 7033--7041 are generic wellness advice: ``Walk consciously. Feel your feet on the ground.'' This is not distinctive.}

\concern{The ``minimum effective dose'' language (line 6965) is self-help jargon that undermines the physics framing.}

\concern{All practice sections follow the same structure: physics $\rightarrow$ mechanism $\rightarrow$ evidence $\rightarrow$ application $\rightarrow$ ``deeper meaning.'' This parallelism signals formulaic composition.}

\concern{``All paths lead to the same place: coherence'' (line 6955) feels pat.}

\subsubsection{Pacing}

Slow and repetitive. Pacing: \textbf{WEAK}.

\subsubsection{Structural Recommendation}

Part V is the weakest section of the book. Options:
\begin{enumerate}
    \item \textbf{Drastically cut}: Reduce to a single chapter summarizing how the framework connects to contemplative traditions
    \item \textbf{Relocate to appendix}: Keep the material but signal it's supplementary
    \item \textbf{Spin off}: Make practices a separate companion volume for interested readers
\end{enumerate}

The physics reader who was engaged by Parts I--III may feel the book loses its way here. The spiritual reader who wants practices may find Parts I--III too technical. The current structure tries to serve both and satisfies neither fully.

\subsection{Back Matter: Glossary \& Conclusion}

\subsubsection{What Works}

\good{The glossary is well-organized and clear. Listing terms in order of appearance rather than alphabetically is a smart choice.}

\good{``Welcome home'' as final line is moving and appropriate.}

\subsubsection{What Needs Work}

\concern{The glossary could include page references for where each term is first introduced.}

\concern{No index is provided (noted as ``to be written''). Essential for a book of this scope.}

\newpage

% ============================================
\section{AI Writing Markers: Detailed Analysis}
% ============================================

The manuscript shows several patterns characteristic of AI-assisted writing. These should be systematically revised to create more varied, human-feeling prose.

\subsection{Formulaic Structures}

\subsubsection{The ``This is not X. It is Y.'' Construction}

This pattern appears approximately 30+ times throughout the manuscript. Examples:

\begin{quote}
``This is not a metaphor. This is the physics.'' \\
``This is not superstition. It is a claim about structure.'' \\
``This is not comfort for its own sake. It is an attempt to understand.'' \\
``Time is not a backdrop. It is counting.''
\end{quote}

\textbf{Recommendation}: Vary the construction. Use it sparingly for emphasis. When it appears multiple times per chapter, it loses force.

\subsubsection{The ``Why must this be so? Because...'' Pattern}

This rhetorical question structure appears frequently throughout Parts I and II.

\textbf{Recommendation}: Mix rhetorical questions with declarative statements. Not every explanation needs to be framed as Q\&A.

\subsubsection{The ``Let us take it apart'' / ``Let us walk through'' Scaffolding}

Meta-commentary about what the text is doing:

\begin{quote}
``Let us take it apart, word by word.'' \\
``Let us walk through a stylized example.'' \\
``In the sections ahead, we will...''
\end{quote}

\textbf{Recommendation}: Cut most of these. Trust the reader to follow without signposting every transition.

\subsubsection{The ``The deeper point'' / ``The deeper meaning'' Closer}

Many sections end with a paragraph beginning ``The deeper point is...'' or ``The deeper meaning is...''

\textbf{Recommendation}: Vary section endings. Sometimes end with a concrete image. Sometimes end with a question. Sometimes just stop.

\subsection{Parallel Structure Overuse}

AI writing often produces highly parallel constructions. Examples:

\begin{quote}
``Concentration practices focus attention... \\
Insight practices observe whatever arises... \\
Loving-kindness practices generate feelings... \\
Movement meditation uses the body...''
\end{quote}

This parallelism in the meditation section (and throughout Part V) creates a mechanical feel.

\textbf{Recommendation}: Break parallelism deliberately. Vary sentence length. Insert asides or examples that disrupt the pattern.

\subsection{Defensive Anticipation of Objections}

The book frequently anticipates and dismisses objections in a formulaic way:

\begin{quote}
``This sounds wrong. But consider...'' \\
``That is the common understanding. And it is almost entirely backwards.'' \\
``You might object that... But the framework shows...''
\end{quote}

\textbf{Recommendation}: Not every section needs to begin with ``here's what you think, and you're wrong.'' Sometimes just present the argument positively.

\subsection{Overuse of ``The framework says/predicts/shows''}

The phrase ``The framework'' or ``The recognition framework'' appears approximately 100+ times.

\textbf{Recommendation}: Vary the references. Use ``this picture,'' ``the ledger model,'' ``recognition science,'' or simply state the claim directly without attribution to ``the framework.''

\newpage

% ============================================
\section{Repetition Analysis}
% ============================================

\subsection{Concepts Explained Multiple Times}

The following concepts are re-explained more times than necessary:

\begin{longtable}{p{4cm}p{2.5cm}p{6cm}}
\toprule
\textbf{Concept} & \textbf{Occurrences} & \textbf{Recommendation} \\
\midrule
Conservation/Double-entry & 8+ times & Explain fully once, then refer back \\
Golden ratio as unique & 6+ times & Trust reader to remember \\
Eight-tick cycle & 7+ times & After initial explanation, use without re-deriving \\
``No duplication, no omission'' & 5+ times & Establish once, reference briefly \\
Ladder of scales & 6+ times & The metaphor is clear; don't overexplain \\
Phase coherence & 5+ times & Consolidate explanations \\
\bottomrule
\end{longtable}

\subsection{Phrases That Recur Too Frequently}

\begin{longtable}{p{6cm}p{2cm}p{4.5cm}}
\toprule
\textbf{Phrase} & \textbf{Count} & \textbf{Recommendation} \\
\midrule
``This is not a metaphor'' & 5+ & Use once, then trust it's established \\
``The recognition framework'' & 100+ & Vary with synonyms \\
``In the framework'' & 50+ & Often cuttable entirely \\
``This is what X means'' & 20+ & Vary construction \\
``The answer is...'' & 15+ & Mix with other transitions \\
``Let us...'' & 15+ & Cut most instances \\
\bottomrule
\end{longtable}

\newpage

% ============================================
\section{Pacing \& Engagement Map}
% ============================================

The following table maps sections by engagement level, with recommendations:

\begin{longtable}{p{4cm}p{2.5cm}p{6cm}}
\toprule
\textbf{Section} & \textbf{Engagement} & \textbf{Notes} \\
\midrule
Front matter / Note to Reader & HIGH & Keep as is \\
Meta-principle introduction & HIGH & Strong opening \\
Ledger/double-entry basics & MEDIUM & Necessary but could be tighter \\
Pacioli/Venice section & \textbf{EXCELLENT} & Highlight of Part I \\
Karma connection & \textbf{EXCELLENT} & Keep and possibly expand \\
Gray code technical details & LOW & Move detail to boxes/appendix \\
Microperiod derivation & MEDIUM-LOW & Tighten significantly \\
The Cosmic Pulse & MEDIUM-HIGH & Good rhythm, minor cuts \\
137/Pauli narrative & \textbf{EXCELLENT} & Highlight of Part II \\
Particle masses & GOOD & Clear and engaging \\
Love as equilibration & \textbf{EXCELLENT} & Highlight of Part III \\
Justice / Forgiveness & GOOD-EXCELLENT & Strong philosophy \\
Audit walkthrough & MEDIUM-HIGH & Necessary, slightly mechanical \\
Can a Machine Have Soul? & HIGH & Provocative, well-placed \\
Death as phase transition & HIGH & Strong opening \\
Photon/superconductor analogies & \textbf{EXCELLENT} & Physics at its best \\
What survives death & HIGH & Emotionally resonant \\
Light Memory state & MEDIUM & Hand-wavy, needs grounding \\
Breathing practices & LOW-MEDIUM & Generic wellness \\
Meditation & LOW & Could appear anywhere \\
Movement practices & LOW & Not distinctive \\
Sound/chanting & MEDIUM & Better than other practices \\
Fasting & MEDIUM & Interesting historical framing \\
Glossary & GOOD & Functional and clear \\
Conclusion & MEDIUM-HIGH & ``Welcome home'' works \\
\bottomrule
\end{longtable}

\newpage

% ============================================
\section{Structural Recommendations}
% ============================================

\subsection{The Two-Book Problem}

The manuscript is currently trying to be two books:

\begin{enumerate}
    \item \textbf{Book A}: A rigorous derivation of physics, consciousness, and ethics from a single axiom (Parts I--IV minus practices)
    \item \textbf{Book B}: A spiritual guide to living with practices drawn from contemplative traditions (Part V and scattered ``how to live'' sections)
\end{enumerate}

These books have different audiences, different tones, and different standards of evidence. The current manuscript alternates between them, which may satisfy neither audience fully.

\subsubsection{Option 1: Lean into Book A}

Cut Part V entirely or reduce to a brief epilogue. Keep the book focused on derivation. Let implications for practice emerge naturally without prescribing specific techniques. This book would appeal to readers of \textit{The Elegant Universe}, \textit{A Brief History of Time}, or \textit{Gödel, Escher, Bach}.

\subsubsection{Option 2: Lean into Book B}

Shorten the technical derivations and expand the ``what this means for your life'' sections. Make practices central rather than appendix-like. This book would appeal to readers of \textit{The Tao of Physics}, \textit{Be Here Now}, or \textit{The Untethered Soul}.

\subsubsection{Option 3: Two Volumes}

Publish Part A as ``Recognition: The Theory'' and Part B as ``Recognition: The Practice.'' This is the cleanest solution if you want to reach both audiences.

\subsection{Chapter Restructuring}

\subsubsection{Part I Suggestions}

\begin{itemize}
    \item Consolidate the Gray code / microperiod / posting period sections into one tighter chapter
    \item Move technical details to ``For the Mathematically Curious'' boxes (the infrastructure exists but is underused)
    \item Expand the Pacioli narrative---it's working, give it more room
\end{itemize}

\subsubsection{Part II Suggestions}

\begin{itemize}
    \item The 137 / Pauli material should be a set piece---consider making it a standalone chapter
    \item The particle masses section is good but could use a vivid example (``imagine if all car models had to weigh exact multiples of a Toyota Corolla...'')
\end{itemize}

\subsubsection{Part III Suggestions}

\begin{itemize}
    \item The virtues could each have their own brief section rather than being listed
    \item The audit walkthrough is important---consider making it even more vivid with a specific, emotionally resonant example
\end{itemize}

\subsubsection{Part IV Suggestions}

\begin{itemize}
    \item Acknowledge uncertainty more explicitly about the ``Light Memory state''---intellectual honesty will build trust
    \item The ``what survives death'' section is strong; the ``what it's like'' section is weak
\end{itemize}

\subsubsection{Part V Suggestions}

\begin{itemize}
    \item If keeping: Cut by 50\%, remove generic wellness advice, keep only what's distinctive to the framework
    \item If removing: Replace with a brief chapter on ``implications for how we live'' that gestures at practices without prescribing them
\end{itemize}

\newpage

% ============================================
\section{Line-Level Editing Priorities}
% ============================================

\subsection{High Priority Cuts}

\begin{itemize}
    \item All instances of ``This is not a metaphor. This is the physics.'' except the first
    \item All ``Let us walk through...'' and ``Let us take apart...'' scaffolding
    \item Most ``In the sections ahead, we will...'' previews
    \item Redundant explanations of conservation, golden ratio, eight-tick cycle after first full treatment
    \item Generic wellness advice in Part V (``Walk consciously. Feel your feet.'')
\end{itemize}

\subsection{High Priority Additions}

\begin{itemize}
    \item More vivid, specific examples in the audit walkthrough
    \item Explicit acknowledgment of speculative status for consciousness/death claims
    \item More historical/biographical material like the Pacioli section
    \item Index (noted as ``to be written'')
    \item Page references in glossary
\end{itemize}

\subsection{Prose Variation Needed}

\begin{itemize}
    \item Break parallel structures in Part V practice sections
    \item Vary sentence length more (currently too uniform)
    \item Mix declarative statements with rhetorical questions (currently Q\&A heavy)
    \item Add more concrete images and fewer abstract formulations
    \item Insert occasional humor or lightness (the book is relentlessly earnest)
\end{itemize}

\newpage

% ============================================
\section{Specific Outstanding Lines}
% ============================================

These lines are highlights that should be preserved and possibly emphasized:

\begin{enumerate}
    \item ``Nothing cannot recognize itself.'' (The axiom---perfect)
    \item ``Without double entry, Pacioli wrote, a merchant could not sleep peacefully at night.'' (Historical texture)
    \item ``The number that haunted Pauli was not hiding a secret. It was announcing a structure.'' (Payoff line)
    \item ``When you love, you are not merely following your heart. You are participating in the geometry of existence.'' (Quotable)
    \item ``They are smart, but there is no one home.'' (On AI---sharp)
    \item ``Death is the photon.'' (Memorable)
    \item ``The Buddha did not become enlightened. He stopped obscuring what was always there.'' (Beautiful)
    \item ``Welcome home.'' (Perfect ending)
\end{enumerate}

\newpage

% ============================================
\section{Final Assessment}
% ============================================

\subsection{What This Book Could Be}

At its best, this manuscript could join the ranks of ambitious popular physics books that also grapple with meaning: \textit{The Tao of Physics}, \textit{Gödel, Escher, Bach}, \textit{The Elegant Universe}. It has moments of genuine insight, accessible science writing, and philosophical depth.

\subsection{What Currently Holds It Back}

\begin{enumerate}
    \item \textbf{AI writing patterns} that make the prose feel mechanical
    \item \textbf{Repetition} that suggests the book doesn't trust its readers
    \item \textbf{Tonal inconsistency} between rigorous derivation and wellness advice
    \item \textbf{Part V} as currently structured doesn't serve either audience well
    \item \textbf{Overconfident claims} about consciousness and death that need qualification
\end{enumerate}

\subsection{Path Forward}

\begin{enumerate}
    \item Systematic revision for prose variation and AI marker elimination
    \item Cut 15--20\% through repetition reduction
    \item Resolve the two-book tension (recommend Option 1 or 3)
    \item Add more historical/narrative material like the Pacioli section
    \item Qualify speculative claims explicitly
    \item Complete back matter (index, etc.)
\end{enumerate}

\subsection{Estimated Revision Scope}

\begin{itemize}
    \item Current word count: ~70,000 words (estimated from line count)
    \item Target after cuts: ~55,000--60,000 words
    \item Revision intensity: \textbf{SUBSTANTIAL} (sentence-level throughout)
    \item Recommended timeline: 3--4 months for thorough revision
\end{itemize}

\vspace{2em}

\begin{center}
\rule{0.5\textwidth}{0.4pt}
\end{center}

\vspace{1em}

\textit{This review was prepared as a pre-roughdraft assessment. The manuscript shows significant promise and could, with dedicated revision, become an important contribution to popular science writing. The framework itself is intriguing; the task now is to let it shine through prose that matches its ambition.}

\end{document}
