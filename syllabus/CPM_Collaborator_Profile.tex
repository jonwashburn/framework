\documentclass[11pt]{article}
\usepackage[margin=1in]{geometry}
\usepackage{helvet}
\usepackage{parskip}
\usepackage{enumitem}
\usepackage[hidelinks]{hyperref}
\usepackage{booktabs}
\usepackage{xcolor}

\renewcommand{\familydefault}{\sfdefault}

\newcommand{\memoto}[1]{\textbf{To:} #1 \\}
\newcommand{\memofrom}[1]{\textbf{From:} #1 \\}
\newcommand{\memodate}[1]{\textbf{Date:} #1 \\}
\newcommand{\memosubject}[1]{\textbf{Subject:} #1 \\}

\definecolor{tier1}{RGB}{0,100,0}
\definecolor{tier2}{RGB}{0,50,140}
\definecolor{tier3}{RGB}{120,80,0}

\title{INTERNAL MEMORANDUM}
\date{}
\author{}

\begin{document}

\noindent
\memoto{Research Team}
\memofrom{Jonathan Washburn, Recognition Physics Institute}
\memodate{\today}
\memosubject{Collaborator Profile for the CPM Paper --- Who Should Work on This?}

\noindent\rule{\textwidth}{1pt}

\section*{Executive Summary}

This memo identifies the mathematician / physicist specialty profile best
suited to collaborate on, review, or extend
\textit{The Coercive Projection Method: Axioms, Theorems, and Applications}.
The recommendation is based on the paper's actual mathematical content,
not its RS framing.

\section{The Paper's Core Mathematics}

The CPM paper's machinery decomposes into four layers:

\begin{enumerate}[nosep]
\item \textbf{Convex optimisation on structured cones.}
      Projection bounds, $\varepsilon$-nets on Grassmannians,
      rank-one / Hermitian estimates.
\item \textbf{Coercivity inequalities.}
      Energy gaps controlling defect with explicit constants
      $(C_{\mathrm{proj}},\, K_{\mathrm{net}},\, C_{\mathrm{eng}})$.
\item \textbf{Local-to-global aggregation.}
      Dispersion bounds, finite covering arguments,
      singular-series lower bounds.
\item \textbf{Four domain instantiations.}
      Calibrated currents (Hodge), circle method (Goldbach),
      boundary certificates (Riemann Hypothesis),
      critical Sobolev estimates (Navier--Stokes).
\end{enumerate}

\section{Best-Fit Specialties}

\subsection*{\textcolor{tier1}{Tier 1 --- Direct match}}

\begin{description}[style=nextline, leftmargin=2em, nosep, font=\bfseries]
\item[Functional analysis / Operator theory]
  The Schur certificates, bounded-real lemmas, KYP/LMI machinery,
  and the entire RSA audit architecture live here.  Someone who works
  with Hardy spaces, Pick interpolation, or dissipative operators
  would read this paper natively.

\item[Convex geometry / Geometric measure theory (GMT)]
  The calibrated cone structure, $\varepsilon$-net covering arguments,
  and the Hodge instantiation (calibrated currents, comass minimisation)
  are squarely in this field.  Think someone in the
  Almgren--De~Lellis--Spadaro lineage.

\item[PDE / Calculus of variations]
  The coercivity-from-energy-gap pattern is bread and butter for
  someone who works on regularity theory, $\Gamma$-convergence, or
  variational methods.  The Navier--Stokes instantiation speaks
  directly to this audience.
\end{description}

\subsection*{\textcolor{tier2}{Tier 2 --- Strong overlap}}

\begin{description}[style=nextline, leftmargin=2em, nosep, font=\bfseries]
\item[Analytic number theory]
  The Goldbach/RH instantiations use the circle method, major/minor
  arc decomposition, large sieve inequalities, and $L$-function
  boundary behaviour.  A number theorist in the Vaughan / Iwaniec
  tradition would engage with half the paper immediately.

\item[Control theory / Systems theory]
  The bounded-real lemma, state-space realisations, Lyapunov / storage
  certificates, and the LMI feasibility formulation are standard tools
  for a controls mathematician.  They would see CPM as a new
  application of robust control machinery.

\item[Optimisation / Semidefinite programming]
  The finite certification step (checking that a semidefinite matrix
  inequality holds) is a computational convex optimisation problem.
  Someone in the Vandenberghe / Boyd tradition would see this as a
  natural application.
\end{description}

\subsection*{\textcolor{tier3}{Tier 3 --- Good fit with some translation}}

\begin{description}[style=nextline, leftmargin=2em, nosep, font=\bfseries]
\item[Complex geometry / Algebraic geometry]
  For the Hodge instantiation specifically.  Someone who works on
  Hodge theory, calibrations, or algebraic cycles.

\item[Mathematical physics / Spectral theory]
  The sensor-as-reciprocal pattern and the Cayley transform are
  natural to someone who works with resolvents, scattering theory,
  or spectral zeta functions.
\end{description}

\section{Summary Table}

\begin{center}
\renewcommand{\arraystretch}{1.3}
\begin{tabular}{@{}lll@{}}
\toprule
\textbf{Tier} & \textbf{Specialty} & \textbf{Paper entry point} \\
\midrule
\textcolor{tier1}{1} & Functional analysis / Operator theory
    & Schur certificates, KYP/LMI \\
\textcolor{tier1}{1} & Convex geometry / GMT
    & Calibrated cones, $\varepsilon$-nets, Hodge case \\
\textcolor{tier1}{1} & PDE / Calculus of variations
    & Coercivity inequalities, Navier--Stokes case \\
\midrule
\textcolor{tier2}{2} & Analytic number theory
    & Circle method, Goldbach/RH cases \\
\textcolor{tier2}{2} & Control theory / Systems theory
    & Bounded-real lemma, state-space realizations \\
\textcolor{tier2}{2} & Optimisation / SDP
    & Finite certification as SDP feasibility \\
\midrule
\textcolor{tier3}{3} & Complex / Algebraic geometry
    & Hodge conjecture instantiation \\
\textcolor{tier3}{3} & Mathematical physics / Spectral theory
    & Resolvent / Cayley transform machinery \\
\bottomrule
\end{tabular}
\end{center}

\section{The Ideal Collaborator}

The \textbf{ideal collaborator} is a \textbf{functional analyst or GMT person
who also has taste for explicit constants} --- someone in the tradition of:

\begin{itemize}[nosep]
\item \textbf{Keith Ball} --- convex geometry, functional analysis
\item \textbf{Emmanuel Cand\`{e}s} --- convex optimisation, structured recovery
\item \textbf{Camillo De~Lellis} --- geometric measure theory, calibrations
\item \textbf{Terence Tao} --- analysis broadly, especially
      structure/randomness decompositions (which is exactly the CPM
      pattern)
\end{itemize}

The CPM pattern --- \emph{decompose into structured + error, control error
by dispersion, aggregate to global} --- is literally the Tao
``structure vs.\ randomness'' philosophy made into a formal proof kernel.
A young analyst or GMT postdoc who works on quantitative regularity with
explicit constants would be the perfect person to sharpen the domain
instantiations and push toward publication.

\section{Presentation Advice}

\textbf{For a talk audience:} present it to an \emph{analysis seminar}
(PDE, harmonic analysis, or geometric analysis), not a physics seminar.
The physics connection (RS bridge) is the motivation, but the paper's
strength is pure mathematics --- and that is where it should be evaluated
first.

\textbf{For a journal:} target \emph{Journal of Functional Analysis},
\emph{Advances in Mathematics}, or \emph{Communications in Mathematical
Physics} (the last if the RS bridge is included as motivation).
The domain instantiations (Hodge, RH, Goldbach, Navier--Stokes) make
the paper automatically relevant to any broad-scope analysis journal.

\end{document}
