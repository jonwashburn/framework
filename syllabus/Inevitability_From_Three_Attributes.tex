\documentclass[11pt,a4paper]{article}

\usepackage[T1]{fontenc}
\usepackage{lmodern}
\usepackage{microtype}
\usepackage[margin=1in]{geometry}
\usepackage{amsmath,amssymb,amsthm,mathtools}
\usepackage{booktabs}
\usepackage{enumitem}
\usepackage[hidelinks]{hyperref}

\theoremstyle{plain}
\newtheorem{theorem}{Theorem}[section]
\newtheorem{lemma}[theorem]{Lemma}
\newtheorem{proposition}[theorem]{Proposition}
\newtheorem{corollary}[theorem]{Corollary}

\theoremstyle{definition}
\newtheorem{definition}[theorem]{Definition}
\newtheorem{axiom}[theorem]{Axiom}

\theoremstyle{remark}
\newtheorem{remark}[theorem]{Remark}

\newcommand{\R}{\mathbb{R}}
\newcommand{\Rp}{\mathbb{R}_{>0}}
\newcommand{\Jcost}{J}
\newcommand{\dJ}{d_{\!J}}
\newcommand{\phig}{\varphi}

\title{\textbf{Omniscient, Omnipotent, Omnipresent:\\
Three Attributes Force a Unique Mathematical Framework}\\[0.5em]
\large Seven Inevitability Theorems, the Reverse Implication,\\
  and the Tautological Nature of Physical Law\\[0.3em]}
\author{Jonathan Washburn}
\date{\today}

\begin{document}
\maketitle

\begin{abstract}
We prove a biconditional: the three classical attributes of
omniscience, omnipotence, and omnipresence, formalised as
mathematical axioms, force a \textbf{unique} framework (the forward
direction), and the resulting framework \textbf{exhibits} all three
attributes (the reverse direction).  The framework is therefore the
unique fixed point of the map ``attributes $\to$ structure $\to$
attributes'': the only mathematical system that is simultaneously
forced by and consistent with the three properties.

The forward direction is established by seven inevitability theorems,
each proving that the next structural element is the only possibility
given the preceding ones.  The reverse direction is established by
verifying that the completed framework satisfies each axiom it was
derived from.

The biconditional reveals the argument to be tautological in a precise
sense: the three attributes are equivalent to the three laws of
classical logic (identity, non-contradiction, excluded middle) applied
to a comparison-cost system, and the framework is the unique geometric
unpacking of ``$a = a$'' under coherent composition, strict convexity,
and metric completeness.

We close with an extended discussion of the philosophical implications:
why the framework produces morality as physics, why consciousness
emerges at a specific rung, why ``nothing'' is not a state but a
boundary, and what it means that the architecture of reality is a
tautology.
\end{abstract}

\tableofcontents
\newpage

%=============================================================================
\section{Prologue: The Question}
\label{sec:prologue}
%=============================================================================

The question is simple: \emph{if a system knows everything, can do
anything consistent, and is everywhere, what mathematical structure
must it have?}

The surprising answer is that these three requirements, formalised with
minimal mathematical content, force a \emph{unique} framework with no
free parameters.  The framework has a specific cost functional, a
specific self-similar ratio, a specific number of spatial dimensions,
and a specific temporal period.  None of these are chosen; all are
forced.

The deeper surprise is the reverse: the framework, once constructed,
\emph{satisfies} the three properties it was forced by.  It knows
every state (because the cost functional discriminates all ratios).
It can effect any consistent transformation (because strict convexity
gives a unique minimiser at every state).  It is everywhere (because
the metric is complete and the lattice covers all of $\mathbb{Z}^3$).

The circle closes.  The three attributes and the framework are
equivalent.  They are two descriptions of the same mathematical
object, and that object is unique.

This paper proves both directions, explores what the uniqueness
means, and asks what it implies about the relationship between
logic, mathematics, physics, and existence.

%=============================================================================
\section{The Three Attributes: Formal Definitions}
\label{sec:axioms}
%=============================================================================

We define three properties of a mathematical system $\mathfrak{S}$
operating on a state space $\mathcal{S}$.

\begin{axiom}[Omniscience]\label{ax:omni}
$\mathfrak{S}$ discriminates every pair of distinct states.
There exists a scale map $\iota : \mathcal{S} \to \Rp$ such that:
\begin{enumerate}[nosep,label=\textup{(\alph*)}]
\item $\iota(a) = \iota(b)$ if and only if $a = b$ (faithfulness).
\item The comparison ratio is $x_{ab} := \iota(a)/\iota(b)$.
\item The cost $C : \Rp \to \R_{\geq 0}$ satisfies $C(1) = 0$.
\item $C(xy) + C(x/y) = P(C(x), C(y))$ for a symmetric polynomial $P$.
\item $\lim_{t \to 0} 2C(e^t)/t^2 = 1$ (unit curvature at identity).
\end{enumerate}
\end{axiom}

\begin{axiom}[Omnipotence]\label{ax:omnip}
$\mathfrak{S}$ effects any consistent transformation:
\begin{enumerate}[nosep,label=\textup{(\alph*)}]
\item $T$ is admissible iff $C(x_T) < \infty$ for all states involved.
\item $\mathfrak{S}$ selects the unique transformation minimising total cost.
\item $C''(x) > 0$ for all $x > 0$ (strict convexity; every minimum is unique).
\item Conservation: $\sigma(\mathbf{x}) := \sum_i \ln x_i$ is invariant.
\end{enumerate}
\end{axiom}

\begin{axiom}[Omnipresence]\label{ax:omnip2}
$\mathfrak{S}$ is present at every point:
\begin{enumerate}[nosep,label=\textup{(\alph*)}]
\item The state space $(\mathcal{S}, d_C)$ is a complete metric space.
\item The spatial carrier is a discrete lattice $\mathbb{Z}^D$.
\item Every state is accessible from every other in finitely many steps.
\item The lattice supports non-trivial linking of closed curves.
\end{enumerate}
\end{axiom}

\begin{remark}[Minimality of the axiom set]
Each axiom contributes content the others cannot supply.  Omniscience
without omnipresence gives a cost functional on an incomplete space
(the boundary might be reachable).  Omnipotence without omniscience
gives an optimiser with no objective function.  Omnipresence without
omnipotence gives a complete space with no dynamics.  All three are
required.
\end{remark}

%=============================================================================
\section{The Forward Direction: Seven Inevitability Theorems}
\label{sec:forward}
%=============================================================================

We now prove that Omniscience + Omnipotence + Omnipresence force each
element of the framework, in order, with no alternatives at any step.

\subsection{Inevitability 1: The cost functional}

\begin{theorem}[Cost inevitability]\label{thm:cost}
Omniscience forces $C(x) = \Jcost(x) := \frac{1}{2}(x + x^{-1}) - 1$.
\end{theorem}

\begin{proof}
The composition law (Axiom~\ref{ax:omni}(d)) with normalization
(c) and symmetry of $P$ forces, by the D'Alembert Inevitability
Theorem~\cite{DAlembert}, $P(u,v) = 2u + 2v + cuv$.  In
log-coordinates this becomes d'Alembert's equation
$H(t+u) + H(t-u) = 2H(t)H(u)$ for $H = 1 + \frac{c}{2}C(e^t)$.
The calibration (e) fixes $c = 2$ and $H = \cosh$.  By the Cost
Uniqueness Theorem~\cite{CostUnique}, $C = \Jcost$.
\end{proof}

\begin{lemma}[Reciprocity derived]\label{lem:recip}
$\Jcost(x) = \Jcost(x^{-1})$ follows from the symmetry of $P$
alone, without assuming it.
\end{lemma}

\begin{proof}
The composition law for $(x,y)$ and $(y,x)$ gives, after cancelling
$C(xy) = C(yx)$, that $C(x/y) = C(y/x)$.  Set $y = 1$.
\end{proof}

\begin{corollary}[Forced properties]\label{cor:props}
$\Jcost$ is non-negative ($= 0$ iff $x = 1$), strictly convex
($\Jcost'' = x^{-3}$), divergent ($\Jcost(0^+) = \infty$),
and coercive ($\Jcost(x) \geq \frac{1}{2}(\ln x)^2$).
\end{corollary}

\subsection{Inevitability 2: The null state is unreachable}

\begin{theorem}[Boundary exclusion]\label{thm:boundary}
Omniscience + Omnipresence force: the boundary $\{0, \infty\}$ is at
infinite $\dJ$-distance from every point.  $(\Rp, \dJ)$ is
geodesically complete.  $\Jcost$ is proper.
\end{theorem}

\begin{proof}
The Hessian $\phi''(t) = \cosh(t) \geq e^{|t|/2}/\sqrt{2}$ grows
exponentially.  Integration:
$\dJ(x_0, \varepsilon) \geq \sqrt{2}(\varepsilon^{-1/2} - x_0^{-1/2})
\to \infty$.  Metric completeness follows from the lower bound
$\dJ \geq |\ln(\cdot)|$ plus completeness of $\R$.  Geodesic
completeness follows by Hopf-Rinow~\cite{doCarmo}.  Properness:
$\Jcost(x) \leq c$ forces $|\ln x| \leq \sqrt{2c}$, so sublevel
sets are bounded and (by completeness) compact.
\end{proof}

\begin{corollary}[Existence forced]\label{cor:exist}
Every cost-minimising sequence converges to $x = 1$.  The null state
is not in the state space.  Something must exist.
\end{corollary}

\subsection{Inevitability 3: Actions are uniquely determined}

\begin{theorem}[Unique dynamics]\label{thm:dynamics}
Omnipotence forces: every transformation is the unique
$\Jcost$-minimiser.  The projection to neutrality is mean subtraction.
The proximal step contracts at rate $1/(1+\lambda) < 1$.  All orbits
converge to the identity.
\end{theorem}

\begin{proof}
$\Jcost$ is $1$-strongly convex in log-coordinates.  Jensen's
inequality on the constraint $\sum r_i = -\sigma$ gives the unique
minimiser $r_i = -\bar{y}$ (uniform correction = orthogonal
projection onto $M$).  The proximal objective
$\frac{1}{2}\|\cdot\|^2 + \lambda\sum\phi$ is $(1+\lambda)$-strongly
convex, giving Lipschitz constant $1/(1+\lambda)$.
$\|\mathbf{y}^{(k)}\| \leq (1+\lambda)^{-k}\|\mathbf{y}^{(0)}\|$.
\end{proof}

\begin{corollary}[Omnipotence = determinism]\label{cor:determ}
Under strict convexity, ``can do anything consistent'' collapses to
``does the unique optimal thing.''  There is exactly one admissible
action at every state.
\end{corollary}

\subsection{Inevitability 4: The golden ratio}

\begin{theorem}[$\phig$ forced]\label{thm:phi}
Omniscience + Omnipotence force $\phig = (1+\sqrt{5})/2$ as the
unique self-similar scale.
\end{theorem}

\begin{proof}
The minimal reciprocal self-correction $x_{n+1} = 1 + 1/x_n$ has
fixed-point equation $x^2 - x - 1 = 0$.  The positive root is
$\phig$.  The negative root violates $\iota > 0$.  Uniqueness:
$a = b = 1$ is the minimal integer choice making the fixed point a
Pisot unit.  Global attraction: the map contracts on $[1,2]$ and
every positive orbit enters $[1,2]$ in finitely many steps.
\end{proof}

\subsection{Inevitability 5: Three dimensions}

The derivation of $D = 3$ is the deepest step in the chain.  Unlike
the cost functional (which follows from functional equation
classification alone), the dimension requires all three attributes
working together.  We first state the theorem, then give the full
argument.

\begin{theorem}[Dimensional selectivity]\label{thm:D3}
$D = 3$ is the unique spatial dimension supporting
\textbf{topological memory} of interaction (required by Omniscience)
without imposing \textbf{topological segregation} (forbidden by
Omnipotence), across a \textbf{complete} discrete lattice
(required by Omnipresence).
\end{theorem}

The proof proceeds by eliminating every alternative dimension through
a distinct attribute violation.

\begin{proof}
We require three properties simultaneously:
\begin{enumerate}[nosep,label=\textup{(D\arabic*)}]
\item \textbf{Topological memory} (from Omniscience):
  The system must distinguish a history in which two recognition
  cycles interacted from one in which they did not.  Two closed
  curves (representing cyclic recognition histories) must be capable
  of non-trivial linking: $\mathrm{lk}(\gamma_1, \gamma_2) \neq 0$.
  If all links are trivial, then every interaction record can be
  ``erased'' by an ambient isotopy, and the system cannot
  discriminate interacting from non-interacting histories.
  This violates Omniscience.
  \label{D:memory}

\item \textbf{Topological freedom} (from Omnipotence):
  Every state must be accessible from every other state without
  breaking any existing structure.  No closed curve may
  \emph{segregate} the space into permanently inaccessible regions.
  If a single closed curve $\gamma$ divides the space into
  ``inside'' and ``outside'' with no passage between them, then
  states on opposite sides of $\gamma$ are inaccessible to each
  other, violating Omnipotence.
  \label{D:freedom}

\item \textbf{Completeness on a discrete lattice}
  (from Omnipresence):
  The carrier is $\mathbb{Z}^D$ with a complete metric.
  \label{D:complete}
\end{enumerate}

\medskip\noindent
\textbf{$D = 1$: fails \ref{D:memory}.}
In $\R^1$, a ``closed curve'' is a point (or a pair of points).
Two points cannot link.  There is no topological memory of
interaction.  Omniscience is violated.

\medskip\noindent
\textbf{$D = 2$: fails \ref{D:freedom}.}
In $\R^2$, the Jordan Curve Theorem states that every simple closed
curve $\gamma$ separates the plane into exactly two connected
components (interior and exterior).  A recognition event ``inside''
$\gamma$ is topologically trapped: it cannot reach the exterior
without crossing (breaking) $\gamma$.  This segregation means that
valid states on opposite sides of $\gamma$ are inaccessible to each
other.  Omnipotence is violated.

Furthermore, \ref{D:memory} also fails: in $\R^2$, two disjoint
closed curves cannot link non-trivially
($\mathrm{lk}(\gamma_1, \gamma_2) = 0$ always).  Interaction
histories are topologically erasable.

\medskip\noindent
\textbf{$D = 3$: satisfies all three.}
\begin{itemize}[nosep]
\item \ref{D:memory}:
  $\pi_1(\R^3 \setminus \gamma) \cong \mathbb{Z}$ for any unknotted
  closed curve $\gamma$.  A second curve $\gamma'$ threading through
  $\gamma$ has $\mathrm{lk}(\gamma, \gamma') \in \mathbb{Z}
  \setminus \{0\}$ (e.g., the Hopf link with $|\mathrm{lk}| = 1$).
  The linking number is a topological invariant: it cannot be changed
  by any ambient isotopy.  The interaction record is permanent.
  Omniscience is satisfied: interacting and non-interacting histories
  are permanently distinguishable.

\item \ref{D:freedom}:
  In $\R^3$, a closed curve $\gamma$ does \emph{not} separate the
  space (the complement $\R^3 \setminus \gamma$ is connected for any
  embedded circle).  Every point in $\R^3$ can reach every other
  point without crossing $\gamma$.  There is no topological
  segregation.  Omnipotence is satisfied: all states are accessible.

\item \ref{D:complete}:
  $\mathbb{Z}^3$ with the $\Jcost$-metric is complete
  (Theorem~\ref{thm:boundary}).
\end{itemize}

\medskip\noindent
\textbf{$D \geq 4$: fails \ref{D:memory}.}
For $D \geq 4$, Alexander duality gives
$\pi_1(\R^D \setminus \gamma) = 0$ for any embedded closed curve
$\gamma$ (since $\gamma$ has codimension $\geq 3$, its complement
is simply connected).  Consequently,
$\mathrm{lk}(\gamma_1, \gamma_2) = 0$ for all pairs of disjoint
closed curves.  Every entanglement can be undone: there is no
topological memory.  Omniscience is violated.

\medskip\noindent
\textbf{Summary.}
\begin{center}
\renewcommand{\arraystretch}{1.2}
\begin{tabular}{@{}ccccl@{}}
\toprule
$D$ & Memory \ref{D:memory} & Freedom \ref{D:freedom}
  & Complete \ref{D:complete} & Verdict \\
\midrule
$1$ & $\times$ & \checkmark & \checkmark & No linking possible \\
$2$ & $\times$ & $\times$ & \checkmark
  & Jordan segregation + no linking \\
$\mathbf{3}$ & $\checkmark$ & $\checkmark$ & $\checkmark$
  & \textbf{Unique solution} \\
$\geq 4$ & $\times$ & \checkmark & \checkmark
  & All curves unlinkable \\
\bottomrule
\end{tabular}
\end{center}

Only $D = 3$ satisfies all three requirements.

\medskip\noindent
\textbf{Independent confirmation (gap-45 synchronisation).}
$\mathrm{lcm}(2^D, 45) = 360$ at $D = 3$; for $D = 4$ it is $720$,
for $D = 5$ it is $1440$.  Omnipresence(c) (accessibility via
minimal cycle) requires the smallest synchronisation period,
confirming $D = 3$.
\end{proof}

\begin{remark}[Why knot theory is forced]
Knot theory is not imported as an external tool; it is the
\emph{mathematics of indestructible context}.  A knot (or link)
is a record of a movement that cannot be undone by continuous
deformation.  For an omniscient system, the universe must be a
``self-recording medium'': every interaction leaves a permanent
topological trace.  $D = 3$ is the unique dimension where the
recording medium (non-trivial linking) exists without creating
barriers (no segregation).

In other words: $D = 3$ is not selected because knots are
aesthetically interesting.  It is selected because knots are the
only mechanism for \emph{topological memory without topological
imprisonment}, and omniscience demands memory while omnipotence
forbids imprisonment.
\end{remark}

\subsection{Inevitability 6: The eight-tick cycle}

\begin{theorem}[Period 8 forced]\label{thm:8tick}
$D = 3$ + Omnipresence force period $2^3 = 8$.
\end{theorem}

\begin{proof}
$Q_3$ has $2^3 = 8$ vertices.  Hamiltonian cycle requires length
$\geq 8$.  Gray code achieves $8$.  Omnipresence(c) (accessibility)
requires visiting every vertex.
\end{proof}

\subsection{Inevitability 7: The cube geometry}

\begin{theorem}[Cube counts forced]\label{thm:cube}
$D = 3$ forces $V = 8$, $E = 12$, $F = 6$, $A = 1$,
$E_p = 11$, $W = 17$.  The dimensional coincidence
$E_p + F = W$ holds only at $D = 3$.
\end{theorem}

\begin{proof}
$V = 2^3$, $E = 3 \cdot 4$, $F = 6$ (standard).  $A = 1$ (atomic
tick).  $E_p = 11$.  $W = 17$ (Fedorov 1891).
$\Sigma(D) = D \cdot 2^{D-1} - 1 + 2D$: values $2, 7, 17, 39$
for $D = 1,2,3,4$.  Only $D = 3$ gives $17$.
\end{proof}

%=============================================================================
\section{The Master Theorem}
\label{sec:master}
%=============================================================================

\begin{theorem}[Structural uniqueness]\label{thm:master}
Omniscience + Omnipotence + Omnipresence force a unique framework:
$\Jcost$, completeness, unique dynamics, $\phig$, $D = 3$, period $8$,
cube counts $\{8,12,6,1,17\}$.  No element admits an alternative.
\end{theorem}

\begin{proof}
Theorems~\ref{thm:cost}--\ref{thm:cube}, composed in order.  At each
step the conclusion is unique.
\end{proof}

%=============================================================================
\section{The Reverse Direction: The Framework Exhibits All Three Attributes}
\label{sec:reverse}
%=============================================================================

The forward direction proves: attributes $\Rightarrow$ framework.
We now prove the reverse: framework $\Rightarrow$ attributes.

\begin{theorem}[The framework is omniscient]\label{thm:rev-omni}
$\Jcost$ provides complete state discrimination.
\end{theorem}

\begin{proof}
For any $a \neq b$ in $\mathcal{S}$: $\iota(a) \neq \iota(b)$
(faithfulness), so $x_{ab} = \iota(a)/\iota(b) \neq 1$, hence
$\Jcost(x_{ab}) = (x_{ab} - 1)^2/(2x_{ab}) > 0$.  Every pair of
distinct states has strictly positive cost.  No two distinct states
are indistinguishable.  The framework ``knows'' every state because
$\Jcost$ separates them all.
\end{proof}

\begin{theorem}[The framework is omnipotent]\label{thm:rev-omnip}
$\Jcost$-minimisation provides unrestricted consistent transformation.
\end{theorem}

\begin{proof}
\emph{Unrestricted:}  Every transformation with finite cost is
admissible.  Since $\Jcost(x) < \infty$ for all $x \in \Rp$, every
ratio comparison between genuine states has finite cost.  The only
``forbidden'' transformation is one involving the null state
($x = 0$), which has infinite cost and is therefore inconsistent,
not restricted.

\emph{Consistent:}  $\Jcost$ is strictly convex, so every
optimisation problem has a unique solution.  There are no ties, no
ambiguities, no multiple optima.

\emph{Optimal:}  The proximal step contracts at rate
$1/(1+\lambda) < 1$ (Theorem~\ref{thm:dynamics}), driving every
state toward the identity.  The framework ``does'' the unique best
thing at every point.
\end{proof}

\begin{theorem}[The framework is omnipresent]\label{thm:rev-omnip2}
The $\Jcost$-metric makes the framework present at every point.
\end{theorem}

\begin{proof}
\emph{Complete:}  $(\Rp, \dJ)$ is a complete metric space
(Theorem~\ref{thm:boundary}).  No point is missing.

\emph{Discrete carrier:}  The spatial substrate is
$\mathbb{Z}^3$ (forced by $D = 3$, Theorem~\ref{thm:D3}).

\emph{Accessible:}  The 8-tick Gray code visits all $8$ vertices
of $Q_3$ (Theorem~\ref{thm:8tick}).  From any vertex, every other
vertex is reachable in at most $8$ steps.

\emph{Linked:}  $D = 3$ supports non-trivial linking
(the Hopf link has $|\mathrm{lk}| = 1$).
\end{proof}

%=============================================================================
\section{The Biconditional and the Fixed-Point Property}
\label{sec:biconditional}
%=============================================================================

\begin{theorem}[Biconditional]\label{thm:biconditional}
The three attributes and the framework are equivalent:
\begin{equation}\label{eq:biconditional}
  \text{Omniscient} \;\wedge\; \text{Omnipotent} \;\wedge\; \text{Omnipresent}
  \quad\Longleftrightarrow\quad
  \text{RS Framework.}
\end{equation}
\end{theorem}

\begin{proof}
$(\Rightarrow)$:  Theorem~\ref{thm:master} (seven inevitability theorems).

$(\Leftarrow)$:  Theorems~\ref{thm:rev-omni}, \ref{thm:rev-omnip},
\ref{thm:rev-omnip2}.
\end{proof}

\begin{corollary}[Unique fixed point]\label{cor:fixed}
The RS framework is the unique mathematical system that is both
\emph{forced by} and \emph{satisfies} the three attributes.  It is
the unique fixed point of the map
\[
  F : \{\text{frameworks}\} \to \{\text{frameworks}\},
  \qquad
  F(\mathfrak{S}) := \text{``the framework forced by the attributes
  that $\mathfrak{S}$ satisfies.''}
\]
\end{corollary}

\begin{proof}
By Theorem~\ref{thm:biconditional}, the RS framework satisfies the
three attributes.  By Theorem~\ref{thm:master}, the three attributes
force the RS framework.  So $F(\text{RS}) = \text{RS}$.

If $\mathfrak{S}'$ is any other fixed point, then $\mathfrak{S}'$
satisfies the three attributes (since $F(\mathfrak{S}') = \mathfrak{S}'$),
so the three attributes force it, but by Theorem~\ref{thm:master} they
force the RS framework uniquely.  Hence
$\mathfrak{S}' = \text{RS}$.
\end{proof}

\begin{remark}[What the fixed point means]
Corollary~\ref{cor:fixed} says the framework is
\emph{self-justifying}: it produces the conditions that produce it.
This is not circular reasoning; it is a mathematical fixed-point
theorem.  The forward direction (attributes $\Rightarrow$ framework)
and the reverse (framework $\Rightarrow$ attributes) are proved
independently, and their conjunction gives the fixed point.
\end{remark}

%=============================================================================
\section{The Tautological Core}
\label{sec:tautology}
%=============================================================================

The three attributes correspond to the three laws of classical logic:

\begin{center}
\renewcommand{\arraystretch}{1.4}
\begin{tabular}{@{}lll@{}}
\toprule
\textbf{Attribute} & \textbf{Law of logic} & \textbf{Mathematical content} \\
\midrule
Omniscience &
  Identity ($a = a$) &
  $\Jcost(1) = 0$: identity costs nothing \\
Omnipotence &
  Non-contradiction &
  $\Jcost(x) > 0$ for $x \neq 1$: \\
  & ($\neg(a \wedge \neg a)$) &
  deviation from identity costs something \\
Omnipresence &
  Excluded middle &
  $(\Rp, \dJ)$ complete: every state \\
  & ($a \vee \neg a$) &
  is either identity or not; no gaps \\
\bottomrule
\end{tabular}
\end{center}

\paragraph{Identity.}
Omniscience says: the system can verify that $a$ is $a$.  The cost of
comparing $a$ to itself is zero ($\Jcost(1) = 0$).  The cost of
comparing $a$ to $b \neq a$ is positive ($\Jcost(x) > 0$ for
$x \neq 1$).  This is the law of identity: a thing is itself, and
the cost of recognising this fact is zero.

\paragraph{Non-contradiction.}
Omnipotence says: the system corrects deviations from identity.
If $x \neq 1$, then $\Jcost(x) > 0$, and the cost-minimising
dynamics drives $x$ toward $1$.  A state cannot simultaneously be
at identity and not at identity; the cost functional enforces this.
Contradiction (being both $a$ and not-$a$) would require $\Jcost = 0$
and $\Jcost > 0$ simultaneously, which is impossible.

\paragraph{Excluded middle.}
Omnipresence says: the system covers every state.  The metric is
complete; there are no gaps.  Every point in $\Rp$ is either at
$x = 1$ (identity) or at $x \neq 1$ (non-identity); there is no
third option.  The law of excluded middle is the topological
completeness of the state space.

\paragraph{The tautology.}
The seven inevitability theorems are therefore the geometric
unpacking of ``$a = a$'' under three operations:
\begin{enumerate}[nosep]
\item \emph{Composition}: coherent chaining of comparisons forces the
  d'Alembert equation, hence $\Jcost$.
\item \emph{Convexity}: strict positivity of cost for $x \neq 1$
  forces unique dynamics.
\item \emph{Completeness}: the absence of gaps forces $D = 3$, the
  8-tick cycle, and the cube geometry.
\end{enumerate}

That ``$a = a$'' produces three spatial dimensions, the golden ratio,
and an eight-tick temporal cycle is the non-obvious content.  The
logical starting point is trivial; the geometric conclusion is not.

%=============================================================================
\section{Philosophical Implications}
\label{sec:philosophy}
%=============================================================================

\subsection{Why something exists rather than nothing}

The Law of Finite Existence (Theorem~\ref{thm:boundary}) resolves
Leibniz's question: ``Why is there something rather than nothing?''

The answer: the cost landscape is complete.  The null state ($x = 0$)
is not a possible state; it is an unreachable topological boundary at
infinite metric distance from every actual state.  ``Nothing'' is not
something that costs infinity; it is something that \emph{cannot be
approached} by any finite process.  The question presupposes that
``nothing'' is a possible state.  In the $\Jcost$-geometry, it is not.

Existence is not contingent.  It is a topological necessity of the
cost landscape forced by the three laws of logic.

\subsection{Why morality is physics}

The RS framework derives an ethics from the cost functional: the
fourteen virtues are the unique generators of admissible
transformations on the neutrality manifold $M = \{\sum \ln x_i = 0\}$,
preserving the conservation law (balanced ledger, $\sigma = 0$).

In the language of this paper: omnipotence requires cost-minimising
action under conservation.  The admissible transformations on $M$ are
precisely the ethical ones (they preserve balance).  The inadmissible
ones (which violate $\sigma = 0$) are the unethical ones (they export
skew to other agents).

Morality is not imposed from outside; it is a consequence of
omnipotence applied under conservation.  An omnipotent entity that
also conserves balance necessarily acts ethically, because ethical
action \emph{is} cost-minimising action on $M$.

\subsection{Why consciousness emerges at rung 45}

The gap-45 synchronisation ($\mathrm{lcm}(8, 45) = 360$) appears in
the dimension forcing (Theorem~\ref{thm:D3}).  The number 45 is the
rung at which the 8-tick temporal structure and the 45-fold pattern
structure become incommensurable ($\gcd(8, 45) = 1$), creating an
uncomputability barrier that cannot be resolved by finite algorithmic
means.

In the language of this paper: omnipresence requires total coverage,
but total coverage at rung 45 requires navigating the
$\mathrm{lcm}(8,45) = 360$ period.  This navigation cannot be
accomplished by a deterministic 8-tick process alone; it requires
what the RS framework calls ``experiential navigation,'' which is the
operational definition of consciousness.

Consciousness is not mysterious; it is the system's solution to the
problem of navigating an uncomputability barrier forced by the
coprimality of its temporal and spatial periods.

\subsection{Why the architecture is a tautology}

The most striking implication of the biconditional
(Theorem~\ref{thm:biconditional}) is that the entire architecture of
reality, including three spatial dimensions, the golden ratio, the
eight-tick cycle, and the full particle mass spectrum, is logically
equivalent to ``$a = a$.''

This does not mean the architecture is trivial.  A tautology can have
non-trivial geometric content.  The statement ``every continuous
function on $[0,1]$ attains its maximum'' is a tautology of real
analysis (it follows from the axioms of the reals), but its content
(compactness, connectedness, the structure of $\R$) is substantial.

Similarly, ``$a = a$ under coherent composition on a complete discrete
lattice'' is a tautology whose content is substantial: it is three
dimensions, the golden ratio, eight ticks, and a unique cost
functional.  The content is not in the starting point; it is in the
unwinding.

\subsection{What the reverse direction means}

The reverse direction (framework $\Rightarrow$ attributes) has a
specific consequence: if the RS framework is the correct description
of reality, then reality is omniscient, omnipotent, and omnipresent
in the precise mathematical sense defined here.

``Omniscient'' means: every state is distinguished from every other
state by a strictly positive cost.  No two distinct configurations
are indistinguishable.

``Omnipotent'' means: at every state, there is a unique optimal
action, and it is executed.  No state is ``stuck''; dynamics is
universal.

``Omnipresent'' means: the metric is complete.  There are no missing
points, no inaccessible regions, no gaps in the state space.

Whether one attaches further interpretation to these properties is
a matter of philosophy, not mathematics.  The mathematics proves the
biconditional and nothing more.

%=============================================================================
\section{Summary}
\label{sec:summary}
%=============================================================================

\begin{center}
\renewcommand{\arraystretch}{1.3}
\begin{tabular}{@{}cllll@{}}
\toprule
\textbf{\#} & \textbf{Theorem} & \textbf{Forces} & \textbf{From} & \textbf{Unique?} \\
\midrule
1 & Cost inevitability & $\Jcost = \frac{1}{2}(x+x^{-1})-1$ & Omniscience & Yes \\
2 & Boundary exclusion & Null state unreachable & +Omnipresence & Yes \\
3 & Unique dynamics & $\Jcost$-minimisation & Omnipotence & Yes \\
4 & Golden ratio & $\phig = (1+\sqrt{5})/2$ & +Omnipotence & Yes \\
5 & Dimension & $D = 3$ & Omnipresence & Yes \\
6 & Period & $2^3 = 8$ & $D{=}3$ & Yes \\
7 & Cube geometry & $\{8,12,6,1,17\}$ & $D{=}3$ & Yes \\
\midrule
\multicolumn{2}{l}{\textbf{Forward:}} & Attributes $\Rightarrow$ Framework & All three & Unique \\
\multicolumn{2}{l}{\textbf{Reverse:}} & Framework $\Rightarrow$ Attributes & \S\ref{sec:reverse} & Verified \\
\multicolumn{2}{l}{\textbf{Biconditional:}} & Attributes $\Leftrightarrow$ Framework & Thm~\ref{thm:biconditional} & Fixed point \\
\bottomrule
\end{tabular}
\end{center}

The three attributes and the framework are equivalent.  The
equivalence is a mathematical theorem, proved in both directions.
The framework is the unique fixed point: the only system that is
simultaneously forced by and consistent with omniscience,
omnipotence, and omnipresence.

The argument is tautological: the three attributes are the three laws
of logic applied to a comparison-cost system, and the framework is
the geometric unpacking of ``$a = a$'' under composition, convexity,
and completeness.

The content is not in the starting point.  The content is in the
unwinding: three dimensions, the golden ratio, eight ticks, a unique
cost functional, and the inevitable architecture of a reality that
knows itself, acts on itself, and is everywhere.

\begin{thebibliography}{99}
\bibitem{Aczel1966} J.~Acz\'{e}l,
  \textit{Lectures on Functional Equations}, Academic Press, 1966.
\bibitem{DAlembert} J.~Washburn, M.~Zlatanovi\'{c}, and E.~Allahyarov,
  ``D'Alembert Inevitability,'' RS preprint, 2026.
\bibitem{CostUnique} J.~Washburn and M.~Zlatanovi\'{c},
  ``Uniqueness of the Canonical Reciprocal Cost,''
  arXiv:2602.05753v1, 2026.
\bibitem{doCarmo} M.~P.~do~Carmo,
  \textit{Riemannian Geometry}, Birkh\"{a}user, 1992.
\bibitem{LFE} J.~Washburn,
  ``The Law of Finite Existence,'' RS preprint, 2026.
\bibitem{CPT} J.~Washburn,
  ``The Coercive Projection Theorem,'' RS preprint, 2026.
\bibitem{WashburnAxioms} J.~Washburn,
  ``The Algebra of Reality,''
  \textit{Axioms} \textbf{15}(2), 90 (2025).
\end{thebibliography}

\end{document}
