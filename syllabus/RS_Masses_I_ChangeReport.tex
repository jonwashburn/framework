\documentclass[11pt,a4paper]{article}
\usepackage[margin=1in]{geometry}
\usepackage[T1]{fontenc}
\usepackage{lmodern}
\usepackage{microtype}
\usepackage{enumitem}
\usepackage{xcolor}
\usepackage{booktabs}
\usepackage[hidelinks]{hyperref}

\definecolor{added}{RGB}{0,100,0}
\definecolor{removed}{RGB}{180,0,0}
\definecolor{changed}{RGB}{0,50,140}

\newcommand{\Added}[1]{\textcolor{added}{\textbf{+}~#1}}
\newcommand{\Removed}[1]{\textcolor{removed}{\textbf{--}~#1}}
\newcommand{\Changed}[1]{\textcolor{changed}{\textbf{$\Delta$}~#1}}

\title{\textbf{Change Report}\\[0.3em]
\large RS\_Masses\_I\_Mechanism.tex\\[0.2em]
\normalsize Session: 2026-02-11}
\author{Prepared by AI assistant for Jonathan Washburn}
\date{\today}

\begin{document}
\maketitle

\section*{Context}

A colleague preparing a talk on the RS particle mass framework identified
a structural vulnerability: the cube combinatorial integers
$\{8, 11, 12, 17, 6\}$ appear in \emph{both} the sector yardstick
exponents and the lepton mass-matching formulas (generation torsion,
electron break $\delta_e$).  A reviewer could interpret this as ad~hoc
fitting.

This report summarises all changes made to Paper~I
(\texttt{RS\_Masses\_I\_Mechanism.tex}) during this session to address
that concern.

\section{Summary of Changes}

\begin{center}
\renewcommand{\arraystretch}{1.3}
\begin{tabular}{@{}p{0.22\textwidth}p{0.72\textwidth}@{}}
\toprule
\textbf{Area} & \textbf{Change} \\
\midrule
Abstract & Tightened to preview: (i)~constraint-based yardsticks, (ii)~generation torsion from edge/face hierarchy, (iii)~the counting-layer vocabulary and over-determination arguments. \\
\midrule
New \S3.1 & \Added{``The Counting-Layer Vocabulary'' (Section~3.1).}  Explains why the same five integers must appear in multiple formulas.  Introduces the exhaustive-vocabulary argument and forward-references the integer budget table. \\
\midrule
\S3.5 Yardsticks & \Changed{Complete rewrite.}  Old version: a flat table with ``structural identifications'' disclaimer.  New version: (i)~two-channel decomposition motivation, (ii)~five explicit constraints (Y1--Y4 + compensation) that fix $B_{\mathrm{pow}}$ from charge ordering with no free choice, (iii)~over-determination argument (8 outputs from 5 inputs), (iv)~honest assessment of the remaining $r_0$ open problem. \\
\midrule
New \S3.7 & \Added{``Generation Torsion from the Cube Hierarchy'' (Section~3.7).}  Previously a 3-line paragraph.  Now a full Hypothesis with the coupling-level derivation ($\tau_1=0$, $\tau_2=E_{\mathrm{passive}}$, $\tau_3=E_{\mathrm{passive}}+F=W$) and an explicit remark distinguishing the physical role of integers in torsion vs.\ yardsticks. \\
\midrule
New Table~1 & \Added{Integer Budget table.}  Every cube integer mapped to its distinct static (sector) and dynamical (generation) roles.  Shows vocabulary is exhaustive: nothing unused, nothing extra needed. \\
\midrule
\S7 Open Problems & \Changed{Restructured into ``high priority'' (predictions may sharpen) and ``structural priority'' (numbers unchanged).}  Added note that solving open problems changes \emph{logical status}, not numerical predictions. \\
\midrule
\S8 Conclusions & \Changed{Expanded.}  Added a three-point rebuttal of the ``same integers used twice'' objection: (i)~exhaustive vocabulary, (ii)~distinct physical roles, (iii)~over-determined system ($>$12 outputs from 5 inputs). \\
\midrule
Macro & \Added{\texttt{\textbackslash Etot} macro} for $E_{\mathrm{total}}$, used consistently in constraint labels. \\
\midrule
Labels & \Added{Cross-reference labels:} \texttt{sec:vocabulary}, \texttt{sec:yardstick}, \texttt{sec:torsion}, \texttt{tab:budget}, \texttt{hyp:sector\_constraints}, \texttt{hyp:torsion}, \texttt{prop:cube}, \texttt{eq:vocab}, \texttt{eq:yardstick}. \\
\bottomrule
\end{tabular}
\end{center}

\section{What Was NOT Changed}

The following elements are \textbf{unchanged}:
\begin{itemize}[nosep]
\item All numerical values ($B_{\mathrm{pow}}$, $r_0$, torsion $\{0,11,17\}$, gap function, $Z$-map).
\item The proved foundation (Section~2): $J$-cost, $\varphi$, $D=3$, cube counts, 8-tick.
\item The master mass law (Hypothesis~3.3).
\item The Yukawa bridge and Higgs reinterpretation (Section~5).
\item The falsifier list (Section~6).
\item The bibliography (only the ordering is preserved).
\end{itemize}

\section{Rationale by Change}

\subsection*{1. Counting-Layer Vocabulary (\S3.1)}

\textbf{Problem:} A reviewer seeing $E_{\mathrm{passive}}=11$ in both the
yardstick and the generation step concludes the authors are recycling
numbers.

\textbf{Solution:} Explain \emph{before} either formula appears that the
cube has a finite, exhaustive vocabulary.  Analogy: the 12 chromatic notes
appear in both melody and harmony.  The new section sets the reader's
expectation so the dual appearance is anticipated, not surprising.

\subsection*{2. Constraint-Based Yardsticks (\S3.5)}

\textbf{Problem:} The old table presented formulas like ``$4W-6$'' as
post-hoc identifications, inviting the reading ``you just searched for a
combination that works.''

\textbf{Solution:} Five constraints (Y1--Y4 + compensation) are stated
\emph{before} the table.  The $B_{\mathrm{pow}}$ column now follows from
the constraints with no remaining freedom.  The $r_0$ column is honestly
flagged as the remaining open problem, but the over-determination argument
($8$ outputs from $5$ inputs) reframes the situation from
``under-constrained fitting'' to ``over-constrained structural match.''

\subsection*{3. Generation Torsion Subsection (\S3.7)}

\textbf{Problem:} The old paper had a 3-line paragraph saying ``see
Paper~VI.''  This left the generation torsion unmotivated in Paper~I itself.

\textbf{Solution:} A full Hypothesis with the coupling-level derivation
and an explicit remark distinguishing the \emph{static} role of integers
(yardstick) from their \emph{dynamical} role (torsion).

\subsection*{4. Integer Budget Table (Table~1)}

\textbf{Problem:} No single display showed all integer appearances side by
side.  A reviewer had to hunt through the text.

\textbf{Solution:} A compact table mapping every cube integer to its
static and dynamical roles.  The table makes it visually immediate that
roles are distinct and the vocabulary is saturated.

\subsection*{5. Conclusions Rebuttal}

\textbf{Problem:} The old conclusions did not address the dual-use concern.

\textbf{Solution:} A three-point argument (exhaustive vocabulary, distinct
roles, over-determination) is now in the conclusions---the section most
likely to be read by a reviewer skimming the paper.

\section{Impact Assessment}

\begin{itemize}[nosep]
\item \textbf{For the talk:} The colleague can cite the constraint-based
  yardstick derivation (Y1--Y4) and the integer budget table as direct
  rebuttals to the ``fitting'' objection.
\item \textbf{For journal submission:} The over-determination argument and
  vocabulary principle significantly strengthen the paper against the
  most predictable reviewer objection.
\item \textbf{For a higher-IF target:} The remaining path to upgrade is
  solving Open Problem O5 (deriving the $r_0$ formulas from an
  admissibility principle), which would close the last gap and eliminate
  all ``structural identification'' language.
\end{itemize}

\end{document}
