\documentclass[11pt,a4paper]{article}
\usepackage[margin=1in]{geometry}
\usepackage{hyperref}
\usepackage{enumitem}
\usepackage{xcolor}
\usepackage{amsmath,amssymb}

\hypersetup{
    colorlinks=true,
    linkcolor=blue,
    urlcolor=blue,
    citecolor=blue
}

\newcommand{\phig}{\varphi}

\title{\textbf{Recognition Science: The Mathematical Spine}\\
\large A Self-Contained Reading Order for Mathematicians}
\author{Jonathan Washburn}
\date{\today}

\begin{document}

\maketitle

\section*{Purpose}

This document extracts from the full Recognition Science (RS) syllabus a
\textbf{self-contained mathematical narrative} that can be read with
\emph{no} physical interpretation.  Every result in Layers~0--3 is pure
mathematics (functional equations, convex analysis, algebraic number theory,
graph theory, topology).  Physical semantics enter only in Layer~4
(``operator theory'') and can be treated as motivation rather than prerequisite.

The dependencies are minimal and explicit: each layer uses only the
definitions and theorems of the layers above it.

\medskip\noindent
\textbf{Companion file:}
\texttt{papers/Recognition\_Science\_Math\_DAG.mermaid} (visual dependency graph).

\bigskip
\hrule
\bigskip

%======================================================================
\section{Layer 0 --- The Functional Equation}
\label{sec:layer0}
%======================================================================

\noindent\textit{``What is the primitive?''}

\subsection*{M1.\; The Recognition Composition Law (RCL)}
\textbf{Syllabus paper:} 2.\ RCL Primer
\hfill\textbf{File:} \texttt{Recognition\_Composition\_Law\_Primer.tex}

\textbf{Statement.}
Let $F:\mathbb{R}_{>0}\to\mathbb{R}$ be continuous and satisfy
\begin{equation}\label{eq:RCL}
F(xy) + F(x/y) \;=\; 2\,F(x)\,F(y) + 2\,F(x) + 2\,F(y).
\end{equation}
This is a calibrated multiplicative form of the d'Alembert functional
equation $f(s{+}t)+f(s{-}t)=2f(s)f(t)$ under the substitution $x=e^s$,
$y=e^t$, and $F=H-1$ where $H$ is the standard d'Alembert solution.

\textbf{Mathematical content.}
The paper states the equation, motivates the normalization side-conditions
(A1)--(A3), and shows that \eqref{eq:RCL} is the natural composition law for
\emph{ratio costs} on $\mathbb{R}_{>0}$.

\textbf{Dependencies:} None (starting point).

%======================================================================
\section{Layer 1 --- Uniqueness and Classification}
\label{sec:layer1}
%======================================================================

\noindent\textit{``What does the equation force?''}

\subsection*{M2.\; Cost Uniqueness (Theorem T5)}
\textbf{Syllabus paper:} 3.\ Uniqueness of the Canonical Reciprocal Cost
\hfill\textbf{File:} \texttt{Cost-9-1.pdf} (latest revision)

\textbf{Statement.}
Under the axioms
\begin{enumerate}[label=(A\arabic*),nosep]
  \item $F(1)=0$ \quad(normalization),
  \item $F$ satisfies the RCL \eqref{eq:RCL} \quad(composition),
  \item $\left.\frac{d^2}{dt^2}F(e^t)\right|_{t=0}=1$ \quad(calibration),
\end{enumerate}
there is a \emph{unique} solution on $\mathbb{R}_{>0}$:
\begin{equation}\label{eq:Jcost}
  J(x) \;=\; \tfrac{1}{2}\!\left(x + x^{-1}\right) - 1
  \;=\; \cosh(\ln x) - 1.
\end{equation}

\textbf{Derived properties} (not assumed):
reciprocity $J(x)=J(x^{-1})$; strict convexity on $\mathbb{R}_{>0}$;
$J\ge 0$ with equality iff $x=1$; $J''(1)=1$.

\textbf{Dependencies:} M1.

\subsection*{M3.\; D'Alembert Inevitability}
\textbf{Syllabus paper:} 5.\ D'Alembert Inevitability
\hfill\textbf{File:} \texttt{DAlembert\_Inevitability.tex}

\textbf{Statement.}
Among all continuous functions $F:\mathbb{R}_{>0}\to\mathbb{R}$ satisfying
$F(x)=F(x^{-1})$, $F(1)=0$, and
$F(xy)+F(x/y) = P(F(x),F(y))$ for a symmetric polynomial $P$,
non-triviality forces $P(u,v) = 2u + 2v + c\,uv$ with $c$ a single scalar.
Calibration fixes $c=2$, recovering the standard d'Alembert family.

\textbf{Mathematical content.}
This is a \emph{classification result} for functional equations.  Under mild
structural assumptions, the bilinear family is the \textbf{only} survivor.
The composition law \eqref{eq:RCL} is therefore not a modeling choice but a
theorem.

\textbf{Dependencies:} M1 (for the composition law setup); M2 (for calibration step).

\subsection*{M4.\; Reciprocal Convex Costs for Ratio Matching}
\textbf{Syllabus paper:} 63.\ Reciprocal Convex Costs
\hfill\textbf{File:} \texttt{submitted-entropy-version-entropy-4136332.pdf}

\textbf{Statement.}
Characterizes the class of reciprocal convex cost functionals on
$\mathbb{R}_{>0}$ satisfying the d'Alembert equation, embedding the cost
uniqueness result in the broader functional-equation literature.

\textbf{Dependencies:} M2.

%======================================================================
\section{Layer 2 --- Algebraic and Geometric Rigidity}
\label{sec:layer2}
%======================================================================

\noindent\textit{``What scale and dimension are forced?''}

\subsection*{M5.\; $\phig$ as Unique Fixed Point (Penrose Bridge)}
\textbf{Syllabus paper:} 10.\ The Golden Ratio as a Universal Coherence Eigenvalue
\hfill\textbf{File:} \texttt{Penrose\_golden\_ratio\_and\_ledger\_structure.tex}

\textbf{Statement.}
The minimal reciprocal self-correction rule $x_{n+1} = 1 + 1/x_n$ has unique
positive fixed point $\phig = (1{+}\sqrt{5})/2$, satisfying
$\phig^2 = \phig + 1$.

\textbf{Key results:}
\begin{itemize}[nosep]
  \item \textbf{Log-ratio isomorphism.}
    $t=\ln x$ gives $(\mathbb{R}_{>0},\times) \cong (\mathbb{R},+)$.
    Multiplicative inflation $x\mapsto\phig\,x$ becomes additive shift
    $t\mapsto t + \ln\phig$.
  \item \textbf{Coherence cost of aperiodicity.}
    $J(\phig) = \phig - 3/2 \approx 0.118$ is the minimal non-trivial cost.
  \item \textbf{Penrose substitution.}  The $2{\times}2$ Penrose substitution
    matrix has Perron--Frobenius eigenvalue $\phig$, giving an independent
    geometric origin of the same constant.
\end{itemize}

\textbf{Dependencies:} M2 (for $J$ and self-similarity).

\subsection*{M6.\; Dimensional Rigidity: $D{=}3$}
\textbf{Syllabus paper:} 11.\ Dimensional Rigidity D=3 (Strengthened)
\hfill\textbf{File:} \texttt{Dimensional\_Rigidity\_D3\_2.tex}

\textbf{Statement.}
Three independent constraints each force spatial dimension $D=3$:
\begin{enumerate}[label=(\Alph*),nosep]
  \item \textbf{Topological.}
    An integer-valued linking invariant for disjoint oriented loops in
    $\mathbb{S}^D$ exists iff $D=3$ (Alexander duality:
    $H_1(\mathbb{S}^D\setminus K)\cong\mathbb{Z}$ iff $D=3$).
  \item \textbf{Dynamical.}
    For the $D$-dimensional Kepler potential $V_D(r)\propto -r^{2-D}$,
    near-circular orbits are stable only for $D<4$ and non-precessing only for
    $D=3$ (Binet equation analysis).
  \item \textbf{Arithmetic.}
    $\mathrm{lcm}(2^D,45)$ is minimized over $D\ge 3$ uniquely at $D=3$,
    where it equals $360$.
\end{enumerate}

\textbf{Dependencies:} M5 (for the arithmetic constraint involving $\phig$-scaling);
standard topology and ODE theory.

\subsection*{M7.\; Stability Audit (RSA)}
\textbf{Syllabus paper:} 53.\ The Recognition Stability Audit
\hfill\textbf{File:} \texttt{Recognition\_Stability\_Audit.tex}

\textbf{Statement.}
Converts existence/non-existence claims into finite certificates using:
(i)~a canonical sensor (Cayley transform), (ii)~Schur/Pick interpolation
theory, (iii)~a realizability class grounded in period-8 constraints.

\textbf{Mathematical content.}
A decision-style pipeline linking complex analysis, control theory, and
computational certificates.  Pure mathematics; physical context is motivational.

\textbf{Dependencies:} M3 (inevitability provides the constraint space).

%======================================================================
\section{Layer 3 --- Structures Forced by Cost}
\label{sec:layer3}
%======================================================================

\noindent\textit{``What mathematical objects emerge from the cost?''}

\subsection*{M8.\; Ledger Dynamics}
\textbf{Syllabus paper:} 4.\ Coherent Comparison as Information Cost
\hfill\textbf{File:} \texttt{papers/pdf/2601.12194v1.pdf}

\textbf{Statement.}
Discrete dynamics on directed graphs with $J$-cost weighted edges, atomic
ticks, and conservation constraints.  Key results:
\begin{itemize}[nosep]
  \item Balanced double-entry postings are \emph{forced} by conservation +
    discreteness.
  \item Minimal period on the $D$-cube is $2^D$ (proved for all $D$).
  \item Closed-chain flux vanishes: $\sum \ln r_i = 0$ over cycles.
\end{itemize}

\textbf{Dependencies:} M2 (for $J$).

\subsection*{M9.\; The Law of Mathematical Inevitability}
\textbf{Syllabus paper:} 35.\ The Law of Mathematical Inevitability
\hfill\textbf{File:} \texttt{Mathematics\_Ledger\_Phenomenon.tex}
\hfill\textcolor{red}{\textbf{NEW --- Feb 2026}}

\textbf{Statement.}
Any continuous cost functional satisfying (A1)--(A3) \emph{necessarily forces}:
\begin{enumerate}[nosep]
  \item \textbf{Natural numbers} as $\phig$-ladder positions $L(n)=\phig^n$,
    with the Fibonacci recursion $L(n{+}2)=L(n{+}1)+L(n)$.
  \item A \textbf{proof concept}: balanced ledger sequences
    ($\sum \ln r_i = 0$), the \emph{unique} admissible balance condition.
  \item \textbf{Incompleteness}: divergent cost of self-referential chains
    ($J(0^+)\to\infty$).
  \item A \textbf{universal referent}: the zero-cost subspace is the unique
    universal reference for all positive-cost objects (Wigner's puzzle resolved).
\end{enumerate}
The cost-consistent metric is $d(m,n) = J(\phig^{m-n})$; it satisfies all
metric axioms including the triangle inequality.

\textbf{Dependencies:} M2, M5, M8.

%======================================================================
\section{Layer 4 --- Operator Theory and Quotient Spaces}
\label{sec:layer4}
%======================================================================

\noindent\textit{``What is the dynamics and what is uniqueness at the level of
observables?''}

\textit{Physical motivation enters here: ``states,'' ``evolution,'' ``observation.''
The mathematics is operator theory on Hilbert and Banach spaces.}

\subsection*{M10.\; Recognition Operator}
\textbf{Syllabus paper:} 9.\ The Recognition Operator: Beyond the Hamiltonian
\hfill\textbf{File:} \texttt{Recognition-Operator.tex}
\hfill\textcolor{red}{\textbf{Updated --- Feb 2026}}

\textbf{Statement.}
A \emph{recognition operator} is a map $\hat{R}:\mathcal{S}\to\mathcal{S}$ on
admissible states satisfying: (i)~cost monotonicity $C(\hat{R}s)\le C(s)$,
(ii)~reciprocity preservation $\sigma(\hat{R}s)=\sigma(s)=0$, (iii)~period-8
update.

\textbf{Key analysis result.}
In the log-deviation coordinate $r=e^\varepsilon$:
\[
  J(e^\varepsilon) = \cosh(\varepsilon)-1
  = \tfrac{1}{2}\varepsilon^2 + \tfrac{1}{24}\varepsilon^4 + \cdots\,.
\]
The quadratic approximation $J\approx\frac{1}{2}\varepsilon^2$ has
relative error $<1\%$ for $|\varepsilon|\le 0.1$.  An effective quadratic
generator $\hat{H}_{\mathrm{eff}}$ exists such that $\hat{R} = \exp(-i\hat{H}_{\mathrm{eff}}\cdot 8\tau_0/\hbar) + O(\tau_0^2)$.
A continuum limit recovers the Schrödinger equation.

\textbf{Dependencies:} M2, M8.

\subsection*{M11.\; Model-Independent Exclusivity on the Quotient}
\textbf{Syllabus paper:} 6.\ Model-Independent Exclusivity
\hfill\textbf{File:} \texttt{Model-Independent-Exclusivity-Quotient.tex}

\textbf{Statement.}
Consider an abstract ``framework'' $(S, R, O, J)$ (state space, evolution,
observables, cost).  If $J$ satisfies (A1)--(A3) and the framework is
zero-parameter + self-similar, then:
\begin{itemize}[nosep]
  \item $J = $ the canonical $J$ (cost uniqueness on quotient),
  \item $\phig = (1{+}\sqrt{5})/2$ (preferred scale),
  \item the quotient state space $S/{\sim_O}$ is a subsingleton.
\end{itemize}
Any two such frameworks are \emph{observationally equivalent}.

\textbf{Dependencies:} M2, M3, M8.

\subsection*{M12.\; The Fredholm Index of the Death Operator}
\textbf{Syllabus paper:} 67.\ The Fredholm Index of Death
\hfill\textbf{File:} \texttt{Fredholm\_Index\_of\_Death.tex}
\hfill\textcolor{red}{\textbf{NEW --- Feb 2026}}

\textbf{Statement.}
A diagonal projection $\mathcal{D}:\mathbb{C}^8\to\mathbb{C}^8$ with survival
factors in $\{0,1\}$, decomposing $\mathbb{C}^8$ into kernel (channels
$1$--$3$) and image (channels $5$--$8$).  Channel $4$ is mixed.  The Fredholm
index is
\[
  \mathrm{ind}(\mathcal{D}) = k - 5,
\]
where $k\in\{0,1,\ldots,8\}$ is a reflexivity parameter.  The dimension of the
preserved subspace is bounded: $\dim(\mathrm{im}\,\mathcal{D}) \le \phig^k$.

\textbf{Dependencies:} M2 (for $\phig$-scaling), M10 (operator framework).

%======================================================================
\section{Layer 5 --- Number-Theoretic Applications}
\label{sec:layer5}
%======================================================================

\noindent\textit{``What does cost geometry say about primes?''}

\subsection*{M13.\; Riemann Hypothesis via Spectral Stability}
\textbf{Syllabus paper:} 34.\ A Weighted Diagonal Operator \ldots
\hfill\textbf{File:} \texttt{Recognition-Riemann-Final.tex}

\textbf{Summary.}
RS approach to RH: prime stiffness, log-prime spectrum, Bernstein inequality
for finite exponential sums, near-field elimination via energy barrier.

\textbf{Dependencies:} M2.

\subsection*{M14.\; Goldbach via a Mod-8 Kernel}
\textbf{Syllabus paper:} 36.\ Goldbach via Mod-8 Kernel
\hfill\textbf{File:} \texttt{goldbach\_rs-arXiv.tex}

\textbf{Summary.}
Classical circle method + $\chi_8$ character; mod-8 kernel concentrates
on the 8-tick-compatible residue classes.

\textbf{Dependencies:} M8 (ledger structure provides mod-8 frame).

\subsection*{M15.\; Prime Stiffness}
\textbf{Syllabus paper:} (within 34)

\textbf{Summary.}
Unconditional chain: prime gaps $\ge 1 \to$ log-prime spectrum strictly
increasing $\to$ effective bandwidth $= k\log T$ $\to$ Bernstein inequality
$\to$ near-field elimination (energy barrier argument).

\textbf{Dependencies:} M13.

%======================================================================
\section*{Summary Table}
%======================================================================

\begin{center}
\begin{tabular}{cllc}
\hline
\textbf{ID} & \textbf{Title} & \textbf{Layer} & \textbf{Deps} \\
\hline
M1  & Recognition Composition Law         & 0 & ---  \\
M2  & Cost Uniqueness (T5)                & 1 & M1   \\
M3  & D'Alembert Inevitability            & 1 & M1,M2 \\
M4  & Reciprocal Convex Costs             & 1 & M2   \\
M5  & $\phig$ Uniqueness / Penrose Bridge & 2 & M2   \\
M6  & Dimensional Rigidity $D{=}3$        & 2 & M5   \\
M7  & Stability Audit (RSA)               & 2 & M3   \\
M8  & Ledger Dynamics                     & 3 & M2   \\
M9  & Mathematical Inevitability          & 3 & M2,M5,M8 \\
M10 & Recognition Operator                & 4 & M2,M8 \\
M11 & Exclusivity on Quotient             & 4 & M2,M3,M8 \\
M12 & Fredholm Index                      & 4 & M2,M10 \\
M13 & Riemann / Spectral Stability        & 5 & M2 \\
M14 & Goldbach / Mod-8 Kernel             & 5 & M8 \\
M15 & Prime Stiffness                     & 5 & M13 \\
\hline
\end{tabular}
\end{center}

\bigskip\noindent
\textbf{How to read this as a mathematician.}
Start at M1 (a functional equation on $\mathbb{R}_{>0}$).  By M2 you know
the unique solution.  M3 shows you had no choice in the equation itself.
M5--M6 pin the algebraic scale and spatial dimension.  M8--M9 show the
equation forces discrete dynamics \emph{and} mathematics.  M10--M11 define the
operator theory and prove observational uniqueness.  M13--M15 are number-theoretic
applications.  At no point are you required to believe anything about physics;
the narrative is pure analysis, algebra, topology, and number theory.

\end{document}
