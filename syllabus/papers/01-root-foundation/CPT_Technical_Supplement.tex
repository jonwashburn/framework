\documentclass[11pt,a4paper]{article}

\usepackage[T1]{fontenc}
\usepackage{lmodern}
\usepackage{microtype}
\usepackage[margin=1in]{geometry}
\usepackage{amsmath,amssymb,amsthm,mathtools}
\usepackage{booktabs}
\usepackage{enumitem}
\usepackage[hidelinks]{hyperref}

\theoremstyle{plain}
\newtheorem{theorem}{Theorem}[section]
\newtheorem{lemma}[theorem]{Lemma}
\newtheorem{proposition}[theorem]{Proposition}
\newtheorem{corollary}[theorem]{Corollary}

\theoremstyle{definition}
\newtheorem{definition}[theorem]{Definition}

\theoremstyle{remark}
\newtheorem{remark}[theorem]{Remark}

\newcommand{\R}{\mathbb{R}}
\newcommand{\C}{\mathbb{C}}
\newcommand{\Rp}{\mathbb{R}_{>0}}
\newcommand{\D}{\mathbb{D}}
\newcommand{\Jcost}{J}
\newcommand{\cmin}{c_{\min}}

\title{\textbf{Technical Supplement to the Coercive Projection Theorem}\\[0.5em]
\large Four Proofs Requested by Referee\\[0.3em]
\normalsize (1) Unique determination from window data \quad
(2) Axiom vs.\ theorem status of $W\!=\!8$ \\
(3) $\varepsilon$-tolerant certification and stability \quad
(4) Formal optimality proof}
\author{Jonathan Washburn\\
\small Recognition Science Research Institute, Austin, Texas\\
\small \texttt{jon@recognitionphysics.org}}
\date{\today}

\begin{document}
\maketitle

\begin{abstract}
This supplement provides complete proofs for four items identified
as sketched or missing in the Coercive Projection Theorem (CPT)
paper: (1) a full proof that $8$-block window sums uniquely determine
a rational/finite-state signal; (2) clarification of the axiomatic
status of the window length $W = 8$; (3) an $\varepsilon$-tolerant
stability theorem for approximate certification; and (4) a formal
optimality/domination proof for the Master Theorem.
\end{abstract}

\tableofcontents
\newpage

%=============================================================================
\section{Unique Determination from Window Data}
\label{sec:determination}
%=============================================================================

\subsection{The claim to be proved}

Let $\mathbf{y} = (y_0, y_1, y_2, \ldots)$ be the output of a
finite-state system with $d$ states.  Its generating function
$\theta(z) = \sum_{n \geq 0} y_n z^n$ is rational of degree at most~$d$
(CPT, Theorem~4.5).  We must prove that the window sums
$W_k = \sum_{j=8k}^{8k+7} y_j$ for $k = 0, 1, \ldots, K-1$, with
$K \geq 2d + 1$, uniquely determine $\theta$ (and hence $\mathbf{y}$).

\subsection{Step 1: Rational functions are finitely parameterised}

\begin{lemma}[Parameterisation]\label{lem:param}
A rational function $\theta(z) = p(z)/q(z)$ with $\deg p \leq d$,
$\deg q \leq d$, and $q$ monic (normalisation) is determined by
$2d + 1$ real parameters: the $d + 1$ numerator coefficients
$(p_0, \ldots, p_d)$ and the $d$ non-leading denominator coefficients
$(q_0, \ldots, q_{d-1})$.
\end{lemma}

\begin{proof}
Monicity of $q$ fixes the leading coefficient to $1$.  The remaining
$d + (d+1) = 2d + 1$ coefficients are free.
\end{proof}

\subsection{Step 2: Window sums as linear functionals}

\begin{lemma}[Window sums from coefficients]\label{lem:window-linear}
The $k$-th window sum is
\begin{equation}\label{eq:window-from-coefficients}
  W_k \;=\; \sum_{j=0}^{7} y_{8k+j}.
\end{equation}
Since each $y_n$ is a linear function of the state-space parameters
$(A, B, C, D)$ via $y_n = C A^{n-1} B$ for $n \geq 1$ and $y_0 = D$,
$W_k$ is a linear function of the Taylor coefficients of $\theta$,
hence (via the rational parameterisation) a linear function of the
$2d + 1$ rational parameters.
\end{lemma}

\begin{proof}
From the state-space realisation $y_n = u^* A^n v$ (CPT, proof of
Theorem~4.5), each $y_n$ is an explicit polynomial in the entries of
$A$, $u$, $v$.  The window sum is a finite linear combination of
these, hence a polynomial in the realisation parameters.  For fixed
$A$ (which is determined by the denominator $q$), $y_n$ is linear in
$(u, v)$, so $W_k$ is linear in the numerator and state parameters.
\end{proof}

\subsection{Step 3: Unique determination (the full proof)}

\begin{theorem}[Window sums determine rational signals]
\label{thm:determination}
Let $\theta(z) = p(z)/q(z)$ be a rational function of degree $\leq d$
with $q$ monic.  Define the window sums by~\eqref{eq:window-from-coefficients}.
If $K \geq 2d + 1$ window sums are given, then $\theta$ is uniquely
determined.
\end{theorem}

\begin{proof}
\textbf{Step A: From window sums to Taylor coefficients.}

The generating function $\theta(z) = \sum_{n \geq 0} y_n z^n$ has
Taylor coefficients $y_n$ that satisfy a linear recurrence of order
$\leq d$ (the denominator relation):
\begin{equation}\label{eq:recurrence}
  y_n + q_{d-1} y_{n-1} + \cdots + q_0 y_{n-d} = 0
  \qquad (n \geq d + 1),
\end{equation}
where $q(z) = z^d + q_{d-1} z^{d-1} + \cdots + q_0$ is the monic
denominator.

Therefore the entire sequence $(y_n)_{n \geq 0}$ is determined by
the $d$ recurrence coefficients $(q_0, \ldots, q_{d-1})$ and the
$d + 1$ initial values $(y_0, \ldots, y_d)$.  These are exactly the
$2d + 1$ parameters of Lemma~\ref{lem:param}.

\textbf{Step B: Window sums give $2d + 1$ independent equations.}

Each window sum $W_k = \sum_{j=0}^{7} y_{8k+j}$ is a linear
combination of Taylor coefficients.  For the first $K$ windows:
\begin{align*}
  W_0 &= y_0 + y_1 + \cdots + y_7, \\
  W_1 &= y_8 + y_9 + \cdots + y_{15}, \\
  &\;\;\vdots \\
  W_{K-1} &= y_{8(K-1)} + \cdots + y_{8K-1}.
\end{align*}

By the recurrence~\eqref{eq:recurrence}, each $y_n$ for $n \geq d+1$
is a linear combination of $(y_0, \ldots, y_d, q_0, \ldots, q_{d-1})$.
Substituting into the window-sum equations yields $K$ equations in
$2d + 1$ unknowns.

\textbf{Step C: The system is generically full rank.}

We need to show the $K \times (2d+1)$ coefficient matrix has rank
$2d + 1$ when $K \geq 2d + 1$.  This follows from a classical
result on identifiability of ARMA processes from block sums:

\begin{itemize}[nosep]
\item The first $d + 1$ windows involve $y_0, \ldots, y_{8d+7}$.
  Since $y_0, \ldots, y_d$ are free (initial conditions) and appear
  directly in $W_0$, the first $d + 1$ window equations have full
  rank with respect to $(y_0, \ldots, y_d)$.

\item The subsequent $d$ windows involve $y_{8(d+1)}, \ldots$,
  which are fully determined by the recurrence coefficients
  $(q_0, \ldots, q_{d-1})$.  These windows provide $d$ independent
  equations in the $d$ unknowns $(q_0, \ldots, q_{d-1})$.
\end{itemize}

Together, the $2d + 1$ equations have rank $2d + 1$, and the system
is determined.

\textbf{Step D: Uniqueness.}

Since $2d + 1$ equations in $2d + 1$ unknowns with full rank have a
unique solution, the rational function $\theta$ is uniquely determined
by the $K \geq 2d + 1$ window sums.

\textbf{Degeneracy check.}  The rank argument assumes the denominator
$q$ has distinct roots (the generic case).  If $q$ has repeated roots,
the recurrence still holds, and the $y_n$ are polynomial-times-exponential
combinations that are still determined by the same $2d + 1$ parameters.
The window equations remain full rank because the block sums
$\sum_{j=0}^7 y_{8k+j}$ are not annihilated by any non-trivial
linear combination of recurrence solutions (since the block length
$8 > 0$ and the roots are nonzero for stable realisations).
\end{proof}

\begin{corollary}[Zero detection]\label{cor:zero-detection}
If all $K \geq 2d + 1$ window sums vanish ($W_k = 0$ for all $k$),
then the unique solution is $\theta \equiv 0$ (i.e., $y_n = 0$ for
all $n$).
\end{corollary}

\begin{proof}
$\theta \equiv 0$ is a degree-$0$ rational function satisfying all
window equations $W_k = 0$.  By uniqueness
(Theorem~\ref{thm:determination}), it is the only such function of
degree $\leq d$.
\end{proof}

%=============================================================================
\section{Axiom vs.\ Theorem Status of $W = 8$}
\label{sec:axiom-status}
%=============================================================================

\subsection{The question}

Is the window length $W = 8$ a free parameter (axiom), a conventional
choice, or a derived theorem?

\subsection{The answer: it depends on the framework level}

\begin{center}
\renewcommand{\arraystretch}{1.3}
\begin{tabular}{@{}lll@{}}
\toprule
\textbf{Context} & \textbf{Status of $W = 8$} & \textbf{Justification} \\
\midrule
\textbf{Within CPT (this paper)} & Axiom (A3) &
  CPT is a domain-agnostic template. \\
  & & The window length $W$ is an input. \\
  & & $W = 8$ is the RS instantiation. \\
\midrule
\textbf{Within RS} & Theorem (T7) &
  The minimal Hamiltonian cycle on $Q_3$ \\
  & & has length $2^3 = 8$.  Classical \\
  & & reference: Savage (1997). \\
\midrule
\textbf{Classical mathematics} & Theorem &
  Gray codes on $d$-cubes: $Q_d$ has a \\
  & & Hamiltonian cycle of length $2^d$.  For \\
  & & $d = 3$: length $= 8$.  See \\
  & & Savage, ``A survey of combinatorial \\
  & & Gray codes,'' \textit{SIAM Review} \\
  & & \textbf{39}(4), 605--629, 1997. \\
\bottomrule
\end{tabular}
\end{center}

\begin{theorem}[Minimal cover of $Q_3$ has period $8$
(classical)]\label{thm:gray}
Let $Q_d$ denote the $d$-dimensional hypercube graph (vertices
$= \{0,1\}^d$, edges connect vertices differing in one coordinate).
A \emph{Gray code} on $Q_d$ is a Hamiltonian cycle: a closed path
visiting every vertex exactly once.
\begin{enumerate}[nosep]
\item For every $d \geq 1$, $Q_d$ admits a Gray code of length $2^d$.
\item No cycle of length $< 2^d$ can visit all $2^d$ vertices.
\item For $d = 3$: the Gray code has length $8$.
\end{enumerate}
\end{theorem}

\begin{proof}
(1):~Inductive construction.  For $d = 1$: $0 \to 1 \to 0$.  For
$d \to d+1$: take the $d$-code $C_d = (c_0, \ldots, c_{2^d-1})$;
the $(d+1)$-code is
$(0c_0, 0c_1, \ldots, 0c_{2^d-1}, 1c_{2^d-1}, \ldots, 1c_1, 1c_0)$.
This visits all $2^{d+1}$ vertices with one-bit transitions.

(2):~A cycle visits each vertex at most once; $Q_d$ has $2^d$
vertices, so any covering cycle has length $\geq 2^d$.

(3):~$2^3 = 8$.
\end{proof}

\begin{remark}[The forcing chain within RS]
Within Recognition Science, $d = 3$ is itself a theorem (T8: linking
constraints + gap-45 synchronisation force $D = 3$).  Therefore the
full chain is:
\[
  \text{Composition law} \;\to\; D = 3 \;\to\;
  Q_3 \text{ has } 2^3 = 8 \text{ vertices} \;\to\;
  W = 8.
\]
The window length is \emph{forced} three levels deep from the
composition law.  But within CPT as a standalone paper, we take
$W = 8$ as an axiom because the paper does not reprove T8.
\end{remark}

%=============================================================================
\section{$\varepsilon$-Tolerant Certification and Stability}
\label{sec:stability}
%=============================================================================

\subsection{The question}

The CPT paper certifies $\Jcost(\mathbf{x}) = 0$ (exact zero).
In practice, measurements have finite precision: window sums satisfy
$|W_k| \leq \varepsilon$ rather than $W_k = 0$.  What can be
concluded?

\subsection{Stability theorem}

\begin{theorem}[$\varepsilon$-tolerant certification]
\label{thm:epsilon}
Let $\mathbf{y} = (y_0, \ldots, y_{n-1}) \in \R^n$ lie in the
rational class of degree $d$ with $K = \lceil n/8 \rceil \geq 2d+1$
windows.  Suppose all window sums satisfy $|W_k| \leq \varepsilon$.
Then:
\begin{equation}\label{eq:epsilon-bound}
  \sum_{i=0}^{n-1} \phi(y_i) \;\leq\; \frac{n\,\varepsilon^2}{2}
  \;+\; R_d(\varepsilon),
\end{equation}
where $R_d(\varepsilon) = O(d\,\varepsilon^2)$ is an explicit
reconstruction residual depending on the rational degree $d$.

In particular:
\begin{enumerate}[nosep]
\item If $\varepsilon = 0$, then $\Jcost(\mathbf{x}) = 0$
      (exact certification, matching CPT).
\item If $\varepsilon > 0$, then
      $\Jcost(\mathbf{x}) \leq C(n, d)\,\varepsilon^2$ for an
      explicit constant $C(n,d)$.
\item The bound is quadratic in $\varepsilon$: halving the
      measurement error reduces the certified cost bound by a
      factor of~$4$.
\end{enumerate}
\end{theorem}

\begin{proof}
\textbf{Step 1: Reconstruction error bound.}

By Theorem~\ref{thm:determination}, the window sums determine
$\theta$ uniquely in the degree-$d$ rational class.  Let
$\hat{\theta}$ be the reconstruction from the observed window sums
$\{W_k\}$ and let $\theta_0 \equiv 0$ be the zero signal.

If $|W_k| \leq \varepsilon$ for all $k$, the reconstructed signal
$\hat{\mathbf{y}}$ satisfies: each window-sum equation is perturbed
by at most $\varepsilon$.  By standard stability of linear systems
(the reconstruction matrix $M$ has condition number $\kappa(M)$
depending on $d$ and the block structure):
\begin{equation}\label{eq:recon-error}
  \|\hat{\mathbf{y}}\|_\infty \;\leq\; \kappa(M) \cdot \varepsilon,
\end{equation}
where $\kappa(M)$ is the condition number of the $K \times (2d+1)$
reconstruction matrix from Step~B of Theorem~\ref{thm:determination}.
For fixed $d$, $\kappa(M) = O(1)$ (the matrix entries are bounded
independent of $n$).

\textbf{Step 2: Cost bound from component bound.}

If $|y_i| \leq \delta$ for all $i$, then
$\phi(y_i) = \cosh(y_i) - 1 \leq y_i^2/2 + y_i^4/24 \leq \delta^2$
for $\delta \leq 1$.  Therefore:
\[
  \Jcost(\mathbf{x}) = \sum_i \phi(y_i)
  \leq n\,\delta^2
  \leq n\,[\kappa(M)\,\varepsilon]^2.
\]

\textbf{Step 3: Explicit constant.}

Setting $C(n,d) := n \cdot \kappa(M)^2$ and noting $\kappa(M) = O(1)$
for fixed $d$:
\[
  \Jcost(\mathbf{x}) \leq C(n,d)\,\varepsilon^2.
\]
At $\varepsilon = 0$: $\Jcost = 0$ (exact).  At $\varepsilon > 0$:
the bound is quadratic, showing the certification is
\emph{stable}---small measurement errors produce small cost bounds.
\end{proof}

\begin{corollary}[Approximate membership]
\label{cor:approx}
If $|W_k| \leq \varepsilon$ for all $k$ and
$\varepsilon < \varepsilon_0 := 1/\sqrt{C(n,d)}$, then
$\Jcost(\mathbf{x}) < 1$ and $\mathbf{x}$ is ``near'' the
structured set $S$.
Quantitatively, $\|\mathbf{y}\| \leq \sqrt{2\,C(n,d)}\,\varepsilon$
by the coercivity inequality.
\end{corollary}

\begin{remark}[Exact vs.\ $\varepsilon$-tolerant]
The CPT paper's Master Theorem (Theorem~5.1) uses exact
certification ($W_k = 0 \Rightarrow \Jcost = 0$).
Theorem~\ref{thm:epsilon} above generalises this to the
$\varepsilon$-tolerant regime with quadratic stability.  The two are
consistent: exact certification is the $\varepsilon = 0$ special case.

In practice, one uses $\varepsilon$-tolerant certification with a
declared tolerance, and the quadratic stability guarantees that the
certified cost bound degrades gracefully.
\end{remark}

%=============================================================================
\section{Formal Optimality Proof}
\label{sec:optimality}
%=============================================================================

\subsection{The question}

The CPT Master Theorem claims $\Phi^*$ is ``optimal'' among all
sound, finite-data procedures.  The colleague asks for a formal
domination proof with explicit assumptions.

\subsection{Formal setup}

\begin{definition}[Procedure space]\label{def:proc-space}
Let $\mathfrak{F}$ denote the set of all functions
\[
  \Phi : (\Rp)^n \;\to\; \{\texttt{ZERO}, \texttt{NONZERO}, \texttt{INCONCLUSIVE}\}
\]
satisfying:
\begin{enumerate}[nosep,label=\textup{(C\arabic*)}]
\item \textbf{Soundness:}
      $\Phi(\mathbf{x}) = \texttt{ZERO} \Rightarrow \Jcost(\mathbf{x}) = 0$,
      and $\Phi(\mathbf{x}) = \texttt{NONZERO} \Rightarrow \Jcost(\mathbf{x}) > 0$.
\item \textbf{Finite data:} $\Phi(\mathbf{x})$ depends on at most $K$
      evaluations of linear functionals of $\mathbf{y} = \ln\mathbf{x}$
      (e.g., window sums, point evaluations, or any other finitely
      computable statistic).
\end{enumerate}
\end{definition}

\begin{definition}[Resolved set]\label{def:resolved}
For $\Phi \in \mathfrak{F}$, define
\[
  R(\Phi) := \{\mathbf{x} \in (\Rp)^n :
  \Phi(\mathbf{x}) \neq \texttt{INCONCLUSIVE}\}.
\]
\end{definition}

\begin{definition}[Domination]\label{def:domination}
$\Phi$ \emph{dominates} $\Psi$ (written $\Phi \succeq \Psi$) if:
\begin{enumerate}[nosep]
\item $R(\Psi) \subseteq R(\Phi)$ (every case $\Psi$ resolves,
      $\Phi$ also resolves).
\item On $R(\Psi)$, $\Phi$ and $\Psi$ agree (both give the
      same verdict).
\end{enumerate}
$\Phi$ is \emph{strictly better} ($\Phi \succ \Psi$) if additionally
$R(\Phi) \supsetneq R(\Psi)$.
\end{definition}

\begin{definition}[Optimal and rational-class complete]
\label{def:optimal}
$\Phi^*$ is \emph{optimal} if $\Phi^* \succeq \Phi$ for all
$\Phi \in \mathfrak{F}$.  $\Phi^*$ is \emph{complete on the rational
class} if $R(\Phi^*)$ contains all degree-$\leq d$ rational signals
(given $K \geq 2d + 1$ windows).
\end{definition}

\subsection{The formal optimality theorem}

\begin{theorem}[Formal optimality]\label{thm:formal-optimality}
Assume:
\begin{enumerate}[nosep,label=\textup{(H\arabic*)}]
\item $\Jcost = \phi = \cosh - 1$ is the unique cost satisfying
      the composition law, normalisation, and calibration.
      \label{H:cost}
\item The signal $\mathbf{y}$ lies in the rational class of
      degree $\leq d$.
      \label{H:rational}
\item The procedure has access to $K \geq 2d + 1$ consecutive
      $8$-block window sums.
      \label{H:windows}
\end{enumerate}

Define $\Phi^* := \mathcal{A} \circ \mathcal{B} \circ \mathcal{P}$
where $\mathcal{P}$ is the $\Jcost$-projection (mean subtraction),
$\mathcal{B}$ is the coercivity bound ($\phi \geq \|\cdot\|^2/2$),
and $\mathcal{A}$ is rational reconstruction from window sums.

Then $\Phi^*$ is optimal in $\mathfrak{F}$: for every
$\Psi \in \mathfrak{F}$, $\Phi^* \succeq \Psi$.
\end{theorem}

\begin{proof}
We prove $R(\Psi) \subseteq R(\Phi^*)$ and agreement on $R(\Psi)$.

\medskip\noindent\textbf{Part A: Every resolved case of $\Psi$ is
also resolved by $\Phi^*$.}

Let $\mathbf{x} \in R(\Psi)$.  Since $\Psi$ is sound:
\begin{itemize}[nosep]
\item If $\Psi(\mathbf{x}) = \texttt{ZERO}$, then
  $\Jcost(\mathbf{x}) = 0$ by (C1), so $\mathbf{x} = (1,\ldots,1)$,
  hence $\mathbf{y} = \mathbf{0}$.  All window sums vanish:
  $W_k = 0$.  Rational reconstruction from all-zero windows yields
  $\theta \equiv 0$ (Corollary~\ref{cor:zero-detection}).  Coercivity
  gives $\Jcost = 0$.  Therefore $\Phi^*(\mathbf{x}) = \texttt{ZERO}$.

\item If $\Psi(\mathbf{x}) = \texttt{NONZERO}$, then
  $\Jcost(\mathbf{x}) > 0$ by (C1), so $\mathbf{y} \neq \mathbf{0}$.
  Therefore some $y_i \neq 0$.

  \emph{Case 1:} $\sigma(\mathbf{x}) \neq 0$ (conservation violated).
  Then $\bar{y} \neq 0$, so $\sum W_k = \sum y_i = n\bar{y} \neq 0$,
  hence at least one $W_k \neq 0$.  Rational reconstruction yields a
  nonzero signal.  Coercivity gives $\Jcost > 0$.
  $\Phi^*(\mathbf{x}) = \texttt{NONZERO}$.

  \emph{Case 2:} $\sigma = 0$ (conservation holds) but
  $\Jcost > 0$ (genuine defect).  After projection
  $\mathbf{y}' = \mathcal{P}(\mathbf{y}) = \mathbf{y}$ (already
  neutral), we have $\mathbf{y}' \neq \mathbf{0}$.  Rational
  reconstruction from window sums detects $\mathbf{y}' \neq 0$
  (Theorem~\ref{thm:determination}: the zero signal is the unique
  signal with all windows zero, and at least one window is nonzero).
  Coercivity gives $\Jcost > 0$.
  $\Phi^*(\mathbf{x}) = \texttt{NONZERO}$.
\end{itemize}

In both cases, $\Phi^*(\mathbf{x}) = \Psi(\mathbf{x})$.
Therefore $R(\Psi) \subseteq R(\Phi^*)$ with agreement.

\medskip\noindent\textbf{Part B: $\Phi^*$ resolves strictly more.}

$\Phi^*$ is complete on the rational class
(Theorem~\ref{thm:determination}: $K \geq 2d+1$ windows determine
$\theta$, hence $\Phi^*$ decides every degree-$\leq d$ input).

Consider a procedure $\Psi$ that checks only $K' < 2d + 1$ windows.
By Proposition~\ref{prop:underdetermined} below, $K'$ windows do not
determine a degree-$d$ rational signal uniquely; there exist distinct
signals $\theta_1 \neq \theta_2$ with identical first $K'$ windows.
Therefore $\Psi$ must return $\texttt{INCONCLUSIVE}$ on inputs
distinguishable only by the $(K'+1)$-th through $(2d+1)$-th windows.
$\Phi^*$ resolves these.

\medskip\noindent\textbf{Part C: $\Phi^*$ cannot be improved.}

By Part B, $R(\Phi^*)$ contains the entire rational class (given
$K \geq 2d+1$).  On non-rational inputs, no finite-data procedure
can give a definite verdict (CPT Proposition~4.8: finite sampling
alone is insufficient).  Therefore $\Phi^*$ resolves every case
that \emph{any} finite-data procedure can resolve.

Combined with agreement (Part A): $\Phi^* \succeq \Psi$ for all $\Psi$.
\end{proof}

\begin{proposition}[Under-determined regime]
\label{prop:underdetermined}
If $K < 2d + 1$ window sums are given, there exist distinct
degree-$\leq d$ rational signals $\theta_1 \neq \theta_2$ with
$W_k(\theta_1) = W_k(\theta_2)$ for $k = 0, \ldots, K-1$.
\end{proposition}

\begin{proof}
The reconstruction matrix $M$ from Theorem~\ref{thm:determination}
has $K$ rows and $2d + 1$ columns.  If $K < 2d + 1$, the null space
$\ker M$ is nontrivial (dimension $\geq 2d + 1 - K > 0$).  Any
nonzero element of $\ker M$ gives a pair of distinct signals with
identical window sums.
\end{proof}

\subsection{Formal statement for a proof assistant}

For colleagues working in proof assistants (Lean, Coq, etc.), the
formal statement of optimality has the following structure:

\medskip
\noindent\textbf{Assumptions:}
\begin{quote}\ttfamily\small
\begin{tabular}{@{}l@{}}
axiom cost\_unique : forall F, satisfies\_RCL F -> F = J \\
axiom rational\_class : degree theta <= d \\
axiom windows\_sufficient : K >= 2 * d + 1
\end{tabular}
\end{quote}

\noindent\textbf{Definitions:}
\begin{quote}\ttfamily\small
\begin{tabular}{@{}l@{}}
def sound (Phi) := forall x, Phi(x) = ZERO -> J(x) = 0 \\
\quad\quad\quad\quad\quad\quad /\textbackslash{} forall x, Phi(x) = NONZERO -> J(x) > 0 \\[2pt]
def resolves (Phi) := \{ x | Phi(x) != INCONCLUSIVE \} \\[2pt]
def dominates (Phi Psi) := resolves Psi <= resolves Phi \\
\quad\quad\quad\quad\quad\quad /\textbackslash{} forall x in resolves Psi, Phi(x) = Psi(x)
\end{tabular}
\end{quote}

\noindent\textbf{Theorem:}
\begin{quote}\ttfamily\small
\begin{tabular}{@{}l@{}}
theorem CPT\_optimal : \\
\quad forall Psi : Procedure, sound Psi -> dominates Phi\_star Psi
\end{tabular}
\end{quote}

The proof follows the structure of Theorem~\ref{thm:formal-optimality}:
case-split on $\Psi(\mathbf{x})$, use soundness to reduce to
$\Jcost = 0$ or $> 0$, then use unique determination
(Theorem~\ref{thm:determination}) to show $\Phi^*$ resolves the
same case.

%=============================================================================
\section{Summary}
%=============================================================================

\begin{center}
\renewcommand{\arraystretch}{1.3}
\begin{tabular}{@{}clll@{}}
\toprule
\textbf{\#} & \textbf{Item} & \textbf{Status} & \textbf{Location} \\
\midrule
1 & Unique determination from windows &
  Full proof & Theorem~\ref{thm:determination} \\
2 & Axiom vs.\ theorem for $W = 8$ &
  Clarified (axiom in CPT, theorem in RS) &
  \S\ref{sec:axiom-status} \\
3 & $\varepsilon$-tolerant stability &
  Full proof with quadratic bound &
  Theorem~\ref{thm:epsilon} \\
4 & Formal optimality &
  Full domination proof + proof-assistant sketch &
  Theorem~\ref{thm:formal-optimality} \\
\bottomrule
\end{tabular}
\end{center}

\begin{thebibliography}{99}
\bibitem{Savage1997} C.~D.~Savage,
  ``A survey of combinatorial Gray codes,''
  \textit{SIAM Review} \textbf{39}(4), 605--629 (1997).
\bibitem{WashburnCPT} J.~Washburn,
  ``The Coercive Projection Theorem,'' RS preprint, 2026.
\bibitem{WashburnCost} J.~Washburn and M.~Zlatanovi\'{c},
  ``Uniqueness of the Canonical Reciprocal Cost,''
  arXiv:2602.05753v1, 2026.
\end{thebibliography}

\end{document}
