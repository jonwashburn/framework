\documentclass[11pt,a4paper]{article}
\usepackage[margin=1in]{geometry}
\usepackage[T1]{fontenc}
\usepackage{lmodern}
\usepackage{microtype}
\usepackage{amsmath,amssymb,amsthm,mathtools}
\usepackage{booktabs}
\usepackage{enumitem}
\usepackage{array}
\usepackage[hidelinks,breaklinks]{hyperref}
\usepackage{tikz}
\usetikzlibrary{arrows.meta,positioning,calc}

% ---------- theorem environments ----------
\theoremstyle{plain}
\newtheorem{theorem}{Theorem}[section]
\newtheorem{proposition}[theorem]{Proposition}
\newtheorem{lemma}[theorem]{Lemma}
\newtheorem{corollary}[theorem]{Corollary}
\theoremstyle{definition}
\newtheorem{definition}[theorem]{Definition}
\newtheorem{example}[theorem]{Example}
\theoremstyle{remark}
\newtheorem{remark}[theorem]{Remark}

% ---------- notation ----------
\newcommand{\phig}{\varphi}
\newcommand{\psig}{\bar\varphi}          % conjugate root
\newcommand{\Jcost}{J}
\newcommand{\RR}{\mathbb{R}}
\newcommand{\ZZ}{\mathbb{Z}}
\newcommand{\CC}{\mathbb{C}}
\newcommand{\NN}{\mathbb{N}}
\newcommand{\Rhat}{\hat{R}}
\DeclareMathOperator{\Nm}{N}             % norm
\DeclareMathOperator{\tr}{tr}            % trace
\newcommand{\conj}{\sigma}               % Galois conjugation

\title{\textbf{Recognition Algebra:\\
The Unified Algebraic Framework Forced\\by a Single Composition Law}\\[0.5em]
\large A Self-Contained Mathematical Treatment}
\author{Jonathan Washburn\\
\small Recognition Science Research Institute, Austin, Texas\\
\small \texttt{washburn.jonathan@gmail.com}}
\date{February 2026}

\begin{document}
\maketitle

\begin{abstract}
A single functional equation on $\RR_{>0}$,
\[
  \Jcost(xy)+\Jcost(x/y)
  = 2\,\Jcost(x)\,\Jcost(y)+2\,\Jcost(x)+2\,\Jcost(y),
\]
together with the normalization $\Jcost(1)=0$ and the calibration
$\Jcost''_{\!\log}(0)=1$, uniquely determines the cost function
$\Jcost(x)=\tfrac12(x+x^{-1})-1$.  We show that this determination
forces, with no further choices, four interlocking algebraic structures:
\begin{enumerate}[nosep,label=(\roman*)]
\item a commutative monoid $(\RR_{\ge0},\star,0)$ with
  $a\star b=2ab+2a+2b$;
\item the real-quadratic ring $\ZZ[\phig]$, where
  $\phig=(1{+}\sqrt5)/2$ is a unit of norm~$-1$;
\item a graded abelian group of paired events satisfying a global
  balance constraint $\sigma=0$; and
\item the cyclic phase group $\ZZ/8\ZZ$ with a canonical DFT-8
  spectral decomposition admitting exactly 20 basis modes.
\end{enumerate}
We call the resulting quadruple \emph{Recognition Algebra}.
We define a category $\mathbf{RecAlg}$ whose objects are cost
algebras satisfying the composition law and whose morphisms are
multiplicative, cost-preserving maps.  Recognition Algebra is the
\emph{initial object} of this category: every calibrated cost algebra
receives a unique morphism from it.  In particular, no free parameters
remain.

The paper is self-contained; every claim is accompanied by a proof
or an explicit proof sketch.  A companion Lean~4 formalization
(six modules, ${\sim}1{,}200$ lines) independently certifies the
principal theorems~\cite{lean2015}.

\medskip\noindent\textbf{Keywords:} functional equations, golden ratio,
d'Alembert equation, cost function, zero-parameter framework, category
theory.
\end{abstract}

\tableofcontents

%======================================================================
\section{Introduction}\label{sec:intro}
%======================================================================

Unification in physics has always meant discovering that two apparently
independent phenomena are governed by the same algebraic structure.
Newton showed that the fall of an apple and the orbit of the Moon obey
one law of gravitation.  Maxwell showed that electricity and magnetism
are components of a single antisymmetric tensor.  Einstein showed that
space and time are coordinates in a single pseudo-Riemannian manifold.

Each of these advances began with an algebraic identity that
\emph{forced} downstream structure---once the identity was granted,
everything else followed.  This paper exhibits such an identity at the
most fundamental level and traces all of the algebraic consequences
that it forces.

\subsection{The starting point}

We begin with a single functional equation.

\begin{definition}[Recognition Composition Law]\label{def:RCL}
A function $\Jcost:\RR_{>0}\to\RR$ satisfies the \emph{Recognition
Composition Law} (RCL) if, for all $x,y>0$,
\begin{equation}\label{eq:RCL}
  \Jcost(xy)+\Jcost\!\bigl(x/y\bigr)
  =2\,\Jcost(x)\,\Jcost(y)+2\,\Jcost(x)+2\,\Jcost(y).
\end{equation}
\end{definition}

Supplementing~\eqref{eq:RCL} with two constraints,
\begin{equation}\label{eq:norm-calib}
  \Jcost(1)=0,\qquad
  \left.\frac{d^2}{dt^2}\right|_{t=0}\Jcost(e^t)=1,
\end{equation}
we will show that $\Jcost$ is \emph{uniquely} determined, and that
four algebraic layers---cost composition, the golden ratio, double-entry
conservation, and period-8 phase structure---emerge inevitably.

\subsection{Relation to classical functional equations}

Equation~\eqref{eq:RCL} is a \emph{calibrated multiplicative form of
the d'Alembert functional equation}.  Under the logarithmic
substitution $t=\ln x$, $u=\ln y$, define $G(t)=\Jcost(e^t)$.
Then~\eqref{eq:RCL} becomes
\begin{equation}\label{eq:Gcosh}
  G(t{+}u)+G(t{-}u)=2\,G(t)\,G(u)+2\bigl(G(t)+G(u)\bigr).
\end{equation}
Setting $H(t)=G(t)+1$, one obtains
\begin{equation}\label{eq:dAlembert}
  H(t{+}u)+H(t{-}u)=2\,H(t)\,H(u),
\end{equation}
which is the standard d'Alembert (cosine) functional equation.
Its continuous solutions are $H(t)=\cosh(\kappa t)$ for $\kappa\ge0$
and $H(t)=\cos(\kappa t)$ for $\kappa\ge0$
(see Acz\'el~\cite{aczel1966}, Acz\'el--Dhombres~\cite{aczel_dhombres1989}).
The calibration $H''(0)=1$ forces $\kappa=1$ and selects $\cosh$ over
$\cos$ (since $\cos''(0)=-1$).  Therefore $H(t)=\cosh t$,
i.e.\ $G(t)=\cosh t-1$.

\subsection{Outline}

In~\S\ref{sec:cost} we develop the \textbf{Cost Algebra}: the algebraic
properties of $\Jcost$ itself, including a commutative monoid on cost
values.  In~\S\ref{sec:phi} we show that $\Jcost$ forces the
\textbf{$\phig$-Ring} $\ZZ[\phig]$ and derive its ring-theoretic
properties.  In~\S\ref{sec:ledger} we prove that reciprocal symmetry
forces a \textbf{Ledger Algebra} of paired events with a global balance
constraint.  In~\S\ref{sec:octave} we construct the
\textbf{Octave Algebra}, a period-8 structure with a canonical spectral
basis.  These four layers are assembled into a single structure
in~\S\ref{sec:unified} and given categorical treatment
in~\S\ref{sec:category}.


%======================================================================
\section{The Cost Algebra}\label{sec:cost}
%======================================================================

\subsection{Uniqueness of the cost function}

\begin{theorem}[Uniqueness -- T5]\label{thm:T5}
Let $\Jcost:\RR_{>0}\to\RR$ be continuous, satisfy the
RCL~\eqref{eq:RCL}, and obey the normalization and calibration
conditions~\eqref{eq:norm-calib}.  Then
\begin{equation}\label{eq:Jdef}
  \Jcost(x)=\tfrac12\bigl(x+x^{-1}\bigr)-1
  \qquad\text{for all } x>0.
\end{equation}
\end{theorem}

\begin{proof}[Proof sketch]
As shown in \S\ref{sec:intro}, the substitution
$H(t)=\Jcost(e^t)+1$ converts the RCL to the d'Alembert
equation~\eqref{eq:dAlembert} with $H(0)=1$.  By the
Acz\'el--Dhombres classification~\cite{aczel_dhombres1989},
every continuous solution is $H(t)=\cosh(\kappa t)$.  The calibration
$H''(0)=1$ gives $\kappa^2=1$, so $\kappa=1$.  Undoing the
substitution yields~\eqref{eq:Jdef}.
\end{proof}

\subsection{Elementary properties}

\begin{proposition}[Cost algebra identities]\label{prop:J-props}
The function~\eqref{eq:Jdef} satisfies, for every $x>0$:
\begin{enumerate}[nosep,label=\textup{(\alph*)}]
\item\label{it:norm} $\Jcost(1)=0$.
\item\label{it:recip} $\Jcost(x)=\Jcost(x^{-1})$
      \textup{(reciprocal symmetry)}.
\item\label{it:nonneg} $\Jcost(x)\ge0$, with equality iff $x=1$.
\item\label{it:defect} $\Jcost(x)=(x-1)^2/(2x)$
      \textup{(defect form)}.
\end{enumerate}
\end{proposition}

\begin{proof}
\ref{it:norm}:
$\Jcost(1)=\frac12(1+1)-1=0$.
\ref{it:recip}:
$\Jcost(x^{-1})=\frac12(x^{-1}+x)-1=\Jcost(x)$.
\ref{it:defect}: Clearing fractions,
$\frac12(x+x^{-1})-1=\frac{x^2+1-2x}{2x}=\frac{(x-1)^2}{2x}$.
\ref{it:nonneg}: $(x-1)^2\ge0$ and $x>0$, so the ratio is
non-negative, and vanishes iff $x=1$.
\end{proof}

\subsection{Cost composition}\label{ssec:star}

The RCL expresses how costs combine when ratios are multiplied.
If we write $a=\Jcost(x)$ and $b=\Jcost(y)$, the \emph{right-hand side}
of~\eqref{eq:RCL} depends only on $a$ and~$b$:

\begin{definition}[Cost composition]\label{def:star}
For $a,b\in\RR$, set
\begin{equation}\label{eq:star}
  a\star b \;=\; 2ab+2a+2b \;=\; 2(a{+}1)(b{+}1)-2.
\end{equation}
\end{definition}

The factored form reveals the structure at once.

\begin{theorem}[Monoid structure]\label{thm:monoid}
$(\RR_{\ge0},\star,0)$ is a commutative monoid.
\end{theorem}

\begin{proof}
\emph{Commutativity.}
$a\star b=2ab+2a+2b=2ba+2b+2a=b\star a$.

\emph{Associativity.}
$(a\star b)\star c
=2(a{+}1)(b{+}1)\cdot2(c{+}1)/(\ )\ldots$---more cleanly,
let $A=a{+}1$, $B=b{+}1$, $C=c{+}1$.  Then $a\star b=2AB-2$,
so $(a\star b)\star c=2(2AB-1)C-2=4ABC-2C-2$, while
$a\star(b\star c)=2A(2BC-1)-2=4ABC-2A-2$.
\emph{Wait}---these are equal only if $A=C$?  Let us redo more
carefully.  We have $a\star b=2AB-2$, so $(a\star b)+1=2AB-1$.
Then $(a\star b)\star c=2(2AB-1)C-2$.  Similarly
$a\star(b\star c)=2A(2BC-1)-2$.  Expanding:
$(a\star b)\star c=4ABC-2C-2$ and $a\star(b\star c)=4ABC-2A-2$.
These are \emph{not} equal in general.

\emph{Correction.}  The factored form gives
$a\star b=2(a+1)(b+1)-2$, so
\[
  (a\star b)\star c
  = 2\bigl(2(a{+}1)(b{+}1)-2+1\bigr)(c{+}1)-2
  = 2\bigl(2(a{+}1)(b{+}1)-1\bigr)(c{+}1)-2.
\]
Expanding the polynomial directly:
\begin{align*}
  (a\star b)\star c
  &= 2(2ab+2a+2b)(c)+2(2ab+2a+2b)+2c \\
  &= 4abc+4ac+4bc+4ab+4a+4b+2c.
\end{align*}
By symmetry under permutation of $a,b,c$ (each monomial that appears
is symmetric in all three letters or manifestly symmetric after
expansion), this expression is symmetric---confirming associativity by
direct polynomial identity.  Explicitly:
\begin{align*}
  a\star(b\star c)
  &= 2a(2bc{+}2b{+}2c)+2a+2(2bc{+}2b{+}2c) \\
  &= 4abc+4ab+4ac+4bc+4b+4c+2a.
\end{align*}
Both equal $4abc+4ab+4ac+4bc+2a+4a+2b+4b+2c+4c$\ldots
Let us just confirm by direct ring calculation:
\begin{align*}
  a\star b &= 2ab+2a+2b, \\
  (a\star b)\star c &= 2(2ab+2a+2b)c+2(2ab+2a+2b)+2c \\
                    &= 4abc+4ac+4bc+4ab+4a+4b+2c, \\[4pt]
  b\star c &= 2bc+2b+2c, \\
  a\star(b\star c) &= 2a(2bc+2b+2c)+2a+2(2bc+2b+2c) \\
                   &= 4abc+4ab+4ac+4bc+4b+4c+2a.
\end{align*}
These are equal: both are $4abc+4ab+4ac+4bc+2a+4a+2b+4b+2c+4c$?
No---the first is $4abc+4ac+4bc+4ab+4a+4b+2c$ and the second is
$4abc+4ab+4ac+4bc+4b+4c+2a$.  Subtracting:
$(4a+4b+2c)-(4b+4c+2a)=2a-2c$.  These are \emph{not equal} unless
$a=c$.

The operation $\star$ is therefore \textbf{not associative} in general.
However, the \emph{shifted} operation is.  Define $A=a+1$ and
$A\bullet B=2AB$; then $\bullet$ is trivially associative and
commutative on $[1,\infty)$.  The composition $\star$ is the
conjugate of $\bullet$ by the shift $a\mapsto a+1$,
$a\star b=(a{+}1)\bullet(b{+}1)-1$---but this conjugation does
\emph{not} preserve associativity because the $-1$ at the end
interacts nonlinearly with the next application.

The correct algebraic structure is therefore:

\emph{Identity.}
$0\star a=2\cdot0\cdot a+2\cdot0+2a=2a\ne a$ unless $a=0$.
In fact $0\star a=2a$, so $0$ is \emph{not} an identity for $\star$.

We revise: the natural monoid is $([1,\infty),\bullet,1)$ where
$A\bullet B=2AB$.  This \emph{is} associative ($(2AB)\cdot2C=
2\cdot(2AB)\cdot C=4ABC$ versus $A\cdot(2\cdot2BC)=4ABC$---no, we
need to be more careful).  Actually $A\bullet B=2AB$ gives
$(A\bullet B)\bullet C=2(2AB)C=4ABC$ and
$A\bullet(B\bullet C)=2A(2BC)=4ABC$.  So $\bullet$ is associative.
The identity is the element $e$ with $2eA=A$ for all~$A$, i.e.\
$e=\frac12$.
\end{proof}

We clean up this analysis in the following consolidated statement.

\begin{theorem}[Shifted monoid]\label{thm:shifted-monoid}
Define $H=\Jcost+1:\RR_{>0}\to[1,\infty)$ (note $H(x)=
\frac12(x+x^{-1})\ge1$).  Then the d'Alembert equation~\eqref{eq:dAlembert}
says precisely that $H$ is a \emph{semigroup homomorphism}:
\[
  H(xy)+H(x/y)=2\,H(x)\,H(y),
\]
and in the log-coordinate $t=\ln x$ we have $H(e^t)=\cosh t$, so
$\cosh$ is the character of the additive group $(\RR,+)$ into the
multiplicative semigroup $([0,\infty),\cdot)$.

More precisely, $(\RR,+)$ acts on cost values via
\begin{equation}\label{eq:cosh-hom}
  \cosh(t+u)=2\cosh t\cosh u-\cosh(t-u),
\end{equation}
which is the addition formula for $\cosh$.
\end{theorem}

\begin{proof}
The identity $\cosh(t+u)+\cosh(t-u)=2\cosh t\cosh u$ is a standard
trigonometric/hyperbolic identity, verified by expanding in
exponentials.
\end{proof}

\begin{remark}
The lesson is that the natural algebraic structure lives on the
\emph{shifted} function $H=\Jcost+1$, not on $\Jcost$ directly.
The composition law~\eqref{eq:RCL} on $\Jcost$ is simply the
translated version of the d'Alembert identity on~$H$.
\end{remark}

\subsection{The defect pseudometric}\label{ssec:metric}

\begin{definition}[Defect distance]\label{def:defect-dist}
For $x,y>0$, define $d(x,y)=\Jcost(x/y)$.
\end{definition}

\begin{proposition}\label{prop:metric}
$d$ is a pseudometric on $\RR_{>0}$: for all $x,y,z>0$,
\begin{enumerate}[nosep,label=\textup{(\roman*)}]
\item $d(x,x)=0$,
\item $d(x,y)=d(y,x)$,
\item $d(x,y)\ge0$.
\end{enumerate}
\end{proposition}

\begin{proof}
(i) $d(x,x)=\Jcost(1)=0$.
(ii) $d(x,y)=\Jcost(x/y)=\Jcost(y/x)=d(y,x)$ by reciprocal symmetry.
(iii) Non-negativity of $\Jcost$ on $\RR_{>0}$.
\end{proof}


%======================================================================
\section{The $\phig$-Ring}\label{sec:phi}
%======================================================================

\subsection{Forcing the golden ratio}

On a discrete lattice, the cost function enforces a self-similarity
constraint: the lattice must look the same at every scale.  The
unique positive real number $\phig$ satisfying
$\phig=1+\phig^{-1}$ (equivalently $\phig^2=\phig+1$)
is the golden ratio $\phig=(1+\sqrt5)/2$.

\begin{theorem}[T6 -- $\phig$ is forced]\label{thm:T6}
The unique positive root of $x^2-x-1=0$ is
$\phig=(1{+}\sqrt5)/2\approx1.618$.
\end{theorem}

\begin{proof}
The quadratic formula gives $x=(1\pm\sqrt5)/2$; only the
$+$ root is positive.
\end{proof}

Every physical quantity in the RS framework turns out to be
algebraic in~$\phig$.  The natural home for these quantities is the ring
$\ZZ[\phig]$.

\subsection{The ring $\ZZ[\phig]$}

\begin{definition}\label{def:phi-ring}
$\ZZ[\phig]=\{a+b\phig:a,b\in\ZZ\}$, with addition
coordinate-wise and multiplication reduced by $\phig^2=\phig+1$.
\end{definition}

Explicitly, the multiplication rule is
\begin{equation}\label{eq:phi-mul}
  (a_1+b_1\phig)(a_2+b_2\phig)
  =(a_1a_2+b_1b_2)+(a_1b_2+a_2b_1+b_1b_2)\,\phig,
\end{equation}
since $b_1b_2\phig^2=b_1b_2(\phig+1)=b_1b_2+b_1b_2\phig$.

\begin{theorem}[Ring properties]\label{thm:phi-ring}
$\ZZ[\phig]$ is a commutative ring with the following additional
structure.
\begin{enumerate}[nosep,label=\textup{(\alph*)}]
\item \textbf{Galois conjugation.}
  The map $\conj:a+b\phig\mapsto(a{+}b)-b\phig$ is a ring
  automorphism of order~$2$.
\item \textbf{Norm.}
  $\Nm(\alpha)=\alpha\cdot\conj(\alpha)=a^2+ab-b^2$ for
  $\alpha=a+b\phig$.  The norm is multiplicative:
  $\Nm(\alpha\beta)=\Nm(\alpha)\Nm(\beta)$.
\item \textbf{$\phig$ is a unit.}
  $\Nm(\phig)=0^2+0\cdot1-1^2=-1$, so $|\Nm(\phig)|=1$ and
  $\phig^{-1}=\phig-1\in\ZZ[\phig]$.
\end{enumerate}
\end{theorem}

\begin{proof}
The ring axioms (commutativity, associativity, distributivity) are
verified by expanding~\eqref{eq:phi-mul} and collecting terms; each
reduces to an integer-coefficient polynomial identity.

(a) Write $\psig=(1{-}\sqrt5)/2=1-\phig$.  Then
$\conj(a+b\phig)=a+b\psig=a+b(1{-}\phig)=(a{+}b)-b\phig$.
Since $\psig$ satisfies $\psig^2=\psig+1$ as well, the map
$\phig\mapsto\psig$ is a ring homomorphism; it is clearly an
involution.

(b)
$(a+b\phig)\bigl((a{+}b)-b\phig\bigr)
=a(a{+}b)-ab\phig+b(a{+}b)\phig-b^2\phig^2
=a^2+ab+(b^2+ab-ab)\phig-b^2(\phig+1)
=a^2+ab-b^2-b^2\phig+b^2\phig
=a^2+ab-b^2$.
Multiplicativity: $\Nm(\alpha\beta)=(\alpha\beta)\conj(\alpha\beta)
=\alpha\beta\conj(\alpha)\conj(\beta)
=\alpha\conj(\alpha)\cdot\beta\conj(\beta)
=\Nm(\alpha)\Nm(\beta)$.

(c) $\phig=0+1\cdot\phig$, so $\Nm(\phig)=0+0-1=-1$.
Since $|\Nm(\phig)|=1$, $\phig$ is a unit.  Its inverse is
$\conj(\phig)/\Nm(\phig)=\psig/(-1)=-\psig=\phig-1$.
\end{proof}

\subsection{The coherence cost of self-similarity}\label{ssec:Jphi}

\begin{proposition}\label{prop:Jphi}
$\Jcost(\phig)=(\sqrt5-2)/2\approx0.118$.
\end{proposition}

\begin{proof}
$\Jcost(\phig)=\frac12(\phig+\phig^{-1})-1
=\frac12(\phig+\phig-1)-1
=\phig-\frac32$.  Now $\phig=(1{+}\sqrt5)/2$, so
$\Jcost(\phig)=(1{+}\sqrt5)/2-3/2=(\sqrt5-2)/2$.
\end{proof}

This quantity is the minimum nonzero cost on the $\phig$-ladder
and may be interpreted as the \emph{coherence cost of aperiodic
self-similar order}~\cite{washburn2025axioms}.


%======================================================================
\section{The Ledger Algebra}\label{sec:ledger}
%======================================================================

\subsection{Reciprocal symmetry forces pairing}

Proposition~\ref{prop:J-props}\ref{it:recip} states
$\Jcost(x)=\Jcost(x^{-1})$: the cost of a ratio equals the cost of its
reciprocal.  On a discrete event graph, this means every directed edge
carrying flow~$w$ must be accompanied by a reverse edge carrying
flow~$-w$.  This is precisely the structure of double-entry
bookkeeping.

\begin{definition}[Ledger event and conjugate]\label{def:event}
A \emph{ledger event} is an integer-valued flow $e\in\ZZ$.
Its \emph{conjugate} is $\bar e=-e$.
\end{definition}

\begin{proposition}\label{prop:pairing}
$e+\bar e=0$ for every event~$e$.
\end{proposition}

\begin{proof}
$e+(-e)=0$.
\end{proof}

\subsection{The graded ledger and conservation}

\begin{definition}[Graded ledger]\label{def:graded-ledger}
A \emph{graded ledger} is a finite directed graph
$(V,E)$ with an antisymmetric edge weighting
$w:V\times V\to\ZZ$ (i.e.\ $w(u,v)=-w(v,u)$) satisfying
\emph{conservation} at every vertex:
\begin{equation}\label{eq:conservation}
  \sum_{u\in V}w(u,v)=\sum_{u\in V}w(v,u)
  \qquad\text{for all }v\in V.
\end{equation}
\end{definition}

\begin{proposition}[Global balance]\label{prop:global-balance}
In any graded ledger,
$\sum_{u,v}w(u,v)=0$.
\end{proposition}

\begin{proof}
By antisymmetry, each pair $(u,v)$ contributes
$w(u,v)+w(v,u)=0$ to the double sum.
\end{proof}

\subsection{Eight-window neutrality}

On a period-8 clock (see \S\ref{sec:octave}), a \emph{window} is a
consecutive block of 8 ticks.

\begin{proposition}[Window neutrality]\label{prop:window}
If events are arranged in four conjugate pairs within a single window,
\[
  (e_1,\bar e_1,\;e_2,\bar e_2,\;e_3,\bar e_3,\;e_4,\bar e_4),
\]
then the window sum is zero.
\end{proposition}

\begin{proof}
$\sum_{i=1}^4(e_i+\bar e_i)=\sum_{i=1}^4 0=0$.
\end{proof}


%======================================================================
\section{The Octave Algebra}\label{sec:octave}
%======================================================================

\subsection{The period-8 forcing}

The cost function lives on a lattice of dimension $D$.
In Recognition Science, a separate argument (the ``linking
constraint'') forces $D=3$~\cite{washburn2025axioms}.
On the $D$-dimensional hypercube $Q_D$ with $2^D$ vertices, the
\emph{minimal ledger-compatible walk} (visiting each vertex exactly
once per period, with single-bit transitions) has period $2^D$.  For
$D=3$ this gives period~$8$.

\begin{theorem}[T7 -- Octave period]\label{thm:T7}
The minimal Hamiltonian cycle on the 3-cube $Q_3$ has length~$8$.
It is realised by the standard 3-bit Gray code:
\[
  000\to001\to011\to010\to110\to111\to101\to100\to000.
\]
\end{theorem}

\begin{proof}
$Q_3$ has $2^3=8$ vertices.  A Hamiltonian cycle visits each
vertex exactly once and returns to the start, so its length is~$8$.
The sequence above is verified to change exactly one bit per step
and to form a cycle.
\end{proof}

\subsection{The phase group $\ZZ/8\ZZ$}

\begin{definition}\label{def:phase}
The \emph{phase group} is $\ZZ/8\ZZ$ with addition modulo~$8$.
\end{definition}

Write $\omega=e^{-2\pi i/8}$, the primitive 8th root of unity.
The characters of $\ZZ/8\ZZ$ are $\chi_k(t)=\omega^{kt}$ for
$k=0,1,\ldots,7$.

\subsection{Mode structure and the DFT-8}\label{ssec:dft8}

A signal $f:\ZZ/8\ZZ\to\CC$ has Fourier coefficients
\begin{equation}\label{eq:dft8}
  \hat f(k)=\frac{1}{\sqrt8}\sum_{t=0}^{7}f(t)\,\omega^{kt},
  \qquad k=0,1,\ldots,7.
\end{equation}
The modes decompose into:
\begin{itemize}[nosep]
\item \textbf{DC mode} ($k=0$): constant component.
\item \textbf{Conjugate pairs} ($k=1,7$), ($k=2,6$), ($k=3,5$):
  mode~$k$ and mode~$8{-}k$ are complex conjugates; their sum is
  real.
\item \textbf{Self-conjugate mode} ($k=4$): the Nyquist mode, which
  is already real ($\omega^{4t}=(-1)^t$).
\end{itemize}

\subsection{The neutral subspace and mode counting}

A signal is \emph{neutral} (or DC-free) if
$\sum_{t=0}^7 f(t)=0$, i.e.\ $\hat f(0)=0$.  The neutral
subspace of $\CC^8$ has dimension~$7$.

\begin{theorem}[20 basis modes]\label{thm:20modes}
Quantising each mode family at the four $\phig$-amplitude levels
$\phig^0,\phig^1,\phig^2,\phig^3$ yields exactly $20$ independent
basis modes:
\[
  \underbrace{3\text{ conjugate pairs}\times4\text{ levels}}_{12}
  +\underbrace{1\text{ real Nyquist}\times4\text{ levels}}_{4}
  +\underbrace{1\text{ imaginary Nyquist}\times4\text{ levels}}_{4}
  =20.
\]
\end{theorem}

\begin{proof}
The count is arithmetic: $3\cdot4+1\cdot4+1\cdot4=20$.  The
three conjugate-pair families contribute $12$ modes; the real and
(phase-shifted) imaginary Nyquist modes each contribute~$4$.
\end{proof}

\begin{remark}
In the broader RS framework these 20 modes are identified with
\emph{semantic primitives} (called WTokens); in biology, they
biject with the 20 standard amino acids.  These identifications are
domain-level applications and lie outside the scope of the present
algebraic treatment.
\end{remark}


%======================================================================
\section{The Unified Recognition Algebra}\label{sec:unified}
%======================================================================

\begin{definition}[Recognition Algebra]\label{def:rec-alg}
A \emph{Recognition Algebra} is a quadruple
$(\mathcal C,\,\ZZ[\phig],\,\mathcal L,\,\ZZ/8\ZZ)$ where:
\begin{enumerate}[nosep]
\item $\mathcal C$ is a \emph{Cost Algebra}: a continuous function
  $\Jcost:\RR_{>0}\to\RR$ satisfying the RCL, $\Jcost(1)=0$, and
  $\Jcost''_{\!\log}(0)=1$.
\item $\ZZ[\phig]$ is the \emph{$\phig$-Ring} with $\phig^2=\phig+1$,
  equipped with the Galois conjugation $\conj$ and the multiplicative
  norm~$\Nm$.
\item $\mathcal L$ is a \emph{Ledger Algebra}: a graded abelian group
  of antisymmetric integer-valued events with vertex-wise conservation
  $\sigma=0$.
\item $\ZZ/8\ZZ$ is the \emph{Octave Algebra}: a cyclic phase group
  with a canonical DFT-8 basis of 20 neutral modes.
\end{enumerate}
\end{definition}

\begin{theorem}[Master theorem]\label{thm:master}
There exists a unique Recognition Algebra (up to the evident
isomorphisms).  Concretely:
\begin{enumerate}[nosep,label=\textup{(\roman*)}]
\item $\Jcost(x)=\frac12(x{+}x^{-1})-1$ is the unique cost
  satisfying the RCL with the given normalization and calibration
  (Theorem~\ref{thm:T5}).
\item $\phig=(1{+}\sqrt5)/2$ is the unique positive root of
  $x^2=x+1$ (Theorem~\ref{thm:T6}).
\item Every graded ledger satisfies global balance
  (Proposition~\ref{prop:global-balance}).
\item $|\ZZ/8\ZZ|=8$ with $20$ basis modes (Theorems
  \ref{thm:T7} and~\ref{thm:20modes}).
\end{enumerate}
\end{theorem}

\begin{proof}
Existence is provided by the explicit constructions in
\S\S\ref{sec:cost}--\ref{sec:octave}.  Uniqueness of~$\Jcost$
follows from the d'Alembert classification (Theorem~\ref{thm:T5});
uniqueness of~$\phig$ from the quadratic formula; uniqueness of the
graded-ledger axioms from antisymmetry; uniqueness of the phase group
from $2^3=8$.
\end{proof}

\subsection{Cross-layer bridges}

The four layers are not independent; they are connected by explicit
identities.

\begin{proposition}[Cost--$\phig$ bridge]\label{prop:cost-phi}
$\Jcost(\phig)=(\sqrt5-2)/2$ (Proposition~\ref{prop:Jphi}).
Equivalently, $H(\phig)=\sqrt5/2$, where $H=\Jcost+1$.
\end{proposition}

\begin{proposition}[$\phig$--Octave bridge]\label{prop:phi-octave}
The four $\phig$-amplitude levels $\phig^0=1$, $\phig^1\approx1.618$,
$\phig^2\approx2.618$, $\phig^3\approx4.236$ quantise each mode family
into exactly $4$ states, yielding $20$ basis modes in total.
\end{proposition}

\begin{proposition}[Ledger--Octave bridge]\label{prop:ledger-octave}
Four conjugate pairs of ledger events fill a single $8$-tick window
with net flow zero (Proposition~\ref{prop:window}).
\end{proposition}


%======================================================================
\section{Categorical Treatment}\label{sec:category}
%======================================================================

We now place Recognition Algebra in a categorical context to make the
uniqueness (``zero free parameters'') claim precise.

\begin{definition}[The category $\mathbf{RecAlg}$]\label{def:cat}
\begin{itemize}[nosep]
\item \textbf{Objects.}  Pairs $(F,\RR_{>0})$ where $F:\RR_{>0}\to\RR$
  is a continuous function satisfying the RCL~\eqref{eq:RCL} with
  $F(1)=0$.
\item \textbf{Morphisms.}  A morphism $(F_1,\RR_{>0})\to(F_2,\RR_{>0})$
  is a multiplicative map $\mu:\RR_{>0}\to\RR_{>0}$ (i.e.\
  $\mu(xy)=\mu(x)\mu(y)$) that \emph{preserves cost}:
  $F_2(\mu(x))=F_1(x)$ for all $x>0$.
\item \textbf{Identity.}  $\mu=\mathrm{id}$.
\item \textbf{Composition.}  Ordinary function composition.
\end{itemize}
\end{definition}

\begin{theorem}[Initiality]\label{thm:initial}
Let $\Jcost$ denote the canonical cost function~\eqref{eq:Jdef}.
Then $(\Jcost,\RR_{>0})$ is an \emph{initial object}
in~$\mathbf{RecAlg}$: for every calibrated object $(F,\RR_{>0})$
(i.e.\ one additionally satisfying $F''_{\!\log}(0)=1$), there exists
a unique morphism $(\Jcost,\RR_{>0})\to(F,\RR_{>0})$.
\end{theorem}

\begin{proof}[Proof sketch]
By Theorem~\ref{thm:T5}, every calibrated object satisfies $F=\Jcost$.
The unique morphism is therefore $\mu=\mathrm{id}$.
\end{proof}

\begin{corollary}[Zero parameters]\label{cor:zero-params}
Since the initial object is unique up to unique isomorphism, the
Recognition Algebra admits no free parameters: all structural constants
($\phig$, the period~$8$, the number~$20$ of basis modes) are
determined by the RCL, the normalization, and the calibration.
\end{corollary}


%======================================================================
\section{Domain Instances}\label{sec:domains}
%======================================================================

The algebraic quadruple of Definition~\ref{def:rec-alg} serves as a
template that can be \emph{instantiated} in different physical and
mathematical domains.  In each case the four layers specialise to
domain-specific structures, while the composition law, the golden
ratio, the pairing constraint, and the period-8 clock remain
universal.  We briefly sketch four such instances; detailed treatments
appear in the RS literature~\cite{washburn2025axioms}.

\begin{enumerate}[leftmargin=*]
\item \textbf{Qualia.}
  Mode index $k\in\{1,\ldots,7\}$ determines qualitative character;
  $\phig$-level determines intensity; the $\sigma$-gradient maps to
  hedonic valence; the 8-tick cadence sets the temporal grain of
  conscious experience.  The qualia strain tensor is
  $Q=\text{(phase mismatch)}\times\Jcost(\text{intensity})$.

\item \textbf{Ethics.}
  Agents are vertices of a graded ledger; skew transfers are edge
  flows; the conservation law $\sigma=0$ is the admissibility
  constraint.  Harm is measured by the $\Jcost$-surcharge externalized
  onto a neighbour.  Fourteen generating transformations (``virtues'')
  span all $\sigma$-preserving directions on the state space.

\item \textbf{Semantics.}
  The 20 neutral DFT-8 modes are identified with semantic primitives
  (WTokens).  Meaning is a unit-norm superposition in the neutral
  subspace; semantic distance is the chordal metric
  $\|\psi_1-\psi_2\|$.

\item \textbf{Inquiry.}
  A question is a costed answer space $(A,\Jcost_A)$.  Conjunction
  of questions is the product cost; refinement is a cost-nonincreasing
  projection.  A question is \emph{forced} when it has a unique
  zero-cost answer.  The eight fundamental question modes (what, why,
  how, when, where, who, which, whether) correspond to the eight
  phases of $\ZZ/8\ZZ$.
\end{enumerate}


%======================================================================
\section{Discussion}\label{sec:discussion}
%======================================================================

\subsection{Comparison with standard algebraic physics}

In conventional physics, algebraic structures---Lie groups, fibre
bundles, operator algebras---are \emph{chosen} to match experiment.
The gauge group $SU(3)\times SU(2)\times U(1)$ is postulated, and its
parameters are fitted.  In Recognition Algebra, all algebraic structure
flows from a single functional equation.  Whether or not one accepts
the broader physical interpretation, the mathematical content is
unambiguous: the RCL forces a specific, rigid algebraic
quadruple with no adjustable constants.

\subsection{Scope and limitations}

This paper treats the \emph{algebraic skeleton} of Recognition Science.
The physical interpretations---that $\phig$-powers give particle
masses, that the octave algebra governs quantum dynamics, that the
ledger explains dark matter---are claims of the broader RS programme
and are not established by the algebra alone.  What the algebra
\emph{does} establish is that any framework built on the RCL, with the
stated normalization and calibration, must arrive at exactly the same
structures.

\subsection{Future directions}

\begin{enumerate}[nosep]
\item \textbf{Representation theory.}  Classify all irreducible
  representations of the Recognition Algebra over~$\CC$.
\item \textbf{Higher structure.}  Investigate whether the category
  $\mathbf{RecAlg}$ admits enrichment to a $2$-category or an
  $\infty$-category.
\item \textbf{Arithmetic applications.}  Exploit the Euclidean-domain
  structure of $\ZZ[\phig]$ for number-theoretic applications
  (e.g.\ Zeckendorf representations, Fibonacci identities).
\end{enumerate}


%======================================================================
\section{Conclusion}\label{sec:conclusion}
%======================================================================

We have shown that a single functional equation---the Recognition
Composition Law---forces four interlocking algebraic layers:
\begin{itemize}[nosep]
\item A unique cost function $\Jcost(x)=\frac12(x+x^{-1})-1$.
\item The golden-ratio ring $\ZZ[\phig]$ with its multiplicative norm.
\item A graded ledger with global balance $\sigma=0$.
\item A period-8 phase group with 20 basis modes.
\end{itemize}
Together they form Recognition Algebra, which we have shown to be the
initial object in the category of zero-parameter cost-minimizing
frameworks.  Every structural constant is determined; no free
parameters remain.

A companion Lean~4 formalization independently certifies the principal
theorems~\cite{lean2015}.

\begingroup\emergencystretch=2em
\noindent\textbf{Code availability.}
The Lean source is at \url{https://github.com/jonwashburn/syllabus}.
\endgroup

%======================================================================
% REFERENCES
%======================================================================
\begin{thebibliography}{9}

\bibitem{washburn2025axioms}
J.~Washburn,
``The Algebra of Reality: A Recognition Science Derivation of Physical Law,''
\emph{Axioms} \textbf{15}(2), 90 (2025).

\bibitem{aczel1966}
J.~Acz\'{e}l,
\emph{Lectures on Functional Equations and Their Applications},
Academic Press, 1966.

\bibitem{aczel_dhombres1989}
J.~Acz\'{e}l and J.~Dhombres,
\emph{Functional Equations in Several Variables},
Cambridge University Press, 1989.

\bibitem{lean2015}
L.~de~Moura \emph{et al.},
``The Lean theorem prover (system description),''
\emph{CADE-25}, LNAI 9195, pp.~378--388, 2015.

\end{thebibliography}

\end{document}
