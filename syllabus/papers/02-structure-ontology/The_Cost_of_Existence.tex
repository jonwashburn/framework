\documentclass[11pt,a4paper]{article}
\usepackage[utf8]{inputenc}
\usepackage{amsmath}
\usepackage{amsfonts}
\usepackage{amssymb}
\usepackage{amsthm}
\usepackage{geometry}
\usepackage{hyperref}
\usepackage{graphicx}

\geometry{margin=1in}

\title{\textbf{The Cost of Existence: A First-Principles Derivation of Physical Law from the Recognition Composition Law}}
\author{Jonathan Washburn \\ \textit{Recognition Science Research Institute}}
\date{\today}

\begin{document}

\maketitle

\begin{abstract}
Standard physical theories typically postulate the existence of a manifold, a set of logical axioms, and initial conditions as irreducible priors. This paper proposes a ``Cost-First'' foundation where these elements are derived rather than assumed. Starting from a single primitive constraint—the \textit{Recognition Composition Law} (RCL)—we derive a unique cost functional $J(x)$. We demonstrate that the laws of logic emerge as the ground state of this functional ($J(1)=0$), while physical existence is forced by the singularity of the vacuum state ($J(0) \to \infty$). This framework, Recognition Science (RS) \cite{AlgebraOfReality}, subsequently forces discreteness, conservation (ledger structure), and the dimensionless constants of nature without free parameters. We establish that the Meta-Principle ``Nothing cannot recognize itself'' is not an axiom, but a derived theorem of the cost structure.
\end{abstract}

\section{Introduction}

The search for a fundamental theory of physics has historically been a search for the correct equations of motion governing matter and energy on a pre-existing manifold. However, this approach suffers from the ``Problem of Priors'': it assumes the existence of the stage (spacetime), the actors (fields or particles), and the script (logic and set theory) before the play begins. A truly fundamental theory must derive these priors. It must explain not only \textit{how} the universe behaves, but \textit{why} it exists in a logical, discrete, and conserved form.

This paper presents a paradigm shift from \textit{Energy Minimization} to \textit{Cost Minimization}. In standard physics, systems evolve to minimize action or energy (via the Hamiltonian $\hat{H}$). We propose that this is a limiting case of a deeper principle: the minimization of \textbf{Recognition Cost} ($J$). By analyzing the structure of recognition—defined strictly as the interaction or comparison of states—we uncover a ``Cost-First'' foundation.

\paragraph{Operational semantics (recognizers and ratios).}
We adopt the measurement-first viewpoint of Recognition Geometry \cite{RecognitionGeometry}. A \emph{recognizer} is a map $R:\mathcal{C}\to\mathcal{E}$ from a configuration space to an event space; observational indistinguishability $c\sim_R c'$ is defined by $R(c)=R(c')$, and observable states are equivalence classes in the recognition quotient $\mathcal{C}_R:=\mathcal{C}/\!\sim_R$. To compare two observable states (or a state and a reference outcome) we attach positive scale maps $\iota$ and form a ratio $x\in\mathbb{R}_{>0}$; in ratio-induced models one may write $x=\iota_S(s)/\iota_O(o)$ \cite{EntropyInterface}. The cost functional $J:(0,\infty)\to[0,\infty)$ assigns a penalty to mismatch in this ratio, normalized so that perfect self-match $x=1$ has zero cost.

The foundation of this theory rests on a single primitive functional equation, the \textbf{Recognition Composition Law (RCL)}. We demonstrate that any system satisfying this law, subject to normalization and calibration, is forced into a unique cost structure:
\begin{equation}
    J(x) = \frac{1}{2}\left(x + \frac{1}{x}\right) - 1
\end{equation}
This function $J(x)$ serves as the ``ontological potential'' of reality.

Crucially, this framework allows us to derive the two most fundamental aspects of reality which are usually treated as axioms: Logic and Existence.
\begin{enumerate}
    \item \textbf{Logic from Cost (T0):} We define logical consistency not as a rule, but as a low-cost state. Since $J(1)=0$, the identity is the unique zero-cost configuration. Logic emerges because contradiction ($x \neq 1$) is energetically expensive.
    \item \textbf{Existence from Cost (T1):} We do not assume that something must exist. Instead, we analyze the cost of ``Nothing'' ($x \to 0$). We find that $\lim_{x \to 0} J(x) = \infty$. The system is violently repelled from non-existence by an infinite cost barrier. Thus, existence is not an accident; it is a geometric necessity forced by the cost functional.
\end{enumerate}

In the sections that follow, we trace the consequences of this cost minimization. We show how finite local resolution in recognition quotients \cite{RecognitionGeometry} yields discrete observable structure (T2), how reciprocity together with a balance-preserving update rule yields a double-entry ledger form (T3), and how these constraints determine dimensionless constants (T5--T8) \cite{AlgebraOfReality}.

\section{The Primitive: The Recognition Composition Law (RCL)}

The foundation of Recognition Science is not a particle or a field, but a constraint on how recognition events combine. We posit a single primitive functional equation, the \textbf{Recognition Composition Law (RCL)}, which governs the cost of interaction between states.

\subsection{Defining the Primitive}
Let $J(x)$ be a cost functional defined on positive real numbers $x \in \mathbb{R}_+$. The primitive law is given by:
\begin{equation}
    J(xy) + J(x/y) = 2J(x)J(y) + 2J(x) + 2J(y)
    \label{eq:RCL}
\end{equation}
This equation describes how the cost of a composite state ($xy$) and a relative state ($x/y$) relates to the costs of the individual components. It is a calibrated multiplicative form of the d'Alembert functional equation.

\subsection{Boundary Conditions}
To select a physical solution from this functional constraint, we impose two natural boundary conditions corresponding to the existence of a neutral identity and a standard scale:

\begin{enumerate}
    \item \textbf{A1: Normalization.} The identity element must have zero cost. Consistency is ``free.''
    \begin{equation}
        J(1) = 0
    \end{equation}
    \item \textbf{A3: Calibration.} We fix the scale of the cost function by defining the curvature at the minimum in logarithmic coordinates. Let $\widetilde{J}(u):=J(e^{u})$ for $u\in\mathbb{R}$. We set:
    \begin{equation}
        J''_{\log}(0) := \widetilde{J}''(0) = 1
    \end{equation}
    This condition rules out the trivial solution $J\equiv 0$ and fixes the units of the cost landscape.
\end{enumerate}

\subsection{Theorem T5: Cost Uniqueness}
\textbf{Theorem (T5: Uniqueness of the calibrated cost).}
Let $J:(0,\infty)\to[0,\infty)$ be continuous and nontrivial, satisfy the Recognition Composition Law (\ref{eq:RCL}), and obey $J(1)=0$. Assume moreover that $\widetilde{J}(u):=J(e^{u})$ is twice differentiable at $u=0$ with $J''_{\log}(0)=1$. Then the cost function is uniquely determined on $\mathbb{R}_{>0}$ as:
\begin{equation}
    J(x) = \frac{1}{2}\left(x + \frac{1}{x}\right) - 1
\end{equation}

\begin{proof}
Define $K:(0,\infty)\to[1,\infty)$ by $K(x):=1+J(x)$. Expanding the right-hand side of (\ref{eq:RCL}) shows that (\ref{eq:RCL}) is equivalent to the multiplicative d'Alembert identity for $K$:
\begin{equation}
    K(xy)+K(x/y)=2K(x)K(y)\qquad(x,y>0).
    \label{eq:K-dalembert-mult}
\end{equation}
Now define $f:\mathbb{R}\to[1,\infty)$ by $f(u):=K(e^{u})$. Continuity of $J$ implies continuity of $f$, and substituting $x=e^{u}$ and $y=e^{v}$ into (\ref{eq:K-dalembert-mult}) yields the additive d'Alembert equation
\begin{equation}
    f(u+v)+f(u-v)=2f(u)f(v)\qquad(u,v\in\mathbb{R}),
    \label{eq:f-dalembert-add}
\end{equation}
with $f(0)=K(1)=1$ and $f(u)\ge 1$.
By the classical classification of continuous solutions to (\ref{eq:f-dalembert-add}), either $f\equiv 1$, or $f(u)=\cos(au)$, or $f(u)=\cosh(au)$ for some $a>0$. The constraint $f(u)\ge 1$ rules out the cosine branch, and nontriviality of $J$ rules out $f\equiv 1$. Hence $f(u)=\cosh(au)$ for some $a>0$, so
\[
K(x)=\cosh(a\log x)\quad\text{and}\quad
J(x)=\cosh(a\log x)-1=\tfrac12(x^{a}+x^{-a})-1.
\]
Finally, $\widetilde{J}(u)=J(e^{u})=\cosh(au)-1$ has $\widetilde{J}''(0)=a^{2}$, so the calibration $J''_{\log}(0)=1$ forces $a=1$.
Thus $J(x)=\cosh(\log x)-1=\frac12(x+x^{-1})-1$, as claimed.
A self-contained version of this characterization (with references) is given in \cite{EntropyInterface}.
\end{proof}

\noindent \textbf{Note:} Theorem T5 establishes the mathematical ``terrain'' of reality. All subsequent physical laws—logic, existence, discreteness, and conservation—are consequences of minimizing this specific potential $J(x)$.

\subsection{The Inevitability of the RCL}
A common objection to foundational theories is the arbitrariness of the starting axiom. Why must the cost function satisfy the specific functional equation (\ref{eq:RCL})?

Rather than postulating (\ref{eq:RCL}), one can ask what structural requirements force a two-variable consistency law for ratio comparison. An ``equation inevitability'' theorem \cite{DAlembertInevitability} shows that if $F:(0,\infty)\to\mathbb{R}$ is continuous and nontrivial, $F(1)=0$, and there exists a \emph{symmetric quadratic polynomial} $P$ such that
\[
F(xy)+F(x/y)=P(F(x),F(y))\qquad(x,y>0),\qquad P(u,v)=P(v,u),
\]
then the law itself is forced into the unique bilinear family
\[
P(u,v)=2u+2v+cuv\qquad(c\in\mathbb{R}).
\]
In particular, reciprocity $F(z)=F(1/z)$ is derived from symmetry of the law (symmetry is imposed on $P$, not on $F$). A natural log-curvature calibration at the identity then fixes $c=2$, recovering (\ref{eq:RCL}) exactly. In this sense, the RCL is not an arbitrary axiom but the canonical symmetric quadratic consistency law for ratio comparison.

\section{Deriving Logic (T0)}

In standard formulations of logic, the Law of Identity ($A=A$) is posited as an axiom. In a recognition-first setting, identity at the observable level is supplied operationally: in Recognition Geometry \cite{RecognitionGeometry}, a recognizer induces an equivalence relation of observational indistinguishability and an observable quotient, where identity is equality of equivalence classes. The cost functional $J$ then acts as a selection principle that singles out self-match ($x=1$) as the unique zero-cost fixed point.

\subsection{Consistency as the Ground State}
We have derived the unique cost function $J(x) = \frac{1}{2}(x + x^{-1}) - 1$. We observe that the global minimum of this function occurs uniquely at $x=1$:
\begin{equation}
    J(1) = \frac{1}{2}(1 + 1) - 1 = 0
\end{equation}
In this framework, the state $x=1$ represents perfect ratio match (self-match) and therefore the operational notion of identity or consistency. The fact that $J(1)=0$ means that self-match is ``free.'' It is the ground state of the ontology.

\subsection{The Cost of Contradiction}
Consider a state of contradiction or inconsistency, represented by any deviation $x \neq 1$. Due to the strict convexity of $J(x)$ on $\mathbb{R}_+$, we have:
\begin{equation}
    \forall x \neq 1, \quad J(x) > 0
\end{equation}
Any deviation from identity incurs a positive cost penalty. For small deviations $x = 1+\epsilon$, the cost rises quadratically ($J \approx \epsilon^2/2$). For large deviations (gross contradictions), the cost grows linearly or exponentially depending on the regime.

Therefore, within this model, identity-consistent recognition corresponds to the unique zero-cost equilibrium of the cost landscape. If physical evolution is cost-minimizing, descriptions that maintain self-match ($x=1$) are selected as stable equilibria, while inconsistent descriptions ($x\neq 1$) carry strictly positive cost and are selected against.

\subsection{The Gödel Dissolution}
Foundational theories are often challenged by Gödel's Incompleteness Theorems, which state that any sufficiently powerful formal system contains undecidable propositions. Does this mean a complete theory of reality is impossible?

We argue that Gödel's theorems do not obstruct the closure of Recognition Science because they apply to different domains:
\begin{itemize}
    \item \textbf{Gödel's Domain:} The \textit{provability} of arithmetic sentences within a formal axiomatic system.
    \item \textbf{RS Domain:} The \textit{selection} of physical configurations via cost minimization.
\end{itemize}

The universe does not compute the digits of $\pi$ to infinity or attempt to prove all true theorems of arithmetic. Instead, it settles into configurations that minimize the cost functional $J$. The process of reality is \textbf{Selection}, not \textbf{Proof}. By reframing reality as a physical selection process rather than a formal axiomatic system, RS is inoculated against logical paradoxes. Self-referential queries that lead to undecidability in logic simply correspond to high-cost or unstable configurations in physics, which are naturally filtered out by the minimization dynamic.

\section{Deriving Existence (T1)}

The most profound question in metaphysics is ``Why is there something rather than nothing?'' In the Cost-First framework, this is not a philosophical mystery but a calculation. We do not assume existence; we derive it from the cost of non-existence.

\subsection{The Definition of ``Nothing''}
Let the state variable $x$ represent the magnitude, scale, or presence of a configuration.
\begin{itemize}
    \item $x=1$ represents ``Something'' (the Identity, the existent state).
    \item $x \to 0$ represents ``Nothing'' (the absence of magnitude, total collapse, or the void).
\end{itemize}

\subsection{The Singularity}
We analyze the behavior of the cost functional $J(x)$ as the system approaches the state of non-existence. Calculating the limit as $x \to 0^+$:
\begin{equation}
    \lim_{x \to 0^+} J(x) = \lim_{x \to 0^+} \left[ \frac{1}{2}\left(x + \frac{1}{x}\right) - 1 \right]
\end{equation}
The term $x$ vanishes, but the reciprocal term $\frac{1}{x}$ diverges:
\begin{equation}
    \lim_{x \to 0^+} \frac{1}{x} = \infty \implies \lim_{x \to 0^+} J(x) = \infty
\end{equation}
This result is fundamental: \textbf{``Nothing'' has infinite cost.}

\subsection{The Forcing Mechanism}
The fundamental dynamic of the theory is the minimization of $J$.
\begin{enumerate}
    \item The system evolves to minimize cost.
    \item The state of ``Nothing'' ($x=0$) sits at the top of an infinitely steep potential barrier.
    \item Therefore, the system is violently repelled from non-existence.
\end{enumerate}
Unlike standard vacuum potentials which often have a minimum at $\phi=0$, the Recognition Cost potential has a singularity at $0$. The system cannot reside in the void because the cost of doing so is infinite.

\subsection{Conclusion}
Consequently, ``Something'' (specifically the finite state $x=1$) is forced to exist. The famous Meta-Principle of Recognition Science—\textit{``Nothing cannot recognize itself''}—is derived here not as a linguistic axiom, but as a physical consequence of the cost singularity. The universe exists because it is infinitely expensive for it not to.

\section{The Geometry of Cost: The Choice Manifold}

To bridge the abstract notion of ``cost'' to the concrete geometry of physics, we must show that the cost functional defines a metric space. We term this space the \textbf{Choice Manifold} (see also \cite{RecognitionGeometry}).

\subsection{Defining the Metric}
In geometric mechanics, a potential function $J(x)$ naturally induces a metric $g(x)$ via its Hessian (second derivative). For our unique cost function $J(x) = \frac{1}{2}(x + x^{-1}) - 1$, the second derivative is:
\begin{equation}
    J'(x) = \frac{1}{2}(1 - x^{-2}) \implies J''(x) = x^{-3}
\end{equation}
Thus, the metric of the Choice Manifold is given by:
\begin{equation}
    g(x) = \frac{1}{x^3}
\end{equation}
This metric diverges as $x \to 0$, consistent with the infinite cost of the void, and equals unity at the ground state $x=1$ ($g(1)=1$), consistent with our calibration condition.

\subsection{Dynamics as Geodesics}
The cost functional is not merely a scalar field; it defines the curvature of the manifold on which reality evolves. In this framework, physical evolution and decision-making are not arbitrary steps but \textbf{geodesics} (shortest paths) across the Choice Manifold.

The equation of motion for a system is the geodesic equation derived from this metric:
\begin{equation}
    \ddot{x} + \Gamma(x) \dot{x}^2 = 0
\end{equation}
where $\Gamma(x)$ is the Christoffel symbol derived from $g(x)$. This connects the abstract logic of Recognition Science directly to the geometric mechanics used in General Relativity. The universe does not just ``minimize cost'' in a vacuum; it follows the curved geometry imposed by the cost of existence.

\section{Deriving Structure (T2 \& T3)}

Having established that something must exist ($x=1$) and that it inhabits a curved manifold, we now derive the fundamental structural properties of physical reality: Discreteness and Conservation.

\subsection{Discreteness (T2)}
In this framework, discreteness is an operational consequence of recognition, not a claim about an underlying continuum. Recognition Geometry introduces locality via neighborhoods and postulates \emph{finite local resolution}: any recognizer can distinguish only finitely many outcomes within a local region \cite{RecognitionGeometry}. Observable space is therefore partitioned into finitely many \emph{resolution cells} at any fixed local scale, yielding a locally discrete quotient description.

The cost landscape $J$ then governs stability across these cells. In particular, the calibrated curvature
\begin{equation}
    J''(1) = 1
\end{equation}
sets the local stiffness of the mismatch penalty near perfect match, so that transitions between distinct resolution cells incur positive cost while variations within a cell are observationally invisible. Dynamics therefore proceeds as a sequence of cell-to-cell updates (discrete observable evolution) even if the underlying configuration space admits continuous representatives.

\subsection{The Ledger (T3)}
The reciprocal symmetry
\begin{equation}
    J(x) = J(1/x)
\end{equation}
implies that a deviation by a ratio $x$ carries the same mismatch penalty as its inverse. To connect this reciprocity to conservation, we make explicit the balance invariant of a closed recognition network.

Let a ledger state be a vector of ratios $X=(x_1,\dots,x_N)\in(\mathbb{R}_{>0})^N$ and define the \emph{balance functional}
\begin{equation}
    B(X):=\prod_{i=1}^N x_i.
\end{equation}
A closed system is balanced when $B(X)=1$ (equivalently $\sum_i \log x_i=0$). A local interaction of strength $r>0$ between two entries $i\neq j$ updates
\[
x_i\mapsto x_i r,\qquad x_j\mapsto x_j/r,
\]
leaving all other coordinates fixed. This update preserves balance: $B(X)$ is invariant.
In this sense every ``debit'' by $r$ is accompanied by a ``credit'' by $1/r$, yielding a double-entry ledger form; standard conservation laws correspond to invariants of such balance-preserving updates. The RCL encodes how the associated recognition costs compose under the paired multiplicative operations $(x,y)\mapsto(xy,x/y)$.
For a proof-bearing discrete-dynamics development of atomic ticks, balance-preserving ledger updates on graphs, and the $2^{d}$-tick hypercube period (including the $8$-tick $d=3$ case), see \cite{CoherentComparisonLedger}.

\section{The Emergence of Constants (T5--T8)}

We have derived the qualitative structure of reality (Logic, Existence, Discreteness, Conservation). We now demonstrate that the quantitative constants of nature are also forced by this structure.

\subsection{Self-Similarity ($\phi$)}
In a discrete ledger system, the simplest non-trivial operation that preserves structure is self-reference or self-similarity. This corresponds to the relation:
\begin{equation}
    x = 1 + \frac{1}{x}
\end{equation}
Multiplying by $x$, we obtain the quadratic equation:
\begin{equation}
    x^2 - x - 1 = 0
\end{equation}
The unique positive solution is the Golden Ratio, $\phi = \frac{1+\sqrt{5}}{2} \approx 1.618$. Thus, $\phi$ is not an arbitrary constant; it is the unique fixed point of the cost function's self-reference.

\subsection{Spacetime Geometry (T6)}
To update the ledger without violating conservation, the system requires a cycle. We consider the minimal ledger-compatible walk on a spatial manifold. Stability analysis (T8) forces the dimension to be $D=3$. The minimal Hamiltonian cycle on a 3-cube ($2^D$ vertices) has a period of:
\begin{equation}
    T = 2^3 = 8
\end{equation}
This \textbf{8-tick cycle} defines the fundamental atomic unit of time ($\tau_0$). Spacetime is not a pre-existing container but the unfolding of this 8-step recognition cycle.

\subsection{Derivation of $\alpha$}
The fine-structure constant $\alpha$ characterizes the strength of electromagnetic interaction. In Recognition Science it is derived from the same primitive structure---$\phi$ and the 8-tick cube cycle---once the relevant projection weights and curvature corrections are defined. We defer the full computation (and the explicit definitions of these geometric quantities) to \cite{AlgebraOfReality}, which derives $\alpha^{-1}\approx 137.036$ without fitting parameters.

\section{Discussion}

\subsection{The Recognition Operator ($\hat{R}$)}
Standard quantum mechanics postulates that time evolution is driven by the Hamiltonian operator $\hat{H}$. In Recognition Science, the fundamental dynamical object is a \textbf{Recognition Operator} ($\hat{R}$).

Let $R:\mathcal{C}\to\mathcal{E}$ be a recognizer and $\mathcal{C}_R$ the induced recognition quotient of observable states \cite{RecognitionGeometry}. We write $s(t)\in\mathcal{C}_R$ for the observable state at time $t$.
\begin{equation}
    s(t+8\tau_0) = \hat{R}(s(t))
\end{equation}
In the simplest deterministic setting $\hat{R}:\mathcal{C}_R\to\mathcal{C}_R$ is a map; more generally, $\hat{R}$ may be taken as a Markov operator acting on probability measures on $\mathcal{C}_R$. In either case, the defining constraints are that $\hat{R}$ preserves the balance invariant of the ledger and selects lower-cost configurations according to the functional $J$.

\subsection{Emergence of the Hamiltonian}
The Hamiltonian description can be recovered as an effective generator of the discrete recognition update. Suppose the 8-tick update is reversible on the observable state space (so that $\hat{R}$ is bijective) and admits a unitary representation $U$ on a Hilbert space $\mathcal{H}$. Then one may define an effective Hamiltonian $\hat{H}$ by
\begin{equation}
    U=\exp\!\left(-\frac{i}{\hbar}\hat{H}\,\Delta t\right),\qquad \Delta t:=8\tau_0,
\end{equation}
so that $\hat{H}$ is the infinitesimal generator (logarithm) of the recognition update in the reversible limit. Near the ground state $x=1$, the quadratic expansion $J(x)\approx \tfrac12(x-1)^2$ provides the local harmonic approximation underlying standard linearized dynamics. In this sense, Hamiltonian evolution appears as a near-equilibrium, representation-dependent limit of cost-based recognition dynamics.

\subsection{Resolving the ``Something from Nothing'' Paradox}
This framework offers a rigorous resolution to the ancient paradox of creation. The question ``Why is there something rather than nothing?'' assumes that ``Nothing'' is a stable, low-energy state from which ``Something'' must be miraculously created.

We have shown that ``Nothing'' ($x \to 0$) is actually a state of \textbf{infinite cost} ($J \to \infty$). It is the most unstable configuration possible. The universe does not require a miracle to exist; it requires a miracle to \textit{stop} existing. Existence is the inevitable ground state of the cost functional.

\section{Conclusion}

We have presented a derivation of physical law that requires zero arbitrary parameters and zero assumed axioms of existence or logic. The derivation chain proceeds as follows:

\begin{enumerate}
    \item \textbf{The Primitive:} The Recognition Composition Law (RCL) is the unique functional equation governing consistent interaction.
    \item \textbf{The Terrain:} Imposing normalization and calibration on the RCL uniquely forces the cost potential $J(x) = \frac{1}{2}(x + x^{-1}) - 1$.
    \item \textbf{Logic (T0):} The ground state $J(1)=0$ defines logical consistency as the thermodynamic equilibrium of reality.
    \item \textbf{Existence (T1):} The singularity $J(0) \to \infty$ creates an infinite cost barrier to non-existence, forcing the universe to exist.
    \item \textbf{Structure (T2--T3):} Finite local resolution in recognition quotients yields discreteness at the observable level; reciprocity together with a balance-preserving update rule yields a double-entry ledger form (conservation laws). The curvature of $J$ sets the stability scale near perfect match.
    \item \textbf{Constants (T5--T8):} The self-reference of the ledger forces $\phi$, and the minimal cycle on the resulting 3D manifold forces the 8-tick atomic clock; further quantitative constants such as $\alpha$ follow from the same structure (see \cite{AlgebraOfReality}).
\end{enumerate}

By replacing the assumption of a pre-existing universe with the derivation of a cost-minimizing one, Recognition Science offers a path to a truly fundamental theory of reality—one where the laws of physics are not just observed, but are shown to be inevitable.

\begin{thebibliography}{99}

\bibitem{AlgebraOfReality}
Washburn, J.; Zlatanović, M.; Allahyarov, E.
The Algebra of Reality: A Recognition Science Derivation of Physical Law.
\textit{Axioms} \textbf{2026}, \textit{15}, 90.
\url{https://doi.org/10.3390/axioms15020090}

\bibitem{DAlembertInevitability}
Washburn, J.; Zlatanović, M.; Allahyarov, E.
D'Alembert Inevitability: Polynomial Consistency Forces the Canonical Composition Law on $\mathbb{R}_{>0}$.
\textit{Manuscript in preparation}, \textbf{2026}.

\bibitem{CoherentComparisonLedger}
Pardo-Guerra, S.; Simons, M.; Thapa, A.; Washburn, J.
Coherent Comparison as Information Cost: A Cost-First Ledger Framework for Discrete Dynamics.
\textit{arXiv preprint} arXiv:2601.12194, \textbf{2026}.
\url{https://doi.org/10.48550/arXiv.2601.12194}

\bibitem{RecognitionGeometry}
Washburn, J.; Zlatanović, M.; Allahyarov, E.
Recognition Geometry.
\textit{Submitted}, \textbf{2026}.

\bibitem{EntropyInterface}
Washburn, J.; Rahnamai Barghi, A.
Reciprocal Convex Costs for Ratio Matching: Functional-Equation Characterization and Decision Geometry.
\textit{Submitted to Entropy}, \textbf{2026}. (Manuscript ID 4136332.)

\end{thebibliography}

\end{document}
