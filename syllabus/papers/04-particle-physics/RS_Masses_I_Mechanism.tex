\documentclass[11pt,a4paper]{article}
\usepackage[margin=1in]{geometry}
\usepackage[T1]{fontenc}
\usepackage{lmodern}
\usepackage{microtype}
\usepackage{amsmath,amssymb,amsthm}
\usepackage{mathtools}
\usepackage{booktabs}
\usepackage{enumitem}
\usepackage{xcolor}
\usepackage[hidelinks]{hyperref}
\usepackage{tikz}
\usetikzlibrary{arrows.meta,positioning,calc}

\newtheorem{theorem}{Theorem}[section]
\newtheorem{proposition}[theorem]{Proposition}
\newtheorem{lemma}[theorem]{Lemma}
\newtheorem{corollary}[theorem]{Corollary}
\newtheorem{definition}[theorem]{Definition}
\newtheorem{hypothesis}[theorem]{Structural Hypothesis}
\newtheorem{remark}[theorem]{Remark}

% Claim hygiene
\newcommand{\PROVED}{\textcolor{blue!70!black}{\textsf{[PROVED]}}}
\newcommand{\HYP}{\textcolor{orange!80!black}{\textsf{[HYP]}}}
\newcommand{\VAL}{\textcolor{purple!70!black}{\textsf{[VAL]}}}

\newcommand{\phig}{\varphi}
\newcommand{\Jcost}{J}
\newcommand{\Rhat}{\hat{R}}
\newcommand{\Ecoh}{E_{\mathrm{coh}}}
\newcommand{\muStar}{\mu_{\star}}
\newcommand{\mRS}{m^{\mathrm{RS}}}
\newcommand{\Epass}{E_{\mathrm{passive}}}
\newcommand{\RS}{Recognition Science}
\newcommand{\SM}{Standard Model}
\newcommand{\RCL}{Recognition Composition Law}

\title{\textbf{The Origin of Mass in Recognition Science:\\
Cost Geometry, Recognition Boundaries, and the $\phig$-Ladder}\\[0.5em]
\large Paper I of VI: Mechanism}
\author{Jonathan Washburn\\
\small Recognition Science Research Institute, Austin, Texas\\
\small \texttt{washburn.jonathan@gmail.com}}
\date{\today}

\begin{document}
\maketitle

\begin{abstract}
In the \SM{}, fermion masses are free parameters encoded by Yukawa
couplings to the Higgs field.  This paper develops an alternative
ontology of mass within \RS{} (RS).  We separate two epistemic layers:

\textbf{Layer 1} \PROVED: The cost functional
$\Jcost(x) = \tfrac{1}{2}(x + x^{-1}) - 1$ is uniquely forced by the
\RCL{} (Theorem~T5, Lean-verified).  The golden ratio
$\phig = (1{+}\sqrt{5})/2$ is the unique self-similar fixed point
(Theorem~T6).  Dimension $D = 3$ is forced (Theorem~T8), giving a
3-cube with $V{=}8$, $E{=}12$, $F{=}6$.  The eight-tick cycle
$2^3 = 8$ is the minimal cover (Theorem~T7).

\textbf{Layer 2} \HYP: Mass emerges as a coordinate on a $\phig$-ladder
whose sector-level scales are fixed by cube combinatorics.  The
\emph{recognition boundary} --- a self-sustaining pattern on a discrete
ledger --- replaces the point-particle ontology.  The recognition
operator $\Rhat$ replaces the Hamiltonian; the Higgs mechanism is
reinterpreted as a low-energy effective description.  Sector yardsticks,
the charge-to-band map, and generation torsion are structural proposals
with explicit falsifiers.

Companion papers develop phenomenological predictions (II), the neutrino
sector (III), transport discipline (IV), the fine-structure constant (V),
and the generation problem (VI).
\end{abstract}

\tableofcontents
\newpage

\noindent\textbf{Claim-hygiene convention.}
Every substantive claim carries one of:
\PROVED\ (derived from RS axioms with complete chain, Lean-verified where noted);
\HYP\ (structural proposal, falsifiable, not yet derived from axioms);
\VAL\ (comparison with external data).
\vspace{0.8em}
\hrule
\vspace{1em}

%=============================================================================
\section{Introduction}
%=============================================================================

\subsection{The mass problem}

The \SM{} contains nine charged fermion masses spanning nearly five orders
of magnitude, from the electron ($0.511\,\mathrm{MeV}$) to the top quark
($173\,\mathrm{GeV}$).  These masses enter as free Yukawa couplings ---
the SM tells us \emph{how} particles acquire mass (electroweak symmetry
breaking) but not \emph{why} they have the particular values they do.

\subsection{The RS approach}

RS begins from a single primitive: the \RCL{}, \PROVED{}
\begin{equation}
  \Jcost(xy) + \Jcost(x/y) = 2\,\Jcost(x)\,\Jcost(y) + 2\,\Jcost(x) + 2\,\Jcost(y),
  \label{eq:RCL}
\end{equation}
together with normalization $\Jcost(1) = 0$ and calibration
$\Jcost''_{\log}(0) = 1$.  These three conditions uniquely determine
$\Jcost(x) = \tfrac{1}{2}(x + x^{-1}) - 1$, proved in Lean~4 via ODE
uniqueness for the d'Alembert functional equation~\cite{WashburnCost2026}.

From this cost functional, a chain of forced consequences (T0--T8)
derives discreteness, a double-entry ledger, recognition events, $\Jcost$
uniqueness, $\phig$, the eight-tick period, and three spatial dimensions.
Within this architecture, we \emph{propose} (Layer~2) that mass is a
coordinate on a discrete multiplicative ladder whose base $\phig$, period
8, and sector structure are all determined by the cube geometry.

\subsection{What this paper does and does not claim}

\begin{itemize}[nosep]
\item We \emph{do} present a structural model for particle mass that uses
  no free parameters and no per-particle fitting.
\item We \emph{do not} claim that every element of the model is derived
  from axioms.  The sector yardsticks, the charge-to-band map, and the
  generation torsion are \emph{structural hypotheses} --- motivated by
  the framework but not yet proved from it.
\item We \emph{do} identify which parts are proved (Layer~1) and which
  are hypothesized (Layer~2), so that a reader can evaluate each on its
  merits.
\end{itemize}

%=============================================================================
\section{Proved Foundation (Layer 1)}
\label{sec:foundation}
%=============================================================================

This section collects only results with complete derivation chains.

\subsection{The cost functional (T5)} \PROVED

\begin{theorem}[Cost uniqueness~\cite{WashburnCost2026}]
Let $F : \mathbb{R}_+ \to \mathbb{R}$ satisfy the \RCL{},
$F(1) = 0$, and $\lim_{t\to 0} 2F(e^t)/t^2 = 1$.  Then
$F(x) = \Jcost(x) := \frac{1}{2}(x + x^{-1}) - 1$ for all $x > 0$.

\emph{Lean:} \texttt{IndisputableMonolith.CostUniqueness.T5\_uniqueness\_complete}.
\end{theorem}

The proof converts the \RCL{} to a d'Alembert equation via
$H(t) := F(e^t) + 1$, yielding $H(t{+}u) + H(t{-}u) = 2H(t)H(u)$.
By Acz\'{e}l's theorem~\cite{Aczel1966}, continuous solutions are
$\cosh(\lambda t)$; calibration fixes $\lambda = 1$.

Key properties: reciprocal symmetry $\Jcost(x) = \Jcost(1/x)$;
non-negativity with equality iff $x = 1$; strict convexity on
$\mathbb{R}_+$; divergence $\Jcost(0^+) = +\infty$.  The divergence
at zero is the ``Meta-Principle'' --- the infinite cost of nothing ---
which is a \emph{derived theorem}, not an axiom.

\subsection{The golden ratio (T6)} \PROVED

\begin{theorem}
The minimal reciprocal self-correction rule $x_{n+1} = 1 + 1/x_n$ has
unique positive fixed point $\phig = (1{+}\sqrt{5})/2$, satisfying
$\phig^2 = \phig + 1$.  The orbit $\{\phig^n : n \in \mathbb{Z}\}$ is
the unique self-similar lattice on $\mathbb{R}_{>0}$ compatible with
$\Jcost$.

\emph{Lean:} \texttt{IndisputableMonolith.Foundation.PhiForcing.phi\_equation}.
\end{theorem}

\subsection{Dimension, cube counts, and the eight-tick cycle (T7--T8)} \PROVED

\begin{theorem}[$D = 3$ forced~\cite{WashburnD3}]
Three independent constraints --- Alexander duality (linking invariant),
Kepler stability (non-precessing orbits), and minimal dyadic
synchronisation ($\mathrm{lcm}(2^D, 45)$ minimised at $D = 3$) ---
each single out $D = 3$.
\end{theorem}

\begin{proposition}[Cube counts]
$V = 2^3 = 8$ vertices, $E = 3 \cdot 2^2 = 12$ edges, $F = 2 \cdot 3 = 6$ faces.
\end{proposition}

\begin{theorem}[Minimal cover (T7)]
The minimal cycle covering all $2^3 = 8$ vertex states has length~8.
\end{theorem}

These four results --- $\Jcost$ uniqueness, $\phig$, $D = 3$ with its
cube counts, and the 8-tick cycle --- constitute the \textbf{proved
foundation}.  Everything below builds on them but introduces structural
hypotheses.

%=============================================================================
\section{The Mass Model (Layer 2)}
\label{sec:model}
%=============================================================================

\noindent\emph{Every claim in this section carries the \HYP\ marker.}

\subsection{What is a particle?} \HYP

\begin{hypothesis}[Recognition boundary]
A \emph{recognition boundary} is a localized, self-sustaining pattern on
the cubic ledger $\mathbb{Z}^3$ with finite nonzero cost, invariant under
the recognition operator $\Rhat$ (up to phase/translation), and satisfying
eight-tick neutrality.
\end{hypothesis}

\begin{remark}[Motivation]
This ontology replaces the point-particle with a structured pattern.  The
hypothesis is motivated by the RS framework (where persistence requires
finite cost and eight-tick closure) but is not derived from the axioms
alone.  It is the \emph{definition} of the model, not a theorem.
\end{remark}

\subsection{Mass as a ladder coordinate} \HYP

\begin{hypothesis}[$\phig$-ladder mass law]\label{hyp:mass}
The mass of boundary $b$ at anchor $\muStar$ is:
\begin{equation}
  \mRS(b;\muStar) = A_{\mathrm{sector}(b)} \cdot
  \phig^{\,r_b - 8 + \mathrm{gap}(Z_b)},
  \label{eq:mass}
\end{equation}
where $A_{\mathrm{sector}}$ is the sector yardstick, $r_b \in \mathbb{Z}$
the rung, $-8$ the octave reference, and
$\mathrm{gap}(Z_b) = \log_\phig(1 + Z_b/\phig)$ the charge-derived band
function.
\end{hypothesis}

\begin{remark}[Status of each element]
\begin{itemize}[nosep]
\item \textbf{Ladder base $\phig$}: \PROVED\ (Theorem~T6, from self-similarity).
\item \textbf{Octave offset $-8$}: \PROVED\ (Theorem~T7, from the minimal cover).
\item \textbf{Integer rung $r_b$}: \HYP\ (the claim that masses sit on
  integer rungs is structural, not derived).
\item \textbf{Sector yardstick $A_S$}: \HYP\ (derived from cube integers
  by structural identification; see Section~\ref{sec:cube}).
\item \textbf{Gap function}: \HYP\ (the specific function
  $\log_\phig(1 + Z/\phig)$ is not yet derived from axioms).
\item \textbf{$Z$-map}: \HYP\ (the polynomial $\tilde{Q}^2 + \tilde{Q}^4$
  is a phenomenological ansatz; see Section~\ref{sec:Zmap}).
\end{itemize}
\end{remark}

\subsection{Cube geometry and the counting layer}
\label{sec:cube}

\begin{hypothesis}[Active/passive decomposition]\label{hyp:passive}
Of the 12 edges, one is ``active'' (traversed) per tick, leaving
$\Epass = E - 1 = 11$ passive edges.  This decomposition is physically
meaningful: the passive count enters the mass formulas.
\end{hypothesis}

\begin{remark}
The edge count $E = 12$ is proved.  The active/passive split is
motivated by the 8-tick update (one edge traversal per tick) but the
claim that $\Epass$ enters the mass formulas is structural.
\end{remark}

\begin{hypothesis}[Wallpaper groups]\label{hyp:W}
The number $W = 17$ of 2D crystallographic groups (Fedorov,
1891~\cite{Fedorov1891}) enters the mass model as a counting constant
for face symmetries of the cubic ledger.
\end{hypothesis}

\begin{remark}[Honest assessment]
$W = 17$ is a mathematical theorem.  That it is \emph{physically
relevant} is the strongest assumption in the paper series.  See
Paper~VI~\cite{PaperVI} for the ``dimensional coincidence theorem''
($\Epass(D) + F(D) = W$ iff $D = 3$), which provides structural
motivation but not a derivation from axioms.
\end{remark}

\begin{hypothesis}[Sector yardsticks]\label{hyp:sector}
Each sector has $A_S = 2^{B_{\mathrm{pow}}(S)} \cdot \Ecoh \cdot
\phig^{r_0(S)}$ where $\Ecoh = \phig^{-5}$:
\begin{center}
\renewcommand{\arraystretch}{1.15}
\begin{tabular}{@{}lrrlr@{}}
\toprule
Sector & $B_{\text{pow}}$ & $r_0$ & Formula & Status \\
\midrule
Lepton & $-22$ & $62$ & $-2\Epass$;\; $4W{-}6$ & \HYP \\
Up quark & $-1$ & $35$ & $-A$;\; $2W{+}A$ & \HYP \\
Down quark & $23$ & $-5$ & $2E{-}1$;\; $E{-}W$ & \HYP \\
Electroweak & $1$ & $55$ & $A$;\; $3W{+}4$ & \HYP \\
\bottomrule
\end{tabular}
\end{center}
\end{hypothesis}

\begin{remark}[What "derived in Lean" means here]
The Lean module \texttt{IndisputableMonolith.Masses.Anchor} verifies the
\emph{integer arithmetic}: e.g., $4 \times 17 - 6 = 62$.  It does
\emph{not} derive \emph{why} the lepton sector uses the formula
$4W - 6$.  The ``formulas'' in the table are \emph{structural
identifications} --- we observe that the working numerical values can be
expressed as simple combinations of the cube integers.  Deriving these
combinations from an admissibility principle remains an open problem.
\end{remark}

\subsection{Charge quantisation and the $Z$-map}
\label{sec:Zmap}

\begin{hypothesis}[$Z$-map] \HYP
Integerise charge as $\tilde{Q} := 6Q$ (note $6 = F$, the face count).
The charge-to-band index is:
\[
Z = \begin{cases}
\tilde{Q}^2 + \tilde{Q}^4 & \text{(leptons)} \\
4 + \tilde{Q}^2 + \tilde{Q}^4 & \text{(quarks)}
\end{cases}
\]
producing three families: $Z_\ell = 1332$, $Z_u = 276$, $Z_d = 24$.
\end{hypothesis}

\begin{remark}
The $Z$-map is a phenomenological ansatz.  The polynomial
$\tilde{Q}^2 + \tilde{Q}^4$ was chosen because it separates the three
sectors into distinct bands.  The factor $6 = F$ is suggestive but not
derived.  Showing that this specific polynomial (and not
$\tilde{Q}^2 + \tilde{Q}^6$, say) is forced by ledger geometry is an
open problem.
\end{remark}

\subsection{Generation torsion} \HYP

Generation torsion $\tau_g \in \{0, \Epass, W\} = \{0, 11, 17\}$ is
universal across sectors.  The derivation --- showing that the three
generations correspond to the three levels of cube combinatorial structure
(vertices, edges, faces) --- is the subject of Paper~VI~\cite{PaperVI}.

%=============================================================================
\section{The Recognition Operator and Dynamics}
\label{sec:dynamics}
%=============================================================================

The fundamental dynamical law is \HYP{}
\begin{equation}
  s(t + 8\tau_0) = \Rhat(s(t)),
\end{equation}
where $\Rhat$ minimises $\Jcost$ (not energy).  The derivation of $\Rhat$
from $\Jcost$ and the eight-tick structure is given in~\cite{WashburnOperator}.

\begin{proposition}[Hamiltonian emergence] \PROVED\ (given $\Rhat$)
In the quadratic regime $|x - 1| \ll 1$:
$\Jcost(x) \approx \frac{1}{2}(x - 1)^2$, so cost minimisation
reduces to stationary action, recovering standard Hamiltonian mechanics
as an approximation valid to $< 1\%$ for $|\varepsilon| \le 0.1$.
\end{proposition}

%=============================================================================
\section{The Yukawa Bridge and the Higgs Reinterpretation}
\label{sec:higgs}
%=============================================================================

\begin{hypothesis}[Yukawa bridge] \HYP
The SM Yukawa coupling at the anchor is the derived quantity:
\begin{equation}
  y_f(\muStar) = \frac{\sqrt{2}}{v} \cdot A_{\mathrm{sector}(f)} \cdot
  \phig^{\,r_f - 8 + \mathrm{gap}(Z_f)}.
\end{equation}
Yukawa couplings are effective parameters encoding $\phig$-ladder
positions, not fundamental.
\end{hypothesis}

\begin{hypothesis}[Higgs reinterpretation] \HYP
The Higgs field is the continuum effective description of discrete
$\phig$-ladder structure.  The VEV $v \approx 246\,\mathrm{GeV}$
corresponds to the electroweak yardstick
$A_{\mathrm{EW}} = 2 \cdot \Ecoh \cdot \phig^{55}$.  The Goldstone
mechanism remains intact as an effective description.
\end{hypothesis}

\begin{remark}
These are among the boldest claims in the paper.  If the mass model
produces the correct Yukawa couplings at $\muStar$, the bridge formula
is validated; if it fails, the structural hypothesis is refuted.  The
Higgs reinterpretation does not modify any SM prediction --- it
reinterprets the origin of the parameters.
\end{remark}

%=============================================================================
\section{Falsifiers}
\label{sec:falsifiers}
%=============================================================================

\begin{enumerate}[nosep]
\item Equal-$Z$ clustering failure at $\muStar$ refutes the $Z$-map
  (Hypothesis~\ref{hyp:W}).
\item Generation ratios inconsistent with $\phig^{11}$, $\phig^6$ refute
  the torsion (Paper~VI).
\item Octave reference $-8$ replaceable by another integer refutes the
  eight-tick connection.
\item An alternative ladder base outperforming $\phig$ refutes the
  self-similarity argument.
\item Sector yardstick formulas achievable from counting-layer inputs
  \emph{other than} $(E, \Epass, F, W, A)$ would weaken the uniqueness
  of the cube identification.
\item Discovery of a fourth fermion generation with SM-like charges would
  falsify the three-level combinatorial argument (Paper~VI).
\end{enumerate}

%=============================================================================
\section{Open Problems}
\label{sec:open}
%=============================================================================

\begin{enumerate}[label=(O\arabic*),nosep]
\item \textbf{Derive W=17.}  Show that the wallpaper-group count enters
  the mass formulas as a consequence of voxel face-symmetry, not as an
  external input.
\item \textbf{Derive the sector formulas.}  Show that
  $r_0 = 4W - 6$ for leptons (etc.) is forced by an admissibility
  constraint on recognition cycles.
\item \textbf{Derive $\Ecoh = \phig^{-5}$.}  Connect the exponent $-5$
  to a structural property of the $\phig$-ladder.
\item \textbf{Derive the $Z$-map polynomial.}  Show that
  $\tilde{Q}^2 + \tilde{Q}^4$ is the unique polynomial compatible with
  ledger charge conservation.
\item \textbf{Derive the gap function.}  Show that
  $\log_\phig(1 + Z/\phig)$ is forced by the geometry.
\end{enumerate}

Each solved open problem would upgrade the corresponding hypothesis to a
proved result, strengthening the framework from a structural model to a
derivation.

%=============================================================================
\section{Conclusions}
%=============================================================================

Mass in RS is proposed as a geometric coordinate on a $\phig$-ladder
determined by the cube's combinatorial structure.  The model uses zero
free parameters and zero per-particle fitting.

The proved foundation (Layer~1) provides $\Jcost$, $\phig$, $D = 3$,
the cube counts, and the 8-tick cycle.  The structural hypothesis
(Layer~2) adds recognition boundaries, sector yardsticks, the charge-band
map, and generation torsion.  Five open problems (O1--O5) identify the
gaps between the two layers.

The companion papers develop phenomenological predictions (II), the
neutrino sector (III), the anchor scale and transport discipline (IV),
the fine-structure constant (V), and the generation-number argument (VI).

\begin{thebibliography}{99}
\bibitem{WashburnCost2026} J.~Washburn and M.~Zlatanovi\'{c},
  ``Uniqueness of the Canonical Reciprocal Cost,''
  arXiv:2602.05753v1, 2026.
\bibitem{WashburnD3} J.~Washburn, M.~Zlatanovi\'{c}, and E.~Allahyarov,
  ``Dimensional Rigidity: D=3,'' RS preprint, 2026.
\bibitem{WashburnOperator} J.~Washburn,
  ``Beyond the Hamiltonian: The Recognition Operator,'' RS preprint, 2026.
\bibitem{PDG2024} R.~L.~Workman \textit{et al.} [PDG],
  PTEP \textbf{2022}, 083C01.
\bibitem{Aczel1966} J.~Acz\'{e}l,
  \textit{Lectures on Functional Equations}, Academic Press (1966).
\bibitem{Fedorov1891} E.~S.~Fedorov,
  ``Simmetriya pravil'nykh sistem figur,''
  Zapiski Imp.\ S.-Peterburgskogo Mineral.\ Obshch.\ \textbf{28},
  1--146 (1891).
\bibitem{Washburn2025} J.~Washburn,
  \textit{Axioms} \textbf{15}(2), 90 (2025).
\bibitem{PaperII} J.~Washburn, Paper~II of this series.
\bibitem{PaperVI} J.~Washburn, Paper~VI of this series.
\end{thebibliography}

\end{document}
