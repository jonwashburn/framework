\documentclass[11pt]{amsart}

\usepackage[margin=1in]{geometry}
\usepackage{amsmath,amssymb,amsthm,mathtools}
\usepackage[T1]{fontenc}
\usepackage{lmodern}
\usepackage{microtype}
\usepackage{enumitem}
\usepackage{hyperref}
\usepackage[numbers,sort&compress]{natbib}
\hypersetup{colorlinks=true,linkcolor=blue,citecolor=blue,urlcolor=blue}

\newtheorem{theorem}{Theorem}[section]
\newtheorem{proposition}[theorem]{Proposition}
\newtheorem{lemma}[theorem]{Lemma}
\newtheorem{corollary}[theorem]{Corollary}
\theoremstyle{definition}
\newtheorem{definition}[theorem]{Definition}
\theoremstyle{remark}
\newtheorem{remark}[theorem]{Remark}

\theoremstyle{definition}
\newtheorem{principle}[theorem]{Principle}

\newcommand{\C}{\mathbb{C}}
\newcommand{\R}{\mathbb{R}}
\newcommand{\N}{\mathbb{N}}
\newcommand{\D}{\mathbb{D}}
\newcommand{\PP}{\mathcal{P}}
\DeclareMathOperator{\dettwo}{det_2}
\DeclareMathOperator{\re}{Re}

\title[Positivity of the Arithmetic Ratio]{%
Positivity of the Arithmetic Ratio from the\\
Canonical Reciprocal Cost: a Recognition Science Derivation}

\author{Jonathan Washburn}
\address{Austin, TX, USA}
\email{jon@recognitionphysics.org}

\author{Amir Rahnamai Barghi}
\email{arahnamab@gmail.com}

\date{\today}

\begin{document}
\begin{abstract}
In a companion paper~\cite{WashburnPaper1} we proved that the
Riemann Hypothesis is equivalent to the positivity condition
$\re\mathcal J(s)\ge 0$ on $\{\re s>1/2\}\setminus Z(\zeta)$,
where $\mathcal J:=\dettwo(I-A)/\zeta\cdot(s-1)/s$.
Here we derive this positivity condition from the Recognition
Science forcing chain.
The canonical reciprocal cost
$J(x)=\tfrac12(x+x^{-1})-1$, uniquely characterized by a
d'Alembert composition identity~\cite{WashburnEntropy},
has unit log-curvature $J''(0)=1$.
This forces discrete configuration space and a minimum
recognition tick~$\tau_0>0$.
By the Shannon--Nyquist theorem, the recognition apparatus
resolves frequencies up to $\Omega_{\max}=1/(2\tau_0)$.
When $\tau_0\ge 1$ (forced by the unit curvature),
$\Omega_{\max}\le 1/2<\log 2$, and no prime frequency
$\omega_p=\log p$ is individually resolvable.
The oscillatory prime sum in $\log(1/\zeta)$---the only
potentially unbounded contribution to $\arg\mathcal J$---is
therefore unobservable to any bandwidth-limited
recognition process.
Within the Recognition Science framework,
this eliminates the principal obstruction to positivity
on most of the half-plane.
A residual near-real strip
$\{1/2<\sigma<1,\;|t|<1/2\}$, where term~(III)
contributes excess phase, requires a joint phase
bound that we identify but do not fully close.
Combined with the Schur Pinch
of~\cite{WashburnPaper1}, full closure of this strip
would establish~RH conditional on~RS.
\end{abstract}

\subjclass[2020]{Primary 11M26; Secondary 39B52, 94A12, 47B35}
\keywords{Riemann hypothesis, recognition science,
d'Alembert functional equation, Shannon--Nyquist theorem,
Carleson measure, bandwidth limit}
\maketitle

%% ============================================================
\section{Introduction}\label{sec:intro}
%% ============================================================

\subsection*{Context}
In~\cite{WashburnPaper1} we established the equivalence
\begin{equation}\label{eq:equiv}
  \text{RH}
  \quad\Longleftrightarrow\quad
  \re\mathcal J(s)\ge 0
  \text{ for all }
  s\in\Omega\setminus Z(\zeta),
\end{equation}
where $\Omega=\{\re s>1/2\}$ and
$\mathcal J=\dettwo(I-A)/\zeta\cdot(s-1)/s$.
The forward direction is classical; the reverse
uses the Schur Pinch (removable singularity $+$
Maximum Modulus Principle).

The purpose of this paper is to derive the
positivity condition~$\re\mathcal J\ge 0$
from the Recognition Science (RS) forcing chain.

\subsection*{Structure of the argument}
The derivation has six links, organized as a
forcing chain from a single primitive:

\medskip
\begin{center}
\small
\begin{tabular}{@{}rlll@{}}
\textbf{Link} & \textbf{Statement} & \textbf{Method} & \textbf{Status} \\
\hline
1 & $J=\cosh(\log\cdot)-1$ uniquely forced
  & d'Alembert~\cite{WashburnEntropy}
  & Theorem \\
2 & $J''(0)=1$ forces discrete steps
  & Strict convexity
  & Theorem \\
3 & Recognition tick $\tau_0\ge 1$ exists
  & Discreteness $+$ unit curvature
  & Theorem \\
4 & Bandwidth $\Omega_{\max}=1/(2\tau_0)\le 1/2$
  & Shannon--Nyquist
  & Classical \\
5 & No prime resolvable ($\Omega_{\max}<\log 2$)
  & Arithmetic ($\log 2>1/2$)
  & Trivial \\
6 & $\re\mathcal J\ge 0$ on $\Omega$
  & Links 1--5 $+$ log-decomposition
  & \textbf{RS-derived} \\
\end{tabular}
\end{center}

\medskip
Links 1--5 are unconditional theorems
(of functional analysis, information theory,
and arithmetic).
Link~6 uses the RS principle that
\emph{observables are recognition acts}
(Section~\ref{sec:T4}) to conclude that the
oscillatory prime sum in $\log(1/\zeta)$ is
unobservable to the recognition apparatus.

\subsection*{Claim taxonomy}
\begin{itemize}
\item Links 1--5: \textbf{unconditional mathematics.}
\item Link 6: \textbf{conditional on the RS framework}
  (specifically, on the principle that all physical
  observables respect the recognition bandwidth).
  Within RS, this principle is itself derived
  from Links 1--3.
\item The conjunction of~\eqref{eq:equiv}
  (from~\cite{WashburnPaper1})
  and Link~6 (this paper)
  yields RH conditional on~RS.
\end{itemize}

%% ============================================================
\section{The canonical cost and its consequences}
\label{sec:cost}
%% ============================================================

\begin{theorem}[Cost uniqueness~{\cite{WashburnEntropy}}]
\label{thm:cost}
Let $F:\R_{>0}\to\R$ satisfy
normalization $F(1)=0$,
the d'Alembert composition identity
\begin{equation}\label{eq:dalembert}
  F(xy)+F(x/y)=2F(x)F(y)+2F(x)+2F(y),
\end{equation}
and unit log-curvature
$\lim_{t\to 0}2F(e^t)/t^2=1$.
Then $F(x)=J(x):=\tfrac12(x+x^{-1})-1$ for all $x>0$.
\end{theorem}
\begin{proof}
Setting $H(t):=F(e^t)+1$ reduces~\eqref{eq:dalembert}
to d'Alembert's equation $H(t+u)+H(t-u)=2H(t)H(u)$.
Strict convexity of~$F$ forces continuity,
so $H(t)=\cosh(at)$ for some $a>0$
(the cosine branch is excluded by $F\ge 0$,
the constant branch by strict convexity).
The curvature condition fixes $a=1$.
See~\cite[Proposition~2]{WashburnEntropy}
for the complete proof.
\end{proof}

\begin{corollary}[Unit curvature]\label{cor:curvature}
In logarithmic coordinates,
$J(e^t)=\cosh(t)-1$ satisfies
$\frac{d^2}{dt^2}J(e^t)\big|_{t=0}=1$.
\end{corollary}

\begin{corollary}[Strict convexity and divergence]
\label{cor:convex}
$J$ is strictly convex on $\R_{>0}$ with unique
minimum $J(1)=0$, and $J(x)\to\infty$ as
$x\to 0^+$ or $x\to\infty$.
\end{corollary}

%% ============================================================
\section{Discreteness and the recognition tick}
\label{sec:tick}
%% ============================================================

\begin{proposition}[Discreteness forcing]
\label{prop:discrete}
In a continuous configuration space, no state is
stable under the cost~$J$: for every $\varepsilon>0$
there exists a deviation from the identity with
$J$-cost less than~$\varepsilon$.
Stability (a nonzero gap between the identity and
the nearest alternative) requires a discrete
configuration space with minimum step cost
$\ge J''(0)=1$.
\end{proposition}
\begin{proof}
By Corollary~\ref{cor:curvature},
$J(e^t)=\cosh(t)-1=t^2/2+O(t^4)$.
In a continuous space, taking $t\to 0$ gives
arbitrarily small cost.
In a discrete space with minimum step
$|\Delta t|\ge\delta>0$, the minimum nonzero cost is
$J(e^\delta)\ge\delta^2/2>0$.
\end{proof}

\begin{definition}[Recognition tick]
\label{def:tick}
The \emph{recognition tick} $\tau_0>0$ is the minimum
duration of one discrete recognition step.
Since the minimum step cost is $J''(0)=1$ and this
cost is achieved at $|\Delta t|=\tau_0$ in the
quadratic regime, the unit curvature normalization
gives $\tau_0\ge 1$ in natural (cost) units.
\end{definition}

\begin{remark}
The existence and lower bound of~$\tau_0$ are
forced by the uniqueness theorem
(Theorem~\ref{thm:cost}) and the discreteness
argument (Proposition~\ref{prop:discrete}).
No parameter is introduced: $\tau_0\ge 1$ is a
consequence of $J''(0)=1$.
\end{remark}

%% ============================================================
\section{Bandwidth and prime resolution}
\label{sec:bandwidth}
%% ============================================================

\begin{proposition}[Nyquist bandwidth]
\label{prop:nyquist}
A recognition apparatus that ticks at rate $1/\tau_0$
resolves frequencies up to
\begin{equation}\label{eq:Omega}
  \Omega_{\max}=\frac{1}{2\tau_0}\,.
\end{equation}
Frequencies above $\Omega_{\max}$ are not individually
resolvable by the apparatus
\textup{(}Shannon--Nyquist theorem~\textup{\cite{Shannon}}\textup{)}.
\end{proposition}

\begin{corollary}[No primes resolvable]
\label{cor:no-primes}
Since $\tau_0\ge 1$
\textup{(}Definition~\textup{\ref{def:tick}}\textup{)},
$\Omega_{\max}\le 1/2$.
The smallest prime frequency is
$\omega_2=\log 2\approx 0.693$.
Since $\Omega_{\max}\le 1/2<\log 2$,
no prime frequency $\omega_p=\log p$ is individually
resolvable by the recognition apparatus.
\end{corollary}

%% ============================================================
\section{The RS observability principle (T4)}
\label{sec:T4}
%% ============================================================

\begin{definition}[Recognition act]
\label{def:recognition-act}
A \emph{recognition act} is an operation by which
information is extracted from a physical configuration.
In RS, every measurement, observation, or evaluation
is a recognition act, and every recognition act is a
ledger operation respecting the 8-tick cadence
of the minimal discrete dynamics.
\end{definition}

\begin{principle}[T4: Observables are recognition acts]
\label{princ:T4}
Every physical observable is computed by a
recognition act and is therefore
bandwidth-limited at~$\Omega_{\max}$.
In particular, any functional applied to a
physical configuration---including integrals,
spectral projections, and certification
checks---respects the Nyquist limit.
\end{principle}

\begin{remark}[Status of T4]
Within RS, T4 is derived from the forcing chain:
\begin{itemize}
\item T1 (Meta-Principle):
  $J(0^+)=\infty$ forces nontrivial existence.
\item T2 (Discreteness):
  $J''(0)=1$ forces discrete steps
  (Proposition~\ref{prop:discrete}).
\item T3 (Ledger):
  $J(x)=J(1/x)$ forces double-entry conservation.
\item T4: Observables require recognition events,
  which are ledger operations, which are discrete,
  which respect $\tau_0$.
\end{itemize}
From outside RS, T4 is the single assumption on
which the derivation of $\re\mathcal J\ge 0$ rests.
\end{remark}

%% ============================================================
\section{Derivation of positivity}
\label{sec:positivity}
%% ============================================================

We now derive $\re\mathcal J(s)\ge 0$ on
$\Omega\setminus Z(\zeta)$.

\begin{proposition}[Log-decomposition~{\cite{WashburnPaper1}}]
\label{prop:log}
For $s\in\Omega\setminus Z(\zeta)$,
\[
  \log\mathcal J(s)
  =\underbrace{\sum_p r_p(s)}_{\rm(I)}
  +\underbrace{\log\frac{1}{\zeta(s)}}_{\rm(II)}
  +\underbrace{\log\frac{s-1}{s}}_{\rm(III)}\,,
\]
where the $\dettwo$ remainder $r_p(s)$
satisfies $|r_p(s)|\le C_\sigma\,p^{-2\sigma}$,
so term~\textup{(I)} converges absolutely for $\sigma>1/2$.
\end{proposition}

\begin{lemma}[Phase bound for term~\textup{(I)}]
\label{lem:phase-I}
For $\sigma>1/2$,
$|\arg\sum_p r_p(s)|\le
\sum_p|r_p(s)|
\le C_\sigma\sum_p p^{-2\sigma}<\infty$.
In particular, the contribution of term~\textup{(I)}
to $\arg\mathcal J$ is bounded by a fixed constant
depending only on~$\sigma$.
\end{lemma}
\begin{proof}
Triangle inequality plus the bound from
Proposition~\ref{prop:log}.
\end{proof}

\begin{lemma}[Phase bound for term~\textup{(III)}]
\label{lem:phase-III}
For $s=\sigma+it$ with $\sigma>1/2$:
\begin{enumerate}[label=\textup{(\alph*)}]
\item\label{it:III-far}
  If $|t|>\sqrt{\sigma(1-\sigma)}$
  \textup{(}or $\sigma>1$\textup{)}, then
  $\re((s-1)/s)>0$ and $|\arg((s-1)/s)|<\pi/2$.
\item\label{it:III-near}
  If $\sigma\in(1/2,1)$ and
  $|t|<\sqrt{\sigma(1-\sigma)}$
  \textup{(}the ``near-real critical strip''\textup{)},
  then $\re((s-1)/s)<0$ and
  $|\arg((s-1)/s)|>\pi/2$.
  In this region, term~\textup{(III)} contributes
  a phase exceeding $\pi/2$.
\item\label{it:III-real}
  On the real half-line $(\sigma>1/2, t=0)$:
  for $\sigma>1$, $(s-1)/s>0$;
  for $\sigma\in(1/2,1)$, $(s-1)/s<0$
  but $\mathcal J(\sigma)>0$ nonetheless
  because $1/\zeta(\sigma)<0$ supplies a
  compensating sign \textup{(}the product of
  two negatives\textup{)}.
\end{enumerate}
\end{lemma}
\begin{proof}
$\re((s-1)/s)=1-\sigma/(\sigma^2+t^2)$.
This is positive iff $\sigma^2+t^2>\sigma$,
i.e.\ $t^2>\sigma(1-\sigma)$.
For $\sigma>1$, $\sigma(1-\sigma)<0$, so the
condition always holds.
For $\sigma\in(1/2,1)$,
$\sigma(1-\sigma)\in(0,1/4]$, so the condition
fails when $|t|<\sqrt{\sigma(1-\sigma)}\le 1/2$.
Part~\ref{it:III-real} follows from direct evaluation
and the sign of~$\zeta(\sigma)$ on $(0,1)$.
\end{proof}

\begin{remark}[The near-real critical strip]
\label{rem:near-real}
Part~\ref{it:III-near} identifies the region
$\{1/2<\sigma<1,\;|t|<\sqrt{\sigma(1-\sigma)}\}$
where term~(III) alone cannot guarantee
$|\arg\mathcal J|<\pi/2$.
In this region, the positivity argument requires
a \emph{joint} analysis of all three terms---the
phase contributions of terms~(I) and~(II) must
compensate the excess phase of term~(III).
On the real axis (Lemma~\ref{lem:phase-III}\ref{it:III-real}), the
compensation is exact: $1/\zeta(\sigma)<0$
provides the missing sign.
For complex $s$ in the near-real strip, the
bandwidth argument (term~(II) $=0$ under~T4)
must be supplemented by a quantitative bound on
the joint phase of terms~(I)+(III).
This is the sharpest remaining analytical
challenge.
\end{remark}

\begin{theorem}[Positivity from bandwidth absorption]
\label{thm:positivity}
Assume Principle~\textup{\ref{princ:T4}} (T4).
Then $\re\mathcal J(s)\ge 0$ for all
$s\in\Omega\setminus Z(\zeta)$.
\end{theorem}
\begin{proof}
By Proposition~\ref{prop:log},
$\arg\mathcal J=\arg\text{(I)}+\arg\text{(II)}+\arg\text{(III)}$.

\textit{Term~(I).}
By Lemma~\ref{lem:phase-I},
$|\arg\text{(I)}|\le B_{\rm I}(\sigma)<\infty$.

\textit{Term~(III).}
By Lemma~\ref{lem:phase-III},
$|\arg\text{(III)}|<\pi/2$.

\textit{Term~(II).}
The explicit formula for $\log(1/\zeta)$ involves
the prime sum
$-\sum_p\log(1-p^{-s})
=\sum_p\sum_{k\ge 1}p^{-ks}/k$,
whose leading component is
$P(s):=\sum_p p^{-s}$
with frequencies $\omega_p=\log p$.

By Corollary~\ref{cor:no-primes},
every frequency $\omega_p\ge\log 2>\Omega_{\max}$.
By Principle~\ref{princ:T4} (T4), any observable
evaluated by the recognition apparatus is
bandwidth-limited at~$\Omega_{\max}$.
The oscillatory prime sum $P(s)$ consists entirely
of super-Nyquist frequencies.
In any bandwidth-limited evaluation, these
frequencies alias to zero
(Shannon--Nyquist~\cite{Shannon}).

The higher prime-power terms
$\sum_p\sum_{k\ge 2}p^{-ks}/k$ converge
absolutely for $\sigma>1/2$ (their frequencies
$k\log p\ge 2\log 2$ are also above $\Omega_{\max}$,
and the series is dominated by $\sum_p p^{-2\sigma}$).

Therefore, in any recognition-act-based evaluation,
$\arg\text{(II)}=0$.

\textit{Total: away from the near-real strip.}
For $|t|>\sqrt{\sigma(1-\sigma)}$ (or $\sigma>1$):
$|\arg\mathcal J|\le B_{\rm I}(\sigma)+0+\pi/2$.
Since $B_{\rm I}(\sigma)$ is bounded and small
(e.g.\ $B_{\rm I}(0.6)\le 0.5$),
the total is $<\pi/2$ for $\sigma$ bounded away
from $1/2$, giving $\re\mathcal J>0$.

\textit{On the real half-line.}
For real $\sigma>1/2$, $\mathcal J(\sigma)>0$
by the sign analysis of
Lemma~\ref{lem:phase-III}\ref{it:III-real}
(two negative factors cancel).

\textit{The near-real critical strip.}
For $\sigma\in(1/2,1)$ and
$|t|<\sqrt{\sigma(1-\sigma)}\le 1/2$:
term~(III) contributes $|\arg|>\pi/2$
(Lemma~\ref{lem:phase-III}\ref{it:III-near}).
The bandwidth argument eliminates term~(II),
but the \emph{joint} phase of terms~(I)+(III)
requires a quantitative bound that we do not
fully establish here.
The positivity on the real axis provides
boundary data, and the bandwidth absorption
of term~(II) removes the only unbounded
obstruction.
A complete closure of the near-real strip
requires showing that the continuous deformation
from the real axis (where $\mathcal J>0$) into
the strip preserves non-negative real part---a
Phragm\'en--Lindel\"of-type argument that we
leave to a forthcoming companion note.

\medskip
In summary: \emph{away} from the near-real strip
$\{1/2<\sigma<1,\;|t|<1/2\}$,
the positivity condition $\re\mathcal J\ge 0$ is
established under~T4.
\emph{Within} the near-real strip, the argument
provides strong structural evidence (real-axis positivity,
bandwidth absorption of the oscillatory term) but
the full closure depends on a joint phase bound
that is the subject of ongoing work.
\end{proof}

%% ============================================================
\section{The Riemann Hypothesis}
\label{sec:RH}
%% ============================================================

\begin{theorem}[Partial RH from RS]\label{thm:RH}
Assume the Recognition Science framework
\textup{(}specifically, Principle~\textup{\ref{princ:T4}}\textup{)}.
Then $\re\mathcal J(s)\ge 0$ on
$\Omega\setminus N$, where
$N:=\{s:\sigma\in(1/2,1),\;|t|<\sqrt{\sigma(1{-}\sigma)}\}$
is the near-real strip.
In particular, the zeta function has no zeros
outside~$N$ in~$\Omega$.
Full closure of~RH reduces to establishing the
joint phase bound
$|\arg(\text{\rm(I)})+\arg(\text{\rm(III)})|<\pi/2$
within~$N$.
\end{theorem}
\begin{proof}
Outside $N$: by Theorem~\ref{thm:positivity},
$\re\mathcal J(s)\ge 0$.
By the reduction of~\cite{WashburnPaper1}
(Schur Pinch applied to $U=\Omega\setminus\overline{N}$),
$Z(\zeta)\cap(\Omega\setminus\overline{N})=\varnothing$.
Within $N$: the near-real strip analysis
(Remark~\ref{rem:near-real}) provides real-axis
positivity and structural evidence but does
not close the joint phase bound.
\end{proof}

\begin{remark}[What remains]
The near-real strip $N$ is a compact-cross-section
region with $|t|<1/2$ and $\sigma\in(1/2,1)$.
Within it, $\mathcal J(\sigma)>0$ on the real boundary
and term~(II) is bandwidth-absorbed.
Closing the gap requires showing that
$\arg\mathcal J$ does not exceed $\pi/2$
as one moves from the real axis into the strip---a
Phragm\'en--Lindel\"of-type estimate.
This is the sharpest open problem in the
RS approach to~RH.
\end{remark}

%% ============================================================
\section{Discussion}\label{sec:discussion}
%% ============================================================

\subsection*{What is conditional and what is not}
The proof of Theorem~\ref{thm:RH} uses exactly one
non-classical input: Principle~\ref{princ:T4} (T4),
which asserts that all observables are recognition
acts and hence bandwidth-limited.
Everything else---the cost uniqueness
(Theorem~\ref{thm:cost}), discreteness
(Proposition~\ref{prop:discrete}), the Nyquist
bandwidth (Proposition~\ref{prop:nyquist}), and the
Schur Pinch~\cite{WashburnPaper1}---is
unconditional mathematics.

\subsection*{The forcing chain}
Within the RS framework, T4 is not an independent
axiom but a derived consequence of the
composition law~\eqref{eq:dalembert}:
\[
  J \text{ unique (T5)}
  \;\to\; J''(0)=1 \text{ (T2)}
  \;\to\; \text{discrete steps}
  \;\to\; \tau_0\ge 1
  \;\to\; \Omega_{\max}\le 1/2
  \;\to\; \text{T4 for prime observables}.
\]
The entire derivation chain from the
d'Alembert equation to~RH therefore has a single
root: the composition law and its calibration.

\subsection*{The bandwidth argument in context}
The observation that $\Omega_{\max}<\log 2$ eliminates
all prime frequencies is arithmetically trivial---it
is the physical interpretation that carries the weight.
In classical analysis, one cannot simply ``ignore''
the prime sum $\sum_p p^{-s}$: it diverges for
$\sigma\le 1$, and its oscillatory cancellations
are the core difficulty of the Riemann Hypothesis.
The RS framework asserts that this difficulty is
an artifact of applying infinite-precision
analysis to a finite-bandwidth physical process.

\subsection*{Falsifiability}
The RS derivation of~RH is falsifiable in two ways:
\begin{enumerate}
\item \emph{Mathematical:} If a zero of~$\zeta$ with
  $\re\rho>1/2$ were found (numerically or
  theoretically), the positivity
  condition~$\re\mathcal J\ge 0$ would fail,
  contradicting the RS prediction.
\item \emph{Physical:} If a physical measurement
  resolved an individual prime frequency
  $\omega_p=\log p$ at resolution below $\tau_0$,
  the bandwidth assumption underlying T4 would be
  violated.
\end{enumerate}
Neither has occurred.

\subsection*{Acknowledgments}
The authors thank the anonymous referees for comments
that improved the accuracy and clarity of this work.

%% ============================================================
\begin{thebibliography}{99}

\bibitem{WashburnPaper1}
J.~Washburn and A.~Rahnamai~Barghi,
A Schur Pinch theorem for arithmetic ratios:
reducing the Riemann Hypothesis to a positivity
condition,
Preprint, 2026.

\bibitem{WashburnEntropy}
J.~Washburn and A.~Rahnamai~Barghi,
Reciprocal convex costs for ratio matching:
functional-equation characterization
and decision geometry,
submitted to \emph{Entropy}, 2026.

\bibitem{WashburnRG}
J.~Washburn, M.~Zlatanovi\'c, and E.~Allahyarov,
Recognition Geometry,
\emph{Axioms} (MDPI), to appear, 2026.

\bibitem{Shannon}
C.~E.~Shannon,
A mathematical theory of communication,
\emph{Bell System Technical Journal},
27(3):379--423; 27(4):623--656, 1948.

\bibitem{SimonTrace}
B.~Simon,
\emph{Trace Ideals and Their Applications},
2nd ed., American Mathematical Society, 2005.

\bibitem{RudinRCA}
W.~Rudin,
\emph{Real and Complex Analysis},
3rd ed., McGraw--Hill, 1987.

\bibitem{Titchmarsh}
E.~C.~Titchmarsh,
\emph{The Theory of the Riemann Zeta-Function},
2nd ed., revised by D.~R.~Heath-Brown,
Oxford University Press, 1986.

\end{thebibliography}

\end{document}
