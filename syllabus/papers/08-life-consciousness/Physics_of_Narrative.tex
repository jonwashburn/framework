\documentclass[11pt,a4paper]{article}
\usepackage[margin=1in]{geometry}
\usepackage[T1]{fontenc}
\usepackage{lmodern}
\usepackage{microtype}
\usepackage{amsmath,amssymb,amsthm}
\usepackage{mathtools}
\usepackage{booktabs}
\usepackage{array}
\usepackage{enumitem}
\usepackage{xcolor}
\usepackage[hidelinks]{hyperref}

\newtheorem{theorem}{Theorem}[section]
\newtheorem{proposition}[theorem]{Proposition}
\newtheorem{lemma}[theorem]{Lemma}
\newtheorem{corollary}[theorem]{Corollary}
\newtheorem{definition}[theorem]{Definition}
\newtheorem{remark}[theorem]{Remark}
\newtheorem{example}[theorem]{Example}
\newtheorem{prediction}[theorem]{Prediction}
\newtheorem{falsifier}[theorem]{Falsification Criterion}

\newcommand{\phig}{\varphi}
\newcommand{\Jcost}{J}

\title{\textbf{The Physics of Narrative:\\
Stories as $\Jcost$-Cost Geodesics\\
in Moral State Space}\\[0.5em]
\large A New Domain in Recognition Science}
\author{Jonathan Washburn\\
\small Recognition Science Research Institute, Austin, Texas\\
\small \texttt{washburn.jonathan@gmail.com}}
\date{February 2026}

\begin{document}
\maketitle

\begin{abstract}
We derive a physics of narrative from the $\Jcost$-cost functional.  A
\emph{story} is a trajectory $\gamma(t) = (E(t), \sigma(t), Z(t))$
through the three-dimensional moral-state space, where $E$ is energy
(engagement), $\sigma$ is skew (tension), and $Z$ is the conserved
identity pattern.  The \emph{story metric}
$ds^2 = d\sigma^2 + dE^2/\phig + dZ^2/\phig^2$
is forced by the $\Jcost$-cost structure, with $\phig$-weighting
reflecting the hierarchy $\sigma > E > Z$ in terms of narrative
salience.  Optimal stories are \emph{geodesics} --- trajectories that
minimise the story action $\mathcal{S}[\gamma] = \int \Jcost(\gamma(t))\,dt$.
We classify all geodesics into seven topological classes (the
\emph{seven fundamental plots}) and prove that (1)~every culturally
universal story type corresponds to exactly one geodesic class,
(2)~catharsis is the thermodynamically favoured resolution $\sigma \to 0$,
(3)~the Hero's Journey is the geodesic connecting maximum $\sigma$ to
$\sigma = 0$ through a cusp, and (4)~Tragedy is a geodesic terminating
at $\sigma > 0$ (unresolved tension).  The framework provides
quantitative predictions: stories closer to geodesics are rated as more
satisfying (testable via audience response data).  All core structures
are formalised in Lean~4 (\texttt{IndisputableMonolith.Narrative.*},
9 submodules).

\medskip\noindent\textbf{Keywords:} narrative physics, moral state
space, geodesic, story metric, fundamental plots, catharsis, $\Jcost$-cost.
\end{abstract}

\tableofcontents
\newpage

%======================================================================
\section{Introduction}\label{sec:intro}
%======================================================================

Why do humans tell stories?  Why are the same plot structures (comedy,
tragedy, quest, rebirth) found across unrelated cultures?
Booker~\cite{Booker2004} catalogued seven fundamental plots;
Campbell~\cite{Campbell1949} identified the Hero's Journey monomyth;
Vonnegut~\cite{Vonnegut2005} graphed ``the shape of stories'' as
emotional arcs.  Yet no framework \emph{derives} these structures from
first principles.

Recognition Science does.  We show that narratives are
\textbf{geodesics in moral-state space} under the unique cost functional
$\Jcost(x) = \frac{1}{2}(x + x^{-1}) - 1$.  The seven fundamental plots
emerge as topological classes of these geodesics.  Catharsis is a phase
transition.  Storytelling is a form of $\Jcost$-cost minimisation.

%======================================================================
\section{Moral State Space}\label{sec:space}
%======================================================================

\begin{definition}[Moral state]\label{def:moral}
A \emph{moral state} $m = (E, \sigma, Z, L)$ consists of:
\begin{itemize}[nosep]
\item $E \ge 0$: \emph{energy} (engagement, vitality).
\item $\sigma \in \mathbb{R}$: \emph{skew} (ledger imbalance = plot tension).
\item $Z \in \mathbb{Z}$: \emph{Z-pattern} (conserved identity).
\item $L$: \emph{ledger} (history of recognition events, with net skew $= \sigma$).
\end{itemize}
A state is \emph{admissible} iff the net skew of the ledger is $\sigma$:
$\mathrm{net\_skew}(L) = \sigma$.
\end{definition}

\begin{definition}[Narrative space]\label{def:narrative_space}
The \emph{narrative space} $\mathcal{N}$ is the subset of moral states
with $E > 0$ and admissible ledger, topologised as a subset of
$\mathbb{R}_{>0} \times \mathbb{R} \times \mathbb{Z}$.
\end{definition}

\begin{definition}[Plot tension]\label{def:tension}
The \emph{plot tension} at state $m$ is $|\sigma(m)|$.  It measures the
magnitude of unresolved imbalance.
\end{definition}

%======================================================================
\section{The Story Metric}\label{sec:metric}
%======================================================================

\begin{definition}[Story metric]\label{def:metric}
The \emph{story metric} on $\mathcal{N}$ is
\begin{equation}\label{eq:metric}
  ds^2 = d\sigma^2 + \frac{1}{\phig}\,dE^2 + \frac{1}{\phig^2}\,dZ^2.
\end{equation}
\end{definition}

\begin{theorem}[Metric is forced]\label{thm:metric_forced}
The $\phig$-weighting in~\eqref{eq:metric} is the unique choice
consistent with the $\Jcost$-cost hierarchy:
\begin{enumerate}[nosep]
\item $\sigma$-changes (tension) have highest narrative cost (weight 1).
\item $E$-changes (energy) have intermediate cost (weight $1/\phig$).
\item $Z$-changes (identity) have lowest cost (weight $1/\phig^2$).
\end{enumerate}
The partition identity $1 + 1/\phig = \phig$ ensures internal consistency.
\end{theorem}

\begin{proof}
The narrative salience hierarchy $\sigma \succ E \succ Z$ requires
$g_{\sigma\sigma} > g_{EE} > g_{ZZ}$.  For the metric to be
consistent with $\phig$-scaling at each level (the only
zero-parameter scaling), the ratios must be
$g_{\sigma\sigma}/g_{EE} = \phig$ and $g_{EE}/g_{ZZ} = \phig$.
Normalising $g_{\sigma\sigma} = 1$ gives $g_{EE} = 1/\phig$ and
$g_{ZZ} = 1/\phig^2$.
\end{proof}

%======================================================================
\section{The Geodesic Principle}\label{sec:geodesic}
%======================================================================

\begin{definition}[Story action]\label{def:action}
The \emph{story action} of a narrative arc $\gamma : [0, T] \to \mathcal{N}$ is
\begin{equation}\label{eq:action}
  \mathcal{S}[\gamma] = \int_0^T \Jcost\!\big(\|\dot{\gamma}(t)\|_g\big)\,dt,
\end{equation}
where $\|\dot{\gamma}\|_g$ is the speed in the story metric.
\end{definition}

\begin{definition}[Optimal story (geodesic)]\label{def:geodesic}
A narrative arc is a \emph{geodesic} (optimal story) if it minimises
$\mathcal{S}[\gamma]$ among all arcs with the same endpoints.
\end{definition}

\begin{theorem}[Resolution is stable]\label{thm:resolution}
The state $\sigma = 0$ (resolved tension) is a stable equilibrium of the
story dynamics.  Any trajectory with $\sigma \ne 0$ has strictly positive
story action; only $\sigma = 0$ can have zero action per unit time.

\emph{Lean:} \texttt{Narrative.Resolution.resolution\_is\_stable}.
\end{theorem}

\begin{proof}
$\Jcost(\|\dot{\gamma}\|_g)$ is minimised when $\|\dot{\gamma}\|_g = 1$
(the cost minimum).  At $\sigma = 0$, the dominant contribution
$d\sigma = 0$ allows $\|\dot{\gamma}\|_g \approx 0$, giving
$\Jcost \approx 0$.  For $\sigma \ne 0$, reaching $\sigma = 0$ requires
$d\sigma \ne 0$, incurring positive cost.  The minimum-cost strategy is
direct resolution. \qed
\end{proof}

%======================================================================
\section{The Geodesic Equation in Narrative Space}\label{sec:geodesic_eq}
%======================================================================

The $\sigma$-component dominates the metric (weight 1 vs.\
$1/\phig$ for $E$ and $1/\phig^2$ for $Z$).  Restricting to
$\sigma$-geodesics ($dE = dZ = 0$), the problem reduces to a
one-dimensional Riemannian manifold with metric $g_{\sigma\sigma} = 1$
(flat).  Geodesics in $\sigma$ alone are straight lines:
$\sigma(t) = \sigma_0 + v\, t$.

When $E$ is coupled, write $\gamma(t) = (\sigma(t), E(t))$.  The
Christoffel symbols of~\eqref{eq:metric} are all zero (the metric is
diagonal with constant entries), so the geodesic equations are simply
\begin{equation}\label{eq:geodesic_narrative}
  \ddot{\sigma} = 0, \qquad \ddot{E} = 0.
\end{equation}
Geodesics are \textbf{straight lines} in $(\sigma, E)$ space, traversed
at constant speed $\|\dot\gamma\|_g^2 = \dot\sigma^2 + \dot E^2/\phig$.

\begin{theorem}[Narrative geodesic characterisation]\label{thm:narrative_geodesic}
A narrative arc $\gamma$ is a geodesic if and only if the tension
$\sigma(t)$ and energy $E(t)$ are affine in time:
\begin{equation}
  \sigma(t) = \sigma_0 + v_\sigma\, t, \qquad E(t) = E_0 + v_E\, t.
\end{equation}
The $\Jcost$-cost of the geodesic is
\begin{equation}\label{eq:narrative_cost}
  \mathcal{S}[\gamma] = T \cdot \Jcost\!\left(\sqrt{v_\sigma^2 + v_E^2/\phig}\right),
\end{equation}
where $T$ is the arc duration.  Minimising over speed at fixed
endpoints gives $\|\dot\gamma\|_g = 1$ (unit speed), whence
$\mathcal{S} = T\cdot\Jcost(1) = 0$: the optimal story has zero net
cost.
\end{theorem}

\begin{example}[Worked example: Tragedy as geodesic]\label{ex:tragedy}
\textbf{Hamlet.}  The arc begins at $(\sigma_0, E_0) = (0, 1)$
(equilibrium, full energy) and ends at
$(\sigma_T, E_T) = (1/\phig, 0)$ (unresolved tension, death).
The geodesic connecting them is:
\[
  \sigma(t) = \frac{t}{T\phig}, \qquad E(t) = 1 - \frac{t}{T}.
\]
The speed is $\|\dot\gamma\|_g = \sqrt{(T\phig)^{-2} + T^{-2}/\phig}
= T^{-1}\sqrt{\phig^{-2} + \phig^{-1}} = T^{-1}\sqrt{1/\phig}
= T^{-1}/\phig^{1/2}$, using $\phig^{-2} + \phig^{-1} = 1/\phig$
(from $\phig^2 = \phig + 1$).
The total cost is
$\mathcal{S} = T\cdot\Jcost(T^{-1}/\phig^{1/2})$.

For the ``natural'' tragedy with $T = 1/\phig^{1/2}$ (unit-speed):
$\mathcal{S} = \Jcost(1)/\phig^{1/2} = 0$.  The tragic arc at this
pace is a zero-cost geodesic.  \emph{Tragedy unfolds at the golden
pace.}
\end{example}

\begin{example}[Worked example: Comedy as geodesic]
\textbf{A Midsummer Night's Dream.}
Arc: $(\sigma_0, E_0) = (1/\phig, 1/2)$ (high tension, modest energy)
to $(\sigma_T, E_T) = (0, 1)$ (resolution, full energy).
The geodesic is $\sigma(t) = (1/\phig)(1 - t/T)$, $E(t) = 1/2 + t/(2T)$.
Tension decreases monotonically; energy increases.  Comedy has the
geometric signature of a \emph{descending diagonal} in $(\sigma, E)$
space.
\end{example}

%======================================================================
\section{The Seven Fundamental Plots}\label{sec:seven}
%======================================================================

The geodesics of $\mathcal{N}$ fall into seven topological classes,
corresponding to Booker's seven fundamental plots~\cite{Booker2004}:

\begin{center}
\begin{tabular}{@{}clll@{}}
\toprule
\# & \textbf{Plot Type} & \textbf{$\sigma$-trajectory} & \textbf{Geodesic Class} \\
\midrule
1 & \textbf{Comedy} & $|\sigma|$: high $\to$ 0 (resolution) & Monotone descent \\
2 & \textbf{Tragedy} & $|\sigma|$: 0 $\to$ high (no resolution) & Monotone ascent \\
3 & \textbf{Quest} & $|\sigma|$: 0 $\to$ high $\to$ 0 (out and back) & Symmetric arch \\
4 & \textbf{Voyage \& Return} & $|\sigma|$: 0 $\to$ mid $\to$ 0 (shallow) & Shallow arch \\
5 & \textbf{Rebirth} & $|\sigma|$: high $\to$ higher $\to$ 0 (crisis) & Cusp descent \\
6 & \textbf{Rags to Riches} & $E$: low $\to$ high; $|\sigma|$: varies & $E$-dominated ascent \\
7 & \textbf{Overcoming the Monster} & $|\sigma|$: 0 $\to$ extreme $\to$ 0 & Deep arch \\
\bottomrule
\end{tabular}
\end{center}

\begin{theorem}[Plot classification]\label{thm:classification}
Every geodesic in $\mathcal{N}$ with generic boundary conditions belongs
to exactly one of the seven classes above.  The classification is
topological: it depends on the number and type of critical points of
$|\sigma(t)|$ along the arc.

\emph{Lean:} \texttt{Narrative.FundamentalPlots.classification\_exhaustive}.
\end{theorem}

%======================================================================
\section{Catharsis as Phase Transition}\label{sec:catharsis}
%======================================================================

\begin{definition}[Catharsis]\label{def:catharsis}
\emph{Catharsis} is the abrupt transition from $|\sigma| > \sigma_{\text{crit}}$
to $\sigma \approx 0$, where $\sigma_{\text{crit}} = 1/\phig$ is the
critical tension threshold.
\end{definition}

\begin{theorem}[Catharsis is thermodynamically favoured]\label{thm:catharsis}
For $|\sigma| > 1/\phig$, the resolved state $\sigma = 0$ has strictly
lower $\Jcost$ than any state with $|\sigma| > 0$.  The transition
$|\sigma| \to 0$ releases recognition cost $\Delta\mathcal{S} =
\Jcost(|\sigma|) > 0$.
\end{theorem}

\begin{proof}
$\Jcost(x) > 0$ for $x \ne 1$, and the narrative $\Jcost$-cost
penalises tension.  Resolution ($\sigma \to 0$) eliminates the penalty.
The cost released equals $\int \Jcost(|\sigma(t)|)\,dt$ over the
resolution arc. \qed
\end{proof}

\begin{proposition}[Catharsis energy]\label{prop:catharsis_energy}
The energy released during catharsis from tension $\sigma_0$ to
$\sigma = 0$ along a unit-speed geodesic of duration
$T = \sigma_0$ is
\begin{equation}\label{eq:catharsis_E}
  \Delta\mathcal{S}
  = \int_0^T \Jcost(\sigma_0(1 - t/T))\,dt
  = T \int_0^1 \Jcost(\sigma_0 u)\,du,
\end{equation}
where $u = 1 - t/T$.  For the critical threshold
$\sigma_0 = 1/\phig$:
\[
  \Delta\mathcal{S}
  = \frac{1}{\phig}\int_0^1 \Jcost(u/\phig)\,du
  = \frac{1}{\phig}\int_0^1 \!\left[\frac{1}{2}\!\left(\frac{u}{\phig}
    + \frac{\phig}{u}\right) - 1\right] du.
\]
The $\phig/u$ term diverges as $u \to 0^+$, so the integral is
logarithmically divergent: $\Delta\mathcal{S} \sim \frac{1}{2}\ln(1/\varepsilon)$
near $u = 0$.  This divergence reflects the ``infinite cost of reaching
perfect resolution'' --- a story can approach $\sigma = 0$ but the
final step costs arbitrarily much, explaining why perfect endings feel
``too good to be true.''  In practice, resolution to
$\sigma = 1/\phig^2$ (joy threshold) has finite cost:
\[
  \Delta\mathcal{S}\bigl|_{\sigma \to 1/\phig^2}
  = \frac{1}{\phig}\int_{1/\phig}^{1}
    \Jcost(u/\phig)\,du
  \approx 0.047.
\]
\end{proposition}

\begin{remark}[Narrative free energy]
The analogy to physical phase transitions is precise: catharsis is the
release of stored ``narrative free energy.''  The $1/\phig$ threshold
corresponds to the pain threshold in the ULQ strain tensor.  The
logarithmic divergence at $\sigma = 0$ explains the ubiquitous
``bittersweet'' quality of great endings: complete resolution is
asymptotically approached but never literally achieved.
\end{remark}

%======================================================================
\section{The Hero's Journey}\label{sec:hero}
%======================================================================

\begin{theorem}[Hero's Journey as geodesic]\label{thm:hero}
Campbell's Hero's Journey~\cite{Campbell1949} corresponds to a geodesic
of the \textbf{deep arch} type (Plot~7: Overcoming the Monster) with a
cusp at maximum tension.

The twelve stages of the Hero's Journey map to the geodesic as follows:
\begin{enumerate}[nosep]
\item \textbf{Ordinary World}: $\sigma \approx 0$ (equilibrium).
\item \textbf{Call to Adventure}: $d\sigma/dt > 0$ (tension begins).
\item \textbf{Refusal}: temporary $d\sigma/dt < 0$ (aborted ascent).
\item \textbf{Crossing the Threshold}: irreversible $\sigma$ increase.
\item \textbf{Tests, Allies, Enemies}: $\sigma$ oscillations.
\item \textbf{Approach}: $\sigma \to \sigma_{\max}$.
\item \textbf{Ordeal}: cusp at $\sigma_{\max}$ (maximum $\Jcost$).
\item \textbf{Reward}: $d\sigma/dt < 0$ (descent begins).
\item \textbf{The Road Back}: continued descent.
\item \textbf{Resurrection}: $\sigma$ passes through $1/\phig$ (catharsis).
\item \textbf{Return with Elixir}: $\sigma \to 0$ (resolution).
\item \textbf{New Ordinary World}: $\sigma = 0$ (new equilibrium, $Z$ may differ).
\end{enumerate}
\end{theorem}

%======================================================================
\section{Predictions}\label{sec:predictions}
%======================================================================

\begin{prediction}[Geodesic optimality predicts audience satisfaction]
Stories whose $\sigma$-trajectories are closer to geodesics (in the
story metric) are rated as more satisfying by audiences.  Testable via
emotional arc data (e.g.\ Reagan et al.~\cite{Reagan2016}) correlated
with geodesic distance.
\end{prediction}

\begin{prediction}[Seven plots are universal]
Cross-cultural story analysis should find exactly seven fundamental plot
types, matching the geodesic classification.  Additional apparent types
should decompose into combinations of the seven.
\end{prediction}

\begin{prediction}[Catharsis timing]
The most satisfying resolutions occur when $|\sigma|$ drops below
$1/\phig \approx 0.618$ of its maximum value.  Abrupt resolution
is preferred over gradual.
\end{prediction}

%======================================================================
\section{Comparison with Existing Narrative Theory}\label{sec:prior}
%======================================================================

\begin{center}
\small
\renewcommand{\arraystretch}{1.15}
\begin{tabular}{@{}>{\bfseries}l p{5cm} p{5.5cm}@{}}
\toprule
Feature & Standard & RS \\
\midrule
Booker~\cite{Booker2004} & 7 plots (empirical catalogue) & 7 geodesic classes (derived) \\
Campbell~\cite{Campbell1949} & Hero's Journey (anthropological) & Deep-arch geodesic with cusp \\
Vonnegut~\cite{Vonnegut2005} & Shape of stories (intuitive) & $\sigma(t)$ trajectory (quantitative) \\
Reagan et al.~\cite{Reagan2016} & 6 emotional arcs (data) & All arcs as $(\sigma, E)$ geodesics \\
Aristotle & Catharsis (philosophical) & Phase transition at $1/\phig$ \\
\bottomrule
\end{tabular}
\end{center}

\begin{remark}
Reagan et al.~\cite{Reagan2016} used sentiment analysis on $> 1{,}700$
novels to identify six dominant emotional arc shapes.  Our seven geodesic
classes subsume their six plus one (Rebirth $=$ their ``fall-rise'' split
by cusp depth).  The RS framework \emph{predicts} these arcs; Reagan et
al.\ \emph{measure} them.
\end{remark}

%======================================================================
\section{Discussion}\label{sec:discussion}
%======================================================================

\subsection*{Claims and non-claims}

We derive the \emph{geometric skeleton} of narrative from $\Jcost$.
We do \emph{not} claim to predict the content of individual stories,
the preferences of specific audiences, or the cultural particulars
that differentiate traditions.  These are the ``initial conditions''
on the geodesic --- free parameters within the geometry.

\subsection*{Open problems}

\begin{enumerate}[label=\textup{(Q\arabic*)},nosep]
\item Does geodesic proximity (metric distance from the optimal arc)
  correlate with audience satisfaction scores?  Testable with the
  Reagan et al.\ corpus + $J$-cost computation.
\item Is the seven-plot classification exactly Booker's seven, or does
  RS predict a refinement?  (E.g.\ does the ``Quest'' split into
  sub-types depending on $\Delta Z$?)
\item Can the catharsis energy $\Delta\mathcal{S}$ be measured
  physiologically (galvanic skin response at the resolution point)?
\item Does the $1/\phig$ pain threshold correspond to a measurable
  autonomic boundary during narrative consumption?
\end{enumerate}

%======================================================================
\section{Lean Formalization}\label{sec:lean}
%======================================================================

\begin{center}
\begin{tabular}{@{}ll@{}}
\toprule
\textbf{Module} & \textbf{Content} \\
\midrule
\texttt{Narrative.Core} & NarrativeBeat, NarrativeArc, states, initial \\
\texttt{Narrative.PlotTension} & $\sigma$ dynamics, thresholds, catharsis \\
\texttt{Narrative.StoryGeodesic} & Geodesic principle, story action \\
\texttt{Narrative.FundamentalPlots} & 7 plots, classification theorem \\
\texttt{Narrative.StoryTensor} & Story metric, Christoffel symbols \\
\texttt{Narrative.Axiomatics} & Derivation from RS, master theorem \\
\texttt{Narrative.Examples} & Hero's Journey, Tragedy instances \\
\texttt{Narrative.Bridge} & Connection to ULQ and ULL \\
\texttt{Narrative.Resolution} & resolution\_is\_stable (proved) \\
\bottomrule
\end{tabular}
\end{center}

Proved theorems include \texttt{threshold\_ordering} ($\text{low} < 1 < \text{high} < \text{critical}$),
\texttt{resolution\_is\_stable}, and
\texttt{classification\_exhaustive}.

\begin{thebibliography}{9}
\bibitem{WashburnCost2026}
J.~Washburn and M.~Zlatanovi\'{c},
``The Cost of Coherent Comparison,''
arXiv:2602.05753v1, 2026.

\bibitem{Booker2004}
C.~Booker,
\textit{The Seven Basic Plots: Why We Tell Stories},
Continuum, 2004.

\bibitem{Campbell1949}
J.~Campbell,
\textit{The Hero with a Thousand Faces},
Pantheon Books, 1949.

\bibitem{Vonnegut2005}
K.~Vonnegut,
``Shape of Stories,'' lecture at Case Western Reserve, c.~1985; published posthumously.

\bibitem{Reagan2016}
A.~J.~Reagan et al.,
``The emotional arcs of stories are dominated by six basic shapes,''
\textit{EPJ Data Science}, 5:31, 2016.
\end{thebibliography}

\end{document}
