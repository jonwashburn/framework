\documentclass[11pt]{article}

\usepackage[margin=1in]{geometry}
\usepackage[T1]{fontenc}
\usepackage[utf8]{inputenc}
\usepackage{lmodern}
\usepackage{microtype}
\usepackage{amsmath,amssymb,amsthm,mathtools}
\usepackage[colorlinks=true,linkcolor=blue,citecolor=blue,urlcolor=blue]{hyperref}
\usepackage[nameinlink]{cleveref}
\usepackage{enumitem}
\usepackage{xcolor}
\setlist{nosep}

% Theorem environments
\newtheorem{theorem}{Theorem}[section]
\newtheorem{lemma}[theorem]{Lemma}
\newtheorem{proposition}[theorem]{Proposition}
\newtheorem{corollary}[theorem]{Corollary}
\newtheorem{definition}[theorem]{Definition}
\newtheorem{remark}[theorem]{Remark}

% Notation
\newcommand{\C}{\mathcal{C}}
\newcommand{\E}{\mathcal{E}}
\newcommand{\CR}{\mathcal{C}_R}
\newcommand{\M}{\mathcal{M}}
\newcommand{\Z}{\mathbb{Z}}
\newcommand{\R}{\mathbb{R}}
\newcommand{\N}{\mathbb{N}}
\newcommand{\lk}{\operatorname{lk}}
\newcommand{\SO}{\operatorname{SO}}

% Colored-addition helper
\newcommand{\ADD}[1]{\textcolor{teal}{#1}}
\newcommand{\ADDBOX}[1]{\par\medskip\noindent\fcolorbox{teal}{teal!8}{\parbox{\dimexpr\linewidth-2\fboxsep-2\fboxrule}{#1}}\medskip}

\title{Version-3 Comment-1 on D3}
\date{}

\begin{document}
\maketitle% Requires: \usepackage{amsmath,amssymb}

\subsection*{List of comments!}

\ADDBOX{\ADD{\textbf{NEW --- Leverage the published Axioms paper (axioms-4140269).}}
\ADD{Now that ``Reciprocal Convex Costs for Ratio Matching: Axiomatic Characterization'' (Washburn \& Rahnamai Barghi, \emph{Axioms} 2026, 15(2), 90; doi:\,\texttt{10.3390/axioms15020090})
is accepted, we should import its main result into this paper and cite it.  This strengthens the
submission by anchoring our cost functional in a \emph{published, peer-reviewed theorem} rather
than re-deriving it.  Concrete insertions are described below in \textcolor{teal}{teal}.}}

\begin{enumerate}
  \item In the Introduction, we still state:
  \begin{quote}
    ``Assume $\CR$ is manifold-like \dots'' 
  \end{quote}
  but never define ``manifold-like'' anywhere as a standalone term. We have replaced it conceptually with ``admits an effective manifold model $\M$'', but we did not propagate that terminology back into Theorem~1.2.

  \medskip
  \noindent\textbf{Fix:}
  Replace the ``manifold-like'' phrasing in Theorem~1.2 with something that directly references
  Definition~2.12. For example:
  \begin{quote}
    ``Assume $(\C,\E,R)$ admits an effective manifold model $\M$ in the sense of Definition~2.12\dots''
  \end{quote}
  That makes paper consistent with the two-scale story.

  \bigskip

  \item We reused the symbol $\omega$ in two slightly different ways (as $\kappa/\Omega$ in Method~1 and as a $(4-D)$-dependent ratio in the
  Binet method). They are the same dimensionless ratio in the linearized regime, thus lets say:
  \begin{quote}
    ``The $\omega$ in Method~2 coincides with the ratio $\kappa/\Omega$ in Method~1.''
  \end{quote}
%%
  \item In Step~(3) of Theorem~4.3, we write
  \begin{equation}
         (\partial Q)\cdot B = Q\cdot (\partial B).
  \end{equation}
  This is not the correct boundary/intersection compatibility identity. The correct schematic identity (up to sign conventions) is
  \begin{equation}
         \partial(Q \pitchfork B)
      = (\partial Q)\pitchfork B \ \pm\  Q \pitchfork (\partial B),
  \end{equation}
  i.e.\ there is an extra $\partial(Q\pitchfork B)$ term that is dropped. 

  \medskip
  \noindent\textbf{How to fix?}
  The fix is to argue at the homology level:
  \begin{itemize}
    \item In an oriented $D$-manifold, the intersection number $Z\cdot B$ depends only on the homology class
      $[Z]\in H_{p+1}(\CR)$.
    \item Since $H_{p+1}(\CR)=0$, we have $[Z]=0$, hence $Z\cdot B = 0$.
    \item Therefore $W\cdot B = W'\cdot B$.
  \end{itemize}
  This avoids all chain-level sign/boundary complications and is standard.

  \medskip
  \noindent\textbf{Another one:} In Proposition~3.3, we say ``$p=0$ gives $D=1$.'' But Theorem~4.3 assumes $0<p<D$, so we cannot cite it to justify $p=0$.

  \medskip
  \noindent\textbf{Fix?}
  \begin{itemize}
    \item easiest: we redefine $A_A=\{3,5,7,\dots\}$ by requiring $p\ge 1$ (still non-singleton, still intersects to $\{3\}$), and drop the
      $D=1$ talk entirely; or
    \item add a short separate remark if we really want to discuss $0$-dimensional ``linking'' (but I think it adds confusion as it doesn't help the selection argument).
  \end{itemize}
  \medskip
  \noindent\textbf{New issue introduced while trying to justify $p$-flexibility:}
  Remark~3.4 is conceptually broken. We call these ``codimension-2 defects,'' but we compute the codimension:
  \[
    D-p = D-\frac{D-1}{2} = \frac{D+1}{2},
  \]
  which equals $2$ only when $D=3$. So as written, the remark effectively says:
  \begin{quote}
    ``These are codimension-2 \dots\ and they are codimension-2 only in $D=3$. ''
  \end{quote}

  \medskip
  \noindent\textbf{Fix?}
  Rename and reframe. The statement is:
  \begin{itemize}
    \item Same-dimension linking requires $p=\frac{D-1}{2}$, which is codimension $p+1$, not ``codimension~2'' in general.
    \item Codimension-2 defects are a different physically motivated class; if we want codim-2 specifically, that specialization
      directly forces $D=3$.
  \end{itemize}
  %%%
  \item In a statement ``Prior Approaches\ldots'':
  \begin{quote}
    ``Freedman's exotic $\mathbb{R}^4$ theorem shows \dots\ $\mathbb{R}^4$ admits uncountably many distinct smooth structures\ldots''
  \end{quote}
  This is not safe as written. At minimum it is misattributed / oversimplified.

  \medskip
  \noindent\textbf{Fix?} We should rewrite cautiously or remove unless we can cite a precise correct attribution and statement.
  %%%%%
    \item \textbf{``Knot theory is nontrivial only in dimensions $D=3,4$''}\\
  This is false and worse, the very next clause says ``surfaces link in $D=5$,'' which contradicts the ``only $3,4$'' part.

  \medskip
  \noindent\textbf{Fix?} we can rewrite as:
  \begin{quote}
    ``Classical knot theory of embeddings $S^1\hookrightarrow \mathbb{R}^3$ is special; in higher dimensions the behavior changes
    dramatically; higher-dimensional knot theory (e.g.\ codimension-2 sphere knots) exists.''
  \end{quote}
  But we should \emph{not} claim it is ``only in $3,4$.''
   %%%%
   \item  The following statement is vague and likely wrong or at least under-specified:
    ``chiral anomalies vanishing only in specific dimensions (e.g., $D=2,6,10$ \dots). ''
  This reads like half-memory. We need to either cut it or replace with it precise statement. 
  %%%%%
  \item Appendix Green-kernel sign convention is inconsistent with the main text. 
  In Appendix~A, we write:
  \begin{quote}
    ``Choosing $C<0$ for an attractive potential \dots''
  \end{quote}
  But in the main text (Proposition~4.2) we take $V_2(r)=k\ln r$ with $k>0$ as attractive (and correctly
  compute $F=-k/r$). For $V(r)=C\ln r$, attraction means
  \[
    F=-V'=-\frac{C}{r}
  \]
  inward, so we need $C>0$, not $C<0$.

  \medskip
  \noindent\textbf{Fix:}
  We need to change that line in the appendix to something like:
  \begin{quote}
    ``Choose the constant so that $F=-\nabla V$ is inward (attractive)\dots''
  \end{quote}
  or explicitly: ``choose $C>0$.''
  %%%%
  \item In Section~5, we state as if $\M$'s rotation group is literally $SO(D)$. However, I think the global
  isometry group of a generic manifold is not $SO(D)$. What we want is local frame rotations, i.e.\ the structure group of the
  oriented orthonormal frame bundle is $SO(D)$ (assuming a Riemannian metric).

  \medskip
  \noindent\textbf{Fix?}
  We can say:
  \begin{quote}
    ``the local orthonormal frame rotation group is $SO(D)$, which is non-abelian iff $D\ge 3$. ''
  \end{quote}
  That removes the ``global isometry group'' objection cleanly.
  %%%
  \item Make sure all the equations have the equation numbers. 
  \item At the end of every proof there is a little box, one need to remove it.
  \item Each equation must end either with comma or period. 
  \item Author names should be in alphabetical order with last name. 

  %%%%%%%%%%%%%%%%%%%%%%%%%%%%%%%%%%%%%%%%%%%%%%%%%%%%%%%%%%%%%%%%%%%%%
  %  NEW ITEMS — coloured teal so they stand out
  %%%%%%%%%%%%%%%%%%%%%%%%%%%%%%%%%%%%%%%%%%%%%%%%%%%%%%%%%%%%%%%%%%%%%

  \bigskip
  \ADD{\item \textbf{Import the published cost-kernel theorem from [Axioms paper].}\\
  The paper ``Reciprocal Convex Costs for Ratio Matching'' (Washburn \& Rahnamai Barghi, \emph{Axioms}~2026, \textbf{15}(2), 90) is now published.  It proves:}

  \ADD{\begin{quote}
  \emph{Under inversion symmetry, strict convexity, coercivity, normalization at~$1$, and a multiplicative d'Alembert identity, the unique admissible mismatch penalty is}
  \[
    J(x)\;=\;\cosh(a\log x)-1\;=\;\tfrac{1}{2}\!\left(x^{a}+x^{-a}\right)-1,
    \qquad x>0,
  \]
  \emph{for some $a>0$; moreover $a$ can be absorbed into the scale maps, giving the canonical choice $a=1$.}
  \end{quote}}

  \ADD{\noindent\textbf{Where to add in D3 paper:}}
  \ADD{\begin{itemize}
    \item \textbf{Introduction (1 paragraph):}  Add a bridge sentence such as:
    \begin{quote}\itshape
      ``This paper builds on the axiomatic characterization of the ratio-induced
      cost functional established in~\cite{WashburnRahnamaiBarghi2026}.
      There it was shown that the assumptions of inversion symmetry, strict convexity, coercivity,
      and a multiplicative d'Alembert compatibility identity uniquely force
      $J(x)=\frac{1}{2}(x^{a}+x^{-a})-1$.
      We take this result as given and focus on the downstream
      topological and dimensional consequences.''
    \end{quote}
    \item \textbf{Preliminaries / Section~2:}  State the result as an imported
    Proposition (or Assumption), e.g.:
    \begin{quote}\itshape
      \textbf{Proposition~2.X} (Washburn--Rahnamai Barghi~\cite{WashburnRahnamaiBarghi2026}).
      Let $J:(0,\infty)\to[0,\infty)$ satisfy (i)~$J(x)=J(1/x)$, (ii)~strict convexity,
      (iii)~$J(1)=0$, (iv)~coercivity, and (v)~the multiplicative d'Alembert identity
      $\bigl(1+J(xy)\bigr)+\bigl(1+J(x/y)\bigr)=2\bigl(1+J(x)\bigr)\bigl(1+J(y)\bigr)$.
      Then there exists $a>0$ such that $J(x)=\cosh(a\log x)-1$.  The parameter~$a$ is
      absorbed by rescaling $\iota_{S},\iota_{O}\mapsto\iota_{S}^{a},\iota_{O}^{a}$, yielding
      $J(x)=\frac{1}{2}(x+x^{-1})-1$ without loss.
    \end{quote}
    \item \textbf{Notation alignment:}  Reuse the Axioms-paper notation $\iota_{S},\iota_{O},J$ exactly,
    so the two publications form a visible chain.
    \item \textbf{Scope sentence (Discussion / Section~1):}  
    \begin{quote}\itshape
      ``The novelty of the present work lies in the geometric and topological
      consequences of the cost kernel --- specifically the forcing of $D=3$
      spatial dimensions via linking constraints --- rather than in the derivation
      of~$J$ itself, which is established in~\cite{WashburnRahnamaiBarghi2026}.''
    \end{quote}
  \end{itemize}}

  \ADD{\item \textbf{Add the argmin / geometric-mean boundary result from the Axioms paper.}\\
  The Axioms paper also proves that for a finite dictionary $\{o_1,\dots,o_n\}$ with ordered scales
  $y_1<\cdots<y_n$, the decision boundary between preferring $o_i$ and $o_{i+1}$ is the
  \emph{geometric mean} $\sqrt{y_i\,y_{i+1}}$.  If the D3~paper uses any discrete selection or
  ``best-matching'' argument on scale sets, we can directly cite this as an already-published lemma
  instead of reproving it inline.  Suggested insertion in the Preliminaries:}
  \ADD{\begin{quote}\itshape
    \textbf{Corollary~2.Y} (Geometric-mean boundaries; \cite{WashburnRahnamaiBarghi2026}, Proposition~4.X).
    For the canonical cost $J(x)=\frac{1}{2}(x+x^{-1})-1$ and an ordered dictionary
    $y_1<y_2<\cdots<y_n$ in $\R_{>0}$, the argmin mapping satisfies
    $\mathrm{Mean}(s)=\{o_i\}$ for $\sqrt{y_{i-1}y_i}<\iota_S(s)<\sqrt{y_i\,y_{i+1}}$.
    In particular, decision boundaries are geometric means.
  \end{quote}}

  \ADD{\item \textbf{Add the compositionality / product-model result from the Axioms paper.}\\
  The Axioms paper establishes exact compositionality: for product models
  $S=S_1\times S_2$, $O=O_1\times O_2$ with product scales, the meaning set factors as
  $\mathrm{Mean}(s_1,s_2)=\mathrm{Mean}(s_1)\times\mathrm{Mean}(s_2)$.
  If the D3~paper invokes any product-structure or factorization argument (e.g.\ for
  composite configurations), this can be cited directly.  Suggested sentence:}
  \ADD{\begin{quote}\itshape
    ``By the compositionality theorem of~\cite{WashburnRahnamaiBarghi2026},
    the argmin rule factors exactly over independent components,
    so the analysis extends to product models without additional assumptions.''
  \end{quote}}

  \ADD{\item \textbf{Add the BibTeX entry.}  Insert in the bibliography:}
  \ADD{\begin{quote}\ttfamily\small
    @article\{WashburnRahnamaiBarghi2026,\\
    \quad author = \{Washburn, Jonathan and Rahnamai Barghi, Amir\},\\
    \quad title  = \{Reciprocal Convex Costs for Ratio Matching:\\
    \quad\quad\quad\quad\quad  Axiomatic Characterization\},\\
    \quad journal = \{Axioms\},\\
    \quad year    = \{2026\},\\
    \quad volume  = \{15\},\\
    \quad number  = \{2\},\\
    \quad pages   = \{90\},\\
    \quad doi     = \{10.3390/axioms15020090\}\\
    \}
  \end{quote}}

\end{enumerate}

\end{document}
