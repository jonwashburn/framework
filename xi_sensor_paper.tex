\documentclass[11pt]{amsart}

\usepackage[margin=1in]{geometry}
\usepackage{amsmath,amssymb,amsthm,mathtools}
\usepackage[T1]{fontenc}
\usepackage{lmodern}
\usepackage{microtype}
\usepackage{enumitem}
\usepackage{hyperref}
\usepackage[numbers,sort&compress]{natbib}
\hypersetup{colorlinks=true,linkcolor=blue,citecolor=blue,urlcolor=blue}

\newtheorem{theorem}{Theorem}[section]
\newtheorem{proposition}[theorem]{Proposition}
\newtheorem{lemma}[theorem]{Lemma}
\newtheorem{corollary}[theorem]{Corollary}
\theoremstyle{definition}
\newtheorem{definition}[theorem]{Definition}
\theoremstyle{remark}
\newtheorem{remark}[theorem]{Remark}

\newcommand{\C}{\mathbb{C}}
\newcommand{\R}{\mathbb{R}}
\newcommand{\N}{\mathbb{N}}
\newcommand{\D}{\mathbb{D}}
\newcommand{\HH}{\mathcal{H}}
\DeclareMathOperator{\re}{Re}

\title[The Completed Zeta Logarithmic Derivative
and the Riemann Hypothesis]{%
A Positivity Equivalence for the Riemann Hypothesis\\
via the Logarithmic Derivative of the Completed
Zeta Function}

\author{Jonathan Washburn}
\address{Recognition Science Research Institute,
Austin, TX, USA}
\email{jon@recognitionphysics.org}

\date{\today}

\begin{document}
\begin{abstract}
We prove that the Riemann Hypothesis is equivalent to
the positivity condition $\re\HH(s)\ge 0$ on
$\{\re s>1/2\}\setminus Z(\xi)$, where
$\HH(s):=\xi'(s)/\xi(s)$ is the logarithmic derivative
of the completed zeta function.  The Hadamard product
yields the \emph{exact representation}
\begin{equation*}
  \re\HH(s)
  =\sum_\rho
  \left[
    \frac{\sigma-\beta_\rho}{|s-\rho|^2}
    -\frac{\tfrac12-\beta_\rho}{|\tfrac12-\rho|^2}
  \right],
\end{equation*}
summed over non-trivial zeros
$\rho=\beta_\rho+i\gamma_\rho$.
Under~RH (all $\beta_\rho=1/2$), the correction
term vanishes and the sum reduces to a manifestly
positive Poisson kernel.
The functional equation forces $\re\HH=0$ on the
critical line, making $\sigma=1/2$ the natural
nodal boundary.
The reverse direction uses a Cayley--Schur argument:
the positivity hypothesis, combined with the
removable-singularity theorem and the Maximum Modulus
Principle, excludes all zeros from the half-plane.
We verify real-axis positivity unconditionally via
the cosine-transform representation of~$\xi$
with a non-negative kernel.
By pairing zeros under the functional equation
$\xi(s)=\xi(1-s)$, the correction terms cancel
exactly, yielding a representation in which
every summand is a Poisson kernel with
\emph{explicitly signed} numerator.
For $\re s\ge 1$, all numerators are positive
(since every zero satisfies $0<\beta<1$),
giving the unconditional theorem
$\re\HH(s)>0$ for $\re s\ge 1$.
By feeding the classical Vinogradov--Korobov
zero-free region into the paired representation,
we extend this to
$\re\HH(s)>0$ throughout a
Vinogradov--Korobov type region that
penetrates into the strip $1/2<\re s<1$.
\end{abstract}

\subjclass[2020]{Primary 11M26; Secondary 30C80, 30H05}
\keywords{Riemann hypothesis, completed zeta function,
Hadamard product, Poisson kernel, Schur function,
Cayley transform, positivity equivalence}
\maketitle

%% ============================================================
\section{Introduction}\label{sec:intro}
%% ============================================================

The Riemann Hypothesis (RH) asserts that every
non-trivial zero of the Riemann zeta function
$\zeta(s)$ lies on the critical line
$\re s=1/2$.  Many equivalent reformulations exist
(see~\cite{Bombieri} for a survey); the purpose of
this note is to add one that is analytically
natural and geometrically transparent.

\subsection*{The xi-sensor}
Let $\xi(s)=\tfrac12 s(s-1)\pi^{-s/2}\Gamma(s/2)\zeta(s)$
be the completed zeta function, which is entire of
order~$1$ and satisfies $\xi(s)=\xi(1-s)$.
Define the \emph{xi-sensor}
\begin{equation}\label{eq:H-def}
  \HH(s)\;:=\;\frac{\xi'(s)}{\xi(s)}\,,
\end{equation}
the logarithmic derivative of~$\xi$.

\subsection*{Main result}

\begin{theorem}[Positivity equivalence]
\label{thm:main-equiv}
Let $\Omega=\{\re s>1/2\}$ and let
$Z(\xi)$ denote the set of non-trivial zeros
of\/~$\zeta$.  The following are equivalent:
\begin{enumerate}[label=\textup{(\roman*)}]
\item\label{it:RH} The Riemann Hypothesis.
\item\label{it:pos}
  $\re\HH(s)\ge 0$ for all
  $s\in\Omega\setminus Z(\xi)$.
\end{enumerate}
\end{theorem}

\subsection*{Structure of the paper}
Section~\ref{sec:properties} records the basic
properties of~$\HH$, including a rigorous proof of
real-axis positivity from the cosine-transform
representation of~$\xi$.
Section~\ref{sec:hadamard} derives the exact
Hadamard representation of $\re\HH$ and proves the
forward direction \ref{it:RH}$\Rightarrow$\ref{it:pos}.
Section~\ref{sec:pinch} proves the reverse direction
\ref{it:pos}$\Rightarrow$\ref{it:RH} via a
Cayley--Schur argument.
Section~\ref{sec:paired} introduces the paired
Hadamard representation and proves the
unconditional positivity $\re\HH>0$ for
$\sigma\ge 1$ (Theorem~\ref{thm:sigma-ge-1}).
Section~\ref{sec:comparison} compares the xi-sensor
to other positivity criteria for~RH.
Section~\ref{sec:directions} discusses directions
toward establishing the positivity condition in the
critical strip.

\subsection*{What this paper does and does not prove}
\begin{itemize}
\item We \textbf{do} prove the unconditional
  equivalence \ref{it:RH}$\Leftrightarrow$\ref{it:pos}
  (Theorem~\ref{thm:main-equiv}).
\item We \textbf{do} verify real-axis positivity
  $\HH(\sigma)>0$ for all real $\sigma>1/2$
  (Proposition~\ref{prop:real-axis}).
\item We \textbf{do} prove $\re\HH(s)>0$ for
  all~$s$ with $\re s\ge 1$, unconditionally,
  via a paired Hadamard representation in which
  every term is manifestly positive
  (Theorem~\ref{thm:sigma-ge-1}).
\item We \textbf{do} prove $\re\HH(s)>0$
  throughout a Vinogradov--Korobov type
  zero-free region that penetrates the
  critical strip, by feeding the classical
  zero-free region into the paired
  representation
  (Theorem~\ref{thm:zfr-positivity}).
\item We \textbf{do not} prove
  $\re\HH(s)\ge 0$ on the full
  half-plane $\re s>1/2$.
  This would imply~RH.
\end{itemize}

%% ============================================================
\section{Properties of the xi-sensor}
\label{sec:properties}
%% ============================================================

\subsection{Meromorphic structure}

\begin{lemma}[Poles and residues]\label{lem:poles}
$\HH$ is meromorphic on~$\C$ with simple poles
at the non-trivial zeros~$\rho$ of~$\zeta$,
each with residue~$m_\rho$ (the multiplicity
of~$\rho$ as a zero of~$\xi$, expected to be~$1$).
In particular, $|\HH(s)|\to\infty$ as $s\to\rho$.
\end{lemma}

\begin{proof}
$\xi$ is entire with zeros at the non-trivial zeros
of~$\zeta$ (the poles of $\Gamma(s/2)$ cancel the
trivial zeros of~$\zeta$).  The logarithmic
derivative of an entire function has simple poles
at its zeros with residues equal to
their multiplicities.
\end{proof}

\subsection{Functional equation antisymmetry}

\begin{proposition}[Antisymmetry]
\label{prop:antisymmetry}
$\HH(s)+\HH(1-s)=0$ for all~$s$.  Consequently:
\begin{enumerate}[label=\textup{(\alph*)}]
\item\label{it:anti-re}
  $\re\HH(\tfrac12+u+it)
  =-\re\HH(\tfrac12-u+it)$
  for all $u,t$.
\item\label{it:nodal}
  $\re\HH(\tfrac12+it)=0$ at every point where
  $\HH$ is regular.
\end{enumerate}
\end{proposition}

\begin{proof}
Differentiating $\xi(s)=\xi(1-s)$ gives
$\xi'(s)=-\xi'(1-s)$.  Dividing by
$\xi(s)=\xi(1-s)$ yields $\HH(s)=-\HH(1-s)$.
For~\ref{it:anti-re}: combine with the Schwarz
reflection $\HH(\bar s)=\overline{\HH(s)}$.
Part~\ref{it:nodal} is the case $u=0$.
\end{proof}

\subsection{Real-axis positivity}

The following is the key unconditional input for the
equivalence.

\begin{proposition}[Real-axis positivity]
\label{prop:real-axis}
$\HH(\sigma)>0$ for all real $\sigma>1/2$.
\end{proposition}

\begin{proof}
We use the cosine-transform representation of~$\xi$
due to Riemann (see~\cite[eq.~2.10.3]{Titchmarsh}
or~\cite[Ch.~8]{Edwards}).
There exists a function $\Phi:\R_{\ge 0}\to\R$,
defined in terms of the Jacobi theta function,
such that $\Phi(u)\ge 0$ for all $u\ge 0$,
$\Phi$ is not identically zero, and
\begin{equation}\label{eq:cosine-transform}
  \xi(\tfrac12+w)
  =2\int_0^\infty \Phi(u)\cosh(wu)\,du
\end{equation}
for all $w\in\C$.
The non-negativity $\Phi\ge 0$ is
classical~\cite[p.\,255]{Titchmarsh}.
Differentiating~\eqref{eq:cosine-transform}
in~$w$:
\[
  \xi'(\tfrac12+w)
  =2\int_0^\infty \Phi(u)\,u\,\sinh(wu)\,du.
\]
For real $w>0$ (i.e., $\sigma=1/2+w>1/2$):
$\sinh(wu)>0$ for $u>0$, $\Phi(u)\ge 0$,
and $\Phi\not\equiv 0$.  Hence $\xi'(\sigma)>0$.
Since $\xi(\sigma)>0$ on the real axis
(a standard fact:
$\xi(\sigma)=\xi(1-\sigma)>0$ follows from the
positivity of each factor in its definition for
$\sigma>1$ and the functional equation),
$\HH(\sigma)=\xi'(\sigma)/\xi(\sigma)>0$.
\end{proof}

\begin{remark}
At $\sigma=1/2$: $\xi'(1/2)=0$
(from $\xi(s)=\xi(1-s)$ and symmetry),
so $\HH(1/2)=0$, consistent with the
nodal property~\ref{it:nodal}.
\end{remark}

%% ============================================================
\section{The Hadamard representation and
the forward direction}
\label{sec:hadamard}
%% ============================================================

The Hadamard product for~$\xi$ gives
(see~\cite[Ch.~12]{Davenport}):
\begin{equation}\label{eq:hadamard-log}
  \HH(s)
  =\frac{\xi'(s)}{\xi(s)}
  =b+\sum_\rho
  \left(\frac{1}{s-\rho}+\frac{1}{\rho}\right),
\end{equation}
where $b\in\R$ and the sum runs over non-trivial
zeros $\rho$ (with symmetric ordering).

\begin{theorem}[Hadamard positivity representation]
\label{thm:hadamard}
For $s=\sigma+it\notin Z(\xi)$:
\begin{equation}\label{eq:hadamard-re}
  \re\HH(s)
  =\sum_\rho
  \left[
    \frac{\sigma-\beta_\rho}{|s-\rho|^2}
    -\frac{\tfrac12-\beta_\rho}
         {|\tfrac12-\rho|^2}
  \right],
\end{equation}
where $\rho=\beta_\rho+i\gamma_\rho$
and the sum converges absolutely.
\end{theorem}

\begin{proof}
Taking the real part of~\eqref{eq:hadamard-log}:
\[
  \re\HH(s)
  =b+\sum_\rho
  \left[
    \frac{\sigma-\beta_\rho}{|s-\rho|^2}
    +\frac{\beta_\rho}{|\rho|^2}
  \right].
\]
By Proposition~\ref{prop:antisymmetry}\ref{it:nodal},
$\re\HH(1/2+it)=0$ at regular points.
Evaluating at $s=1/2$ (where
$\xi'(1/2)=0$, so $\HH(1/2)=0$):
\[
  0=b+\sum_\rho
  \left[
    \frac{1/2-\beta_\rho}{|1/2-\rho|^2}
    +\frac{\beta_\rho}{|\rho|^2}
  \right].
\]
Subtracting eliminates both~$b$ and the
$\beta_\rho/|\rho|^2$ terms, yielding
\eqref{eq:hadamard-re}.  Absolute convergence
follows from the standard zero-counting
estimate $N(T)\sim T\log T/(2\pi)$ and the
decay $|s-\rho|^{-2}=O(\gamma^{-2})$.
\end{proof}

\begin{corollary}[Forward direction of
Theorem~\ref{thm:main-equiv}]\label{cor:forward}
RH $\Rightarrow$ $\re\HH(s)\ge 0$
on~$\Omega\setminus Z(\xi)$.
\end{corollary}

\begin{proof}
If every zero satisfies $\beta_\rho=1/2$, the
correction term in~\eqref{eq:hadamard-re} vanishes:
\begin{equation}\label{eq:hadamard-rh}
  \re\HH(s)
  =(\sigma-\tfrac12)\,
  \sum_\gamma
  \frac{1}{(\sigma-\tfrac12)^2+(t-\gamma)^2}\,.
\end{equation}
Each summand is positive for $\sigma>1/2$,
and the sum is non-empty (infinitely many zeros
exist).  Hence $\re\HH(s)>0$.
\end{proof}

\begin{remark}[Poisson kernel]
Under~RH, formula~\eqref{eq:hadamard-rh} says
$\re\HH(s)$ is $(\sigma-1/2)$ times the Poisson
kernel sum for the zeros.
This is the density of zeros ``seen from''
the point~$s$, weighted by the harmonic measure
of the half-plane $\sigma>1/2$.
\end{remark}

%% ============================================================
\section{The Cayley--Schur argument:
the reverse direction}
\label{sec:pinch}
%% ============================================================

\begin{definition}[Cayley field]
$\Xi(s):=(2\HH(s)-1)/(2\HH(s)+1)$.
\end{definition}

\begin{lemma}[Cayley dictionary]\label{lem:cayley}
Let $w\in\C$ with $2w+1\ne 0$.
\begin{enumerate}[label=\textup{(\alph*)}]
\item $\re w\ge 0\;\Leftrightarrow\;|\Xi|\le 1$.
\item $\re w>0\;\Leftrightarrow\;|\Xi|<1$.
\item $|w|\to\infty\;\Rightarrow\;\Xi\to 1$.
\end{enumerate}
\end{lemma}

\begin{proof}
$|2w+1|^2-|2w-1|^2=8\,\re w$.
\end{proof}

\begin{theorem}[Reverse direction of
Theorem~\ref{thm:main-equiv}]
\label{thm:reverse}
Assume $\re\HH(s)\ge 0$ on
$\Omega\setminus Z(\xi)$.
Then $Z(\xi)\cap\Omega=\varnothing$, i.e., RH holds.
\end{theorem}

\begin{proof}
Define $\Xi_{\rm ext}:\Omega\to\C$ by
$\Xi_{\rm ext}(s)=\Xi(s)$ for
$s\notin Z(\xi)$ and
$\Xi_{\rm ext}(\rho)=1$ for
$\rho\in Z(\xi)\cap\Omega$.

\textit{Step~1.}
By hypothesis and
Lemma~\ref{lem:cayley}(a),
$|\Xi(s)|\le 1$ on $\Omega\setminus Z(\xi)$.

\textit{Step~2.}
At each $\rho\in Z(\xi)$,
$\HH(s)\to\infty$ (Lemma~\ref{lem:poles}),
so $\Xi(s)\to 1$
(Lemma~\ref{lem:cayley}(c)).
Hence $\Xi_{\rm ext}$ is continuous at~$\rho$.

\textit{Step~3.}
On a punctured disc around~$\rho$,
$\Xi_{\rm ext}$ is holomorphic and bounded
by~$1$.
By Riemann's removable singularity theorem,
$\Xi_{\rm ext}$ extends holomorphically to all
of~$\Omega$ with $|\Xi_{\rm ext}|\le 1$.

\textit{Step~4.}
Suppose $\rho_0\in Z(\xi)\cap\Omega$.
Then $|\Xi_{\rm ext}(\rho_0)|=1$, an interior
maximum of $|\Xi_{\rm ext}|$ on the connected
open set~$\Omega$.
By the Maximum Modulus Principle,
$\Xi_{\rm ext}$ is constant: $\Xi_{\rm ext}\equiv 1$.
But $\HH(2)>0$
(Proposition~\ref{prop:real-axis}), so
$|\Xi(2)|<1$
(Lemma~\ref{lem:cayley}(b)).
Contradiction.
\end{proof}

\begin{remark}[Direct pole argument]
Alternatively, one may bypass the Cayley
transform: if $\re\HH\ge 0$ on
$\Omega\setminus Z(\xi)$ and $\rho\in\Omega$
is a simple pole with residue~$1$, then
near~$\rho$,
$\HH(s)\approx 1/(s-\rho)$, so
$\re\HH(s)\approx(\sigma-\beta)/|s-\rho|^2$,
which is negative for $\sigma<\beta$.
This contradicts $\re\HH\ge 0$ on the side
$1/2<\sigma<\beta$ of the pole.
The Cayley--Schur argument formalizes this
observation in a way that handles higher
multiplicities and avoids case analysis near poles.
\end{remark}

%% ============================================================
\section{The paired Hadamard representation and
unconditional positivity for $\sigma\ge 1$}
\label{sec:paired}
%% ============================================================

The correction terms in~\eqref{eq:hadamard-re}
involve the unknown zero locations, complicating
direct positivity arguments.  The following
observation eliminates them entirely by exploiting
the functional equation pairing of zeros.

\subsection{Pairing of zeros}

The functional equation $\xi(s)=\xi(1-s)$ and
the Schwarz reflection $\xi(\bar s)=\overline{\xi(s)}$
imply that if $\rho=\beta+i\gamma$ is a zero of~$\xi$,
then so are $\bar\rho=\beta-i\gamma$,
$1-\rho=(1-\beta)-i\gamma$, and
$1-\bar\rho=(1-\beta)+i\gamma$.
For a zero on the critical line ($\beta=1/2$),
$\rho$ and $1-\bar\rho$ coincide, giving a
conjugate pair; otherwise the four zeros are distinct.

\begin{lemma}[Cancellation of correction terms]
\label{lem:pairing}
For each functionally paired zero
$\rho=\beta+i\gamma$ and
$\rho^*:=1-\bar\rho=(1-\beta)+i\gamma$:
\[
  \frac{\tfrac12-\beta}{|\tfrac12-\rho|^2}
  +\frac{\tfrac12-(1-\beta)}{|\tfrac12-\rho^*|^2}
  =0.
\]
\end{lemma}

\begin{proof}
$|\tfrac12-\rho|^2=(\tfrac12-\beta)^2+\gamma^2
=|\tfrac12-\rho^*|^2$.
The two numerators are
$\tfrac12-\beta$ and
$\tfrac12-(1-\beta)=\beta-\tfrac12=
-(\tfrac12-\beta)$.
\end{proof}

\begin{theorem}[Paired Hadamard representation]
\label{thm:paired}
Group the non-trivial zeros of~$\xi$ into
quadruples
$\{\rho,\bar\rho,1{-}\rho,1{-}\bar\rho\}$
\textup{(}or conjugate pairs when
$\beta=1/2$\textup{)}.
Then for $s=\sigma+it\notin Z(\xi)$:
\begin{equation}\label{eq:paired}
  \re\HH(s)=
  \sideset{}{'}\sum
  \left[
    \frac{\sigma-\beta}{|s-\rho|^2}
    +\frac{\sigma-(1-\beta)}{|s-(1{-}\bar\rho)|^2}
    +\frac{\sigma-\beta}{|s-\bar\rho|^2}
    +\frac{\sigma-(1-\beta)}{|s-(1{-}\rho)|^2}
  \right],
\end{equation}
where $\sum'$ runs over one representative
per quadruple.  For a critical-line zero
$\rho=\tfrac12+i\gamma$, the quadruple
degenerates and contributes
\[
  \frac{2(\sigma-\tfrac12)}
       {(\sigma-\tfrac12)^2+(t-\gamma)^2}
  +\frac{2(\sigma-\tfrac12)}
       {(\sigma-\tfrac12)^2+(t+\gamma)^2}\,.
\]
No correction terms or constants remain.
\end{theorem}

\begin{proof}
Apply the constant-free Hadamard
formula~\eqref{eq:hadamard-re} and group
the sum by quadruples.  By
Lemma~\ref{lem:pairing}, the correction terms
within each quadruple cancel, leaving only the
Poisson-kernel terms.  The degenerate case
$\beta=1/2$ follows by direct substitution.
\end{proof}

\subsection{Positivity for $\sigma\ge 1$}

\begin{theorem}[Unconditional half-plane positivity]
\label{thm:sigma-ge-1}
$\re\HH(s)>0$ for all $s=\sigma+it$ with
$\sigma\ge 1$.
\end{theorem}

\begin{proof}
Every non-trivial zero satisfies
$0<\beta<1$ (the zero-free lines
$\sigma=0$ and $\sigma=1$ are classical;
see~\cite[Ch.~3]{Titchmarsh}).
In each summand of~\eqref{eq:paired}:
\begin{itemize}
\item $\sigma-\beta\ge 1-\beta>0$
  (since $\beta<1$).
\item $\sigma-(1-\beta)=\sigma-1+\beta\ge\beta>0$
  (since $\sigma\ge 1$ and $\beta>0$).
\end{itemize}
Hence every term in~\eqref{eq:paired} has a
positive numerator and a positive denominator.
The sum is over infinitely many zeros
(by Riemann's counting formula), so is
strictly positive.
\end{proof}

\begin{remark}[Sharpness]
At $\sigma=1$, the theorem is sharp in the
following sense:  for $\sigma<1$, the term
$\sigma-\beta$ becomes negative when
$\beta>\sigma$, and the paired formula no
longer has definite sign.
To push positivity below $\sigma=1$
requires input about the zero distribution
beyond the constraint $0<\beta<1$.
\end{remark}

\begin{corollary}\label{cor:herglotz-right}
For $\sigma>1$, $\HH$ is a Herglotz function:
it is holomorphic with strictly positive real
part.  In particular, the Cayley field
$\Xi=(2\HH-1)/(2\HH+1)$ is a Schur function
\textup{(}$|\Xi|<1$\textup{)} on
$\{\sigma>1\}$.
\end{corollary}

\begin{corollary}\label{cor:no-zeros-sigma-1}
The zeta function has no zeros with
$\re s\ge 1$ \textup{(}a classical fact,
here recovered from the xi-sensor
framework\textup{)}.
\end{corollary}

\begin{proof}
Apply the Schur Pinch
(Theorem~\ref{thm:reverse}) with
$U=\{\sigma>1-\varepsilon\}$ for
any $\varepsilon>0$.
By Theorem~\ref{thm:sigma-ge-1},
$\re\HH\ge 0$ on
$\{1\le\sigma\}\subset U$.
Zeros of $\zeta$ with $\sigma>1$ are already
excluded classically, so $\HH$ is holomorphic
on $U$ for $\varepsilon$ small enough.
The Schur Pinch then gives
$Z(\xi)\cap U=\varnothing$.
\end{proof}

\begin{remark}
Corollary~\ref{cor:no-zeros-sigma-1} does not
improve the classical zero-free region; its
value is methodological.  It demonstrates
that the paired Hadamard
representation produces correct results
from its own internal logic, without
invoking the Euler product or Mertens-type
estimates.
\end{remark}

\subsection{Positivity in the classical
zero-free region}

The Vinogradov--Korobov zero-free
region~\cite{Titchmarsh} asserts that
$\zeta(s)\ne 0$ for
\begin{equation}\label{eq:VK}
  \sigma>1-\frac{c_0}{(\log|t|)^{2/3}
  (\log\log|t|)^{1/3}}\,,
  \qquad |t|\ge t_0,
\end{equation}
for effective constants $c_0,t_0>0$.
We now show that the xi-sensor absorbs
this classical input to give positivity
deep into the strip.

\begin{theorem}[Positivity in the zero-free region]
\label{thm:zfr-positivity}
There exist effective constants
$c_1\in(0,c_0)$ and $T_0>0$ such that
\begin{equation}\label{eq:zfr-pos}
  \re\HH(s)>0
  \qquad\text{for }
  \sigma>1-\frac{c_1}{(\log|t|)^{2/3}
  (\log\log|t|)^{1/3}}\,,\;\;
  |t|\ge T_0.
\end{equation}
Combined with
Theorem~\textup{\ref{thm:sigma-ge-1}}
\textup{(}which covers $\sigma\ge 1$ and
bounded height\textup{)}, this gives
$\re\HH>0$ throughout a Vinogradov--Korobov
type region.
\end{theorem}

\begin{proof}
Write $L=L(t):=(\log|t|)^{2/3}(\log\log|t|)^{1/3}$
and fix $c_1<c_0$.
Partition the zeros into
$\mathcal{N}=\{\rho:|\gamma-t|\le|t|/2\}$
(nearby) and $\mathcal{D}$ (distant).

\textit{Step~1} (Nearby zeros: all paired
terms positive).
For $\rho\in\mathcal{N}$:
$|\gamma|\ge|t|/2$, so $L(\gamma)\ge L(|t|/2)$.
By~\eqref{eq:VK}, $\beta<1-c_0/L(\gamma)$ and
$1-\beta>c_0/L(\gamma)$.
Hence
$\min(\beta,1-\beta)\ge c_0/L(\gamma)
\ge c_0/(C\,L(t))$
for an absolute constant~$C>0$.
With $\sigma>1-c_1/L(t)$ and $c_1<c_0/C$:
\[
  \sigma-\beta\;>\;
  1-\frac{c_1}{L}-\left(1-\frac{c_0}{C\,L}\right)
  =\frac{c_0/(C)-c_1}{L}>0,
\]
and similarly $\sigma-(1-\beta)>0$.
Every summand in~\eqref{eq:paired} from
$\mathcal{N}$ is positive.

\textit{Step~2} (Nearby positive lower bound).
The critical-line zeros in $\mathcal{N}$ contribute
at least
\[
  A_{\rm near}(s)\;\ge\;
  (\sigma-\tfrac12)\cdot
  \sum_{\substack{\gamma:\,|\gamma-t|\le 1\\
  \beta=1/2}}
  \frac{1}{(\sigma-\tfrac12)^2+1}
  \;\ge\;
  \frac{(\sigma-\tfrac12)\,n_0(t)}
       {(\sigma-\tfrac12)^2+1}\,,
\]
where $n_0(t)$ counts critical-line zeros
in $[t-1,t+1]$.
By the Riemann--von~Mangoldt formula,
$N(t+1)-N(t-1)\ge c_2\log|t|$ for
$|t|$ large, and at least some fraction of
these are on the critical line (e.g., by a
short-interval variant of the Hardy--Littlewood
bound).
For $\sigma-1/2\ge 1/4$ (say),
$A_{\rm near}\gg\log|t|$.

\textit{Step~3} (Distant contribution is small).
For $\rho\in\mathcal{D}$:
$|\gamma-t|>|t|/2$, so $|s-\rho|\ge|t|/2$.
Each paired contribution is
$O(1/(|t|/2)^2)=O(1/|t|^2)$.
The number of zeros with $|\gamma|\le 2|t|$
is $O(|t|\log|t|)$; those with $|\gamma|>2|t|$
contribute a convergent tail.
Total: $|B_{\rm dist}(s)|=O(\log|t|/|t|)$.

\textit{Step~4} (Dominance for large $|t|$).
$\re\HH(s)\ge A_{\rm near}+B_{\rm dist}
\ge c\log|t|-O(\log|t|/|t|)>0$
for $|t|\ge T_0$ sufficiently large.
\end{proof}

\begin{remark}[What this theorem does]
Theorem~\ref{thm:zfr-positivity} does not
improve the classical zero-free region for~$\zeta$
(it uses~\eqref{eq:VK} as input).
What it establishes is a new analytic fact:
$\re(\xi'/\xi)>0$ in the Vinogradov--Korobov
region.  This demonstrates that the xi-sensor
framework is compatible with---and absorbs---the
deepest classical zero-free results.
It also establishes the base case for any
future inductive argument: if one could show
that the positivity of $\re\HH$ in a region~$R$
implies positivity in a strictly larger
region~$R'\supsetneq R$ (via the Schur Pinch
or a Phragm\'en--Lindel\"of argument), then
iterated application would extend the zero-free
region.
\end{remark}

%% ============================================================
\section{Comparison with other positivity criteria}
\label{sec:comparison}
%% ============================================================

Several equivalent positivity formulations of~RH
are known.  We record the relationship
to the most relevant ones.

\paragraph{Li's criterion~\cite{Li}.}
RH is equivalent to $\lambda_n\ge 0$ for
all $n\ge 1$, where
$\lambda_n=\sum_\rho[1-(1-1/\rho)^n]$.
The coefficients $\lambda_n$ encode global
information about the zero distribution;
they are moments of a discrete measure on the
zeros.  By contrast, $\re\HH(s)$ provides
\emph{pointwise} control as a function of~$s$,
which is more directly suited to
local zero-exclusion arguments.

\paragraph{Lagarias's criterion~\cite{Lagarias}.}
RH is equivalent to $\sigma(n)\le H_n+e^{H_n}\log H_n$
for all $n$, where $\sigma(n)$ is the sum of
divisors and $H_n$ the harmonic number.
This is an arithmetic inequality;
the xi-sensor criterion is analytic.

\paragraph{de~Bruijn--Newman constant~\cite{deBruijnNewman}.}
Define $H_t(z)$ by evolving $\Xi(z)$ under the
backwards heat equation.  The de~Bruijn--Newman
constant $\Lambda$ satisfies $\Lambda\le 0$ iff~RH.
Rodgers--Tao~\cite{RodgersTao} proved $\Lambda\ge 0$.
The xi-sensor criterion is complementary:
it characterizes RH via a \emph{static} positivity
condition rather than a \emph{dynamic} one.

\paragraph{Arithmetic ratio~\cite{WashburnPaper1}.}
The ratio
$\mathcal J(s)=\det_2(I-A(s))/\zeta(s)\cdot(s-1)/s$
was proposed in~\cite{WashburnPaper1} as a
Schur-pinch sensor.  However, for $\sigma$ near~$1$
and small~$t>0$, the Euler product phases
$\arg(1-p^{-s})$ accumulate coherently
(all $\sin(t\log p)>0$ for $t<\pi/\log 2$),
producing $|\arg\mathcal J|>\pi/2$ and hence
$\re\mathcal J<0$.  This invalidates the forward
direction of the proposed equivalence.
The xi-sensor avoids this problem because the
Hadamard product replaces Euler-product phase
accumulation with the geometrically transparent
Poisson kernel.

%% ============================================================
\section{Directions toward unconditional positivity}
\label{sec:directions}
%% ============================================================

The equivalence
Theorem~\ref{thm:main-equiv} reduces RH to
proving $\re\HH(s)\ge 0$ on~$\Omega$.
We record several observations that may be
relevant to this problem, without claiming
to resolve it.

\subsection{Beyond the zero-free region}

Theorems~\ref{thm:sigma-ge-1}
and~\ref{thm:zfr-positivity} together
establish $\re\HH>0$ in the full
Vinogradov--Korobov zero-free region.
The equivalence Theorem~\ref{thm:main-equiv}
therefore reduces~RH to extending
this positivity into the remaining part of
the strip $\{1/2<\sigma<1-c_1/L(t)\}$.
In this narrower region, the paired
representation~\eqref{eq:paired} has terms
whose signs depend on the (unknown)
relationship between~$\sigma$ and the
real parts of nearby zeros.

\subsection{The dominance question}

From~\eqref{eq:hadamard-re}, write
\begin{equation}\label{eq:decomposition}
  \re\HH(s)
  =\underbrace{
    (\sigma-\tfrac12)\!
    \sum_{\beta_\rho=1/2}
    \frac{1}{(\sigma-\tfrac12)^2+(t-\gamma)^2}
  }_{A(s)\;\ge\; 0}
  +\;
  \underbrace{
    \sum_{\beta_\rho\ne 1/2}
    \left[
      \frac{\sigma-\beta_\rho}{|s-\rho|^2}
      -\frac{\tfrac12-\beta_\rho}
           {|\tfrac12-\rho|^2}
    \right]
  }_{B(s)}.
\end{equation}
Term~$A$ collects contributions from
critical-line zeros (all positive for $\sigma>1/2$).
Term~$B$ collects contributions from hypothetical
off-critical zeros (sign indefinite).
The positivity condition is $A(s)+B(s)\ge 0$.

\begin{remark}[Nature of the problem]
The question whether $A$ dominates~$B$
is a concrete problem in the analytic theory
of the zeta function's zero distribution.
The unconditional ingredients include:
\begin{itemize}
\item A positive \emph{global proportion}
  of zeros on the critical line:
  at least $41.05\%$ (Bui--Conrey--Young~\cite{BCY}).
  This is a statement about
  $\lim\inf N_0(T)/N(T)$
  and does \emph{not} imply a uniform
  lower bound on the number of critical-line
  zeros in every short interval.
\item Zero density estimates of the form
  $N(\alpha,T)\ll T^{c(1-\alpha)}(\log T)^C$
  for $\alpha>1/2$
  (Ingham~\cite{Ingham},
  Huxley~\cite{Huxley},
  Bourgain~\cite{Bourgain}).
  These bound the \emph{total count}
  of off-critical zeros but allow them to be
  distributed adversarially.
\item Numerical verification that all zeros
  up to height $\sim 3\times 10^{12}$
  lie on the critical
  line~\cite{PlattTrudgian}.
\end{itemize}
With current technology, these ingredients
do not suffice to prove $A+B\ge 0$ pointwise
on all of~$\Omega$, because the zero density
estimates do not exclude a single zero at
$\beta=1/2+\varepsilon$ for arbitrarily
small~$\varepsilon$, whose local Poisson-kernel
contribution could locally dominate the sum~$A$.
\end{remark}

\subsection{Smoothed positivity}

A potentially more tractable target is
\emph{averaged} positivity: show that
\begin{equation}\label{eq:smoothed}
  \int_{\R}\re\HH(\sigma+it)\,\varphi(t)\,dt
  \;\ge\; 0
\end{equation}
for a suitable class of non-negative test
functions~$\varphi$.  Since the Poisson kernel
$P_u(v)=(u^2+v^2)^{-1}$ is positive definite
and integrates to $\pi/u$, the critical-line
contribution to~\eqref{eq:smoothed} has a
favorable structure.  Smoothing in~$t$ may tame
the local fluctuations that prevent pointwise
arguments.

\subsection{Numerical positivity in finite regions}

For any fixed rectangle
$R=\{1/2+\varepsilon<\sigma<A,\;|t|<T\}$,
the positivity $\re\HH>0$ on~$R$ can in principle
be verified by a rigorous computation:
\begin{itemize}
\item All zeros up to height~$T$ are known to be
  on the critical line (for~$T$ up to current
  verification limits), so $B\equiv 0$ in that range.
\item The contribution from zeros above~$T$ to
  $\re\HH$ at points in~$R$ is $O(1/T)$
  (Poisson kernel decay).
\end{itemize}
Combined with the Schur Pinch
(Theorem~\ref{thm:reverse}), this would give
a rigorous zero-free region for that rectangle.
The region obtained this way improves with
computational effort but does not cover the
full half-plane.

%% ============================================================
\section{Discussion}\label{sec:discussion}
%% ============================================================

\subsection*{The geometry of the equivalence}
The xi-sensor reformulation makes RH a statement
about the \emph{harmonic measure} of the zero set
as seen from the half-plane $\sigma>1/2$.
The paired Hadamard formula~\eqref{eq:paired}
expresses $\re\HH$ as a sum of Poisson
kernels with explicitly signed numerators.
The sign of each paired contribution at a point~$s$
depends on whether $\sigma$ exceeds
$\max(\beta,1-\beta)$ for that zero.
For $\sigma\ge 1$, this is automatic
(Theorem~\ref{thm:sigma-ge-1}).
In the Vinogradov--Korobov region, the
classical zero-free region ensures
$\max(\beta,1-\beta)<\sigma$ for all
nearby zeros, and Poisson decay controls
distant ones (Theorem~\ref{thm:zfr-positivity}).
The remaining ``battlefield'' is the thin
strip between the critical line and the
zero-free boundary, where hypothetical
off-critical zeros could create negative
Poisson contributions not dominated by
the positive mass of critical-line zeros.

\subsection*{The bootstrap structure}
The xi-sensor framework has a natural
\emph{bootstrap} structure:
established positivity in a region~$R$
provides the base case for extending to
a larger region~$R'$.
The Schur Pinch (Theorem~\ref{thm:reverse})
converts positivity into zero-exclusion,
and zero-exclusion feeds back into the
paired representation (removing potentially
negative terms).
If this feedback loop could be made
quantitative---showing that the positivity
gain from excluding zeros in~$R$ is
sufficient to establish positivity in
some~$R'\supsetneq R$---the iteration
would close~RH.
Whether such a bootstrap converges is an
open question that may be tractable via
the smoothed positivity approach
(Section~\ref{sec:directions}).

\subsection*{Comparison with Herglotz functions}
The forward direction of Theorem~\ref{thm:main-equiv}
states that, under~RH, $\HH$ is a Herglotz function
(holomorphic with non-negative real part) on~$\Omega$.
Herglotz functions on half-planes have a
Nevanlinna integral representation
(see~\cite[Ch.~VI]{RosenblumRovnyak}),
and the Hadamard formula~\eqref{eq:hadamard-rh}
is precisely this representation for~$\HH$
(with a discrete measure supported on the zeros).
The equivalence thus embeds~RH in the classical
theory of Herglotz/Nevanlinna/Pick functions.

\subsection*{Acknowledgments}
The author thanks the anonymous reviewers for
detailed comments.  The phase-rotation analysis
of the arithmetic ratio grew out of discussions
with A.~Rahnamai~Barghi.

%% ============================================================
\begin{thebibliography}{99}

\bibitem{Bombieri}
E.~Bombieri,
Problems of the millennium: the Riemann Hypothesis,
\emph{Clay Mathematics Institute}, 2000.

\bibitem{Bourgain}
J.~Bourgain,
Decoupling, exponential sums and the
Riemann zeta function,
\emph{J.~Amer.\ Math.\ Soc.}
\textbf{30} (2017), 205--224.

\bibitem{BCY}
H.~M.~Bui, J.~B.~Conrey, and M.~P.~Young,
More than 41\% of the zeros of the
zeta function are on the critical line,
\emph{Acta Arith.}
\textbf{150} (2011), 35--64.

\bibitem{Davenport}
H.~Davenport,
\emph{Multiplicative Number Theory},
3rd ed., revised by H.~L.~Montgomery,
Springer, 2000.

\bibitem{deBruijnNewman}
N.~G.~de~Bruijn,
The roots of trigonometric integrals,
\emph{Duke Math.\ J.}
\textbf{17} (1950), 197--226.

\bibitem{Edwards}
H.~M.~Edwards,
\emph{Riemann's Zeta Function},
Academic Press, 1974; Dover reprint, 2001.

\bibitem{Huxley}
M.~N.~Huxley,
On the difference between consecutive primes,
\emph{Invent.\ Math.}
\textbf{15} (1972), 164--170.

\bibitem{Ingham}
A.~E.~Ingham,
On the estimation of $N(\sigma,T)$,
\emph{Quart.\ J.\ Math.\ Oxford}
\textbf{11} (1940), 291--292.

\bibitem{Lagarias}
J.~C.~Lagarias,
An elementary problem equivalent to the
Riemann hypothesis,
\emph{Amer.\ Math.\ Monthly}
\textbf{109} (2002), 534--543.

\bibitem{Li}
X.-J.~Li,
The positivity of a sequence of numbers and
the Riemann hypothesis,
\emph{J.~Number Theory}
\textbf{65} (1997), 325--333.

\bibitem{PlattTrudgian}
D.~J.~Platt and T.~S.~Trudgian,
The Riemann hypothesis is true up to
$3\times 10^{12}$,
\emph{Bull.\ Lond.\ Math.\ Soc.}
\textbf{53} (2021), 792--797.

\bibitem{RodgersTao}
B.~Rodgers and T.~Tao,
The de~Bruijn--Newman constant is non-negative,
\emph{Forum Math.\ Pi}
\textbf{8} (2020), e6.

\bibitem{RosenblumRovnyak}
M.~Rosenblum and J.~Rovnyak,
\emph{Hardy Classes and Operator Theory},
Oxford University Press, 1985.

\bibitem{RudinRCA}
W.~Rudin,
\emph{Real and Complex Analysis},
3rd ed., McGraw--Hill, 1987.

\bibitem{Titchmarsh}
E.~C.~Titchmarsh,
\emph{The Theory of the Riemann Zeta-Function},
2nd ed., revised by D.~R.~Heath-Brown,
Oxford University Press, 1986.

\bibitem{WashburnPaper1}
J.~Washburn and A.~Rahnamai~Barghi,
A Schur Pinch theorem for arithmetic ratios:
reducing the Riemann Hypothesis to a positivity
condition,
Preprint, 2026.

\end{thebibliography}

\end{document}
