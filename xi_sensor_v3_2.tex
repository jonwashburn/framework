\documentclass[11pt]{amsart}

\usepackage[margin=1in]{geometry}
\usepackage{amsmath,amssymb,amsthm,mathtools}
\usepackage[T1]{fontenc}
\usepackage{lmodern}
\usepackage{microtype}
\usepackage{enumitem}
\usepackage{hyperref}
\usepackage[numbers,sort&compress]{natbib}
\hypersetup{colorlinks=true,linkcolor=blue,citecolor=blue,urlcolor=blue}

\newtheorem{theorem}{Theorem}[section]
\newtheorem{proposition}[theorem]{Proposition}
\newtheorem{lemma}[theorem]{Lemma}
\newtheorem{corollary}[theorem]{Corollary}
\theoremstyle{definition}
\newtheorem{definition}[theorem]{Definition}
\theoremstyle{remark}
\newtheorem{remark}[theorem]{Remark}

\newcommand{\R}{\mathbb{R}}
\newcommand{\C}{\mathbb{C}}
\newcommand{\angles}[1]{\langle #1\rangle}
\DeclareMathOperator{\re}{Re}
\DeclareMathOperator{\im}{Im}

\title[Xi-Sensor v3.2]{%
Xi-Sensor v3.2:\\
Resolution of the M4--M5 Engineering Targets}

\author{Jonathan Washburn}
\address{Recognition Science Research Institute, Austin, TX, USA}
\email{jon@recognitionphysics.org}
\date{\today}

\begin{document}
\maketitle

\begin{abstract}
We resolve the two engineering targets introduced in v3.1.
First, we prove that the originally stated M4 inequality
\[
\left|\int \psi_L\,\partial_\sigma U\right|
\le C\,L\,\sqrt{E_D(U)},
\qquad
E_D(U)=\iint_{Q}|\,\nabla U|^2\,\sigma\,d\sigma\,dt,
\]
is false in general (even for a single Blaschke potential source).
Second, we prove the M5 Whitney-box energy estimate unconditionally
for the corrected boundary-neutralized field:
\[
E_D(U_D)\le C_{M5}\,\log^2\angles{t_0}\,|I|,
\]
with $C_{M5}$ independent of the Whitney parameter $c$.
We then state the corrected next target M4$^\ast$ and quantify why
M4$^\ast$+M5 does not yet close RH.
\end{abstract}

%% ============================================================
\section{Setup}
%% ============================================================

For an off-line zero
\[
\rho=\frac12+\delta+i\gamma,\qquad \delta>0,\qquad
\rho^\#=1-\overline{\rho}=\frac12-\delta+i\gamma,
\]
define the half-plane Blaschke potential
\[
G_\rho(s):=\log\left|\frac{s-\rho^\#}{s-\rho}\right|.
\]
Then $G_\rho$ is harmonic on $\{\re s>1/2\}\setminus\{\rho\}$,
$G_\rho(1/2+it)=0$, and
\[
\partial_\sigma G_\rho(1/2+it)
=\frac{2\delta}{\delta^2+(t-\gamma)^2}\ge 0.
\]

Fix $t_0\in\R$ and
\[
L=\frac{c}{\log^2\angles{t_0}},\qquad
I=[t_0-L,t_0+L],\qquad
Q(\alpha'I)=\{1/2+\sigma+it:0<\sigma\le \alpha' L,\ t\in \alpha' I\}.
\]

%% ============================================================
\section{M4 (as stated in v3.1) is false}
%% ============================================================

\begin{theorem}[Counterexample to M4]\label{thm:M4-false}
The inequality
\begin{equation}\label{eq:M4-old}
\left|\int_{\R}\psi_{L,t_0}(t)\,\partial_\sigma U(1/2+it)\,dt\right|
\le C\,L\,\sqrt{E_D(U)},
\qquad
E_D(U)=\iint_{Q(\alpha'I)}|\nabla U|^2\,\sigma\,d\sigma\,dt,
\end{equation}
cannot hold with $C$ independent of $L$ for all harmonic
$U$ with zero boundary trace on $\sigma=0$.
\end{theorem}

\begin{proof}
Take
\[
U(s)=G_{\rho_0}(s),\qquad
\rho_0=\frac12+\delta_0+i t_0,\quad \delta_0>0,
\]
and assume $L\le \delta_0/(4\alpha')$.

Lower bound:
\[
\int \psi_{L,t_0}\,\partial_\sigma U(1/2+it)\,dt
\ge \int_{t_0-L}^{t_0+L}\frac{2\delta_0}{\delta_0^2+(t-t_0)^2}\,dt
=4\arctan(L/\delta_0)
\asymp \frac{L}{\delta_0}.
\]

Energy bound:
on $Q(\alpha'I)$, $|\nabla U|\ll 1/\delta_0$, hence
\[
E_D(U)\ll \frac{1}{\delta_0^2}
\iint_{Q(\alpha'I)}\sigma\,d\sigma\,dt
\ll \frac{L^3}{\delta_0^2}.
\]
Therefore
\[
L\sqrt{E_D(U)}\ll \frac{L^{5/2}}{\delta_0}.
\]
As $L\to 0$, $L^{5/2}/\delta_0=o(L/\delta_0)$, contradicting
\eqref{eq:M4-old} for any $L$-independent constant $C$.
\end{proof}

\begin{remark}
So the v3.1 target M4 is not merely unproved; it is false.
This is the decisive scaling obstruction.
\end{remark}

%% ============================================================
\section{M5 is true unconditionally (corrected neutralization)}
%% ============================================================

\begin{definition}[Unit-strip neutralization (except target)]
Fix a distinguished off-line zero
\(
\rho_0=\frac12+\delta_0+i\gamma_0
\)
and define
\[
\mathcal N:=\{\rho\neq \rho_0:\ |\im\rho-\gamma_0|\le 1,\ \re\rho>1/2\}.
\]
Define
\[
U_D(s):=\sum_{\rho\in Z_+\setminus \mathcal N} m_\rho\,G_\rho(s).
\]
\end{definition}

\begin{proposition}[M5]\label{prop:M5-true}
There exists $C_{M5}$ independent of $c$ such that
\[
E_D(U_D):=
\iint_{Q(\alpha'I)}|\nabla U_D|^2\,\sigma\,d\sigma\,dt
\le C_{M5}\,\log^2\angles{t_0}\,|I|.
\]
\end{proposition}

\begin{proof}
\textit{Step 1: boundary bound.}
For $s\in \partial Q(\alpha''I)$ and
$\rho\in Z_+\setminus\mathcal N$, either $\rho=\rho_0$ or
$|\im\rho-\gamma_0|>1$.
Assume $L\le 1/(4\alpha'')$. Then for $|\im\rho-\gamma_0|>1$:
\[
|\,\im s-\im\rho\,|\ge |\im\rho-\gamma_0|-\alpha''L\ge \tfrac12.
\]
Using $\log(1+x)\le x$ and $\delta_\rho\le 1/2$:
\[
G_\rho(s)
=\frac12\log\!\left(
1+\frac{4\sigma\delta_\rho}{(\sigma-\delta_\rho)^2+(\im s-\im\rho)^2}
\right)
\ll \frac{\sigma}{(\im s-\im\rho)^2}
\ll \frac{L}{(\im s-\im\rho)^2}.
\]
Hence
\[
\sum_{\substack{\rho\in Z_+\setminus\mathcal N\\ \rho\neq \rho_0}}
m_\rho G_\rho(s)
\ll L\sum_{\rho} \frac{m_\rho}{|\im\rho-\gamma_0|^2}.
\]
By Riemann--von Mangoldt in unit shells:
\[
\sum_{\rho}\frac{m_\rho}{|\im\rho-\gamma_0|^2}
\ll \sum_{n\ge 1}\frac{\log(\angles{t_0}+n)}{n^2}
\ll \log\angles{t_0}.
\]
So far part is $\ll L\log\angles{t_0}$.

For $\rho_0$:
\[
G_{\rho_0}(s)\ll \frac{\sigma}{\delta_0}\ll \frac{L}{\delta_0}.
\]
Therefore
\[
M:=\sup_{\partial Q(\alpha''I)}|U_D|
\ll L\log\angles{t_0}+\frac{L}{\delta_0}
\ll \log\angles{t_0},
\]
since $L\le 1$.

\textit{Step 2: interior gradient estimate.}
$U_D$ is harmonic on $Q(\alpha''I)$ by construction.
Standard interior gradient bounds for harmonic functions give
\[
\sup_{Q(\alpha'I)}|\nabla U_D|^2\ll \frac{M^2}{L^2}.
\]
Integrating with weight $\sigma$:
\[
E_D(U_D)\ll \frac{M^2}{L^2}
\iint_{Q(\alpha'I)}\sigma\,d\sigma\,dt
\ll \frac{M^2}{L^2}\cdot L^2|I|
\ll M^2 |I|
\ll \log^2\angles{t_0}\,|I|.
\]
This proves the claim.
\end{proof}

%% ============================================================
\section{What is now solved, and the corrected next target}
%% ============================================================

\begin{itemize}
\item M5 is solved unconditionally (Proposition~\ref{prop:M5-true}).
\item M4 in v3.1 form is false (Theorem~\ref{thm:M4-false}).
\end{itemize}

The corrected candidate is:
\[
\text{M4}^\ast:\quad
\left|\int \psi_{L,t_0}\,\partial_\sigma U_D\right|
\le C_{M4^\ast}\,\sqrt{\frac{E_D(U_D)}{L}}.
\]
This scaling is dimensionally consistent and compatible with
Theorem~\ref{thm:M4-false}.

\begin{remark}
M4$^\ast$ + M5 gives an upper bound of order $\log\angles{t_0}$,
while the single-zero lower bound is order $L$.
So M4$^\ast$+M5 does \emph{not} close RH.
To close, one needs either:
\begin{itemize}
\item a stronger energy estimate than M5 (sub-logarithmic enough to beat the
window scale), or
\item a different boundary functional with a lower bound not decaying like $L$.
\end{itemize}
\end{remark}

\begin{thebibliography}{99}
\bibitem{Davenport}
H.~Davenport, \emph{Multiplicative Number Theory}, 3rd ed., Springer, 2000.
\bibitem{Titchmarsh}
E.~C.~Titchmarsh, \emph{The Theory of the Riemann Zeta-Function},
2nd ed., Oxford University Press, 1986.
\end{thebibliography}

\end{document}
